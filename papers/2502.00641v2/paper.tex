% This must be in the first 5 lines to tell arXiv to use pdfLaTeX, which is strongly recommended.
\pdfoutput=1
% In particular, the hyperref package requires pdfLaTeX in order to break URLs across lines.

\documentclass[11pt]{article}

% Change "review" to "final" to generate the final (sometimes called camera-ready) version.
% Change to "preprint" to generate a non-anonymous version with page numbers.
\usepackage[final]{acl}

% Standard package includes
\usepackage{times}
\usepackage{latexsym}
\usepackage{enumitem}

% For proper rendering and hyphenation of words containing Latin characters (including in bib files)
\usepackage[T1]{fontenc}
% For Vietnamese characters
% \usepackage[T5]{fontenc}
% See https://www.latex-project.org/help/documentation/encguide.pdf for other character sets

% This assumes your files are encoded as UTF8
\usepackage[utf8]{inputenc}

% This is not strictly necessary, and may be commented out,
% but it will improve the layout of the manuscript,
% and will typically save some space.
\usepackage{microtype}

% This is also not strictly necessary, and may be commented out.
% However, it will improve the aesthetics of text in
% the typewriter font.
\usepackage{inconsolata}

%Including images in your LaTeX document requires adding
%additional package(s)
\usepackage{graphicx}

\usepackage{multirow}
\usepackage{adjustbox}
\usepackage{amssymb}
% If the title and author information does not fit in the area allocated, uncomment the following
%
%\setlength\titlebox{<dim>}
%
% and set <dim> to something 5cm or larger.

\title{Evaluating Small Language Models for News Summarization: Implications and Factors Influencing Performance}

% Author information can be set in various styles:
% For several authors from the same institution:
% \author{Author 1 \and ... \and Author n \\
%         Address line \\ ... \\ Address line}
% if the names do not fit well on one line use
%         Author 1 \\ {\bf Author 2} \\ ... \\ {\bf Author n} \\
% For authors from different institutions:
% \author{Author 1 \\ Address line \\  ... \\ Address line
%         \And  ... \And
%         Author n \\ Address line \\ ... \\ Address line}
% To start a separate ``row'' of authors use \AND, as in
% \author{Author 1 \\ Address line \\  ... \\ Address line
%         \AND
%         Author 2 \\ Address line \\ ... \\ Address line \And
%         Author 3 \\ Address line \\ ... \\ Address line}

\author{Borui Xu \\
  Shandong University \\
  \texttt{boruixu@mail.sdu.edu.cn} \\\And
  Yao Chen\thanks{Corresponding authors} \\
  National University of Singapore \\
  \texttt{yaochen@nus.edu.sg} \\
  \And
  Zeyi Wen \\
  HKUST(GZ) \& HKUST \\
  \texttt{wenzeyi@ust.hk}
  \AND
  Weiguo Liu\textsuperscript{*} \\
  Shandong University \\
  \texttt{weiguo.liu@sdu.edu.cn}
  \And
  Bingsheng He \\
  National University of Singapore \\
  \texttt{hebs@comp.nus.edu.sg}}

% \author{
%  \textbf{First Author\textsuperscript{1}},
%  \textbf{Second Author\textsuperscript{1,2}},
%  \textbf{Third T. Author\textsuperscript{1}},
%  \textbf{Fourth Author\textsuperscript{1}},
% \\
%  \textbf{Fifth Author\textsuperscript{1,2}},
%  \textbf{Sixth Author\textsuperscript{1}},
%  \textbf{Seventh Author\textsuperscript{1}},
%  \textbf{Eighth Author \textsuperscript{1,2,3,4}},
% \\
%  \textbf{Ninth Author\textsuperscript{1}},
%  \textbf{Tenth Author\textsuperscript{1}},
%  \textbf{Eleventh E. Author\textsuperscript{1,2,3,4,5}},
%  \textbf{Twelfth Author\textsuperscript{1}},
% \\
%  \textbf{Thirteenth Author\textsuperscript{3}},
%  \textbf{Fourteenth F. Author\textsuperscript{2,4}},
%  \textbf{Fifteenth Author\textsuperscript{1}},
%  \textbf{Sixteenth Author\textsuperscript{1}},
% \\
%  \textbf{Seventeenth S. Author\textsuperscript{4,5}},
%  \textbf{Eighteenth Author\textsuperscript{3,4}},
%  \textbf{Nineteenth N. Author\textsuperscript{2,5}},
%  \textbf{Twentieth Author\textsuperscript{1}}
% \\
% \\
%  \textsuperscript{1}Affiliation 1,
%  \textsuperscript{2}Affiliation 2,
%  \textsuperscript{3}Affiliation 3,
%  \textsuperscript{4}Affiliation 4,
%  \textsuperscript{5}Affiliation 5
% \\
%  \small{
%    \textbf{Correspondence:} \href{mailto:email@domain}{email@domain}
%  }
% }

\begin{document}
\maketitle
\begin{abstract}

% While large language models (LLMs) excel at text summarization, significant computing and memory resources limit their widespread application. A more accessible alternative is using small language models (SLMs), which have fewer parameters and can run on consumer devices like mobile phones and PCs. However, the abilities of SLMs in text summarization, as well as the gaps and characteristics compared to LLMs, require further study.
% Large language models (LLMs) deliver superior summarization quality but demand high computational resources, limiting their practical use. Small language models (SLMs) offer a more accessible alternative, capable of efficient real-time summarization on edge devices. However, their summarization capabilities, and how they compare to LLMs, require further investigation. In this paper, we present a comprehensive evaluation of SLMs for news summarization, focusing on relevance, coherence, factual consistency, and summary length. First, we evaluate 19 SLMs on 2,000 news samples and find that, despite significant variations in SLM summarization capabilities, the top-performing SLMs like Phi3-Mini and Llama3.2-3B-Ins exhibit performance comparable to that of 70B LLMs and generate more concise summaries. Then our further exploration reveals that SLMs are better suited for simple prompts, as overly complex prompts may lead to a decline in the summary quality. Additionally, we find that instruction tuning does not consistently enhance the news summarization capabilities of SLMs.

The increasing demand for efficient summarization tools in resource-constrained environments highlights the need for effective solutions. While large language models (LLMs) deliver superior summarization quality, their high computational resource requirements limit practical use applications. In contrast, small language models (SLMs) present a more accessible alternative, capable of real-time summarization on edge devices. However, their summarization capabilities and comparative performance against LLMs remain underexplored. This paper addresses this gap by presenting a comprehensive evaluation of 19 SLMs for news summarization across 2,000 news samples, focusing on relevance, coherence, factual consistency, and summary length. Our findings reveal significant variations in SLM performance, with top-performing models such as Phi3-Mini and Llama3.2-3B-Ins achieving results comparable to those of 70B LLMs while generating more concise summaries. Notably, SLMs are better suited for simple prompts, as overly complex prompts may lead to a decline in summary quality. Additionally, our analysis indicates that instruction tuning does not consistently enhance the news summarization capabilities of SLMs. This research not only contributes to the understanding of SLMs but also provides practical insights for researchers seeking efficient summarization solutions that balance performance and resource use.



\end{abstract}

%生成式两个字?
% what is the conclusion?
% what is the insight?
%2000 instances? more than previous works?
%limitation: reference bias? traditional math model?幻觉检测?缺少对数据质量对SLMs影响的进一步探索


%contributation
%我们使用LLM增强了基于reference评估与人类偏好的一致性
%不同SLM之间的差异巨大,一般3B左右的模型可以取得较好的结果,phi3目前最好
%在文本总结领域,LLM和SOTA SLM的差距很小(远小于其他任务?),使用SLM在终端设备上进行文本总结是可能的。
%和之前结论的不同,instruct不能更好summary
%不同数据集上保持稳定
%随着SLM的发展,summary的提升越来越小?

% 现有指标都只有用rougeL

%conclusion
% 我们认为现在一些小模型已经适合在端侧直接进行新闻文本的总结,其中相关性和连贯性已经很好,幻觉问题需要解决,相比新闻总结能力,提高它们在端侧的推理速度更加重要?
%要不要添加一些选择建议
%细化比较了instruct的影响,格式的影响
%和LLM稳定性比较之前工作没比较过
%细化和llm的比较
%效率对比?
%之前论文也没做过指令遵从的比较,有一个做过,但比较难,我们想简单的比如控制字数。这也是以后发展的一个方向
%SLM在relevance和coherence方面已经很好,未来更应该考虑fact的问题

% Large language models (LLMs) show significant performance in various downstream
tasks~\citep{brown_language_2020,openai_gpt-4_2024,dubey_llama_2024}. Studies
have found that training on high quality corpus improves the ability of LLMs
to solve different problems such as writing code, doing math exercises, and
answering logic questions~\citep{cai_internlm2_2024,deepseek-ai_deepseek-v3_2024,qwen_qwen25_2024}.
Therefore, effectively selecting high-quality text data is an important subject for
training LLM.

\begin{figure}[t]
    \centering
    \includegraphics[width=\linewidth]{figures/head.pdf}
    \caption{The overview of CritiQ. We (1) employ human annotators to annotate $\sim$30
    pairwise quality comparisons, (2) use CritiQ Flow to mine quality criteria, (3)
    use the derived criteria to annotate 25k pairs, and (4) train the CritiQ Scorer to
    perform efficient data selection.}
    \label{fig:overview}
\end{figure}

To select high-quality data from a large corpus, researchers manually design heuristics~\citep{dubey_llama_2024,rae_scaling_2022},
calculate perplexity using existing LLMs~\citep{marion2023moreinvestigatingdatapruning,wenzek2019ccnetextractinghighquality},
train classifiers~\citep{brown_language_2020,dubey_llama_2024,xie_data_2023} and
query LLMs for text quality through careful prompt engineering~\citep{gunasekar_textbooks_2023,wettig_qurating_2024,sachdeva_how_2024}.
Large-scale human annotation and prompt engineering require a lot of human
effort. Giving a comprehensive description of what high-quality data is like is also
challenging. As a result, manually designing heuristics lacks robustness and introduces
biases to the data processing pipeline, potentially harming model performance
and generalization. In addition, quality standards vary across different
domains. These methods can not be directly applied to other domains without significant
modifications.

To address these problems, we introduce CritiQ, a novel method to automatically
and effectively capture human preferences for data quality and perform efficient data
selection. Figure~\ref{fig:overview} gives an overview of CritiQ, comprising an agent
workflow, CritiQ Flow, and a scoring model, CritiQ Scorer. Instead of manually describing
how high quality is defined, we employ LLM-based agents to summarize quality
criteria from only $\sim$30 human-annotated pairs.

CritiQ Flow starts from a knowledge base of data quality criteria. The worker
agents are responsible to perform pairwise judgment under a given
criterion. The manager agent generates new criteria and refines them through reflection
on worker agents' performance. The final judgment is made by majority voting among
all worker agents, which gives a multi-perspective view of data quality.

To perform efficient data selection, we employ the worker agents to annotate a randomly
selected pairwise subset, which is ~1000x larger than the human-annotated one.
Following \citet{korbak_pretraining_2023,wettig_qurating_2024}, we train CritiQ
Scorer, a lightweight Bradley-Terry model~\citep{bradley_rank_1952} to convert
pairwise preferences into numerical scores for each text. We use CritiQ Scorer to
score the entire corpus and sample the high-quality subset.

For our experiments, we established human-annotated test sets to quantitatively
evaluate the agreement rate with human annotators on data quality preferences. We implemented the manager agent by \texttt{GPT-4o} and the worker
agent by \texttt{Qwen2.5-72B-Insruct}. We conducted experiments on different
domains including code, math, and logic, in which CritiQ Flow shows a consistent
improvement in the accuracies on the test sets, demonstrating the effectiveness
of our method in capturing human preferences for data quality. To validate the quality
of the selected dataset, we continually train \texttt{Llama 3.1}~\citep{dubey_llama_2024}
models and find that the models achieve better performance on downstream tasks
compared to models trained on the uniformly sampled subsets.

We highlight our contributions as follows. We will release the code to facilitate
future research.

\begin{itemize}
    \item We introduce CritiQ, a method that captures human preferences for data
        quality and performs efficient data selection at little cost of human
        annotation effort.

    \item Continual pretraining experiments show improved model performance in code,
        math, and logic tasks trained on our selected high-quality subset compared to the raw dataset.

    \item Ablation studies demonstrate the effectiveness of the knowledge base and
        the the reflection process.
\end{itemize}

\begin{figure*}[t]
    \centering
    \includegraphics[width=\linewidth]{figures/method.pdf}
    \caption{CritiQ Flow comprises two major components: multi-criteria pairwise
    judgment and the criteria evolution process. The multi-criteria pairwise
    judgment process employs a series of worker agents to make quality
    comparisons under a certain criterion. The criteria evolution process aims to
    obtain data quality criteria that highly align with human judgment through
    an iterative evolution. The initial criteria are retrieved from the
    knowledge base. After evolution, we select the final criteria to annotate
    the dataset for training CritiQ Scorer.}
    \label{fig:method}
\end{figure*}

% \section{Background and Related Work}



\textbf{CLIP}.
Contrastive Language-Image Pretraining (CLIP)~\citep{radford2021} trains multi-modal encoders to map text-image pairs into a shared latent space with semantically equal representations.
%(\citet{radford2021}).
The core of CLIP is a two-encoder architecture with an image encoder $\fimg$ and a text encoder $\ftxt$ that are trained to maximize the similarity between the image and text features for correct text-image pairs, while minimizing the similarity for incorrect pairs. This is achieved using a contrastive loss function $\mathcal{L}$ defined as:
\begin{align*}
\mathcal{L} = - \frac{1}{N} \sum_{i=1}^{N} \log \frac{\exp(\text{sim}(\fimg(x_i), \ftxt(y_i)) / \tau)}{\sum_{j=1}^{N} \exp(\text{sim}(\fimg(x_i), \ftxt(y_j)) / \tau)},
\end{align*}
where $\text{sim}(\cdot, \cdot)$ is the cosine similarity,  $\tau$  is the temperature parameter, and $N$ is the batch size.
This training makes CLIP versatile across various downstream tasks, including image classification, retrieval, captioning, and object recognition.
%It trains vision and text encoders by matching images and texts through its contrastive learning objective; it maximizes the cosine similarity between the correct image-text pairs while minimizing the similarity for incorrect pairs.
%The model projects both images and texts into a shared latent space, where related image-text pairs are pulled together and unrelated pairs are pulled apart.
%CLIP generalizes well to downstream tasks by training on vast amounts of labeled images scraped from the web, making it effective and cost-efficient for vision-language tasks.
There are different versions of CLIP.
%OpenCLIP is an open-source version that supports different architectures and datasets while still achieving performance competitive with the original CLIP model (\citet{cherti2023}).
The popular Language augmented CLIP (LaCLIP)~\citep{fan2023} augments original CLIP by introducing text augmentations during training in addition to the image augmentations (crops) performed in the original CLIP training to reduce overfitting.
We study the impact of this practice on memorization and find it to be a suitable mitigation method.

%builds upon CLIP by introducing text augmentations through language rewrites generated by large language models (\citet{fan2023}).
%This variation addresses the asymmetry in CLIP training, where only images are augmented but texts remain the same, improving model performance by mitigating text overfitting (\citet{fan2023}).
%In our experiments, we leverage both OpenCLIP and LaCLIP.

\textbf{Memorization.}
Memorization refers to a model's tendency to store specific details of individual training examples, rather than generalizing patterns across the dataset~\citep{zhang2016understanding,arpit2017closer,chatterjee2018learning,feldman2020does}. This becomes problematic when models memorize sensitive data, as it has been shown to increase privacy risks~\citep{carlini2019secret, carlini2021extracting, carlini2022privacy,song2017machine}. 
To date, memorization has been studied within \textit{single modalities} for supervised and self-supervised learning.
In \textbf{supervised learning}, it has been shown that models tend to memorize \textit{mislabeled}~\citep{feldman2020does}, \textit{difficult}, or \textit{atypical} examples~\citep{arpit2017closer,sadrtdinov2021memorization}, and that this memorization improves generalization, especially on long-tailed data~\citep{feldman2020does,feldman2020neural}.
Similar findings have been observed in \textbf{self-supervised learning} (SSL) in the vision domain~\citep{wang2024memorization}, where atypical samples experience high memorization, and a reduction of memorization in SSL encoders leads to decreased performance in
% a reduction of memorization in the SSL encoders reduces performance of
various downstream tasks, such as classification, depth-estimation, and segmentation.
A connection between memorization and generalization has also been observed in the language domain~\citep{antoniades2024generalization,tirumala2022memorization}.
In contrast to our work, these papers consider single-modality models. How those insights transfer to multi-modal models remains unclear.
%\franzi{Here, we can think about whether it is to our advantage or disadvantage to mention memorization in diffusion models.}\todo{@Adam, what do you think?}\adam{I'd say that above is enough.}
%\wenhao{What I am thinking is that the image generate by diffusion models are quite different with the original coco images. Although we use the same normalization parameters (from ImageNet) to normalize them, the distribution of dataset is still not proper enough to directly make comparison}
%\franzi{No, I mean work on mitigating memorization in diffusion models, like the one that I shared in the channel. They also do text masking. These are exactly things that are what we also do.}

\textbf{Memorization in self-supervised learning.}
Our \ours builds on concepts from the SSLMem metric introduced by \citet{wang2024memorization}. This metric measures the memorization of an individual data point $x$  by an SSL encoder, based on the alignment of representations from augmented views of $x$. Let $f:\mathbb{R}^n \to \mathbb{R}^d$ be an SSL encoder trained using an SSL algorithm $\mathcal{A}$ on an unlabeled dataset $S = \{x_i\}_{i=1}^{m}$. The data augmentations are represented as $\Aug(x) = \{a(x) | a \in \Aug\}$, 
% data augmentations can be represented as $\Aug(x) = \{a(x) | a \in \Aug\}$,
where $a$ is a transformation function applied to the data point $x$, mapping from $\mathbb{R}^n \to \mathbb{R}^n$.
The encoder's output representation for a given data point $x$ is denoted as $f(x)$. For a trained SSL encoder $f$, the alignment loss for a data point $x$ is defined as
\begin{equation}
    \lalign(f, x) = \alignexp{f}{x}\text{,}
    \label{eq:alignment_loss}
\end{equation}
where $x', x''$ are augmented views of $x$ and $d(\cdot, \cdot)$ is a distance metric, typically the $\ell_2$ distance.
SSLMem is then defined as
\begin{equation}\label{eq:memdef}
    \begin{split}
    \text{SSLMem}(x) =  \underset{g \sim \mathcal{A}(S \setminus x)}{\mathbb{E}} \ \lalign(g, x)
    - \underset{f \sim \mathcal{A} (S)}{\mathbb{E}} \ \lalign(f, x)  
    \end{split}
    \end{equation}
with $f$ being an SSL encoder trained with data point $x$, and $g$, an encoder trained without $x$ but otherwise on the same dataset.
While this framework measures memorization using alignment loss for single-modality encoders, this approach is unsuitable to leverage the signal over both modalities from multi-modal encoders like CLIP, as we also highlight empirically in \Cref{sub:sslmem_not_for_clip}. 
However, we can build on the main concepts from SSLMem to define a new metric that can evaluate memorization in CLIP, by considering both image and text representations, as we will detail in \Cref{sec:clipmem}.
%Memorization refers to a model's tendency to store specific details of individual training examples, rather than generalizing patterns across the dataset.
%In supervised learning (SL), memorization has been actively studied, with the \textit{leave-one-out} approach, where the model is retrained with a data point excluded from the training set (\citet{feldman2020}).
%If this results in a change in the model’s predictions, memorization of that specific data point can be directly measured.
%In self-supervised learning (SSL), memorization can be measured with alignment loss (\citet{wang2024}).
%Alignment measures how similar representations of different augmented views of the same data point are after training.
%A model that has memorized the training data will show a large difference in alignment loss compared to a model that hasn’t seen that data point.
%By calculating the alignment loss between an encoder trained with a data point and another encoder trained without it, the difference provides a measure of how much the model memorizes that specific data point.
%Additionally, in both SL and SSL, atypical (or outlier) data points that deviate from the general training data distribution were found to be more likely to be memorized. Memorization can impose data privacy risks as a model that memorizes can potentially leak sensitive information it was trained on.
%However, prior work suggests that some degree of memorization is necessary in enhancing model performance, and that atypical examples are harder to generalize from, therefore requiring memorization for better generalization and performance on downstream tasks.

\textbf{Memorization in CLIP}.
Even though CLIP is a widely used vision-language encoder, there has been limited work on measuring memorization in CLIP.
% work on measuring memorization in CLIP is still limited.
The only existing work~\citep{jayaraman2024} applies the empirical Déjà Vu memorization framework from~\citep{meehan2023ssl} to CLIP. It measures memorization by computing the overlap between unique objects in potentially memorized images and their nearest neighbors---identified in the CLIP embedding space---from a public dataset.
However, the reliance on external public data from the same distribution, along with the required accuracy of the object detection (which may not perform well for all samples, especially atypical ones~\citep{kumar2023normalizing,dhamija2020overlooked}, limits the applicability of this approach. We further expand on this in \Cref{app:deja_vu_comparison}.
% The reliance on external public data from the same distribution and the accuracy of the object detection, which might not work equally well for all samples, in particularly not atypical ones~\citep{kumar2023normalizing,dhamija2020overlooked} limit the applicability of this approach.
In contrast, our \ours operates directly on CLIP's output representations and returns a joint score over both modalities. 
% Additionally, it evaluates several memorization mitigation strategies, identifies early stopping and weight decay as effective, and additionally proposes text randomization, \ie using multiple differently masked captions for the same image during training as a defense to mitigate Déjà Vu memorization while preserving utility.
% Due to its reliance on an additional public dataset from the same distribution as the potentially memorized data, and due to the required fine-grained object detection, the CLIP Déjà Vu memorization framework~\citep{jayaraman2024} is not generally practical or applicable to all data distributions.
%  In contrast, our \ours metric does not make such strong assumptions. Finally, their work is focused on quantifying memorization in CLIP whereas we take the sample perspective and seek to understand which data points experience high memorization and why.\todo{Here, we could also highlight that their mitigation reduces performance whereas we can leverage CLIPMem to increase performance.} 
 % providing empirical insights into memorization in CLIP while we seek to understand the formal connection to generalization in the same way to~\citet{feldman2020does,wang2024memorization}.

% To date, the only work addressing this gap is the Déjà Vu framework (\citet{jayaraman2024}).
% In this framework, memorization in CLIP is measured by performing nearest neighbor searches to retrieve images from a public set of images, separate from the training set, using a target caption.
% If the model has memorized the image-text pair, the retrieved images will closely resemble the training image, beyond what would be expected from a simple correlation.
% This memorization is measured by counting the number of ground-truth objects from the original image that appear in the retrieved images, and comparing the results to a model that wasn’t trained on the same image-text pair.
% If the model retrieves more detailed objects than described in the caption, it indicates memorization, based on the belief that this behavior will not be observed in models that were trained without that specific image-text pair.
% However, the Déjà Vu framework is limited, as it relies on indirect measurements and is heavily dependent on the additional datasets selected for evaluation.
% Rather than directly evaluating whether a model reproduces specific information from its training set, it indirectly measures memorization by checking whether objects in the retrieved images align with those in the training images beyond what the captions describe.
% A significant issue with this approach is its dependence on external public datasets for nearest-neighbor searches.
% To determine whether a model has memorized certain training data, images are retrieved from a separate public dataset based on the model’s interpretation of the caption, and then compared to the original training images.
% This dependency introduces challenges, as the memorization test is then dependent on the quality, diversity, and size of the public dataset.
% For example, the dataset must be large and diverse enough to include images similar to the training data, which is not very practical and can undermine the reliability of the results.
%Our definition of memorization in CLIP builds upon SSLMem by leveraging the concept of alignment but extending it to handle multimodal data.
%Specifically, we measure the difference in alignment between representations of text-image pairs from encoders trained on these pairs and those that were not.
%By doing so, we can quantify how much the model memorizes specific text-image pairs across both modalities, reflecting CLIP’s contrastive learning objectives.
% Additionally, our approach uses canary data to directly measure elimination, eliminating the need for an external dataset.


% \section{Background and Setup}


% %In supervised learning (SL), memorization has been actively studied, with the \textit{leave-one-out} approach, where the model is retrained with a data point excluded from the training set (\citet{feldman2020}).
% %If this results in a change in the model’s predictions, memorization of that specific data point can be directly measured.
% %In self-supervised learning (SSL), memorization can be measured with alignment loss (\citet{wang2024}).
% %Alignment measures how similar representations of different augmented views of the same data point are after training.
% %A model that has memorized the training data will show a large difference in alignment loss compared to a model that hasn’t seen that data point.
% %By calculating the alignment loss between an encoder trained with a data point and another encoder trained without it, the difference provides a measure of how much the model memorizes that specific data point.

% We present concepts relevant to defining memorization in CLIP by leveraging main ideas from the SSLMem framework \citep{wang2024memorization}. This framework measures memorization of individual data points in SSL encoders, based on their augmentations and alignment. Let $f:\mathbb{R}^n \to \mathbb{R}^d$ be an encoder trained using an SSL algorithm $\mathcal{A}$ on an unlabeled dataset $S = \{x_i\}_{i=1}^{m}$. To evaluate memorization, data augmentations are used, represented as $\Aug(x) = \{a(x) | a \in \Aug\}$, where $a$ is a transformation function applied to the data point $x$, mapping from $\mathbb{R}^n \to \mathbb{R}^n$. The encoder's output representation for a given data point $x$ is represented as $f(x)$.

% \subsection{Alignment and Memorization in SSL}

% % Alignment Loss
% \textbf{Alignment Loss} quantifies the similarity between representations of different augmentations of a data point. For a trained SSL encoder $f$, the alignment loss for a data point $x$ is defined as:
% \begin{equation}
%     \lalign(f, x) = \alignexp{f}{x}\text{.}
%     \label{eq:alignment_loss}
% \end{equation}
% where $x', x''$ are augmented views of $x$ and $d(\cdot, \cdot)$ is a distance metric, typically the $\ell_2$ distance.

% % Memorization Score
% \textbf{Memorization Score} measures the degree to which an SSL model retains information about a particular data point, by comparing alignment losses between two encoders: $f$, trained with a data point $x$, and $g$, trained without it. The memorization score for $x$ is given by:
% \begin{equation}\label{eq:memdef}
%     \begin{split}
%     m(x) =  \underset{g \sim \mathcal{A}(S \setminus x)}{\mathbb{E}} \ \lalign(g, x)
%     - \underset{f \sim \mathcal{A} (S)}{\mathbb{E}} \ \lalign(f, x)  \text{.} 
%     \end{split}
% \end{equation}

% This score measures the extent to which the presence of $x$ in the training set affects the alignment. A positive score indicates that the encoder $f$ trained with $x$ shows a higher alignment compared to the encoder $g$, suggesting higher memorization.

% this is a very good description, I moved it to the appendix.
% \subsection{Experimental Setup}
% To experimentally evaluate memorization using the SSLMem framework, the training dataset $S$ is split into three sets: \textit{shared set} ($S_S$) used for training both encoders $f$ and $g$, \textit{candidate set} ($S_C$) used only for training encoder $f$, and \textit{independent set} ($S_I$) data used only for training encoder $g$. For training encoders, encoder $f$ is trained on $S_S \cup S_C$, while encoder $g$ is trained on $S_S \cup S_I$. The alignment losses $\lalign(f, x)$ and $\lalign(g, x)$ are computed for both encoders, and the memorization score $m(x)$ for each data point is derived as the difference between these alignment losses, normalized to a range between $-1$ and $1$. A score of $0$ indicates no memorization, $+1$ indicates the strongest memorization by $f$, and $-1$ indicates the strongest memorization by $g$.
% % \section{Evaluation Method}
% In this section, we first analyze the shortcomings of existing news summarization evaluation methods. Subsequently, we illustrate the new news summarization evaluation method used in this paper.


%1. Why use system level corr?
%2. different human different score?
%3. why this method? compared with llm evaluation is still well
%加一个coherence的图
% \section{Text Summarization with SLMs}
% \section{Methodology}
% \label{sec:method}

% In this section, we outline the process of news summarization with general-purpose SLMs and introduce reference summary generation for evaluation.






% \subsection{News Summarization with SLMs}



\section{LLM-Augmented Reference-Based Evaluation}\label{sec:method}


% Reference-based evaluation methods are widely used for evaluating text summarization~\cite{papineni-etal-2002-bleu,fabbri2021summeval,google-2023-benchmarking,tiny_titan}. These methods typically rely on statistical metrics (e.g., ROUGE~\cite{lin-2004-rouge}, BLEU~\cite{papineni-etal-2002-bleu}) or language models (e.g., BertScore~\cite{bert-score}, BLEURT~\cite{sellam-etal-2020-bleurt}) to measure the similarity between generated and reference summaries, with higher similarity indicating better quality. However, existing research~\cite{how_well,fabbri2021summeval} indicates that the poor quality of reference summaries in current datasets significantly weakens the correlation between these evaluation methods and human preferences. Moreover, \citet{zhang2024benchmarking} found that using summaries generated by freelance writers as references can improve alignment with human preferences. Numerous studies have shown that LLM-generated summaries are comparable in quality to those written by freelance writers and are often preferred for their style~\cite{goyal2022news,dead_summarization}. Therefore, we use LLM-generated summaries as high-quality reference summaries for the evaluation of SLMs. Figure~\ref{fig:workflow} illustrates how LLM-generated summaries are used as references to evaluate SLMs. The LLM generates summaries autoregressively, while the SLM-generated summaries are scored based on selected metrics.


% LLM-generated reference summaries not only match the quality of those created by freelance writers but also offer two distinct advantages. First, LLMs produce summaries faster than freelance writers. We can use LLMs to collect hundreds of summaries within a few hours, which is beneficial for large-scale news evaluations. In contrast, a freelance writer typically takes about 15 minutes to complete a single summary~\cite{zhang2024benchmarking}. Second, the financial cost of generating reference summaries with LLMs is lower. Hiring a freelance writer to summarize a news article costs around four dollars~\cite{zhang2024benchmarking}. In Section~\ref{sec:Verification}, we verify that LLM-generated reference summaries significantly improve reference-based evaluation.


As the reference-based method offers higher efficiency, better reproducibility, and lower cost compared to human and LLM evaluation~\cite{fabbri2021summeval,zhang2024benchmarking}, we use the reference-based method for large-scale SLM evaluation. However, some existing research~\cite{how_well,fabbri2021summeval} indicates that the poor quality of heuristic reference summaries in current datasets weakens their correlation with human preferences. Moreover, \citet{zhang2024benchmarking} find that using high-quality summaries as references can significantly improve alignment with human preferences. Thus, we use LLM-generated summaries as references in our SLM evaluation instead of relying on original dataset references. 
% Figure~\ref{fig:workflow} illustrates how LLM-generated summaries serve as references for evaluating SLMs.

LLM-generated reference summaries offer two key advantages.
First, they have higher quality than original heuristic summaries; \citet{zhang2024benchmarking} find LLMs averaged 4.5 on a five-point scale, while original references scored only 3.6. Additionally, \citet{dead_summarization} find that LLM-generated summaries are preferred to those written by freelance authors by up to 84\% of the time. 
Second, LLMs produce high-quality summaries quickly, generating hundreds in a few hours, which benefits large-scale news evaluations. In contrast, a freelance writer typically takes about 15 minutes for a single summary~\cite{zhang2024benchmarking}. 
% \end{itemize}



% Section~\ref{sec:Verification} further verifies that LLM-generated references enhance reference-based evaluation.

% \section{LLM-Generated Reference Verification}
% \label{sec:Verification}
We also verify whether the LLM-generated reference summaries can improve the effectiveness of reference-based evaluation on three human evaluation datasets~\cite{zhang2024benchmarking,fabbri2021summeval} in Table~\ref{tab:compare_kt_corr}. These datasets include summaries generated by various models as well as human ratings. We compare the correlation of three popular similarity metrics with human scoring using different references. BertScore and BLEURT calculate similarity based on fine-tuned language models, while RougeL measures similarity through textual overlap. We use prompt 2 in Figure~\ref{fig:prompt} as the input of the LLM. For all datasets, we consistently use two sentences to summarize the article. LLM-generated summaries significantly improve consistency than original references. BertScore performs best in both the relevance and coherence aspects.

\begin{table*}[]
% \caption{System-level Kendall's tau correlation coefficients with different reference summary settings.}
\begin{center}
% \begin{adjustbox}{max width=2.05\columnwidth}
\resizebox{2\columnwidth}{!}{
\begin{tabular}{cccccccccc}
\hline
\multirow{2}{*}{Metric} &
  \multirow{2}{*}{\begin{tabular}[c]{@{}c@{}}Reference\\ type\end{tabular}} &
  \multicolumn{4}{c}{Relevance evaluation} &
  \multicolumn{4}{c}{Coherence evaluation} \\
 &
   &
  \begin{tabular}[c]{@{}c@{}}Bench\\ -CNN/DM\end{tabular} &
  \begin{tabular}[c]{@{}c@{}}Bench\\ -XSum\end{tabular} &
  SummEval &
  Average &
  \begin{tabular}[c]{@{}c@{}}Bench\\ -CNN/DM\end{tabular} &
  \begin{tabular}[c]{@{}c@{}}Bench\\ -XSum\end{tabular} &
  SummEval &
  Average \\ \hline
\multirow{3}{*}{RougeL}   & Original    & 0.6732 & 0.2647 & 0.3714 & 0.4364          & 0.4248 & 0.6765 & 0.1238 & 0.4084          \\
                           & Qwen1.5-72B & 0.7778 & 0.7059 & 0.4476 & \textbf{0.6438} & 0.5556 & 0.5000 & 0.2381 & \textbf{0.4312} \\
                           & llama2-70B  & 0.7778 & 0.6324 & 0.4095 & 0.6066          & 0.5294 & 0.3676 & 0.2000 & 0.3657          \\ \hline
\multirow{3}{*}{BertScore} & Original    & 0.6209 & 0.2206 & 0.4476 & 0.4297          & 0.4771 & 0.6912 & 0.4286 & 0.5323          \\
                           & Qwen1.5-72B & 0.7516 & 0.6324 & 0.8095 & 0.7312          & 0.5817 & 0.6029 & 0.6000 & \textbf{0.5949} \\
                           & llama2-70B  & 0.7516 & 0.6618 & 0.8286 & \textbf{0.7413} & 0.5294 & 0.6029 & 0.6190 & 0.5838          \\ \hline
\multirow{3}{*}{BLEURT}    & Original    & 0.5425 & 0.2647 & 0.181  & 0.3294          & 0.3203 & 0.6176 & 0.2381 & 0.3929          \\
                           & Qwen1.5-72B & 0.5817 & 0.7206 & 0.4476 & 0.5833          & 0.4902 & 0.4853 & 0.5429 & 0.5061          \\
                           & llama2-70B  & 0.5556 & 0.7059 & 0.5429 & \textbf{0.6015} & 0.5425 & 0.4706 & 0.6762 & \textbf{0.5631} \\ \hline
\end{tabular}}
% \end{adjustbox}
\end{center}
\caption{{System-level Kendall's tau correlation coefficients between reference-based and human evaluations under different reference summary settings. "Original" is the original heuristic reference from the dataset; others are generated by LLMs. References generated by LLMs improve the effectiveness of reference-based evaluations.}}
\label{tab:compare_kt_corr}
\end{table*}






\begin{figure}
    \centering
    \includegraphics[width=1\linewidth]{ARR_version/fig/prompt2.pdf}
    \caption{ The prompt templates for the language model to generate summaries. }
    \label{fig:prompt}
\end{figure}




% %实验部分介绍数据集,如何生成的数据集,为什么选择500个
%这种测评方法的具体实现,数据清洗?prompt设计
%验证这种方法的结果和llm做测评是相媲美的。
%选择了那些tiny LLM进行测评,给出测评结果
%llm测评和我们测评的相关性。
%tiny llm的能力和34b,70b相比如何?
%每B的得分值
%与人类写手得分相比
%some example 给出具体的例子
%平均长度
%多少分算好?60分以下,65分以下,70分以上
%小模型上有什么问题?
%参数从多少到多少提升最大?模型越大性能越强,但到了一定程度提升不明显了
%instruct对性能的影响
%提出的方法的robust,不同prompt,不同LLM
%加入summary level?
%gpt-4o的测评?不用啊,本身就是测的和大模型的相似度
%data相关?为什么更好?因为训练数据方面么?
%template的影响?
%影响summarization能力的因素有哪些
%输出太长
%给出例子说明bertscore的有效性,相比原始reference更好
%instruct 对长度的影响
%小模型生成总结长度更短
%解释指标的含义,为什么要衡量总结的长短
%稳定性
%常见存储16GB?
%原始新闻的质量?给出例子
%为什么采用这种比较方法,因为不受reference长度影响结果
% 所有figure to fig.


\section{Benchmark Design}\label{sec:setup}


In this section, we first introduce the dataset for news summarization evaluation, followed by the SLM details, evaluation metrics, and reference summary generation.

% \subsection{Experimental Setup}
\subsection{News Article Selection }

All experiments are conducted on the following four datasets: CNN/DM~\cite{cnndm}, XSum~\cite{xsum}, Newsroom~\cite{newsroom}, and BBC2024. The first three datasets have been widely used for verifying model text summarization~\cite{goyal2022news,google-2023-benchmarking,zhang2024benchmarking}. However, their older data may overlap with the training datasets of some models. To address this, we create BBC2024, featuring news articles from January to March 2024 on the BBC website~\footnote{\url{http://dracos.co.uk/made/bbc-news-archive/}}. Based on prior research, 500 samples per dataset are sufficient to differentiate the models~\cite{google-2023-benchmarking}. We randomly select 500 samples from each of the four test sets and construct 2000 samples in total. Each sample contains 500 to 1500 tokens according to the Qwen1.5-72B-Chat tokenizer.



\subsection{Model Selection}
We select 21 popular models with parameter sizes not exceeding 4 billion for benchmarking, including 19 SLMs and 2 models specifically designed for text summarization, Pegasus-Large and Brio, which have been fine-tuned on text summarization datasets, detailed in Table~\ref{tab:model_list}.

% \textbf{LiteLlama~\cite{huggingface2024litelama}:} LiteLlama is an open-source reproduction of Llama2~\cite{touvron2023llama2openfoundation}. It has 460M parameters and is trained on 1T tokens from the RedPajama dataset~\cite{together2023redpajama}.

% \textbf{Bloom-560M~\cite{bloom}:} Bloom-560M is part of the BLOOM (BigScience Large Open-science Open-access Multilingual) family of models developed by the BigScience project. It has 560M parameters and is trained on 1.5T pre-processed text.

% \textbf{TinyLlama~\cite{zhang2024tinyllama}:} TinyLlama is an SLM which has the same architecture and tokenizer as Llama 2. It has 1.1B parameters and is trained on 3T tokens. We use the chat finetuned version.

% \textbf{GPT-Neo series~\cite{gpt-neo}:} GPT-Neo Series are open-source language models developed by EleutherAI to reproduce GPT-3 architecture. It has two versions with 1.3B and 2.7B parameters. They are trained on 380B and 420B tokens, respectively, on the Pile dataset~\cite{gao2020pile,biderman2022datasheet_pile}.

% \textbf{Qwen2 series~\cite{qwen}:} Qwen2 comprises a series of language models pre-trained on multilingual datasets. And some models use instruction tuning to align with human preferences. We select models with no more than 7B parameters for benchmarking, including 0.5B, 0.5B-Ins, 1.5B, 1.5B-Ins, 7B, and 7B-Ins. Models with the "Ins" suffix indicate the instruction tuning version.

% \textbf{InterLM2 series~\cite{cai2024internlm2}:} The InternLM2 series includes models with various parameter sizes. They all support long contexts of up to 200,000 characters. We select the 1.8B version and the 1.8B-Chat version for evaluation.

% \textbf{Gemma-1.1~\cite{team2024gemma}:} Gemma-1.1 is developed by Google and is trained using a novel RLHF method. We select the 2B instruction tuning version for benchmarking.

% \textbf{MiniCPM~\cite{hu2024minicpm}:} MiniCPM is an end-size language model with 2.7B parameters. It employs various post-training methods to align with human preferences. We select the supervised tuning (SFT) version for benchmarking.

% \textbf{Phi series~\cite{microsoft2023phi2,abdin2024phi3}:} Phi series are some SLMs developed by Microsoft. They are trained on high-quality datasets. We select Phi-2(2.7B parameters) and Phi-3-Mini(3.8B parameters, 4K context length) for evaluation. 

% \textbf{Brio~\cite{brio}: } Brio is one of the state-of-the-art specialized language models designed for text summarization, which leverages reinforcement learning to optimize the selection of sentences, ensuring high-quality, informative, and coherent summaries. It only has 406M parameters, and we select the version trained on the CNN/DM dataset.

% \textbf{Pegasus-Large~\cite{pegasus}:} Pegasus-Large also is a specialized language model designed for abstractive text summarization, utilizing gap-sentence generation pre-training to achieve exceptional performance on summarization tasks. It has 568M parameters.


%model details

% Meanwhile, to compare the performance of tiny LLMs with larger LLMs, we also consider the Llama3-70B. 

All models are sourced from the Hugging Face\footnote{\url{https://huggingface.co}} and tested based on the lm-evaluation-harness tool~\cite{eval-harness}. For general-purpose SLMs, we use the prompt from the huggingface hallucination leaderboard\footnote{\url{https://huggingface.co/spaces/hallucinations-leaderboard/leaderboard}} (Prompt 1 in Figure~\ref{fig:prompt}). To ensure reproducibility, we employ the greedy decoding strategy, generating until a newline character or <EOS> token is reached. Outputs are post-processed to remove prompt words and incomplete sentences. Given the 2048-token limit of some models, we only evaluate the zero-shot approach.

\begin{table}[]
\centering
\begin{adjustbox}{max width=1\columnwidth}
% \resizebox{\columnwidth}{!}{
\begin{tabular}{lccc}
\hline
Model          & Parameters & \begin{tabular}[c]{@{}c@{}}Instruction \\ Tuning\end{tabular} & Reference                                                              \\ \hline
Brio $^{\ast}$          & 406M         & $\checkmark$                                                  & \cite{brio} \\
LiteLlama      & 460M         & $\times$                                                      & \cite{huggingface2024litelama}       \\
Qwen2-0.5B     & 500M         & $\times$                                                      & \cite{qwen}              \\
Qwen2-0.5B-Ins & 500M         & $\checkmark$                                                  & \cite{qwen}     \\
Bloom-560M          & 560M         & $\times$                                                      & \cite{bloom}        \\
Pegasus-Large$^{\ast}$  & 568M         & $\checkmark$                                                  & \cite{pegasus}         \\
TinyLlama      & 1.1B         & $\checkmark$                                                  & \cite{zhang2024tinyllama} \\
Llama3.2-1B    & 1.2B         & $\times$                                                      & \cite{llama3.2} \\
Llama3.2-1B-Ins& 1.2B         & $\checkmark$                                                  & \cite{llama3.2} \\
GPT-Neo-1.3B   & 1.3B         & $\times$                                                      & \cite{gpt-neo}      \\
Qwen2-1.5B     & 1.5B         & $\times$                                                      & \cite{qwen}              \\
Qwen2-1.5B-Ins & 1.5B         & $\checkmark$                                                  & \cite{qwen}     \\
InternLM2-1.8B & 1.8B         & $\times$                                                      & \cite{cai2024internlm2}     \\
InternLM2-1.8B-Chat & 1.8B    & $\checkmark$                                                  & \cite{cai2024internlm2}     \\
MiniCPM        & 2.4B         & $\checkmark$                                                  & \cite{hu2024minicpm}  \\
Gemma-1.1      & 2.5B         & $\checkmark$                                                  & \cite{team2024gemma}       \\
GPT-Neo-2.7B   & 2.7B         & $\times$                                                      & \cite{gpt-neo}       \\
Phi-2          & 2.7B         & $\times$                                                      & \cite{microsoft2023phi2}             \\
Llama3.2-3B    & 3.2B         & $\times$                                                      & \cite{llama3.2} \\
Llama3.2-3B-Ins& 3.2B         & $\checkmark$                                                  & \cite{llama3.2} \\
Phi-3-Mini     & 3.8B         & $\checkmark$                                                  & \cite{abdin2024phi3}   \\ \hline
\end{tabular}
\end{adjustbox}
\caption{\upshape{Models for news summarization benchmark. Models marked with $^{\ast}$ have been fine-tuned on text summarization datasets.} }
\label{tab:model_list}
\end{table}


% \begin{figure}
%     \centering
%     \includegraphics[width=1\linewidth]{latex/fig/benchmark_prompt.pdf}
%     \caption{The benchmark prompt template for news summary generation. }
%     \label{fig:benchmark_prompt}
% \end{figure}


\subsection{Evaluation Metric }

We evaluate the summary quality in relevance, coherence, factual consistency, and text compression using BertScore~\cite{bert-score}, HHEM-2.1-Open~\cite{HHEM-2.1-Open}, and summary length.

% As mentioned earlier, BertScore aligns well with human preferences for relevance and coherence when using high-quality references. It measures how well the summary captures key information (relevance) and maintains logical flow (coherence). 

BertScore is a robust semantic similarity metric designed to assess the quality of the generated text and achieves the best correlation in Table~\ref{tab:compare_kt_corr}. It compares the embeddings of candidate and reference sentences through contextualized word representations from BERT~\cite{devlin-etal-2019-bert}. It can measure how well the summary captures key information (relevance) and maintains logical flow (coherence). We use the F1 score of BertScore\footnote{\url{https://huggingface.co/microsoft/deberta-xlarge-mnli}}, ranging from 0 to 100, where higher values indicate better quality.

HHEM-2.1-Open is a hallucination detection model which outperforms GPT-3.5-Turbo and even GPT-4. It can evaluate whether news summaries are factually consistent with the original article. It is based on a fine-tuned T5 model to flag hallucinations with a score. When the score falls below 0.5, it indicates a summary is inconsistent with the source. We report the percentage of summaries deemed factually consistent; higher values indicate better performance.

Since BertScore is insensitive to summary length, we also track the average summary length to assess text compression. With similar BertScore results, shorter summaries indicate better text compression ability.


\subsection{Reference Summary Generation}
To mitigate the impact of occasional low-quality reference summaries on evaluation results, as well as the tendency for models within the same series to score higher due to possible similar output content, we utilize two LLMs, Qwen1.5-72B-Chat and Llama2-70B-Chat, to generate two sets of reference summaries. We then average the scores during the SLM evaluation. We use Prompt 2 from Figure~\ref{fig:prompt} as the prompt template and apply a greedy strategy to generate summaries based on the vLLM inference framework~\cite{vllm}.



% Figure 5 shows the average length of the generated summaries for each dataset.






% \subsubsection{Parameter Setup}

% \subsection{Evaluation Results}

% \begin{table*}[]
% \caption{}
% \label{tab:my-table}
% \begin{adjustbox}{max width=2.05\columnwidth}
% \begin{tabular}{lcccccccccccc}
% \hline
% Model                & \multicolumn{3}{c}{XSum}   & \multicolumn{3}{c}{Newsroom} & \multicolumn{3}{c}{CNN/DM} & \multicolumn{3}{c}{BBC2024} \\
% \multicolumn{1}{c}{} & Qwen1.5 & Llama2 & Average & Qwen1.5  & Llama2  & Average & Qwen1.5 & Llama2 & Average & Qwen1.5  & Llama2 & Average \\ \hline
% LiteLama      & 38.68 & 39.24 & 38.96 & 38.02 & 38.70 & 38.36 & 40.58 & 41.38 & 40.98 & 38.05 & 38.86 & 38.45 \\
% Qwen2-0.5     & 69.57 & 70.71 & 70.14 & 65.94 & 66.92 & 66.43 & 69.50 & 70.48 & 69.99 & 68.23 & 69.67 & 68.95 \\
% Qwen2-0.5-Ins & 61.67 & 62.35 & 62.01 & 60.17 & 60.91 & 60.54 & 63.34 & 63.98 & 63.66 & 60.72 & 61.47 & 61.09 \\
% TinyLlama     & 68.07 & 69.16 & 68.61 & 64.70 & 65.63 & 65.16 & 68.66 & 69.23 & 68.95 & 66.93 & 68.07 & 67.50 \\
% GPT-Neo       & 59.67 & 59.55 & 59.61 & 52.54 & 53.11 & 52.83 & 58.37 & 58.81 & 58.59 & 58.16 & 58.18 & 58.17 \\
% Qwen2-1.5     & 72.07 & 72.65 & 72.36 & 69.12 & 69.86 & 69.49 & 71.30 & 72.05 & 71.67 & 71.08 & 71.99 & 71.53 \\
% Qwen2-1.5-Ins & 71.52 & 72.36 & 71.94 & 69.43 & 69.56 & 69.50 & 71.72 & 72.39 & 72.06 & 70.83 & 71.78 & 71.30 \\
% InternLM2-1.8 & 68.64 & 69.20 & 68.92 & 64.27 & 65.00 & 64.64 & 68.81 & 69.49 & 69.15 & 67.88 & 68.74 & 68.31 \\
% InternLM2-1.8-chat   & 67.77   & 68.82  & 68.30   & 64.99    & 65.57   & 65.28   & 68.78   & 69.38  & 69.08   & 68.32    & 69.62  & 68.97   \\
% Gemma-1.1     & 70.80 & 72.27 & 71.53 & 69.42 & 70.30 & 69.86 & 71.88 & 72.83 & 72.35 & 70.60 & 71.67 & 71.13 \\
% MiniCPM       & 65.59 & 65.77 & 65.68 & 62.48 & 63.13 & 62.80 & 66.82 & 67.51 & 67.17 & 65.54 & 65.96 & 65.75 \\
% Phi-2         & 70.95 & 72.03 & 71.49 & 68.15 & 69.14 & 68.64 & 71.19 & 72.21 & 71.70 & 70.38 & 71.50 & 70.94 \\
% Phi-3-mini    & 74.31 & 74.65 & 74.48 & 72.07 & 72.64 & 72.36 & 74.54 & 75.24 & 74.89 & 73.63 & 74.05 & 73.84 \\
% ChatGLM3      & 73.51 & 74.08 & 73.79 & 70.77 & 71.11 & 70.94 & 73.66 & 74.27 & 73.97 & 72.57 & 73.41 & 72.99 \\
% Mistral       & 51.73 & 51.98 &       & 57.13 & 57.45 &       & 51.19 & 52.10 &       & 59.85 & 59.47 &       \\
% Mistral-Ins   & 74.51 & 75.33 & 74.92 & 71.18 & 71.81 & 71.49 & 74.29 & 74.81 & 74.55 & 73.90 & 74.32 & 74.11 \\
% Qwen2-7B      & 75.49 & 75.73 & 75.61 & 73.65 & 73.73 & 73.69 & 74.94 & 75.90 & 75.42 & 74.89 & 75.27 & 75.08 \\
% Qwen2-7B-Ins  & 75.04 & 75.50 & 75.27 & 73.12 & 73.01 & 73.07 & 75.14 & 74.28 & 74.71 & 74.24 & 74.71 & 74.48 \\
% Llama3-instruct      & 77.39   & 76.77  & 77.08   & 75.47    & 74.67   & 75.07   & 77.63   & 76.95  & 77.29   & 76.79    & 76.06  & 76.42   \\ \hline
% \end{tabular}
% \end{adjustbox}
% \end{table*}



\section{Evaluation Results }
\label{sec:benchmark}
% In this section, we first evaluate the news summarization. Then we compare them with larger LLMs and give model recommendations with different model sizes.



%根据分数划分结果

\subsection{Relevance and Coherence Evaluation}


\begin{table*}[]
\begin{center}
\resizebox{2\columnwidth}{!}{
% \begin{adjustbox}{max width=2.05\columnwidth}
\begin{tabular}{llccccccccr}
\hline
Score Range                  & Model         & \multicolumn{2}{c}{XSum} & \multicolumn{2}{c}{Newsroom} & \multicolumn{2}{c}{CNN/DM} & \multicolumn{2}{c}{BBC2024} & Average \\
                    &                    & Qwen1.5 & Llama2 & Qwen1.5 & Llama2 & Qwen1.5 & Llama2 & Qwen1.5 & Llama2 & \multicolumn{1}{l}{} \\ \hline
\multirow{4}{*}{$< 60$} & LiteLlama           & 38.68   & 39.24  & 38.02   & 38.70  & 40.58   & 41.38  & 38.05   & 38.86  & 39.19                \\
                    & Bloom-560M              & 48.23   & 48.43  & 42.73   & 43.28  & 49.70   & 50.04  & 47.56   & 48.01  & 47.25                \\
                    & GPT-Neo-1.3B            & 59.67   & 59.55  & 52.54   & 53.11  & 58.37   & 58.81  & 58.16   & 58.18  & 57.30                \\ 
                    & GPT-Neo-2.7B            & 60.11   & 60.15  & 53.65   & 54.14  & 59.15   & 59.97  & 59.29   & 59.50  & 58.25                \\ \hline
\multirow{10}{*}{$60 \sim 70$}
                    & Llama3.2-1B       &62.44    &62.33   &55.25    &55.86   &60.32    &60.67  &62.17&62.33      &60.17    \\
                    & Pegasus-Large      &61.39    &61.44   &61.26    &61.64   &61.94    &62.26  &63.12&63.47      &62.07    \\
                    & Llama3.2-3B       &66.14    &65.84   &59.93    &60.68   &64.19    &64.51  &66.62&66.57      &64.31    \\
                    & MiniCPM            & 65.59   & 65.77  & 62.48   & 63.13  & 66.82   & 67.51  & 65.54   & 65.96  & 65.35                \\
                    & TinyLlama          & 68.07   & 69.16  & 64.70   & 65.63  & 68.66   & 69.23  & 66.93   & 68.07  & 67.56                \\
                    & InternLM2-1.8B      & 68.64   & 69.20  & 64.27   & 65.00  & 68.81   & 69.49  & 67.88   & 68.74  & 67.75                \\
                    & InternLM2-1.8B-Chat & 67.77   & 68.82  & 64.99   & 65.57  & 68.78   & 69.38  & 68.32   & 69.62  & 67.91                \\
                    & Qwen2-0.5B-Ins & 69.30       & 70.01      & 65.85   & 66.65   &69.31   & 70.06 & 67.54 & 68.64        & 68.42   \\
                    & Qwen2-0.5B          & 69.57   & 70.71  & 65.94   & 66.92  & 69.50   & 70.48  & 68.23   & 69.67  & 68.88                \\
                    &Brio   & 70.67  & 70.71  &68.43  & 68.41   &69.86  &70.34  &70.39  & 70.21  & 69.88  \\ \hline
\multirow{7}{*}{$> 70$} & Phi-2              & 70.95   & 72.03  & 68.15   & 69.14  & 71.19   & 72.21  & 70.38   & 71.50  & 70.69                \\
                    & Gemma-1.1          & 70.80   & 72.27  & 69.42   & 70.30  & 71.88   & 72.83  & 70.60   & 71.67  & 71.22                \\
                    & Qwen2-1.5B          & 72.03   & 72.90  & 69.19   & 69.91  & 71.33   & 72.42  & 71.08   & 72.18  & 71.38                \\
                    & Qwen2-1.5B-Ins      & 72.21   & 73.06  & 69.49   & 69.86  & 71.64   & 71.95  & 71.01   & 71.97  & 71.40                \\
                    & Llama3.2-1B-Ins     & 72.24   & 73.27  & 70.54   & 70.98  & 72.43   & 73.61  & 71.68   & 72.55  & 72.16                \\
                    & Phi-3-Mini         & \textbf{74.67}   & {75.14}  & {72.42}   & {72.32}  & \textbf{74.86}   & {74.88}  & {74.08}   & {73.72}  & {74.01} \\  
                    & Llama3.2-3B-Ins     & 74.40   & \textbf{75.33}  & \textbf{72.81}   & \textbf{73.41}  & 74.54   & \textbf{75.29}  & \textbf{74.51}   & \textbf{74.95}  & \textbf{74.41}   \\ \hline
                  
\end{tabular}}
\end{center}
% \end{adjustbox}
\caption{\upshape {BertScore of SLMs on four text summarization datasets. Qwen1.5-72B-Chat and Llama2-70B-Chat are used to generate reference summaries.}}
\label{tab:bertscore_res}
\end{table*}


\begin{figure}
    \centering
    \includegraphics[width=1\linewidth]{ARR_version/fig/bertscore_example5.pdf}
    \caption{ Example summaries from SLMs and LLMs. The bold part is the same as the reference, the underline indicates irrelevant content, and the red indicates incorrect content. Summaries with BertScore above 70, such as those from Llama3.2-3B-Ins, demonstrate similar quality to LLMs. }
    \label{fig:bertscore_example}
\end{figure}

Table~\ref{tab:bertscore_res} shows the average BertScore results of SLMs. All models demonstrate consistent performance across datasets. Due to the significant score differences, we categorize the models into three approximate ranges, using two specialized summarization models, Pegasus-Large and Brio, as reference points.


The first range includes scores below 60. \textbf{Models scoring below 60 struggle to effectively summarize articles.} LiteLlama, Bloom, and GPT-Neo series fall into this range. In terms of relevance, these models often miss key points; in terms of coherence, these models sometimes produce repetitive outputs, leading to overly long summaries. Figure~\ref{fig:bertscore_example} shows an example from LiteLlama, it scores only 40.81, deviating significantly from the news.


The second range covers scores between 60 and 70. \textbf{Models in this range produce useful summaries but occasionally lack key points.} Models like TinyLlama and Qwen-0.5B series fall into this range. In relevance, these models generally relate to the original content but may omit crucial details. In coherence, the sentences can be well-organized. Fig~\ref{fig:bertscore_example} shows examples from Qwen2-0.5B. Although its summary is relevant and coherent to the news article, it omits the final score.


The third range includes scores above 70. \textbf{Models in this range produce summaries comparable to LLMs, with occasional inconsistencies.} They effectively capture key points and maintain coherent structure. Phi3-Mini and Llama3.2-3B-Ins stand out in this group. As shown in Figure~\ref{fig:bertscore_example}, Llama3.2-3B-Ins generates a concise, accurate summary, correctly capturing the match outcome even though the source news lacks a direct score description.







\subsection{Factual Consistency Evaluation}


% \begin{table}[]
% \caption{\upshape{Factual consistency rate of SLMs on news summarization datasets.}}
% \label{tab:fact}
% \begin{adjustbox}{max width=1\columnwidth}
% \begin{tabular}{lccccc}
% \hline
% Model          & XSum  & Newsroom & CNN/DM & BBC2024 & Average \\ \hline
% Qwen2-1.5B     & 95.39 & 92.86    & 97.65  & 93.27   & 94.79   \\
% Qwen2-1.5B-Ins & 94.77 & 95.28    & 98.51  & 94.57   & 95.78   \\
% Gemma1.1       & 95.35 & 96.46    & 98.60  & \textbf{94.68}   & \textbf{96.27}   \\
% Phi2           & 93.00 & 95.46    & 97.41  & 93.43   & 94.82   \\
% Phi3-Mini      & \textbf{96.29} & \textbf{97.09}    & \textbf{99.14}  & 92.27   & 96.20   \\ \hline
% \end{tabular}
% \end{adjustbox}
% \end{table}




During summary generation, SLMs may produce false information due to hallucinations. Therefore, we report the factual consistency rate in Table~\ref{tab:fact}. Models with high BertScore generally have better factual consistency as shown in Figre~\ref{fig:bertscore_example}. SLMs that perform well in fact consistency include Phi3-Mini (96.7\%) and Qwen2-1.5B-Ins (96.5\%), maintaining over 95\% consistency across all datasets. The traditional model, Pegasus-large, outperforms SLMs in fact consistency, which achieves an exceptional 99.9\% average. This is because it often copies sentences from the original text, ensuring alignment but sometimes missing key information. 


% This table also includes LLM results, showing minimal gaps between LLMs and top SLMs, with Phi3-Mini and Qwen2-1.5B even outperforming Llama3-70B-Ins on multiple datasets.


% fig3中加上fact分数





\subsection{Summary Length Evaluation}



\begin{figure}
    \centering
    \includegraphics[width=1\linewidth]{ARR_version/fig/hist_summary_length.pdf}
    \caption{Average summary length comparison. SLMs with high BertScore generate 50-70 word summaries.}
    \label{fig:summary_length}
\end{figure}




% Average summary length changes with BertScore. The summary length first decreases, then increases, and finally remains stable. 
Figure~\ref{fig:summary_length} shows the average summary lengths on 4 datasets produced by SLMs, which vary significantly across models. For scores below 60, summaries typically exceed 100 words due to redundant content and poor summarization. In the 60–70 range, length varies widely; e.g., the Llama3.2 series includes too many unnecessary details, while TinyLlama omits key information, leading to shorter summaries. Models with scores above 70 generate more consistent summaries, averaging around 50 to 70 words in length. For the top-performing models, Phi3-Mini and Llama3.2 series, Phi3-Mini generates more concise summaries while maintaining similar relevance.




\begin{table}[]
\begin{adjustbox}{max width=1\columnwidth}
\begin{tabular}{lccccc}
\hline
Model              & XSUM &Newsroom & CNN/DM & BBC2024 & Average \\ \hline
LiteLama           & 60.6 & 63.4    & 68.8   & 63.6    & 64.1    \\
Bloom-560M         & 83.6 & 77.4                          & 87.0   & 83.4    & 82.9    \\
GPT-Neo-1.3B       & 91.2 & 76.8                          & 92.4   & 92.0    & 88.1    \\
GPT-Neo-2.7B       & 83.2 & 67.0                          & 87.6   & 84.0    & 80.5    \\
Llama3.2-1B        & 97.6 & 85.8                         & 97.6  & 95.8      & 94.2    \\
Pegasus-Large      & 99.6 & 100.0                         & 100.0  & 100.0   & \textbf{99.9}    \\
Llama3.2-3B        & 98.8 & 85.6                         & 98.8    & 97.4    & 95.2    \\
MiniCPM            & 85.8 & 72.8                          & 93.6   & 89.6    & 85.5    \\
TinyLlama          & 95.4 & 92.4                          & 98.4   & 98.0    & 96.1    \\
InternLM2-1.8B     & 91.8 & 87.4                          & 96.0   & 94.8    & 92.5    \\
InternLM2-1.8B-Chat & 88.6 & 84.6                          & 93.8   & 90.8    & 89.5    \\
Qwen2-0.5B-Ins     & 87.2 & 86.0                          & 93.2   & 89.6    & 89.0    \\
Qwen2-0.5B         & 93.0 & 92.8                          & 96.8   & 95.2    & 94.5    \\
Brio               & 94.8 & 89.0                          & 97.8   & 95.4    & 94.3    \\
Phi2               & 95.2 & 93.4                          & 97.8   & 96.8    & 95.8    \\
Gemma-1.1          & 96.6 & 93.6                          & 97.8   & 94.6    & 95.7    \\
Qwen2-1.5B         & 95.6 & 96.2                          & 97.4   & 96.6    & 96.5    \\
Qwen2-1.5B-Ins     & 91.8 & 91.2                          & 94.4   & 92.4    & 92.5    \\
Llama3.2-1B-Ins    & 92.2 & 88.0                          & 95.4   & 94.0    & 92.4    \\
Phi3-Mini          & 96.0 & 94.8                          & 98.2   & 97.6    & \textbf{96.7}    \\ 
Llama3.2-3B-Ins    & 93.2 & 93.2                          & 96.6   & 95.6    & 94.7    \\ \hline
% ChatGLM3           & 96.0 & 95.2                          & 99.2   & 96.0    & 96.6    \\
% Mistral-7B-Ins     & 92.6 & 88.0                          & 96.8   & 93.0    & 92.6    \\
% Qwen2-7B-Ins       & 94.0 & 95.0                          & 98.6   & 95.6    & 95.8    \\
% Llama3-70B-Ins     & 94.6 & 94.8                          & 97.6   & 95.8    & 95.7    \\
% Qwen2-72B-Ins      & 97.0 & 97.6                          & 98.8   & 98.4    & 98.0    \\ \hline
\end{tabular}
\end{adjustbox}
\caption{\upshape{Factual consistency rate (\%) on news datasets.}}
\label{tab:fact}
\end{table}

% Summary length reflects the model's ability to compress information. Although TinyLlama produces the shortest summaries, its low BertScore suggests it misses critical details. Therefore, we consider text compression only when BertScore exceeds 70. Among these, Gemma-1.1 (43 words) and Qwen2-1.5-Ins (41 words) demonstrate the best compression performance.

%总结
\subsection{Comparison with LLMs}

% \begin{figure}
%     \centering
%     \includegraphics[width=1\linewidth]{latex/fig/vsLLM_bertscore.pdf}
%     \caption{ Bertscore between two models. }
%     \label{fig:vsllm_bertscore}
% \end{figure}



% \begin{table}[]
% \centering
% % \resizebox{\columnwidth}{!}{
% \begin{adjustbox}{max width=1\columnwidth}
% \begin{tabular}{lcccc}
% \hline
% Model &
%   \begin{tabular}[c]{@{}c@{}}Minimum\\ length\end{tabular} &
%   \begin{tabular}[c]{@{}c@{}}Maximum\\ length\end{tabular} &
%   \begin{tabular}[c]{@{}c@{}}Average\\ length\end{tabular} &
%   \begin{tabular}[c]{@{}c@{}}Standard\\ Deviation\end{tabular} \\ \hline
% Phi2           & 17 & 219 & 51.00 & 38.27 \\
% Gemma-1.1      & 1  & 72  & 41.84 & 8.68  \\
% Qwen2-1.5B-Ins & 16 & 206 & 40.97 & 17.85 \\
% Phi3-Mini      & 25 & 182 & 50.69 & 13.09 \\
% ChatGLM3       & 15 & 139 & 59.31 & 19.33 \\
% Mistral-7B-Ins & 26 & 191 & 59.62 & 27.85 \\
% Qwen2-7B-Ins   & 33 & 169 & 58.57& 16.06 \\
% Llama3-70B-Ins & 51 & 95  & 71.06 & 7.31  \\
% Qwen2-72B-Ins  & 31 & 254 & 85.75 & 21.67 \\ \hline
% \end{tabular}
% \end{adjustbox}
% \caption{\upshape{Summary length comparison of SLMs and LLMs on BBC2024 dataset.}}
% \label{tab:vs_llm_length}
% \end{table}

To further evaluate the SLM performance in news summarization, we compare the well-performing SLMs with LLMs, including ChatGLM3~\cite{glm2024chatglm}, Mistral-7B-Ins~\cite{jiang2023mistral}, Llam3-70B-Ins~\cite{llama3modelcard}, and Qwen2 series, all in instruction-tuned versions. As the models show similar performance across various reference summaries and datasets, we use Llama2-70B-Chat to generate references and present the average results on BBC2024 in Table~\ref{tab:vs_llm_bertscore}. 


We highlight the top two metrics within each category. As can be seen, the summarization quality produced by the SLMs is on par with that of the LLMs. The difference in BertScore and factual consistency is minimal, with SLMs occasionally performing better. Notably, when BertScore scores are similar, smaller models tend to generate shorter summaries that facilitate quicker reading and comprehension of news outlines. 

\begin{table}[]
\centering
% \resizebox{\columnwidth}{!}{
\begin{adjustbox}{max width=1\columnwidth}
\begin{tabular}{lccc}
\hline
Model & BertScore & \begin{tabular}[c]{@{}c@{}}Factual \\ consistency\end{tabular} & \begin{tabular}[c]{@{}c@{}}Summary\\ length\end{tabular} \\ \hline
Qwen2-1.5B-Ins  & 71.97 & 92.4 & \textbf{41} \\
Phi3-Mini       & 73.72 & \textbf{97.6} & \textbf{51} \\
Llama3.2-3B-Ins & \textbf{74.95} & 95.6 & 70 \\
ChatGLM3        & 73.41 & 96.0 & 59 \\
Mistral-7B-Ins  & 74.32 & 93.0 & 60 \\
Qwen2-7B-Ins    & 74.71 & 95.6 & 59 \\
Llama3-70B-Ins  & \textbf{76.06} & 95.8 & 71 \\
Qwen2-72B-Ins   & 73.78 & \textbf{98.4} & 86 \\ \hline
\end{tabular}
\end{adjustbox}
\caption{\upshape{Average metric comparison of SLMs and LLMs on BBC2024 dataset.}}
\label{tab:vs_llm_bertscore}
\end{table}


\begin{table}[]
\centering
\begin{adjustbox}{max width=1\columnwidth}
% \resizebox{\linewidth}{!}{
\begin{tabular}{lccc}
\hline
Model & \multicolumn{1}{r}{Prompt 1} & \multicolumn{1}{r}{Prompt 2} & \multicolumn{1}{r}{Prompt 3} \\ \hline
Qwen2-0.5B-Ins  & 68.64 & 60.31 & 62.46 \\
Qwen2-1.5B-Ins  & 71.97 & 69.74 & 72.30 \\
Llama3.2-1B-Ins & 72.55 & 72.16 & 67.53 \\
Phi3-Mini       & 73.72 & 73.27 & 73.83 \\
Llama3.2-3B-Ins & 74.95 & 74.94 & 75.12 \\ \hline
\end{tabular}
\end{adjustbox}
\caption{\upshape{BertScore comparison with different prompts.} }
\label{tab:diff_prompt_bertscore}
\end{table}

% Although SLM trails behind LLM in both BertScore and stability, the difference is minimal. Llama3-70B-Ins achieves the highest average BertScore of 76.06, reflecting its superior summary quality, and also has the lowest standard deviation (5.12), indicating consistent performance. Although Phi3-Mini is an SLM, its scores and stability are comparable to LLMs. Its BertScore surpasses that of the ChatGLM3, and its standard deviation is lower than 7B LLMs like Qwen2-7B-Ins and Mistral-7B-Ins.


% Table~\ref{tab:vs_llm_length} shows summary length statistics. SLMs generally produce more concise summaries than LLMs, especially those around 70B parameters. Although SLMs occasionally generate longer summaries, their average length is typically shorter due to the use of simpler sentences and fewer descriptive words. In Figure~\ref{fig:bertscore_example}, while both Phi3-Mini and Llama2-70B-Chat summarize match results and rankings, Llama2-70B-Chat uses more descriptive language for additional details. In terms of summary length stability, there is little difference between SLMs and LLMs. Models like Gemma-1.1 and Llama3-70B-Ins show consistent summary lengths, while Qwen2-72B-Ins and Phi2 exhibit greater variability.


\subsection{Human Evaluation}


To further validate the effectiveness of our metrics, we performed a small-scale human evaluation, referencing existing studies~\cite{fabbri2021summeval}. Specifically, six highly educated annotators evaluated the relevance of news summaries generated by five SLMs based on the original news articles (scored on a 1-5 scale, with higher scores indicating greater relevance). We randomly selected 20 news instances from the BBC2024 dataset as our evaluation set. As shown in Table~\ref{tab:small_human_eva}, the average scores of each model demonstrate a strong alignment between BertScore and human evaluations, achieving a Kendall correlation coefficient of 1.

\begin{table*}[]
\begin{tabular}{lrrrrrrrr}
\hline
Model & \multicolumn{1}{l}{Ann. 1} & \multicolumn{1}{l}{Ann. 2} & \multicolumn{1}{l}{Ann. 3} & \multicolumn{1}{l}{Ann. 4} & \multicolumn{1}{l}{Ann. 5} & \multicolumn{1}{l}{Ann. 6} & \multicolumn{1}{l}{Average} & \multicolumn{1}{l}{BertScore} \\ \hline
Bloom-560M & 1.95 & 1.25 & 1.43 & 1.60 & 1.00 & 1.00 & 1.37 & 46.30 \\
Llama3.2-3B & 3.15 & 2.10 & 2.25 & 2.50 & 1.35 & 1.15 & 2.08 & 66.36 \\
Qwen2-0.5B & 4.05 & 2.15 & 3.28 & 2.70 & 3.60 & 3.25 & 3.17 & 69.85 \\
Phi3-Mini & 4.45 & 4.60 & 4.00 & 3.45 & 4.60 & 4.25 & 4.20 & 73.14 \\
Llama3.2-3B-Ins & 4.65 & 4.45 & 4.35 & 3.65 & 4.00 & 4.10 & 4.23 & 75.15 \\ \hline
\end{tabular}
\caption{Humance evaluation in relevance on 5 SLMs.}
\label{tab:small_human_eva}
\end{table*}



\subsection{Overall Evaluation and Model Selection}
SLMs exhibit considerable variability in text summarization performance, with larger and newer models generally showing stronger capabilities. Among them, Phi3-Mini and Llama3.2-3B-Ins perform the best, matching the 70B LLM in relevance, coherence, and factual consistency, while producing shorter summaries. All of these comparisons suggest that SLMs can effectively replace LLMs on edge devices for news summarization.

Although we evaluate 19 SLMs, there is considerable variation in their parameter sizes. To optimize deployment on edge devices, we provide model selection recommendations based on size. For models under 1B parameters, Brio and Qwen2-0.5B are the top choices overall. General-purpose language models in this range offer no clear advantage over those specialized in text summarization. For models between 1B and 2B parameters, Llama3.2-1B-Ins demonstrates a clear edge. For models above 2B parameters, Llama3.2-3B-Ins and Phi3-Mini outperform others across all criteria.

% \begin{table}[]
% \caption{\upshape{Overall SLM performance evaluation for different model sizes}}
% \label{tab:by_size}
% \begin{adjustbox}{max width=1\columnwidth}
% \begin{tabular}{ccccc}
% \hline
% Parameters & Model & BertScore($\uparrow$) & Faithfulness($\uparrow$) & Length($\downarrow$) \\ \hline
% \multirow{6}{*}{$< 1B$} & Brio                & \textbf{69.88} & 94.3          & \textbf{53.70} \\
%                   & LiteLama            & 39.19          & 64.1          & 186.88         \\
%                   & Bloom-560M          & 47.25          & 82.9          & 198.60         \\
%                   & Qwen2-0.5B          & \textbf{68.88} & \textbf{94.5} & \textbf{44.03} \\
%                   & Qwen2-0.5B-Ins      & 68.42          & 89.0          & 61.38          \\
%                   & Pegasus-Large       & 62.07          & \textbf{99.9} & 97.37          \\ \hline
% \multirow{6}{*}{$1B \sim 2B$} & TinyLlama           & 67.56          & \textbf{96.1} & \textbf{36.80} \\
%                   & GPT-Neo-1.3B        & 57.30          & 88.1          & 169.60         \\
%                   & Qwen2-1.5B          & \textbf{71.38} & \textbf{96.5} & 59.10          \\
%                   & Qwen2-1.5B-Ins      & \textbf{71.40} & 92.5          & \textbf{41.20} \\
%                   & InternLM2-1.8B      & 67.75          & 92.5          & 45.29          \\
%                   & InternLM2-1.8B-Chat & 67.91          & 89.5          & 43.58          \\ \hline
% \multirow{5}{*}{$> 2B$} & MiniCPM             & 65.35          & 85.5          & 105.90         \\
%                   & Gemma-1.1           & \textbf{71.22} & \textbf{95.7} & \textbf{42.78} \\
%                   & GPT-Neo-2.7B        & 58.25          & 80.5          & 121.55         \\
%                   & Phi-2               & 70.69          & 95.8          & 52.72          \\
%                   & Phi-3-Mini          & \textbf{74.01} & \textbf{96.7} & \textbf{51.88} \\\hline
% \end{tabular}
% \end{adjustbox}
% \end{table}



% \begin{table*}[]
% % \caption{System-level Kendall's tau correlation coefficients with different reference summary settings.}
% \begin{center}
% % \begin{adjustbox}{max width=2.05\columnwidth}
% \resizebox{2\columnwidth}{!}{
% \begin{tabular}{cccccccccc}
% \hline
% \multirow{2}{*}{Metric} &
%   \multirow{2}{*}{\begin{tabular}[c]{@{}c@{}}Reference\\ type\end{tabular}} &
%   \multicolumn{4}{c}{Relevance evaluation} &
%   \multicolumn{4}{c}{Coherence evaluation} \\
%  &
%    &
%   \begin{tabular}[c]{@{}c@{}}Bench\\ -CNN/DM\end{tabular} &
%   \begin{tabular}[c]{@{}c@{}}Bench\\ -XSum\end{tabular} &
%   SummEval &
%   Average &
%   \begin{tabular}[c]{@{}c@{}}Bench\\ -CNN/DM\end{tabular} &
%   \begin{tabular}[c]{@{}c@{}}Bench\\ -XSum\end{tabular} &
%   SummEval &
%   Average \\ \hline
% \multirow{3}{*}{RougeL}   & Original    & 0.6732 & 0.2647 & 0.3714 & 0.4364          & 0.4248 & 0.6765 & 0.1238 & 0.4084          \\
%                            & Qwen1.5-72B & 0.7778 & 0.7059 & 0.4476 & \textbf{0.6438} & 0.5556 & 0.5000 & 0.2381 & \textbf{0.4312} \\
%                            & llama2-70B  & 0.7778 & 0.6324 & 0.4095 & 0.6066          & 0.5294 & 0.3676 & 0.2000 & 0.3657          \\ \hline
% \multirow{3}{*}{BertScore} & Original    & 0.6209 & 0.2206 & 0.4476 & 0.4297          & 0.4771 & 0.6912 & 0.4286 & 0.5323          \\
%                            & Qwen1.5-72B & 0.7516 & 0.6324 & 0.8095 & 0.7312          & 0.5817 & 0.6029 & 0.6000 & \textbf{0.5949} \\
%                            & llama2-70B  & 0.7516 & 0.6618 & 0.8286 & \textbf{0.7413} & 0.5294 & 0.6029 & 0.6190 & 0.5838          \\ \hline
% \multirow{3}{*}{BLEURT}    & Original    & 0.5425 & 0.2647 & 0.181  & 0.3294          & 0.3203 & 0.6176 & 0.2381 & 0.3929          \\
%                            & Qwen1.5-72B & 0.5817 & 0.7206 & 0.4476 & 0.5833          & 0.4902 & 0.4853 & 0.5429 & 0.5061          \\
%                            & llama2-70B  & 0.5556 & 0.7059 & 0.5429 & \textbf{0.6015} & 0.5425 & 0.4706 & 0.6762 & \textbf{0.5631} \\ \hline
% \end{tabular}}
% % \end{adjustbox}
% \end{center}
% \caption{{System-level Kendall's tau correlation coefficients between reference-based and human evaluations under different reference summary settings. "Original" is the original heuristic reference from the dataset; others are generated by LLMs. References generated by LLMs improve the effectiveness of reference-based evaluations.}}
% \label{tab:compare_kt_corr}
% \end{table*}

\section{Influencing Factor Analysis}
\label{sec:factor}
% In this section, we further select some well-performing SLMs to explore factors influencing the quality of text summarization, including prompt design and instruction tuning. 


\subsection{Prompt Design}
Prompt engineering has demonstrated significant power in many tasks, helping to improve the output quality of LLMs~\cite{zhao2021calibrate,surveypromptengineering}. Therefore, we use different prompt templates shown in Figure~\ref{fig:prompt} to analyze the impact of prompt engineering on BertScore scores, factual consistency, and summary length. The instructions become more detailed from Prompt 1 to Prompt 3.



\begin{figure}
    \centering
    \includegraphics[width=1\linewidth]{ARR_version/fig/diff_prompt_example2.pdf}
    \caption{Examples summaries generated by Llama3.2-3B-Ins with different prompts. The bold parts show the same content, suggesting that the prompt design has a limited improvement in summary quality.}
    \label{fig:diff_prompt_example}
\end{figure}




Table~\ref{tab:diff_prompt_bertscore} compares the BertScore. The small score differences among prompts suggest that detailed descriptions do not significantly improve summary relevance and coherence and may even negatively affect some models. This suggests that \textbf{simple instructions are more suitable for SLMs, and overly complex prompts may degrade the performance}. For instance, the BertScore of Qwen2-0.5B-Ins drops significantly. The decline is likely due to prompt noise, where the model struggles to balance multiple instructions, resulting in less coherent summaries. Additionally, Figure~\ref{fig:diff_prompt_example} shows an example of Llama3.2-3B-Ins with different prompts. All summaries capture the key information, with only slight differences in details.



Table~\ref{tab:diff_prompt_fact} provides a comparison of factual consistency rates for various models using three different prompts. Many models, such as Qwen2-0.5B-Ins, exhibit a significant drop in factual consistency when using prompt 3. This further demonstrates that models with very small parameters are not well-suited for complex prompt design. The summary length variations across different prompts in Table~\ref{tab:diff_prompt_length} also indicate the limitations of prompt design. Some models fail to generate summaries of appropriate length given complex prompts.





\begin{table}[]
\centering
\begin{adjustbox}{max width=1\columnwidth}
% \resizebox{\linewidth}{!}{
\begin{tabular}{lccc}
\hline
Model & \multicolumn{1}{r}{Prompt 1} & \multicolumn{1}{r}{Prompt 2} & \multicolumn{1}{r}{Prompt 3} \\ \hline
Qwen2-0.5B-Ins  & 89.6 & 89.8  & 78.2 \\
Qwen2-1.5B-Ins  & 92.4 & 90.8  & 89.0 \\
Llama3.2-1B-Ins & 94.0 & 94.8  & 91.6 \\
Phi3-Mini       & 97.6 & 96.4  & 97.2 \\
Llama3.2-3B-Ins & 95.6 & 96.2  & 94.6 \\ \hline
\end{tabular}
\end{adjustbox}
\caption{\upshape{Factual consistency rate (\%)  comparison with different prompts.} }
\label{tab:diff_prompt_fact}
\end{table}


\begin{table}[]
\centering
\begin{adjustbox}{max width=1\columnwidth}
% \resizebox{\linewidth}{!}{
\begin{tabular}{lccc}
\hline
Model & \multicolumn{1}{r}{Prompt 1} & \multicolumn{1}{r}{Prompt 2} & \multicolumn{1}{r}{Prompt 3} \\ \hline
Qwen2-0.5B-Ins  & 65 & 150 & 151 \\
Qwen2-1.5B-Ins  & 41 & 99  & 53 \\
Llama3.2-1B-Ins & 63 & 60  & 58 \\
Phi3-Mini       & 51 & 57  & 65  \\
Llama3.2-3B-Ins & 70 & 67  & 61 \\ \hline
\end{tabular}
\end{adjustbox}
\caption{\upshape{Summary length with different prompts. All prompts require models to summarize in two sentences.} }
\label{tab:diff_prompt_length}
\end{table}



Overall, the improvement of prompt design on the quality of SLM summaries is limited. For SLMs that already perform well with simple prompts, they can capture the key points without the need for complex prompts. In contrast, SLMs that perform poorly may struggle with the logical relationships in complex prompts, leading to lengthy summaries. For practical applications, we recommend using simple and clear prompts when deploying SLMs for news summarization.










% Models that already perform well do not benefit from prompt adjustments, while models with insufficient capabilities are not improved by prompt design and may even produce lower-quality summaries.






\subsection{Instruction Tuning}
Instruction tuning plays a crucial role in training LLMs by enhancing their ability to follow specific instructions. And some existing studies claim that instruction-tuned language models have stronger summarization capabilities~\cite{goyal2022news,zhang2024benchmarking}. However, in our extensive evaluations, with the exception of the Llama3.2 series, models perform very similarly before and after instruction tuning. In fact, models without instruction tuning often exhibit higher factual consistency. In the sample analysis, we find that the Llama3.2 models without instruction tuning include too many details, while instruction tuning helps it produce more general summaries. However, this trend is not observed in other models like Qwen2 and InternLM2 series, which is different from the conclusions of previous work~\cite{zhang2024benchmarking}. We leave the deep study of instruction tuning on summarization ability as a future research direction.


% \section{LLM-Generated Reference Verification}
% \label{sec:Verification}
% In this section, we verify whether the LLM-generated reference summaries can improve the effectiveness of reference-based evaluation on three human evaluation datasets~\cite{zhang2024benchmarking,fabbri2021summeval}. These datasets include summaries generated by various models as well as human ratings. We compare the consistency with human evaluations using different reference summaries and show results in Table~\ref{tab:compare_kt_corr}. As can be seen, LLM-generated summaries significantly improve consistency than original references. BertScore shows the best performance, which validates the effectiveness of using LLM-generated reference summaries in our evaluations.



% method on three human evaluation datasets~\cite{zhang2024benchmarking,fabbri2021summeval} and show results in Table~\ref{tab:compare_kt_corr}. We use Prompt 2 in Figure~\ref{fig:prompt} as the LLM input. The output is post-processed to remove introductory phrases and retain only complete sentences for the final summaries. For the LLM model, we select Qwen1.5-72B-Chat~\cite{qwen1} and Llama2-70B Chat~\cite{touvron2023llama2openfoundation}. As can be seen, LLM-generated summaries significantly improve consistency with human evaluation, particularly for relevance.  According to previous work~\cite{kendeer}, a Kendall coefficient above 0.3 indicates a strong correlation, making BertScore a suitable metric for the following SLM summarization benchmark.






% T 检验?
% instruction tuning更容易停止,更短
% 指令调优效果在不同模型上不同,以及可能有负面作用,这个是需要进一步研究的东西
%prompt design 对表现不好的模型有帮助么?


% Additionally, the impact of instruction tuning on summary length did not display a consistent trend. It is worth noting that after instruction tuning, the summaries generated by Qwen2-0.5B-Ins increased significantly in length.

% We also explore whether instruction tuning can enhance the model ability to control summary length. We ask the models to generate summaries of a specified length. Specifically, based on the prompt 1 template, we specify the word count or the sentence count for the generated summaries. 
% Tables~\ref{tab:summary_len_words} and~\ref{tab:summary_len_sentence} show the results, and model names with the suffix "Ins" or "Chat" are instruction-tuned versions. we find that instruction tuning does not significantly help in controlling summary length. Except for Phi3-Mini, the instruction-tuned models perform similarly to the original versions in limiting summary words. Under different summary length constraints, only Phi3-Mini meets the requirements well. When considering the word limit, although the summary length increases with the word limit increasing, most models output a length that exceeds the word limit. Only Phi3-Mini generates summaries that match it closely. When sentence count is limited, Phi3-Mini produces summaries with each sentence containing 20-30 words, effectively adhering to the specified number of sentences. Phi2 and the Qwen2 series perform well only when the number of sentences is specified. Regardless of whether they have undergone instruction tuning, they are unable to effectively limit the number of words. 

% Although instruction tuning has limited effects on controlling summary length, it helps models better terminate their output. During testing, we find that the instruction-tuned versions more accurately adhere to the specified token (such as <eos>) limit, effectively stopping output. In contrast, for the non-instruction-tuned versions, we had to detect the newline character to halt the output, which sometimes resulted in truncating the generated content.
% how to stop

% \begin{table}[]
% \caption{\upshape{Summary length under different prompt word limits.}}
% \label{tab:summary_len_words}
% \centering
% \begin{adjustbox}{max width=1\columnwidth}
% \begin{tabular}{lcccc}
% \hline
% Model          & 50 words & 60 words & 70 words & 80 words \\ \hline
% Qwen2-1.5B-Ins & 75.01    & 71.99    & 79.19    & 84.40    \\
% Qwen2-1.5B     & 84.01    & 86.84    & 98.64    & 105.57   \\
% InternLM2-1.8B-Chat & 91.59 & 93.62  & 98.76  & 106.14  \\
% InternLM2-1.8B & 77.47 & 101.86 & 110.29  & 116.47 \\
% Phi2           & 115.68   & 137.78   & 148.54   & 157.19  \\
% Phi3-Mini      & 49.78    & 56.31    & 67.78    & 76.32    \\
% Qwen2-7B-Ins   & 78.57    & 86.00    & 91.00    & 99.07    \\
% Qwen2-7B       & 71.78    & 86.81    & 87.32    & 91.54    \\ \hline
% \end{tabular}
% \end{adjustbox}
% \end{table}

% \begin{table}[]
% \caption{\upshape{Summary length under different prompt sentence limits.}}
% \label{tab:summary_len_sentence}
% \centering
% \begin{adjustbox}{max width=1\columnwidth}
% \begin{tabular}{lcccc}
% \hline
% Model &
%   \begin{tabular}[c]{@{}c@{}}One\\ sentence\end{tabular} &
%   \begin{tabular}[c]{@{}c@{}}Two\\ sentences\end{tabular} &
%   \begin{tabular}[c]{@{}c@{}}Three\\ sentences\end{tabular} &
%   \begin{tabular}[c]{@{}c@{}}Four\\ sentences\end{tabular} \\ \hline
% Qwen2-1.5B-Ins & 31.61 & 40.90 & 61.80 & 75.17 \\
% Qwen2-1.5B     & 38.08 & 58.89 & 78.95 & 93.78 \\
% InternLM2-1.8B-Chat & 44.48 & 57.69 & 75.75 & 95.10 \\
% InternLM2-1.8B & 23.45 & 41.17 & 66.01 & 78.90 \\
% Phi2           & 35.50 & 60.00 & 69.96 & 97.09 \\
% Phi3-Mini      & 35.25 & 50.69 & 78.39 & 97.49 \\
% Qwen2-7B-Ins   & 44.92 & 59.80 & 80.32 & 96.35 \\
% Qwen2-7B       & 42.38 & 59.44 & 74.21 & 86.58 \\ \hline
% \end{tabular}
% \end{adjustbox}
% \end{table}







% \subsection{Prompt Design}

% \begin{table}[]
% \caption{\upshape{BertScore comparison with different prompts.} }
% \label{tab:diff_prompt_bertscore}
% \centering
% \begin{adjustbox}{max width=1\columnwidth}
% % \resizebox{\linewidth}{!}{
% \begin{tabular}{lrrr}
% \hline
% Model          & Prompt 1 & Prompt 2 & Prompt 3 \\ \hline
% Qwen2-0.5B-Ins & 68.64   & 68.09   & 68.30   \\
% Qwen2-0.5B     & 69.67   & 69.58   & 69.67   \\
% Qwen2-1.5B-Ins & 71.78   & 72.02   & 71.70   \\
% Qwen2-1.5B     & 71.99   & 72.12   & 72.22   \\
% Phi2           & 71.50   & 70.97   & 71.87   \\
% Phi3-Mini      & 73.72   & 74.13   & 74.49   \\
% Qwen2-7B-Ins   & 74.71   & 73.40   & 73.88   \\
% Qwen2-7B       & 75.27   & 75.37   & 74.99   \\ \hline
% \end{tabular}
% \end{adjustbox}

% \end{table}




% \begin{table}[]
% \caption{\upshape{BertScore and summary length comparison with different prompts}}
% \label{tab:diff_detail_prompt}
% \centering
% \begin{adjustbox}{max width=1\columnwidth}
% \begin{tabular}{lrrrrrr}
% \hline
% Model          & \multicolumn{3}{c}{BertScore}  & \multicolumn{3}{c}{Summary Length} \\
%                & Prompt 1 & Prompt 2 & Prompt 3 & Prompt 1   & Prompt 2  & Prompt 3  \\ \hline
% Qwen2-0.5B-Ins & 61.47    & 68.09    & 68.30    & 140.75     & 51.52     & 64.79     \\
% Qwen2-0.5B     & 69.67    & 69.58    & 69.67    & 40.84      & 30.93     & 37.90     \\
% Qwen2-1.5B-Ins & 71.78    & 72.02    & 71.70    & 53.36      & 42.68     & 41.38     \\
% Qwen2-1.5B     & 72.18    & 72.12    & 72.22    & 58.89      & 48.64     & 56.98     \\
% Phi2           & 71.50    & 70.97    & 71.87    & 60.00      & 100.88    & 72.554    \\
% Phi3-Mini      & 73.72    & 74.13    & 74.49    & 50.69      & 58.27     & 58.66     \\
% Qwen2-7B-Ins   & 74.71    & 73.40    & 73.88    & 58.57      & 64.19     & 63.74     \\
% Qwen2-7B       & 75.27    & 75.37    & 74.99    & 56.41      & 50.78     & 51.64     \\ \hline
% \end{tabular}
% \end{adjustbox}
% \end{table}


% Prompt engineering has demonstrated significant power in many tasks, helping to improve the output quality of LLMs~\cite{zhao2021calibrate,surveypromptengineering}. Therefore, we use different prompt templates shown in Fig~\ref{fig:prompt} to analyze the impact of prompt engineering on BertScore scores, factual consistency, and summary length. From Prompt 1 to Prompt 3, the instructions become increasingly detailed. 
% % Tables~\ref{tab:diff_detail_prompt} present the results.

% Table~\ref{tab:diff_prompt_bertscore} compares the BertScore for various models using three different prompts. The changes across different prompts are minimal, indicating that more detailed prompt words cannot significantly improve the relevance and coherence of the summary. This consistency is beneficial for practical applications where prompts may vary, ensuring stable and reliable text generation performance. Fig~\ref{fig:diff_prompt_example} shows some examples of Qwen-0.5B-Ins with different prompts. All summaries can capture the key information, and prompts 2 and 3 result in only minor changes in connective words. 

% Table~\ref{tab:diff_prompt_fact} provides a comparison of factual consistency rates for various models using three different prompts. Although most models show slight variations in factual consistency rates, they show consistent changes under some prompts. On most language models, using prompt 2 improves fact consistency by about 2\% compared with prompt 1. The BBC2024 test dataset includes 500 instances. A 2\% improvement means improving the consistency of facts on 10 summaries.

% However, prompt design has a significant impact on summary length, especially for SLMs. Although they are all asked to summarize the article in two sentences, some SLMs exhibit significant fluctuations, as shown in Table~\ref{tab:diff_prompt_len}. Only Phi3-Mini can maintain the stable summary length. Prompts affect different models in varying ways. For the Qwen2 series, although prompt 2 is more detailed than prompt 1, it generates shorter summaries. However, for the Phi series, summaries generated using prompt 2 are longer than those generated using prompt 1. Fig~\ref{fig:diff_prompt_example} shows examples of summaries generated by Qwen2-0.5B-Ins under different prompts. While all summaries capture the key points, the summary generated with prompt 1 is significantly longer due to more details.



% \begin{table}[]
% \caption{\upshape{Factual consistency rate (\%) with different prompts.} }
% \label{tab:diff_prompt_fact}
% \centering
% \begin{adjustbox}{max width=1\columnwidth}
% % \resizebox{\linewidth}{!}{
% \begin{tabular}{lrrr}
% \hline
% Model          & Prompt 1 & Prompt 2 & Prompt 3 \\ \hline
% Qwen2-0.5B-Ins & 89.6    & 88.8   & 88.2   \\
% Qwen2-0.5B     & 95.2    & 97.4   & 96.0   \\
% Qwen2-1.5B-Ins & 92.4    & 94.0   & 91.8   \\
% Qwen2-1.5B     & 96.6    & 97.6   & 97.6   \\
% Phi2           & 96.8    & 99.0   & 97.2   \\
% Phi3-Mini      & 97.6    & 98.2   & 98.6   \\
% Qwen2-7B-Ins   & 95.6    & 96.0   & 95.0   \\
% Qwen2-7B       & 96.8    & 97.4   & 96.0    \\ \hline
% \end{tabular}
% \end{adjustbox}

% \end{table}


% \begin{table}[]
% \caption{\upshape{Summary length comparison with different prompts.}}
% \label{tab:diff_prompt_len}
% \centering
% \begin{adjustbox}{max width=1\columnwidth}
% \begin{tabular}{lrrr}
% \hline
% Model          & Prompt 1 & Prompt 2 & Prompt 3 \\ \hline
% Qwen2-0.5B-Ins & 64.80  & 51.52   & 64.79   \\
% Qwen2-0.5B     & 40.84   & 30.93   & 37.90   \\
% Qwen2-1.5B-Ins & 53.36   & 42.68   & 41.38   \\
% Qwen2-1.5B     & 58.89   & 48.64   & 56.98   \\
% Phi2           & 60.00   & 100.88  & 72.55    \\
% Phi3-Mini      & 53.87   & 58.27   & 58.66   \\
% Qwen2-7B-Ins   & 58.57   & 64.19   & 63.74   \\
% Qwen2-7B       & 56.41   & 50.78   & 51.64   \\ \hline
% \end{tabular}
% \end{adjustbox}
% \end{table}




% \subsection{Model Size and Training Data}
% To ensure that model size is the only variable, we can focus on models from the same series, such as the Qwen2 series and GPT-Neo series. As can be seen in Table~\ref{tab:bertscore_res} and~\ref{tab:diff_prompt_bertscore}, it is evident that increasing the model size improves the performance on BertScore, but the effect diminishes with larger models, which is consistent with existing studies~\cite{zhang2024benchmarking}. For instance, Qwen2-7B-Ins scores on the BBC2024 dataset even surpass those of Qwen2-72B-Ins. In terms of factual consistency, having more parameters does not necessarily lead to improvement. For example, as shown in Table~\ref{tab:fact}, the summaries generated by GPT-Neo-1.3B are more aligned with the original articles than those generated by GPT-Neo-2.7B by around 8\%.

% Besides model size, high-quality training data is also crucial. When the model sizes are similar, there is also a large gap between different model series. For example, Qwen2-0.5B and Bloom-560M have a BertScore difference of nearly 30 points. Some models with
% fewer parameters even score higher. Although Qwen2-1.5B has fewer parameters than Phi2 and Gemma-1.1, it scores higher average BertScore. And Phi3-Mini even outperforms some 70B models on factual consistency. We attribute this to the differences in training data and training methods. Thus, both model size and the quality of training data are crucial for the text summarization capability.

% \subsection{Findings}
% Among all the factors influencing the summarization capabilities of SLMs, we find that model size and the training data quality are crucial for ensuring the relevance, coherence, and factual consistency of summaries. Prompt design and instruction tuning have a limited impact on most SLMs, which maintain consistent summary quality across different prompts. Additionally, instruction tuning only marginally improves the control of output text length.

\section{Conclusion}
\label{sec:conclusion}
This paper presents the comprehensive evaluation of SLMs for news summarization, comparing 19 models across diverse datasets. Our results demonstrate that top-performing models like Phi3-Mini and Llama3.2-3B-Ins can match the performance of larger 70B LLMs while producing shorter, more concise summaries. These findings highlight the potential of SLMs for real-world applications, particularly in resource-constrained environments. Further exploration shows that simple prompts are more effective for SLM summarization, while instruction tuning provides inconsistent benefits, necessitating further research.




% we comprehensively evaluate 17 SLMs on their text summarization performance. In terms of evaluation metrics, to improve the evaluation results more aligned with human preferences, we use high-quality reference summaries generated by LLMs to replace the original reference summaries in reference-based evaluation methods. In the evaluation experiments, we find that there is significant variability among SLMs in text summarization, with Phi3-Mini exhibiting the best performance among them. Compared with LLMs, while state-of-the-art SLMs still lag behind LLMs in summary relevance and coherence, the difference is minimal, with LLMs typically including more detailed content. There is no significant difference in factual consistency. Furthermore, we find that model size and training data remain the most critical factors influencing summary quality, with prompt design and instruction tuning having limited impact.


\section*{Limitations}

Although using LLM-generated summaries improved the reliability of reference-based summarization evaluation methods, it may introduce bias. For instance, if the LLM fails to accurately summarize the news, an SLM that produces similar content might receive an undeservedly high score. To mitigate this potential bias, we use summaries generated by multiple LLMs as references and calculate the average scores. Furthermore, although BertScore demonstrates a high level of consistency with human evaluations in coherence, it is not specifically designed to evaluate coherence. In future work, we plan to explore the use of more accurate metrics for coherence evaluation. Due to the input length limits of most SLMs, we only considered news articles with a maximum of 1,500 tokens. In the future, we will explore how to effectively use SLMs to summarize much longer news articles. Additionally, we do not consider quantized models in this study. In future work, we will also incorporate an analysis of the performance of quantized models into our evaluation.
% \subsection{Text Order}

% title issue (done)
% abstract (done)
% sec1 figure 1(done) 
% sec1 SLM intorduction (done)
% figure 2 (done)
% sec3.2 provide some detailed number (done)

\section{Introduction}


Automatic text summarization is a fundamental challenge in Natural Language Processing (NLP) that plays a crucial role in various applications, including search engine optimization~\cite{2019soe}, financial forecasting~\cite{fintech}, and public sentiment analysis~\cite{twitter_news}. The ability to distill large volumes of information into concise summaries is particularly important in today's fast-paced digital environment, where users often seek quick insights from news articles. Recent advancements in large language models (LLMs) such as GPT-3~\cite{NEURIPS2020_gpt3}, OPT~\cite{opt}, and Llama2~\cite{touvron2023llama2openfoundation} have significantly enhanced the quality and coherence of generated summaries compared to traditional models~\cite{luhn1958automatic, dong-etal-2018-banditsum, zhang-etal-2018-neural, t5, genest-lapalme-2012-fully}. However, the hundreds of billions of parameters in these LLMs require substantial computational and storage resources. For example, deploying the Llama2-70B model at FP16 precision typically necessitates two Nvidia A100 GPUs with 80GB of memory each. This means that only organizations with significant computing power can offer LLM services, raising concerns about service stability and data privacy.


\begin{figure}
    \centering
    \includegraphics[width=1\linewidth]{ARR_version/fig/model_parameter_score.pdf}
    \caption{Variation of SLMs in size and BertScore for news summarization. }
    \label{fig:intro}
\end{figure}




As a result, general-purpose small language models (SLMs) have emerged as a potential alternative, as shown in Figure~\ref{fig:intro}. They share the same decoder-only architecture as LLMs but have fewer than 4B parameters~\footnote{Currently, there is no clear definition of SLMs, but they generally have fewer than 4 billion parameters.}~\cite{azure2024phi,qwen,abdin2024phi3,llama3.2}. Same as the LLMs, they generate specified content based on prompts. SLMs are designed to run efficiently on edge devices like smartphones and personal computers, making them more accessible. This accessibility makes SLMs an attractive option for applications requiring quick, stable, and low-cost solutions that also protect user privacy, such as summarizing news articles for mobile users or enhancing data processing privacy in sensitive environments.

Despite the promising capabilities of SLMs, their specific performance in the domain of news summarization remains underexplored. Previous studies, such as Tiny Titans~\cite{tiny_titan}, have evaluated specifically fine-tuned SLMs in meeting summarization. The adopted ROUGE metric~\cite{lin-2004-rouge} only reflects the quality in terms of relevance. In addition, broader evaluations of SLMs across various tasks have been evaluated but omitted news summarization~\cite{lu2024smalllanguagemodelssurvey}, highlighting the gap in understanding SLM in this area.

This paper aims to address these gaps by conducting a comprehensive evaluation of pre-trained general-purpose SLMs in news summarization, a process that abstracts news articles into shorter versions while preserving key information. We seek to answer the following research questions:
% \begin{itemize}
%     \item How do different SLMs compare to each other, and how do they compare to LLMs?
%     \item What is the impact of prompt complexity on the performance of SLMs?
%     \item How does instruction tuning affect the summarization capabilities of different SLMs?
% \end{itemize}


First, \textbf{How do different SLMs compare to each other, and how do they compare to LLMs?} We benchmark 19 SLMs on 2,000 news samples\footnote{The codes and results are available at \url{https://github.com/Xtra-Computing/SLM_Summary_Benchmark}}. Our findings reveal that the news summarization capabilities of different SLMs vary significantly as shown in Figure~\ref{fig:intro}. And the best-performing SLMs can generate news summaries comparable in quality to those produced by LLMs while producing shorter summaries. Notably, Phi3-Mini and Llama3.2-3B-Ins emerge as the top performers.


Second, \textbf{What is the impact of prompt complexity on the performance of SLMs?} We compare the impact of prompts with varying levels of detail on the quality of the summaries. Our findings indicate that prompt engineering has a limited impact on enhancing summarization abilities, with even complex prompts potentially degrading model performance. Thus, simple prompts suffice in practical applications.

Third, \textbf{How does instruction tuning affect the summarization capabilities of different SLMs?} We find that the effects of instruction tuning on SLMs vary significantly. While the Llama3.2 series models show substantial improvements after instruction tuning, which aligns with existing works~\cite{zhang2024benchmarking}, models like Qwen2 and InternLM2 exhibit minimal changes.




% The rest of the paper is organized as follows: Section~\ref{sec:background} covers the background and related work, Section~\ref{sec:method} introduces our reference summary generation, Section~\ref{sec:setup} illustrates the benchmark design, Section~\ref{sec:benchmark} presents the benchmarking results of SLMs, Section~\ref{sec:factor} analyzes the influencing factors on SLM summarization, Section~\ref{sec:Verification} verifies our evaluation method, and Section~\ref{sec:conclusion} concludes with a summary of findings and future work.

The rest of the paper is organized as follows: Section~\ref{sec:background} covers the background and related work, Section~\ref{sec:method} introduces reference summary generation, Section~\ref{sec:setup} illustrates the benchmark design, Section~\ref{sec:benchmark} presents presents the benchmarking results of SLMs, Section~\ref{sec:factor} analyzes the influencing factors on SLM summarization, and Section~\ref{sec:conclusion} concludes our findings.
\section{Background and Related Work}
\label{sec:background}




\subsection{Small Language Model}


Small language models (SLMs) are characterized by their reduced parameter count and computational requirements compared to larger LLMs. These models can be general-purpose or specialized, and this paper focuses on the evaluation of pre-trained general-purpose SLMs, which will be referred to simply as SLMs throughout the remainder of the paper. Structurally, SLMs and LLMs share a common architecture, consisting of stacked decoder-only layers from the transformer framework~\cite{attention_all_you_need}, generating outputs in an autoregressive manner. SLMs typically have fewer decoder layers, attention heads, and smaller hidden dimensions, and they are trained on smaller, high-quality datasets~\cite{lu2024smalllanguagemodelssurvey}. While there is no clear definition for the parameter count that qualifies a model as an SLM, models that can operate on consumer-grade devices are generally considered to fall within this category. This paper focuses on models smaller than 8GB in FP16 precision, corresponding to fewer than 4B parameters.




% 文本总结评估是什么主要涉及哪些方面,常用的方法有哪些,以及现有的evaluation集中在传统方法和LLM。
\subsection{Text Summarization and Its Evaluation}

% Text summarization evaluation involves assessing the quality of automatically generated summaries from aspects such as relevance, coherence, and factual consistency, ensuring they capture the necessary information from the original text. Existing summarization evaluation methods can be broadly categorized into three types: human evaluation, reference-based evaluation, and LLM evaluation. 

\begin{figure}
    \centering
    \includegraphics[width=1\linewidth]{ARR_version/fig/evaluation_method_compare.pdf}
    \caption{Comparison of text summarization evaluation. }
    \label{fig:evaluation_compare}
\end{figure}


% \begin{figure}
%     \centering
%     \includegraphics[width=1\linewidth]{ARR_version/fig/workflow3.pdf}
%     \caption{SLM news summary evaluation workflow. }
%     \label{fig:workflow}
% \end{figure}

Text summarization is the process of compressing a large volume of text into a shorter version while retaining its main ideas and essential information. Traditional text summarization models usually take a single news article as input and output a summary. However, SLMs require both a prompt and an article to generate a summary. 

% The bottom of Figure~\ref{fig:workflow} illustrates the summary generation with SLMs.


The evaluation methods of text summarization can be categorized into three categories, as shown in Figure~\ref{fig:evaluation_compare}: human evaluation, reference-based evaluation, and LLM evaluation. 


\noindent \textbf{Human evaluation} is an intuitive and effective method, involving the hiring of annotators to score the generated text. For instance, \citet{huang-etal-2020-achieved} employed human evaluation to evaluate 10 representative summarization models, revealing that extractive summary methods often performed better in human evaluation. The Summeval benchmark~\cite{fabbri2021summeval} utilized human evaluation methods to evaluate 23 traditional models and found Pegasus~\cite{pegasus} performed best. \citet{dead_summarization} used human evaluation to assess LLMs and found that they were more favored by evaluators. \citet{zhang2024benchmarking} also used human evaluation to evaluate the LLMs, discovering that instruction tuning had a greater impact on performance than model size. INSTRUSUM~\cite{liu2023benchmarking} evaluated instruction controllable summarization of LLMs by annotators and found them still unsatisfactory in summarizing text based on complex instructions.


\noindent \textbf{Reference-based evaluation} is a commonly used method. It scores generated text by calculating its similarity to reference texts using various metrics, and many such metrics have been proposed~\cite{lin-2004-rouge,papineni-etal-2002-bleu,sellam-etal-2020-bleurt,bert-score,bartscore}. In addition to human evaluation, \citet{huang-etal-2020-achieved} and ~\citet{fabbri2021summeval} also used many reference-based metrics to evaluated models and found metrics like Rouge and BertScore are very close to human evaluations. \citet{google-2023-benchmarking} used the Rouge metric to measure text similarity. Tiny Titans~\cite{tiny_titan} assessed the performance of small models in meeting summarization and found FLAN-T5 performed best.


\noindent \textbf{LLM evaluation} is explored by recent studies~\cite{goyal2022news,kendeer,gao2023human-like,liu2023benchmarking,wang2023chatgpt}. This approach involves guiding LLMs with prompts and providing both the summary and the original article for scoring. However, \citet{not-yet} indicated that LLMs may exhibit biases when evaluating different models, warranting further research to ensure fairness.

In conclusion, existing evaluations primarily evaluate the news summarization capabilities of traditional models or LLMs, leaving a gap in the evaluation of SLMs.
% This paper aims to address this with reference-based evaluation methods.





% The evaluation of text summarization models has been a significant area of research, addressing various dimensions such as relevance, coherence, and factual consistency~\cite{huang-etal-2020-achieved,fabbri2021summeval,dead_summarization,zhang2024benchmarking,liu2023benchmarking}. For instance, \citet{huang-etal-2020-achieved} employed ROUGE and human evaluation to assess 10 representative summarization models, revealing that extractive summarizers often performed better in human assessments. The Summeval benchmark~\cite{fabbri2021summeval} analyzed existing metrics and utilized human evaluation methods to evaluate 23 models, providing a comprehensive overview of summarization performance. Additionally, \citet{dead_summarization} examined the performance of LLMs in text summarization, noting a preference for these models among evaluators, which has implications for the perceived effectiveness of LLMs over SLMs. In a study by \citet{zhang2024benchmarking}, human evaluation indicated that instruction tuning significantly influenced performance, more so than model size, suggesting that the methodology of training can be as crucial as the model architecture itself. The INSTRUSUM benchmark~\cite{liu2023benchmarking} focused on instruction-controllable summarization of LLMs, highlighting ongoing challenges in summarizing text based on complex instructions.

% Despite the promising capabilities of SLMs, comprehensive evaluations of their text summarization performance remain limited. The Tiny Titans study~\cite{tiny_titan} assessed SLMs in the context of meeting summarization but primarily focused on older models, excluding newer iterations such as Phi3 and Llama3. Furthermore, its evaluation relied solely on the ROUGE metric, which may not capture the full spectrum of summarization quality. While \citet{lu2024smalllanguagemodelssurvey} provided a broader evaluation of SLMs across various tasks, news summarization was notably absent. This paper aims to fill this gap by providing a thorough evaluation of SLM performance specifically in news summarization, introducing innovative methodologies that highlight the unique strengths of SLMs in this domain. 

% To justify our choice of SLMs, we selected models based on their performance in prior studies, their architectural innovations, and their applicability to resource-constrained environments. Our findings suggest that SLMs not only match but can sometimes exceed the performance of LLMs in specific summarization tasks, thereby offering a compelling alternative for practical applications. By addressing the limitations of previous studies, particularly the lack of focus on news summarization and the reliance on outdated models, this work contributes to a deeper understanding of SLM capabilities and their potential in real-world applications. Furthermore, we contextualize our findings within the broader landscape of summarization techniques, discussing the implications of using SLMs compared to LLMs and traditional summarization methods, particularly in scenarios where computational resources are limited. This discussion underscores the relevance of our research in advancing the field of NLP and highlights the critical need for further exploration of SLMs in diverse summarization contexts, thereby establishing a clear connection between the identified gaps in the literature and our research objectives.


% \section{Evaluation Method}
% In this section, we first analyze the shortcomings of existing news summarization evaluation methods. Subsequently, we illustrate the new news summarization evaluation method used in this paper.


%1. Why use system level corr?
%2. different human different score?
%3. why this method? compared with llm evaluation is still well
%加一个coherence的图
% \section{Text Summarization with SLMs}
% \section{Methodology}
% \label{sec:method}

% In this section, we outline the process of news summarization with general-purpose SLMs and introduce reference summary generation for evaluation.






% \subsection{News Summarization with SLMs}



\section{LLM-Augmented Reference-Based Evaluation}\label{sec:method}


% Reference-based evaluation methods are widely used for evaluating text summarization~\cite{papineni-etal-2002-bleu,fabbri2021summeval,google-2023-benchmarking,tiny_titan}. These methods typically rely on statistical metrics (e.g., ROUGE~\cite{lin-2004-rouge}, BLEU~\cite{papineni-etal-2002-bleu}) or language models (e.g., BertScore~\cite{bert-score}, BLEURT~\cite{sellam-etal-2020-bleurt}) to measure the similarity between generated and reference summaries, with higher similarity indicating better quality. However, existing research~\cite{how_well,fabbri2021summeval} indicates that the poor quality of reference summaries in current datasets significantly weakens the correlation between these evaluation methods and human preferences. Moreover, \citet{zhang2024benchmarking} found that using summaries generated by freelance writers as references can improve alignment with human preferences. Numerous studies have shown that LLM-generated summaries are comparable in quality to those written by freelance writers and are often preferred for their style~\cite{goyal2022news,dead_summarization}. Therefore, we use LLM-generated summaries as high-quality reference summaries for the evaluation of SLMs. Figure~\ref{fig:workflow} illustrates how LLM-generated summaries are used as references to evaluate SLMs. The LLM generates summaries autoregressively, while the SLM-generated summaries are scored based on selected metrics.


% LLM-generated reference summaries not only match the quality of those created by freelance writers but also offer two distinct advantages. First, LLMs produce summaries faster than freelance writers. We can use LLMs to collect hundreds of summaries within a few hours, which is beneficial for large-scale news evaluations. In contrast, a freelance writer typically takes about 15 minutes to complete a single summary~\cite{zhang2024benchmarking}. Second, the financial cost of generating reference summaries with LLMs is lower. Hiring a freelance writer to summarize a news article costs around four dollars~\cite{zhang2024benchmarking}. In Section~\ref{sec:Verification}, we verify that LLM-generated reference summaries significantly improve reference-based evaluation.


As the reference-based method offers higher efficiency, better reproducibility, and lower cost compared to human and LLM evaluation~\cite{fabbri2021summeval,zhang2024benchmarking}, we use the reference-based method for large-scale SLM evaluation. However, some existing research~\cite{how_well,fabbri2021summeval} indicates that the poor quality of heuristic reference summaries in current datasets weakens their correlation with human preferences. Moreover, \citet{zhang2024benchmarking} find that using high-quality summaries as references can significantly improve alignment with human preferences. Thus, we use LLM-generated summaries as references in our SLM evaluation instead of relying on original dataset references. 
% Figure~\ref{fig:workflow} illustrates how LLM-generated summaries serve as references for evaluating SLMs.

LLM-generated reference summaries offer two key advantages.
First, they have higher quality than original heuristic summaries; \citet{zhang2024benchmarking} find LLMs averaged 4.5 on a five-point scale, while original references scored only 3.6. Additionally, \citet{dead_summarization} find that LLM-generated summaries are preferred to those written by freelance authors by up to 84\% of the time. 
Second, LLMs produce high-quality summaries quickly, generating hundreds in a few hours, which benefits large-scale news evaluations. In contrast, a freelance writer typically takes about 15 minutes for a single summary~\cite{zhang2024benchmarking}. 
% \end{itemize}



% Section~\ref{sec:Verification} further verifies that LLM-generated references enhance reference-based evaluation.

% \section{LLM-Generated Reference Verification}
% \label{sec:Verification}
We also verify whether the LLM-generated reference summaries can improve the effectiveness of reference-based evaluation on three human evaluation datasets~\cite{zhang2024benchmarking,fabbri2021summeval} in Table~\ref{tab:compare_kt_corr}. These datasets include summaries generated by various models as well as human ratings. We compare the correlation of three popular similarity metrics with human scoring using different references. BertScore and BLEURT calculate similarity based on fine-tuned language models, while RougeL measures similarity through textual overlap. We use prompt 2 in Figure~\ref{fig:prompt} as the input of the LLM. For all datasets, we consistently use two sentences to summarize the article. LLM-generated summaries significantly improve consistency than original references. BertScore performs best in both the relevance and coherence aspects.

\begin{table*}[]
% \caption{System-level Kendall's tau correlation coefficients with different reference summary settings.}
\begin{center}
% \begin{adjustbox}{max width=2.05\columnwidth}
\resizebox{2\columnwidth}{!}{
\begin{tabular}{cccccccccc}
\hline
\multirow{2}{*}{Metric} &
  \multirow{2}{*}{\begin{tabular}[c]{@{}c@{}}Reference\\ type\end{tabular}} &
  \multicolumn{4}{c}{Relevance evaluation} &
  \multicolumn{4}{c}{Coherence evaluation} \\
 &
   &
  \begin{tabular}[c]{@{}c@{}}Bench\\ -CNN/DM\end{tabular} &
  \begin{tabular}[c]{@{}c@{}}Bench\\ -XSum\end{tabular} &
  SummEval &
  Average &
  \begin{tabular}[c]{@{}c@{}}Bench\\ -CNN/DM\end{tabular} &
  \begin{tabular}[c]{@{}c@{}}Bench\\ -XSum\end{tabular} &
  SummEval &
  Average \\ \hline
\multirow{3}{*}{RougeL}   & Original    & 0.6732 & 0.2647 & 0.3714 & 0.4364          & 0.4248 & 0.6765 & 0.1238 & 0.4084          \\
                           & Qwen1.5-72B & 0.7778 & 0.7059 & 0.4476 & \textbf{0.6438} & 0.5556 & 0.5000 & 0.2381 & \textbf{0.4312} \\
                           & llama2-70B  & 0.7778 & 0.6324 & 0.4095 & 0.6066          & 0.5294 & 0.3676 & 0.2000 & 0.3657          \\ \hline
\multirow{3}{*}{BertScore} & Original    & 0.6209 & 0.2206 & 0.4476 & 0.4297          & 0.4771 & 0.6912 & 0.4286 & 0.5323          \\
                           & Qwen1.5-72B & 0.7516 & 0.6324 & 0.8095 & 0.7312          & 0.5817 & 0.6029 & 0.6000 & \textbf{0.5949} \\
                           & llama2-70B  & 0.7516 & 0.6618 & 0.8286 & \textbf{0.7413} & 0.5294 & 0.6029 & 0.6190 & 0.5838          \\ \hline
\multirow{3}{*}{BLEURT}    & Original    & 0.5425 & 0.2647 & 0.181  & 0.3294          & 0.3203 & 0.6176 & 0.2381 & 0.3929          \\
                           & Qwen1.5-72B & 0.5817 & 0.7206 & 0.4476 & 0.5833          & 0.4902 & 0.4853 & 0.5429 & 0.5061          \\
                           & llama2-70B  & 0.5556 & 0.7059 & 0.5429 & \textbf{0.6015} & 0.5425 & 0.4706 & 0.6762 & \textbf{0.5631} \\ \hline
\end{tabular}}
% \end{adjustbox}
\end{center}
\caption{{System-level Kendall's tau correlation coefficients between reference-based and human evaluations under different reference summary settings. "Original" is the original heuristic reference from the dataset; others are generated by LLMs. References generated by LLMs improve the effectiveness of reference-based evaluations.}}
\label{tab:compare_kt_corr}
\end{table*}






\begin{figure}
    \centering
    \includegraphics[width=1\linewidth]{ARR_version/fig/prompt2.pdf}
    \caption{ The prompt templates for the language model to generate summaries. }
    \label{fig:prompt}
\end{figure}




%实验部分介绍数据集,如何生成的数据集,为什么选择500个
%这种测评方法的具体实现,数据清洗?prompt设计
%验证这种方法的结果和llm做测评是相媲美的。
%选择了那些tiny LLM进行测评,给出测评结果
%llm测评和我们测评的相关性。
%tiny llm的能力和34b,70b相比如何?
%每B的得分值
%与人类写手得分相比
%some example 给出具体的例子
%平均长度
%多少分算好?60分以下,65分以下,70分以上
%小模型上有什么问题?
%参数从多少到多少提升最大?模型越大性能越强,但到了一定程度提升不明显了
%instruct对性能的影响
%提出的方法的robust,不同prompt,不同LLM
%加入summary level?
%gpt-4o的测评?不用啊,本身就是测的和大模型的相似度
%data相关?为什么更好?因为训练数据方面么?
%template的影响?
%影响summarization能力的因素有哪些
%输出太长
%给出例子说明bertscore的有效性,相比原始reference更好
%instruct 对长度的影响
%小模型生成总结长度更短
%解释指标的含义,为什么要衡量总结的长短
%稳定性
%常见存储16GB?
%原始新闻的质量?给出例子
%为什么采用这种比较方法,因为不受reference长度影响结果
% 所有figure to fig.


\section{Benchmark Design}\label{sec:setup}


In this section, we first introduce the dataset for news summarization evaluation, followed by the SLM details, evaluation metrics, and reference summary generation.

% \subsection{Experimental Setup}
\subsection{News Article Selection }

All experiments are conducted on the following four datasets: CNN/DM~\cite{cnndm}, XSum~\cite{xsum}, Newsroom~\cite{newsroom}, and BBC2024. The first three datasets have been widely used for verifying model text summarization~\cite{goyal2022news,google-2023-benchmarking,zhang2024benchmarking}. However, their older data may overlap with the training datasets of some models. To address this, we create BBC2024, featuring news articles from January to March 2024 on the BBC website~\footnote{\url{http://dracos.co.uk/made/bbc-news-archive/}}. Based on prior research, 500 samples per dataset are sufficient to differentiate the models~\cite{google-2023-benchmarking}. We randomly select 500 samples from each of the four test sets and construct 2000 samples in total. Each sample contains 500 to 1500 tokens according to the Qwen1.5-72B-Chat tokenizer.



\subsection{Model Selection}
We select 21 popular models with parameter sizes not exceeding 4 billion for benchmarking, including 19 SLMs and 2 models specifically designed for text summarization, Pegasus-Large and Brio, which have been fine-tuned on text summarization datasets, detailed in Table~\ref{tab:model_list}.

% \textbf{LiteLlama~\cite{huggingface2024litelama}:} LiteLlama is an open-source reproduction of Llama2~\cite{touvron2023llama2openfoundation}. It has 460M parameters and is trained on 1T tokens from the RedPajama dataset~\cite{together2023redpajama}.

% \textbf{Bloom-560M~\cite{bloom}:} Bloom-560M is part of the BLOOM (BigScience Large Open-science Open-access Multilingual) family of models developed by the BigScience project. It has 560M parameters and is trained on 1.5T pre-processed text.

% \textbf{TinyLlama~\cite{zhang2024tinyllama}:} TinyLlama is an SLM which has the same architecture and tokenizer as Llama 2. It has 1.1B parameters and is trained on 3T tokens. We use the chat finetuned version.

% \textbf{GPT-Neo series~\cite{gpt-neo}:} GPT-Neo Series are open-source language models developed by EleutherAI to reproduce GPT-3 architecture. It has two versions with 1.3B and 2.7B parameters. They are trained on 380B and 420B tokens, respectively, on the Pile dataset~\cite{gao2020pile,biderman2022datasheet_pile}.

% \textbf{Qwen2 series~\cite{qwen}:} Qwen2 comprises a series of language models pre-trained on multilingual datasets. And some models use instruction tuning to align with human preferences. We select models with no more than 7B parameters for benchmarking, including 0.5B, 0.5B-Ins, 1.5B, 1.5B-Ins, 7B, and 7B-Ins. Models with the "Ins" suffix indicate the instruction tuning version.

% \textbf{InterLM2 series~\cite{cai2024internlm2}:} The InternLM2 series includes models with various parameter sizes. They all support long contexts of up to 200,000 characters. We select the 1.8B version and the 1.8B-Chat version for evaluation.

% \textbf{Gemma-1.1~\cite{team2024gemma}:} Gemma-1.1 is developed by Google and is trained using a novel RLHF method. We select the 2B instruction tuning version for benchmarking.

% \textbf{MiniCPM~\cite{hu2024minicpm}:} MiniCPM is an end-size language model with 2.7B parameters. It employs various post-training methods to align with human preferences. We select the supervised tuning (SFT) version for benchmarking.

% \textbf{Phi series~\cite{microsoft2023phi2,abdin2024phi3}:} Phi series are some SLMs developed by Microsoft. They are trained on high-quality datasets. We select Phi-2(2.7B parameters) and Phi-3-Mini(3.8B parameters, 4K context length) for evaluation. 

% \textbf{Brio~\cite{brio}: } Brio is one of the state-of-the-art specialized language models designed for text summarization, which leverages reinforcement learning to optimize the selection of sentences, ensuring high-quality, informative, and coherent summaries. It only has 406M parameters, and we select the version trained on the CNN/DM dataset.

% \textbf{Pegasus-Large~\cite{pegasus}:} Pegasus-Large also is a specialized language model designed for abstractive text summarization, utilizing gap-sentence generation pre-training to achieve exceptional performance on summarization tasks. It has 568M parameters.


%model details

% Meanwhile, to compare the performance of tiny LLMs with larger LLMs, we also consider the Llama3-70B. 

All models are sourced from the Hugging Face\footnote{\url{https://huggingface.co}} and tested based on the lm-evaluation-harness tool~\cite{eval-harness}. For general-purpose SLMs, we use the prompt from the huggingface hallucination leaderboard\footnote{\url{https://huggingface.co/spaces/hallucinations-leaderboard/leaderboard}} (Prompt 1 in Figure~\ref{fig:prompt}). To ensure reproducibility, we employ the greedy decoding strategy, generating until a newline character or <EOS> token is reached. Outputs are post-processed to remove prompt words and incomplete sentences. Given the 2048-token limit of some models, we only evaluate the zero-shot approach.

\begin{table}[]
\centering
\begin{adjustbox}{max width=1\columnwidth}
% \resizebox{\columnwidth}{!}{
\begin{tabular}{lccc}
\hline
Model          & Parameters & \begin{tabular}[c]{@{}c@{}}Instruction \\ Tuning\end{tabular} & Reference                                                              \\ \hline
Brio $^{\ast}$          & 406M         & $\checkmark$                                                  & \cite{brio} \\
LiteLlama      & 460M         & $\times$                                                      & \cite{huggingface2024litelama}       \\
Qwen2-0.5B     & 500M         & $\times$                                                      & \cite{qwen}              \\
Qwen2-0.5B-Ins & 500M         & $\checkmark$                                                  & \cite{qwen}     \\
Bloom-560M          & 560M         & $\times$                                                      & \cite{bloom}        \\
Pegasus-Large$^{\ast}$  & 568M         & $\checkmark$                                                  & \cite{pegasus}         \\
TinyLlama      & 1.1B         & $\checkmark$                                                  & \cite{zhang2024tinyllama} \\
Llama3.2-1B    & 1.2B         & $\times$                                                      & \cite{llama3.2} \\
Llama3.2-1B-Ins& 1.2B         & $\checkmark$                                                  & \cite{llama3.2} \\
GPT-Neo-1.3B   & 1.3B         & $\times$                                                      & \cite{gpt-neo}      \\
Qwen2-1.5B     & 1.5B         & $\times$                                                      & \cite{qwen}              \\
Qwen2-1.5B-Ins & 1.5B         & $\checkmark$                                                  & \cite{qwen}     \\
InternLM2-1.8B & 1.8B         & $\times$                                                      & \cite{cai2024internlm2}     \\
InternLM2-1.8B-Chat & 1.8B    & $\checkmark$                                                  & \cite{cai2024internlm2}     \\
MiniCPM        & 2.4B         & $\checkmark$                                                  & \cite{hu2024minicpm}  \\
Gemma-1.1      & 2.5B         & $\checkmark$                                                  & \cite{team2024gemma}       \\
GPT-Neo-2.7B   & 2.7B         & $\times$                                                      & \cite{gpt-neo}       \\
Phi-2          & 2.7B         & $\times$                                                      & \cite{microsoft2023phi2}             \\
Llama3.2-3B    & 3.2B         & $\times$                                                      & \cite{llama3.2} \\
Llama3.2-3B-Ins& 3.2B         & $\checkmark$                                                  & \cite{llama3.2} \\
Phi-3-Mini     & 3.8B         & $\checkmark$                                                  & \cite{abdin2024phi3}   \\ \hline
\end{tabular}
\end{adjustbox}
\caption{\upshape{Models for news summarization benchmark. Models marked with $^{\ast}$ have been fine-tuned on text summarization datasets.} }
\label{tab:model_list}
\end{table}


% \begin{figure}
%     \centering
%     \includegraphics[width=1\linewidth]{latex/fig/benchmark_prompt.pdf}
%     \caption{The benchmark prompt template for news summary generation. }
%     \label{fig:benchmark_prompt}
% \end{figure}


\subsection{Evaluation Metric }

We evaluate the summary quality in relevance, coherence, factual consistency, and text compression using BertScore~\cite{bert-score}, HHEM-2.1-Open~\cite{HHEM-2.1-Open}, and summary length.

% As mentioned earlier, BertScore aligns well with human preferences for relevance and coherence when using high-quality references. It measures how well the summary captures key information (relevance) and maintains logical flow (coherence). 

BertScore is a robust semantic similarity metric designed to assess the quality of the generated text and achieves the best correlation in Table~\ref{tab:compare_kt_corr}. It compares the embeddings of candidate and reference sentences through contextualized word representations from BERT~\cite{devlin-etal-2019-bert}. It can measure how well the summary captures key information (relevance) and maintains logical flow (coherence). We use the F1 score of BertScore\footnote{\url{https://huggingface.co/microsoft/deberta-xlarge-mnli}}, ranging from 0 to 100, where higher values indicate better quality.

HHEM-2.1-Open is a hallucination detection model which outperforms GPT-3.5-Turbo and even GPT-4. It can evaluate whether news summaries are factually consistent with the original article. It is based on a fine-tuned T5 model to flag hallucinations with a score. When the score falls below 0.5, it indicates a summary is inconsistent with the source. We report the percentage of summaries deemed factually consistent; higher values indicate better performance.

Since BertScore is insensitive to summary length, we also track the average summary length to assess text compression. With similar BertScore results, shorter summaries indicate better text compression ability.


\subsection{Reference Summary Generation}
To mitigate the impact of occasional low-quality reference summaries on evaluation results, as well as the tendency for models within the same series to score higher due to possible similar output content, we utilize two LLMs, Qwen1.5-72B-Chat and Llama2-70B-Chat, to generate two sets of reference summaries. We then average the scores during the SLM evaluation. We use Prompt 2 from Figure~\ref{fig:prompt} as the prompt template and apply a greedy strategy to generate summaries based on the vLLM inference framework~\cite{vllm}.



% Figure 5 shows the average length of the generated summaries for each dataset.






% \subsubsection{Parameter Setup}

% \subsection{Evaluation Results}

% \begin{table*}[]
% \caption{}
% \label{tab:my-table}
% \begin{adjustbox}{max width=2.05\columnwidth}
% \begin{tabular}{lcccccccccccc}
% \hline
% Model                & \multicolumn{3}{c}{XSum}   & \multicolumn{3}{c}{Newsroom} & \multicolumn{3}{c}{CNN/DM} & \multicolumn{3}{c}{BBC2024} \\
% \multicolumn{1}{c}{} & Qwen1.5 & Llama2 & Average & Qwen1.5  & Llama2  & Average & Qwen1.5 & Llama2 & Average & Qwen1.5  & Llama2 & Average \\ \hline
% LiteLama      & 38.68 & 39.24 & 38.96 & 38.02 & 38.70 & 38.36 & 40.58 & 41.38 & 40.98 & 38.05 & 38.86 & 38.45 \\
% Qwen2-0.5     & 69.57 & 70.71 & 70.14 & 65.94 & 66.92 & 66.43 & 69.50 & 70.48 & 69.99 & 68.23 & 69.67 & 68.95 \\
% Qwen2-0.5-Ins & 61.67 & 62.35 & 62.01 & 60.17 & 60.91 & 60.54 & 63.34 & 63.98 & 63.66 & 60.72 & 61.47 & 61.09 \\
% TinyLlama     & 68.07 & 69.16 & 68.61 & 64.70 & 65.63 & 65.16 & 68.66 & 69.23 & 68.95 & 66.93 & 68.07 & 67.50 \\
% GPT-Neo       & 59.67 & 59.55 & 59.61 & 52.54 & 53.11 & 52.83 & 58.37 & 58.81 & 58.59 & 58.16 & 58.18 & 58.17 \\
% Qwen2-1.5     & 72.07 & 72.65 & 72.36 & 69.12 & 69.86 & 69.49 & 71.30 & 72.05 & 71.67 & 71.08 & 71.99 & 71.53 \\
% Qwen2-1.5-Ins & 71.52 & 72.36 & 71.94 & 69.43 & 69.56 & 69.50 & 71.72 & 72.39 & 72.06 & 70.83 & 71.78 & 71.30 \\
% InternLM2-1.8 & 68.64 & 69.20 & 68.92 & 64.27 & 65.00 & 64.64 & 68.81 & 69.49 & 69.15 & 67.88 & 68.74 & 68.31 \\
% InternLM2-1.8-chat   & 67.77   & 68.82  & 68.30   & 64.99    & 65.57   & 65.28   & 68.78   & 69.38  & 69.08   & 68.32    & 69.62  & 68.97   \\
% Gemma-1.1     & 70.80 & 72.27 & 71.53 & 69.42 & 70.30 & 69.86 & 71.88 & 72.83 & 72.35 & 70.60 & 71.67 & 71.13 \\
% MiniCPM       & 65.59 & 65.77 & 65.68 & 62.48 & 63.13 & 62.80 & 66.82 & 67.51 & 67.17 & 65.54 & 65.96 & 65.75 \\
% Phi-2         & 70.95 & 72.03 & 71.49 & 68.15 & 69.14 & 68.64 & 71.19 & 72.21 & 71.70 & 70.38 & 71.50 & 70.94 \\
% Phi-3-mini    & 74.31 & 74.65 & 74.48 & 72.07 & 72.64 & 72.36 & 74.54 & 75.24 & 74.89 & 73.63 & 74.05 & 73.84 \\
% ChatGLM3      & 73.51 & 74.08 & 73.79 & 70.77 & 71.11 & 70.94 & 73.66 & 74.27 & 73.97 & 72.57 & 73.41 & 72.99 \\
% Mistral       & 51.73 & 51.98 &       & 57.13 & 57.45 &       & 51.19 & 52.10 &       & 59.85 & 59.47 &       \\
% Mistral-Ins   & 74.51 & 75.33 & 74.92 & 71.18 & 71.81 & 71.49 & 74.29 & 74.81 & 74.55 & 73.90 & 74.32 & 74.11 \\
% Qwen2-7B      & 75.49 & 75.73 & 75.61 & 73.65 & 73.73 & 73.69 & 74.94 & 75.90 & 75.42 & 74.89 & 75.27 & 75.08 \\
% Qwen2-7B-Ins  & 75.04 & 75.50 & 75.27 & 73.12 & 73.01 & 73.07 & 75.14 & 74.28 & 74.71 & 74.24 & 74.71 & 74.48 \\
% Llama3-instruct      & 77.39   & 76.77  & 77.08   & 75.47    & 74.67   & 75.07   & 77.63   & 76.95  & 77.29   & 76.79    & 76.06  & 76.42   \\ \hline
% \end{tabular}
% \end{adjustbox}
% \end{table*}



\section{Evaluation Results }
\label{sec:benchmark}
% In this section, we first evaluate the news summarization. Then we compare them with larger LLMs and give model recommendations with different model sizes.



%根据分数划分结果

\subsection{Relevance and Coherence Evaluation}


\begin{table*}[]
\begin{center}
\resizebox{2\columnwidth}{!}{
% \begin{adjustbox}{max width=2.05\columnwidth}
\begin{tabular}{llccccccccr}
\hline
Score Range                  & Model         & \multicolumn{2}{c}{XSum} & \multicolumn{2}{c}{Newsroom} & \multicolumn{2}{c}{CNN/DM} & \multicolumn{2}{c}{BBC2024} & Average \\
                    &                    & Qwen1.5 & Llama2 & Qwen1.5 & Llama2 & Qwen1.5 & Llama2 & Qwen1.5 & Llama2 & \multicolumn{1}{l}{} \\ \hline
\multirow{4}{*}{$< 60$} & LiteLlama           & 38.68   & 39.24  & 38.02   & 38.70  & 40.58   & 41.38  & 38.05   & 38.86  & 39.19                \\
                    & Bloom-560M              & 48.23   & 48.43  & 42.73   & 43.28  & 49.70   & 50.04  & 47.56   & 48.01  & 47.25                \\
                    & GPT-Neo-1.3B            & 59.67   & 59.55  & 52.54   & 53.11  & 58.37   & 58.81  & 58.16   & 58.18  & 57.30                \\ 
                    & GPT-Neo-2.7B            & 60.11   & 60.15  & 53.65   & 54.14  & 59.15   & 59.97  & 59.29   & 59.50  & 58.25                \\ \hline
\multirow{10}{*}{$60 \sim 70$}
                    & Llama3.2-1B       &62.44    &62.33   &55.25    &55.86   &60.32    &60.67  &62.17&62.33      &60.17    \\
                    & Pegasus-Large      &61.39    &61.44   &61.26    &61.64   &61.94    &62.26  &63.12&63.47      &62.07    \\
                    & Llama3.2-3B       &66.14    &65.84   &59.93    &60.68   &64.19    &64.51  &66.62&66.57      &64.31    \\
                    & MiniCPM            & 65.59   & 65.77  & 62.48   & 63.13  & 66.82   & 67.51  & 65.54   & 65.96  & 65.35                \\
                    & TinyLlama          & 68.07   & 69.16  & 64.70   & 65.63  & 68.66   & 69.23  & 66.93   & 68.07  & 67.56                \\
                    & InternLM2-1.8B      & 68.64   & 69.20  & 64.27   & 65.00  & 68.81   & 69.49  & 67.88   & 68.74  & 67.75                \\
                    & InternLM2-1.8B-Chat & 67.77   & 68.82  & 64.99   & 65.57  & 68.78   & 69.38  & 68.32   & 69.62  & 67.91                \\
                    & Qwen2-0.5B-Ins & 69.30       & 70.01      & 65.85   & 66.65   &69.31   & 70.06 & 67.54 & 68.64        & 68.42   \\
                    & Qwen2-0.5B          & 69.57   & 70.71  & 65.94   & 66.92  & 69.50   & 70.48  & 68.23   & 69.67  & 68.88                \\
                    &Brio   & 70.67  & 70.71  &68.43  & 68.41   &69.86  &70.34  &70.39  & 70.21  & 69.88  \\ \hline
\multirow{7}{*}{$> 70$} & Phi-2              & 70.95   & 72.03  & 68.15   & 69.14  & 71.19   & 72.21  & 70.38   & 71.50  & 70.69                \\
                    & Gemma-1.1          & 70.80   & 72.27  & 69.42   & 70.30  & 71.88   & 72.83  & 70.60   & 71.67  & 71.22                \\
                    & Qwen2-1.5B          & 72.03   & 72.90  & 69.19   & 69.91  & 71.33   & 72.42  & 71.08   & 72.18  & 71.38                \\
                    & Qwen2-1.5B-Ins      & 72.21   & 73.06  & 69.49   & 69.86  & 71.64   & 71.95  & 71.01   & 71.97  & 71.40                \\
                    & Llama3.2-1B-Ins     & 72.24   & 73.27  & 70.54   & 70.98  & 72.43   & 73.61  & 71.68   & 72.55  & 72.16                \\
                    & Phi-3-Mini         & \textbf{74.67}   & {75.14}  & {72.42}   & {72.32}  & \textbf{74.86}   & {74.88}  & {74.08}   & {73.72}  & {74.01} \\  
                    & Llama3.2-3B-Ins     & 74.40   & \textbf{75.33}  & \textbf{72.81}   & \textbf{73.41}  & 74.54   & \textbf{75.29}  & \textbf{74.51}   & \textbf{74.95}  & \textbf{74.41}   \\ \hline
                  
\end{tabular}}
\end{center}
% \end{adjustbox}
\caption{\upshape {BertScore of SLMs on four text summarization datasets. Qwen1.5-72B-Chat and Llama2-70B-Chat are used to generate reference summaries.}}
\label{tab:bertscore_res}
\end{table*}


\begin{figure}
    \centering
    \includegraphics[width=1\linewidth]{ARR_version/fig/bertscore_example5.pdf}
    \caption{ Example summaries from SLMs and LLMs. The bold part is the same as the reference, the underline indicates irrelevant content, and the red indicates incorrect content. Summaries with BertScore above 70, such as those from Llama3.2-3B-Ins, demonstrate similar quality to LLMs. }
    \label{fig:bertscore_example}
\end{figure}

Table~\ref{tab:bertscore_res} shows the average BertScore results of SLMs. All models demonstrate consistent performance across datasets. Due to the significant score differences, we categorize the models into three approximate ranges, using two specialized summarization models, Pegasus-Large and Brio, as reference points.


The first range includes scores below 60. \textbf{Models scoring below 60 struggle to effectively summarize articles.} LiteLlama, Bloom, and GPT-Neo series fall into this range. In terms of relevance, these models often miss key points; in terms of coherence, these models sometimes produce repetitive outputs, leading to overly long summaries. Figure~\ref{fig:bertscore_example} shows an example from LiteLlama, it scores only 40.81, deviating significantly from the news.


The second range covers scores between 60 and 70. \textbf{Models in this range produce useful summaries but occasionally lack key points.} Models like TinyLlama and Qwen-0.5B series fall into this range. In relevance, these models generally relate to the original content but may omit crucial details. In coherence, the sentences can be well-organized. Fig~\ref{fig:bertscore_example} shows examples from Qwen2-0.5B. Although its summary is relevant and coherent to the news article, it omits the final score.


The third range includes scores above 70. \textbf{Models in this range produce summaries comparable to LLMs, with occasional inconsistencies.} They effectively capture key points and maintain coherent structure. Phi3-Mini and Llama3.2-3B-Ins stand out in this group. As shown in Figure~\ref{fig:bertscore_example}, Llama3.2-3B-Ins generates a concise, accurate summary, correctly capturing the match outcome even though the source news lacks a direct score description.







\subsection{Factual Consistency Evaluation}


% \begin{table}[]
% \caption{\upshape{Factual consistency rate of SLMs on news summarization datasets.}}
% \label{tab:fact}
% \begin{adjustbox}{max width=1\columnwidth}
% \begin{tabular}{lccccc}
% \hline
% Model          & XSum  & Newsroom & CNN/DM & BBC2024 & Average \\ \hline
% Qwen2-1.5B     & 95.39 & 92.86    & 97.65  & 93.27   & 94.79   \\
% Qwen2-1.5B-Ins & 94.77 & 95.28    & 98.51  & 94.57   & 95.78   \\
% Gemma1.1       & 95.35 & 96.46    & 98.60  & \textbf{94.68}   & \textbf{96.27}   \\
% Phi2           & 93.00 & 95.46    & 97.41  & 93.43   & 94.82   \\
% Phi3-Mini      & \textbf{96.29} & \textbf{97.09}    & \textbf{99.14}  & 92.27   & 96.20   \\ \hline
% \end{tabular}
% \end{adjustbox}
% \end{table}




During summary generation, SLMs may produce false information due to hallucinations. Therefore, we report the factual consistency rate in Table~\ref{tab:fact}. Models with high BertScore generally have better factual consistency as shown in Figre~\ref{fig:bertscore_example}. SLMs that perform well in fact consistency include Phi3-Mini (96.7\%) and Qwen2-1.5B-Ins (96.5\%), maintaining over 95\% consistency across all datasets. The traditional model, Pegasus-large, outperforms SLMs in fact consistency, which achieves an exceptional 99.9\% average. This is because it often copies sentences from the original text, ensuring alignment but sometimes missing key information. 


% This table also includes LLM results, showing minimal gaps between LLMs and top SLMs, with Phi3-Mini and Qwen2-1.5B even outperforming Llama3-70B-Ins on multiple datasets.


% fig3中加上fact分数





\subsection{Summary Length Evaluation}



\begin{figure}
    \centering
    \includegraphics[width=1\linewidth]{ARR_version/fig/hist_summary_length.pdf}
    \caption{Average summary length comparison. SLMs with high BertScore generate 50-70 word summaries.}
    \label{fig:summary_length}
\end{figure}




% Average summary length changes with BertScore. The summary length first decreases, then increases, and finally remains stable. 
Figure~\ref{fig:summary_length} shows the average summary lengths on 4 datasets produced by SLMs, which vary significantly across models. For scores below 60, summaries typically exceed 100 words due to redundant content and poor summarization. In the 60–70 range, length varies widely; e.g., the Llama3.2 series includes too many unnecessary details, while TinyLlama omits key information, leading to shorter summaries. Models with scores above 70 generate more consistent summaries, averaging around 50 to 70 words in length. For the top-performing models, Phi3-Mini and Llama3.2 series, Phi3-Mini generates more concise summaries while maintaining similar relevance.




\begin{table}[]
\begin{adjustbox}{max width=1\columnwidth}
\begin{tabular}{lccccc}
\hline
Model              & XSUM &Newsroom & CNN/DM & BBC2024 & Average \\ \hline
LiteLama           & 60.6 & 63.4    & 68.8   & 63.6    & 64.1    \\
Bloom-560M         & 83.6 & 77.4                          & 87.0   & 83.4    & 82.9    \\
GPT-Neo-1.3B       & 91.2 & 76.8                          & 92.4   & 92.0    & 88.1    \\
GPT-Neo-2.7B       & 83.2 & 67.0                          & 87.6   & 84.0    & 80.5    \\
Llama3.2-1B        & 97.6 & 85.8                         & 97.6  & 95.8      & 94.2    \\
Pegasus-Large      & 99.6 & 100.0                         & 100.0  & 100.0   & \textbf{99.9}    \\
Llama3.2-3B        & 98.8 & 85.6                         & 98.8    & 97.4    & 95.2    \\
MiniCPM            & 85.8 & 72.8                          & 93.6   & 89.6    & 85.5    \\
TinyLlama          & 95.4 & 92.4                          & 98.4   & 98.0    & 96.1    \\
InternLM2-1.8B     & 91.8 & 87.4                          & 96.0   & 94.8    & 92.5    \\
InternLM2-1.8B-Chat & 88.6 & 84.6                          & 93.8   & 90.8    & 89.5    \\
Qwen2-0.5B-Ins     & 87.2 & 86.0                          & 93.2   & 89.6    & 89.0    \\
Qwen2-0.5B         & 93.0 & 92.8                          & 96.8   & 95.2    & 94.5    \\
Brio               & 94.8 & 89.0                          & 97.8   & 95.4    & 94.3    \\
Phi2               & 95.2 & 93.4                          & 97.8   & 96.8    & 95.8    \\
Gemma-1.1          & 96.6 & 93.6                          & 97.8   & 94.6    & 95.7    \\
Qwen2-1.5B         & 95.6 & 96.2                          & 97.4   & 96.6    & 96.5    \\
Qwen2-1.5B-Ins     & 91.8 & 91.2                          & 94.4   & 92.4    & 92.5    \\
Llama3.2-1B-Ins    & 92.2 & 88.0                          & 95.4   & 94.0    & 92.4    \\
Phi3-Mini          & 96.0 & 94.8                          & 98.2   & 97.6    & \textbf{96.7}    \\ 
Llama3.2-3B-Ins    & 93.2 & 93.2                          & 96.6   & 95.6    & 94.7    \\ \hline
% ChatGLM3           & 96.0 & 95.2                          & 99.2   & 96.0    & 96.6    \\
% Mistral-7B-Ins     & 92.6 & 88.0                          & 96.8   & 93.0    & 92.6    \\
% Qwen2-7B-Ins       & 94.0 & 95.0                          & 98.6   & 95.6    & 95.8    \\
% Llama3-70B-Ins     & 94.6 & 94.8                          & 97.6   & 95.8    & 95.7    \\
% Qwen2-72B-Ins      & 97.0 & 97.6                          & 98.8   & 98.4    & 98.0    \\ \hline
\end{tabular}
\end{adjustbox}
\caption{\upshape{Factual consistency rate (\%) on news datasets.}}
\label{tab:fact}
\end{table}

% Summary length reflects the model's ability to compress information. Although TinyLlama produces the shortest summaries, its low BertScore suggests it misses critical details. Therefore, we consider text compression only when BertScore exceeds 70. Among these, Gemma-1.1 (43 words) and Qwen2-1.5-Ins (41 words) demonstrate the best compression performance.

%总结
\subsection{Comparison with LLMs}

% \begin{figure}
%     \centering
%     \includegraphics[width=1\linewidth]{latex/fig/vsLLM_bertscore.pdf}
%     \caption{ Bertscore between two models. }
%     \label{fig:vsllm_bertscore}
% \end{figure}



% \begin{table}[]
% \centering
% % \resizebox{\columnwidth}{!}{
% \begin{adjustbox}{max width=1\columnwidth}
% \begin{tabular}{lcccc}
% \hline
% Model &
%   \begin{tabular}[c]{@{}c@{}}Minimum\\ length\end{tabular} &
%   \begin{tabular}[c]{@{}c@{}}Maximum\\ length\end{tabular} &
%   \begin{tabular}[c]{@{}c@{}}Average\\ length\end{tabular} &
%   \begin{tabular}[c]{@{}c@{}}Standard\\ Deviation\end{tabular} \\ \hline
% Phi2           & 17 & 219 & 51.00 & 38.27 \\
% Gemma-1.1      & 1  & 72  & 41.84 & 8.68  \\
% Qwen2-1.5B-Ins & 16 & 206 & 40.97 & 17.85 \\
% Phi3-Mini      & 25 & 182 & 50.69 & 13.09 \\
% ChatGLM3       & 15 & 139 & 59.31 & 19.33 \\
% Mistral-7B-Ins & 26 & 191 & 59.62 & 27.85 \\
% Qwen2-7B-Ins   & 33 & 169 & 58.57& 16.06 \\
% Llama3-70B-Ins & 51 & 95  & 71.06 & 7.31  \\
% Qwen2-72B-Ins  & 31 & 254 & 85.75 & 21.67 \\ \hline
% \end{tabular}
% \end{adjustbox}
% \caption{\upshape{Summary length comparison of SLMs and LLMs on BBC2024 dataset.}}
% \label{tab:vs_llm_length}
% \end{table}

To further evaluate the SLM performance in news summarization, we compare the well-performing SLMs with LLMs, including ChatGLM3~\cite{glm2024chatglm}, Mistral-7B-Ins~\cite{jiang2023mistral}, Llam3-70B-Ins~\cite{llama3modelcard}, and Qwen2 series, all in instruction-tuned versions. As the models show similar performance across various reference summaries and datasets, we use Llama2-70B-Chat to generate references and present the average results on BBC2024 in Table~\ref{tab:vs_llm_bertscore}. 


We highlight the top two metrics within each category. As can be seen, the summarization quality produced by the SLMs is on par with that of the LLMs. The difference in BertScore and factual consistency is minimal, with SLMs occasionally performing better. Notably, when BertScore scores are similar, smaller models tend to generate shorter summaries that facilitate quicker reading and comprehension of news outlines. 

\begin{table}[]
\centering
% \resizebox{\columnwidth}{!}{
\begin{adjustbox}{max width=1\columnwidth}
\begin{tabular}{lccc}
\hline
Model & BertScore & \begin{tabular}[c]{@{}c@{}}Factual \\ consistency\end{tabular} & \begin{tabular}[c]{@{}c@{}}Summary\\ length\end{tabular} \\ \hline
Qwen2-1.5B-Ins  & 71.97 & 92.4 & \textbf{41} \\
Phi3-Mini       & 73.72 & \textbf{97.6} & \textbf{51} \\
Llama3.2-3B-Ins & \textbf{74.95} & 95.6 & 70 \\
ChatGLM3        & 73.41 & 96.0 & 59 \\
Mistral-7B-Ins  & 74.32 & 93.0 & 60 \\
Qwen2-7B-Ins    & 74.71 & 95.6 & 59 \\
Llama3-70B-Ins  & \textbf{76.06} & 95.8 & 71 \\
Qwen2-72B-Ins   & 73.78 & \textbf{98.4} & 86 \\ \hline
\end{tabular}
\end{adjustbox}
\caption{\upshape{Average metric comparison of SLMs and LLMs on BBC2024 dataset.}}
\label{tab:vs_llm_bertscore}
\end{table}


\begin{table}[]
\centering
\begin{adjustbox}{max width=1\columnwidth}
% \resizebox{\linewidth}{!}{
\begin{tabular}{lccc}
\hline
Model & \multicolumn{1}{r}{Prompt 1} & \multicolumn{1}{r}{Prompt 2} & \multicolumn{1}{r}{Prompt 3} \\ \hline
Qwen2-0.5B-Ins  & 68.64 & 60.31 & 62.46 \\
Qwen2-1.5B-Ins  & 71.97 & 69.74 & 72.30 \\
Llama3.2-1B-Ins & 72.55 & 72.16 & 67.53 \\
Phi3-Mini       & 73.72 & 73.27 & 73.83 \\
Llama3.2-3B-Ins & 74.95 & 74.94 & 75.12 \\ \hline
\end{tabular}
\end{adjustbox}
\caption{\upshape{BertScore comparison with different prompts.} }
\label{tab:diff_prompt_bertscore}
\end{table}

% Although SLM trails behind LLM in both BertScore and stability, the difference is minimal. Llama3-70B-Ins achieves the highest average BertScore of 76.06, reflecting its superior summary quality, and also has the lowest standard deviation (5.12), indicating consistent performance. Although Phi3-Mini is an SLM, its scores and stability are comparable to LLMs. Its BertScore surpasses that of the ChatGLM3, and its standard deviation is lower than 7B LLMs like Qwen2-7B-Ins and Mistral-7B-Ins.


% Table~\ref{tab:vs_llm_length} shows summary length statistics. SLMs generally produce more concise summaries than LLMs, especially those around 70B parameters. Although SLMs occasionally generate longer summaries, their average length is typically shorter due to the use of simpler sentences and fewer descriptive words. In Figure~\ref{fig:bertscore_example}, while both Phi3-Mini and Llama2-70B-Chat summarize match results and rankings, Llama2-70B-Chat uses more descriptive language for additional details. In terms of summary length stability, there is little difference between SLMs and LLMs. Models like Gemma-1.1 and Llama3-70B-Ins show consistent summary lengths, while Qwen2-72B-Ins and Phi2 exhibit greater variability.


\subsection{Human Evaluation}


To further validate the effectiveness of our metrics, we performed a small-scale human evaluation, referencing existing studies~\cite{fabbri2021summeval}. Specifically, six highly educated annotators evaluated the relevance of news summaries generated by five SLMs based on the original news articles (scored on a 1-5 scale, with higher scores indicating greater relevance). We randomly selected 20 news instances from the BBC2024 dataset as our evaluation set. As shown in Table~\ref{tab:small_human_eva}, the average scores of each model demonstrate a strong alignment between BertScore and human evaluations, achieving a Kendall correlation coefficient of 1.

\begin{table*}[]
\begin{tabular}{lrrrrrrrr}
\hline
Model & \multicolumn{1}{l}{Ann. 1} & \multicolumn{1}{l}{Ann. 2} & \multicolumn{1}{l}{Ann. 3} & \multicolumn{1}{l}{Ann. 4} & \multicolumn{1}{l}{Ann. 5} & \multicolumn{1}{l}{Ann. 6} & \multicolumn{1}{l}{Average} & \multicolumn{1}{l}{BertScore} \\ \hline
Bloom-560M & 1.95 & 1.25 & 1.43 & 1.60 & 1.00 & 1.00 & 1.37 & 46.30 \\
Llama3.2-3B & 3.15 & 2.10 & 2.25 & 2.50 & 1.35 & 1.15 & 2.08 & 66.36 \\
Qwen2-0.5B & 4.05 & 2.15 & 3.28 & 2.70 & 3.60 & 3.25 & 3.17 & 69.85 \\
Phi3-Mini & 4.45 & 4.60 & 4.00 & 3.45 & 4.60 & 4.25 & 4.20 & 73.14 \\
Llama3.2-3B-Ins & 4.65 & 4.45 & 4.35 & 3.65 & 4.00 & 4.10 & 4.23 & 75.15 \\ \hline
\end{tabular}
\caption{Humance evaluation in relevance on 5 SLMs.}
\label{tab:small_human_eva}
\end{table*}



\subsection{Overall Evaluation and Model Selection}
SLMs exhibit considerable variability in text summarization performance, with larger and newer models generally showing stronger capabilities. Among them, Phi3-Mini and Llama3.2-3B-Ins perform the best, matching the 70B LLM in relevance, coherence, and factual consistency, while producing shorter summaries. All of these comparisons suggest that SLMs can effectively replace LLMs on edge devices for news summarization.

Although we evaluate 19 SLMs, there is considerable variation in their parameter sizes. To optimize deployment on edge devices, we provide model selection recommendations based on size. For models under 1B parameters, Brio and Qwen2-0.5B are the top choices overall. General-purpose language models in this range offer no clear advantage over those specialized in text summarization. For models between 1B and 2B parameters, Llama3.2-1B-Ins demonstrates a clear edge. For models above 2B parameters, Llama3.2-3B-Ins and Phi3-Mini outperform others across all criteria.

% \begin{table}[]
% \caption{\upshape{Overall SLM performance evaluation for different model sizes}}
% \label{tab:by_size}
% \begin{adjustbox}{max width=1\columnwidth}
% \begin{tabular}{ccccc}
% \hline
% Parameters & Model & BertScore($\uparrow$) & Faithfulness($\uparrow$) & Length($\downarrow$) \\ \hline
% \multirow{6}{*}{$< 1B$} & Brio                & \textbf{69.88} & 94.3          & \textbf{53.70} \\
%                   & LiteLama            & 39.19          & 64.1          & 186.88         \\
%                   & Bloom-560M          & 47.25          & 82.9          & 198.60         \\
%                   & Qwen2-0.5B          & \textbf{68.88} & \textbf{94.5} & \textbf{44.03} \\
%                   & Qwen2-0.5B-Ins      & 68.42          & 89.0          & 61.38          \\
%                   & Pegasus-Large       & 62.07          & \textbf{99.9} & 97.37          \\ \hline
% \multirow{6}{*}{$1B \sim 2B$} & TinyLlama           & 67.56          & \textbf{96.1} & \textbf{36.80} \\
%                   & GPT-Neo-1.3B        & 57.30          & 88.1          & 169.60         \\
%                   & Qwen2-1.5B          & \textbf{71.38} & \textbf{96.5} & 59.10          \\
%                   & Qwen2-1.5B-Ins      & \textbf{71.40} & 92.5          & \textbf{41.20} \\
%                   & InternLM2-1.8B      & 67.75          & 92.5          & 45.29          \\
%                   & InternLM2-1.8B-Chat & 67.91          & 89.5          & 43.58          \\ \hline
% \multirow{5}{*}{$> 2B$} & MiniCPM             & 65.35          & 85.5          & 105.90         \\
%                   & Gemma-1.1           & \textbf{71.22} & \textbf{95.7} & \textbf{42.78} \\
%                   & GPT-Neo-2.7B        & 58.25          & 80.5          & 121.55         \\
%                   & Phi-2               & 70.69          & 95.8          & 52.72          \\
%                   & Phi-3-Mini          & \textbf{74.01} & \textbf{96.7} & \textbf{51.88} \\\hline
% \end{tabular}
% \end{adjustbox}
% \end{table}



% \begin{table*}[]
% % \caption{System-level Kendall's tau correlation coefficients with different reference summary settings.}
% \begin{center}
% % \begin{adjustbox}{max width=2.05\columnwidth}
% \resizebox{2\columnwidth}{!}{
% \begin{tabular}{cccccccccc}
% \hline
% \multirow{2}{*}{Metric} &
%   \multirow{2}{*}{\begin{tabular}[c]{@{}c@{}}Reference\\ type\end{tabular}} &
%   \multicolumn{4}{c}{Relevance evaluation} &
%   \multicolumn{4}{c}{Coherence evaluation} \\
%  &
%    &
%   \begin{tabular}[c]{@{}c@{}}Bench\\ -CNN/DM\end{tabular} &
%   \begin{tabular}[c]{@{}c@{}}Bench\\ -XSum\end{tabular} &
%   SummEval &
%   Average &
%   \begin{tabular}[c]{@{}c@{}}Bench\\ -CNN/DM\end{tabular} &
%   \begin{tabular}[c]{@{}c@{}}Bench\\ -XSum\end{tabular} &
%   SummEval &
%   Average \\ \hline
% \multirow{3}{*}{RougeL}   & Original    & 0.6732 & 0.2647 & 0.3714 & 0.4364          & 0.4248 & 0.6765 & 0.1238 & 0.4084          \\
%                            & Qwen1.5-72B & 0.7778 & 0.7059 & 0.4476 & \textbf{0.6438} & 0.5556 & 0.5000 & 0.2381 & \textbf{0.4312} \\
%                            & llama2-70B  & 0.7778 & 0.6324 & 0.4095 & 0.6066          & 0.5294 & 0.3676 & 0.2000 & 0.3657          \\ \hline
% \multirow{3}{*}{BertScore} & Original    & 0.6209 & 0.2206 & 0.4476 & 0.4297          & 0.4771 & 0.6912 & 0.4286 & 0.5323          \\
%                            & Qwen1.5-72B & 0.7516 & 0.6324 & 0.8095 & 0.7312          & 0.5817 & 0.6029 & 0.6000 & \textbf{0.5949} \\
%                            & llama2-70B  & 0.7516 & 0.6618 & 0.8286 & \textbf{0.7413} & 0.5294 & 0.6029 & 0.6190 & 0.5838          \\ \hline
% \multirow{3}{*}{BLEURT}    & Original    & 0.5425 & 0.2647 & 0.181  & 0.3294          & 0.3203 & 0.6176 & 0.2381 & 0.3929          \\
%                            & Qwen1.5-72B & 0.5817 & 0.7206 & 0.4476 & 0.5833          & 0.4902 & 0.4853 & 0.5429 & 0.5061          \\
%                            & llama2-70B  & 0.5556 & 0.7059 & 0.5429 & \textbf{0.6015} & 0.5425 & 0.4706 & 0.6762 & \textbf{0.5631} \\ \hline
% \end{tabular}}
% % \end{adjustbox}
% \end{center}
% \caption{{System-level Kendall's tau correlation coefficients between reference-based and human evaluations under different reference summary settings. "Original" is the original heuristic reference from the dataset; others are generated by LLMs. References generated by LLMs improve the effectiveness of reference-based evaluations.}}
% \label{tab:compare_kt_corr}
% \end{table*}

\section{Influencing Factor Analysis}
\label{sec:factor}
% In this section, we further select some well-performing SLMs to explore factors influencing the quality of text summarization, including prompt design and instruction tuning. 


\subsection{Prompt Design}
Prompt engineering has demonstrated significant power in many tasks, helping to improve the output quality of LLMs~\cite{zhao2021calibrate,surveypromptengineering}. Therefore, we use different prompt templates shown in Figure~\ref{fig:prompt} to analyze the impact of prompt engineering on BertScore scores, factual consistency, and summary length. The instructions become more detailed from Prompt 1 to Prompt 3.



\begin{figure}
    \centering
    \includegraphics[width=1\linewidth]{ARR_version/fig/diff_prompt_example2.pdf}
    \caption{Examples summaries generated by Llama3.2-3B-Ins with different prompts. The bold parts show the same content, suggesting that the prompt design has a limited improvement in summary quality.}
    \label{fig:diff_prompt_example}
\end{figure}




Table~\ref{tab:diff_prompt_bertscore} compares the BertScore. The small score differences among prompts suggest that detailed descriptions do not significantly improve summary relevance and coherence and may even negatively affect some models. This suggests that \textbf{simple instructions are more suitable for SLMs, and overly complex prompts may degrade the performance}. For instance, the BertScore of Qwen2-0.5B-Ins drops significantly. The decline is likely due to prompt noise, where the model struggles to balance multiple instructions, resulting in less coherent summaries. Additionally, Figure~\ref{fig:diff_prompt_example} shows an example of Llama3.2-3B-Ins with different prompts. All summaries capture the key information, with only slight differences in details.



Table~\ref{tab:diff_prompt_fact} provides a comparison of factual consistency rates for various models using three different prompts. Many models, such as Qwen2-0.5B-Ins, exhibit a significant drop in factual consistency when using prompt 3. This further demonstrates that models with very small parameters are not well-suited for complex prompt design. The summary length variations across different prompts in Table~\ref{tab:diff_prompt_length} also indicate the limitations of prompt design. Some models fail to generate summaries of appropriate length given complex prompts.





\begin{table}[]
\centering
\begin{adjustbox}{max width=1\columnwidth}
% \resizebox{\linewidth}{!}{
\begin{tabular}{lccc}
\hline
Model & \multicolumn{1}{r}{Prompt 1} & \multicolumn{1}{r}{Prompt 2} & \multicolumn{1}{r}{Prompt 3} \\ \hline
Qwen2-0.5B-Ins  & 89.6 & 89.8  & 78.2 \\
Qwen2-1.5B-Ins  & 92.4 & 90.8  & 89.0 \\
Llama3.2-1B-Ins & 94.0 & 94.8  & 91.6 \\
Phi3-Mini       & 97.6 & 96.4  & 97.2 \\
Llama3.2-3B-Ins & 95.6 & 96.2  & 94.6 \\ \hline
\end{tabular}
\end{adjustbox}
\caption{\upshape{Factual consistency rate (\%)  comparison with different prompts.} }
\label{tab:diff_prompt_fact}
\end{table}


\begin{table}[]
\centering
\begin{adjustbox}{max width=1\columnwidth}
% \resizebox{\linewidth}{!}{
\begin{tabular}{lccc}
\hline
Model & \multicolumn{1}{r}{Prompt 1} & \multicolumn{1}{r}{Prompt 2} & \multicolumn{1}{r}{Prompt 3} \\ \hline
Qwen2-0.5B-Ins  & 65 & 150 & 151 \\
Qwen2-1.5B-Ins  & 41 & 99  & 53 \\
Llama3.2-1B-Ins & 63 & 60  & 58 \\
Phi3-Mini       & 51 & 57  & 65  \\
Llama3.2-3B-Ins & 70 & 67  & 61 \\ \hline
\end{tabular}
\end{adjustbox}
\caption{\upshape{Summary length with different prompts. All prompts require models to summarize in two sentences.} }
\label{tab:diff_prompt_length}
\end{table}



Overall, the improvement of prompt design on the quality of SLM summaries is limited. For SLMs that already perform well with simple prompts, they can capture the key points without the need for complex prompts. In contrast, SLMs that perform poorly may struggle with the logical relationships in complex prompts, leading to lengthy summaries. For practical applications, we recommend using simple and clear prompts when deploying SLMs for news summarization.










% Models that already perform well do not benefit from prompt adjustments, while models with insufficient capabilities are not improved by prompt design and may even produce lower-quality summaries.






\subsection{Instruction Tuning}
Instruction tuning plays a crucial role in training LLMs by enhancing their ability to follow specific instructions. And some existing studies claim that instruction-tuned language models have stronger summarization capabilities~\cite{goyal2022news,zhang2024benchmarking}. However, in our extensive evaluations, with the exception of the Llama3.2 series, models perform very similarly before and after instruction tuning. In fact, models without instruction tuning often exhibit higher factual consistency. In the sample analysis, we find that the Llama3.2 models without instruction tuning include too many details, while instruction tuning helps it produce more general summaries. However, this trend is not observed in other models like Qwen2 and InternLM2 series, which is different from the conclusions of previous work~\cite{zhang2024benchmarking}. We leave the deep study of instruction tuning on summarization ability as a future research direction.


% \section{LLM-Generated Reference Verification}
% \label{sec:Verification}
% In this section, we verify whether the LLM-generated reference summaries can improve the effectiveness of reference-based evaluation on three human evaluation datasets~\cite{zhang2024benchmarking,fabbri2021summeval}. These datasets include summaries generated by various models as well as human ratings. We compare the consistency with human evaluations using different reference summaries and show results in Table~\ref{tab:compare_kt_corr}. As can be seen, LLM-generated summaries significantly improve consistency than original references. BertScore shows the best performance, which validates the effectiveness of using LLM-generated reference summaries in our evaluations.



% method on three human evaluation datasets~\cite{zhang2024benchmarking,fabbri2021summeval} and show results in Table~\ref{tab:compare_kt_corr}. We use Prompt 2 in Figure~\ref{fig:prompt} as the LLM input. The output is post-processed to remove introductory phrases and retain only complete sentences for the final summaries. For the LLM model, we select Qwen1.5-72B-Chat~\cite{qwen1} and Llama2-70B Chat~\cite{touvron2023llama2openfoundation}. As can be seen, LLM-generated summaries significantly improve consistency with human evaluation, particularly for relevance.  According to previous work~\cite{kendeer}, a Kendall coefficient above 0.3 indicates a strong correlation, making BertScore a suitable metric for the following SLM summarization benchmark.






% T 检验?
% instruction tuning更容易停止,更短
% 指令调优效果在不同模型上不同,以及可能有负面作用,这个是需要进一步研究的东西
%prompt design 对表现不好的模型有帮助么?


% Additionally, the impact of instruction tuning on summary length did not display a consistent trend. It is worth noting that after instruction tuning, the summaries generated by Qwen2-0.5B-Ins increased significantly in length.

% We also explore whether instruction tuning can enhance the model ability to control summary length. We ask the models to generate summaries of a specified length. Specifically, based on the prompt 1 template, we specify the word count or the sentence count for the generated summaries. 
% Tables~\ref{tab:summary_len_words} and~\ref{tab:summary_len_sentence} show the results, and model names with the suffix "Ins" or "Chat" are instruction-tuned versions. we find that instruction tuning does not significantly help in controlling summary length. Except for Phi3-Mini, the instruction-tuned models perform similarly to the original versions in limiting summary words. Under different summary length constraints, only Phi3-Mini meets the requirements well. When considering the word limit, although the summary length increases with the word limit increasing, most models output a length that exceeds the word limit. Only Phi3-Mini generates summaries that match it closely. When sentence count is limited, Phi3-Mini produces summaries with each sentence containing 20-30 words, effectively adhering to the specified number of sentences. Phi2 and the Qwen2 series perform well only when the number of sentences is specified. Regardless of whether they have undergone instruction tuning, they are unable to effectively limit the number of words. 

% Although instruction tuning has limited effects on controlling summary length, it helps models better terminate their output. During testing, we find that the instruction-tuned versions more accurately adhere to the specified token (such as <eos>) limit, effectively stopping output. In contrast, for the non-instruction-tuned versions, we had to detect the newline character to halt the output, which sometimes resulted in truncating the generated content.
% how to stop

% \begin{table}[]
% \caption{\upshape{Summary length under different prompt word limits.}}
% \label{tab:summary_len_words}
% \centering
% \begin{adjustbox}{max width=1\columnwidth}
% \begin{tabular}{lcccc}
% \hline
% Model          & 50 words & 60 words & 70 words & 80 words \\ \hline
% Qwen2-1.5B-Ins & 75.01    & 71.99    & 79.19    & 84.40    \\
% Qwen2-1.5B     & 84.01    & 86.84    & 98.64    & 105.57   \\
% InternLM2-1.8B-Chat & 91.59 & 93.62  & 98.76  & 106.14  \\
% InternLM2-1.8B & 77.47 & 101.86 & 110.29  & 116.47 \\
% Phi2           & 115.68   & 137.78   & 148.54   & 157.19  \\
% Phi3-Mini      & 49.78    & 56.31    & 67.78    & 76.32    \\
% Qwen2-7B-Ins   & 78.57    & 86.00    & 91.00    & 99.07    \\
% Qwen2-7B       & 71.78    & 86.81    & 87.32    & 91.54    \\ \hline
% \end{tabular}
% \end{adjustbox}
% \end{table}

% \begin{table}[]
% \caption{\upshape{Summary length under different prompt sentence limits.}}
% \label{tab:summary_len_sentence}
% \centering
% \begin{adjustbox}{max width=1\columnwidth}
% \begin{tabular}{lcccc}
% \hline
% Model &
%   \begin{tabular}[c]{@{}c@{}}One\\ sentence\end{tabular} &
%   \begin{tabular}[c]{@{}c@{}}Two\\ sentences\end{tabular} &
%   \begin{tabular}[c]{@{}c@{}}Three\\ sentences\end{tabular} &
%   \begin{tabular}[c]{@{}c@{}}Four\\ sentences\end{tabular} \\ \hline
% Qwen2-1.5B-Ins & 31.61 & 40.90 & 61.80 & 75.17 \\
% Qwen2-1.5B     & 38.08 & 58.89 & 78.95 & 93.78 \\
% InternLM2-1.8B-Chat & 44.48 & 57.69 & 75.75 & 95.10 \\
% InternLM2-1.8B & 23.45 & 41.17 & 66.01 & 78.90 \\
% Phi2           & 35.50 & 60.00 & 69.96 & 97.09 \\
% Phi3-Mini      & 35.25 & 50.69 & 78.39 & 97.49 \\
% Qwen2-7B-Ins   & 44.92 & 59.80 & 80.32 & 96.35 \\
% Qwen2-7B       & 42.38 & 59.44 & 74.21 & 86.58 \\ \hline
% \end{tabular}
% \end{adjustbox}
% \end{table}







% \subsection{Prompt Design}

% \begin{table}[]
% \caption{\upshape{BertScore comparison with different prompts.} }
% \label{tab:diff_prompt_bertscore}
% \centering
% \begin{adjustbox}{max width=1\columnwidth}
% % \resizebox{\linewidth}{!}{
% \begin{tabular}{lrrr}
% \hline
% Model          & Prompt 1 & Prompt 2 & Prompt 3 \\ \hline
% Qwen2-0.5B-Ins & 68.64   & 68.09   & 68.30   \\
% Qwen2-0.5B     & 69.67   & 69.58   & 69.67   \\
% Qwen2-1.5B-Ins & 71.78   & 72.02   & 71.70   \\
% Qwen2-1.5B     & 71.99   & 72.12   & 72.22   \\
% Phi2           & 71.50   & 70.97   & 71.87   \\
% Phi3-Mini      & 73.72   & 74.13   & 74.49   \\
% Qwen2-7B-Ins   & 74.71   & 73.40   & 73.88   \\
% Qwen2-7B       & 75.27   & 75.37   & 74.99   \\ \hline
% \end{tabular}
% \end{adjustbox}

% \end{table}




% \begin{table}[]
% \caption{\upshape{BertScore and summary length comparison with different prompts}}
% \label{tab:diff_detail_prompt}
% \centering
% \begin{adjustbox}{max width=1\columnwidth}
% \begin{tabular}{lrrrrrr}
% \hline
% Model          & \multicolumn{3}{c}{BertScore}  & \multicolumn{3}{c}{Summary Length} \\
%                & Prompt 1 & Prompt 2 & Prompt 3 & Prompt 1   & Prompt 2  & Prompt 3  \\ \hline
% Qwen2-0.5B-Ins & 61.47    & 68.09    & 68.30    & 140.75     & 51.52     & 64.79     \\
% Qwen2-0.5B     & 69.67    & 69.58    & 69.67    & 40.84      & 30.93     & 37.90     \\
% Qwen2-1.5B-Ins & 71.78    & 72.02    & 71.70    & 53.36      & 42.68     & 41.38     \\
% Qwen2-1.5B     & 72.18    & 72.12    & 72.22    & 58.89      & 48.64     & 56.98     \\
% Phi2           & 71.50    & 70.97    & 71.87    & 60.00      & 100.88    & 72.554    \\
% Phi3-Mini      & 73.72    & 74.13    & 74.49    & 50.69      & 58.27     & 58.66     \\
% Qwen2-7B-Ins   & 74.71    & 73.40    & 73.88    & 58.57      & 64.19     & 63.74     \\
% Qwen2-7B       & 75.27    & 75.37    & 74.99    & 56.41      & 50.78     & 51.64     \\ \hline
% \end{tabular}
% \end{adjustbox}
% \end{table}


% Prompt engineering has demonstrated significant power in many tasks, helping to improve the output quality of LLMs~\cite{zhao2021calibrate,surveypromptengineering}. Therefore, we use different prompt templates shown in Fig~\ref{fig:prompt} to analyze the impact of prompt engineering on BertScore scores, factual consistency, and summary length. From Prompt 1 to Prompt 3, the instructions become increasingly detailed. 
% % Tables~\ref{tab:diff_detail_prompt} present the results.

% Table~\ref{tab:diff_prompt_bertscore} compares the BertScore for various models using three different prompts. The changes across different prompts are minimal, indicating that more detailed prompt words cannot significantly improve the relevance and coherence of the summary. This consistency is beneficial for practical applications where prompts may vary, ensuring stable and reliable text generation performance. Fig~\ref{fig:diff_prompt_example} shows some examples of Qwen-0.5B-Ins with different prompts. All summaries can capture the key information, and prompts 2 and 3 result in only minor changes in connective words. 

% Table~\ref{tab:diff_prompt_fact} provides a comparison of factual consistency rates for various models using three different prompts. Although most models show slight variations in factual consistency rates, they show consistent changes under some prompts. On most language models, using prompt 2 improves fact consistency by about 2\% compared with prompt 1. The BBC2024 test dataset includes 500 instances. A 2\% improvement means improving the consistency of facts on 10 summaries.

% However, prompt design has a significant impact on summary length, especially for SLMs. Although they are all asked to summarize the article in two sentences, some SLMs exhibit significant fluctuations, as shown in Table~\ref{tab:diff_prompt_len}. Only Phi3-Mini can maintain the stable summary length. Prompts affect different models in varying ways. For the Qwen2 series, although prompt 2 is more detailed than prompt 1, it generates shorter summaries. However, for the Phi series, summaries generated using prompt 2 are longer than those generated using prompt 1. Fig~\ref{fig:diff_prompt_example} shows examples of summaries generated by Qwen2-0.5B-Ins under different prompts. While all summaries capture the key points, the summary generated with prompt 1 is significantly longer due to more details.



% \begin{table}[]
% \caption{\upshape{Factual consistency rate (\%) with different prompts.} }
% \label{tab:diff_prompt_fact}
% \centering
% \begin{adjustbox}{max width=1\columnwidth}
% % \resizebox{\linewidth}{!}{
% \begin{tabular}{lrrr}
% \hline
% Model          & Prompt 1 & Prompt 2 & Prompt 3 \\ \hline
% Qwen2-0.5B-Ins & 89.6    & 88.8   & 88.2   \\
% Qwen2-0.5B     & 95.2    & 97.4   & 96.0   \\
% Qwen2-1.5B-Ins & 92.4    & 94.0   & 91.8   \\
% Qwen2-1.5B     & 96.6    & 97.6   & 97.6   \\
% Phi2           & 96.8    & 99.0   & 97.2   \\
% Phi3-Mini      & 97.6    & 98.2   & 98.6   \\
% Qwen2-7B-Ins   & 95.6    & 96.0   & 95.0   \\
% Qwen2-7B       & 96.8    & 97.4   & 96.0    \\ \hline
% \end{tabular}
% \end{adjustbox}

% \end{table}


% \begin{table}[]
% \caption{\upshape{Summary length comparison with different prompts.}}
% \label{tab:diff_prompt_len}
% \centering
% \begin{adjustbox}{max width=1\columnwidth}
% \begin{tabular}{lrrr}
% \hline
% Model          & Prompt 1 & Prompt 2 & Prompt 3 \\ \hline
% Qwen2-0.5B-Ins & 64.80  & 51.52   & 64.79   \\
% Qwen2-0.5B     & 40.84   & 30.93   & 37.90   \\
% Qwen2-1.5B-Ins & 53.36   & 42.68   & 41.38   \\
% Qwen2-1.5B     & 58.89   & 48.64   & 56.98   \\
% Phi2           & 60.00   & 100.88  & 72.55    \\
% Phi3-Mini      & 53.87   & 58.27   & 58.66   \\
% Qwen2-7B-Ins   & 58.57   & 64.19   & 63.74   \\
% Qwen2-7B       & 56.41   & 50.78   & 51.64   \\ \hline
% \end{tabular}
% \end{adjustbox}
% \end{table}




% \subsection{Model Size and Training Data}
% To ensure that model size is the only variable, we can focus on models from the same series, such as the Qwen2 series and GPT-Neo series. As can be seen in Table~\ref{tab:bertscore_res} and~\ref{tab:diff_prompt_bertscore}, it is evident that increasing the model size improves the performance on BertScore, but the effect diminishes with larger models, which is consistent with existing studies~\cite{zhang2024benchmarking}. For instance, Qwen2-7B-Ins scores on the BBC2024 dataset even surpass those of Qwen2-72B-Ins. In terms of factual consistency, having more parameters does not necessarily lead to improvement. For example, as shown in Table~\ref{tab:fact}, the summaries generated by GPT-Neo-1.3B are more aligned with the original articles than those generated by GPT-Neo-2.7B by around 8\%.

% Besides model size, high-quality training data is also crucial. When the model sizes are similar, there is also a large gap between different model series. For example, Qwen2-0.5B and Bloom-560M have a BertScore difference of nearly 30 points. Some models with
% fewer parameters even score higher. Although Qwen2-1.5B has fewer parameters than Phi2 and Gemma-1.1, it scores higher average BertScore. And Phi3-Mini even outperforms some 70B models on factual consistency. We attribute this to the differences in training data and training methods. Thus, both model size and the quality of training data are crucial for the text summarization capability.

% \subsection{Findings}
% Among all the factors influencing the summarization capabilities of SLMs, we find that model size and the training data quality are crucial for ensuring the relevance, coherence, and factual consistency of summaries. Prompt design and instruction tuning have a limited impact on most SLMs, which maintain consistent summary quality across different prompts. Additionally, instruction tuning only marginally improves the control of output text length.

\section{Conclusion}
\label{sec:conclusion}
This paper presents the comprehensive evaluation of SLMs for news summarization, comparing 19 models across diverse datasets. Our results demonstrate that top-performing models like Phi3-Mini and Llama3.2-3B-Ins can match the performance of larger 70B LLMs while producing shorter, more concise summaries. These findings highlight the potential of SLMs for real-world applications, particularly in resource-constrained environments. Further exploration shows that simple prompts are more effective for SLM summarization, while instruction tuning provides inconsistent benefits, necessitating further research.




% we comprehensively evaluate 17 SLMs on their text summarization performance. In terms of evaluation metrics, to improve the evaluation results more aligned with human preferences, we use high-quality reference summaries generated by LLMs to replace the original reference summaries in reference-based evaluation methods. In the evaluation experiments, we find that there is significant variability among SLMs in text summarization, with Phi3-Mini exhibiting the best performance among them. Compared with LLMs, while state-of-the-art SLMs still lag behind LLMs in summary relevance and coherence, the difference is minimal, with LLMs typically including more detailed content. There is no significant difference in factual consistency. Furthermore, we find that model size and training data remain the most critical factors influencing summary quality, with prompt design and instruction tuning having limited impact.


\section*{Limitations}

Although using LLM-generated summaries improved the reliability of reference-based summarization evaluation methods, it may introduce bias. For instance, if the LLM fails to accurately summarize the news, an SLM that produces similar content might receive an undeservedly high score. To mitigate this potential bias, we use summaries generated by multiple LLMs as references and calculate the average scores. Furthermore, although BertScore demonstrates a high level of consistency with human evaluations in coherence, it is not specifically designed to evaluate coherence. In future work, we plan to explore the use of more accurate metrics for coherence evaluation. Due to the input length limits of most SLMs, we only considered news articles with a maximum of 1,500 tokens. In the future, we will explore how to effectively use SLMs to summarize much longer news articles. Additionally, we do not consider quantized models in this study. In future work, we will also incorporate an analysis of the performance of quantized models into our evaluation.
% \subsection{Text Order}

\section*{Acknowledgments}
This work is funded by the National Key R\&D Program of China (No. 2019YFA0709400), the NSFC Project (No. 62306256), the Guangzhou Science and Technology Development Projects (No. 2023A03J0143 and No. 2024A04J4458), and the Guangzhou Municipality Big Data Intelligence Key Lab (NO. 2023A03J0012). This work is also supported by the National Research Foundation, Singapore under its Industry Alignment Fund – Pre-positioning (IAF-PP) Funding Initiative. Any opinions, findings and conclusions or recommendations expressed in this material are those of the author(s) and do not reflect the views of National Research Foundation, Singapore. Borui's work was done when he was a visiting student at the National University of Singapore.


% Bibliography entries for the entire Anthology, followed by custom entries
%\bibliography{anthology,custom}
% Custom bibliography entries only
\bibliography{custom}

\appendix

\section{Challenges in Existing Summary Evaluation Method}
\label{sec:appendix}
In this section, we will supplement the detailed analysis of existing evaluation methods and explain why we use the reference-based evaluation method.

\subsection{Human evaluation}
Human evaluation remains the most intuitive and effective method for text summarization evaluation~\cite{fabbri2021summeval,pu2023summarization,liu2023benchmarking,zhang2024benchmarking}. Annotators generally score generated summaries on aspects such as relevance and coherence (scale 1-5) based on the source text, and the final score is the average across all samples. Despite its reliability, human evaluation faces three main challenges for large-scale use: (1) Time-consuming—even evaluating 100 samples for each model can take weeks~\cite{zhang2024benchmarking}; (2) Low reproducibility—scores may vary due to different annotators; (3) High cost—only expert annotators are suitable, but hiring them is costly~\cite{fabbri2021summeval}. As shown in Figure~\ref{fig:kd_heatmap_relevance}, experts demonstrate significantly higher agreement than crowd-sourced annotators.

\begin{figure}
    \centering
    \includegraphics[width=1\linewidth]{ARR_version/fig/kendall_correlation_heatmap_relevance.pdf}
    \caption{Pairwise system-level Kendall’s tau correlation on relevance for different annotators in SummEval benchmark. "CS" is the abbreviation for crowd-sourced. Darker colors indicate better consistency. }
    \label{fig:kd_heatmap_relevance}
\end{figure}



\subsection{Reference-based Evaluation}
\label{sec:reference eva}
Reference-based methods are among the earliest for evaluating text summarization~\cite{papineni-etal-2002-bleu}. They are more cost-effective and faster than human evaluation, using statistical metrics or small models to compute the similarity between generated and reference summaries. Higher similarity indicates better quality. There are two main categories: one measures textual overlap (e.g., ROUGE~\cite{lin-2004-rouge}, BLEU~\cite{papineni-etal-2002-bleu}); the other uses language models like BERT~\cite{devlin-etal-2019-bert} to compute semantic similarity (e.g., BertScore~\cite{bert-score}, BLEURT~\cite{sellam-etal-2020-bleurt}). However, the reference quality affects the evaluation reliability heavily~\cite{how_well,zhang2024benchmarking}. Heuristic methods used to create references in many datasets often lead to poor-quality summaries. For instance, XSum~\cite{xsum} uses only the first sentence of a news article as the reference, resulting in 54\% of summaries missing key information~\cite{how_well}. Moreover, since different datasets use varied reference generation methods, evaluation reliability fluctuates. Table~\ref{tab:compare_kt_corr} illustrates the impact of reference quality on reference-based methods. The correlation with human evaluation varies significantly across datasets when using the original reference for each dataset. It highlights the need for a unified, high-quality reference generation method.


\subsection{LLM evaluation}
Recent studies~\cite{goyal2022news,kendeer,gao2023human-like,liu2023benchmarking,wang2023chatgpt} have explored using LLMs for human-like summarization evaluation to achieve better alignment with human judgment. This approach involves guiding LLMs with prompts and providing both the summary and the original article for scoring. However, while LLM-based evaluation is reproducible and promising, it shares cost and efficiency challenges with human evaluation. Using commercial LLM APIs for large datasets can be expensive, and deploying open-source LLMs (e.g., 70B models) locally requires costly GPUs like the Nvidia A100 for reasonable speed. Additionally, studies~\cite{not-yet} indicate that LLMs may exhibit biases when evaluating different models, warranting further research to ensure fairness.

Overall, existing evaluation methods face issues related to financial cost, reproducibility, and effectiveness. Among them, reference-based methods are the most cost-effective and efficient. Therefore, we consider the LLM-augmented reference-based evaluation method as our metrics.


% \section{Experimental Setup for LLM-Generated Reference Verification}

% We use prompt 2 in Figure~\ref{fig:prompt} as the input of the LLM. For all datasets, we consistently use two sentences to summarize the article. We also perform post-processing on the output of the LLM. We filter out some prompting words, such as "The news summary is:", and retain only complete sentences to form the final reference summaries. And the reference summaries are generated by Qwen1.5-72B-Chat~\cite{qwen1} and Llama2-70B-Chat~\cite{touvron2023llama2openfoundation} with vLLM framework~\cite{vllm}. All models do not restrict output length, and the temperature was 0. In Table~\ref{tab:compare_kt_corr}, BertScore exhibits the best performance among the evaluated metrics. The correlation between BertScore and human evaluation in terms of relevance is approaching that observed among different expert evaluations.



\section{Model Details}
In this section, we describe all benchmarked SLMs in our experiments.

\textbf{LiteLlama~\cite{huggingface2024litelama}:} LiteLlama is an open-source reproduction of Llama2~\cite{touvron2023llama2openfoundation}. It has 460M parameters and is trained on 1T tokens from the RedPajama dataset~\cite{together2023redpajama}.

\textbf{Bloom-560M~\cite{bloom}:} Bloom-560M is part of the BLOOM (BigScience Large Open-science Open-access Multilingual) family of models developed by the BigScience project. It has 560M parameters and is trained on 1.5T pre-processed text.

\textbf{TinyLlama~\cite{zhang2024tinyllama}:} TinyLlama has the same architecture and tokenizer as Llama 2. It has 1.1B parameters and is trained on 3T tokens. We use the chat finetuned version.

\textbf{GPT-Neo series~\cite{gpt-neo}:} GPT-Neo Series are open-source language models developed by EleutherAI to reproduce GPT-3 architecture. It has two versions with 1.3B and 2.7B parameters. They are trained on 380B and 420B tokens, respectively, on the Pile dataset~\cite{gao2020pile,biderman2022datasheet_pile}.

\textbf{Qwen2 series~\cite{qwen}:} Qwen2 comprises a series of language models pre-trained on multilingual datasets. And some models use instruction tuning to align with human preferences. We select models with no more than 7B parameters for benchmarking, including 0.5B, 0.5B-Ins, 1.5B, 1.5B-Ins, 7B, and 7B-Ins. Models with the "Ins" suffix indicate the instruction tuning version.

\textbf{InterLM2 series~\cite{cai2024internlm2}:} The InternLM2 series includes models with various parameter sizes. They all support long contexts of up to 200,000 characters. We select the 1.8B version and the 1.8B-Chat version for evaluation.

\textbf{Gemma-1.1~\cite{team2024gemma}:} Gemma-1.1 is developed by Google and is trained using a novel RLHF method. We select the 2B instruction tuning version for benchmarking.

\textbf{MiniCPM~\cite{hu2024minicpm}:} MiniCPM is an end-size language model with 2.7B parameters. It employs various post-training methods to align with human preferences. We select the supervised fine-tuning (SFT) version for benchmarking.

\textbf{Llama3.2 series~\cite{llama3.2}:} In the Llama 3.2 series, Meta introduced two SLMs, 1B and 3B, designed for edge devices. While Meta highlighted their capability to summarize social media messages, their performance in news summarization has yet to be evaluated.


\textbf{Phi series~\cite{microsoft2023phi2,abdin2024phi3}:} Phi series are some SLMs developed by Microsoft. They are trained on high-quality datasets. We select Phi-2(2.7B parameters) and Phi-3-Mini(3.8B parameters, 4K context length) for evaluation. 

\textbf{Brio~\cite{brio}: } Brio is one of the state-of-the-art specialized language models designed for text summarization, which leverages reinforcement learning to optimize the selection of sentences, ensuring high-quality, informative, and coherent summaries. It only has 406M parameters, and we select the version trained on the CNN/DM dataset.

\textbf{Pegasus-Large~\cite{pegasus}:} Pegasus-Large also is a specialized language model designed for abstractive text summarization, utilizing gap-sentence generation pre-training to achieve exceptional performance on summarization tasks. It has 568M parameters.



\end{document}
