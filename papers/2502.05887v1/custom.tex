% This must be in the first 5 lines to tell arXiv to use pdfLaTeX, which is strongly recommended.
\pdfoutput=1
% In particular, the hyperref package requires pdfLaTeX in order to break URLs across lines.

\documentclass[11pt]{article}

% Change "review" to "final" to generate the final (sometimes called camera-ready) version.
% Change to "preprint" to generate a non-anonymous version with page numbers.
\usepackage[final]{acl}

% Standard package includes
\usepackage{times}
\usepackage{latexsym}

% For proper rendering and hyphenation of words containing Latin characters (including in bib files)
\usepackage[T1]{fontenc}
% For Vietnamese characters
% \usepackage[T5]{fontenc}
% See https://www.latex-project.org/help/documentation/encguide.pdf for other character sets

% This assumes your files are encoded as UTF8
\usepackage[utf8]{inputenc}

% This is not strictly necessary, and may be commented out,
% but it will improve the layout of the manuscript,
% and will typically save some space.
\usepackage{microtype}

% This is also not strictly necessary, and may be commented out.
% However, it will improve the aesthetics of text in
% the typewriter font.
\usepackage{inconsolata}

%Including images in your LaTeX document requires adding
%additional package(s)
\usepackage{graphicx}
\usepackage{booktabs}
\usepackage{multirow}
\usepackage{graphicx}
\usepackage{makecell}
\usepackage{xcolor}
\usepackage{amsmath}
% If the title and author information does not fit in the area allocated, uncomment the following
%
%\setlength\titlebox{<dim>}
%
% and set <dim> to something 5cm or larger.
\newcommand{\mf}[1]{\textcolor{orange}{MF: #1}}

\title{MTPChat: A Multimodal Time-Aware Persona Dataset \\ for Conversational Agents}

% Author information can be set in various styles:
% For several authors from the same institution:
% \author{Author 1 \and ... \and Author n \\
%         Address line \\ ... \\ Address line}
% if the names do not fit well on one line use
%         Author 1 \\ {\bf Author 2} \\ ... \\ {\bf Author n} \\
% For authors from different institutions:
% \author{Author 1 \\ Address line \\  ... \\ Address line
%         \And  ... \And
%         Author n \\ Address line \\ ... \\ Address line}
% To start a separate ``row'' of authors use \AND, as in
% \author{Author 1 \\ Address line \\  ... \\ Address line
%         \AND
%         Author 2 \\ Address line \\ ... \\ Address line \And
%         Author 3 \\ Address line \\ ... \\ Address line}

\author{
 \bfseries
 Wanqi Yang$^{1 \ast}$
 \
 Yanda Li$^{1}$\thanks{Equal contributions}
 \
 Meng Fang$^{2}$
 \
 Ling Chen$^1$
 \\
 \normalsize 
 $ ^1$ University of Technology Sydney
 \
 $ ^2 $ University of Liverpool
 \\
 {\normalsize \tt   wanqi.yang-1@student.uts.edu.au, 
 \normalsize \tt Yanda.Li@student.uts.edu.au}\\
 \normalsize \tt  Meng.Fang@liverpool.ac.uk,
 \normalsize \tt ling.chen@uts.edu.au
 }


%\author{
%  \textbf{First Author\textsuperscript{1}},
%  \textbf{Second Author\textsuperscript{1,2}},
%  \textbf{Third T. Author\textsuperscript{1}},
%  \textbf{Fourth Author\textsuperscript{1}},
%\\
%  \textbf{Fifth Author\textsuperscript{1,2}},
%  \textbf{Sixth Author\textsuperscript{1}},
%  \textbf{Seventh Author\textsuperscript{1}},
%  \textbf{Eighth Author \textsuperscript{1,2,3,4}},
%\\
%  \textbf{Ninth Author\textsuperscript{1}},
%  \textbf{Tenth Author\textsuperscript{1}},
%  \textbf{Eleventh E. Author\textsuperscript{1,2,3,4,5}},
%  \textbf{Twelfth Author\textsuperscript{1}},
%\\
%  \textbf{Thirteenth Author\textsuperscript{3}},
%  \textbf{Fourteenth F. Author\textsuperscript{2,4}},
%  \textbf{Fifteenth Author\textsuperscript{1}},
%  \textbf{Sixteenth Author\textsuperscript{1}},
%\\
%  \textbf{Seventeenth S. Author\textsuperscript{4,5}},
%  \textbf{Eighteenth Author\textsuperscript{3,4}},
%  \textbf{Nineteenth N. Author\textsuperscript{2,5}},
%  \textbf{Twentieth Author\textsuperscript{1}}
%\\
%\\
%  \textsuperscript{1}Affiliation 1,
%  \textsuperscript{2}Affiliation 2,
%  \textsuperscript{3}Affiliation 3,
%  \textsuperscript{4}Affiliation 4,
%  \textsuperscript{5}Affiliation 5
%\\
%  \small{
%    \textbf{Correspondence:} \href{mailto:email@domain}{email@domain}
%  }
%}

\begin{document}
\maketitle

%\renewcommand{\thefootnote}{\fnsymbol{footnote}}
%\footnotetext[1]{Equal contribution} 

%\renewcommand{\thefootnote}{\arabic{footnote}} 

\begin{abstract}
%\mf{To build self-consistent personalized dialogue agents, previous research has primarily focused on aligning conversations to deliver personal facts or personalities. However, understanding temporal dynamics is crucial for fully capturing the essence of a persona, as it can more effectively reveal the speaker's personal characteristics and experiences in episodic memory. In this work, we address the often-overlooked aspect of temporal dynamics in persona chats. Firstly, we introduce MPTChat, a multimodal time-sensitive dialogue dataset that integrates linguistic, visual, and temporal elements in dialogue and persona memory. Secondly, we design two time-sensitive tasks: Temporal Next Response Prediction (TNRP) and Temporal Grounding Memory Prediction (TGMP), which utilize implicit temporal cues and dynamic aspects to challenge models' temporal sensitivity. Furthermore, we present an innovative framework with an adaptive temporal module to effectively integrate these multimodal streams and build interconnections. Experimental results confirm the novel challenges posed by MPTChat and demonstrate the effectiveness of our framework in multimodal time-sensitive scenarios.} \mf{The following motivation is not correct or too general} 
%Understanding temporal dynamics is critical for applications ranging from multimedia content analysis to decision-making. However, existing time-sensitive datasets are predominantly focus on QA tasks and rely on explicit time information, which narrows their scope and diminishes their complexity. 
Understanding temporal dynamics is critical for conversational agents, enabling effective content analysis and informed decision-making. However, time-aware datasets, particularly for persona-grounded conversations, are still limited, which narrows their scope and diminishes their complexity. To address this gap, we introduce MTPChat, a multimodal, time-aware persona dialogue dataset that integrates linguistic, visual, and temporal elements within dialogue and persona memory. Leveraging MTPChat, we propose two time-sensitive tasks: Temporal Next Response Prediction (TNRP) and Temporal Grounding Memory Prediction (TGMP), both designed to assess a model’s ability to understand implicit temporal cues and dynamic interactions. Additionally, we present an innovative framework featuring an adaptive temporal module to effectively integrate multimodal streams and capture temporal dependencies. Experimental results validate the challenges posed by MTPChat and demonstrate the effectiveness of our framework in multimodal time-sensitive scenarios.
%To overcome these limitations, we introduce MTPChat, a multimodal time-aware persona dialogue dataset that integrates linguistic, visual, and temporal elements in dialogue and persona memory. Based on MTPChat, we design two time-sensitive tasks, Temporal Next Response Prediction (TNRP) and Temporal Grounding Memory Prediction (TGMP), utilizing implicit temporal cues and dynamic aspects to challenge model's temporal awareness. Furthermore, we present an innovative framework with an adaptive temporal module to integrate these multimodal streams and build interconnections effectively. The experimental results confirm that novel challenges of MTPChat and effectiveness of our framework in multimodal time-sensitive scenarios.
\end{abstract}

% 
% 
The widespread integration of communication networks and smart devices in modern control systems has increased the vulnerability of industrial systems to online cyber-attacks, e.g., Industroyer, Blackenergy, etc \citep{osti_1505628}.
% Modern control systems have seen a large push to include communication networks and smart devices to increase performance, made possible by improvements in communication device cost and energy consumption. This trend has been coupled with the usage of open-standard communication protocols among industrial control systems, making them vulnerable to online cyber-attacks such as Industroyer, Blackenergy, etc \citep{osti_1505628}. 
To counter this, methods have been developed to improve security by achieving attack detection, mitigation, and monitoring, among others \citep{sandberg2022secure}. This paper focuses on active attack diagnosis to mitigate stealthy attacks. 
%
%\subsection{Literature review}

Active diagnosis techniques rely on the inclusion of additional moduli to control systems
% inclusion within the control system of additional moduli 
to alter the behavior of the system compared to information known by the attacker. 
For instance, the concept of additive watermarking was introduced in \cite{mo2015physical}, where noise signals of known mean and variance are added at the plant and compensated for it at the controller. 
This compensation, however, is not exact, causing some performance degradation. Thus, trade-offs between performance and detectability  are necessary \citep{zhu2023detection}.
% A later work \citep{zhu2023detection} designs the watermark signal by trading performance for detection. Thus, although additive watermarking serves as a good detection scheme, they endure performance losses even in the nominal case. 

In encrypted control \citep{darup2021encrypted}, the sensor data is encrypted, sent to the controller, and then operated on directly. Encrypted input signals are sent back to the plant for decryption. Although encryption is widespread in IT security, in control systems it presents some concerns, such as the introduction of time delays \citep{stabile2024verifiable}, while it may present inherent weaknesses \citep{alisic2023model}.
% they are not preferred as they introduce time delays \citep{stabile2024verifiable} which can cause instability, and some encryption schemes can be very weak  \citep{alisic2023model}. 

In moving target defense \citep{griffioen2020moving}, the plant is augmented with fictitious dynamics, known to the controller. The plant output is transmitted to the controller along with the fictitious states over a network under attack. 
The additional measurements then aide in the detection of attacks. 
This comes at the cost of higher communication bandwidth needs, which increases rapidly with the dimension of the augmented systems.
% Since the dynamics of the fictitious dynamics are exactly known to the controller, the attack is detected easily. However, when the scale of the system increases, the communication bandwidth used by moving the target defense approach increases rapidly. 

Other recently proposed works include two-way coding \citep{fang2019two}, a weak encryuption technique, and dynamic masking \citep{abdalmoaty2023privacy}, which enhances privacy as well as security, have been shown to be effective against zero-dynamics attacks.
% Two-way coding \citep{fang2019two} and dynamic masking \citep{abdalmoaty2023privacy} are other recently proposed approaches. Two-way coding is another form of weak encryption technique whilst dynamic masking proposes an architecture that enhances both privacy and security. These schemes are shown to be effective against zero dynamics attacks but remain to be studied for other classes of attacks. 
% Recent extensions include \citep{mukherjee2021secure,ramos2024privacy}.
% Some other works which are related are \citep{mukherjee2021secure}, an extension of \cite{fang2019two}. The work \citep{ramos2024privacy} is an extension of moving target defense for multi-agent systems. 
Furthermore, filtering techniques for attack detection are proposed by \cite{murguia2020security,hashemi2022codesign,escudero2023safety}, while not focusing on stealthy attacks.
% The works \citep{murguia2020security,hashemi2022codesign,escudero2023safety} develop filtering techniques to guarantee safety, without being focused on stealthy covert attacks.

Multiplicative watermarking (mWM) has been proposed by the authors as a diagnosis technique \citep{ferrari2020switching}. mWM consists of a pair of filters on each communication channel between the plant and its controller; the scheme is affine to weak encryption, whereby ``encoding'' and ``decoding'' are done by changing signals' dynamic characteristics through inverse pairs of filters. This enables original signals to be recovered exactly, and thus does not lead to performance degradation.
% A multiplicative watermark is an affine to a weak encryption technique, through which the signal is ``encoded'' by a filter, changing its dynamic behavior. The use of inverse pairs means that the original signal can be recovered, through ``decoding'' via an inverse filter. As such, differently to techniques based on additive watermarking, no performance is lost due to the injection of noise, and there are no bandwidth limitations.

%\subsection{Contributions}
One of the critical features of multiplicative watermarking is that to detect stealthy attacks, the mWM filter parameters must be switched over time. In this paper, an algorithm to optimally design the mWM parameters after a switching event is presented, enhancing detection performance, without changing the switching time.
% This is done without changing the switching time, which is taken as given.

\textcolor{black}{
To formalize the filter design problem, we suppose the defender is interested in optimal performance against adversaries injecting covert attacks with matched system parameters \citep{smith2015covert}, including the mWM parameters prior to the switch. This scenario represents a worst case where malicious agents can take full control of the system while remaining undetected.
Thus, the attack strategy is explicitly included within the formulation of the closed-loop system, and the mWM filters are chosen by solving an optimization problem minimizing the attack-energy-constrained output-to-output gain (AEC-OOG) \citep{anand2023risk}, a variation of the output-to-output gain proposed in  \cite{teixeira2015strategic}.
}
The main contributions of this paper are:
% We consider an adversary injecting a covert attack with matched system parameters \citep{smith2015covert}, i.e., an attacker with full knowledge of the control system parameters, including those of the mWM filters before the switch. This scenario is taken as a worst case, as it has been shown that this class of attacks can be made stealthy. To quantitatively define a cost, the output-to-output gain (OOG) \citep{teixeira2015strategic} is leveraged,
% a metric introduced to evaluate the impact of an additive attack in a control system. %Specifically, OOG evaluates the worst-case performance loss that an attacker injecting an undetectable attack can obtain. 
% Here, the maximum performance loss caused by a stealthy adversary with limited energy is taken, the attack-energy-constrained OOG (AEC-OOG) \citep{anand2023risk}. The main contributions of this paper are:
\begin{enumerate}
%[label=\alph*.]
\item The problem of optimally designing the switching mWM filters is formulated as an optimization problem, with the AEC-OOG is taken as the objective;%where the AEC-OOG is taken as the impact metric; 
\item The worst-case scenario of a covert attack with exact knowledge of plant and mWM filter parameters is embedded within the design problem;
% The optimization problem is defined to incorporate the worst-case scenario of a covert attack with exact knowledge of plant and mWM filter parameters;
\item The feasibility of the optimization problem is shown to be dependent only on stability conditions; 
\item A solution scheme is proposed to promote randomization of the mWM filter parameters such that an eavesdropping adversary cannot remain stealthy.
\end{enumerate} 

This builds on the results of \cite{ferrari2020switching}, where the focus was on the design of the switching protocols, rather than the parameters themselves.
Compared to previous work \citep{gallo2021design}, this paper introduces an optimization problem which is always feasible (thanks to the use of AEC-OOG in the objective), while also considering a more sophisticated class of covert attacks, where the presence of watermark is known to the adversary. 
Moreover, this paper poses a different objective than \citep{zhang2023hybrid}; indeed, while \citep{zhang2023hybrid} provided a design strategy to ensure certain privacy properties, in this paper we address the problem of optimal parameter design following a switching event.


%\subsection{Organization}
The rest of the paper is organized as follows. 
After formulating the problem in Section~\ref{sec:PF}, we propose our design algorithm in Section~\ref{sec:main}, and analyze its properties. It is then evaluated through a numerical example in Section~\ref{sec:NE}, and concluding remarks are given Section~\ref{sec:Con}.
% We provide the problem background in Section~\ref{sec:PF}. We formulate the design problem in Section~\ref{sec:main}, together with an analysis of its properties. The proposed algorithm is evaluated through a numerical example in Section \ref{sec:NE}. Concluding remarks are offered in Section \ref{sec:Con}.
%\subsection{Comparison with Other Datasets}
\section{Comparison with Existing Datasets}
We start with a brief comparison of existing datasets, emphasizing multi-modal and time-aware strategies (see Table~\ref{comparison} for an overview).

\paragraph{Time-Sensitive QA Datasets}

Time-Sensitive Question Answering (TSQA) involves interpreting and responding to questions that are dependent on specific time points or intervals. We analyse a set of TSQA datasets~\cite{dhingra2022time,chen2021dataset,liska2022streamingqa,tan2023towards}, as shown in the upper part of Table~\ref{comparison}. 
%These datasets challenge a model's ability to handle time by modifying the temporal information within the questions. 
Currently, TSQA datasets typically use free-text form or knowledge graphs (KGs) and are structured as QA tasks. However, our work introduces the first multimodal time-aware dataset based on conversation. Similar to TSQA, we modify the time of dialogues, which affects the responses and the related grounding memory, thereby testing the model's ability to understand time.

\paragraph{MultiModal Dialogue Datasets}

Multimodal dialogue datasets generally comprise one or more images and multi-turn textual dialogues. As depicted in the lower half of Table~\ref{comparison}, we analyse two representative datasets~\cite{zang2021photochat,feng2022mmdialog}. These datasets are designed for models to interpret images and utterances within a dialogue framework and generate coherent responses. Our MTPChat dataset, although drawing on the conversational structure and task, distinctively emphasizes the annotation and manipulation of time information. MTPChat allows the model to acknowledge the influence of temporal dynamics on dialogue interaction and memory processes, demonstrating temporal awareness. 
%This feature marks a significant distinction from other current multimodal dialogue datasets.

\paragraph{Time-Sensitive Video-Centric Dataset}

TimeIT~\cite{ren2023timechat} is a novel dataset focused on video-based instructions, encompassing a collection of long-video datasets annotated with timestamps. It requires models to describe video content across specified time intervals. The description follows a structured format, such as ``<timestamp\_start> to <timestamp\_end> seconds: <event\_description>''. 
%TimeIT aims to enable models to intuitively describe content corresponding to timestamps in videos, testing their temporal sensitivity. 
Ingeniously, our dataset integrates time of dialogues and memories, making model awareness of the time order of dialogue and memory significant influence on dialogue responses and memory recall. In contrast to TimeIT's tasks that directly answer timestamp and associated content, MTPChat offers a more complex challenge with implicit time factor, pushing the boundaries of temporal understanding in multimodal dialogue models.

\section{MTPChat Dataset}

Our dataset is built on the basis of MPChat~\cite{ahn2023mpchat}, a comprehensive multimodal persona-grounded dialogue dataset that includes both linguistic and visual components derived from episodic-memory-based personas. MPChat gathered from the social media platform Reddit, consists of memory image-sentence pairs and dialogue instances grounded on the speakers' multimodal memories. 

A significant challenge is the ingenious integration of time information and multimodal dialogue, aiming to establish a multimodal time-aware dataset. Based on MPChat dataset, we develop a novel methodology that involves three primary steps: 1) Time annotations, 2) Constructing time-aware conversations, and 3) Memory annotations. These efforts achieve the creation of a pioneering multimodal time-aware dialogue dataset. MTPChat breaks away from the limitations of current time-sensitive datasets confined to QA tasks, free-text formats and relying on explicit time information. We believe that our work fosters the development of more diverse time-sensitive datasets and advancing research toward achieving human-level temporal understanding in models.

\subsection{Time Annotations}
%\subsection{Constructing Time Information}%构建时间信息
\label{section:2.1}

We converted the UTC strings in MPChat dataset into date format ``yyyy/mm/dd'' and incorporated this feature into both the dialogue and memory components. The dialogue in our dataset is structured as a triplet consisting of (dialogue context, dialogue image, dialogue time), while each memory of the speaker is similarly organized as a triplet (memory description context, memory image, memory time).

\subsection{Time-Aware Conversations}
%\subsection{Constructing Dialogues Before Grounding Memory Takes Place} 
\label{section:2.2}

In real-world scenarios, conversations can vary significantly based on the time they occur, even with similar contexts. For instance, as a high school student asked, ``What is machine learning?'', you might respond with no knowledge on the subject. However, after three years of studying machine learning at university, your response to the same conversation would be more detailed, potentially including discussions about deep learning and related topics.

Inspired by how the temporal order of conversation and memories influences human responses, we constructed conversational data with temporal orders:
\begin{itemize}
\vspace{-2mm}
\item Later Stage Conversations: We used the original memories and conversations from the MPChat dataset, adding time annotations as described in Section~\ref{section:2.1}. For instance, if you are a university student with three years of study in machine learning and are asked, ``What is machine learning?'', your response might include topics like deep learning.
\vspace{-1mm}
    \item Early Stage Conversations: To simulate conversations from earlier times, we assumed there was no prior memory of the discussion topic. We used the context of the original conversations but removed the original responses. We then add new, earlier time annotations and responses. The newly created responses differ from the original ones and contain minimal information about the discussion topic due to the lack of relevant memory. For example, if you are a high school student asked, ``What is machine learning?'', you might respond with little to no knowledge on the subject.

    Specifically, we utilized GPT-4 \cite{achiam2023gpt} to process a combination of inputs: the dialogue context, dialogue image, newly modified dialogue time, and speaker memories predating this new dialogue time. GPT-4 generated responses under the following guidelines: 1) responses could not exceed 40 words; 2) if the provided memories' topics significantly differed from the conversation, the response should indicate the speaker’s lack of familiarity with the conversations topic; 3) if the provided memories and conversation topics were only slightly different, the response should reflect the speaker's intention to engage with and explore the conversation topic.
\end{itemize}


\subsection{Memory Annotations}

%\subsection{Constructing Grounding Memory After Dialogue Takes Place}
\vspace{-1mm}
To gain a more precise understanding of the model's capabilities in temporal awareness, we align conversations with memory. For the memory component, we add time annotations as outlined in Section~\ref{section:2.1}. Since the memories of the speakers are sourced from real users on Reddit, we avoid creating fabricated memories to preserve data authenticity. Additionally, we incorporate a ``No Memory'' category into the speaker's memory set. Structured similarly to existing memory triplets (memory description context, memory image, memory time), the ``No Memory'' category is assigned as the description context, indicating that there is no memory to align with the response. \footnote{We also correlate ``No Memory'' with a plain white image as the memory image.} This memory category is used to align early-stage conversations. We then synchronize the memory time with the conversation's time information.

\subsection{Dataset Statistics}


MTPChat comprises 18,973 conversations and 25,877 users. We divided MTPChat into training, validation, and test sets with 15,056, 1,994, and 1,923 conversations respectively. We analyzed the proportion of later stage conversations and early stage conversations, finding a ratio of 3:1. As well as later stage conversations with grounding memories (some later stage conversations lack grounding memory) and early stage conversations with ``No Memory'', resulting in a ratio of 2:1. Furthermore, to gain deeper insight into the time information within MTPChat, we charted the distribution of times across conversations and memories in Fig~\ref{data}.


\begin{figure}[t]
\centering
\includegraphics[width=0.85\columnwidth]{data.png}
\vspace{-2mm}
\caption{Distribution of times across conversations and memories in training, validation, and test set.
}
\vspace{-2mm}
\label{data}
\end{figure}






\section{Task Definition}

\begin{figure*}[h]
\centering
  \includegraphics[width=0.95\textwidth]{task.png}
  \vspace{-2mm}
  \caption{Overview of the Temporal Next Response Prediction (TNRP) and Temporal Grounding Memory Prediction (TGMP) tasks. The left panel displays a user’s episodic memories, represented as image-sentence-time triplets with various creation dates. The dialogue instance on the right highlights the corresponding response and task setup. }
    \label{task}
\vspace{-4mm}
 \end{figure*}

%\subsection{Preliminaries}

The MTPChat datasets consist of N examples $\mathcal{D}=\{({d}_n, r_n, \mathcal{M}_n)\}_{n=1}^N$, where \( \forall n \in \{1, \ldots, N\} \) and each example contains a dialogue ${d}_n$, the speaker's response ${r}_n$ to the dialogue ${d}_n$ and a memory set $\mathcal{M}_n$ from the speaker. Each dialogue $d_n=(c^{d_n}, {i}^{d_n}, t^{d_n})$ encompasses the context $c^{d_n}$ (context utterances), an associated image ${i}^{d_n}$ and the time 
marking ${t}^{d_n}$ (formatted as yyyy/mm/dd) when the dialogue occurred. The memory set for the speaker consists of m distinct memories $\mathcal{M}_n=\{M_{n_1},\ldots,M_{n_m}\}$, where each memory $M_{n_m} = (c^{M_{n_m}}, i^{M_{n_m}}, t^{M_{n_m}})$ characterized by a description context $c^{M_{n_m}}$ (context utterances), an image $i^{M_{n_m}}$ and the time marking $t^{M_{n_m}}$ (formatted as yyyy/mm/dd) when the memory occurred.



\subsection{Temporal Next Response Prediction}

As illustrated in the Fig~\ref{task}, Temporal Next Response Prediction (TNRP) is a retrieval task aimed at predicting the next response $\tilde{r}$ from a set $R_c$ containing $C$ response candidates based on the dialogue $d=(c^{d}, {i}^{d}, t^{d})$ and the speaker's memories $\mathcal{M}=\{M_{1}=(c^{M_{1}}, i^{M_{1}}, t^{M_{1}}),\ldots,M_{m}\}$. The response candidate set $R_c$ comprises one ground truth and $C-1$ distractor responses. It is essential to emphasize that, 1) Identical dialogue content and speaker memories can lead to vastly different responses depending on the time of the dialogue. 2) To intensify the task's complexity and underline the temporal factor's significance, our response candidate set includes responses from later-stage dialogue and early-stage dialogue. The remainder of the response candidates are randomly selected from other dialogues. 

%This task makes time a subtle but crucial factor within the dialogue, presenting a challenge in evaluating the model's temporal awareness.


\subsection{Temporal Grounding Memory Prediction}

Temporal Grounding Memory Prediction (TGMP) task is also a retrieval task that requires predicting the most likely memory element from a set $M_c$ containing $C$ memory candidates based on a given dialogue $d=(c^{d}, {i}^{d}, t^{d})$ and a remainder memory set (except grounding memory) before producing a response. The memory candidate set $M_c$ comprises one grounding memory, one ``No Memory'' option and $C-2$ distractor memories randomly selected from other speakers. As shown in Fig~\ref{task}, time variations within the dialogue substantially influence the choice of the grounding memory. Specifically, when the time of the dialogue is later than the time of the grounding memory, suggesting the availability of memory related to the dialogue for supporting the speaker’s response, the model is capable of predicting the grounding memory. Conversely, if the time of the dialogue is earlier than that of the grounding memory, indicating an absence of relevant dialogue memory, the model must predict a ``No Memory'' outcome.

In TGMP task, we deliberately exclude the speaker's response from the input. This decision is based on the consideration that potential responses of early-stage dialogue can vary significantly—ranging from disinterest in the dialogue topic to expressing a desire to learn. These different but reasonable responses could potentially confuse the model to predict grounding memory. The principal objective of the TGMP task is making model recognize the critical temporal order between dialogue and memory. %rather than to match the content of the response with memory. 
By focusing on whether the model can identify the appropriate grounding memory or its absence for a given time information, we obtain a clearer measure of its temporal awareness capabilities.
\begin{figure}[t]
\centering
\includegraphics[width=0.9\columnwidth]{model.png}
\vspace{-2mm}
\caption{Architecture of our framework with Adaptive Temporal Module (ATM).
}
\label{model}
\vspace{-4mm}
\end{figure}
\vspace{-2mm}
\section{Framework}
In this section, we present a framework to perform above retrieval tasks based on dialogue and memory. The inputs include dialogue ${d}_n$, the speaker's response ${r}_n$ to the dialogue and a memory set $\mathcal{M}_n$. We define various encoders to process different modalities of data, fuse the extracted features, and achieve both the temporal next response prediction task and the temporal grounding memory prediction task. The architecture of our framework is shown in Fig~\ref{model}.

\textbf{Text Encoder}
In this study, we employ the text encoder to process textual components within tasks, specifically extracting representations of text and date information from dialogues, memories, and responses. For both dialogue and speaker memories, which may contain multiple entries, we first concatenate the text and date information for each entry. These concatenated strings are then further combined using a delimiter, forming unified representations. This method ensures comprehensive feature extraction by the text encoder, facilitating a more robust analysis of the textual data involved.

\textbf{Vision Encoder}
%\vspace{-1mm}
Similarly, our vision encoder to extract features from images embedded in dialogues and memories. In datasets featuring speaker memories with multiple images, each image is processed by this vision encoder. The extracted features are then aggregated via mean-pool operation to create a consolidated visual representation. This methodology ensures a coherent integration of visual data, significantly enhancing the model's capacity to process multi-image features effectively.

\textbf{Adaptive Temporal Module}
Following the extraction of textual and visual representations, it is essential to effectively integrate these features. As the inclusion of date information into textual representations can impact the correspondence between the text and vision features extracted by text encoder and vision encoder, we propose a method to dynamically balance these modalities to maintain the alignment of text and visual information within the same set of memories and dialogues. We introduce a module called the Adaptive Temporal Module (ATM), which is designed to be both simple and effective.

First, we concatenate the corresponding text and vision features and map them through a linear layer. Subsequently, a sigmoid layer is used to derive the weights for both text and vision features. These weights are then employed to merge the features based on their relevance, ensuring better alignment and integration. This approach allows for a more coherent and contextually appropriate fusion of multimodal features, enhancing the overall interpretative capability of the model.

In this section, we empirically compare the proposed algorithm on both sequence windows and time windows with existing methods.
\paragraph{Datasets} For the sequence-based model, we used two synthetic datasets and two cross-language datasets. The statistics of the datasets are provided in Table \ref{table:statistics}:

\begin{table}[t]
    \centering
    \caption{The statistics of the datasets. The datasets satisfy $1 \leq \|\vx\|\|\vy\| \leq R $.}
    \label{table:statistics}
    \begin{tabular}{|c|c|c|c|c|c|}
    \hline
        Dataset & $n$ & $m_x$ & $m_y$ & $N$ & $R$ \\ \hline
        SYNTHETIC(1) & 100,000 & 1,000 & 2,000 & 50,000 & 65 \\ \hline
        SYNTHETIC(2) & 100,000 & 1,000 & 2,000 & 50,000 & 724 \\ \hline
        APR & 23,235 & 28,017 & 42,833 & 10,000 & 773 \\ \hline
        PAN11 & 88,977 & 5,121 & 9,959 & 10,000 & 5,548 \\ \hline
        EURO & 475,834 & 7,247 & 8,768 & 100,000 & 107,840 \\ \hline
    \end{tabular}
\end{table}

\begin{itemize}
    \item Synthetic: The elements of the two synthetic datasets are initially uniformly sampled from the range (0,1), then multiplied by a coefficient to adjust the maximum column squared norm $R$. The X matrix has 1,000 rows, and the Y matrix has 2,000 rows, each with 100,000 columns. The window size is set to 50,000.
    \item APR: The Amazon Product Reviews (APR) dataset is a publicly available collection containing product reviews and related information from the Amazon website. This dataset consists of millions of sentences in both English and French. We structured it into a review matrix where the X matrix has 28,017 rows, and the Y matrix has 42,833 rows, with both matrices sharing 23,235 columns. The window size is 10,000.
    \item PAN11: PANPC-11 (PAN11) is a dataset designed for text analysis, particularly for tasks such as plagiarism detection, author identification, and near-duplicate detection. The dataset includes texts in English and French. The X and Y matrices contain 5,121 and 9,959 rows, respectively, with both matrices having 88,977 columns. The window size is 10,000.
\end{itemize}
We evaluate the time-based model on another real-world dataset:
\begin{itemize}
    \item EURO: The Europarl (EURO) dataset is a widely used multilingual parallel corpus, comprising the proceedings of the European Parliament. We selected a subset of its English and French text portions. The X and Y matrices contain 7,247 and 8,768 rows, respectively, and both matrices share 475,834 columns. Timestamps are generated using the $Poisson$ $Arrival$ $Process$ with a rate parameter of $\lambda=2$. The window size is set to 100,000, with approximately 30,000 columns of data on average in each window.
\end{itemize}

\paragraph{Setup} For the sequence-based model, we compare the proposed hDS-COD and  aDS-COD with EH-COD~\cite{yao2024approximate} and DI-COD~\cite{yao2024approximate}. We do not consider the Sampling algorithm as a baseline, as its performance is inferior to that of EH-COD and DI-CID, as demonstrated in \cite{yao2024approximate}. %The hDS-COD is adjusted by the parameter $\ell$ and the maximum number of levels $L = \log{R}$, where $R$ is the prior estimate of the maximum squared column norm of the dataset. DI-COD similarly requires a prior estimate of $R$ to limit the maximum number of levels $L = \log{(R/\varepsilon})$. In contrast, aDS-COD and EH-COD do not require an estimate of $R$; their error-space balance is controlled by the parameter $\ell = \frac{1}{\varepsilon}$. 
For the time-based model, we compare the proposed hDS-COD and  aDS-COD with EH-COD and the Sampling algorithm since DI-COD cannot be applied to time-based sliding window model. To achieve the same error bound, the maximum number of levels for hDS-COD is set to $L = \log{(\varepsilon NR)}$, and the initial threshold for aDS-COD is set to $1$.

Our experiments aim to illustrate the trade-offs between space and approximation errors. The x-axis represents two metrics for space: final sketch size and total space cost. The final sketch size refers to the number of columns in the result sketches $\mA$ and $\mB$ generated by the algorithm, representing a compression ratio. The total space cost refers to the maximum space required during the algorithm's execution, measured by the number of columns.We evaluate the approximation performance of all algorithms based on correlation errors $\operatorname{corr-err}(\mathbf{X}_W \mathbf{Y}_W^\top, \mathbf{A} \mathbf{B}^\top)$, which is reflected on the y-axis. Every 1,000 iterations, all algorithms query the window and record the average and maximum errors across all sampled windows.

The experiments for all algorithms were conducted using MATLAB (R2023a), with all algorithms running on a Windows server equipped with 32GB of memory and a single processor of Intel i9-13900K.

\paragraph{Performance} Figure \ref{fig:error vs l} and Figure \ref{fig:error vs space} illustrate the space efficiency comparison of the algorithms on sequence-based datasets. Panels (a-d) show the average errors across all sampled windows, while panels (e-h) display the maximum errors.

Figure \ref{fig:error vs l} evaluates the compression effect of the final sketch. The hDS-COD, aDS-COD, and EH-COD show similar compression performances. But the DS series is more stable, particularly on the synthetic datasets, where they significantly outperform EH-COD and DI-COD. The performance of hDS-COD and aDS-COD is nearly the same, indicating that the adaptive threshold trick in aDS-COD does not have a noticeable negative impact on it, maintaining the same error as hDS-COD.

Figure \ref{fig:error vs space} measures the total space cost of the algorithms. hDS-COD and aDS-COD show a significant advantage over existing methods, as they can achieve the  $\varepsilon$-approximation error with much less space. For the same space cost, the correlation errors of hDS-COD and aDS-COD are much smaller than those of EH-COD and DI-COD. Also, aDS-COD has better space efficiency than hDS-COD because aDS only uses a single-level structure while hDS requires $\log R+1$ levels. We find that hDS-COD requires more space on  SYNTHETIC(2) dataset compared to SYNTHETIC(1) dataset. This phenomenon occurs because SYNTHETIC(2) dataset has a larger $R$, which confirms the dependence on $R$ as stated in Theorem~\ref{thm:hds}. 

Figure \ref{fig:time-based} compares the performance of algorithms on time-based windows. Panels (a) and (b) present the error against the final sketch size, which show that our aDS-COD and hDS-COD algorithms enjoy similar performance as EH-COD and significantly outperform the sampling algorithm. On the other hand, as shown in panels (c) and (d), our methods outperform baselines in terms of total space cost.

\section{Related Work}
\subsection{Multimodal Large Language Models}
% Building on the success of large language models (LLMs) \citep{yao2024tree, glm2024chatglm, achiam2023gpt, touvron2023llama, brown2020language}, multimodal large language models (MLLMs) \citep{liu2024improved, li2023blip, zhu2023minigpt, wang2023cogvlm, liu2024visual} extend these capabilities by integrating vision and text processing, achieving remarkable performance in tasks involving images, videos, and multimodal reasoning. However, handling visual data poses computational challenges due to the redundancy and low information density of high-resolution tokens \citep{liang2022evit} and the quadratic scaling of attention mechanisms \citep{vaswani2017attention}.
% For instance, models like LLaVA \citep{liu2023improvedllava} and mini-Gemini-HD \citep{li2024mini} encode high-resolution images into thousands of tokens, while video-based models such as VideoLLaVA \citep{lin2023video} and VideoPoet \citep{kondratyuk2023videopoet} allocate even more tokens to process multiple frames. These challenges highlight the need for more efficient token representations and longer context lengths to enable scalability. Recent advancements, such as Gemini \citep{geminiteam2023gemini} and LWM \citep{liu2024world}, have focused on addressing these issues by optimizing token efficiency and extending the context length, paving the way for more scalable and effective MLLMs.

The remarkable success of large language models (LLMs) \citep{radford2019language, brown2020language} has spurred a growing trend of extending their advanced reasoning capabilities to multi-modal tasks, leading to the development of vision-language models (VLMs) \citep{huang2023languageneedaligningperception, driess2023palmeembodiedmultimodallanguage, liu2024visual, Qwen-VL}. These VLMs typically consist of a visual encoder \citep{radford2021learning} that serializes input image representations and an LLM responsible for text generation. To enable the LLM to process visual inputs, an alignment module is employed to bridge the gap between visual and textual modalities. This module can take various forms, such as a simple linear layer, an MLP projector, or a more complex query-based network. While this integration allows the LLM to gain visual perception, it also introduces significant computational challenges due to the long sequences of visual tokens.

Moreover, existing VLMs often exhibit limitations, such as visual shortcomings or hallucinations, which hinder their performance. Efforts to enhance VLM capabilities by increasing input image resolution have further exacerbated computational demands. For instance, encoding higher-resolution images results in a substantial increase in the number of visual tokens. A model like LLaVA-1.5 \citep{liu2024improved} generates 576 visual tokens for a single image, while its successor, LLaVA-NeXT \citep{liu2024llavanext}, produces up to 2880 tokens at double the resolution, far exceeding the length of typical textual prompts.
Optimizing the inference efficiency of VLMs is thus a critical task to facilitate their deployment in real-world scenarios with limited computational resources.

\subsection{Visual Token Compression}
% Visual tokens often exceed text tokens by tens to hundreds of times, with visual signals being more spatially redundant compared to information dense text \citep{marr2010vision}.
% Various methods have been proposed to address this issue. For instance, LLaMA-VID \citep{li2023llama} uses a Q-Former with context tokens, and DeCo \citep{yao2024deco} applies adaptive pooling to downsample visual tokens at the patch level.
% However, these approaches require modifying model components and additional training, increasing computational and training costs.
% ToMe~\citep{bolya2022tome} reduces tokens without training by adding a token merge module to ViTs, but this disrupts early cross-modal interactions in language models~\citep{xing2024PyramidDrop}. FastV~\citep{chen2024image} selects important visual tokens using attention scores, while SparseVLM~\citep{zhang2024sparsevlm} incorporates text guidance via cross-modal attention.
% However, these methods forgo flash-attention~\citep{dao2022flashattention, dao2023flashattention2} and primarily focus on token importance, overlooking the impact of token duplication.
% In our work, we preserve hardware acceleration compatibility, including flash attention, while considering both token importance and duplication for token reduction.

Visual tokens are often significantly more numerous than text tokens, with higher spatial redundancy and lower information density. To address this issue, various methods have been proposed for reducing visual token counts in vision language models. For instance, some approaches modify model components, such as using context tokens in Q-Former \citep{li2023llama} or applying adaptive pooling at the patch level, but these typically require additional training and increase computational costs. Other techniques, like Token Merging (ToMe) \citep{bolya2022tome} and FastV \citep{chen2024image}, focus on reducing tokens without retraining by merging tokens or selecting important ones based on attention scores. SparseVLM \cite{zhang2024sparsevlm} incorporates text guidance through cross-modal attention to refine token selection. However, these methods often overlook hardware acceleration compatibility and fail to account for token duplication alongside token importance. Furthermore, while token pruning has been extensively explored in natural language processing and computer vision to improve inference efficiency, its application to VLMs remains under-explored. Existing pruning strategies, such as those in FastV and SparseVLM, rely on text-visual attention within large language models (LLMs) to evaluate token importance, which may not align well with actual visual token relevance.


Software development is increasingly conceived as a collaboration activity between developers and AIs. Indeed, IDEs already implement features to enable interactive development, with AI suggesting implementations that are reused by developers.

Although multiple studies show this interaction can be successful, there is still limited understanding of how the models must be configured and used in the context of code generation tasks. This study addresses this gap, systematically investigating the impact of several key parameters, including the repeated submission of a prompt to accommodate for the non-deterministic nature of the models.

Our study reveals several key findings about the usage of ChatGPT. In particular, we discovered how creativity, although up to a limited extent, is useful to increase the range of methods whose code can be generated correctly. A major role is played by parameter top-p, which is commonly underrated, and instead has a major impact on the correctness of the results, with lower values producing better results. Finally, prompts should be submitted multiple times, with $5$ repetitions combined with a temperature of $1.2$ resulting in an effective configuration in our experiments.  

Future work concerns two main research directions. One is about replicating this experiment with other AI assistants, to validate our findings in multiple contexts. The second research direction concerns finding strategies to deal with the need to submit the same prompt multiple times to obtain a useful result, and thus developing approaches able to select or merge multiple responses automatically. 



% Bibliography entries for the entire Anthology, followed by custom entries
%\bibliography{anthology,custom}
% Custom bibliography entries only
\bibliography{custom}


\clearpage
\begin{center}\large\bfseries
Appendix
\end{center}

\appendix

\section{Detailed Prompt of GPT-4}
\label{sec:appendix1}

\begin{table}[h]

\centering
\resizebox{\columnwidth}{!}{
\begin{tabular}{l}
\toprule
\textbf{Prompt of GPT-4 for generating response to early-stage conversation}  \\
\midrule
Given the topic of a conversation, the context of the dialogue, and multiple memories \\of the speaker, please write a response to the conversation. \\
 \\
It is important to note:\\
1. responses could not exceed 40 words.\\
2. If the memories are almost unrelated to the conversation, the generated response \\should reflect the speaker's lack of expertise in the conversation topic. \\If appropriate, consider incorporating the current content of the speaker's memories. \\
3. If the memories are related to the conversation, the response should express \\a willingness to try or explore it in the future. \\
 \\
Conversation Topic: [topic]\\
Dialogue Context: [context]\\
Memories: [context]\\
\bottomrule
\end{tabular} }
\caption{\label{Parameters1} Detailed prompt of GPT-4 for generating response to early-stage conversation.}
\end{table}

\section{Detailed Parameters}
\label{sec:appendix2}
The parameter settings of Temporal Next Response Prediction (TNRP) and Temporal Grounding Memory Prediction (TGMP) tasks used in our paper are illustrated in Table~\ref{Parameters1}.


\begin{table}[h]

\centering
\resizebox{0.8\columnwidth}{!}{
\begin{tabular}{l|c|c}
\toprule
\textbf{Parameters} & \textbf{TNRP} & \textbf{TGMP} \\
\midrule
per\_gpu\_train\_batch\_size & 8 & 8 \\ 
per\_gpu\_eval\_batch\_size& 1 & 4  \\
num\_train\_epoch& 5 & 5  \\
max\_num\_candidates& 100 & 100   \\
max\_num\_image& 20 & 20  \\
image\_size &224 & 224 \\
learning\_rate& $3e^{-6}$ & $3e^{-6}$\\
weight\_decay &0.05 &0.05    \\
\bottomrule
\end{tabular} }
\caption{\label{Parameters1} Detailed Parameters of Temporal Next Response Prediction (TNRP) and Temporal Grounding Memory Prediction (TGMP) tasks.}
\end{table}

\end{document}
