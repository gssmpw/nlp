\section{Task Definition}

\begin{figure*}[h]
\centering
  \includegraphics[width=0.95\textwidth]{task.png}
  \vspace{-2mm}
  \caption{Overview of the Temporal Next Response Prediction (TNRP) and Temporal Grounding Memory Prediction (TGMP) tasks. The left panel displays a user’s episodic memories, represented as image-sentence-time triplets with various creation dates. The dialogue instance on the right highlights the corresponding response and task setup. }
    \label{task}
\vspace{-4mm}
 \end{figure*}

%\subsection{Preliminaries}

The MTPChat datasets consist of N examples $\mathcal{D}=\{({d}_n, r_n, \mathcal{M}_n)\}_{n=1}^N$, where \( \forall n \in \{1, \ldots, N\} \) and each example contains a dialogue ${d}_n$, the speaker's response ${r}_n$ to the dialogue ${d}_n$ and a memory set $\mathcal{M}_n$ from the speaker. Each dialogue $d_n=(c^{d_n}, {i}^{d_n}, t^{d_n})$ encompasses the context $c^{d_n}$ (context utterances), an associated image ${i}^{d_n}$ and the time 
marking ${t}^{d_n}$ (formatted as yyyy/mm/dd) when the dialogue occurred. The memory set for the speaker consists of m distinct memories $\mathcal{M}_n=\{M_{n_1},\ldots,M_{n_m}\}$, where each memory $M_{n_m} = (c^{M_{n_m}}, i^{M_{n_m}}, t^{M_{n_m}})$ characterized by a description context $c^{M_{n_m}}$ (context utterances), an image $i^{M_{n_m}}$ and the time marking $t^{M_{n_m}}$ (formatted as yyyy/mm/dd) when the memory occurred.



\subsection{Temporal Next Response Prediction}

As illustrated in the Fig~\ref{task}, Temporal Next Response Prediction (TNRP) is a retrieval task aimed at predicting the next response $\tilde{r}$ from a set $R_c$ containing $C$ response candidates based on the dialogue $d=(c^{d}, {i}^{d}, t^{d})$ and the speaker's memories $\mathcal{M}=\{M_{1}=(c^{M_{1}}, i^{M_{1}}, t^{M_{1}}),\ldots,M_{m}\}$. The response candidate set $R_c$ comprises one ground truth and $C-1$ distractor responses. It is essential to emphasize that, 1) Identical dialogue content and speaker memories can lead to vastly different responses depending on the time of the dialogue. 2) To intensify the task's complexity and underline the temporal factor's significance, our response candidate set includes responses from later-stage dialogue and early-stage dialogue. The remainder of the response candidates are randomly selected from other dialogues. 

%This task makes time a subtle but crucial factor within the dialogue, presenting a challenge in evaluating the model's temporal awareness.


\subsection{Temporal Grounding Memory Prediction}

Temporal Grounding Memory Prediction (TGMP) task is also a retrieval task that requires predicting the most likely memory element from a set $M_c$ containing $C$ memory candidates based on a given dialogue $d=(c^{d}, {i}^{d}, t^{d})$ and a remainder memory set (except grounding memory) before producing a response. The memory candidate set $M_c$ comprises one grounding memory, one ``No Memory'' option and $C-2$ distractor memories randomly selected from other speakers. As shown in Fig~\ref{task}, time variations within the dialogue substantially influence the choice of the grounding memory. Specifically, when the time of the dialogue is later than the time of the grounding memory, suggesting the availability of memory related to the dialogue for supporting the speaker’s response, the model is capable of predicting the grounding memory. Conversely, if the time of the dialogue is earlier than that of the grounding memory, indicating an absence of relevant dialogue memory, the model must predict a ``No Memory'' outcome.

In TGMP task, we deliberately exclude the speaker's response from the input. This decision is based on the consideration that potential responses of early-stage dialogue can vary significantly—ranging from disinterest in the dialogue topic to expressing a desire to learn. These different but reasonable responses could potentially confuse the model to predict grounding memory. The principal objective of the TGMP task is making model recognize the critical temporal order between dialogue and memory. %rather than to match the content of the response with memory. 
By focusing on whether the model can identify the appropriate grounding memory or its absence for a given time information, we obtain a clearer measure of its temporal awareness capabilities.