\section{Related Work}
Previous work suggested that TTRPG dialogues provide an appropriate challenge for artificial intelligence____.
Datasets of \textit{written} TTRPGs conversations____ have been applied to different tasks, such as text generating____ and character understanding____.

We extend the application of TTRPG dialogues from the written to the audio domain. Audio processing is done, for example, in automatic speech recognition____, voice-based writing____, or diarization____ which we focus on.

Previous work showed that people can spoof speaker identification models by changing their voice____ and diarization models have poor performance for dialogues in which people pretend to be someone else____.
While TTRPG players usually do not try to mimick an existing person, many do change their voices while speaking in-character, thus posing a naturalistic challenge for diarization. %