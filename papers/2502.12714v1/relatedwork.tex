\section{Related Work}
Previous work suggested that TTRPG dialogues provide an appropriate challenge for artificial intelligence~\cite{ellis2017computers,martin2018dungeons}.
Datasets of \textit{written} TTRPGs conversations~\cite{callison2022dungeons,louis2018deep,rameshkumar2020storytelling} have been applied to different tasks, such as text generating~\cite{callison2022dungeons,newman2022generating,si2021telling} and character understanding~\cite{louis2018deep}.

We extend the application of TTRPG dialogues from the written to the audio domain. Audio processing is done, for example, in automatic speech recognition~\cite{huang2023mclf}, voice-based writing~\cite{goswami2023weakly}, or diarization~\cite{qamar2023speaking,qasemi2021paco} which we focus on.

Previous work showed that people can spoof speaker identification models by changing their voice~\cite{lau2004vulnerability,lau2005testing} and diarization models have poor performance for dialogues in which people pretend to be someone else~\cite{medaramitta2021evaluating}.
While TTRPG players usually do not try to mimick an existing person, many do change their voices while speaking in-character, thus posing a naturalistic challenge for diarization. %