%===============================================================================
% ifacconf.tex 2022-02-11 jpuente  
% 2022-11-11 jpuente change length of abstract
% Template for IFAC meeting papers
% Copyright (c) 2022 International Federation of Automatic Control
%===============================================================================
\documentclass{ifacconf}
\usepackage{natbib}        % required for bibliography
%===============================================================================
% Added packages
\usepackage{float}
\usepackage{graphicx}
\usepackage{amsfonts,amsmath}
\usepackage{multirow}
\usepackage{mathtools}
\usepackage[most]{tcolorbox}
\usepackage{xcolor}
\usepackage{adjustbox}
\usepackage{algorithm}
% \usepackage{caption}
% \usepackage{subcaption}
\usepackage{enumitem}
\usepackage{matlab-prettifier}
\usepackage{courier}
\usepackage{url}
%\usepackage{hyperref}
\usepackage{siunitx}
\usepackage{tabularx}
% \usepackage{booktabs}

%===============================================================================

% Custom commands
\newcommand{\com}[1]{\textcolor{pink}{#1}}
\newcommand{\addcite}{\textcolor{red}{CITE }}
\newcommand{\nx}{n_\mathrm{x}}
\newcommand{\dnu}{n_\mathrm{u}}
\newcommand{\ny}{n_\mathrm{y}}
\newcommand{\np}{n_\mathrm{p}}
\newcommand{\diff}{\xi}
\newcommand{\R}{\mathbb{R}}
\newcommand{\Pset}{\mathcal{P}}
\newcommand*\crule[3]{%
  {\color[rgb]{#1}\rule[0.4ex]{#2}{#3}}}

\DeclarePairedDelimiterX{\norm}[1]{\lVert}{\rVert}{#1}


\lstset{%
    style = Matlab-editor,
    linewidth = \columnwidth,
    xleftmargin = 0pt
}
%===============================================================================

\begin{document}
\begin{frontmatter}

    \title{Automated Linear Parameter-Varying Modeling of Nonlinear Systems: A Global Embedding Approach\thanksref{footnoteinfo}}
    % Title, preferably not more than 10 words.

    \thanks[footnoteinfo]{This research was supported by the European Space Agency (grant number: 4000145530) and The MathWorks Inc. Opinions, findings, conclusions or recommendations expressed in this abstract are those of the authors and do not necessarily reflect the views of The MathWorks Inc. or the European Space Agency.}

    \author[TUE]{E. Javier Olucha}
    \author[TUE]{Patrick J. W. Koelewijn}
    \author[TUE]{Amritam Das}
    \author[TUE, HUNREN]{Roland Tóth}

    \address[TUE]{Control Systems Group, Eindhoven University of Technology, Eindhoven, The Netherlands \\(e-mail: \{e.j.olucha.delgado, am.das, r.toth\}@tue.nl).}
    \address[ HUNREN]{Systems and Control Laboratory, HUN-REN Institute for Computer Science and Control, Budapest, Hungary}
        %\address[Second]{Independent researcher (e-mail: research@patrickkoelewijn.nl)}

    \begin{abstract}                % Abstract of 50--100 words
        In this paper, an automated \emph{linear parameter-varying} (LPV) model conversion approach is proposed for nonlinear dynamical systems. The proposed method achieves global embedding of the original nonlinear behavior of the system by leveraging the second fundamental theorem of calculus to factorize matrix function expressions without any approximation. The implementation of the proposed method in the \textsc{LPVcore} toolbox for \textsc{Matlab} is discussed, %The paper contains an overview of the conversion method and the corresponding functions implemented in the toolbox. 
        %Moreover, a 
        and its performance is showcased on a comprehensive example of automated LPV model conversion of an unbalanced disk system, which is then used to design an LPV controller that is deployed on the original nonlinear system. In addition, the conversion capabilities are further demonstrated by obtaining an LPV embedding of a three-degree-of-freedom control moment gyroscope. All software implementations are available at \url{www.lpvcore.net}.
    \end{abstract}

    \begin{keyword}
        Model conversion, LPV embedding, LPV systems, nonlinear systems.
    \end{keyword}

\end{frontmatter}
%===============================================================================

\section{Introduction}


\begin{figure}[t]
\centering
\includegraphics[width=0.6\columnwidth]{figures/evaluation_desiderata_V5.pdf}
\vspace{-0.5cm}
\caption{\systemName is a platform for conducting realistic evaluations of code LLMs, collecting human preferences of coding models with real users, real tasks, and in realistic environments, aimed at addressing the limitations of existing evaluations.
}
\label{fig:motivation}
\end{figure}

\begin{figure*}[t]
\centering
\includegraphics[width=\textwidth]{figures/system_design_v2.png}
\caption{We introduce \systemName, a VSCode extension to collect human preferences of code directly in a developer's IDE. \systemName enables developers to use code completions from various models. The system comprises a) the interface in the user's IDE which presents paired completions to users (left), b) a sampling strategy that picks model pairs to reduce latency (right, top), and c) a prompting scheme that allows diverse LLMs to perform code completions with high fidelity.
Users can select between the top completion (green box) using \texttt{tab} or the bottom completion (blue box) using \texttt{shift+tab}.}
\label{fig:overview}
\end{figure*}

As model capabilities improve, large language models (LLMs) are increasingly integrated into user environments and workflows.
For example, software developers code with AI in integrated developer environments (IDEs)~\citep{peng2023impact}, doctors rely on notes generated through ambient listening~\citep{oberst2024science}, and lawyers consider case evidence identified by electronic discovery systems~\citep{yang2024beyond}.
Increasing deployment of models in productivity tools demands evaluation that more closely reflects real-world circumstances~\citep{hutchinson2022evaluation, saxon2024benchmarks, kapoor2024ai}.
While newer benchmarks and live platforms incorporate human feedback to capture real-world usage, they almost exclusively focus on evaluating LLMs in chat conversations~\citep{zheng2023judging,dubois2023alpacafarm,chiang2024chatbot, kirk2024the}.
Model evaluation must move beyond chat-based interactions and into specialized user environments.



 

In this work, we focus on evaluating LLM-based coding assistants. 
Despite the popularity of these tools---millions of developers use Github Copilot~\citep{Copilot}---existing
evaluations of the coding capabilities of new models exhibit multiple limitations (Figure~\ref{fig:motivation}, bottom).
Traditional ML benchmarks evaluate LLM capabilities by measuring how well a model can complete static, interview-style coding tasks~\citep{chen2021evaluating,austin2021program,jain2024livecodebench, white2024livebench} and lack \emph{real users}. 
User studies recruit real users to evaluate the effectiveness of LLMs as coding assistants, but are often limited to simple programming tasks as opposed to \emph{real tasks}~\citep{vaithilingam2022expectation,ross2023programmer, mozannar2024realhumaneval}.
Recent efforts to collect human feedback such as Chatbot Arena~\citep{chiang2024chatbot} are still removed from a \emph{realistic environment}, resulting in users and data that deviate from typical software development processes.
We introduce \systemName to address these limitations (Figure~\ref{fig:motivation}, top), and we describe our three main contributions below.


\textbf{We deploy \systemName in-the-wild to collect human preferences on code.} 
\systemName is a Visual Studio Code extension, collecting preferences directly in a developer's IDE within their actual workflow (Figure~\ref{fig:overview}).
\systemName provides developers with code completions, akin to the type of support provided by Github Copilot~\citep{Copilot}. 
Over the past 3 months, \systemName has served over~\completions suggestions from 10 state-of-the-art LLMs, 
gathering \sampleCount~votes from \userCount~users.
To collect user preferences,
\systemName presents a novel interface that shows users paired code completions from two different LLMs, which are determined based on a sampling strategy that aims to 
mitigate latency while preserving coverage across model comparisons.
Additionally, we devise a prompting scheme that allows a diverse set of models to perform code completions with high fidelity.
See Section~\ref{sec:system} and Section~\ref{sec:deployment} for details about system design and deployment respectively.



\textbf{We construct a leaderboard of user preferences and find notable differences from existing static benchmarks and human preference leaderboards.}
In general, we observe that smaller models seem to overperform in static benchmarks compared to our leaderboard, while performance among larger models is mixed (Section~\ref{sec:leaderboard_calculation}).
We attribute these differences to the fact that \systemName is exposed to users and tasks that differ drastically from code evaluations in the past. 
Our data spans 103 programming languages and 24 natural languages as well as a variety of real-world applications and code structures, while static benchmarks tend to focus on a specific programming and natural language and task (e.g. coding competition problems).
Additionally, while all of \systemName interactions contain code contexts and the majority involve infilling tasks, a much smaller fraction of Chatbot Arena's coding tasks contain code context, with infilling tasks appearing even more rarely. 
We analyze our data in depth in Section~\ref{subsec:comparison}.



\textbf{We derive new insights into user preferences of code by analyzing \systemName's diverse and distinct data distribution.}
We compare user preferences across different stratifications of input data (e.g., common versus rare languages) and observe which affect observed preferences most (Section~\ref{sec:analysis}).
For example, while user preferences stay relatively consistent across various programming languages, they differ drastically between different task categories (e.g. frontend/backend versus algorithm design).
We also observe variations in user preference due to different features related to code structure 
(e.g., context length and completion patterns).
We open-source \systemName and release a curated subset of code contexts.
Altogether, our results highlight the necessity of model evaluation in realistic and domain-specific settings.






\section{Problem definition\label{sec:probdef}}
%\com{I just wrote an initial problem definition as a starting point. Probably overcomplicated. I did not explicitly state that this problem is describing a global embedding, but I think it should be specified. Also, for simplicity, I formulated the problem of embedding the system for the complete $\mathcal{X} \times \mathcal{U}$ space, instead of for a region of it.}

Consider a nonlinear dynamical system given by the \emph{state-space} (SS) representation 
\begin{subequations}
	\label{eq:nl_dyn}
\begin{align}
	%\begin{aligned}
		\diff x(t) & = f(x(t), u(t)), \\
		y(t)       & = h(x(t), u(t)),
	%\end{aligned}
\end{align}
\end{subequations}
where $t \in \mathbb{T}$ is time, $\diff$ is $\diff x(t)= \frac{d}{dt}x(t)$ in the continuous-time case with $\mathbb{T}=\mathbb{R}$ and $\diff x(t)= x(t+1)$ in the discrete-time case with $\mathbb{T}=\mathbb{Z}$, $x(t) \in \mathcal{X} \subseteq \R^{\nx}$, $u(t) \in \mathcal{U} \subseteq \R^{\dnu}$ and $y(t) \in \mathcal{Y} \subseteq \R^{\ny}$ with $\nx, \dnu, \ny \in \mathbb{N}$ are the state, input, and output signals associated with the system, respectively. The functions $f: \R^{\nx} \times \R^{\dnu} \to \R^{\nx}$ and $h: \R^{\nx} \to \R^{\ny}$ are continuously differentiable once, i.e., $f, h \in \mathcal{C}_1$, and the solutions of \eqref{eq:nl_dyn} have left compact support, they are forward complete and unique, i.e., for any initial condition $x(t_0)$ and any input trajectory $u(t)$, the solutions of \eqref{eq:nl_dyn} are uniquely determined for all $t \geq t_0$. Moreover, $(0, 0) \in \mathcal{X} \times \mathcal{U}$. Finally we denote the behavior of \eqref{eq:nl_dyn}, i.e., the set of all possible trajectories by
%
\begin{equation}
	\label{eq:nl_behavior}
	\mathcal{B} \coloneq \left\{ (x, u, y) \in \left(\mathcal{X}, \mathcal{U}, \mathcal{Y}\right)^{\mathbb{T}} \mid (x, u, y) ~ \text{satisfy}~\eqref{eq:nl_dyn} \right\},
\end{equation}
%
where $\mathcal{X}^{\mathbb{T}}$ denotes the set of all signals $\mathbb{T} \rightarrow \mathcal{X}$ with left compact support.

In this paper, our objective is to automatically convert the system description \eqref{eq:nl_dyn} into a \emph{linear parameter-varying} (LPV) representation of the form
%
\begin{subequations}
	\label{eq:lpv_dyn}
	\begin{align}
		\diff x(t) & = A(p(t))x(t) + B(p(t))u(t); \\
		y(t)       & = C(p(t))x(t) + D(p(t))u(t),
	\end{align}
\end{subequations}
%
where $p \in \Pset \subseteq \R^{\np}$ with $\np \in \mathbb{N}$ is the scheduling variable, $A : \Pset \rightarrow \R^{\nx \times \nx}$, $B : \Pset \rightarrow \R^{\nx \times \dnu}$, $C : \Pset \rightarrow \R^{\ny \times \nx}$, and $D : \Pset \rightarrow \R^{\ny \times \dnu}$ are smooth real-valued matrix functions, and the solutions of \eqref{eq:lpv_dyn} with left compact support are forward complete and unique. For a given scheduling trajectory $p(t)\in\mathcal{P}$, the behavior of \eqref{eq:lpv_dyn} is defined as
%
\begin{equation}
	\label{eq:p_behavior}
	\mathcal{B}_p \coloneq \left\{ (x, u, y) \in \left(\mathcal{X}, \mathcal{U}, \mathcal{Y}\right)^{\mathbb{T}} \mid (x, u, y, p) ~ \text{satisfy}~\eqref{eq:lpv_dyn} \right\},
\end{equation}
%
and the behavior of \eqref{eq:lpv_dyn} for all possible scheduling trajectories is defined as
%
\begin{equation}
	\label{eq:lpv_behavior}
	\mathcal{B}_{\text{LPV}} \coloneq \bigcup_{p \in \Pset^{\mathbb{T}}} \mathcal{B}_p(p).
\end{equation}


We consider the LPV representation \eqref{eq:lpv_dyn} to be a so-called global LPV embedding of \eqref{eq:nl_dyn} if in addition, we can construct a so-called \emph{scheduling map} $\eta : \mathcal{X} \times \mathcal{U} \rightarrow \Pset$,
%
\begin{equation}
	\label{eq:scheduling_map}
	p(t) = \eta(x(t), u(t)),
\end{equation}
such that
%
\begin{subequations}
\label{eq:LPV_realization}
	\begin{align}
		f(x, u) & = A(\eta(x, u)) x + B(\eta(x, u)) u, \\
		h(x, u) & = C(\eta(x, u)) x + D(\eta(x, u)) u, 
	\end{align}
\end{subequations}
%
for all $(x, u) \in \mathcal{X} \times \mathcal{U}$. This gives that $\eta(\mathcal{X},\mathcal{U})\subseteq \mathcal{P} \subseteq \mathbb{R}^{n_\mathrm{p}}$, where $\mathcal{P}$ is often chosen be a compact convex set, if the model is further utilized for analysis or control synthesis.
Consequently, the behavior of the nonlinear system is included (embedded) in the behavior of the LPV system, i.e., $\mathcal{B} \subseteq \mathcal{B}_{\text{LPV}}$. In the next section, we will discuss the proposed method to achieve this objective.


% , and the scheduling variable $p(t)$ can be expressed as a function of the states and inputs of the system via the so-called \emph{scheduling map} $\eta$ as $p(t) = \eta(x(t), u(t))$, \com{which is often expected to belong to a certain function class, i.e. affine, polynomial, etc.} In the next section, we will discuss the proposed method to achieve this objective
\section{Method}\label{sec:method}
\begin{figure}
    \centering
    \includegraphics[width=0.85\textwidth]{imgs/heatmap_acc.pdf}
    \caption{\textbf{Visualization of the proposed periodic Bayesian flow with mean parameter $\mu$ and accumulated accuracy parameter $c$ which corresponds to the entropy/uncertainty}. For $x = 0.3, \beta(1) = 1000$ and $\alpha_i$ defined in \cref{appd:bfn_cir}, this figure plots three colored stochastic parameter trajectories for receiver mean parameter $m$ and accumulated accuracy parameter $c$, superimposed on a log-scale heatmap of the Bayesian flow distribution $p_F(m|x,\senderacc)$ and $p_F(c|x,\senderacc)$. Note the \emph{non-monotonicity} and \emph{non-additive} property of $c$ which could inform the network the entropy of the mean parameter $m$ as a condition and the \emph{periodicity} of $m$. %\jj{Shrink the figures to save space}\hanlin{Do we need to make this figure one-column?}
    }
    \label{fig:vmbf_vis}
    \vskip -0.1in
\end{figure}
% \begin{wrapfigure}{r}{0.5\textwidth}
%     \centering
%     \includegraphics[width=0.49\textwidth]{imgs/heatmap_acc.pdf}
%     \caption{\textbf{Visualization of hyper-torus Bayesian flow based on von Mises Distribution}. For $x = 0.3, \beta(1) = 1000$ and $\alpha_i$ defined in \cref{appd:bfn_cir}, this figure plots three colored stochastic parameter trajectories for receiver mean parameter $m$ and accumulated accuracy parameter $c$, superimposed on a log-scale heatmap of the Bayesian flow distribution $p_F(m|x,\senderacc)$ and $p_F(c|x,\senderacc)$. Note the \emph{non-monotonicity} and \emph{non-additive} property of $c$. \jj{Shrink the figures to save space}}
%     \label{fig:vmbf_vis}
%     \vspace{-30pt}
% \end{wrapfigure}


In this section, we explain the detailed design of CrysBFN tackling theoretical and practical challenges. First, we describe how to derive our new formulation of Bayesian Flow Networks over hyper-torus $\mathbb{T}^{D}$ from scratch. Next, we illustrate the two key differences between \modelname and the original form of BFN: $1)$ a meticulously designed novel base distribution with different Bayesian update rules; and $2)$ different properties over the accuracy scheduling resulted from the periodicity and the new Bayesian update rules. Then, we present in detail the overall framework of \modelname over each manifold of the crystal space (\textit{i.e.} fractional coordinates, lattice vectors, atom types) respecting \textit{periodic E(3) invariance}. 

% In this section, we first demonstrate how to build Bayesian flow on hyper-torus $\mathbb{T}^{D}$ by overcoming theoretical and practical problems to provide a low-noise parameter-space approach to fractional atom coordinate generation. Next, we present how \modelname models each manifold of crystal space respecting \textit{periodic E(3) invariance}. 

\subsection{Periodic Bayesian Flow on Hyper-torus \texorpdfstring{$\mathbb{T}^{D}$}{}} 
For generative modeling of fractional coordinates in crystal, we first construct a periodic Bayesian flow on \texorpdfstring{$\mathbb{T}^{D}$}{} by designing every component of the totally new Bayesian update process which we demonstrate to be distinct from the original Bayesian flow (please see \cref{fig:non_add}). 
 %:) 
 
 The fractional atom coordinate system \citep{jiao2023crystal} inherently distributes over a hyper-torus support $\mathbb{T}^{3\times N}$. Hence, the normal distribution support on $\R$ used in the original \citep{bfn} is not suitable for this scenario. 
% The key problem of generative modeling for crystal is the periodicity of Cartesian atom coordinates $\vX$ requiring:
% \begin{equation}\label{eq:periodcity}
% p(\vA,\vL,\vX)=p(\vA,\vL,\vX+\vec{LK}),\text{where}~\vec{K}=\vec{k}\vec{1}_{1\times N},\forall\vec{k}\in\mathbb{Z}^{3\times1}
% \end{equation}
% However, there does not exist such a distribution supporting on $\R$ to model such property because the integration of such distribution over $\R$ will not be finite and equal to 1. Therefore, the normal distribution used in \citet{bfn} can not meet this condition.

To tackle this problem, the circular distribution~\citep{mardia2009directional} over the finite interval $[-\pi,\pi)$ is a natural choice as the base distribution for deriving the BFN on $\mathbb{T}^D$. 
% one natural choice is to 
% we would like to consider the circular distribution over the finite interval as the base 
% we find that circular distributions \citep{mardia2009directional} defined on a finite interval with lengths of $2\pi$ can be used as the instantiation of input distribution for the BFN on $\mathbb{T}^D$.
Specifically, circular distributions enjoy desirable periodic properties: $1)$ the integration over any interval length of $2\pi$ equals 1; $2)$ the probability distribution function is periodic with period $2\pi$.  Sharing the same intrinsic with fractional coordinates, such periodic property of circular distribution makes it suitable for the instantiation of BFN's input distribution, in parameterizing the belief towards ground truth $\x$ on $\mathbb{T}^D$. 
% \yuxuan{this is very complicated from my perspective.} \hanlin{But this property is exactly beautiful and perfectly fit into the BFN.}

\textbf{von Mises Distribution and its Bayesian Update} We choose von Mises distribution \citep{mardia2009directional} from various circular distributions as the form of input distribution, based on the appealing conjugacy property required in the derivation of the BFN framework.
% to leverage the Bayesian conjugacy property of von Mises distribution which is required by the BFN framework. 
That is, the posterior of a von Mises distribution parameterized likelihood is still in the family of von Mises distributions. The probability density function of von Mises distribution with mean direction parameter $m$ and concentration parameter $c$ (describing the entropy/uncertainty of $m$) is defined as: 
\begin{equation}
f(x|m,c)=vM(x|m,c)=\frac{\exp(c\cos(x-m))}{2\pi I_0(c)}
\end{equation}
where $I_0(c)$ is zeroth order modified Bessel function of the first kind as the normalizing constant. Given the last univariate belief parameterized by von Mises distribution with parameter $\theta_{i-1}=\{m_{i-1},\ c_{i-1}\}$ and the sample $y$ from sender distribution with unknown data sample $x$ and known accuracy $\alpha$ describing the entropy/uncertainty of $y$,  Bayesian update for the receiver is deducted as:
\begin{equation}
 h(\{m_{i-1},c_{i-1}\},y,\alpha)=\{m_i,c_i \}, \text{where}
\end{equation}
\begin{equation}\label{eq:h_m}
m_i=\text{atan2}(\alpha\sin y+c_{i-1}\sin m_{i-1}, {\alpha\cos y+c_{i-1}\cos m_{i-1}})
\end{equation}
\begin{equation}\label{eq:h_c}
c_i =\sqrt{\alpha^2+c_{i-1}^2+2\alpha c_{i-1}\cos(y-m_{i-1})}
\end{equation}
The proof of the above equations can be found in \cref{apdx:bayesian_update_function}. The atan2 function refers to  2-argument arctangent. Independently conducting  Bayesian update for each dimension, we can obtain the Bayesian update distribution by marginalizing $\y$:
\begin{equation}
p_U(\vtheta'|\vtheta,\bold{x};\alpha)=\mathbb{E}_{p_S(\bold{y}|\bold{x};\alpha)}\delta(\vtheta'-h(\vtheta,\bold{y},\alpha))=\mathbb{E}_{vM(\bold{y}|\bold{x},\alpha)}\delta(\vtheta'-h(\vtheta,\bold{y},\alpha))
\end{equation} 
\begin{figure}
    \centering
    \vskip -0.15in
    \includegraphics[width=0.95\linewidth]{imgs/non_add.pdf}
    \caption{An intuitive illustration of non-additive accuracy Bayesian update on the torus. The lengths of arrows represent the uncertainty/entropy of the belief (\emph{e.g.}~$1/\sigma^2$ for Gaussian and $c$ for von Mises). The directions of the arrows represent the believed location (\emph{e.g.}~ $\mu$ for Gaussian and $m$ for von Mises).}
    \label{fig:non_add}
    \vskip -0.15in
\end{figure}
\textbf{Non-additive Accuracy} 
The additive accuracy is a nice property held with the Gaussian-formed sender distribution of the original BFN expressed as:
\begin{align}
\label{eq:standard_id}
    \update(\parsn{}'' \mid \parsn{}, \x; \alpha_a+\alpha_b) = \E_{\update(\parsn{}' \mid \parsn{}, \x; \alpha_a)} \update(\parsn{}'' \mid \parsn{}', \x; \alpha_b)
\end{align}
Such property is mainly derived based on the standard identity of Gaussian variable:
\begin{equation}
X \sim \mathcal{N}\left(\mu_X, \sigma_X^2\right), Y \sim \mathcal{N}\left(\mu_Y, \sigma_Y^2\right) \Longrightarrow X+Y \sim \mathcal{N}\left(\mu_X+\mu_Y, \sigma_X^2+\sigma_Y^2\right)
\end{equation}
The additive accuracy property makes it feasible to derive the Bayesian flow distribution $
p_F(\boldsymbol{\theta} \mid \mathbf{x} ; i)=p_U\left(\boldsymbol{\theta} \mid \boldsymbol{\theta}_0, \mathbf{x}, \sum_{k=1}^{i} \alpha_i \right)
$ for the simulation-free training of \cref{eq:loss_n}.
It should be noted that the standard identity in \cref{eq:standard_id} does not hold in the von Mises distribution. Hence there exists an important difference between the original Bayesian flow defined on Euclidean space and the Bayesian flow of circular data on $\mathbb{T}^D$ based on von Mises distribution. With prior $\btheta = \{\bold{0},\bold{0}\}$, we could formally represent the non-additive accuracy issue as:
% The additive accuracy property implies the fact that the "confidence" for the data sample after observing a series of the noisy samples with accuracy ${\alpha_1, \cdots, \alpha_i}$ could be  as the accuracy sum  which could be  
% Here we 
% Here we emphasize the specific property of BFN based on von Mises distribution.
% Note that 
% \begin{equation}
% \update(\parsn'' \mid \parsn, \x; \alpha_a+\alpha_b) \ne \E_{\update(\parsn' \mid \parsn, \x; \alpha_a)} \update(\parsn'' \mid \parsn', \x; \alpha_b)
% \end{equation}
% \oyyw{please check whether the below equation is better}
% \yuxuan{I fill somehow confusing on what is the update distribution with $\alpha$. }
% \begin{equation}
% \update(\parsn{}'' \mid \parsn{}, \x; \alpha_a+\alpha_b) \ne \E_{\update(\parsn{}' \mid \parsn{}, \x; \alpha_a)} \update(\parsn{}'' \mid \parsn{}', \x; \alpha_b)
% \end{equation}
% We give an intuitive visualization of such difference in \cref{fig:non_add}. The untenability of this property can materialize by considering the following case: with prior $\btheta = \{\bold{0},\bold{0}\}$, check the two-step Bayesian update distribution with $\alpha_a,\alpha_b$ and one-step Bayesian update with $\alpha=\alpha_a+\alpha_b$:
\begin{align}
\label{eq:nonadd}
     &\update(c'' \mid \parsn, \x; \alpha_a+\alpha_b)  = \delta(c-\alpha_a-\alpha_b)
     \ne  \mathbb{E}_{p_U(\parsn' \mid \parsn, \x; \alpha_a)}\update(c'' \mid \parsn', \x; \alpha_b) \nonumber \\&= \mathbb{E}_{vM(\bold{y}_b|\bold{x},\alpha_a)}\mathbb{E}_{vM(\bold{y}_a|\bold{x},\alpha_b)}\delta(c-||[\alpha_a \cos\y_a+\alpha_b\cos \y_b,\alpha_a \sin\y_a+\alpha_b\sin \y_b]^T||_2)
\end{align}
A more intuitive visualization could be found in \cref{fig:non_add}. This fundamental difference between periodic Bayesian flow and that of \citet{bfn} presents both theoretical and practical challenges, which we will explain and address in the following contents.

% This makes constructing Bayesian flow based on von Mises distribution intrinsically different from previous Bayesian flows (\citet{bfn}).

% Thus, we must reformulate the framework of Bayesian flow networks  accordingly. % and do necessary reformulations of BFN. 

% \yuxuan{overall I feel this part is complicated by using the language of update distribution. I would like to suggest simply use bayesian update, to provide intuitive explantion.}\hanlin{See the illustration in \cref{fig:non_add}}

% That introduces a cascade of problems, and we investigate the following issues: $(1)$ Accuracies between sender and receiver are not synchronized and need to be differentiated. $(2)$ There is no tractable Bayesian flow distribution for a one-step sample conditioned on a given time step $i$, and naively simulating the Bayesian flow results in computational overhead. $(3)$ It is difficult to control the entropy of the Bayesian flow. $(4)$ Accuracy is no longer a function of $t$ and becomes a distribution conditioned on $t$, which can be different across dimensions.
%\jj{Edited till here}

\textbf{Entropy Conditioning} As a common practice in generative models~\citep{ddpm,flowmatching,bfn}, timestep $t$ is widely used to distinguish among generation states by feeding the timestep information into the networks. However, this paper shows that for periodic Bayesian flow, the accumulated accuracy $\vc_i$ is more effective than time-based conditioning by informing the network about the entropy and certainty of the states $\parsnt{i}$. This stems from the intrinsic non-additive accuracy which makes the receiver's accumulated accuracy $c$ not bijective function of $t$, but a distribution conditioned on accumulated accuracies $\vc_i$ instead. Therefore, the entropy parameter $\vc$ is taken logarithm and fed into the network to describe the entropy of the input corrupted structure. We verify this consideration in \cref{sec:exp_ablation}. 
% \yuxuan{implement variant. traditionally, the timestep is widely used to distinguish the different states by putting the timestep embedding into the networks. citation of FM, diffusion, BFN. However, we find that conditioned on time in periodic flow could not provide extra benefits. To further boost the performance, we introduce a simple yet effective modification term entropy conditional. This is based on that the accumulated accuracy which represents the current uncertainty or entropy could be a better indicator to distinguish different states. + Describe how you do this. }



\textbf{Reformulations of BFN}. Recall the original update function with Gaussian sender distribution, after receiving noisy samples $\y_1,\y_2,\dots,\y_i$ with accuracies $\senderacc$, the accumulated accuracies of the receiver side could be analytically obtained by the additive property and it is consistent with the sender side.
% Since observing sample $\y$ with $\alpha_i$ can not result in exact accuracy increment $\alpha_i$ for receiver, the accuracies between sender and receiver are not synchronized which need to be differentiated. 
However, as previously mentioned, this does not apply to periodic Bayesian flow, and some of the notations in original BFN~\citep{bfn} need to be adjusted accordingly. We maintain the notations of sender side's one-step accuracy $\alpha$ and added accuracy $\beta$, and alter the notation of receiver's accuracy parameter as $c$, which is needed to be simulated by cascade of Bayesian updates. We emphasize that the receiver's accumulated accuracy $c$ is no longer a function of $t$ (differently from the Gaussian case), and it becomes a distribution conditioned on received accuracies $\senderacc$ from the sender. Therefore, we represent the Bayesian flow distribution of von Mises distribution as $p_F(\btheta|\x;\alpha_1,\alpha_2,\dots,\alpha_i)$. And the original simulation-free training with Bayesian flow distribution is no longer applicable in this scenario.
% Different from previous BFNs where the accumulated accuracy $\rho$ is not explicitly modeled, the accumulated accuracy parameter $c$ (visualized in \cref{fig:vmbf_vis}) needs to be explicitly modeled by feeding it to the network to avoid information loss.
% the randomaccuracy parameter $c$ (visualized in \cref{fig:vmbf_vis}) implies that there exists information in $c$ from the sender just like $m$, meaning that $c$ also should be fed into the network to avoid information loss. 
% We ablate this consideration in  \cref{sec:exp_ablation}. 

\textbf{Fast Sampling from Equivalent Bayesian Flow Distribution} Based on the above reformulations, the Bayesian flow distribution of von Mises distribution is reframed as: 
\begin{equation}\label{eq:flow_frac}
p_F(\btheta_i|\x;\alpha_1,\alpha_2,\dots,\alpha_i)=\E_{\update(\parsnt{1} \mid \parsnt{0}, \x ; \alphat{1})}\dots\E_{\update(\parsn_{i-1} \mid \parsnt{i-2}, \x; \alphat{i-1})} \update(\parsnt{i} | \parsnt{i-1},\x;\alphat{i} )
\end{equation}
Naively sampling from \cref{eq:flow_frac} requires slow auto-regressive iterated simulation, making training unaffordable. Noticing the mathematical properties of \cref{eq:h_m,eq:h_c}, we  transform \cref{eq:flow_frac} to the equivalent form:
\begin{equation}\label{eq:cirflow_equiv}
p_F(\vec{m}_i|\x;\alpha_1,\alpha_2,\dots,\alpha_i)=\E_{vM(\y_1|\x,\alpha_1)\dots vM(\y_i|\x,\alpha_i)} \delta(\vec{m}_i-\text{atan2}(\sum_{j=1}^i \alpha_j \cos \y_j,\sum_{j=1}^i \alpha_j \sin \y_j))
\end{equation}
\begin{equation}\label{eq:cirflow_equiv2}
p_F(\vec{c}_i|\x;\alpha_1,\alpha_2,\dots,\alpha_i)=\E_{vM(\y_1|\x,\alpha_1)\dots vM(\y_i|\x,\alpha_i)}  \delta(\vec{c}_i-||[\sum_{j=1}^i \alpha_j \cos \y_j,\sum_{j=1}^i \alpha_j \sin \y_j]^T||_2)
\end{equation}
which bypasses the computation of intermediate variables and allows pure tensor operations, with negligible computational overhead.
\begin{restatable}{proposition}{cirflowequiv}
The probability density function of Bayesian flow distribution defined by \cref{eq:cirflow_equiv,eq:cirflow_equiv2} is equivalent to the original definition in \cref{eq:flow_frac}. 
\end{restatable}
\textbf{Numerical Determination of Linear Entropy Sender Accuracy Schedule} ~Original BFN designs the accuracy schedule $\beta(t)$ to make the entropy of input distribution linearly decrease. As for crystal generation task, to ensure information coherence between modalities, we choose a sender accuracy schedule $\senderacc$ that makes the receiver's belief entropy $H(t_i)=H(p_I(\cdot|\vtheta_i))=H(p_I(\cdot|\vc_i))$ linearly decrease \emph{w.r.t.} time $t_i$, given the initial and final accuracy parameter $c(0)$ and $c(1)$. Due to the intractability of \cref{eq:vm_entropy}, we first use numerical binary search in $[0,c(1)]$ to determine the receiver's $c(t_i)$ for $i=1,\dots, n$ by solving the equation $H(c(t_i))=(1-t_i)H(c(0))+tH(c(1))$. Next, with $c(t_i)$, we conduct numerical binary search for each $\alpha_i$ in $[0,c(1)]$ by solving the equations $\E_{y\sim vM(x,\alpha_i)}[\sqrt{\alpha_i^2+c_{i-1}^2+2\alpha_i c_{i-1}\cos(y-m_{i-1})}]=c(t_i)$ from $i=1$ to $i=n$ for arbitrarily selected $x\in[-\pi,\pi)$.

After tackling all those issues, we have now arrived at a new BFN architecture for effectively modeling crystals. Such BFN can also be adapted to other type of data located in hyper-torus $\mathbb{T}^{D}$.

\subsection{Equivariant Bayesian Flow for Crystal}
With the above Bayesian flow designed for generative modeling of fractional coordinate $\vF$, we are able to build equivariant Bayesian flow for each modality of crystal. In this section, we first give an overview of the general training and sampling algorithm of \modelname (visualized in \cref{fig:framework}). Then, we describe the details of the Bayesian flow of every modality. The training and sampling algorithm can be found in \cref{alg:train} and \cref{alg:sampling}.

\textbf{Overview} Operating in the parameter space $\bthetaM=\{\bthetaA,\bthetaL,\bthetaF\}$, \modelname generates high-fidelity crystals through a joint BFN sampling process on the parameter of  atom type $\bthetaA$, lattice parameter $\vec{\theta}^L=\{\bmuL,\brhoL\}$, and the parameter of fractional coordinate matrix $\bthetaF=\{\bmF,\bcF\}$. We index the $n$-steps of the generation process in a discrete manner $i$, and denote the corresponding continuous notation $t_i=i/n$ from prior parameter $\thetaM_0$ to a considerably low variance parameter $\thetaM_n$ (\emph{i.e.} large $\vrho^L,\bmF$, and centered $\bthetaA$).

At training time, \modelname samples time $i\sim U\{1,n\}$ and $\bthetaM_{i-1}$ from the Bayesian flow distribution of each modality, serving as the input to the network. The network $\net$ outputs $\net(\parsnt{i-1}^\mathcal{M},t_{i-1})=\net(\parsnt{i-1}^A,\parsnt{i-1}^F,\parsnt{i-1}^L,t_{i-1})$ and conducts gradient descents on loss function \cref{eq:loss_n} for each modality. After proper training, the sender distribution $p_S$ can be approximated by the receiver distribution $p_R$. 

At inference time, from predefined $\thetaM_0$, we conduct transitions from $\thetaM_{i-1}$ to $\thetaM_{i}$ by: $(1)$ sampling $\y_i\sim p_R(\bold{y}|\thetaM_{i-1};t_i,\alpha_i)$ according to network prediction $\predM{i-1}$; and $(2)$ performing Bayesian update $h(\thetaM_{i-1},\y^\calM_{i-1},\alpha_i)$ for each dimension. 

% Alternatively, we complete this transition using the flow-back technique by sampling 
% $\thetaM_{i}$ from Bayesian flow distribution $\flow(\btheta^M_{i}|\predM{i-1};t_{i-1})$. 

% The training objective of $\net$ is to minimize the KL divergence between sender distribution and receiver distribution for every modality as defined in \cref{eq:loss_n} which is equivalent to optimizing the negative variational lower bound $\calL^{VLB}$ as discussed in \cref{sec:preliminaries}. 

%In the following part, we will present the Bayesian flow of each modality in detail.

\textbf{Bayesian Flow of Fractional Coordinate $\vF$}~The distribution of the prior parameter $\bthetaF_0$ is defined as:
\begin{equation}\label{eq:prior_frac}
    p(\bthetaF_0) \defeq \{vM(\vm_0^F|\vec{0}_{3\times N},\vec{0}_{3\times N}),\delta(\vc_0^F-\vec{0}_{3\times N})\} = \{U(\vec{0},\vec{1}),\delta(\vc_0^F-\vec{0}_{3\times N})\}
\end{equation}
Note that this prior distribution of $\vm_0^F$ is uniform over $[\vec{0},\vec{1})$, ensuring the periodic translation invariance property in \cref{De:pi}. The training objective is minimizing the KL divergence between sender and receiver distribution (deduction can be found in \cref{appd:cir_loss}): 
%\oyyw{replace $\vF$ with $\x$?} \hanlin{notations follow Preliminary?}
\begin{align}\label{loss_frac}
\calL_F = n \E_{i \sim \ui{n}, \flow(\parsn{}^F \mid \vF ; \senderacc)} \alpha_i\frac{I_1(\alpha_i)}{I_0(\alpha_i)}(1-\cos(\vF-\predF{i-1}))
\end{align}
where $I_0(x)$ and $I_1(x)$ are the zeroth and the first order of modified Bessel functions. The transition from $\bthetaF_{i-1}$ to $\bthetaF_{i}$ is the Bayesian update distribution based on network prediction:
\begin{equation}\label{eq:transi_frac}
    p(\btheta^F_{i}|\parsnt{i-1}^\calM)=\mathbb{E}_{vM(\bold{y}|\predF{i-1},\alpha_i)}\delta(\btheta^F_{i}-h(\btheta^F_{i-1},\bold{y},\alpha_i))
\end{equation}
\begin{restatable}{proposition}{fracinv}
With $\net_{F}$ as a periodic translation equivariant function namely $\net_F(\parsnt{}^A,w(\parsnt{}^F+\vt),\parsnt{}^L,t)=w(\net_F(\parsnt{}^A,\parsnt{}^F,\parsnt{}^L,t)+\vt), \forall\vt\in\R^3$, the marginal distribution of $p(\vF_n)$ defined by \cref{eq:prior_frac,eq:transi_frac} is periodic translation invariant. 
\end{restatable}
\textbf{Bayesian Flow of Lattice Parameter \texorpdfstring{$\boldsymbol{L}$}{}}   
Noting the lattice parameter $\bm{L}$ located in Euclidean space, we set prior as the parameter of a isotropic multivariate normal distribution $\btheta^L_0\defeq\{\vmu_0^L,\vrho_0^L\}=\{\bm{0}_{3\times3},\bm{1}_{3\times3}\}$
% \begin{equation}\label{eq:lattice_prior}
% \btheta^L_0\defeq\{\vmu_0^L,\vrho_0^L\}=\{\bm{0}_{3\times3},\bm{1}_{3\times3}\}
% \end{equation}
such that the prior distribution of the Markov process on $\vmu^L$ is the Dirac distribution $\delta(\vec{\mu_0}-\vec{0})$ and $\delta(\vec{\rho_0}-\vec{1})$, 
% \begin{equation}
%     p_I^L(\boldsymbol{L}|\btheta_0^L)=\mathcal{N}(\bm{L}|\bm{0},\bm{I})
% \end{equation}
which ensures O(3)-invariance of prior distribution of $\vL$. By Eq. 77 from \citet{bfn}, the Bayesian flow distribution of the lattice parameter $\bm{L}$ is: 
\begin{align}% =p_U(\bmuL|\btheta_0^L,\bm{L},\beta(t))
p_F^L(\bmuL|\bm{L};t) &=\mathcal{N}(\bmuL|\gamma(t)\bm{L},\gamma(t)(1-\gamma(t))\bm{I}) 
\end{align}
where $\gamma(t) = 1 - \sigma_1^{2t}$ and $\sigma_1$ is the predefined hyper-parameter controlling the variance of input distribution at $t=1$ under linear entropy accuracy schedule. The variance parameter $\vrho$ does not need to be modeled and fed to the network, since it is deterministic given the accuracy schedule. After sampling $\bmuL_i$ from $p_F^L$, the training objective is defined as minimizing KL divergence between sender and receiver distribution (based on Eq. 96 in \citet{bfn}):
\begin{align}
\mathcal{L}_{L} = \frac{n}{2}\left(1-\sigma_1^{2/n}\right)\E_{i \sim \ui{n}}\E_{\flow(\bmuL_{i-1} |\vL ; t_{i-1})}  \frac{\left\|\vL -\predL{i-1}\right\|^2}{\sigma_1^{2i/n}},\label{eq:lattice_loss}
\end{align}
where the prediction term $\predL{i-1}$ is the lattice parameter part of network output. After training, the generation process is defined as the Bayesian update distribution given network prediction:
\begin{equation}\label{eq:lattice_sampling}
    p(\bmuL_{i}|\parsnt{i-1}^\calM)=\update^L(\bmuL_{i}|\predL{i-1},\bmuL_{i-1};t_{i-1})
\end{equation}
    

% The final prediction of the lattice parameter is given by $\bmuL_n = \predL{n-1}$.
% \begin{equation}\label{eq:final_lattice}
%     \bmuL_n = \predL{n-1}
% \end{equation}

\begin{restatable}{proposition}{latticeinv}\label{prop:latticeinv}
With $\net_{L}$ as  O(3)-equivariant function namely $\net_L(\parsnt{}^A,\parsnt{}^F,\vQ\parsnt{}^L,t)=\vQ\net_L(\parsnt{}^A,\parsnt{}^F,\parsnt{}^L,t),\forall\vQ^T\vQ=\vI$, the marginal distribution of $p(\bmuL_n)$ defined by \cref{eq:lattice_sampling} is O(3)-invariant. 
\end{restatable}


\textbf{Bayesian Flow of Atom Types \texorpdfstring{$\boldsymbol{A}$}{}} 
Given that atom types are discrete random variables located in a simplex $\calS^K$, the prior parameter of $\boldsymbol{A}$ is the discrete uniform distribution over the vocabulary $\parsnt{0}^A \defeq \frac{1}{K}\vec{1}_{1\times N}$. 
% \begin{align}\label{eq:disc_input_prior}
% \parsnt{0}^A \defeq \frac{1}{K}\vec{1}_{1\times N}
% \end{align}
% \begin{align}
%     (\oh{j}{K})_k \defeq \delta_{j k}, \text{where }\oh{j}{K}\in \R^{K},\oh{\vA}{KD} \defeq \left(\oh{a_1}{K},\dots,\oh{a_N}{K}\right) \in \R^{K\times N}
% \end{align}
With the notation of the projection from the class index $j$ to the length $K$ one-hot vector $ (\oh{j}{K})_k \defeq \delta_{j k}, \text{where }\oh{j}{K}\in \R^{K},\oh{\vA}{KD} \defeq \left(\oh{a_1}{K},\dots,\oh{a_N}{K}\right) \in \R^{K\times N}$, the Bayesian flow distribution of atom types $\vA$ is derived in \citet{bfn}:
\begin{align}
\flow^{A}(\parsn^A \mid \vA; t) &= \E_{\N{\y \mid \beta^A(t)\left(K \oh{\vA}{K\times N} - \vec{1}_{K\times N}\right)}{\beta^A(t) K \vec{I}_{K\times N \times N}}} \delta\left(\parsn^A - \frac{e^{\y}\parsnt{0}^A}{\sum_{k=1}^K e^{\y_k}(\parsnt{0})_{k}^A}\right).
\end{align}
where $\beta^A(t)$ is the predefined accuracy schedule for atom types. Sampling $\btheta_i^A$ from $p_F^A$ as the training signal, the training objective is the $n$-step discrete-time loss for discrete variable \citep{bfn}: 
% \oyyw{can we simplify the next equation? Such as remove $K \times N, K \times N \times N$}
% \begin{align}
% &\calL_A = n\E_{i \sim U\{1,n\},\flow^A(\parsn^A \mid \vA ; t_{i-1}),\N{\y \mid \alphat{i}\left(K \oh{\vA}{KD} - \vec{1}_{K\times N}\right)}{\alphat{i} K \vec{I}_{K\times N \times N}}} \ln \N{\y \mid \alphat{i}\left(K \oh{\vA}{K\times N} - \vec{1}_{K\times N}\right)}{\alphat{i} K \vec{I}_{K\times N \times N}}\nonumber\\
% &\qquad\qquad\qquad-\sum_{d=1}^N \ln \left(\sum_{k=1}^K \out^{(d)}(k \mid \parsn^A; t_{i-1}) \N{\ydd{d} \mid \alphat{i}\left(K\oh{k}{K}- \vec{1}_{K\times N}\right)}{\alphat{i} K \vec{I}_{K\times N \times N}}\right)\label{discdisc_t_loss_exp}
% \end{align}
\begin{align}
&\calL_A = n\E_{i \sim U\{1,n\},\flow^A(\parsn^A \mid \vA ; t_{i-1}),\N{\y \mid \alphat{i}\left(K \oh{\vA}{KD} - \vec{1}\right)}{\alphat{i} K \vec{I}}} \ln \N{\y \mid \alphat{i}\left(K \oh{\vA}{K\times N} - \vec{1}\right)}{\alphat{i} K \vec{I}}\nonumber\\
&\qquad\qquad\qquad-\sum_{d=1}^N \ln \left(\sum_{k=1}^K \out^{(d)}(k \mid \parsn^A; t_{i-1}) \N{\ydd{d} \mid \alphat{i}\left(K\oh{k}{K}- \vec{1}\right)}{\alphat{i} K \vec{I}}\right)\label{discdisc_t_loss_exp}
\end{align}
where $\vec{I}\in \R^{K\times N \times N}$ and $\vec{1}\in\R^{K\times D}$. When sampling, the transition from $\bthetaA_{i-1}$ to $\bthetaA_{i}$ is derived as:
\begin{equation}
    p(\btheta^A_{i}|\parsnt{i-1}^\calM)=\update^A(\btheta^A_{i}|\btheta^A_{i-1},\predA{i-1};t_{i-1})
\end{equation}

The detailed training and sampling algorithm could be found in \cref{alg:train} and \cref{alg:sampling}.




% The commented chapter is the FTC formulation from the MPC paper Roland sent me.
% \section{FTC embedding from paper Roland sent}

The core idea is that if given a continuously differentiable function $g:\mathbb{R}^n\rightarrow\mathbb{R}^m$, then based on the FTC~\cite{Koelewijn2023}:
Let us assume we have a continuously differentiable function $g:\mathbb{R}^n\rightarrow\mathbb{R}^m$, with $g(0)=0$, for which we have that $g(\eta)=\begin{bmatrix}
		g_1(\eta) & \cdots & g_m(\eta)
	\end{bmatrix}$, where $g_i(\eta):\mathbb{R}^n\rightarrow\mathbb{R}$ for $ i = 1,\,\dots,\,m$ and $\eta\in\mathbb{R}^n$. Let us define the element\nobreakdash-wise functions $\bar{g}_i(\lambda)=g_i(\lambda \eta)$ for $i = 1,\,\dots,\,m$, with $\bar{g}_i:[0,1]\rightarrow\mathbb{R}$. Using the second fundamental theorem of calculus:
\begin{equation}
	\bar{g}_i(1)-\bar{g}_i(0)=\int_0^1 \frac{d \bar{g}_i}{d\lambda}(\bar{\lambda})\;d\bar{\lambda},\quad\text{for }i=1,\dots\,m,
\end{equation}
results in
\begin{align}
	g_i(\eta)-\underbrace{g_i(0)}_0 & =\int_0^1\left(\frac{d g_i}{d\eta}(\bar{\lambda}\eta)\right)\eta \;d\bar{\lambda}, \quad\text{for }i=1,\dots\,m, \\
	g_i(\eta)                       & =\left(\int_0^1\frac{d g_i}{d\eta}(\bar{\lambda}\eta)\,d\bar{\lambda}\right)\eta.
\end{align}
Combining the elements, $g$ is expressed as
\begin{subequations}
	\begin{align}
		g(\eta)- g(0) & =                                                                                                                        %\begin{bmatrix}
		%g_1(\eta)\\\vdots\\g_m(\eta)
		%\end{bmatrix} 
		{\small
		\begin{bmatrix}
			\left(\int_0^1\frac{d g_1}{d\eta}({\lambda}\eta)\,d{\lambda}\right)(\eta-0) \\\vdots\\\left(\int_0^1\frac{d g_m}{d\eta}({\lambda}\eta)\,d{\lambda}\right)(\eta-0)
		\end{bmatrix}} \\
		g(\eta)- g(0) & =\left(\int_0^1\frac{d g}{d\eta}({\lambda}\eta)\,d{\lambda}\right)\eta,
	\end{align}
\end{subequations}
where $\frac{d g}{d\eta}({\lambda}\eta) \in \mathbb{R}^{m \times n}$ is the Jacobian of $g$ evaluated in ${\lambda}\eta$. Provided that functions $f$ and $h$ in \eqref{eq:nl_dyn} are differentiable, choosing  $\eta=[\
	\hat{x}^\top \ \ u^\top \ ]^\top$ gives via the FTC that
%\begin{equation*}%\label{eq:abcd_def}
\begin{align}
	\tilde{f}_{\hat \theta}(\hat{x},u) & \!=\!\underbrace{  \left({\int_0^1\!\!\frac{\partial f}{\partial \hat{x}}({\lambda}\hat{x},{\lambda}u)\,d\lambda}\right)}_{{A}(\hat{x},u)}\!\hat{x}\!+\!\underbrace{\left(\int_0^1\!\!\frac{\partial f}{\partial u}({\lambda}\hat{x},{\lambda}u)\,d\lambda\right)}_{{B}(\hat{x},u)}\!u, \notag \\
	\tilde{h}_{\hat \theta}(\hat{x})   & \!=\!\underbrace{\left(\int_0^1\!\!\frac{\partial h}{\partial \hat{x}}({\lambda}\hat{x})\,d\lambda\right)}_{{C}(\hat{x})}\!\hat{x}, \label{eq:abcd_def}                                                                                                                                        %\!+\!\underbrace{\left(\int_0^1\!\!\frac{\partial h}{\partial u}({\lambda}\hat{x},{\lambda}u)\,d\lambda\right)}_{{D}(\hat{x},u)}\!u,
\end{align}
%\end{equation*}
\vskip -2mm \noindent where $	\tilde{f}_{\hat \theta}(\hat{x},u)=f_{\hat \theta}(\hat{x},u)-	f_{\hat \theta}(0,0)$ with $\tilde{h}_{\hat \theta}$ similarly defined. With $p_k=[\
	\hat{x}_k^\top \ \ u_k^\top \ ]^\top$ and $\mu$ the identity function, this gives an LPV form of the ANN-SS model \eqref{eq:model_structure} with affine terms $V=	f_{\hat \theta}(0,0)$ and  $W=	h_{\hat \theta}(0)$ as
\begin{subequations}
	\label{eq:lpv_model}\begin{align}
		\hat{x}_{k+1} & = A(p_k) \hat{x}_k + B(p_k)  u_k + V, \\
		y_k           & = C(p_k)\hat{x}_k                     %+ D(p_k)  u_k
		+ W.
	\end{align}
\end{subequations}
\chapter{Implementation}{\label{ch:implementation}}
In this chapter, we present the implementation of the final product. We start by discussing how the four steps introduced in \hyperref[ch:high_level_approach]{chapter \ref*{ch:high_level_approach}} are integrated. We then outline the main system components of our score follower, presenting each as an independent, self-contained module. We then combine this into an overall system architecture and finally introduce the open-source score renderer used to display the score and evaluate the score follower.       

% \section{Aims and Requirements}
% The overall aim of the score follower was to 


\section{Score Follower Framework Details}
Our score follower conforms to the high-level framework presented in \hyperref[section:score_follower_framework]{section \ref*{section:score_follower_framework}}. In step 1, two score features are extracted from a MIDI file (see \hyperref[subsection:midi]{subsection \ref*{subsection:midi}}), namely MIDI note numbers\footnote{\href{https://inspiredacoustics.com/en/MIDI_note_numbers_and_center_frequencies}{https://inspiredacoustics.com/en/MIDI\_note\_numbers\_and\_center\_frequencies}} (corresponding to pitch) and note onsets (corresponding to duration). In step 2, the audio is streamed (whether from a file or into a microphone) and audioframes that exceed some predefined energy threshold are extracted. Here, audioframes are groups of contiguous audio samples, whose length can be specified by the argument \verb|frame_length|, usually between 800 and 2000 samples. The period between consecutive audioframes can also be defined by the argument \verb|hop_length|, typically between 2000 and 5000 audio samples. In step 3, score following is performed via a `Windowed' Viterbi algorithm (see  \hyperref[subsection:adjusting_viterbi]{subsection \ref*{subsection:adjusting_viterbi}}) which uses the Gaussian Process (GP) log marginal likelihoods (LMLs) for emission probabilities (see \hyperref[section:state_duration_model]{section \ref*{section:state_duration_model}}) and a state duration model for transition probabilities (see \hyperref[section:state_duration_model]{section \ref*{section:state_duration_model}}). Finally, in step 4 we render our results using an adapted version of the open source user interface, \textit{Flippy Qualitative Testbench}.

\section{Following Modes}
Two modes are available to the user: Pre-recorded Mode and Live Mode. The former requires a pre-recorded $\verb|.wav|$ file, whereas the latter takes an input stream of audio via the device's microphone. Note that both modes are still forms of score following, as opposed to score alignment, since in each mode we receive audioframes at the sampling rate, not all at once.\\

Live Mode offers a practical example of a score follower, displaying a score and position marker which a musician can read off while playing. However, this mode is not suitable for evaluation because the input and results cannot be easily replicated. Even ignoring repeatability, Live Mode is not suitable for one-off testing since a musician using this application may be influenced by the movement of the marker. For instance, the performer may speed up if the score follower `gets ahead' or slow down if the position marker lags or `gets lost'. To avoid this, we use Pre-recorded Mode when evaluating the performance of our score follower. Furthermore, Pre-recorded Mode offers the advantage of testing away from the music room, providing the opportunity to evaluate a variety of recordings available online. 

\section{System Architecture}
Our guiding principle for development was to build modular code in order to create a streamlined system where each component performs a specific task independently. This structure facilitates easy testing and debugging. \hyperref[fig:black_box]{Figure \ref*{fig:black_box}} presents a high-level architecture diagram, where each black box abstracts a key component of the score follower. When operating in Pre-recorded Mode, there is the option to stream the recording during run-time, which outputs to the device's speakers (as indicated by the dashed lines).

\begin{figure}[H]
    \centering
    \includegraphics[width=1\textwidth]{figs/Part_4_Implementation_And_Results/black_box.png}
    \caption{Abstracted system architecture diagram displaying inputs in grey, the 4 main components of the score follower in black and the outputs in green.}
    \label{fig:black_box}
\end{figure}

\subsection{Score Preprocessor}
The architecture for the Score Preprocesor is given in \hyperref[fig:score_preprocessor]{Figure \ref*{fig:score_preprocessor}}. First, MIDI note number and note onset times are extracted from each MIDI event. Simultaneous notes can be gathered into states and returned as a time-sorted list of lists called \verb|score|, where each element of the outer list is a list of simultaneous note onsets. Similarly, a list of note durations calculated as the time difference between consecutive states is returned as \verb|times_to_next|. Finally, all covariance matrices are precalculated and stored in a dictionary, where the key of the dictionary is determined by the notes present. This is because the distribution of notes and chords in a score is not random: notes tend to belong to a home \gls{key} and melodies tend to be repeated or related (similar to subject fields in speech processing). Therefore, states tend to be reused often, allowing us to achieve amortised time and space savings (by avoiding repeated calculation of the same covariance matrices). 

\begin{figure}[H]
    \centering
    \includegraphics[width=1\textwidth]{figs/Part_3_Implementation/Stage_2_Alignment/score_preprocessor.png}
    \caption{System architecture diagram representing the Score Preprocessor with inputs in grey, processes in blue and objects in yellow.}
    \label{fig:score_preprocessor}
\end{figure}


\subsection{Audio Preprocessor}
The architecture for the Audio Preprocessor is illustrated in \hyperref[fig:audio_preprocessor]{Figure \ref*{fig:audio_preprocessor}}. In Pre-recorded Mode, the Slicer receives a $\verb|.wav|$ file and returns audioframes separated by the \verb|hop_length|. These audioframes are periodically added to a multiprocessing queue, \verb|AudioFramesQueue|, to simulate real-time score following. In Live Mode, we use the python module \verb|sounddevice| to receive a stream of audio, using a periodic callback function to place audioframes on \verb|AudioFramesQueue|. 

\begin{figure}[H]
    \centering
    \includegraphics[width=1\textwidth]{figs/Part_4_Implementation_And_Results/audio_preprocessor.png}
    \caption{System architecture diagram representing the Audio Preprocessor with inputs in grey, processes in blue and objects in yellow.}
    \label{fig:audio_preprocessor}
\end{figure}

\subsection{Follower and Backend}
The joint Follower and Backend architecture diagram is shown in \hyperref[fig:follwer_and_backend]{Figure \ref*{fig:follwer_and_backend}}. The Viterbi Follower (detailed in \hyperref[subsection:adjusting_viterbi]{section \ref*{subsection:adjusting_viterbi}}) calculates the most probable state in the score, given audioframes continually taken from \verb|AudioFramesQueue|. These states are placed on another multiprocessing queue, the \verb|FollowerOutputQueue|, for the Backend to process and send. This prevents any bottle-necking occurring at the Follower stage. The Backend first sets up a UDP connection and then reads off values from \verb|FollowerOutputQueue|, sending them via UDP packets to the score renderer.

\begin{figure}[H]
    \centering
    \includegraphics[width=1\textwidth]{figs/Part_4_Implementation_And_Results/follower_and_backend.png}
    \caption{System architecture diagram representing the Follower and Backend processes with processes in blue, objects in yellow and outputs in green.}
    \label{fig:follwer_and_backend}
\end{figure}

\subsection{Player}
In Pre-recorded Mode, the Player sets up a new process and begins streaming the recording once the Follower process begins. This provides a baseline for testing purposes, as a trained musician can observe the score position marker and judge how well it matches the music. 

\subsection{Overall System Architecture}
The overall system architecture is presented in \hyperref[fig:overall_system_architecture]{Figure \ref*{fig:overall_system_architecture}}. Since the Follower runs a real-time, time sensitive process, parallelism is employed to reduce the total system latency. We use two \verb|multiprocessing| queues\footnote{\href{https://docs.python.org/3/library/multiprocessing.html}{https://docs.python.org/3/library/multiprocessing.html}} to avoid bottle-necking, which allows us to run 4 concurrent processes (Audio Preprocessor, Follower, Backend, and Audio Player). Hence, this architecture allows the components to run independently of one another to avoid blocking. Furthermore, this allows the system to take advantage of the multiple cores and high computational power offered by most modern machines.  

\begin{figure}[H]
    \centering
    \includegraphics[width=1\textwidth]{figs/Part_4_Implementation_And_Results/overall_score_follower_2.png}
    \caption{System architecture diagram representing the overall score follower running in Pre-recorded mode, with inputs in grey, processes in blue, objects in yellow and outputs in green.}
    \label{fig:overall_system_architecture}
\end{figure}


\section{Rendering Results}{\label{section:renderer}}
To visualise the results of our score follower, we adapted an open source tool for testing different score followers.\footnote{\href{https://github.com/flippy-fyp/flippy-qualitative-testbench/blob/main/README.md}{https://github.com/flippy-fyp/flippy-qualitative-testbench/blob/main/README.md}} \hyperref[fig:flippy_example]{Figure \ref*{fig:flippy_example}} shows the user interface of the score position renderer, where the green bar indicates score position. 

\begin{figure}[H]
    \centering
    \includegraphics{figs/Part_4_Implementation_And_Results/example_renderer.png}
    \caption{Screenshot of the score renderer user interface which displays a score (here we show a keyboard arrangement of \textit{O Haupt voll Blut und Wunden} by Bach). The green marker represents the score follower position.}
    \label{fig:flippy_example}
\end{figure}




\begin{figure*}[ht!]
    \centering
    \includegraphics[width=0.99\linewidth]{plots/squares_p4_and_z1z2_d5000_10seeds_overlaps.pdf}
    \caption{Overlap $\normf{\bM}^2 / \Tr(\bQ)$ as a function of the sample complexity $\alpha$. The dots represent numerical simulation results, computed for $n = 5000$ (for the asymmetric method) or $d = 5000$ (for the symmetric method) and averaging over $10$ instances. (\textbf{Left}) Link function $g(z_1,z_2) = z_1z_2$. Solid lines are obtained from state evolution predictions eq. (\ref{eq:overlap_prod_zk},\ref{eq:examples_symmetric_general}). Dashed line at $\alpha_c \approx 0.59375$. (\textbf{Right}) Link function $g(\bz) = p^{-1}\norm{\bz}^2$, $p = 4$. Solid lines are obtained from state evolution predictions eq. (\ref{eq:overlap_asymmetric_squares},\ref{eq:examples_symmetric_general}). Dashed line at $\alpha_c = 2$.}
    \label{fig:z1z2_overlap}
\end{figure*}
In this section we illustrate the framework introduced in Section \ref{sec:main_results} to predict the asymptotic performance of the spectral estimators (\ref{eq:def:spectral_asymmetric},\ref{eq:def:spectral_symmetric}) for specific examples of link functions, providing a comparison between our asymptotic analytical results and finite size numerical simulations for the overlap between the spectral estimators and the weights $\mat{W}_\star$, defined as $m \coloneqq \nicefrac{\normf{\bM}}{\sqrt{\Tr(\bQ)}}$, where $\bM$ and $\bQ$ are the overlap matrices defined in eq. (\ref{eq:def:overlaps_amp}) correspondent to the fixed points in Lemmas \ref{result:1}, \ref{result:2}, \ref{result:3}, \ref{result:4}. In Figure \ref{fig:z1z2_overlap} we compare these theoretical predictions to numerical simulations at finite dimensions, respectively for the link functions $g(z_1,z_2) = z_1z_2$ and $g(\bz) = p^{-1}\norm{\bz}^2$. Additional numerical experiments are presented in Appendix \ref{app:example_details}.
\subsection{Asymmetric spectral method}\label{sec:examples_asymmetric}
%\subsection{$\dgout(y)$ jointly diagonalizable $\forall y$}
We provide closed-form expressions for the overlap parameter $m \coloneqq \nicefrac{\normf{\bM}}{\sqrt{\Tr(\bQ)}}$ of the spectral estimator $\matwhat_{\tens{L}}$ (\ref{eq:def:spectral_asymmetric}), for a selection of examples of link functions. The details of the derivation are given in Appendix \ref{app:example_details}.
\begin{itemize}[leftmargin=2em,wide=1pt]
    \item $g(z\in\R)$ (single-index model):
    \begin{align}
        \alpha_c &= \left(\E_{\rdm{y}\sim\Zout}\left[\left(\Var[z\big|\rdm{y}] - 1\right)^2\right]\right)^{-1},\\
        m^2 &= \Bigg(\frac{\alpha - \alpha_c}{\alpha + \alpha_c^2\E_{\rdm{y}\sim\Zout}\left[\left(\Var[z\big|\rdm{y}] - 1\right)^3\right]}\Bigg)_+
    \end{align}
    \item $g(\bz) = p^{-1}||\bz||^2$:
    \begin{equation}\label{eq:overlap_asymmetric_squares}
        \alpha_c = \frac{p}{2},\quad m^2 = \left(\frac{(2\alpha-p)}{2(\alpha+2)}\right)_+
    \end{equation}
    \item $g(\bz) = \operatorname{sign}(z_1z_2)$: 
    \begin{equation}
        \alpha_c = \frac{\pi^2}{4},\quad m^2 = \left(1-\frac{\pi^2}{4\alpha}\right)_+
    \end{equation}
    \item $g(\bz) = \prod_{k=1}^pz_k$:
   \begin{align}\label{eq:overlap_prod_zk}
        \alpha_c &= \left(\E_{\rdm{y}\sim\Zout}\left[\lambda(\rdm{y})^2\right]\right)^{-1},\\
        m^2 &= \left(\frac{\alpha - \alpha_c}{\alpha + \alpha_c^2\E_{\rdm{y}\sim\Zout}\left[\lambda(\rdm{y})^3\right]}\right)_+,
    \end{align}
    where
    \begin{equation}\label{eq:lambda_prod_zk}
        \lambda(y) = \begin{cases}
            \begin{array}{ll}
              |y|\frac{K_1(|y|)}{K_0(|y|)} +\rdm{y}- 1,   & p = 2 \\
              \frac{2G^{p, 0}_{0, p} \left( y^22^{-p}  \, \bigg| \, \begin{array}{c}
0 \\
\vect{e}_p
\end{array} \right)}{ G^{p, 0}_{0, p} \left( y^22^{-p} \, \bigg| \, \begin{array}{c}
0 \\
\bzero_p
\end{array} \right)} - 1,   & p\geq 3
            \end{array}
        \end{cases}
    \end{equation}
    and the previous expression are written in terms of the modified Bessel function of the second kind and Meijer $G$-function, with the notations $\bzero_p\in\R^p = (0,\ldots,0)^T$ and $\vect{e}_p\in\R^p=(0,\ldots,0,1)^T$.
    \item $g(z_1,z_2) = z_1z_2^{-1}$: $
        \alpha_c = 1,\quad m^2 = (1 -\alpha^{-1})_+$
\end{itemize}

\subsection{Symmetric spectral method}
We provide expressions for the overlap parameter $m \coloneqq \nicefrac{\normf{\bM}}{\sqrt{\Tr(\bQ)}}$ of the spectral estimator $\matwhat_{\tens{T}}$ (\ref{eq:def:spectral_symmetric}), for a selection of examples of link functions.
In all the following cases, the state evolution equations simplify, allowing to write the results as functionals of $\lambda:\R\to\R$, which is a specific to each problem:
\begin{itemize}
    \item $g(z\in\R)$ (single-index model): $\lambda(y) = \Var[z\big|y] - 1$;
    
    \item $g(\bz) = p^{-1}||\bz||^2$: $\lambda(y) = y - 1$;
    \item $g(\bz) = \operatorname{sign}(z_1z_2)$: $\lambda(y) = 2\pi^{-1}y$;
    \item $g(\bz) = \prod_{k=1}^pz_k$: $\lambda(y)$ defined in eq. (\ref{eq:lambda_prod_zk}).
\end{itemize}
For all these example, the value $\alpha_c$ can be found in Section \ref{sec:examples_asymmetric}.
For $\alpha>\alpha_c$, consider $a$ and $\gamma$ solutions of
\begin{align}\label{eq:examples_symmetric_a}
&\E_{\rdm{y}\sim\Zout}\left[\frac{\lambda(\rdm{y})^2}{a(1 + \lambda(\rdm{y})) - \lambda(\rdm{y}) }\right] = \frac{1}{\alpha}\\
\label{eq:examples_symmetric_gamma}
&\gamma = 1 + \alpha\E_{\rdm{y}\sim\Zout}\left[\frac{\lambda(\rdm{y})}{a(1 + \lambda(\rdm{y})) - \lambda(\rdm{y}) }\right].
\end{align}
Then, for any $\alpha$, the overlap $m \coloneqq \nicefrac{\normf{\bM}}{\sqrt{\Tr(\bQ)}} $ is given by
\begin{equation}\label{eq:examples_symmetric_general}
    m^2 = \Bigg(\frac{1 - \alpha\E_{\rdm{y}\sim\Zout}[\lambda^2(\rdm{y})\left(a(1 + \lambda(\rdm{y})) - \lambda(\rdm{y})\right)^{-2}]}{1 + \alpha\E_{\rdm{y}\sim\Zout}[\lambda^3(\rdm{y})\left(a(1+ \lambda(\rdm{y})) - \lambda(\rdm{y})\right)^{-2}] } \Bigg)_+,
\end{equation}
which is strictly positive $\forall \alpha > \alpha_c$. Additional details on the derivation of this result can be found in Appendix \ref{app:details_examples_symmetric}.

\section{Conclusion}
In this work, we propose a simple yet effective approach, called SMILE, for graph few-shot learning with fewer tasks. Specifically, we introduce a novel dual-level mixup strategy, including within-task and across-task mixup, for enriching the diversity of nodes within each task and the diversity of tasks. Also, we incorporate the degree-based prior information to learn expressive node embeddings. Theoretically, we prove that SMILE effectively enhances the model's generalization performance. Empirically, we conduct extensive experiments on multiple benchmarks and the results suggest that SMILE significantly outperforms other baselines, including both in-domain and cross-domain few-shot settings.

\bibliography{LPVS25,ieeetr}
\end{document}
