\section{Introduction\label{sec:intro}}

The general concept of \emph{linear parameter-varying} (LPV) systems has been introduced to provide
an easily applicable and deployable analysis and control synthesis framework for \emph{nonlinear} (NL) systems based on the extension of powerful approaches of the \emph{linear time-invariant} (LTI) framework \citep{mohammadpour2011lpv,Toth2010SpringerBook,APKARIAN19951251}. In LPV systems, the signal relations between inputs and outputs are considered linear but, at the same time, dependent on a so-called \emph{scheduling variable} $p$. %The variable $p$ is intentionally assumed to be a measurable and free signal in the modeled system. 
In this way, the variation of $p$ represents changing operating conditions, nonlinear effects, etc., described by an underlying \emph{scheduling map}, and aims at complete embedding of the original NL behavior into the solution set of an LPV system representation \citep{Toth2010SpringerBook,Rugh00}. While the former objective is pursued by the so-called \emph{global} LPV modeling approaches; alternatively, one can aim for only an approximation of the NL behavior by interpolation of various linearizations of the system around operating points or signal trajectories, often referred to as \emph{local} modeling; see, \emph{e.g.}, \citep{Toth14JPC,Shamma90c}.

Although many practical approaches have been introduced for local LPV modeling and have also been implemented in various software packages such as \textsc{Matlab}, the development of global LPV modeling methods has been limited, especially in terms of widely usable software implementations.
Existing approaches for \emph{global} LPV modeling of NL dynamical systems can be classified into two main categories: \emph{substitution based transformation} (SBT) methods \citep{Shamma93, Papageorgiou00,Bokor2007,Toth11ACC_Philips,Rugh00,Leith98b,Marcos04} and \emph{automated conversion procedures} \citep{Donida2009,Kwiatkowski06,Kwiatkowski4494453,Toth2010SpringerBook,Hoffmann2015,Toth19TAC,Toth20bIET,ROTONDO201544}. Specifically, \cite{Donida2009} proposed a promising symbolic conversion method based on \textsc{Modelica}, but the approach has never been released as an available software package. In \cite{Toth19TAC}, a multipath feedback linearization-based LPV model conversion method has been proposed that provides global embedding guarantees, but can generally result in dependence of $p$ on various derivatives of the input and output of the system.
Summand decomposition and factorization of nonlinear terms have been proposed in various forms such as in \cite{Kwiatkowski06,Toth2010SpringerBook,ROTONDO201544}, resulting in applicable conversion methods, which also allow choosing between complexity and conservativeness of possible LPV embeddings. However, these techniques need the exploration of all possible decomposition and division possibilities, leading to decision trees with a combinatorial explosion of possibilities and suffering from high computational load due to the involved symbolic manipulations and lack of reliable software implementation. In \cite{Hoffmann2015}, a \emph{linear matrix inequalities}-based optimization approach has been proposed to efficiently solve the choice of optimal trade-off between the combination of various conversion possibilities. Even if this provided a powerful tool, it is limited to small or moderate state and scheduling dimensions and requires pre-proposed conversion terms. All these problems have been overcome by data-driven techniques \citep{Kwiatkowski4494453,Toth20bIET}, representing an evolution of conversion methods based on \emph{principal components analysis} (PCA), capable of automatic determination of the relevant scheduling dimension and the construction of a data-based scheduling map. These methods have also been implemented in toolboxes such as \textsc{LPVcore}. However, success of these modeling techniques depends largely on the available data sets and they are inherently approximative in their nature. Hence, in general there is a lack of reliable and efficiently computable analytic LPV model conversion methods that ensure global embedding of the original NL system and available in terms of an off-the-shelf software solution.

Recently, the idea of a novel LPV modeling procedure has been proposed in the PhD thesis \citep{koelewijnAnalysisControlNonlinear2023} for general NL models using the \emph{Second Fundamental Theorem of Calculus} (FTC) to factorize matrix function expressions without any approximation and alleviating the need for summand decompositions, decision trees, or other complex embedding processes. Then, the approach has been successfully applied to construct a solution to various problems, such as in \citep{Toth23CDCj}, but has never been properly formulated and discussed as a global LPV embedding and model conversion tool and no software implementation has been developed for this purpose. In this paper, our main contributions are as follows:
\begin{itemize}
    \item Mathematically rigorous formulation of the global LPV embedding of a rather general class of NL systems using an FTC-based factorization of nonlinearites.
    \item Analysis of the properties of the resulting systematic LPV modeling process, discussing its pros and cons.
    \item  Developing a publicly available software implementation of the proposed FTC-based conversion in the toolbox \textsc{LPVcore} and demonstrating the capabilities of the approach in practical examples.
\end{itemize}

The paper is structured as follows: In Section~\ref{sec:probdef}, the problem of global embedding-based LPV model construction for nonlinear systems is introduced. For this problem, the proposed concept of FTC-based model conversion is proposed in Section~\ref{sec:method}, also discussing its properties. Then, in Section~\ref{sec:implementation} implementation of the approach in \textsc{Matlab} is discussed, followed in Section~\ref{sec:examples} by the demonstration of the conversion capabilities of the developed tooling on LPV model-based controller design for an unbalanced disk system and LPV modeling of a three-degree-of-freedom control moment gyroscope. Finally, in Section~\ref{sec:conclusion}, the main conclusions on the achieved results are drawn.





