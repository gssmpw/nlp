\section{FTC embedding from paper Roland sent}

The core idea is that if given a continuously differentiable function $g:\mathbb{R}^n\rightarrow\mathbb{R}^m$, then based on the FTC~\cite{Koelewijn2023}:
Let us assume we have a continuously differentiable function $g:\mathbb{R}^n\rightarrow\mathbb{R}^m$, with $g(0)=0$, for which we have that $g(\eta)=\begin{bmatrix}
		g_1(\eta) & \cdots & g_m(\eta)
	\end{bmatrix}$, where $g_i(\eta):\mathbb{R}^n\rightarrow\mathbb{R}$ for $ i = 1,\,\dots,\,m$ and $\eta\in\mathbb{R}^n$. Let us define the element\nobreakdash-wise functions $\bar{g}_i(\lambda)=g_i(\lambda \eta)$ for $i = 1,\,\dots,\,m$, with $\bar{g}_i:[0,1]\rightarrow\mathbb{R}$. Using the second fundamental theorem of calculus:
\begin{equation}
	\bar{g}_i(1)-\bar{g}_i(0)=\int_0^1 \frac{d \bar{g}_i}{d\lambda}(\bar{\lambda})\;d\bar{\lambda},\quad\text{for }i=1,\dots\,m,
\end{equation}
results in
\begin{align}
	g_i(\eta)-\underbrace{g_i(0)}_0 & =\int_0^1\left(\frac{d g_i}{d\eta}(\bar{\lambda}\eta)\right)\eta \;d\bar{\lambda}, \quad\text{for }i=1,\dots\,m, \\
	g_i(\eta)                       & =\left(\int_0^1\frac{d g_i}{d\eta}(\bar{\lambda}\eta)\,d\bar{\lambda}\right)\eta.
\end{align}
Combining the elements, $g$ is expressed as
\begin{subequations}
	\begin{align}
		g(\eta)- g(0) & =                                                                                                                        %\begin{bmatrix}
		%g_1(\eta)\\\vdots\\g_m(\eta)
		%\end{bmatrix} 
		{\small
		\begin{bmatrix}
			\left(\int_0^1\frac{d g_1}{d\eta}({\lambda}\eta)\,d{\lambda}\right)(\eta-0) \\\vdots\\\left(\int_0^1\frac{d g_m}{d\eta}({\lambda}\eta)\,d{\lambda}\right)(\eta-0)
		\end{bmatrix}} \\
		g(\eta)- g(0) & =\left(\int_0^1\frac{d g}{d\eta}({\lambda}\eta)\,d{\lambda}\right)\eta,
	\end{align}
\end{subequations}
where $\frac{d g}{d\eta}({\lambda}\eta) \in \mathbb{R}^{m \times n}$ is the Jacobian of $g$ evaluated in ${\lambda}\eta$. Provided that functions $f$ and $h$ in \eqref{eq:nl_dyn} are differentiable, choosing  $\eta=[\
	\hat{x}^\top \ \ u^\top \ ]^\top$ gives via the FTC that
%\begin{equation*}%\label{eq:abcd_def}
\begin{align}
	\tilde{f}_{\hat \theta}(\hat{x},u) & \!=\!\underbrace{  \left({\int_0^1\!\!\frac{\partial f}{\partial \hat{x}}({\lambda}\hat{x},{\lambda}u)\,d\lambda}\right)}_{{A}(\hat{x},u)}\!\hat{x}\!+\!\underbrace{\left(\int_0^1\!\!\frac{\partial f}{\partial u}({\lambda}\hat{x},{\lambda}u)\,d\lambda\right)}_{{B}(\hat{x},u)}\!u, \notag \\
	\tilde{h}_{\hat \theta}(\hat{x})   & \!=\!\underbrace{\left(\int_0^1\!\!\frac{\partial h}{\partial \hat{x}}({\lambda}\hat{x})\,d\lambda\right)}_{{C}(\hat{x})}\!\hat{x}, \label{eq:abcd_def}                                                                                                                                        %\!+\!\underbrace{\left(\int_0^1\!\!\frac{\partial h}{\partial u}({\lambda}\hat{x},{\lambda}u)\,d\lambda\right)}_{{D}(\hat{x},u)}\!u,
\end{align}
%\end{equation*}
\vskip -2mm \noindent where $	\tilde{f}_{\hat \theta}(\hat{x},u)=f_{\hat \theta}(\hat{x},u)-	f_{\hat \theta}(0,0)$ with $\tilde{h}_{\hat \theta}$ similarly defined. With $p_k=[\
	\hat{x}_k^\top \ \ u_k^\top \ ]^\top$ and $\mu$ the identity function, this gives an LPV form of the ANN-SS model \eqref{eq:model_structure} with affine terms $V=	f_{\hat \theta}(0,0)$ and  $W=	h_{\hat \theta}(0)$ as
\begin{subequations}
	\label{eq:lpv_model}\begin{align}
		\hat{x}_{k+1} & = A(p_k) \hat{x}_k + B(p_k)  u_k + V, \\
		y_k           & = C(p_k)\hat{x}_k                     %+ D(p_k)  u_k
		+ W.
	\end{align}
\end{subequations}