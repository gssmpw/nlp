\section{Problem definition\label{sec:probdef}}
%\com{I just wrote an initial problem definition as a starting point. Probably overcomplicated. I did not explicitly state that this problem is describing a global embedding, but I think it should be specified. Also, for simplicity, I formulated the problem of embedding the system for the complete $\mathcal{X} \times \mathcal{U}$ space, instead of for a region of it.}

Consider a nonlinear dynamical system given by the \emph{state-space} (SS) representation 
\begin{subequations}
	\label{eq:nl_dyn}
\begin{align}
	%\begin{aligned}
		\diff x(t) & = f(x(t), u(t)), \\
		y(t)       & = h(x(t), u(t)),
	%\end{aligned}
\end{align}
\end{subequations}
where $t \in \mathbb{T}$ is time, $\diff$ is $\diff x(t)= \frac{d}{dt}x(t)$ in the continuous-time case with $\mathbb{T}=\mathbb{R}$ and $\diff x(t)= x(t+1)$ in the discrete-time case with $\mathbb{T}=\mathbb{Z}$, $x(t) \in \mathcal{X} \subseteq \R^{\nx}$, $u(t) \in \mathcal{U} \subseteq \R^{\dnu}$ and $y(t) \in \mathcal{Y} \subseteq \R^{\ny}$ with $\nx, \dnu, \ny \in \mathbb{N}$ are the state, input, and output signals associated with the system, respectively. The functions $f: \R^{\nx} \times \R^{\dnu} \to \R^{\nx}$ and $h: \R^{\nx} \to \R^{\ny}$ are continuously differentiable once, i.e., $f, h \in \mathcal{C}_1$, and the solutions of \eqref{eq:nl_dyn} have left compact support, they are forward complete and unique, i.e., for any initial condition $x(t_0)$ and any input trajectory $u(t)$, the solutions of \eqref{eq:nl_dyn} are uniquely determined for all $t \geq t_0$. Moreover, $(0, 0) \in \mathcal{X} \times \mathcal{U}$. Finally we denote the behavior of \eqref{eq:nl_dyn}, i.e., the set of all possible trajectories by
%
\begin{equation}
	\label{eq:nl_behavior}
	\mathcal{B} \coloneq \left\{ (x, u, y) \in \left(\mathcal{X}, \mathcal{U}, \mathcal{Y}\right)^{\mathbb{T}} \mid (x, u, y) ~ \text{satisfy}~\eqref{eq:nl_dyn} \right\},
\end{equation}
%
where $\mathcal{X}^{\mathbb{T}}$ denotes the set of all signals $\mathbb{T} \rightarrow \mathcal{X}$ with left compact support.

In this paper, our objective is to automatically convert the system description \eqref{eq:nl_dyn} into a \emph{linear parameter-varying} (LPV) representation of the form
%
\begin{subequations}
	\label{eq:lpv_dyn}
	\begin{align}
		\diff x(t) & = A(p(t))x(t) + B(p(t))u(t); \\
		y(t)       & = C(p(t))x(t) + D(p(t))u(t),
	\end{align}
\end{subequations}
%
where $p \in \Pset \subseteq \R^{\np}$ with $\np \in \mathbb{N}$ is the scheduling variable, $A : \Pset \rightarrow \R^{\nx \times \nx}$, $B : \Pset \rightarrow \R^{\nx \times \dnu}$, $C : \Pset \rightarrow \R^{\ny \times \nx}$, and $D : \Pset \rightarrow \R^{\ny \times \dnu}$ are smooth real-valued matrix functions, and the solutions of \eqref{eq:lpv_dyn} with left compact support are forward complete and unique. For a given scheduling trajectory $p(t)\in\mathcal{P}$, the behavior of \eqref{eq:lpv_dyn} is defined as
%
\begin{equation}
	\label{eq:p_behavior}
	\mathcal{B}_p \coloneq \left\{ (x, u, y) \in \left(\mathcal{X}, \mathcal{U}, \mathcal{Y}\right)^{\mathbb{T}} \mid (x, u, y, p) ~ \text{satisfy}~\eqref{eq:lpv_dyn} \right\},
\end{equation}
%
and the behavior of \eqref{eq:lpv_dyn} for all possible scheduling trajectories is defined as
%
\begin{equation}
	\label{eq:lpv_behavior}
	\mathcal{B}_{\text{LPV}} \coloneq \bigcup_{p \in \Pset^{\mathbb{T}}} \mathcal{B}_p(p).
\end{equation}


We consider the LPV representation \eqref{eq:lpv_dyn} to be a so-called global LPV embedding of \eqref{eq:nl_dyn} if in addition, we can construct a so-called \emph{scheduling map} $\eta : \mathcal{X} \times \mathcal{U} \rightarrow \Pset$,
%
\begin{equation}
	\label{eq:scheduling_map}
	p(t) = \eta(x(t), u(t)),
\end{equation}
such that
%
\begin{subequations}
\label{eq:LPV_realization}
	\begin{align}
		f(x, u) & = A(\eta(x, u)) x + B(\eta(x, u)) u, \\
		h(x, u) & = C(\eta(x, u)) x + D(\eta(x, u)) u, 
	\end{align}
\end{subequations}
%
for all $(x, u) \in \mathcal{X} \times \mathcal{U}$. This gives that $\eta(\mathcal{X},\mathcal{U})\subseteq \mathcal{P} \subseteq \mathbb{R}^{n_\mathrm{p}}$, where $\mathcal{P}$ is often chosen be a compact convex set, if the model is further utilized for analysis or control synthesis.
Consequently, the behavior of the nonlinear system is included (embedded) in the behavior of the LPV system, i.e., $\mathcal{B} \subseteq \mathcal{B}_{\text{LPV}}$. In the next section, we will discuss the proposed method to achieve this objective.


% , and the scheduling variable $p(t)$ can be expressed as a function of the states and inputs of the system via the so-called \emph{scheduling map} $\eta$ as $p(t) = \eta(x(t), u(t))$, \com{which is often expected to belong to a certain function class, i.e. affine, polynomial, etc.} In the next section, we will discuss the proposed method to achieve this objective