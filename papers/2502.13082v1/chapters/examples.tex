\section{Examples \label{sec:examples}}
In this section, the capabilities of the proposed LPV embedding method are demonstrated through two examples. The first example involves a simple nonlinear model of an unbalanced disk. This example is used to analyze the LPV model obtained with the proposed approach, evaluate the accuracy of the LPV embedding via time-domain simulations, and to design an LPV controller using the obtained model. The second example considers a more complex system to assess the scalability of the proposed method. Specifically, the Quanser 3DOF \emph{control moment gyroscope} (CMG) is considered, and the performance of the method is demonstrated through time-domain simulations.

\subsection{Unbalanced disk \label{sec:unbalanced_disk}}
The unbalanced disk system consists of an actuated disk with one degree of freedom with a mass mounted on it, as shown in Fig.~\ref{fig:unbalanced_disk}. The motion of the unbalanced disk can be described in the form of \eqref{eq:nl_dyn} by
\begin{figure}[t]
    \centering
    \includegraphics[width=0.4\columnwidth]{fig/unbalanced_disk_figure.jpg}
    \caption{Picture of the unbalanced disk system.}
    \label{fig:unbalanced_disk}
\end{figure}
%
\begin{equation} \label{eq:unbalanced_disk}
    \hspace{-2mm}  \Pi_\mathrm{NL} \coloneq \left\{
    \begin{aligned}
        \dot{x}_1(t) & = x_2(t);                                                                                   \\
        \dot{x}_2(t) & = \tfrac{M g l}{J} \sin(x_1(t)) - \tfrac{1}{\tau} x_2(t) + \tfrac{K_\mathrm{m}}{\tau} u(t); \\
        y(t)         & = x_1(t),
    \end{aligned}\right.
\end{equation}
where $x_1$ and $x_2$ represent the angle of the disk in radians and its angular velocity in radians per second, respectively, $u$ the input voltage to the motor in Volts, $y$ the output of the system, and $M = \num{7e-2} \ \mathrm{kg}$, $g = \num{9.8} \ \mathrm{m \cdot s^{-2}}$, $l = \num{4.2e-2} \ \mathrm{m}$, $J = \num{2.2e-4} \ \mathrm{kg \cdot m^2}$, $\tau = \num{5.971e-1} \ \mathrm{s}$, and $K_\mathrm{m} = \num{1.531e1} \ \mathrm{rad \cdot s^{-1} \cdot V^{-1}}$ are the physical parameters.
%\footnote{\com{If we have additional paper room, I have prepared below a table to put these parameters better.}}. This is fine.
%
% \begin{table}[t]
%     \tiny
%     \centering
%     \caption{Physical parameters of the unbalanced disk}
%     \renewcommand{\arraystretch}{1.2}
%     \begin{tabular}{l c c c c c c}
%         \hline
%         \textbf{Parameter} & $g$       & $J$          & $K_\mathrm{m}$     & $l$          & $M$        & $\tau$           \\
%         \hline
%         \textbf{Value}     & \num{9.8} & \num{2.2e-4} & \com{\num{1.1e10}} & \num{4.2e-2} & \num{7e-2} & \com{\num{4e-1}} \\
%         \hline
%     \end{tabular}
%     \label{tab:unbaldisk_param}
% \end{table}

% \begin{table}[t]
%     \centering
%     \caption{Physical parameters of the unbalanced disk}
%     \renewcommand{\arraystretch}{1.2} % Adjust row height for better readability
%     \begin{tabular}{c||c|c|c|c|c|c}
%         Parameter & \( g \) & \( J \) & \( K_m \) & \( l \) & \( M \) & \( \tau \) \\
%         \hline
%         Value & 9.8 & \num{2.2e-4} & 15.31 & 0.042 & 0.076 & 0.59 \\
%     \end{tabular}
%      \label{tab:unbaldisk_param}
% \end{table}

For illustration, we apply the proposed approach manually first and then we compare the resulting LPV model to the model obtained by the implementation in \textsc{LPVcore}. By applying~\eqref{eq:simplified_factorized_nl} to~\eqref{eq:unbalanced_disk}, we get
%
\begin{subequations}
    \begin{align}
        \dot{x}(t) & = \bar{A}(x, u)x(t) + \bar{B}(x, u) u(t); \\
        y(t)       & = \bar{C}(x, u) x(t),
    \end{align}
    with
    \begin{align*}
        \bar{A}(x, u) & = \begin{bmatrix}
                              0                                                                          & 1               \\
                              \frac{M g l}{J} \! \int_{0}^{1} \cos\left(\lambda x_1(t) \right) d \lambda & -\frac{1}{\tau}
                          \end{bmatrix};                    \\
        \bar{B}(x, u) & = \begin{bmatrix}
                              0 & \frac{K_\mathrm{m}}{\tau}
                          \end{bmatrix}^\top; \qquad \bar{C}(x, u)                                                       = \begin{bmatrix}
                                                                                                                               1 & 0
                                                                                                                           \end{bmatrix},
    \end{align*}
\end{subequations}
where $x = \operatorname{col}(x_1, x_2)$. The analytical solution of the integral provides $\int_{0}^{1} \cos\left(\lambda x_1(t) \right) \; d \lambda = \frac{\sin(x_1(t))}{x_1(t)} = \operatorname{sinc}(x_1(t))$. Then, we can define the scheduling map in \eqref{eq:scheduling_map} to be $\eta(x, u) = \operatorname{sinc}(x_1)$ to obtain the following  LPV representation with affine coefficient dependence:
%
\begin{subequations} \label{eq:unbalanced_LPV_manual}
    \begin{equation}
        \hspace{-2mm} \Pi_\mathrm{LPV}^\mathrm{ana} \coloneq \left\{
        \begin{aligned}
            \dot{x}(t) & = \begin{bmatrix}
                               0 & 1 \\ \frac{M g l}{J} p(t) & -\frac{1}{\tau}
                           \end{bmatrix}x(t) + \begin{bmatrix}
                                                   0 \\ \frac{K_\mathrm{m}}{\tau}
                                               \end{bmatrix} u(t); \\
            y(t)       & = \begin{bmatrix}
                               0 & 1
                           \end{bmatrix} x(t),
        \end{aligned} \right.
    \end{equation}
    with
    \begin{align}
        p(t) & = \operatorname{sinc}(x_1(t)),
    \end{align}
\end{subequations}
%
which is an LPV embedding of~\eqref{eq:unbalanced_disk} with $\mathcal{P}= [-0.22, \, 1]$. Now, as indicated in Section~\ref{sec:implementation}, we implement~\eqref{eq:unbalanced_disk} as a \lstinline{nlss} object and execute the LPV embedding with the function \lstinline{nlss2lpvss} specifying an analytical integration of \eqref{eq:simplified_factorized_nl}. By specifying the input argument \mbox{\lstinline{'factor'}}, \lstinline{nlss2lpvss} produces the same solution as in~\eqref{eq:unbalanced_LPV_manual}. With the input argument \mbox{\lstinline{'element'}}, the following equivalent LPV model is obtained
%
\begin{subequations} \label{eq:unbalanced_LPV_auto}
    \begin{equation}
        \begin{aligned}
            \dot{x}(t) & = \begin{bmatrix}
                               0 & 1 \\ p(t) & -\frac{1}{\tau}
                           \end{bmatrix}x(t) + \begin{bmatrix}
                                                   0 \\ \frac{K_\mathrm{m}}{\tau}
                                               \end{bmatrix} u(t); \\
            y(t)       & = \begin{bmatrix}
                               0 & 1
                           \end{bmatrix} x(t),
        \end{aligned}
    \end{equation}
    with
    \begin{align}
        p(t) & = \frac{M g l}{J} \operatorname{sinc}(x_1(t)),
    \end{align}
\end{subequations}
resulting in a $\mathcal{P}=[-28.45, \, 130.96]$.
To show what happens if symbolic integration is substituted with numerical integration of~\eqref{eq:simplified_factorized_nl}, an LPV model $\Pi_\mathrm{LPV}^\mathrm{num}$ is produced. Note that, due to symbolic integration in Matlab, the $\operatorname{sinc}$ function in the scheduling map is expressed as $\frac{\sin(x_1)}{x_1}$, which is theoretically well-defined in the limit, but leads to computational issues when directly evaluated at $x_1 = 0$. This issue can be solved in the software by redefining the produced scheduling map as a piecewise function, where the output at $x_1 = 0$ is explicitly defined using limit.

As the proposed LPV embedding approach is exact, the solutions of the nonlinear system are expected to exactly match those of the LPV models produced.
% To verify this, we compare the simulation of the nonlinear system in~\eqref{eq:unbalanced_disk} with the self-scheduled simulation of the LPV models produced by \lstinline{nlss2lpvss} using the \lstinline{'analytical'} and \lstinline{'numerical'} integration settings. 
To verify this, we compare the simulation of the nonlinear system $\Pi_\mathrm{NL}$ in~\eqref{eq:unbalanced_disk} with the self-scheduled simulation of the LPV models $\Pi_\mathrm{LPV}^\mathrm{ana}$ and $\Pi_\mathrm{LPV}^\mathrm{num}$ produced by \lstinline{nlss2lpvss}. We use the \textsc{Matlab} in-built variable-step solver \lstinline{ode45} with the default parameters and step size, and simulate for 15 seconds.
\begin{figure}[t]
    \centering
    \includegraphics[width=\columnwidth]{fig/unbaldisk_embeddingtest_nolegend.pdf}
    \caption{Simulation of the nonlinear unbalanced disk $\Pi_\mathrm{NL}$ (\crule{0, 0.447, 0.7410}{8pt}{1.2pt}), self-scheduled simulation of the LPV models $\Pi_\mathrm{LPV}^\mathrm{ana}$ (\crule{0.85, 0.325, 0.098}{4pt}{1.2pt}\,\crule{0.85, 0.325, 0.098}{4pt}{1.2pt}) and  $\Pi_\mathrm{LPV}^\mathrm{num}$ (\crule{0.929, 0.694, 0.125}{3pt}{1.2pt}\,\crule{0.929, 0.694, 0.125}{1pt}{1.2pt}\,\crule{0.929, 0.694, 0.125}{3pt}{1.2pt}).}
    \label{fig:sim_unbaldisk_LPVembedding}
\end{figure}
The system is initialized with the mass at the upwards position with zero velocity, i.e. $x(0) = \operatorname{col}(0,0)$, and we consider $u(t) = 2 \sin(0.2 \pi t)$ as the input to the system. The results in Fig.~\ref{fig:sim_unbaldisk_LPVembedding} show that the solution of the nonlinear system is equivalent to the solutions of the LPV models.
The \emph{root-mean-square error} (RMSE) between the state responses of the nonlinear simulation and the LPV models are reported in Table.~\ref{tab:RMSE_lpvembeddings}, indicating a negligible error due to numerical precision.
\begin{table}[t]
    \centering
    % \captionsetup{width=1.2\columnwidth}
    \caption{RMSE between the simulation of the nonlinear model $\Pi_\mathrm{NL}$ and the self-scheduled simulation of the LPV models $\Pi_\mathrm{LPV}^\mathrm{ana}$ and $\Pi_\mathrm{LPV}^\mathrm{num}$.}
    %\renewcommand{\arraystretch}{1.2}
    \begin{tabular}{l l l}
        \hline
                                                              & \textbf{State} ($x_1$) & \textbf{State} ($x_2$) \\
        \hline
        \vspace{1mm} RMSE for $\Pi_\mathrm{LPV}^\mathrm{ana}$ & \num{3.18e-13}         & \num{3.67e-12}         \\
        RMSE for $\Pi_\mathrm{LPV}^\mathrm{num}$              & \num{5.82e-14}         & \num{6.67e-13}         \\[0.5mm]
        \hline
    \end{tabular}
    \label{tab:RMSE_lpvembeddings}
\end{table}

Lastly, we test the obtained LPV models for LPV controller synthesis. For this, we construct the generalized plant structure %\footnote{\com{I literally used the same gen plant and filters as in page 95 of Patrick's thesis. Should we cite again here?}} No....
as depicted in Fig.~\ref{fig:genSS},
\begin{figure}[t]
    \centering
    \includegraphics[width=0.85\columnwidth]{fig/genSS_diagram-eps-converted-to.pdf}
    \caption{Generalized plant structure for the controller synthesis of the unbalanced disk.}
    \label{fig:genSS}
\end{figure}
using the \textsc{LPVcore} extension of the \textsc{Matlab} function \lstinline{connect}. In the generalized plant, $G$ is an LPV representation of the unbalanced disk, $K$ is the to-be-synthesized LPV controller, $r$, $d_\mathrm{i}$ and $d_\mathrm{o}$ are disturbance channels and $z_1$ and $z_2$ the generalized performance channels. Specifically, $r$ is the reference, $d_\mathrm{i}$ is an input load and $d_\mathrm{o}$ is an output disturbance. The controller $K$ is designed in a two-degrees of freedom configuration, corresponding to a joint design of a feedforward and a feedback controller. This can be seen from the reference trajectory and tracking error being two separate inputs to the controller. The weighting filters are chosen as
    %
    \begin{align}
         & W_{\mathrm{z1}}(s)  = \frac{0.5012s + 3.0071}{s + 0.0301}; &  & W_\mathrm{z2} (s)  = \frac{10s + 400}{s + 4000}, \nonumber \\
         & W_\mathrm{di}       = 0.5;                                 &  & W_\mathrm{do}      = 0.1,
    \end{align}
    where $W_{\mathrm{z1}}$ has low-pass characteristics to approximate integral action and ensure good tracking performance at low frequencies, $W_{\mathrm{z2}}$ has high-pass characteristics to enforce roll-off at high frequencies, and each filter $W_\mathrm{d\ast}$ considers an upper bound on the expected magnitude of the input disturbances.
    %
    Next, we invoke the \textsc{LPVcore} \lstinline{lpvsyn} function to synthesize the LPV controllers $K^\mathrm{ana}$ and $K^\mathrm{num}$ using an $\mathcal{L}_2$-gain optimal polytopic LPV controller synthesis approach \cite{APKARIAN19951251}. Each controller is synthesized by using either $\Pi_\mathrm{LPV}^\mathrm{ana}$ with $\mathcal{P}=[-0.22, \, 1]$ or $\Pi_\mathrm{LPV}^\mathrm{num}$ with $\mathcal{P}=[-28.45, \, 130.96]$ as $G$ in the generalized plant. Both controllers result in an $\mathcal{L}_2$-gain of $\gamma = 0.954$. Then, we test the resulting controllers with a closed-loop self-scheduled simulation, as illustrated in Fig.~\ref{fig:clic_genss}.
\begin{figure}[b]
    \centering
    \includegraphics[width=0.7\columnwidth]{fig/selfsched_clic_diagram-eps-converted-to.pdf}
    \caption{Schematic representation of the self-scheduled closed-loop simulation.}
    \label{fig:clic_genss}
\end{figure}
Now, we use the \textsc{Matlab} variable-step solver \lstinline{ode15s} with the default settings and step size as the closed-loop interconnection becomes a stiff system, and simulate for 4 seconds. The system is initialized with the mass at the downwards position with zero velocity, i.e. $x(0) = \operatorname{col}(\pi, 0)$, while the controller states are initialized at zero. The input disturbances are generated as discrete-time signals with a sampling time of $T_s = 0.01 \ (s)$, with $d_i(t) \sim  \mathcal{U}(-0.5, 0.5)$ and $d_o(t)\sim  \mathcal{U}(-0.1, 0.1)$ where $\mathcal{U}$ denotes a uniform distribution, and are applied in a zeroth-order hold setting. In Fig.~\ref{fig:clic_sim}, the output of the unbalanced disk is shown together with the controller output for a reference signal that induces a swing-up motion of the mass at time zero, and recovers the downwards position afterwards. The difference between $K^\mathrm{ana}$ and $K^\mathrm{num}$ is negligible and both achieve the desired reference tracking and disturbance rejection objectives.
    %
    % \begin{equation}
    %     r(t) = \begin{cases}
    %         0 \quad \text{if} \ t < 2;       \\
    %         -\pi \quad \text{if} \ t \geq 2, \\
    %     \end{cases}
    % \end{equation}
    %

    %
    \begin{figure}[t]
        \centering
        \includegraphics[width=\columnwidth]{fig/unbaldisk_clic_nolegend.pdf}
        \caption{At the top, the angle of the unbalanced disk in closed-loop with the controllers $K^\mathrm{ana}$ (\crule{0.85, 0.325, 0.098}{4pt}{1.2pt}\,\crule{0.85, 0.325, 0.098}{4pt}{1.2pt}) and $K^\mathrm{num}$ (\crule{0.929, 0.694, 0.125}{3pt}{1.2pt}\,\crule{0.929, 0.694, 0.125}{1pt}{1.2pt}\,\crule{0.929, 0.694, 0.125}{3pt}{1.2pt}) under the reference (\crule{0, 0.447, 0.7410}{8pt}{1.2pt}). At the bottom, the inputs to the plant generated by the controllers.}
        \label{fig:clic_sim}
    \end{figure}


    \subsection{3DOF control moment gyroscope}
    The 3DOF CMG consists of three actuated gimbals and a flydisk mounted on the inner gimbal, as illustrated in Fig.~\ref{fig:geom_gyro}.
    \begin{figure}[b]
        \centering
        \includegraphics[width=0.45\columnwidth]{fig/gyro_sketch-eps-converted-to.pdf}
        \caption{Schematics of the Quanser 3DOF control moment gyroscope, where the angular positions are indicated by $q_i$ and the motor inputs by $i_i$.}
        \label{fig:geom_gyro}
    \end{figure}
    The nonlinear dynamics that describe the CMG are presented in~\citep{5841d138cdd94f17aaa413781677ac2a}, resulting in the differential equation
    %
    \begin{equation}
        M(q(t))\ddot{q}(t) + \left( C\left(q(t), \dot{q}(t) \right) + F_\mathrm{v} \right) \dot{q}(t) = K_\mathrm{m} u(t),
    \end{equation}
    %
    where $t \in \R$ is time, $q(t) \in \R^4$ are the angular positions of the gimbals, $i(t) \in \R^4$ are the input currents to the motors, $M$ is the inertia matrix, $C$ is the Coriolis matrix, $F_\mathrm{v}$ is the viscous friction matrix, and $K_\mathrm{m}$ is the motor gain matrix. For this example, we consider that the third gimbal is locked, i.e. $q_3, \ \dot{q}_3 \equiv 0$ %$\forall t \in \R$ 
    and only the first two gimbals are actuated, i.e. $i_3, \ i_4 \equiv 0$. % \ \forall \, t \in \R$. 
    As outputs of the system, we consider the angular position of the unlocked gimbals, $y = \operatorname{col}\left(q_1, q_2, q_4 \right)$. Then, the dynamics of the CMG system can be represented by a continuous-time nonlinear model of the form of~\eqref{eq:nl_dyn}, where $x(t) = \operatorname{col}\left(q_1, q_2, q_4, \dot{q}_1, \dot{q}_2, \dot{q}_4 \right)(t) \in \R^6$, $u(t) = \operatorname{col}\left(i_1, i_2 \right)(t) \in \R^2$, and $y(t) = \operatorname{col}\left(q_1, q_2, q_4 \right)(t) \in \R^3$.

    Now, two LPV embeddings $\Omega_\mathrm{LPV}^\mathrm{ana}$ and $\Omega_\mathrm{LPV}^\mathrm{num}$ of the CMG model are obtained using \lstinline{nlss2lpvss} with the respective options \lstinline{'analytical'} and \lstinline{'numerical'}, and both using \lstinline{'factor'}. In this case, while $\Omega_\mathrm{LPV}^\mathrm{ana}$ has $n_\mathrm{p} = 15$ scheduling variables, $\Omega_\mathrm{LPV}^\mathrm{num}$ contains only $n_\mathrm{p} = 13$. Note that this difference in scheduling dimensions is caused by the extraction process of the nonlinear elements from the matrix functions of the system, but both LPV models should be equivalent. Moreover, if required for any downstream tasks (such as controller synthesis), the scheduling dimension can be reduced using the LPV-SS scheduling dimension reduction methods present in \textsc{LPVcore}. To verify that the produced LPV models are equivalent to the original CMG model, we compare the simulation of the CMG model with the self-scheduled simulation of $\Omega_\mathrm{LPV}^\mathrm{ana}$ and $\Omega_\mathrm{LPV}^\mathrm{num}$. The simulation is run for 15 seconds using the \textsc{Matlab} variable step solver \mbox{\lstinline{ode15s}} with the default settings and step size. The gimbals are initialized at the angular position of $(q_1(0), q_2(0), q_4(0)) = (0, \pi/2, -\pi)$ with zero angular velocity, and we consider $i_1(t) = 0.1 \sin(t)$ and $i_2(t) = -0.3 \cos(t)$ as inputs to the system. In Fig.~\ref{fig:sim_gyro_LPVembedding}, the simulation results show that the output trajectories are equivalent. Moreover, we compute the RMSE error between the simulation outputs of the original CMG model and the LPV models and it resulted in less than \num{1e-12} in all cases.
\begin{figure}[t]
    \centering
    \includegraphics[width=\columnwidth]{fig/gyro_embeddingtest.pdf}
    \caption{Simulation of the nonlinear CMG (\crule{0, 0.447, 0.7410}{8pt}{1.2pt}), self-scheduled simulation of the LPV model $\Omega_\mathrm{LPV}^\mathrm{ana}$ (\crule{0.85, 0.325, 0.098}{4pt}{1.2pt}\,\crule{0.85, 0.325, 0.098}{4pt}{1.2pt}) and the LPV model $\Omega_\mathrm{LPV}^\mathrm{num}$ (\crule{0.929, 0.694, 0.125}{3pt}{1.2pt}\,\crule{0.929, 0.694, 0.125}{1pt}{1.2pt}\,\crule{0.929, 0.694, 0.125}{3pt}{1.2pt}).}
    \label{fig:sim_gyro_LPVembedding}
\end{figure}





