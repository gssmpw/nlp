\section{Related Work}
Achieving robust task planning in robotics requires addressing two interrelated challenges: precise 3D spatial understanding and the effective utilization of large pre-trained models. Precise 3D perception provides the geometric and spatial foundation necessary for fine-grained operations, such as grasping and trajectory planning. However, integrating these perceptual insights into large models, such as Vision-Language Models (VLMs) and Large Language Models (LLMs), is equally critical to ensure task-relevant outputs and dynamic adaptability. This section reviews progress in these areas, focusing on multimodal perception and model adaptation strategies, and highlights their limitations.
\subsection{Multimodal Perception for Robotic Operations}

Multimodal perception enhances 3D scene understanding by combining RGB, depth, and textual inputs. Current approaches can be broadly categorized into \textbf{End-to-End Models} and \textbf{Modular Architectures}. End-to-End Models, for example "RT series"____ and "RobotFlamingo"____, employ transformer-based frameworks to encode sensory inputs into unified cross-modal embeddings, enabling robust multimodal reasoning. These methods excel in learning rich representations but often lack geometric fidelity, struggle with task-specific adaptation, and require costly retraining on large annotated datasets, making them less effective in dynamic environments.

Modular architectures decouple perception and reasoning processes, leveraging pre-trained VLMs to reduce computational demands. For example, CLIP-based methods extract semantic features, while auxiliary modules handle spatial reasoning ____. Another approach integrates Reinforcement Learning (RL) modules to generate desired outputs directly ____. Despite their flexibility, these architectures face significant challenges in multimodal data fusion, particularly in addressing sensor uncertainties and prioritizing task-relevant features in cluttered or dynamic scenarios. These limitations highlight the need for adaptive frameworks capable of integrating 3D perception with dynamic task planning.


\begin{figure}[t]
	\centering
	\includegraphics[width=0.5\textwidth]{VLM3.png}
	\caption{
		The architecture of the Vision-Language Model (VLM). Inputs include a \textbf{robot task} (top-left), \textbf{segmented image} (bottom-left), and \textbf{text template} (top-right), processed by encoders to generate feature vectors. The \textbf{cross-modal alignment module} (center) integrates text and image features, producing a unified representation. The \textbf{multimodal decoder} (right) generates executable robot control codes.
	}
	
	\label{fig_2}
\end{figure}



\subsection{Large Pre-trained Models In Robotic Tasks}

Large pre-trained models (VLMs and LLMs) have emerged as powerful tools for robotic task planning ____ and navigation ____ through their multimodal reasoning capabilities ____. Their integration into robotics employs two complementary approaches: \textit{prompt engineering} that steers models through input design without parameter modification, and \textit{architectural innovations} involving structural adaptations or domain-specific retraining. 

Prompt engineering strategies, which preserve original model parameters, include textual instructions for task decomposition ____ and visual prompts encoding spatial layouts via images ____. Advanced techniques like chain-of-thought (CoT) prompting ____ further enhance complex task handling by breaking operations into logical sequences. However, their reliance on textual or 2D visual inputs introduces geometric reasoning limitations, leading to failures in precision tasks such as sub-centimeter alignment. 
Architectural innovations address these limitations at the cost of flexibility: Models like XComposer ____ leverage cross-modal attention mechanisms to align visual-textual embeddings obtaining geometric grounding ability. Mini-CPM ____, a new cross-modal attention work, optimizes computational efficiency through layer pruning. This type of mechanism(multimodal pretrained method) underpin diverse applications: Well-designed domain-specific VLMs guide navigation in large-scale environments ____, while multimodal LLMs like PALM-E ____ enable robotic arm control through embodied reasoning. Although performing well in specific domain, structure-modified models suffer from resource-intensive retraining for new environment. 

These challenges highlight three key unresolved issues: the semantic-geometric divide in complex 3D tasks, the computational burden of architectural adaptations, and the absence of automatic geometric grounding in prompt-based approaches.