\section{Related Work}

\paragraph{Role-based Instruction Management}

Recent work has highlighted the importance of role-based controls in LLM deployments through system messages. System messages have emerged as a specialized component for developers to configure model behavior, introduced prominently with ChatGPT \citep{achiam2023gpt} and adopted by various models including Mistral \citep{jiang2024mixtralexperts}, Claude \citep{claude21modelcard}, and Command R.\footnote{\url{https://docs.cohere.com/docs/responsible-use}} The evolution from early models like Llama \citep{touvron2023llama}, which used fixed system messages primarily for consistency, to more sophisticated approaches that enable dynamic behavioral control \citep{kung2023models,lee2024aligningthousandspreferencesmessage}, reflects the growing importance of instruction management in LLM systems.

\paragraph{Instruction Hierarchies and LLM Safety}
The management of instruction hierarchies has become particularly crucial in the context of LLM safety and security. Research on prompt injection attacks has revealed how end users can potentially bypass developer-intended constraints, leading to important insights about LLM instruction processing and deployment practices \citep{wu2024instructional,Hines2024DefendingAI, toyer2023tensor}. 
Another approach is to treat user inputs as data rather than instructions \citep{Chen2024StruQDA,liu2023prompt,zverev2024can} to prevent such bypasses. \citet{wallace2024instruction} further expanded this understanding by investigating how models prioritize different prompt elements, including system prompts, user messages, and tool outputs. The significance of instruction hierarchy in LLM safety is underscored by \citet{li2024libraleaderboardresponsibleaibalanced}, who identify it as a core safety aspect of LLMs.