\section{Introduction}
Food pairing has traditionally relied on the intuition and experience of chefs, yet scientific analysis and optimization of food combinations remain underexplored. Recent research has leveraged data-driven approaches to model the relationships between food ingredients and chemical compounds to predict novel food pairings.

Several computational approaches have been developed to model food pairings and ingredient relationships. {Kitchenette} \cite{Park2019}, for instance, applies Siamese neural networks to predict and recommend ingredient pairings based on a large annotated dataset. However, it suffers from key limitations, such as a lack of chemical interpretability and heavy reliance on labeled data, making it less generalizable across different cuisines and novel food combinations.

One of the key advancements in this domain is {FlavorGraph} \cite{Park2019}, a large-scale food-chemical deep neural network model comprising {6,653 ingredient nodes} and {1,645 compound nodes}. This graph captures two primary relationships: (1) {ingredient-ingredient relations}, representing co-occurrence patterns in recipes, and (2) {ingredient-compound relations}, indicating chemical composition links. These relationships are constructed using datasets such as {Recipe1M} \cite{Marin2019}, {FlavorDB}, and {HyperFoods}. {FlavorGraph} incorporates food-chemical associations into a neural network by leveraging the {metapath2vec} \cite{Dong2017} algorithm, which embeds ingredient-compound relationships in a word2vec-like manner. Expanding on this approach, {WineGraph} \cite{Gawrysiak2023} extends the framework by integrating wine-related datasets to define optimal food-wine pairings.

Despite progress in computational food science, major challenges remain. Chromatography-based methods, while precise, are costly and limit the acquisition of large-scale chemical interaction data. {FlavorGraph} effectively captures ingredient-compound relationships using {metapath-based embeddings}, but its reliance on {random-walk sampling} makes it difficult to incorporate edge weights and spatial information within the graph structure. These limitations hinder the full exploitation of food-chemical associations, leading to suboptimal ingredient relationship modeling. To address these challenges, we introduce {FlavorDiffusion}, a Diffusion Model-based framework that refines the representation of food-chemical interactions and elevates the quality of food pairing predictions.

\subsection*{Contributions}
\begin{itemize}
    \item We propose a {graph-based diffusion modeling approach} that leverages {DIFUSCO} \cite{Sun2023} to capture richer and more structured representations of food-chemical interactions.
    \item We introduce a {balanced subgraph sampling} strategy to address data imbalance issues, ensuring fair representation across different ingredient-chemical associations.
    \item Our experimental results demonstrate improvements in {Normalized Pointwise Mutual Information (NPMI)} scores for node embeddings, facilitating more effective chemical inference.
    \item We establish a foundation for predicting chromatography results for {non-hub chemicals}, extending the applicability of our model beyond frequently occurring compounds.
    \item Our approach enables {pairing inference using chemical properties}, providing structured and interpretable recommendations for novel ingredient combinations.
\end{itemize}
