\section{Dataset}

Our study builds upon {FlavorGraph} \cite{Park2019} by utilizing the same large-scale datasets to construct a robust food-chemical network. These datasets provide a structured representation of ingredient relationships and chemical interactions. In the following sections, we summarize the key characteristics of these datasets and outline the preprocessing steps applied to ensure data consistency and usability in our framework.

\begin{table}[h]
  \centering
  \begin{tabular}{lccc}
    \hline
    {Type} & {Source} & {Nodes} & {Edges} \\
    \hline
    I-I & Recipe1M & 6,653 & 111,355 \\
    I-FC & FlavorDB & 1,561 & 35,440 \\
    I-DC & HyperFoods & 84 & 386 \\
    \hline
    {Total} & - & 8,298 & 147,179 \\
    \hline
  \end{tabular}
  \caption{Summary of the heterogeneous food-compound graph. I-I represents ingredient ingredient co-occurrence from Recipe1M, I-FC denotes ingredient-flavor compound associations from FlavorDB, and I-DC refers to ingredient-drug compound relations}

  \label{tab:graph_summary}
\end{table}

\subsection{Data Sources}
This study utilizes the same datasets as {FlavorGraph} \cite{Park2019} to construct a structured food-chemical network.

{Recipe1M} \cite{Marin2019} contains {65,284} recipes with ingredient lists and cooking instructions, capturing ingredient co-occurrence patterns in real-world culinary practices.

{FlavorDB} compiles chemical composition data from multiple sources, including \textit{FooDB}, \textit{Flavornet}, and \textit{BitterDB}. It originally includes {2,254 flavor compounds} linked to {936 food ingredients}, but only {400 commonly used ingredients} were selected to align with Recipe1M, resulting in {1,561 flavor compound nodes} and {164,531 ingredient-flavor compound edges}.

{HyperFoods} maps drug compounds to food ingredients using machine learning based on food-gene interactions. From the original {206 food ingredients}, {104 were selected}, yielding {84 drug compound nodes} and {386 ingredient-drug compound edges}.

\subsection{Data Processing}

To construct a structured representation of food-chemical relationships, we build upon {FlavorGraph} \cite{Park2019}, a heterogeneous graph that integrates both culinary and chemical associations. The graph construction process follows a structured approach. First, an ingredient-ingredient graph is built by extracting co-occurrence patterns from Recipe1M \cite{Marin2019}, where edges between ingredients are established based on their {Normalized Pointwise Mutual Information (NPMI)} scores. Only statistically significant ingredient pairs appearing together in a substantial number of recipes are retained, resulting in a total of {111,355 edges}. Second, an ingredient-chemical graph is formed by linking ingredients to their corresponding chemical compounds using FlavorDB and HyperFoods, leading to {35,440 edges} between food ingredients and known chemical compounds. The final graph structure comprises {6,653 ingredient nodes} and {1,645 compound nodes}, forming a {heterogeneous graph} that encodes both culinary co-occurrence relationships and chemical interactions.

\subsection{Chemical Property Encoding}
To ensure chemically informed ingredient representations, each compound is characterized using {CACTVS chemical fingerprints}, which are encoded as {881-dimensional binary vectors}. These vectors represent molecular descriptors such as molecular weight, functional groups, and substructure patterns, using a {binary encoding scheme} where each bit indicates the presence or absence of a specific chemical substructure.


