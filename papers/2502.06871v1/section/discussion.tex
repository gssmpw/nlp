\section{Discussion}

The visualization results highlight the impact of the proposed Flavor Diffusion framework on embedding quality, particularly with the CSP (Chemical Structure Prediction) layer, as shown in Figures~\ref{fig:embedding_comparison} and~\ref{fig:generation_example}.

\paragraph{Embedding Space Analysis}
Figure~\ref{fig:embedding_comparison} illustrates the differences in embedding spaces across model configurations. The baseline embeddings (left) fail to separate chemical compounds and ingredients effectively, resulting in diffuse and isotropic clusters dominated by non-hub ingredients. 

In contrast, the embeddings generated by **Flavor Diffusion (200 nodes)** without CSP (center) show improved clustering, with chemical compounds and hub ingredients forming clearer groups. However, some overlap persists between hubs and non-hubs. The inclusion of the CSP layer (right) further refines the embeddings, creating well-structured, anisotropic clusters that reflect meaningful relationships between ingredients and compounds.

\begin{figure*}[t!]
  \centering
  \includegraphics[width=0.32\linewidth]{./images/plot_baseline.png} \hfill
  \includegraphics[width=0.32\linewidth]{./images/plot_FlavorDiffusion_200.png} \hfill
  \includegraphics[width=0.32\linewidth]{./images/plot_FlavorDiffusion_200_CSP.png}
  \caption{Embedding space comparison under different configurations. 
  (Left) Baseline embeddings show poor separation between ingredients and compounds. 
  (Center) Flavor Diffusion (200 nodes) without CSP achieves improved clustering of chemical compounds and hub ingredients. 
  (Right) Flavor Diffusion (200 nodes) with CSP results in well-defined clusters, leveraging chemical fingerprints to enhance separation.}
  \label{fig:embedding_comparison}
\end{figure*}

\begin{figure}[h!]
    \centering
    \includegraphics[width=\linewidth]{./images/generation.png}
    \caption{Progression of edge scores over diffusion steps for a 25-node subgraph. The color intensity represents edge scores normalized between 0 and 1. The reconstructed graph increasingly aligns with the ground truth structure.}
    \label{fig:generation_example}
\end{figure}

\paragraph{Performance and Structural Insights}
The CSP layer significantly enhances clustering performance, as evidenced by the highest NMI mean (\textbf{0.3410}) achieved by Flavor Diffusion (200 nodes) with CSP. The transition from isotropic to anisotropic embedding spaces reflects the model's ability to learn diverse, domain-specific relationships. Furthermore, the iterative refinement process highlights the framework's capacity to generate realistic ingredient-ingredient graphs that align with culinary and chemical properties.

\paragraph{Dynamic Reconstruction for Novel Insights}
The iterative reconstruction process visualized in Figure~\ref{fig:generation_example} showcases the Flavor Diffusion framework's ability to refine ingredient-ingredient relationships progressively. Starting from random initialization (Step 0), the edge scores evolve over diffusion steps, ultimately converging towards the ground truth structure by Step 10. The color intensity of the edges reflects their normalized scores, with higher values indicating stronger relationships. This gradual alignment with the ground truth demonstrates the model's capacity to encode meaningful relational patterns in a structured manner.

\paragraph{Potential for Ingredient Innovation}

The progressive nature of the diffusion process suggests that Flavor Diffusion is not only capable of reconstructing known ingredient-ingredient relationships but also has the potential to generalize and infer connections beyond the training data. The inclusion of chemical fingerprints and iterative edge refinement allows the model to generate plausible ingredient combinations, even in scenarios involving diverse or sparse subgraph configurations. This characteristic is particularly valuable for fields such as computational gastronomy, where discovering unique and harmonious flavor pairings is a central goal.

\paragraph{Alignment with Culinary and Chemical Properties}
The alignment of the reconstructed graphs with ground truth structures further underscores the model’s fidelity in capturing culinary and chemical properties. As the diffusion process unfolds, the model demonstrates an increasing ability to balance local (ingredient-specific) and global (chemical-based) relationships. This balance not only enhances clustering quality but also provides a robust framework for extending ingredient networks in a meaningful way.