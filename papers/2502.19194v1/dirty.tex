\subsection{Constructing the Prediction Set}

The conformal prediction set \( \hat{C}(y) \) is defined as:
\[
\hat{C}(y) = \{x \in \mathbb{R}^n : s(x, y) \leq q_\alpha\},
\]
where \( q_\alpha \) is the \((1 - \alpha)\)-quantile of the calibrated score distribution. Using SURE for calibration, we estimate \( q_\alpha \) based on the variability of the residual error in the measurement space, ensuring that the resulting prediction sets are compact and centered around \( \hat{x}(y) \).

\subsection{ Advantages of the Proposed Method}

\begin{itemize}
    \item  \textbf{Adaptivity to Observations:} By calibrating the score function with SURE, our method adapts to the observed data \( y \), capturing the local variability in the measurement space.

    \item \textbf{Practical Applicability:} The reliance on observed measurements rather than ground truth \( x^\star \) makes the approach feasible in real-world applications where paired data is unavailable.


    \item  \textbf{Robustness in High Dimensions:} The focus on the measurement space mitigates the challenges posed by the high dimensionality of \( x \), resulting in prediction sets that remain compact and informative.
\end{itemize}


In the next sections, we demonstrate the effectiveness of this method through experiments on image restoration tasks, including denoising and deblurring.

### Advantages of the Updated MC-SURE Method
1. **Hyperparameter-Free:** Unlike conventional MC-SURE methods that require tuning \( \epsilon \) for finite-difference approximations, this approach avoids introducing hyperparameters.
2. **Computational Efficiency:** By leveraging automatic differentiation, the trace of the Jacobian is estimated without explicitly constructing the full Jacobian matrix, significantly reducing computational overhead.
3. **Scalability:** The method is well-suited for large-scale problems, as it only involves matrix-vector products, which scale efficiently with the problem size.

By adopting this divergence approximation method, we ensure that the proposed calibration strategy is both computationally feasible and robust to the challenges posed by high-dimensional imaging problems.
\begin{figure}
    \centering
    \includegraphics[width=0.5\linewidth]{Graphs/histogramdeblurring.png}
    \caption{Enter Caption}
    \label{fig:enter-label}
\end{figure}

\begin{figure}
    \centering
    \includegraphics[width=0.5\linewidth]{Graphs/histogramdenoising.png}
    \caption{Enter Caption}
    \label{fig:enter-label}
\end{figure}

\begin{figure}
    \centering
    \includegraphics[width=0.5\linewidth]{Graphs/histogramdeblurring.png}
    \caption{Enter Caption}
    \label{fig:enter-label}
\end{figure}