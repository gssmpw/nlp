


\section{Main Results}\label{sec:main_result}


In Section~\ref{sec:mr_privacy}, we will provide the privacy of our algorithm. Then, we will examine the utility implications of our algorithm applying a random response mechanism. 
In Section~\ref{sec:main_result:utility}, we introduce the utility guarantees of our algorithm.
In Section~\ref{sec:main_result:running_complexity}, we demonstrate that DPBloomfilter does not import the running complexity burden to the standard Bloom filter.

\subsection{Privacy for DPBloomfilter}\label{sec:mr_privacy}

Algorithm~\ref{alg:init} illustrates the application of the random response mechanism to the standard Bloom filter, thereby accomplishing differential privacy. In detail, once the Bloom filter is initialized, each bit in the $m$-bit array is independently toggled with a probability of $\frac{1}{\epsilon_0 + 1}$. Our algorithm will ensure that modifications to any element within the dataset are protected to a degree, as the DPBloomfilter maintains the privacy of the altered element. Then, we present the Theorem demonstrating that our algorithm is $(\epsilon,\delta)$-DP.

\begin{theorem}[Privacy for Query, informal version of Theorem~\ref{thm:query_privacy:formal}]\label{thm:query_privacy:informal}
Let $N := F_W^{-1}(1 - \delta)$ and $\epsilon_0 = \epsilon / N$.
Then, we can show,
the output of \textsc{Query} procedure of Algorithm~\ref{alg:init} achieves $(\epsilon, \delta)$-DP. 
\end{theorem}

Theorem~\ref{thm:query_privacy:informal} shows that our DPBloomfilter in Algorithm~\ref{alg:init} is $(\epsilon, \delta)$-DP. 
Our main technique leverages the single-bit random response technique to enhance the privacy properties of the traditional Bloom filter by composition rule (Lemma~\ref{lem:pre_com_lem}). 

\subsection{Utility for DPBloomfilter}\label{sec:main_result:utility}

Despite the introduction of privacy-preserving mechanisms, our algorithm still ensures that the utility of the Bloom Filter remains acceptable. This is achieved through careful calibration of the Random Response technique parameters, balancing the need for privacy with the requirement for accurate set membership queries. 
Here, we present the theorem for the entire utility loss between the output of our algorithm and ground truth.

\begin{theorem}[Accuracy (compare DPBloom with true-answer) for Query, informal version of Theorem~\ref{thm:dpbloom_true_accuracy:formal}]\label{thm:dpbloom_true_accuracy:informal}
If the following conditions hold
\begin{itemize}
    \item Let $z \in \{0,1\}$ denote the true answer for whether $x \in A$. 
    \item Let $\wh{z} \in \{0,1\}$ denote the answer for whether $x \in A$ output by Bloom Filter.
    \item Let $\alpha: = \Pr[ z = 0 ] \in [0,1]$, $t := e^{\epsilon_0} / (e^{\epsilon_0} + 1)$, and  $\delta_{\mathrm{err}} > 0$.
\end{itemize}
Then, we can show 
\begin{align*}
\Pr[ \wt{z} = z ] \geq \delta_{\mathrm{err}}\cdot\alpha\cdot(1-t-t^{k}) + \alpha \cdot t.
\end{align*}
\end{theorem}

Theorem~\ref{thm:dpbloom_true_accuracy:informal} shows that when most queries are not in $A$, the above theorem can ensure that the utility of DPBloomfilter has a good guarantee. Namely, in such cases, answers from DPBloomfilter are correct with high probability. 


\subsection{Running Complexity of DPBloomfilter}\label{sec:main_result:running_complexity}

Now, we introduce the running complexity for the DPBloomfilter in the following theorem. 

\begin{theorem} [Running complexity of DPBloomfilter] \label{thm:running_complexity}
Let $\mathcal{T}_h$ denote the time of evaluation of function $h$ at any point. 
Then, for the DPBloomfilter (Algorithm~\ref{alg:init}) we have
\begin{itemize}
    \item The running complexity for the initialization procedure is $O(|A| \cdot k \cdot \mathcal{T}_h + m)$.
    \item The running complexity $O(k\cdot \mathcal{T}_h)$ for a single query. 
\end{itemize}
\end{theorem}

\begin{proof}
It can be proved by combining Lemma~\ref{lem:init_time} and \ref{lem:query_time}. 
\end{proof}

Our Theorem~\ref{thm:running_complexity} shows that DPBloomfilter not only addresses the critical need to protect the privacy of elements stored with Bloom filter but also ensures that the data structure's utility remains acceptable, with minimal impact on its computational efficiency.
By keeping the running time within the same order of magnitude as the standard Bloom filter, our approach is practical for real-world applications requiring fast and scalable set operations. 

