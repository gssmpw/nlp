
{\bf Roadmap.} The Appendix is organized as follows:
In Section~\ref{sec:appendix_basic_tools}, we introduce the notations used in the paper and differential privacy tools.
In Section~\ref{sec:appendix_quantile_proof}, we elaborate the derivations for the closed-form distribution of the random variable $W$, where $N$ is the $1 - \delta$ quantile of $W$. 
Section~\ref{sec:appendix_running_time} restates the analysis results of running time for DPBloomfilter.
Section~\ref{sec:appendix_utility} presents a detailed analysis of utility guarantees for DPBloomfilter.
Section~\ref{sec:appendix_privacy_guarantees} contains the proof of privacy guarantees for DPBloomfilter.


\section{Basic Tools} \label{sec:appendix_basic_tools}
In this section, we display the notations and basic tools for a better understanding of the readers. In Section~\ref{sec:appendix:notation},  we introduce the notations used in this paper. In Section~\ref{sec:apendix:composition}, we provide an essential basic composition Lemma for Differential Privacy.
\subsection{Notations}\label{sec:appendix:notation}
In this section, we describe the notations we use in this paper.

For any positive integer $n$, let $[n]$ denote the set $\{1, 2, \cdots , n\}$. We use $\E[]$ to denote the expectation operator and $\Pr[]$ to denote probability. We use $n!$ to denote the factorial of integer $n$. We use $A_{m}^{n}:=\frac{m!}{(m-n)!}$ to denote the number of permutation ways to choose $n$ elements from $m$ elements considering the order of selection. We use $\binom{m}{n}:=\frac{m!}{n!(m-n!)}$ to denote the number of combination ways to choose $n$ elements from $m$ elements without considering the order of selection. We use $F_{X}(x)$ to denote the Cumulative Distribution Function (CDF) of a random variable $X$ and use $F_{X}^{-1}(1-\delta)$ to denote the $1-\delta$ quantile of $F_{X}(x)$.

\subsection{Differential Privacy Tools}\label{sec:apendix:composition}
In this section, we introduce the basic composition Lemma for Differential Privacy.
\begin{lemma}[Basic composition, \cite{gkk+23}]\label{lem:append_com_lem}
    Let $M_1$ be an $(\epsilon_1,\delta_1)$-DP algorithm and $M_2$ be an $(\epsilon_2,\delta_2)$-DP algorithm. 
    
    Then $M(X) = (M_1(X),M_2(M_1(X),X)$ is an $(\epsilon_1+\epsilon_2,\delta_1+\delta_2)$-DP algorithm.
\end{lemma}


