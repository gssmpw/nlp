\section*{Ethical Considerations}

In the era of big data, many carefully designed data structures are used in various scenarios, and they usually contain a large amount of sensitive user privacy information. 
Some malicious attackers will restore the user information contained in the published data structure, causing a bad social impact. 
Based on this problem, this work uses the Bloom filter data structure as an example to explore the possibility of protecting sensitive information in the Bloom filter through the properties of differential privacy.

Since these data structures often contain subtle structures, naively applying classical differential mechanisms like Gaussian or Laplace mechanisms on them will have a greater impact on the utility of the data structure.
Therefore, designing differential privacy on such data structures requires more effort and exploration. Our work is just the first step in this direction, and we have made preliminary explorations in protecting sensitive data in data structures in the context of big data.


