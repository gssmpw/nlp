


\section{Discussion} \label{sec:discussion}

Section~\ref{sec:flip_compare_gaussian_laplace} discusses why the random response mechanism is preferred over Gaussian and Laplace mechanisms for achieving differential privacy. 
In Section~\ref{sec:flip_both_directions}, we consider the underlying reasons for applying the random response mechanism to both $1$ and $0$.


\subsection{Why Random Response but not Gaussian or Laplace Noise?}\label{sec:flip_compare_gaussian_laplace}

As mentioned in Section~\ref{sec:related_work}, Gaussian and Laplace noise are two classical mechanisms to achieve differential privacy. 

The advantage of the Laplace mechanism is that its distribution is concentrated on its mean. Under the same privacy budget, it will not introduce too much noise like the Gaussian mechanism due to the long-tail nature of its distribution. The advantage of the Gaussian mechanism is that it has good mathematical properties and makes it easy to analyze the utility of private data structures.

However, the above two mechanisms are not as effective as the random response (flip coin) mechanism when dealing with discrete values. Here, we consider the case where the discrete values are integers. Under certain privacy budgets, the noise added by Gaussian and Laplace mechanisms does not reach the threshold of $0.5$, resulting in attackers being able to remove the noise through rounding operations easily, and the privacy of the data structure no longer exists.

In our case, each bit of the Bloom filter can only be $1$ or $0$, which is consistent with the above situation. Hence, our work only considers the random response mechanism instead of classical Gaussian and Laplace mechanisms.

\subsection{Why Flip Both \texorpdfstring{$0$}{} and \texorpdfstring{$1$}{}?} \label{sec:flip_both_directions}

In our work, we apply random response mechanism to each bit in the Bloom filter, either it is $0$ or $1$. Although this will lead to a certain probability of false negatives in the Bloom filter, we argue that it is necessary to make the Bloom filter differentially private.

Let's consider what will happen if we don't apply random response mechanism like this. Suppose we only apply random responses to bits that are $1$ in the Bloom filter and leave the bits with 0 untouched. 
Following the notations used in Lemma A, we use $g \in \{0, 1\}^m$ to represent the bit array generated by inserting the original dataset into the Bloom filter and $g' \in \{0, 1\}^m$ to represent the bit array generated by inserting the neighboring dataset into the Bloom filter. 
We use $\wt{g}$ and $\wt{g}'$ to denote their private version, respectively. 
Without loss of generality, for some $j \in [m]$, we assume $g[j] = 1$ and $g'[j] = 0$. 
Since we only apply random response mechanism on bits with value $1$, then $\Pr[\wt{g}' [j] = 1] = 0$. 
Therefore, we cannot calculate $\Pr[\wt{g} [j] = 1] / \Pr[\wt{g}' [j] = 1]$, since the denominator is $0$. 
Hence, we cannot have any privacy guarantees under this setting. 
Similar situations occur when we apply a random response mechanism on bits with value $0$. We also cannot prove the differential privacy property of the Bloom filter. 
Therefore, we have to apply the random response mechanism on bits either with value $0$ or $1$. 





