\section{Privacy guarantees for one coordinate}\label{sec:appendix_privacy_guarantees}
In this section, we provide proof of the privacy guarantees of the DPBloomfilter.

In Section~\ref{sec:single_bit_private}, we demonstrate the privacy guarantees for single bit of array in Bloom filter.

Then in Section~\ref{sec:query_privacy}, we provide the proof of privacy guarantees for our entire algorithm.

\subsection{Single bit is private} \label{sec:single_bit_private}
We first consider the privacy guarantees of single bit of array in Bloom filter.
\begin{lemma} [Single bit is private
] \label{lem:eps0_DP:formal}
If the following conditions hold:
\begin{itemize}
    \item Let $\epsilon_0 \geq 0$. 
    \item Let $\wt{g}[j] \in \{0,1\}$ be the $i$-th element of array output by DPBloomfilter  
    
    
\end{itemize}

Then, we can show that, for all
$j \in [m]$, $\wt{g}[j]$ is $\epsilon_0$-DP. 
\end{lemma}

\begin{proof}

$\forall j \in [m]$, $g[j]$ is the ground truth value generated by dataset $A \subset [n]$. (An alternative view of $g$ is $g:[m] \rightarrow \{0,1\}$.) Suppose $g[j] = u$, $u \in \{0, 1\}$. For any neighboring dataset $A' \subset [n]$, we denote the ground truth value generated by it as $g'[j]$. Similarly, we can define the $\wt{g}'[j]$. 

We consider the following two cases to prove $\wt{g}[j]$ is $\epsilon_0$-DP, for all $j \in [m]$.

{\bf Case 1}. Suppose $g'[j] = u$. We know
\begin{align*}
    \Pr [ \wt{g}[j] = u ] = & ~ \frac{e^{\epsilon_0}}{ e^{\epsilon_0} + 1 }, \\
    \Pr[ \wt{g}'[j] = u ] = & ~ \frac{e^{\epsilon_0}}{ e^{\epsilon_0} + 1 }.
\end{align*}
Combining the above two equations, then we obtain
\begin{align*}
\frac{ \Pr [ \wt{g}[j] = u ] }{ \Pr[ \wt{g}'[j] = u ] } = 1.
\end{align*}

Similarly, we know 
\begin{align*}
    \Pr [ \wt{g}[j] = 1-u ] = & ~ \frac{ 1 }{ e^{\epsilon_0} + 1 }, \\
    \Pr[ \wt{g}'[j] = 1-u ] = & ~ \frac{ 1 }{ e^{\epsilon_0} + 1 }.
\end{align*}
Combining the above two equations, then we obtain
\begin{align*}
\frac{ \Pr [ \wt{g}[j] = 1- u ] }{ \Pr[ \wt{g}'[j] = 1-u ] } = 1.
\end{align*}
Thus, we know for all $v\in \{0,1\}$,
\begin{align*}
\frac{ \Pr [ \wt{g}[j] = v ] }{ \Pr[ \wt{g}'[j] = v ] } = 1.
\end{align*}

{\bf Case 2}. Suppose $g'[j] \neq u$.

We know
\begin{align*}
    \Pr [ \wt{g}[j] = u ] = & ~ \frac{e^{\epsilon_0}}{ e^{\epsilon_0} + 1 }, \\
    \Pr[ \wt{g}'[j] = u ] = & ~ \frac{ 1 }{ e^{\epsilon_0} + 1 }.
\end{align*}
Combining the above two equations, then we obtain
\begin{align*}
\frac{ \Pr [ \wt{g}[j] = u ] }{ \Pr[ \wt{g}'[j] = u ] } = e^{\epsilon_0}.
\end{align*}

Similarly, we know 
\begin{align*}
    \Pr [ \wt{g}[j] = 1-u ] = & ~ \frac{ 1 }{ e^{\epsilon_0} + 1 }, \\
    \Pr[ \wt{g}'[j] = 1-u ] = & ~ \frac{ e^{\epsilon_0} }{ e^{\epsilon_0} + 1 }.
\end{align*}
Combining the above two equations, then we obtain
\begin{align*}
\frac{ \Pr [ \wt{g}[j] = 1- u ] }{ \Pr[ \wt{g}'[j] = 1-u ] } = e^{-\epsilon_0}.
\end{align*}


For $v \in \{0, 1\}$, we have 
\begin{align*}
e^{- \epsilon_0} \leq \frac{ \Pr [ \wt{g}[j] = v ] }{ \Pr [ \wt{g}'[j] = v ] } \leq e^{\epsilon_0}.
\end{align*}



Therefore, $\forall j \in [m]$, $\wt{g}[j]$ is $\epsilon_0$-DP. 
\end{proof}

\subsection{Privacy guarantees for DPBloomfilter}\label{sec:query_privacy}
Then, we can prove that our entire algorithm is differentially private.
\begin{theorem}[Privacy for Query, formal version of Lemma~\ref{thm:query_privacy:informal}]\label{thm:query_privacy:formal}
If the following conditions hold
\begin{itemize}
    \item Let $N = F_W^{-1}(1 - \delta)$ denote the $1 - \delta$ quantile of the random variable $W$ (see Definition~\ref{def:W}).
    \item Let  $\epsilon_0 = \epsilon / N$.
\end{itemize}

Then, we can show,
the output of \textsc{Query} procedure of Algorithm~\ref{alg:init} achieves $(\epsilon, \delta)$-DP. 
\end{theorem}

\begin{proof}
Let $A$ and $A'$ are neighboring datasets. Let $g \in \{0, 1\}^m$ is the ground truth value generated by dataset $A$, and $g' \in \{0, 1\}^m$ is the ground truth value generated by dataset $A'$. 


We define
\begin{align*}
    S := \{j \in [m] ~:~ g[j] \neq g'[j]\}.
\end{align*}
We further define
\begin{align*}
    \ov{S} := [m] \backslash S.
\end{align*}

We consider two cases, {\bf Case 1} is $j \in \ov{S}$ and {\bf Case 2} is $j \in S$.

{\bf Case 1}. $j \in \ov{S}$. 

We can show that
\begin{align*}
\frac{ \Pr [ \wt{g}[j] = v ] }{\Pr[ \wt{g'}[j] = v ] } = 1.
\end{align*}
holds for $\forall v \in \{0, 1\}$.

{\bf Case 2.} $j \in S$.

We can show that
\begin{align}\label{eqn:query_privacy_single}
    e^{-\epsilon_0}\leq \frac{ \Pr[ \wt{g}[j] = v ] }{ \Pr[ \wt{g'}[j] = v] } \leq e^{\epsilon_0}.
\end{align}
holds for $\forall v \in \{0, 1\}$. 

Thus, for any $Z\in \{0,1\}^m$, the absolute privacy loss can be bounded by
\begin{align}\label{eqn:query_privacy_prod}
     |\ln \frac{ \Pr[ \wt{g} = Z ] }{ \Pr[ \wt{g'} = Z ] } | 
     = & ~  |\ln \prod_{j\in S} \frac{ \Pr[ \wt{g}[j] = v ] }{ \Pr[ \wt{g'}[j] = v ] }  | \notag \\
     \leq & ~ |S| \epsilon_0 \notag \\
     = & ~  |S|\frac{\epsilon}{N}.  
\end{align}
where the first step follows from each entry of $g$ is independent, the second step follows from Eq.~\eqref{eqn:query_privacy_single}, and the last step follows from choice of $\epsilon_0$.

By the definition of $N$, we know that with probability at least $1-\delta$, $|S|\leq F^{-1}(1-\delta)=N$. Hence, Eq.~\eqref{eqn:query_privacy_prod} is upper bounded by $\epsilon$ with probability $1-\delta$. 

This proves the $(\epsilon,\delta)$-DP.
\end{proof}