\def\isarxiv{1} %%% for icml submission version, we comment this line

\ifdefined\isarxiv
\documentclass[11pt]{article}

\usepackage[numbers]{natbib}

\else

\documentclass[conference,compsoc]{IEEEtran}

\fi

\usepackage{amsmath}
\usepackage{amsthm}
\usepackage{amssymb}
\usepackage{algorithm}
\usepackage{subfig}
\usepackage{algpseudocode}
\usepackage{graphicx}
\usepackage{grffile}
\usepackage{wrapfig,epsfig}
\usepackage{url}
\usepackage{xcolor}
\usepackage{epstopdf}
\usepackage{bbm}
\usepackage{dsfont}
\usepackage{gradient-text} % Zhizhou: I add this packge for the fancy color in the paper title.

 %%% print refs in table of contents
%\displaybreak
\allowdisplaybreaks

%\usepackage[lmargin=1in,rmargin=1in,tmargin=0.8in,bmargin=0.8in]{geometry}

\ifdefined\isarxiv

\let\C\relax
\usepackage{tikz}
\usepackage{hyperref}  %%% arxiv don't allow this.
\hypersetup{colorlinks=true,citecolor=blue,linkcolor=blue} %%% Zhao : maybe we should comment this in submission.
\usetikzlibrary{arrows}
\usepackage[margin=1in]{geometry}

\else
%\usepackage[pagebackref,breaklinks,colorlinks]{hyperref}


% Support for easy cross-referencing
%\usepackage[capitalize]{cleveref}
%\usepackage{microtype}
\usepackage{hyperref}
\definecolor{mydarkblue}{rgb}{0,0.08,0.45}
\hypersetup{colorlinks=true, citecolor=mydarkblue,linkcolor=mydarkblue}
%\usepackage[capitalize,noabbrev]{cleveref}
%\usepackage{colortbl}

\fi
%\linespread{1}
%\newcommand{\QED}{\hfill$\qed$}
\graphicspath{{./figs/}}

%\theoremstyle{plain} %%%Zhao: let me comment it out.
\newtheorem{theorem}{Theorem}[section]
\newtheorem{lemma}[theorem]{Lemma}
\newtheorem{definition}[theorem]{Definition}
\newtheorem{notation}[theorem]{Notation}
%\newtheorem{proof}[theorem]{Proof}
\newtheorem{proposition}[theorem]{Proposition}
\newtheorem{corollary}[theorem]{Corollary}
\newtheorem{conjecture}[theorem]{Conjecture}
\newtheorem{assumption}[theorem]{Assumption}
\newtheorem{observation}[theorem]{Observation}
\newtheorem{fact}[theorem]{Fact}
\newtheorem{remark}[theorem]{Remark}
\newtheorem{claim}[theorem]{Claim}
\newtheorem{example}[theorem]{Example}
\newtheorem{problem}[theorem]{Problem}
\newtheorem{open}[theorem]{Open Problem}
\newtheorem{property}[theorem]{Property}
\newtheorem{hypothesis}[theorem]{Hypothesis}
\newtheorem{condition}[theorem]{Condition}

\newcommand{\wh}{\widehat}
\newcommand{\wt}{\widetilde}
\newcommand{\ov}{\overline}
\newcommand{\N}{\mathcal{N}}
\newcommand{\R}{\mathbb{R}}
\newcommand{\RHS}{\mathrm{RHS}}
\newcommand{\LHS}{\mathrm{LHS}}
\renewcommand{\d}{\mathrm{d}}
\renewcommand{\i}{\mathbf{i}}
\renewcommand{\tilde}{\wt}
\renewcommand{\hat}{\wh}
\newcommand{\Tmat}{{\cal T}_{\mathrm{mat}}}

\newcommand{\ac}{ac} %%% Zhao: use this for $\delta_{\ac}$ , which is accuracy
\newcommand{\pa}{pa} %%% Zhao: use this for $\delta_{\pa}$ , which is privacy

\DeclareMathOperator*{\E}{{\mathbb{E}}}
\DeclareMathOperator*{\var}{\mathrm{Var}}
\DeclareMathOperator*{\Z}{\mathbb{Z}}
\DeclareMathOperator*{\C}{\mathbb{C}}
\DeclareMathOperator*{\D}{\mathcal{D}}
\DeclareMathOperator*{\median}{median}
\DeclareMathOperator*{\mean}{mean}
\DeclareMathOperator{\OPT}{OPT}
\DeclareMathOperator{\supp}{supp}
\DeclareMathOperator{\poly}{poly}

\DeclareMathOperator{\nnz}{nnz}
\DeclareMathOperator{\sparsity}{sparsity}
\DeclareMathOperator{\rank}{rank}
\DeclareMathOperator{\diag}{diag}
\DeclareMathOperator{\dist}{dist}
\DeclareMathOperator{\cost}{cost}
\DeclareMathOperator{\vect}{vec}
\DeclareMathOperator{\tr}{tr}
\DeclareMathOperator{\dis}{dis}
\DeclareMathOperator{\cts}{cts}



\makeatletter
\newcommand*{\RN}[1]{\expandafter\@slowromancap\romannumeral #1@}
\makeatother
\newcommand{\Zhao}[1]{{\color{red}[Zhao: #1]}}
\newcommand{\Zhenmei}[1]{{\color{orange}[Zhenmei: #1]}}
\newcommand{\Zhizhou}[1]{{\color{blue}[Zhizhou: #1]}} 
\newcommand{\Yekun}[1]{{\color{purple}[Yekun: #1]}} 
% \newcommand{\InernName}[1]{{\color{blue}[InternName: #1]}} %%%Change to intern name


\usepackage{lineno}
\def\linenumberfont{\normalfont\small}

\ifdefined\isarxiv
\else
% Zhizhou: This line is added for S&P submission

% correct bad hyphenation here
\hyphenation{op-tical net-works semi-conduc-tor}
\fi

\begin{document}

\ifdefined\isarxiv

\date{}


% \title{DPBloomfilter: Securing Bloom Filters with Differential Privacy}
\title{\includegraphics[scale=0.045]{filter.png} \textbf{\textsc{\gradientRGB{DPBloomfilter}{11,47,159}{152,222,217}}}: Securing Bloom Filters with Differential Privacy}

\author{
Yekun Ke\thanks{\texttt{
keyekun0628@gmail.com}. Independent Researcher.}
\and
Yingyu Liang\thanks{\texttt{
yingyul@hku.hk}. The University of Hong Kong. \texttt{
yliang@cs.wisc.edu}. University of Wisconsin-Madison.} 
\and
Zhizhou Sha\thanks{\texttt{ shazz20@mails.tsinghua.edu.cn}. Tsinghua University.}
\and
Zhenmei Shi\thanks{\texttt{
zhmeishi@cs.wisc.edu}. University of Wisconsin-Madison.}
\and 
Zhao Song\thanks{\texttt{ magic.linuxkde@gmail.com}. The Simons Institute for the Theory of Computing at UC Berkeley.}
}
\else

% \title{DPBloomfilter: Securing Bloom Filters with Differential Privacy}
\title{\includegraphics[scale=0.035]{filter.png} \textbf{\textsc{\gradientRGB{DPBloomfilter}{11,47,159}{152,222,217}}}: Securing Bloom Filters with Differential Privacy}

% % author names and affiliations
% % use a multiple column layout for up to three different
% % affiliations
% \author{\IEEEauthorblockN{Michael Shell}
% \IEEEauthorblockA{School of Electrical and\\Computer Engineering\\
% Georgia Institute of Technology\\
% Atlanta, Georgia 30332--0250\\
% Email: http://www.michaelshell.org/contact.html}
% \and
% \IEEEauthorblockN{Homer Simpson}
% \IEEEauthorblockA{Twentieth Century Fox\\
% Springfield, USA\\
% Email: homer@thesimpsons.com}
% \and
% \IEEEauthorblockN{James Kirk\\ and Montgomery Scott}
% \IEEEauthorblockA{Starfleet Academy\\
% San Francisco, California 96678-2391\\
% Telephone: (800) 555--1212\\
% Fax: (888) 555--1212}}


% Zhizhou: I am not sure how to anonymize the author in IEEEtran.cls So I adopt the answer provided in this link https://tex.stackexchange.com/questions/676097/remove-author-name-in-latex-with-ieeetran-template
\author{\IEEEauthorblockN{Anonymous Authors}}

% \author{
% Yekun Ke\thanks{\texttt{
% keyekun0628@gmail.com}.}
% \and
% Yingyu Liang\thanks{\texttt{
% yingyul@hku.hk}. The University of Hong Kong. \texttt{
% yliang@cs.wisc.edu}. University of Wisconsin-Madison.} 
% \and
% Zhizhou Sha\thanks{\texttt{ shazz20@mails.tsinghua.edu.cn}. Tsinghua University.}
% \and
% Zhenmei Shi\thanks{\texttt{
% zhmeishi@cs.wisc.edu}. University of Wisconsin-Madison.}
% \and 
% Zhao Song\thanks{\texttt{ zsong@adobe.com}. Adobe Research.}
% }


\fi


\ifdefined\isarxiv
\begin{titlepage}
  \maketitle
  \begin{abstract}
  \begin{abstract}

% Recent works to jointly reconstruct 3D human and object from a single RGB image, are mostly model-based, that fail to capture the fine details of the clothed human body and object surface. In this paper, we introduce ReCHOR, a novel, model-free, first-method to produce realistic clothed human-object reconstructions from a monocular view. This is extremely challenging due to human-object occlusions, diverse interactions and depth ambiguity, as it needs to infer both 3D spatial awareness and high resolution details. Our core idea is based on estimating neural implicit representations for human and object respectively by an attention-based neural implicit model that attends to pixel-aligned features from both the global human-object image for spatial awareness and  the local separate view of human and object images for high quality details. Additionally, the network is conditioned on semantic features from an initial estimated human-object pose prior and a generative diffusion model that inpaints occluded regions, thus enabling the retrieval of details from them.
% We also propose a synthetic dataset with rendered scenes of diverse, inter-occluded 3D human and object scans, to train our network. We evaluate our method on the synthetic and real world BEHAVE dataset. Our experiments show that our method outperforms the SOTA in achieving realistic clothed human-object reconstructions.
Recent approaches to jointly reconstruct 3D humans and objects from a single RGB image represent 3D shapes with template-based or coarse models, which fail to capture details of loose clothing on human bodies. In this paper, we introduce a novel implicit approach for jointly reconstructing realistic 3D clothed humans and objects from a monocular view. For the first time, we model both the human and the object with an implicit representation, allowing to capture more realistic details such as clothing. This task is extremely challenging due to human-object occlusions and the lack of 3D information in 2D images, often leading to poor detail reconstruction and depth ambiguity. To address these problems, we propose a novel attention-based neural implicit model that leverages image pixel alignment from both the input human-object image for a global understanding of the human-object scene and from local separate views of the human and object images to improve realism with, for example, clothing details. Additionally, the network is conditioned on semantic features derived from an estimated human-object pose prior, which provides 3D spatial information about the shared space of humans and objects. To handle human occlusion caused by objects, we use a generative diffusion model that inpaints the occluded regions, recovering otherwise lost details. For training and evaluation, we introduce a synthetic dataset featuring rendered scenes of inter-occluded 3D human scans and diverse objects. Extensive evaluation on both synthetic and real-world datasets demonstrates the superior quality of the proposed human-object reconstructions over competitive methods.
\end{abstract}
  \end{abstract}
  \thispagestyle{empty}
\end{titlepage}

{\hypersetup{linkcolor=black}
\tableofcontents
}
\newpage

\else

\maketitle 

\begin{abstract}
\begin{abstract}

% Recent works to jointly reconstruct 3D human and object from a single RGB image, are mostly model-based, that fail to capture the fine details of the clothed human body and object surface. In this paper, we introduce ReCHOR, a novel, model-free, first-method to produce realistic clothed human-object reconstructions from a monocular view. This is extremely challenging due to human-object occlusions, diverse interactions and depth ambiguity, as it needs to infer both 3D spatial awareness and high resolution details. Our core idea is based on estimating neural implicit representations for human and object respectively by an attention-based neural implicit model that attends to pixel-aligned features from both the global human-object image for spatial awareness and  the local separate view of human and object images for high quality details. Additionally, the network is conditioned on semantic features from an initial estimated human-object pose prior and a generative diffusion model that inpaints occluded regions, thus enabling the retrieval of details from them.
% We also propose a synthetic dataset with rendered scenes of diverse, inter-occluded 3D human and object scans, to train our network. We evaluate our method on the synthetic and real world BEHAVE dataset. Our experiments show that our method outperforms the SOTA in achieving realistic clothed human-object reconstructions.
Recent approaches to jointly reconstruct 3D humans and objects from a single RGB image represent 3D shapes with template-based or coarse models, which fail to capture details of loose clothing on human bodies. In this paper, we introduce a novel implicit approach for jointly reconstructing realistic 3D clothed humans and objects from a monocular view. For the first time, we model both the human and the object with an implicit representation, allowing to capture more realistic details such as clothing. This task is extremely challenging due to human-object occlusions and the lack of 3D information in 2D images, often leading to poor detail reconstruction and depth ambiguity. To address these problems, we propose a novel attention-based neural implicit model that leverages image pixel alignment from both the input human-object image for a global understanding of the human-object scene and from local separate views of the human and object images to improve realism with, for example, clothing details. Additionally, the network is conditioned on semantic features derived from an estimated human-object pose prior, which provides 3D spatial information about the shared space of humans and objects. To handle human occlusion caused by objects, we use a generative diffusion model that inpaints the occluded regions, recovering otherwise lost details. For training and evaluation, we introduce a synthetic dataset featuring rendered scenes of inter-occluded 3D human scans and diverse objects. Extensive evaluation on both synthetic and real-world datasets demonstrates the superior quality of the proposed human-object reconstructions over competitive methods.
\end{abstract}
\end{abstract}

% For peerreview papers, this IEEEtran command inserts a page break and
% creates the second title. It will be ignored for other modes.
\IEEEpeerreviewmaketitle



\fi

\section{Introduction}
\label{sec:intro}
% Image editing methods in diffusion models depend on user-defined control directions - users can unlock their creativity using these methods by specifying the desired manipulation through prompts~\cite{gandikota2023concept}, reference images~\cite{ruiz2022dreambooth, kumari2022customdiffusion, gal2022image, chen2024trainingfreeregionalpromptingdiffusion}, or attribute vectors~\cite{parmar2023zero,hertz2022prompt}. In this work, we ask a fundamentally different question: \emph{Can we automatically discover the underlying visual structure of a concept within diffusion model's knowledge?} %Rather than requiring user-specified controls, we aim to decompose the model's internal knowledge into meaningful directions.

% This question touches on a fundamental limitation in how we interact with diffusion models. Current control methods ~\cite{zhang2023addingconditionalcontroltexttoimage, gandikota2023concept, ye2023ipadaptertextcompatibleimage,ye2023ipadaptertextcompatibleimage, hertz2024stylealignedimagegeneration, li2023photomaker, shi2024instantbooth, chen2024trainingfreeregionalpromptingdiffusion} require users to specify their desired manipulations in advance, limiting interactive creativity. This contrasts with natural human artistic workflows, where creators dynamically explore creative ideas while jointly refining them toward meaningful artistic outcomes~\cite{hoffmann2016modeling}. This synergy between specification and exploration is not new to generative models. Early GAN architectures naturally developed disentangled latent spaces that enabled continuous\cite{harkonen2020ganspace,radford2015unsupervised, wu2021stylespace, shen2020interfacegan}, compositional control over generated images. Users could explore these spaces to discover interesting variations that would be difficult to describe in words~\cite{wu2021stylespace}, then combine them to achieve their creative goals~\cite{grabe2022towards}. 


% While diffusion models have largely superseded GANs in conditional image synthesis~\cite{dhariwal2021diffusion},  their underlying structure remains less understood. Diffusion models achieve remarkable diversity through high-dimensional latents, unlike GANs' compact latent spaces.  With a single prompt, diffusion models can generate radically different variations through different random initializations of input noise. We ask - Is it possible to discover interpretable structure within this vast space of variations?

Text-to-image diffusion models are capable of generating remarkable visual variations from a single prompt through different random initializations. However, this vast creative potential remains largely opaque to users---while we can generate diverse images, we lack understanding of the underlying structure of these variations. This presents a fundamental challenge: how can we discover and expose the latent visual capabilities encoded within these models?

\let\thefootnote\relax \footnote{$^{*}$Correspondence to \texttt{gandikota.ro@northeastern.edu}}

The challenge touches on a key limitation in how we interact with diffusion models today. Current control methods require users to explicitly specify their desired edits in advance through prompts~\cite{gandikota2023concept}, reference images~\cite{zhang2023addingconditionalcontroltexttoimage, chen2024trainingfreeregionalpromptingdiffusion, ruiz2022dreambooth,kumari2022customdiffusion, Ryu_lora, hu2021lora}, or attribute vectors~\cite{ye2023ipadaptertextcompatibleimage, hertz2024stylealignedimagegeneration, li2023photomaker, shi2024instantbooth,parmar2023zero,hertz2022prompt}. That contrasts sharply with natural human creative workflows, where artists dynamically explore creative ideas and jointly refine them toward meaningful artistic outcomes~\cite{hoffmann2016modeling}. The need for pre-specified controls creates a barrier between users and the full creative potential of these models.

Interestingly, earlier generative models like GANs~\cite{gans,karras2019style,brock2018large} naturally developed more interpretable internal structures. Their compact latent spaces often exhibited emergent disentanglement~\cite{harkonen2020ganspace,radford2015unsupervised, wu2021stylespace, shen2020interfacegan}, enabling continuous and compositional control over generated images. Users could explore these spaces to discover interesting variations that would be difficult to describe in words~\cite{wu2021stylespace}, then combine them to achieve their creative goals~\cite{grabe2022towards}.

Diffusion models have largely superseded GANs in conditional image synthesis~\cite{dhariwal2021diffusion}, achieving greater diversity through much higher-dimensional latents. And yet an understanding of the underlying structure of these larger latent spaces has remained elusive. In this work, we ask a fundamental question: \emph{Can we automatically discover the visual structure within a diffusion model's knowledge of a concept?} Rather than requiring user-specified controls, we aim to decompose the model's internal representations into expressive directions that users can explore and combine.

To address these needs, we present \textbf{SliderSpace}, a framework that brings systematic explorability to diffusion models. Given just a text prompt, SliderSpace discovers a canonical set of meaningful, diverse, and controllable directions within the model's knowledge of that concept. Each direction is implemented as a low-rank adapter~\cite{hu2021lora} that can be scaled and composed with others, allowing users to explore and smoothly combine different aspects of variation, as shown in Figure~\ref{fig:intro}.

We ground SliderSpace discovery in three key requirements for meaningful decomposition of a diffusion model's visual manifold: 
\begin{enumerate}
    \item \textbf{Unsupervised Discovery:} The decomposition process should emerge from the intrinsic structure of the model's learned representation, rather than being guided by predefined attributes. This ensures we capture the true topology of the model's knowledge space rather than projecting our assumptions onto it.
    
    \item \textbf{Semantic Orthogonality:} Each discovered control must represent a distinct semantic direction. This is enforced in a semantic feature space, like CLIP, where every slider has an orthogonal effect in embeddings. This prevents discovering multiple controls that create similar semantic effects, making the system more efficient and easier.
    
    \item \textbf{Distribution Consistency:} Directions must induce consistent transformations across both random seeds and prompt variations. 
\end{enumerate}

These requirements naturally lead to our proposed framework, which we formalize in Section~\ref{sec:method}. As we show in our experiments, SliderSpace is architecture-agnostic, working with both conventional U-Net based models like Stable Diffusion~\cite{rombach2022high, rombach2022sd20, podell2023sdxl, turbo, dmd} and recent transformer-based architectures like Flux~\cite{flux}.

We demonstrate the expressiveness of SliderSpace through three applications: First, we show how SliderSpace can decompose high-level concepts into diverse and expressive components, revealing the natural axes of variation in the model's understanding. Second, we explore artistic style variation, where SliderSpace discovers directions that match or exceed the diversity of manually curated artist lists while being judged more useful by human evaluators. Finally, we show how SliderSpace can help reverse the mode collapse commonly observed in distilled diffusion models, restoring diversity while maintaining generation speed.

Beyond providing practical creative control, SliderSpace opens new avenues for understanding and utilizing the latent capabilities of diffusion models. By mapping these models' visual potential into intuitive, composable directions, we take a step toward making their creative possibilities more accessible and interpretable to users.

% Image editing methods in diffusion models unlock the creativity of users. In this work we ask an alternate question: \emph{Can we organize and expose what of the diffusion model is already capable of?}.
% Existing methods for controlling image generation typically require users to manually specify edit directions for desired changes. This process is time-consuming, requires technical expertise, and limits the spontaneity of the creative process. For instance, if a user wants to adjust the smile of a generated person, they must explicitly request this edit, often through imprecise prompt engineering or model fine-tuning. This approach of predefined controls or manual specifications restricts users from fully exploring the latent capabilities of the model. There may be interesting stylistic variations or attributes that the model can generate, but users have no easy way to discover or utilize these.

% Natural visual disentanglement was an emergent property in the latent space of Generative Adversarial Models (GANs) \cite{harkonen2020ganspace,radford2015unsupervised, wu2021stylespace, shen2020interfacegan}. In particular, it has been observed that StyleGAN~\cite{karras2019style} stylespace neurons offer detailed control over many meaningful aspects of images that would be difficult to describe in words~\cite{wu2021stylespace}. However, diffusion models do not share such a compact latent space~\cite{park2023unsupervised}; and efforts to uncover such a space in the semantic embeddings of the text conditioning have met with limited success \nik{Nick - is there a specific citation you were thinking about?}.

% In this work we introduce \textbf{SliderSpace}, which takes a step towards uncovering an analogous low dimensional representation of diffusion models' visual breadth; in essence treating the diffusion model as many generators sharing parameters, where a particular generator is defined by a specific prompt. For a given prompt we sample many random seeds (and optionally prompt expansions using an LLM), generate the corresponding images, and apply an off the shelf feature extractor (in this work CLIP, but our method can be applied to any differentiable feature extractor). We use PCA to analyze these features, and for each of the leading $k$ principal components we train a LoRA \cite{} which causes the diffusion model to produces images which increase the feature magnitude along that component when passed back through the same feature extractor. This leads to a 'Slider' for each principal component, because each LoRA can be scaled and applied to the original diffusion model, continuously varying those visual features in the generated results (as measured, in our case, by CLIP).

% There are many other works that enhance the controllability of diffusion models. One common approach is enabling users to add spatial constraints to a generation either manually, or via a reference image \cite{zhang2023addingconditionalcontroltexttoimage, chen2024trainingfreeregionalpromptingdiffusion}, a second is leveraging more abstract embeddings (e.g. identity, style) extracted from a reference image \cite{ye2023ipadaptertextcompatibleimage, hertz2024stylealignedimagegeneration, li2023photomaker, shi2024instantbooth}, a third is finetuning a foundation model to better generate a concept important to the user \cite{ruiz2022dreambooth, kumari2022customdiffusion, Ryu_lora, hu2021lora}, and a fourth (most relevant to this work) is finding low-rank adaptors of the model based on a prompt or small training set which can be scaled to provide continous control over one aspect of generated image (e.g. night vs day, basic vs luxury, etc.) \cite{gandikota2023concept}. SliderSpace is complementary to all of these methods and offers something distinct. All of the other methods we are aware require the user (and / or model designer) to know in advance what type of control they want. In contrast SliderSpace assists users in discovering and controlling hidden capabilities present in the diffusion model's distribution of possible generations.

%We propose that truly intuitive creative control in a text-to-image model should meet three key criteria: \emph{discoverability}, \emph{intuitiveness}, and \emph{specificity}. The model should reveal controllable attributes that may not be immediately obvious, offer controls that are easy to understand and manipulate, and ensure each control affects a distinct attribute of the generated image.

% We demonstrate the utility and power of SliderSpace using three applications built on top of SDXL-DMD \cite{dmd}, because its fast generation speed lends itself well to the continuous control offered by SliderSpace.

% First, we study concept decomposition (Section \ref{sec:concept_exp}), where we learn sliders for a specific concept (e.g. 'monster', 'waterfall', 'car'). Through quantitative metrics of diversity and text alignment we demonstrate that the learned sliders dramatically boost the diversity of generations when randomly applied without harming text alignment; we also ask humans to qualitatively judge these results in a user study where they find the SliderSpace results to be more 'Diverse', 'Useful', and 'Creative' than our baselines.

% Second, we attempt to compare the automatic discoveries of SliderSpace to a large scale manual study of artistic styles (Section \ref{sec:art_exp}), open-sourced by ParrotZone \cite{parrotzone}. In this study SDXL was prompted with over 4300 artist names,  and based on visual inspection the cases of successful stylistic mimicry recorded. Quantitatively SliderSpace more closely matches the distribution of artistic variation discovered by ParrotZone than other baselines, and in our user studies was judged to be significantly more 'Diverse' and 'Useful' than the baselines. To our surprise humans even judged SliderSpace results to be slightly more 'Diverse' than the results generated by the manually discovered artist names of \cite{parrotzone}.

% Third, we attempt to use SliderSpace to reverse the mode collapse commonly observed in distilled few-step diffusion models relative to the original teacher model (Section \ref{sec:diverse_exp}). We quantitatively demonstrate that applying SliderSpace to SDXL-DMD leads to more closely matching the distribution of images by the original teacher, SDXL.

%Through extensive experiments on various state-of-the-art text-to-image models, we demonstrate that SliderSpace significantly enhances user control and creative expression in AI-assisted image generation tasks. Our method enables a range of applications, including concept decomposition and control, diversity improvement in generated images, customization dissection and edits, and the exploration of artistic styles inherent in the model.

% SliderSpace goes beyond providing a practical tool for enhanced creative control. By mapping the visual potential of diffusion models it can open new avenues for generative creativity and deepens our understanding of each model's hidden potential.
\section{Related Work}
\label{sec:related_work}

The original investigation \cite{gibson1979ecological} on the relationship between visual perception and human action defines \emph{affordance} as the opportunities for interaction with the surrounding environment. Behavioral studies on regular and cognitively impaired persons have shown evidence that perception results in both visual and motor signals in the human brain. An extended study \cite{anderson2002attentional} shows that visual attention to the spatial characteristics of the perceived objects initiates automatic motor signals for different actions. In computer vision, human affordance learning involves novel pose prediction such that the estimated pose represents a valid human action within the scene context. The task is fundamental to many problems requiring robust semantic reasoning about the environment, such as human motion synthesis \cite{wang2021scene} and scene-aware human pose generation \cite{wang2017binge, roy2016multi, zhang2022inpaint, yao2023scene}.

Earlier methods of affordance learning have explored knowledge mining \cite{zhu2014reasoning} and multimodal feature cues \cite{roy2016multi} to address the problem. In \cite{zhu2014reasoning}, the authors use a Markov Logic Network for constructing a knowledge base by extracting several object attributes from different image and metadata sources, which can perform various downstream visual inference tasks without any additional classifier, including zero-shot affordance prediction. In \cite{roy2016multi}, the authors use depth map, surface normals, and segmentation map as multimodal cues to train a multi-scale convolutional neural network (CNN) for scene-level semantic label assignment associated with specific human actions. In \cite{do2018affordancenet}, the authors design a multi-branch end-to-end CNN with two separate pathways for object detection and affordance label assignment to achieve high real-time inference throughput. Researchers \cite{chuang2018learning} have also explored socially imposed constraints for affordance learning. In \cite{chuang2018learning}, the authors propose a graph neural network (GNN) to propagate contextual scene information from egocentric views for action-object affordance reasoning.

Probabilistic modeling of scene-aware human motion generation also involves semantic reasoning of human interaction with the environment. Initial works on human motion synthesis have taken different architectural approaches, such as sequence-to-sequence models \cite{barsoum2018hp}, generative adversarial networks (GAN) \cite{barsoum2018hp, cai2018deep, yang2018pose}, graph convolutional networks (GCN) \cite{yan2019convolutional}, and variational autoencoders (VAE) \cite{guo2020action2motion}. However, these methods have mostly ignored the role of environmental semantics. Due to potential uncertainty in human motion, in a recent approach \cite{wang2021scene}, the authors address such motion synthesis with a GAN conditioned on scene attributes and motion trajectory to predict probable body pose dynamics.

One key challenge of human affordance generation in 2D scenes is the lack of large-scale datasets with rich pose annotations. In \cite{wang2017binge}, the authors compile the only public dataset of annotated human body poses in complex 2D indoor scenes by extracting frames from sitcom videos. Aiming to generate a contextually valid human affordance at a user-defined location, the authors propose sampling the scale and deformation parameters for an existing human pose template using a VAE conditioned on the localized image patches as scene context. In \cite{zhang2022inpaint}, the authors introduce a two-stage GAN architecture for achieving a similar goal by estimating the affine bounding box parameters to localize a probable human in the scene and then generating a potential body pose at that location. The method uses the input scene, corresponding depth, and segmentation maps as semantic guidance. In \cite{yao2023scene}, the authors propose a transformer-based approach with knowledge distillation for generating human affordances in 2D indoor scenes.


\section{Preliminaries}
\noindent \textbf{Autoregressive Language Modeling}.
Language provides a versatile way to represent tasks, data, inputs, and outputs, all as a sequence of tokens. Autoregressive language modeling is the basis for LLMs like GPT~\citep{gpt2,gpt3}. This approach predicts the probability of a sequence of words or tokens, with each prediction conditioned on the previous elements in the sequence. 

Formally, given a language token sequence $\vx = (x_1, x_2, \cdots, x_n)$, autoregressive language modeling decomposes the joint distribution of the sequence as the product of a series of conditional probabilities: $p(\vx) = \prod_{i=1}^n p(x_i|x_1, ..., x_{i-1})$,
where $p(x_1|x_0) = p(x_1)$ is the marginal probability. With the factorized distribution and a parameterized model (e.g., Transformers~\cite{vaswani2017attention}), the parameterized distribution $p_{\theta}(x)$ can be optimized via minimizing the negative log-likelihood loss:
\begin{equation}\label{eqn:loss-ar}
    \gL(\theta) = -\log p_{\theta}(\vx) = - \sum\limits_{i=1}^n \log p(x_i|x_{1},\cdots, x_{i-1}).
\end{equation}

\vspace{3pt}
\noindent \textbf{Query-based Data Analytic Tasks}. This paper focuses on query-based data analytic tasks, represented as $\{task, data, query, answer\}$. Given a task description, the data to be analyzed, and a natural language query, the goal is to predict the answer, i,e, $p(answer|task,$ $data, query)$. For example, we can format the input text for a table selection task as: "$\texttt{find tables},$ $\texttt{table schema},~$ $\texttt{who was the}$ $\texttt{only athlete...}$". Then, the model is expected to return the table name(s) that can answer the question, e.g., "$\texttt{<tables>Final\_1},$ $\texttt{Athletics\_1</tables>}$".

\vspace{3pt}

\noindent \textbf{Supervised Instruction Tuning} is a critical fine-tuning process employed to enhance the performance of LLMs on specific tasks by leveraging labeled datasets. Supervised instruction tuning focuses on adapting the model to follow explicit instructions and produce task-specific outputs. This process involves training the model on input-output pairs, where the inputs are typically natural language instructions or prompts, and the outputs are the desired responses or completions. The loss function of supervised instruction tuning is computed only on the "output" tokens to optimize the ability to execute specific tasks and understand instructions.

\section{Theoretical Analysis} \label{sec:main_result}


In this section, we will introduce our main result, the approximation error of the second order flow matching. The theory for higher order flow matching is deferred to Section~\ref{sec:app:higher_order_flow_matching}.

\begin{algorithm}[!ht]\caption{HOMO Training}
\begin{algorithmic}[1]
\Procedure{HOMOTraining}{$\theta, D, p, k$}
\State \Comment{Parameter $\theta$ for HOMO model $u_1$ and $u_2$.}
\State \Comment{Training dataset $D$}
\State \Comment{Stepsize and time index distribution $p$}
\State \Comment{Batch size $k$}
\While{not converged}
\State $x_0 \sim \N (0, I), x_1 \sim D, (d, t) \sim p$
\State $\beta_t \gets \sqrt{1-\alpha_t^2}$
\State $x_t \gets \alpha_t \cdot x_0 + \beta_t \cdot x_1$ \Comment{Noise data point}
\For{first $k$ batch elements}
\State $\dot s_t^{\True} \gets \dot{\alpha_t} x_0 + \dot{\beta_t} x_1$ \Comment{First-order target}
\State $\ddot s_t^{\True} \gets \ddot{\alpha_t} x_0 + \ddot{\beta_t} x_1$ \Comment{Second-order target}
\State $d \gets 0$
\EndFor
\For{other batch elements}
\State $s_t \gets u_1 ( x_t, t, d)$ \Comment{First small step of first order}
\State $\dot s_t \gets u_2 (u_1 ( x_t, t, d), x_t, t, d)$ \Comment{First small step of second order}
\State $x_{t + d} \gets x_t + d \cdot s_t + \frac{d^2}{2} \dot s_t $ \Comment{Follow ODE}
\State $s_{t + d} \gets u_1 ( x_{t + d}, t + d, d )$ \Comment{Second small step of first order}
\State $\dot s_t^{\mathrm{target}} \gets$ stopgrad $(s_t + s_{t+d}) / 2$ \Comment{Self-consistency target of first order }
\EndFor
\State $\theta \gets \nabla_\theta ( \| u_1 ( x_t, t, 2d ) - \dot s_t^{\True} \|^2$
\Statex \hspace{4.2em} $ + \| u_2 (u_1 (x_t, t, 2d), x_t, t, 2d) - \ddot s_t^{\True} \|^2$
\Statex \hspace{4.2em} $ + \| u_{1}(x_t, t, 2 d) - \dot{s}_t^{\mathrm{target}}\|^2)$
\EndWhile
\State \Return{$\theta$}
\EndProcedure
\end{algorithmic}
\end{algorithm}

\begin{algorithm}
[!ht]
\caption{HOMO Sampling}
\begin{algorithmic}[1]
\Procedure{HOMOSampling}{$\theta, M$}
\State \Comment{Parameter $\theta$ for the HOMO model $u_1$ and $u_2$}
\State \Comment{The number of sampling steps $M$}
\State $x \sim \N (0, I)$
\State $d \gets 1 / M$
\State $t \gets 0$
\For{$n \in [0, \dots, M - 1]$}
\State $x \gets x + d \cdot u_1 (x, t, d) + \frac{d^2}{2} \cdot u_2 (u_1 (x, t, d), x, t, d)$
\State $t \gets t + d$
\EndFor
\State \textbf{return} $x$
    \EndProcedure
\end{algorithmic}
\end{algorithm}



We first present the approximation error result for the early stage of the diffusion process. This result establishes theoretical guarantees on how well a neural network can approximate the first and second order flows during the initial phases of the trajectory evolution.

\begin{theorem}[Approximation error of second order flow matching for small $t$, informal version of Theorem~\ref{thm:secon_order_small_t:formal}]\label{thm:secon_order_small_t:informal}
    Let $N$ be a value associated with sample size $n$. Let $T_0 := N^{-R_0}$ and $T_* := N - \frac{\kappa^{-1} - \delta}{d}$ where $R_0, \kappa, \delta$ are some parameters.  Let $s$ be the order of smoothness of the Besov space that the target distribution belongs to.
    Under some mild assumptions, there exist neural networks $\phi_{1},\phi_2$ from a class of neural networks such that, for sufficiently large $N$, we have
\begin{align*}
    &~ \int (\|\phi_1(x, t) - \dot{x}_t^\mathrm{true}\|_2^2 + \|\phi_2(x, t) - \ddot{x}_t^\mathrm{true}\|_2^2) p_t(x) \d x \\ 
    \lesssim &~ (\dot{\alpha}_t^2 \log N + \dot{\beta}_t^2 ) N^{- \frac{2s}{d}} +
    \E_{x \sim P_t}[\|\dot{x}^\mathrm{true}_t - \ddot{x}^\mathrm{true}_t\|_2^2]
\end{align*}
    holds for any $t \in [T_{0}, 3T_{*}]$. In addition, $\phi_1, \phi_2$ can be taken so we have
    \begin{align*}
         \|\phi_1(\cdot,t) \|_\infty = &~ O(  |\dot{\alpha}_t | \sqrt{\log n} +  |\dot{\beta}_t |), \\ \|\phi_2(\cdot,t) \|_\infty = &~ O(  |\dot{\alpha}_t | \sqrt{\log n} +  |\dot{\beta}_t |).
    \end{align*}
\end{theorem}

Next, we present the approximation error result for the later stages, confirming that the second-order flow matching remains effective throughout the generative process.

\begin{theorem}[Approximation error of second order flow matching for large $t$, informal version of Theorem~\ref{thm:secon_order_large_t:formal}]\label{thm:secon_order_large_t:informal}
    Let $N$ be a value associated with sample size $n$. Let $T_0 := N^{-R_0}$ and $T_* := N - \frac{\kappa^{-1} - \delta}{d}$ where $R_0, \kappa, \delta$ are some parameters.  Let $s$ be the order of smoothness of the Besov space that the target distribution belongs to.
    Fix $t_{*} \in [T_{*},1]$ and let $\eta>0$ be arbitrary. Under some mild assumptions, there exists neural networks $\phi_{1},\phi_2$ from a class of neural networks such that
\begin{align*}
    &~ \int (\|\phi_1(x, t) - \dot{x}_t^\mathrm{true}\|_2^2 + \|\phi_2(x, t) - \ddot{x}_t^\mathrm{true}\|_2^2) p_t(x) \d x \\ \lesssim &~ (\dot{\alpha}_t^{2} \log N  +   \dot{\beta}_t^{2} ) N^{-\eta} +
    \E_{x \sim P_t}[\|\dot{x}^\mathrm{true}_t - \ddot{x}^\mathrm{true}_t\|_2^2]
\end{align*}
    holds for any $t \in [2t_*, 1]$. In addition, $\phi_1, \phi_2$ can be taken so we have
    \begin{align*}
         \|\phi_1(\cdot,t) \|_\infty = &~ O(  |\dot{\alpha}_t | \log N +  |\dot{\beta}_t |), \\ \|\phi_2(\cdot,t) \|_\infty = &~ O(  |\dot{\alpha}_t | \log N +  |\dot{\beta}_t |).
    \end{align*}
\end{theorem}

Overall, these two results demonstrate the effectiveness across different phases of the generative process.


\section{Proof for 
\texorpdfstring{$1 - \delta$}{} Quantile}\label{sec:appendix_quantile_proof}
In this section, we provide the calculation of the probability distribution of random variable $W := \sum_{j=1}^{m} \mathbbm 1\{g[j] \neq g'[j]\}$, which plays an important part in the proof of the privacy guarantee for our algorithm (see Section~\ref{sec:appendix_privacy_guarantees}).
In Section~\ref{sec:definition_quantile}, we present the definition of random variables $W, Y, Z$ used in this section.
In Section~\ref{sec:distribution_Y}, we calculate the probability distribution of $Y$.
In Section~\ref{sec:distribution_Z}, we calculate the probability distribution of $Z$ conditioned on $Y$.
In Section~\ref{sec:distribution_W}, we calculate the probability distribution of $W$.

\subsection{Definition} \label{sec:definition_quantile}
In this section, we present the definitions of random variables which will be used in the section.
\begin{definition}[Definition of $W$]\label{def:W}
Let $W := \sum_{j=1}^{m} \mathbbm 1\{g[j] \neq g'[j]\}$, where $g \in \{0, 1\}^m$ denotes the ground truth values generated by dataset $A$, and $g' \in \{0, 1\}^m$ denotes the ground truth values generated by neighboring dataset $A'$. 
\end{definition}

\begin{definition}[Definition of $Y$]\label{def:Y}
Consider a $x \in [n]$. 

Let $y_1, y_2, \cdots , y_k$ denotes the $k$ hash values generated by the standard Bloom filter (Definition~\ref{def:bloom_filter}). 

We define $Y$ as the set of distinct values among $y_1, y_2, \cdots, y_k$, where $|Y| \in { 1, 2, \cdots, k }$.

\end{definition}

\begin{definition}[Definition of $Z$]\label{def:Z}
Consider two data $x, x' \in [n]$. 

Let $y_1, y_2, \cdots , y_k$ denotes the $k$ hash values generated by $x$, and $y_1', y_2', \cdots , y_k'$ denotes the $k$ hash values generated by $x'$. 

Follow the Definition~\ref{def:Y}, let $Y_x$ denotes the set of distinct values in $y_1, y_2, \cdots , y_k$, and $Y_{x'}$ denotes the set of distinct values in $y_1', y_2', \cdots , y_k'$.

Suppose $|Y_x| = a, |Y_{x'}| = b$, where $a, b \in \{1, 2, \cdots , k \}$

We define $Z$ is the set of distinct values in $Y_x \cup Y_{x'}$, where $|Z| \in \{1, 2, \cdots , 2k \}$

\end{definition}


\subsection{Distribution of \texorpdfstring{$Y$}{}}\label{sec:distribution_Y}
Then we proceed to calculate the probability distribution of $Y$ in this section.
\begin{lemma}[Distribution of $Y$]\label{lem:distribution_of_Y}
If the following conditions hold
\begin{itemize}
    \item Let $y_1, y_2, \cdots , y_k$ be defined in Definition~\ref{def:Y}.
    \item Let $Y$ be defined as Definition~\ref{def:Y}.
\end{itemize}

Then, we can show, for $y = 1, 2, \cdots, k$, 
\begin{align*}
    & ~ \Pr[|Y| = y] \\
    = & ~ \begin{cases}
        1 / m^{k-1},  & y = 1 \\
        \binom{m}{y} \cdot y ^k / m^k
        - \sum_{i=1}^{k-1} \binom{m - i}{y -i} \Pr[Y = i], & y = 2, \cdots , k
    \end{cases}
\end{align*}
\end{lemma}

\begin{proof}

{\bf Step 1.} We consider $Y = 1$ case. 

Without any constraints, there are total $m^k$ situations. This is because each hash value can be freely chosen from $m$ positions, and there are $k$ hash values. Therefore, there are total $m^k$ situations. 

Then, with constraint $Y = 1$, $k$ hash values must be assigned to the same position. The position can be chosen from a total of $m$ positions. Therefore, in this case, there are $m$ situations. 

Combining the above two analysis, we have
\begin{align*}
    \Pr[Y = 1] = & ~ \frac{m}{m^k} \notag \\
    = & ~ \frac{1}{m^{k - 1}}.
\end{align*}

{\bf Step 2.} We consider $Y = 2, \cdots , k$ cases.

Similarly, without any constraints, there are total $m^k$ situations. 

Since we need $Y = y$, we must choose $y$ from different positions in the total $m$ positions. Therefore, we have $\binom{m}{y}$ term.

Note that in each position, we need at least one hash value. We first compute the number of freely assigning $k$ hash values to the $y$ positions. Then we remove the failure cases.  

As there are $y$ positions and $k$ hash values, we have the $y^k$ term for freely assigning $k$ hash values to $y$ positions.

For the failure case, we have $\sum_{i=1}^{k-1} \Pr[Y = i] \cdot \binom{m - i}{y -i}$. The $\binom{m - i}{y -i}$ term is due to repeated counting for each $i \in [k-1]$, where we first fix $i$ positions and then randomly pick the other $y-i$ different positions in the total $m-i$ positions. 

Thus, in all, we have the following formula,
\begin{align*}
    \Pr[Y = y] = \frac{\binom{m}{y} \cdot y ^k}{m^k} - \sum_{i=1}^{k-1} \Pr[Y = i] \cdot \binom{m - i}{y -i}.
\end{align*}

\end{proof}

\subsection{Distribution of \texorpdfstring{$Z$}{} conditioned on \texorpdfstring{$Y$}{}}\label{sec:distribution_Z}
In this section, we calculate the probability distribution of $Z$ condition on $Y$.

\begin{lemma}[Probability of $Z$ conditioned on $Y_x$ and $ Y_{x'}$]\label{lem:distribution_of_Z}
If the following conditions hold
\begin{itemize}
    \item Let $Y_x, Y_{x'}, Z$ be defined as Definition~\ref{def:Z}.
    \item Let $A_n^m$ denotes $n! / (n-m)!$.
    \item Let $t := z - \max(a, b)$. 
\end{itemize}

Then, we can show, for $z = \max(a, b), \cdots, (a + b)$, 
\begin{align*}
    \Pr[|Z| = z | |Y_x| = a, |Y_{x'}| = b] = \frac{A_m^a \cdot \binom{b}{t} \cdot A_{m - a}^t \cdot A_{a}^{b-t}}{A_m^a \cdot A_m^b}.
\end{align*}
\end{lemma}

\begin{proof}

Since the minimum value of $Z$ is $\max(a, b)$, without loss of generality, we assume $ a \geq b$. Then we have $a \leq z \leq (a + b)$.

Recall we have $t = z - \max(a, b) = z - a, t \in \{0, 1, \cdots , b\}$. Then we have
\begin{align*}
    & ~ \Pr[|Z| = a + t | |Y_x| = a, |Y_{x'}| = b] \\
    = & ~ \frac{A_m^a \cdot \binom{b}{t} \cdot A_{m - a}^t \cdot A_{a}^{b-t}}{A_m^a \cdot A_m^b}.
\end{align*}

We explain why we have the above equation in the following steps.

{\bf Step 1.} We consider the denominator. 

Without any constraints, since $|Y_x| = a$, we need to choose $a$ from different positions in the total $m$ positions. Therefore, we have the $A_m^a$ term in the denominator. Similarly, since $|Y_{x'}| = b$, we have the $A_m^b$ term in the denominator. 

{\bf Step 2.} We consider the numerator. 

Firstly, since $|Y_x| = a$, we need to choose $a$ different positions in total $m$ positions. Therefore, we have the $A_m^a$ term in the numerator. 

Since $Z$ is defined as Definition~\ref{def:Z}, we can have the following
\begin{align*}
    |Y_x \cap Y_{x'}| = & ~ a + b - z \notag \\
    |Y_{x'}| - |Y_x \cap Y_{x'}| = & ~ z - a \notag \\
    = & ~ t
\end{align*}

Then, we need to choose $t$ values from $Y_{x'}$ to construct $|Y_{x'}| - |Y_x \cap Y_{x'}|$ part. Therefore, we have the $\binom{b}{t}$ term in the numerator. 

We also need to choose $t$ different positions in the rest $m - a$ positions for  $|Y_{x'}| - |Y_x \cap Y_{x'}|$ part. Hence, we have the $A_{m - a} ^ t$ term in the numerator. 

Lastly, let's consider the $b - t$ part. For this part, we need to choose $b - t$ different positions from $a$ positions. Therefore, we have the $A_a^{b - t}$ term in the numerator. 

Combining all analyses together, finally, we have 
\begin{align*}
   \Pr[|Z| = z | |Y_x| = a, |Y_{x'}| = b] = \frac{A_m^a \cdot \binom{b}{t} \cdot A_{m - a}^t \cdot A_{a}^{b-t}}{A_m^a \cdot A_m^b}.
\end{align*}

\end{proof}

\subsection{Distribution of \texorpdfstring{$W$}{}}\label{sec:distribution_W}


\begin{figure}[!ht]
\centering
\includegraphics[width=0.45\textwidth]{w_figs/w_pmf.pdf}
\caption{
Let $W := |S|$ denote the number of bits in the Bloom filter changed by substituting an element in the inserted set $A$ (Definition~\ref{def:pre_neighbor_dataset}). We achieve $\epsilon_0$-DP for each single bit and $(\epsilon, \delta)$-DP for the entire Bloom filter via the random response (Definition~\ref{def:random_response}), where $\epsilon_0 = \epsilon / N$. 
The $N$ is $1 - \delta$ quantile of the random variable $W$. 
We visualize the distribution of the random variable $W$ (see Lemma~\ref{lem:distribution_of_W}) under the setting described in the experiments section (Section~\ref{sec:experiments}). Namely, we have the bit array length in the Bloom filter $m = 2^{19}$, the number of elements inserted into the Bloom filter $|A| = 10^{5}$, and the number of hash functions $k=3$. It can be inferred from this visualization that the values of random variable $W$ have good concentration properties, mostly concentrated around its mean. 
}
\label{fig:w_distribution}
\end{figure}

Finally, we present the calculation of the probability distribution of $W$ in this section.
\begin{lemma}[Distribution of $W$]\label{lem:distribution_of_W}
If the following conditions hold
\begin{itemize}
    \item Let $Y_x, Y_{x'}, Z$ be defined as Definition~\ref{def:Z}.
    \item Let $W$ be defined as Definition~\ref{def:W}.
    \item Let $A_n^m$ denotes $\frac{n!}{(n - m)!}$. 
    \item Let $p_0 := (1 - \frac{1}{m})^{(|A| - 1)k}$ denotes the proportion of bits which are still $0$ in the bit-array.
    \item Let $n_1 := |Y_x \cap Y_{x'}|= a + b - z$ denotes the number of overlap elements in $Y_x$ and $Y_{x'}$. 
    \item Let $n_2 := |Y_x \cup Y_{x'}| - |Y_x \cap Y_{x'}| =  z  -(a + b - z) = 2z -a -b$ denotes the number of exclusive or elements in $Y_x$ and $Y_{x'}$.
\end{itemize}

Then, we can show, for $w=0, \cdots 2k$,
\begin{align*}
    & ~ \Pr[W = w] \notag \\
    = & ~ \sum_{a = 1}^k \sum_{b = 1}^k \sum_{z = 1}^{a+b} \Pr[W = w | |Z| = z, |Y_x| = a, |Y_{x'}| = b] \notag \\
    & ~ \cdot \Pr[|Z| = z | |Y_x| = a, |Y_{x'}| = b] \\
    & ~ \cdot \Pr[|Y_x| = a] \cdot  \Pr[|Y_{x'}| = b].
\end{align*}

where

\begin{align*}
    & ~ \Pr[W = w | |Z| = z, |Y_x| = a, |Y_{x'}| = b] \\
    = & ~
    \begin{cases}
        0,  &  n_2 < w \\
        \binom{n_2}{w} \cdot p_0^w \cdot (1 - p_0)^{n_2 - w}, & n_2 \geq w
    \end{cases}
\end{align*}
\end{lemma}

\begin{proof}

By basic probability rules, we have the following equation
\begin{align*}
    & ~\Pr[W = w] \notag \\
    =& ~ \sum_{a = 1}^k \sum_{b = 1}^k \sum_{z = 1}^{a+b} \Pr[W = w | |Z| = z, |Y_x| = a, |Y_{x'}| = b] \notag \\
    & ~ \cdot \Pr[|Z| = z | |Y_x| = a, |Y_{x'}| = b] \\
    & ~ \cdot \Pr[|Y_x| = a, |Y_{x'}| = b] \notag \\
    =& ~ \sum_{a = 1}^k \sum_{b = 1}^k \sum_{z = 1}^{a+b} \Pr[W = w | |Z| = z, |Y_x| = a, |Y_{x'}| = b] \notag \\
    & ~ \cdot \Pr[|Z| = z | |Y_x| = a, |Y_{x'}| = b] \\
    & ~ \cdot \Pr[|Y_x| = a] \cdot \Pr[|Y_{x'}| = b].
\end{align*}
where the first step follows from basic probability rules, the second step follows from $Y_x$, and $Y_{x'}$ are independent. 

We can get the probability of $\Pr[|Y_x| = a]$ and $\Pr[|Y_{x'}| = b$ from Lemma~\ref{lem:distribution_of_Y}. 

We can get the probability of $\Pr[|Z| = z | |Y_x| = a, |Y_{x'}| = b]$ from Lemma~\ref{lem:distribution_of_Z}. 

Now, let's consider the $\Pr[W = w | |Z| = z, |Y_x| = a, |Y_{x'}| = b]$ term. 

Note that only elements in the exclusive-or set may contribute to the final $W$. Therefore, we have $w \leq n_2$. Namely, when $n_2 < w$, we have $\Pr[W = w | |Z| = z, |Y_x| = a, |Y_{x'}| = b] = 0$. 

Now, let's calculate $\Pr[W = w | |Z| = z, |Y_x| = a, |Y_{x'}| = b]$ under $n_2 \geq w$ condition. 

Recall $x$ denotes the element deleted from $A$, and $x'$ denotes the element added to $A$ for constructing the neighbor dataset $A'$. 

Let $A_{fix} := A - x$ denote the fixed set of elements during the modifications. We have $|A_{fix}| = |A| - 1$. 

Consider the following steps:
\begin{itemize}
    \item We construct a new Bloom filter.
    \item We insert all elements in $A_{fix}$ in the Bloom filter.
    \item We define $Z_{zero}$ as the set of positions of bits which are still $0$ after the insertion of $A_{fix}$.
\end{itemize}

We define $Z_{xor}$ as the exclusive-or set of $Y_x$ and $Y_{x'}$. We have
\begin{align*}
    Z_{xor} = & ~ (Y_x \cup Y_{x'}) - (Y_x \cap Y_{x'}), \notag \\
    |Z_{xor}| = & ~ |Y_x \cup Y_{x'}| - |Y_x \cap Y_{x'}| \notag \\
    = & ~ z - (a +b - z) \notag \\
    = & ~ 2z - a - b \notag \\
    = & ~ n_2.
\end{align*}

Note that only positions in $Z_{xor} \cap Z_{zero}$ will contribute to $W$. Namely, we need $|Z_{xor} \cap Z_{zero}| = w$. 

We achieve the above condition by selecting $w$ elements in $Z_{xor}$ and let them satisfy the condition of $Z_{zero}$. 

Therefore, we have 
\begin{align*}
    & ~ \Pr[|Z_{xor} \cap Z_{zero}| = w] \\
    = & ~ \binom{n_2}{w} \cdot (1 - \frac{1}{m})^{(|A| - 1)kw} \cdot (1 - (1 - \frac{1}{m})^{(|A| - 1)k})^{n_2 - w}.
\end{align*}

Combining the above analysis, we have
\begin{align*}
    & ~ \Pr[W = w | |Z| = z, |Y_x| = a, |Y_{x'}| = b] \\
    = & ~ 
    \begin{cases}
        0,  &  n_2 < w \\
        \binom{n_2}{w} \cdot p_0^w \cdot (1 - p_0)^{n_2 - w}, & n_2 \geq w
    \end{cases}.
\end{align*}


\end{proof}



\section{Privacy guarantees for one coordinate}\label{sec:appendix_privacy_guarantees}
In this section, we provide proof of the privacy guarantees of the DPBloomfilter.

In Section~\ref{sec:single_bit_private}, we demonstrate the privacy guarantees for single bit of array in Bloom filter.

Then in Section~\ref{sec:query_privacy}, we provide the proof of privacy guarantees for our entire algorithm.

\subsection{Single bit is private} \label{sec:single_bit_private}
We first consider the privacy guarantees of single bit of array in Bloom filter.
\begin{lemma} [Single bit is private
] \label{lem:eps0_DP:formal}
If the following conditions hold:
\begin{itemize}
    \item Let $\epsilon_0 \geq 0$. 
    \item Let $\wt{g}[j] \in \{0,1\}$ be the $i$-th element of array output by DPBloomfilter  
    
    
\end{itemize}

Then, we can show that, for all
$j \in [m]$, $\wt{g}[j]$ is $\epsilon_0$-DP. 
\end{lemma}

\begin{proof}

$\forall j \in [m]$, $g[j]$ is the ground truth value generated by dataset $A \subset [n]$. (An alternative view of $g$ is $g:[m] \rightarrow \{0,1\}$.) Suppose $g[j] = u$, $u \in \{0, 1\}$. For any neighboring dataset $A' \subset [n]$, we denote the ground truth value generated by it as $g'[j]$. Similarly, we can define the $\wt{g}'[j]$. 

We consider the following two cases to prove $\wt{g}[j]$ is $\epsilon_0$-DP, for all $j \in [m]$.

{\bf Case 1}. Suppose $g'[j] = u$. We know
\begin{align*}
    \Pr [ \wt{g}[j] = u ] = & ~ \frac{e^{\epsilon_0}}{ e^{\epsilon_0} + 1 }, \\
    \Pr[ \wt{g}'[j] = u ] = & ~ \frac{e^{\epsilon_0}}{ e^{\epsilon_0} + 1 }.
\end{align*}
Combining the above two equations, then we obtain
\begin{align*}
\frac{ \Pr [ \wt{g}[j] = u ] }{ \Pr[ \wt{g}'[j] = u ] } = 1.
\end{align*}

Similarly, we know 
\begin{align*}
    \Pr [ \wt{g}[j] = 1-u ] = & ~ \frac{ 1 }{ e^{\epsilon_0} + 1 }, \\
    \Pr[ \wt{g}'[j] = 1-u ] = & ~ \frac{ 1 }{ e^{\epsilon_0} + 1 }.
\end{align*}
Combining the above two equations, then we obtain
\begin{align*}
\frac{ \Pr [ \wt{g}[j] = 1- u ] }{ \Pr[ \wt{g}'[j] = 1-u ] } = 1.
\end{align*}
Thus, we know for all $v\in \{0,1\}$,
\begin{align*}
\frac{ \Pr [ \wt{g}[j] = v ] }{ \Pr[ \wt{g}'[j] = v ] } = 1.
\end{align*}

{\bf Case 2}. Suppose $g'[j] \neq u$.

We know
\begin{align*}
    \Pr [ \wt{g}[j] = u ] = & ~ \frac{e^{\epsilon_0}}{ e^{\epsilon_0} + 1 }, \\
    \Pr[ \wt{g}'[j] = u ] = & ~ \frac{ 1 }{ e^{\epsilon_0} + 1 }.
\end{align*}
Combining the above two equations, then we obtain
\begin{align*}
\frac{ \Pr [ \wt{g}[j] = u ] }{ \Pr[ \wt{g}'[j] = u ] } = e^{\epsilon_0}.
\end{align*}

Similarly, we know 
\begin{align*}
    \Pr [ \wt{g}[j] = 1-u ] = & ~ \frac{ 1 }{ e^{\epsilon_0} + 1 }, \\
    \Pr[ \wt{g}'[j] = 1-u ] = & ~ \frac{ e^{\epsilon_0} }{ e^{\epsilon_0} + 1 }.
\end{align*}
Combining the above two equations, then we obtain
\begin{align*}
\frac{ \Pr [ \wt{g}[j] = 1- u ] }{ \Pr[ \wt{g}'[j] = 1-u ] } = e^{-\epsilon_0}.
\end{align*}


For $v \in \{0, 1\}$, we have 
\begin{align*}
e^{- \epsilon_0} \leq \frac{ \Pr [ \wt{g}[j] = v ] }{ \Pr [ \wt{g}'[j] = v ] } \leq e^{\epsilon_0}.
\end{align*}



Therefore, $\forall j \in [m]$, $\wt{g}[j]$ is $\epsilon_0$-DP. 
\end{proof}

\subsection{Privacy guarantees for DPBloomfilter}\label{sec:query_privacy}
Then, we can prove that our entire algorithm is differentially private.
\begin{theorem}[Privacy for Query, formal version of Lemma~\ref{thm:query_privacy:informal}]\label{thm:query_privacy:formal}
If the following conditions hold
\begin{itemize}
    \item Let $N = F_W^{-1}(1 - \delta)$ denote the $1 - \delta$ quantile of the random variable $W$ (see Definition~\ref{def:W}).
    \item Let  $\epsilon_0 = \epsilon / N$.
\end{itemize}

Then, we can show,
the output of \textsc{Query} procedure of Algorithm~\ref{alg:init} achieves $(\epsilon, \delta)$-DP. 
\end{theorem}

\begin{proof}
Let $A$ and $A'$ are neighboring datasets. Let $g \in \{0, 1\}^m$ is the ground truth value generated by dataset $A$, and $g' \in \{0, 1\}^m$ is the ground truth value generated by dataset $A'$. 


We define
\begin{align*}
    S := \{j \in [m] ~:~ g[j] \neq g'[j]\}.
\end{align*}
We further define
\begin{align*}
    \ov{S} := [m] \backslash S.
\end{align*}

We consider two cases, {\bf Case 1} is $j \in \ov{S}$ and {\bf Case 2} is $j \in S$.

{\bf Case 1}. $j \in \ov{S}$. 

We can show that
\begin{align*}
\frac{ \Pr [ \wt{g}[j] = v ] }{\Pr[ \wt{g'}[j] = v ] } = 1.
\end{align*}
holds for $\forall v \in \{0, 1\}$.

{\bf Case 2.} $j \in S$.

We can show that
\begin{align}\label{eqn:query_privacy_single}
    e^{-\epsilon_0}\leq \frac{ \Pr[ \wt{g}[j] = v ] }{ \Pr[ \wt{g'}[j] = v] } \leq e^{\epsilon_0}.
\end{align}
holds for $\forall v \in \{0, 1\}$. 

Thus, for any $Z\in \{0,1\}^m$, the absolute privacy loss can be bounded by
\begin{align}\label{eqn:query_privacy_prod}
     |\ln \frac{ \Pr[ \wt{g} = Z ] }{ \Pr[ \wt{g'} = Z ] } | 
     = & ~  |\ln \prod_{j\in S} \frac{ \Pr[ \wt{g}[j] = v ] }{ \Pr[ \wt{g'}[j] = v ] }  | \notag \\
     \leq & ~ |S| \epsilon_0 \notag \\
     = & ~  |S|\frac{\epsilon}{N}.  
\end{align}
where the first step follows from each entry of $g$ is independent, the second step follows from Eq.~\eqref{eqn:query_privacy_single}, and the last step follows from choice of $\epsilon_0$.

By the definition of $N$, we know that with probability at least $1-\delta$, $|S|\leq F^{-1}(1-\delta)=N$. Hence, Eq.~\eqref{eqn:query_privacy_prod} is upper bounded by $\epsilon$ with probability $1-\delta$. 

This proves the $(\epsilon,\delta)$-DP.
\end{proof}

\section{Utility analysis}\label{sec:appendix_utility}
In this section, we establish the utility guarantees for our algorithm. Initially, we calculate the accuracy for the query of the standard Bloom filter in Section~\ref{sec:acc_bloom}. We then assess the utility loss caused by introducing the random response technique by comparing the output of the DPBloomfilter with the output of the standard Bloom filter in Section~\ref{sec:acc_dpbloom_bloom}. Ultimately, we present the assessment of our algorithm's utility in Section~\ref{sec:acc_dpbloom_true}.

We begin by defining the notation we will use in this section.
\begin{definition}\label{def:three_z}
    Let $z \in \{0,1\}$ denote the true answer for whether $x \in A$. Let $\wh{z} \in \{0,1\}$ denote the answer outputs by \textsc{Bloom filter}. Let $\wt{z} \in \{0,1\}$ denote the answer output by \textsc{DPBloomFilter} (Algorithm~\ref{alg:init}).
\end{definition}

\subsection{Accuracy for query of Standard Bloom Filter}\label{sec:acc_bloom}

We first present the accuracy of the query of the standard bloom filter, as follows.

\begin{lemma}[Accuracy for query of Standard Bloom Filter
]\label{lem:bloom_true_accuracy:formal}
If the following conditions hold
\begin{itemize}
    \item Assume that a hash function selects each array position with equal probability. 
    \item Let $\wh{z}$ be defined as Definition~\ref{def:three_z}.
    \item Let $z$ be defined as Definition~\ref{def:three_z}.
    \item Let $\alpha := \Pr[z=0]$
\end{itemize}
Then, we can show
\begin{align*}
    \Pr [ \wh{z} = z ] \geq 1 - (1 - e^{-2|A| k / m})^k \cdot \alpha.
\end{align*}
Further if $m = \Omega(|A| k)$ and $k = \Theta(\log(1/\delta_{err}))$, we have
\begin{align*}
     \Pr [ \wh{z} = z ] \geq 1 - \delta_{err} \cdot \alpha.
\end{align*}
\end{lemma}

\begin{proof}
Recall that we have defined Bloom filter in Definition~\ref{def:bloom_filter}, it only has false positive error. Therefore, we only need to calculate the following
\begin{align*}
    \Pr[\wh{z} = 1 | z = 0]
\end{align*}

Recall that $A \subset [n]$ denotes the set of elements inserted into the Bloom filter. And $h_i : [n] \rightarrow [m]$ for each $i \in [k]$ denotes $k$ hash functions used in the Bloom filter. 

For a query $y \notin A$, we denotes event $E_1$ happens if the following happens:
\begin{align*}
    h_i[y] = 1, \forall i \in [k]
\end{align*}

Recall that we have defined Bloom filter in Definition~\ref{def:bloom_filter}, we have 
\begin{align}\label{eq:def_E_1}
    \Pr[\wh{z} = 1 | z = 0] = \Pr[E_1].
\end{align}

Now, we start calculating $\Pr[E_1]$.

Recall that we assume a hash function selects each array position with equal probability in the lemma statement. 

During one inserting operation, the probability of a certain bit is not set to $1$ is 
\begin{align*}
    (1 - \frac{1}{m})^k
\end{align*}



If we have inserted $|A|$ elements, the probability that a certain bit is still $0$ is
\begin{align*}
    (1 - \frac{1}{m})^{|A| k} = ( (1-\frac{1}{m})^{m} )^ {|A| k/m } \geq e^{-2 |A| k / m}
\end{align*}
where the last step follows from $(1-1/m)^m \geq e^{-2}$ for all $m \geq 2$.

Thus the probability that a certain bit is $1$ is
\begin{align*}
    1 - (1 - \frac{1}{m})^{ |A| k} \leq 1 - e^{-2 |A| k / m}.
\end{align*}

Combining the above fact, we have
\begin{align}\label{eq:pr_e}
    \Pr[E_1] = & ~ (1 - (1 - \frac{1}{m})^{|A|k})^k \notag \\
    \leq & ~ (1 - e^{-2 |A| k / m})^k.
\end{align}
where the first step follows from the definition of event $E_1$, the second step follows from $(1-1/m)^m \geq e^{-2}$ for all $m \geq 2$. 

Therefore, the accuracy of Bloom filter is
\begin{align*}
    \Pr[\wh{z} = z] 
    = & ~ 1 - \Pr[\wh{z} = 1 | z = 0] \Pr[z=0] \\
    = & ~ 1 - \Pr[E_1] \alpha \\
    \geq & ~ 1 - (1 - e^{-2 |A| k / m})^k \alpha.
\end{align*}
where the first step follows from Bloom filter only has false positive error, the second step follows from the definition of event $E_1$ and the definition of $\alpha$, the third step follows from Eq.~\eqref{eq:pr_e}. 

\end{proof}

\subsection{Accuracy (compare DPBloomFilter with Standard BloomFilter) for Query}\label{sec:acc_dpbloom_bloom}
We then assess the accuracy loss caused by the introduction of the random response technique by comparing the outputs of the DPBloomfilter with those of the standard Bloom filter.

\begin{lemma}[Accuracy (compare DPBloomFilter with Standard BloomFilter) for Query
]\label{lem:dpbloom_bloom_accuracy:formal}

If the following conditions hold
\begin{itemize}
    \item Let $\wh{z}$ be defined as Definition~\ref{def:three_z}.
    \item Let $\wt{z}$ be defined as Definition~\ref{def:three_z}.
    \item Let $\alpha: = \Pr[ z = 0 ] \in [0,1]$
    \item Let $t := \frac{ e^{\epsilon_0} }{ e^{\epsilon_0} + 1 }$. 
    \item Let $\delta_{\mathrm{err}}$ be defined as in Lemma~\ref{lem:bloom_true_accuracy:formal}. 
\end{itemize}

Then, we can show
\begin{align*}
\Pr[ \wt{z} = \wh{z}] \geq t \cdot (\alpha - \delta_{\mathrm{err}}).
\end{align*}

\end{lemma}
\begin{proof}
We denote the query as $q$. 

We define
\begin{align}\label{def:Q}
    Q := \{j \in [m] ~:~ h_i(q) = j,~ i \in [k]\}
\end{align}



We denote $Q[i]$ as the $i$-th element in $Q$. 

Using basic probability rules, we have
\begin{align*}
    & ~ \Pr[\wt{z} = \wh{z}] \\
    = & ~ \Pr[\wt{z} = 1 | \wh{z} = 1] \Pr[\wh{z} = 1] \\
    + & ~ \Pr[\wt{z} = 0 | \wh{z} = 0] \Pr[\wh{z} = 0].
\end{align*}

{\bf Step 1}. Calculate $\Pr[\wt{z} = 1 | \wh{z} = 1]$


We denote event $E_2$ happens as the following happens:
\begin{align*}
    \wt{g}[j] = g[j], \forall j \in Q.
\end{align*}

Recall that we have defined Bloom filter in Definition~\ref{def:bloom_filter}, we have 

\begin{align*}
    \Pr[\wt{z} = 1 | \wh{z} = 1] = \Pr[E_2].
\end{align*}

 
Now, we calculate the probability that $E_2$ happens. 
\begin{align*}
    \Pr [E_2] = & ~ \prod_{i = 1}^k \Pr [\wt{g}[Q[i]] = g[Q[i]]] \notag \\
    = & ~ (\frac{e^{\epsilon_0}}{e^{\epsilon_0} + 1})^k.
\end{align*}
where the first step follows from each entry of $g$ is independent, the second steps follows from the definition of $\wt{g}$. 

Therefore, we have
\begin{align}\label{eq:pr_wtz1_z1}
    \Pr[\wt{z} = 1 | \wh{z} = 1] 
    = & ~ (\frac{e^{\epsilon_0}}{e^{\epsilon_0} + 1})^k.
\end{align}

{\bf Step 2}. Calculate $\Pr[\wt{z} = 0 | \wh{z} = 0]$

Recall we have defined $Q \subset [m]$ in Eq.~\eqref{def:Q}. We further define
\begin{align*}
    Z := \{j \in Q ~:~ g[j] = 0\}.
\end{align*}

We denote $Z[i]$ as the $i$-th element in $Z$. 

We further define
\begin{align*}
    \ov{Q} := Q \backslash Z.
\end{align*}


By basic probability rules, we have
\begin{align*}
    \Pr[\wt{z} = 0 | \wh{z} = 0] = & ~ 1 - \Pr[\wt{z} = 1 | \wh{z} = 0].
\end{align*}

Now, let's calculate $\Pr[\wt{z} = 1 | \wh{z} = 0]$

$[\wt{z} = 1 | \wh{z} = 0]$ happens only if the following conditions hold:
\begin{enumerate}
    \item All elements in $Z$ flip from $0$ to $1$.
    \item All elements in $\ov{Q}$ remain $1$.
\end{enumerate}

Then, we have
\begin{align*}
    \Pr[\wt{z} = 1 | \wh{z} = 0]  = & ~ \prod_{i = 1}^{|Z|} \Pr [\wt{g}[Z[i]] = 1] \prod_{i = 1}^{|\ov{Q}|} \Pr [\wt{g}[\ov{Q}[i]] = 1] \notag \\
    = & ~ (\frac{1}{e^{\epsilon_0} + 1})^{|Z|} (\frac{e^{\epsilon_0}}{e^{\epsilon_0} + 1})^{|\ov{Q}|} \notag \\
    \leq & ~ (\frac{1}{e^{\epsilon_0} + 1})^{|Z|} \notag \\
    \leq & ~ \frac{1}{e^{\epsilon_0} + 1}.
\end{align*}
where the first step follows from the above analysis, the second step follows from the definition of $\wt{g}$, the third step follows from $|\ov{Q}| \geq 0$ and $\frac{e^{\epsilon_0}}{e^{\epsilon_0} + 1} < 1$, the fourth step follows from $|Z| \geq 1$ and $\frac{1}{e^{\epsilon_0} + 1} < 1$. 

Therefore, we have
\begin{align}\label{eq:pr_wtz0_z0}
    \Pr[\wt{z} = 0 | \wh{z} = 0] = & ~ 1 - \Pr[\wt{z} = 1 | \wh{z} = 0] \notag \\
    \geq & ~ 1 -  \frac{1}{e^{\epsilon_0} + 1} \notag \\
    = & ~ \frac{ e^{\epsilon_0} }{ e^{\epsilon_0} + 1 }.
\end{align}
Let $\hat \alpha := \Pr[ \wh{z} = 0 ]$, then we have $1- \wh{\alpha} = \Pr[ \wh{z} = 1 ]$. 
Let $\alpha := \Pr[ z = 0 ]$.
Note that $ \wh{\alpha} = \alpha (1 - \delta_{\mathrm{err}}) $.

Let $t := \frac{e^{\epsilon_0}}{e^{\epsilon_0} + 1}$. 

The final accuracy is 
\begin{align*}
& ~ \Pr[\wt{z} = 0 | \wh{z} = 0]  \cdot \Pr[ \wh{z} = 0 ] + \Pr[\wt{z} = 1 | \wh{z} = 1]  \cdot \Pr[ \wh{z} = 1 ] \\
= & ~ \Pr[\wt{z} = 0 | \wh{z} = 0]  \cdot \wh{\alpha} + \Pr[\wt{z} = 1 | \wh{z} = 1]  \cdot (1- \wh{\alpha}) \\
= & ~ \Pr[\wt{z} = 0 | \wh{z} = 0]  \cdot \alpha (1 - \delta_{err}) \\
+ & ~ \Pr[\wt{z} = 1 | \wh{z} = 1]  \cdot (1- \alpha + \alpha \cdot \delta_{err}) \\
\geq & ~ \frac{ e^{\epsilon_0} }{ e^{\epsilon_0} + 1 }  \cdot \alpha (1 - \delta_{err}) + (\frac{ e^{\epsilon_0} }{ e^{\epsilon_0} + 1 })^k  \cdot (1- \alpha + \alpha \cdot \delta_{err}) \notag \\ 
= & ~ t \cdot (\alpha - \alpha \cdot \delta_{err}) + t^k  \cdot (1- \alpha + \alpha \cdot \delta_{err}) \\
\geq & ~ t \cdot \alpha \cdot (1 - \delta_{err}).
\end{align*}



where the first step follows from the definition of $\wh{\alpha}$, the second step follows from $ \wh{\alpha} = \alpha (1 - \delta) $, the third step follows from  Eq.~\eqref{eq:pr_wtz1_z1} Eq.~\eqref{eq:pr_wtz0_z0}, the fourth step follows from basic algebra rules, the fifth step follows from $(1 - \alpha + \alpha \cdot \delta_{\mathrm{err}}) \geq 0$. 

Therefore, the final accuracy is $t \cdot (\alpha - \delta_{err})$. 
\end{proof}

\subsection{Accuracy (compare DPBloomfilter with true-answer) for Query}\label{sec:acc_dpbloom_true}
Now we can examine the utility guarantees of DPBloomfilter by calculating the error between the ground truth for query and the output of DPBloomfilter.

\begin{theorem}[Accuracy (compare DPBloomfilter with true-answer) for Query, formal version of Lemma~\ref{thm:dpbloom_true_accuracy:informal}]\label{thm:dpbloom_true_accuracy:formal}

If the following conditions hold
\begin{itemize}
    \item Let $\wh{z}$ be defined as Definition~\ref{def:three_z}.
    \item Let $z$ be defined as Definition~\ref{def:three_z}.
    \item Let $\alpha: = \Pr[ z = 0 ] \in [0,1]$
    \item Let $t := e^{\epsilon_0} / (e^{\epsilon_0} + 1)$. 
    \item Let $\delta_{\mathrm{err}}$ be defined as in Lemma~\ref{lem:bloom_true_accuracy:formal}. 
\end{itemize}

Then, we can show 
\begin{align*}
\Pr[ \wt{z} = z ] \geq \alpha (1-t-t^k) \delta_{\mathrm{err}} + \alpha t .
\end{align*}
\end{theorem}

\begin{proof}

We have
\begin{align*}
    & ~ \Pr[ \wt{z} = z ] \\
    = & ~ \Pr [\wt{z} = 0 | \wh{z} = 0] \Pr [\wh{z} = 0 | z = 0] \Pr[z=0] \\
    + & ~ \Pr [\wt{z} = 0 | \wh{z} = 1] \Pr [\wh{z} = 1 | z = 0] \Pr[z=0] \\
    + & ~ \Pr [\wt{z} = 1 | \wh{z} = 1] \Pr [\wh{z} = 1 | z = 1] \Pr[z=1] \\
    + & ~ \Pr [\wt{z} = 1 | \wh{z} = 0] \Pr [\wh{z} = 0 | z = 1] \Pr[z=1]\\
    \geq & ~ t \cdot (1 - \Pr[E_1]) \cdot \alpha + (1- t^k) \cdot \Pr[E_1]\cdot \alpha + t^k \cdot 1 \cdot (1-\alpha)\\
     = & ~ \alpha (1-t-t^k) \delta_{\mathrm{err}} + \alpha t + t^k(1-\alpha)\\
     \geq & ~ \alpha (1-t-t^k) \delta_{\mathrm{err}} + \alpha t.
\end{align*}
where the first step from basic probability rules, the secod step follows from Equation~\ref{eq:def_E_1}, Equation \ref{eq:pr_wtz0_z0} and definition of $\alpha$ and $t$, the third step follows from basic algebra,  the fourth step follows from the fact that $t,\alpha \in [0,1]$.

\end{proof}

To make it easier to understand, we also provide the utility analysis of the Bloom filter under the case of random guess.  

\begin{lemma}[Accuracy for Query under Random Guess]\label{lem:random_guess}
If the following conditions hold
\begin{itemize}
    \item Let $\wh{z}$ be defined as Definition~\ref{def:three_z}.
    \item $\epsilon_0 = 0$. Namely, each bit in the bit-array of the DP Bloom has $\frac{1}{2}$ probability to be set to $0$, and  $\frac{1}{2}$ probability to be set to $1$. 
\end{itemize}

Then, we can show 
\begin{align*}
    \Pr[\wt{z} = 0] = & ~ 1 - \frac{1}{2^k}, \notag \\
    \Pr[\wt{z} = 1] = & ~ \frac{1}{2^k}.
\end{align*}
\end{lemma}

\begin{proof}
    By the definition of Bloom filter~\ref{def:bloom_filter}, the answer $\wt{z} = 1$ requires $k$ corresponding positions in the bit-array of the query are all set to $1$. 

    Note that each bit has $\frac{1}{2}$ probability to be set to $1$. Therefore, we have
    \begin{align*}
        \Pr[\wt{z} = 1] = \frac{1}{2^k} .
    \end{align*}

    Then, we have $\Pr[\wt{z} = 0] = 1 - \Pr[\wt{z} = 1] = 1 - \frac{1}{2^k}.$
\end{proof}





\section{Running Time}\label{sec:appendix_running_time}
In this section, we provide the proof of running time for Algorithm~\ref{alg:init}. The running time for our algorithm consists of two parts: time for initialization in Section~\ref{sec:time_init} and time for query 
in Section~\ref{sec:time_query}. 
\subsection{Running time for initialization}\label{sec:time_init}
Now we calculate the time of initialization for our algorithm. 
\begin{lemma}[Running time for initialization]\label{lem:init_time}
Let $\mathcal{T}_h$ denote the time of evaluation of function $h$ at any point. 

It takes $O(|A| \cdot k \cdot \mathcal{T}_h + m)$ time to run the initialization function.
\end{lemma}
\begin{proof}
 
{\bf Step 1} Let's consider the initialization of the standard Bloom filter. 

A single element $x$ needs $O(k \cdot \mathcal{T}_h)$ time to compute over $k$ hash functions. 

There are $|A|$ elements which need to be inserted. 

Combining the above two facts, it needs $O(|A| \cdot k \cdot \mathcal{T}_h)$ time to initialise the standard Bloom filter. 

{\bf Step 2} Let's consider the ``Flip each bit" part. 

Since there are $m$ bits in the Bloom filter, it takes $O(m)$ time to flip each bit.

Therefore, the initialization function needs $O(|A| \cdot k \cdot \mathcal{T}_h + m)$ time to run. 

\end{proof}

\subsection{Running time for query}\label{sec:time_query}
Then, we proceed to calculate the query time for our algorithm.

\begin{lemma}[Running time for query]\label{lem:query_time}
Let $\mathcal{T}_h$ denote the time of evaluation of function $h$ at any point. 
It takes $O(k \cdot \mathcal{T}_h)$ time to run each query $y$ in the query function.
\end{lemma}
\begin{proof}

For each query $y$, the algorithm needs $O(k \cdot \mathcal{T}_h)$ time to compute the hash values of $y$ over $k$ hash functions. 

Therefore, it takes $O(k\cdot \mathcal{T}_h)$ time to run the query function for each query. 
\end{proof}

By combing the result of Lemma~\ref{lem:init_time} and Lemma~\ref{lem:query_time}, we can obtain the running of our entire algorithm is $O(|A|\cdot k \cdot \mathcal{T}_h + m)$.



% \section{Technical Overview}\label{sec:tech_overview}
In Section~\ref{sec:tec_privacy_sb}, we will provide the privacy guarantees of Single Bit in DPBloomfilter. Then, we will present the privacy guarantees of our entire algorithm in Section~\ref{sec:tec_privacy_dp}. In Section~\ref{sec:tec_utility_dp}, we will examine the utility guarantees of DPBloomfilter. Additionally, we include a remark that analyzes the trade-off between privacy and utility inherent in our approach. In Section~\ref{sec:tec_time_dp}, we discuss the running time of our algorithm.


\subsection{Privacy Guarantees of Single Bit}\label{sec:tec_privacy_sb}

To accomplish differential privacy, Algorithm~\ref{alg:init} applies a random response mechanism to each bit of the standard Bloom Filter. In this section, we aim to examine the privacy guarantees for a single bit of our algorithm.

Recall that in Definition~\ref{sec:pre_def_bf}, for dataset $A \subset [n]$, we use $g[j]$  to denote the $j$-th element of array output by standard Bloom Filter. Here, we use $\wh{g}[j]$ to denote the $j$-th element of array output by DPBloomfilter. Similarly, for any neighboring dataset $A' \subset [n]$, we use $g'[j]$ and $\wh{g}'[j]$ to denote the $j$-th element of array output by standard Bloom Filter and DPBloomfilter. 
To examine the privacy guarantees for the $i$-th bit, we must consider two distinct cases.

{\bf Case 1}. Suppose $g'[j] = g[j]$, then we can obtain (See also Lemma~\ref{lem:eps0_DP:formal}) that  for all $v \in \{0,1\}$, we have
\begin{align*}
    \frac{\Pr[\wt{g}[j]=v]}{\Pr[\wt{g}'[j]=v]} = 1.
\end{align*}

{\bf Case 2}. Suppose $g'[j] \neq g[j]$, then we can obtain (See also Lemma~\ref{lem:eps0_DP:formal}) that for all $v \in {0,1}$, we have
\begin{align*}
    e^{-\epsilon_{0}} \leq \frac{\Pr[\wt{g}[j]=v]}{\Pr[\wt{g}'[j]=v]} \leq e^{\epsilon_{0}}.
\end{align*}

By combining the above two cases, we can demonstrate the privacy guarantees of single bit for our algorithm.

\begin{lemma} [Differential Privacy for single Bit, informal version of Lemma~\ref{lem:eps0_DP:formal}] \label{lem:eps0_DP:informal}
Let $\epsilon_0 \geq 0$ and $\wt{g}[i] \in \{0,1\}$ be the $i$-th element of array output by DPBloomfilter. 
Then, we can show that, for all
$j \in [m]$, $\wt{g}[j]$ is $\epsilon_0$-DP. 
\end{lemma}



\subsection{Privacy Guarantees of DPBloomFilter}\label{sec:tec_privacy_dp}
Here, we comprehensively analyze the DP guarantees for our DPBloomFilter. Recall that in Definition~\ref{def:bloom_filter}, for dataset $A$, we use $g$ to denote the array output by standard Bloom Filter. Here, we use $\wt{g}$ to denote the array output by DPBloomfilter. Similarly, for any neighboring dataset $A'$, we use $g'$ and $\wh{g}'$ to denote the array output by standard Bloom Filter and DPBloomfilter, respectively.

Here, we consider the set of indices $j$ within the range $m$ where the value of $g[j]$ and $g'[j]$ differs.
\begin{align*}
    S := \{j \in [m] : g[j] \neq g'[j]\}.
\end{align*}
Thus, the set of indices $j$ where the value of $g[j]$ and $g'[j]$ are the same can be defined as
$
    \ov{S} := [m] \backslash S.
$
We can use the result of privacy guarantees of a single bit in Section~\ref{sec:tec_privacy_sb}, for any $j \in S$ and $v \in \{0,1\}$, we have
\begin{align*}
\frac{\Pr[\wt{g}[j]=v]}{\Pr[\wt{g}'[j]=v]}  = 1,
\end{align*}
and for any $j \in \ov{S}$ and $v \in \{0,1\}$, we have
\begin{align*}
e^{-\epsilon_{0}} \leq \frac{\Pr[\wt{g}[j]=v]}{\Pr[\wt{g}'[j]=v]} \leq e^{\epsilon_{0}}.
\end{align*}
By applying the composition lemma (refer to Lemma~\ref{lem:pre_com_lem}) , we obtain the following for any $Z \in \{0,1\}^m$,
\begin{align}\label{equ:epsilon_0_bound}
|\ln{\frac{\Pr[\wt{g} = Z]}{\Pr[\wt{g}' = Z]}}| \leq  |S|\epsilon_0.
\end{align}
Here, we define $W := |S|$ for convenience. To get a better bound for Equation~\ref{equ:epsilon_0_bound}, we need to calculate the probability distribution function of the random variable $W$. Before that, we need to define two random variables we will use. Firstly, we define $Y$ as the set of distinct values among the $k$ hash values generated by the standard Bloom filter considering one $x \in [n]$. Then we consider two data $x, x' \in [n]$. We define $Z$ as the set of distinct values in $Y_{x} \cup Y_{x'}$.

Then firstly we proceed to calculate the distribution of $|Y|$ (see details in Lemma~\ref{lem:distribution_of_Y}), we can show for any $y = 1,2,\dots,k$
\begin{align*}
    & ~ \Pr[|Y| = y] \\
    = & ~ 
    \begin{cases}
        1 / m^{k-1},  & y = 1 \\
        \binom{m}{y} \cdot y ^k / m^k
        - \sum_{i=1}^{k-1} \Pr[Y = i] \cdot \binom{m - i}{y -i}, & y = 2, \cdots , k
    \end{cases}
\end{align*}
Given the probability of $|Y|$, we can calculate the conditional probability of $|Z|$ conditioned on $|Y_{x}| = a$ and $|Y_{x'}| = b$, where $a,b \in [k]$ (see details in Lemma~\ref{lem:distribution_of_Z})
\begin{align*}
    \Pr[|Z| = z | |Y_x| = a, |Y_{x'}| = b] = \frac{A_m^a \cdot \binom{b}{t} \cdot A_{m - a}^t \cdot A_{a}^{b-t}}{A_m^a \cdot A_m^b}.
\end{align*}
Finally, we use the property of union probability. We can calculate the probability of $W$ (see details in Lemma~\ref{lem:distribution_of_W}). 
Fig.~\ref{fig:w_distribution} visualize the distribution of the random variable $W$ under the setting described in the experiments section (Section~\ref{sec:experiments}). It shows that the distribution of $W$ has a good concentration property, i.e., it concentrates on its mean.

Recall in Section~\ref{sec:pre_notations}, we use $F_{X}^{-1}$ to denote the $1-\delta$ quantile of the Cumulative Distribution Function $F_{X}(x)$ of random variable $X$. 

Here, we define
\begin{align*}
    N:= F_{W}^{-1}(1-\delta)
\end{align*}
Hence, by the properties of the quantile function, we have
\begin{align*}
    \Pr[N \leq W] = 1-\delta.
\end{align*}
By choosing the appropriate value of $\epsilon_0 = \epsilon/N$, we have
\begin{align*}
|\ln{\frac{\Pr[\wt{g} = Z]}{\Pr[\wt{g}' = Z]}}| \leq W\frac{\epsilon}{N}.
\end{align*}
Then we have, with probability $1-\delta$,
\begin{align*}
    |\ln{\frac{\Pr[\wt{g} = Z]}{\Pr[\wt{g}' = Z]}}| \leq \epsilon.
\end{align*}
Then, we can demonstrate the privacy guarantees for DPBloomfilter (see also Theorem~\ref{thm:query_privacy:informal}).



\subsection{Utility Guarantees of DPBloomfilter}\label{sec:tec_utility_dp}
This section will present a comprehensive analysis of the utility guarantees for DPBloomfilter.
We start by introducing the following conditions for the Utility guarantee of DPBloomFilter.
\begin{condition} \label{con:utility_condition}
We need the following conditions for Utility guarantees of DPBloomfilter:
\begin{itemize}
    \item \textbf{Condition 1.} Assume that a hash function selects each array position with equal probability.
    \item \textbf{Condition 2.} Let $z \in \{0,1\}$ denote the ground truth for whether an element $y \in A$.
    \item \textbf{Condition 3.} Let $\wh{z} \in \{0,1\}$ denote the answer output by standard Bloom Filter for whether an element $y \in A$.
    \item \textbf{Condition 4.} Let $\wt{z} \in \{0,1\}$ denote the answer output by DPBloomfilter for whether an element $y \in A$
    \item \textbf{Condition 5.} Let $\alpha:=\Pr[z=0] \in [0,1]$
    \item \textbf{Condition 6.} Let $t := e^{\epsilon_0} / (e^{\epsilon_0} + 1)$. 
\end{itemize}
    
\end{condition}

Firstly, we proceed to derive the utility of the standard Bloom Filter by calculating 
\begin{align*}
    \Pr[\wh{z} = z] = 1 - \Pr[\wh{z} = 1 | z = 0] \Pr[z=0].
\end{align*}
The above equation comes from the fact that Bloom Filter will not introduce a false negative. After the initialization process of Bloom Filter, the probability of one certain bit is not set to $1$ is (see also Lemma~\ref{lem:bloom_true_accuracy:formal})
\begin{align*}
    (1-\frac{1}{m})^{|A|k} \geq e^{-2|A|k/m}.
\end{align*}
A false positive occurs when, for all $i \in [k]$, the elements $g[h_i(y)]$ are all set to $1$ after initialization. In this case, we have:
\begin{align*}
    \Pr[\wh{z} = 1 | z = 0] = & ~( 1 - (1 - \frac{1}{m})^{|A|k})^k \leq  ~ (1 - e^{-2|A|k/m})^k.
\end{align*}
Therefore, we have
\begin{align*}
    \Pr[\wh{z} = z] \geq & ~ 1 - (1 - e^{-2|A|k/m})^{k} \alpha.
\end{align*}
Further if $m = \Omega(|A|k)$ and $k = \Theta(log(\alpha/\delta_{err}))$, we have
\begin{align*}
    \Pr[\wh{z} = z] = 1 - \delta_{\mathrm{err}} \cdot \alpha.
\end{align*}
\begin{lemma} [Accuracy for query of Standard Bloom filter, informal version of Lemma~\ref{lem:bloom_true_accuracy:formal}]\label{lem:bloom_true_accuracy:informal}
If Condition~\ref{con:utility_condition} holds, we have
\begin{align*}
    \Pr [ \wh{z} = z ] \geq 1 - (1 - e^{-2|A| k / m})^k \cdot \alpha.
\end{align*}
Further if $m = \Omega(|A| k)$ and $k = \Theta(\log(1/\delta_{err}))$, we have
\begin{align*}
     \Pr [ \wh{z} = z ] \geq 1 - \delta_{\mathrm{err}} \cdot \alpha.
\end{align*}
\end{lemma}

We then quantify the error introduced by applying the random response mechanism in the DPBloomfilter by calculating $\Pr[\wt{z} = \wh{z}]$. Using basic probability rules, we have
\begin{align*}
    \Pr[\wt{z} = \wh{z}] = & ~ \Pr[\wt{z}=1|\wh{z}=1]\Pr[\wh{z}=1] \\
    & ~ +\Pr[\wt{z}=0|\wh{z}=0]\Pr[\wh{z}=0].
\end{align*}

We can compute the following term by using the definition of DPBloomfilter in Algorithm~\ref{alg:init}  (see details in Lemma~\ref{lem:dpbloom_bloom_accuracy:formal})
\begin{align*}
    \Pr[\wt{z}=1|\wh{z}=1] = & ~ (\frac{e^{\epsilon_0}}{e^{\epsilon_0}+1})^k, \\
     \Pr[\wt{z}=0|\wh{z}=0] \geq & ~ \frac{e^{\epsilon_0}}{e^{\epsilon_0}+1}.
\end{align*}
Here we let $\Pr[\wh{z}=0] = \wh{\alpha}$, note that $\wh{\alpha} = \alpha (1 - \delta_{\mathrm{err}})$. Hence, $\Pr[\wh{z} = 1] = 1 - \Pr[\wh{z}=0] = 1 - \alpha + \alpha \cdot \delta_{err}$. Then we will have (see details in Lemma~\ref{lem:dpbloom_bloom_accuracy:formal})
\begin{align*}
    \Pr[\wh{z} = z] \geq t \cdot \alpha \cdot (1 - \delta_{err}).
\end{align*}

\begin{lemma}[Accuracy (compare DPBloomFilter with Bloom) for Query, informal version of Lemma~\ref{lem:dpbloom_bloom_accuracy:formal}]\label{lem:dpbloom_bloom_accuracy:informal}
If Condition~\ref{con:utility_condition} holds, we can show
Then, we can show
\begin{align*}
\Pr[ \wt{z} = \wh{z}] \geq t \cdot \alpha \cdot (1 - \delta_{err}).
\end{align*}
\end{lemma}
Now, we can proceed to examine the utility guarantees of DPBloomfilter by calculating $\Pr[\wt{z} = z]$, i.e., comparing the output of DPBloomfilter with the ground truth for the query. 
By combining the result of the analysis above, we will have (see more details in Theorem~\ref{thm:dpbloom_true_accuracy:formal})
\begin{align*}
    \Pr[ \wt{z} = z ] \geq \alpha \cdot (1-t-t^k)\cdot \delta_{\mathrm{err}}+\alpha\cdot t. 
\end{align*}
Then, we have demonstrated the utility guarantees of our algorithm while simultaneously ensuring privacy (see Theorem~\ref{thm:dpbloom_true_accuracy:informal}).


Similar to other differential privacy algorithms, our algorithm encounters a trade-off between privacy and utility, where increased privacy typically results in a reduction in utility, and conversely. An in-depth examination of this trade-off is provided as follows.

\begin{remark} [Trade-off between Privacy and Utility of DPBloomfilter]
An inherent trade-off exists between the privacy and utility guarantees of our algorithm. To ensure privacy, we must lower the value of $\epsilon_0$ in Theorem~\ref{thm:query_privacy:informal}. On the other hand, for utility considerations (in Theorem~\ref{thm:dpbloom_true_accuracy:informal}), we define the lower bound of $\Pr[\wt{z} = z]$ as $u = \alpha(1-t-t^k)\delta_{\mathrm{err}}+\alpha t$
, a reduction in $\epsilon_0$ will lead to a reduction in $t$ then finally result in a reduction in $u$. This, in turn, leads to diminished utility.
\end{remark}

\subsection{Running Time of DPBloomfilter}\label{sec:tec_time_dp}
In this section, we will analyze the running time of our DPBloomfilter. 
Recall in Definition~\ref{def:bloom_filter}, we let $\mathcal{T}_{h}$ denote the computation time per execution for all hash functions. To analyze the algorithm's running time, firstly, we consider the running time of initialization in Algorithm~\ref{alg:init}. It contains two steps as follows

\textbf{Step 1.} Let's consider the initialization of the standard Bloom Filter. For a single element $x \in A$, it needs $O(k\cdot \mathcal{T}_h)$ time to compute over $k$ hash functions. And $|A|$ elements need to be inserted. Combining these two facts, it needs $|A|\cdot k \cdot \mathcal{T}_h$ time to initialize the standard Bloom Filter.

\textbf{Step 2.} Let's consider the ``Flip each bit'' part in DPBloomfilter. Since there are $m$ bits in the Bloom Filter, it takes $O(m)$ time to flip each bit.

Hence, it takes $O(|A|\cdot k \cdot \mathcal{T}_{h}+m)$ time to run the initialization function in Algorithm~\ref{alg:init}. (see also in Lemma~\ref{lem:init_time})

Then we consider the running time of a single query in Algorithm~\ref{alg:init}. For each query $y$, the algorithm needs $O(k \cdot \mathcal{T}_{h})$ time to compute the hash values of $y$ over $k$ hash functions. Hence, it takes $O(k \cdot \mathcal{T}_{h})$ time to run each query $y$ in. (see also in Lemma~\ref{lem:query_time})

By combining the two running time together, we can obtain the running time of our entire algorithm is $O(|A|\cdot k \cdot \mathcal{T}_{h}+m)$. This highlights the advantage of our algorithm: it matches the time complexity of a standard Bloom Filter while providing a strong privacy guarantee.





\section{Experiments}\label{sec:experiments}


\begin{figure*}[!ht]
\centering
\includegraphics[width=0.32\textwidth]{eps_figs/eps_eps_diff_m_Random.pdf}
\includegraphics[width=0.32\textwidth]{eps_figs/eps_eps_diff_m_False_Negative.pdf}
\includegraphics[width=0.32\textwidth]{eps_figs/eps_eps_diff_m_False_Positive.pdf}
\caption{
Three kinds of error rates with different bit-array lengths $m$. We fix the number of inserted elements $|A|=10^5$, the number of hash functions $k = 3$, and $\delta = 0.01$ in $(\epsilon, \delta)$-DP. 
In the figure, $\log$ denotes $\log_2$. 
{\bf Left:} Total error denotes the case when we randomly choose queries from the universe $[n]$; 
{\bf Middle:} False negative denotes the case when we randomly choose queries from the set $S$, which represents the set of elements inserted into the DP Bloom filter; 
{\bf Right:} False positive denotes the case when we randomly choose queries from the set $\ov{S} = [n] \backslash S$.  
As $m$ increases, the total error rate and false positive error rate decrease accordingly, while false negative error rate remains constant. 
As $\epsilon$ approaches $0$, the DP Bloom filter gets closer to random guessing. In this case, the false positive error rate converges to $\frac{1}{2^k}$, and the false negative error rate converges to $1 - \frac{1}{2^k}$. This is consistent with our result in Lemma~\ref{lem:random_guess}
Our \textsc{DPBloomFilter} achieves practical utility when $\epsilon$ is small(e.g. $\epsilon < 10$).
}
\label{fig:eps_diff_m}
\end{figure*}


\begin{figure*}[!ht]
\centering
\includegraphics[width=0.32\textwidth]{eps_figs/eps_eps_diff_na_Random.pdf}
\includegraphics[width=0.32\textwidth]{eps_figs/eps_eps_diff_na_False_Negative.pdf}
\includegraphics[width=0.32\textwidth]{eps_figs/eps_eps_diff_na_False_Positive.pdf}
\caption{
Three kinds of error rates with different numbers of inserted elements $|A|$. We fix the length of bit-array $m=2^{19}$, the number of hash functions $k = 3$, and $\delta = 0.01$ in $(\epsilon, \delta)$-DP.
As $|A|$ increases, the Total Error Rate and false positive error rate increase accordingly, while the false negative error rate remains constant. 
}
\label{fig:eps_diff_na}
\end{figure*}

\begin{figure*}[!ht]
\centering
\includegraphics[width=0.32\textwidth]{eps_figs/eps_eps_diff_k_Random.pdf}
\includegraphics[width=0.32\textwidth]{eps_figs/eps_eps_diff_k_False_Negative.pdf}
\includegraphics[width=0.32\textwidth]{eps_figs/eps_eps_diff_k_False_Positive.pdf}
\caption{
Three kinds of error rates with different numbers of hash function $k$.  
We fix the length of bit-array $m=2^{19}$, the number of inserted elements $|A| = 10^5$, and $\delta = 0.01$ in $(\epsilon, \delta)$-DP.
As $k$ increases, the Total Error Rate and false positive error rate decrease accordingly, while the false negative error rate increases accordingly. 
}
\label{fig:eps_diff_k}
\end{figure*}

In this section, we introduce the simulation experiments conducted on the DPBloomfilter.
In Section~\ref{sec:exp:setup}, we introduce the basic setup of our experiments and restate basic definitions of three kinds of error.
In Section~\ref{sec:exp:main_result}, we discuss the results of our experiments, which align with our theoretical analysis. 

\subsection{Experiments Setup and Basic Notations} \label{sec:exp:setup}


Recall that we have the following notations. 
Let $m$ denote the length of the bit array in the DPBloomfilter.
Let $|A|$ denote the number of elements inserted into the DPBloomfilter. 
Let $k$ denote the number of hash functions used in the DPBloomfilter.
Let $\epsilon, \delta$ denote the differential privacy parameters of the DPBloomfilter. 
Let $N$ denotes the $1 - \delta$ quantile of $W$ (see Definition~\ref{def:W}), and the close-form of the distribution of $W$ is shown in Lemma~\ref{lem:distribution_of_W}. 
Let $\epsilon_0 = \epsilon / N$. By Theorem~\ref{thm:query_privacy:informal}, we choose $\epsilon_0$ in this way can guarantee to $(\epsilon, \delta)$-DP in the whole algorithm. 
Unless specified, we adopt $m = 2^{19}, |A| = 10^5, k=8, n = 2^{63} \approx 10^{19}$ in the following experiments. 
We choose this $n$ because this $n$ is the biggest integer that can be represented on our server.

Recall that $[n]$ denotes the universe. 
Let $S$ denote the elements inserted into the DPBloomfilter. 
Let $\ov{S} = [n] \backslash S$ denote the elements not inserted into the DPBloomfilter. Let $\wt{z} \in \{ 0, 1 \}$ denote the answer output by DPBloomfilter. 

We report three kinds of error rates in our experiments. They are the following: 
(1) {\bf total error}, where we randomly choose queries from the universe $[n]$ and report the error rate of our DPBloomfilter;
(2) {\bf false positive error}, where we random choose queries from $\ov{S}$. When the DPBloomfilter outputs $\wt{z} = 1$, this will cause a false positive error; 
(3) {\bf false negative error}, where we random choose queries from $S$. When the DPBloomfilter outputs $\wt{z} = 0$, this will cause a false negative error. 

\subsection{Experiment Results} \label{sec:exp:main_result}

In this section, we conduct experiments based on the setting mentioned in the previous section. Specifically, we run simulation experiments on different $m$, $|A|$, and $k$ to demonstrate the utility of our algorithm under differential privacy guarantees. 

In Figure~\ref{fig:eps_diff_m}, we conduct experiments on different $m$, whereas $m$ increases, the total error rate and false positive error rate decrease accordingly, while the false negative error rate remains constant. 

In Figure~\ref{fig:eps_diff_na}, we also conduct experiments on different $|A|$, whereas $|A|$ increases, the total error rate and false positive error rate increase accordingly. At the same time, the false negative error rate remains constant.
This phenomenon is consistent with our theoretical analysis of the utility of DPBloomfilter (Theorem~\ref{thm:dpbloom_true_accuracy:informal}). Recall that we have $\alpha = \Pr[z=0]$, denoting the probability of an arbitrary query $q \notin A$. 
Since $|A|$ increases, $\alpha$ decreases, the utility guarantee in Theorem~\ref{thm:dpbloom_true_accuracy:informal}, which is consistent with higher error rate in our experiment results. 


In Figure~\ref{fig:eps_diff_k}, we conduct experiments on different $k$ as well, whereas $k$ increases, the total error rate, and false positive error rate decrease, while the false negative error rate increases accordingly. 

Note that in Figure~\ref{fig:eps_diff_m}, Figure~\ref{fig:eps_diff_na}, and Figure~\ref{fig:eps_diff_k}, as $\epsilon$ approaches $0$, the DPBloomfilter gets closer to random guessing. In this case, the false positive error rate converges to $\frac{1}{2^k}$, and the false negative error rate converges to $1 - \frac{1}{2^k}$. This is consistent with our result in Lemma~\ref{lem:random_guess}. 
Also, as $\epsilon$ increases, the three types of error rates in the Bloom filter with differential privacy (DP) approach the error rates observed when DP is not applied. This is consistent with the intuition that when $\epsilon$ increases, there is less privacy. Therefore, the performance approaches the performance of a Bloom filter without any privacy guarantees. 


This paper presents a planning approach for effective and efficient joint motion generation for manipulators to cover a surface, aiming to minimize specific joint space costs.

\textit{Limitations} -- Our work has several limitations that suggest potential directions for future research. First, our method uses a heuristic to accelerate the traditional Joint-GTSP approach. While we provide empirical evidence of its efficiency in producing high-quality solutions, we cannot guarantee consistent performance in all scenarios.
Second, our bi-level hierarchical method reduces the size of GTSP. Future research could extend it to multiple levels to further improve performance, though this may produce misleading guide paths.
Third, we observe that both Joint-GTSP and H-Joint-GTSP tend to generate paths with frequent turns, a pattern also observed in the motions of prior work \cite{kaljaca2020coverage, zhang2024jpmdp}.  Future work should explore strategies to balance joint movements with other objectives such as motion smoothness.

\footnotetext{Visualization tool: \url{https://github.com/uwgraphics/MotionComparator}}
\textit{Implications} -- The hierarchical approach presented in this work enables effective and efficient coverage path planning for robot manipulators. 
This approach is beneficial to applications that require dexterous surface coverage, such as sanding, polishing, wiping, and sensor scanning. 


\section{Summary and Conclusion}
\label{sec:conclusion}


In this paper, we introduced \ToolName{}, a method for discovering fine-grained \emph{sub-activities} from unlabeled smart home sensor data without relying on pre-segmentation. Our pipeline is organized into two core steps: Clustering and Labeling. 
The \textbf{Clustering step} consists of:

\begin{itemize}
    \item \textbf{Encoder Pre-Training:} We leverage a pre-trained BERT model adapted with sensor-specific tokens and train it using a masked language modeling (MLM) objective to generate context-rich embeddings for raw sensor sequences.
    
    \item \textbf{Clustering Model Fine-Tuning:} Using the SCAN loss function, we fine-tune these embeddings to form more homogeneous and distinct clusters of sensor sequences.
\end{itemize}

The \textbf{Labeling step} comprises:

\begin{itemize}
    \item \textbf{Cluster Centroid Annotation:} Representative sequences from each cluster are visualized with a custom tool, enabling expert annotators to assign meaningful sub-activity labels to the centroids.
    
    \item \textbf{Label Propagation:} The centroid labels are propagated to all sequences within their respective clusters, resulting in a fully labeled dataset with minimal manual effort.
    
    \item \textbf{Re-annotation of Original Time-Series Data:} 
    Finally, these propagated labels are mapped back onto the original time-series data, preserving temporal continuity and facilitating the analysis of longitudinal activity patterns.
\end{itemize}


Our approach addresses important challenges in HAR, including the high cost and effort of manual data annotation, the limitations of coarse activity labels, and the need for scalable and generalizable models. \ToolName{} offers an open source tool that facilitates the HAR annotation and re-annotation process and enables the dynamic discovery and validation of sub-activities, thus capturing a broader spectrum of behaviors observed in real homes.
% \section*{Ethical Considerations}

In the era of big data, many carefully designed data structures are used in various scenarios, and they usually contain a large amount of sensitive user privacy information. 
Some malicious attackers will restore the user information contained in the published data structure, causing a bad social impact. 
Based on this problem, this work uses the Bloom filter data structure as an example to explore the possibility of protecting sensitive information in the Bloom filter through the properties of differential privacy.

Since these data structures often contain subtle structures, naively applying classical differential mechanisms like Gaussian or Laplace mechanisms on them will have a greater impact on the utility of the data structure.
Therefore, designing differential privacy on such data structures requires more effort and exploration. Our work is just the first step in this direction, and we have made preliminary explorations in protecting sensitive data in data structures in the context of big data.




\ifdefined\isarxiv
\section*{Acknowledgement}
Research is partially supported by the National Science Foundation (NSF) Grants 2023239-DMS, CCF-2046710, and Air Force Grant FA9550-18-1-0166.
\else

% \bibliographystyle{ACM-Reference-Format}

% \bibliographystyle{alpha}
\bibliographystyle{plain}
% \bibliographystyle{plainnat}

\bibliography{ref}
% \bibliographystyle{plain}
\fi

\newpage
\onecolumn
% Zhizhou: if added the \appendix command, the all the \section will disappear in Appendix, only left \subsection. So I comment \appendix off. 
% \appendix

% \begin{center}
%     \textbf{\LARGE Appendix }
% \end{center}

%%%% Cut-line between first 10 pages and appendix





%%% some writing rules

%% Writing rule for creating tags.
%% Tags :
%% Theorem    \ref{thm:bla_bla}
%% Lemma      \ref{lem:bla_bla}
%% Claim      \ref{cla:bla_bla}
%% Corollary  \ref{cor:bla_bla}
%% Fact       \ref{fac:bla_bla}
%% Definition \ref{def:bla_bla}
%% Section    \ref{sec:bla_bla}
%% Subsection \ref{sub:bla_bla}
%% Equation   \ref{eq:bla_bla}

\ifdefined\isarxiv
%\section*{Acknowledgments}
\bibliographystyle{alpha}
\bibliography{ref}
\fi

\end{document}



%%%%%%%%%%%%%%%%%%%%%%%%%%%%%%%%%%%%%%%%%%%%%%%%%%%%%%%%%%%%%%%%%%%%%%%%%%%%%%%%%%%%%%%%%%%%%%%%%%%%%%%%%%%%%%%%%%%%%%%%%%%%%%%%%%%%%%%%%%%%%%%%%%%%%%%%%%%%%%%%%%%%%%%%%%%%%%%%%%%%%%%%%%%%%%%%%%%%%%%%%%%%%%%%%%%%%%%%%%%%%%%%%%%%%%%%%%%%%%%%%%%%%%%%%%%%%%%%%%%%%%%%%%%%%%%%%%%%%%%%%%%%%%%%%%%%%%%%%%%%%%%%%%%%%%%%%%%%%%%%%%%%%%%%%%%%%%%%%%%%%%%%%%%%%%%%%%%%%%%%%%%%%%%%%%%%%%%%%%%%%%%%%%%%%%%%%%%%%%%%%%%%%%%%%%%%%%%%%%%%%%%%%%%%%%%%%%%%%%%%%%%%%%%%%%%%%%%%%%%%%%
