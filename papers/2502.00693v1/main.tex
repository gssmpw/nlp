\def\isarxiv{1} %%% for icml submission version, we comment this line

\ifdefined\isarxiv
\documentclass[11pt]{article}

\usepackage[numbers]{natbib}

\else

\documentclass[conference,compsoc]{IEEEtran}

\fi

\usepackage{amsmath}
\usepackage{amsthm}
\usepackage{amssymb}
\usepackage{algorithm}
\usepackage{subfig}
\usepackage{algpseudocode}
\usepackage{graphicx}
\usepackage{grffile}
\usepackage{wrapfig,epsfig}
\usepackage{url}
\usepackage{xcolor}
\usepackage{epstopdf}
\usepackage{bbm}
\usepackage{dsfont}
\usepackage{gradient-text} % Zhizhou: I add this packge for the fancy color in the paper title.

 %%% print refs in table of contents
%\displaybreak
\allowdisplaybreaks

%\usepackage[lmargin=1in,rmargin=1in,tmargin=0.8in,bmargin=0.8in]{geometry}

\ifdefined\isarxiv

\let\C\relax
\usepackage{tikz}
\usepackage{hyperref}  %%% arxiv don't allow this.
\hypersetup{colorlinks=true,citecolor=blue,linkcolor=blue} %%% Zhao : maybe we should comment this in submission.
\usetikzlibrary{arrows}
\usepackage[margin=1in]{geometry}

\else
%\usepackage[pagebackref,breaklinks,colorlinks]{hyperref}


% Support for easy cross-referencing
%\usepackage[capitalize]{cleveref}
%\usepackage{microtype}
\usepackage{hyperref}
\definecolor{mydarkblue}{rgb}{0,0.08,0.45}
\hypersetup{colorlinks=true, citecolor=mydarkblue,linkcolor=mydarkblue}
%\usepackage[capitalize,noabbrev]{cleveref}
%\usepackage{colortbl}

\fi
%\linespread{1}
%\newcommand{\QED}{\hfill$\qed$}
\graphicspath{{./figs/}}

%\theoremstyle{plain} %%%Zhao: let me comment it out.
\newtheorem{theorem}{Theorem}[section]
\newtheorem{lemma}[theorem]{Lemma}
\newtheorem{definition}[theorem]{Definition}
\newtheorem{notation}[theorem]{Notation}
%\newtheorem{proof}[theorem]{Proof}
\newtheorem{proposition}[theorem]{Proposition}
\newtheorem{corollary}[theorem]{Corollary}
\newtheorem{conjecture}[theorem]{Conjecture}
\newtheorem{assumption}[theorem]{Assumption}
\newtheorem{observation}[theorem]{Observation}
\newtheorem{fact}[theorem]{Fact}
\newtheorem{remark}[theorem]{Remark}
\newtheorem{claim}[theorem]{Claim}
\newtheorem{example}[theorem]{Example}
\newtheorem{problem}[theorem]{Problem}
\newtheorem{open}[theorem]{Open Problem}
\newtheorem{property}[theorem]{Property}
\newtheorem{hypothesis}[theorem]{Hypothesis}
\newtheorem{condition}[theorem]{Condition}

\newcommand{\wh}{\widehat}
\newcommand{\wt}{\widetilde}
\newcommand{\ov}{\overline}
\newcommand{\N}{\mathcal{N}}
\newcommand{\R}{\mathbb{R}}
\newcommand{\RHS}{\mathrm{RHS}}
\newcommand{\LHS}{\mathrm{LHS}}
\renewcommand{\d}{\mathrm{d}}
\renewcommand{\i}{\mathbf{i}}
\renewcommand{\tilde}{\wt}
\renewcommand{\hat}{\wh}
\newcommand{\Tmat}{{\cal T}_{\mathrm{mat}}}

\newcommand{\ac}{ac} %%% Zhao: use this for $\delta_{\ac}$ , which is accuracy
\newcommand{\pa}{pa} %%% Zhao: use this for $\delta_{\pa}$ , which is privacy

\DeclareMathOperator*{\E}{{\mathbb{E}}}
\DeclareMathOperator*{\var}{\mathrm{Var}}
\DeclareMathOperator*{\Z}{\mathbb{Z}}
\DeclareMathOperator*{\C}{\mathbb{C}}
\DeclareMathOperator*{\D}{\mathcal{D}}
\DeclareMathOperator*{\median}{median}
\DeclareMathOperator*{\mean}{mean}
\DeclareMathOperator{\OPT}{OPT}
\DeclareMathOperator{\supp}{supp}
\DeclareMathOperator{\poly}{poly}

\DeclareMathOperator{\nnz}{nnz}
\DeclareMathOperator{\sparsity}{sparsity}
\DeclareMathOperator{\rank}{rank}
\DeclareMathOperator{\diag}{diag}
\DeclareMathOperator{\dist}{dist}
\DeclareMathOperator{\cost}{cost}
\DeclareMathOperator{\vect}{vec}
\DeclareMathOperator{\tr}{tr}
\DeclareMathOperator{\dis}{dis}
\DeclareMathOperator{\cts}{cts}



\makeatletter
\newcommand*{\RN}[1]{\expandafter\@slowromancap\romannumeral #1@}
\makeatother
\newcommand{\Zhao}[1]{{\color{red}[Zhao: #1]}}
\newcommand{\Zhenmei}[1]{{\color{orange}[Zhenmei: #1]}}
\newcommand{\Zhizhou}[1]{{\color{blue}[Zhizhou: #1]}} 
\newcommand{\Yekun}[1]{{\color{purple}[Yekun: #1]}} 
% \newcommand{\InernName}[1]{{\color{blue}[InternName: #1]}} %%%Change to intern name


\usepackage{lineno}
\def\linenumberfont{\normalfont\small}

\ifdefined\isarxiv
\else
% Zhizhou: This line is added for S&P submission

% correct bad hyphenation here
\hyphenation{op-tical net-works semi-conduc-tor}
\fi

\begin{document}

\ifdefined\isarxiv

\date{}


% \title{DPBloomfilter: Securing Bloom Filters with Differential Privacy}
\title{\includegraphics[scale=0.045]{filter.png} \textbf{\textsc{\gradientRGB{DPBloomfilter}{11,47,159}{152,222,217}}}: Securing Bloom Filters with Differential Privacy}

\author{
Yekun Ke\thanks{\texttt{
keyekun0628@gmail.com}. Independent Researcher.}
\and
Yingyu Liang\thanks{\texttt{
yingyul@hku.hk}. The University of Hong Kong. \texttt{
yliang@cs.wisc.edu}. University of Wisconsin-Madison.} 
\and
Zhizhou Sha\thanks{\texttt{ shazz20@mails.tsinghua.edu.cn}. Tsinghua University.}
\and
Zhenmei Shi\thanks{\texttt{
zhmeishi@cs.wisc.edu}. University of Wisconsin-Madison.}
\and 
Zhao Song\thanks{\texttt{ magic.linuxkde@gmail.com}. The Simons Institute for the Theory of Computing at UC Berkeley.}
}
\else

% \title{DPBloomfilter: Securing Bloom Filters with Differential Privacy}
\title{\includegraphics[scale=0.035]{filter.png} \textbf{\textsc{\gradientRGB{DPBloomfilter}{11,47,159}{152,222,217}}}: Securing Bloom Filters with Differential Privacy}

% % author names and affiliations
% % use a multiple column layout for up to three different
% % affiliations
% \author{\IEEEauthorblockN{Michael Shell}
% \IEEEauthorblockA{School of Electrical and\\Computer Engineering\\
% Georgia Institute of Technology\\
% Atlanta, Georgia 30332--0250\\
% Email: http://www.michaelshell.org/contact.html}
% \and
% \IEEEauthorblockN{Homer Simpson}
% \IEEEauthorblockA{Twentieth Century Fox\\
% Springfield, USA\\
% Email: homer@thesimpsons.com}
% \and
% \IEEEauthorblockN{James Kirk\\ and Montgomery Scott}
% \IEEEauthorblockA{Starfleet Academy\\
% San Francisco, California 96678-2391\\
% Telephone: (800) 555--1212\\
% Fax: (888) 555--1212}}


% Zhizhou: I am not sure how to anonymize the author in IEEEtran.cls So I adopt the answer provided in this link https://tex.stackexchange.com/questions/676097/remove-author-name-in-latex-with-ieeetran-template
\author{\IEEEauthorblockN{Anonymous Authors}}

% \author{
% Yekun Ke\thanks{\texttt{
% keyekun0628@gmail.com}.}
% \and
% Yingyu Liang\thanks{\texttt{
% yingyul@hku.hk}. The University of Hong Kong. \texttt{
% yliang@cs.wisc.edu}. University of Wisconsin-Madison.} 
% \and
% Zhizhou Sha\thanks{\texttt{ shazz20@mails.tsinghua.edu.cn}. Tsinghua University.}
% \and
% Zhenmei Shi\thanks{\texttt{
% zhmeishi@cs.wisc.edu}. University of Wisconsin-Madison.}
% \and 
% Zhao Song\thanks{\texttt{ zsong@adobe.com}. Adobe Research.}
% }


\fi


\ifdefined\isarxiv
\begin{titlepage}
  \maketitle
  \begin{abstract}
  \begin{abstract}


The choice of representation for geographic location significantly impacts the accuracy of models for a broad range of geospatial tasks, including fine-grained species classification, population density estimation, and biome classification. Recent works like SatCLIP and GeoCLIP learn such representations by contrastively aligning geolocation with co-located images. While these methods work exceptionally well, in this paper, we posit that the current training strategies fail to fully capture the important visual features. We provide an information theoretic perspective on why the resulting embeddings from these methods discard crucial visual information that is important for many downstream tasks. To solve this problem, we propose a novel retrieval-augmented strategy called RANGE. We build our method on the intuition that the visual features of a location can be estimated by combining the visual features from multiple similar-looking locations. We evaluate our method across a wide variety of tasks. Our results show that RANGE outperforms the existing state-of-the-art models with significant margins in most tasks. We show gains of up to 13.1\% on classification tasks and 0.145 $R^2$ on regression tasks. All our code and models will be made available at: \href{https://github.com/mvrl/RANGE}{https://github.com/mvrl/RANGE}.

\end{abstract}


  \end{abstract}
  \thispagestyle{empty}
\end{titlepage}

{\hypersetup{linkcolor=black}
\tableofcontents
}
\newpage

\else

\maketitle 

\begin{abstract}
\begin{abstract}


The choice of representation for geographic location significantly impacts the accuracy of models for a broad range of geospatial tasks, including fine-grained species classification, population density estimation, and biome classification. Recent works like SatCLIP and GeoCLIP learn such representations by contrastively aligning geolocation with co-located images. While these methods work exceptionally well, in this paper, we posit that the current training strategies fail to fully capture the important visual features. We provide an information theoretic perspective on why the resulting embeddings from these methods discard crucial visual information that is important for many downstream tasks. To solve this problem, we propose a novel retrieval-augmented strategy called RANGE. We build our method on the intuition that the visual features of a location can be estimated by combining the visual features from multiple similar-looking locations. We evaluate our method across a wide variety of tasks. Our results show that RANGE outperforms the existing state-of-the-art models with significant margins in most tasks. We show gains of up to 13.1\% on classification tasks and 0.145 $R^2$ on regression tasks. All our code and models will be made available at: \href{https://github.com/mvrl/RANGE}{https://github.com/mvrl/RANGE}.

\end{abstract}


\end{abstract}

% For peerreview papers, this IEEEtran command inserts a page break and
% creates the second title. It will be ignored for other modes.
\IEEEpeerreviewmaketitle



\fi

\section{Introduction}
Backdoor attacks pose a concealed yet profound security risk to machine learning (ML) models, for which the adversaries can inject a stealth backdoor into the model during training, enabling them to illicitly control the model's output upon encountering predefined inputs. These attacks can even occur without the knowledge of developers or end-users, thereby undermining the trust in ML systems. As ML becomes more deeply embedded in critical sectors like finance, healthcare, and autonomous driving \citep{he2016deep, liu2020computing, tournier2019mrtrix3, adjabi2020past}, the potential damage from backdoor attacks grows, underscoring the emergency for developing robust defense mechanisms against backdoor attacks.

To address the threat of backdoor attacks, researchers have developed a variety of strategies \cite{liu2018fine,wu2021adversarial,wang2019neural,zeng2022adversarial,zhu2023neural,Zhu_2023_ICCV, wei2024shared,wei2024d3}, aimed at purifying backdoors within victim models. These methods are designed to integrate with current deployment workflows seamlessly and have demonstrated significant success in mitigating the effects of backdoor triggers \cite{wubackdoorbench, wu2023defenses, wu2024backdoorbench,dunnett2024countering}.  However, most state-of-the-art (SOTA) backdoor purification methods operate under the assumption that a small clean dataset, often referred to as \textbf{auxiliary dataset}, is available for purification. Such an assumption poses practical challenges, especially in scenarios where data is scarce. To tackle this challenge, efforts have been made to reduce the size of the required auxiliary dataset~\cite{chai2022oneshot,li2023reconstructive, Zhu_2023_ICCV} and even explore dataset-free purification techniques~\cite{zheng2022data,hong2023revisiting,lin2024fusing}. Although these approaches offer some improvements, recent evaluations \cite{dunnett2024countering, wu2024backdoorbench} continue to highlight the importance of sufficient auxiliary data for achieving robust defenses against backdoor attacks.

While significant progress has been made in reducing the size of auxiliary datasets, an equally critical yet underexplored question remains: \emph{how does the nature of the auxiliary dataset affect purification effectiveness?} In  real-world  applications, auxiliary datasets can vary widely, encompassing in-distribution data, synthetic data, or external data from different sources. Understanding how each type of auxiliary dataset influences the purification effectiveness is vital for selecting or constructing the most suitable auxiliary dataset and the corresponding technique. For instance, when multiple datasets are available, understanding how different datasets contribute to purification can guide defenders in selecting or crafting the most appropriate dataset. Conversely, when only limited auxiliary data is accessible, knowing which purification technique works best under those constraints is critical. Therefore, there is an urgent need for a thorough investigation into the impact of auxiliary datasets on purification effectiveness to guide defenders in  enhancing the security of ML systems. 

In this paper, we systematically investigate the critical role of auxiliary datasets in backdoor purification, aiming to bridge the gap between idealized and practical purification scenarios.  Specifically, we first construct a diverse set of auxiliary datasets to emulate real-world conditions, as summarized in Table~\ref{overall}. These datasets include in-distribution data, synthetic data, and external data from other sources. Through an evaluation of SOTA backdoor purification methods across these datasets, we uncover several critical insights: \textbf{1)} In-distribution datasets, particularly those carefully filtered from the original training data of the victim model, effectively preserve the model’s utility for its intended tasks but may fall short in eliminating backdoors. \textbf{2)} Incorporating OOD datasets can help the model forget backdoors but also bring the risk of forgetting critical learned knowledge, significantly degrading its overall performance. Building on these findings, we propose Guided Input Calibration (GIC), a novel technique that enhances backdoor purification by adaptively transforming auxiliary data to better align with the victim model’s learned representations. By leveraging the victim model itself to guide this transformation, GIC optimizes the purification process, striking a balance between preserving model utility and mitigating backdoor threats. Extensive experiments demonstrate that GIC significantly improves the effectiveness of backdoor purification across diverse auxiliary datasets, providing a practical and robust defense solution.

Our main contributions are threefold:
\textbf{1) Impact analysis of auxiliary datasets:} We take the \textbf{first step}  in systematically investigating how different types of auxiliary datasets influence backdoor purification effectiveness. Our findings provide novel insights and serve as a foundation for future research on optimizing dataset selection and construction for enhanced backdoor defense.
%
\textbf{2) Compilation and evaluation of diverse auxiliary datasets:}  We have compiled and rigorously evaluated a diverse set of auxiliary datasets using SOTA purification methods, making our datasets and code publicly available to facilitate and support future research on practical backdoor defense strategies.
%
\textbf{3) Introduction of GIC:} We introduce GIC, the \textbf{first} dedicated solution designed to align auxiliary datasets with the model’s learned representations, significantly enhancing backdoor mitigation across various dataset types. Our approach sets a new benchmark for practical and effective backdoor defense.



\section{Related Work}

\subsection{Large 3D Reconstruction Models}
Recently, generalized feed-forward models for 3D reconstruction from sparse input views have garnered considerable attention due to their applicability in heavily under-constrained scenarios. The Large Reconstruction Model (LRM)~\cite{hong2023lrm} uses a transformer-based encoder-decoder pipeline to infer a NeRF reconstruction from just a single image. Newer iterations have shifted the focus towards generating 3D Gaussian representations from four input images~\cite{tang2025lgm, xu2024grm, zhang2025gslrm, charatan2024pixelsplat, chen2025mvsplat, liu2025mvsgaussian}, showing remarkable novel view synthesis results. The paradigm of transformer-based sparse 3D reconstruction has also successfully been applied to lifting monocular videos to 4D~\cite{ren2024l4gm}. \\
Yet, none of the existing works in the domain have studied the use-case of inferring \textit{animatable} 3D representations from sparse input images, which is the focus of our work. To this end, we build on top of the Large Gaussian Reconstruction Model (GRM)~\cite{xu2024grm}.

\subsection{3D-aware Portrait Animation}
A different line of work focuses on animating portraits in a 3D-aware manner.
MegaPortraits~\cite{drobyshev2022megaportraits} builds a 3D Volume given a source and driving image, and renders the animated source actor via orthographic projection with subsequent 2D neural rendering.
3D morphable models (3DMMs)~\cite{blanz19993dmm} are extensively used to obtain more interpretable control over the portrait animation. For example, StyleRig~\cite{tewari2020stylerig} demonstrates how a 3DMM can be used to control the data generated from a pre-trained StyleGAN~\cite{karras2019stylegan} network. ROME~\cite{khakhulin2022rome} predicts vertex offsets and texture of a FLAME~\cite{li2017flame} mesh from the input image.
A TriPlane representation is inferred and animated via FLAME~\cite{li2017flame} in multiple methods like Portrait4D~\cite{deng2024portrait4d}, Portrait4D-v2~\cite{deng2024portrait4dv2}, and GPAvatar~\cite{chu2024gpavatar}.
Others, such as VOODOO 3D~\cite{tran2024voodoo3d} and VOODOO XP~\cite{tran2024voodooxp}, learn their own expression encoder to drive the source person in a more detailed manner. \\
All of the aforementioned methods require nothing more than a single image of a person to animate it. This allows them to train on large monocular video datasets to infer a very generic motion prior that even translates to paintings or cartoon characters. However, due to their task formulation, these methods mostly focus on image synthesis from a frontal camera, often trading 3D consistency for better image quality by using 2D screen-space neural renderers. In contrast, our work aims to produce a truthful and complete 3D avatar representation from the input images that can be viewed from any angle.  

\subsection{Photo-realistic 3D Face Models}
The increasing availability of large-scale multi-view face datasets~\cite{kirschstein2023nersemble, ava256, pan2024renderme360, yang2020facescape} has enabled building photo-realistic 3D face models that learn a detailed prior over both geometry and appearance of human faces. HeadNeRF~\cite{hong2022headnerf} conditions a Neural Radiance Field (NeRF)~\cite{mildenhall2021nerf} on identity, expression, albedo, and illumination codes. VRMM~\cite{yang2024vrmm} builds a high-quality and relightable 3D face model using volumetric primitives~\cite{lombardi2021mvp}. One2Avatar~\cite{yu2024one2avatar} extends a 3DMM by anchoring a radiance field to its surface. More recently, GPHM~\cite{xu2025gphm} and HeadGAP~\cite{zheng2024headgap} have adopted 3D Gaussians to build a photo-realistic 3D face model. \\
Photo-realistic 3D face models learn a powerful prior over human facial appearance and geometry, which can be fitted to a single or multiple images of a person, effectively inferring a 3D head avatar. However, the fitting procedure itself is non-trivial and often requires expensive test-time optimization, impeding casual use-cases on consumer-grade devices. While this limitation may be circumvented by learning a generalized encoder that maps images into the 3D face model's latent space, another fundamental limitation remains. Even with more multi-view face datasets being published, the number of available training subjects rarely exceeds the thousands, making it hard to truly learn the full distibution of human facial appearance. Instead, our approach avoids generalizing over the identity axis by conditioning on some images of a person, and only generalizes over the expression axis for which plenty of data is available. 

A similar motivation has inspired recent work on codec avatars where a generalized network infers an animatable 3D representation given a registered mesh of a person~\cite{cao2022authentic, li2024uravatar}.
The resulting avatars exhibit excellent quality at the cost of several minutes of video capture per subject and expensive test-time optimization.
For example, URAvatar~\cite{li2024uravatar} finetunes their network on the given video recording for 3 hours on 8 A100 GPUs, making inference on consumer-grade devices impossible. In contrast, our approach directly regresses the final 3D head avatar from just four input images without the need for expensive test-time fine-tuning.



\section{Brief Review of 3D Gaussian Splatting}
\label{sec:prelim}
For the sake of clarity, we first briefly review 3D Gaussian Splatting (3DGS)~\cite{kerbl202333dgs}, an explicit representation of a 3D scene for providing effective image rendering. 
% We also provide brief reviews of two powerful extensions of 3DGS, Gaussian Grouping~\cite{ye2023gaussiangrouping} and Relightable Gaussian~\cite{gao2023relightable}, which equip 3DGS with segmentation and relighting abilities and are utilized together with 3DGS as the backbone representation in our work. 


Given $K$ multi-view images $I_{1:K} = \{I_1, I_2, ..., I_K\}$ with corresponding camera poses $\xi_{1:K} = \{\xi_1, \xi_2, ..., \xi_K\}$ of a 3D scene, a scene-specific 3DGS is applied to model the scene with $N$ learnable 3D Gaussian ellipsoids (i.e., $G_{1:N} = \{G_1, G_2, ..., G_N \}$). Each Gaussian $G_i$ is parameterized with its 3-dimensional centroid $\mathbf{p}_i \in \mathbb{R}^{3}$, a 3-dimensional standard deviation $\mathbf{s}_i \in \mathbb{R}^{3}$, a 4-dimensional rotational quaternion $\mathbf{q}_i \in \mathbb{R}^{4}$, an opacity ${\alpha}_i \in [0,1]$, and color coefficients $\mathbf{c}_i$ for spherical harmonics in degree of 3. Hence, $G_i$ is represented with a set of the above parameters (i.e., $G_i = \{\mathbf{p}_i, \mathbf{s}_i, \mathbf{q}_i, {\alpha}_i, \mathbf{c}_i\}$). To model the scene with $G_{1:N}$, 2D images $\hat{I_{1:K}} = \{\hat{I_1}, \hat{I_2}, ..., \hat{I_K}\}$ are sequentially rendered from $G_{1:N}$ using $\xi_{1:K} = \{\xi_1, \xi_2, ..., \xi_K\}$ (please refer to~\cite{kerbl202333dgs} for a detailed rendering process), and supervised with $I_{1:K}$ by the rendering loss:
\begin{equation} \label{Limage}
    \mathcal{L}_{image} = \sum_{k\in {1...K}}\lambda\| I_k - {\hat{I_k}}\|_1 + \mathcal{L}_{SSIM}(I_k, \hat{I_k}),
\end{equation}
where $\mathcal{L}_{SSIM}(\cdot)$ represents a SSIM loss and $\lambda$ is a hyper-parameter (set to $0.2$ as mentioned in~\cite{kerbl202333dgs}).



% \subsection{Gaussian Grouping}
% To overcome the lack of fine-grained scene understanding in 3DGS, Gaussian Grouping~\cite{ye2023gaussiangrouping} extends 3DGS by incorporating segmentation capabilities. Along with $I_{1:K}$, Gaussian Grouping additionally takes the Segment Anything Model (SAM) to produce 2D semantic segmentation masks $S_{1:K} = \{S_1, S_2, ..., S_K\}$ from multiple views as inputs, and an additional 16-dimensional parameter $\mathbf{e}_i \in \mathbb{R}^{16}$ is introduced to represent a 3D Identity Encoding for each Gaussian $G_i$. Therefore, each Gaussian $G_i$ is extended as $G_i = \{\mathbf{p}_i, \mathbf{s}_i, \mathbf{q}_i, {\alpha}_i, \mathbf{c}_i, \mathbf{e}_i\}$. To make sure $G_{1:K}$ learns to segment each object represented by $S_{1:K}$ in the scene, a 2D identity loss $\mathcal{L}_{id}$ is applied by calculating cross-entropy between $\hat{S}_{1:K}$ and $S_{1:K}$, where $\hat{S}_{1:K} = \{\hat{S}_1, \hat{S}_2, ... , S_K\}$ denotes the rendered segmentation maps from $G_{1:K}$. Additionally, to further ensure that the Gaussians having the same identities are grouped together, a 3D regularization loss $\mathcal{L}_{3D}$ is applied to enforce each $G_i$'s k-nearest 3D spatial neighbors to be close in their feature distance of Identity Encodings. Please refer to the original paper~\cite{ye2023gaussiangrouping} for detailed formulations of segmentation map rendering and $\mathcal{L}_{3D}$. The design of Gaussian Grouping ensures that the segmentation results are coherent across multiple views, enabling the automatic generation of binary masks for any queried object in the scene.

% \subsection{Relightable Gaussians}
% Different from Gaussian Grouping, Relightable Gaussians~\cite{gao2023relightable} extends the capabilities of Gaussian Splatting by incorporating Disney-BRDF~\cite{burley2012brdf} decomposition and ray tracing to achieve realistic point cloud relighting. 
% % Unlike traditional Gaussian Splatting, which primarily focuses on appearance and geometry modeling, Relightable Gaussians also aim to model the physical interaction of light with different surfaces in the scene.
% Specifically, for each Gaussian $G_i$, the original color coefficients $\mathbf{c}_i$ is decomposed into a 3-dimensional base color $\mathbf{b}_i \in [0,1]^3$, a 1-dimensional roughness $r \in [0,1]$, and incident light coefficients $\mathbf{l}_i$ for spherical harmonics in degree of 3. Subsequently, the Physical-Based Rendering (PBR) process and a point-based ray tracing are applied to obtain the colored 2D images $\hat{I}^{PBR}_{1:K}$ and supervised by $I_{1:K}$ using the aforementioned $\mathcal{L}_{image}$ in Eqn.~\ref{Limage}. Besides the above extensions on PBR for relighting, Relightable Gaussians also introduces a 3-dimensional normal $\mathbf{n}_i$ for $G_i$ and leverages several techniques, including an unsupervised estimation of a depth map $D_i$ from each input view $\xi_i$, to enhance the geometry accuracy and smoothness. Please refer to the original paper of Relightable Gaussians~\cite{gao2023relightable} for detailed explanations.  

\pj{Our pipeline for 3D Inpainting is built on top of the 3DGS model. Additionally, we incorporate the design of Gaussian Grouping~\cite{ye2023gaussiangrouping} to introduce a 16-dimensional semantic feature $\mathbf{e}_i \in \mathbb{R}^{16}$ for each Gaussian $G_i$, so that the 2D segmentation maps of the Gaussians $G_{1:K}$ is rendered and the object mask for the object to be removed can be directly produced, as mentioned in Sect.~\ref{subsec:3Dinpaint}.
By combining these methods as our backbone, we are able to perform an automatic inpainting mask generation and a reliable depth estimation for depth-guided 3D inpainting. Please refer to our Supplementary material for a more detailed explanation of our backbones.}

% the backbone representation by parameterizing each Gaussian $G_i$ as $G_i = \{\mathbf{p}_i, \mathbf{s}_i, \mathbf{q}_i, {\alpha}_i, \mathbf{c}_i, \mathbf{e}_i, \mathbf{b}_i, r,  \mathbf{l}_i, \mathbf{n}_i\}$. By combining these methods, we are able to perform an automatic inpainting mask generation and a reliable depth estimation for depth-guided 3D inpainting.



\section{Main Results}\label{sec:main_result}


In Section~\ref{sec:mr_privacy}, we will provide the privacy of our algorithm. Then, we will examine the utility implications of our algorithm applying a random response mechanism. 
In Section~\ref{sec:main_result:utility}, we introduce the utility guarantees of our algorithm.
In Section~\ref{sec:main_result:running_complexity}, we demonstrate that DPBloomfilter does not import the running complexity burden to the standard Bloom filter.

\subsection{Privacy for DPBloomfilter}\label{sec:mr_privacy}

Algorithm~\ref{alg:init} illustrates the application of the random response mechanism to the standard Bloom filter, thereby accomplishing differential privacy. In detail, once the Bloom filter is initialized, each bit in the $m$-bit array is independently toggled with a probability of $\frac{1}{\epsilon_0 + 1}$. Our algorithm will ensure that modifications to any element within the dataset are protected to a degree, as the DPBloomfilter maintains the privacy of the altered element. Then, we present the Theorem demonstrating that our algorithm is $(\epsilon,\delta)$-DP.

\begin{theorem}[Privacy for Query, informal version of Theorem~\ref{thm:query_privacy:formal}]\label{thm:query_privacy:informal}
Let $N := F_W^{-1}(1 - \delta)$ and $\epsilon_0 = \epsilon / N$.
Then, we can show,
the output of \textsc{Query} procedure of Algorithm~\ref{alg:init} achieves $(\epsilon, \delta)$-DP. 
\end{theorem}

Theorem~\ref{thm:query_privacy:informal} shows that our DPBloomfilter in Algorithm~\ref{alg:init} is $(\epsilon, \delta)$-DP. 
Our main technique leverages the single-bit random response technique to enhance the privacy properties of the traditional Bloom filter by composition rule (Lemma~\ref{lem:pre_com_lem}). 

\subsection{Utility for DPBloomfilter}\label{sec:main_result:utility}

Despite the introduction of privacy-preserving mechanisms, our algorithm still ensures that the utility of the Bloom Filter remains acceptable. This is achieved through careful calibration of the Random Response technique parameters, balancing the need for privacy with the requirement for accurate set membership queries. 
Here, we present the theorem for the entire utility loss between the output of our algorithm and ground truth.

\begin{theorem}[Accuracy (compare DPBloom with true-answer) for Query, informal version of Theorem~\ref{thm:dpbloom_true_accuracy:formal}]\label{thm:dpbloom_true_accuracy:informal}
If the following conditions hold
\begin{itemize}
    \item Let $z \in \{0,1\}$ denote the true answer for whether $x \in A$. 
    \item Let $\wh{z} \in \{0,1\}$ denote the answer for whether $x \in A$ output by Bloom Filter.
    \item Let $\alpha: = \Pr[ z = 0 ] \in [0,1]$, $t := e^{\epsilon_0} / (e^{\epsilon_0} + 1)$, and  $\delta_{\mathrm{err}} > 0$.
\end{itemize}
Then, we can show 
\begin{align*}
\Pr[ \wt{z} = z ] \geq \delta_{\mathrm{err}}\cdot\alpha\cdot(1-t-t^{k}) + \alpha \cdot t.
\end{align*}
\end{theorem}

Theorem~\ref{thm:dpbloom_true_accuracy:informal} shows that when most queries are not in $A$, the above theorem can ensure that the utility of DPBloomfilter has a good guarantee. Namely, in such cases, answers from DPBloomfilter are correct with high probability. 


\subsection{Running Complexity of DPBloomfilter}\label{sec:main_result:running_complexity}

Now, we introduce the running complexity for the DPBloomfilter in the following theorem. 

\begin{theorem} [Running complexity of DPBloomfilter] \label{thm:running_complexity}
Let $\mathcal{T}_h$ denote the time of evaluation of function $h$ at any point. 
Then, for the DPBloomfilter (Algorithm~\ref{alg:init}) we have
\begin{itemize}
    \item The running complexity for the initialization procedure is $O(|A| \cdot k \cdot \mathcal{T}_h + m)$.
    \item The running complexity $O(k\cdot \mathcal{T}_h)$ for a single query. 
\end{itemize}
\end{theorem}

\begin{proof}
It can be proved by combining Lemma~\ref{lem:init_time} and \ref{lem:query_time}. 
\end{proof}

Our Theorem~\ref{thm:running_complexity} shows that DPBloomfilter not only addresses the critical need to protect the privacy of elements stored with Bloom filter but also ensures that the data structure's utility remains acceptable, with minimal impact on its computational efficiency.
By keeping the running time within the same order of magnitude as the standard Bloom filter, our approach is practical for real-world applications requiring fast and scalable set operations. 



\section{Proof for 
\texorpdfstring{$1 - \delta$}{} Quantile}\label{sec:appendix_quantile_proof}
In this section, we provide the calculation of the probability distribution of random variable $W := \sum_{j=1}^{m} \mathbbm 1\{g[j] \neq g'[j]\}$, which plays an important part in the proof of the privacy guarantee for our algorithm (see Section~\ref{sec:appendix_privacy_guarantees}).
In Section~\ref{sec:definition_quantile}, we present the definition of random variables $W, Y, Z$ used in this section.
In Section~\ref{sec:distribution_Y}, we calculate the probability distribution of $Y$.
In Section~\ref{sec:distribution_Z}, we calculate the probability distribution of $Z$ conditioned on $Y$.
In Section~\ref{sec:distribution_W}, we calculate the probability distribution of $W$.

\subsection{Definition} \label{sec:definition_quantile}
In this section, we present the definitions of random variables which will be used in the section.
\begin{definition}[Definition of $W$]\label{def:W}
Let $W := \sum_{j=1}^{m} \mathbbm 1\{g[j] \neq g'[j]\}$, where $g \in \{0, 1\}^m$ denotes the ground truth values generated by dataset $A$, and $g' \in \{0, 1\}^m$ denotes the ground truth values generated by neighboring dataset $A'$. 
\end{definition}

\begin{definition}[Definition of $Y$]\label{def:Y}
Consider a $x \in [n]$. 

Let $y_1, y_2, \cdots , y_k$ denotes the $k$ hash values generated by the standard Bloom filter (Definition~\ref{def:bloom_filter}). 

We define $Y$ as the set of distinct values among $y_1, y_2, \cdots, y_k$, where $|Y| \in { 1, 2, \cdots, k }$.

\end{definition}

\begin{definition}[Definition of $Z$]\label{def:Z}
Consider two data $x, x' \in [n]$. 

Let $y_1, y_2, \cdots , y_k$ denotes the $k$ hash values generated by $x$, and $y_1', y_2', \cdots , y_k'$ denotes the $k$ hash values generated by $x'$. 

Follow the Definition~\ref{def:Y}, let $Y_x$ denotes the set of distinct values in $y_1, y_2, \cdots , y_k$, and $Y_{x'}$ denotes the set of distinct values in $y_1', y_2', \cdots , y_k'$.

Suppose $|Y_x| = a, |Y_{x'}| = b$, where $a, b \in \{1, 2, \cdots , k \}$

We define $Z$ is the set of distinct values in $Y_x \cup Y_{x'}$, where $|Z| \in \{1, 2, \cdots , 2k \}$

\end{definition}


\subsection{Distribution of \texorpdfstring{$Y$}{}}\label{sec:distribution_Y}
Then we proceed to calculate the probability distribution of $Y$ in this section.
\begin{lemma}[Distribution of $Y$]\label{lem:distribution_of_Y}
If the following conditions hold
\begin{itemize}
    \item Let $y_1, y_2, \cdots , y_k$ be defined in Definition~\ref{def:Y}.
    \item Let $Y$ be defined as Definition~\ref{def:Y}.
\end{itemize}

Then, we can show, for $y = 1, 2, \cdots, k$, 
\begin{align*}
    & ~ \Pr[|Y| = y] \\
    = & ~ \begin{cases}
        1 / m^{k-1},  & y = 1 \\
        \binom{m}{y} \cdot y ^k / m^k
        - \sum_{i=1}^{k-1} \binom{m - i}{y -i} \Pr[Y = i], & y = 2, \cdots , k
    \end{cases}
\end{align*}
\end{lemma}

\begin{proof}

{\bf Step 1.} We consider $Y = 1$ case. 

Without any constraints, there are total $m^k$ situations. This is because each hash value can be freely chosen from $m$ positions, and there are $k$ hash values. Therefore, there are total $m^k$ situations. 

Then, with constraint $Y = 1$, $k$ hash values must be assigned to the same position. The position can be chosen from a total of $m$ positions. Therefore, in this case, there are $m$ situations. 

Combining the above two analysis, we have
\begin{align*}
    \Pr[Y = 1] = & ~ \frac{m}{m^k} \notag \\
    = & ~ \frac{1}{m^{k - 1}}.
\end{align*}

{\bf Step 2.} We consider $Y = 2, \cdots , k$ cases.

Similarly, without any constraints, there are total $m^k$ situations. 

Since we need $Y = y$, we must choose $y$ from different positions in the total $m$ positions. Therefore, we have $\binom{m}{y}$ term.

Note that in each position, we need at least one hash value. We first compute the number of freely assigning $k$ hash values to the $y$ positions. Then we remove the failure cases.  

As there are $y$ positions and $k$ hash values, we have the $y^k$ term for freely assigning $k$ hash values to $y$ positions.

For the failure case, we have $\sum_{i=1}^{k-1} \Pr[Y = i] \cdot \binom{m - i}{y -i}$. The $\binom{m - i}{y -i}$ term is due to repeated counting for each $i \in [k-1]$, where we first fix $i$ positions and then randomly pick the other $y-i$ different positions in the total $m-i$ positions. 

Thus, in all, we have the following formula,
\begin{align*}
    \Pr[Y = y] = \frac{\binom{m}{y} \cdot y ^k}{m^k} - \sum_{i=1}^{k-1} \Pr[Y = i] \cdot \binom{m - i}{y -i}.
\end{align*}

\end{proof}

\subsection{Distribution of \texorpdfstring{$Z$}{} conditioned on \texorpdfstring{$Y$}{}}\label{sec:distribution_Z}
In this section, we calculate the probability distribution of $Z$ condition on $Y$.

\begin{lemma}[Probability of $Z$ conditioned on $Y_x$ and $ Y_{x'}$]\label{lem:distribution_of_Z}
If the following conditions hold
\begin{itemize}
    \item Let $Y_x, Y_{x'}, Z$ be defined as Definition~\ref{def:Z}.
    \item Let $A_n^m$ denotes $n! / (n-m)!$.
    \item Let $t := z - \max(a, b)$. 
\end{itemize}

Then, we can show, for $z = \max(a, b), \cdots, (a + b)$, 
\begin{align*}
    \Pr[|Z| = z | |Y_x| = a, |Y_{x'}| = b] = \frac{A_m^a \cdot \binom{b}{t} \cdot A_{m - a}^t \cdot A_{a}^{b-t}}{A_m^a \cdot A_m^b}.
\end{align*}
\end{lemma}

\begin{proof}

Since the minimum value of $Z$ is $\max(a, b)$, without loss of generality, we assume $ a \geq b$. Then we have $a \leq z \leq (a + b)$.

Recall we have $t = z - \max(a, b) = z - a, t \in \{0, 1, \cdots , b\}$. Then we have
\begin{align*}
    & ~ \Pr[|Z| = a + t | |Y_x| = a, |Y_{x'}| = b] \\
    = & ~ \frac{A_m^a \cdot \binom{b}{t} \cdot A_{m - a}^t \cdot A_{a}^{b-t}}{A_m^a \cdot A_m^b}.
\end{align*}

We explain why we have the above equation in the following steps.

{\bf Step 1.} We consider the denominator. 

Without any constraints, since $|Y_x| = a$, we need to choose $a$ from different positions in the total $m$ positions. Therefore, we have the $A_m^a$ term in the denominator. Similarly, since $|Y_{x'}| = b$, we have the $A_m^b$ term in the denominator. 

{\bf Step 2.} We consider the numerator. 

Firstly, since $|Y_x| = a$, we need to choose $a$ different positions in total $m$ positions. Therefore, we have the $A_m^a$ term in the numerator. 

Since $Z$ is defined as Definition~\ref{def:Z}, we can have the following
\begin{align*}
    |Y_x \cap Y_{x'}| = & ~ a + b - z \notag \\
    |Y_{x'}| - |Y_x \cap Y_{x'}| = & ~ z - a \notag \\
    = & ~ t
\end{align*}

Then, we need to choose $t$ values from $Y_{x'}$ to construct $|Y_{x'}| - |Y_x \cap Y_{x'}|$ part. Therefore, we have the $\binom{b}{t}$ term in the numerator. 

We also need to choose $t$ different positions in the rest $m - a$ positions for  $|Y_{x'}| - |Y_x \cap Y_{x'}|$ part. Hence, we have the $A_{m - a} ^ t$ term in the numerator. 

Lastly, let's consider the $b - t$ part. For this part, we need to choose $b - t$ different positions from $a$ positions. Therefore, we have the $A_a^{b - t}$ term in the numerator. 

Combining all analyses together, finally, we have 
\begin{align*}
   \Pr[|Z| = z | |Y_x| = a, |Y_{x'}| = b] = \frac{A_m^a \cdot \binom{b}{t} \cdot A_{m - a}^t \cdot A_{a}^{b-t}}{A_m^a \cdot A_m^b}.
\end{align*}

\end{proof}

\subsection{Distribution of \texorpdfstring{$W$}{}}\label{sec:distribution_W}


\begin{figure}[!ht]
\centering
\includegraphics[width=0.45\textwidth]{w_figs/w_pmf.pdf}
\caption{
Let $W := |S|$ denote the number of bits in the Bloom filter changed by substituting an element in the inserted set $A$ (Definition~\ref{def:pre_neighbor_dataset}). We achieve $\epsilon_0$-DP for each single bit and $(\epsilon, \delta)$-DP for the entire Bloom filter via the random response (Definition~\ref{def:random_response}), where $\epsilon_0 = \epsilon / N$. 
The $N$ is $1 - \delta$ quantile of the random variable $W$. 
We visualize the distribution of the random variable $W$ (see Lemma~\ref{lem:distribution_of_W}) under the setting described in the experiments section (Section~\ref{sec:experiments}). Namely, we have the bit array length in the Bloom filter $m = 2^{19}$, the number of elements inserted into the Bloom filter $|A| = 10^{5}$, and the number of hash functions $k=3$. It can be inferred from this visualization that the values of random variable $W$ have good concentration properties, mostly concentrated around its mean. 
}
\label{fig:w_distribution}
\end{figure}

Finally, we present the calculation of the probability distribution of $W$ in this section.
\begin{lemma}[Distribution of $W$]\label{lem:distribution_of_W}
If the following conditions hold
\begin{itemize}
    \item Let $Y_x, Y_{x'}, Z$ be defined as Definition~\ref{def:Z}.
    \item Let $W$ be defined as Definition~\ref{def:W}.
    \item Let $A_n^m$ denotes $\frac{n!}{(n - m)!}$. 
    \item Let $p_0 := (1 - \frac{1}{m})^{(|A| - 1)k}$ denotes the proportion of bits which are still $0$ in the bit-array.
    \item Let $n_1 := |Y_x \cap Y_{x'}|= a + b - z$ denotes the number of overlap elements in $Y_x$ and $Y_{x'}$. 
    \item Let $n_2 := |Y_x \cup Y_{x'}| - |Y_x \cap Y_{x'}| =  z  -(a + b - z) = 2z -a -b$ denotes the number of exclusive or elements in $Y_x$ and $Y_{x'}$.
\end{itemize}

Then, we can show, for $w=0, \cdots 2k$,
\begin{align*}
    & ~ \Pr[W = w] \notag \\
    = & ~ \sum_{a = 1}^k \sum_{b = 1}^k \sum_{z = 1}^{a+b} \Pr[W = w | |Z| = z, |Y_x| = a, |Y_{x'}| = b] \notag \\
    & ~ \cdot \Pr[|Z| = z | |Y_x| = a, |Y_{x'}| = b] \\
    & ~ \cdot \Pr[|Y_x| = a] \cdot  \Pr[|Y_{x'}| = b].
\end{align*}

where

\begin{align*}
    & ~ \Pr[W = w | |Z| = z, |Y_x| = a, |Y_{x'}| = b] \\
    = & ~
    \begin{cases}
        0,  &  n_2 < w \\
        \binom{n_2}{w} \cdot p_0^w \cdot (1 - p_0)^{n_2 - w}, & n_2 \geq w
    \end{cases}
\end{align*}
\end{lemma}

\begin{proof}

By basic probability rules, we have the following equation
\begin{align*}
    & ~\Pr[W = w] \notag \\
    =& ~ \sum_{a = 1}^k \sum_{b = 1}^k \sum_{z = 1}^{a+b} \Pr[W = w | |Z| = z, |Y_x| = a, |Y_{x'}| = b] \notag \\
    & ~ \cdot \Pr[|Z| = z | |Y_x| = a, |Y_{x'}| = b] \\
    & ~ \cdot \Pr[|Y_x| = a, |Y_{x'}| = b] \notag \\
    =& ~ \sum_{a = 1}^k \sum_{b = 1}^k \sum_{z = 1}^{a+b} \Pr[W = w | |Z| = z, |Y_x| = a, |Y_{x'}| = b] \notag \\
    & ~ \cdot \Pr[|Z| = z | |Y_x| = a, |Y_{x'}| = b] \\
    & ~ \cdot \Pr[|Y_x| = a] \cdot \Pr[|Y_{x'}| = b].
\end{align*}
where the first step follows from basic probability rules, the second step follows from $Y_x$, and $Y_{x'}$ are independent. 

We can get the probability of $\Pr[|Y_x| = a]$ and $\Pr[|Y_{x'}| = b$ from Lemma~\ref{lem:distribution_of_Y}. 

We can get the probability of $\Pr[|Z| = z | |Y_x| = a, |Y_{x'}| = b]$ from Lemma~\ref{lem:distribution_of_Z}. 

Now, let's consider the $\Pr[W = w | |Z| = z, |Y_x| = a, |Y_{x'}| = b]$ term. 

Note that only elements in the exclusive-or set may contribute to the final $W$. Therefore, we have $w \leq n_2$. Namely, when $n_2 < w$, we have $\Pr[W = w | |Z| = z, |Y_x| = a, |Y_{x'}| = b] = 0$. 

Now, let's calculate $\Pr[W = w | |Z| = z, |Y_x| = a, |Y_{x'}| = b]$ under $n_2 \geq w$ condition. 

Recall $x$ denotes the element deleted from $A$, and $x'$ denotes the element added to $A$ for constructing the neighbor dataset $A'$. 

Let $A_{fix} := A - x$ denote the fixed set of elements during the modifications. We have $|A_{fix}| = |A| - 1$. 

Consider the following steps:
\begin{itemize}
    \item We construct a new Bloom filter.
    \item We insert all elements in $A_{fix}$ in the Bloom filter.
    \item We define $Z_{zero}$ as the set of positions of bits which are still $0$ after the insertion of $A_{fix}$.
\end{itemize}

We define $Z_{xor}$ as the exclusive-or set of $Y_x$ and $Y_{x'}$. We have
\begin{align*}
    Z_{xor} = & ~ (Y_x \cup Y_{x'}) - (Y_x \cap Y_{x'}), \notag \\
    |Z_{xor}| = & ~ |Y_x \cup Y_{x'}| - |Y_x \cap Y_{x'}| \notag \\
    = & ~ z - (a +b - z) \notag \\
    = & ~ 2z - a - b \notag \\
    = & ~ n_2.
\end{align*}

Note that only positions in $Z_{xor} \cap Z_{zero}$ will contribute to $W$. Namely, we need $|Z_{xor} \cap Z_{zero}| = w$. 

We achieve the above condition by selecting $w$ elements in $Z_{xor}$ and let them satisfy the condition of $Z_{zero}$. 

Therefore, we have 
\begin{align*}
    & ~ \Pr[|Z_{xor} \cap Z_{zero}| = w] \\
    = & ~ \binom{n_2}{w} \cdot (1 - \frac{1}{m})^{(|A| - 1)kw} \cdot (1 - (1 - \frac{1}{m})^{(|A| - 1)k})^{n_2 - w}.
\end{align*}

Combining the above analysis, we have
\begin{align*}
    & ~ \Pr[W = w | |Z| = z, |Y_x| = a, |Y_{x'}| = b] \\
    = & ~ 
    \begin{cases}
        0,  &  n_2 < w \\
        \binom{n_2}{w} \cdot p_0^w \cdot (1 - p_0)^{n_2 - w}, & n_2 \geq w
    \end{cases}.
\end{align*}


\end{proof}



\section{Privacy guarantees for one coordinate}\label{sec:appendix_privacy_guarantees}
In this section, we provide proof of the privacy guarantees of the DPBloomfilter.

In Section~\ref{sec:single_bit_private}, we demonstrate the privacy guarantees for single bit of array in Bloom filter.

Then in Section~\ref{sec:query_privacy}, we provide the proof of privacy guarantees for our entire algorithm.

\subsection{Single bit is private} \label{sec:single_bit_private}
We first consider the privacy guarantees of single bit of array in Bloom filter.
\begin{lemma} [Single bit is private
] \label{lem:eps0_DP:formal}
If the following conditions hold:
\begin{itemize}
    \item Let $\epsilon_0 \geq 0$. 
    \item Let $\wt{g}[j] \in \{0,1\}$ be the $i$-th element of array output by DPBloomfilter  
    
    
\end{itemize}

Then, we can show that, for all
$j \in [m]$, $\wt{g}[j]$ is $\epsilon_0$-DP. 
\end{lemma}

\begin{proof}

$\forall j \in [m]$, $g[j]$ is the ground truth value generated by dataset $A \subset [n]$. (An alternative view of $g$ is $g:[m] \rightarrow \{0,1\}$.) Suppose $g[j] = u$, $u \in \{0, 1\}$. For any neighboring dataset $A' \subset [n]$, we denote the ground truth value generated by it as $g'[j]$. Similarly, we can define the $\wt{g}'[j]$. 

We consider the following two cases to prove $\wt{g}[j]$ is $\epsilon_0$-DP, for all $j \in [m]$.

{\bf Case 1}. Suppose $g'[j] = u$. We know
\begin{align*}
    \Pr [ \wt{g}[j] = u ] = & ~ \frac{e^{\epsilon_0}}{ e^{\epsilon_0} + 1 }, \\
    \Pr[ \wt{g}'[j] = u ] = & ~ \frac{e^{\epsilon_0}}{ e^{\epsilon_0} + 1 }.
\end{align*}
Combining the above two equations, then we obtain
\begin{align*}
\frac{ \Pr [ \wt{g}[j] = u ] }{ \Pr[ \wt{g}'[j] = u ] } = 1.
\end{align*}

Similarly, we know 
\begin{align*}
    \Pr [ \wt{g}[j] = 1-u ] = & ~ \frac{ 1 }{ e^{\epsilon_0} + 1 }, \\
    \Pr[ \wt{g}'[j] = 1-u ] = & ~ \frac{ 1 }{ e^{\epsilon_0} + 1 }.
\end{align*}
Combining the above two equations, then we obtain
\begin{align*}
\frac{ \Pr [ \wt{g}[j] = 1- u ] }{ \Pr[ \wt{g}'[j] = 1-u ] } = 1.
\end{align*}
Thus, we know for all $v\in \{0,1\}$,
\begin{align*}
\frac{ \Pr [ \wt{g}[j] = v ] }{ \Pr[ \wt{g}'[j] = v ] } = 1.
\end{align*}

{\bf Case 2}. Suppose $g'[j] \neq u$.

We know
\begin{align*}
    \Pr [ \wt{g}[j] = u ] = & ~ \frac{e^{\epsilon_0}}{ e^{\epsilon_0} + 1 }, \\
    \Pr[ \wt{g}'[j] = u ] = & ~ \frac{ 1 }{ e^{\epsilon_0} + 1 }.
\end{align*}
Combining the above two equations, then we obtain
\begin{align*}
\frac{ \Pr [ \wt{g}[j] = u ] }{ \Pr[ \wt{g}'[j] = u ] } = e^{\epsilon_0}.
\end{align*}

Similarly, we know 
\begin{align*}
    \Pr [ \wt{g}[j] = 1-u ] = & ~ \frac{ 1 }{ e^{\epsilon_0} + 1 }, \\
    \Pr[ \wt{g}'[j] = 1-u ] = & ~ \frac{ e^{\epsilon_0} }{ e^{\epsilon_0} + 1 }.
\end{align*}
Combining the above two equations, then we obtain
\begin{align*}
\frac{ \Pr [ \wt{g}[j] = 1- u ] }{ \Pr[ \wt{g}'[j] = 1-u ] } = e^{-\epsilon_0}.
\end{align*}


For $v \in \{0, 1\}$, we have 
\begin{align*}
e^{- \epsilon_0} \leq \frac{ \Pr [ \wt{g}[j] = v ] }{ \Pr [ \wt{g}'[j] = v ] } \leq e^{\epsilon_0}.
\end{align*}



Therefore, $\forall j \in [m]$, $\wt{g}[j]$ is $\epsilon_0$-DP. 
\end{proof}

\subsection{Privacy guarantees for DPBloomfilter}\label{sec:query_privacy}
Then, we can prove that our entire algorithm is differentially private.
\begin{theorem}[Privacy for Query, formal version of Lemma~\ref{thm:query_privacy:informal}]\label{thm:query_privacy:formal}
If the following conditions hold
\begin{itemize}
    \item Let $N = F_W^{-1}(1 - \delta)$ denote the $1 - \delta$ quantile of the random variable $W$ (see Definition~\ref{def:W}).
    \item Let  $\epsilon_0 = \epsilon / N$.
\end{itemize}

Then, we can show,
the output of \textsc{Query} procedure of Algorithm~\ref{alg:init} achieves $(\epsilon, \delta)$-DP. 
\end{theorem}

\begin{proof}
Let $A$ and $A'$ are neighboring datasets. Let $g \in \{0, 1\}^m$ is the ground truth value generated by dataset $A$, and $g' \in \{0, 1\}^m$ is the ground truth value generated by dataset $A'$. 


We define
\begin{align*}
    S := \{j \in [m] ~:~ g[j] \neq g'[j]\}.
\end{align*}
We further define
\begin{align*}
    \ov{S} := [m] \backslash S.
\end{align*}

We consider two cases, {\bf Case 1} is $j \in \ov{S}$ and {\bf Case 2} is $j \in S$.

{\bf Case 1}. $j \in \ov{S}$. 

We can show that
\begin{align*}
\frac{ \Pr [ \wt{g}[j] = v ] }{\Pr[ \wt{g'}[j] = v ] } = 1.
\end{align*}
holds for $\forall v \in \{0, 1\}$.

{\bf Case 2.} $j \in S$.

We can show that
\begin{align}\label{eqn:query_privacy_single}
    e^{-\epsilon_0}\leq \frac{ \Pr[ \wt{g}[j] = v ] }{ \Pr[ \wt{g'}[j] = v] } \leq e^{\epsilon_0}.
\end{align}
holds for $\forall v \in \{0, 1\}$. 

Thus, for any $Z\in \{0,1\}^m$, the absolute privacy loss can be bounded by
\begin{align}\label{eqn:query_privacy_prod}
     |\ln \frac{ \Pr[ \wt{g} = Z ] }{ \Pr[ \wt{g'} = Z ] } | 
     = & ~  |\ln \prod_{j\in S} \frac{ \Pr[ \wt{g}[j] = v ] }{ \Pr[ \wt{g'}[j] = v ] }  | \notag \\
     \leq & ~ |S| \epsilon_0 \notag \\
     = & ~  |S|\frac{\epsilon}{N}.  
\end{align}
where the first step follows from each entry of $g$ is independent, the second step follows from Eq.~\eqref{eqn:query_privacy_single}, and the last step follows from choice of $\epsilon_0$.

By the definition of $N$, we know that with probability at least $1-\delta$, $|S|\leq F^{-1}(1-\delta)=N$. Hence, Eq.~\eqref{eqn:query_privacy_prod} is upper bounded by $\epsilon$ with probability $1-\delta$. 

This proves the $(\epsilon,\delta)$-DP.
\end{proof}

\section{Utility analysis}\label{sec:appendix_utility}
In this section, we establish the utility guarantees for our algorithm. Initially, we calculate the accuracy for the query of the standard Bloom filter in Section~\ref{sec:acc_bloom}. We then assess the utility loss caused by introducing the random response technique by comparing the output of the DPBloomfilter with the output of the standard Bloom filter in Section~\ref{sec:acc_dpbloom_bloom}. Ultimately, we present the assessment of our algorithm's utility in Section~\ref{sec:acc_dpbloom_true}.

We begin by defining the notation we will use in this section.
\begin{definition}\label{def:three_z}
    Let $z \in \{0,1\}$ denote the true answer for whether $x \in A$. Let $\wh{z} \in \{0,1\}$ denote the answer outputs by \textsc{Bloom filter}. Let $\wt{z} \in \{0,1\}$ denote the answer output by \textsc{DPBloomFilter} (Algorithm~\ref{alg:init}).
\end{definition}

\subsection{Accuracy for query of Standard Bloom Filter}\label{sec:acc_bloom}

We first present the accuracy of the query of the standard bloom filter, as follows.

\begin{lemma}[Accuracy for query of Standard Bloom Filter
]\label{lem:bloom_true_accuracy:formal}
If the following conditions hold
\begin{itemize}
    \item Assume that a hash function selects each array position with equal probability. 
    \item Let $\wh{z}$ be defined as Definition~\ref{def:three_z}.
    \item Let $z$ be defined as Definition~\ref{def:three_z}.
    \item Let $\alpha := \Pr[z=0]$
\end{itemize}
Then, we can show
\begin{align*}
    \Pr [ \wh{z} = z ] \geq 1 - (1 - e^{-2|A| k / m})^k \cdot \alpha.
\end{align*}
Further if $m = \Omega(|A| k)$ and $k = \Theta(\log(1/\delta_{err}))$, we have
\begin{align*}
     \Pr [ \wh{z} = z ] \geq 1 - \delta_{err} \cdot \alpha.
\end{align*}
\end{lemma}

\begin{proof}
Recall that we have defined Bloom filter in Definition~\ref{def:bloom_filter}, it only has false positive error. Therefore, we only need to calculate the following
\begin{align*}
    \Pr[\wh{z} = 1 | z = 0]
\end{align*}

Recall that $A \subset [n]$ denotes the set of elements inserted into the Bloom filter. And $h_i : [n] \rightarrow [m]$ for each $i \in [k]$ denotes $k$ hash functions used in the Bloom filter. 

For a query $y \notin A$, we denotes event $E_1$ happens if the following happens:
\begin{align*}
    h_i[y] = 1, \forall i \in [k]
\end{align*}

Recall that we have defined Bloom filter in Definition~\ref{def:bloom_filter}, we have 
\begin{align}\label{eq:def_E_1}
    \Pr[\wh{z} = 1 | z = 0] = \Pr[E_1].
\end{align}

Now, we start calculating $\Pr[E_1]$.

Recall that we assume a hash function selects each array position with equal probability in the lemma statement. 

During one inserting operation, the probability of a certain bit is not set to $1$ is 
\begin{align*}
    (1 - \frac{1}{m})^k
\end{align*}



If we have inserted $|A|$ elements, the probability that a certain bit is still $0$ is
\begin{align*}
    (1 - \frac{1}{m})^{|A| k} = ( (1-\frac{1}{m})^{m} )^ {|A| k/m } \geq e^{-2 |A| k / m}
\end{align*}
where the last step follows from $(1-1/m)^m \geq e^{-2}$ for all $m \geq 2$.

Thus the probability that a certain bit is $1$ is
\begin{align*}
    1 - (1 - \frac{1}{m})^{ |A| k} \leq 1 - e^{-2 |A| k / m}.
\end{align*}

Combining the above fact, we have
\begin{align}\label{eq:pr_e}
    \Pr[E_1] = & ~ (1 - (1 - \frac{1}{m})^{|A|k})^k \notag \\
    \leq & ~ (1 - e^{-2 |A| k / m})^k.
\end{align}
where the first step follows from the definition of event $E_1$, the second step follows from $(1-1/m)^m \geq e^{-2}$ for all $m \geq 2$. 

Therefore, the accuracy of Bloom filter is
\begin{align*}
    \Pr[\wh{z} = z] 
    = & ~ 1 - \Pr[\wh{z} = 1 | z = 0] \Pr[z=0] \\
    = & ~ 1 - \Pr[E_1] \alpha \\
    \geq & ~ 1 - (1 - e^{-2 |A| k / m})^k \alpha.
\end{align*}
where the first step follows from Bloom filter only has false positive error, the second step follows from the definition of event $E_1$ and the definition of $\alpha$, the third step follows from Eq.~\eqref{eq:pr_e}. 

\end{proof}

\subsection{Accuracy (compare DPBloomFilter with Standard BloomFilter) for Query}\label{sec:acc_dpbloom_bloom}
We then assess the accuracy loss caused by the introduction of the random response technique by comparing the outputs of the DPBloomfilter with those of the standard Bloom filter.

\begin{lemma}[Accuracy (compare DPBloomFilter with Standard BloomFilter) for Query
]\label{lem:dpbloom_bloom_accuracy:formal}

If the following conditions hold
\begin{itemize}
    \item Let $\wh{z}$ be defined as Definition~\ref{def:three_z}.
    \item Let $\wt{z}$ be defined as Definition~\ref{def:three_z}.
    \item Let $\alpha: = \Pr[ z = 0 ] \in [0,1]$
    \item Let $t := \frac{ e^{\epsilon_0} }{ e^{\epsilon_0} + 1 }$. 
    \item Let $\delta_{\mathrm{err}}$ be defined as in Lemma~\ref{lem:bloom_true_accuracy:formal}. 
\end{itemize}

Then, we can show
\begin{align*}
\Pr[ \wt{z} = \wh{z}] \geq t \cdot (\alpha - \delta_{\mathrm{err}}).
\end{align*}

\end{lemma}
\begin{proof}
We denote the query as $q$. 

We define
\begin{align}\label{def:Q}
    Q := \{j \in [m] ~:~ h_i(q) = j,~ i \in [k]\}
\end{align}



We denote $Q[i]$ as the $i$-th element in $Q$. 

Using basic probability rules, we have
\begin{align*}
    & ~ \Pr[\wt{z} = \wh{z}] \\
    = & ~ \Pr[\wt{z} = 1 | \wh{z} = 1] \Pr[\wh{z} = 1] \\
    + & ~ \Pr[\wt{z} = 0 | \wh{z} = 0] \Pr[\wh{z} = 0].
\end{align*}

{\bf Step 1}. Calculate $\Pr[\wt{z} = 1 | \wh{z} = 1]$


We denote event $E_2$ happens as the following happens:
\begin{align*}
    \wt{g}[j] = g[j], \forall j \in Q.
\end{align*}

Recall that we have defined Bloom filter in Definition~\ref{def:bloom_filter}, we have 

\begin{align*}
    \Pr[\wt{z} = 1 | \wh{z} = 1] = \Pr[E_2].
\end{align*}

 
Now, we calculate the probability that $E_2$ happens. 
\begin{align*}
    \Pr [E_2] = & ~ \prod_{i = 1}^k \Pr [\wt{g}[Q[i]] = g[Q[i]]] \notag \\
    = & ~ (\frac{e^{\epsilon_0}}{e^{\epsilon_0} + 1})^k.
\end{align*}
where the first step follows from each entry of $g$ is independent, the second steps follows from the definition of $\wt{g}$. 

Therefore, we have
\begin{align}\label{eq:pr_wtz1_z1}
    \Pr[\wt{z} = 1 | \wh{z} = 1] 
    = & ~ (\frac{e^{\epsilon_0}}{e^{\epsilon_0} + 1})^k.
\end{align}

{\bf Step 2}. Calculate $\Pr[\wt{z} = 0 | \wh{z} = 0]$

Recall we have defined $Q \subset [m]$ in Eq.~\eqref{def:Q}. We further define
\begin{align*}
    Z := \{j \in Q ~:~ g[j] = 0\}.
\end{align*}

We denote $Z[i]$ as the $i$-th element in $Z$. 

We further define
\begin{align*}
    \ov{Q} := Q \backslash Z.
\end{align*}


By basic probability rules, we have
\begin{align*}
    \Pr[\wt{z} = 0 | \wh{z} = 0] = & ~ 1 - \Pr[\wt{z} = 1 | \wh{z} = 0].
\end{align*}

Now, let's calculate $\Pr[\wt{z} = 1 | \wh{z} = 0]$

$[\wt{z} = 1 | \wh{z} = 0]$ happens only if the following conditions hold:
\begin{enumerate}
    \item All elements in $Z$ flip from $0$ to $1$.
    \item All elements in $\ov{Q}$ remain $1$.
\end{enumerate}

Then, we have
\begin{align*}
    \Pr[\wt{z} = 1 | \wh{z} = 0]  = & ~ \prod_{i = 1}^{|Z|} \Pr [\wt{g}[Z[i]] = 1] \prod_{i = 1}^{|\ov{Q}|} \Pr [\wt{g}[\ov{Q}[i]] = 1] \notag \\
    = & ~ (\frac{1}{e^{\epsilon_0} + 1})^{|Z|} (\frac{e^{\epsilon_0}}{e^{\epsilon_0} + 1})^{|\ov{Q}|} \notag \\
    \leq & ~ (\frac{1}{e^{\epsilon_0} + 1})^{|Z|} \notag \\
    \leq & ~ \frac{1}{e^{\epsilon_0} + 1}.
\end{align*}
where the first step follows from the above analysis, the second step follows from the definition of $\wt{g}$, the third step follows from $|\ov{Q}| \geq 0$ and $\frac{e^{\epsilon_0}}{e^{\epsilon_0} + 1} < 1$, the fourth step follows from $|Z| \geq 1$ and $\frac{1}{e^{\epsilon_0} + 1} < 1$. 

Therefore, we have
\begin{align}\label{eq:pr_wtz0_z0}
    \Pr[\wt{z} = 0 | \wh{z} = 0] = & ~ 1 - \Pr[\wt{z} = 1 | \wh{z} = 0] \notag \\
    \geq & ~ 1 -  \frac{1}{e^{\epsilon_0} + 1} \notag \\
    = & ~ \frac{ e^{\epsilon_0} }{ e^{\epsilon_0} + 1 }.
\end{align}
Let $\hat \alpha := \Pr[ \wh{z} = 0 ]$, then we have $1- \wh{\alpha} = \Pr[ \wh{z} = 1 ]$. 
Let $\alpha := \Pr[ z = 0 ]$.
Note that $ \wh{\alpha} = \alpha (1 - \delta_{\mathrm{err}}) $.

Let $t := \frac{e^{\epsilon_0}}{e^{\epsilon_0} + 1}$. 

The final accuracy is 
\begin{align*}
& ~ \Pr[\wt{z} = 0 | \wh{z} = 0]  \cdot \Pr[ \wh{z} = 0 ] + \Pr[\wt{z} = 1 | \wh{z} = 1]  \cdot \Pr[ \wh{z} = 1 ] \\
= & ~ \Pr[\wt{z} = 0 | \wh{z} = 0]  \cdot \wh{\alpha} + \Pr[\wt{z} = 1 | \wh{z} = 1]  \cdot (1- \wh{\alpha}) \\
= & ~ \Pr[\wt{z} = 0 | \wh{z} = 0]  \cdot \alpha (1 - \delta_{err}) \\
+ & ~ \Pr[\wt{z} = 1 | \wh{z} = 1]  \cdot (1- \alpha + \alpha \cdot \delta_{err}) \\
\geq & ~ \frac{ e^{\epsilon_0} }{ e^{\epsilon_0} + 1 }  \cdot \alpha (1 - \delta_{err}) + (\frac{ e^{\epsilon_0} }{ e^{\epsilon_0} + 1 })^k  \cdot (1- \alpha + \alpha \cdot \delta_{err}) \notag \\ 
= & ~ t \cdot (\alpha - \alpha \cdot \delta_{err}) + t^k  \cdot (1- \alpha + \alpha \cdot \delta_{err}) \\
\geq & ~ t \cdot \alpha \cdot (1 - \delta_{err}).
\end{align*}



where the first step follows from the definition of $\wh{\alpha}$, the second step follows from $ \wh{\alpha} = \alpha (1 - \delta) $, the third step follows from  Eq.~\eqref{eq:pr_wtz1_z1} Eq.~\eqref{eq:pr_wtz0_z0}, the fourth step follows from basic algebra rules, the fifth step follows from $(1 - \alpha + \alpha \cdot \delta_{\mathrm{err}}) \geq 0$. 

Therefore, the final accuracy is $t \cdot (\alpha - \delta_{err})$. 
\end{proof}

\subsection{Accuracy (compare DPBloomfilter with true-answer) for Query}\label{sec:acc_dpbloom_true}
Now we can examine the utility guarantees of DPBloomfilter by calculating the error between the ground truth for query and the output of DPBloomfilter.

\begin{theorem}[Accuracy (compare DPBloomfilter with true-answer) for Query, formal version of Lemma~\ref{thm:dpbloom_true_accuracy:informal}]\label{thm:dpbloom_true_accuracy:formal}

If the following conditions hold
\begin{itemize}
    \item Let $\wh{z}$ be defined as Definition~\ref{def:three_z}.
    \item Let $z$ be defined as Definition~\ref{def:three_z}.
    \item Let $\alpha: = \Pr[ z = 0 ] \in [0,1]$
    \item Let $t := e^{\epsilon_0} / (e^{\epsilon_0} + 1)$. 
    \item Let $\delta_{\mathrm{err}}$ be defined as in Lemma~\ref{lem:bloom_true_accuracy:formal}. 
\end{itemize}

Then, we can show 
\begin{align*}
\Pr[ \wt{z} = z ] \geq \alpha (1-t-t^k) \delta_{\mathrm{err}} + \alpha t .
\end{align*}
\end{theorem}

\begin{proof}

We have
\begin{align*}
    & ~ \Pr[ \wt{z} = z ] \\
    = & ~ \Pr [\wt{z} = 0 | \wh{z} = 0] \Pr [\wh{z} = 0 | z = 0] \Pr[z=0] \\
    + & ~ \Pr [\wt{z} = 0 | \wh{z} = 1] \Pr [\wh{z} = 1 | z = 0] \Pr[z=0] \\
    + & ~ \Pr [\wt{z} = 1 | \wh{z} = 1] \Pr [\wh{z} = 1 | z = 1] \Pr[z=1] \\
    + & ~ \Pr [\wt{z} = 1 | \wh{z} = 0] \Pr [\wh{z} = 0 | z = 1] \Pr[z=1]\\
    \geq & ~ t \cdot (1 - \Pr[E_1]) \cdot \alpha + (1- t^k) \cdot \Pr[E_1]\cdot \alpha + t^k \cdot 1 \cdot (1-\alpha)\\
     = & ~ \alpha (1-t-t^k) \delta_{\mathrm{err}} + \alpha t + t^k(1-\alpha)\\
     \geq & ~ \alpha (1-t-t^k) \delta_{\mathrm{err}} + \alpha t.
\end{align*}
where the first step from basic probability rules, the secod step follows from Equation~\ref{eq:def_E_1}, Equation \ref{eq:pr_wtz0_z0} and definition of $\alpha$ and $t$, the third step follows from basic algebra,  the fourth step follows from the fact that $t,\alpha \in [0,1]$.

\end{proof}

To make it easier to understand, we also provide the utility analysis of the Bloom filter under the case of random guess.  

\begin{lemma}[Accuracy for Query under Random Guess]\label{lem:random_guess}
If the following conditions hold
\begin{itemize}
    \item Let $\wh{z}$ be defined as Definition~\ref{def:three_z}.
    \item $\epsilon_0 = 0$. Namely, each bit in the bit-array of the DP Bloom has $\frac{1}{2}$ probability to be set to $0$, and  $\frac{1}{2}$ probability to be set to $1$. 
\end{itemize}

Then, we can show 
\begin{align*}
    \Pr[\wt{z} = 0] = & ~ 1 - \frac{1}{2^k}, \notag \\
    \Pr[\wt{z} = 1] = & ~ \frac{1}{2^k}.
\end{align*}
\end{lemma}

\begin{proof}
    By the definition of Bloom filter~\ref{def:bloom_filter}, the answer $\wt{z} = 1$ requires $k$ corresponding positions in the bit-array of the query are all set to $1$. 

    Note that each bit has $\frac{1}{2}$ probability to be set to $1$. Therefore, we have
    \begin{align*}
        \Pr[\wt{z} = 1] = \frac{1}{2^k} .
    \end{align*}

    Then, we have $\Pr[\wt{z} = 0] = 1 - \Pr[\wt{z} = 1] = 1 - \frac{1}{2^k}.$
\end{proof}





\section{Running Time}\label{sec:appendix_running_time}
In this section, we provide the proof of running time for Algorithm~\ref{alg:init}. The running time for our algorithm consists of two parts: time for initialization in Section~\ref{sec:time_init} and time for query 
in Section~\ref{sec:time_query}. 
\subsection{Running time for initialization}\label{sec:time_init}
Now we calculate the time of initialization for our algorithm. 
\begin{lemma}[Running time for initialization]\label{lem:init_time}
Let $\mathcal{T}_h$ denote the time of evaluation of function $h$ at any point. 

It takes $O(|A| \cdot k \cdot \mathcal{T}_h + m)$ time to run the initialization function.
\end{lemma}
\begin{proof}
 
{\bf Step 1} Let's consider the initialization of the standard Bloom filter. 

A single element $x$ needs $O(k \cdot \mathcal{T}_h)$ time to compute over $k$ hash functions. 

There are $|A|$ elements which need to be inserted. 

Combining the above two facts, it needs $O(|A| \cdot k \cdot \mathcal{T}_h)$ time to initialise the standard Bloom filter. 

{\bf Step 2} Let's consider the ``Flip each bit" part. 

Since there are $m$ bits in the Bloom filter, it takes $O(m)$ time to flip each bit.

Therefore, the initialization function needs $O(|A| \cdot k \cdot \mathcal{T}_h + m)$ time to run. 

\end{proof}

\subsection{Running time for query}\label{sec:time_query}
Then, we proceed to calculate the query time for our algorithm.

\begin{lemma}[Running time for query]\label{lem:query_time}
Let $\mathcal{T}_h$ denote the time of evaluation of function $h$ at any point. 
It takes $O(k \cdot \mathcal{T}_h)$ time to run each query $y$ in the query function.
\end{lemma}
\begin{proof}

For each query $y$, the algorithm needs $O(k \cdot \mathcal{T}_h)$ time to compute the hash values of $y$ over $k$ hash functions. 

Therefore, it takes $O(k\cdot \mathcal{T}_h)$ time to run the query function for each query. 
\end{proof}

By combing the result of Lemma~\ref{lem:init_time} and Lemma~\ref{lem:query_time}, we can obtain the running of our entire algorithm is $O(|A|\cdot k \cdot \mathcal{T}_h + m)$.



% \section{Technical Overview}\label{sec:tech_overview}
In Section~\ref{sec:tec_privacy_sb}, we will provide the privacy guarantees of Single Bit in DPBloomfilter. Then, we will present the privacy guarantees of our entire algorithm in Section~\ref{sec:tec_privacy_dp}. In Section~\ref{sec:tec_utility_dp}, we will examine the utility guarantees of DPBloomfilter. Additionally, we include a remark that analyzes the trade-off between privacy and utility inherent in our approach. In Section~\ref{sec:tec_time_dp}, we discuss the running time of our algorithm.


\subsection{Privacy Guarantees of Single Bit}\label{sec:tec_privacy_sb}

To accomplish differential privacy, Algorithm~\ref{alg:init} applies a random response mechanism to each bit of the standard Bloom Filter. In this section, we aim to examine the privacy guarantees for a single bit of our algorithm.

Recall that in Definition~\ref{sec:pre_def_bf}, for dataset $A \subset [n]$, we use $g[j]$  to denote the $j$-th element of array output by standard Bloom Filter. Here, we use $\wh{g}[j]$ to denote the $j$-th element of array output by DPBloomfilter. Similarly, for any neighboring dataset $A' \subset [n]$, we use $g'[j]$ and $\wh{g}'[j]$ to denote the $j$-th element of array output by standard Bloom Filter and DPBloomfilter. 
To examine the privacy guarantees for the $i$-th bit, we must consider two distinct cases.

{\bf Case 1}. Suppose $g'[j] = g[j]$, then we can obtain (See also Lemma~\ref{lem:eps0_DP:formal}) that  for all $v \in \{0,1\}$, we have
\begin{align*}
    \frac{\Pr[\wt{g}[j]=v]}{\Pr[\wt{g}'[j]=v]} = 1.
\end{align*}

{\bf Case 2}. Suppose $g'[j] \neq g[j]$, then we can obtain (See also Lemma~\ref{lem:eps0_DP:formal}) that for all $v \in {0,1}$, we have
\begin{align*}
    e^{-\epsilon_{0}} \leq \frac{\Pr[\wt{g}[j]=v]}{\Pr[\wt{g}'[j]=v]} \leq e^{\epsilon_{0}}.
\end{align*}

By combining the above two cases, we can demonstrate the privacy guarantees of single bit for our algorithm.

\begin{lemma} [Differential Privacy for single Bit, informal version of Lemma~\ref{lem:eps0_DP:formal}] \label{lem:eps0_DP:informal}
Let $\epsilon_0 \geq 0$ and $\wt{g}[i] \in \{0,1\}$ be the $i$-th element of array output by DPBloomfilter. 
Then, we can show that, for all
$j \in [m]$, $\wt{g}[j]$ is $\epsilon_0$-DP. 
\end{lemma}



\subsection{Privacy Guarantees of DPBloomFilter}\label{sec:tec_privacy_dp}
Here, we comprehensively analyze the DP guarantees for our DPBloomFilter. Recall that in Definition~\ref{def:bloom_filter}, for dataset $A$, we use $g$ to denote the array output by standard Bloom Filter. Here, we use $\wt{g}$ to denote the array output by DPBloomfilter. Similarly, for any neighboring dataset $A'$, we use $g'$ and $\wh{g}'$ to denote the array output by standard Bloom Filter and DPBloomfilter, respectively.

Here, we consider the set of indices $j$ within the range $m$ where the value of $g[j]$ and $g'[j]$ differs.
\begin{align*}
    S := \{j \in [m] : g[j] \neq g'[j]\}.
\end{align*}
Thus, the set of indices $j$ where the value of $g[j]$ and $g'[j]$ are the same can be defined as
$
    \ov{S} := [m] \backslash S.
$
We can use the result of privacy guarantees of a single bit in Section~\ref{sec:tec_privacy_sb}, for any $j \in S$ and $v \in \{0,1\}$, we have
\begin{align*}
\frac{\Pr[\wt{g}[j]=v]}{\Pr[\wt{g}'[j]=v]}  = 1,
\end{align*}
and for any $j \in \ov{S}$ and $v \in \{0,1\}$, we have
\begin{align*}
e^{-\epsilon_{0}} \leq \frac{\Pr[\wt{g}[j]=v]}{\Pr[\wt{g}'[j]=v]} \leq e^{\epsilon_{0}}.
\end{align*}
By applying the composition lemma (refer to Lemma~\ref{lem:pre_com_lem}) , we obtain the following for any $Z \in \{0,1\}^m$,
\begin{align}\label{equ:epsilon_0_bound}
|\ln{\frac{\Pr[\wt{g} = Z]}{\Pr[\wt{g}' = Z]}}| \leq  |S|\epsilon_0.
\end{align}
Here, we define $W := |S|$ for convenience. To get a better bound for Equation~\ref{equ:epsilon_0_bound}, we need to calculate the probability distribution function of the random variable $W$. Before that, we need to define two random variables we will use. Firstly, we define $Y$ as the set of distinct values among the $k$ hash values generated by the standard Bloom filter considering one $x \in [n]$. Then we consider two data $x, x' \in [n]$. We define $Z$ as the set of distinct values in $Y_{x} \cup Y_{x'}$.

Then firstly we proceed to calculate the distribution of $|Y|$ (see details in Lemma~\ref{lem:distribution_of_Y}), we can show for any $y = 1,2,\dots,k$
\begin{align*}
    & ~ \Pr[|Y| = y] \\
    = & ~ 
    \begin{cases}
        1 / m^{k-1},  & y = 1 \\
        \binom{m}{y} \cdot y ^k / m^k
        - \sum_{i=1}^{k-1} \Pr[Y = i] \cdot \binom{m - i}{y -i}, & y = 2, \cdots , k
    \end{cases}
\end{align*}
Given the probability of $|Y|$, we can calculate the conditional probability of $|Z|$ conditioned on $|Y_{x}| = a$ and $|Y_{x'}| = b$, where $a,b \in [k]$ (see details in Lemma~\ref{lem:distribution_of_Z})
\begin{align*}
    \Pr[|Z| = z | |Y_x| = a, |Y_{x'}| = b] = \frac{A_m^a \cdot \binom{b}{t} \cdot A_{m - a}^t \cdot A_{a}^{b-t}}{A_m^a \cdot A_m^b}.
\end{align*}
Finally, we use the property of union probability. We can calculate the probability of $W$ (see details in Lemma~\ref{lem:distribution_of_W}). 
Fig.~\ref{fig:w_distribution} visualize the distribution of the random variable $W$ under the setting described in the experiments section (Section~\ref{sec:experiments}). It shows that the distribution of $W$ has a good concentration property, i.e., it concentrates on its mean.

Recall in Section~\ref{sec:pre_notations}, we use $F_{X}^{-1}$ to denote the $1-\delta$ quantile of the Cumulative Distribution Function $F_{X}(x)$ of random variable $X$. 

Here, we define
\begin{align*}
    N:= F_{W}^{-1}(1-\delta)
\end{align*}
Hence, by the properties of the quantile function, we have
\begin{align*}
    \Pr[N \leq W] = 1-\delta.
\end{align*}
By choosing the appropriate value of $\epsilon_0 = \epsilon/N$, we have
\begin{align*}
|\ln{\frac{\Pr[\wt{g} = Z]}{\Pr[\wt{g}' = Z]}}| \leq W\frac{\epsilon}{N}.
\end{align*}
Then we have, with probability $1-\delta$,
\begin{align*}
    |\ln{\frac{\Pr[\wt{g} = Z]}{\Pr[\wt{g}' = Z]}}| \leq \epsilon.
\end{align*}
Then, we can demonstrate the privacy guarantees for DPBloomfilter (see also Theorem~\ref{thm:query_privacy:informal}).



\subsection{Utility Guarantees of DPBloomfilter}\label{sec:tec_utility_dp}
This section will present a comprehensive analysis of the utility guarantees for DPBloomfilter.
We start by introducing the following conditions for the Utility guarantee of DPBloomFilter.
\begin{condition} \label{con:utility_condition}
We need the following conditions for Utility guarantees of DPBloomfilter:
\begin{itemize}
    \item \textbf{Condition 1.} Assume that a hash function selects each array position with equal probability.
    \item \textbf{Condition 2.} Let $z \in \{0,1\}$ denote the ground truth for whether an element $y \in A$.
    \item \textbf{Condition 3.} Let $\wh{z} \in \{0,1\}$ denote the answer output by standard Bloom Filter for whether an element $y \in A$.
    \item \textbf{Condition 4.} Let $\wt{z} \in \{0,1\}$ denote the answer output by DPBloomfilter for whether an element $y \in A$
    \item \textbf{Condition 5.} Let $\alpha:=\Pr[z=0] \in [0,1]$
    \item \textbf{Condition 6.} Let $t := e^{\epsilon_0} / (e^{\epsilon_0} + 1)$. 
\end{itemize}
    
\end{condition}

Firstly, we proceed to derive the utility of the standard Bloom Filter by calculating 
\begin{align*}
    \Pr[\wh{z} = z] = 1 - \Pr[\wh{z} = 1 | z = 0] \Pr[z=0].
\end{align*}
The above equation comes from the fact that Bloom Filter will not introduce a false negative. After the initialization process of Bloom Filter, the probability of one certain bit is not set to $1$ is (see also Lemma~\ref{lem:bloom_true_accuracy:formal})
\begin{align*}
    (1-\frac{1}{m})^{|A|k} \geq e^{-2|A|k/m}.
\end{align*}
A false positive occurs when, for all $i \in [k]$, the elements $g[h_i(y)]$ are all set to $1$ after initialization. In this case, we have:
\begin{align*}
    \Pr[\wh{z} = 1 | z = 0] = & ~( 1 - (1 - \frac{1}{m})^{|A|k})^k \leq  ~ (1 - e^{-2|A|k/m})^k.
\end{align*}
Therefore, we have
\begin{align*}
    \Pr[\wh{z} = z] \geq & ~ 1 - (1 - e^{-2|A|k/m})^{k} \alpha.
\end{align*}
Further if $m = \Omega(|A|k)$ and $k = \Theta(log(\alpha/\delta_{err}))$, we have
\begin{align*}
    \Pr[\wh{z} = z] = 1 - \delta_{\mathrm{err}} \cdot \alpha.
\end{align*}
\begin{lemma} [Accuracy for query of Standard Bloom filter, informal version of Lemma~\ref{lem:bloom_true_accuracy:formal}]\label{lem:bloom_true_accuracy:informal}
If Condition~\ref{con:utility_condition} holds, we have
\begin{align*}
    \Pr [ \wh{z} = z ] \geq 1 - (1 - e^{-2|A| k / m})^k \cdot \alpha.
\end{align*}
Further if $m = \Omega(|A| k)$ and $k = \Theta(\log(1/\delta_{err}))$, we have
\begin{align*}
     \Pr [ \wh{z} = z ] \geq 1 - \delta_{\mathrm{err}} \cdot \alpha.
\end{align*}
\end{lemma}

We then quantify the error introduced by applying the random response mechanism in the DPBloomfilter by calculating $\Pr[\wt{z} = \wh{z}]$. Using basic probability rules, we have
\begin{align*}
    \Pr[\wt{z} = \wh{z}] = & ~ \Pr[\wt{z}=1|\wh{z}=1]\Pr[\wh{z}=1] \\
    & ~ +\Pr[\wt{z}=0|\wh{z}=0]\Pr[\wh{z}=0].
\end{align*}

We can compute the following term by using the definition of DPBloomfilter in Algorithm~\ref{alg:init}  (see details in Lemma~\ref{lem:dpbloom_bloom_accuracy:formal})
\begin{align*}
    \Pr[\wt{z}=1|\wh{z}=1] = & ~ (\frac{e^{\epsilon_0}}{e^{\epsilon_0}+1})^k, \\
     \Pr[\wt{z}=0|\wh{z}=0] \geq & ~ \frac{e^{\epsilon_0}}{e^{\epsilon_0}+1}.
\end{align*}
Here we let $\Pr[\wh{z}=0] = \wh{\alpha}$, note that $\wh{\alpha} = \alpha (1 - \delta_{\mathrm{err}})$. Hence, $\Pr[\wh{z} = 1] = 1 - \Pr[\wh{z}=0] = 1 - \alpha + \alpha \cdot \delta_{err}$. Then we will have (see details in Lemma~\ref{lem:dpbloom_bloom_accuracy:formal})
\begin{align*}
    \Pr[\wh{z} = z] \geq t \cdot \alpha \cdot (1 - \delta_{err}).
\end{align*}

\begin{lemma}[Accuracy (compare DPBloomFilter with Bloom) for Query, informal version of Lemma~\ref{lem:dpbloom_bloom_accuracy:formal}]\label{lem:dpbloom_bloom_accuracy:informal}
If Condition~\ref{con:utility_condition} holds, we can show
Then, we can show
\begin{align*}
\Pr[ \wt{z} = \wh{z}] \geq t \cdot \alpha \cdot (1 - \delta_{err}).
\end{align*}
\end{lemma}
Now, we can proceed to examine the utility guarantees of DPBloomfilter by calculating $\Pr[\wt{z} = z]$, i.e., comparing the output of DPBloomfilter with the ground truth for the query. 
By combining the result of the analysis above, we will have (see more details in Theorem~\ref{thm:dpbloom_true_accuracy:formal})
\begin{align*}
    \Pr[ \wt{z} = z ] \geq \alpha \cdot (1-t-t^k)\cdot \delta_{\mathrm{err}}+\alpha\cdot t. 
\end{align*}
Then, we have demonstrated the utility guarantees of our algorithm while simultaneously ensuring privacy (see Theorem~\ref{thm:dpbloom_true_accuracy:informal}).


Similar to other differential privacy algorithms, our algorithm encounters a trade-off between privacy and utility, where increased privacy typically results in a reduction in utility, and conversely. An in-depth examination of this trade-off is provided as follows.

\begin{remark} [Trade-off between Privacy and Utility of DPBloomfilter]
An inherent trade-off exists between the privacy and utility guarantees of our algorithm. To ensure privacy, we must lower the value of $\epsilon_0$ in Theorem~\ref{thm:query_privacy:informal}. On the other hand, for utility considerations (in Theorem~\ref{thm:dpbloom_true_accuracy:informal}), we define the lower bound of $\Pr[\wt{z} = z]$ as $u = \alpha(1-t-t^k)\delta_{\mathrm{err}}+\alpha t$
, a reduction in $\epsilon_0$ will lead to a reduction in $t$ then finally result in a reduction in $u$. This, in turn, leads to diminished utility.
\end{remark}

\subsection{Running Time of DPBloomfilter}\label{sec:tec_time_dp}
In this section, we will analyze the running time of our DPBloomfilter. 
Recall in Definition~\ref{def:bloom_filter}, we let $\mathcal{T}_{h}$ denote the computation time per execution for all hash functions. To analyze the algorithm's running time, firstly, we consider the running time of initialization in Algorithm~\ref{alg:init}. It contains two steps as follows

\textbf{Step 1.} Let's consider the initialization of the standard Bloom Filter. For a single element $x \in A$, it needs $O(k\cdot \mathcal{T}_h)$ time to compute over $k$ hash functions. And $|A|$ elements need to be inserted. Combining these two facts, it needs $|A|\cdot k \cdot \mathcal{T}_h$ time to initialize the standard Bloom Filter.

\textbf{Step 2.} Let's consider the ``Flip each bit'' part in DPBloomfilter. Since there are $m$ bits in the Bloom Filter, it takes $O(m)$ time to flip each bit.

Hence, it takes $O(|A|\cdot k \cdot \mathcal{T}_{h}+m)$ time to run the initialization function in Algorithm~\ref{alg:init}. (see also in Lemma~\ref{lem:init_time})

Then we consider the running time of a single query in Algorithm~\ref{alg:init}. For each query $y$, the algorithm needs $O(k \cdot \mathcal{T}_{h})$ time to compute the hash values of $y$ over $k$ hash functions. Hence, it takes $O(k \cdot \mathcal{T}_{h})$ time to run each query $y$ in. (see also in Lemma~\ref{lem:query_time})

By combining the two running time together, we can obtain the running time of our entire algorithm is $O(|A|\cdot k \cdot \mathcal{T}_{h}+m)$. This highlights the advantage of our algorithm: it matches the time complexity of a standard Bloom Filter while providing a strong privacy guarantee.





\section{Experiments}\label{sec:experiments}


\begin{figure*}[!ht]
\centering
\includegraphics[width=0.32\textwidth]{eps_figs/eps_eps_diff_m_Random.pdf}
\includegraphics[width=0.32\textwidth]{eps_figs/eps_eps_diff_m_False_Negative.pdf}
\includegraphics[width=0.32\textwidth]{eps_figs/eps_eps_diff_m_False_Positive.pdf}
\caption{
Three kinds of error rates with different bit-array lengths $m$. We fix the number of inserted elements $|A|=10^5$, the number of hash functions $k = 3$, and $\delta = 0.01$ in $(\epsilon, \delta)$-DP. 
In the figure, $\log$ denotes $\log_2$. 
{\bf Left:} Total error denotes the case when we randomly choose queries from the universe $[n]$; 
{\bf Middle:} False negative denotes the case when we randomly choose queries from the set $S$, which represents the set of elements inserted into the DP Bloom filter; 
{\bf Right:} False positive denotes the case when we randomly choose queries from the set $\ov{S} = [n] \backslash S$.  
As $m$ increases, the total error rate and false positive error rate decrease accordingly, while false negative error rate remains constant. 
As $\epsilon$ approaches $0$, the DP Bloom filter gets closer to random guessing. In this case, the false positive error rate converges to $\frac{1}{2^k}$, and the false negative error rate converges to $1 - \frac{1}{2^k}$. This is consistent with our result in Lemma~\ref{lem:random_guess}
Our \textsc{DPBloomFilter} achieves practical utility when $\epsilon$ is small(e.g. $\epsilon < 10$).
}
\label{fig:eps_diff_m}
\end{figure*}


\begin{figure*}[!ht]
\centering
\includegraphics[width=0.32\textwidth]{eps_figs/eps_eps_diff_na_Random.pdf}
\includegraphics[width=0.32\textwidth]{eps_figs/eps_eps_diff_na_False_Negative.pdf}
\includegraphics[width=0.32\textwidth]{eps_figs/eps_eps_diff_na_False_Positive.pdf}
\caption{
Three kinds of error rates with different numbers of inserted elements $|A|$. We fix the length of bit-array $m=2^{19}$, the number of hash functions $k = 3$, and $\delta = 0.01$ in $(\epsilon, \delta)$-DP.
As $|A|$ increases, the Total Error Rate and false positive error rate increase accordingly, while the false negative error rate remains constant. 
}
\label{fig:eps_diff_na}
\end{figure*}

\begin{figure*}[!ht]
\centering
\includegraphics[width=0.32\textwidth]{eps_figs/eps_eps_diff_k_Random.pdf}
\includegraphics[width=0.32\textwidth]{eps_figs/eps_eps_diff_k_False_Negative.pdf}
\includegraphics[width=0.32\textwidth]{eps_figs/eps_eps_diff_k_False_Positive.pdf}
\caption{
Three kinds of error rates with different numbers of hash function $k$.  
We fix the length of bit-array $m=2^{19}$, the number of inserted elements $|A| = 10^5$, and $\delta = 0.01$ in $(\epsilon, \delta)$-DP.
As $k$ increases, the Total Error Rate and false positive error rate decrease accordingly, while the false negative error rate increases accordingly. 
}
\label{fig:eps_diff_k}
\end{figure*}

In this section, we introduce the simulation experiments conducted on the DPBloomfilter.
In Section~\ref{sec:exp:setup}, we introduce the basic setup of our experiments and restate basic definitions of three kinds of error.
In Section~\ref{sec:exp:main_result}, we discuss the results of our experiments, which align with our theoretical analysis. 

\subsection{Experiments Setup and Basic Notations} \label{sec:exp:setup}


Recall that we have the following notations. 
Let $m$ denote the length of the bit array in the DPBloomfilter.
Let $|A|$ denote the number of elements inserted into the DPBloomfilter. 
Let $k$ denote the number of hash functions used in the DPBloomfilter.
Let $\epsilon, \delta$ denote the differential privacy parameters of the DPBloomfilter. 
Let $N$ denotes the $1 - \delta$ quantile of $W$ (see Definition~\ref{def:W}), and the close-form of the distribution of $W$ is shown in Lemma~\ref{lem:distribution_of_W}. 
Let $\epsilon_0 = \epsilon / N$. By Theorem~\ref{thm:query_privacy:informal}, we choose $\epsilon_0$ in this way can guarantee to $(\epsilon, \delta)$-DP in the whole algorithm. 
Unless specified, we adopt $m = 2^{19}, |A| = 10^5, k=8, n = 2^{63} \approx 10^{19}$ in the following experiments. 
We choose this $n$ because this $n$ is the biggest integer that can be represented on our server.

Recall that $[n]$ denotes the universe. 
Let $S$ denote the elements inserted into the DPBloomfilter. 
Let $\ov{S} = [n] \backslash S$ denote the elements not inserted into the DPBloomfilter. Let $\wt{z} \in \{ 0, 1 \}$ denote the answer output by DPBloomfilter. 

We report three kinds of error rates in our experiments. They are the following: 
(1) {\bf total error}, where we randomly choose queries from the universe $[n]$ and report the error rate of our DPBloomfilter;
(2) {\bf false positive error}, where we random choose queries from $\ov{S}$. When the DPBloomfilter outputs $\wt{z} = 1$, this will cause a false positive error; 
(3) {\bf false negative error}, where we random choose queries from $S$. When the DPBloomfilter outputs $\wt{z} = 0$, this will cause a false negative error. 

\subsection{Experiment Results} \label{sec:exp:main_result}

In this section, we conduct experiments based on the setting mentioned in the previous section. Specifically, we run simulation experiments on different $m$, $|A|$, and $k$ to demonstrate the utility of our algorithm under differential privacy guarantees. 

In Figure~\ref{fig:eps_diff_m}, we conduct experiments on different $m$, whereas $m$ increases, the total error rate and false positive error rate decrease accordingly, while the false negative error rate remains constant. 

In Figure~\ref{fig:eps_diff_na}, we also conduct experiments on different $|A|$, whereas $|A|$ increases, the total error rate and false positive error rate increase accordingly. At the same time, the false negative error rate remains constant.
This phenomenon is consistent with our theoretical analysis of the utility of DPBloomfilter (Theorem~\ref{thm:dpbloom_true_accuracy:informal}). Recall that we have $\alpha = \Pr[z=0]$, denoting the probability of an arbitrary query $q \notin A$. 
Since $|A|$ increases, $\alpha$ decreases, the utility guarantee in Theorem~\ref{thm:dpbloom_true_accuracy:informal}, which is consistent with higher error rate in our experiment results. 


In Figure~\ref{fig:eps_diff_k}, we conduct experiments on different $k$ as well, whereas $k$ increases, the total error rate, and false positive error rate decrease, while the false negative error rate increases accordingly. 

Note that in Figure~\ref{fig:eps_diff_m}, Figure~\ref{fig:eps_diff_na}, and Figure~\ref{fig:eps_diff_k}, as $\epsilon$ approaches $0$, the DPBloomfilter gets closer to random guessing. In this case, the false positive error rate converges to $\frac{1}{2^k}$, and the false negative error rate converges to $1 - \frac{1}{2^k}$. This is consistent with our result in Lemma~\ref{lem:random_guess}. 
Also, as $\epsilon$ increases, the three types of error rates in the Bloom filter with differential privacy (DP) approach the error rates observed when DP is not applied. This is consistent with the intuition that when $\epsilon$ increases, there is less privacy. Therefore, the performance approaches the performance of a Bloom filter without any privacy guarantees. 


This paper presents a planning approach for effective and efficient joint motion generation for manipulators to cover a surface, aiming to minimize specific joint space costs.

\textit{Limitations} -- Our work has several limitations that suggest potential directions for future research. First, our method uses a heuristic to accelerate the traditional Joint-GTSP approach. While we provide empirical evidence of its efficiency in producing high-quality solutions, we cannot guarantee consistent performance in all scenarios.
Second, our bi-level hierarchical method reduces the size of GTSP. Future research could extend it to multiple levels to further improve performance, though this may produce misleading guide paths.
Third, we observe that both Joint-GTSP and H-Joint-GTSP tend to generate paths with frequent turns, a pattern also observed in the motions of prior work \cite{kaljaca2020coverage, zhang2024jpmdp}.  Future work should explore strategies to balance joint movements with other objectives such as motion smoothness.

\footnotetext{Visualization tool: \url{https://github.com/uwgraphics/MotionComparator}}
\textit{Implications} -- The hierarchical approach presented in this work enables effective and efficient coverage path planning for robot manipulators. 
This approach is beneficial to applications that require dexterous surface coverage, such as sanding, polishing, wiping, and sensor scanning. 


\section{Conclusion \& Future Work}\label{conclusion}
This work presents XAMBA, the first framework optimizing SSMs on COTS NPUs, removing the need for specialized accelerators. XAMBA mitigates key bottlenecks in SSMs like CumSum, ReduceSum, and activations using ActiBA, CumBA, and ReduBA, transforming sequential operations into parallel computations. These optimizations improve latency, throughput (Tokens/s), and memory efficiency. Future work will extend XAMBA to other models, explore compression, and develop dynamic optimizations for broader hardware platforms.



% This work introduces XAMBA, the first framework to optimize SSMs on COTS NPUs, eliminating the need for specialized hardware accelerators. XAMBA addresses key bottlenecks in SSM execution, including CumSum, ReduceSum, and activation functions, through techniques like ActiBA, CumBA, and ReduBA, which restructure sequential operations into parallel matrix computations. These optimizations reduce latency, enhance throughput, and improve memory efficiency. 
% Experimental results show up to 2.6$\times$ performance improvement on Intel\textregistered\ Core\texttrademark\ Ultra Series 2 AI PC. 
% Future work will extend XAMBA to other models, incorporate compression techniques, and explore dynamic optimization strategies for broader hardware platforms.


% This work presents XAMBA, an optimization framework that enhances the performance of SSMs on NPUs. Unlike transformers, SSMs rely on structured state transitions and implicit recurrence, which introduce sequential dependencies that challenge efficient hardware execution. XAMBA addresses these inefficiencies by introducing CumBA, ReduBA, and ActiBA, which optimize cumulative summation, ReduceSum, and activation functions, respectively, significantly reducing latency and improving throughput. By restructuring sequential computations into parallelizable matrix operations and leveraging specialized hardware acceleration, XAMBA enables efficient execution of SSMs on NPUs. Future work will extend XAMBA to other state-space models, integrate advanced compression techniques like pruning and quantization, and explore dynamic optimization strategies to further enhance performance across various hardware platforms and frameworks.
% This work presents XAMBA, an optimization framework that enhances the performance of SSMs on NPUs. Key techniques, including CumBA, ReduBA, and ActiBA, achieve significant latency reductions by optimizing operations like cumulative summation, ReduceSum, and activation functions. Future work will focus on extending XAMBA to other state-space models, integrating advanced compression techniques, and exploring dynamic optimization strategies to further improve performance across various hardware platforms and frameworks.

% This work introduces XAMBA, an optimization framework for improving the performance of Mamba-2 and Mamba models on NPUs. XAMBA includes three key techniques: CumBA, ReduBA, and ActiBA. CumBA reduces latency by transforming cumulative summation operations into matrix multiplication using precomputed masks. ReduBA optimizes the ReduceSum operation through matrix-vector multiplication, reducing execution time. ActiBA accelerates activation functions like Swish and Softplus by mapping them to specialized hardware during the DPU’s drain phase, avoiding sequential execution bottlenecks. Additionally, XAMBA enhances memory efficiency by reducing SRAM access, increasing data reuse, and utilizing Zero Value Compression (ZVC) for masks. The framework provides significant latency reductions, with CumBA, ReduBA, and ActiBA achieving up to 1.8X, 1.1X, and 2.6X reductions, respectively, compared to the baseline.
% Future work includes extending XAMBA to other state-space models (SSMs) and exploring further hardware optimizations for emerging NPUs. Additionally, integrating advanced compression techniques like pruning and quantization, and developing adaptive strategies for dynamic optimization, could enhance performance. Expanding XAMBA's compatibility with other frameworks and deployment environments will ensure broader adoption across various hardware platforms.
% \section*{Ethical Considerations}

In the era of big data, many carefully designed data structures are used in various scenarios, and they usually contain a large amount of sensitive user privacy information. 
Some malicious attackers will restore the user information contained in the published data structure, causing a bad social impact. 
Based on this problem, this work uses the Bloom filter data structure as an example to explore the possibility of protecting sensitive information in the Bloom filter through the properties of differential privacy.

Since these data structures often contain subtle structures, naively applying classical differential mechanisms like Gaussian or Laplace mechanisms on them will have a greater impact on the utility of the data structure.
Therefore, designing differential privacy on such data structures requires more effort and exploration. Our work is just the first step in this direction, and we have made preliminary explorations in protecting sensitive data in data structures in the context of big data.




\ifdefined\isarxiv
\section*{Acknowledgement}
Research is partially supported by the National Science Foundation (NSF) Grants 2023239-DMS, CCF-2046710, and Air Force Grant FA9550-18-1-0166.
\else

% \bibliographystyle{ACM-Reference-Format}

% \bibliographystyle{alpha}
\bibliographystyle{plain}
% \bibliographystyle{plainnat}

\bibliography{ref}
% \bibliographystyle{plain}
\fi

\newpage
\onecolumn
% Zhizhou: if added the \appendix command, the all the \section will disappear in Appendix, only left \subsection. So I comment \appendix off. 
% \appendix

% \begin{center}
%     \textbf{\LARGE Appendix }
% \end{center}

%%%% Cut-line between first 10 pages and appendix





%%% some writing rules

%% Writing rule for creating tags.
%% Tags :
%% Theorem    \ref{thm:bla_bla}
%% Lemma      \ref{lem:bla_bla}
%% Claim      \ref{cla:bla_bla}
%% Corollary  \ref{cor:bla_bla}
%% Fact       \ref{fac:bla_bla}
%% Definition \ref{def:bla_bla}
%% Section    \ref{sec:bla_bla}
%% Subsection \ref{sub:bla_bla}
%% Equation   \ref{eq:bla_bla}

\ifdefined\isarxiv
%\section*{Acknowledgments}
\bibliographystyle{alpha}
\bibliography{ref}
\fi

\end{document}



%%%%%%%%%%%%%%%%%%%%%%%%%%%%%%%%%%%%%%%%%%%%%%%%%%%%%%%%%%%%%%%%%%%%%%%%%%%%%%%%%%%%%%%%%%%%%%%%%%%%%%%%%%%%%%%%%%%%%%%%%%%%%%%%%%%%%%%%%%%%%%%%%%%%%%%%%%%%%%%%%%%%%%%%%%%%%%%%%%%%%%%%%%%%%%%%%%%%%%%%%%%%%%%%%%%%%%%%%%%%%%%%%%%%%%%%%%%%%%%%%%%%%%%%%%%%%%%%%%%%%%%%%%%%%%%%%%%%%%%%%%%%%%%%%%%%%%%%%%%%%%%%%%%%%%%%%%%%%%%%%%%%%%%%%%%%%%%%%%%%%%%%%%%%%%%%%%%%%%%%%%%%%%%%%%%%%%%%%%%%%%%%%%%%%%%%%%%%%%%%%%%%%%%%%%%%%%%%%%%%%%%%%%%%%%%%%%%%%%%%%%%%%%%%%%%%%%%%%%%%%%
