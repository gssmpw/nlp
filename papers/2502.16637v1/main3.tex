
%% bare_jrnl_compsoc.tex
%% V1.4b
%% 2015/08/26
%% by Michael Shell
%% See:
%% http://www.michaelshell.org/
%% for current contact information.
%%\cdots
%% This is a skeleton file demonstr\appendixating the use of IEEEtran.cls
%% (requires IEEEtran.cls version 1.8b or later) with an IEEE
%% Computer Society journal paper.
%%
%% Support sites:
%% http://www.michaelshell.org/tex/ieeetran/
%% http://www.ctan.org/pkg/ieeetran
%% and
%% http://www.ieee.org/

%%*************************************************************************
%% Legal Notice:
%% This code is offered as-is without any warranty either expressed or
%% implied; without even the implied warranty of MERCHANTABILITY or
%% FITNESS FOR A PARTICULAR PURPOSE! 
%% User assumes all risk.
%% In no event shall the IEEE or any contributor to this code be liable for
%% any damages or losses, including, but not limited to, incidental,
%% consequential, or any other damages, resulting from the use or misuse
%% of any information contained here.
%%
%% All comments are the opinions of their respective authors and are not
%% necessarily endorsed by the IEEE.
%%
%% This work is distributed under the LaTeX Project Public License (LPPL)
%% ( http://www.latex-project.org/ ) version 1.3, and may be freely used,
%% distributed and modified. A copy of the LPPL, version 1.3, is included
%% in the base LaTeX documentation of all distributions of LaTeX released
%% 2003/12/01 or later.
%% Retain all contribution notices and credits.
%% ** Modified files should be clearly indicated as such, including  **
%% ** renaming them and changing author support contact information. **
%%*************************************************************************


% *** Authors should verify (and, if needed, correct) their LaTeX system  ***
% *** with the testflow diagnostic prior to trusting their LaTeX platform ***
% *** with production work. The IEEE's font choices and paper sizes can   ***
% *** trigger bugs that do not appear when using other class files.       ***                          ***
% The testflow support page is at:
% http://www.michaelshell.org/tex/testflow/


\documentclass[10pt,journal,compsoc]{IEEEtran}

%%%%% NEW MATH DEFINITIONS %%%%%

\usepackage{amsmath,amsfonts,bm}
\usepackage{derivative}
% Mark sections of captions for referring to divisions of figures
\newcommand{\figleft}{{\em (Left)}}
\newcommand{\figcenter}{{\em (Center)}}
\newcommand{\figright}{{\em (Right)}}
\newcommand{\figtop}{{\em (Top)}}
\newcommand{\figbottom}{{\em (Bottom)}}
\newcommand{\captiona}{{\em (a)}}
\newcommand{\captionb}{{\em (b)}}
\newcommand{\captionc}{{\em (c)}}
\newcommand{\captiond}{{\em (d)}}

% Highlight a newly defined term
\newcommand{\newterm}[1]{{\bf #1}}

% Derivative d 
\newcommand{\deriv}{{\mathrm{d}}}

% Figure reference, lower-case.
\def\figref#1{figure~\ref{#1}}
% Figure reference, capital. For start of sentence
\def\Figref#1{Figure~\ref{#1}}
\def\twofigref#1#2{figures \ref{#1} and \ref{#2}}
\def\quadfigref#1#2#3#4{figures \ref{#1}, \ref{#2}, \ref{#3} and \ref{#4}}
% Section reference, lower-case.
\def\secref#1{section~\ref{#1}}
% Section reference, capital.
\def\Secref#1{Section~\ref{#1}}
% Reference to two sections.
\def\twosecrefs#1#2{sections \ref{#1} and \ref{#2}}
% Reference to three sections.
\def\secrefs#1#2#3{sections \ref{#1}, \ref{#2} and \ref{#3}}
% Reference to an equation, lower-case.
\def\eqref#1{equation~\ref{#1}}
% Reference to an equation, upper case
\def\Eqref#1{Equation~\ref{#1}}
% A raw reference to an equation---avoid using if possible
\def\plaineqref#1{\ref{#1}}
% Reference to a chapter, lower-case.
\def\chapref#1{chapter~\ref{#1}}
% Reference to an equation, upper case.
\def\Chapref#1{Chapter~\ref{#1}}
% Reference to a range of chapters
\def\rangechapref#1#2{chapters\ref{#1}--\ref{#2}}
% Reference to an algorithm, lower-case.
\def\algref#1{algorithm~\ref{#1}}
% Reference to an algorithm, upper case.
\def\Algref#1{Algorithm~\ref{#1}}
\def\twoalgref#1#2{algorithms \ref{#1} and \ref{#2}}
\def\Twoalgref#1#2{Algorithms \ref{#1} and \ref{#2}}
% Reference to a part, lower case
\def\partref#1{part~\ref{#1}}
% Reference to a part, upper case
\def\Partref#1{Part~\ref{#1}}
\def\twopartref#1#2{parts \ref{#1} and \ref{#2}}

\def\ceil#1{\lceil #1 \rceil}
\def\floor#1{\lfloor #1 \rfloor}
\def\1{\bm{1}}
\newcommand{\train}{\mathcal{D}}
\newcommand{\valid}{\mathcal{D_{\mathrm{valid}}}}
\newcommand{\test}{\mathcal{D_{\mathrm{test}}}}

\def\eps{{\epsilon}}


% Random variables
\def\reta{{\textnormal{$\eta$}}}
\def\ra{{\textnormal{a}}}
\def\rb{{\textnormal{b}}}
\def\rc{{\textnormal{c}}}
\def\rd{{\textnormal{d}}}
\def\re{{\textnormal{e}}}
\def\rf{{\textnormal{f}}}
\def\rg{{\textnormal{g}}}
\def\rh{{\textnormal{h}}}
\def\ri{{\textnormal{i}}}
\def\rj{{\textnormal{j}}}
\def\rk{{\textnormal{k}}}
\def\rl{{\textnormal{l}}}
% rm is already a command, just don't name any random variables m
\def\rn{{\textnormal{n}}}
\def\ro{{\textnormal{o}}}
\def\rp{{\textnormal{p}}}
\def\rq{{\textnormal{q}}}
\def\rr{{\textnormal{r}}}
\def\rs{{\textnormal{s}}}
\def\rt{{\textnormal{t}}}
\def\ru{{\textnormal{u}}}
\def\rv{{\textnormal{v}}}
\def\rw{{\textnormal{w}}}
\def\rx{{\textnormal{x}}}
\def\ry{{\textnormal{y}}}
\def\rz{{\textnormal{z}}}

% Random vectors
\def\rvepsilon{{\mathbf{\epsilon}}}
\def\rvphi{{\mathbf{\phi}}}
\def\rvtheta{{\mathbf{\theta}}}
\def\rva{{\mathbf{a}}}
\def\rvb{{\mathbf{b}}}
\def\rvc{{\mathbf{c}}}
\def\rvd{{\mathbf{d}}}
\def\rve{{\mathbf{e}}}
\def\rvf{{\mathbf{f}}}
\def\rvg{{\mathbf{g}}}
\def\rvh{{\mathbf{h}}}
\def\rvu{{\mathbf{i}}}
\def\rvj{{\mathbf{j}}}
\def\rvk{{\mathbf{k}}}
\def\rvl{{\mathbf{l}}}
\def\rvm{{\mathbf{m}}}
\def\rvn{{\mathbf{n}}}
\def\rvo{{\mathbf{o}}}
\def\rvp{{\mathbf{p}}}
\def\rvq{{\mathbf{q}}}
\def\rvr{{\mathbf{r}}}
\def\rvs{{\mathbf{s}}}
\def\rvt{{\mathbf{t}}}
\def\rvu{{\mathbf{u}}}
\def\rvv{{\mathbf{v}}}
\def\rvw{{\mathbf{w}}}
\def\rvx{{\mathbf{x}}}
\def\rvy{{\mathbf{y}}}
\def\rvz{{\mathbf{z}}}

% Elements of random vectors
\def\erva{{\textnormal{a}}}
\def\ervb{{\textnormal{b}}}
\def\ervc{{\textnormal{c}}}
\def\ervd{{\textnormal{d}}}
\def\erve{{\textnormal{e}}}
\def\ervf{{\textnormal{f}}}
\def\ervg{{\textnormal{g}}}
\def\ervh{{\textnormal{h}}}
\def\ervi{{\textnormal{i}}}
\def\ervj{{\textnormal{j}}}
\def\ervk{{\textnormal{k}}}
\def\ervl{{\textnormal{l}}}
\def\ervm{{\textnormal{m}}}
\def\ervn{{\textnormal{n}}}
\def\ervo{{\textnormal{o}}}
\def\ervp{{\textnormal{p}}}
\def\ervq{{\textnormal{q}}}
\def\ervr{{\textnormal{r}}}
\def\ervs{{\textnormal{s}}}
\def\ervt{{\textnormal{t}}}
\def\ervu{{\textnormal{u}}}
\def\ervv{{\textnormal{v}}}
\def\ervw{{\textnormal{w}}}
\def\ervx{{\textnormal{x}}}
\def\ervy{{\textnormal{y}}}
\def\ervz{{\textnormal{z}}}

% Random matrices
\def\rmA{{\mathbf{A}}}
\def\rmB{{\mathbf{B}}}
\def\rmC{{\mathbf{C}}}
\def\rmD{{\mathbf{D}}}
\def\rmE{{\mathbf{E}}}
\def\rmF{{\mathbf{F}}}
\def\rmG{{\mathbf{G}}}
\def\rmH{{\mathbf{H}}}
\def\rmI{{\mathbf{I}}}
\def\rmJ{{\mathbf{J}}}
\def\rmK{{\mathbf{K}}}
\def\rmL{{\mathbf{L}}}
\def\rmM{{\mathbf{M}}}
\def\rmN{{\mathbf{N}}}
\def\rmO{{\mathbf{O}}}
\def\rmP{{\mathbf{P}}}
\def\rmQ{{\mathbf{Q}}}
\def\rmR{{\mathbf{R}}}
\def\rmS{{\mathbf{S}}}
\def\rmT{{\mathbf{T}}}
\def\rmU{{\mathbf{U}}}
\def\rmV{{\mathbf{V}}}
\def\rmW{{\mathbf{W}}}
\def\rmX{{\mathbf{X}}}
\def\rmY{{\mathbf{Y}}}
\def\rmZ{{\mathbf{Z}}}

% Elements of random matrices
\def\ermA{{\textnormal{A}}}
\def\ermB{{\textnormal{B}}}
\def\ermC{{\textnormal{C}}}
\def\ermD{{\textnormal{D}}}
\def\ermE{{\textnormal{E}}}
\def\ermF{{\textnormal{F}}}
\def\ermG{{\textnormal{G}}}
\def\ermH{{\textnormal{H}}}
\def\ermI{{\textnormal{I}}}
\def\ermJ{{\textnormal{J}}}
\def\ermK{{\textnormal{K}}}
\def\ermL{{\textnormal{L}}}
\def\ermM{{\textnormal{M}}}
\def\ermN{{\textnormal{N}}}
\def\ermO{{\textnormal{O}}}
\def\ermP{{\textnormal{P}}}
\def\ermQ{{\textnormal{Q}}}
\def\ermR{{\textnormal{R}}}
\def\ermS{{\textnormal{S}}}
\def\ermT{{\textnormal{T}}}
\def\ermU{{\textnormal{U}}}
\def\ermV{{\textnormal{V}}}
\def\ermW{{\textnormal{W}}}
\def\ermX{{\textnormal{X}}}
\def\ermY{{\textnormal{Y}}}
\def\ermZ{{\textnormal{Z}}}

% Vectors
\def\vzero{{\bm{0}}}
\def\vone{{\bm{1}}}
\def\vmu{{\bm{\mu}}}
\def\vtheta{{\bm{\theta}}}
\def\vphi{{\bm{\phi}}}
\def\va{{\bm{a}}}
\def\vb{{\bm{b}}}
\def\vc{{\bm{c}}}
\def\vd{{\bm{d}}}
\def\ve{{\bm{e}}}
\def\vf{{\bm{f}}}
\def\vg{{\bm{g}}}
\def\vh{{\bm{h}}}
\def\vi{{\bm{i}}}
\def\vj{{\bm{j}}}
\def\vk{{\bm{k}}}
\def\vl{{\bm{l}}}
\def\vm{{\bm{m}}}
\def\vn{{\bm{n}}}
\def\vo{{\bm{o}}}
\def\vp{{\bm{p}}}
\def\vq{{\bm{q}}}
\def\vr{{\bm{r}}}
\def\vs{{\bm{s}}}
\def\vt{{\bm{t}}}
\def\vu{{\bm{u}}}
\def\vv{{\bm{v}}}
\def\vw{{\bm{w}}}
\def\vx{{\bm{x}}}
\def\vy{{\bm{y}}}
\def\vz{{\bm{z}}}

% Elements of vectors
\def\evalpha{{\alpha}}
\def\evbeta{{\beta}}
\def\evepsilon{{\epsilon}}
\def\evlambda{{\lambda}}
\def\evomega{{\omega}}
\def\evmu{{\mu}}
\def\evpsi{{\psi}}
\def\evsigma{{\sigma}}
\def\evtheta{{\theta}}
\def\eva{{a}}
\def\evb{{b}}
\def\evc{{c}}
\def\evd{{d}}
\def\eve{{e}}
\def\evf{{f}}
\def\evg{{g}}
\def\evh{{h}}
\def\evi{{i}}
\def\evj{{j}}
\def\evk{{k}}
\def\evl{{l}}
\def\evm{{m}}
\def\evn{{n}}
\def\evo{{o}}
\def\evp{{p}}
\def\evq{{q}}
\def\evr{{r}}
\def\evs{{s}}
\def\evt{{t}}
\def\evu{{u}}
\def\evv{{v}}
\def\evw{{w}}
\def\evx{{x}}
\def\evy{{y}}
\def\evz{{z}}

% Matrix
\def\mA{{\bm{A}}}
\def\mB{{\bm{B}}}
\def\mC{{\bm{C}}}
\def\mD{{\bm{D}}}
\def\mE{{\bm{E}}}
\def\mF{{\bm{F}}}
\def\mG{{\bm{G}}}
\def\mH{{\bm{H}}}
\def\mI{{\bm{I}}}
\def\mJ{{\bm{J}}}
\def\mK{{\bm{K}}}
\def\mL{{\bm{L}}}
\def\mM{{\bm{M}}}
\def\mN{{\bm{N}}}
\def\mO{{\bm{O}}}
\def\mP{{\bm{P}}}
\def\mQ{{\bm{Q}}}
\def\mR{{\bm{R}}}
\def\mS{{\bm{S}}}
\def\mT{{\bm{T}}}
\def\mU{{\bm{U}}}
\def\mV{{\bm{V}}}
\def\mW{{\bm{W}}}
\def\mX{{\bm{X}}}
\def\mY{{\bm{Y}}}
\def\mZ{{\bm{Z}}}
\def\mBeta{{\bm{\beta}}}
\def\mPhi{{\bm{\Phi}}}
\def\mLambda{{\bm{\Lambda}}}
\def\mSigma{{\bm{\Sigma}}}

% Tensor
\DeclareMathAlphabet{\mathsfit}{\encodingdefault}{\sfdefault}{m}{sl}
\SetMathAlphabet{\mathsfit}{bold}{\encodingdefault}{\sfdefault}{bx}{n}
\newcommand{\tens}[1]{\bm{\mathsfit{#1}}}
\def\tA{{\tens{A}}}
\def\tB{{\tens{B}}}
\def\tC{{\tens{C}}}
\def\tD{{\tens{D}}}
\def\tE{{\tens{E}}}
\def\tF{{\tens{F}}}
\def\tG{{\tens{G}}}
\def\tH{{\tens{H}}}
\def\tI{{\tens{I}}}
\def\tJ{{\tens{J}}}
\def\tK{{\tens{K}}}
\def\tL{{\tens{L}}}
\def\tM{{\tens{M}}}
\def\tN{{\tens{N}}}
\def\tO{{\tens{O}}}
\def\tP{{\tens{P}}}
\def\tQ{{\tens{Q}}}
\def\tR{{\tens{R}}}
\def\tS{{\tens{S}}}
\def\tT{{\tens{T}}}
\def\tU{{\tens{U}}}
\def\tV{{\tens{V}}}
\def\tW{{\tens{W}}}
\def\tX{{\tens{X}}}
\def\tY{{\tens{Y}}}
\def\tZ{{\tens{Z}}}


% Graph
\def\gA{{\mathcal{A}}}
\def\gB{{\mathcal{B}}}
\def\gC{{\mathcal{C}}}
\def\gD{{\mathcal{D}}}
\def\gE{{\mathcal{E}}}
\def\gF{{\mathcal{F}}}
\def\gG{{\mathcal{G}}}
\def\gH{{\mathcal{H}}}
\def\gI{{\mathcal{I}}}
\def\gJ{{\mathcal{J}}}
\def\gK{{\mathcal{K}}}
\def\gL{{\mathcal{L}}}
\def\gM{{\mathcal{M}}}
\def\gN{{\mathcal{N}}}
\def\gO{{\mathcal{O}}}
\def\gP{{\mathcal{P}}}
\def\gQ{{\mathcal{Q}}}
\def\gR{{\mathcal{R}}}
\def\gS{{\mathcal{S}}}
\def\gT{{\mathcal{T}}}
\def\gU{{\mathcal{U}}}
\def\gV{{\mathcal{V}}}
\def\gW{{\mathcal{W}}}
\def\gX{{\mathcal{X}}}
\def\gY{{\mathcal{Y}}}
\def\gZ{{\mathcal{Z}}}

% Sets
\def\sA{{\mathbb{A}}}
\def\sB{{\mathbb{B}}}
\def\sC{{\mathbb{C}}}
\def\sD{{\mathbb{D}}}
% Don't use a set called E, because this would be the same as our symbol
% for expectation.
\def\sF{{\mathbb{F}}}
\def\sG{{\mathbb{G}}}
\def\sH{{\mathbb{H}}}
\def\sI{{\mathbb{I}}}
\def\sJ{{\mathbb{J}}}
\def\sK{{\mathbb{K}}}
\def\sL{{\mathbb{L}}}
\def\sM{{\mathbb{M}}}
\def\sN{{\mathbb{N}}}
\def\sO{{\mathbb{O}}}
\def\sP{{\mathbb{P}}}
\def\sQ{{\mathbb{Q}}}
\def\sR{{\mathbb{R}}}
\def\sS{{\mathbb{S}}}
\def\sT{{\mathbb{T}}}
\def\sU{{\mathbb{U}}}
\def\sV{{\mathbb{V}}}
\def\sW{{\mathbb{W}}}
\def\sX{{\mathbb{X}}}
\def\sY{{\mathbb{Y}}}
\def\sZ{{\mathbb{Z}}}

% Entries of a matrix
\def\emLambda{{\Lambda}}
\def\emA{{A}}
\def\emB{{B}}
\def\emC{{C}}
\def\emD{{D}}
\def\emE{{E}}
\def\emF{{F}}
\def\emG{{G}}
\def\emH{{H}}
\def\emI{{I}}
\def\emJ{{J}}
\def\emK{{K}}
\def\emL{{L}}
\def\emM{{M}}
\def\emN{{N}}
\def\emO{{O}}
\def\emP{{P}}
\def\emQ{{Q}}
\def\emR{{R}}
\def\emS{{S}}
\def\emT{{T}}
\def\emU{{U}}
\def\emV{{V}}
\def\emW{{W}}
\def\emX{{X}}
\def\emY{{Y}}
\def\emZ{{Z}}
\def\emSigma{{\Sigma}}

% entries of a tensor
% Same font as tensor, without \bm wrapper
\newcommand{\etens}[1]{\mathsfit{#1}}
\def\etLambda{{\etens{\Lambda}}}
\def\etA{{\etens{A}}}
\def\etB{{\etens{B}}}
\def\etC{{\etens{C}}}
\def\etD{{\etens{D}}}
\def\etE{{\etens{E}}}
\def\etF{{\etens{F}}}
\def\etG{{\etens{G}}}
\def\etH{{\etens{H}}}
\def\etI{{\etens{I}}}
\def\etJ{{\etens{J}}}
\def\etK{{\etens{K}}}
\def\etL{{\etens{L}}}
\def\etM{{\etens{M}}}
\def\etN{{\etens{N}}}
\def\etO{{\etens{O}}}
\def\etP{{\etens{P}}}
\def\etQ{{\etens{Q}}}
\def\etR{{\etens{R}}}
\def\etS{{\etens{S}}}
\def\etT{{\etens{T}}}
\def\etU{{\etens{U}}}
\def\etV{{\etens{V}}}
\def\etW{{\etens{W}}}
\def\etX{{\etens{X}}}
\def\etY{{\etens{Y}}}
\def\etZ{{\etens{Z}}}

% The true underlying data generating distribution
\newcommand{\pdata}{p_{\rm{data}}}
\newcommand{\ptarget}{p_{\rm{target}}}
\newcommand{\pprior}{p_{\rm{prior}}}
\newcommand{\pbase}{p_{\rm{base}}}
\newcommand{\pref}{p_{\rm{ref}}}

% The empirical distribution defined by the training set
\newcommand{\ptrain}{\hat{p}_{\rm{data}}}
\newcommand{\Ptrain}{\hat{P}_{\rm{data}}}
% The model distribution
\newcommand{\pmodel}{p_{\rm{model}}}
\newcommand{\Pmodel}{P_{\rm{model}}}
\newcommand{\ptildemodel}{\tilde{p}_{\rm{model}}}
% Stochastic autoencoder distributions
\newcommand{\pencode}{p_{\rm{encoder}}}
\newcommand{\pdecode}{p_{\rm{decoder}}}
\newcommand{\precons}{p_{\rm{reconstruct}}}

\newcommand{\laplace}{\mathrm{Laplace}} % Laplace distribution

\newcommand{\E}{\mathbb{E}}
\newcommand{\Ls}{\mathcal{L}}
\newcommand{\R}{\mathbb{R}}
\newcommand{\emp}{\tilde{p}}
\newcommand{\lr}{\alpha}
\newcommand{\reg}{\lambda}
\newcommand{\rect}{\mathrm{rectifier}}
\newcommand{\softmax}{\mathrm{softmax}}
\newcommand{\sigmoid}{\sigma}
\newcommand{\softplus}{\zeta}
\newcommand{\KL}{D_{\mathrm{KL}}}
\newcommand{\Var}{\mathrm{Var}}
\newcommand{\standarderror}{\mathrm{SE}}
\newcommand{\Cov}{\mathrm{Cov}}
% Wolfram Mathworld says $L^2$ is for function spaces and $\ell^2$ is for vectors
% But then they seem to use $L^2$ for vectors throughout the site, and so does
% wikipedia.
\newcommand{\normlzero}{L^0}
\newcommand{\normlone}{L^1}
\newcommand{\normltwo}{L^2}
\newcommand{\normlp}{L^p}
\newcommand{\normmax}{L^\infty}

\newcommand{\parents}{Pa} % See usage in notation.tex. Chosen to match Daphne's book.

\DeclareMathOperator*{\argmax}{arg\,max}
\DeclareMathOperator*{\argmin}{arg\,min}

\DeclareMathOperator{\sign}{sign}
\DeclareMathOperator{\Tr}{Tr}
\let\ab\allowbreak

\usepackage{subfigure}
\usepackage{booktabs}
\usepackage{enumitem}
\usepackage{makecell}
\usepackage{algorithm}
\usepackage{amsfonts, amssymb}
\usepackage{algorithmic}
\usepackage{mathtools}
\usepackage{amsthm}
\usepackage{setspace}
\usepackage{bm}
\usepackage{graphicx}
\usepackage{rotating}
\usepackage{multirow}
\usepackage{amssymb} 
% \usepackage{ulem}
\usepackage{hyperref}
\usepackage{cleveref}
\usepackage[utf8]{inputenc}
\usepackage{ragged2e}
\usepackage[english]{babel}
\usepackage{chngcntr}
\usepackage{color}

\counterwithout{figure}{section}
\counterwithout{table}{section}

\newtheorem{theorem}{Theorem}
\newtheorem{corollary}{Corollary}[theorem]
\newtheorem{prop}[theorem]{Proposition}
\newtheorem{lemma}[theorem]{Lemma}
\newtheorem{definition}{Definition}
\newtheorem{assumption}{Assumption}

\renewcommand{\thefigure}{\arabic{figure}}
\renewcommand{\thetable}{\arabic{table}}


\hypersetup{hidelinks,
	colorlinks=true,
	allcolors=black,
	pdfstartview=Fit,
	breaklinks=true}

% \newtheorem{lem}{Lemma}
% \newtheorem{theorem}{Theorem}
% \newtheorem{corollary}{Corollary}
% \newtheorem{assumption}{Assumption}

% \theoremstyle{plain}
% \newtheorem{thm}{Theorem}[section]
% % \newtheorem{lem}[thm]{Lemma}
% \newtheorem{prop}[thm]{Proposition}
% \newtheorem*{cor}{Corollary}

% \theoremstyle{definition}
% \newtheorem{defn}{Definition}[section]
% \newtheorem{conj}{Conjecture}[section]
% \newtheorem{exmp}{Example}[section]

% \theoremstyle{remark}
% \newtheorem*{rem}{Remark}
% \newtheorem*{note}{Note}

% Some very useful LaTeX packages include:
% (uncomment the ones you want to load)

\usepackage{color}
\newcommand{\syf}[1]{%
  \textcolor{black}{#1}%
}

% *** MISC UTILITY PACKAGES ***
%
%\usepackage{ifpdf}
% Heiko Oberdiek's ifpdf.sty is very useful if you need conditional
% compilation based on whether the output is pdf or dvi.
% usage:
% \ifpdf
%   % pdf code
% \else
%   % dvi code
% \fi
% The latest version of ifpdf.sty can be obtained from:
% http://www.ctan.org/pkg/ifpdf
% Also, note that IEEEtran.cls V1.7 and later provides a builtin
% \ifCLASSINFOpdf conditional that works the same way.
% When switching from latex to pdflatex and vice-versa, the compiler may
% have to be run twice to clear warning/error messages.






% *** CITATION PACKAGES ***
%
\ifCLASSOPTIONcompsoc
  % IEEE Computer Society needs nocompress option
  % requires cite.sty v4.0 or later (November 2003)
  \usepackage[nocompress]{cite}
\else
  % normal IEEE
  \usepackage{cite}
\fi
% cite.sty was written by Donald Arseneau
% V1.6 and later of IEEEtran pre-defines the format of the cite.sty package
% \cite{} output to follow that of the IEEE. Loading the cite package will
% result in citation numbers being automatically sorted and properly
% "compressed/ranged". e.g., [1], [9], [2], [7], [5], [6] without using
% cite.sty will become [1], [2], [5]--[7], [9] using cite.sty. cite.sty's
% \cite will automatically add leading space, if needed. Use cite.sty's
% noadjust option (cite.sty V3.8 and later) if you want to turn this off
% such as if a citation ever needs to be enclosed in parenthesis.
% cite.sty is already installed on most LaTeX systems. Be sure and use
% version 5.0 (2009-03-20) and later if using hyperref.sty.
% The latest version can be obtained at:
% http://www.ctan.org/pkg/cite
% The documentation is contained in the cite.sty file itself.
%
% Note that some packages require special options to format as the Computer
% Society requires. In particular, Computer Society  papers do not use
% compressed citation ranges as is done in typical IEEE papers
% (e.g., [1]-[4]). Instead, they list every citation separately in order
% (e.g., [1], [2], [3], [4]). To get the latter we need to load the cite
% package with the nocompress option which is supported by cite.sty v4.0
% and later. Note also the use of a CLASSOPTION conditional provided by
% IEEEtran.cls V1.7 and later.





% *** GRAPHICS RELATED PACKAGES ***
%
\ifCLASSINFOpdf
  % \usepackage[pdftex]{graphicx}
  % declare the path(s) where your graphic files are
  % \graphicspath{{../pdf/}{../jpeg/}}
  % and their extensions so you won't have to specify these with
  % every instance of \includegraphics
  % \DeclareGraphicsExtensions{.pdf,.jpeg,.png}
\else
  % or other class option (dvipsone, dvipdf, if not using dvips). graphicx
  % will default to the driver specified in the system graphics.cfg if no
  % driver is specified.
  % \usepackage[dvips]{graphicx}
  % declare the path(s) where your graphic files are
  % \graphicspath{{../eps/}}
  % and their extensions so you won't have to specify these with
  % every instance of \includegraphics
  % \DeclareGraphicsExtensions{.eps}
\fi
% graphicx was written by David Carlisle and Sebastian Rahtz. It is
% required if you want graphics, photos, etc. graphicx.sty is already
% installed on most LaTeX systems. The latest version and documentation
% can be obtained at: 
% http://www.ctan.org/pkg/graphicx
% Another good source of documentation is "Using Imported Graphics in
% LaTeX2e" by Keith Reckdahl which can be found at:
% http://www.ctan.org/pkg/epslatex
%
% latex, and pdflatex in dvi mode, support graphics in encapsulated
% postscript (.eps) format. pdflatex in pdf mode supports graphics
% in .pdf, .jpeg, .png and .mps (metapost) formats. Users should ensure
% that all non-photo figures use a vector format (.eps, .pdf, .mps) and
% not a bitmapped formats (.jpeg, .png). The IEEE frowns on bitmapped formats
% which can result in "jaggedy"/blurry rendering of lines and letters as
% well as large increases in file sizes.
%
% You can find documentation about the pdfTeX application at:
% http://www.tug.org/applications/pdftex






% *** MATH PACKAGES ***
%
%\usepackage{amsmath}
% A popular package from the American Mathematical Society that provides
% many useful and powerful commands for dealing with mathematics.
%
% Note that the amsmath package sets \interdisplaylinepenalty to 10000
% thus preventing page breaks from occurring within multiline equations. Use:
%\interdisplaylinepenalty=2500
% after loading amsmath to restore such page breaks as IEEEtran.cls normally
% does. amsmath.sty is already installed on most LaTeX systems. The latest
% version and documentation can be obtained at:
% http://www.ctan.org/pkg/amsmath





% *** SPECIALIZED LIST PACKAGES ***
%
%\usepackage{algorithmic}
% algorithmic.sty was written by Peter Williams and Rogerio Brito.
% This package provides an algorithmic environment fo describing algorithms.
% You can use the algorithmic environment in-text or within a figure
% environment to provide for a floating algorithm. Do NOT use the algorithm
% floating environment provided by algorithm.sty (by the same authors) or
% algorithm2e.sty (by Christophe Fiorio) as the IEEE does not use dedicated
% algorithm float types and packages that provide these will not provide
% correct IEEE style captions. The latest version and documentation of
% algorithmic.sty can be obtained at:
% http://www.ctan.org/pkg/algorithms
% Also of interest may be the (relatively newer and more customizable)
% algorithmicx.sty package by Szasz Janos:
% http://www.ctan.org/pkg/algorithmicx




% *** ALIGNMENT PACKAGES ***
%
%\usepackage{array}
% Frank Mittelbach's and David Carlisle's array.sty patches and improves
% the standard LaTeX2e array and tabular environments to provide better
% appearance and additional user controls. As the default LaTeX2e table
% generation code is lacking to the point of almost being broken with
% respect to the quality of the end results, all users are strongly
% advised to use an enhanced (at the very least that provided by array.sty)
% set of table tools. array.sty is already installed on most systems. The
% latest version and documentation can be obtained at:
% http://www.ctan.org/pkg/array


% IEEEtran contains the IEEEeqnarray family of commands that can be used to
% generate multiline equations as well as matrices, tables, etc., of high
% quality.




% *** SUBFIGURE PACKAGES ***
%\ifCLASSOPTIONcompsoc
%  \usepackage[caption=false,font=footnotesize,labelfont=sf,textfont=sf]{subfig}
%\else
%  \usepackage[caption=false,font=footnotesize]{subfig}
%\fi
% subfig.sty, written by Steven Douglas Cochran, is the modern replacement
% for subfigure.sty, the latter of which is no longer maintained and is
% incompatible with some LaTeX packages including fixltx2e. However,
% subfig.sty requires and automatically loads Axel Sommerfeldt's caption.sty
% which will override IEEEtran.cls' handling of captions and this will result
% in non-IEEE style figure/table captions. To prevent this problem, be sure
% and invoke subfig.sty's "caption=false" package option (available since
% subfig.sty version 1.3, 2005/06/28) as this is will preserve IEEEtran.cls
% handling of captions.
% Note that the Computer Society format requires a sans serif font rather
% than the serif font used in traditional IEEE formatting and thus the need
% to invoke different subfig.sty package options depending on whether
% compsoc mode has been enabled.
%
% The latest version and documentation of subfig.sty can be obtained at:
% http://www.ctan.org/pkg/subfig




% *** FLOAT PACKAGES ***
%
%\usepackage{fixltx2e}
% fixltx2e, the successor to the earlier fix2col.sty, was written by
% Frank Mittelbach and David Carlisle. This package corrects a few problems
% in the LaTeX2e kernel, the most notable of which is that in current
% LaTeX2e releases, the ordering of single and double column floats is not
% guaranteed to be preserved. Thus, an unpatched LaTeX2e can allow a
% single column figure to be placed prior to an earlier double column
% figure.
% Be aware that LaTeX2e kernels dated 2015 and later have fixltx2e.sty's
% corrections already built into the system in which case a warning will
% be issued if an attempt is made to load fixltx2e.sty as it is no longer
% needed.
% The latest version and documentation can be found at:
% http://www.ctan.org/pkg/fixltx2e


%\usepackage{stfloats}
% stfloats.sty was written by Sigitas Tolusis. This package gives LaTeX2e
% the ability to do double column floats at the bottom of the page as well
% as the top. (e.g., "\begin{figure*}[!b]" is not normally possible in
% LaTeX2e). It also provides a command:
%\fnbelowfloat
% to enable the placement of footnotes below bottom floats (the standard
% LaTeX2e kernel puts them above bottom floats). This is an invasive package
% which rewrites many portions of the LaTeX2e float routines. It may not work
% with other packages that modify the LaTeX2e float routines. The latest
% version and documentation can be obtained at:
% http://www.ctan.org/pkg/stfloats
% Do not use the stfloats baselinefloat ability as the IEEE does not allow
% \baselineskip to stretch. Authors submitting work to the IEEE should note
% that the IEEE rarely uses double column equations and that authors should try
% to avoid such use. Do not be tempted to use the cuted.sty or midfloat.sty
% packages (also by Sigitas Tolusis) as the IEEE does not format its papers in
% such ways.
% Do not attempt to use stfloats with fixltx2e as they are incompatible.
% Instead, use Morten Hogholm'a dblfloatfix which combines the features
% of both fixltx2e and stfloats:
%
% \usepackage{dblfloatfix}
% The latest version can be found at:
% http://www.ctan.org/pkg/dblfloatfix




%\ifCLASSOPTIONcaptionsoff
%  \usepackage[nomarkers]{endfloat}
% \let\MYoriglatexcaption\caption
% \renewcommand{\caption}[2][\relax]{\MYoriglatexcaption[#2]{#2}}
%\fi
% endfloat.sty was written by James Darrell McCauley, Jeff Goldberg and 
% Axel Sommerfeldt. This package may be useful when used in conjunction with 
% IEEEtran.cls'  captionsoff option. Some IEEE journals/societies require that
% submissions have lists of figures/tables at the end of the paper and that
% figures/tables without any captions are placed on a page by themselves at
% the end of the document. If needed, the draftcls IEEEtran class option or
% \CLASSINPUTbaselinestretch interface can be used to increase the line
% spacing as well. Be sure and use the nomarkers option of endfloat to
% prevent endfloat from "marking" where the figures would have been placed
% in the text. The two hack lines of code above are a slight modification of
% that suggested by in the endfloat docs (section 8.4.1) to ensure that
% the full captions always appear in the list of figures/tables - even if
% the user used the short optional argument of \caption[]{}.
% IEEE papers do not typically make use of \caption[]'s optional argument,
% so this should not be an issue. A similar trick can be used to disable
% captions of packages such as subfig.sty that lack options to turn off
% the subcaptions:
% For subfig.sty:
% \let\MYorigsubfloat\subfloat
% \renewcommand{\subfloat}[2][\relax]{\MYorigsubfloat[]{#2}}
% However, the above trick will not work if both optional arguments of
% the \subfloat command are used. Furthermore, there needs to be a
% description of each subfigure *somewhere* and endfloat does not add
% subfigure captions to its list of figures. Thus, the best approach is to
% avoid the use of subfigure captions (many IEEE journals avoid them anyway)
% and instead reference/explain all the subfigures within the main caption.
% The latest version of endfloat.sty and its documentation can obtained at:
% http://www.ctan.org/pkg/endfloat
%
% The IEEEtran \ifCLASSOPTIONcaptionsoff conditional can also be used
% later in the document, say, to conditionally put the References on a 
% page by themselves.




% *** PDF, URL AND HYPERLINK PACKAGES ***
%
%\usepackage{url}
% url.sty was written by Donald Arseneau. It provides better support for
% handling and breaking URLs. url.sty is already installed on most LaTeX
% systems. The latest version and documentation can be obtained at:
% http://www.ctan.org/pkg/url
% Basically, \url{my_url_here}.





% *** Do not adjust lengths that control margins, column widths, etc. ***
% *** Do not use packages that alter fonts (such as pslatex).         ***
% There should be no need to do such things with IEEEtran.cls V1.6 and later.
% (Unless specifically asked to do so by the journal or conference you plan
% to submit to, of course. )


% correct bad hyphenation here
\hyphenation{op-tical net-works semi-conduc-tor}


\begin{document}
\renewcommand\arraystretch{0.5}
%
% paper title
% Titles are generally capitalized except for words such as a, an, and, as,
% at, but, by, for, in, nor, of, on, or, the, to and up, which are usually
% not capitalized unless they are the first or last word of the title.
% Linebreaks \\ can be used within to get better formatting as desired.
% Do not put math or special symbols in the title.
\title{Time Series Domain Adaptation via Latent Invariant Causal Mechanism}
%
%
% author names and IEEE memberships
% note positions of commas and nonbreaking spaces ( ~ ) LaTeX will not break
% a structure at a ~ so this keeps an author's name from being broken across
% two lines.
% use \thanks{} to gain access to the first footnote area
% a separate \thanks must be used for each paragraph as LaTeX2e's \thanks
% was not built to handle multiple paragraphs
%
%
%\IEEEcompsocitemizethanks is a special \thanks that produces the bulleted
% lists the Computer Society journals use for "first footnote" author
% affiliations. Use \IEEEcompsocthanksitem which works much like \item
% for each affiliation group. When not in compsoc mode,
% \IEEEcompsocitemizethanks becomes like \thanks and
% \IEEEcompsocthanksitem becomes a line break with idention. This
% facilitates dual compilation, although admittedly the differences in the
% desired content of \author between the different types of papers makes a
% one-size-fits-all approach a daunting prospect. For instance, compsoc 
% journal papers have the author affiliations above the "Manuscript
% received ..."  text while in non-compsoc journals this is reversed. Sigh.

\author{Ruichu Cai,
        Junxian Huang,
        Zhenhui Yang,
        Zijian Li,
        Emadeldeen Eldele,
        Min Wu,
        Fuchun Sun, \textit{Fellow, IEEE}
        % <-this % stops a space
\IEEEcompsocitemizethanks{
\IEEEcompsocthanksitem Ruichu Cai is with the School of Computer Science, Guangdong University of Technology, Guangzhou, China, 510006 and Peng Cheng Laboratory, Shenzhen, China, 518066.
Email: cairuichu@gmail.com\protect
\IEEEcompsocthanksitem Junxian Huang is with the School of Computer Science, Guangdong University of Technology, Guangzhou China, 510006.\protect Email: huangjunxian459@gmail.com\protect
\IEEEcompsocthanksitem
Zhenhui Yang is with the School of Computing, Guangdong University of Technology, Guangzhou China, 510006.\protect
% note need leading \protect in front of \\ to get a newline within \thanks as
% \\ is fragile and will error, could use \hfil\break instead.
E-mail: yangzhenhui8@gmail.com
\IEEEcompsocthanksitem Zijian Li is with the Machine Learning Department, Mohamed bin Zayed University of Artificial Intelligence, Abu Dhabi.\protect Email: leizigin@gmail.com\protect

\IEEEcompsocthanksitem Emadeldeen Eldele is with the Institute for Infocomm Research, A*STAR, Singapore, Centre for Frontier AI Research, A*STAR, Singapore \protect Email: emad0002@ntu.edu.sg\protect
% \IEEEcompsocthanksitem Zhifeng Hao is with the College of Science, Shantou University, Shantou, Guangdong, 515063\protect
\IEEEcompsocthanksitem Min Wu is with the Institute for Infocomm Research, A*STAR, Singapore. E-mail: wumin@i2r.a-star.edu.sg\protect
\IEEEcompsocthanksitem Fuchun Sun is with the Department of Computer Science and Technology, Tsinghua University, Beijing, China. E-mail: fcsun@tsinghua.edu.cn\protect
}% <-this % stops an unwanted space
\thanks{This research was supported in part by the National Key R\&D Program of China (2021ZD0111501), National Science Fund for Excellent Young Scholars (62122022), Natural Science Foundation of China (61876043, 61976052), the major key project of PCL (PCL2021A12). (Min Wu is the Corresponding author.)}}

% note the % following the last \IEEEmembership and also \thanks - 
% these prevent an unwanted space from occurring between the last author name
% and the end of the author line. i.e., if you had this:
% 
% \author{....lastname \thanks{...} \thanks{...} }
%                     ^------------^------------^----Do not want these spaces!
%
% a space would be appended to the last name and could cause every name on that
% line to be shifted left slightly. This is one of those "LaTeX things". For
% instance, "\textbf{A} \textbf{B}" will typeset as "A B" not "AB". To get
% "AB" then you have to do: "\textbf{A}\textbf{B}"
% \thanks is no different in this regard, so shield the last } of each \thanks
% that ends a line with a % and do not let a space in before the next \thanks.
% Spaces after \IEEEmembership other than the last one are OK (and needed) as
% you are supposed to have spaces between the names. For what it is worth,
% this is a minor point as most people would not even notice if the said evil
% space somehow managed to creep in.



% The paper headers
\markboth{Journal of \LaTeX\ Class Files,~Vol.~14, No.~8, August~2015}%
{Shell \MakeLowercase{\textit{et al.}}: Bare Demo of IEEEtran.cls for Computer Society Journals}
% The only time the second header will appear is for the odd numbered pages
% after the title page when using the twoside option.
% 
% *** Note that you probably will NOT want to include the author's ***
% *** name in the headers of peer review papers.                   ***
% You can use \ifCLASSOPTIONpeerreview for conditional compilation here if
% you desire.



% The publisher's ID mark at the bottom of the page is less important with
% Computer Society journal papers as those publications place the marks
% outside of the main text columns and, therefore, unlike regular IEEE
% journals, the available text space is not reduced by their presence.
% If you want to put a publisher's ID mark on the page you can do it like
% this:
%\IEEEpubid{0000--0000/00\$00.00~\copyright~2015 IEEE}
% or like this to get the Computer Society new two part style.
%\IEEEpubid{\makebox[\columnwidth]{\hfill 0000--0000/00/\$00.00~\copyright~2015 IEEE}%
%\hspace{\columnsep}\makebox[\columnwidth]{Published by the IEEE Computer Society\hfill}}
% Remember, if you use this you must call \IEEEpubidadjcol in the second
% column for its text to clear the IEEEpubid mark (Computer Society jorunal
% papers don't need this extra clearance.)



% use for special paper notices
%\IEEEspecialpapernotice{(Invited Paper)}



% for Computer Society papers, we must declare the abstract and index terms
% PRIOR to the title within the \IEEEtitleabstractindextext IEEEtran
% command as these need to go into the title area created by \maketitle.
% As a general rule, do not put math, special symbols or citations
% in the abstract or keywords.
\IEEEtitleabstractindextext{%
\begin{abstract}\justifying
% Time series domain adaptation methods aim to transfer discriminative patterns from the labeled source domain to the unlabeled target domain. Although recent methods achieve empirically ideal transportation results by enforcing the alignment of causal structures across different domains, they are not suitable for the high-dimension data e.g., video data, which do not have direct causal edges on the observation level. Moreover, existing methods based on causality are heuristic and lack
% a proper theoretical foundation. To address these challenges, we propose a general invariant latent causal mechanism learning framework for time series domain adaptation. Specifically, we assume that the causal relationships occur among latent variables instead of observed variables. Moreover, we further theoretically show that the learned causal mechanisms are uniquely identified. Based on these theoretical results, we develop a Latent Causality Alignment (\textbf{LCA}) model that leverages variational inference. Furthermore, the \textbf{LCA} model incorporates flow-based prior networks to extract the latent mechanisms and latent mechanism alignment restriction to reduce the diversity of latent causal structures. Experiment results on eight benchmarks show a general improvement in the domain-adaptive time series classification and forecasting tasks., highlighting the effectiveness of our method in real-world scenarios.
Time series domain adaptation aims to transfer the complex temporal dependence from the labeled source domain to the unlabeled target domain. Recent advances leverage the stable causal mechanism over observed variables to model the domain-invariant temporal dependence. However, modeling precise causal structures in high-dimensional data, such as videos, remains challenging. Additionally, direct causal edges may not exist among observed variables (e.g., pixels). These limitations hinder the \textcolor{black}{applicability} of existing approaches to real-world scenarios. To address these challenges, we find that the high-dimension time series data are generated from the low-dimension latent variables, which motivates us to model the causal mechanisms of the temporal latent process. Based on this intuition, we propose a latent causal mechanism identification framework that guarantees the uniqueness of the reconstructed latent causal structures. Specifically, we first identify latent variables by utilizing sufficient changes in historical information. Moreover, by enforcing the sparsity of the relationships of latent variables, we can achieve identifiable latent causal structures. Built on the theoretical results, we develop the Latent Causality Alignment (\textbf{LCA}) model that leverages variational inference, which incorporates an intra-domain latent sparsity constraint for latent structure reconstruction and an inter-domain latent sparsity constraint for domain-invariant structure reconstruction. Experiment results on eight benchmarks show a general improvement in the domain-adaptive time series classification and forecasting tasks, highlighting the effectiveness of our method in real-world scenarios. Codes are available at \url{https://github.com/DMIRLAB-Group/LCA}.

% we establish a general latent causal mechanism learning framework that guarantees unique reconstruction of 

% By leveraging the stable Granger causality, recent advances have proposed the causal conditional shift assumption to model the domain-invariant temporal dependence. 

% Recent advances have proposed to leverage the stable Granger causality to address the 
% 时间序列的da很重要
% 观测变量层面上的因果结构没有意义
% 现有的工作难以学习到高维不变的时序特征
% 现有的工作学到的因果结构是启发性的,并没有理论保证 

% 基于时间序列的领域自适应方法旨在从src到tgt提取领域不变的parttern。虽然现有的方法利用不变因果机制在低维数中取得实验性的好效果,但是


\end{abstract}
% 1. 解决什么问题(DA)
% 2. 难点在哪儿,联合分布不知道,可以说什么是变的
% 3. 解决方法,提出motvation(一句话让别人知道大概怎么做,而且是惊人的想法)花两三句话来详细介绍自己的方法
% 4. 这样做的好处有什么,实验结果,代码等等
% no keywords
% 我的版本
% 解决时间序列预测的DA问题
% 难点在于基于多维时间序列的复杂依赖是根据domain变化而变化的,这个变化主要体现在三个方面:Offset,Time Lag, 和条件分布的变化。
% 为了解决这个问题,我们使用稳定的简洁的格兰杰因果关系来建模这种复杂的domain-specific的依赖关系。这种简洁的格兰杰因果关系不但可以通过避免直接对齐数据值从而绕开offset问题,而且通过考虑不同time lag之间的因果图从而可以很好地刻画条件分布的变和不变的部分。
% 通过这样做,我们提供了一种端对端的时间序列预测迁移模型,它不仅仅能够跨领域学习格兰杰因果关系,而且很好地通过学习不同domain的依赖关系来预测多维时间序列,更能为预测结果提过一定的可解释性。我们进一步从理论上分析了基于格兰杰因果关系对齐的时间序列预测迁移模型的优越性,我们在生成数据集和真实数据集上验证了我们的方法。代码公布xxxx


% Note that keywords are not normally used for peerreview papers.
\begin{IEEEkeywords}
Time Series Data, Latent Causal Mechanism, Domain Adaptation
\end{IEEEkeywords}}


% make the title area
\maketitle


% To allow for easy dual compilation without having to reenter the
% abstract/keywords data, the \IEEEtitleabstractindextext text will
% not be used in maketitle, but will appear (i.e., to be "transported")
% here as \IEEEdisplaynontitleabstractindextext when the compsoc 
% or transmag modes are not selected <OR> if conference mode is selected 
% - because all conference papers position the abstract like regular
% papers do.
\IEEEdisplaynontitleabstractindextext
% \IEEEdisplaynontitleabstractindextext has no effect when using
% compsoc or transmag under a non-conference mode.



% For peer review papers, you can put extra information on the cover
% page as needed:
% \ifCLASSOPTIONpeerreview
% \begin{center} \bfseries EDICS Category: 3-BBND \end{center}
% \fi
%
% For peerreview papers, this IEEEtran command inserts a page break and
% creates the second title. It will be ignored for other modes.




\IEEEpeerreviewmaketitle



%\IEEEraisesectionheading{\section{Introduction}\label{sec:introduction}
%\textbf{INTRODUCTION}
% 起,科学的第一要义是泛化,在很多多维时间序列的应用也是如此,例如,人体血糖关系中,有关的血糖,胰岛素,胰高血糖素的关系应该在不同年龄,性别,人中都使用。领域自适应的目的就是利用source domain的数据在target domain上做预测,它的一个核心问题在于解决domain shift.
% 
% 承1,考虑到source domain和target domain中包含共享的可以用来预测的信息,一个常见的假设是covariate shift assumption,它假设p(x)是变化的,但是p(y|x)是固定的。基于这个假设,在非时序领域例如CV已经获得了巨大的成功,例如MMD,adversarial。不少研究者直接将这个假设拓展到时序领域,采用RNN之类的特征提取器。但是由于时间序列的依赖关系是很复杂,即使一阶马尔可夫依赖造成很大不同,所以在时间序列数据上p(y|x)也会随着domain变化,i.e. p_S(x_t+1|\phi(x_1,...x_t)) \neq p_T(x_t+1|\phi(x_1,...x_t)),因此直接在数据层面上对齐很难提取领域不变特征。
% 
% 
% 转,
% 
% 合,基于以上的intuition,我们提出了Granger Causality Alignment (GCA) approach for time-series domain adaptation by assuming the full graph of granger causality is stable across different domains.
% 我们GCA方法主要包含两个不同的挑战:
%1.如何学full graph不同summary相同的格兰杰因果结构。
%2.如何结合因果图产生简介的表达?
%3.如何结合因果图刻画不同domain数据的条件分布?
% 为了解决以上问题,我们首先将不同lag的因果图作为一个隐变量,然后使用在变分自动编码器和离散重参数化的框架上重构格兰杰因果结构。然后我们进一步设计了领域敏感的预测器来预测下一个时间步的结果。我们理论上证明了我们提出方法的优越性。我们的方法不仅仅可以学到不同domain上的格兰杰因果关系,而且在生成和真实数据集上超过了现有sota的方法。
% Science is all about generalizations. Scientific applications of new theorems and discoveries also need to transfer the experiment conclusions from the lab or the virtual environments to real environments. For example, the physiological mechanism in the human body among ``Blood Glucose'', `` Glucagon'' and ``Insulin'' should be held among people of different ages, genders, and even races. Semi-supervised domain adaptation, which leverages the labeled source domain data and a few labeled target domain data to make a prediction in the unlabeled target domain, essentially aims to address the notorious \textit{domain shift} phenomenon and find a robust forecasting model.



% 存在问题:
% 1. 格兰杰因果出现得有点突然
% 2. 应该强调时序的因果结构
% 3. 强调时序的可迁移性质
% 4. 推出SASA仅仅是一个特例,再提及SASA的不足

% GCA的行文思路:
% 泛化问题很重要,要从机器到现实,因此迁移学习也很重要
% 然而现在的主流迁移学习方法主要是假设covariance shift,这些方法取得了很大的成功在非时序数据上,很多场景下不满足,张从因果的角度考虑了很多的场景。
% 最近时序预测迁移学习的方法受到越来越多的关注,然而直接平推不行,因为复杂的依赖关系,举例子,然而考虑到背后的因果机制是一致的。
%然后举出自己的模型
%
% 

%我们的行文思路:
%迁移学习很重要(1,减少数据标签成本,收集成分,2。更好泛化问题 3.减少训练代价)
%跟GCA一样,非时序数据上,时序数据上,然后复杂不行。因此SASA,GCA考虑从因果的角度,然而它们面对高维数据时,不能有效学习到数据的因果关系,图片。因此我们更一般的考虑时序数据的生成。然后

\IEEEraisesectionheading{\section{Introduction}\label{sec:introduction}}
% \begin{figure*}
% 	\centering
%     \label{fig:motivation}
% 	\includegraphics[width=2\columnwidth, trim=0cm 7.5cm 0cm 0cm ,clip]{fig/motivation.pdf}
%     \caption{Domain adaptation methods for time series data with different temporal domains, where \( x \) represents observed data and \( z \) denotes latent variables. (a) Existing methods that reuse the covariate shift assumption consider all relationships, including spurious ones, which can lead to performance degradation. (b) A method addressing domain adaptation in low-dimensional time series data by extracting causal structures and modeling conditional distributions specific to each domain, but struggles with capturing causal mechanisms when observed variables are high-dimensional. (c) Our approach, which starts from the perspective of latent variables, more generally considers the data generation process and effectively extracts causal mechanisms.}
% \end{figure*}


To overcome the challenges of distribution shift between the training and the test time series datasets \cite{pan2009survey}, time series domain adaptation seeks to transfer the temporal dependence from labeled source data to unlabeled target data. Mathematically, in the context of time series domain adaptation, we let $X=(\rvx_1,\cdots,\rvx_{\tau},\cdots,\rvx_t)$ be multivariate time series with $t$ timestamps \textcolor{black}{and a channel size of $n$}, where $\rvx_{\tau} \in \mathbb{R}^n$ and $Y$ is the corresponding label\textcolor{black}{, such that $Y$ can be a time series or scalar, depending on the forecasting or classification tasks.} By considering $\rvu$ as the domain label, we further assume that $P(X, Y|\rvu^S)$ and $P(X, Y|\rvu^T)$ are the source and target distributions, respectively. In the source domain, we can access $m^S$ annotated $X$-$Y$ pairs, represented as $(X^S, Y^S)=(X^S_{i}, Y^S_{i})_{i=1}^{m^S}$. While in the target domain, only $m_T$ unannotated time series data can be observed, denoted by $(X^T)=(X^T_{i})_{i=1}^{m_T}$. The primary goal of unsupervised time series domain adaptation is to model the target joint distribution $P(X,Y|\rvu^T)$ by leveraging the labeled source and the unlabeled target data.

% 将方法分成几类
% 静态方法拓展的
% 基于时序特性的

Several methods have been proposed to address this problem. Previous works have extended the conventional assumptions of domain adaptation for static data \cite{zhang2015multi,cai2019learning} to time series data, including covariate shift \cite{sugiyama2007covariate}, label shift \cite{lipton2018detecting}, and conditional shift \cite{zhang2013domain}. For instance, leveraging the covariate shift assumption, i.e., $P(Y|X)$ is stable, \textcolor{black}{some researchers \cite{da2020remaining,purushotham2022variational}} employ recursive neural network-based architecture as feature extractor and adopt adversarial training strategy to extract domain-invariant information. \textcolor{black}{Others, e.g., Wilson et al.} \cite{wilson2023calda}, utilize adversarial and contrastive learning to extract the domain-invariant representation for time series data. Moreover, other researchers assume that the distribution shift of $Y$ varies across different domains, known as label shift. Specifically, \textcolor{black}{He et al.} \cite{he2023domain} tackle the feature shift and label shift by aligning both temporal and frequency features. Conversely, \textcolor{black}{Hoyez et al. \cite{hoyeztime}} argue that the content shift and style shift in time series data belong to the conditional shift, where $P(X|Y)$ varies with different domains in time series domain adaptation.

% Whereas \cite{hoyeztime} consider the conditional shift, i.e., $P(X|Y)$ varies with different domains in time series domain adaptation.
\begin{figure*}[h]
    \centering
    \includegraphics[width=0.95\textwidth]{fig/lca_motication1.pdf}
    \caption{A toy domain adaptation example for video data, where the blue arrows denote the estimated causal relationships. (a) Two three-frame videos, one featuring a walking human and the other a monster, represent the source and target domains, respectively. (b) Since the skeleton variables are latent confounders and there are no causal relationships among pixels, the existing methods may learn dense and fault causal relationships among adjacent two frames, failing to extract the domain-invariant causal mechanism. (c) By identifying the latent variables like the joints in the skeleton, it is convenient to model the latent invariant causal mechanisms behind observed variables. Hence, we can address the time series domain adaptation problem in complex real-world scenarios.}
    \label{fig:motivation}
\end{figure*}
Instead of introducing assumptions from a statistical perspective, another promising direction is to harness the invariant temporal dependencies for time series domain adaptation. Specifically, Cai et al. \cite{Cai_Chen_Li_Chen_Zhang_Ye_Li_Yang_Zhang_2021} address time series domain adaptation by aligning sparse associate structures across different domains. \textcolor{black}{Li et al. }\cite{LI2024106659} further extend this approach by considering the domain-specific strengths of the domain-invariant sparse associative structures. Recently, Wang et al. \cite{wang2023sensor} proposed reducing distribution shift at both local and global sensor levels by aligning sensor relationships. Moreover, \textcolor{black}{Wang et al.} \cite{wang2024sea++} exploit multi-graph-based higher-order sensor alignment to achieve more robust adaptation. To further utilize domain-invariant temporal causal dependencies, \textcolor{black}{Li et al. \cite{li2023transferable}} proposed leveraging stable causal structures over observed variables, introducing the concept of causal conditional shift.

Although existing methods \cite{li2023transferable} demonstrate effective transfer performance on time series by extracting domain-invariant causal mechanisms, they implicitly assume that the dimension of observed variables is small and that direct causal edges exist among these variables. Specifically, most of the existing methods for Granger causal discovery \cite{9376668,khanna2019economy,marcinkevivcs2021interpretable,li2023transferable} are designed for low-dimension time series data (e.g., $n \leq 20$). However, many real-world time series datasets, such as electricity load diagrams or weather, are high-dimensional, limiting the applicability of these methods. 

Furthermore, direct causal relationships may not usually exist in these high-dimension data. For instance,  Figure \ref{fig:motivation} (a) provides a toy example of three-frame video data, where the walking human and monster denote the source and target domains, respectively. As shown in Figure \ref{fig:motivation} (b), existing methods like \cite{li2023transferable}, that estimate Granger causal structure over observed variables, may generate incorrect and dense causal structures since there are no causal relationships in the pixel level. As a result, these methods fail to extract domain-invariant causal mechanisms, limiting their effectiveness in addressing the time series domain adaptation problem.

% To overcome the aforementioned problems, we propose a latent causality alignment (\textbf{LCA}) framework for time series domain adaptation by assuming that the high-dimension time series data are generated from low-dimension time series data, where the invariant causal mechanisms exist. Intuitively, as shown in Figure \ref{fig:motivation} (c), the videos with the motion of walking from different domains are generated from a domain-invariant skeleton, which motivates us to exploit the causal structures over latent variables for time series domain adaptation. Based on this intuition, we 
% they can hardly extend to complex real-world scenarios. 

To address the aforementioned problems, an intuitive solution is based on the observation that videos depicting walking motion from different domains are generated from a domain-invariant skeleton, as shown in Figure \ref{fig:motivation} (c). This insight motivates us to exploit the causal structures over latent variables for time series domain adaptation. Building on this intuition, we prove that the latent causal process can be uniquely reconstructed, i.e., achieving identifiability, with the aid of nonlinear independent component analysis. 

Guided by these theoretical results, we develop a latent causality alignment model (\textbf{LCA} in short). Specifically, the proposed \textbf{LCA} model employs variational inference to model the time series data from the source and target domains, utilizing a flow-based prior estimator. Moreover, we incorporate a gradient-based sparsity constraint to discover the Granger causal structures on latent variables. Furthermore, we harness the latent causality regularization to restrict the discrepancy of the causal structures from different domains. Our method is validated on eight time-series domain adaptation datasets, including video and high-dimension electricity load diagram datasets. The impressive performance, outperforming state-of-the-art methods, demonstrates the effectiveness of our approach.

%1. 从观测变量的因果结构到隐变量的因果结构对其,首次将因果表征对其方法应用到时间序列的领域自适应中
%2. 隐变量的因果结构提供可识别性
% 建立了一个通用的时间序列领域自适应框架,并且在分类和预测两个下游任务共8个数据集上做验证,实验结果表明有通用的提升


\textcolor{black}{Our contributions can be summarized as follows:}
\begin{itemize}
    \item \textcolor{black}{Breaking out the limitation of causality alignment on observed variables, we develop a latent causal structure alignment method for time-series domain adaptation from the view of causal representation learning. we are the first to employ causal representation learning to time series domain adaptation.}
    \item \textcolor{black}{Different from previous works that leverage causality structure for time series domain adaptation, we provide formal identification guarantees for both the causal representation and the latent causal structures.}
    \item \textcolor{black}{We propose a general framework for time series domain adaptation, validated through extensive experiments on time series classification and forecasting tasks across eight datasets. The results consistently demonstrate significant performance improvements over state-of-the-art methods.} 
\end{itemize}

% The remainder of this paper is organized as follows. Section 2 provides a review of existing domain adaptation methods for time series and video data, Granger causal discovery, and identifiability of generative models.
% In Section 3, we provide the background of this paper like unsupervised domain adaptation (UDA), the time series data generation process under multiple distributions, and the definition of identifiability. Section 4 presents theoretical proofs and discussions regarding the identifiability of the data generation process. Section 5 details the implementation of our proposed model. Section 6 presents and analyzes the experimental results. Finally, Section 7 concludes the paper and discusses potential future research directions.

% \clearpage
% Deep learning has achieved significant success in many practical applications, but it comes with considerable costs, including data collection, annotation, and model training expenses. However, domain adaptation techniques\cite{pan2009survey,zhang2019bridging,li2021learning,stojanov2021domain} not only reduce these costs but also enhance the model's generalization ability, making them a subject of extensive research and application.

% Numerous strategies have been explored for domain adaptation \cite{cai2021graph,hao2021semi,shui2021aggregating,li2021causal}, typically relying on the assumption that while the marginal distribution \(P(X)\) differs across domains, the conditional distribution \(P(Y|X)\) remains unchanged. Using this assumption, methods utilizing MDD \cite{long2015learning,pan2010domain, Cai_Chen_Li_Chen_Zhang_Ye_Li_Yang_Zhang_2021} and adversarial training \cite{cai2019learning,ganin2015unsupervised,xie2018learning,cai2019learning} have been developed to extract domain-invariant features, leading to notable success in non-time-series data. However, Zhang et al. \cite{zhang2013domain} recognized that the covariate shift assumption might not always be valid and instead examined three alternative scenarios through a causal perspective: target shift, conditional shift, and generalized target shift. More recently, Zhang et al. \cite{zhang2020domain} employed a Bayesian framework to describe distributional changes and introduced a causality-driven method to address the challenge of unsupervised domain adaptation.

% Recently, there has been an increasing focus on domain adaptation for time-series data. Although many researchers have extended the principles used for non-time series data to time series contexts \cite{da2020remaining,purushotham2016variational}, these approaches often continue to rely on the covariate shift assumption. This traditional assumption, commonly used in existing time series domain adaptation methods, assumes that \( P_S(z_{t+1} \mid z_t, \ldots, z_1) = P_T(z_{t+1} \mid z_t, \ldots, z_1) \), which considers all variable relationships, as illustrated in Figure~\ref{fig:motivation} (a). However, due to the intricate dependencies in time-series data—such as the first-order Markov dependence that creates correlations between any two-time steps—the conditional distribution \( P(z_{t+1} \mid z_1, \ldots, z_t) \) can vary significantly across different domains. This variability makes it challenging for the covariate shift assumption to hold in time-series data, posing difficulties in developing transferable and robust models.

% Fortunately, as shown in Figure~\ref{fig:motivation}(b), \cite{cai2021time,li2023transferable} found that the causal structures in different domains are stable and compact. This stability arises because time-series data generation generally follows underlying physical laws that embody causality. For example, in finance, the causal relationships among ``Interest Rates,'' ``Stock Prices,'' and ``Inflation'' remain consistently stable across various market conditions. Building on this insight, \cite{cai2021time,li2023transferable}move beyond the conventional covariate shift assumption and approach the adaptation of the time-series domain using these stable causal mechanisms.

% However, when it comes to uncovering causal mechanisms between observed variables, the methods \cite{cai2021time,li2023transferable} face challenges in handling high-dimensional data, making it difficult to effectively reveal causal relationships within the data. For example, in the case of video data, the feature dimensions of each frame typically reach tens of thousands, which makes extracting effective causal mechanisms from them very challenging. In reality, the observed phenomena are often driven by more underlying logic, and the causal mechanisms of these underlying factors are usually stable. As shown in Figure~\ref{fig:motivation}(c), each image captures the posture and gait of the person at that moment, and these postures and gaits are typically determined by more fundamental factors such as stride length, stride frequency, and body balance. A larger stride length usually results in a lower stride frequency, while the body becomes more prone to losing balance, and vice versa. These patterns are consistent and stable between different individuals.

% Therefore, we consider the causal generation process of time series data in a more general framework. We assume that the observed variables at each time point are determined by the corresponding latent variables and that the causal mechanisms formed by the latent variables at adjacent time points are consistent across different domains. Moreover, we theoretically ensure that the latent variables and the structure between them are identifiable. Ultimately, based on the data generation process, we derive the evidence lower bound and use it to build our model. To ensure the sparsity of causal relationships and the consistency of causal mechanisms, we incorporate sparsity and consistency constraints into the model. To validate the effectiveness of our method, we conducted extensive experiments. Specifically, we first performed prediction tasks on five time series datasets to comprehensively evaluate the performance of our method. To further verify its applicability in handling high-dimensional data, we also selected video datasets (where each frame is an image) for classification tasks. The experimental results demonstrate that our method performs exceptionally well in these tasks, further confirming its effectiveness.

% The remainder of this paper is organized as follows: Section 2 provides a review of existing time series transfer learning methods, time series forecasting methods, video transfer learning methods, and identifiability theory. In Section 3, we define unsupervised domain adaptation (UDA), the time series data generation process, and identifiability. Section 4 presents theoretical proofs and discussions regarding the identifiability of the data generation process. Section 5 details the implementation of our proposed model. Section 6 presents and analyzes the experimental results. Finally, Section 7 concludes the paper and discusses potential future research directions.


\section{Related Works}\label{related_works}

% \subsection{Unsupervised Domain Adaptation}

% \subsection{Domain Adaptation on Temporal Data}

% \subsection{Identification of Generative Model}

\subsection{Unsupervised Domain Adaptation}
Unsupervised domain adaptation \cite{cai2019learning,kong2022partial,shui2021aggregating,stojanov2021domain,wen2019bayesian,zhang2013domain} aims to leverage the knowledge from a labeled source domain to an unlabeled target domain, by training a model to domain-invariant representations \cite{bousmalis2016domain}. Researchers have adopted different directions to tackle this problem. For example, Long et al. \cite{long2017deep} trained the model to minimize a similarity measure, i.e., maximum mean discrepancy (MMD), to guide learning domain-invariant representations. Tzeng et al. \cite{tzeng2014deep} used an adaptation layer and a domain confusion loss. Another direction is to assume the stability of conditional distributions across domains and extract the label-wise domain-invariant representation \cite{chen2019joint,chen2019progressive,kang2020contrastive}. For instance, Xie et al. \cite{xie2018learning} constrained the label-wise domain discrepancy, and Shu et al. \cite{shu2018dirt} considered that the decision boundaries should not cross high-density data regions, so they propose the virtual adversarial domain adaptation model. Another type of assumption is the target shift \cite{lipton2018detecting,roberts2022unsupervised,wen2020domain,zhang2013domain}, which assumes $p(Y|\rvu)$ varies across different domains. 

Besides, several methods address the domain adaptation problem from a causality perspective. Specifically, Zhang et al. \cite{zhang2013domain} proposed the target shift, conditional shift, and generalized target shift assumptions, based on the premise that $p(Y)$ and $P(X|Y)$ vary independently. Cai et al. \cite{cai2019learning} leveraged the data generation process to extract the disentangled semantic representations. Building on causality analysis, Petar et al. \cite{stojanov2021domain} highlighted the significance of incorporating domain-specific knowledge for learning domain-invariant representation. Recently, Kong et al. \cite{kong2022partial} addressed the multi-source domain adaptation by identifying the latent variables, and Li et al. \cite{li2024subspace} further relaxed the identifiability assumptions.


\subsection{Domain Adaptation on Temporal Data}
In recent years, domain adaptation for time series data has garnered significant attention. Da Costa et al. \cite{da2020remaining} is one of the earliest time series domain adaptation works, where authors adopted techniques originally designed for non-time series data to this domain, incorporating recurrent neural networks as feature extractors to capture domain-invariant representations. Purushotham et al. \cite{purushotham2022variational} further refined this approach by employing variational recurrent neural networks \cite{chung2015recurrent} to enhance extracting domain-invariant features. However, such methods face challenges in effectively capturing domain-invariant information due to the intricate dependencies between time points.
Subsequently, Cai et al. \cite{cai2021time} proposed the Sparse Associative Structure Alignment (SASA) method, based on the assumption that sparse associative structures among variables remain stable across domains. This method has been successfully applied to adaptive time series classification and regression tasks. Additionally, Jin et al. \cite{jin2022domain} introduced the Domain Adaptation Forecaster (DAF), which leverages statistical relationships from relevant source domains to improve performance in target domains. Li et al. \cite{li2023transferable} hypothesized that causal structures are consistent across domains, leading to the development of the Granger Causality Alignment (GCA) approach. This method uncovers underlying causal structures while modeling shifts in conditional distributions across domains.

Our work also relates to domain adaptation for video data, which could be considered a form of high-dimensional time series data. Video data offers a robust benchmark for evaluating the performance of our method. Unsupervised domain adaptation for video data has recently attracted substantial interest. For instance, Chen et al. \cite{chen2019temporal} proposed a Temporal Attentive Adversarial Adaptation Network (TA3N), which integrates a temporal relation module to enhance temporal alignment. Choi et al. \cite{choi2020shuffle} proposed the SAVA method, which leverages self-supervised clip order prediction and attention-based alignment across clips. In addition, Pan et al. proposed the Temporal Co-attention Network (TCoN) \cite{pan2020adversarial}, which employs a cross-domain co-attention mechanism to identify key frames shared across domains, thereby improving alignment. 

Luo et al. \cite{luo2020adversarial} focused on domain-agnostic classification using a bipartite graph network topology to model cross-domain correlations. Rather than relying on adversarial learning, Sahoo et al. \cite{sahoo2021contrast} developed CoMix, an end-to-end temporal contrastive learning framework that employs background mixing and target pseudo-labels. More recently, Chen et al. \cite{chen2022multi} introduced multiple domain discriminators for multi-level temporal attentive features to achieve superior alignment, while Turrisi et al.~\cite{da2022dual} utilized a dual-headed deep architecture that combines cross-entropy and contrastive losses to learn a more robust target classifier. Additionally, Wei et al.~\cite{wei2023unsupervised} employed contrastive and adversarial learning to disentangle dynamic and static information in videos, leveraging shared dynamic information across domains for more accurate prediction.


\subsection{Granger Causal Discovery}
Several works have been raised to infer causal structures for time series data based on Granger causality \cite{diks2006new,granger1969investigating,marcinkevivcs2021interpretable,lowe2020amortized,lin2024root,gong2023causal}. Previously, several researchers used the vector autoregressive (\textbf{VAR}) model \cite{lozano2009grouped,hamilton2020time} with the sparsity constraint like Lasso or Group Lasso \cite{yuan2006model,tibshirani1996regression} to learn Granger causality. Recently, several works have inferred Granger causality with the aid of neural networks. For instance, Tank et al. \cite{tank2021neural} developed a neural network-based autoregressive model with sparsity penalties applied to network weights. Inspired by the interpretability of self-explaining neural networks, Marcinkevivcs et al. \cite{marcinkevivcs2021interpretable} introduced a generalized vector autoregression model to learn Granger causality. Li et al. \cite{li2023transferable} considered the Granger causal structure as latent variables. Cheng et al. \cite{cheng2023cuts,cheng2024cuts+} proposed a neural Granger causal discovery algorithm to discover Granger causality from irregular time series data. Lin et al. \cite{lin2024root} used a neural architecture with contrastive learning to learn Granger causality. However, these methods usually consider the Granger causal structures over low-dimension observed time series data, which can hardly address the time series data with high dimension or latent causal relationships. To address this limitation, we identify the latent variables and infer the latent Granger causal structures behind high-dimensional time series data.

\subsection{Identifiability of Generative Model}
To achieve causal representation \cite{rajendran2024learning,mansouri2023object,wendong2024causal} for time series data, many researchers leverage Independent Component Analysis (ICA) to recover latent variables with identifiability guarantees \cite{yao2023multi,scholkopf2021toward,Liu2023CausalTriplet,gresele2020incomplete}. Conventional methods typically assume a linear mixing function from the latent variables to the observed variables \cite{comon1994independent,hyvarinen2013independent,lee1998independent,zhang2007kernel}. To relax the linear assumption, researchers achieve the identifiability of latent variables in nonlinear ICA by using different types of assumptions, such as auxiliary variables or sparse generation processes \cite{zheng2022identifiability,hyvarinen1999nonlinear,hyvarinen2023identifiability,khemakhem2020ice,li2023identifying}. Aapo et al. \cite{hyvarinen2017nonlinear} first achieved identifiability for methods employing auxiliary variables by assuming the latent sources follow an exponential family distribution and introducing auxiliary variables, such as domain indices, time indices, and class labels. To further relax the exponential family assumption, Zhang et al. \cite{kong2022partial,xie2022multi, kong2023identification,yan2023counterfactual, xie2022multi} proposed component-wise identifiability results for nonlinear ICA, requiring $2n+1$ auxiliary variables for $n$ latent variables. To seek identifiability in an unsupervised manner, researchers employed the assumption of structural sparsity to achieve identifiability \cite{ng2024identifiability,lachapelle2022disentanglement,zheng2022identifiability, xu2024sparsity}. Recently, Zhang et al. \cite{zhang2024causal} achieved identifiability under distribution shift by leveraging the sparse structures of latent variables. Li et al. \cite{li2024identification} further employed sparse causal influence to achieve identifiability for time series data with instantaneous dependency.


% \clearpage

\section{PRELIMINARIES}
% \begin{figure*}
% 	\centering
%     \label{fig: data generation process}
% 	\includegraphics[width=2\columnwidth, trim=0cm 10cm 0cm 0cm, clip]{fig/new_data generation.pdf}
%     \caption{(a) The data generation process for the time series $\rvx$ and its corresponding labels $y$. (b) The causal mechanisms between latent variables are consistent across different domains, although the weight coefficients within causal relationships may vary.}
% \end{figure*}
% \subsection{Problem Definition}
% Unsupervised time series domain adaptation aims to improve the generalization performance of the model by transferring knowledge from a labeled source domain to an unlabeled target domain. Specifically, for the prediction, we have a source domain $\mathcal{S}$ and a target domain $\mathcal{T}$. The source domain $\mathcal{S}$ contains historical data and future data, denoted $\{(X_{i,1:t}^\mathcal{S}, X_{i,t+1:l}^\mathcal{S})\}_{i=1}^{N_{\mathcal{S}}}$, where $X_{i,1:t}^\mathcal{S}$ is a historical time series of length $t$ and $X_{i,t+1:l}^\mathcal{S}$ is a corresponding future time series data of length $l-t$. However, the target domain $\mathcal{T}$ contains only historical time series data, denoted as $\{X_{i,1:t}^\mathcal{T}\}_{i=1}^{N_{\mathcal{T}}}$.
% We aim to use data from $\mathcal{S}$ and $\mathcal{T}$ to more accurately predict future time series in the target domain $\mathcal{T}$. 
% \\\\
% Similarly, for classification, Source domain $\mathcal{S}$ contains $\{(X_{i,1:t}^\mathcal{S}, y^\mathcal{S}_i)\}_{i=1}^{N_{\mathcal{S}}}$, where $y^\mathcal{S}_i$ is the class label. The target domain $\mathcal{T}$ still contains only time series data $\{X_{i,1:t}^\mathcal{T}\}_{i=1}^{N_{\mathcal{T}}}$. We use them to better predict the labels on the target.

In this section, we first describe the data generation process from multiple distributions under latent Granger Causality. Sequentially, we further provide the definition of generalized causal conditional shift as well as identifiability of latent causal process.

\subsection{Data Generation Process from Multiple Distributions under Latent Granger Causality}
To illustrate how we address time series domain adaptation with latent causality alignment, we first consider the data generation process under latent Granger causality. Specifically, we first let $X=(\rvx_1,\cdots,\rvx_{\tau}, \cdots, \rvx_t)$ be multivariate time series with $t$ time steps. Each $\rvx_{\tau} \in \mathbb{R}^m$ is generated from latent variables $\rvz_{\tau} \in \mathbb{R}^n, m \gg n$ via an invertible nonlinear
mixing function $\rvg$ as follows:
\begin{equation}
% \small
\label{equ:g1}
    \rvx_{t}=\rvg(\rvz_{t}).
\end{equation}
Moreover, the $i$-th dimension of the latent variables $\rvz_{t,i}$ is generated through the latent causal process, which is assumed to be influenced by the time-delayed parent variables $\text{Pa}(\rvz_{t,i})$ and the different domains, simultaneously. Formally, this relationship can be expressed using
a structural equation model as follows:
\begin{equation}
% \small
\label{equ:g2}
\begin{split}
    z_{t,i}= f_i(\text{Pa}(z_{t,i}), \rvu ,\epsilon_{t,i}) \quad \text{with} \quad \epsilon_{t,i}\sim p_{\epsilon_{i}},
    % z_{t-\tau,k}
\end{split}
\end{equation}
where $\epsilon_{\tau,i}$ denotes the temporally and spatially
independent noise extracted from a distribution $p_{\epsilon_{i}}$, and $\rvu$ denotes the domain index that could be either the source ($S$) or target ($T$) domains.

\subsection{Generalized Causal Conditional Shift Assumption}

Inspired by the domain-invariant causal mechanism between the source and target domains, we generalize the causal conditional shift assumption \cite{li2023transferable} from observed variables to latent variables, which is discussed next.
\begin{assumption}
(\textbf{Generalized Causal Conditional Shift}) Given latent causal structure $\mA$ and latent variables $\rvz_1,\cdots,\rvz_{t}, \rvz_{t+1}$, we assume that the conditional distribution $P(\rvz_{t+1}|\rvz_1,\cdots,\rvz_{t})$ varies with different domains, while the latent causal structures are stable across different domains. This can be formalized as follows:

\begin{equation}
\begin{split}
    P(\rvz_{t+1}|\rvz_1,\cdots,\rvz_{t},S)&\neq P(\rvz_{t+1}|\rvz_1,\cdots,\rvz_{t},T)\\
    A_S&=A_T,
\end{split}
\end{equation}
where $A_S$ and $A_T$ are the causal structures over latent variables from the source and target domains, respectively.
\end{assumption}

Based on the aforementioned assumption, we consider both the forecasting and the classification time series domain adaptation tasks. Starting with the time series forecasting, the label $Y=(\rvx_{t+1},\cdots,\rvx_{t+\rho})$ denotes the values of the future $\rho$ time steps. On the other hand, $Y$ denotes the discrete class labels for the classification tasks, which are generated as follows:
\begin{equation}
    Y=C(\rvz_1,\cdots,\rvz_t),
\end{equation}
where $C$ is the labeling function. Our objective in this paper is to use the labeled source data and unlabeled target data to identify the target joint distribution.

% 我们考虑两种time series da任务

% In this section, we analyze the time series data generation process illustrated in Figure~\ref{fig: data generation process}(a), where the latent causal structure is sparse. Assume we have a time series dataset with discrete timestamps, denoted by $X=\{x_1,\cdots,x_t,\cdots,x_l\}$. Each time step $x_t$ is generated from latent variables $\rvz_t\in \mathcal{Z}\subseteq \mathbb{R}^n$ using an invertible nonlinear mixing function $\rvg$, as detailed in Equation~\ref{equ:g1}.

% The labels $y$ corresponding to the time series dataset are also generated by latent variables, which can be formalized as follows:
% \begin{equation}
% \small
% \label{equ:g3}
%     y=\rvf(\rvz_{1:l}).
% \end{equation}
% Specifically, the $i$-th dimension of the latent variable $z_{t,i}$ is generated through the latent causal process, which is assumed to be influenced by the time-delayed parent variables $\text{Pa}d(z{t,i})$. Formally, this relationship can be expressed using a structural equation model as follows:
% \begin{equation}
% \small
% \label{equ:g2}
% \begin{split}
%     z_{t,i}= f_i(\text{Pa}_d(z_{t,i}),\epsilon_{t,i}) \quad \text{with} \quad \epsilon_{t,i}\sim p_{\epsilon_{i}},
%     % z_{t-\tau,k}
% \end{split}
% \end{equation}
% where $\epsilon_{t,i}$ denotes the temporally and spatially
% independent noise extracted from a distribution $p_{\epsilon_{i}}$. 

% As discussed previously, the causal mechanisms of the latent variables exhibit consistency across different domains, as illustrated in Figure \ref{fig: data generation process}(b). This is evident in the fact that, while the causal relationships between latent variables at adjacent time points may have different weights across domains, the underlying causal mechanisms remain unchanged.
%为了更加有效的在时序数据上进行领域自适应学习,SASA,以及GCA工作中分别考虑观测数据上的系数关联结构以及格兰杰因果结构在不同领域上是一致的,但这些方法不能很好的面对高维数据。因此为了解决这一问题,正如图B所示,我们考虑不同领域的隐变量之间的因果机制是一致的。
% This data generation process can be used to explain several real-world scenarios.

% To enhance domain adaptation learning on time series data more effectively, the SASA approach considers the sparse t correlation structure, while the GCA approach focuses on the Granger causality structure, both assuming consistency across different domains. However, these methods struggle with high-dimensional data. To address this issue, as illustrated in Figure \ref{fig:data generation process}(b), we propose considering the causal mechanisms between latent variables across different domains as consistent.

\subsection{Identifiability of Target Joint Distribution}
In this subsection, we discuss the mechanism of identifying the target joint distribution. By introducing the latent variables and combining the data generation process, we can factorize the target joint distribution as shown in Equation (\ref{eq: prob_int}).

\begin{equation}
\label{eq: prob_int}
\begin{split}
    &P(X,Y|T)=\int_{Z}P(X|Z)P(Y|Z)P(Z|T)dZ,
\end{split}
\end{equation}
where $Z=(\rvz_1,\cdots,\rvz_{t})$ denotes the latent causal process. According to the aforementioned equation, we can identify the target joint distribution by modeling the conditional distribution of observed data given latent variables and identifying the latent causal process under theoretical guarantees. Specifically, if estimated latent processes can be uniquely identified up to permutation and component-wise invertible transformations, then the latent causal relationships are also identifiable. This can be regarded to the conditional independence relations that fully characterize causal relationships within a causal system, assuming the absence of causal confounders within latent causal processes.

\begin{definition}[Identifiable Latent Causal Process]
\label{def: identifibility}
Let $\mathbf{X} = \{\mathbf{x}_1, \dots, \mathbf{x}_l\}$ represent a sequence of observed variables generated according to the true latent causal processes characterized by $(f_i, p(\epsilon_i), \mathbf{g})$, as described in Equations (\ref{equ:g1}) and (\ref{equ:g2}). A learned generative model $(\hat{f}_i, \hat{p}(\epsilon_i), \hat{\mathbf{g}})$ is said to be observationally equivalent to $(f_i, p(\epsilon_i), \mathbf{g})$ if the distribution $p_{\hat{f}_i, \hat{p}(\epsilon_i), \hat{\mathbf{g}}}(\{\mathbf{x}\}_{t=1}^l)$ produced by the model matches the true data distribution $p_{f_i, p(\epsilon_i), \mathbf{g}}(\{\mathbf{x}\}_{t=1}^l)$ for all possible values of $\mathbf{x}_t$. We say that the latent causal processes are identifiable if such observational equivalence implies that the generative model can be transformed into the true generative process by means of a permutation $\pi$ and a component-wise invertible transformation $\mathcal{T}$:
\begin{equation}
% \small
\label{eq:iden}
     p_{\hat{f}_i, \hat{p}_{\epsilon_i},\hat{\mathbf{g}}}\big(\{\mathbf{x}\}_{t=1}^T\big) = p_{f_i, p_{\epsilon_i}, \mathbf{g}} \big(\{\mathbf{x}\}_{t=1}^T\big) 
     \ \ \Rightarrow\ \   \hat{\mathbf{g}} = \mathbf{g}  \circ  \pi  \circ \mathcal{T}.
\end{equation}
\end{definition} 

Upon the identification of the mixing process, the latent variables will be identified up to a permutation and a component-wise invertible transformation:
\begin{align}
% \small
    \mathbf{\hat{z}}_t 
    &= \mathbf{\hat{g}}^{-1}(\mathbf{x}_t) 
    = \big(\mathcal{T}^{-1} \circ \pi^{-1} \circ \mathbf{g}^{-1}\big)(\mathbf{x}_t) \\
 \nonumber
    &= \big(\mathcal{T}^{-1} \circ \pi^{-1} \big)\big(\mathbf{g}^{-1}(\mathbf{x}_t)\big) 
    = \big(\mathcal{T}^{-1} \circ \pi^{-1} \big)(\mathbf{z}_t).
\end{align}

% Furthermore, once the latent variables are identified, the entire causal structure among them can be fully determined, as interventions can be readily applied to any individual latent variable

\section{\textcolor{black}{Identifying Latent Causal Process}}
Based on the definition of the identification of latent causal processes, we demonstrate how to identify the latent causal process within the context of a sparse latent process. Specifically, we first utilize the connection between conditional independence and cross derivatives \cite{lin1997factorizing}, along with the sufficient variability of temporal data, to uncover the relationships between the estimated and true latent variables. Furthermore, we establish the identifiability of the latent variables by imposing constraints on sparse causal influences, as shown in Lemma \ref{Th1}.



% In this section, we first reveal the relationships between ground-truth and estimated latent variables. Without loss of generality, we consider a simplified case with $\tau=1$, implying that any $z_{t, i}$ is subject to time-delayed effects of at most one step and instantaneous effects.

\syf{
We demonstrate that given certain assumptions, the latent variables are also identifiable. To integrate contextual information for identifying latent variables $\rvz_t$, we consider latent variables across $L$ consecutive timestamps, including $\rvz_t$. For simplicity, we analyze the case where the sequence length is 2 (i.e., $L=2$) and the time lag is 1. 
% The general case involving multiple lags and varying sequence lengths is discussed in Appendix A%\ref{app: Extension to Multiple Lags and Sequence Lengths}.
}

\begin{lemma}
\label{Th1}
\textbf{(Identifiability of Temporally Latent Process)} \cite{yao2022temporally} Suppose there exists invertible function $\hat{\mathbf{g}}$ that maps $\mathbf{x}_t$ to $\hat{\mathbf{z}}_t$, i.e., $\hat{\mathbf{z}}_t=\hat{\mathbf{g}}(\mathbf{x}_t)$, such that the components of  $\hat{\mathbf{z}}_t$ are mutually independent conditional on $\hat{\mathbf{z}}_{t-1}$.Let
\begin{equation}
\label{eq: independent_condition}
 % \small
 \begin{split}
    \mathbf{v}_{t,k} =
    \Big(\frac{\partial^{2}\log p(z_{t,k}|\mathbf{z}_{t-1}) }{\partial z_{t,k}\partial z_{t-1,1}},\frac{\partial^{2}\log p(z_{t,k}|\mathbf{z}_{t-1})}{\partial z_{t,k}\partial z_{t-1,2}},...,\\
        \frac{\partial^{2}\log p(z_{t,k}|\mathbf{z}_{t-1})}{\partial z_{t,k}\partial z_{t-1,n}}\Big)^{\mathsf{T}},\\
    \mathring{\mathbf{v}}_{t,k}=
\Big(\frac{\partial^{3}\log p(z_{t,k}|\mathbf{z}_{t-1})}{\partial z_{t,k}^{2}\partial z_{t-1,1}},\frac{\partial^{3}\log p(z_{t,k}|\mathbf{z}_{t-1})}{\partial z_{t,k}^{2}\partial z_{t-1,2}},...,\\ 
    \frac{\partial^{3}\log p(z_{t,k}|\mathbf{z}_{t-1})}{\partial z_{t,k}^{2}\partial z_{t-1,n}}\Big)^{\mathsf{T}}.
    \end{split}
\end{equation}
If for each value of $\mathbf{z}_t,\mathbf{v}_{t,1},\mathring{\mathbf{v}}_{t,1},\mathbf{v}_{t,2},\mathring{\mathbf{v}}_{t,2},...,,\mathbf{v}_{t,n},\mathring{\mathbf{v}}_{t,n}$, as 2n vector function in $z_{t-1,1},z_{t-1,2},...,z_{t-1,n}$, are linearly independent, then $\mathbf{z}_t$ must be an invertible, component-wise transformation of a permuted version of $\hat{\mathbf{z}}_t$.
\end{lemma}
% The


% \syf{
% \begin{theorem}
% Suppose that the observations are generated by Equation (\ref{equ:g1}) and (\ref{equ:g2}) and the process subject to observational equivalence $\mathbf{x}_t = \mathbf{\hat{g}}(\mathbf{\hat{z}}_{t})$, for a series of observations $\mathbf{x}_t\in\mathbb{R}^n$ and estimated latent variables $\mathbf{\hat{z}}_{t}\in\mathbb{R}^n$ with the corresponding process $\hat{f}_i, \hat{p}(\epsilon), \hat{g}$, where $\hat{g}$ is invertible. Let $\mathbf{c}_t \triangleq \{\mathbf{z}_{t-1},\mathbf{z}_{t}\}$ and $\mathcal{M}_{\mathbf{c}_t}$ be the variable set of two consecutive timestamps and the corresponding Markov network respectively. Suppose the following assumptions hold:
%     \begin{itemize}[leftmargin=*]
%         \item \underline{A1 (Smooth and Positive Density):} The probability function of the latent variables $\rvc_t$ is smooth and positive, i.e., $p_{\rvc_t}$ is third-order differentiable and $p_{\rvc_t}>0$ over $\mathbb{R}^{2n}$,
%         % \item \underline{A2 (Independent Noise):} The noise term $\epsilon_{t,i}$ are mutually independent, i.e., spatially and temporally independent,
%         \item \underline{A2 (linear independence):}
%         % Let $\rvc_t=\{\rvz_{t-1},\rvz_t\} \in \mathbb{R}^{2n}$ be the latent variables in the adjacent two timestamps and $c_{t,i},c_{t,j}$ be the $i$-th and $j$-th latent variables of $\rvc_t$.
%         Denote $|\mathcal{M}_{\rvc_t}|$ as the number of edges in Markov network $\mathcal{M}_{\rvc_t}$. Let
%         \begin{equation}
%         \small
%         \begin{split}
%             w(m)=
%             &\Big(\frac{\partial^3 \log p(\rvc_t|\rvz_{t-2})}{\partial c_{t,1}^2\partial z_{t-2,m}},\cdots,\frac{\partial^3 \log p(\rvc_t|\rvz_{t-2})}{\partial c_{t,2n}^2\partial z_{t-2,m}}\Big)\oplus \\
%             &\Big(\frac{\partial^2 \log p(c_t|\rvz_{t-2})}{\partial c_{t,1}\partial z_{t-2,m}},\cdots,\frac{\partial^2 \log p(c_t|\rvz_{t-2})}{\partial c_{t,2n}\partial z_{t-2,m}}\Big)\oplus \\ &\Big(\frac{\partial^3 \log p(\rvc_t|\rvz_{t-2})}{\partial c_{t,i}\partial c_{t,j}\partial z_{t-2,m}}\Big)_{(i,j)\in \mathcal{E}(\mathcal{M}_{\rvc_t})},
%         \end{split}
%         \end{equation}
%         where $\oplus$ denotes concatenation operation and $(i,j)\in\mathcal{E}(\mathcal{M}_{\rvc_t})$ denotes all pairwise indice such that $c_{t,i},c_{t,j}$ are adjacent in $\mathcal{M}_{\rvc_t}$.
%         For $m\in[1,\cdots,n]$, there exist $4n+2|\mathcal{M}_{\rvc_t}|$ different values of $\rvz_{t-2,m}$, as the $4n+2|\mathcal{M}_{\rvc_t}|$ values of vector functions $w(m)$ are linearly independent.
%     \end{itemize}
%     \syf{When the observational equivalence is achieved} with the minimal number of edges of the estimated Markov network of $\mathcal{M}_{\hat{\rvc}_t}$, then we have the following two statements:\\
%     (i) The estimated Markov network $\mathcal{M}_{\hat{\rvc}_t}$ is isomorphic to the Markov network with ground truth $\mathcal{M}_{\rvc_t}$.\\
% % , i.e., the latent Markov network $\mathcal{M}_{\rvc}$ is identifiable.
%     (ii) There exists a permutation $\pi$ of the estimated latent variables, such that $z_{t,i}$ and $\hat{z}_{t,\pi(i)}$ is one-to-one corresponding, i.e., $z_{t,i}$ is component-wise identifiable.
%     % at most one of them is a function of $\hat{c}_{t,k}$ or $\hat{c}_{t,l}$.
% \end{theorem}
% }

%\ref{app:the1}
\textbf{Proof Sketch.} The proof can be found in Appendix A.1. First, we establish a bijective transformation between the ground truth $\rvz_t$ and the estimated $\hat{\rvz}_t$ to connect them. Sequentially, we leverage the historical information to construct a full-rank linear system, where the only solution of $\frac{\partial z_{t,i}}{\partial \hat{z}_{t,j}} \cdot \frac{\partial z_{t,i}}{\partial \hat{z}_{t,k}}$ is zero. Since the Jacobian of $h$ is invertible, for each $z_{t,i}$, there exists a $h_i$ such that $z_{t,i}=h(\hat{z}_{t,j})$, implying that $z_{t,i}$ is component-wise identifiable.

By identifying the latent variables, they can be considered as observed variables up to a permutation and invertible transformation. As a result, Granger Causality among $\rvz_t$ and $\rvz_{t-1}$ can be identified by using sparsity constraint on the relationships among latent variables, which is shown in Proposition \ref{prop1} (also refer to the proof in Appendix A.3 for more details).%\ref{app: Granger non-causality of Latent Variables}.

\begin{prop}
\label{prop1}
\textbf{(Identification of Granger Causality among latent variables)} Suppose the estimated transition function $f_i$ accurately models the relationship between $\rvz_{t}$ and $\rvz_{t-1}$. Let the ground truth Granger Causal structure be represented as $\mathcal{G}=(V, E_V)$ with the maximum lag of 1, where $V$ and $E_V$ denote the nodes and edges, respectively. If the data are generated according to Equation (\ref{equ:g1}) and (\ref{equ:g2}), and sparsity is enforced among the latent variables, then $z_{t-1,j}\rightarrow z_{t,i} \notin E_V$ if and only if $\frac{\partial z_{t,i}}{\partial z_{t-1,j}}=0$. 
\end{prop}

% Next, we utilize the structural properties of the ground-truth Markov network $\mathcal{M}_{\rvc_t}$ to constrain the structure of the estimated Markov network $\mathcal{M}_{\hat{\rvc}_t}$ through the connection between conditional independence and cross derivatives\cite{lin1997factorizing}, i.e., $c_{t, i} \perp c_{t,j} | \rvc_t \backslash \{c_{t, i}, c_{t,j}\}$ implies $\frac{\partial^2 \log p(\rvc_t)}{\partial c_{t, i}\partial c_{t,j}}=0$. By introducing the sufficient variability assumption, we further construct a linear system with a full rank coefficient matrix to ensure that the only solution holds, i.e.,

% \begin{equation} 
% \label{eq: solutions of th1}
% \small
%     \frac{\partial c_{t,i}}{\partial \hat{c}_{t,k}}\cdot\frac{\partial c_{t,i}}{\partial \hat{c}_{t,l}}=0, \quad
%     \frac{\partial c_{t,i}}{\partial \hat{c}_{t,k}}\cdot\frac{\partial c_{t,j}}{\partial \hat{c}_{t,l}}=0, \quad
%     \frac{\partial^2 c_{t,i}}{\partial c_{t,k}\partial c_{t,l}}=0, 
% \end{equation}
% where $\frac{\partial c_{t,i}}{\partial \hat{c}_{t,k}}\cdot\frac{\partial c_{t,i}}{\partial \hat{c}_{t,l}}=0$ and $\frac{\partial c_{t,i}}{\partial \hat{c}_{t,k}}\cdot\frac{\partial c_{t,j}}{\partial \hat{c}_{t,l}}=0$ correspond to following statement (a) statement (b),respectively.\\
% For any two different entries $\hat{c}_{t,k},\hat{c}_{t,l} \in \hat{\rvc}_t$ that are \textbf{not adjacent} in the Markov network $\mathcal{M}_{\hat{\rvc}_t}$ over estimated $\hat{\rvc}_t$,\\
% \textbf{(a)} Each ground-truth latent variable $c_{t,i}\in \rvc_t$ is a function of at most one of $\hat{c}_{k}$ and $\hat{c}_{l}$,\\
% \textbf{(b)} For each pair of ground-truth latent variables $c_{t,i}$ and $c_{t,j}$ that are \textbf{adjacent} in $\mathcal{M}_{\rvc_t}$ over $\rvc_{t}$, they can not be a function of $\hat{c}_{t,k}$ and $\hat{c}_{t,l}$ respectively.

% Given the fact that there always exists a row permutation for each invertible matrix such that the permuted diagonal entries are non-zero \cite{zheng2024generalizing}, we utilize the conclusion of above statement (a), to demonstrate that any edge present in the true Markov network will necessarily exist in the estimated one. Furthermore, when the number of estimated edges reaches a minimum, the identifiability of Markov network can be achieved. Finally, we once again leverage statement (b) to illustrate that the permutation which allows the Markov network to be identified can further lead to a component-wise level identifiability of latent variables.

% \textbf{Discussion.}
% The assumption of Smooth, Positive is common in the existing identification results \cite{yao2022temporally,yao2021learning}. In real-world scenarios, smooth and positive density implies continuous changes in historical information, such as temperature variations in weather data. To achieve this, we should sample as much data as possible to learn the transition probabilities more accurately. 
% Linear Independence.
% The linear independence assumption is standard in the identification of nonlinear ICA \cite{allman2009identifiability}. It implies that linear independence is necessary for a unique solution to a system of equations. Though this assumption is untestable, we may investigate whether it is satisfied through the prior knowledge of the applications.  

\section{Latent Causality Alignment Model}
Based on the theoretical results, we propose the latent causality alignment (\textbf{LCA}), shown in Figure \ref{fig:model}, for time series domain adaptation. This figure shows a variational-inference-based neural architecture to model time series data, a prior estimation network, and a downstream neural forecaster for different downstream tasks.
\subsection{Temporally Variational Inference Architecture}
\begin{figure*}
	\centering
	\includegraphics[width=1.9\columnwidth]{fig/lca_model.pdf}
    \caption{Model architecture: Blue arrows indicate the flow of source data, while purple arrows represent the flow of target data. The loss function is highlighted in bold as \( L \). Subfigures (a) and (b) depict the architectures for prediction and classification tasks, respectively.}
    \label{fig:model}
\end{figure*}
We first derive the evidence lower bound (ELBO) to model the time series data, as follows: 
\begin{equation}
\label{eq:elbop}
\begin{split}
&\ln P(X, Y) = \ln {P(\rvx_{1:t}, Y)}=\ln\frac{P(\rvx_{1:t}, Y, \rvz_{1:t})}{P(\rvz_{1:t}|\rvx_{1:t}, Y)}\\\geq&\underbrace{\mathbb{E}_{Q(\rvz_{1:t}|\rvx_{1:t})}\ln P(\rvx_{1:t}|\rvz_{1:t})}_{L_{R}} +\underbrace{\mathbb{E}_{Q(\rvz_{1:t})}\ln P(Y|\rvz_{1:t})}_{L_{Y}} \\
&\underbrace{
-D_{KL}(Q(\rvz_{1:t}|\rvx_{1:t})||P(\rvz_{1:t}))
}_{L_{KL}} ,
\end{split}
\end{equation}
where \(Q(\rvz_{1:t}|\rvx_{1:t})\) and \(P(\rvx_{1:t}|\rvz_{1:t})\) are used to approximate the prior distribution of latent variables and reconstruct the observations, respectively. Technologically, we consider $Q(\rvz_{1:t}|\rvx_{1:t})$ and $P(\rvx_{1:t}|\rvz_{1:t})$ as the encoder and decoder networks, respectively, which can be formalized as follows:
\begin{equation}
    \hat{\rvz}_{1:t} = \psi(\rvx_{1:t}) \quad \hat{\rvx}_{1:t} = \phi(\rvx_{1:t}),
\end{equation}
where $\psi$ and $\phi$ denote the encoder and decoder respectively.
%Please refer to Appendix X for the implementation details.

% And $P(Y|\rvz_{1:t})$ is used to generate predicted labels. \(D_{KL}\) denotes the Kullback–Leibler divergence between prior and posterior distributions. 

% As illustrated in Figure~\ref{fig:model}, we employ distinct neural networks to learn the aforementioned distributions. Specifically, the Encoder, History Reconstructor, Latent Predictor, Future Predictor, and Classifier are utilized to approximate \( q(\rvz_{1:t}|\rvx_{1:t}) \), \( p(\rvx_{1:t}|\rvz_{1:t}) \), \( q(\rvz_{t+1:T}|\rvz_{1:t}) \), \( p(\rvx_{t+1:T}|\rvz_{t+1:T}) \), and \( p(y|\rvz_{1:t}) \), respectively. The corresponding inputs and outputs for each network are detailed below, with \(\theta_{e}\), \(\theta_{h}\), \(\theta_{l}\),  \(\theta_{f}\), and \(\theta_{c}\) representing their respective parameters.
% \begin{align}
% \small
%     z_{1:t}&=Encoder(\rvx_{1:t};\theta_{e})\\
%     \hat{x}_{1:t}&=History \ Reconstructor(z_{1:t};\theta_h)\\
%     z_{t+1:T}&=Latent Predictor(z_{1:t};\theta_l)\\
%     \hat{x}_{t+1:T}&=Future Predictor(z_{t+1:T};\theta_f)\\
%     \hat{y}&=Classifier(z_{1:t};\theta_f)
% \end{align}
% To avoid the assumption that the latent variables at each time step adhere to a standard Gaussian distribution, which may obscure the complex dependencies present in time series data, we have developed a network to estimate the true prior distributions \(p(\rvz_{1:t})\) and \(p(\rvz_{t+1:T}|\rvz_{1:t})\), as outlined in Section~\ref{sec:prior}. Furthermore, we introduce a gradient-based sparsity penalty to promote sparse causal relationships and a gradient-based alignment penalty to ensure the consistency of causal mechanisms between the source and target domains. These approaches are elaborated in Section~\cref{sec: spar,sec:align}.
% Please refer to Appendix \ref{app:imple} for details of the implementation.
\subsection{Prior Estimation Networks} \label{sec:prior}
To estimate the prior distribution $P(\rvz_{1:t})$, we propose the prior estimation networks. Specifically, we first let ${r_i}$ be a set of learned inverse transition functions that receive the estimated latent variables \textcolor{black}{with the superscript symbol $\hat{}$} as input, and use it to estimate the noise term $\hat{\epsilon}_i$, i.e., $\hat{\epsilon}_{t,i}=r_i(\hat{\rvz}_{t,i},\hat{\rvz}_{t-1})$, where each $r_i$ is implemented by Multi-layer Perceptron networks (MLPs). Sequentially, we devise a transformation $\kappa := \{\hat{\rvz}_{t-1},\hat{\rvz}_t\}\rightarrow\{\hat{\rvz}_{t-1},\hat{\bm{\epsilon}}_{t}\}$, whose Jacobian can be formalized as ${\mathbf{J}_{\kappa}=
    \begin{pmatrix}
        \mathbb{I}&0\\
        \mathbf{J}_d & \mathbf{J}_e
    \end{pmatrix}}$, where $\mathbf{J}_d(i,j) = \frac{\partial r_i}{\partial \hat{z}_{t-1,j}}
$ and $\mathbf{J}_e\!\!=\!\!\text{diag}\!\left(\frac{\partial r_i}{\partial \hat{z}_{t,i}}\!\right)$. Hence we have Equation (\ref{eq:pri_1}) via the change of variables formula. 
\begin{equation}
\label{eq:pri_1}
\small
    \log p(\hat{\rvz}_t, \hat{\rvz}_{t-1})=\log p(\hat{\rvz}_{t-1},\epsilon_t) + \log |\frac{\partial r_i}{\partial z_{t,i}}|.
\end{equation}
According to the generation process, the noise term \(\epsilon_{t,i}\) is independent of \(\rvz_{t-1}\). Therefore, we can impose independence on the estimated noise term \(\hat{\epsilon}_{t,i}\). Consequently, Equation~(\ref{eq:pri_1}) can be further expressed as:
\begin{equation}
\small
    \log p(\hat{\rvz}_t \mid \{\hat{\rvz}_{t-\tau}\}) = \sum_{i=1}^{n} \log p(\hat{\epsilon}_{t,i}) + \sum_{i=1}^{n} \log \left| \frac{\partial r_i}{\partial \hat{z}_{t,i}} \right|.
\end{equation}
Assuming a \(\tau\)-order Markov process, the prior \( p(\hat{\rvz}_{1:T}) \) can be expressed as \( p(\hat{z}_{1:\tau}) \prod_{t=\tau+1}^{T} p(\hat{\rvz}_{t} \mid \{ \hat{\rvz}_{t-\tau} \}) \). Hence, the log-likelihood \( \log p(\hat{\rvz}_{1:T}) \) can be estimated as:
\begin{equation}
\small
    \log p(\hat{\rvz}_{1:T}) = \log p(\hat{z}_{1:\tau}) + \sum_{t=\tau+1}^T \left( \sum_{i=1}^n \log p(\hat{\epsilon}_{t,i}) + \sum_{i=1}^n \log \left| \frac{\partial r_i}{\partial \hat{z}_{t,i}} \right| \right).
\end{equation}
Here, we assume that the noise term \( p(\hat{\epsilon}_{\tau,i}) \) and the initial latent variables \( p(\hat{\rvz}_{1:\tau}) \) follow Gaussian distributions. For \(\tau = 1\), this simplifies to:
\begin{equation}
\small
    \log p(\hat{\rvz}_{1:T}) = \log p(\hat{z}_{1}) + \sum_{t=2}^T \left( \sum_{i=1}^n \log p(\hat{\epsilon}_{t,i}) + \sum_{i=1}^n \log \left| \frac{\partial r_i}{\partial \hat{z}_{t,i}} \right| \right).
\end{equation}
Similarly, the distribution \( p(\hat{\rvz}_{t+1:T} \mid \hat{\rvz}_{1:t}) \) can be estimated using analogous methods. For further details on the derivation of the prior, please refer to Appendix B.%\ref{app:prior_derivation}


\subsection{Sparsity Constraint for Latent Granger Causality} 
\begin{figure}
	\centering
	\includegraphics[width=0.9\columnwidth]{fig/noise_meaning.pdf}
    \caption{To describe the relationship between partial derivatives and the existence of edges clearly, we assume that the data follow a simple linear generation process.}
    \label{fig: noise_meaning}
\end{figure}
Based on the theoretical results, we can identify the latent variables by using the variational-influenced-based architecture and prior estimation networks. However, without any further constraints, it is hard for us to infer the causal structures over latent variables. To address this problem, we propose the sparsity constraint on partial derivatives regarding independent noise and latent variables for the latent Granger Causality. To provide a clearer understanding of its implications, we use a straightforward example with ground truth and estimated generation processes as shown in Figure \ref{fig: noise_meaning}, where the boiled and dashed arrows denote the ground truth and spurious relationships. According to Figure \ref{fig: noise_meaning}, $z_{t,1}$ and $\hat{z}_{t,1}$ are generated as follows:
\begin{equation}
\begin{split}
    z_{t,1}&=f_1(z_{t-1,1},\epsilon_{t,1}) \\
    \hat{z}_{t,1}&=f_1(\hat{z}_{t-1,1}, \hat{z}_{t-1,2},\hat{\epsilon}_{t,1}),
\end{split}
\end{equation}
where the estimated generation process of $\hat{z}_{t,1}$ includes the spurious dependence from $\hat{z}_{t-1,2}$. To remove these spurious dependencies, we find that the partial derivatives of ground truth generation process $\frac{\partial \epsilon_{t,1}}{\partial z_{t-1,2}} = 0$ and $\frac{\partial \epsilon_{t,1}}{\partial z_{t-1,1}} \neq 0$, meaning that the $\mJ$ can provide an intuitive representation of the causal structures among the latent variables, as it quantifies the influence of \( \hat{\rvz}_{t-1} \) on \( \hat{z}_{t,i} \). Therefore, we can apply sparsity constraint on the partial derivatives regarding the estimated noise term and latent variables, i.e., $\frac{\partial \hat{\epsilon}_{t,1}}{\partial \hat{z}_{t-1,2}}$, to remove the spurious dependencies of latent variables. As a result, we propose to employ the $\mathcal{L}_1$ regularization on the  Jacobian matrix $\mJ$ as shown in Equation (\ref{equ:sparsity}).
\begin{equation}
% \small
\label{equ:sparsity}
    L_{S} = ||\mathbf{J}||_1,
\end{equation}
where $||*||_1$ denotes the $\mathcal{L}_1$ Norm of a matrix. 
% By employing the gradient-based sparsity penalty on the estimated latent causal processes, we can indirectly restrict the sparsity of Markov networks to satisfy the sparse latent process.
% which motivate us to apply sparsity constraint on the partial derivatives regarding the estimated noise term and latent variables, i.e., $\frac{\partial \hat{\epsilon}_{t,1}}{\partial \hat{z}_{t-1,2}}$. Based on this intuition, we can directly 


% \label{sec: spar}

% In this section, we begin by elucidating the significance of the Jacobian matrix $\mathbf{J}_d(i,j) = \frac{\partial r_i}{\partial \hat{z}_{t-1,j}}
% $.

% To provide a clearer understanding of its implications, we use a straightforward linear generative process as an illustrative example. As shown in the figure~\ref{fig: noise_meaning}, \( z_{t,1} \) is generated from the noise term \( \epsilon_{t,i} \) and \( z_{t-1,1} \). It is apparent that $\frac{\epsilon_{t,i}}{z_{t-1,1}}$ is non-zero, while $\frac{\epsilon_{t,i}}{z_{t-1,2}}$ equals zero. Consequently, by computing the derivatives, we can infer the existence of causal relationships and determine the presence of edges within the causal structure.

% Ideally, the MLPs-based architecture $r_i$ can capture the causal structure of latent variables by restricting the independence of noise $\hat{\epsilon}_{t, i}$. However, without any further constraint, $r_i$ may bring redundant causal edges from $\hat{\rvz}_{t-1}$ to $\hat{z}_{t,i}$, leading to the incorrect estimation of prior distribution and further the suboptimization of ELBO. 

% As mentioned previously, the matrix \( \mathbf{J}_d \) provides an intuitive representation of the causal structures among the latent variables, as it quantifies the influence of \( \hat{\rvz}_{t-1} \) on \( \hat{z}_{t,i} \). This insight leads us to eliminate redundant causal edges by incorporating a sparsity regularization term \( \mathcal{L}_S \). Specifically, we apply \( \mathcal{L}_1 \) regularization to \( \mathbf{J}_d \) to promote sparsity. Formally, this is expressed as:



\subsection{Latent Causality Alignment Constraint}\label{sec:align}
% 由于源域和目标域的隐结构是稳定不变的,我们需要对其源域和目标域的隐结构进行对齐。但是和观测变量上的结构对齐不同,隐变量的对齐是一件困难的事情。原因有2,首先隐变量之间的结构是隐式的,不能像GCA那样直接提取出因果结构进行对齐,另外虽然梯度一定可以反映隐变量之间的因果结构,但是直接对齐梯度会导致次优化,导致下游任务的性能变差。为了解决这两个问题,本文提出了Latent Causality Alignment Constraint

Since the latent structures are stable across different domains, we need to align the latent structures of the source domain and the target domain. However, unlike the structural alignment of the observed variables \cite{li2023transferable}, the alignment of latent variables can be a more challenging task for two main reasons. First, the structure between latent variables is implicit, and the causal structure cannot be directly extracted for alignment as in GCA \cite{li2023transferable}. In addition, although the partial derivatives regarding estimated noise and latent variables can reflect the causal structure over latent variables, directly aligning the gradient can affect the correct gradient descent direction and result in suboptimal performance downstream tasks. To overcome these challenges, we propose the latent causality alignment constraint.

% 首先,我们先通过一个阈值确定源域和目标域的结构,然后通过异或操作获取源域和目标域之间的差异,然后只需要约束差异的比分即可在保留结构
Specifically, we choose a threshold $u$ to determine the causal relations among the latent variables. For instance, if $\mathbf{J}^*_{i,j}>u$, then there is an edge from $z_{t-1, i}$ to $z_{t,j}$, otherwise, no edge is present. Formally, we can obtain the estimated causal relationships of the source or target domains (the superscript * shows the source or target domains) as follows: 
\begin{equation}  
\hat{\mathbf{J}}^*_{i,j} = \left\{
\begin{array}{l}
1, \text{ if }   \mathbf{J}^*_{i,j}>u ;\\
0, \text{ otherwise}.
\end{array}
\right.
\end{equation}
A direct solution to align the latent causal structures is to restrict the discrepancy between the latent structures of source and target domains, following \cite{li2023transferable}. However, since these causal structures are represented using gradients with respect to latent variables, the direct alignment of the gradients can interfere with the model optimization, thereby increasing the difficulty of training. To overcome these issues, we find that it is sufficient to focus on reducing the differences in $\mathbf{J}^*$ between the source and target domains while ignoring the identical parts. This approach achieves causal structure alignment while minimizing the impact on the gradients. Based on this idea, we first obtain a masking matrix through an XOR operation. In this matrix, elements with a value of 0 represent identical structures between the source and target domains, while elements with a value of 1 represent differing structures. We then constrain only the differing parts, which are formalized as follows:
\begin{equation}
\small
\begin{split}
    \mathcal{M} &= (\mathbf{J}^S_d  > u)  \oplus  (\mathbf{J}^T_d  > u) \\
    L_{A} &= ||\mathbf{C}(\mathbf{J}_d^S \odot \mathcal{M})-\mathbf{J}_d^T\odot \mathcal{M}||_1,
\end{split}
\end{equation}
where $\oplus$ and $\odot$ denote the XOR and element-wise product operations, respectively, and $\mathbf{C}(\cdot)$ denotes the gradient-stoping operation \cite{li2023transferable}, which is used to enforce the target latent structures closer to the source latent structures.

By combining Equations (\ref{eq:elbop}) and (\ref{equ:sparsity}), we can achieve the total loss of our method, as follows:
\begin{equation}
    L_{total} = L_Y + \alpha L_R + \beta L_{KL} + \gamma L_S + \delta L_A, 
\label{eqn:overall_loss}
\end{equation}
where $\alpha, \beta, \gamma$, and $\delta$ are tunable hyper-parameters.


% 但是由于这儿的因果结构使用关于隐变量的梯度来表示的,这种直接的在梯度上进行对齐会影响模型的优化,从而增加训练的难度。为了解决这些问题,我们发现我们只需要减少源域和目标域的J^*中不同的部分,而且不用管相同的部分,从而达到因果结构对齐的目的的同时,减少对梯度的最小影响。基于这个想法,我们首先通过异或的操作获得一个掩盖矩阵,这个掩盖矩阵中值为0的元素表示源域和目标域结构相同的部分,而值为1的部分表示结构不同的部分。然后我们仅仅约束结构不同的部分,实现如下所示。

% \begin{equation}
%     \mathbf{J}^*=\left\{  
%              \begin{array}{**lr**}  
%              x=\dfrac{3\pi}{2}(1+2t)\cos(\dfrac{3\pi}{2}(1+2t)), &  \\ 
%              y=s, & 0\leq s\leq L,|t|\leq1.\\  
%              z=\dfrac{3\pi}{2}(1+2t)\sin(\dfrac{3\pi}{2}(1+2t)), &    
%              \end{array}  
% \right.  
% \end{equation}


% Since the latent causal structures are stable across the source and target domains, we impose constraints on the causal structures of both domains. However, directly minimizing the distance between matrices \( \mathbf{J}^S \) and \( \mathbf{J}^T  \) could lead to a convergence in the weights of the causal relationships across the two domains. While this promotes alignment of causal mechanisms, it also risks losing domain-specific information and does not accurately reflect real-world scenarios. To address this issue, we introduce a threshold \( \mu \). If \( \mathbf{J}^S_d  > \mu \), we consider the presence of an edge in the source domain; otherwise, no edge exists. The same applies to target domain. We focus on aligning only the edges that differ in their existence between the two domains. Specifically, if \( z^s_{t,2} \) points to \( z^s_{t-1,1} \) in the source domain but \( z^t_{t,2} \) does not point to \( z^t_{t-1,1} \) in the target domain, alignment is required. In unsupervised domain adaptation, the causal relationships in the source domain can be learned more accurately due to the availability of labels. Therefore, during the alignment process, we align the target domain toward the source domain. This approach can be formalized as:
% \begin{equation}
% \small
% \begin{split}
%     mask = (\mathbf{J}^S_d  > u)  \oplus  (\mathbf{J}^T_d  > u) \\
%     L_{align}=||\mathbf{C}(\mathbf{J}_d^S[mask])-\mathbf{J}_d^T[mask]||_1
% \end{split}
% \end{equation}
% Where the symbol \( \oplus \) denotes the XOR operation, and \( \mathbf{C} \) represents an operation that prevents the backpropagation of gradients.

% However, the threshold \( u \) is a hyperparameter, and determining its 
% optimal value can be challenging. To address this, we make \( u \) a learnable and adaptive parameter. Specifically, \( u \) is treated as a learnable parameter, but because the operations \( > \) and \( \oplus \) are non-differentiable, \( u \) cannot be updated through gradient backpropagation.To overcome this issue, we adopt a reparameterization technique similar to the Gumbel-Softmax approach, which allows the output of one-hot vectors to be differentiable. The technique reintroduces the threshold \( u \) into the computational graph, allowing it to be updated through gradients. We denote this operation as \( G \), where \( G(\mathbf{J}_d, u) \) outputs a matrix with values of 0 or 1.
% Additionally, we replace the XOR operation \((\mathbf{J}^S_d > u) \oplus (\mathbf{J}^T_d > u)\) with \( |G(\mathbf{J}^S_d, u) - G(\mathbf{J}^T_d, u)| \) to make it differentiable. Therefore, the adaptive Alignment Regularization can be formulated as:
% \begin{equation}
% \small
% \begin{split}
%     \text{mask} &= |G(\mathbf{J}^S_d > u) - G(\mathbf{J}^T_d > u)| \\
%     L_{\text{align}} &= \|\mathbf{C}(\mathbf{J}_d^S \cdot \text{mask}) - (\mathbf{J}_d^T \cdot \text{mask})\|_1
% \end{split}
% \end{equation}

% \subsection{Model Summary}
% \textbf{In UDA prediction tasks}, where future data from the target domain is inaccessible, the final loss function can be formulated as follows:
% \begin{equation}
% \small
% \begin{split}
%         \mathcal{L}_{total}&=-\mathcal{L}_{pre}^S-\alpha(\mathcal{L}_{rec}^S+\mathcal{L}_{rec}^T ) - \beta (\mathcal{L}_{KL}^S+\mathcal{L}_{KL}^T )\\
%         &\quad~+\gamma (\mathcal{L}_{spa}^S+\mathcal{L}_{spa}^T  )+\delta \mathcal{L}_{align} \\
%         &=-\mathcal{L}_{pre}^S-\alpha \mathcal{L}_{rec} - \beta \mathcal{L}_{KL}
%         +\gamma \mathcal{L}_{spa}+\delta \mathcal{L}_{align}
% \end{split}
% \end{equation}
% where \(\alpha\), \(\beta\), and \(\gamma\) are weighting coefficients.
% \\
% \textbf{In UDA classification tasks}, although we cannot access the labels of the target domain, we employ the common approach of pseudo-labeling. Consequently, the final loss function can be formulated as follows:
% \begin{equation}
% \small
% \begin{split}
% \mathcal{L}_{total}&=\mathcal{L}^S_{CE}+\mathcal{L}^T_{CE}-\alpha(\mathcal{L}_{rec}^S+\mathcal{L}_{rec}^T )
%         \\ &\quad~ -\beta (\mathcal{L}_{KL}^S+\mathcal{L}_{KL}^T )
%         +\gamma (\mathcal{L}_{spa}^S+\mathcal{L}_{spa}^T  )+\delta \mathcal{L}_{align} \\
%         &=\mathcal{L}_{CE}-\alpha \mathcal{L}_{rec} - \beta \mathcal{L}_{KL}
%         +\gamma \mathcal{L}_{spa}+\delta \mathcal{L}_{align}
% \end{split}
% \end{equation}
% where \(\alpha\), \(\beta\), and \(\gamma\) are weighting coefficients.

\section{EXPERIMENTS}

% \subsection{Simulation Experiment

% \subsection{Realworld Experimetns}
% \subsubsection{Domain Adaptative Time Series Forecasting}
\begin{table*}[t]
\centering
\caption{MAE and MSE of various methods on the PPG-DaLiA dataset, where R-COAT, iTrans, and TMixer are the abbreviation of RAINCOAT, iTransformer, and TimeMixer, respectively.}
\label{tab:ppg}
\setlength{\tabcolsep}{3mm}
\resizebox{\textwidth}{!}{%
\begin{tabular}{@{}c|c|ccccccccccc@{}}
\toprule
Metric                & Task                        & SASA   & GCA    & DAF    & CLUDA  & R-COAT & AdvSKM & iTrans & TMixer & TSLANet & SegRNN & Ours             \\ \midrule
\multirow{12}{*}{\rotatebox{90}{MSE}} & \multicolumn{1}{c|}{ $\text{C} \rightarrow \text{D}$ } & 0.7421 & 0.8260 & 0.7085 & 0.8869 & 0.9117   & 0.7795 & 0.6036       & 0.5901    & 0.6162  & 0.6333 & \textbf{0.5797} \\
                      & \multicolumn{1}{c|}{$\text{C} \rightarrow \text{S}$} & 0.6279 & 0.6898 & 0.5726 & 0.7851 & 0.8675   & 0.6998 & 0.3433       & 0.2962    & 0.3139  & 0.4259 & \textbf{0.2842} \\
                      & \multicolumn{1}{c|}{$\text{C} \rightarrow \text{W}$} & 0.8767 & 0.9076 & 0.8565 & 0.9709 & 0.9356   & 0.8846 & 0.8117       & 0.8162    & 0.8226  & 0.8452 & \textbf{0.7946} \\ \cmidrule(l){2-13} 
                      & \multicolumn{1}{c|}{D $\rightarrow$ C} & 0.9415 & 0.9858 & 0.9258 & 1.0562 & 1.0134   & 0.9539 & 0.8603       & 0.8476    & 0.8623  & 0.8599 & \textbf{0.8294} \\
                      & \multicolumn{1}{c|}{D $\rightarrow$ S} & 0.3643 & 0.4976 & 0.3751 & 0.5719 & 0.4645   & 0.4179 & 0.3121       & 0.3012    & 0.3128  & 0.3668 & \textbf{0.2697} \\
                      & \multicolumn{1}{c|}{D $\rightarrow$ W} & 0.8666 & 0.9197 & 0.8542 & 0.9844 & 0.9270   & 0.9055 & 0.8365       & 0.8211    & 0.8335  & 0.8366 & \textbf{0.7989} \\ \cmidrule(l){2-13} 
                      & \multicolumn{1}{c|}{S $\rightarrow$ C} & 0.9613 & 0.9568 & 0.9224 & 1.0834 & 1.1211   & 0.9459 & 0.8552       & 0.8360    & 0.8569  & 0.8571 & \textbf{0.8212} \\
                      & \multicolumn{1}{c|}{S $\rightarrow$ D} & 0.5708 & 0.6306 & 0.5756 & 0.8250 & 0.7340   & 0.6727 & 0.5855       & 0.5909    & 0.6178  & 0.5936 & \textbf{0.5358} \\
                      & \multicolumn{1}{c|}{S $\rightarrow$ W} & 0.8660 & 0.9002 & 0.8439 & 1.0098 & 1.0181   & 0.8852 & 0.8310       & 0.8128    & 0.8358  & 0.8283 & \textbf{0.7999} \\ \cmidrule(l){2-13} 
                      & \multicolumn{1}{c|}{W $\rightarrow$ C} & 0.9302 & 0.9786 & 0.8965 & 0.9321 & 0.9396   & 0.8992 & 0.8690       & 0.8340    & 0.8532  & 0.8768 & \textbf{0.8171} \\
                      & \multicolumn{1}{c|}{W $\rightarrow$ D} & 0.7634 & 0.7800 & 0.6983 & 0.8647 & 0.8124   & 0.7736 & 0.6010       & 0.5997    & 0.6287  & 0.6408 & \textbf{0.5577} \\
                      & \multicolumn{1}{c|}{W $\rightarrow$ S} & 0.7244 & 0.6561 & 0.5982 & 0.7484 & 0.8592   & 0.6459 & 0.3365       & 0.3000    & 0.3238  & 0.4535 & \textbf{0.2807} \\ \midrule
\multirow{12}{*}{\rotatebox{90}{MAE}} & \multicolumn{1}{c|}{C $\rightarrow$ D} & 0.5963 & 0.6565 & 0.5817 & 0.7061 & 0.7147   & 0.6371 & 0.5007       & 0.4836    & 0.5001  & 0.5306 & \textbf{0.4678} \\
                      & \multicolumn{1}{c|}{C $\rightarrow$ S} & 0.5139 & 0.5520 & 0.4817 & 0.6354 & 0.6982   & 0.5663 & 0.3300       & 0.2656    & 0.3044  & 0.3956 & \textbf{0.2373} \\
                      & \multicolumn{1}{c|}{C $\rightarrow$ W} & 0.6453 & 0.6750 & 0.6350 & 0.7214 & 0.6983   & 0.6658 & 0.6121       & 0.6138    & 0.6124  & 0.6354 & \textbf{0.5898} \\ \cmidrule(l){2-13} 
                      & \multicolumn{1}{c|}{D $\rightarrow$ C} & 0.6723 & 0.6927 & 0.6597 & 0.7578 & 0.7384   & 0.6820 & 0.6172       & 0.6128    & 0.6237  & 0.6334 & \textbf{0.5942} \\
                      & \multicolumn{1}{c|}{D $\rightarrow$ S} & 0.3620 & 0.4326 & 0.3695 & 0.5130 & 0.4442   & 0.4032 & 0.2782       & 0.2736    & 0.2784  & 0.3465 & \textbf{0.2264} \\
                      & \multicolumn{1}{c|}{D $\rightarrow$ W} & 0.6484 & 0.6733 & 0.6372 & 0.7295 & 0.6983   & 0.6780 & 0.6116       & 0.6066    & 0.6182  & 0.6214 & \textbf{0.5886} \\ \cmidrule(l){2-13} 
                      & \multicolumn{1}{c|}{S $\rightarrow$ C} & 0.6980 & 0.6787 & 0.6858 & 0.7816 & 0.7972   & 0.6918 & 0.6137       & 0.6060    & 0.6139  & 0.6279 & \textbf{0.5920} \\
                      & \multicolumn{1}{c|}{S $\rightarrow$ D} & 0.4814 & 0.5169 & 0.4861 & 0.6663 & 0.5941   & 0.5675 & 0.4750       & 0.4745    & 0.4894  & 0.4873 & \textbf{0.4381} \\
                      & \multicolumn{1}{c|}{S $\rightarrow$ W} & 0.6387 & 0.6583 & 0.6319 & 0.7500 & 0.7344   & 0.6760 & 0.6130       & 0.6051    & 0.6116  & 0.6094 & \textbf{0.5927} \\ \cmidrule(l){2-13} 
                      & \multicolumn{1}{c|}{W $\rightarrow$ C} & 0.6648 & 0.7078 & 0.6448 & 0.6921 & 0.6900   & 0.6675 & 0.6258       & 0.6134    & 0.6233  & 0.6546 & \textbf{0.5911} \\
                      & \multicolumn{1}{c|}{W $\rightarrow$ D} & 0.6060 & 0.6321 & 0.5673 & 0.6922 & 0.6490   & 0.6252 & 0.5032       & 0.4991    & 0.5078  & 0.5391 & \textbf{0.4595} \\
                      & \multicolumn{1}{c|}{W $\rightarrow$ S} & 0.5330 & 0.5557 & 0.4813 & 0.6283 & 0.6249   & 0.5328 & 0.3361       & 0.2858    & 0.3171  & 0.4241 & \textbf{0.2447} \\
\bottomrule
\end{tabular}%
}
\end{table*}

% , where R-COAT, iTRANS, and TMixer are the abbreviation of RAINCOAT, iTransformer, and TimeMixer, respectively

\begin{table*}[]
\caption{MAE and MSE for various methods on the Human Motion dataset.}
\label{tab:human}
\setlength{\tabcolsep}{3mm}
\resizebox{\textwidth}{!}{%
\begin{tabular}{@{}c|c|ccccccccccc@{}}
\toprule
Metric                & Task              & SASA   & GCA    & DAF    & CLUDA  & R-COAT & AdvSKM & iTrans & TMixer & TSLANet & SegRNN & Ours            \\ \midrule
\multirow{12}{*}{\rotatebox{90}{MSE}} & G $\rightarrow$  E & 0.1845 & 0.2701 & 0.2210 & 0.7697 & 0.4857 & 0.4330 & 0.1061 & 0.0611 & 0.0845  & 0.0680 & \textbf{0.0543} \\
                      & G $\rightarrow$ W & 0.1666 & 0.2499 & 0.2033 & 0.6710 & 0.4078 & 0.3856 & 0.1387 & 0.1363 & 0.1354  & 0.1378 & \textbf{0.1066} \\
                      & G $\rightarrow$ S & 0.1405 & 0.1645 & 0.1655 & 0.5964 & 0.3492 & 0.3562 & 0.0630 & 0.0593 & 0.0512  & 0.0599 & \textbf{0.0439} \\ \cmidrule{2-13} 
                      & E $\rightarrow$ G & 0.2220 & 0.2535 & 0.2460 & 0.7796 & 0.5953 & 0.5227 & 0.0806 & 0.0787 & 0.0930  & 0.0821 & \textbf{0.0588} \\
                      & E $\rightarrow$ W & 0.2250 & 0.2353 & 0.2264 & 0.6706 & 0.4090 & 0.4043 & 0.2412 & 0.1414 & 0.1748  & 0.1469 & \textbf{0.1199} \\
                      & E $\rightarrow$ S & 0.1173 & 0.1210 & 0.1247 & 0.5465 & 0.3794 & 0.3204 & 0.0568 & 0.0502 & 0.0456  & 0.0493 & \textbf{0.0399} \\ \cmidrule{2-13} 
                      & W $\rightarrow$ G & 0.2295 & 0.2719 & 0.2754 & 0.8068 & 0.6679 & 0.5823 & 0.0992 & 0.0775 & 0.0840  & 0.0858 & \textbf{0.0582} \\
                      & W $\rightarrow$ E & 0.2183 & 0.2616 & 0.2386 & 0.8086 & 0.6266 & 0.5130 & 0.1382 & 0.0741 & 0.0605  & 0.0965 & \textbf{0.0566} \\
                      & W $\rightarrow$ S & 0.1455 & 0.1506 & 0.1687 & 0.6043 & 0.3765 & 0.3481 & 0.0705 & 0.0525 & 0.0501  & 0.0712 & \textbf{0.0423} \\ \cmidrule{2-13} 
                      & S $\rightarrow$ G & 0.1637 & 0.1958 & 0.1790 & 0.7551 & 0.5090 & 0.3823 & 0.0909 & 0.0767 & 0.0728  & 0.0841 & \textbf{0.0637} \\
                      & S $\rightarrow$ E & 0.1555 & 0.1716 & 0.1505 & 0.7503 & 0.5252 & 0.3510 & 0.0866 & 0.0567 & 0.0677  & 0.0794 & \textbf{0.0564} \\
                      & S $\rightarrow$ W & 0.2012 & 0.2597 & 0.2524 & 0.6825 & 0.4572 & 0.4051 & 0.2182 & 0.1514 & 0.1632  & 0.1419 & \textbf{0.1274} \\ \midrule
\multirow{12}{*}{\rotatebox{90}{MAE}} & G $\rightarrow$ E & 0.2982 & 0.3700 & 0.3307 & 0.6734 & 0.5035 & 0.4933 & 0.2159 & 0.1274 & 0.1656  & 0.1469 & \textbf{0.1146} \\
                      & G $\rightarrow$ W & 0.2816 & 0.3479 & 0.3231 & 0.6339 & 0.4431 & 0.4618 & 0.2043 & 0.1910 & 0.2069  & 0.1958 & \textbf{0.1595} \\
                      & G $\rightarrow$ S & 0.2517 & 0.2751 & 0.2714 & 0.5885 & 0.3939 & 0.4386 & 0.1450 & 0.0975 & 0.1088  & 0.1278 & \textbf{0.0807} \\ \cmidrule{2-13} 
                      & E $\rightarrow$ G & 0.3200 & 0.3499 & 0.3502 & 0.6759 & 0.5531 & 0.5384 & 0.1628 & 0.1601 & 0.1890  & 0.1696 & \textbf{0.1310} \\
                      & E $\rightarrow$ W & 0.3204 & 0.3345 & 0.3301 & 0.6032 & 0.4569 & 0.4661 & 0.3211 & 0.1879 & 0.2171  & 0.2060 & \textbf{0.1658} \\
                      & E $\rightarrow$ S & 0.2383 & 0.2331 & 0.2474 & 0.5641 & 0.4390 & 0.4222 & 0.1280 & 0.0821 & 0.0898  & 0.0950 & \textbf{0.0785} \\ \cmidrule{2-13} 
                      & W $\rightarrow$ G & 0.3164 & 0.3466 & 0.3612 & 0.6755 & 0.5688 & 0.5534 & 0.2058 & 0.1688 & 0.1798  & 0.1674 & \textbf{0.1327} \\
                      & W $\rightarrow$ E & 0.3138 & 0.3566 & 0.3517 & 0.7005 & 0.5800 & 0.5419 & 0.2510 & 0.1606 & 0.1221  & 0.1827 & \textbf{0.1191} \\
                      & W $\rightarrow$ S & 0.2505 & 0.2556 & 0.2738 & 0.5988 & 0.4265 & 0.4373 & 0.1517 & 0.1034 & 0.0989  & 0.1534 & \textbf{0.0819} \\ \cmidrule{2-13} 
                      & S $\rightarrow$ G & 0.2783 & 0.3091 & 0.3004 & 0.6746 & 0.5065 & 0.4577 & 0.1917 & 0.1574 & 0.1521  & 0.1690 & \textbf{0.1369} \\
                      & S $\rightarrow$ E & 0.2799 & 0.2971 & 0.2784 & 0.6591 & 0.4972 & 0.4494 & 0.1812 & 0.1163 & 0.1319  & 0.1526 & \textbf{0.1152} \\
                      & S $\rightarrow$ W & 0.2946 & 0.3423 & 0.3422 & 0.6153 & 0.4572 & 0.4766 & 0.2833 & 0.1949 & 0.2007  & 0.1885 & \textbf{0.1689} \\ \bottomrule
\end{tabular}
}
\end{table*}

% , where R-COAT, iTRANS, and TMixer are the abbreviation of RAINCOAT, iTransformer, and TimeMixer, respectively.

\begin{table*}[]
\caption{MAE and MSE of various methods on the ETT dataset.}
\label{tab:ett}
\setlength{\tabcolsep}{3mm}
\resizebox{\textwidth}{!}{%
\begin{tabular}{@{}c|c|ccccccccccc@{}}
\toprule
Metric               & Task               & SASA   & GCA    & DAF    & CLUDA  & R-COAT & AdvSKM & iTrans & TMixer & TSLANet & SegRNN & Ours            \\ \midrule
\multirow{2}{*}{MSE} & 1 $\rightarrow$  2 & 0.2843 & 0.2820 & 0.1812 & 0.4097 & 0.3930 & 0.2839 & 0.1439 & 0.1306 & 0.1419  & 0.1775 & \textbf{0.1023} \\
                     & 2 $\rightarrow$ 1  & 0.8068 & 0.8421 & 0.6438 & 0.9711 & 0.8921 & 0.8352 & 0.5848 & 0.6620 & 0.6790  & 0.8740 & \textbf{0.4395} \\ \midrule
\multirow{2}{*}{MAE} & 1 $\rightarrow$  2 & 0.3977 & 0.4188 & 0.3248 & 0.5126 & 0.5083 & 0.4117 & 0.2764 & 0.2585 & 0.2751  & 0.3141 & \textbf{0.2195} \\
                     & 2 $\rightarrow$ 1  & 0.6686 & 0.6670 & 0.5925 & 0.7636 & 0.6796 & 0.6609 & 0.5493 & 0.5696 & 0.5693  & 0.6508 & \textbf{0.4482} \\ \bottomrule
\end{tabular}%

}
\end{table*}

\begin{table*}[]
\caption{MAE and MSE of various methods on the PEMS dataset.}
\label{tab:traffc}
\setlength{\tabcolsep}{3mm}
\resizebox{\textwidth}{!}{%
\begin{tabular}{@{}c|c|ccccccccccc@{}}
\toprule
Metric               & Task              & SASA   & GCA    & DAF    & CLUDA  & R-COAT & AdvSKM & iTrans & TMixer & TSLANet & SegRNN & Ours            \\ \midrule
\multirow{6}{*}{\rotatebox{90}{MSE}} & 1 $\rightarrow$ 2 & 0.3249 & 0.5068 & 0.4412 & 0.4223 & 0.4751 & 0.4176 & 0.4134 & 0.3698 & 0.3558  & 0.3312 & \textbf{0.2629} \\
                     & 1 $\rightarrow$ 3 & 0.3210 & 0.4788 & 0.3806 & 0.4055 & 0.4220 & 0.4045 & 0.3695 & 0.3342 & 0.3055  & 0.2769 & \textbf{0.2244} \\ \cmidrule{2-13} 
                     & 2 $\rightarrow$ 1 & 0.3035 & 0.4866 & 0.4356 & 0.4139 & 0.4024 & 0.3975 & 0.3499 & 0.3023 & 0.2886  & 0.2996 & \textbf{0.2393} \\
                     & 2 $\rightarrow$ 3 & 0.3100 & 0.4383 & 0.3694 & 0.3832 & 0.3858 & 0.3802 & 0.3682 & 0.2980 & 0.2871  & 0.2681 & \textbf{0.2221} \\ \cmidrule{2-13} 
                     & 3 $\rightarrow$ 1 & 0.3423 & 0.5112 & 0.4492 & 0.4627 & 0.4553 & 0.4155 & 0.4064 & 0.3377 & 0.3225  & 0.2952 & \textbf{0.2372} \\
                     & 3 $\rightarrow$ 2 & 0.2894 & 0.4427 & 0.3820 & 0.3620 & 0.4485 & 0.3572 & 0.4083 & 0.3729 & 0.3657  & 0.3515 & \textbf{0.2655} \\ \midrule
\multirow{6}{*}{\rotatebox{90}{MAE}} & 1 $\rightarrow$ 2 & 0.3237 & 0.3988 & 0.3906 & 0.3585 & 0.4041 & 0.3577 & 0.3415 & 0.3398 & 0.3162  & 0.2944 & \textbf{0.2392} \\
                     & 1 $\rightarrow$ 3 & 0.3463 & 0.4209 & 0.3780 & 0.3715 & 0.4044 & 0.3778 & 0.3533 & 0.3393 & 0.3061  & 0.2815 & \textbf{0.2327} \\ \cmidrule{2-13} 
                     & 2 $\rightarrow$ 1 & 0.3332 & 0.4105 & 0.4025 & 0.3727 & 0.3996 & 0.3630 & 0.3308 & 0.3200 & 0.2961  & 0.3058 & \textbf{0.2445} \\
                     & 2 $\rightarrow$ 3 & 0.3385 & 0.3872 & 0.3629 & 0.3532 & 0.3727 & 0.3565 & 0.3472 & 0.3139 & 0.3009  & 0.2848 & \textbf{0.2327} \\ \cmidrule{2-13} 
                     & 3 $\rightarrow$ 1 & 0.3534 & 0.4326 & 0.4063 & 0.3980 & 0.4238 & 0.3813 & 0.3855 & 0.3311 & 0.3310  & 0.2971 & \textbf{0.2425} \\
                     & 3 $\rightarrow$ 2 & 0.3116 & 0.3691 & 0.3496 & 0.3163 & 0.3928 & 0.3164 & 0.3503 & 0.3343 & 0.3245  & 0.3163 & \textbf{0.2433} \\ \bottomrule
\end{tabular}%
}
\end{table*}


\subsection{Domain Adaptation for Time Series Forecasting}

\subsubsection{Datasets}\quad In this section, we provide an overview of the five real-world datasets used to evaluate the \textbf{LCA} model. \textbf{PPG-DaLiA\footnote{\scriptsize 
https://archive.ics.uci.edu/ml/datasets/PPG-DaLiA}} is a publicly available multimodal dataset, used for PPG-based heart rate estimation. It includes physiological and motion data collected from 15 volunteers using wrist and chest-worn devices during various activities. We categorize the data into four domains based on activities: Cycling (C), Sitting (S), Working (W), and Driving (D). \textbf{Human Motion\footnote{\scriptsize 
 http://vision.imar.ro/human3.6m/description.php}} is a dataset for human motion prediction and cross-domain adaptation. We select four motion types as domains: Walking (W), Greeting (G), Eating (E), and Smoking (S). \textbf{Electricity Load Diagrams\footnote{\scriptsize https://archive.ics.uci.edu/dataset/321/electricityloaddiagrams20112014}} is a dataset containing electricity consumption data from 370 substations in Portugal, recorded from January 2011 to December 2014. We account for seasonal domain shifts by dividing the data into four domains based on the months: Domain 1 (January, February, March), Domain 2 (April, May, June), Domain 3 (July, August, September), and Domain 4 (October, November, December). \textbf{PEMS\footnote{\scriptsize 
 https://pems.dot.ca.gov/}} dataset comprises traffic speed data collected by the California Transportation Agencies over a 6-month period from January 1st, 2017 to May 31st, 2017. To account for seasonal domain shifts, we divide the data into three domains based on months: Domain 1 (January), Domain 2 (February), and Domain 3 (March). \textbf{ETT\footnote{\scriptsize 
 https://github.com/zhouhaoyi/ETDataset}} dataset describes the electric power deployment. We use the ETT-small subset, which includes data from two stations, treating each station as a separate domain.
% \begin{itemize}
%     % \item \textbf{PPG-DaLiA\footnote{\scriptsize 
%  % https://archive.ics.uci.edu/ml/datasets/PPG-DaLiA}:} This publicly available multimodal dataset is used for PPG-based heart rate estimation. It includes physiological and motion data collected from 15 volunteers using wrist and chest-worn devices during various activities. We categorize the data into four domains based on activities: Cycling (C), Sitting (S), Working (W), and Driving (D).
    
%  %    \item \textbf{Human Motion\footnote{\scriptsize 
%  % http://vision.imar.ro/human3.6m/description.php}:} A benchmark dataset for human motion prediction and cross-domain adaptation. We select four motion types as domains: Walking(W), Greeting(G), Eating(E), and Smoking(S).
    
%     % \item \textbf{Electricity Load Diagrams\footnote{\scriptsize https://archive.ics.uci.edu/dataset/321/electricityloaddiagrams20112014}:} This dataset contains electricity consumption data from 370 substations in Portugal, recorded from January 2011 to December 2014. We account for seasonal domain shifts by dividing the data into four domains based on the months: Domain 1 (January, February, March), Domain 2 (April, May, June), Domain 3 (July, August, September), and Domain 4 (October, November, December).

%  %    \item \textbf{PEMS\footnote{\scriptsize 
%  % https://pems.dot.ca.gov/}:} This dataset comprises traffic speed data collected by the California Transportation Agencies over a 6-month period from January 1st, 2017 to May 31st, 2017. To account for seasonal domain shifts, we divide the data into three domains based on months: Domain 1 (January), Domain 2 (February), and Domain 3 (March).

%     \item \textbf{ETT\footnote{\scriptsize 
%  https://github.com/zhouhaoyi/ETDataset}:} The ETT dataset focuses on electric power deployment. We use the ETT-small subset, which includes data from two stations, treating each station as a separate domain.
% \end{itemize}

% These datasets provide varied and challenging scenarios for testing the LCA model’s performance.
% \vspace{0.5em}\\
% \noindent 
\subsubsection{Baselines}  Here, we introduce the benchmarks for unsupervised domain adaptation in time series forecasting, including SASA\cite{Cai_Chen_Li_Chen_Zhang_Ye_Li_Yang_Zhang_2021}, AdvSKM\cite{liu2021adversarial}, DAF\cite{jin2022domain}, GCA\cite{li2023transferable}, CLUDA\cite{ozyurt2022contrastive}, and Raincoat\cite{he2023domain}. In addition, we include approaches that integrate the Gradient Reversal Layer (GRL) with state-of-the-art time series forecasting techniques such as SegRNN\cite{lin2023segrnn}, TSLANet\cite{tslanet}, TimeMixer\cite{wang2023timemixer}, and iTransformer\cite{liu2023itransformer}, which are based on RNN, CNN, MLP, and Transformer architectures, respectively. For all the above methods, we ensure consistency by employing the same experimental settings among them.

\noindent\subsubsection{Experimental Settings}\quad We performed multivariate time series forecasting across all datasets, using an input length of 30 and an output length of 10. Each dataset is partitioned into train, validation, and test sets. To ensure robustness, we run each method three times with different random seeds and report the average performance. The model yielding the best validation results is selected, and its performance is subsequently assessed on the test set. In all experiments, we employed the ADAM optimizer \cite{kingma2014adam} and used Mean Squared Error (MSE) as the loss function for prediction. We report the Mean Squared Error (MSE) and Mean Absolute Error (MAE) as evaluation metrics. 

\noindent\subsubsection{Quantization Result}\quad We conducted comparative experiments between our method and baseline models across five datasets, and the results are presented in Tables~\ref{tab:ppg},~\ref{tab:human},~\ref{tab:ett},~\ref{tab:traffc},~and~\ref{tab:eld}, respectively. The results show that our method significantly outperforms the baseline models across all metrics, particularly on the Human Motion Dataset, where causal mechanisms are more pronounced (since the relationships between human joints represent a natural causal structure). This further highlights the superiority of our approach.

Additionally, we find that combining state-of-the-art time series forecasting methods with a gradient reversal layer to address UDA in time series forecasting also yielded excellent results. This is clear in the results of iTransformer, TimeMixer, TSLAnet, and SegRNN specifically, where these methods leverage the most advanced models in their model architectures (iTransformer for Transformers, TimeMixer for MLPs, TSLAnet for CNNs, and SegRNN for RNNs). When combined with a gradient reversal layer, these methods effectively capture domain-shared features, enhancing prediction performance. Despite their strong backbones, our approach still outperforms these baselines. Unlike TimeMixer, which captures complex macro and micro temporal information by decomposing trends and seasonal information using a carefully designed MLP, our method uses a simpler MLP learning network, further demonstrating our superiority.

Furthermore, existing UDA methods in the time series domain, such as DAF, CLUDA, Raincoat, and AdvSKM, exhibited 
relatively weaker performance, possibly due to their less optimal backbone models.  However, we notice that SASA and GCA perform better than other time series UDA methods, such as DAF, CLUDA, Raincoat, and AdvSKM. This can be attributed to their consideration of sparse correlation structures among observed variables and the consistency of Granger causality structures, despite having relatively simple backbones. 
% The relatively weaker performance of GCA compared to SASA can be attributed to GCA's use of recursive forecasting, where errors tend to accumulate over multiple time steps, ultimately affecting prediction accuracy.

% \begin{figure*}[h]
%     \centering
%     \includegraphics[width=1.01\textwidth, trim=0cm  12cm 0cm 0cm, clip]{fig/ett_1_to_2_predict.pdf}
%     \caption{Visualization of prediction results across varying forecast lengths for the transition from domain 1 to domain 2 in the ETT dataset. Subfigures (a), (b), (c), and (d) represent forecast lengths of 10, 20, 30, and 40, respectively.}
%     \label{fig:diff_len_mase}
% \end{figure*}

\begin{figure}[h]
    \centering
    \includegraphics[width=0.5\textwidth, trim=0.7cm  0cm 0.5cm 0.7cm, clip]{fig/new_length_compare.png}
    \caption{The MSE and MAE values after predicting different lengths in the transition from domain 1 to domain 2 in the ETT dataset. Subfigures (a), (b), (c), and (d) show the forecasting results on lengths of 10, 20, 30, and 40 time steps, respectively.}
    \label{fig:diff_len_v}
\end{figure}

\noindent\textbf{Visualization Result:}\quad To demonstrate the robustness of our method, we performed predictions of varying lengths on the ETT dataset, specifically from domain 1 to domain 2. For each input of the 30 time steps, we predicted future values over 10, 20, 30, and 40 steps. The resulting MSE and MAE values for these different forecast lengths are illustrated in Figure \ref{fig:diff_len_mase}. Notably, our method consistently outperforms most baseline methods that only predict 10 steps, particularly for longer forecast horizons (20, 30, and 40 steps), highlighting the superior performance of our approach. Additionally, we visualize the predictions for the last dimension of the ETT dataset, as shown in Figure \ref{fig:diff_len_v}. The visualization confirms that our method effectively captures the temporal variations in the data.


% 我们在5个数据集上与将我们的方法与baseline进行了实验,实验结果如表1,2,3,4,5所示。从实验结果中,我们可以看出我们的方法大幅度领先于baseline方法,尤其在human Motion Capture Dataset上,从中可以看出我们方法的优越性。此外,1.我们发现将最新的时序预测方法,iTransformer,TimeMixer,TSLAnet以及SegRNN与梯度反转层结合得到的用于解决时序预测UDA的方法,R-iTransformer,R-TimeMixer,R-TSLAnet和R-SegRNN在UDA预测效果上也有不错的表现。这主要归功于这些方法都是分别基于Transformer,MLP,CNN,RNN中的最先进的,表现最好的时序预测方法,因此这些方法能够提取更有效的时序表征,因此当与梯度反转层结合后,能够更加有效的提取到领域共有的表征,从而获得更好的预测效果。然而尽管这些方法拥有强大的模型作为backbone,但是我们的方法依旧能够领先于它们,与基于MLP的,采用多尺度地分解趋势和季节信息,用于得到复杂的宏观信息以及微观时序信息的TimeMixer方法相比,我们的方法只是简单的使用MLP作为学习网络,更显示出了我们方法的优越性。2.原本时序领域上的UDA方法,DAF,CLUDA,Raincoat,AdvSKM表现较次,可能的原因是采用的backbone不是最优的。3.考虑观测变量上稀疏关联结构以及格兰杰因果结构是一致的方法,尽管采用的backbone也是普通的,但是相比与DAF,CLUDA,Raincoat,AdvSKM相比,有更好的表现。之所以GCA相比于SASA有更差的表现,是因为GCA是递归预测,因此当预测多个时间步时,会导致误差积累。
% Please add the following required packages to your document preamble:
% \usepackage{multirow}
% \begin{table*}[]
% \centering
% \small
% \caption{The MAE and MSE on PPG-DaLiA dataset for the baselines and }
% \resizebox{\textwidth}{!}{%
% \begin{tabular}{c|cccccccccccc}
% \toprule
% Task                                & Metric                   & SASA   & GCA    & DAF    & CLUDA  & Raincoat & AdvSKM & iTransformer & TimeMixer & TSLANet & SegRNN & Our             \\ \midrule
% \multirow{2}{*}{C to D}   & \multicolumn{1}{c|}{mse} & 0.7421 & 0.8260 & 0.7085 & 0.8869 & 0.9117   & 0.7795 & 0.6036       & 0.5901    & 0.6162  & 0.6333 & \textbf{0.5797} \\
%                                     & \multicolumn{1}{c|}{mae} & 0.5963 & 0.6565 & 0.5817 & 0.7061 & 0.7147   & 0.6371 & 0.5007       & 0.4836    & 0.5001  & 0.5306 & \textbf{0.4678} \\ \midrule
% \multirow{2}{*}{C to S} & \multicolumn{1}{c|}{mse} & 0.6279 & 0.6898 & 0.5726 & 0.7851 & 0.8675   & 0.6998 & 0.3433       & 0.2962    & 0.3139  & 0.4259 & \textbf{0.2842} \\
%                                     & \multicolumn{1}{c|}{mae} & 0.5139 & 0.5520 & 0.4817 & 0.6354 & 0.6982   & 0.5663 & 0.3300       & 0.2656    & 0.3044  & 0.3956 & \textbf{0.2373} \\ \midrule
% \multirow{2}{*}{C to W} & \multicolumn{1}{c|}{mse} & 0.8767 & 0.9076 & 0.8565 & 0.9709 & 0.9356   & 0.8846 & 0.8117       & 0.8162    & 0.8226  & 0.8452 & \textbf{0.7946} \\
%                                     & \multicolumn{1}{c|}{mae} & 0.6453 & 0.6750 & 0.6350 & 0.7214 & 0.6983   & 0.6658 & 0.6121       & 0.6138    & 0.6124  & 0.6354 & \textbf{0.5898} \\ \midrule
% \multirow{2}{*}{D to C}   & \multicolumn{1}{c|}{mse} & 0.9415 & 0.9858 & 0.9258 & 1.0562 & 1.0134   & 0.9539 & 0.8603       & 0.8476    & 0.8623  & 0.8599 & \textbf{0.8294} \\
%                                     & \multicolumn{1}{c|}{mae} & 0.6723 & 0.6927 & 0.6597 & 0.7578 & 0.7384   & 0.6820 & 0.6172       & 0.6128    & 0.6237  & 0.6334 & \textbf{0.5942} \\ \midrule
% \multirow{2}{*}{D to S}   & \multicolumn{1}{c|}{mse} & 0.3643 & 0.4976 & 0.3751 & 0.5719 & 0.4645   & 0.4179 & 0.3121       & 0.3012    & 0.3128  & 0.3668 & \textbf{0.2697} \\
%                                     & \multicolumn{1}{c|}{mae} & 0.3620 & 0.4326 & 0.3695 & 0.5130 & 0.4442   & 0.4032 & 0.2782       & 0.2736    & 0.2784  & 0.3465 & \textbf{0.2264} \\ \midrule
% \multirow{2}{*}{D to W}   & \multicolumn{1}{c|}{mse} & 0.8666 & 0.9197 & 0.8542 & 0.9844 & 0.9270   & 0.9055 & 0.8365       & 0.8211    & 0.8335  & 0.8366 & \textbf{0.7989} \\
%                                     & \multicolumn{1}{c|}{mae} & 0.6484 & 0.6733 & 0.6372 & 0.7295 & 0.6983   & 0.6780 & 0.6116       & 0.6066    & 0.6182  & 0.6214 & \textbf{0.5886} \\ \midrule
% \multirow{2}{*}{S to C} & \multicolumn{1}{c|}{mse} & 0.9613 & 0.9568 & 0.9224 & 1.0834 & 1.1211   & 0.9459 & 0.8552       & 0.8360    & 0.8569  & 0.8571 & \textbf{0.8212} \\
%                                     & \multicolumn{1}{c|}{mae} & 0.6980 & 0.6787 & 0.6858 & 0.7816 & 0.7972   & 0.6918 & 0.6137       & 0.6060    & 0.6139  & 0.6279 & \textbf{0.5920} \\ \midrule
% \multirow{2}{*}{S to D}   & \multicolumn{1}{c|}{mse} & 0.5708 & 0.6306 & 0.5756 & 0.8250 & 0.7340   & 0.6727 & 0.5855       & 0.5909    & 0.6178  & 0.5936 & \textbf{0.5358} \\
%                                     & \multicolumn{1}{c|}{mae} & 0.4814 & 0.5169 & 0.4861 & 0.6663 & 0.5941   & 0.5675 & 0.4750       & 0.4745    & 0.4894  & 0.4873 & \textbf{0.4381} \\ \midrule
% \multirow{2}{*}{S to W} & \multicolumn{1}{c|}{mse} & 0.8660 & 0.9002 & 0.8439 & 1.0098 & 1.0181   & 0.8852 & 0.8310       & 0.8128    & 0.8358  & 0.8283 & \textbf{0.7999} \\
%                                     & \multicolumn{1}{c|}{mae} & 0.6387 & 0.6583 & 0.6319 & 0.7500 & 0.7344   & 0.6760 & 0.6130       & 0.6051    & 0.6116  & 0.6094 & \textbf{0.5927} \\ \midrule
% \multirow{2}{*}{W to C} & \multicolumn{1}{c|}{mse} & 0.9302 & 0.9786 & 0.8965 & 0.9321 & 0.9396   & 0.8992 & 0.8690       & 0.8340    & 0.8532  & 0.8768 & \textbf{0.8171} \\
%                                     & \multicolumn{1}{c|}{mae} & 0.6648 & 0.7078 & 0.6448 & 0.6921 & 0.6900   & 0.6675 & 0.6258       & 0.6134    & 0.6233  & 0.6546 & \textbf{0.5911} \\ \midrule
% \multirow{2}{*}{W to D}   & \multicolumn{1}{c|}{mse} & 0.7634 & 0.7800 & 0.6983 & 0.8647 & 0.8124   & 0.7736 & 0.6010       & 0.5997    & 0.6287  & 0.6408 & \textbf{0.5577} \\
%                                     & \multicolumn{1}{c|}{mae} & 0.6060 & 0.6321 & 0.5673 & 0.6922 & 0.6490   & 0.6252 & 0.5032       & 0.4991    & 0.5078  & 0.5391 & \textbf{0.4595} \\ \midrule
% \multirow{2}{*}{W to S} & \multicolumn{1}{c|}{mse} & 0.7244 & 0.6561 & 0.5982 & 0.7484 & 0.8592   & 0.6459 & 0.3365       & 0.3000    & 0.3238  & 0.4535 & \textbf{0.2807} \\
%                                     & \multicolumn{1}{c|}{mae} & 0.5330 & 0.5557 & 0.4813 & 0.6283 & 0.6249   & 0.5328 & 0.3361       & 0.2858    & 0.3171  & 0.4241 & \textbf{0.2447} \\ \bottomrule
% \end{tabular}
% }
% % Please add the following required packages to your document preamble:
% % \usepackage{multirow}
% \label{tab:ppg}
% \end{table*}
% Please add the following required packages to your document preamble:
% \usepackage{booktabs}
% \usepackage{multirow}
% \usepackage{graphicx}
% Please add the following required packages to your document preamble:
% \usepackage{booktabs}
% \usepackage{multirow}
% \usepackage{graphicx}


% Please add the following required packages to your document preamble:
% \usepackage{multirow}
% \usepackage{graphicx}


% Please add the following required packages to your document preamble:
% \usepackage{multirow}
% \usepackage{graphicx}


% Please add the following required packages to your document preamble:
% \usepackage{multirow}
% \usepackage{graphicx}


% Please add the following required packages to your document preamble:
% \usepackage{multirow}
% \usepackage{graphicx}








\subsection{Domain Adaptation for Time Series Classification}
In classification tasks, we experimented on the UCIHAR and HHAR datasets, following AdaTime \cite{ragab2023adatime} framework, which is a benchmarking suite for domain adaptation on time series data. Furthermore, we validate the performance of our method on the high-dimensional video classification datasets.

% \begin{table*}[htbp]
% \caption{The MAE and MSE on the Electricity Load Diagrams dataset, where R-COAT, iTRANS, and TMixer are the abbreviation of RAINCOAT, iTransformer, and
% TimeMixer, respectively.}
% \label{tab:eld}
% \setlength{\tabcolsep}{3mm}
% \resizebox{\textwidth}{!}{%
% \begin{tabular}{@{}c|c|ccccccccccc@{}}
% \toprule
% Metric                & Task              & SASA            & GCA    & DAF    & CLUDA  & R-COAT & AdvSKM & iTRANS & TMixer & TSLANet & SegRNN & Ours            \\ \midrule
% \multirow{12}{*}{\rotatebox{90}{MSE}} & 1 $\rightarrow$ 2 & 0.1781          & 0.2919 & 0.2142 & 0.2412 & 0.2472 & 0.2143 & 0.1778 & 0.2066 & 0.2551  & 0.2675 & \textbf{0.1452} \\
%                       & 1 $\rightarrow$ 3 & 0.1544          & 0.2810 & 0.1772 & 0.2133 & 0.2251 & 0.1879 & 0.1723 & 0.2064 & 0.2464  & 0.2356 & \textbf{0.1410} \\
%                       & 1 $\rightarrow$ 4 & 0.1855          & 0.3421 & 0.2043 & 0.2689 & 0.2804 & 0.2205 & 0.2121 & 0.3375 & 0.2763  & 0.3018 & \textbf{0.1586} \\ \cmidrule{2-13} 
%                       & 2 $\rightarrow$ 1 & 0.1216          & 0.2062 & 0.1426 & 0.1503 & 0.1523 & 0.1217 & 0.1758 & 0.2742 & 0.2401  & 0.2372 & \textbf{0.1171} \\
%                       & 2 $\rightarrow$ 3 & 0.1315          & 0.2097 & 0.1527 & 0.1643 & 0.1625 & 0.1372 & 0.1459 & 0.2249 & 0.2172  & 0.2144 & \textbf{0.1129} \\
%                       & 2 $\rightarrow$ 4 & 0.1892          & 0.3018 & 0.2480 & 0.2528 & 0.2285 & 0.2157 & 0.2084 & 0.3372 & 0.3161  & 0.2903 & \textbf{0.1587} \\ \cmidrule{2-13} 
%                       & 3 $\rightarrow$ 1 & \textbf{0.1360} & 0.2514 & 0.1550 & 0.1560 & 0.1922 & 0.1473 & 0.1927 & 0.2840 & 0.2612  & 0.2512 & \textbf{0.1384} \\
%                       & 3 $\rightarrow$ 2 & 0.1181          & 0.2262 & 0.1305 & 0.1468 & 0.1625 & 0.1190 & 0.1577 & 0.2371 & 0.2281  & 0.2250 & \textbf{0.1113} \\
%                       & 3 $\rightarrow$ 4 & 0.1874          & 0.3277 & 0.2166 & 0.2835 & 0.2868 & 0.2505 & 0.2327 & 0.3328 & 0.3054  & 0.2928 & \textbf{0.1839} \\ \cmidrule{2-13} 
%                       & 4 $\rightarrow$ 1 & 0.1203          & 0.2496 & 0.1277 & 0.1416 & 0.1589 & 0.1206 & 0.1791 & 0.2380 & 0.2449  & 0.2308 & \textbf{0.1063} \\
%                       & 4 $\rightarrow$ 2 & 0.1442          & 0.2591 & 0.1636 & 0.1851 & 0.1829 & 0.1600 & 0.1655 & 0.2140 & 0.2493  & 0.2340 & \textbf{0.1176} \\
%                       & 4 $\rightarrow$ 3 & 0.1271          & 0.2153 & 0.1372 & 0.1670 & 0.1734 & 0.1386 & 0.1558 & 0.2779 & 0.2435  & 0.2155 & \textbf{0.1100} \\ \midrule
% \multirow{12}{*}{\rotatebox{90}{MAE}} & 1 $\rightarrow$ 2 & 0.3103          & 0.4131 & 0.3508 & 0.3682 & 0.3866 & 0.3494 & 0.3116 & 0.3264 & 0.3794  & 0.3873 & \textbf{0.2821} \\
%                       & 1 $\rightarrow$ 3 & 0.2959          & 0.4107 & 0.3228 & 0.3531 & 0.3717 & 0.3343 & 0.3127 & 0.3305 & 0.3760  & 0.3606 & \textbf{0.2816} \\
%                       & 1 $\rightarrow$ 4 & 0.3270          & 0.4548 & 0.3498 & 0.3999 & 0.4167 & 0.3667 & 0.3511 & 0.4419 & 0.3972  & 0.4147 & \textbf{0.3007} \\ \cmidrule{2-13} 
%                       & 2 $\rightarrow$ 1 & 0.2593          & 0.3396 & 0.2842 & 0.2946 & 0.2954 & 0.2595 & 0.3065 & 0.3883 & 0.3638  & 0.3570 & \textbf{0.2519} \\
%                       & 2 $\rightarrow$ 3 & 0.2722          & 0.3491 & 0.2988 & 0.3061 & 0.3105 & 0.2817 & 0.2842 & 0.3569 & 0.3512  & 0.3421 & \textbf{0.2531} \\
%                       & 2 $\rightarrow$ 4 & 0.3286          & 0.4238 & 0.3879 & 0.3752 & 0.3662 & 0.3447 & 0.3411 & 0.4391 & 0.4321  & 0.4046 & \textbf{0.2961} \\ \cmidrule{2-13} 
%                       & 3 $\rightarrow$ 1 & 0.2744          & 0.3797 & 0.2967 & 0.2976 & 0.3305 & 0.2886 & 0.3238 & 0.4012 & 0.3880  & 0.3760 & \textbf{0.2740} \\
%                       & 3 $\rightarrow$ 2 & 0.2532          & 0.3584 & 0.2707 & 0.2860 & 0.3059 & 0.2564 & 0.2960 & 0.3635 & 0.3577  & 0.3508 & \textbf{0.2472} \\
%                       & 3 $\rightarrow$ 4 & 0.3240          & 0.4367 & 0.3572 & 0.3975 & 0.4084 & 0.3802 & 0.3650 & 0.4405 & 0.4227  & 0.4090 & \textbf{0.3202} \\ \cmidrule{2-13} 
%                       & 4 $\rightarrow$ 1 & 0.2564          & 0.3819 & 0.2672 & 0.2833 & 0.3021 & 0.2592 & 0.3114 & 0.3549 & 0.3734  & 0.3517 & \textbf{0.2391} \\
%                       & 4 $\rightarrow$ 2 & 0.2794          & 0.3910 & 0.3028 & 0.3212 & 0.3236 & 0.2952 & 0.3004 & 0.3388 & 0.3742  & 0.3599 & \textbf{0.2527} \\
%                       & 4 $\rightarrow$ 3 & 0.2642          & 0.3537 & 0.2802 & 0.3108 & 0.3185 & 0.2831 & 0.2950 & 0.3797 & 0.3744  & 0.3396 & \textbf{0.2458} \\ \bottomrule
% \end{tabular}%
% }
% \end{table*}



\begin{table*}[]
\small
\caption{F1-scores of various methods on the UCIHAR time series dataset.}
\resizebox{\textwidth}{!}{%
% Please add the following required packages to your document preamble:
% \usepackage{multirow}
\begin{tabular}{ccccccccccc}
\toprule
\multicolumn{1}{c|}{Task}       & AdvSKM & CoDATS & CoTimix & DANN   & DDC    & DeepCoral & DIRT   & MMDA   & SASA   & Ours            \\ \midrule
\multicolumn{1}{c|}{18 $\rightarrow$ 14} & 0.9773 & 0.8988 & 0.9232  & 0.7729 & 0.9804 & 0.9902      & 1.0000 & 0.9869 & 0.9869 & \textbf{1.0000} \\ \midrule
\multicolumn{1}{c|}{6 $\rightarrow$ 13}  & 0.9864 & 0.9932 & 0.9386  & 0.7064 & 0.9931 & 1.0000      & 1.0000 & 0.9778 & 0.9935 & \textbf{1.0000} \\ \midrule
\multicolumn{1}{c|}{20 $\rightarrow$ 9}  & 0.3835 & 0.5649 & 0.5527  & 0.3885 & 0.3730 & 0.5871      & 0.5458 & 0.5356 & 0.4860 & \textbf{0.6946} \\ \midrule
\multicolumn{1}{c|}{7 $\rightarrow$ 18}  & 0.7826 & 0.7813 & 0.8429  & 0.4479 & 0.7604 & 0.8246      & 0.8501 & 0.7723 & 0.7414 & \textbf{0.9108} \\ \midrule
\multicolumn{1}{c|}{19 $\rightarrow$ 11} & 0.6765 & 0.9890 & 0.9388  & 0.5348 & 0.6999 & 0.9823      & 0.8651 & 0.9386 & 0.8341 & \textbf{0.9963} \\ \midrule
\multicolumn{1}{c|}{17 $\rightarrow$ 18} & 0.8742 & 0.8730 & 0.9011  & 0.4884 & 0.8767 & 0.9507      & 0.8820 & 0.9191 & 0.7520 & \textbf{0.9601} \\ \midrule
\multicolumn{1}{c|}{9 $\rightarrow$ 19}  & 0.4269 & 0.6956 & 0.8939  & 0.6113 & 0.3869 & 0.7467      & 0.8290 & 0.6182 & 0.6000 & \textbf{0.9692} \\ \midrule
\multicolumn{1}{c|}{2 $\rightarrow$ 12}  & 0.9633 & 0.8191 & 0.8775  & 0.6701 & 0.9857 & 0.9886      & 0.9859 & 0.9963 & 1.0000 & \textbf{1.0000} \\ \midrule
\multicolumn{1}{c|}{12 $\rightarrow$ 3}  & 0.9609 & 0.9744 & 0.9904  & 0.5566 & 0.9678 & 0.9968      & 0.9266 & 0.9936 & 0.9936 & \textbf{1.0000} \\ \midrule
\multicolumn{1}{c|}{17 $\rightarrow$ 14}                     & 0.8804 & 0.7389 & 0.8423  & 0.4021 & 0.8435 & 0.8572      & 0.7738 & 0.9062 & 0.8694 & \textbf{0.9673} \\ \bottomrule
\end{tabular}
}
\label{tab:har}
\end{table*}




\noindent\subsubsection{Datasets}
\noindent\textbf{Time Series Datasets:} \quad We consider the UCIHAR\cite{anguita2013public} dataset, which comprises sensor data from accelerometers, gyroscopes, and body sensors collected from 30 subjects performing six activities: walking, walking upstairs, walking downstairs, standing, sitting, and lying down. Due to the variability between individuals, each subject is treated as a separate domain. We also consider the HHAR\cite{stisen2015smart} dataset, which contains sensor readings from smartphones and smartwatches, collected from 9 subjects, with each subject similarly treated as an individual domain.

\noindent\textbf{Video Datasets:} We adopted two widely recognized and frequently used datasets, i.e., UCF101 and HMDB51. The UCF101 dataset, curated by the University of Central Florida, consists of videos primarily sourced from YouTube, capturing a wide variety of daily activities and sports actions. In contrast, the HMDB51 dataset, collected by the University of Massachusetts Amherst, includes videos from movies, public databases, and YouTube, offering a more diverse range of real-world action scenes. Both datasets serve as key benchmarks for evaluating video classification algorithms. For data processing, we followed the approach used in TranSVAE, extracting relevant and overlapping action categories from both UCF101 and HMDB51. This resulted in a combined dataset of 3,209 videos, where UCF101 provided 1,438 training videos and 571 validation videos, and HMDB51 contributed 840 training videos and 360 validation videos. Based on these datasets, we defined two video-based unsupervised domain adaptation tasks: U → H and H → U.

% 对于,由于标签是离散的,不少方法中的约束只能施加在离散标签中,因此对于分类的domain adaptation,我们考虑经典
\noindent\subsubsection{Baselines} \textcolor{black}{Since the time series classification UDA methods use constraints that can be only applied to discrete labels}, we selected several state-of-the-art methods implemented in the AdaTime framework as baselines for comparison. These methods include AdvSKM\cite{liu2021adversarial}, CODATS\cite{wilson2020multi}, CoTMix\cite{eldele2022cotmix}, DANN\cite{ganin2016domain}, DDC\cite{tzeng2014deep}, DeepCoral\cite{sun2017correlation}, DIRT\cite{shu2018dirt}, MMDA\cite{rahman2020minimum}, and SASA\cite{cai2021time}. 




For the video classification task, we conducted a comprehensive comparison based on the experimental settings of the latest TranSVAE approach~\cite{wei2023unsupervised}. Specifically, we included five widely-used image-based UDA methods, which perform domain adaptation by disregarding temporal information: DANN~\cite{ganin2016domain}, JAN~\cite{long2017deep}, ADDA~\cite{tzeng2017adversarial}, AdaBN~\cite{li2018adaptive}, and MCD~\cite{sahoo2021contrast}. Additionally, we benchmarked our approach against the latest state-of-the-art video-based UDA methods, including TA3N~\cite{chen2019temporal}, SAVA~\cite{choi2020shuffle}, TCoN~\cite{pan2020adversarial}, ABG~\cite{luo2020adversarial}, CoMix~\cite{sahoo2021contrast}, CO2A~\cite{da2022dual}, and MA2L-TD~\cite{chen2022multi}.


\noindent\subsubsection{Experimental Settings} For each of the time series datasets, we randomly generated 10 transfer directions. Each experiment was run three times and the average F1-score is reported. We generally followed the validation methods provided by AdaTime benchmark.

For video classification, we strictly followed the data processing procedure used in TranSVAE. TranSVAE employs a pretrained I3D model on the Kinetics dataset as the backbone to extract features from each video frame, which are stored and subsequently used as input to the network. To maintain consistency, we directly downloaded these precomputed features from TranSVAE and used them as input for our method. For the comparison methods, we referred to the experimental data from the TranSVAE network, where classification accuracy was recorded. Consistent with the experimental settings for prediction tasks, we employed the Adam optimizer. All experiments were implemented in PyTorch and conducted on a single NVIDIA GTX 3090 GPU with 24GB of memory.
% \vspace{0.5em}\\




\noindent\subsubsection{Quantization and Visual Results}\quad 
Here, we provide a detailed analysis of our method’s performance on classification tasks.
Tables~\ref{tab:har} and \ref{tab:hhar} present the performance of our proposed method alongside several baseline methods on the UCIHAR and HHAR time series classification datasets. It is worth noting that AdvSKM, CoDATS, DANN, and DIRT are adversarial-based methods, while DDC, Deep-CORAL, and MMDA methods align domains by minimizing a distance metric. On the other hand, CoTMix utilizes contrastive learning, while SASA enforces the consistency of sparse correlation structures to align domains. 

\textcolor{black}{The experimental results on the UCIHAR dataset show that our approach achieves the highest F1-scores across multiple cross-domain scenarios (e.g., 18 → 14, 6 → 13, 20 → 9), reaching a perfect score of 1 in some of them, significantly outperforming other comparative models such as AdvSKM, CoDATS, CoTMix, and DANN. Moreover, our model demonstrates exceptional stability across all tasks, further validating its robustness in temporal classification tasks. On the other temporal classification dataset, HHAR, our method remains the best-performing approach. For the cross-domain task 0 → 5, our method achieved an F1-score of 0.7899, significantly outperforming other comparative models, such as CoDATS (0.5714), MMDA (0.5444), and CoTMix (0.5064). Additionally, our method consistently achieves the best results across other tasks, including 3 → 1, 7 → 4, and 5 → 8.}

\begin{table*}[]
\small
\caption{F1-scores of various methods on the HHAR time series dataset.}
\resizebox{\textwidth}{!}{%
% Please add the following required packages to your document preamble:
% \usepackage{multirow}

\begin{tabular}{c|cccccccccc}
\toprule
Task   & AdvSKM & CoDATS & CoTMix & DANN   & DDC    & DeepCoral & DIRT   & MMDA   & SASA   & Ours            \\ \midrule
3 $\rightarrow$ 1 & 0.9077 & 0.9582 & 0.9601 & 0.8925 & 0.9120 & 0.9731      & 0.9727 & 0.9472 & 0.9747 & \textbf{0.9802} \\  \midrule
7 $\rightarrow$ 4 & 0.8477 & 0.9207 & 0.9391 & 0.9587 & 0.8245 & 0.8990      & 0.8932 & 0.9274 & 0.9216 & \textbf{0.9723} \\  \midrule
2 $\rightarrow$ 0 & 0.7286 & 0.6855 & 0.6447 & 0.7594 & 0.7390 & 0.7426      & 0.7474 & 0.7835 & 0.7398 & \textbf{0.8058} \\  \midrule
5 $\rightarrow$ 7 & 0.5064 & 0.8987 & 0.9218 & 0.9186 & 0.5028 & 0.8234      & 0.8540 & 0.8200 & 0.8200 & \textbf{0.9378} \\  \midrule
0 $\rightarrow$ 5 & 0.4140 & 0.5714 & 0.5064 & 0.4410 & 0.4138 & 0.4365      & 0.4794 & 0.5444 & 0.5055 & \textbf{0.7899} \\  \midrule
3 $\rightarrow$ 5 & 0.8653 & 0.9701 & 0.9704 & 0.9799 & 0.8828 & 0.9317      & 0.9595 & 0.9750 & 0.9696 & \textbf{0.9824} \\  \midrule
0 $\rightarrow$ 2 & 0.5332 & 0.6743 & 0.7588 & 0.7083 & 0.5723 & 0.6217      & 0.6673 & 0.7239 & 0.6980 & \textbf{0.8028} \\  \midrule
5 $\rightarrow$ 8 & 0.9279 & 0.9796 & 0.9739 & 0.9915 & 0.8242 & 0.9818      & 0.9776 & 0.9874 & 0.9868 & \textbf{0.9927} \\  \midrule
1 $\rightarrow$ 6 & 0.6978 & 0.9047 & 0.9313 & 0.9353 & 0.6160 & 0.8751      & 0.9089 & 0.9144 & 0.8890 & \textbf{0.9415} \\  \midrule
7 $\rightarrow$ 3 & 0.9155 & 0.9338 & 0.9637 & 0.9547 & 0.8912 & 0.9314      & 0.9379 & 0.9575 & 0.9397 & \textbf{0.9688} \\  \bottomrule
\end{tabular}
}
\label{tab:hhar}
\end{table*}

\begin{table}[]
\small
\renewcommand{\arraystretch}{1.3}
\caption{Classification accuracy on the HMDB-UCF video dataset.}
\begin{tabular}{c|c|c|c|c}
\toprule
Method   & Backbone   & U $\rightarrow$ H & H $\rightarrow$ U & Average \\ \midrule
DANN     & ResNet-101 & 75.28  & 76.36  & 75.82   \\
JAN      & ResNet-102 & 74.72  & 76.69  & 75.71   \\
AdaBN    & ResNet-103 & 72.22  & 77.41  & 74.82   \\
MCD      & ResNet-104 & 73.89  & 79.34  & 76.62   \\
TA3N     & ResNet-105 & 78.33  & 81.79  & 80.06   \\
ABG      & ResNet-106 & 79.17  & 85.11  & 82.14   \\
TCoN     & ResNet-107 & 87.22  & 89.14  & 88.18   \\
MA2L-TD  & ResNet-108 & 85     & 86.59  & 85.8    \\ \midrule
DANN     & I3D        & 80.83  & 88.09  & 84.46   \\
ADDA     & I3D        & 79.17  & 88.44  & 83.81   \\
TA3N     & I3D        & 81.38  & 90.54  & 85.96   \\
SAVA     & I3D        & 82.22  & 91.24  & 86.73   \\
CoMix    & I3D        & 86.66  & 93.87  & 90.22   \\
CO2A     & I3D        & 87.78  & 95.79  & 91.79   \\
TranSVAE & I3D        & 87.78  & 98.95  & 93.37   \\ \midrule
Ours      & I3D        & \textbf{89.44}  & \textbf{99.12}  & \textbf{94.28}   \\ \bottomrule
\end{tabular}
\label{classification result}
\end{table}

\begin{figure*}[h]
    \centering
    \label{fig:scatter}
    \includegraphics[width=1.01\textwidth, trim=0cm  11.5cm 0cm 0cm, clip]{fig/scatter_pic.pdf}
    \caption{Classification Visualization: (a) and (b) show scatter plots of the latent variables and probability vectors, respectively, after dimensionality reduction using t-SNE for the H$\rightarrow$U scenario. (c) and (d) present the corresponding scatter plots for the U$\rightarrow$H scenario. These visualizations illustrate how the model’s latent representations and probability vectors effectively capture and distinguish between categories.}
\end{figure*}




% As shown in the tables, our method consistently outperforms all comparative methods across all transfer directions, demonstrating the effectiveness of our approach. 

Table~\ref{classification result} presents a comparative analysis of our method against the baseline on the UCF-HMDB video classification dataset. We notice that methods utilizing the I3D backbone\cite{carreira2017quo} generally outperform those using ResNet-101. Our method consistently exceeds the performance of all prior approaches, achieving an average accuracy of 94.28\%. These results highlight the exceptional performance of our approach even on high-dimensional data.

TranSVAE represents the previous state-of-the-art method, modeling videos as being generated by both dynamic and static latent variables, and has demonstrated remarkable performance in video unsupervised domain adaptation (UDA). Similarly, our approach models the data generation process for time series data—in which videos are a specific instance—from a causal perspective. It assumes that the underlying data generation mechanisms remain consistent across different domains. This causal modeling approach, combined with advanced deep learning architectures, has also delivered impressive results, reinforcing the potential of modeling problems from a causal perspective and leveraging deep learning networks to fit the data generation process.

Additionally, we visualize the classification results for both H$\rightarrow$U and U$\rightarrow$H scenarios. Specifically, we reduced the dimensionality of the model’s output probability vectors or latent variable representations using t-SNE and depicted the results in scatter plots shown in Figure~\ref{fig:scatter}. The visualizations reveal that both the latent variables and probability vectors effectively distinguish between categories.


% \begin{figure*}[h]
%     \centering
%     \label{fig:scatter}
%     \includegraphics[width=1.01\textwidth, trim=0cm  11.5cm 0cm 0cm, clip]{fig/scatter_pic.pdf}
%     \caption{Classification Visualization: (a) and (b) show scatter plots of latent variables and probability vectors, respectively, after dimensionality reduction using t-SNE for the H$\rightarrow$U scenario. (c) and (d) present the corresponding scatter plots for the U$\rightarrow$H scenario. These visualizations illustrate how the model’s latent representations and probability vectors effectively capture and distinguish between categories.}
% \end{figure*}

% \begin{figure*}[h]
%     \centering
%     \label{fig:sensetivty}
%     \includegraphics[width=1.01\textwidth, trim=0.1cm  11.5cm 0.5cm 0cm, clip]{fig/sen.pdf}
%     \caption{Sensitivity analysis of parameters \(\alpha\), \(\beta\), \(\gamma\), and \(\delta\) is shown in (a), (b), (c), and (d). The x-axis represents parameter values, and the y-axis shows mse.}
% \end{figure*}


\begin{figure*}[h]
    \centering
    \includegraphics[width=1.01\textwidth, trim=0.1cm  11.5cm 0.5cm 0cm, clip]{fig/sen.pdf}
    \caption{Sensitivity analysis of the parameters \(\alpha\), \(\beta\), \(\gamma\), and \(\delta\) (from Equation~\ref{eqn:overall_loss}) is shown in (a), (b), (c), and (d), respectively. The x-axis represents parameter values, and the y-axis shows the MSE value.}
    \label{fig:sensetivty}
\end{figure*}

\subsection{Ablation Study}
\begin{figure}[h]
    \centering
    \label{fig:ablation}
    \includegraphics[width=1.01\textwidth, trim=0cm  4.2cm 3.2cm 0cm, clip]{fig/ablation.pdf}
    \caption{Ablation Results Visualization: (a) presents the results for the H$\rightarrow$U scenario, while (b) shows the results for the U$\rightarrow$H scenario. These visualizations illustrate the impact of different components on the performance of our model across the two scenarios.}
\end{figure}
To rigorously evaluate the contribution of each component in the proposed loss function, we devised four model variants, described as follows.

\begin{itemize}[leftmargin=*]
\setlength{\itemsep}{0pt}
    \item \textbf{LCA-$L_{KL}$}: We remove $L_{KL}$, a critical term in the ELBO.
    \item \textbf{LCA-$L_{R}$}: We remove $L_{R}$, another essential term in the ELBO.
    \item \textbf{LCA-$L_{S}$}: We remove $L_{S}$, which enforces sparsity in the causal structure.
    \item \textbf{LCA-$L_{A}$}: We remove $L_{A}$, which ensures consistency of causal structures across different domains.
\end{itemize}
We conducted the ablation experiments on the high-dimensional video datasets, HMDB, and UCF. In these experiments, we systematically set the weight of each corresponding loss term to zero to observe the performance of the resulting model variants. The results are presented in Figure~\ref{fig:ablation}.
Our findings demonstrate that each loss component plays a crucial role in enhancing the model's performance. Specifically, $\mathcal{L}_{KL}$ and $\mathcal{L}_{R}$ are vital elements of the ELBO, and maximizing them enables the model to accurately learn the joint distribution of the data. From a theoretical standpoint, $\mathcal{L}_{KL}$ ensures that the learned distribution of the latent variables closely approximates the true distribution, while $\mathcal{L}_{R}$ ensures that the latent variables capture as much information from the observed variables as possible.
The term $\mathcal{L}_{S}$ enforces the sparsity of relationships between adjacent latent variables, a requirement rooted in identifiability theory. This constraint aligns with real-world scenarios, where causal relationships are typically sparse, and it helps prevent the model from learning redundant associations. Lastly, $\mathcal{L}_{A}$ aligns the causal mechanisms in the target domain with those in the source domain. The substantial improvement in performance observed in the experiments validates the importance of enforcing consistency in causal mechanisms across domains, thereby further affirming the soundness of our approach.

\subsection{Sensitivity Analysis}
We performed a sensitivity analysis on the parameters \(\alpha\), \(\beta\), \(\gamma\), and \(\delta\) in the loss function (Equation~\ref{eqn:overall_loss}), focusing on the transfer directions from G to W and W to G in the Human Motion dataset. We set several values for each parameter, ranging from 10 to 0.001 and decreasing by a factor of 10 each time. The experimental results are shown in Figure \ref{fig:sensetivty}. 
% The results indicate that our method consistently outperforms the baseline for different parameter selections, demonstrating its robustness. 
Notably, as the parameters decrease, the MSE exhibits a trend of initially decreasing and then increasing, which matches our expectations: when the parameters are large, the constraints on the corresponding modules are strongest, causing the model to over-focus on certain areas and leading to a decline in performance. Conversely, when these parameters are small, the corresponding modules receive insufficient attention, resulting in suboptimal performance.


\section{Conclusion}\label{conclusion}
In this paper, we propose a Latent Causality Alignment (LCA) model for time series domain adaptation. By identifying low-dimensional latent variables and reconstructing their causal structures with sparsity constraints, we effectively transfer domain knowledge through latent causal mechanism alignment. The proposed method not only addresses the challenges of modeling latent causal mechanisms in high-dimensional time series data but also guarantees the uniqueness of latent causal structures, providing improved performance on domain-adaptive time series classification and forecasting tasks. However, our method assumes sufficient changes in historical information for latent variable identification. Exploring how to relax this assumption and extend our framework to more complex scenarios would be a promising direction for future work.

% use section* for acknowledgment
% \ifCLASSOPTIONcompsoc
%   % The Computer Society usually uses the plural form
%   \section*{Acknowledgments}
% \else
%   % regular IEEE prefers the singular form
%   \section*{Acknowledgment}
% \fi




\bibliographystyle{IEEEtran}
\bibliography{ref}



% \newpage
% % \clearpage
% \vspace{-7ex}
% \begin{IEEEbiography}[{\includegraphics[width=1in, height=1.25in, clip, keepaspectratio]{fig/lizijian3}}]{Zijian Li}
% received a B.S. degree in computer science from the Guangdong University of Technology, Guangzhou, China, in 2017. 
% He is currently a Ph.D. student at the School of Computer Science, Guangdong University of Technology. He was a research intern at the Advanced Digital Sciences Center, Illinois at Singapore Pte in 2018.3-2018-6, and a visiting student at Nanyang Technological University in 2019.9-2019.12. His research interests cover a variety of different topics including causality-inspired machine learning, transfer learning, and their applications.
% \end{IEEEbiography}
% \vspace{-7ex}

% \begin{IEEEbiography}[{\includegraphics[width=1in, height=1.25in, clip, keepaspectratio]{fig/Biogra_Ruichu_Cai}}]{Ruichu Cai} (M'17) received his B.S. degree in Applied Mathematics and Ph.D. degree in Computer Science from South China University of Technology in 2005 and 2010, respectively. He is currently a Professor in the School of Computer and the director of the data mining and retrieval lab. , Guangdong University of Technology. 
	
% His research interests cover a variety of different topics including causality, deep learning and their applications.  He was a recipient of the National Science Fund for Excellent Young Scholars, Natural Science Award of Guangdong and so on awards. He has served as the area chair of ICML 2022, senior PC for AAAI 2019-2022, IJCAI 2019-2022, and so on. 
% \end{IEEEbiography}
% \vspace{-7ex}
% \begin{IEEEbiography}[{\includegraphics[width=1in, height=1.25in, clip, keepaspectratio]{fig/Tom-bio}}]{Tom Z. J. Fu } 
% obtained the B.Eng degree in Information Engineering from Shanghai Jiao Tong University in 2006. He received the M.Phil and the Ph.D degrees from the Department of Information Engineering at the Chinese University of Hong Kong in 2008 and 2013, respectively. He is currently leading the network transmission algorithm team of Bigo Technology. Before joining Bigo, he was a senior research scientist and analytics area programme manager of the Advanced Digital Sciences Center, a Singapore-based research center affiliated with the University of Illinois at Urbana-Champaign. His research interests include data driven networking, software defined networking (SDN), Internet measurement and monitoring, Peer-to-Peer content distribution, cloud computing and real-time distributed stream analytics.
% \end{IEEEbiography}
% % \vspace{15ex}
% \vspace{-7ex}
% \begin{IEEEbiography}
% 	[{\includegraphics[width=1in, height=1.25in, clip, keepaspectratio]{fig/zhifenghao}}]{Zhifeng Hao} received his B.S. degree in Mathematics from the Sun Yat-Sen University in 1990, and his Ph.D. degree in Mathematics from Nanjing University in 1995. He is currently a Professor in the School of Computer, Guangdong University of Technology, and College of Science, Shantou University. 
	
% 	His research interests involve various aspects of Algebra, Machine Learning, Data Mining, Evolutionary Algorithms.
% \end{IEEEbiography}

% \vspace{-7ex}
% \begin{IEEEbiography}[{\includegraphics[width=1in, height=1.25in, clip, keepaspectratio]{fig/kun-zhang}}]{Kun Zhang}
% received the B.S. degree in automation
% from the University of Science and Technology of
% China, Hefei, China, in 2001, and the Ph.D. degree
% in computer science from The Chinese University of
% Hong Kong, Hong Kong, in 2005.

% He is currently an Associate Professor with the
% Philosophy Department and an Affiliate Faculty
% Member with the Machine Learning Department,
% Carnegie Mellon University, Pittsburgh, PA, USA
% His research interests lie in causality, machine
% learning, and artificial intelligence, especially in
% causal discovery, hidden causal representation learning, transfer learning, and
% general-purpose artificial intelligence
% \end{IEEEbiography}



\newpage
\appendix

\renewcommand{\figurename}{Supplementary Figure}
\renewcommand{\tablename}{Supplementary Table}
\setcounter{figure}{0}
\setcounter{table}{0}

    



\section{Details of datasets}
This section provides additional details about the dataset used to evaluate the downstream tasks. \Cref{tab:disease_definition} lists the ICD-10 codes and medications used to identify the diagnoses for each disease. \Cref{tab:characteristic} presents the distribution of patient characteristics for each disease. \Cref{fig:nyu_langone_prevalence,fig:nyu_longisland_prevalence} illustrates the prevalence of each disease in the downstream tasks for the NYU Langone and NYU Long Island datasets, highlighting the imbalances present in these tasks.

\begin{table}[!htpb]
    \centering
    \caption{The definition of diseases in EHR by diagnosis codes and medications.}
    \begin{tabular}{lr}
    \toprule
         Disease &  Definition in EHR \\
    \midrule
       IPH  &  I61.0, I61.1, I61.2, I61.3, I61.4, I61.8, I61.9 \\
       IVH  &  I61.5, P52.1, P52.2, P52.3  \\
       ICH  &  IPH + IVH + I61.6, I62.9, P10.9, P52.4, P52.9 \\
       SDH  &  S06.5, I62.0 \\
       EDH  &  S06.4, I62.1 \\
       SAH  &  I60.*, S06.6, P52.5, P10.3  \\
       Tumor  &  C71.*, C79.3, D33.0, D33.1, D33.2, D33.3, D33.7, D33.9  \\
       Hydrocephalus  &  G91.* \\
       Edema  &  G93.1, G93.5, G93.6, G93.82, S06.1 \\
       \multirow{2}{*}{ADRD}  &  G23.1, G30.*, G31.01, G31.09, G31.83, G31.85, G31.9, F01.*, F02.*, F03.*, G31.84, G31.1, \\ 
       & \textbf{Medication:} DONEPEZIL, RIVASTIGMINE, GALANTAMINE, MEMANTINE, TACRINE \\ 
    \bottomrule
    \end{tabular}
    \label{tab:disease_definition}
\end{table}

\begin{table}[!htbp]
\centering
\caption{Demographic characteristics of patients associated with scans from the NYU Langone dataset, matched with electronic health records (EHR) and utilized in downstream tasks.}
\label{tab:characteristic}

 The characteristic table on NYU Langone dataset matched with EHR.
\begin{tabular}{ll|rr|r}
\toprule
                       \textbf{Cohort} &  &           \textbf{Male (\%)} &          \textbf{Female (\%)} &     \textbf{Age (std)} \\
\midrule
 --- & All (n=270,205) & 128,113 (47.41\%) & 142,092 (52.59\%) & 63.64 (19.68) \\
\midrule
       Tumor & Neg (n=260,704) & 123,338 (47.31\%) & 137,366 (52.69\%) & 63.85 (19.72) \\
             & Pos (n=9,501) &   4,775 (50.26\%) &   4,726 (49.74\%) & 57.80 (17.67) \\
\midrule
HCP & Neg (n=253,000) & 118,881 (46.99\%) & 134,119 (53.01\%) & 63.67 (19.72) \\
              & Pos (n=17,205) &   9,232 (53.66\%) &   7,973 (46.34\%) & 63.18 (19.11) \\
\midrule
Edema & Neg (n=242,576) & 112,987 (46.58\%) & 129,589 (53.42\%) & 63.96 (19.84) \\
      & Pos (n=27,629) &  15,126 (54.75\%) &  12,503 (45.25\%) & 60.81 (17.97) \\
\midrule
ADRD  & Neg (n=232,667) & 111,159 (47.78\%) & 121,508 (52.22\%) & 61.31 (19.55) \\
      & Pos (n=37,538) &  16,954 (45.16\%) &  20,584 (54.84\%) & 78.09 (13.30) \\
\midrule
          IPH & Neg (n=251,308) & 117,692 (46.83\%) & 133,616 (53.17\%) & 63.58 (19.82) \\
              & Pos (n=18,897) &  10,421 (55.15\%) &   8,476 (44.85\%) & 64.39 (17.69) \\
\midrule
          IVH & Neg (n=258,232) & 121,686 (47.12\%) & 136,546 (52.88\%) & 63.65 (19.79) \\
              & Pos (n=11,973) &   6,427 (53.68\%) &   5,546 (46.32\%) & 63.45 (17.19) \\
\midrule
          SDH & Neg (n=248,468) & 114,869 (46.23\%) & 133,599 (53.77\%) & 63.44 (19.78) \\
              & Pos (n=21,737) &  13,244 (60.93\%) &   8,493 (39.07\%) & 65.95 (18.33) \\
\midrule
          EDH & Neg (n=265,431) & 125,113 (47.14\%) & 140,318 (52.86\%) & 63.77 (19.64) \\
              & Pos (n=4,774) &   3,000 (62.84\%) &   1,774 (37.16\%) & 56.53 (20.75) \\
\midrule
          SAH & Neg (n=251,594) & 118,424 (47.07\%) & 133,170 (52.93\%) & 63.79 (19.76) \\
              & Pos (n=18,611) &   9,689 (52.06\%) &   8,922 (47.94\%) & 61.59 (18.49) \\
\midrule
          ICH & Neg (n=229,851) & 105,498 (45.90\%) & 124,353 (54.10\%) & 63.41 (19.93) \\
              & Pos (n=40,354) &  22,615 (56.04\%) &  17,739 (43.96\%) & 64.93 (18.14) \\
\bottomrule
\end{tabular}
\end{table}


\begin{table}[!h]
    \centering
    \caption*{\textbf{Supplementary \Cref{tab:characteristic} Continued.} Demographic characteristics of patients associated with scans from the NYU Long Island dataset, matched with electronic health records (EHR) and utilized in downstream tasks.}
\begin{tabular}{ll|rr|r}
\toprule
                       \textbf{Cohort} &  &           \textbf{Male (\%)} &          \textbf{Female (\%)} &     \textbf{Age (std)} \\
\midrule
--- & All (n=22,158) & 9,580 (43.23\%) & 12,578 (56.77\%) & 68.33 (18.14) \\
\midrule
Tumor & Neg (n=21,578) & 9,275 (42.98\%) & 12,303 (57.02\%) & 68.59 (18.08) \\
      & Pos (n=580) &   305 (52.59\%) &    275 (47.41\%) & 58.78 (17.79) \\
\midrule
HCP   & Neg (n=20,653) & 8,718 (42.21\%) & 11,935 (57.79\%) & 69.05 (17.90) \\
      & Pos (n=1,505) &   862 (57.28\%) &    643 (42.72\%) & 58.52 (18.48) \\
\midrule
Edema & Neg (n=19,402) & 8,068 (41.58\%) & 11,334 (58.42\%) & 68.89 (18.27) \\
      & Pos (n=2,756) & 1,512 (54.86\%) &  1,244 (45.14\%) & 64.36 (16.66) \\
\midrule
ADRD  & Neg (n=19,537) & 8,391 (42.95\%) & 11,146 (57.05\%) & 66.78 (18.28) \\
      & Pos (n=2,621) & 1,189 (45.36\%) &  1,432 (54.64\%) & 79.90 (11.77) \\
\midrule
IPH   & Neg (n=19,357) & 7,974 (41.19\%) & 11,383 (58.81\%) & 68.97 (18.27) \\
      & Pos (n=2,801) & 1,606 (57.34\%) &  1,195 (42.66\%) & 63.89 (16.48) \\
\midrule
IVH   & Neg (n=19,636) & 8,164 (41.58\%) & 11,472 (58.42\%) & 68.96 (18.22) \\
      & Pos (n=2,522) & 1,416 (56.15\%) &  1,106 (43.85\%) & 63.43 (16.66) \\
\midrule
SDH   & Neg (n=20,885) & 8,870 (42.47\%) & 12,015 (57.53\%) & 68.33 (18.21) \\
      & Pos (n=1,273) &   710 (55.77\%) &    563 (44.23\%) & 68.37 (16.83) \\
\midrule
EDH   & Neg (n=21,912) & 9,443 (43.10\%) & 12,469 (56.90\%) & 68.33 (18.16) \\
      & Pos (n=246) &   137 (55.69\%) &    109 (44.31\%) & 68.19 (15.59) \\
\midrule
SAH   & Neg (n=20,652) & 8,824 (42.73\%) & 11,828 (57.27\%) & 68.68 (18.12) \\
      & Pos (n=1,506) &   756 (50.20\%) &    750 (49.80\%) & 63.58 (17.65) \\
\midrule
ICH   & Neg (n=18,388) & 7,456 (40.55\%) & 10,932 (59.45\%) & 68.92 (18.35) \\
      & Pos (n=3,770) & 2,124 (56.34\%) &  1,646 (43.66\%) & 65.48 (16.77) \\
\bottomrule
\end{tabular}
\end{table}

\begin{figure}[!ht]
    \centering
    \includegraphics[width=0.8\textwidth]{images/NYU_Langone_prevalence.pdf}
    \caption{Disease prevalence of NYU Langone }
    \label{fig:nyu_langone_prevalence}
\end{figure}

\begin{figure}[!h]
    \centering
    \includegraphics[width=0.8\textwidth]{images/NYU_Longisland_prevalence.pdf}
    \caption{Disease prevalence of NYU Longisland dataset}
    \label{fig:nyu_longisland_prevalence}
\end{figure}



\section{Data augmentation details}
\label{sec:dataaug_details}
We applied Random Flipping across all three dimensions, Random Shift Intensity with offset $0.1$ for both pre-training and fine-tuning. For DINO training. random Gaussian Smoothing with sigma=$(0.5-1.0)$ is applied across all dimensions, Random Gamma Adjust is applied with gamma=$(0.2-1.0)$.


\section{Additional experiment results}
This section provides additional experimental results with more details.
Supplementary \Cref{fig:channels-ablation,fig:patches-ablation} compares the performance of the foundation model using different numbers of channels and patch sizes, demonstrating that the architecture design of our foundation model is optimal. 

Supplementary \Cref{fig:radar-comparison-merlin} compares our foundation model with a foundation CT model from previous studies, Merlin\cite{blankemeier2024merlinvisionlanguagefoundation}, which was trained on abdomen CT scans with corresponding radiology report pairs. Our model demonstrates superior performance on head CT scans.

Supplementary \Cref{fig:probing-comparison-gemini} compares our foundation model with Google CT Foundation model~\cite{yang2024advancingmultimodalmedicalcapabilities}, which was trained on large scale and diverse CT scans from different anatomy with corresponding radiology report pairs. Our model consistently shows improved performance across the board even though our model was pre-trained with less samples.

Supplementary \Cref{fig:probing_comparison} compares the performance on downstream tasks with various supervised tuning methods applied to foundation models pretrained with the MAE and DINO frameworks. Per-pathology comparisons are shown in Supplementary \Cref{fig:probing-comparison-perpath,fig:probing-comparison-perpath-dino}. Meanwhile, supplementary \Cref{fig:boxplot_scaling} complements \Cref{fig:scaling_law}, illustrating the per-pathology performances of foundation models pretrained with different scales of training data.

Supplementary \Cref{fig:batch_effect,fig:thickness-ablation} studies the impact of batch effect caused by different CT scan protocols of slice thickness and machine manufacturer. Detailed per-pathology performances are shown in Supplementary \Cref{fig:slice_thickness_per_pathology,fig:manufacturer_per_pathology}.

\begin{figure}[!htpb]
    \centering
    \makebox[\textwidth][l]{%
        \hspace{0.3\textwidth}\textbf{NYU Langone}
    } \\[0.2cm]
    \includegraphics[trim={0 0 0 0},clip,height=0.3\textwidth, width=0.3\textwidth]{figures/abla_chans/AUC_chans_NYU.pdf}
    \includegraphics[trim={0 0 0 0},clip,height=0.3\textwidth, width=0.55\textwidth]{figures/abla_chans/AP_chans_NYU.pdf}\\
    \makebox[\textwidth][l]{
        \hspace{0.34\textwidth}\textbf{RSNA}
    } \\[0.2cm]
    \includegraphics[trim={0 0 0 0},clip,height=0.3\textwidth, width=0.3\textwidth]{figures/abla_chans/AUC_chans_RSNA.pdf}
    \includegraphics[height=0.3\textwidth, width=0.55\textwidth]{figures/abla_chans/AP_chans_RSNA.pdf} 
    \caption{\textbf{Comparison of Different Channels Performance.} This plot compares the performance of models trained using different numbers of channels (channels from multiple HU intervals vs. a single HU interval). Across two datasets, models using three channels from different HU intervals consistently outperform those using a single channel with a fixed HU interval. All models were pre-trained on $100\%$ of the pretraining data with MAE.}
    \label{fig:channels-ablation}
\end{figure}


\begin{figure}[!htpb]
    \centering
    \makebox[\textwidth][l]{%
        \hspace{0.3\textwidth}\textbf{NYU Langone}
    } \\[0.2cm]
    \includegraphics[trim={0 0 0 0},clip,height=0.3\textwidth, width=0.3\textwidth]{figures/abla_patches/AUC_patches_NYU.pdf}
    \includegraphics[trim={0 0 0 0},clip,height=0.3\textwidth, width=0.55\textwidth]{figures/abla_patches/AP_patches_NYU.pdf}\\
    \makebox[\textwidth][l]{
        \hspace{0.34\textwidth}\textbf{RSNA}
    } \\[0.2cm]
    \includegraphics[trim={0 0 0 0},clip,height=0.3\textwidth, width=0.3\textwidth]{figures/abla_patches/AUC_patches_RSNA.pdf}
    \includegraphics[height=0.3\textwidth, width=0.55\textwidth]{figures/abla_patches/AP_patches_RSNA.pdf} 
    \caption{\textbf{Comparison of Different Patches Performance.} This plot compares the performance of models trained with different patch sizes (12 vs. 16). The results demonstrate that smaller patch sizes consistently achieve better performance. All models were pre-trained on $100\%$ of the pretraining data with MAE.}
    \label{fig:patches-ablation}
\end{figure}


\begin{figure*}
    \centering
    \makebox[\textwidth][l]{%
        \hspace{0.06\textwidth}
        \textbf{NYU Langone} \hspace{0.06\textwidth} \textbf{NYU Long Island} \hspace{0.11\textwidth} \textbf{RSNA} \hspace{0.18\textwidth} \textbf{CQ500}
    } \\[0.2cm]
    \includegraphics[trim={0 0 0 0},clip,height=0.21\textwidth, width=0.21\textwidth]{figures/abla_radarplot_merlin/AUC_NYU.pdf}
    \includegraphics[trim={0 0 0 0},clip,height=0.21\textwidth, width=0.21\textwidth]{figures/abla_radarplot_merlin/AUC_Longisland.pdf}
    \includegraphics[trim={0 0 0 0},clip,height=0.21\textwidth, width=0.21\textwidth]{figures/abla_radarplot_merlin/AUC_RSNA.pdf}
    \includegraphics[trim={0 0 0 0},clip,height=0.21\textwidth, width=0.35\textwidth]{figures/abla_radarplot_merlin/AUC_CQ500.pdf}\\[0.2cm]
    \includegraphics[height=0.21\textwidth, width=0.21\textwidth]{figures/abla_radarplot_merlin/AP_NYU.pdf} 
    \includegraphics[height=0.21\textwidth, width=0.21\textwidth]{figures/abla_radarplot_merlin/AP_Longisland.pdf} 
    \includegraphics[height=0.21\textwidth, width=0.21\textwidth]{figures/abla_radarplot_merlin/AP_RSNA.pdf}
    \includegraphics[height=0.21\textwidth, width=0.35\textwidth]{figures/abla_radarplot_merlin/AP_CQ500.pdf}
    \caption{\textbf{Comparison to previous 3D Foundation Model.} This plot compares the performance of our model with Merlin~\cite{blankemeier2024merlinvisionlanguagefoundation} and models trained from scratch across four datasets for our model and ResNet50-3D. Our DINO trained model is used in this comparison. Our model demonstrates consistently superior performance across majority of diseases, with the exception of epidural hemorrhage (EDH) in the CQ500 dataset.}
    \label{fig:radar-comparison-merlin}
\end{figure*}



\begin{figure*}
    \centering
    \makebox[\textwidth][l]{%
        \hspace{0.10\textwidth}
        \textbf{NYU Langone} \hspace{0.08\textwidth} \textbf{NYU Long Island} \hspace{0.1\textwidth} \textbf{RSNA} \hspace{0.15\textwidth} \textbf{CQ500}
    } \\[0.2cm]
    \includegraphics[trim={0 0 0 0},clip, width=0.22\textwidth]{figures/abla_probing/AUC_NYU.pdf}
    \includegraphics[trim={0 0 0 0},clip, width=0.22\textwidth]{figures/abla_probing/AUC_Longisland.pdf}
    \includegraphics[trim={0 0 0 0},clip, width=0.22\textwidth]{figures/abla_probing/AUC_RSNA.pdf}
    \includegraphics[trim={0 0 0 0},clip, width=0.28\textwidth]{figures/abla_probing/AUC_CQ500.pdf}
    \\[0.2cm]
    \includegraphics[width=0.22\textwidth]{figures/abla_probing/AP_NYU.pdf} 
    \includegraphics[width=0.22\textwidth]{figures/abla_probing/AP_Longisland.pdf} 
    \includegraphics[width=0.22\textwidth]{figures/abla_probing/AP_RSNA.pdf}
    \includegraphics[width=0.28\textwidth]{figures/abla_probing/AP_CQ500.pdf}
    \caption{\textbf{Comparison of Different Downstream Training Methods.} This plot illustrates the downstream performance of models evaluated using fine-tuning and various probing methods across four datasets. In most cases, the DINO pre-trained model outperforms the MAE pre-trained model. All models were pre-trained on $100\%$ of the available pretraining data.}
    \label{fig:probing_comparison}
\end{figure*}


\begin{figure}
\centering
\makebox[\textwidth][l]{%
    \hspace{0.39\textwidth}\textbf{RSNA}
} \\[0.2cm]
\includegraphics[trim={0 0 0mm 0},clip,height=0.27\textwidth]{figures/abla_gemini/AUC_RSNA_Gemini.pdf}
\includegraphics[trim={0 0 5mm 0},clip,height=0.27\textwidth]{figures/abla_gemini/AP_RSNA_Gemini.pdf}

\makebox[\textwidth][l]{%
    \hspace{0.38\textwidth}\textbf{CQ500}
} \\[0.2cm]
\includegraphics[trim={0 0 10mm 0},clip,height=0.345\textwidth]{figures/abla_gemini/AUC_CQ500_Gemini.pdf}
\includegraphics[trim={0 0 5mm 0},clip,height=0.345\textwidth]{figures/abla_gemini/AP_CQ500_Gemini.pdf}

\caption{\textbf{Performance comparison of linear probing for Our Model vs. Google CT Foundation model} This plot compares our model performance vs. Google CT Foundation model\cite{yang2024advancing} and Merlin \cite{blankemeier2024merlinvisionlanguagefoundation} across all diseases on RSNA and CQ500. Since Google CT Foundation moudel requires uploading data to Google Cloud (not allowed on our private data) for requesting model embeddings with model weights inaccessible, only public dataset comparison is provided in this study. Similar to other evaluations, we observed that our model outperforms Google CT Foundation model across the board with the only exception on Midline Shift for Google CT Foundation model and EDH for Merlin.}
\label{fig:probing-comparison-gemini}
\end{figure}



\begin{figure}
    \centering
    \makebox[\textwidth][l]{%
        \hspace{0.35\textwidth}\textbf{NYU Langone}
    } \\[0.2cm]
    \includegraphics[trim={0 0 120mm 0},clip,height=0.255\textwidth]{figures/abla_probing_perpath/DINO_AUC_NYU_Langone.pdf}
    \includegraphics[trim={0 0 0 0},clip,height=0.255\textwidth]{figures/abla_probing_perpath/DINO_AP_NYU_Langone.pdf} \\
    \makebox[\textwidth][l]{
        \hspace{0.35\textwidth}\textbf{NYU Long Island}
    } \\[0.2cm]
    \includegraphics[trim={0 0 120mm 0},clip,height=0.255\textwidth]{figures/abla_probing_perpath/DINO_AUC_NYU_Long_Island.pdf}
    \includegraphics[trim={0 0 0 0},clip,height=0.255\textwidth]{figures/abla_probing_perpath/DINO_AP_NYU_Long_Island.pdf} 
    \makebox[\textwidth][l]{
        \hspace{0.4\textwidth}\textbf{RSNA}
    } \\[0.2cm]
    \includegraphics[trim={0 0 120mm 0},clip,height=0.24\textwidth]{figures/abla_probing_perpath/DINO_AUC_RSNA.pdf}
    \hspace{5mm}
    \includegraphics[trim={0 0 0 0},clip,height=0.24\textwidth]{figures/abla_probing_perpath/DINO_AP_RSNA.pdf} 
    \makebox[\textwidth][l]{
        \hspace{0.4\textwidth}\textbf{CQ500}
    } \\[0.2cm]
    \includegraphics[trim={0 0 120mm 0},clip,height=0.30\textwidth]{figures/abla_probing_perpath/DINO_AUC_CQ500.pdf} \hspace{5mm}
    \includegraphics[trim={0 0 0 0},clip,height=0.30\textwidth]{figures/abla_probing_perpath/DINO_AP_CQ500.pdf} 
    \caption{\textbf{Performance comparison of supervised finetuning methods per pathology on the foundation model trained with DINO.} This plot breaks down the average performance across all diseases shown in Supplementary \Cref{fig:probing_comparison}. The results show that fine-tuning the entire network achieves the best performance in most scenarios. However, linear probing closely approaches finetuning performance for many diseases especially on small or imbalanced dataset, underscoring the capability of our pre-trained models to generate representations that adapt effectively to diverse disease detection tasks.}
    \label{fig:probing-comparison-perpath-dino}
\end{figure}

\begin{figure}
    \centering
    \makebox[\textwidth][l]{%
        \hspace{0.35\textwidth}\textbf{NYU Langone}
    } \\[0.2cm]
    \includegraphics[trim={0 0 0 0},clip,height=0.24\textwidth, width=0.3\textwidth]{figures/abla_probing_perpath/AUC_NYU.pdf}
    \includegraphics[trim={0 0 0 0},clip,height=0.24\textwidth, width=0.45\textwidth]{figures/abla_probing_perpath/AP_NYU.pdf}\\
    \makebox[\textwidth][l]{
        \hspace{0.35\textwidth}\textbf{NYU Long Island}
    } \\[0.2cm]
    \includegraphics[trim={0 0 0 0},clip,height=0.24\textwidth, width=0.3\textwidth]{figures/abla_probing_perpath/AUC_Longisland.pdf}
    \includegraphics[trim={0 0 0 0},clip,height=0.24\textwidth, width=0.45\textwidth]{figures/abla_probing_perpath/AP_Longisland.pdf} 
    \makebox[\textwidth][l]{
        \hspace{0.4\textwidth}\textbf{RSNA}
    } \\[0.2cm]
    \includegraphics[trim={0 0 0 0},clip,height=0.24\textwidth, width=0.3\textwidth]{figures/abla_probing_perpath/AUC_RSNA.pdf}
    \includegraphics[height=0.24\textwidth, width=0.45\textwidth]{figures/abla_probing_perpath/AP_RSNA.pdf} 
    \makebox[\textwidth][l]{
        \hspace{0.4\textwidth}\textbf{CQ500}
    } \\[0.2cm]
    \includegraphics[trim={0 0 120mm 0},clip,height=0.24\textwidth]{figures/abla_probing_perpath/AUC_CQ500.pdf}
    \includegraphics[trim={0 0 0 0},clip,height=0.24\textwidth]{figures/abla_probing_perpath/AP_CQ500.pdf} 
    \caption{\textbf{Performance comparison of supervised finetuning methods per pathology on the foundation model trained with MAE.} The results reveal that attentive probing is significantly more effective than linear probing, consistent with findings from~\cite{Chen2024}. Furthermore, for many diseases, the performance of probing models approaches that of fine-tuning, demonstrating that our pre-trained models produce adaptable representations capable of detecting diverse diseases.}
    \label{fig:probing-comparison-perpath}
\end{figure}









\begin{figure}
    \centering
    \textbf{NYU Langone} \\
    \includegraphics[trim={0 0 0 0},clip,height=0.24\textwidth, width=0.38\textwidth]{figures/abla_perpath_perf/AUC_NYU.pdf}
    \includegraphics[height=0.24\textwidth, width=0.45\textwidth]{figures/abla_perpath_perf/AP_NYU.pdf} \\
    \textbf{NYU Long Island} \\
    \includegraphics[trim={0 0 0 0},clip,height=0.24\textwidth, width=0.38\textwidth]{figures/abla_perpath_perf/AUC_Longisland.pdf}
    \includegraphics[height=0.24\textwidth, width=0.45\textwidth]{figures/abla_perpath_perf/AP_Longisland.pdf} \\
    \textbf{RSNA} \\
    \includegraphics[trim={0 0 0 0},clip,height=0.24\textwidth, width=0.38\textwidth]{figures/abla_perpath_perf/AUC_RSNA.pdf}
    \includegraphics[height=0.24\textwidth, width=0.45\textwidth]{figures/abla_perpath_perf/AP_RSNA.pdf}\\
    \textbf{CQ500} \\
    \includegraphics[trim={0 0 0 0},clip,height=0.24\textwidth, width=0.38\textwidth]{figures/abla_perpath_perf/AUC_CQ500.pdf}
    \includegraphics[height=0.24\textwidth, width=0.45\textwidth]{figures/abla_perpath_perf/AP_CQ500.pdf}
    \caption{\textbf{Performance for Different Percentage of Pre-training Samples (Per-Pathology).} This plot illustrates label efficiency for individual pathologies using Tukey plots, alongside the average performance across all diseases shown in \Cref{fig:scaling_law}. The results indicate that the majority of pathologies show improved downstream performance as the amount of pretraining data increases.}
    \label{fig:boxplot_scaling}
\end{figure}


\newpage

\section{Time complexity increase with reduced patch size}
\label{apd:self_attention_rate}
Assume we have 3D image input of shape $H\times W\times D$, patch size $P$ and reducing factor $s$. By time complexity of self-attention $O(n^2 d)$ for sequence length $n$ and embedding dimension $d$, the new time complexity after reducing patch size can be derived as
\begin{align*}
    O(n^2d)&=O((\frac{H\times W\times D}{(\frac{P}{s})^3})^2d) \\
           &=O((\frac{H\times W\times D}{P^3})^2 s^6d)  \\
           &=O(s^6)O(n_{ori}^2d)
\end{align*}
where $n_{ori}=\frac{H\times W\times D}{P^3}$ is the original sequence length before reducing patch size.



















\newpage
\begin{figure}[ht]
    \centering
    \includegraphics[width=\textwidth]{images/tsne_embedding_visualization_per_pathology.png}
    \caption{The 2D projection with t-SNE of CT volume representation extracted from the foundation model. Interestingly, certain subgroups still exhibited slightly better AUCs. For instance, scans with slice thicknesses between 1–4 mm (represented by light green points in the upper panel of \Cref{fig:batch_effect}) align with a specialized protocol for CT angiography (CTA), which uses contrast dye to improve diagnosis on particular diseases.}
    \label{fig:batch_effect}
\end{figure}


\begin{figure*}[ht]
    \centering
    \begin{subfigure}[b]{0.33\textwidth}
        \centering
        \includegraphics[width=\textwidth]{images/AUROC_vs_Slice_thickness_binned.png}
        \caption{AUROC Performance}
    \end{subfigure}
    \hfill
    \begin{subfigure}[b]{0.33\textwidth}
        \centering
        \includegraphics[width=\textwidth]{images/AUPRC_vs_Slice_thickness_binned.png}
        \caption{AUPRC Performance}
    \end{subfigure}
    \hfill
    \begin{subfigure}[b]{0.33\textwidth}
        \centering
        \includegraphics[width=\textwidth]{images/Histogram_of_slice_thickness_distribution_across_scans.png}
        \caption{Histogram of slice thickness distribution}
    \end{subfigure}
    \caption{The downstream task performances on various ranges of slice thickness.}
    \label{fig:thickness-ablation}
\end{figure*}


\begin{figure*}[ht]
    \centering
    \begin{subfigure}[b]{\textwidth}
        \centering
        \includegraphics[width=\textwidth]{images/AUROC_vs_slice_thickness_for_each_disease_category.png}
        \caption{AUROC Performance}
    \end{subfigure}
    \hfill
    \begin{subfigure}[b]{\textwidth}
        \centering
        \includegraphics[width=\textwidth]{images/AUPRC_vs_slice_thickness_for_eachdisease_category.png}
        \caption{AUPRC Performance}
    \end{subfigure}
    \hfill
    \begin{subfigure}[b]{\textwidth}
        \centering
        \includegraphics[width=\textwidth]{images/Ratio_of_positive_labels_vs_slice_thickness_for_each_disease_category.png}
        \caption{Ratio of Positive Labels}
    \end{subfigure}
    \caption{Performance for Each Slice Thickness Bin (Per Pathology).}
    \label{fig:slice_thickness_per_pathology}
\end{figure*}


\begin{figure*}[ht]
    \centering
    \begin{subfigure}[b]{0.3\textwidth}
        \centering
        \includegraphics[width=\textwidth]{images/AUROC_by_Disease_and_Manufacturer.png}
        \caption{AUROC Performance}
    \end{subfigure}
    \hfill
    \begin{subfigure}[b]{0.3\textwidth}
        \centering
        \includegraphics[width=\textwidth]{images/AUPRC_by_Disease_and_Manufacturer.png}
        \caption{AUPRC Performance}
    \end{subfigure}
    \hfill
    \begin{subfigure}[b]{0.39\textwidth}
        \centering
        \includegraphics[width=\textwidth]{images/Positive_Label_Ratio_by_Disease_and_Manufacturer.png}
        \caption{Distribution of Scans from Each Manufacturer}
    \end{subfigure}
    \caption{Performance for Each Manufacturer (Per Pathology).}
    \label{fig:manufacturer_per_pathology}
\end{figure*}















% Can use something like this to put references on a page
% by themselves when using endfloat and the captionsoff option.
\ifCLASSOPTIONcaptionsoff
  \newpage
\fi

\clearpage

% trigger a \newpage just before the given reference
% number - used to balance the columns on the last page
% adjust value as needed - may need to be readjusted if
% the document is modified later
%\IEEEtriggeratref{8}
% The "triggered" command can be changed if desired:
%\IEEEtriggercmd{\enlargethispage{-5in}}

% references section

% can use a bibliography generated by BibTeX as a .bbl file
% BibTeX documentation can be easily obtained at:
% http://mirror.ctan.org/biblio/bibtex/contrib/doc/
% The IEEEtran BibTeX style support page is at:
% http://www.michaelshell.org/tex/ieeetran/bibtex/
%\bibliographystyle{IEEEtran}
% argument is your BibTeX string definitions and bibliography database(s)
%\bibliography{IEEEabrv,../bib/paper}
%
% <OR> manually copy in the resultant .bbl file
% set second argument of \begin to the number of references
% (used to reserve space for the reference number labels box)

% biography section
% 
% If you have an EPS/PDF photo (graphicx package needed) extra braces are
% needed around the contents of the optional argument to biography to prevent
% the LaTeX parser from getting confused when it sees the complicated
% \includegraphics command within an optional argument. (You could create
% your own custom macro containing the \includegraphics command to make things
% simpler here.)
%\begin{IEEEbiography}[{\includegraphics[width=1in,height=1.25in,clip,keepaspectratio]{mshell}}]{Michael Shell}
% or if you just want to reserve a space for a photo:



% insert where needed to balance the two columns on the last page with
% biographies
%\newpage
% \newpage
\appendix

\renewcommand{\figurename}{Supplementary Figure}
\renewcommand{\tablename}{Supplementary Table}
\setcounter{figure}{0}
\setcounter{table}{0}

    



\section{Details of datasets}
This section provides additional details about the dataset used to evaluate the downstream tasks. \Cref{tab:disease_definition} lists the ICD-10 codes and medications used to identify the diagnoses for each disease. \Cref{tab:characteristic} presents the distribution of patient characteristics for each disease. \Cref{fig:nyu_langone_prevalence,fig:nyu_longisland_prevalence} illustrates the prevalence of each disease in the downstream tasks for the NYU Langone and NYU Long Island datasets, highlighting the imbalances present in these tasks.

\begin{table}[!htpb]
    \centering
    \caption{The definition of diseases in EHR by diagnosis codes and medications.}
    \begin{tabular}{lr}
    \toprule
         Disease &  Definition in EHR \\
    \midrule
       IPH  &  I61.0, I61.1, I61.2, I61.3, I61.4, I61.8, I61.9 \\
       IVH  &  I61.5, P52.1, P52.2, P52.3  \\
       ICH  &  IPH + IVH + I61.6, I62.9, P10.9, P52.4, P52.9 \\
       SDH  &  S06.5, I62.0 \\
       EDH  &  S06.4, I62.1 \\
       SAH  &  I60.*, S06.6, P52.5, P10.3  \\
       Tumor  &  C71.*, C79.3, D33.0, D33.1, D33.2, D33.3, D33.7, D33.9  \\
       Hydrocephalus  &  G91.* \\
       Edema  &  G93.1, G93.5, G93.6, G93.82, S06.1 \\
       \multirow{2}{*}{ADRD}  &  G23.1, G30.*, G31.01, G31.09, G31.83, G31.85, G31.9, F01.*, F02.*, F03.*, G31.84, G31.1, \\ 
       & \textbf{Medication:} DONEPEZIL, RIVASTIGMINE, GALANTAMINE, MEMANTINE, TACRINE \\ 
    \bottomrule
    \end{tabular}
    \label{tab:disease_definition}
\end{table}

\begin{table}[!htbp]
\centering
\caption{Demographic characteristics of patients associated with scans from the NYU Langone dataset, matched with electronic health records (EHR) and utilized in downstream tasks.}
\label{tab:characteristic}

 The characteristic table on NYU Langone dataset matched with EHR.
\begin{tabular}{ll|rr|r}
\toprule
                       \textbf{Cohort} &  &           \textbf{Male (\%)} &          \textbf{Female (\%)} &     \textbf{Age (std)} \\
\midrule
 --- & All (n=270,205) & 128,113 (47.41\%) & 142,092 (52.59\%) & 63.64 (19.68) \\
\midrule
       Tumor & Neg (n=260,704) & 123,338 (47.31\%) & 137,366 (52.69\%) & 63.85 (19.72) \\
             & Pos (n=9,501) &   4,775 (50.26\%) &   4,726 (49.74\%) & 57.80 (17.67) \\
\midrule
HCP & Neg (n=253,000) & 118,881 (46.99\%) & 134,119 (53.01\%) & 63.67 (19.72) \\
              & Pos (n=17,205) &   9,232 (53.66\%) &   7,973 (46.34\%) & 63.18 (19.11) \\
\midrule
Edema & Neg (n=242,576) & 112,987 (46.58\%) & 129,589 (53.42\%) & 63.96 (19.84) \\
      & Pos (n=27,629) &  15,126 (54.75\%) &  12,503 (45.25\%) & 60.81 (17.97) \\
\midrule
ADRD  & Neg (n=232,667) & 111,159 (47.78\%) & 121,508 (52.22\%) & 61.31 (19.55) \\
      & Pos (n=37,538) &  16,954 (45.16\%) &  20,584 (54.84\%) & 78.09 (13.30) \\
\midrule
          IPH & Neg (n=251,308) & 117,692 (46.83\%) & 133,616 (53.17\%) & 63.58 (19.82) \\
              & Pos (n=18,897) &  10,421 (55.15\%) &   8,476 (44.85\%) & 64.39 (17.69) \\
\midrule
          IVH & Neg (n=258,232) & 121,686 (47.12\%) & 136,546 (52.88\%) & 63.65 (19.79) \\
              & Pos (n=11,973) &   6,427 (53.68\%) &   5,546 (46.32\%) & 63.45 (17.19) \\
\midrule
          SDH & Neg (n=248,468) & 114,869 (46.23\%) & 133,599 (53.77\%) & 63.44 (19.78) \\
              & Pos (n=21,737) &  13,244 (60.93\%) &   8,493 (39.07\%) & 65.95 (18.33) \\
\midrule
          EDH & Neg (n=265,431) & 125,113 (47.14\%) & 140,318 (52.86\%) & 63.77 (19.64) \\
              & Pos (n=4,774) &   3,000 (62.84\%) &   1,774 (37.16\%) & 56.53 (20.75) \\
\midrule
          SAH & Neg (n=251,594) & 118,424 (47.07\%) & 133,170 (52.93\%) & 63.79 (19.76) \\
              & Pos (n=18,611) &   9,689 (52.06\%) &   8,922 (47.94\%) & 61.59 (18.49) \\
\midrule
          ICH & Neg (n=229,851) & 105,498 (45.90\%) & 124,353 (54.10\%) & 63.41 (19.93) \\
              & Pos (n=40,354) &  22,615 (56.04\%) &  17,739 (43.96\%) & 64.93 (18.14) \\
\bottomrule
\end{tabular}
\end{table}


\begin{table}[!h]
    \centering
    \caption*{\textbf{Supplementary \Cref{tab:characteristic} Continued.} Demographic characteristics of patients associated with scans from the NYU Long Island dataset, matched with electronic health records (EHR) and utilized in downstream tasks.}
\begin{tabular}{ll|rr|r}
\toprule
                       \textbf{Cohort} &  &           \textbf{Male (\%)} &          \textbf{Female (\%)} &     \textbf{Age (std)} \\
\midrule
--- & All (n=22,158) & 9,580 (43.23\%) & 12,578 (56.77\%) & 68.33 (18.14) \\
\midrule
Tumor & Neg (n=21,578) & 9,275 (42.98\%) & 12,303 (57.02\%) & 68.59 (18.08) \\
      & Pos (n=580) &   305 (52.59\%) &    275 (47.41\%) & 58.78 (17.79) \\
\midrule
HCP   & Neg (n=20,653) & 8,718 (42.21\%) & 11,935 (57.79\%) & 69.05 (17.90) \\
      & Pos (n=1,505) &   862 (57.28\%) &    643 (42.72\%) & 58.52 (18.48) \\
\midrule
Edema & Neg (n=19,402) & 8,068 (41.58\%) & 11,334 (58.42\%) & 68.89 (18.27) \\
      & Pos (n=2,756) & 1,512 (54.86\%) &  1,244 (45.14\%) & 64.36 (16.66) \\
\midrule
ADRD  & Neg (n=19,537) & 8,391 (42.95\%) & 11,146 (57.05\%) & 66.78 (18.28) \\
      & Pos (n=2,621) & 1,189 (45.36\%) &  1,432 (54.64\%) & 79.90 (11.77) \\
\midrule
IPH   & Neg (n=19,357) & 7,974 (41.19\%) & 11,383 (58.81\%) & 68.97 (18.27) \\
      & Pos (n=2,801) & 1,606 (57.34\%) &  1,195 (42.66\%) & 63.89 (16.48) \\
\midrule
IVH   & Neg (n=19,636) & 8,164 (41.58\%) & 11,472 (58.42\%) & 68.96 (18.22) \\
      & Pos (n=2,522) & 1,416 (56.15\%) &  1,106 (43.85\%) & 63.43 (16.66) \\
\midrule
SDH   & Neg (n=20,885) & 8,870 (42.47\%) & 12,015 (57.53\%) & 68.33 (18.21) \\
      & Pos (n=1,273) &   710 (55.77\%) &    563 (44.23\%) & 68.37 (16.83) \\
\midrule
EDH   & Neg (n=21,912) & 9,443 (43.10\%) & 12,469 (56.90\%) & 68.33 (18.16) \\
      & Pos (n=246) &   137 (55.69\%) &    109 (44.31\%) & 68.19 (15.59) \\
\midrule
SAH   & Neg (n=20,652) & 8,824 (42.73\%) & 11,828 (57.27\%) & 68.68 (18.12) \\
      & Pos (n=1,506) &   756 (50.20\%) &    750 (49.80\%) & 63.58 (17.65) \\
\midrule
ICH   & Neg (n=18,388) & 7,456 (40.55\%) & 10,932 (59.45\%) & 68.92 (18.35) \\
      & Pos (n=3,770) & 2,124 (56.34\%) &  1,646 (43.66\%) & 65.48 (16.77) \\
\bottomrule
\end{tabular}
\end{table}

\begin{figure}[!ht]
    \centering
    \includegraphics[width=0.8\textwidth]{images/NYU_Langone_prevalence.pdf}
    \caption{Disease prevalence of NYU Langone }
    \label{fig:nyu_langone_prevalence}
\end{figure}

\begin{figure}[!h]
    \centering
    \includegraphics[width=0.8\textwidth]{images/NYU_Longisland_prevalence.pdf}
    \caption{Disease prevalence of NYU Longisland dataset}
    \label{fig:nyu_longisland_prevalence}
\end{figure}



\section{Data augmentation details}
\label{sec:dataaug_details}
We applied Random Flipping across all three dimensions, Random Shift Intensity with offset $0.1$ for both pre-training and fine-tuning. For DINO training. random Gaussian Smoothing with sigma=$(0.5-1.0)$ is applied across all dimensions, Random Gamma Adjust is applied with gamma=$(0.2-1.0)$.


\section{Additional experiment results}
This section provides additional experimental results with more details.
Supplementary \Cref{fig:channels-ablation,fig:patches-ablation} compares the performance of the foundation model using different numbers of channels and patch sizes, demonstrating that the architecture design of our foundation model is optimal. 

Supplementary \Cref{fig:radar-comparison-merlin} compares our foundation model with a foundation CT model from previous studies, Merlin\cite{blankemeier2024merlinvisionlanguagefoundation}, which was trained on abdomen CT scans with corresponding radiology report pairs. Our model demonstrates superior performance on head CT scans.

Supplementary \Cref{fig:probing-comparison-gemini} compares our foundation model with Google CT Foundation model~\cite{yang2024advancingmultimodalmedicalcapabilities}, which was trained on large scale and diverse CT scans from different anatomy with corresponding radiology report pairs. Our model consistently shows improved performance across the board even though our model was pre-trained with less samples.

Supplementary \Cref{fig:probing_comparison} compares the performance on downstream tasks with various supervised tuning methods applied to foundation models pretrained with the MAE and DINO frameworks. Per-pathology comparisons are shown in Supplementary \Cref{fig:probing-comparison-perpath,fig:probing-comparison-perpath-dino}. Meanwhile, supplementary \Cref{fig:boxplot_scaling} complements \Cref{fig:scaling_law}, illustrating the per-pathology performances of foundation models pretrained with different scales of training data.

Supplementary \Cref{fig:batch_effect,fig:thickness-ablation} studies the impact of batch effect caused by different CT scan protocols of slice thickness and machine manufacturer. Detailed per-pathology performances are shown in Supplementary \Cref{fig:slice_thickness_per_pathology,fig:manufacturer_per_pathology}.

\begin{figure}[!htpb]
    \centering
    \makebox[\textwidth][l]{%
        \hspace{0.3\textwidth}\textbf{NYU Langone}
    } \\[0.2cm]
    \includegraphics[trim={0 0 0 0},clip,height=0.3\textwidth, width=0.3\textwidth]{figures/abla_chans/AUC_chans_NYU.pdf}
    \includegraphics[trim={0 0 0 0},clip,height=0.3\textwidth, width=0.55\textwidth]{figures/abla_chans/AP_chans_NYU.pdf}\\
    \makebox[\textwidth][l]{
        \hspace{0.34\textwidth}\textbf{RSNA}
    } \\[0.2cm]
    \includegraphics[trim={0 0 0 0},clip,height=0.3\textwidth, width=0.3\textwidth]{figures/abla_chans/AUC_chans_RSNA.pdf}
    \includegraphics[height=0.3\textwidth, width=0.55\textwidth]{figures/abla_chans/AP_chans_RSNA.pdf} 
    \caption{\textbf{Comparison of Different Channels Performance.} This plot compares the performance of models trained using different numbers of channels (channels from multiple HU intervals vs. a single HU interval). Across two datasets, models using three channels from different HU intervals consistently outperform those using a single channel with a fixed HU interval. All models were pre-trained on $100\%$ of the pretraining data with MAE.}
    \label{fig:channels-ablation}
\end{figure}


\begin{figure}[!htpb]
    \centering
    \makebox[\textwidth][l]{%
        \hspace{0.3\textwidth}\textbf{NYU Langone}
    } \\[0.2cm]
    \includegraphics[trim={0 0 0 0},clip,height=0.3\textwidth, width=0.3\textwidth]{figures/abla_patches/AUC_patches_NYU.pdf}
    \includegraphics[trim={0 0 0 0},clip,height=0.3\textwidth, width=0.55\textwidth]{figures/abla_patches/AP_patches_NYU.pdf}\\
    \makebox[\textwidth][l]{
        \hspace{0.34\textwidth}\textbf{RSNA}
    } \\[0.2cm]
    \includegraphics[trim={0 0 0 0},clip,height=0.3\textwidth, width=0.3\textwidth]{figures/abla_patches/AUC_patches_RSNA.pdf}
    \includegraphics[height=0.3\textwidth, width=0.55\textwidth]{figures/abla_patches/AP_patches_RSNA.pdf} 
    \caption{\textbf{Comparison of Different Patches Performance.} This plot compares the performance of models trained with different patch sizes (12 vs. 16). The results demonstrate that smaller patch sizes consistently achieve better performance. All models were pre-trained on $100\%$ of the pretraining data with MAE.}
    \label{fig:patches-ablation}
\end{figure}


\begin{figure*}
    \centering
    \makebox[\textwidth][l]{%
        \hspace{0.06\textwidth}
        \textbf{NYU Langone} \hspace{0.06\textwidth} \textbf{NYU Long Island} \hspace{0.11\textwidth} \textbf{RSNA} \hspace{0.18\textwidth} \textbf{CQ500}
    } \\[0.2cm]
    \includegraphics[trim={0 0 0 0},clip,height=0.21\textwidth, width=0.21\textwidth]{figures/abla_radarplot_merlin/AUC_NYU.pdf}
    \includegraphics[trim={0 0 0 0},clip,height=0.21\textwidth, width=0.21\textwidth]{figures/abla_radarplot_merlin/AUC_Longisland.pdf}
    \includegraphics[trim={0 0 0 0},clip,height=0.21\textwidth, width=0.21\textwidth]{figures/abla_radarplot_merlin/AUC_RSNA.pdf}
    \includegraphics[trim={0 0 0 0},clip,height=0.21\textwidth, width=0.35\textwidth]{figures/abla_radarplot_merlin/AUC_CQ500.pdf}\\[0.2cm]
    \includegraphics[height=0.21\textwidth, width=0.21\textwidth]{figures/abla_radarplot_merlin/AP_NYU.pdf} 
    \includegraphics[height=0.21\textwidth, width=0.21\textwidth]{figures/abla_radarplot_merlin/AP_Longisland.pdf} 
    \includegraphics[height=0.21\textwidth, width=0.21\textwidth]{figures/abla_radarplot_merlin/AP_RSNA.pdf}
    \includegraphics[height=0.21\textwidth, width=0.35\textwidth]{figures/abla_radarplot_merlin/AP_CQ500.pdf}
    \caption{\textbf{Comparison to previous 3D Foundation Model.} This plot compares the performance of our model with Merlin~\cite{blankemeier2024merlinvisionlanguagefoundation} and models trained from scratch across four datasets for our model and ResNet50-3D. Our DINO trained model is used in this comparison. Our model demonstrates consistently superior performance across majority of diseases, with the exception of epidural hemorrhage (EDH) in the CQ500 dataset.}
    \label{fig:radar-comparison-merlin}
\end{figure*}



\begin{figure*}
    \centering
    \makebox[\textwidth][l]{%
        \hspace{0.10\textwidth}
        \textbf{NYU Langone} \hspace{0.08\textwidth} \textbf{NYU Long Island} \hspace{0.1\textwidth} \textbf{RSNA} \hspace{0.15\textwidth} \textbf{CQ500}
    } \\[0.2cm]
    \includegraphics[trim={0 0 0 0},clip, width=0.22\textwidth]{figures/abla_probing/AUC_NYU.pdf}
    \includegraphics[trim={0 0 0 0},clip, width=0.22\textwidth]{figures/abla_probing/AUC_Longisland.pdf}
    \includegraphics[trim={0 0 0 0},clip, width=0.22\textwidth]{figures/abla_probing/AUC_RSNA.pdf}
    \includegraphics[trim={0 0 0 0},clip, width=0.28\textwidth]{figures/abla_probing/AUC_CQ500.pdf}
    \\[0.2cm]
    \includegraphics[width=0.22\textwidth]{figures/abla_probing/AP_NYU.pdf} 
    \includegraphics[width=0.22\textwidth]{figures/abla_probing/AP_Longisland.pdf} 
    \includegraphics[width=0.22\textwidth]{figures/abla_probing/AP_RSNA.pdf}
    \includegraphics[width=0.28\textwidth]{figures/abla_probing/AP_CQ500.pdf}
    \caption{\textbf{Comparison of Different Downstream Training Methods.} This plot illustrates the downstream performance of models evaluated using fine-tuning and various probing methods across four datasets. In most cases, the DINO pre-trained model outperforms the MAE pre-trained model. All models were pre-trained on $100\%$ of the available pretraining data.}
    \label{fig:probing_comparison}
\end{figure*}


\begin{figure}
\centering
\makebox[\textwidth][l]{%
    \hspace{0.39\textwidth}\textbf{RSNA}
} \\[0.2cm]
\includegraphics[trim={0 0 0mm 0},clip,height=0.27\textwidth]{figures/abla_gemini/AUC_RSNA_Gemini.pdf}
\includegraphics[trim={0 0 5mm 0},clip,height=0.27\textwidth]{figures/abla_gemini/AP_RSNA_Gemini.pdf}

\makebox[\textwidth][l]{%
    \hspace{0.38\textwidth}\textbf{CQ500}
} \\[0.2cm]
\includegraphics[trim={0 0 10mm 0},clip,height=0.345\textwidth]{figures/abla_gemini/AUC_CQ500_Gemini.pdf}
\includegraphics[trim={0 0 5mm 0},clip,height=0.345\textwidth]{figures/abla_gemini/AP_CQ500_Gemini.pdf}

\caption{\textbf{Performance comparison of linear probing for Our Model vs. Google CT Foundation model} This plot compares our model performance vs. Google CT Foundation model\cite{yang2024advancing} and Merlin \cite{blankemeier2024merlinvisionlanguagefoundation} across all diseases on RSNA and CQ500. Since Google CT Foundation moudel requires uploading data to Google Cloud (not allowed on our private data) for requesting model embeddings with model weights inaccessible, only public dataset comparison is provided in this study. Similar to other evaluations, we observed that our model outperforms Google CT Foundation model across the board with the only exception on Midline Shift for Google CT Foundation model and EDH for Merlin.}
\label{fig:probing-comparison-gemini}
\end{figure}



\begin{figure}
    \centering
    \makebox[\textwidth][l]{%
        \hspace{0.35\textwidth}\textbf{NYU Langone}
    } \\[0.2cm]
    \includegraphics[trim={0 0 120mm 0},clip,height=0.255\textwidth]{figures/abla_probing_perpath/DINO_AUC_NYU_Langone.pdf}
    \includegraphics[trim={0 0 0 0},clip,height=0.255\textwidth]{figures/abla_probing_perpath/DINO_AP_NYU_Langone.pdf} \\
    \makebox[\textwidth][l]{
        \hspace{0.35\textwidth}\textbf{NYU Long Island}
    } \\[0.2cm]
    \includegraphics[trim={0 0 120mm 0},clip,height=0.255\textwidth]{figures/abla_probing_perpath/DINO_AUC_NYU_Long_Island.pdf}
    \includegraphics[trim={0 0 0 0},clip,height=0.255\textwidth]{figures/abla_probing_perpath/DINO_AP_NYU_Long_Island.pdf} 
    \makebox[\textwidth][l]{
        \hspace{0.4\textwidth}\textbf{RSNA}
    } \\[0.2cm]
    \includegraphics[trim={0 0 120mm 0},clip,height=0.24\textwidth]{figures/abla_probing_perpath/DINO_AUC_RSNA.pdf}
    \hspace{5mm}
    \includegraphics[trim={0 0 0 0},clip,height=0.24\textwidth]{figures/abla_probing_perpath/DINO_AP_RSNA.pdf} 
    \makebox[\textwidth][l]{
        \hspace{0.4\textwidth}\textbf{CQ500}
    } \\[0.2cm]
    \includegraphics[trim={0 0 120mm 0},clip,height=0.30\textwidth]{figures/abla_probing_perpath/DINO_AUC_CQ500.pdf} \hspace{5mm}
    \includegraphics[trim={0 0 0 0},clip,height=0.30\textwidth]{figures/abla_probing_perpath/DINO_AP_CQ500.pdf} 
    \caption{\textbf{Performance comparison of supervised finetuning methods per pathology on the foundation model trained with DINO.} This plot breaks down the average performance across all diseases shown in Supplementary \Cref{fig:probing_comparison}. The results show that fine-tuning the entire network achieves the best performance in most scenarios. However, linear probing closely approaches finetuning performance for many diseases especially on small or imbalanced dataset, underscoring the capability of our pre-trained models to generate representations that adapt effectively to diverse disease detection tasks.}
    \label{fig:probing-comparison-perpath-dino}
\end{figure}

\begin{figure}
    \centering
    \makebox[\textwidth][l]{%
        \hspace{0.35\textwidth}\textbf{NYU Langone}
    } \\[0.2cm]
    \includegraphics[trim={0 0 0 0},clip,height=0.24\textwidth, width=0.3\textwidth]{figures/abla_probing_perpath/AUC_NYU.pdf}
    \includegraphics[trim={0 0 0 0},clip,height=0.24\textwidth, width=0.45\textwidth]{figures/abla_probing_perpath/AP_NYU.pdf}\\
    \makebox[\textwidth][l]{
        \hspace{0.35\textwidth}\textbf{NYU Long Island}
    } \\[0.2cm]
    \includegraphics[trim={0 0 0 0},clip,height=0.24\textwidth, width=0.3\textwidth]{figures/abla_probing_perpath/AUC_Longisland.pdf}
    \includegraphics[trim={0 0 0 0},clip,height=0.24\textwidth, width=0.45\textwidth]{figures/abla_probing_perpath/AP_Longisland.pdf} 
    \makebox[\textwidth][l]{
        \hspace{0.4\textwidth}\textbf{RSNA}
    } \\[0.2cm]
    \includegraphics[trim={0 0 0 0},clip,height=0.24\textwidth, width=0.3\textwidth]{figures/abla_probing_perpath/AUC_RSNA.pdf}
    \includegraphics[height=0.24\textwidth, width=0.45\textwidth]{figures/abla_probing_perpath/AP_RSNA.pdf} 
    \makebox[\textwidth][l]{
        \hspace{0.4\textwidth}\textbf{CQ500}
    } \\[0.2cm]
    \includegraphics[trim={0 0 120mm 0},clip,height=0.24\textwidth]{figures/abla_probing_perpath/AUC_CQ500.pdf}
    \includegraphics[trim={0 0 0 0},clip,height=0.24\textwidth]{figures/abla_probing_perpath/AP_CQ500.pdf} 
    \caption{\textbf{Performance comparison of supervised finetuning methods per pathology on the foundation model trained with MAE.} The results reveal that attentive probing is significantly more effective than linear probing, consistent with findings from~\cite{Chen2024}. Furthermore, for many diseases, the performance of probing models approaches that of fine-tuning, demonstrating that our pre-trained models produce adaptable representations capable of detecting diverse diseases.}
    \label{fig:probing-comparison-perpath}
\end{figure}









\begin{figure}
    \centering
    \textbf{NYU Langone} \\
    \includegraphics[trim={0 0 0 0},clip,height=0.24\textwidth, width=0.38\textwidth]{figures/abla_perpath_perf/AUC_NYU.pdf}
    \includegraphics[height=0.24\textwidth, width=0.45\textwidth]{figures/abla_perpath_perf/AP_NYU.pdf} \\
    \textbf{NYU Long Island} \\
    \includegraphics[trim={0 0 0 0},clip,height=0.24\textwidth, width=0.38\textwidth]{figures/abla_perpath_perf/AUC_Longisland.pdf}
    \includegraphics[height=0.24\textwidth, width=0.45\textwidth]{figures/abla_perpath_perf/AP_Longisland.pdf} \\
    \textbf{RSNA} \\
    \includegraphics[trim={0 0 0 0},clip,height=0.24\textwidth, width=0.38\textwidth]{figures/abla_perpath_perf/AUC_RSNA.pdf}
    \includegraphics[height=0.24\textwidth, width=0.45\textwidth]{figures/abla_perpath_perf/AP_RSNA.pdf}\\
    \textbf{CQ500} \\
    \includegraphics[trim={0 0 0 0},clip,height=0.24\textwidth, width=0.38\textwidth]{figures/abla_perpath_perf/AUC_CQ500.pdf}
    \includegraphics[height=0.24\textwidth, width=0.45\textwidth]{figures/abla_perpath_perf/AP_CQ500.pdf}
    \caption{\textbf{Performance for Different Percentage of Pre-training Samples (Per-Pathology).} This plot illustrates label efficiency for individual pathologies using Tukey plots, alongside the average performance across all diseases shown in \Cref{fig:scaling_law}. The results indicate that the majority of pathologies show improved downstream performance as the amount of pretraining data increases.}
    \label{fig:boxplot_scaling}
\end{figure}


\newpage

\section{Time complexity increase with reduced patch size}
\label{apd:self_attention_rate}
Assume we have 3D image input of shape $H\times W\times D$, patch size $P$ and reducing factor $s$. By time complexity of self-attention $O(n^2 d)$ for sequence length $n$ and embedding dimension $d$, the new time complexity after reducing patch size can be derived as
\begin{align*}
    O(n^2d)&=O((\frac{H\times W\times D}{(\frac{P}{s})^3})^2d) \\
           &=O((\frac{H\times W\times D}{P^3})^2 s^6d)  \\
           &=O(s^6)O(n_{ori}^2d)
\end{align*}
where $n_{ori}=\frac{H\times W\times D}{P^3}$ is the original sequence length before reducing patch size.



















\newpage
\begin{figure}[ht]
    \centering
    \includegraphics[width=\textwidth]{images/tsne_embedding_visualization_per_pathology.png}
    \caption{The 2D projection with t-SNE of CT volume representation extracted from the foundation model. Interestingly, certain subgroups still exhibited slightly better AUCs. For instance, scans with slice thicknesses between 1–4 mm (represented by light green points in the upper panel of \Cref{fig:batch_effect}) align with a specialized protocol for CT angiography (CTA), which uses contrast dye to improve diagnosis on particular diseases.}
    \label{fig:batch_effect}
\end{figure}


\begin{figure*}[ht]
    \centering
    \begin{subfigure}[b]{0.33\textwidth}
        \centering
        \includegraphics[width=\textwidth]{images/AUROC_vs_Slice_thickness_binned.png}
        \caption{AUROC Performance}
    \end{subfigure}
    \hfill
    \begin{subfigure}[b]{0.33\textwidth}
        \centering
        \includegraphics[width=\textwidth]{images/AUPRC_vs_Slice_thickness_binned.png}
        \caption{AUPRC Performance}
    \end{subfigure}
    \hfill
    \begin{subfigure}[b]{0.33\textwidth}
        \centering
        \includegraphics[width=\textwidth]{images/Histogram_of_slice_thickness_distribution_across_scans.png}
        \caption{Histogram of slice thickness distribution}
    \end{subfigure}
    \caption{The downstream task performances on various ranges of slice thickness.}
    \label{fig:thickness-ablation}
\end{figure*}


\begin{figure*}[ht]
    \centering
    \begin{subfigure}[b]{\textwidth}
        \centering
        \includegraphics[width=\textwidth]{images/AUROC_vs_slice_thickness_for_each_disease_category.png}
        \caption{AUROC Performance}
    \end{subfigure}
    \hfill
    \begin{subfigure}[b]{\textwidth}
        \centering
        \includegraphics[width=\textwidth]{images/AUPRC_vs_slice_thickness_for_eachdisease_category.png}
        \caption{AUPRC Performance}
    \end{subfigure}
    \hfill
    \begin{subfigure}[b]{\textwidth}
        \centering
        \includegraphics[width=\textwidth]{images/Ratio_of_positive_labels_vs_slice_thickness_for_each_disease_category.png}
        \caption{Ratio of Positive Labels}
    \end{subfigure}
    \caption{Performance for Each Slice Thickness Bin (Per Pathology).}
    \label{fig:slice_thickness_per_pathology}
\end{figure*}


\begin{figure*}[ht]
    \centering
    \begin{subfigure}[b]{0.3\textwidth}
        \centering
        \includegraphics[width=\textwidth]{images/AUROC_by_Disease_and_Manufacturer.png}
        \caption{AUROC Performance}
    \end{subfigure}
    \hfill
    \begin{subfigure}[b]{0.3\textwidth}
        \centering
        \includegraphics[width=\textwidth]{images/AUPRC_by_Disease_and_Manufacturer.png}
        \caption{AUPRC Performance}
    \end{subfigure}
    \hfill
    \begin{subfigure}[b]{0.39\textwidth}
        \centering
        \includegraphics[width=\textwidth]{images/Positive_Label_Ratio_by_Disease_and_Manufacturer.png}
        \caption{Distribution of Scans from Each Manufacturer}
    \end{subfigure}
    \caption{Performance for Each Manufacturer (Per Pathology).}
    \label{fig:manufacturer_per_pathology}
\end{figure*}






% \begin{IEEEbiographynophoto}{Jane Doe}
% Biography text here.
% \end{IEEEbiographynophoto}

% You can push biographies down or up by placing
% a \vfill before or after them. The appropriate
% use of \vfill depends on what kind of text is
% on the last page and whether or not the columns
% are being equalized.

%\vfill

% Can be used to pull up biographies so that the bottom of the last one
% is flush with the other column.
%\enlargethispage{-5in}
% that's all folks
\end{document}


