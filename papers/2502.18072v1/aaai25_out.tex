
\documentclass[letterpaper]{article}
\usepackage{xeCJK}
\usepackage{amsmath}
 
\usepackage{aaai25}  
\usepackage{times}  
\usepackage{helvet}  
\usepackage{courier}  
\usepackage[hyphens]{url}  
\usepackage{graphicx} 
\urlstyle{rm} 
\def\UrlFont{\rm}  
\usepackage{natbib}  
\usepackage{caption} 
\frenchspacing  
\setlength{\pdfpagewidth}{8.5in}  
\setlength{\pdfpageheight}{11in}  
\usepackage{algorithm}
\usepackage[noend]{algorithmic} 
\usepackage{multirow}
\usepackage{graphicx}
\usepackage{subcaption}
\usepackage{pifont}
\newcommand{\xmark}{\ding{55}}
\usepackage{my}
\usepackage{booktabs}
\usepackage{amsmath}
\usepackage{makecell}
\allowdisplaybreaks  
\usepackage{newfloat}
\usepackage{listings}
\DeclareCaptionStyle{ruled}{labelfont=normalfont,labelsep=colon,strut=off} 
\lstset{
	basicstyle={\footnotesize\ttfamily},
	numbers=left,numberstyle=\footnotesize,xleftmargin=2em,
	aboveskip=0pt,belowskip=0pt,
	showstringspaces=false,tabsize=2,breaklines=true}
\floatstyle{ruled}
\newfloat{listing}{tb}{lst}{}
\floatname{listing}{Listing}
% \pdfinfo{
% /TemplateVersion (2025.1)
% }
\setcounter{secnumdepth}{0} 
\makeatletter
\newcounter{corrfn}\setcounter{corrfn}{0}
\def\corrauthor{
  \ifnum\value{corrfn}=0
    \thanks{Corresponding Author}
    \setcounter{corrfn}{\value{footnote}}
  \else
    \footnotemark[\value{corrfn}]
  \fi
}
\makeatother
\title{MRBTP:高效的多机器人行为树计划和协作  }
\author{
    蔡怡帅\equalcontrib,
    陈星霖\equalcontrib,
    蔡中轩
  \ifnum\value{corrfn}=0
    \thanks{Corresponding Author}
    \setcounter{corrfn}{\value{footnote}}
  \else
    \footnotemark[\value{corrfn}]
  \fi
,
    毛韵欣, 
    李明龙
  \ifnum\value{corrfn}=0
    \thanks{Corresponding Author}
    \setcounter{corrfn}{\value{footnote}}
  \else
    \footnotemark[\value{corrfn}]
  \fi
 ,
    杨文婧,
    王戟
}
    
\affiliations{
    \textsuperscript{\rm }计算机学院, 国防科技大学 \\ 
     \{ caiyishuai, chenxinglin, caizhongxuan, maoyunxin, liminglong10, wenjing.yang, wj \} @nudt.edu.cn
}
\iffalse
\title{我的出版标题---单一作者  }
\author {
    Author Name
}
\affiliations{
    Affiliation \\ 
    Affiliation Line 2 \\ 
    name@example.com
}
\fi
\iffalse
\title{我的出版标题---多个作者  }
\author {
    First Author Name\textsuperscript{\rm 1,\rm 2},
    Second Author Name\textsuperscript{\rm 2},
    Third Author Name\textsuperscript{\rm 1}
}
\affiliations {
    \textsuperscript{\rm 1}Affiliation 1 \\ 
    \textsuperscript{\rm 2}Affiliation 2 \\ 
    firstAuthor@affiliation1.com, secondAuthor@affilation2.com, thirdAuthor@affiliation1.com
}
\fi
\usepackage{bibentry}
\begin{document}

  

   \maketitle     

   \begin{abstract}多机器人任务计划和协作是机器人技术的关键挑战。尽管行为树(BTS)已成为一种流行的控制架构,并且是单个机器人计划的,但由于协调各种动作空间的复杂性,有效的多机器人BT计划算法的开发仍然具有挑战性。我们提出了多机器人行为树计划(MRBTP)算法,并具有理论上的声音和完整性保证。 MRBTP具有跨树扩展,以协调不同BTS的异质行动,以实现团队的目标。对于同质行动,我们保留BT之间的备份结构,以确保稳健性并通过意图共享来防止冗余执行。尽管MRBTP能够为同质和异质机器人团队生成BTS,但其效率可以进一步提高。然后,当可以使用大型语言模型(LLMS)来推理每个机器人与目标相关的操作时,我们为MRBTP提供了一个可选插件。可以预先计划这些相关的动作以形成长马子树,从而大大提高了MRBTP的计划速度和协作效率。我们在仓库管理和日常服务方案中评估我们的算法。结果表明,在不同的设置下,MRBTP的鲁棒性和执行效率以及预训练的LLM为MRBTP生成有效特定任务子树的能力。  \end{abstract}    
    \begin{links}
    \link{Code}{https://github.com/DIDS-EI/MRBTP}
\end{links}     

   \section{介绍  }     

与单机器人解决方案   \cite{colledanchise2016advantages}   相比,涉及具有多种功能的机器人的多机器人系统(MRS)为提高性能和容忍度的提高提供了潜力。开发自主的MRS需要有效,可靠的控制体系结构,以及将其适应特定任务的方法。由于其模块化,解释性,反应性和鲁棒性,行为树(BT)已成为流行的控制体系结构,因此非常适合单杆和多机器人系统   \cite{heppner2024behavior,heppnerl2023distributed,neupane2019learning,colledanchise2016advantages}   。随着BTS的潜力增加了关注,已经提出了各种自动生成BTS的方法,包括进化计算   \cite{neupane2019learning,colledanchise2019learning}   ,增强学习   \cite{banerjee2018autonomous,pereira2015framework}   和BT SYNTHESIS    \cite{tadewos2022specificationguided,neupane2023designing}   。在这些方法中,BT规划   \cite{chen2024integrating,chen2024efficient,cai2021bt,colledanchise2019blended}   在利用可解释的动作模型和生产可靠的BT来实现目标方面具有优势,这使其成为为自动驾驶机器人系统生成BTS的有希望的方法。  

   \begin{figure}[t]
	\centering
	\includegraphics[width=0.47\textwidth]{images/introduction.png}
	\caption{MRBTP计划的两个BTS的示例:(1)跨树扩展,(2)意图共享,(3)可选插件:子树预先计划。  }
	\label{fig:introduction}
\end{figure}     

但是,当前的BT计划主要关注一个机器人,而有效的多机器人BT计划算法的开发仍然具有挑战性。挑战主要来自两个方面:
    \begin{itemize}

   \item   对于异质行动,如何在不同的BTS上协调它们以实现团队的目标。   \item   对于同质动作,如何使用它们来提高容错而无需冗余执行。  \end{itemize}     

在本文中,我们提出了多机器人行为树计划(MRBTP),这是第一种声音和完整的算法,用于为MRS生成可靠且强大的BTS。 MRBTP解决了上述挑战,如下所示:
    \begin{itemize}

   \item   我们采用越树的扩展,其中所有BT将进一步扩展一个BT的条件。这意味着一个机器人可能会采取行动来满足他人的前提条件,从而实现多树的协作。   \item   我们允许备份结构通过具有同质动作的机器人扩展,以确保容错,同时使用意图共享以避免冗余执行。在执行过程中,每个机器人都会广播其当前动作,以便其他机器人可以预测其效果并避免执行具有相同效果的动作。  \end{itemize}     

如图   \ref{fig:introduction}   所示,在日常服务方案中,团队的目标是准备   \constant{Salad}   。如果满足前提条件   \constant{Has(Fruit)}   ,则类人机器人可以执行动作   \constant{Make(Fruit,Salad)}   。尽管它不能执行   \constant{Unload(Fruit,Package)}   ,但它可以将   \constant{Has}       \constant{(Fruit)}   推到计划队列,另一个四倍的机器人将通过跨树扩展扩展此操作,从而实现多树协调。  

在满足   \constant{In(Salad,Refrigerator)}   的另一种情况下,两个机器人都可以执行   \constant{Open(Refrigerator)}   和   \constant{Get(Salad)}   。在这种情况下,MRBTP将在两棵树中扩展相同的结构,以确保耐失功能。如果两个机器人都可用,则具有较高优先级的人形机器人将执行   \constant{Open}       \constant{(Refrigerator)}   并共享其意图。然后,四倍的机器人将假设   \constant{IsOpen}       \constant{(Refrigerator)}   是正确的,然后步行到   \constant{Refrigerator}   ,等待执行   \constant{Get(Salad,}       \constant{Refrigerator)}   ,只要   \constant{Refrigerator}   真正打开。意图共享确保并行化并提高机器人团队的执行效率。  

尽管MRBTP是一种与域无关的算法,但如果大型语言模型(LLMS)可用于域依赖性推理,则可以提高计划和执行效率。因此,我们进一步提出了一个名为Subtree Predning的可选插件。假设LLM可以根据每个机器人的功能为每个机器人提供一些有用的动作,我们可以使用这些动作在长期计划过程之前快速计划有用的子树结构。这些子树不仅可以提高计划速度,还可以降低执行过程中的通信费用。仓库管理和日常服务方案的实验证明了MRBTP在不同的设置下的稳健性和执行效率,以及预先训练的LLM为MRBTP生成有效特定任务子树的能力。  

   \section{背景  }    
    \paragraph{行为树。  }    a bt是一个有向根的树,执行节点与环境相互作用,并且控制流节点处理其子女的触发逻辑   \cite{colledanchise2018behavior}   。在每个时间步骤,BT都会启动通过控制节点的壁虱,确定机器人将根据环境状态执行的操作。本文主要关注四个典型的BT节点:  

   \begin{itemize}

   \item   条件   \includegraphics[height=0.75em]{images/BTnodes/condition.png}   :检查环境状态是否满足指定条件的执行节点,相应地返回 {    \ttfamily   成功   } 或 {    \ttfamily   失败   } 。   \item   操作   \includegraphics[height=0.75em]{images/BTnodes/action.png}   :控制机器人执行操作的执行节点,返回 {    \ttfamily   成功   } , {    \ttfamily   失败   } , {    \ttfamily   运行   } ,这取决于执行的结果。   \item   序列   \includegraphics[height=0.75em]{images/BTnodes/sequence.png}   :一个控制流节点,该节点仅返回 {    \ttfamily   成功   } ,如果其所有孩子成功。否则,它会将其从左到右勾选的孩子,并且第一个返回 {    \ttfamily   失败   } 或 {    \ttfamily   运行   } 的孩子将确定其返回状态。   \item   后备   \includegraphics[height=0.75em]{images/BTnodes/selector.png}   :逻辑与序列节点相反的控制流节点。它仅在所有孩子失败的情况下才返回 {    \ttfamily   失败   } 。如果不是,则在滴答作用期间的第一次出现 {    \ttfamily   成功   } 或 {    \ttfamily   运行   } 成为其返回状态。  \end{itemize}     

   \paragraph{BT计划。  }   在BT计划的单个机器人   \cite{cai2021bt}   中,我们代表BT为三翼   $\mathcal{T} = <f, r, \Delta t>$   。    $f:2^{n}\rightarrow 2^{n}$   是其对环境状态的影响,   $\Delta t$   是时间步骤,   $r:2^{n}\mapsto  \{  $       \constant{S}   ,   \constant{R}   ,   \constant{F}    \}分区分为三个区域,其中   $\mathcal{T}$   分别返回成功,运行,失败。  

然后,BT计划问题可以描述为:   \(<\mathcal{S},\mathcal{L},\mathcal{A},\mathcal{M}, s_0,g>\)   ,其中   \( \mathcal{S} \)   是有限环境状态集,   $\mathcal{L}$   是形成的有限文字集,   \( \mathcal{A} \)   是有限的操作集,   $\mathcal{M}$   是动作模型,   $s_0$   是   $s_0$   是   $s_0$   初始状态,   $g$   是目标条件。  

BT中的条件   $c$   通常是状态   $s$   的子集。如果   $c\subseteq s$   ,则说明条件   $c$   在该状态   $s$   中保存。受动作   $a\in \mathcal{A}$   影响的状态转变可以定义为三重态   \( \mathcal{M}(a)=<pre(a),add(a),del(a)> \)   ,包括前提条件,添加效果和动作的删除效果。如果   $a$   在   $k$   时间步骤之后完成,则随后的状态   $s_{t'}$   为:
    \begin{equation}\label{eqn:s_f}
	s_{t'}=f_a(s_t)=s_t\cup add(a)\setminus del(a), t'=t + k
\end{equation}     

   \section{问题制定  }   我们首先将BT表示从单个机器人扩展到多机器人系统。  

   \begin{definition}[Multi-BT System]
A         $n$        -robot BT system is a four-tuple         $\left<\Phi, f_\Phi, r_\Phi, \Delta t_\Phi\right>$        , where         $\Phi = \left \{  \mathcal{T}_i \right \} _{i=1}^n$         is the set of BTs,         $f_\Phi: \mathcal{S} \mapsto \mathcal{S}$         is the team state transition function,         $\Delta t_\Phi$         is the team time step,         $r_\Phi: \mathcal{S} \mapsto  \{ $         \constant{S}, \constant{R}, \constant{F}         $ \} $         is the team region partition.
\end{definition}   由于硬件性能的可变性,我们允许每个机器人的BT具有不同的响应频率,   $\Delta t_\Phi$   代表共同的最小响应间隔。国家过渡可以计算如下:
    \begin{align}
    s_{t+\Delta t_\Phi} = f_\Phi(s_t) = s_t \cup \bigcup_{i=1}^n \left( add(a_i) \setminus del(a_i) \right)
\end{align}   其中   $a_i$   是机器人   $i$   的动作   $t$   。如果机器人   $i$   没有操作或其操作正在运行,我们让   $add(a_i) = del(a_i) = \emptyset$   。  

团队区域分区可以按以下方式计算:
    \begin{equation}
    r_\Phi(s) = 
\begin{cases}
    \text{\constant{R}} & \text{if } \exists i, r_i(s) = \text{\constant{R}}  \\ 
    \text{\constant{S}} & \text{if } \forall i, r_i(s) \neq \text{\constant{R}} \text{ and } \exists i, r_i(s) = \text{\constant{S}}  \\ 
    \text{\constant{F}} & \text{if } \forall i, r_i(s) = \text{\constant{F}}
\end{cases}
\end{equation}       $\Phi$   的状态是   \constant{R}   如果BT仍在运行,则   \constant{S}   如果某些BT返回成功并且没有人正在运行,并且   \constant{F}   如果所有BT失败。
    \begin{definition}[Finite Time Successful]
        $\Phi$         is finite time successful (FTS) from region of attraction (ROA)         $R$         to condition         $c$        , if         $\forall s_0 \in R$          there is 
 a finite time         $\tau$         such that for any         $t<\tau$        ,         $r_\Phi(s_t)=$         \constant{R}, and for any         $t\geq\tau, r_\Phi(s_t)=$         \constant{S},         $c\in s_t$        .
\end{definition}   具有上述定义,最终可以定义多机器人BT计划问题。  

   \begin{problem}[Multi-Robot BT Planning]
The problem is a tuple         \(\left<\mathcal{S},\mathcal{L}, \{ \mathcal{A}_i \} _{i=1}^n,\mathcal{M}, s_0,g\right>\)        , where         \( \mathcal{S} \)         is the finite set of environment states,         $\mathcal{L}$         is the finite set literals that form states and conditions,         $\mathcal{A}_i$         is the finite action set of robot         $i$        ,         $\mathcal{M}$         is the action model,         $s_0$          is the initial state,         $g$         is the goal condition. A solution to this problem is a BT set         $\Phi =  \{ \mathcal{T}_i \} _{i=1}^n$         built with         $ \{ \mathcal{A}_i \} _{i=1}^n$        , such that         $\Phi$         is FTS from         $R\ni s_0$         to         $g$        .
\end{problem}     

   \begin{algorithm}[t]
\caption{一步横树扩展  }
\label{alg:one_step}
\begin{algorithmic}[1]
\State  function  \constant{ExpandOneRobot(}        $\mathcal{T},\mathcal{A}, c$        \constant{)}
\begin{ALC@g}
\StateComment{        $\mathcal{T}_{new} \leftarrow c$        }{newly expanded subtree}
\StateComment{        $\mathcal{C}_{new} \gets \emptyset$        }{newly expanded conditions}
\FOR{ each  action         $a \in \mathcal{A}$        }
    \IF{        $c \cap (pre(a) \cup add(a) \setminus del(a)) \neq \emptyset$         and         $c \setminus del(a) = c$        }
    \STATE         $c_a \leftarrow pre(a) \cup c \setminus add(a)$        
    \STATE         $\mathcal{T}_a \leftarrow Sequence(c_a, a)$        
    \STATE         $\mathcal{T}_{new} \leftarrow Fallback(\mathcal{T}_{new}, \mathcal{T}_a)$        
    \State{        $\mathcal{C}_{new} \gets \mathcal{C}_{new} \cup  \{ c_a \} $        }
    \ENDIF
\ENDFOR
\IF{        $\mathcal{C}_{new}\neq \emptyset$        }
    \IF{\constant{ConditionInTree}(        $c$        ,         $\mathcal{T}$        )}
        \StateComment{Replace         $c$         with         $\mathcal{T}_{new}$         in         $\mathcal{T}$        }{in-tree expand} \label{InsideExpand}
    \ELSIF{        $\mathcal{T}_{new} \neq c$        } 
        \StateComment{        $\mathcal{T} \gets Fallback(\mathcal{T},\mathcal{T}_{new}) $        }{cross-tree expand}
    \ENDIF
\ENDIF
\end{ALC@g}
\RETURN         $\mathcal{T}, \mathcal{C}_{new}$        
\end{algorithmic}
\end{algorithm}     

   \section{方法  }     

我们首先详细介绍了MRBTP,分析了其健全性,完整性和计算复杂性。然后,我们演示了执行过程中BTS之间的意图共享功能。最后,我们介绍了可选的插件,“子树预先计划”,以进一步提高效率。
    \subsection{多机器人行为树计划  }    
    \subsubsection{一步横树扩展  }   算法   \ref{alg:one_step}   给出了一个机器人的一步横树扩展的伪代码。鉴于其当前的BT    $\mathcal{T}$   ,ACTION SPACE    $\mathcal{A}$   和扩展   $c$   (第1行)的条件,该函数将返回扩展的BT    $\mathcal{T}$   ,以及新扩展的条件SET    $\mathcal{C}_{new}$   (第15行)。与单个机器人   \cite{cai2021bt}   的一步扩展类似,扩展以   $\mathcal{T}_{new}=c$   和   $\mathcal{C}_{new}=\emptyset$   (第2-3行)开头。然后,我们浏览动作空间   $\mathcal{A}$   ,以找到可能导致   $c$   (第4-5行)的所有前提动作。对于每个前提动作   $a$   ,我们将其相应的先决条件   $c_a$   (第6行)计算,形成序列结构   $\mathcal{T}_a$   (第7行),然后将   $\mathcal{T}_a$   添加到   $\mathcal{T}_{new}$   (第8行)的根落后节点的尾部。现在,   $\mathcal{T}_{new}$   可以使用这些扩展的动作实现   $c$   ,如果满足其先决条件。我们将这些先决条件存储在   $\mathcal{C}_{new}$   (第9行)中。  

如果   $\mathcal{T}_{new}$   扩展(第10行),我们需要确定   $\mathcal{T}$   中的位置以放置它。有两种情况:(1)   $c$   在   $\mathcal{T}$   中,这意味着以前由   $\mathcal{T}$   本身扩展。因此,我们在   $\mathcal{T}$   中将   $c$   替换为   $\mathcal{T}_{new}$   ,就像在单机器人BT扩展中一样(第11-12行); (2)   $c$   不在   $\mathcal{T}$   中,这意味着它是由其他BTS扩展的。为了允许该BT采取动作来实现   $c$   ,我们将其添加到   $\mathcal{T}$   的根部后退节点的尾部(第13-14行)。 
    \begin{proposition}\label{pro:inside}
    Given         $\mathcal{T}$         is FTS from         $R$         to         $g$        , if         $\mathcal{T}$         is expanded by Algorithm \ref{alg:one_step} to         $\mathcal{T}'$         given         $c$        ,         $c$         is in         $\mathcal{T}$         and         $\mathcal{C}_{new} \neq \emptyset$        , then         $\mathcal{T}'$         is FTS from         $R'=R\cup  \{  s\in \mathcal{S}| c_a \subseteq s, c_a\in \mathcal{C}_{new}  \} $         to         $g$        .
\end{proposition}     

   \begin{proposition}\label{pro:outside}
    If         $\mathcal{T}$         is expanded by Algorithm \ref{alg:one_step} to         $\mathcal{T}'$         given         $c$        ,         $c$         is not in         $\mathcal{T}$         and         $\mathcal{C}_{new} \neq \emptyset$        , then         $\mathcal{T}'$         is FTS from         $\mathcal{S}_{new}= \{  s\in \mathcal{S}| c_a \subseteq s, c_a\in \mathcal{C}_{new}  \} $         to         $c$        .
\end{proposition}   上述两个命题在一步横树扩展后说明了ROA的变化。如果   $c$   在   $\mathcal{T}$   (命题   \ref{pro:inside}   )中,则   $\mathcal{C}_{new}$   的ROA将扩展为   $\mathcal{C}_{new}$   ,以实现   $g$   。如果   $c$   不在   $\mathcal{T}$   (命题   \ref{pro:outside}   )中,则   $c$   将被视为   $\mathcal{T}$   的新子目标,可以从   $\mathcal{S}_{new}$   中实现。    \footnote{所有正式的证明都在附录中。  }     

   \begin{algorithm}[t]
\caption{MRBTP  }
\label{alg:MABTP}
 Input : problem         $\tuple{\mathcal{S},\mathcal{L},\multi{\mathcal{A}},\mathcal{M},s_0,c}$          \\ 
 Output : solution         $\Phi = \multi{\mathcal{T}}$         
\begin{algorithmic}[1]

\STATE         $\mathcal{C}_U\gets  \{ g \} $         \label{line:cu}
		\hfill         $\triangleright$         conditions to be explored
\StateCommentLabel{        $\mathcal{C}_E\gets \emptyset$        }{expanded conditions}{line:ce}
\FOR{        $i = 1$         \TO         $n$        }
    \STATE         $\mathcal{T}_i \gets Fallback(g)$         \label{line:initBT} \Comment{init the BTs}
\ENDFOR
\WHILE{         $\mathcal{C}_U \neq \emptyset$        }
\STATE         $c\gets $         \constant{Pop(}        $\mathcal{C}_U$        \constant{)} \label{line:pickC}
\hfill         $\triangleright$         explore         $c$        
\StateComment{ if  \algofunc{HasSubSet}{        $c,\mathcal{C}_E$        }  then   continue }{prune}  

\FOR{        $i = 1$         \TO         $n$        }
    \STATE         $\mathcal{T}_i, \mathcal{C}_{new} \gets$         \algofunc{ExpandOneRobot}{        $\mathcal{T}_i,\mathcal{A}_i,c$        } \label{line:initBT}
    \IF{\algofunc{HasSubSet}{        $s_0,\mathcal{C}_{new}$        }}
        \RETURN         $\Phi= \{ \mathcal{T}_i \} _{i=1}^n$         \Comment{return a solution}
    \ENDIF
    \State{        $\mathcal{C}_E \gets \mathcal{C}_E \cup \mathcal{C}_{new}$        }
    \StateCommentLabel{        $\mathcal{C}_U \gets \mathcal{C}_U \cup \mathcal{C}_{new}$        }{add new conditions}{}
\ENDFOR


\ENDWHILE
  
\RETURN \constant{Unsolvable}
    
\end{algorithmic}
\end{algorithm}     

   \subsubsection{MRBTP  }   算法   \ref{alg:MABTP}   提供了MRBTP的伪代码,以计划整个机器人团队的BTS。该算法初始化了一组要探索的条件   $\mathcal{C}_U= \{ g \} $   和一组扩展的条件   $\mathcal{C}_E=\emptyset$   (第1-2行)。每个机器人   $i$   的BT初始化为   $\mathcal{T}_i=Fallback(g)$   (第4行),即从   $\emptyset$   到   $g$   。然后,该算法将不断探索   $\mathcal{C}_U$   (第5-6行)中的条件,直到找到解决方案为止,否则返回   \constant{Unsolvable}   (第14行)。对于每个探索的   $c$   ,如果   $\exists c' \in \mathcal{C}_E, c'\subseteq c$   (第7行)进行修剪,或者由所有机器人通过一步横树扩展扩展(第8-9行)。在每个机器人的一步扩展之后,新扩展的条件   $\mathcal{C}_{new}$   将附加到   $\mathcal{C}_E$   和   $\mathcal{C}_U$   (第12-13行)。如果当时找到解决方案的   $\exists c'\in \mathcal{C}_{new}, c'\subseteq s_0$   ,则该算法将   $\Phi= \{ \mathcal{T}_i \} _{i=1}^n$   返回作为解决方案(第10-11行)。
    \begin{proposition}\label{pro-k-loop}: After the         $k$        -th (        $k\geq1$        ) iteration of the while loop in Algorithm \ref{alg:MABTP}, where the explored condition is         $c^{k}$        ,         $\Phi^{k}=\multi{\mathcal{T}^{k}}$         is FTS from ROA         $R^k=R^{k-1} \cup \bigcup_{i=1}^n \mathcal{S}^k_i$         to goal         $g$        , where         $\mathcal{S}^{k}_i= \{ s\in \mathcal{S}|c_a\subseteq s,c_a\in C^k_{i,new} \} $        .
\end{proposition}   注意,命题   \ref{pro-k-loop}   不能自然地从单个BT计划中扩展。该命题要求机器人以适当的顺序执行(在任何时间步骤中,只有优先级的机器人都可以执行诉讼,如果满足其先决条件,则可以执行机器人;否则,可能会发生僵局或离开ROA。幸运的是,我们始终可以在执行过程中使用诸如僵局检测之类的机制,以确保仅在特殊情况下使用该串行执行。在绝大多数情况下,机器人可以并行执行,因此无需担心此假设会降低机器人团队的执行效率。  

   \begin{proposition}\label{pro-sound}
Algorithm \ref{alg:MABTP} is sound, i.e. if it returns a result         $\Phi$         rather than \constant{Unsolvable}, then         $\Phi$         is a solution of Problem 1.
\end{proposition}     

   \begin{figure*}[t]
	\centering
	\includegraphics[width=1\textwidth]{images/framework.png}
	\caption{我们论文的框架。 (1)MRBTP。多机器人BT计划问题的声音和完整算法,能够通过跨树扩展来协调不同BT的各种动作。 (2)意图共享。机器人在执行过程中彼此共享意图,从而在不损害故障公差的情况下实现多BT并行化。 (3)可选插件:子树预先计划。该插件将LLMS用于预先计划的特定任务特定子树,建立长马操作序列以提高MRBTP的计划和执行效率。  }

 
	\label{fig:failure_example}
\end{figure*}     

   \begin{proposition}\label{pro-complete}
Algorithm \ref{alg:MABTP} is complete, i.e., if Problem 1 is solvable, the algorithm returns a         $\Phi$         which is a solution.
\end{proposition}     

命题   \ref{pro-sound}   可以通过基于命题   \ref{pro-k-loop}   的强归纳来证明,并且可以根据命题   \ref{pro-sound}   来证明命题   \ref{pro-complete}   。这两个命题指出了MRBTP的健全性和完整性,这使其成为解决多机器人BT计划问题的有效算法。  

在最坏情况下,MRBTP的时间复杂性是   $O(|\bigcup_{i=1}^n \mathcal{A}_i||\mathcal{S}||\mathcal{L}|)$   ,它是系统大小的多项式。在这种情况下,该算法必须探索所有状态   $s \in \mathcal{S}$   才能找到解决方案。在每个探索中,都将检查所有机器人的动作,并检查   $O(|\mathcal{L}|)$   的复杂性。
    \subsection{多BT执行的意图共享  }     

从MRBTP计划过程中,我们可以观察到,如果多个机器人具有相同的动作(或具有相同效果的相似动作),则MRBTP将同时在不同的BT中扩展它们。这可能会导致备份结构。这些结构对容错性是有益的,因为如果一个机器人失败,其他机器人可以接管并完成该动作。但是,当有多个机器人可用时,备份结构可能会导致冗余执行。为了避免这种情况,我们介绍了基于通信的多BT意图共享方法。  

   \paragraph{意图队列  }   在执行过程中,每个机器人   $i$   保持意图队列   $\mathcal{I}_i = (a_1, a_2, \dots, a_m)$   ,该   $\mathcal{I}_i = (a_1, a_2, \dots, a_m)$   指示其他机器人执行的操作。在沟通良好的情况下,所有机器人的意图队列都应保持一致。因此,在以下文本中,我们使用   $\mathcal{I}$   参考所有机器人的意图队列。基于意图队列,我们可以计算机器人   $i$   的信念成功空间   $\mathcal{B}^S_i$   和信念故障空间   $\mathcal{B}^F_i$   :
    \setlength{\jot}{5pt}    
    \begin{gather}
\mathcal{B}^S_{i} = \bigcup_{k=1}^{j-1} \left( add(a_k) \setminus del(a_k) \right)  \\ 
\mathcal{B}^F_i = \bigcup_{k=1}^{j-1} \left( del(a_k) \setminus add(a_k) \right) 
\end{gather}   其中   $j$   是意图队列   $\mathcal{I}$   中其自己的操作   $a_j$   的索引。如果   $j=1$   ,则   $\mathcal{B}^S_i = \mathcal{B}^F_i = \emptyset$   ,这意味着该动作不取决于任何其他人的意图。如果机器人当前没有动作,则在计算信念空间时将其视为   $j = m+1$   。  

   $\mathcal{B}^S$   和   $\mathcal{B}^F$   将在每个BT的滴答过程中使用。对于由单个字面   $c = {l}$   表示的每个原子条件节点,它将首先检查   $l$   是否在勾选时是否在信念空间中。如果   $l \in \mathcal{B}^S$   ,它将返回   \constant{S}   而无需与环境交互,并在   $l \in \mathcal{B}^F$   时返回   \constant{F}   。  

每当机器人   $i$   退出操作或进入新操作时,它都会向其他所有机器人广播。然后,每个机器人从意图队列   $\mathcal{I}$   (如果存在)将机器人   $i$   的旧操作删除,并将新操作推入其中(如果适用)。此后,每个机器人将更新其信念空间   $\mathcal{B}^S$   和   $\mathcal{B}^F$   ,以反应地调整其动作。请注意,退出或进入的动作可能是由于两种情况引起的:(1)环境状态发生了变化,或(2)信念空间已经改变。结果,意图队列   $\mathcal{I}$   中的任何添加或删除都可能导致其他动作的调整,从而产生链反应。换句话说,我们的意图共享方法维持了对不确定环境的反应性和鲁棒性。  

   \subsubsection{并行性和阻塞  }     

尽管意图共享可以避免冗余执行,但它还增强了机器人团队内的动作并行性。例如,如图   \ref{fig:failure_example}   所示,在仓库管理方案中,有两个机器人能够打开门和运输包装。他们已经扩展了几乎相同的树结构,依次执行   \constant{Open}       \constant{(Door)}   ,   \constant{Walk}       \constant{(Package)}   ,???    \constant{(Package)}   。但是,由于   \constant{IsClose(Door)}       $\in s_0$   ,两个机器人都满足执行   \constant{Open}       \constant{(Door)}   的前提条件。他们没有意图共享,他们会同时执行此操作,从而导致冗余。但是,在意图共享的情况下,如果Robot    $1$   首先勾选其BT    $\mathcal{T}_1$   ,它将执行   \constant{Open(Door)}   并将此意图发送到Robot    $2$   。对于机器人   $2$   ,   \constant{IsOpen(Door)}       $\in \mathcal{B}^S_2$   更新意图队列   $\mathcal{I}$   ,因此   \constant{IsOpen(Door)}   的相应条件节点将返回???,允许BT    $\mathcal{T}_2$   继续tick tick tick tick和启动tick ???。这将串行BT结构转换为并行执行。  

但是,当机器人   $2$   尝试执行   \constant{Move}       \constant{(Package)}   (   \constant{(Package)}   ),该   \constant{(Package)}   依赖于前提   $a$    ???时,Robot    $2$   将等待直到车门实际上由Robot    $1$   打开。正式地说,如果   $l\in \mathcal{B}^S_i$   ,但在当前状态   $l\notin s$   中,当Robot    $i$   尝试执行   $a$   时,将阻止   $l\in pre(a)$   ,   $a$   将被阻止。在这种情况下,Robot    $i$   共享   $a$   的意图,而   $a$   返回   \constant{R}   ,就像执行一样,但实际上什么都不做。阻止机制可以防止在不正确的先决条件下执行操作,同时还可以并行执行后续操作,从而进一步提高了机器人团队的执行效率。  

   \subsection{可选插件:子树预先计划  }     

尽管事实证明,具有意图共享的MRBTP是多机器人BT计划问题的有效和有效算法,但仍有进一步提高其效率的空间。为了实现这一目标,我们首先考虑以下观察结果。
    \begin{itemize}

   \item   在计划期间,可能会多次生成相同的树结构,尤其是当多个机器人重叠的动作空间时。   \item   在执行过程中,分享每个短术语原子能动作的意图不仅增加了沟通开销,而且对于长期任务计划也无效。  \end{itemize}     

一个自然的想法是,如果我们可以为每个机器人获得一些对任务有益的长马操作,我们将其称为子树,并将这些动作添加到相应的机器人的动作空间中。在计划期间,我们让这些子树优先考虑原子能的行动,从而加快了寻找解决方案并避免冗余计划。在执行过程中,我们仅共享这些子树的意图。如果子树是精心设计的,则此方法可以减少沟通开销,同时也提高并行执行的效率。  

   \subsubsection{子树预先计划  }     

首先假设我们获得了用于计划子树的动作序列   $A=(a_1,a_2,\dots, a_m)$   ,然后考虑如何使用LLMS为每个机器人生成与任务相关的动作序列。由于BT的模块化,我们可以将动作序列   $A$   视为长胜下动作,并且可以计算其动作模型:
    \setlength{\jot}{3pt}    
    \begin{align}
pre(A) &= \bigcup_{j=1}^m \left(pre(a_j) \setminus \bigcup_{k=1}^j add(a_j)\right) \\ 
add(A) &= \bigcup_{j=1}^m \left( add(a_j) \setminus del(a_j) \right) - pre(A)  \\ 
del(A) &= \bigcup_{j=1}^m \left( del(a_j) \setminus add(a_j) \right) 
\end{align}     

我们可以通过运行单重BT BT计划算法来依次获得   $A$   中的执行操作的树结构,并在要扩展的操作顺序上有限制,我们称之为SubTree预先计划的过程。  

但是,要使子树的行为像原子动作一样,即,在以中间状态运行时不要退出   $A$   的先决条件,我们需要引入一个附加的子树控制结构,如图   \ref{fig-subtree_bt}   所示。子树   $\mathcal{T}_A$   具有前提条件   \constant{Close(Door)}   和   \constant{Empty(Hand)}   ,但是在   \constant{Get(Key)}   之后,不再满足   \constant{Empty(Hand)}   的条件。在常规的BT计划算法中,这将导致随后的动作未被勾选,从而导致整个子树   $\mathcal{T}_A$   失败。为了解决此问题,我们介绍了三个子树控制节点:   \constant{EnterSubtree}   ,   \constant{ExitSubtree}   和   \constant{RunningSubtree}   。如果满足   $pre(A)$   ,并且机器人当前未运行此子树   $\mathcal{T}_A$   ,则将执行   \constant{EnterSubtree}   。此操作将将子树的状态更改为运行。    \constant{RunningSubtree}   将返回   \constant{S}   ,直到执行   \constant{ExitSubtree}   ,或者由于环境状态的变化,BT开始执行新操作。三个节点的参数可以是子树的任何标识符。一种简单的方法是将动作序列   $A$   中最后一个动作的添加效果   $add(a_m)$   用作标识符。  

   \begin{figure}[t]
	\centering
	\includegraphics[width=0.47\textwidth]{images/subtree.png}
	\caption{一个预先计划的子树结构的示例,用于打开门。  }
	\label{fig-subtree_bt}
\end{figure}     

   \begin{figure*}[ht]
    \centering
    \begin{subfigure}[b]{0.49\textwidth}
        \centering
        \includegraphics[width=\textwidth]{exp-robust-combined.pdf}
        \caption{同质性对鲁棒性的影响  }
        \label{fig:robust_results}
    \end{subfigure}
    \hfill
    \begin{subfigure}[b]{0.49\textwidth}
        \centering
        \includegraphics[width=\textwidth]{images/exp-parallel-combined.pdf}
        \caption{意图共享对执行效率的影响  }
        \label{fig:subtask_chaining_results}
    \end{subfigure}
       \caption{在4和8机器人的不同条件下的成功率和团队步骤的比较。每个数据点代表500个试验的平均值。  }
    \label{fig:combined_results}
\end{figure*}     

   \subsubsection{提示和反馈LLM  }   每个机器人的适当子树可以显着提高计划和执行的效率。但是,在计划之前获得这些子树是一项非常棘手的任务。幸运的是,已证明预训练的LLM具有任务推理功能   \cite{liu2023llm,song2023llmplanner}   。当可用的模型可用时,子树预先计划可能是MRBTP的高效插件。  

要从预训练的LLM中获得每个机器人的合适动作序列,该模型的提示应包括:(1)任务信息,包括初始环境状态和目标; (2)环境中的对象; (3)每个机器人的动作空间; (4)几次示威; (5)系统提示指导模型正确输出。  

LLM产生输出后,我们设计了一个检查器来自动验证它。我们在三种情况下向LLM提供反馈:(1)输出具有语法错误; (2)不能将动作序列预先计划为子树,这意味着它们不是连贯的; (3)每个机器人生成的动作序列数量不足。一旦输出足够好,或者已经达到了最大反馈尝试,我们就开始进行subtree预先计划。预先计划完成后,我们将每个子树添加到包含子树中所有操作的所有机器人的操作空间中,以充分利用子树。  

   \section{实验  }     

我们在两个模拟方案中评估了MRBTP的性能:(1)仓库管理具有粗糙的动作粒度和较小的动作空间,以及(2)具有更精细的粒度和较大动作空间的家庭服务。首先,我们通过引入每个动作的失败概率来评估MRBTP方法在不同水平的同质性水平下的鲁棒性。接下来,我们进行了一项消融研究,以验证其对多机器人BTS执行效率的贡献的意图。然后,鉴于在家庭服务方案中采取更精细的动作粒度,我们进行了一项消融研究,以评估子树预先计划,研究预训练的LLM在产生任务相关的动作序列及其对MRBTP的整体效率的影响方面的有效性。所有实验均在配备AMD Ryzen 9 5900X 12核处理器的系统上进行,其基本时钟和128 GB的DDR4 RAM。  

   \subsection{实验设置  }     

   \subsubsection{方案  }   (a)仓库管理。我们将Minigrid    \cite{chevalier-boisvert2023minigrid}   环境扩展为多机器人模拟的环境,其中4-8个机器人在包含随机放置包装的4个房间中。机器人具有不同的动作空间,包括房间检查和包装搬迁,其中一些具有专门功能或限制访问权限。目标是优化仓库空间利用率。 (b)家庭服务。在VirtualHome    \cite{puig2018virtualhome}   环境中,2-4个机器人与数十个对象进行交互并执行数百个潜在的动作。每个机器人的动作空间都是多样的,旨在完成复杂的家庭任务,例如设置桌子或准备饭菜。  

   \subsubsection{评估指标  }   使用以下指标评估了算法的性能:(a)成功率(SR):在多个试验中成功完成任务的百分比,这考虑了动作故障概率。 (b)团队步骤(TS):所有机器人并行完成任务所需的步骤总数。 (c)总机器人步骤(RS):独立于每个机器人采取的步骤之和。 (d)沟通开销(通信):由于意图共享而引起的机器人之间的广播通信数量。 (e)扩展条件的数量(EC):在多机器人BTS计划过程中扩大条件节点的数量,包括SubTree预先计划(如果有)的情况。 (f)计划时间(PT):用于多机器人BT计划所需的时间,包括在可用的情况下进行subtree预先计划。  

   \subsubsection{设置  }   (A)同质性(   $\boldsymbol{\alpha}$   ):分配给机器人的冗余动作的比例,其中   $\boldsymbol{\alpha} = 1$   表示完全异质性(在动作空间中无重叠)和   $\boldsymbol{\alpha} = 0$   表示完全同质性(相同的动作空间)。 (b)动作故障概率(FP):机器人无法执行操作的概率。 (c)子树的意图共享(子树IS)和原子行动节点意图共享(Atomic IS):这些条款是指在子树中或在个别原子行动节点的级别中的意图共享的应用。 (d)反馈(F)和无反馈(NF):此设置区分了在子树生成期间使用反馈的LLM与那些不使用反馈的LLM。在反馈条件下,LLM最多收到3个反馈迭代,而在无反馈条件下,则没有提供反馈。  

   \begin{table}[t]
\centering
\small
\setlength{\tabcolsep}{3pt}  
\label{tab:planning_time}
\begin{tabular}{c ccc c c c c}
\toprule
\multirow{2}{*}{ Method } & \multicolumn{3}{c}{         $\boldsymbol{\alpha = 1}$          } & &  \textbf{        $\boldsymbol{\alpha \approx 0.5}$         }  & &  \textbf{        $\boldsymbol{\alpha = 0}$         }   \\ 
\cline{2-4} \cline{6-6} \cline{8-8}  
 \\ [-1.5ex] 
&   SR( \% )  &  TS  &  RS  & &   SR( \% )  & &   SR( \% )   \\ 
\midrule
 BT-Expansion  & 100 & 8.8 & 33.8 & & 12.4 & & 4.6  \\ 
 MRBTP  & 100 & 5.8 & 15.3 & & 100 & & 100  \\ 
\bottomrule
\end{tabular}
\caption{与仓库管理中基线的性能比较(4个机器人,平均进行了500多次试验)。  }
\label{tab:comparison_with_baseline}
\end{table}     

   \begin{table*}[t]
    \centering
    \small
    \setlength{\tabcolsep}{2pt} 
    \begin{tabular}{cc |cc cccc| cc cccc | cc cccc}
        \toprule
        \multicolumn{2}{c}{ Homogeneity } & \multicolumn{6}{c}{         $\boldsymbol{\alpha = 1}$          } & \multicolumn{6}{c}{         $\boldsymbol{\alpha \approx 0.5 } $         } & \multicolumn{6}{c}{         $\boldsymbol{\alpha=0}$         }  \\  


       \midrule 
      \multicolumn{2}{c|}{ Subtree }   & - & - & \ding{51} &\ding{51} &\ding{51} &\ding{51} & - & - &  \ding{51} &\ding{51} &\ding{51} &\ding{51}  &- & -&  \ding{51} &\ding{51} &\ding{51} &\ding{51}   \\  

        
        \multicolumn{2}{c|}{ Subtree IS }          &- & - & -&-  & \ding{51} & \ding{51} &- & - & -&- & \ding{51} & \ding{51} &- & - &- &- & \ding{51} & \ding{51}  \\  
        \multicolumn{2}{c|}{ Atomic IS }           &  - & \ding{51} & -  & \ding{51} & - & \ding{51}  & - &  \ding{51} & - & \ding{51} & - & \ding{51} & - & \ding{51} &- &\ding{51}  & -& \ding{51}  \\  



        \midrule
        \multirow{2}{*}{ TS }   &   NF     & 161 & 159.4 & 114.4 & 109.6 & 78.9 & 79.8 & 139.7 & 137.5 & 126.2 & 119.6 & 86.9 & 102.5 & 73.7 & 68.5 & 96.1 & 94.8 & 75.6 & 78.0  \\  
        
         &   F  & - & - & 116.7 & 114.2 &  77.1  & 79.16 & - & - & 124.6 & 126.0 &  80.8  & 96.2 & - & - & 107.1 & 106.4 &  70.4  & 76.8  \\   
        \midrule

        
        \multirow{2}{*}{ RS }   &   NF       & 570.8 & 557.3 & 374 & 359.4 & 217.4 & 219.1 & 385.2 & 380.1 & 345.8 & 326.6 & 209.6 & 222 & 128.6 & 128.6 & 128.6 & 128.6 & 128.6 & 128.6  \\  
        &   F   & - & - & 377 & 370.9 &  205.2  & 208 & - & - & 380.7 & 348.2 &  192.2  & 209 & - & - & 128.6 & 128.6 & 128.6 & 128.6  \\  

        \midrule
        \multirow{2}{*}{ Comm. }  &   NF   & 0.0 & 63.8 & 0.0 & 7.1 & 6.7 & 14.1 & 0.0 & 43.5 & 0.0 & 8.0 & 7.1 & 20.9 & 0.0 & 15.2 & 0.0 & 2.8 & 6.5 & 9.3  \\  
         &  F  & - & - & 0.0 & 4.8 &  6.6  & 12.2 & - & - & 0.0 & 4.4 &  6.4  & 13.2 & - & - & 0.0 & 0.7 &  5.2  & 6.0  \\  
        
        
    \bottomrule
    \end{tabular}
    \caption{执行效率通过子树预先计划和意图共享。  }
    \label{tab:llm_sutree_result}
\end{table*}    
    \begin{table}[ht]
\centering
\small
\setlength{\tabcolsep}{5.5pt}  
\renewcommand{\arraystretch}{0.9} 
\begin{tabular}{ccccc}
\toprule
 Homogeneity  &  Subtree  &  Feedback  &  EC    &  PT (s)    \\  \midrule

\multirow{3}{*}{          $\boldsymbol{\alpha = 1}$           } & - & - & 8033.3  &  Timeout  \\ 
 & \ding{51} & - & 998.1  & 12.4  \\ 
  & \ding{51} & \ding{51}&  384.3   &  3.7   \\ 
\midrule

\multirow{3}{*}{         $\boldsymbol{\alpha \approx 0.5 } $          } & - & - & 7882.5 &  Timeout  \\ 
 & \ding{51} & - &  623.8 & 7.2  \\  
  & \ding{51} &\ding{51} &  267.9  &  2.6   \\  

\midrule
\multirow{3}{*}{         $\boldsymbol{\alpha=0}$         } &- & - & 2695.5 & 20.2  \\ 
& \ding{51} & - &  576.6  & 5.6 \\   
& \ding{51} & \ding{51} &   146.8   &  1.4  \\  

\bottomrule
\end{tabular}
\caption{通过预先计划的子树计划效率。每个LLM调用的平均响应时间为4.2秒。  }
\label{tab:planning_time}
\end{table}     

   \subsubsection{基线  }    BT计划算法通常利用行动模型进行计划。为了确保在相同的问题假设下的一致性,我们建议将BT-Expansion    \cite{cai2021bt}   算法直接调整,该算法已被证明是在单机器人设置中被证明且完整的,以作为我们的基线。在BT-Expansion中,每个机器人都独立执行了向团队目标的落后计划,而无需结合跨树的扩展或意图共享。  

   \subsection{实验结果  }     

   \subsubsection{性能比较  }    
在各种设置下,我们随机生成可解决的多机BT计划问题。表   \ref{tab:comparison_with_baseline}   显示了随着均匀性的降低,BT-expansion的成功率显着下降。相比之下,由于其跨树的扩展,MRBTP在所有设置中保持了100 \%的完美成功率。为了避免由计划失败引起的执行效率(TS,RS)的偏见,我们仅比较了两种算法成功的案例。值得注意的是,即使在完全同质性下,由于意图共享,MRBTP在执行效率方面的表现优于BT-BT扩张。  

   \subsubsection{鲁棒性  }   如图   \ref{fig:robust_results}   所示,我们算法的鲁棒性随着同质性的增加而改善,并通过大量机器人进一步增强。这种改进是由于其他机器人补偿动作故障的可能性增加而导致的。即使每个动作都有50 \%的故障概率,该系统仍保留了大约50 \%通过8个机器人和完整的动作空间同质性实现目标的机会。  

   \subsubsection{执行效率  }     

如图   \ref{fig:subtask_chaining_results}   所示,在完全异质的场景中,可以以更少的团队步骤共享意图共享结果,这表明我们的MRBTP算法在这些条件下固有地保持了卓越的执行效率。此外,随着同质性的增加,机器人执行冗余动作的可能性会增加,从而降低了并行任务执行的可能性。但是,随着意图共享,冗余行动被显着最小化,从而阻止了进一步的效率损失。在这种情况下,增加的同质性带来了更多的备份结构,从而进一步提高了执行效率。  

   \subsubsection{特定于任务的子树预先计划的有效性  }   我们构建了一个在三个均等级别的75个实例的数据集。用于生成子树的模型是GPT-4O-MINI-2024-07-18    \cite{openai2023gpt4}   。表   \ref{tab:llm_sutree_result}   提供了对介绍特定任务子树对计划效率,执行效率和通信开销的影响的比较分析。随着意图共享,随着均匀性的增加,效率提高。当子树与子任务链结合使用时,执行效率最高。在执行过程中,长匹马子树促进了远期计划,与较精细的原子动作相比,效率更高且频繁的沟通较低。表   \ref{tab:planning_time}   显示,子树预先计划可显着减少60秒限制下的BTS计划时间,通过通过子树重复使用和类似的机器人动作空间最大程度地减少冗余。此外,表   \ref{tab:llm_sutree_result}   和表   \ref{tab:planning_time}   都证明了反馈有效提高了计划和执行效率,尤其是与子树的共享和意图集成时。  

   \subsubsection{跨不同LLM的执行效率  }   我们测试了大型语言模型的不同版本,包括GPT-3.5-Turbo(2024.12)和GPT-4O-2024-08-08-06    \cite{openai2023gpt4}   ,以协助进行子树预先计划。如表   \ref{tab:different_llms}   所示,随着语言模型的推理能力的提高,执行效率的提高略有提高,而交流开销仍然很大程度上没有变化。这可以归因于以下事实:随着模型的推理能力的提高,子树预先计划变得更加适当和有效。此外,结果进一步表明,反馈机制提高了所有LLM的执行效率。  

   \begin{table}[t]
\centering
\small
\setlength{\tabcolsep}{3pt}  
\begin{tabular}{c ccc c ccc}
\toprule
\multirow{2}{*}{ Models } & \multicolumn{3}{c}{ No Feedback } &  & \multicolumn{3}{c}{ Feedback }  \\ 
\cline{2-4} \cline{6-8}
 \\ [-1.5ex] 
    &  TS  &  RS   &  Comm.  & &  TS  &  RS   &  Comm.    \\ 
\midrule
 GPT-3.5-turbo    & 81.6   & 223.6   & 5.1   & & 80.0   & 219.0   & 5.1    \\ 
 GPT-4o-mini     & 78.9   & 217.4   & 6.7   & & 77.1   & 205.2   & 6.6    \\ 
 GPT-4o     & 77.4   & 200.9   & 6.3   & & 74.9   & 190.7   & 6.3    \\  
\bottomrule
\end{tabular}
\caption{   $\boldsymbol{\alpha = 1}$   下不同LLMS的执行效率,以及子树和子树意图共享。  }
\label{tab:different_llms}
\end{table}     

   \section{相关工作  }     

   \paragraph{BT计划。  }   许多作品都集中在自动生成BTS来执行任务,例如Evolutionary Computing    \cite{neupane2019learning,colledanchise2019learning,lim2010evolving}   ,强化学习   \cite{banerjee2018autonomous,pereira2015framework}   ,模仿学习   \cite{french2019learning}   ,MCTS    \cite{scheide2021behavior}   和正式的合成   \cite{li2021reactive,tadewos2022specificationguided,neupane2023designing}   。最近,一些作品直接使用LLMS    \cite{lykov2023llmbrain,lykov2023llmmars}   直接生成BTS。但是,上述方法要么需要复杂的环境建模,要么不能保证BT的可靠性。相比之下,基于条形式建模   \cite{fikes1971strips}   的BT规划   \cite{cai2021bt,chen2024integrating}   不仅提供直观的环境建模,而且还确保了生成的BTS的可靠性和鲁棒性。  

使用各种方法研究了用于多机器人系统(MRS)的   \paragraph{Bt Mrs。  }    BT生成(MRS)。进化计算   \cite{neupane2019learning}   是一种通用启发式搜索方法,用于MRS中的BT生成。尽管多才多艺,但由于缺乏与动作模型的集成,这种方法通常会遭受缓慢的搜索效率。鉴于BT系统的模块化性质,动作模型并不难获得   \cite{arora2018review}   ,从而可以开发可以产生更有效的解决方案的方法。还探索了基于LLMS    \cite{lykov2023llmmars}   或其他机器学习技术   \cite{fu2016reinforcement}   的BT生成方法。这些方法需要大量的培训数据,从而使数据收集和模型培训资源密集型。此外,上述方法缺乏对生成BTS的完整性和正确性的保证。基于拍卖的方法   \cite{dahlquist2023reactive,heppner2024behavior,colledanchise2016advantages}   ,其中一些结合了行动模型计划,依赖于可靠的通信和低传输延迟的假设以确保有效的任务完成。但是,这种条件并不总是保证,使这些方法在交流不可靠的环境中降低了。相比之下,我们的方法在机器人团队开始执行之前生成BTS,即使在执行过程中没有通信的情况下,也确保任务完成。执行过程中的沟通仅用于提高协调效率,而不是作为必要的假设。  

   \paragraph{LLM用于任务推理。  }   最近,在使用LLMS进行任务推理   \cite{song2023llmplanner,liu2023llm,ahn2022can,chen2023robogpt}   (例如Prog Prompt    \cite{singh2022progprompt}   ,PlanBench    \cite{valmeekam2023planbench}   和Voyager    \cite{wang2023voyager}   )方面取得了重大进展。此外,LLM表明将任务分解为子目标   \cite{gao2024dagplan,singh2024twostep}   的能力与我们的子树预先计划密切相关,以用于多机手BT计划。随着LLM的任务推理能力继续发展和增强,我们的子树预先计划的技术有望变得越来越相关和有效。  

   \section{结论  }     

我们提出了MRBTP,这是第一种声音和完整的算法,用于解决多机器人BT计划问题。跨树扩展协调BTS实现目标,同时分享意图提高了执行效率和稳健性。 LLM插件进一步提高了计划速度并降低了开销。这些贡献代表了在可扩展的可靠多机器人系统中向前迈出的关键一步。未来的研究将完善该算法的性能,并将其应用扩展到更复杂,动态的环境,从而巩固MRBTP作为多机器人计划中的基础方法。此外,将探索该算法在实际机器人系统上的潜在部署,以评估其在现实世界中的有效性,可伸缩性和实用性。
    \section{致谢  }   这项工作得到了中国国家自然科学基金会(62106278,62032024)的支持。  

   \bigskip     

   \bibliography{aaai25}     


  
\end{document}

