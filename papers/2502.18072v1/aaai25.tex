%File: formatting-instructions-latex-2025.tex
%release 2025.0
\documentclass[letterpaper]{article} % DO NOT CHANGE THIS
\usepackage{aaai25}  % DO NOT CHANGE THIS
\usepackage{times}  % DO NOT CHANGE THIS
\usepackage{helvet}  % DO NOT CHANGE THIS
\usepackage{courier}  % DO NOT CHANGE THIS
\usepackage[hyphens]{url}  % DO NOT CHANGE THIS
\usepackage{graphicx} % DO NOT CHANGE THIS
\urlstyle{rm} % DO NOT CHANGE THIS
\def\UrlFont{\rm}  % DO NOT CHANGE THIS
\usepackage{natbib}  % DO NOT CHANGE THIS AND DO NOT ADD ANY OPTIONS TO IT
\usepackage{caption} % DO NOT CHANGE THIS AND DO NOT ADD ANY OPTIONS TO IT
\frenchspacing  % DO NOT CHANGE THIS
\setlength{\pdfpagewidth}{8.5in}  % DO NOT CHANGE THIS
\setlength{\pdfpageheight}{11in}  % DO NOT CHANGE THIS
%
% These are recommended to typeset algorithms but not required. See the subsubsection on algorithms. Remove them if you don't have algorithms in your paper.
\usepackage{algorithm}
% \usepackage{algorithmic}
\usepackage[noend]{algorithmic} % 只加载一次并加上 noend 选项
% cys
\usepackage{multirow}
\usepackage{graphicx}
\usepackage{subcaption}
\usepackage{pifont}
\newcommand{\xmark}{\ding{55}}
\usepackage{my}
\usepackage{booktabs}
\usepackage{amsmath}
\usepackage{makecell}
% \usepackage{lipsum}    % 用于生成占位文本
\allowdisplaybreaks  % 允许公式跨页


% appendix
\usepackage[breakable]{tcolorbox}
\usepackage{enumitem}
\usepackage{tikz}
% \usepackage{titlesec}
% % 保存原始 \section 格式
% \newcommand{\sectionbackup}{\titleformat*{\section}{\Large\bfseries\centering}}
 

% \usepackage{fontspec}
% \usepackage{tcolorbox} %
% \tcbuselibrary{breakable}
%
% These are are recommended to typeset listings but not required. See the subsubsection on listing. Remove this block if you don't have listings in your paper.
\usepackage{newfloat}
\usepackage{listings}
\DeclareCaptionStyle{ruled}{labelfont=normalfont,labelsep=colon,strut=off} % DO NOT CHANGE THIS
\lstset{%
	basicstyle={\footnotesize\ttfamily},% footnotesize acceptable for monospace
	numbers=left,numberstyle=\footnotesize,xleftmargin=2em,% show line numbers, remove this entire line if you don't want the numbers.
	aboveskip=0pt,belowskip=0pt,%
	showstringspaces=false,tabsize=2,breaklines=true}
\floatstyle{ruled}
\newfloat{listing}{tb}{lst}{}
\floatname{listing}{Listing}
%
% Keep the \pdfinfo as shown here. There's no need
% for you to add the /Title and /Author tags.
\pdfinfo{
/TemplateVersion (2025.1)
}


% DISALLOWED PACKAGES
% \usepackage{authblk} -- This package is specifically forbidden
% \usepackage{balance} -- This package is specifically forbidden
% \usepackage{color (if used in text)
% \usepackage{CJK} -- This package is specifically forbidden
% \usepackage{float} -- This package is specifically forbidden
% \usepackage{flushend} -- This package is specifically forbidden
% \usepackage{fontenc} -- This package is specifically forbidden
% \usepackage{fullpage} -- This package is specifically forbidden
% \usepackage{geometry} -- This package is specifically forbidden
% \usepackage{grffile} -- This package is specifically forbidden
% \usepackage{hyperref} -- This package is specifically forbidden
% \usepackage{navigator} -- This package is specifically forbidden
% (or any other package that embeds links such as navigator or hyperref)
% \indentfirst} -- This package is specifically forbidden
% \layout} -- This package is specifically forbidden
% \multicol} -- This package is specifically forbidden
% \nameref} -- This package is specifically forbidden
% \usepackage{savetrees} -- This package is specifically forbidden
% \usepackage{setspace} -- This package is specifically forbidden
% \usepackage{stfloats} -- This package is specifically forbidden
% \usepackage{tabu} -- This package is specifically forbidden
% \usepackage{titlesec} -- This package is specifically forbidden
% \usepackage{tocbibind} -- This package is specifically forbidden
% \usepackage{ulem} -- This package is specifically forbidden
% \usepackage{wrapfig} -- This package is specifically forbidden
% DISALLOWED COMMANDS
% \nocopyright -- Your paper will not be published if you use this command
% \addtolength -- This command may not be used
% \balance -- This command may not be used
% \baselinestretch -- Your paper will not be published if you use this command
% \clearpage -- No page breaks of any kind may be used for the final version of your paper
% \columnsep -- This command may not be used
% \newpage -- No page breaks of any kind may be used for the final version of your paper
% \pagebreak -- No page breaks of any kind may be used for the final version of your paperr
% \pagestyle -- This command may not be used
% \tiny -- This is not an acceptable font size.
% \vspace{- -- No negative value may be used in proximity of a caption, figure, table, section, subsection, subsubsection, or reference
% \vskip{- -- No negative value may be used to alter spacing above or below a caption, figure, table, section, subsection, subsubsection, or reference

\setcounter{secnumdepth}{0} %May be changed to 1 or 2 if section numbers are desired.

% The file aaai25.sty is the style file for AAAI Press
% proceedings, working notes, and technical reports.
%

% Title

% Your title must be in mixed case, not sentence case.
% That means all verbs (including short verbs like be, is, using,and go),
% nouns, adverbs, adjectives should be capitalized, including both words in hyphenated terms, while
% articles, conjunctions, and prepositions are lower case unless they
% directly follow a colon or long dash

%  自己编写通讯作者右上角的脚标
\makeatletter
\newcounter{corrfn}\setcounter{corrfn}{0}
\def\corrauthor{%
  \ifnum\value{corrfn}=0%
    \thanks{Corresponding Author}%
    \setcounter{corrfn}{\value{footnote}}%
  \else%
    \footnotemark[\value{corrfn}]%
  \fi%
}
\makeatother


\title{MRBTP: Efficient Multi-Robot Behavior Tree Planning and Collaboration}
\author{
    %Authors
    % All authors must be in the same font size and format.
    Yishuai Cai\equalcontrib,
    Xinglin Chen\equalcontrib,
    Zhongxuan Cai\corrauthor,
    Yunxin Mao,\\
    Minglong Li\corrauthor ,
    Wenjing Yang,
    Ji Wang
}

% Written by AAAI Press Staff\textsuperscript{\rm 1}\thanks{With help from the AAAI Publications Committee.}\\
% AAAI Style Contributions by Pater Patel Schneider,
% Sunil Issar,\\
% J. Scott Penberthy,
% George Ferguson,
% Hans Guesgen,
% Francisco Cruz\equalcontrib,
% Marc Pujol-Gonzalez\equalcontrib
    
\affiliations{
    %Afiliations
    \textsuperscript{\rm }College of Computer Science and Technology, National University of Defense Technology\\
    % If you have multiple authors and multiple affiliations
    % use superscripts in text and roman font to identify them.
    % For example,

    % Sunil Issar\textsuperscript{\rm 2}, 
    % J. Scott Penberthy\textsuperscript{\rm 3}, 
    % George Ferguson\textsuperscript{\rm 4},
    % Hans Guesgen\textsuperscript{\rm 5}
    % Note that the comma should be placed after the superscript

    % 1101 Pennsylvania Ave, NW Suite 300\\
    % Washington, DC 20004 USA\\
    % Changsha, Hunan 410000, China
    % email address must be in roman text type, not monospace or sans serif
    \{caiyishuai, chenxinglin, caizhongxuan, maoyunxin, liminglong10, wenjing.yang, wj\}@nudt.edu.cn
    % proceedings-questions@aaai.org
    
%
% See more examples next
}

%Example, Single Author, ->> remove \iffalse,\fi and place them surrounding AAAI title to use it
\iffalse
\title{My Publication Title --- Single Author}
\author {
    Author Name
}
\affiliations{
    Affiliation\\
    Affiliation Line 2\\
    name@example.com
}
\fi

\iffalse
%Example, Multiple Authors, ->> remove \iffalse,\fi and place them surrounding AAAI title to use it
\title{My Publication Title --- Multiple Authors}
\author {
    % Authors
    First Author Name\textsuperscript{\rm 1,\rm 2},
    Second Author Name\textsuperscript{\rm 2},
    Third Author Name\textsuperscript{\rm 1}
}
\affiliations {
    % Affiliations
    \textsuperscript{\rm 1}Affiliation 1\\
    \textsuperscript{\rm 2}Affiliation 2\\
    firstAuthor@affiliation1.com, secondAuthor@affilation2.com, thirdAuthor@affiliation1.com
}
\fi


% REMOVE THIS: bibentry
% This is only needed to show inline citations in the guidelines document. You should not need it and can safely delete it.
\usepackage{bibentry}
% END REMOVE bibentry

\begin{document}

\maketitle

\begin{abstract}

% 行为树因其卓越的可解释性、可靠性和反应性,在单机器人系统中广泛应用,并已成功实现行为树的自动规划。然而,在多机器人系统中,机器人的异构性和并行执行的复杂性使得多行为树规划和协同面临新的挑战。挑战主要有两个,一是如何规划鲁棒的多行为树,并让他们在运行时高效协同。二是如何提高规划和执行的效率,让算法在实际应用中成为可能。为了应对第一个挑战,我们对单机器人行为树规划算法进行扩展,引入树外扩展来让异构机器人可以互相配合,同时执行时通过分布式通信构造子目标队列,成功实现分布式多机器人行为树规划与协同。我们在理论上证明了该规划算法具有健全性和完备性。为了应对第二个挑战,我们利用行为树模块化的特点,在规划前预先规划出一些task-specific long-horizon的subtree,将这些subtree加入动作空间能够大大提高规划和协同执行的效率。我们使用大模型的场景理解和推理能力来生成合适的subtree partterns,最终实现了高效的系统架构。我们设计了多机器人行为树货物搬运场景,探究了我们规划算法的鲁棒性和并行性随同构性和动作失效率的变化。随后在家庭服务场景中测试了大模型生成subtree的能力,以及基于subtree的规划和协同效率能够提高多少。

Multi-robot task planning and collaboration are critical challenges in robotics. While Behavior Trees (BTs) have been established as a popular control architecture and are plannable for a single robot, the development of effective multi-robot BT planning algorithms remains challenging due to the complexity of coordinating diverse action spaces. We propose the Multi-Robot Behavior Tree Planning (MRBTP) algorithm, with theoretical guarantees of both soundness and completeness. MRBTP features cross-tree expansion to coordinate heterogeneous actions across different BTs to achieve the team's goal. For homogeneous actions, we retain backup structures among BTs to ensure robustness and prevent redundant execution through intention sharing. While MRBTP is capable of generating BTs for both homogeneous and heterogeneous robot teams, its efficiency can be further improved. We then propose an optional plugin for MRBTP when Large Language Models (LLMs) are available to reason goal-related actions for each robot. These relevant actions can be pre-planned to form long-horizon subtrees, significantly enhancing the planning speed and collaboration efficiency of MRBTP. We evaluate our algorithm in warehouse management and everyday service scenarios. Results demonstrate MRBTP's robustness and execution efficiency under varying settings, as well as the ability of the pre-trained LLM to generate effective task-specific subtrees for MRBTP.

\end{abstract}

% Uncomment the following to link to your code, datasets, an extended version or similar.
%
% \begin{links}
%     \link{Code}{https://aaai.org/example/code}
%     \link{Datasets}{https://aaai.org/example/datasets}
%     \link{Extended version}{https://aaai.org/example/extended-version}
% \end{links}
\begin{links}
    \link{Code}{https://github.com/DIDS-EI/MRBTP}
\end{links}

\section{Introduction}
Multi-robot systems (MRS) that involve robots with diverse capabilities offer the potential for improved performance and fault tolerance compared to single-robot solutions \cite{colledanchise2016advantages}. Developing an autonomous MRS requires an efficient and robust control architecture, along with methods to adapt them for specific tasks. Behavior Trees (BTs) have emerged as a popular control architecture due to their modularity, interpretability, reactivity, and robustness, making them well-suited for both single- and multi-robot systems \cite{heppner2024behavior,heppnerl2023distributed,neupane2019learning,colledanchise2016advantages}. As the potential of BTs gains more attention, various methods for automatically generating BTs have been proposed, including evolutionary computing \cite{neupane2019learning,colledanchise2019learning}, reinforcement learning \cite{banerjee2018autonomous,pereira2015framework}, and BT synthesis \cite{tadewos2022specificationguided,neupane2023designing}. Among these methods, BT planning \cite{chen2024integrating,chen2024efficient,cai2021bt,colledanchise2019blended} has advantages in leveraging interpretable action models and producing reliable BTs to achieve goals, which make it a promising approach to generate BTs for autonomous robot systems.

\begin{figure}[t]
	\centering
	\includegraphics[width=0.47\textwidth]{images/introduction.png}
	\caption{An example of two BTs planned by MRBTP: (1) Cross-tree expansion, (2) Intention sharing, (3) Optional plugin: subtree pre-planning.}
	\label{fig:introduction}
\end{figure}


However, current BT planning focuses primarily on a single robot, and the development of effective multi-robot BT planning algorithms remains challenging. The challenges mainly arise from two aspects:
\begin{itemize}
\item For heterogeneous actions, how to coordinate them across different BTs to accomplish the team's goal.
\item For homogeneous actions, how to use them to improve fault tolerance without redundant execution.
\end{itemize}

In this paper, we propose Multi-Robot Behavior Tree Planning (MRBTP), the first sound and complete algorithm for generating reliable and robust BTs for MRS. MRBTP addresses the above challenges as follows:
\begin{itemize}
\item We employ cross-tree expansion, where the condition expanded in one BT will be further expanded by all BTs. This means one robot may take an action to satisfy another's precondition, enabling multi-tree collaboration.
\item We allow backup structures to be expanded by robots with homogeneous actions to ensure fault tolerance, while using intention sharing to avoid redundant execution. During execution, each robot broadcasts its current action so that others can predict its effects and avoid performing actions with the same effects.
\end{itemize}

As shown in Figure \ref{fig:introduction}, in an everyday service scenario, the team's goal is to prepare a \constant{Salad}. The humanoid robot can perform the action \constant{Make(Fruit,Salad)}, if the precondition \constant{Has(Fruit)} is satisfied. Although it cannot do \constant{Unload(Fruit,Package)}, it can push \constant{Has} \constant{(Fruit)} to the planning queue, and another quadruped robot will expand this action through cross-tree expansion, enabling multi-tree coordination. 

In another case where \constant{In(Salad,Refrigerator)} is satisfied, and both robots can do \constant{Open(Refrigerator)} and \constant{Get(Salad)}. In this case, MRBTP will expand the same structure in both trees to ensure failure tolerance. If both robots are available, the humanoid robot with higher priority will do \constant{Open}\constant{(Refrigerator)} and share its intention. The quadruped robot will then assume \constant{IsOpen}\constant{(Refrigerator)} is true and walk to the \constant{Refrigerator}, waiting to do \constant{Get(Salad,} \constant{Refrigerator)} as long as the \constant{Refrigerator} is truly open. The intention sharing ensures parallelization and improves execution efficiency of the robot team.

Although MRBTP is a domain-independent algorithm, it is possible to enhance the planning and execution efficiency if Large Language Models (LLMs) are available for domain-dependent reasoning. Therefore, we further propose an optional plugin named subtree pre-planning. Assuming the LLM can reason some useful actions for each robot according to its capabilities, we can use these actions to plan useful subtree structures quickly before the long-horizon planning process. These subtrees can not only increase the planning speed, but can also decrease the communication expenses during execution. Experiments in warehouse management and everyday service scenarios demonstrate MRBTP's robustness and execution efficiency under varying settings, as well as the ability of pre-trained LLMs to generate effective task-specific subtrees for MRBTP.


% \section{Copyright}
% All papers submitted for publication by AAAI Press must be accompanied by a valid signed copyright form. They must also contain the AAAI copyright notice at the bottom of the first page of the paper. There are no exceptions to these requirements. If you fail to provide us with a signed copyright form or disable the copyright notice, we will be unable to publish your paper. There are \textbf{no exceptions} to this policy. You will find a PDF version of the AAAI copyright form in the AAAI AuthorKit. Please see the specific instructions for your conference for submission details.

\section{Background}
\paragraph{Behavior Tree.} 
A BT is a directed rooted tree where the execution nodes interact with the environment and the control flow nodes handle the triggering logic of their children\cite{colledanchise2018behavior}. 
At each time step, the BT initiates a tick that goes through control nodes, determining the action that the robot will execute according to the environmental state. This paper mainly focuses on four typical BT nodes:

\begin{itemize}
\item \textbf{Condition} \includegraphics[height=0.75em]{images/BTnodes/condition.png}: 
An execution node that checks whether the environment state satisfies the specified condition, returning either {\ttfamily success} or {\ttfamily failure} accordingly.

\item \textbf{Action} \includegraphics[height=0.75em]{images/BTnodes/action.png}: An execution node that controls the robot to perform an action, returning {\ttfamily success}, {\ttfamily failure}, {\ttfamily running} depending upon the outcome of execution.

\item \textbf{Sequence} \includegraphics[height=0.75em]{images/BTnodes/sequence.png}: A control flow node that only returns {\ttfamily success} if all its children succeed. Otherwise, it ticks its children from left to right, and the first child to return {\ttfamily failure} or {\ttfamily running} will determine its return status. 

\item \textbf{Fallback} \includegraphics[height=0.75em]{images/BTnodes/selector.png}:  A control flow node with logic opposite to the sequence node. It returns {\ttfamily failure} only if all of its children fail. If not, the first occurrence of {\ttfamily success} or {\ttfamily running} during ticking becomes its return status.
\end{itemize}


\paragraph{BT Planning.}
In BT planning for a single robot \cite{cai2021bt}, we represent a BT as a three-tuple $\mathcal{T} = <f, r, \Delta t>$. $f:2^{n}\rightarrow 2^{n}$ is its effect on the environment state, $\Delta t$ is the time step, and $r:2^{n}\mapsto \{ $\constant{S}, \constant{R}, \constant{F}\} partitions states into three regions, where $\mathcal{T}$ returns success, running, failure, respectively.

Then the BT planning problem can be described as: \(<\mathcal{S},\mathcal{L},\mathcal{A},\mathcal{M}, s_0,g>\), where \( \mathcal{S} \) is the finite set of environment states, $\mathcal{L}$ is the finite set of literals that form states, \( \mathcal{A} \) is the finite set of actions, $\mathcal{M}$ is the action model, $s_0$  is the initial state, $g$ is the goal condition. 


A condition $c$ in BT is usually a subset of a state $s$. If $c\subseteq s$, it is said condition $c$ holds in that state $s$. The state transition affected by action $a\in \mathcal{A}$ can be defined as a triplet \( \mathcal{M}(a)=<pre(a),add(a),del(a)> \), comprising the precondition, add effects, and delete effects of the action. If $a$ is finished after $k$ time step, the subsequent state $s_{t'}$ will be:
\begin{equation}\label{eqn:s_f}
	s_{t'}=f_a(s_t)=s_t\cup add(a)\setminus del(a), t'=t + k
\end{equation}


\section{Problem Formulation}
We first extend the BT representation from a single robot to a multi-robot system. 



\begin{definition}[Multi-BT System]
A $n$-robot BT system is a four-tuple $\left<\Phi, f_\Phi, r_\Phi, \Delta t_\Phi\right>$, where $\Phi = \left\{ \mathcal{T}_i \right\}_{i=1}^n$ is the set of BTs, $f_\Phi: \mathcal{S} \mapsto \mathcal{S}$ is the team state transition function, $\Delta t_\Phi$ is the team time step, $r_\Phi: \mathcal{S} \mapsto \{$ \constant{S}, \constant{R}, \constant{F} $\}$ is the team region partition.
\end{definition}
Due to variability in hardware performance, we allow each robot's BT to have a different response frequency, with $\Delta t_\Phi$ representing the common minimum response interval. The state transition can be calculated as follows:
\begin{align}
    s_{t+\Delta t_\Phi} = f_\Phi(s_t) = s_t \cup \bigcup_{i=1}^n \left( add(a_i) \setminus del(a_i) \right)
\end{align}
where $a_i$ is the action of robot $i$ in time $t$. If robot $i$ does not have an action or its action is running, we let $add(a_i) = del(a_i) = \emptyset$. 

The team region partition can be calculated as follows:
\begin{equation}
    r_\Phi(s) = 
\begin{cases}
    \text{\constant{R}} & \text{if } \exists i, r_i(s) = \text{\constant{R}} \\
    \text{\constant{S}} & \text{if } \forall i, r_i(s) \neq \text{\constant{R}} \text{ and } \exists i, r_i(s) = \text{\constant{S}} \\
    \text{\constant{F}} & \text{if } \forall i, r_i(s) = \text{\constant{F}}
\end{cases}
\end{equation}
The status of $\Phi$ is \constant{R} if any BT is still running, \constant{S} if some BT returns success and no one is running, and \constant{F} if all BT fails.
% Finite time successful (FTS) in region of attraction (ROA).
\begin{definition}[Finite Time Successful]
$\Phi$ is finite time successful (FTS) from region of attraction (ROA) $R$ to condition $c$, if $\forall s_0 \in R$  there is 
 a finite time $\tau$ such that for any $t<\tau$, $r_\Phi(s_t)=$ \constant{R}, and for any $t\geq\tau, r_\Phi(s_t)=$ \constant{S}, $c\in s_t$.
\end{definition}
With definitions above, the multi-robot BT planning problem can finally be defined.

\begin{problem}[Multi-Robot BT Planning]
The problem is a tuple \(\left<\mathcal{S},\mathcal{L},\{\mathcal{A}_i\}_{i=1}^n,\mathcal{M}, s_0,g\right>\), where \( \mathcal{S} \) is the finite set of environment states, $\mathcal{L}$ is the finite set literals that form states and conditions, $\mathcal{A}_i$ is the finite action set of robot $i$, $\mathcal{M}$ is the action model, $s_0$  is the initial state, $g$ is the goal condition. A solution to this problem is a BT set $\Phi = \{\mathcal{T}_i\}_{i=1}^n$ built with $\{\mathcal{A}_i\}_{i=1}^n$, such that $\Phi$ is FTS from $R\ni s_0$ to $g$.

% , and for all $s$ in the team success region $S=\{s|r_\Phi(s)=$\constant{S}$\}$, $g\subseteq s$).
\end{problem}


\begin{algorithm}[t]
\caption{One-step cross-tree expansion}
\label{alg:one_step}
\begin{algorithmic}[1]
% \Function{Expand}{$\mathcal{T}, c$}
\State \textbf{function} \constant{ExpandOneRobot(}$\mathcal{T},\mathcal{A}, c$\constant{)}
\begin{ALC@g}
\StateComment{$\mathcal{T}_{new} \leftarrow c$}{newly expanded subtree}
\StateComment{$\mathcal{C}_{new} \gets \emptyset$}{newly expanded conditions}
\FOR{\textbf{each} action $a \in \mathcal{A}$}
    \IF{$c \cap (pre(a) \cup add(a) \setminus del(a)) \neq \emptyset$ and $c \setminus del(a) = c$}
    \STATE $c_a \leftarrow pre(a) \cup c \setminus add(a)$
    \STATE $\mathcal{T}_a \leftarrow Sequence(c_a, a)$
    \STATE $\mathcal{T}_{new} \leftarrow Fallback(\mathcal{T}_{new}, \mathcal{T}_a)$
    \State{$\mathcal{C}_{new} \gets \mathcal{C}_{new} \cup \{c_a\}$}
    \ENDIF
\ENDFOR
\IF{$\mathcal{C}_{new}\neq \emptyset$}
    \IF{\constant{ConditionInTree}($c$, $\mathcal{T}$)}
        \StateComment{Replace $c$ with $\mathcal{T}_{new}$ in $\mathcal{T}$}{in-tree expand} \label{InsideExpand2}
    \ELSIF{$\mathcal{T}_{new} \neq c$} 
        \StateComment{$\mathcal{T} \gets Fallback(\mathcal{T},\mathcal{T}_{new}) $}{cross-tree expand}
    \ENDIF
\ENDIF
\end{ALC@g}
\RETURN $\mathcal{T}, \mathcal{C}_{new}$
% \EndFunction
\end{algorithmic}
\end{algorithm}


\section{Methods}

We first detail MRBTP, analyzing its soundness, completeness, and computational complexity. Then, we demonstrate how intention sharing functions among BTs during execution. Finally, we introduce the optional plugin, subtree pre-planning, to further enhance efficiency.

% \subsection{异构多智能体行为树规划算法}
\subsection{Multi-Robot Behavior Tree Planning}
% Like BT planning for single robots, we initialize each robot's BT with a goal condition. 


% \subsubsection{算法流程}
\subsubsection{One-Step Cross-Tree Expansion}
Algorithm \ref{alg:one_step} gives the pseudocode of one-step cross-tree expansion for one robot. Given its current BT $\mathcal{T}$, action space $\mathcal{A}$ and the condition to expand $c$ (line 1), the function returns an expanded BT $\mathcal{T}$ along with the newly expanded condition set $\mathcal{C}_{new}$ (line 15). Similar to one-step expansion for a single robot \cite{cai2021bt}, the expansion begins with $\mathcal{T}_{new}=c$ and $\mathcal{C}_{new}=\emptyset$ (line 2-3). Then we go through the action space $\mathcal{A}$ to find all premise actions that can lead to $c$ (line 4-5). For each premise action $a$, we calculate its corresponding precondition $c_a$ (line 6), form a sequence structure $\mathcal{T}_a$ (line 7), and add $\mathcal{T}_a$ to the tail of the root fallback node of $\mathcal{T}_{new}$ (line 8). Now $\mathcal{T}_{new}$ can achieve $c$ using these expanded actions if their precondition are met. We store these preconditions in $\mathcal{C}_{new}$ (line 9). 

If $\mathcal{T}_{new}$ is expanded (line 10), we need to decide where in $\mathcal{T}$ to place it. There are two cases: (1) $c$ is in $\mathcal{T}$, which means it was previously expanded by $\mathcal{T}$ itself. So we replace $c$ with $\mathcal{T}_{new}$ in $\mathcal{T}$ just like in single-robot BT expansion (line 11-12); (2) $c$ is not in $\mathcal{T}$, which means it was expanded by other BTs. To allow this BT to take actions to fulfill $c$, we add it to the tail of the root fallback node of $\mathcal{T}$ (line 13-14). 
\begin{proposition}\label{pro:inside}
    Given $\mathcal{T}$ is FTS from $R$ to $g$, if $\mathcal{T}$ is expanded by Algorithm \ref{alg:one_step} to $\mathcal{T}'$ given $c$, $c$ is in $\mathcal{T}$ and $\mathcal{C}_{new} \neq \emptyset$, then $\mathcal{T}'$ is FTS from $R'=R\cup \{ s\in \mathcal{S}| c_a \subseteq s, c_a\in \mathcal{C}_{new} \}$ to $g$.
\end{proposition}

\begin{proposition}\label{pro:outside}
    If $\mathcal{T}$ is expanded by Algorithm \ref{alg:one_step} to $\mathcal{T}'$ given $c$, $c$ is not in $\mathcal{T}$ and $\mathcal{C}_{new} \neq \emptyset$, then $\mathcal{T}'$ is FTS from $\mathcal{S}_{new}=\{ s\in \mathcal{S}| c_a \subseteq s, c_a\in \mathcal{C}_{new} \}$ to $c$.
\end{proposition}
The above two propositions state the changes in the ROA after one-step cross-tree expansion. If $c$ is in $\mathcal{T}$ (Proposition \ref{pro:inside}), the ROA of $\mathcal{T}$ will be expanded by $\mathcal{C}_{new}$ to achieve $g$. If $c$ is not in $\mathcal{T}$ (Proposition \ref{pro:outside}), then $c$ will be treated as a new sub-goal for $\mathcal{T}$ to be achieved from $\mathcal{S}_{new}$.\footnote{All the formal proofs are in the Appendix.}




\begin{algorithm}[t]
\caption{MRBTP}
\label{alg:MABTP}
\textbf{Input}: problem $\tuple{\mathcal{S},\mathcal{L},\multi{\mathcal{A}},\mathcal{M},s_0,c}$ \\
\textbf{Output}: solution $\Phi = \multi{\mathcal{T}}$ 
\begin{algorithmic}[1]

\STATE $\mathcal{C}_U\gets \{g\}$ \label{line:cu}
		\hfill $\triangleright$ conditions to be explored
\StateCommentLabel{$\mathcal{C}_E\gets \emptyset$}{expanded conditions}{line:ce}
\FOR{$i = 1$ \TO $n$}
    \STATE $\mathcal{T}_i \gets Fallback(g)$ \label{line:initBT} \Comment{init the BTs}
\ENDFOR
\WHILE{ $\mathcal{C}_U \neq \emptyset$}
\STATE $c\gets $ \constant{Pop(}$\mathcal{C}_U$\constant{)} \label{line:pickC}
\hfill $\triangleright$ explore $c$

% \IF{\algofunc{HasSupperSet}{$c,\mathcal{C}_E$}}
%     \StateComment{\textbf{continue}}{prune if any suppe already expanded}
% \ENDIF
\StateComment{\textbf{if} \algofunc{HasSubSet}{$c,\mathcal{C}_E$} \textbf{then} \textbf{continue}}{prune}  

\FOR{$i = 1$ \TO $n$}
    \STATE $\mathcal{T}_i, \mathcal{C}_{new} \gets$ \algofunc{ExpandOneRobot}{$\mathcal{T}_i,\mathcal{A}_i,c$} \label{line:initBT2}
    \IF{\algofunc{HasSubSet}{$s_0,\mathcal{C}_{new}$}}
        \RETURN $\Phi=\{\mathcal{T}_i\}_{i=1}^n$ \Comment{return a solution}
    \ENDIF
    \State{$\mathcal{C}_E \gets \mathcal{C}_E \cup \mathcal{C}_{new}$}
    \StateCommentLabel{$\mathcal{C}_U \gets \mathcal{C}_U \cup \mathcal{C}_{new}$}{add new conditions}{}
\ENDFOR


\ENDWHILE
  
\RETURN \constant{Unsolvable}
    
\end{algorithmic}
\end{algorithm}


\subsubsection{MRBTP}
Algorithm \ref{alg:MABTP} gives the pseudocode of MRBTP to plan BTs for the whole robot team. The algorithm initializes a set of conditions to be explored $\mathcal{C}_U=\{g\}$ and a set of expanded conditions $\mathcal{C}_E=\emptyset$ (line 1-2). The BT for each robot $i$ is initialized as $\mathcal{T}_i=Fallback(g)$ (line 4), which is FTS from $\emptyset$ to $g$. Then the algorithm continually explores conditions in $\mathcal{C}_U$ (line 5-6) until a solution is found, otherwise it returns \constant{Unsolvable} (line 14). For each explored $c$, it is either pruned if $\exists c' \in \mathcal{C}_E, c'\subseteq c$ (line 7), or expanded by all robots through one-step cross-tree expansion (line 8-9). After the one-step expansion for each robot, the newly expanded conditions $\mathcal{C}_{new}$ will be appended to $\mathcal{C}_E$ and $\mathcal{C}_U$ (line 12-13). If at that time $\exists c'\in \mathcal{C}_{new}, c'\subseteq s_0$, which means a solution is found, the algorithm returns $\Phi=\{\mathcal{T}_i\}_{i=1}^n$ as the solution (line 10-11).
\begin{proposition}\label{pro-k-loop2}: After the $k$-th ($k\geq1$) iteration of the while loop in Algorithm \ref{alg:MABTP}, where the explored condition is $c^{k}$, $\Phi^{k}=\multi{\mathcal{T}^{k}}$ is FTS from ROA $R^k=R^{k-1} \cup \bigcup_{i=1}^n \mathcal{S}^k_i$ to goal $g$, where $\mathcal{S}^{k}_i=\{s\in \mathcal{S}|c_a\subseteq s,c_a\in C^k_{i,new}\}$.
\end{proposition}
Note that Proposition \ref{pro-k-loop2} cannot be naturally extended from the single BT planning as it might seem. This proposition requires the assumption that robots execute in an appropriate order (at any time step, only the robot with the highest priority can execute an action if its precondition is satisfied); otherwise, deadlocks or departures from the ROA could occur. Fortunately, we can always use mechanisms such as deadlock detection during execution, ensuring that this serial execution is only employed in exceptional cases. In the vast majority of cases, robots can safely execute in parallel, so there is no need to worry that this assumption will reduce the execution efficiency of the robot team.

\begin{proposition}\label{pro-sound}
Algorithm \ref{alg:MABTP} is sound, i.e. if it returns a result $\Phi$ rather than \constant{Unsolvable}, then $\Phi$ is a solution of Problem 1.
\end{proposition}


\begin{figure*}[t]
	\centering
	\includegraphics[width=1\textwidth]{images/framework.png}
	\caption{The framework of our paper. (1) MRBTP. A sound and complete algorithm for the multi-robot BT planning problem, capable of coordinating diverse actions across different BTs through cross-tree expansion. (2) Intention Sharing. Robots share intentions with each other during execution, enabling multi-BT parallelization without compromising failure tolerance. (3) Optional Plugin: Subtree Pre-planning. This plugin utilizes LLMs to pre-plan task-specific subtrees, establishing long-horizon action sequences to enhance MRBTP's planning and execution efficiency.}

%  (1) Multi-Robot Behavior Tree Planning.  We first target the collective goal and plan individual BTs for multiple robots with both homogeneous and heterogeneous action spaces through in-tree expansion and cross-tree expansion strategies.
% (2) Intention Sharing for Multi-BT Execution: During execution, robots share their intentions via an intention queue, enabling parallel execution and preventing redundant tasks.
% (3) Optional Plugin: his plugin leverages LLMs to pre-plan relevant subtrees, establishing long-horizon action sequences before the main planning process begins.

 
	\label{fig:failure_example}
\end{figure*}


\begin{proposition}\label{pro-complete}
Algorithm \ref{alg:MABTP} is complete, i.e., if Problem 1 is solvable, the algorithm returns a $\Phi$ which is a solution.
\end{proposition}

Proposition \ref{pro-sound} can be proven by strong induction based on Proposition \ref{pro-k-loop}, and Proposition \ref{pro-complete} can be proven based on Proposition \ref{pro-sound}. These two propositions state the soundness and completeness of MRBTP, which makes it an effective algorithm to solve the multi-robot BT planning problem.


The time complexity of MRBTP in the worst case is $O(|\bigcup_{i=1}^n \mathcal{A}_i||\mathcal{S}||\mathcal{L}|)$, which is polynomial to the system size. In this case, the algorithm has to explore all states $s \in \mathcal{S}$ to find a solution. And in each exploration, the actions of all robots will be checked, with the checking complexity of $O(|\mathcal{L}|)$.
% the  where the size of the environment states $|\mathcal{S}|$ can reach $O(|\mathcal{A}|!)$ in the worst case. Massive computational consumption leads to a strong need for more efficient BT planning algorithms to make it practical for real-world applications.




% \subsubsection{基于子任务链的多行为树协同执行框架}
\subsection{Intention Sharing for Multi-BT Execution}

From the MRBTP planning process, we can observe that if multiple robots have identical actions (or similar actions with the same effect), MRBTP will expand them simultaneously in different BTs. This could lead to backup structures. These structures are beneficial for fault tolerance because if one robot fails, others can take over and complete the action. However, when multiple robots are available, backup structures can result in redundant execution. To avoid this, we introduce the multi-BT intention sharing method based on communication.

% However, when the robots' action spaces overlap, meaning that the same state has multiple robots capable of executing actions, but the sub-tasks they want to complete are the same, we can say these two robots have similar capabilities. Capability can be seen as the set of all sub-tasks a robot can complete. For example:

% In other words, higher heterogeneity means stronger parallelism, while higher homogeneity means stronger robustness (because if one robot fails, another can still complete the task).

% So, in the presence of homogeneity, is there a method that ensures both robustness and parallelism? To address this issue, this section proposes a multi-behavior tree coordination framework based on sub-task chains, building on the multi-robot behavior tree planning discussed above.

\paragraph{Intention Queue}
During execution, each robot $i$ maintains an intention queue $\mathcal{I}_i = (a_1, a_2, \dots, a_m)$ that indicates the actions being performed by other robots. In a situation with good communication, all robots' intention queues should remain consistent. Therefore, in the following text, we use $\mathcal{I}$ to refer to the intention queues of all robots. Based on the intention queue, we can calculate the belief success space $\mathcal{B}^S_i$ and the belief failure space $\mathcal{B}^F_i$ for robot $i$:
\setlength{\jot}{5pt}  % 设置公式之间的垂直间距
\begin{gather}
\mathcal{B}^S_{i} = \bigcup_{k=1}^{j-1} \left( add(a_k) \setminus del(a_k) \right) \\
\mathcal{B}^F_i = \bigcup_{k=1}^{j-1} \left( del(a_k) \setminus add(a_k) \right) 
\end{gather}
where $j$ is the index of its own action $a_j$ in the intention queue $\mathcal{I}$. If $j=1$, then $\mathcal{B}^S_i = \mathcal{B}^F_i = \emptyset$, which means the action is not dependent on any other's intention. If the robot currently has no action, it will be treated as $j = m+1$ when calculating belief spaces.

$\mathcal{B}^S$ and $\mathcal{B}^F$ will be used during the ticks of each BT. For each atomic condition node represented by a single literal $c = {l}$, it will first check if $l$ is in the belief spaces when ticked. If $l \in \mathcal{B}^S$, it returns \constant{S} without interacting with the environment, and returns \constant{F} when $l \in \mathcal{B}^F$. 

Whenever a robot $i$ exits an action or enters a new one, it will be broadcast to every other robot. Each robot then removes the old action of robot $i$ from the intention queue $\mathcal{I}$ (if it exists) and pushes the new action into it (if applicable). After this, each robot will update its belief spaces $\mathcal{B}^S$ and $\mathcal{B}^F$ to adjust its actions reactively. Note that an action exiting or entering may be due to two cases: (1) the environment state has changed, or (2) the belief spaces have changed. As a result, any addition or removal of actions in the intention queue $\mathcal{I}$ may lead to adjustments in other actions, creating a chain reaction. In other words, our intention-sharing method maintains the reactivity and robustness of BTs in response to uncertain environments.
% treat each action performed by the robot as a sub-task, and whenever the robot performs an action, that sub-task is assigned to other robots. When the next robot ticks in the behavior tree, it assumes that the sub-task has been completed, and the assumed environmental conditions can be calculated as follows:

\subsubsection{Parallelism and Blocking}

While intention sharing can avoid redundant execution, it also enhances action parallelism within the robot team. For example, as shown in Figure \ref{fig:failure_example}, in a warehouse management scenario, there are two robots capable of opening doors and transporting packages. They have expanded almost identical tree structures, sequentially executing \constant{Open}\constant{(Door)}, \constant{Walk}\constant{(Package)}, and \constant{Move}\constant{(Package)}. However, since \constant{IsClose(Door)}$\in s_0$, both robots satisfy the precondition to execute \constant{Open} \constant{(Door)}. Without intention sharing, they would perform this action simultaneously, causing redundancy. With intention sharing, however, if robot $1$ ticks its BT $\mathcal{T}_1$ first, it will execute \constant{Open(Door)} and send this intention to robot $2$. For robot $2$, \constant{IsOpen(Door)} $\in \mathcal{B}^S_2$ after updating the intention queue $\mathcal{I}$, so the corresponding condition node for \constant{IsOpen(Door)} will return \constant{S}, allowing the BT $\mathcal{T}_2$ to continue ticking and start executing \constant{Walk(Package)}. This transforms a serial BT structure into parallel execution.

However, when robot $2$ attempts to execute \constant{Move} \constant{(Package)}, which relies on the precondition \constant{IsOpen} \constant{(Door)}, robot $2$ will wait until the door is actually opened by robot $1$. Formally speaking, if $l\in \mathcal{B}^S_i$ but not in the current state $l\notin s$, when robot $i$ attempts to perform an action $a$ where $l\in pre(a)$, $a$ will be blocked. In this case, robot $i$ shares the intention of $a$, and $a$ returns \constant{R} as if it were executing, but it is actually doing nothing. The blocking mechanism prevents actions from being executed under incorrect preconditions, while also enabling the parallel execution of subsequent actions, thereby further enhancing the execution efficiency of the robot team.






\subsection{Optional Plugin: Subtree Pre-planning}

While MRBTP with intention sharing is proven to be an effective and efficient algorithm for the multi-robot BT planning problem, there is still room for further improving its efficiency. To achieve this, we begin by considering the following observations.
% It’s possible to plan and coordinate heterogeneous multi-robot behavior trees. However, in practice, we’ve found that designing the action space for multiple robots is a challenge. To ensure diversity in the actions of a multi-robot system and adapt to different planning tasks, we usually break down the action space into finer granularity. But this brings some issues:
\begin{itemize}
\item During planning, the same tree structure might be generated multiple times, especially when multiple robots have overlapping action spaces.
\item During execution, sharing the intentions of every short-horizon atomic action not only increases the communication overhead but is also ineffective for long-term task scheduling.
\end{itemize}


A natural idea is that if we can obtain some long-horizon actions for each robot that are beneficial to the task, which we call subtrees, and add these actions to the corresponding robot's action space. During planning, we let these subtrees be prioritized over atomic actions, thereby speeding up the search for solutions and avoiding redundant planning. During execution, we only share the intentions of these subtrees. If the subtrees are well-designed, this approach can reduce communication overhead while also improving the efficiency of parallel execution.

\subsubsection{Subtree Pre-planning}

Let's first assume that we have obtained an action sequence $A=(a_1,a_2,\dots, a_m)$ for planning the subtree, and then consider how to use LLMs to generate task-related action sequences for each robot. Due to the modularity of the BT, we can treat the action sequence $A$ as a long-horizon action, and its action model can be calculated:
\setlength{\jot}{3pt}  % 设置公式之间的垂直间距
\begin{align}
pre(A) &= \bigcup_{j=1}^m \left(pre(a_j) \setminus \bigcup_{k=1}^j add(a_j)\right)\\
add(A) &= \bigcup_{j=1}^m \left( add(a_j) \setminus del(a_j) \right) - pre(A) \\
del(A) &= \bigcup_{j=1}^m \left( del(a_j) \setminus add(a_j) \right) 
\end{align}

We can obtain the tree structure of execution actions in $A$ sequentially by running a single-robot BT planning algorithm, with constraints on the order of actions to be expanded, a process we call subtree pre-planning.

However, to make a subtree behave like an atomic action, i.e., not to exit the precondition of $A$ while running in intermediate states, we need to introduce an additional subtree control structure, as illustrated in Figure \ref{fig-subtree_bt}. The subtree $\mathcal{T}_A$ has the preconditions \constant{Close(Door)} and \constant{Empty(Hand)}, but after \constant{Get(Key)}, the \constant{Empty(Hand)} condition is no longer satisfied. In the conventional BT planning algorithm, this would result in the subsequent actions not being ticked, causing the entire subtree $\mathcal{T}_A$ to fail. To address this issue, we introduce three subtree control nodes: \constant{EnterSubtree}, \constant{ExitSubtree}, and \constant{RunningSubtree}. If $pre(A)$ is satisfied and the robot is not currently running this subtree $\mathcal{T}_A$, then \constant{EnterSubtree} will be executed. This action will change the status of the subtree to running. The \constant{RunningSubtree} will return \constant{S} until \constant{ExitSubtree} is executed, or the BT begins executing a new action due to a change in the environment state. The parameter for the three nodes can be any identifier of the subtree. A simple way is to use the add effect $add(a_m)$ of the last action in the action sequence $A$ as the identifier.


% In theory, due to the modular and hierarchical nature of behavior trees, any structure of a subtree can be seen as a behavior tree. However, in our planning problem, to automatically compute the action model of a subtree, we define the Task-specific Subtree as a subtree with a specific structure containing a sequence of actions meaningful for completing a task. A typical pre-planned subtree is shown in the figure.




\begin{figure}[t]
	\centering
	\includegraphics[width=0.47\textwidth]{images/subtree.png}
	\caption{ An example of a pre-planned subtree structure for open a door.}
	\label{fig-subtree_bt}
\end{figure}

\begin{figure*}[ht]
    \centering
    \begin{subfigure}[b]{0.49\textwidth}
        \centering
        \includegraphics[width=\textwidth]{exp-robust-combined.pdf}
        \caption{The impact of homogeneity on robustness}
        % \caption{Impact of Heterogeneity on Robustness}
        % \caption{ \textbf{Heterogeneity(H)} is the proportion of actions not redundantly assigned to robots.}
        % \caption{Task success rates for 4 and 8 robots under varying heterogeneity levels and action failure probabilities (FP). Each data point is the average success rate over 500 trials. \textbf{Heterogeneity(H)} is the proportion of actions not redundantly assigned to robots, with 1 indicating complete heterogeneity and 0 complete homogeneity. FP denotes the probability of an action failing for each robot.}
        \label{fig:robust_results}
    \end{subfigure}
    \hfill
    \begin{subfigure}[b]{0.49\textwidth}
        \centering
        \includegraphics[width=\textwidth]{images/exp-parallel-combined.pdf}
        \caption{The impact of intention sharing on execution efficiency}
        % \caption{Total steps taken by 4 and 8 robots under varying heterogeneity levels, with subtask chaining enabled and disabled. }
        % \caption{Total steps taken by 4 and 8 robots under varying heterogeneity levels, with subtask chaining enabled and disabled. Each data point represents the average total steps over 500 trials.}
        \label{fig:subtask_chaining_results}
    \end{subfigure}
       \caption{Comparison of success rate and team steps under different conditions for 4 and 8 robots. Each data point represents the average of 500 trials.}
    \label{fig:combined_results}
\end{figure*}





\subsubsection{Prompt and Feedback for LLMs}
Appropriate subtrees for each robot can significantly improve the efficiency of planning and execution. However, obtaining these subtrees before planning is a very tricky task. Fortunately, pre-trained LLMs have been proven to possess task reasoning capabilities \cite{liu2023llm,song2023llmplanner}. When such models are available, subtree pre-planning can be a highly effective plugin for MRBTP.

To obtain suitable action sequences for each robot from pre-trained LLMs, the model's prompt should include: (1) task information, including the initial environment state and goal; (2) objects in the environment; (3) the action space for each robot; (4) few-shot demonstrations; and (5) a system prompt to guide the model to output correctly. 

After the LLM produces an output, we designed a checker to automatically verify it. We provide feedback to the LLM in three cases: (1) the output has grammar errors; (2) the action sequences cannot be pre-planned into a subtree, meaning they are not coherent; (3) the number of action sequences generated for each robot is insufficient. Once the output is good enough or the maximum number of feedback attempts has been reached, we begin subtree pre-planning. After pre-planning is completed, we add each subtree to the action spaces of all robots that contain all of the actions in the subtree, to fully utilize the subtrees.





\section{Experiments}



We evaluate the performance of MRBTP in two simulated scenarios: (1) Warehouse management with coarse action granularity and a smaller action space, and (2) Home service with finer granularity and a larger action space. First, we assess the robustness of the MRBTP method under varying levels of homogeneity by introducing a failure probability for each action. Next, we conduct an ablation study on intention sharing to verify its contribution to the execution efficiency of multi-robot BTs. Then, given the finer action granularity in the home service scenario, we perform an ablation study to evaluate subtree pre-planning, examining the effectiveness of pre-trained LLMs in generating task-related action sequences and their impact on the overall efficiency of the MRBTP. All experiments were conducted on a system equipped with an AMD Ryzen 9 5900X 12-core processor with a 3.70 GHz base clock and 128 GB of DDR4 RAM.


\subsection{Experimental Setup}


\subsubsection{Scenarios} (a) Warehouse Management. We extend the Minigrid \cite{chevalier-boisvert2023minigrid} environment for multi-robot simulations with 4-8 robots in 4 rooms containing randomly placed packages. Robots have diverse action spaces, including room inspection and package relocation, with some possessing specialized capabilities or restricted access. The goal is to optimize warehouse space utilization. (b) Home Service. In the VirtualHome \cite{puig2018virtualhome} environment, 2-4 robots interact with dozens of objects and perform hundreds of potential actions. Each robot’s action space is diverse, aiming to complete complex household tasks, such as setting the table or preparing a meal.


\subsubsection{Evaluation Metrics}
The algorithm's performance was evaluated using the following metrics: (a) Success Rate (SR): The percentage of successfully completed tasks across multiple trials, accounting for action failure probabilities. (b) Team Steps (TS): The total number of steps required for all robots to complete their tasks in parallel. (c) Total Robot Steps (RS): The sum of steps taken by each robot independently. (d) Communication Overhead (Comm.): The number of broadcast communications between robots due to intention sharing. (e) Number of Expanded Conditions (EC): The number of condition nodes expanded during the multi-robot BTs planning process, including those from subtree pre-planning if available. (f) Planning Time (PT): The time taken for multi-robot BT planning, including subtree pre-planning when available. 




\subsubsection{Settings}
(a)  Homogeneity ($\boldsymbol{\alpha}$): The proportion of redundant actions assigned to robots, where $\boldsymbol{\alpha} = 1$ denotes complete heterogeneity (no overlap in action spaces) and $\boldsymbol{\alpha} = 0$ denotes complete homogeneity (identical action spaces).
(b) Action Failure Probability (FP): The probability that a robot fails to execute an action.
(c) Subtree Intention Sharing (Subtree IS) and Atomic Action Nodes Intention Sharing (Atomic IS): These terms refer to the application of Intention Sharing either among subtrees or at the level of individual atomic action nodes.
(d) Feedback (F) and No Feedback (NF): This setting distinguishes between LLMs that use feedback during subtree generation and those that do not. In the Feedback condition, the LLM receives up to 3 feedback iterations, while in the No Feedback condition, no feedback is provided.
% (d) Feedback (F) and No Feedback (NF): This setting differentiates between large language models (LLMs) that incorporate feedback during subtree generation and those that do not. In the Feedback (F) condition, the LLM receives up to three feedback iterations, whereas in the No Feedback (NF) condition, the LLM generates subtrees without any feedback.




\begin{table}[t]
\centering
\small
\setlength{\tabcolsep}{3pt}  % 调整列间距,默认是6pt,可以减少为3pt或者更小
% \label{tab:planning_time}
\begin{tabular}{c ccc c c c c}
\toprule
\multirow{2}{*}{\textbf{Method}} & \multicolumn{3}{c}{\textbf{$\boldsymbol{\alpha = 1}$ }} & & \textbf{\textbf{$\boldsymbol{\alpha \approx 0.5}$ }} & & \textbf{\textbf{$\boldsymbol{\alpha = 0}$ }} \\
\cline{2-4} \cline{6-6} \cline{8-8}  % 为每一条线单独指定范围
\\[-1.5ex] % 控制横线后的间距,-2ex 可以让间距更小
&  \textbf{SR(\%)} & \textbf{TS} & \textbf{RS} & &  \textbf{SR(\%)} & &  \textbf{SR(\%)} \\
\midrule
\textbf{BT-Expansion} & 100 & 8.8 & 33.8 & & 12.4 & & 4.6 \\
\textbf{MRBTP} & 100 & 5.8 & 15.3 & & 100 & & 100 \\
\bottomrule
\end{tabular}
\caption{Performance comparison with baseline in warehouse management (4 robots, averaged over 500 trials).}
\label{tab:comparison_with_baseline}
\end{table}






\begin{table*}[t]
    \centering
    \small
    \setlength{\tabcolsep}{2pt} % Adjust the column spacing
    \begin{tabular}{cc |cc cccc| cc cccc | cc cccc}
        \toprule
        \multicolumn{2}{c}{\textbf{Homogeneity}} & \multicolumn{6}{c}{\textbf{$\boldsymbol{\alpha = 1}$ }} & \multicolumn{6}{c}{\textbf{$\boldsymbol{\alpha \approx 0.5 } $}} & \multicolumn{6}{c}{\textbf{$\boldsymbol{\alpha=0}$}} \\ 


       \midrule 
      \multicolumn{2}{c|}{\textbf{Subtree}}   & - & - & \ding{51} &\ding{51} &\ding{51} &\ding{51} & - & - &  \ding{51} &\ding{51} &\ding{51} &\ding{51}  &- & -&  \ding{51} &\ding{51} &\ding{51} &\ding{51}  \\ 

        
        \multicolumn{2}{c|}{\textbf{Subtree IS}}          &- & - & -&-  & \ding{51} & \ding{51} &- & - & -&- & \ding{51} & \ding{51} &- & - &- &- & \ding{51} & \ding{51} \\ 
        \multicolumn{2}{c|}{\textbf{Atomic IS}}           &  - & \ding{51} & -  & \ding{51} & - & \ding{51}  & - &  \ding{51} & - & \ding{51} & - & \ding{51} & - & \ding{51} &- &\ding{51}  & -& \ding{51} \\ 



        \midrule
        \multirow{2}{*}{\textbf{TS}}   &  \textbf{NF}    & 161 & 159.4 & 114.4 & 109.6 & 78.9 & 79.8 & 139.7 & 137.5 & 126.2 & 119.6 & 86.9 & 102.5 & 73.7 & 68.5 & 96.1 & 94.8 & 75.6 & 78.0 \\ 
        
         &  \textbf{F} & - & - & 116.7 & 114.2 & \textbf{77.1} & 79.16 & - & - & 124.6 & 126.0 & \textbf{80.8} & 96.2 & - & - & 107.1 & 106.4 & \textbf{70.4} & 76.8 \\  
        \midrule

        
        \multirow{2}{*}{\textbf{RS}}   &  \textbf{NF}      & 570.8 & 557.3 & 374 & 359.4 & 217.4 & 219.1 & 385.2 & 380.1 & 345.8 & 326.6 & 209.6 & 222 & 128.6 & 128.6 & 128.6 & 128.6 & 128.6 & 128.6 \\ 
        &  \textbf{F}  & - & - & 377 & 370.9 & \textbf{205.2} & 208 & - & - & 380.7 & 348.2 & \textbf{192.2} & 209 & - & - & 128.6 & 128.6 & 128.6 & 128.6 \\ 

        \midrule
        \multirow{2}{*}{\textbf{Comm.}}  &  \textbf{NF}  & 0.0 & 63.8 & 0.0 & 7.1 & 6.7 & 14.1 & 0.0 & 43.5 & 0.0 & 8.0 & 7.1 & 20.9 & 0.0 & 15.2 & 0.0 & 2.8 & 6.5 & 9.3 \\ 
         & \textbf{F} & - & - & 0.0 & 4.8 & \textbf{6.6} & 12.2 & - & - & 0.0 & 4.4 & \textbf{6.4} & 13.2 & - & - & 0.0 & 0.7 & \textbf{5.2} & 6.0 \\ 
        
        
    \bottomrule
    \end{tabular}
    \caption{Execution efficiency with subtree pre-planning and intention sharing.}
    \label{tab:llm_sutree_result}
\end{table*}


% The time cost for each LLM invocation is around 0-5 seconds, depending on the number of robots and atomic actions. 
\begin{table}[ht]
\centering
% \footnotesize 
\small
\setlength{\tabcolsep}{5.5pt}  % 调整列间距,默认是6pt,可以减少为3pt或者更小
%  LLM invocation 的每次平均时长大约为4.2s
% \setlength{\tabcolsep}{3pt} % 调整列间距
% \setlength{\tabcolsep}{2pt} % 调整列间距
\renewcommand{\arraystretch}{0.9} % 调整行间距,数值越小行间距越小
\begin{tabular}{ccccc}
\toprule
\textbf{Homogeneity} & \textbf{Subtree} & \textbf{Feedback} & \textbf{EC}   & \textbf{PT (s)}  \\ \midrule

\multirow{3}{*}{ \textbf{$\boldsymbol{\alpha = 1}$ } } & - & - & 8033.3  &  Timeout \\
 & \ding{51} & - & 998.1  & 12.4 \\
  & \ding{51} & \ding{51}& \textbf{384.3}  & \textbf{3.7} \\
\midrule

\multirow{3}{*}{\textbf{$\boldsymbol{\alpha \approx 0.5 } $} } & - & - & 7882.5 &  Timeout \\
 & \ding{51} & - &  623.8 & 7.2 \\ 
  & \ding{51} &\ding{51} & \textbf{267.9} & \textbf{2.6} \\ 

\midrule
\multirow{3}{*}{\textbf{$\boldsymbol{\alpha=0}$}} &- & - & 2695.5 & 20.2 \\
& \ding{51} & - &  576.6  & 5.6\\  
& \ding{51} & \ding{51} &  \textbf{146.8}  & \textbf{1.4}\\ 

\bottomrule
\end{tabular}
\caption{Planning efficiency with pre-planned subtrees. The average response time per LLM invocation is 4.2 seconds.}
\label{tab:planning_time}
\end{table}


% \begin{table*}[t]
%     \centering
%     \small
%     \caption{Execution efficiency with subtree pre-planning and intention sharing.}
%     \setlength{\tabcolsep}{3pt} % 调整列间距
%     \begin{tabular}{c| c c c  c |c  c | c c | c c }
%         \toprule
%         \multirow{2}{*}{\textbf{Homogeneity}} & \multirow{2}{*}{\textbf{}} & \multirow{2}{*}{\textbf{Subtree}} & \multirow{2}{*}{\textbf{Subtree IS}} & \multirow{2}{*}{\textbf{Atom IS}} & \multicolumn{2}{c|}{\textbf{Team Steps}} & \multicolumn{2}{c|} {\textbf{Robot Steps}} & \multicolumn{2}{c}{\textbf{Comm.}} \\
%         & & \textbf{} & \textbf{} & \textbf{} & \textbf{No Feedback} & \textbf{Feedback} & \textbf{No Feedback} & \textbf{Feedback} & \textbf{No Feedback} & \textbf{Feedback} \\
%         \midrule
%         \multirow{6}{*}{\textbf{$\boldsymbol{\alpha = 1}$ }}  
%         &  & - & - & - & 161 & - & 570.8 & - & 0.0 & - \\
%         &  & - & - & \ding{51} & 159.4 & - & 557.3 & - & 63.8 & - \\
%         &  & \ding{51} & - & - & 114.4 & 108.2 & 469.2 & 445.0 & 98.4 & 92.1 \\
%         &  & \ding{51} & \ding{51} & - & 111.1 & 104.5 & 458.2 & 430.5 & 105.9 & 97.4 \\
%         &  & \ding{51} & - & \ding{51} & 108.0 & 101.7 & 447.8 & 419.2 & 111.4 & 102.3 \\
%         &  & \ding{51} & \ding{51} & \ding{51} & 106.0 & 99.9 & 438.5 & 410.6 & 118.2 & 108.9 \\
%         \midrule
%         \multirow{6}{*}{\textbf{$\boldsymbol{\alpha \approx 0.5}$}}  
%         &  & - & - & - & 139.7 & - & 385.2 & - & 0.0 & - \\
%         &  & - & - & \ding{51} & 137.5 & - & 380.7 & - & 0.0 & 0.0 \\
%         &  & \ding{51} & - & - & 119.6 & 126.0 & 473.6 & 451.3 & 94.5 & 88.2 \\
%         &  & \ding{51} & \ding{51} & - & 115.2 & 108.9 & 462.9 & 436.5 & 101.3 & 92.9 \\
%         &  & \ding{51} & - & \ding{51} & 111.8 & 105.4 & 451.7 & 423.4 & 107.2 & 97.8 \\
%         &  & \ding{51} & \ding{51} & \ding{51} & 109.6 & 103.1 & 442.6 & 413.4 & 114.0 & 104.5 \\
%         \midrule
%         \multirow{6}{*}{\textbf{$\boldsymbol{\alpha = 0}$}}  
%         &  & - & - & - & 173.3 & - & 573.7 & - & 0.0 & - \\
%         &  & - & - & \ding{51} & 169.7 & - & 559.4 & - & 61.3 & - \\
%         &  & \ding{51} & - & - & 122.2 & 115.8 & 477.5 & 455.0 & 90.2 & 83.1 \\
%         &  & \ding{51} & \ding{51} & - & 118.5 & 112.2 & 466.2 & 439.8 & 97.5 & 89.4 \\
%         &  & \ding{51} & - & \ding{51} & 114.9 & 108.5 & 454.8 & 426.2 & 103.9 & 94.6 \\
%         &  & \ding{51} & \ding{51} & \ding{51} & 112.7 & 106.1 & 445.5 & 416.8 & 110.4 & 101.1 \\
%         \bottomrule
%     \end{tabular}
% \end{table*}


\subsubsection{Baselines}
% 怎么表达?
% 由于目前没有多机器人行为树规划的算法,我们考虑将单机器人行为树规划算法 BT-Expansion 直接迁移到多机器人上,与我们的算法进行对比。被证明是 sound 和 complete

% 行为树规划算法通常利用动作模型进行规划,为保证在相同的问题假设下,我们考虑将在单机器人行为树规划算法  BT-Expansion 直接迁移到多机器人上,作为我们的 baseline。每个机器人针对群体目标,分别进行反向扩展规划,缺少Cross-tree expansion和ntention sharing。
BT planning algorithms typically utilize action models for planning. To ensure consistency under the same problem assumptions, we propose directly adapting the BT-Expansion \cite{cai2021bt} algorithm, which has been proven sound and complete in single-robot settings, to multi-robot scenarios as our baseline. In BT-Expansion, each robot independently performs backward planning towards the team's goal, without incorporating cross-tree expansion or intention sharing. 




\subsection{Experimental Results}

\subsubsection{Performance Comparison}
 
We randomly generated solvable multi-robot BT planning problems under various settings. Table \ref{tab:comparison_with_baseline} shows a significant drop in BT-Expansion's success rate as homogeneity decreases. In contrast, MRBTP maintains a perfect success rate of 100\% across all settings due to its cross-tree expansion. To avoid bias in execution efficiency (TS, RS) caused by planning failures, we only compared cases where both algorithms succeeded. Notably, even under full homogeneity, MRBTP outperforms BT-Expansion in execution efficiency due to intention sharing.


\subsubsection{Robustness}
As shown in Figure \ref{fig:robust_results}, the robustness of our algorithm improves with increasing homogeneity and is further enhanced by a larger number of robots. This improvement results from the increased likelihood of other robots compensating for action failures. Even with a 50\% failure probability per action, the system retains approximately a 50\% chance of achieving the goal with 8 robots and complete action space homogeneity. 






\subsubsection{Execution Efficiency}

As shown in Figure \ref{fig:subtask_chaining_results}, in fully heterogeneous scenarios, enabling intention sharing results in fewer team steps, indicating that our MRBTP algorithm inherently maintains superior execution efficiency under these conditions. Additionally, as homogeneity increases, the likelihood of robots performing redundant actions rises, reducing the probability of parallel task execution. However, with intention sharing, redundant actions are significantly minimized, preventing further efficiency loss. In this context, increased homogeneity brings more backup structures, further improving execution efficiency. 



\subsubsection{Effectiveness of Task-Specific Subtree Pre-Planning}
We constructed a dataset of 75 instances across three levels of homogeneity. The model used to generate subtrees is \textit{gpt-4o-mini-2024-07-18} \cite{openai2023gpt4}. Table \ref{tab:llm_sutree_result} provides a comparative analysis of the impact of introducing task-specific subtrees on execution efficiency, and communication overhead. The improvement in execution efficiency brought about by intention sharing increases with higher homogeneity. Execution efficiency is highest when subtrees are combined with intention sharing. During execution, the long-horizon subtrees facilitate forward planning, leading to more efficient and less frequent communication compared to finer-grained atomic actions. Table \ref{tab:planning_time} shows that subtree pre-planning significantly reduces BTs planning time under a 60-second constraint by minimizing redundancy through subtree reuse and similar robot action spaces. Additionally, both Table \ref{tab:llm_sutree_result} and Table \ref{tab:planning_time} demonstrate that feedback effectively enhances both planning and execution efficiency, especially when integrated with the sharing of the subtree and intentions.


% With subtree usage, the first row of Team Steps, Total Robot Steps, and Comm. reflects outcomes without LLM feedback, while the second row includes feedback (max 3 times). 



\subsubsection{Execution Efficiency across Different LLMs}
% 我们更换了不同版本的大语言模型,尝试了 GPT-3.5-turbo-2024-12-15 和 GPT-4o-2024-08-06 ,使用它们辅助subtree pre-planning。表3 中显示出随着大语言模型推理能力的增强,执行效率略有提升,通信开销变化较小。这是由于 subtree pre-planning will also become more appropriate and effective。
% 此外也再次看出,反馈机制在所有模型中均能提高执行效率
We tested different versions of LLMs, including \textit{gpt-3.5-turbo (2024.12)} and \textit{gpt-4o-2024-08-06} \cite{openai2023gpt4}, for assisting in subtree pre-planning. As shown in Table \ref{tab:different_llms}, with the increased reasoning capability of the LLMs, there is a slight improvement in execution efficiency, while communication overhead remains largely unchanged. This can be attributed to the fact that subtree pre-planning becomes more appropriate and effective as the model's reasoning ability improves. Additionally, the results further demonstrate that the feedback mechanism enhances execution efficiency across all LLMs.




\begin{table}[t]
\centering
\small
\setlength{\tabcolsep}{3pt}  % 调整列间距
% \caption{Execution efficiency across different LLMs.}
\begin{tabular}{c ccc c ccc}
\toprule
\multirow{2}{*}{\textbf{Models}} & \multicolumn{3}{c}{\textbf{No Feedback}} &  & \multicolumn{3}{c}{\textbf{Feedback}} \\
\cline{2-4} \cline{6-8}
\\[-1.5ex] % 控制横线后的间距,-2ex 可以让间距更小
  % & \makecell{\textbf{Team} \\ \textbf{Steps}} & \makecell{\textbf{Robot} \\ \textbf{Steps}} & \textbf{Comm.} & & \makecell{\textbf{Team} \\ \textbf{Steps}} & \makecell{\textbf{Robot} \\ \textbf{Steps}} & \textbf{Comm.} \\
    & \textbf{TS} & \textbf{RS}  & \textbf{Comm.} & & \textbf{TS} & \textbf{RS}  & \textbf{Comm.}  \\
\midrule
% \textbf{llma}      & x1   & y1   & z1   & & a1   & b1   & c1   \\
\textbf{GPT-3.5-turbo}   & 81.6   & 223.6   & 5.1   & & 80.0   & 219.0   & 5.1   \\
% \textbf{Gemini-1.5}& x3   & y3   & z3   & & a3   & b3   & c3   \\
\textbf{GPT-4o-mini}    & 78.9   & 217.4   & 6.7   & & 77.1   & 205.2   & 6.6   \\
\textbf{GPT-4o}    & 77.4   & 200.9   & 6.3   & & 74.9   & 190.7   & 6.3   \\ %gpt-4o-2024-08-06
\bottomrule
\end{tabular}
\caption{Execution efficiency across different LLMs under $\boldsymbol{\alpha = 1}$ with subtree and subtree intention sharing.}
\label{tab:different_llms}
\end{table}


\section{Related Work}

\paragraph{BT Planning.}
Many works have focused on automatically generating BTs to perform tasks, such as evolutionary computing \cite{neupane2019learning,colledanchise2019learning,lim2010evolving}, reinforcement learning \cite{banerjee2018autonomous,pereira2015framework}, imitation learning \cite{french2019learning}, MCTS \cite{scheide2021behavior}, and formal synthesis \cite{li2021reactive,tadewos2022specificationguided,neupane2023designing}. Recently, some works directly generate BTs using LLMs \cite{lykov2023llmbrain,lykov2023llmmars}. However, the above methods either require complex environment modeling or cannot guarantee the reliability of BTs. In contrast, BT planning \cite{cai2021bt,chen2024integrating} based on STRIPS-style modeling \cite{fikes1971strips} not only offers intuitive environment modeling but also ensures the reliability and robustness of the generated BTs. 



\paragraph{BT in MRS.}
BT generation for Multi-Robot Systems (MRS) has been investigated using various methodologies. Evolutionary computing \cite{neupane2019learning} is a general heuristic search method applied to BT generation in MRS. While versatile, this approach often suffers from slow search efficiency due to its lack of integration with the action model. Given the modular nature of BT systems, the action model is not difficult to obtain \cite{arora2018review}, enabling the development of methods that can yield more efficient solutions. MRS BT generation methods based on LLMs \cite{lykov2023llmmars} or other machine learning techniques \cite{fu2016reinforcement} have also been explored. These methods require substantial training data, making data collection and model training resource-intensive. Moreover, the aforementioned methods lack guarantees for the completeness and correctness of the generated BTs. Auction-based methods \cite{dahlquist2023reactive,heppner2024behavior,colledanchise2016advantages}, some of which incorporate action model planning, rely on the assumption of reliable communication and low transmission delay to ensure efficient task completion. However, such conditions are not always guaranteed, rendering these approaches less robust in environments with unreliable communication. In contrast, our method generates BTs before the robot team begins execution, ensuring task completion even in the absence of communication during execution. Communication during execution serves only to improve coordination efficiency, rather than being a necessary assumption.



% Therefore, developing an effective multi-robot BT planning algorithm that leverages the action model to ensure completeness and correctness, while maintaining robustness under diverse communication conditions, remains a crucial and underexplored area.




% This reliance on extensive datasets can limit the practicality and scalability of ML-based BT generation in diverse and dynamic multi-robot environments.
% Given the modular nature of BT systems, the action model is is not difficult to obtain, suggesting that BT planning methods leveraging this information could offer more efficient and effective solutions for MRS.
% In BT systems, due to their modular nature, the action model is not difficult to obtain.
% As for BT generation in MRS, there are also many methods including evolutionary computing \cite{neupane2019learning}, auction-based methods \cite{dahlquist2023reactive,heppner2024behavior,colledanchise2016advantages}, BT synthesis \cite{tadewos2023automatic,colledanchise2018behavior}, and LLM-based methods \cite{lykov2023llmmars}. 
% However, just like single-robot BT generation methods, the approaches mentioned above suffer from complex formalization or lack of interpretability. Therefore, developing an effective multi-robot BT planning algorithm is crucial.


% There are many methods designed for BT generation in MRS with different problem settings and assumptions. Evolutionary computing [23] is a general heuristic search method that can be used for BT generation in MRS. However, since this method does not utilize information from the action model, its search efficiency is relatively slow. In BT systems, due to their modular nature, the action model is not difficult to obtain. Therefore, research that leverages the action model for BT planning in MRS offers a more promising application scenario. MRS BT generation methods based on LLMs or other machine learning techniques [21] typically assume that large amounts of data are available for training. However, the data collection and training process is often time-consuming. Auction-based methods [12, 16, 9, 33] assume that communication between robots is reliable and low-latency during execution. However, good communication conditions cannot be guaranteed in all scenarios. In summary, we believe that the multi-robot BT planning problem, with its unique advantages and application scenarios, is a novel and highly worthwhile research area for BT generation in MRS, and the algorithm proposed in this paper addresses a significant gap in this field.

\paragraph{LLM for Task Reasoning.} Recently, significant progress has been made in using LLMs for task reasoning \cite{song2023llmplanner,liu2023llm,ahn2022can,chen2023robogpt}, such as progprompt \cite{singh2022progprompt}, PlanBench \cite{valmeekam2023planbench}, and Voyager \cite{wang2023voyager}. Furthermore, the LLM has shown the ability to decompose the task into subgoals \cite{gao2024dagplan,singh2024twostep}, which is closely related to our subtree pre-planning for multi-robot BT planning. As the task reasoning abilities of LLMs continue to evolve and strengthen, our subtree pre-planning technique is poised to become increasingly relevant and effective.
% BT planning has shown as a promising for BT generation which can guarantee the success of the BTs \cite{chen2024integrating,chen2024efficient,cai2021bt,colledanchise2019blended}. However, there are no BT planning algorithms for multi-robot system. 
% \cite{ghzouli2020behavior} This article explores how to use subtrees to enhance the reusability of behavior tree systems.




\section{Conclusion}
% 终稿 We propose MRBTP, the first sound and complete algorithm for solving the multi-robot BT planning problem. Intention sharing significantly improves execution efficiency and robustness. The plugin LLM further enhances the planning speed and reducing communication overhead. These contributions represent a key step forward in scalable, reliable multi-robot systems. Future research will refine the algorithm's performance and extend its application to more complex, dynamic environments, solidifying MRBTP as a foundational approach in multi-robot planning.

We propose MRBTP, the first sound and complete algorithm for solving the multi-robot BT planning problem. The cross-tree expansion coordinates BTs for achieving goals, while intention sharing improves execution efficiency and robustness. The LLM plugin further enhances planning speed and reduces communication overhead. These contributions represent a key step forward in scalable, reliable multi-robot systems. Future research will refine the algorithm's performance and extend its application to more complex, dynamic environments, solidifying MRBTP as a foundational approach in multi-robot planning. Furthermore, the potential deployment of the algorithm on actual robotic systems will be explored, evaluating its effectiveness, scalability, and practicality in real-world scenarios.

% This work proposed the MRBTP algorithm, which advances multi-robot task planning by ensuring both soundness and completeness while effectively managing heterogeneous and homogeneous actions. The algorithm's cross-tree expansion and intention sharing mechanisms demonstrated significant improvements in execution efficiency and robustness across various scenarios. Additionally, the integration of LLMs as an optional plugin provided notable enhancements in planning speed and communication overhead. These contributions mark a step forward in the development of scalable and reliable multi-robot systems. Future research will aim to refine the algorithm's performance and explore its application to more complex and dynamic environments, further establishing MRBTP as a foundational approach in multi-robot planning.


% This paper proposed an iterative expansion algorithm for multi-robot behavior trees to efficiently coordinate task planning and execution in multi-robot systems. The algorithm ensures soundness and completeness, handling dynamic changes and external disturbances to enhance robustness and adaptability. Experimental results in simulated environments showed the algorithm's effectiveness in reducing task completion times and optimizing resource utilization, with significant improvements from parallel execution. Despite challenges related to computational overhead and non-interference, the algorithm provides a robust framework for multi-robot planning. Future work will focus on refining synchronization mechanisms, optimizing the expansion process, and integrating learning-based approaches to further enhance performance and scalability. This research contributes to advancing multi-robot systems, enabling efficient and reliable autonomous coordination in dynamic environments.

% \newpage

% \section{Acknowledgments}
% This work was supported by the National Natural Science
% Foundation of China (Grant Nos. 62106278, 91948303-1, 
% 611803375, 12002380, 62101575, 62032024), and the National Key
% R\&D Program of China (Grant No. 2021ZD0140301). 
\section{Acknowledgments}
This work was supported by the National Natural Science
Foundation of China (Grant Nos. 62106278, 62032024). 

\bigskip

\bibliography{aaai25}



\newpage
\newpage
\onecolumn
\appendix
% 设置 \section 居左
% \titleformat*{\section}{\Large\bfseries\raggedright}

\section{Appendix}

\section{A.Proofs of MRBTP}
In this section, we present a comprehensive version of the formalizations and proofs for Multi-Robot Behavior Tree Planning (MRBTP).

\subsection{Definitions}

\begin{definition}[Behavior Tree]
A behavior tree (BT) is a three-tuple $\mathcal{T} = <f, r, \Delta t>$. $f:2^{n}\rightarrow 2^{n}$ is its effect on the environment state, $\Delta t$ is the time step, and $r:2^{n}\mapsto \{ $\constant{S}, \constant{R}, \constant{F}\} partitions states into three regions, where $\mathcal{T}$ returns success, running, failure, respectively.
\end{definition}

In BT planning for a single robot \cite{cai2021bt}, we represent the problem as: \(<\mathcal{S},\mathcal{L},\mathcal{A},\mathcal{M}, s_0,g>\), where \( \mathcal{S} \) is the finite set of environment states, $\mathcal{L}$ is the finite set of literals that form states, \( \mathcal{A} \) is the finite set of actions, $\mathcal{M}$ is the action model, $s_0$  is the initial state, $g$ is the goal condition. 


A condition $c$ in BT is usually a subset of a state $s$. If $c\subseteq s$, it is said condition $c$ holds in that state $s$. The state transition affected by action $a\in \mathcal{A}$ can be defined as a triplet \( \mathcal{M}(a)=<pre(a),add(a),del(a)> \), comprising the precondition, add effects, and delete effects of the action. We assume that an action always finishes in finite time.  If $a$ is finished after $k$ time step, the subsequent state $s_{t'}$ will be:
\begin{equation}\label{eqn-s_f}
	s_{t'}=f_a(s_t)=s_t\cup add(a)\setminus del(a), t'=t + k
\end{equation}
The following property holds for $\forall a \in \mathcal{A}$:
\begin{align}
	add(a) \cap del(a) = \emptyset \label{eqn:a_d}\\
	add(a) \cap pre(a) = \emptyset \label{eqn:a_p}
\end{align}

We then extend the BT representation from a single robot to a multi-robot system. 



\begin{definition}[Multi-BT System]
A $n$-robot BT system is a four-tuple $\left<\Phi, f_\Phi, r_\Phi, \Delta t_\Phi\right>$, where $\Phi = \left\{ \mathcal{T}_i \right\}_{i=1}^n$ is the set of BTs, $f_\Phi: \mathcal{S} \mapsto \mathcal{S}$ is the team state transition function, $\Delta t_\Phi$ is the team time step, $r_\Phi: \mathcal{S} \mapsto \{$ \constant{S}, \constant{R}, \constant{F} $\}$ is the team region partition.
\end{definition}
Due to variability in hardware performance, we allow each robot's BT to have a different response frequency, with $\Delta t_\Phi$ representing the common minimum response interval. The state transition can be calculated as follows:
\begin{align}
    s_{t+\Delta t_\Phi} = f_\Phi(s_t) = s_t \cup \bigcup_{i=1}^n \left( add(a_i) \setminus del(a_i) \right)
\end{align}
where $a_i$ is the action of robot $i$ in time $t$. If robot $i$ do not have an action or its action is running, we let $add(a_i) = del(a_i) = \emptyset$. 

The team region partition can be calculated as follows:
\begin{equation}\label{equ-multiregion}
    r_\Phi(s) = 
\begin{cases}
    \text{\constant{R}} & \text{if } \exists i, r_i(s) = \text{\constant{R}} \\
    \text{\constant{S}} & \text{if } \forall i, r_i(s) \neq \text{\constant{R}} \text{ and } \exists i, r_i(s) = \text{\constant{S}} \\
    \text{\constant{F}} & \text{if } \forall i, r_i(s) = \text{\constant{F}}
\end{cases}
\end{equation}
The status of $\Phi$ is \constant{R} if any BT is still running, \constant{S} if some BT returns success and no one is running, and \constant{F} if all BT fails.
% Finite time successful (FTS) in region of attraction (ROA).
\begin{definition}[Finite Time Successful]
$\Phi$ is finite time successful (FTS) from region of attraction (ROA) $R$ to condition $c$, if $\forall s_0 \in R$  there is 
 a finite time $\tau$ such that for any $t<\tau$, $r_\Phi(s_t)=$ \constant{R}, and for any $t\geq\tau, r_\Phi(s_t)=$ \constant{S}, $c\subseteq s_t$.
\end{definition}
With definitions above, the multi-robot BT planning problem can finally be defined.

\begin{problem}[Multi-Robot BT Planning]\label{problem-MRBTP}
The problem is a tuple \(\left<\mathcal{S},\mathcal{L},\{\mathcal{A}_i\}_{i=1}^n,\mathcal{M}, s_0,g\right>\), where \( \mathcal{S} \) is the finite set of environment states, $\mathcal{L}$ is the finite set literals that form states and conditions, $\mathcal{A}_i$ is the finite action set of robot $i$, $\mathcal{M}$ is the action model, $s_0$  is the initial state, $g$ is the goal condition. A solution to this problem is a BT set $\Phi = \{\mathcal{T}_i\}_{i=1}^n$ built with $\{\mathcal{A}_i\}_{i=1}^n$, such that $\Phi$ is FTS from $R\ni s_0$ to $g$.

% , and for all $s$ in the team success region $S=\{s|r_\Phi(s)=$\constant{S}$\}$, $g\subseteq s$).
\end{problem}


% Assuming that the effects of any action only take effect when the action is completed, the state transition function of the system is:


\begin{algorithm}[t]
\caption{One-step cross-tree expansion}
\label{alg-one_step}
\begin{algorithmic}[1]
% \Function{Expand}{$\mathcal{T}, c$}
\State \textbf{function} \constant{ExpandOneRobot(}$\mathcal{T},\mathcal{A}, c$\constant{)}
\begin{ALC@g}
\StateComment{$\mathcal{T}_{new} \leftarrow c$}{newly expanded subtree}
\StateComment{$\mathcal{C}_{new} \gets \emptyset$}{newly expanded conditions}
\FOR{\textbf{each} action $a \in \mathcal{A}$}
    \IF{$c \cap (pre(a) \cup add(a) \setminus del(a)) \neq \emptyset$ and $c \setminus del(a) = c$}  \label{line-ifexpanded}
    \STATE $c_a \leftarrow pre(a) \cup c \setminus add(a)$ \label{line-ca}
    \STATE $\mathcal{T}_a \leftarrow Sequence(c_a, a)$ \label{line-ta}
    \STATE $\mathcal{T}_{new} \leftarrow Fallback(\mathcal{T}_{new}, \mathcal{T}_a)$ \label{line-tnew}
    \State{$\mathcal{C}_{new} \gets \mathcal{C}_{new} \cup \{c_a\}$}
    \ENDIF
\ENDFOR
\IF{$\mathcal{C}_{new}\neq \emptyset$}
    \IF{\constant{ConditionInTree}($c$, $\mathcal{T}$)}
        \StateComment{Replace $c$ with $\mathcal{T}_{new}$ in $\mathcal{T}$}{in-tree expand} \label{InsideExpand}
    \ELSIF{$\mathcal{T}_{new} \neq c$} 
        \StateComment{$\mathcal{T} \gets Fallback(\mathcal{T},\mathcal{T}_{new}) $}{cross-tree expand} \label{line-outside}
    \ENDIF
\ENDIF
\end{ALC@g}
\RETURN $\mathcal{T}, \mathcal{C}_{new}$
% \EndFunction
\end{algorithmic}
\end{algorithm}

\subsection{Propositions and Proofs}

\begin{lemma}\label{lem-ta}
Given a condition $c$, the sequence structure $\mathcal{T}_a = sequence(c_a,a)$ expanded in Algorithm \ref{alg-one_step} (line \ref{line-ta}) is FTS from $\mathcal{S}_a=\{s\in \mathcal{S} | c_a\subseteq s\}$ to $c$.
\end{lemma}

\begin{proof}
Starting from any $s_t\in \mathcal{S}_{a}=\{s\in \mathcal{S}| c_a \subseteq s\}$, $\exists c_a, c_a\subseteq s_t$. According to Equation \ref{eqn-s_f}, there exists a finite $k$ such that the action returns success and $s_{t+k}=s_t\cup add(a) \setminus del(a)\supseteq c_a \cup add(a) \setminus del(a)=pre(a)\cup c \setminus del(a)$. Since action selection (line \ref{line-ifexpanded}) ensures that $c\setminus del(a) = c$, we have $s_{t+k}\supseteq pre(a)\cup c \supseteq c$. Therefore $\mathcal{T}_a$ is FTS from $\mathcal{S}_{a}$ to $c$.
\end{proof}

\begin{lemma}\label{lem-tnew}
Given a condition $c$ and $\mathcal{C}_{new}\neq \emptyset$, $\mathcal{T}_{new}$ expanded in Algorithm \ref{alg-one_step} (line \ref{line-tnew}) is FTS from $\mathcal{S}_{new}=\{s\in \mathcal{S} | c_a\subseteq s, c_a\in \mathcal{C}_{new}\}$ to $c$.
\end{lemma}

\begin{proof}
Since $\mathcal{C}_{new}\neq\emptyset$, $\mathcal{T}_{new}$ has and only has one fallback node as the root, given $Fallback(\mathcal{T}_1,Fallback(\mathcal{T}_2,\mathcal{T}_3)) =
Fallback(\mathcal{T}_1,\mathcal{T}_2,\mathcal{T}_3)$. The fallback node returns success as long as any child returns success. According to Lemma \ref{lem-ta}, $\forall c_a\in \mathcal{C}_{new}$, the corresponding $\mathcal{T}_a$ is FTS from $\mathcal{S}_a$ to $c$, and $\mathcal{C}_{new}$ is a finite set as $\mathcal{A}$ is a finite set.  Therefore $\mathcal{T}_{new}$ is FTS from $\mathcal{S}_{new}=\bigcup_{c_a\in \mathcal{C}_{new}} \mathcal{S}_a = \{s\in \mathcal{S} | c_a\subseteq s, c_a\in \mathcal{C}_{new}\}$ to $c$. 
\end{proof}

\begin{proposition}\label{pro-inside}
    Given $\mathcal{T}$ is FTS from $R$ to $g$, if $\mathcal{T}$ is expanded by Algorithm \ref{alg-one_step} to $\mathcal{T}'$ given $c$, $c$ is in $\mathcal{T}$ and $\mathcal{C}_{new} \neq \emptyset$, then $\mathcal{T}'$ is FTS from $R'=R\cup \{ s\in \mathcal{S}| c_a \subseteq s, c_a\in \mathcal{C}_{new} \}$ to $g$.
\end{proposition}

\begin{proof}
% Because $\mathcal{T}$ is FTS from $R$ to $g$. 
% If $s\in \mathcal{S}_{new}=\{ s\in \mathcal{S}| c_a \subseteq s, c_a\in \mathcal{C}_{new} \}$, according to Lemma \ref{lem-tnew}, the expended $\mathcal{T}_{new}$ in $\mathcal{T}'$ is FTS from $\mathcal{S}_{new}$ to $c$. 

For any $s\in R'$, we consider two cases:

(1) $s\in R$. In this case, either $\mathcal{T}_{new}$ in the BT $\mathcal{T}'$ is not ticked ($c\nsubseteq s$), or $\mathcal{T}_{new}$ returns success ($c\subseteq s$). In either situation, the execution logic of the BT is the same as that of the original $\mathcal{T}$. Since $\mathcal{T}$ is FTS from $R$ to $g$, the expanded $\mathcal{T}'$ is also FTS from $R$ to $g$.

(2) $s\notin R$ and $s\in \mathcal{S}_{new}=\{ s\in \mathcal{S}| c_a \subseteq s, c_a\in \mathcal{C}_{new} \}$. In this case, the expended $\mathcal{T}_{new}$ in BT $\mathcal{T}'$ will be ticked. According to Lemma \ref{lem-tnew}, $\mathcal{T}_{new}$ is FTS from $\mathcal{S}_{new}$ to $c$, which means there is a finite time $\tau_1$ such that for any $t<\tau_1$, $r_{\mathcal{T}'}(s_t)=$\constant{R}, and for $t=\tau_1$, $r_{\mathcal{T}_{new}}(s_{\tau_1})=$\constant{S}, $c\subseteq s_{\tau_1}$. Since $s_{\tau_1}\in R$, there is a finite time $t=\tau_2$ such that for any $t<\tau_1+\tau_2$, 
$r_{\mathcal{T}'}(s_t)=$\constant{R}, and for any $t\geq \tau_1+\tau_2$, $r_{\mathcal{T}'}(s_t)=$\constant{S}, $g\subseteq s_t$. Therefore $\mathcal{T}'$ is FTS from $\mathcal{S}_{new}\setminus R$ to $g$.

According (1) and (2), $\mathcal{T}'$ is FTS from $R'= R\cup \{ s\in \mathcal{S}| c_a \subseteq s, c_a\in \mathcal{C}_{new} \}$ to $g$.
\end{proof}

\begin{proposition}\label{pro-outside}
    If $\mathcal{T}$ is expanded by Algorithm \ref{alg-one_step} to $\mathcal{T}'$ given $c$, $c$ is not in $\mathcal{T}$ and $\mathcal{C}_{new} \neq \emptyset$, then $\mathcal{T}'$ is FTS from $S_{new}=\{ s\in \mathcal{S}| c_a \subseteq s, c_a\in \mathcal{C}_{new} \}$ to $c$.
\end{proposition}

\begin{proof}
Because $c$ is not in $\mathcal{T}$, the expanded subtree $\mathcal{T}_{new}$ is added to the tail of the root fallback node of $\mathcal{T}$ (line \ref{line-outside}). $\forall s \in \mathcal{S}_{new}$, $\mathcal{T}_{new}$ in $\mathcal{T}'$ will be ticked. According to Lemma \ref{lem-tnew}, $\mathcal{T}_{new}$ is FTS from $\mathcal{S}_{new}$ to $c$, which means $\mathcal{T}'$ is also FTS from $\mathcal{S}_{new}$ to $c$.
\end{proof} 



% forming a subtree $\mathcal{T}^i_{new}$. If the current expanded state is already in the robot’s behavior tree, we replace that state with subtree $\mathcal{T}^i_{new}$, which we call an internal expansion. If it's not in the robot's behavior tree, we extend the subtree to the end of the root node, which we call an external expansion.

% After the step-by-step expansion, we add all newly expanded conditions {c} from all robots to the global queue, and the loop continues. The algorithm stops when a condition in the new expansions {c} satisfies the initial state $s_0$, or when the queue is empty.



\begin{algorithm}[t]
\caption{MABTP}
\label{alg:MABTP}
\textbf{Input}: problem $\tuple{\mathcal{S},\mathcal{L},\multi{\mathcal{A}},\mathcal{M},s_0,c}$ \\
\textbf{Output}: solution $\Phi = \multi{\mathcal{T}}$ 
\begin{algorithmic}[1]

\STATE $\mathcal{C}_U\gets \{g\}$ \label{line:cu}
		\hfill $\triangleright$ conditions to be explored
\StateCommentLabel{$\mathcal{C}_E\gets \emptyset$}{expanded conditions}{line:ce}
\FOR{$i = 1$ \TO $n$}
    \STATE $\mathcal{T}_i \gets Fallback(g)$ \label{line-initBT} \Comment{init the BTs}
\ENDFOR
\WHILE{ $\mathcal{C}_U \neq \emptyset$}
\STATE $c\gets $ \constant{Pop(}$\mathcal{C}_U$\constant{)} \label{line:pickC}
\hfill $\triangleright$ explore $c$

\StateComment{\textbf{if} \algofunc{HasSubSet}{$c,\mathcal{C}_E$} \textbf{then} \textbf{continue}}{prune}  

\FOR{$i = 1$ \TO $n$}
    \STATE $\mathcal{T}_i, \mathcal{C}_{new} \gets$ \algofunc{ExpandOneRobot}{$\mathcal{T}_i,\mathcal{A}_i,c$} \label{line:initBT}
    \IF{\algofunc{HasSubSet}{$s_0,\mathcal{C}_{new}$}} \label{line-returnif}
        \RETURN $\Phi=\{\mathcal{T}_i\}_{i=1}^n$ \Comment{return a solution} \label{line-return}
    \ENDIF
    \State{$\mathcal{C}_E \gets \mathcal{C}_E \cup \mathcal{C}_{new}$}
    \StateCommentLabel{$\mathcal{C}_U \gets \mathcal{C}_U \cup \mathcal{C}_{new}$}{add new conditions}{}
\ENDFOR

\ENDWHILE
  
\RETURN \constant{Unsolvable} \label{line-returnu}
    
\end{algorithmic}
\end{algorithm}

\begin{lemma}\label{lem-treeonestep}
After the one-step expansion of the multi-BT system \( \Phi = \{\mathcal{T}_i\}_{i=1}^{n} \) with respect to condition \( c \), let \( \Phi_{new} = \{\mathcal{T}_{i,new}\}_{i=1}^{n} \) represent the set of all extended subtrees. $\Phi_{new}$ is FTS from $\bigcup_{i=1}^n \mathcal{S}_{i,new}$ to $c$, where $\mathcal{S}_{i,new}=\{s\in \mathcal{S}|c_a\subseteq s,c_a\in C_{i,new}\}$.
\end{lemma}

\begin{proof}
According to Lemma \ref{lem-tnew}, $\forall \mathcal{T}_{i,new} \in \Phi_{new}$, $\mathcal{T}_{i,new}$ is FTS from $\mathcal{S}_{i,new}$ to $c$. Therefore, for any $s\in \bigcup_{i=1}^n \mathcal{S}_{i,new}$, at least one subtree can run in $s$. Since we assume that at each time, only one robot with the highest priority can execute an action if its precondition is satisfied, we designate the executed subtree as $\mathcal{T}_{j,new}$. Because $\mathcal{T}_{j,new}$ is FTS to $c$, we have that $\Phi_{new}$ is FTS from $\bigcup_{i=1}^n \mathcal{S}_{i,new}$ to $c$.
\end{proof}

\begin{proposition}\label{pro-k-loop}: After the $k$-th ($k\geq 1$) iteration of the while loop in Algorithm \ref{alg:MABTP}, where the explored condition is $c^{k}$, $\Phi^{k}=\multi{\mathcal{T}^{k}}$ is FTS from ROA $R^k=R^{k-1} \cup \bigcup_{i=1}^n \mathcal{S}^k_i$ to goal $g$, where $\mathcal{S}^{k}_i=\{s\in \mathcal{S}|c_a\subseteq s,c_a\in C^k_{i,new}\}$.
\end{proposition}

\begin{proof}
This proposition can be proved by strong induction. In the basis step (before the first expansion), all the BT $\mathcal{T}^0_i\in \Phi^0$ is $Fallback(g)$ (line \ref{line-initBT}). According to \ref{equ-multiregion}, $\Phi^0=\multio{\mathcal{T}^0_i}$ if FTS from $R^0=\{s\in \mathcal{S}|g\subseteq s\}$ to $g$.

For the inductive step, we assume after the $k$-th ($k\geq 1$) iteration of the while loop, $\Phi^k$ is FTS from $R^k$ to $g$. Then after the $(k+1)$-th iteration, for $s\in R^{k} \cup \bigcup_{i=1}^n \mathcal{S}^{k+1}_i$, there are two cases:

(1) $s\in R^{k}$. In this case, $\Phi^{k+1}$ has the same execution logic as $\Phi^{k}$, just similar to case (1) in Proposition \ref{pro-inside}. Therefore $\Phi^{k+1}$ is FTS from $R^{k}$ to $g$.

(2) $s\notin R^{k}$ and $s\in \bigcup_{i=1}^n \mathcal{S}^{k+1}_i$. In this case, one of the expended $\mathcal{T}^{k+1}_{i,new}\in\Phi^{k+1}_{new}$ will be ticked and the state will go to some $s'\supseteq c$ in finite time according to Lemma \ref{lem-treeonestep}. Since $s'\in R^{k}$, $\Phi^{k+1}$ will succeed to $g$ in finite time according to (1).

According (1) and (2), after the $k$-th interation, $\Phi^{k}$ is FTS from ROA $R^k=R^{k-1} \cup \bigcup_{i=1}^n \mathcal{S}^k_i$ to goal $g$.
\end{proof}

\begin{proposition}\label{pro-sound}
Algorithm \ref{alg:MABTP2} is sound, i.e. if it returns a result $\Phi$ rather than \constant{Unsolvable}, then $\Phi$ is a solution of Problem 1.
\end{proposition}

\begin{proof}
If Algorithm \ref{alg:MABTP2} returns a result $\Phi^k$ after $k$-th iteration, that means $\exists c\in \mathcal{C}^k_{new}, c\subseteq s_0$ (line \ref{line-returnif}). So $s_0$ is in the ROA of $\Phi^k$: $R^k=R^{k-1} \cup \bigcup_{i=1}^n \mathcal{S}^k_i$. According to Proposition \ref{pro-k-loop}, $\Phi^k$ if FTS from $R^k$ to $g$. That makes $\Phi^k$ a solution of Problem \ref{problem-MRBTP}.
\end{proof}

\begin{lemma}\label{lemma_terminate}
    Algorithm \ref{alg:MABTP2} terminates in finite time.
\end{lemma}

\begin{proof}
This is because the totoal number of states $|\mathcal{S}|$ is finite, therefore the number of conditions is also finite. The algorithm either returns prematurely (line \ref{line-return}) or terminates when \( \mathcal{C}_U = \emptyset \) (line \ref{line-returnu}). Within the while loop, the loop for one-step expansion also terminates in finite time because the total number of actions $|\multi{\mathcal{A}}|$ is finite. Therefore, Algorithm \ref{alg:MABTP2} terminates in finite time.
\end{proof}


\begin{proposition}\label{pro-complete}
Algorithm \ref{alg:MABTP2} is complete, i.e., if Problem \ref{problem-MRBTP} is solvable, the algorithm returns a $\Phi$ which is a solution.
\end{proposition}

\begin{proof}
According to Lemma \ref{lemma_terminate}, Algorithm \ref{alg:MABTP2} either returns prematurely or terminates when $\mathcal{C}_U=\emptyset$ in finite time.

If it returns a $\Phi$ prematurely, according to Proposition \ref{pro-sound}, $\Phi$ is a solution of Problem \ref{problem-MRBTP}.

If it terminates when $\mathcal{C}_U=\emptyset$, it implies that all conditions reachable from the goal condition \( g \) have been explored, but no viable path from the goal \( g \) to the initial state \( s_0 \) has been found. This suggests that Problem \ref{problem-MRBTP} is not solvable. 

This can be proved by contradiction. We assume that there exists a solution $\Phi$ but our algorithm fails. $\Phi$ is FTS from $R\ni s_0$ to $g$, so we can utilize a finite sequence to represent the state transition of the solution BT: $(s_0,a_1,s_1,a_2,\dots,s_{m-1},a_m,s_m)$ where the goal condition $g\subseteq s_m$. We can prove, via mathematical induction from $s_m$ to $s_0$ in this sequence, that every state in this sequence must satisfy some condition that was traversed during the while loop.

For the basis step, we consider the state $s_{m-k}$ with $k=0$, i.e. $s_m\supseteq g$. Obviously $g$ is in all BTs of $\Phi$.

For the inductive step, we consider a transition $(s_{m-k-1},a_{m-k},s_{m-k})$ and there exists a condition $c\subseteq s_{m-k}$ in some BT $\mathcal{T}_i\in \Phi$ as the inductive premise. We need to prove that there also exists a condition satisfied by $s_{m-k-1}$. The inductive premise and the transition gives: 
\begin{equation}\label{eqn:snk1}
s_{m-k-1}\cup add(a_{m-k})\setminus del(a_{m-k}) = s_{m-k}\supseteq c
\end{equation}
which can be deduced to:
\begin{equation}\label{eqn:snk1s}
s_{m-k-1}\supseteq c\setminus add(a_{m-k})
\end{equation}

We then analyze in two cases, whether $c \cap (pre(a_{m-k})\cup add(a_{m-k}) \setminus del(a_{m-k}))=\emptyset$.

(1) $c \cap (pre(a_{m-k})\cup add(a_{m-k}) \setminus del(a_{m-k}))=\emptyset$: From the case premise, $c\cap add(a_{m-k}) = \emptyset$ holds because the add effects and delete effects has no common member (Equation \ref{eqn:a_d}). Therefore Equation \ref{eqn:snk1s} can further be deduced to $s_{m-k-1}\supseteq c$, which means $s_{m-k-1}$ satisfies the condition $c$ of the BT.

(2) $c \cap (pre(a_{m-k})\cup add(a_{m-k}) \setminus del(a_{m-k}))\neq\emptyset$: The transition shown by eqn. \ref{eqn:snk1} indicate  $s_{m-k}\setminus del(a_{m-k})=s_{m-k}$. Therefore when expanding $c$, action $a_{m-k}$ will be explored because the selective condition holds (line \ref{line-ifexpanded}), which creates a condition node $c_{a_{m-k}}=pre(a_{m-k})\cup c \setminus add(a_{m-k})$. The state transition also provides that $s_{m-k-1}\supseteq pre(a_{m-k})$, then with Equation \ref{eqn:snk1s} we can deduce:
\begin{equation}\label{eqn:snk1sp}
	s_{m-k-1}\supseteq pre(a_{m-k}) \cup c\setminus add(a_{m-k})
\end{equation}
Therefore $s_{m-k-1}$ satisfies the condition $c_{a_{m-k}}$. Assume $a_{m-k}$ is in the action space of robot $i$, $a_{m-k}\in \mathcal{A}_i$, then $c_{a_{m-k}}$ will be expanded to its BT $\mathcal{T}_i\in \Phi$.

The two cases complete the proof of the inductive step.

The mathematical induction proves that any state in the sequence must satisfy some condition node expanded in the while-loop. Therefore $s_0$ will satisfy some condition
node $c\in \mathcal{C}_{new}$ after one-step expansion, then the algorithm will return a $\Phi$, which contradicts the initial assumption that a solution $\Phi$ exists but our algorithm fails. 

Therefore if Problem \ref{problem-MRBTP} is solvable, the algorithm returns a solution $\Phi$.
\end{proof}



\section{B.Limitations and Future Work}


\subsection{Assumption}

As mentioned above, we assume that at each time, only one robot with the highest priority can execute an action if its precondition is satisfied. Otherwise, two exceptions may occur:

(1) Deadlocks. For example, at time $t$, the robot $1$ takes an action $a_1$ that changes the state from $s_1$ to $s_2$, but the robot $2$ then takes an action $a_2$ and changes $s_2$ back to $s_1$. This loop will continue indefinitely.

(2) Departures from the ROA. For example, the state is $\{1,2\}$ at time $t$, the robot $1$ attempts to take $a_1$ and the robot $2$ attempts to take $a_2$. The action models are:
\begin{align}
pre(a_1)=\{1,2\}, add(a_1)=\{3\}, del(a_1)=\{1\} \nonumber \\
pre(a_2)=\{1,2\}, add(a_2)=\{4\}, del(a_2)=\{2\} \nonumber
\end{align}

If only $a_1$ is applied, the state will transfer to $\{2,3\}$. If only $a_2$ is applied, the state will transfer to $\{1,4\}$. Both $\{2,3\}$ and $\{1,4\}$ is in the ROA of $\Phi$. However, if $a_1$ and $a_2$ are applied simultaneously, the state will transfer to $\{3,4\}$, which is out of the ROA.

As we demonstrated in the paper, we can always use mechanisms such as deadlock detection and intention sharing during execution, ensuring that this serial execution is only employed in exceptional cases. In the vast majority of cases, robots can execute in parallel safely, so there is no need to worry that this assumption will reduce the efficiency of the robot team.

Nevertheless, we believe it is valuable to explore in future work whether it is possible to design an improved algorithm that addresses the aforementioned issues during the planning process.


\subsection{Optimality}
MRBTP can guarantee finding a solution for Problem \ref{problem-MRBTP}; however, since the cost of actions is not considered, MRBTP does not account for the optimality of the solution obtained. In the future, designing an algorithm for solving the optimal multi-robot BT planning problem is also a valuable research field.

\subsection{Parallelization}
MRBTP can achieve a certain degree of multi-robot parallel execution through intention sharing during execution. However, this parallelism is not carefully scheduled. In the future, we may explore the development of an algorithm that can automatically identify parallelizable subtrees and combine it with a runtime scheduling algorithm to achieve more efficient multi-robot parallelism.


\subsection{Subtree Generation}
Currently, we rely on the task reasoning capabilities of LLMs to generate task-specific subtrees. This approach becomes ineffective when LLMs are unavailable. Therefore, we might explore the development of an efficient subtree pre-planning algorithm that does not depend on LLMs, to quickly obtain useful subtree structures before the formal planning process of MRBTP.



\section{C.Experiments}
% All experiments were conducted on a system equipped with an AMD Ryzen 9 5900X 12-Core Processor, operating at a base clock speed of 3.70 GHz.
\subsection{Computing Infrastructure}
All experiments were conducted on a system equipped with an AMD Ryzen 9 5900X 12-Core Processor (3.70 GHz base clock), 128 GB DDR4 RAM, and running Windows 10, 64-bit.
% The computational experiments were performed on a system equipped with advanced hardware to ensure efficient handling of complex computational tasks. Here are the specifications of the computing infrastructure used in our research:

% \begin{itemize}
%     \item \textbf{Processor:} AMD Ryzen 9 5900X 12-Core Processor, operating at 3.70 GHz
%     \item \textbf{Memory:} 32.0 GB RAM
%     \item \textbf{System Type:} 64-bit operating system, x64-based processor
%     \item \textbf{Operating System:} Windows 10
% \end{itemize}

% These hardware resources provided the necessary computational power required for running extensive simulations and processing large datasets. The system's capabilities were essential for executing the demanding tasks associated with our research.




\subsection{Simulation Scenarios}
% The two simulation scenarios used in our experiments are illustrated in Figure \ref{fig:scenario}. The first scenario, \textit{Warehouse Management}, and the second scenario, \textit{Home Service}, are designed to test the effectiveness of our task-specific subtree pre-planning approach.

% 每个场景有两种仿真模式:计算模拟和仿真模拟。{Warehouse Management, 最多8个机器人在5个房间里整理物资。计算模拟模式下动作谓词有2个为: \constant{OpenRoom} 和  \constant{MovePackage},条件谓词有4个: \constant{IsOpen} , \constant{IsClose} ,\constant{IsHolding},\constant{IsInRoom}。  Home Service 中最多4个机器人在家庭服务,计算模拟模式和仿真模拟下具有相同的动作谓词和条件谓词。The two simulation scenarios used in our experiments are illustrated in Figure \ref{fig:scenario}.,它们在仿真场景模拟下动作谓词和条件谓词见表,
The experimental design encompasses two distinct scenarios, each with two simulation modes: computational and scenario simulations. In the Warehouse Management scenario, up to 8 robots are tasked with organizing supplies across 5 designated rooms. This setup utilizes two action predicates, \constant{OpenRoom} and \constant{MovePackage}, and four condition predicates: \constant{IsOpen}, \constant{IsClose}, \constant{IsHolding}, and \constant{IsInRoom}. Similarly, the Home Service scenario involves up to 4 robots assigned to household tasks, employing the same set of action and condition predicates across both simulation modes. Figure \ref{fig:scenario} depicts the two simulation scenarios, and Table \ref{table:action_predicates} details the action and condition predicates used in the scenario simulation mode.




\begin{figure}[h]
    \centering
    \begin{subfigure}[b]{0.47\linewidth}
        \includegraphics[width=\linewidth]{images/exp_warehouse.pdf}
        \caption{Warehouse Management}
        \label{fig:warehouse}
    \end{subfigure}
    % \hspace{0.04\textwidth} % 调整两个 subfigure 之间的间距
    \begin{subfigure}[b]{0.50\linewidth}
        \includegraphics[width=\linewidth]{images/exp_virtualhome.pdf}
        \caption{Home Service}
        \label{fig:homeservice}
    \end{subfigure}
    \caption{Simulation Scenario Overview}
    \label{fig:scenario}
\end{figure}


\begin{table*}[h]
	\centering
	\caption{Action predicates for different scenarios}
	\label{table:action_predicates}
	\begin{tabular}{>{\bfseries}c c}
		\toprule
		Scenarios & \textbf{Included Action Predicates} \\
		\midrule
		\multirow{2}{*}{\textbf{Warehouse Mangement}} & \constant{GoToInRoom}, \constant{GoBtwRoom}, \constant{PickUp},\\
		 & \constant{PutInRoom}, \constant{PutNearInRoom}, \constant{Toggle} \\
  
		\midrule
  
		\multirow{3}{*}{\textbf{Home Service}} & \constant{Walk},  \constant{LeftPut}, \constant{LeftPutIn}, \\
  
             & \constant{LeftGrab}, \constant{RightGrab} , \constant{Open}, \constant{Close} \\
             
		& \constant{RightPut}, \constant{RightPutIn},  \constant{SwitchOn},\constant{SwitchOff} \\
		
		\bottomrule
	\end{tabular}
\end{table*}



\subsection{Goal Distribution in Datasets}
In the \textit{Effectiveness of Task-Specific Subtree Pre-Planning} experiments, the datasets involved goals that were composed of 1 to 5 conditions. The overall distribution of these conditions across the datasets is presented in Table \ref{tab:condition_distribution}.

\begin{table}[h]
\centering
\small
\caption{Overall distribution of conditions}
\label{tab:condition_distribution}
\begin{tabular}{lc}
\toprule
\textbf{Condition} & \textbf{Overall Percentage} \\
\midrule
IsOpen/Close(obj) & 11.9\% \\
IsSwitchOn/Off(obj) & 19.5\% \\
In(obj,container) & 27.0\% \\
On(obj,surface) & 41.6\% \\
\bottomrule
\end{tabular}
\end{table}

\subsubsection{Action Allocation Process Based on Homogeneity 
($\boldsymbol{\alpha}$) Parameter}
% % 关于 Homogeneity (α) 的动作分配细节。假设有 $n$ 个 robot,与机器人无关的总的动作集合为  大写A,动作a属于大写A,A 的数据为 xx. 我们首先将 大写A 中的每个 动作a 随机分给每个机器人,保证一个动作只分配给一个机器人。接着,我们根据 alpha 的值,计算还需要荣誉分配的动作数量为 xxx = int(xx * alpha) , 然后再将  大写A 中的每个 动作a 每次再都随机分配给 xxx 个机器人,保证分到的机器人不包含前面第一次该动作分到的机器人。
Let $\alpha$ represent the Homogeneity parameter, where $0 \leq \alpha \leq 1$. Assume there are $n$ robots, and let $\mathcal{A}$ denote the set of actions, with each action $a \in \mathcal{A}$. The cardinality of $\mathcal{A}$ is denoted by $|\mathcal{A}| = m$.

The action allocation process is described as follows:

\begin{enumerate}
    \item Initially, each action $a_i \in \mathcal{A}$ is randomly assigned to a unique robot $r_j \in \{r_1, r_2, \dots, r_n\}$, ensuring that each action is allocated to exactly one robot.
    \item The number of additional robots to which each action will be redundantly assigned is determined by $k = \text{int}(m \times \alpha)$.
    \item Each action $a_i$ is then randomly reallocated to $k$ additional robots, ensuring that these robots do not include the one initially assigned to the action.
\end{enumerate}



\subsection{LLM}

\subsubsection{Prompt}

% 我们希望大模型能够根据任务信息给出每个机器人的 goal-related actions,以便于后续进行 subtree pre-planning。输入的 prompt 包括条件谓词、动作谓词,任务信息(目标,初始状态,物体,机器人可执行的动作等)。希望大模型的输出为与机器人数量相同的字典组成的列表,每个字典里可以包含多组 goal-related actions,代表某个机器人的大模型推理的动作。本研究使用了GPT-4模型的支持JSON结构化输出的功能。它支持定义JSON模式,在生成输出时会严格遵循这一模式,确保了输出的一致性和可预测性,简化了后续的数据处理工作。详细的 prompt 见 引用 tcolorbox。

The objective of this study is to employ the GPT-4 model to generate goal-oriented actions for each robot, utilizing detailed task information to support the pre-planning of subtrees. The input prompt is systematically structured in JSON format, encompassing condition predicates, action predicates, and comprehensive task details, including goals, initial states, objects, and the actions available to each robot. The model is programmed to produce a list of dictionaries, also in JSON format, where each dictionary corresponds to a specific robot and contains multiple sets of goal-related actions derived from the model’s inference. This study takes advantage of GPT-4's capability to handle JSON-structured inputs and outputs, allowing for the definition of a precise JSON schema that ensures strict adherence to the predefined format. This method enhances consistency and predictability, while also streamlining subsequent data processing. For a detailed example of the input prompt, refer to \ref{my_prompt}.

\tcbset{colframe=black, colback=white, arc=2mm, boxrule=0.3mm}

\begin{tcolorbox}
% fontupper=\fontspec{Carlito},
[breakable,title=Prompt,label=my_prompt]
\begin{flushleft}
\textbf{[Condition]}\\
IsNear\_self\_<ALL>, IsOn\_<GRABBABLE>\_<SURFACES>, IsIn\_<GRABBABLE>\_<CONTAINERS>,
IsOpen\_<CAN\_OPEN>, IsClose\_<CAN\_OPEN>,
IsSwitchedOn\_<HAS\_SWITCH>, IsSwitchedOff\_<HAS\_SWITCH>
\vspace{0.5cm}

\textbf{[Action]}\\
Walk\_<ALL>, RightGrab\_<GRABBABLE>, LeftGrab\_<GRABBABLE>,
RightPut\_<GRABBABLE>\_<SURFACES>, LeftPut\_<GRABBABLE>\_<SURFACES>,
RightPutIn\_<GRABBABLE>\_<CONTAINERS>, LeftPutIn\_<GRABBABLE>\_<CONTAINERS>,
RightGrabFrom\_<GRABBABLE>\_<CONTAINERS>, LeftGrabFrom\_<GRABBABLE>\_<CONTAINERS>,
Open\_<CAN\_OPEN>, Close\_<CAN\_OPEN>, SwitchOn\_<HAS\_SWITCH>, SwitchOff\_<HAS\_SWITCH>
\vspace{0.5cm}

\textbf{[Example]}\\
\textbf{[Task Information]}\\
\textbf{"goal"}: ["IsOn(mug,nightstand)", "IsSwitchedOn(tablelamp)", "IsOpen(book)"],\\
\textbf{"init\_state"}: ["IsSwitchedOff(tablelamp)", "IsClose(book)", "IsClose(nightstand)"],\\
\textbf{"objects"}: ["mug", "nightstand", "tablelamp", "book", "remotecontrol", "drawer"],\\
\textbf{"action\_space"}: [\\
  {["Walk", "SwitchOn", "Open", "Close"]},\\ % 使用 {} 包裹以避免解析错误
  {["Walk", "RightGrab", "RightPut", "SwitchOn", "RightPutIn"]},\\
  {["Walk", "RightGrab", "RightPut", "SwitchOn", "Open", "Close", "RightPutIn"]}
]

\textbf{[Output]}\\
\textbf{"multi\_robot\_subtree\_ls"}: [\{\\
    "WalkToSwitchOntablelamp": ["Walk(self,tablelamp)", "SwitchOn(self,tablelamp)"],\\
    "WalkToOpenBook": ["Walk(self,book)", "Open(self,book)"],\\
    "WalkToOpenNightstand": ["Walk(self,nightstand)", "Open(self,nightstand)"]\\
  \},\\
  \{\\
    "WalkToPutMugOnNightstand": [ "Walk(self,mug)", "RightGrab(self,mug)", "Walk(self,nightstand)", "RightPut(self,mug,nightstand)" ],\\
    "WalkToSwitchOntablelamp": ["Walk(self,tablelamp)", "SwitchOn(self,tablelamp)"],\\
    "WalkToPutRemoteInNightstand": ["Walk(self,remotecontrol)", "RightGrab(self,remotecontrol)", "Walk(self,nightstand)", "RightPutIn(self,remotecontrol,nightstand)"]\\
  \},\\
  \{\\
    "WalkToPutMugOnNightstand": [ "Walk(self,mug)", "RightGrab(self,mug)", "Walk(self,nightstand)", "RightPut(self,mug,nightstand)" ],\\
    "WalkToSwitchOntablelamp": [ "Walk(self,tablelamp)", "SwitchOn(self,tablelamp)"],\\
    "WalkToOpenBook": ["Walk(self,book)", "Open(self,book)" ],\\
    "WalkToPutRemoteInNightstand": ["Walk(self,remotecontrol)", "RightGrab(self,remotecontrol)", "Walk(self,nightstand)", "RightPutIn(self,remotecontrol,nightstand)"],\\
    "WalkToOpenNightstand": [ "Walk(self,nightstand)", "Open(self,nightstand)"] \}]
\vspace{0.5cm}

\textbf{[System]}\\
	\begin{enumerate}[leftmargin=*,topsep=0pt, partopsep=0pt, parsep=0pt, itemsep=1pt]
	 \item For each task, generate all possible composite actions for each robot based on its goals, initial state, and available action space. Repetition of composite actions is permissible. 

	 \item \textbf{[multi\_robot\_subtree\_ls]} is a list where each entry is a dictionary [subtree\_dict] containing all task-related composite actions for a robot. Keys in [subtree\_dict] are composite action names, and values are sequences of atomic actions, ordered such that each action’s effect serves as the precondition for the next. Using the current [Task Information] and [Example], construct \textbf{[multi\_robot\_subtree\_ls]} for each robot.
	 \item The length of \textbf{[multi\_subtree\_list]} equals the number of robots and corresponds to the number of action lists in [action\_space]. With \textbf{\{num\_agent\}} robots, \textbf{[multi\_subtree\_list]} contains \textbf{\{num\_agent\}} dictionaries, each with 1-5 key-value pairs.
	\end{enumerate}



\end{flushleft}
\end{tcolorbox}


\subsubsection{Automatic Reflective Feedback}

% 对于大模型输出的结果,We designed an automated checker to verify the LLM output, providing feedback in cases of grammar errors, action sequences that cannot be pre-planned into a subtree, or an insufficient number of action sequences for each robot.
 To validate the outputs generated by the LLM, we developed an automated checker that provides feedback in cases of grammatical errors, action sequences that cannot be pre-planned into a subtree, or an insufficient number of action sequences generated for each robot. \\
 % 假设两个机器人都在0号房间,钥匙也在0号房间,钥匙可以打开通往1号房间的门,它们的目标是把球搬到1号房间。LLM 能够准确理解任务意图,推理除了两个 action sequences ,分别是 \constant{UseKeyToOpenDoor} 和 \constant{MoveBallBetweenRooms}, 然而其中存在grammar errors,存在错误的物体、动作谓词,或者动作的参数缺失等。此外,"action_space" 列表包含了两个列表,说明有2个机器人,而LLM只给出了一个机器人的action sequences字典。我们的自动检测程序能够解析和捕捉这些错误,并转其拼接为自然语言反馈给LLM。接着LLM改进回答,再次输出,此时没有再出现语法错误,也成功输出了分别代表两个机器人"action_space"  的2个字典。接着,将收集所有这些 action_space ,尝试依次运行 subtree pre-planning 算法生成 subtree,但是这次在生成过程中,我们发现 \constant{UseKeyToOpenDoor} 运行失败,在 \constant{PickUp(self,ball-0)} 之后无法再反向扩展出下一个动作 \constant{Toggle(self,door-0)}。这个因为反向扩展的条件是  \constant{Toggle(self,door-0)} 的增加效果属于 \constant{PickUp(self,ball-0)} 的 precondition,删除效果不属于precondition。而\constant{Toggle(self,door-0)}的增add effects为 \constant{IsOpen(door-0)}, \constant{PickUp(self,ball-0)} 的 precondition 为 \constant{IsHandEmpty(self), IsIn(ball-0,room-0), IsIn(self,room-0)}。因此进一步自动根据这些信息套用subtree规划失败的Reflective Feedback模板,反馈给大模型。最终LLM成功推理了可用的结果。
A specific example is provided in \ref{box:feedback}, where two robots and a key are in Room 0, with the key able to unlock the door to Room 1. The goal is to move a ball from Room 0 to Room 1. The LLM correctly inferred two action sequences: \constant{UseKeyToOpenDoor} and \constant{MoveBallBetweenRooms}. However, the output contained grammatical errors, including incorrect objects, action predicates, and missing action parameters.
Additionally, the \constant{action\_space} list indicated two robots, but the LLM only provided an action sequence dictionary for one robot. Our automated checker detected these issues, generating natural language feedback for the LLM to refine its response. Upon revision, the LLM corrected the grammatical errors and produced two dictionaries representing the \constant{action\_space} for both robots. Subsequently, the pre-planning algorithm attempted to generate subtrees. However, it failed when expanding the \constant{UseKeyToOpenDoor} subtree, as \constant{Toggle(self,door-0)} could not be extended after \constant{PickUp(self,ball-0)}. The failure occurred because the add effects of \constant{Toggle(self,door-0)} (\constant{IsOpen(door-0)}) do not satisfy the preconditions of \constant{PickUp(self,ball-0)} (\constant{IsHandEmpty(self), IsIn(ball-0,room-0), IsIn(self,room-0)}). The automated checker identified this mismatch and applied the reflective feedback template, which was then provided to the LLM for further refinement. Ultimately, the LLM successfully inferred a valid sequence.



\begin{tcolorbox}
% fontupper=\fontspec{Carlito},
[breakable,title=An Example of LLM Process with Automatic Reflective Feedback,label=my_feedback,breakable]
\begin{flushleft}
\textbf{[Task Information]}\\
\textbf{"goal"}: ["IsInRoom(ball-0,room-1)"],\\
\textbf{"init\_state"}: ["IsInRoom(ball-0,room-0)", "IsInRoom(key-0,room-0)","IsClose(door-0)"],"IsInRoom(robot-0,room-0)", "IsInRoom(robot-1,room-0)","IsInRoom(robot-1,room-0)","IsHandEmpty(robot-0)","IsHandEmpty(robot-1)"]\\
\textbf{"objects"}: ["key-0", "ball-0", "door-0", "room-0", "room-1"],\\
\textbf{"action\_space"}: [\\
  {["GoToInRoom", "PickUp", "PutInRoom", "Toggle"]},\\ % 使用 {} 包裹以避免解析错误
  {["GoToInRoom", "GoBtwRoom", "PickUp", "PutInRoom"]}]\\
\vspace{0.5cm}
\textbf{[LLM]}\\
\textbf{"multi\_robot\_subtree\_ls"}:[\\
    "UseKeyToOpenDoor": ["GoToInRoom(self,key-0,room-0)", "PickUp\underline{(key-0)}", "GoToInRoom(self,door-0,room-0)", "\underline{Open}(self,door-0)"],\\

     "MoveBallBetweenRooms": ["GoToInRoom(self,ball-0,room-0)", "PickUp(self,\underline{ball-1})", "GoBtwRoom(self,room-0,room-1)","PutInRoom(self,ball-0,room-1)"]\\

  \}]



\vspace{0.2cm}
\tikz{\draw[dashed,line width=1pt] (0,0) -- (\linewidth,0);}\\ 
\vspace{0.2cm}
\textbf{Reflective Feedback:}\newline
% 1. The following objects: "ball-1" do not exist and must belong to the set \textbf{{"objects"}}.
% 2. The following 动作谓词: “PickUp(key-0)'”、"Open" do not exist 或者格式错误 and must belong to the set \textbf{{"action_space"}}.
% 3. The number of robots in this task is \textbf{{num_agent}}, meaning \textbf{[multi\_subtree\_list]} contains \textbf{{num_agent}} dictionaries. Each of the \textbf{{num_agent}} dictionaries in the list should contain 1-4 key-value pairs. Please revise accordingly.
1. The object \textbf{["ball-1"]} does not exist and must be an element of the set \textbf{[objects]}.\\
2. The action predicates \textbf{["PickUp(key-0)","Open"]} either do not exist or are incorrectly formatted. They must be part of the set \textbf{[action\_space]}.\\
3. The number of robots involved in this task is \textbf{\{num\_agent\}}, which implies that the \textbf{[multi\_subtree\_list]} should contain \textbf{\{num\_agent\}} dictionaries. Each of these dictionaries should have 1-4 key-value pairs. Please revise accordingly.\\


\vspace{0.5cm}

\textbf{[LLM]}\\
\textbf{"multi\_robot\_subtree\_ls"}: [\{\\
    "UseKeyToOpenDoor": ["GoToInRoom(self,key-0,room-0)", "PickUp(self,key-0)", "GoToInRoom(self,door-0,room-0)", "Toggle(self,door-0)", "\underline{PickUp(self,ball-0)}"]\\
  \},\\
  \{\\
    "MoveBallBetweenRooms": ["GoToInRoom(self,ball-0,room-0)", "PickUp(self,ball-0)", "GoBtwRoom(self,room-0,room-1)", "PutInRoom(self,ball-0,room-1)"]\\
  \}]

\vspace{0.2cm}
\tikz{\draw[dashed,line width=1pt] (0,0) -- (\linewidth,0);}\\ 
\vspace{0.2cm}
\textbf{Reflective Feedback:}\newline
%  "UseKeyToOpenDoor" pre-planning subtree 失败。Toggle(self,door-0)' 和 "PickUp(self,ball-0)}" cannot be pre-planned into a subtree。Toggle(self,door-0)'  的 add = ("IsOpen(door-0)"), 不属于 "PickUp(self,ball-0)}"  的pre = ("IsHandEmpty(self)","IsIn(ball-0,room-0)","IsIn(self,room-0)")
% The pre-planning of the subtree \textbf{"UseKeyToOpenDoor"} failed. The actions \textbf{"Toggle(self,door-0)"} and \textbf{"PickUp(self,ball-0)"} cannot be pre-planned into a single subtree. The \textbf{add} condition for \textbf{"Toggle(self,door-0)"}, which is \textbf{("IsOpen(door-0)")}, does not belong to the \textbf{pre} conditions for \textbf{"PickUp(self,ball-0)"}, which are \textbf{("IsHandEmpty(self)", "IsIn(ball-0,room-0)", "IsIn(self,room-0)")}.


1.\textbf{"UseKeyToOpenDoor"} cannot be pre-planned. \\ 
2.\textbf{"Toggle(self,door-0)"} and \textbf{"PickUp(self,ball-0)"} cannot be pre-planned together.\\
3.[Toggle(self,door-0).add] = ("IsOpen(door-0)")\\
4.[PickUp(self,ball-0).pre] = ("IsHandEmpty(self)", "IsIn(ball-0,room-0)", "IsIn(self,room-0)")\\
5.[Toggle(self,door-0).add] condition does not belong to the [PickUp(self,ball-0).pre] conditions.



\vspace{0.5cm}

\textbf{[LLM]}\\
\textbf{"multi\_robot\_subtree\_ls"}: [\{\\
    "UseKeyToOpenDoor": ["GoToInRoom(self,key-0,room-0)", "PickUp(self,key-0)", "GoToInRoom(self,door-0,room-0)", "Toggle(self,door-0)"]\\
  \},\\
  \{\\
    "MoveBallBetweenRooms": ["GoToInRoom(self,ball-0,room-0)", "PickUp(self,ball-0)", "GoBtwRoom(self,room-0,room-1)", "PutInRoom(self,ball-0,room-1)"]\\
  \}]

\end{flushleft}
\label{box:feedback}
\end{tcolorbox}


\end{document}



