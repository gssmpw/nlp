\section{Methodology}
In this section, we introduce the Judge-Consistency (\method{}) method, which optimizes Large Language Models (LLMs) as judgment models to evaluate the effectiveness of the Retrieval-Augmented Generation (RAG) system. First, we discuss the evaluation methods used in \method{} (Sec.\ref{sec:3.1}). Next, we present how the \method{} method is utilized to optimize the judgment model, improving its judgment accuracy (Sec.\ref{sec:3.2}). Finally, we apply the \method{} method as the reward model to optimize the RAG system (Sec.~\ref{sec:3.3}). To illustrate equilibria and dynamics of performative prediction games, we focus on a scenario in which a \emph{duopoly} of mortgage companies, i.e. banks, compete to sell loans to customers.

\paragraph{Customer Model:} In our game, each bank is trying to attract customers from a given population $\mathcal{P}$. We model this population as comprised of individuals with a single-dimensional type: we denote individual $j$'s type as $y_j \in [0,1]$. For simplicity, we assume that \(y\) represents the customer’s probability of repaying the loan\footnote{In practice, a customer's (normalized) credit score can be interpreted as a noisy observation of $y_j$. This also corresponds to credit scores being \emph{calibrated}.}, i.e., $y_j := \P[Y_j = 1]$, where $Y_j$ is a random variable such that $Y_j = 0$ means that $j$ defaults on their loan, and $Y_j = 1$ means they repay their loan. Customer types in the population are drawn from a known distribution $D_y$ supported on $[0,1]$. 

\paragraph{Game between Banks:} Each Bank \(i \in \{1, 2\}\) selects two parameters \( (\tau_i, \gamma_i) := \theta_i\), where:
\begin{itemize}
    \item \(\tau_i \in \{\tau_l,\tau_h\}\) is the credit score threshold for approving a customer\footnote{We restrict the bank to only pick between two thresholds, $\tau_l$ and $\tau_h$. However, we highlight how our results are affected when we expand the strategy space to $n > 2$ actions in our experiments of Appendix \ref{app:3gamma}.}. Specifically, a customer $j$ with credit score \(y_j\) is approved by Bank $i$ if and only if \(y_j \geq \tau_i\);
    \item \(\gamma_i \in \{\gamma_l, \gamma_h\}\) is the interest rate offered to approved customers.
\end{itemize}
We denote as shorthand the space of allowable thresholds by $\Gamma := [0,1]$ and allowable interests rates by $\Lambda := [0,1]$. %The latter is set without loss of generality---we simply normalize the rates to be at most $1$. 
% {\color{red} Vidya: just thinking about this but is it natural to restrict interest rate to $1$? I don't think it would affect the equilibrium structure of the game but theoretically I think the interest rate could be anything in $[0,\infty)$.} {\color{green} Guanghui: Could we say something like this is without loss of generality} \gua{changed.}\juba{I think we repeated this twice, the next sentence already had this}
The loan amount is normalized to $1$ in the entire paper, without loss of generality; in this case, if a customer chooses Bank $i$, and the customer is approved by the bank at an interest rate of $\gamma_i$, the expected utility for the bank is equal to
\[
(1+\gamma_i)\cdot \P[Y_i = 1]-\P[Y_i = 0] = (1+\gamma_i)y_i-(1-y_i).
\]


%In practice, the credit score \(y\) serves as a noisy observation of the true likelihood of the customer's repayment. 

\paragraph{Banks' Utilities:} For given parameter choices \(\theta_1 = (\tau_1, \gamma_1)\) by Bank 1 and \(\theta_2 = (\tau_2, \gamma_2)\) by Bank 2, a \emph{rational} customer with credit score $y$ acts as follows:

\begin{enumerate}
    \item \textbf{Qualified for a single bank}: 
        \begin{itemize}
        \item If \(\tau_1 \leq y < \tau_2\), the customer goes to Bank 1, as the score qualifies for Bank 1 but not Bank 2. Conversely, if \(\tau_2 \leq y < \tau_1\), the customer chooses Bank 2.
    \end{itemize}
    \item \textbf{Qualified for both banks}:
     \begin{itemize}
        \item If \(\tau_1, \tau_2 \leq y\) and \(\gamma_1 < \gamma_2\), the customer selects Bank 1 for its lower interest rate. Conversely, if \(\gamma_1 > \gamma_2\), the customer chooses Bank 2.
        \item If \(\gamma_1 = \gamma_2\), the customer picks each bank with probability $1/2$. 
    \end{itemize}
    \item \textbf{Unqualified for both banks}:
    \begin{itemize}
        \item If \(y < \tau_1\) and \(y < \tau_2\), the customer is rejected by both banks.
    \end{itemize}
\end{enumerate}

The expected reward for Bank 1, denoted as \(u_1(\theta_1, \theta_2)\), can then be expressed as:
\begin{align}\label{eq:utility}
    u_1(\theta_1, \theta_2) 
    &=  \mathbb{E}_{y \sim D_y} \left[ \mathbb{I}\{\underbrace{\tau_1 \leq y < \tau_2 \ \cup \ (\tau_1, \tau_2 \leq y \ \cap \ \gamma_1 < \gamma_2)}_{\text{accepted by Bank 1}}\} \cdot \big((1+\gamma_1)y - (1-y)\big) \right] \nonumber\\
    & + \frac{1}{2} \mathbb{E}_{y \sim D_y} \left[ \mathbb{I}\{\underbrace{\tau_1, \tau_2 \leq y \ \cap \ \gamma_1 = \gamma_2}_{\text{accepted by both Banks}}\} \cdot \big((1+\gamma_1)y - (1-y)\big) \right].
\end{align}
Note that the problem is \emph{symmetric}, i.e., the utility function for Bank 2 can be derived by swapping the roles of \(\theta_1\) and \(\theta_2\). I.e., $u_2(\theta_1, \theta_2) = u_1(\theta_2, \theta_1)$. 

% If a bank only attracts customers between thresholds $\tau_a$ and $\tau_b$, for $\tau_a<\tau_b$, we call $[\tau_a,\tau_b]$ the \emph{threshold} range for that bank. For example, if Bank $1$ sets a threshold of $\tau_1$, Bank $2$ a threshold of $\tau_2 > \tau_1$, and $\gamma_1 > \gamma_2$, then Bank 1 has a threshold range of $[\tau_1,\tau_2]$, while bank $2$ has a threshold range of $[\tau_2,1]$.
% Note that the parameters set by \emph{both} banks, i.e. $(\theta_1,\theta_2)$ both influence the threshold range for each of Bank 1 and 2.  If $\tau_1>\tau_2$, $\gamma_1>\gamma_2$, then $\tau_a>\tau_b$, and the bank does not attract any customers. 
% {\color{red} is it possible for $\tau_a > \tau_b$, leading to the bank never attracting customers?} \gua{if $\gamma_1>\gamma_2$, $\tau_1>\tau_2$, then it gets no customer. I think it also makes sense.}\juba{I think we said we wanted to delete the discussion of the threshold range, no?}

% \noindent \textbf{Discrete Model}   
% We now present the discrete version of our model, where the interest rates and thresholds are selected from finite sets \(\Gamma\) and \(\Lambda\), respectively, with $\tau\in[0,1], \gamma\in[0,1]$,  for all $\tau\in\Lambda$ and $\gamma\in\Gamma$, \(|\Gamma| = n\) and \(|\Lambda| = m\). Let \(p_1, p_2 \in \Delta(\Gamma \times \Lambda)\) represent the mixed strategies of the two banks, where \(\Delta(\Gamma \times \Lambda)\) denotes the set of probability distributions over the discrete decision space \(\Gamma \times \Lambda\).


% \begin{Remark}
%    Note that our proposed problem can be reformulated as a standard multi-player performative prediction problem \citep{narang2023multiplayer}. However, in our problem, the data distribution faced by each learner breaks the Lipschitzness assumption of previous work~\citep{hardt2023performative,narang2023multiplayer}. A small modification in one of the learner's thresholds can completely change how demand is allocated across both learners, as is often the case in Bertrand-style games. 
% \end{Remark} 

% \gua{I made some changes to Remark 1, please have a look}
\begin{Remark}
   Previous works in multi-learner performative prediction~\citep{narang2023multiplayer} resort to an insensitivity assumption, i.e., the data distribution faced by each player can only changes slightly when the parameters also change slightly; formally, the data distribution faced by each player is Lipschitz in their decisions. This is immediately not true in our setting: the bank slightly changing its parameters can completely changes the demand distribution of customers it faces. Intuitively, this is because of Bertrand-competition-style effects, where if two banks have similar rates, one bank that lowers their rate by a small amount suddenly captures the entire customer demand that is eligible for that rate.%\juba{made further light edits adding intuition}
   
   In Appendix \ref{Appendix:refumulation}, we discuss this problem more carefully by reformulating our problem in the standard multi-learner performative prediction form given by~\citep{narang2023multiplayer}. We show the distribution is not Lipschitz with respect to the parameters, and thus does not satisfy the insensitivity assumption. 
%Prior work~\citep{hardt2023performative,narang2023multiplayer} showed that, for a general multi-agent performative prediction framework to work, insensitivity assumptions are needed: in the \textbf{worst case}, they can construct settings where the insensitivity assumption does not hold and simple dynamics do not converge anymore. We add nuance to this picture. We will show that our dynamics often converge, even absent insensitivity assumptions, highlighting that while the impossibility results of previous work hold in the worst case, they may not hold in the ``average case'' and especially not in problems motivated by applications. In particular, we will show convergence to a variety of equilibria of our game, and often to symmetric Nash equilibria where insensitivity is immediately violated.
     
\end{Remark}



% \paragraph{Relationship to Performative Prediction} A central point of our work is to highlight that \textcolor{red}{needs writing from intro}. We highlight how our work specifically ties to ``Performative Prediction'' below:


%\textcolor{red}{needs a definition environment}



%Here, \(\E_{\theta_1, \theta_2}\) represents the expected utility of the banks over their respective strategies \((\theta_1, \theta_2)\). These inequalities ensure that neither bank can unilaterally improve its expected utility by deviating from its mixed strategy in the equilibrium.



%and  for all $\tau\in\Gamma$, we have $\tau\in\Lambda$, $(\tau,\gamma)\in[0,1]^2$. Let $\Gamma\times\Lambda$
%In this paper, we focus on the most fundamental case, where there are two choices for each parameter: $0\leq\tau_{\ell}<\tau_{h}\leq 1$, and $0\leq \gamma_{\ell}< \gamma_{h}\leq 1$. In this case, the utility for each pair of decisions forms a $4\times4$ matrix (given in Table \ref{tab:my-table}). We consider the canonical case where $\tau_{\ell}=\frac{1}{2+\gamma_{h}}$, and $\tau_{h}=\frac{1}{2+\gamma_{\ell}}.$ Note that these are natural choices for the thresholds, in the sense that, if there is only one bank and the interest rate is set to be $\gamma$, then $\frac{1}{2+\gamma}$ is the optimal threshold corresponding to the fixed $\gamma$.


%and the thresholds are chosen in $\Lambda=\{\tau^{(1)},\dots,\tau^{(m)}\}$. Here, we only assume that, for each $\gamma\in\Gamma$, there at least exist one $\tau\in\Lambda$ such that $f(\gamma,\tau,1)>0$. Note that this is a very minor assumption, in the sense that, if for a $\gamma$ such that $f(\gamma,\tau,1)<0$ for all $\tau\in\Lambda$, then adopting this decision will lead to negative utility regardless of the opponent's decision, and thus is not an interesting case. 

%\textcolor{red}{The model section is missing the dynamic version of the game. We should clearly define the one-shot and the dynamic game}
% we only considered one-shot case in our paper




\subsection{Preliminary of RAG Evaluation Methods} \label{sec:3.1}
Recent work, such as LLM-as-a-Judge~\cite{zheng2023judging}, typically regards LLMs as judgment models for rating responses in various NLP tasks. These methods use specially designed prompts to employ LLMs in evaluating generated responses across different dimensions, such as hallucination and accuracy~\cite{li2024llms}. In this section, we describe the evaluation dimensions and modeling methods used in \method{}.

\textbf{Evaluation Dimensions.} Some studies~\cite{Rageval2024Zhu,zhang2025rag} evaluate the generation quality of RAG models based on multiple criteria. \method{} follows these approaches by designing different evaluation prompts for four key dimensions: hallucination, completeness, coherence, and semantic consistency.

\textit{Hallucination.} Hallucination refers to the inclusion of information in the response that contradicts the ground truth. This dimension aims to detect whether the generated responses contain factual errors due to hallucinations~\cite{xu2024hallucination}.

\textit{Completeness.} Completeness evaluates whether the generated responses contain as much relevant information as possible from the ground truth. This dimension primarily focuses on identifying whether the responses omit some key points from the ground truth answers.

\textit{Coherence.} Coherence evaluates whether the responses are logically consistent and whether the language flows fluently between sentences. This dimension is primarily concerned with ensuring that the responses are both coherent and fluent.

\textit{Semantic Consistency.} Semantic consistency checks whether the generated response is semantically aligned with the ground truth answer, rather than simply matching it lexically. This dimension helps avoid misjudging responses that differ from the ground truth in terms of tokens but share the same meaning with ground truth answers.

\textbf{Evaluation Modeling.} For evaluation modeling, our \method{} method uses a multiple-choice selection approach~\cite{SurveyonLLM-as-a-Judge2024Gu}. In this method, LLMs evaluate all candidate responses and choose the best or the worst based on different evaluation dimensions. This facilitates RAG training using the judgment model (Sec.~\ref{sec:3.3}).

Existing methods typically rely on Pointwise Evaluation or Listwise Comparison for candidate response evaluation. Pointwise Evaluation directly prompts LLMs to score each candidate based on predefined evaluation dimensions.
However, this method fails to capture the differences between responses, leading to evaluation bias~\cite{kim2023prometheus,wang2023learning}. 
In contrast, Listwise Comparison prompts LLMs to evaluate an entire list of candidate responses and rank them~\cite{niu2024judgerank,yan2024consolidating}, allowing for a more comprehensive evaluation~\cite{li2024llms}. The \method{} model adopts a multiple-choice selection method, which performs Listwise Comparison to evaluate all candidate responses.



% prompting LLMs to score each candidate responses, or the listwise comparison method~\cite{niu2024judgerank,yan2024consolidating}, prompting LLMs to rank the candidate responses and select the best and worst ones. 


%some studies train judgment models using high-quality human-annotated data to align the judgment model with human judgments~\cite{ouyang2022training,Rewardbanch2024Lambert}. However, this method is both expensive and time-consuming~\cite{zhang2025rag}. Inspired by the previous work~\cite{self-improve2023Li,wangself},
% we propose the \method{} method to prompt the judgment model to sample multiple evaluation responses and synthesize the training data based on the consistency among them, thereby applying preference optimization to train the judgment model $R$. 

\subsection{Training Evaluation Models through Judge-Consistency} \label{sec:3.2}
Although LLMs have demonstrated effectiveness in evaluating their own responses, judgment models may still suffer from issues such as Position Bias, Verbosity Bias, and Evaluation Metric Bias, which can compromise the quality of judgments~\cite{zheng2023judging,chen2024humans,li2024llms}. To address these challenges, we propose the Judge-Consistency method to optimize the judgment model $\mathcal{M}$ based on the consistency of judgments across different evaluation dimensions. This process self-improves the judgment model by selecting more suitable evaluation dimensions, ultimately allowing for more precise assessments.


\textbf{Evaluation of LLM Responses.} To optimize the judgment model, we begin by selecting $m$ LLMs and sampling responses from each using different temperatures. Next, we randomly select one response $y$ from each LLM, forming a response set $Y=\{y_1, \dots, y_m\}$. We then combine the four evaluation dimensions introduced in Sec.~\ref{sec:3.1}, generating $k$ distinct hybrid evaluation aspects:
\begin{equation}\small
\mathcal{I} = \{I_1,...,I_k\},
\end{equation}
where $I_i$ represents a hybrid evaluation aspect, which could be a single evaluation dimension or a combination of multiple dimensions. For each evaluation aspect $I_i \in \mathcal{I}$, we create an evaluation prompt, yielding $k$ distinct prompts:
\begin{equation}\small
\mathcal{P} = \{P^1,...,P^k\},
\end{equation}
where $P^i$ is the $i$-th evaluation prompt. The judgment model $\mathcal{M}$ then generates a judgment result $r_i$ for evaluating LLM-generated responses $Y$, based on the $i$-th evaluation aspect $I_i$:
\begin{equation}\small
r_i = \mathcal{M}(P^i,q,y^*, Y),
\end{equation}
where $y^*$ is the ground truth for question $q$. The judgment result $r_i$ includes both the best ($y^+$) and the worst ($y^-$) selections from the candidate responses $Y = \{y_1, \dots, y_m\}$, along with chain-of-thoughts~\cite{Chain-of-Thought2022Wei} for the judgment. By utilizing all evaluation prompts in $\mathcal{P}$, we obtain $k$ judgment results, denoted as $R = \{r_1, \dots, r_k\}$.


\textbf{Judge Consistency Evaluation.} We follow the previous work~\cite{self-improve2023Li} to introduce a ``Judge as a judge'' approach that evaluates the consistency of judgments across different prompts.

Specifically, after obtaining the judgment results $R = \{r_1, \dots, r_k\}$, we use a text embedding model $\text{Emb}(\cdot)$ to compute the similarity score between the $i$-th judgment $r_i$ and all other judgment results $R$. The average of these similarity scores provides the consistency score $S_i$ for the judgment $r_i$:
\begin{equation}\small\label{eq:score}
S_i = \frac{1}{k} \sum_{j=1}^{k} \cos (\text{Emb}(r_i),\text{Emb}(r_j)).
\end{equation}
Judgments exhibiting higher consistency scores are considered positive results, while those with lower consistency scores are interpreted as negative results, indicating potential judge bias.



\textbf{Judgment Model Optimization.} We then optimize the judgment model $\mathcal{M}$ to better select the appropriate evaluation aspect from the set $\mathcal{I}$ to make more accurate judgments.

To achieve this, we treat the judgment with the highest consistency score as a positive judgment $r^+$, and the judgment with the lowest score as the negative one $r^-$. We collect the instance $(q, y^*, r^+, r^-)$ to form the training dataset $\mathcal{T}$. The judgment model $\mathcal{M}$ then uses prompts $\mathcal{P}$ for evaluating the quality of responses $Y$ and is optimized to assign a higher probability to the positive judgment $r^+$ than to the negative judgment $r^-$. This is accomplished through the Direct Preference Optimization (DPO) method~\cite{DPO2023Rafailov}:
\begin{equation}\small\label{eq:dpo}
\begin{aligned}
 & \mathcal{L}= - \mathbb{E}_{(q, y^*, r^+, r^-) \sim \mathcal{T}} [\log \sigma(\beta \\ &\log \frac{\mathcal{M}(r^+ \mid q, y^*)}{\mathcal{M}^\text{ref}(r^+ \mid q, y^*)} - \beta \log \frac{\mathcal{M}(r^- \mid q, y^*)}{\mathcal{M}^\text{ref} (r^- \mid q, y^*)})],
\end{aligned}
\end{equation}
where $r^+, r^- \in R$, and $\beta$ is a hyperparameter. $\mathcal{M}^\text{ref}$ denotes the reference model, which remains fixed during training. 

These judgment results in $R$ are generated using different evaluation prompts $\mathcal{P}$ that combine various evaluation dimensions. \method{} optimizes the judgment model $\mathcal{M}$ to reproduce the positive judgment $r^+$ that shares the most consistency score with others, making the judgment model $\mathcal{M}$ select more appropriate dimensions to evaluate the response quality of LLMs.









% Then, based on Multiple-Choice Selection method, we formulate a multiple-choice question $Q$ by combining the candidate responses $Y$, the question $q$, and the ground truth $q^*$, which requires the LLM to select the best and worst responses from $Y$:


%use $m$ different LLMs to generate $m$ responses $R=\left\{r_1,...,r_m \right\}$ for the given question $q$. 



% Specifically, we use existing QA datasets to feed queries into existing LLMs to obtain multiple responses. In order to increase the diversity of responses,we selected multiple popular models and set different temperatures for each model to sample. This is defined as:
% \begin{equation}\small
% R\ =\ LLM\left( Q,M,T \right) 
% \end{equation}
% where \(Q\) represents query, \(R=\left\{ r_1,r_2,...,r_m \right\} \) represents the \(m\) responses obtained after sending query to LLMs, \(M=\left\{ m_1,m_2,...,m_n \right\} \) represents the \(n\) LLMs we use to generate multiple responses, \(T=\left\{ t_1,t_{2,},...,t_k \right\} \) represents the k different temperatures set when each model is sampled. The relationship between the parameters is: \(m=n\times k\).



% The query in the original QA dataset is used to synthesize multiple different responses, and the answer in original QA pair can be used as the ground truth \(G\). Then the data now becomes a multi-tuple \(<Q,G,r_1,r_2,...,r_m>\). To simplify the task, we only randomly sample one response from a model. So now the data become \(<Q,G,r_1,r_2,...,r_n>\).

% we use the model which will be fine-tuned to evaluate the quality of the responses itself. Since the dimensional preference required by the model for making fine-grained evaluations is not fixed, in order to simulate this preference, we carefully designed several different prompts. Each prompt is a hybrid evaluation of the four evaluation dimensions.
% \begin{equation}\small
% P=\ Combine\left( Hall,Com,Coh,Sem \right) 
% \end{equation}
% In the above formula, \(P=\left\{ p_1,p_2,...,p_n \right\} \) represents the many prompts obtained by free combination. 
% After that, we use the ground truth \(G\) and query \(Q\) as references, and use the model to be fine-tuned to evaluate responses comprehensively through multiple fine-grained evaluation prompts to obtain multiple evaluations \(E\). In all these hybrid evaluations, some of the dimensions used in this evaluation may be truly needed, thus giving a correct evaluation,while some are not, thus may lead to an incorrect evaluation.


% Our goal is to build a preference dataset, so we need to select the best and worst evaluations from the lots of evaluations \(E\). So we adopted the Judge-Consistency mechanism, A correct evaluation will have a higher semantic similarity with most evaluations,conversely,a worse evaluation will have a low semantic similarity with other evaluations.The semantic similarity is calculated as:
% \begin{equation}\small
% f\left( y,\widehat{y} \right) =Sim\left( Emb\left( y \right) ,Emb\left( \tilde{y} \right) \right) 
% \end{equation}
% where \(Emb(y)\) represents the embedding representation of evaluation \(y\), we employ a lightweight MiniCPM-Embedding model to obtain embedding representation. Then, We use the voting mechanism, which prioritizes evaluations that exhibit higher similarity with others. This process can be described simply as:
% \begin{equation}\small
% s\left( y \right) \approx \frac{1}{N}\sum_{i=1}^n{\left[ f\left( y,\tilde{y_i} \right) \right]}
% \end{equation}
% Here, \(s(y)\) is the score assigned to evaluation \(y\). Therefore, we only need to sort the scores \(s(y)\) obtained by each evaluation, select the ones with the highest score as the positive example of the preference dataset, and select the ones with the lowest score as the negative example. So, the preference training dataset will be used to train our judgment model.


\subsection{Applying Judgment Models to Optimize Retrieval-Augmented Generation Systems}
\label{sec:3.3}
To evaluate the effectiveness of judgment model $\mathcal{M}$, we use it as the reward model and apply DPO to optimize the RAG system~\cite{rag-ddr2024Li}.

For a given query $q$, the current RAG system typically utilizes a dense retriever model to retrieve the Top-$n$ relevant documents $\mathcal{D} = \{d_1, \dots, d_n\}$ from the external knowledge bases. The generation model ($\text{Gen}$) then samples outputs $\Tilde{y}$, either with or without the retrieved documents $\mathcal{D}$:
\begin{equation}\small
\begin{aligned}
    \Tilde{y} &\sim \text{Gen} (\mathcal{D} \oplus q), \\
    \Tilde{y} &\sim \text{Gen} (q),
\end{aligned}
\end{equation}
where $\oplus$ is the concatenation operation. Then, we collect all sampled responses $\Tilde{y}$ in the set $\Tilde{Y}$ and utilize the judgment model $\mathcal{M}$ to generate the judgment result $r_\text{all}$ based on all evaluation dimensions:
\begin{equation}\small
r_\text{all} = \mathcal{M}(P_\text{all}, q,y^*,\Tilde{Y}),
\end{equation}
where $P_\text{all}$ indicates the evaluation prompt that involves all evaluation dimensions. Based on $r_\text{all}$, we select the best response $\Tilde{y}^+$ and worst response $\Tilde{y}^-$ from $\Tilde{Y}$, respectively. This allows us to construct a preference dataset $(q, \mathcal{P}, \Tilde{y}^+, \Tilde{y}^-)$ to optimize the generation model ($\text{Gen}$) via DPO training. If the optimized generation model ($\text{Gen}$) demonstrates improved performance on the downstream RAG tasks, this indicates that the judgment model $\mathcal{M}$ provides more precise judgments, effectively serving as the reward for optimizing the RAG system.



%To further improve the effectiveness of the RAG system $V$, some work uses reinforcement learning methods~\cite{RLRAGforchatbots2024Kulkarni}, such as Direct Preference Optimization~\cite{DPO2023Rafailov} (DPO) to optimize the generation module $G$ in RAG system. In these methods, the judgment model $\mathcal{R}$ plays a crucial role, which evaluates the candidate responses of the RAG system and generates reward signals to optimize the LLM $\mathcal{M}$, guiding its output preference toward to responses with higher reward. However, evaluating the quality of the judgment model in the RAG system remains a challenge~\cite{jin2024rag}. Since the quality of the judgment model directly affects the optimization effectiveness of the RAG system $V$, its quality can be evaluated based on how well it optimizes the RAG system.








% Inspired by~\cite{rag-ddr2024Li}, we use DPO to optimize the LLM in the generation module $G$, evaluating the effectiveness of the judgment model $R$. The generation module $G$ utilizes the query $q$ with retrieved documents $D$ or the query alone as inputs and prompts the LLM for sampling:
% \begin{equation}\small
% \Tilde{y} \sim \mathcal{M} (D \oplus q), \Tilde{y} \sim \mathcal{M} (q),
% \end{equation}
% where $\Tilde{y}$ is sampled output. Then, we utilize the judgment model $\mathcal{R}$ to calculate the reward for the $\Tilde{y}$ and judge the outputs with the highest and lowest rewards, $y^+$ and $y^-$, respectively:
% \begin{equation}\small
% (y^+, y^-) = \mathcal{R}(\Tilde{y}).
% \end{equation}
% The LLM $\mathcal{M}$ in generation module $G$ can be trained using the DPO training loss, guiding the LLM to better balance internal and external knowledge by comparative learning from the positive ($\Tilde{y}_t^+$) and negative ($\Tilde{y}_t^-$) outputs:
% \begin{equation}\small\label{eq:dpo2}
% \begin{aligned}
%  & \mathcal{L}= -\mathbb{E}_{(x, \Tilde{y}^+,\Tilde{y}^-) \sim \mathcal{I}} [\log \sigma (\beta \log \frac{\mathcal{M}(\Tilde{y}_t^+ \mid x)}{\mathcal{M}^\text{ref}(\Tilde{y}^+ \mid x)}  \\ &- \beta \log \frac{\mathcal{M}(\Tilde{y}^- \mid x)}{\mathcal{M}^\text{ref} (\Tilde{y}^- \mid x)})],
% \end{aligned}
% \end{equation}


% where $\beta$ is a hyperparameter. $\mathcal{I}$ is the dataset containing the query $q$ and its corresponding preference data pairs $(\Tilde{y}_t^+, \Tilde{y}_t^-)$. $V_t^\text{ref}$ is the reference model, which is frozen during training. By this method, we can obtain the optimized LLM $\mathcal{M}^*$. Then, according to Eq~\ref{eq:rag}, $\mathcal{M}^*$ can generate a response $y$ to answer the given question $q$ with the retrieved documents $D$. Finally, we use the evaluation metric corresponding to question $q$ to compute an evaluation score $S$ for $y$:
% \begin{equation}\small
% S =\text{Metric}(y,y^*),
% \end{equation}
% where $\text{Metric}$ can be evaluation metrics such as accuracy. $y^*$ is the ground truth answer of the question $q$. Finally, we can evaluate the quality of the judgment model based on the score $S$. If the score $S$ is higher, it indicates that the judgment model $\mathcal{R}$ generates more accurate rewards for the sampled output $\Tilde{y}$ and the quality of the judgment model is higher.

 

% Inspired by~\cite{}, we incorporate the judgment model into the optimization of the RAG system and evaluate the quality of the judgment model based on how well the RAG system is optimized.


% primarily of rewarding in rag system.
% %如何评价专门用于RAG的rm性能
% %我们是为了评价rm才做强化学习
% With the rise of reinforcement learning in LLM training, some studies have recognized the potential of reinforcement learning and have attempted to incorporate it into the RAG system~\cite{RewardRAG2024Thang}. In reinforcement learning within RAG system, the judgment model plays a crucial role by providing feedback to the responses generated by $G$, and we can use the feedback to optimize $G$. A good judgment model should give higher rewards to responses that better align with human preferences. However, it is still a challenge that how to evaluate the quality of the feedback provided by the judgment model. ~\cite{rag-ddr2024Li} propose a new RAG optimization method: Differentiable Data Rewards(DDR), DDR uses a judgment model to provide rewards to the texts generated by $G$, and uses the reward signals to build a new preference dataset to optimize $G$. In DDR, the quality of the judgment model effects the optimization of $G$ directly, thus, we can follow the same approach to evaluate the performance of our judgment model.




% \subsection{Optimizing RAG Models Using Reinforcement Learning}
% \label{sec:Optimizing RAG Models Using Reinforcement Learning}

% RAG is widely applied in a variety of real-world scenarios. However, RAG may not fully meet our actual needs in certain scenarios, so additional training methods are required to enhance the performance of RAG systems.Conventional supervised fine-tuning~\cite{Radit2023Lin} methods play a limited role in improving RAG systems. So, ~\cite{rag-ddr2024Li} proposed a new RAG optimization method: Differentiable Data Rewards(DDR).DDR aims to optimize the modules by aligning preferences with RAG system, ensuring that the RAG system provides a higher quality response. To achieve this goal, DDR utilizes system reward \(r\left( x,\tilde{y} \right) \) to train the target module.Specifically,it instructs module to input \(x\) and sample diverse responses \(\tilde{y}\) through diverse settings.
% Then using raw metric to provide system reward and constructing a preference training dataset.
% \begin{equation}\small
% r\left( x,\tilde{y} \right) =Metric\left( x,\tilde{y} \right) 
% \end{equation}
% Finally, this module is trained using the DPO to improve performance.
% The whole process can be iterated multiple times to greatly improve the performance of the RAG system.

% However, using raw metric to evaluate the outputs of RAG systems and serve as reward signals for training has its limitations.So LLM-as-a-judge has become a new evaluation approach. ~\cite{Rageval2024Zhu} proposes a relatively complete RAG evaluation framework. For the generator module in the RAG system, they proposed fine-grained evaluation dimensions that can comprehensively evaluate the quality of the text generated by the generator. Firstly, they use GPT-4o to extract 3-5 keywords from the generated text and the ground truth respectively, and then define three evaluation metrics which utilizes the keywords to judge whether the generated responses are informative, accurate, and relevant.

% Therefore,it is a good choice to optimize RAG models using LLMs as judgment models which have better evaluation capabilities. Specifically,We train a more powerful judge model as a judgment model by aligning the model's dimensional preferences during fine-grained evaluation. Then system reward in RAG system can be provided this judge model:
% \begin{equation}\small
% r\left( x,\tilde{y} \right) =Judge\left( x,\tilde{y} \right) 
% \end{equation}
% Finally, we can use this better system reward to obtain better RAG models.

