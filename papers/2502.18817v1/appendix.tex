\section{Appendix}
\label{sec:appendix}

\begin{table}[t]
\centering
\small
\begin{tabular}{l|r|r|r}
\hline
\textbf{Dataset} & \textbf{Total} & \textbf{Train} & \textbf{Dev}\\
\hline
\rowcolor{gray!8}\multicolumn{4}{l}{\textit{\method{} Training Data}}\\
\hline
FinQA~\shortcite{chen2021finqa} &4,000 &3,800 &200\\
FiQA~\shortcite{maia201818} & 8,000 &7,600 &400\\
MeDiaQA~\shortcite{suri2021mediaqa} & 2,000 &1,900 &100\\
PubMedQA~\shortcite{jin2019pubmedqa}& 2,000 &1,900 &100 \\
ScienceQA~\shortcite{lu2022learn} & 7,095 &6,740 &355\\
NQ~\shortcite{nq2019Kwiatkowski}& 4,000 & 3,800 & 200 \\
ELI5~\shortcite{fan-etal-2019-eli5} & 4,000 &3,800 &200\\
NarrativeQA~\shortcite{kovcisky2018narrativeqa} & 2,000 &1,900 &100 \\
PopQA~\shortcite{mallen2023not} & 14,267 &13,554 & 713\\
CNNSum~\shortcite{wei2024cnnsum} & 8,000 &7,600 &400\\
MuSiQue~\shortcite{trivedi2022musique} & 22,355 &21,237 &1,118\\
\hline
\rowcolor{gray!8}\multicolumn{4}{l}{\textit{RAG Training Data}}\\
\hline
ECQA~\shortcite{ecqa2021Aggarwal}&4,200 &4,000 &200 \\
MARCOQA~\shortcite{bajaj2016ms}&4,200 &4,000 &200 \\
Web Questions~\shortcite{webquestion2014Berant}&3,778 &3,578 &200 \\
WiKiQA~\shortcite{wikiqa2015Bajaj}&1,040 &840 &200 \\

Yahoo!QA&4,200 &4,000 &200 \\
StrategyQA~\shortcite{strategyqa2021Geva}&2,060 &1,860 &200 \\
AQUA-RAT~\shortcite{aquarat2017Ling}&2,727 &2,527 &200 \\
\hline
\end{tabular}
\caption{Statistics of the Data Used for ConsJudge Training and RAG Training.}
\label{table1:reward_and_traindataset}  % 设置表格的标签
\end{table}



\subsection{License}
We show the licenses of the datasets that we use. ELI5 and Yahoo!QA do not report the license of the dataset in the paper or a repository. ELI5 shows its terms of use at website\footnote{\url{https://facebookresearch.github.io/ELI5/}}. Yahoo!QA shows its terms of use at website\footnote{\url{https://tensorflow.google.cn/datasets/community_catalog/huggingface/yahoo_answers_qa}}. We summarize the licenses of the remaining datasets as follows:

All of these licenses and agreements allow their data for academic use: NQ (CC BY-SA 3.0 license); FiQA (CC BY-SA 4.0 license); ScienceQA, CNNSum, MuSiQue, Web Questions and HotpotQA (CC BY 4.0 license); WoW (CC BY-NC license); FinQA, MeDiaQA, PubMedQA, PopQA, MARCOQA, WiKiQA, and StrategyQA (MIT license); NarrativeQA, AQUA-RAT, TriviaQA and ASQA (Apache 2.0 license); ECQA (CDLA-Sharing 1.0 license); 


% nq cc-by-sa-3.0
% Finqa MIT
% Fiqa cc-by-sa-4.0
% mediaqa MIT
% pubmedqa MIT
% sciqa cc-by-4.0
% Eli5 uunknown
% narrativeqa Apache-2.0
% popqa MIT
% CNNsum cc-by-4.0
% Musique cc-by-4.0
% commonsenseqa MIT
% ECQA CDLA-sharing-1.0
% marcoqa MIT
% webquestions cc-by-4.0
% wikiqa MIT
% yahoo unknown
% gsm8k MIT
% mathqa Apache-2.0
% strategyqa MIT
% aquarat Apache-2.0
% triviaqa Apache-2.0
% asqa  Apache-2.0
% hotpotqa cc-by-4.0
% trex cc-by-sa-4.0
% wow cc-by-nc
\begin{figure}[t]
    \subfigure[Llama3-8B.] { 
    \label{fig:llama:consistency:distribution} 
    \includegraphics[width=0.48\linewidth]{image/llama_consistency_distribution.png}}  
    \subfigure[Qwen2.5-14B.] { 
    \label{fig:qwen:consistency:distribution} 
    \includegraphics[width=0.48\linewidth]{image/qwen_consistency_distribution.png}}  
    \caption{Distribution of Judgment Consistency Score of Both Vanilla LLMs and ConsJudge.}
    \label{fig:consistency:distribution}
\end{figure}

% \begin{figure}[t]
    \subfigure[Llama3-8B.] { 
    \label{fig:human_consist_llama} 
    \includegraphics[width=0.48\linewidth]{image/llama_human_consistency.pdf}}  
    \subfigure[Qwen2.5-14B.] { 
    \label{fig:human_consist_qwen} 
    \includegraphics[width=0.48\linewidth]{image/qwen_human_consistency.pdf}}  
    \caption{Judge Agreement Evaluation on RAG Training Dataset. We analyze the agreements of different judgment models with Humans. We use Llama3-8B-Instruct and Qwen2.5-14B-Instruct as the backbone models of \method{}, respectively.}
    \label{fig:human_consist}
\end{figure}
% \begin{figure}[t]
%     \centering
%     \includegraphics[width=0.7\columnwidth]{image/human_consistency.pdf}
%     \caption{Judge Agreement Evaluation for RAG Training. We analyze the agreements of different judgment models with Human.}
%     \label{fig:human_consist}
% \end{figure}
\begin{figure}[t]
    \subfigure[Llama3-8B.] { 
    \label{fig:human_consist_llama} 
    \includegraphics[width=0.48\linewidth]{image/llama_human_consistency.pdf}}  
    \subfigure[Qwen2.5-14B.] { 
    \label{fig:human_consist_qwen} 
    \includegraphics[width=0.48\linewidth]{image/qwen_human_consistency.pdf}}  
    \caption{Judge Agreement Evaluation on RAG Training Dataset. We analyze the agreements of different judgment models with Humans. We use Llama3-8B-Instruct and Qwen2.5-14B-Instruct as the backbone models of \method{}, respectively.}
    \label{fig:human_consist}
\end{figure}
% \begin{figure}[t]
%     \centering
%     \includegraphics[width=0.7\columnwidth]{image/human_consistency.pdf}
%     \caption{Judge Agreement Evaluation for RAG Training. We analyze the agreements of different judgment models with Human.}
%     \label{fig:human_consist}
% \end{figure}
\subsection{Judgment Consistency Score Distribution of Vanilla LLMs and \method{}}
\label{sec:consistency score distribution}
In this section, we further analyze the consistency score distributions of judgments generated by the vanilla LLM and \method{} based on different hybrid evaluation aspects. To construct a dataset for this analysis, we randomly sample 1,000 queries from both HotpotQA and TriviaQA. We employ both vanilla LLM and \method{} to generate judgments for each query, using different hybrid evaluation aspects. Then, we refer to Eq.~\ref{eq:score} to use MiniCPM-Embedding to compute the consistency scores of these judgments.

As shown in Figure~\ref{fig:consistency:distribution}, the results demonstrate that \method{} not only achieves higher consistency scores but also exhibits a more concentrated distribution of consistency scores compared to the vanilla LLM. Notably, \method{} consistently maintains its advantage across LLMs of different scales, highlighting its robust generalization ability.


\subsection{Judge Agreement Evaluation between Human and \method{}}
In this section, we further analyze the agreement between different judgment models and human evaluators.

First, we randomly sample 200 queries from the RAG training dataset to assess the agreement in judgment, aiming to evaluate the effectiveness of \method{} in assisting the RAG training process. We then collect responses from various models and ask both judgment models and human evaluators to assess these responses. As shown in Figure~\ref{fig:human_consist}, we use four different judgment methods: Human, GLM-4-plus~\cite{du2022glm}, \method{}, and Raw Metrics, to select the best response for each query. For human annotators, we also provide them with the instructions shown in Table~\ref{table1:eightprompt} to guide their evaluation.

The Raw Metric exhibits the lowest agreement with the other judgment methods, underscoring that character-matching metrics are inadequate for fairly evaluating generated responses. In contrast, \method{} shows higher agreement with both humans and the superior LLM, GLM-4-plus. Furthermore, the agreement between \method{} and humans is comparable to that between GLM-4-plus and humans. This indicates that \method{} has the ability to produce judgments that are more consistent with human evaluators, making it an effective tool for constructing high-quality preference pairs during training RAG models~\cite{rag-ddr2024Li}.

\begin{table}[t]
\centering
\small
\begin{tabular}{l|l}
\hline
\textbf{Quantity} & \textbf{Hybrid Evaluation Aspects}  \\
\hline
Single & Hallucination, Completeness, \\  
& Coherence, Semantic Consistency \\ \hline
Two & Hallucination + Completeness, \\  
& Coherence + Semantic Consistency \\ \hline

Three & Hallucination + Completeness + \\  
& Semantic Consistency \\ \hline
Four & Hallucination + Completeness +\\  
& Coherence + Semantic Consistency \\ 
\hline
\end{tabular}
\caption{Statistics of the Hybrid Evaluation Aspects.}
\label{table1:hyb_dimensions}  % 设置表格的标签
\end{table}




\subsection{More Experimental Details}
\label{app:dataset details}
In this section, we introduce more details of our experiments. We first show the details of training \method{}. Then, we describe the details of applying \method{} to optimize the RAG model. 

\textbf{\method{} Training.} To construct the \method{} Training dataset, as shown in Table~\ref{table1:reward_and_traindataset}, we collect multiple queries from these datasets and use four different LLMs, MiniCPM-2.4B~\cite{minicpm-2b2024Hu}, MiniCPM3-4B~\cite{minicpm-2b2024Hu}, Llama3-8B-Instruct~\cite{touvron2023llama} and Qwen1.5-14B-Chat~\cite{qwen2.5-14b2023Bai} to generate responses for each query. Specifically, each LLM generates three responses using three different temperatures, 0.5, 0.6, and 0.7 and we randomly sample one response from them, resulting in a total of four responses for each query. Furthermore, we combine the four different evaluation dimensions, resulting in eight hybrid evaluation aspects. As shown in Table~\ref{table1:hyb_dimensions}, these include four individual evaluation dimensions, two combinations of two dimensions, one combination of three dimensions, and one that integrates all evaluation dimensions. For the hybrid evaluation aspects combining two evaluation dimensions, one integrates Coherence and Semantic Consistency, focusing on evaluating the logical coherence and fluency of the response, while another combines Hallucination and Completeness, emphasizing whether the response is factually accurate and complete. For the hybrid aspects involving three dimensions, we exclude the Coherence, as it is less relevant in the RAG scenario than the other evaluation dimensions.


\begin{table*}[t]
\centering
\resizebox{\textwidth}{!}{ 
\begin{tabular}{|l|}
\hline
\rowcolor{gray!8} \textbf{\textit{Dimensions Descriptions}} \\ 
\hline
\textbf{Hallucination}: Hallucination refers to the presence of information in the option that contradicts ground truth, \\
it is an incorrect answer to the question.\\
\textbf{Completeness}: Completeness refers to whether the choice contains as complete information as possible \\
from the ground truth. it did not fully answer the question correctly.\\
\textbf{Coherence}: coherence refers to whether the choice is logically coherent and whether the language between \\each sentence is fluent.\\
\textbf{Semantic Consistency}: Semantic Consistency refers to whether the choice is semantically consistent with \\the ground truth, rather than just having lexical repetition.\\

\hline
\rowcolor{gray!8} \textbf{\textit{Prompt}} \\
\hline
You are an excellent evaluation 
expert. Please select the best answer and the worst answer from four choices \\based on the 
ground truth and the query from the \{\textbf{\textit{Name of the hybrid evaluation aspects.}}\} aspect. \\ 
\{\textbf{\textit{Here is the descriptions of the hybrid evaluation aspects.}}\} \\
Note: your result format must strictly be "COT:\{.there 
is your analysis\}. \\Answer : Best answer:\{a choice 
must be one of[A,B,C,D]\}.\\
Worst answer :\{a choice must be one of[A, B, C, D]\}".\\ Output the content of COT first and 
then output the Answer.\\
Here is the query:\{query\}, Here is the ground truth:\{Ground Truth\}\\
Here is the A choice:\{choiceA\}, Here is the B choice:\{choiceB\},\\
Here is the C choice:\{choiceC\}, Here is the D choice:\{choiceD\}. \\
Result: \\
\hline
\end{tabular}
}
\caption{The Prompt Templates Used in \method{}.}
\label{table1:eightprompt}
\end{table*}
\textbf{RAG Training.} To construct the DPO training data to optimize the RAG models, we employ \method{} to select the best and worst responses from the sampling responses generated by RAG models. As shown in Table~\ref{table1:reward_and_traindataset}, we collect multiple queries from these datasets and use bge-large~\cite{bge_embedding} to retriever top-$5$ relevant documents for each query. To enhance sampling diversity, RAG models generate responses under two different input conditions: the query alone (without RAG) and the query with the top-$5$ retrieved documents. RAG model samples two responses for each different input, yielding a total of four sampled responses. After that, we use the judgment model model to select the best and worst responses from them to construct the DPO training dataset.



\begin{figure*}[t]
    \centering
    \subfigure[MiniCPM-2.4B.] { 
        \label{fig:cpm} 
        \includegraphics[width=0.48\linewidth]{image/cpm_training_and_evaluation_prompt.pdf}
    }  
    \subfigure[Llama3-8B.] { 
        \label{fig:llama3} 
        \includegraphics[width=0.48\linewidth]{image/llama_training_and_evaluation_prompt.pdf}
    }  
    \caption{The Prompt Templates Used in Training Processes of RAG Models.}
    \label{fig:RAGtrain_and_evaluation_prompt}
\end{figure*}

\begin{figure*}[t]
    \centering
    \includegraphics[width=\textwidth]{image/RAGmodel_inference.pdf}
    \caption{The Prompt Templates Used in Evaluation Processes of RAG Models.}
    \label{fig:RAGevaluation_prompt}
\end{figure*}
\begin{figure*}[t]
    \centering
    \includegraphics[width=\textwidth]{image/eval_prompt.pdf}
    \caption{The Prompt Templates Used for GLM-4-plus to Evaluate the Performance of RAG Models on the MARCO QA and WoW Datasets.}
    \label{fig:overall_eval_prompt}
\end{figure*}





\subsection{Prompt Templates Used in Experiment}
\label{sec:prompt details}
In this section, we present the prompt templates used in our experiment. 

First, we present the prompt designed for \method{} to evaluate the responses generated by RAG models, as shown in Table~\ref{table1:eightprompt}. Next, as illustrated in Figures~\ref{fig:RAGtrain_and_evaluation_prompt} and \ref{fig:RAGevaluation_prompt}, we introduce the prompts used for training and evaluating the RAG models. These prompt templates are based on RA-DIT~\cite{Radit2023Lin} and RAG-DDR~\cite{rag-ddr2024Li}, specifically tailored to different LLMs and tasks to facilitate the generation of more effective responses. Additionally, the prompt designed for evaluating the performance of RAG models across the MARCO QA and WoW datasets using the GLM-4-plus model is displayed in Figure~\ref{fig:overall_eval_prompt}. Finally, Figure~\ref{fig:GLM_eval_prompt} presents the prompt used to instruct the GLM-4-plus to compare the judge quality between different judgment models.

\begin{figure*}[t]
    \centering
    \includegraphics[width=\textwidth]{image/GLM_eval.pdf}
    \caption{The Prompt Templates Used for GLM-4-plus to Evaluate the Judge Quality of the Different Judgment Models.}
    \label{fig:GLM_eval_prompt}
\end{figure*}






























%从两个不同维度区分大一点,hallu和comp判断有没有幻觉
%三个的是在RAG场景下
%cohe和sem从文本的角度是否流利,想不想人说的,回答精准性,内部逻辑性,形式层面
%剩下的两个有没有幻觉,语义上