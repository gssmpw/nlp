\section{Related work}
\label{sec:2}

Web Usage Mining is the process of analyzing weblogs to extract meaningful patterns from users' online interactions. The data, such as users' browsing habits, pages visited, and session durations, form the core of WUM's information sources \cite{Yau2020}. This process is supported by machine learning and data mining techniques to handle and analyze large datasets. \cref{tab1} summarizes recent studies on web usage mining, focusing on prediction and anomaly detection, along with the methods and techniques used.


\begingroup
\setlength{\extrarowheight}{3pt}  % Table row spaces
\begin{table*}[htb]
	\setlength{\parindent}{0pt} % Tablo girintisini sıfırla
	\centering
	\caption{Recent studies on web usage mining, prediction, and anomaly detection methods.}
	\label{tab1}
	\small
	\begin{tabular}{ m{0.465\textwidth} m{0.37\textwidth}
		>{\centering\arraybackslash}m{0.035\textwidth} >{\centering\arraybackslash}m{0.035\textwidth}  }
	\toprule
	\textbf{Key Contribution} & \textbf{Methods \& Techniques Used} & \textbf{Year} & \textbf{Ref.} \\ \hline 
	\midrule
	
	Predicts e-commerce users' shopping intentions using LSTM recurrent neural networks. & Prediction, LSTM Recurrent Neural Networks & 2021 & \cite{Diamantaras2021} \\ 
	Provides an analysis of usage patterns and prediction through web usage mining techniques. & Pattern identification, Prediction, Clustering, Classification & 2024 & \cite{Dubey2024} \\ 
	Extracts patterns from proxy logs and predicts website requests. & Prediction, Fuzzy data mining, Fuzzy frequent mining & 2023 & \cite{Gangadwala2023} \\ 
	Proposes a neural network model for web page prediction using adaptive deer hunting and chicken swarm optimization. & Prediction, Apriori Algorithm, Chicken Swarm Optimization, Neural Network & 2022 & \cite{Gangurde2022} \\ 
	Develops a model for anomaly and intrusion detection using multi-demeanor fusion techniques. & Intrusion detection, Anomaly detection, Stochastic Latent Semantic Analysis & 2021 & \cite{Gupta2021} \\ 
	Enhances next-page prediction performance using web graphs and session reconstruction techniques. & Session reconstruction, Prediction, Bayesian network, Complete Session Reconstruction & 2023 & \cite{Jors2023} \\ 
	Uses a context-aware cohesive Markov model and Apriori algorithm for web usage pattern discovery. & Prediction, Cohesive Markov Model, Apriori Algorithm & 2022 & \cite{Luckose2022} \\ 
	Predicts user navigation patterns on websites using web usage mining techniques. & Prediction, Maximum frequent pattern, Classification & 2021 & \cite{Om2021} \\ 
	Proposes a system to predict users' learning styles on e-learning platforms. & Prediction, Spectral Clustering, Quadratic Support Vector Machine (E-SVM) & 2024 & \cite{Prashanth2024} \\ 
	Analyzes user behavior on e-commerce platforms and develops recommendation systems based on the findings. & Prediction, Random Forest classification, Event listeners & 2023 & \cite{Rajapaksha2023} \\ 
	Utilizes web access logs for semantic clustering to improve web prefetching accuracy. & Web prefetching, Prediction, Semantic clustering, Thesaurus (WordNet), SPUDK & 2024 & \cite{Setia2024} \\ 
	Improves user session clustering and prediction using semantic-based web session clustering methods. & Session clustering, Prediction, K-Means, Hierarchical Agglomerative Clust., Semantic distance & 2022 & \cite{Sowmya2022} \\
	
	\bottomrule
\end{tabular}
\end{table*}
\endgroup




In recent years, predictive modeling and anomaly detection have emerged as two prominent application areas within WUM \cite{Marcin2023, Hardik2023}. The main stages of web usage mining consist of data preprocessing, pattern discovery, and pattern analysis \cite{Suguna2017}. The first stage, data preprocessing, is essential for making raw web log data analyzable, involving tasks such as data cleaning, user identification, and session identification \cite{Panwar2018}. However, inaccuracies and errors in web log data can negatively impact the accuracy of analyses. For instance, incorrect session merging or user identification errors can lead to misleading results during modeling processes \cite{Sisodia2018}. Therefore, careful execution of data cleaning and session management processes is critical to the success of WUM.

Predictive modeling is a strategic method used to forecast users' future behaviors. These models utilize large-scale data analysis and machine learning algorithms to predict users' browsing habits, interactions, or purchasing tendencies \cite{Choudhary2023}. Such predictions provide valuable contributions, especially in dynamic fields like e-commerce and online services \cite{Ashraf2020}. On the other hand, anomaly detection identifies activities that deviate from standard user behavior patterns, providing crucial feedback in terms of security and performance \cite{Ilieva2021}. This section will examine recent developments in predictive modeling and anomaly detection within WUM, exploring studies and new approaches to enhance these processes' effectiveness.




\subsection{ Data preprocessing and session identification techniques}

Data preprocessing is critical in web usage mining as it optimizes large datasets and creates a more suitable environment for successful prediction and anomaly detection. Correcting errors in raw data and filtering out irrelevant information improves the accuracy of results significantly obtained from WUM processes. A recent study highlights the importance of data preprocessing in modeling user behavior \cite{Ali2020}. This study analyzed page views during user sessions, enabling more accurate session identification.

Session identification, an essential step in data preprocessing, plays a critical role in accurately analyzing users' navigation behavior. A new method for identifying web user sessions was developed, successfully generating all possible maximal paths \cite{Bayir2022}. This approach enabled more accurate structuring of user sessions, leading to superior results in subsequent page predictions. This process plays a significant role in making predictive modeling more efficient. Similarly, the Online Web Navigation Assistant (OWNA) analyzes real-time data streams during session identification, providing recommendations to users \cite{Ali2021}. This model optimizes user navigation behavior throughout sessions, improving the prediction of their actions on the web.

Tools used for data preprocessing in web usage mining also enhance the efficiency of processes. A hybrid approach has been developed that combines techniques like Ant Colony Optimization and Genetic Algorithm to improve classification accuracy during the data preprocessing stage \cite{Malik2021b}. Such tools filter errors and anomalies in large datasets, contributing to more successful prediction and anomaly detection outcomes.


\subsection{ Predictive modeling techniques}


Predictive modeling is one of the critical techniques used to anticipate users' next steps and predict potential actions on the web. One study demonstrated that the Compact Prediction Tree algorithm offers higher accuracy in predicting web pages than traditional methods such as k-nearest neighbor (k-NN) and decision trees \cite{Mani2020}. Similarly, recent research has highlighted the effectiveness of hybrid machine learning methods like Random Forest and Gradient Boosting in predicting web page transitions by utilizing both static and dynamic page features \cite{Dang2023}. Algorithms like these, employed to predict users' following pages, stand out as some of the most potent approaches in Web Usage Mining (WUM). Another study developed a web session reconstruction algorithm using a dynamic link repository \cite{Jors2023}. In this approach, web sessions were modeled graphically, and Bayesian networks were used to predict the next page, providing a dynamic prediction mechanism to optimize user movements across the web.

Another practical approach for predicting user behavior is the use of fuzzy logic-based algorithms. Using fuzzy data mining, one study analyzed proxy server log files to predict users' subsequent web requests \cite{Gangadwala2023}. By analyzing users' browsing frequency and behaviors, fuzzy association rules were created, enabling the accurate prediction of their next steps. Similarly, picture fuzzy logic was applied in a multi-criteria decision-making framework to evaluate website performance \cite{Karahan2024}. In another approach, fuzzy association rules were extracted from web data using learning automata, where trapezoidal membership functions (TMF) were used to optimize the time users spent on web pages, resulting in improved prediction accuracy \cite{Anari2021}. When classical machine learning methods fall short, these approaches offer a more flexible and adaptive prediction mechanism.

Clustering and classification techniques are also widely used in the predictive modeling process. One study employed hybrid methods combining classification techniques such as Random Forest and genetic algorithms to improve prediction accuracy \cite{Malik2021a}. These hybrid approaches, used in the context of WUM, allow for a more accurate classification of web log data. Similarly, another study used the fuzzy C-means algorithm to cluster user behaviors, making predictions based on these clusters \cite{Serin2022}. Following this line of research, an approach applied K-means and hierarchical clustering algorithms to group web sessions, extracting meaningful patterns from session data and predicting future user actions \cite{Sowmya2022}. Clustering algorithms are particularly effective in grouping user behaviors in large datasets and predicting future trends based on these groups.


\subsection{Anomaly detection approaches}

One key aspect of Web Usage Mining is the detection of anomalous user behaviors. Such anomalies can stem from various sources, including security threats, unusual user activities, or incorrect data inputs. Techniques that combine WUM with anomaly detection not only optimize user behavior analysis but also contribute to enhancing security measures. A modified hybrid method, combining PSO, GA, and K-Means, was developed for anomaly and misuse detection in computer networks \cite{Yuan2022}. This model detects abnormal behaviors in network traffic, allowing for minimizing security vulnerabilities.

Big data analytics plays a crucial role in anomaly detection within web mining. The IRPDP\_HT2 algorithm, developed as a scalable data preprocessing method based on Hadoop MapReduce, enables faster and more efficient detection of anomalies in large datasets \cite{Zhang2023}. Analyzing large-scale data sets for critical tasks such as robot detection proves to be an effective method for identifying abnormal activities. Similarly, dimensionality reduction techniques have been employed to detect anomalies in large datasets, providing scalable solutions for managing large volumes of web data \cite{Liu2022}. These approaches provide scalable solutions for managing large volumes of web data and detecting anomalies.

Hybrid methods are another approach to anomaly detection. A hybrid method combining the Grey Wolf Algorithm and CNN was developed to detect anomalous behavior in network data streams \cite{Wang2023}. Hybrid methods offer more flexible and efficient solutions for anomaly detection by integrating machine learning techniques with traditional approaches such as data compression \cite{Prasanth2015}.