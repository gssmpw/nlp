
% In this section, we present our framework comprising two main stages: generalized class discovery and dynamic learning. In the first stage, we propose a novel generalized class discovery model $f_d$ to cluster unlabeled objects in $\calD^u$ discovered by an open-world instance segmentation model $f_o$ pre-trained using $\calD^l$. In the second stage, we utilize both $\calD^l$ and $\calD^u$ with generated pseudo-labels to train a novel instance segmentation network $f_s$. The overall architecture of the proposed framework is shown in~\fref{fig:overview_training}. 


% We first define the GCD problem in instance segmentation, which aims to discover novel classes and learn to segment instances of both known and novel categories, given labeled and unlabeled data. Then, we present our GCD method including instance-wise temperature assignment (ITA) and soft attention modules (SAM). Lastly, we introduce the reliability-based dynamic learning (RDL) method for training an instance segmentation network using pseudo-labels from the GCD model. An overview of the proposed framework during training is illustrated in~\fref{fig:overview_training}.

\section{Proposed Method}
\label{sec:method}
We first define the GCD problem in instance segmentation, which aims to discover novel classes and learn to segment instances of both known and novel categories, given labeled and unlabeled data. We then present our GCD method in~\sref{sec:class_discovery} and the method for training an instance segmentation network in~\sref{sec:dynamic_learning}. An overview of the proposed framework during training is illustrated in~\fref{fig:overview_training}.



% We assume that a labeled dataset $\calD^l$ and one or more unlabeled datasets $\calD^u$ are given where $\calD^l$ contains images $\{\mI^l\}$, instance-wise class labels $\{\vy^l\}$, and instance-wise mask labels $\{\mM^l\}$ for known object classes $\calC^l$, whereas $\calD^u$ comprises only images $\{\mI^u\}$. The images $\{\mI^u\}$ in $\calD^u$ may contain instances of both the known classes $\calC^l$ and novel classes $\calC^u$. The novel classes are disjoint from the known classes (\ie $\calC^l \cap \calC^u=\emptyset$). Given $\calD^l$ and $\calD^u$, we aim to obtain an instance segmentation network capable of segmenting instances of both the known and novel classes $\calC = \calC^l \cup \calC^u$. For example, given an arbitrary image $\mI$, the network is expected to segment instances of known classes (\eg person and car) and novel classes (\eg $\text{unknown}_1$ and $\text{unknown}_2$).

% \subsection{Problem Formulation}
\subsection{Preliminaries}
\label{sec:prelim}
\noindent \textbf{Problem Formulation}.
We are given a labeled dataset $\calD^l$ and an unlabeled dataset $\calD^u$. $\calD^l$ contains images $\{\mI^l\}$ along with instance-wise class and mask labels ($\{\vy^l\}$, $\{\mM^l\}$) for known classes $\calC^k$, while $\calD^u$ comprises only images $\{\mI^u\}$. Given $\calD^l$ and $\calD^u$, GCD in instance segmentation aims to discover novel categories $\calC^n$ (\ie $\calC^k \cap \calC^n = \emptyset$) and to obtain a model capable of segmenting instances of both the known and novel classes $\calC = \calC^k \cup \calC^n$. Hence, during inference, the network is expected to segment instances of known classes (\eg person and car) as well as novel categories (\eg $\text{unknown}_1$ and $\text{unknown}_2$) given an image $\mI$. The images in $\calD^l$ and $\calD^u$ may contain instances of both the known and novel classes.


\vspace{1mm}
\noindent \textbf{Contrastive Learning for GCD}. 
\cite{vaze2022generalized} introduced a contrastive learning (CL) method for GCD in balanced image classification datasets. They first pre-trained a backbone with DINO~\cite{dino2021} on the ImageNet dataset~\cite{imagenet2015} without labels. Subsequently, they fine-tuned the backbone and a projection head using supervised CL on the labeled data and unsupervised CL on both the labeled and unlabeled data. 

% While \cite{vaze2022generalized} explored contrastive learning on a curated and balanced dataset, we focus on GCD in instance segmentation, which is naturally imbalanced (\ie certain objects appear more frequently than others).


Following~\cite{wen2023parametric, vaze2022generalized}, our GCD model $f_d(\cdot)$ consists of a backbone $b(\cdot)$ and a projection head $g(\cdot)$. We utilize an MLP for $g(\cdot)$ and a ResNet-50 backbone~\cite{resnet} for $b(\cdot)$, which is pre-trained on the unlabeled ImageNet dataset~\cite{imagenet2015} using DINO~\cite{dino2021}. Additionally, we employ a momentum encoder $f'_d(\cdot)$ whose parameters are momentum-based moving averages of the parameters of $f_d(\cdot)$ during training, following MoCo~\cite{he2020momentum}. We also use a queue to store the embeddings from $f'_d(\cdot)$ for the samples of both the previous and current mini-batches.


% (\ie $\mI_i \in \calI^u_o \cup \calI^l_o$)

% $f'_d(\cdot)$ is utilized to extract consistent representations between mini-batches while training $f_d(\cdot)$. 
% Additionally, we use a queue $\bar{\calZ}$ to store the representations $\vz'_j = f'_d(\mI_j)$ of the samples from both the previous and current mini-batches, generated by the momentum encoder $\bar{f}_d(\cdot)$.

% where $\calZ'$ denotes the queue containing the representations from the momentum encoder; 


Formally, given an image, we generate two views (random augmentations) $\mI_i$ and $\mI'_i$. We then encode them using $f_d(\cdot)$ and $f'_d(\cdot)$ to obtain $\vz_i = f_d(\mI_i) = g(b(\mI_i))$ and $\vz'_i = f'_d(\mI'_i)$, respectively. We store $\vz'_i$ from the samples in the previous and current mini-batches in the queue $\calZ'$. In~\cite{vaze2022generalized}, the unsupervised contrastive loss $\calL^u_{rep}$ and supervised contrastive loss $\calL^s_{rep}$ are computed as follows:
\begin{equation}
\begin{split}
& \calL^u_{rep}:= -\log \frac{\exp(\vz_i^\mathsf{T} \vz'_i/\tau)}{\sum_{\vz'_j \in \hat{\calZ}_i } \exp(\vz_i^\mathsf{T} \vz'_j/\tau)}, \\
& \calL^s_{rep} := - \frac{1}{|\calZ^p_i|}\sum_{\vz'_k \in \calZ^p_i} \log \frac{\exp(\vz_i^\mathsf{T} \vz'_k/\tau)}{\sum_{\vz'_j \in \hat{\calZ}_i } \exp(\vz_i^\mathsf{T} \vz'_j/\tau)}
\label{eqn:cont_loss_prelim}
\end{split}
\end{equation}
where $\hat{\calZ}_i$ represents the set $\calZ'$ excluding $\vz'_i$ (\ie $\hat{\calZ}_i = \calZ' \setminus \vz'_i$); $\calZ^p_i$ denotes the subset of $\calZ'$ containing the representations that belong to the same class as $\mI_i$; $\tau$ is a temperature hyperparameter; and $|\cdot|$ denotes the number of samples in the set.


% $\calL^s_{rep}$ and $\calL^u_{rep}$ are computed on the labeled data and both the labeled and unlabeled data, respectively. 


% Firstly, a source (labeled) dataset $\calD^l$ is given where $\calD^l$ contains the images $\{\mI^l\}$, instance-wise class labels $\{\vy^l\}$, and instance-wise mask labels $\{\mM^l\}$ for known object classes $\calC^l$. Additionally, one or more target (unlabeled) datasets $\calD^u$ are provided where $\calD^u$ comprises images $\{\mI^u\}$ that may contain instances of both the known classes $\calC^l$ and novel classes $\calC^u$. The two sets of classes are disjoint (\ie, $\calC^l \cap \calC^u=\emptyset$). Given both the labeled $\calD^l$ and unlabeled $\calD^u$ datasets, our aim is to train an instance segmentation network that is capable of segmenting instances of both the known and novel classes $\calC = \calC^l \cup \calC^u$. For example, given an arbitrary image $\mI$, the network is expected to segment instances of known classes (\eg person and car) as well as instances of novel classes (\eg $\text{unknown}_1$ and $\text{unknown}_2$). 

% We assume that the number $|\calC^u|$ of novel classes is larger than the number $|\calC^l|$ of known classes. 

% We further assume that the quantity of novel classes $\calC^u$ is an established prior following~\cite{fomenko2022learning}. 

%==============================================================

% check later 0815
% To predict instance masks and classes for both known and novel categories given labeled and unlabeled datasets, we first employ GGNs~\cite{wang2022open} to obtain class-agnostic instance masks for all instances in the unlabeled dataset. We then apply the proposed class discovery method to generate pseudo-class labels for these instances. Subsequently, we train SOLOv2~\cite{wang2020solov2} using the proposed reliability-based dynamic learning approach on the labeled dataset and the unlabeled images with pseudo-labels.


% check later 0815
% While we also generate pseudo-labels for unlabeled images and train an instance segmentation network using them, similar to~\cite{zhao2022novel, riz2023novel} in semantic segmentation, we take into account the long-tail distribution of classes by proposing instance-wise temperature assignment and class-wise reliability-based dynamic learning. In instance segmentation, similar to~\cite{fomenko2022learning}, our method does not require clustering during inference~\cite{weng2021unsupervised}. Different from~\cite{fomenko2022learning}, we do not depend on the pre-selected prior distribution and use self-computed target distribution for optimization. 


% check later 0815
% While \cite{vaze2022generalized, wen2023parametric} used the same constant values for $\tau_u$ and $\tau_s$ across all the data, we propose to assign varying values to each sample based on its headness or tailness.


% check later 0815
% Following~\cite{fomenko2022learning}, we also utilize an online clustering method to avoid the separate semi-supervised clustering step used in~\cite{vaze2022generalized}. However, while \cite{fomenko2022learning} employed online constrained clustering with an experimentally selected target prior distribution, we investigate contrastive learning for clustering.

\subsection{Generalized Class Discovery}
\label{sec:class_discovery}
Given $\calD^l$ and $\calD^u$, we aim to discover novel categories in $\calD^u$ using the knowledge from $\calD^l$. To achieve this, we first train an instance segmentation network using $\calD^l$ and $\calD^u$ to generate class-agnostic instance masks $\mM^u$ for all objects in $\calD^u$. Subsequently, we crop the unlabeled images using $\mM^u$ and the labeled images using the ground-truth masks $\mM^l$. Then, we train a GCD model using the cropped unlabeled images $\calI^u_o$ and the cropped labeled images $\calI^l_o$ with class labels $\vy^l$ to generate pseudo-class labels $\vy^u$ for $\calI^u_o$.

% In this work , 

% Hence, the class discovery model needs to discover novel classes that are distinct from the known classes given instance images of both known and unseen classes. In this work, we investigate a novel generalized class discovery model for a long-tailed and large-scale dataset considering numerous objects in the real world and their long-tail distribution, .

% We train an open-world instance segmentation network $f_o(\cdot)$ using $\calD^l$ and $\calD^u$. We then apply the trained network $f_o$ to the unlabeled images $\mI^u \in \calD^u$ to generate class-agnostic instance masks for the instances of both known and unseen classes $\calC = \calC^l \cup \calC^u$. Specifically, we employ the Generic Grouping Network (GGN)~\cite{wang2022open} for $f_o(\cdot)$. 
\vspace{1mm}
\noindent \textbf{Class-Agnostic Instance Mask Generation}.
Similar to~\cite{fomenko2022learning}, we first train an instance segmentation network $f_o(\cdot)$ to obtain class-agnostic instance masks for both known and unseen classes $\calC = \calC^k \cup \calC^n$. We train a class-agnostic instance segmentation network, the Generic Grouping Network (GGN) from~\cite{wang2022open}, using both $\calD^l$ and $\calD^u$. We experimentally demonstrate that our GCD method is robust when applied to other class-agnostic instance segmentation methods.

% While \cite{fomenko2022learning} trained Mask R-CNN~\cite{he2017mask} using $\calD^l$, w


% , including Mask R-CNN~\cite{he2017mask}, OLN~\cite{Kim2022Learning}, LDET~\cite{Saito2022Learning}, and UDOS~\cite{kalluri2023open}

Once the training terminates, we apply $f_o(\cdot)$ to the images $\mI^u \in \calD^u$ to obtain instance masks $\mM^u$. We then construct an unlabeled object image set $\calI^u_o$ by cropping the rectangular regions of $\mI^u$ based on $\mM^u$. Similarly, a labeled object image set $\calI^l_o$ is prepared by cropping $\mI^l$ in $\calD^l$ based on the mask labels $\mM^l$.

% Additionally, $\mM^u$ is retained to compute a loss during the training of a discovery model.

% We train an open-world instance segmentation network $f_o(\cdot)$ using the labeled dataset $\calD^l$, which contains images and mask/class labels. 

% The images and masks are padded to square shapes and resized to $224 \times 224$ for further processes.








% Then, we train $f_d(\cdot)$ using images $\mI_i$ in $\calI^u_o$ and $\calI^l_o$ (\ie $\mI_i \in \calI^u_o \cup \calI^l_o$), a momentum encoder, and a novel loss function. The loss function is designed to cluster the unlabeled object images $\calI^u_o$ into classes $\calC = \calC^k \cup \calC^n$.

% , by leveraging the semi-supervised $k$-means clustering method in~\cite{vaze2022generalized}.


% Hence, we propose a novel adaptive temperature assignment method that estimates a temperature value for each sample based on its likelihood of belonging to head or tail classes. 

\vspace{1mm}
\noindent \textbf{Contrastive Learning for GCD in Instance Segmentation}. 
We propose a contrastive learning method for GCD in instance segmentation by modifying the losses in~\eref{eqn:cont_loss_prelim}. While these losses are designed for curated and balanced data, instances in typical instance segmentation datasets are naturally imbalanced (\ie certain objects appear more frequently than others). To address the long-tail distribution of instances for each class, we propose to adjust the temperature parameters in $\calL^u_{rep}$ and $\calL^s_{rep}$ for each instance based on its likelihood of belonging to head classes. 


% check later 0815
% \textcolor{red}{more details}. 

\fref{fig:ITA} visualizes $t$-SNE projections of two semantically similar classes: `book' and `booklet'. Previously, \cite{kukleva2023temperature} showed that a low temperature value tends to uniformly discriminate all instances while a high temperature value leads to group-wise discrimination, as shown in~\fref{fig:ITA} (a) and (b). Based on this, \cite{kukleva2023temperature} introduced a cosine temperature scheduling (TS) method to alternate between instance-wise and group-wise discrimination. In contrast, we propose to estimate the headness of each instance and assign high/low temperature values to instances belonging to head/tail class samples. \fref{fig:ITA} demonstrates the superiority of our instance-wise temperature assignment (ITA) method compared to static assignments and TS~\cite{kukleva2023temperature}.

% Specifically, we assign high-temperature values to instances belonging to head classes to relax instance discrimination and emphasize group-wise discrimination, and vice versa.

% In related work, \cite{kukleva2023temperature} introduced a cosine temperature scheduling method to alternate between instance and group-wise discrimination. However, they assigned the same temperature parameter to all samples in each iteration. We experimentally demonstrate the superiority of our temperature assignment method compared to~\cite{kukleva2023temperature} in~\tref{tab:analysis_temperature}.


% The training of the proposed generalized discovery model is illustrated in~\fref{fig:overgcd}. 

% \input{sections/over_gcd}

\begin{figure}[!t] 
\begin{minipage}{0.49\linewidth}
\centerline{\includegraphics[width=0.97\linewidth,height=0.07\textheight]{fig/low_temp.png}}
\end{minipage}
\begin{minipage}{0.49\linewidth}
\centerline{\includegraphics[width=0.97\linewidth,height=0.07\textheight]{fig/high_temp.png}}
\end{minipage}

\begin{minipage}{0.49\linewidth}
\centerline{\footnotesize (a) $\tau=0.07$}
\end{minipage}
\begin{minipage}{0.49\linewidth}
\centerline{\footnotesize (b) $\tau=1$}
\end{minipage}

\begin{minipage}{0.49\linewidth}
\centerline{\includegraphics[width=0.97\linewidth,height=0.07\textheight]{fig/ts_temp.png}}
\end{minipage}
\begin{minipage}{0.49\linewidth}
\centerline{\includegraphics[width=0.97\linewidth,height=0.07\textheight]{fig/optimal_temp.png}}
\end{minipage}

\begin{minipage}{0.49\linewidth}
\centerline{\footnotesize (c) TS}
\end{minipage}
\begin{minipage}{0.49\linewidth}
\centerline{\footnotesize (d) Ours (ITA)}
\end{minipage}
\caption{$t$-SNE visualization of two semantically close classes. Green: single head class (`book'); Blue: single tail class (`booklet').}
\label{fig:ITA}
\end{figure}


In detail, we first compute a headness score $\hat{h}_i$ for each instance $\mI_i$ by estimating the density of the neighborhood of $\vz_i$ in the embedding space. A higher score $\hat{h}_i$ indicates a higher probability of $\mI_i$ belonging to a head class. We then apply a momentum update to the headness scores to enhance the robustness of the estimation. Specifically, $\hat{h}_i$ and the momentum-updated headness score $h_i$ at the $t$-th epoch are computed as follows:
\begin{equation}
\begin{split}
& \hat{h}_i^t := \frac{\sum_{\vz'_j \in \hat{\calZ}^{top_K}_i }\exp(\vz_i^\mathsf{T} \vz'_j)}{\sum_{\vz'_j \in \hat{\calZ}_i} \exp(\vz_i^\mathsf{T} \vz'_j)}, \\
& h_i^t := \rho h_i^{t-1}+(1-\rho) \hat{h}_i^t 
\label{eqn:headness}
\end{split}
\end{equation}
where $\hat{\calZ}_i = \calZ' \setminus \vz'_i$; $\hat{\calZ}^{top_K}_i$ denotes the set containing the $K\%$ most similar representations to $\vz_i$ in $\hat{\calZ}$; $\rho$ denotes a momentum hyperparameter with a value between 0 and 1.


Subsequently, we determine a temperature value $\tau_i$ for each instance $\mI_i$ using $h_i^t$. To avoid extreme values, we constrain $h_i^t$ to fall within the lowest 10\% ($h^{low}$) and highest 10\% ($h^{high}$) of the scores. We then apply min-max normalization to adjust the score to fall within the range between $\tau^{min}$ and $\tau^{max}$. Specifically, the temperature value $\tau_i$ for $\mI_i$ is calculated as follows:
\begin{equation}
\begin{split}
\tau_i := \frac{\bar{h}_i^t - \min(\calH^t)}{\max(\calH^t)-\min(\calH^t)}(\tau^{max} - \tau^{min})+\tau^{min} 
\label{eqn:temperature}
\end{split}
\end{equation}
where $\bar{h}_i^t := \min ( \max (h_i^t, h^{low}), h^{high} )$; $\calH^t$ represents the set containing the headness scores for all samples. For efficiency, $\tau_i$ and $h_i^t$ are updated at every epoch.


The two contrastive losses $\calL^u_{rep}$ and $\calL^s_{rep}$ in~\eref{eqn:cont_loss_prelim} are modified using the estimated instance-wise temperature value $\tau_i$ as follows:
\begin{equation}
\begin{split}
& \calL^u_{rep}:= -\log \frac{\exp(\vz_i^\mathsf{T} \vz'_i/\tau_i)}{\sum_{\vz'_j \in \hat{\calZ}_i } \exp(\vz_i^\mathsf{T} \vz'_j/\tau_i)}, \\
& \calL^s_{rep} := - \frac{1}{|\calZ^p_i|}\sum_{\vz'_k \in \calZ^p_i} \log \frac{\exp(\vz_i^\mathsf{T} \vz'_k/\tau_i)}{\sum_{\vz'_j \in \hat{\calZ}_i } \exp(\vz_i^\mathsf{T} \vz'_j/\tau_i)}_.
\label{eqn:loss_unsupervised}
\end{split}
\end{equation}
% where $\calL^u_{rep}$ is computed using both $\calI^u_o$ and $\calI^l_o$, while $\calL^s_{rep}$ is calculated using $\calI^l_o$ and their class labels $\{\vy^l\}$.

% When the anchor $\mI_i$ has a high probability of belonging to head classes, a high temperature parameter $\tau_i$ is assigned to relax instance discrimination and emphasize group-wise discrimination, and vice versa. 





%\textcolor{blue}{
%In addition to $\calL^u_{cls}$, we compute the typical cross-entropy loss $\calL^s_{cls}$ for supervised learning on $\calD^l$, as follows:
%}
%\begin{equation}
%\begin{split}
%\calL^s_{cls} := \sum_{c=1}^C - y_{ic} \log q_{ic}
%\label{eqn:loss_cross_entropy}
%\end{split}
%\end{equation}
%where $\vy_i$ denotes the one-hot encoded label for $\mI_i$.




%\textcolor{blue}{
%In detail, we first compute class scores by measuring the cosine similarities between the embeddings $\vv_i=b(\mI_i)$ from the backbone of $f_d(\cdot)$ and the class-wise prototypes $\hat{\vv}_c$ and by applying a softmax function. Specifically, the probability $q_{ic}$ of $\mI_i$ belonging to class $c$ is computed as follows:
%}
%\begin{equation}
%q_{ic} := \frac{\exp(\vv_i \cdot \hat{\vv}_c/\tau)}{\sum_{c=1}^C \exp(\vv_i \cdot \hat{\vv}_c / \tau)}
%\label{eqn:class_score}
%\end{equation}
%where $C$ is the total number of prototypes, and $\tau$ is a temperature hyperparameter. Similarly, we compute $\bar{q}_{ic}$ using $\bar{\vv}_i$ from the backbone of the momentum encoder $f'_d(\cdot)$ and $\hat{\vv}_c$.
%
%
%\textcolor{blue}{
%Subsequently, we estimate the auxiliary target class probability $\bar{p}_{ic}$, inspired by~\cite{xie2016unsupervised}, as follows:
%}
%\begin{equation}
%\bar{p}_{ic} := \frac{\bar{q}_{ic}^2 / n_c}{\sum_{c=1}^C \bar{q}_{ic}^2 / n_c}
%\label{eqn:target_class_prob}
%\end{equation}
%where $n_c = \sum_{i} \bar{q}_{ic}^2$. Squaring $\bar{q}_{ic}$ increases reliance on highly confident samples while reducing the impact of low-confidence samples. $n_c$ balances classes with varying sample counts using the samples in the queue.




\vspace{1mm}
\noindent \textbf{Soft Attention Module}. 
Since $\calI^u_o$ and $\calI^l_o$ contain target objects along with background or adjacent objects, we investigate a soft attention module (SAM) to encode object-specific features. Although we generate pseudo-masks $\mM^u$ for $\calD^u$ using $f_o(\cdot)$, directly using these pseudo-masks to encode object-specific features is risky due to noisy boundaries. To address this, we train an efficient attention module using the pseudo-masks $\mM^u$ for $\calD^u$ and ground-truth masks $\mM^l$ for $\calD^l$. We integrate this attention module into every stage of the CNN backbone. Additionally, we utilize pooled feature maps and embedding functions with depth reduction to reduce computational complexity.

% (\ie $\mP_i := \eta_i(p^s_i(\mF))$)

Given a feature map $\mF \in \mathbb{R}^{D \times \bar{H} \times \bar{W}}$ at the end of each stage in the backbone, we first reduce its dimensions to decrease subsequent computations. Specifically, we apply spatial average pooling to $\mF$ with varying receptive fields. Then, we use $M$ embedding functions $\eta_i(\cdot)$ to generate $M$ outputs $\mP_i \in \mathbb{R}^{d \times s_i \times s_i}$. Here, $i$ indexes the $M$ outputs, $d$ represents the reduced depth produced by the embedding functions, and $s_i \times s_i$ denotes the resulting spatial dimension after pooling. Subsequently, we reshape $\mP_i$ into $\hat{\mP}_i \in \mathbb{R}^{d \times s_i^2}$ and concatenate them to obtain $\bar{\mP} \in \mathbb{R}^{d \times (s_1^2 + s_2^2 + \cdots + s_M^2)}$.

Then, we compute a pairwise affinity matrix $\mA$ by projecting $\mF$ using an embedding function $\phi(\cdot)$ and multiplying the result by $\bar{\mP}$ (\ie $\mA := \bar{\mP}^T \phi(\mF)$). The matrix $\mA \in \mathbb{R}^{(s_1^2 + s_2^2 + \cdots + s_M^2) \times \bar{H} \times \bar{W}}$ represents the spatial relations between the pooled feature map $\mP$ and $\mF$. Additionally, we project $\mF$ using a function $\psi(\cdot)$ and apply global average pooling along the channel dimension, generating a map $\mG$.


Finally, we concatenate $\mA$ with $\mG$ and apply an embedding function $\nu(\cdot)$ followed by a sigmoid function to obtain an attention map $\mS \in \mathbb{R}^{1 \times \bar{H} \times \bar{W}}$. This map $\mS$ is then element-wise multiplied with each channel of $\mF$ to produce the output $\mO \in \mathbb{R}^{D \times \bar{H} \times \bar{W}}$ of the attention module.
\begin{equation}
\begin{split}
\mO := \mS \odot \mF := \sigma( \nu ( [\mA, \mG] ) ) \odot \mF
\label{eqn:soft_attention}
\end{split}
\end{equation}
where $\odot$ and $\sigma(\cdot)$ denote element-wise multiplication and the sigmoid function, respectively; $[\cdot, \cdot]$ represents concatenation. Each of the embedding functions ($\eta(\cdot)$, $\psi(\cdot)$, $\phi(\cdot)$) consists of a $1 \times 1$ convolution layer, batch normalization, and ReLU activation. In comparison, $\nu(\cdot)$ contains only a $1 \times 1$ convolution layer and batch normalization.


% On the other hand, when $\mI_i$ has a high probability of belonging to tail classes, a low value is assigned to $\tau_i$ to focus on instance discrimination. We experimentally demonstrate the effectiveness of the proposed adaptive temperature assignment method in separating classes with a long-tail distribution. 

To train the soft attention modules, we use object masks $\mM^u$ from $f_o(\cdot)$ for $\calD^u$ and ground-truth masks $\mM^l$ for $\calD^l$. Because the pseudo-masks for $\calD^u$ are noisy, especially near object boundaries~\cite{wang2022noisy}, we utilize a weight map $\mW$ to reduce reliance on these regions. Specifically, the attention loss $\calL_{att}$ is computed as follows:
\begin{equation}
\begin{split}
& \calL_{att} := \frac{1}{HW} \sum_{i=1}^H \sum_{j=1}^W \mW_{ij} \norm{\mS_{ij}-\mM_{ij}}^2_2, \\
\end{split}
\end{equation}
where $\mW_{ij}$ is set to $w$ if $d_{ij} \leq \bar{d}$ and $\mM \in \{\mM^u\}$, and to 1 otherwise. Here, $d_{ij}$ denotes the Euclidean distance from ($i, j$) to the nearest object boundary; $\bar{d}$ is a hyperparameter that defines the boundary regions; and $w$ is a weighting coefficient ($\leq 1$). Since SAM is applied after each stage of the backbone, $\calL_{att}$ is obtained by averaging all the corresponding losses.

%  that varies depending on the stage in the backbone

\begin{figure}[!t] 
\begin{minipage}{0.24\linewidth}
\centerline{\includegraphics[width=0.97\linewidth,height=0.06\textheight]{fig/1fore_ground11.png}}
\end{minipage}
\begin{minipage}{0.24\linewidth}
\centerline{\includegraphics[width=0.97\linewidth,height=0.06\textheight]{fig/1back_ground11.png}}
\end{minipage}
\hspace{0.1mm}
\begin{minipage}{0.24\linewidth}
\centerline{\includegraphics[width=0.97\linewidth,height=0.06\textheight]{fig/1fore_ground1.png}}
\end{minipage}
\begin{minipage}{0.24\linewidth}
\centerline{\includegraphics[width=0.97\linewidth,height=0.06\textheight]{fig/1back_ground1.png}}
\end{minipage}

\begin{minipage}{0.486\linewidth}
\centerline{\footnotesize (a) RGA-S~\cite{zhang2020relation}}
\end{minipage}
\begin{minipage}{0.486\linewidth}
\centerline{\footnotesize (b) Ours (SAM)}
\end{minipage}
\caption{Visualization of the pairwise affinity between the white cross marked location and other pixels.}
\label{fig:SAM}
\end{figure}




\fref{fig:SAM} visualizes the pairwise affinity between a marked position and other pixels. The results show that our method, which uses pooled feature maps, is more robust than the previous approach~\cite{zhang2020relation}.











\vspace{1mm}
\noindent \textbf{Deep Clustering for GCD}. 
To avoid the separate semi-supervised clustering step in~\cite{vaze2022generalized}, we employ a deep clustering method similar to those in~\cite{fomenko2022learning, wen2023parametric}. However, unlike~\cite{fomenko2022learning}, our method does not rely on an experimentally selected target prior distribution. Additionally, it is designed to handle imbalanced data, in contrast to~\cite{wen2023parametric}. Specifically, we use the method from~\cite{Zhang2021Supporting} for clustering $\calI^u_o$, with a minor modification: replacing L2 distance with cosine similarity.

We compute the KL-divergence-based loss $\calL^u_{cls}$ on $\calI^u_o$ for unsupervised clustering in~\cite{Zhang2021Supporting}, as follows:
\begin{equation}
\begin{split}
\calL^u_{cls} := \sum_{c=1}^C \bar{p}_{ic} \log \frac{\bar{p}_{ic}}{q_{ic}}
\label{eqn:loss_KL_divergence}
\end{split}
\end{equation}
where $q_{ic}$ is the probability of $\mI_i$ belonging to class $c$; $\bar{p}_{ic}$ is the auxiliary target class probability; and $C$ is the total number of clusters/classes. % Additional details are provided in the supplementary material.

Independent from deep clustering, we additionally compute the typical cross-entropy loss $\calL^s_{cls}$ on $\calD^l$ for supervised classification, as follows:
\begin{equation}
\begin{split}
\calL^s_{cls} := \sum_{c=1}^C - y_{ic} \log q_{ic}
\label{eqn:loss_cross_entropy}
\end{split}
\end{equation}
where $\vy_i$ denotes the one-hot encoded label for $\mI_i$. 



\vspace{1mm}
\noindent \textbf{Total Loss for GCD}. 
The total loss $\calL_{gcd}$ for $f_d(\cdot)$ is computed as the weighted sum of the two contrastive losses, the two classification losses, and the attention loss, as follows:
\begin{equation}
\calL_{gcd} := \calL_{att} + (1-\lambda) \calL^u_{rep} + \lambda \calL^s_{rep} + (1-\lambda) \calL^u_{cls} + \lambda \calL^s_{cls}
\label{eqn:gcd_loss}
\end{equation}
where $\lambda$ is a hyperparameter used to balance the losses, following~\cite{wen2023parametric}. 



% \begin{table*}[ht!]
    \caption{We compare the results obtained using the sub-hypothesis decomposition method with those obtained without it. The results without hypothesis decomposition are presented at the top of the table, while those with hypothesis decomposition are shown below. Benchmark ARC-ID37.}
    \centering
    \scalebox{0.8}{
    \begin{tabular}{ccc}
\toprule
\textbf{Observations} & \textbf{Rounds} & \textbf{Executable Function} \\ \midrule
\multirow{2}{*}{\thead{\includegraphics[width=4cm]{fig/arc_case2.pdf}}} & \thead{\textbf{Round 1:}\\Step 1: Identify the third row\\Step 2: Check for a non-zero \\ numbers in the third row.\\ Step 3: Replace the number at \\the center position of the sequence. \\ Step 4: Change all numbers \\in rows 4 and 5 to zero.} & \thead{\lstinputlisting[language=Python, style=mystyle]{codes/ARC2_round1.py}} \\ \cline{2-3} 
& \thead{\textbf{Round 2:}\\Step 1: Identify the third row\\Step 2: Check for a non-zero\\ numbers in the third row.\\ Step 3: Replace identical numbers\\ in the third row with the\\ corresponding non-zero  number\\ from the first row.} & \thead{\lstinputlisting[language=Python, style=mystyle]{codes/ARC2_round2.py}} \\
\bottomrule
\end{tabular}
    }
    \label{tab:case_study_feed2}
\vspace{-4mm}
\end{table*}
% \input{sections/soft_att}

% During novel class discovery, we generate pseudo-masks $\mM^u$ and pseudo-class labels $\vy^u$ for the target (unlabeled) dataset $\calD^u$. Given the source (labeled) dataset $\calD^l$ containing images $\{\mI^l\}$, ground-truth masks $\{\mM^l\}$, and ground-truth class labels $\{\vy^l\}$ and the target (unlabeled) dataset $\calD^u$ containing images $\{\mI^u\}$, pseudo-masks $\{\mM^u\}$, and pseudo-class labels $\{\vy^u\}$, we train an instance segmentation network that can segment instances of both known and novel classes $\calC = \calC^l \cup \calC^u$. Because the pseudo-labels for the unlabeled dataset may contain inaccurate information, we propose a novel dynamic learning method to rely on accurate pseudo-labels while reducing the dependency on inaccurate pseudo-labels. 


\subsection{Reliability-Based Dynamic Learning}
\label{sec:dynamic_learning}
We generate pseudo-masks $\mM^u$ and pseudo-class labels $\vy^u$ for $\calD^u$ using the method described in~\sref{sec:class_discovery}. Subsequently, we train an instance segmentation network $f_s(\cdot)$ that can segment instances of both known and novel classes using $\calD^l$ and $\calD^u$ with pseudo-labels. To address the issues of inaccurate pseudo-labels and imbalanced instance distributions across classes, we propose a reliability-based dynamic learning (RDL) method. It applies different reliability criteria to each class to avoid excluding all samples from tail classes. Additionally, it adjusts these criteria during training to use diverse data in the early stages while relying only on reliable pseudo-labels in the later stages.


Inspired by~\cite{yang2022st++}, we use holistic stability to measure the reliability of the pseudo-labels. At every fixed number of epochs during training $f_d(\cdot)$, we save the model at that point in time. We then apply these saved models to object images $\mI_i \in \calI^u_o$ to compute the probability $q^{\bar{t}}_{ic}$ of $\mI_i$ belonging to class $c$, where $\bar{t}$ denotes the index of the stored models, ranging from 1 to $\bar{T}$. Subsequently, we compute the stability $s_{i}$ of the probabilities by comparing $q^{\bar{T}}_{ic}$ from the final model with $q^{\bar{t}}_{ic}$ from the intermediate models, as follows:
\begin{equation}
s_{i} = \sum_{\bar{t}=1}^{\bar{T} - 1} \frac{1}{KL(\vq^{\bar{T}}_{i} || \vq^{\bar{t}}_{i})}
\label{eqn:stability}
\end{equation}
where $KL(\cdot||\cdot)$ represents the Kullback-Leibler divergence, and $\vq^{\bar{t}}_{i} \in \mathbb{R}^{|\calC|}$. Since a higher $s_{i}$ indicates greater stability, \cite{yang2022st++} considered pseudo-labels with the lowest $r\%$ scores unreliable.


% $s_{i} = \sum_{\bar{t}=1}^{\bar{T} - 1} \frac{1}{KL(\vq^{\bar{T}}_{i} || \vq^{\bar{t}}_{i})}$


However, applying the same criteria to all data may result in categorizing most samples of a certain class as unreliable. Specifically, for imbalanced data, instances of tail classes tend to have lower $s_{i}$ than those of head classes due to the smaller number of training samples. Additionally, because neural networks tend to first memorize easy samples and then gradually learn harder instances during training~\cite{arpit2017closer}, difficult samples/classes often have lower $s_{i}$ than easier ones. To address these issues, we propose to use class-wise reliability criteria, which consider pseudo-labels with the lowest $r\%$ scores per class as unreliable. 


% and use the rest with the reliability-based adjusting weight $\kappa^t_i$ during training.

% We then assign a discrete value $t_i$ between 0 and $T_d-1$ to each instance $i$ based on 

Additionally, we gradually increase the portion $r\%$ of unreliable samples to initially learn from all data and later optimize using only reliable samples. The idea is that, in the early stages, having a larger number of diverse samples is more important than the accuracy of pseudo-labels, while pseudo-label quality becomes more crucial in later stages. Specifically, we first compute $\bar{t}_i$ for each instance $i$ by finding the rank of $s_i$ among the lowest values within its class. We assign $\gamma$ to $\bar{t}_i$ if its proportional rank falls between $\frac{\gamma}{T_{is}}$ and $\frac{\gamma+1}{T_{is}}$. Then, the reliability-based adjustment weight $\kappa^t_i$ is computed as follows:
\begin{equation}
\kappa^t_i = \sqrt{1-\Big( \max \Big(\frac{ t - \bar{t}_i}{T_{is}}, 0 \Big) \Big)^2} \\
% \kappa^t_i = \sqrt{1-\Big( \max \Big(\frac{\bar{s}^{t}_i - s_i }{T_{is}}, 0 \Big) \Big)^2} \\
\label{eqn:adjusting_weight}
\end{equation}
where $t$ and $T_{is}$ denote the current epoch and the total number of epochs for training $f_s(\cdot)$, respectively.


% When $s_i$ is lower than $\bar{s}^{t}_i$, $\kappa^t_i$ is smaller than 1, reducing reliance on the corresponding instance $i$. 

% $\bar{s}^{t}_i$ denotes the stability threshold for the instance $i$ at the $t$-th epoch;
% 

Finally, we employ SOLOv2~\cite{wang2020solov2} for $f_s(\cdot)$ and use its loss function with modifications for training. First, we replace the focal loss~\cite{lin2017focal} with the equalized focal loss~\cite{li2022equalized}, which performs better on imbalanced data. We use this modified loss $\calL_{s}$ for $\calD^l$ and this loss multiplied by $\kappa^t_i$ for $\calD^u$. Therefore, the total instance segmentation loss $\calL_{is}$ is computed as follows:
\begin{equation}
\begin{split}
% \calL_{is} = & \frac{1}{|\calB^l|} \sum_{i \in \calB^l} \calL_{s}(\mI^l_i,\mM^l_i,\vy^l_i) \\ 
%  & + \frac{1}{|\calB^u|} \sum_{i \in \calB^u} \kappa^t_i \calL_{s}(\mI^u_i,\mM^u_i,\vy^u_i) \\
\calL_{is} = \sum_{i \in \calB^l} \calL_{s}(\mI^l_i,\mM^l_i,\vy^l_i) + \sum_{i \in \calB^u} \kappa^t_i \calL_{s}(\mI^u_i,\mM^u_i,\vy^u_i)
\label{eqn:loss_instance_segmentation}
\end{split}
\end{equation}
where $\calB^l$ and $\calB^u$ denote the sets containing the indices of the data from $\calD^l$ and $\calD^u$, respectively.



% The overall process is described in Algorithm 1. 

% \section{The general case: Proof of \texorpdfstring{\Cref{thm:main-decomp}}{Theorem 1.6}}\label{sec:algo}

First, we show that data structure of \Cref{l:max_min_query} can be used to compute distances witnessed by shortest paths that pass through a constant-size separator.

\begin{lemma}\label{l:single_adhesion}
Fix a constant $k \in \mathbb{N}$. There exists an algorithm which as the input receives an edge-weighted graph $G$ on $n$ vertices and $m$ edges together with a partition of its vertices into three sets $A, B, C$ such that $|B| \leq k$ and there are no edges between $A$ and $C$, and as the output computes $\max_{c \in C} \dist(a, c)$ for every $a \in A$. The running time is $\Oh(m \log n + n \log^{k - 1} n)$.
\end{lemma}

\begin{proof}
Let $B = \{b_1, \ldots, b_k\}$. For any $a \in A, c \in C$, we have $\dist(a, c) = \min_{i \in [k]} \dist(a, b_i) + \dist(c, b_i)$. First, we run Dijkstra's algorithm from every vertex in $B$ to find $\dist(v, b_i)$ for every $v \in V(G)$ and $i \in [k]$. Next, we use \Cref{l:max_min_query} to construct a data structure $\mathbb{D}$ for the point set $\{(\dist(c, b_1), \dots, \dist(c, b_k))\colon c\in C\}\subseteq \mathbb{R}^k$. Now, the value $\max_{c \in C} \dist(a, c)$ for any given $a$ is equal to the answer of $\mathbb{D}$ to the query with argument $(\dist(a, b_1), \dots, \dist(a, b_k))$.
\end{proof}

After computing the distances over a constant-size separator, we will use the following observation to simplify one of the sides of the separation.

\begin{lemma}\label{l:inserting_paths}
Let $G$ be a edge-weighted connected graph and let $A, B, C$ be a partition of its vertices such that there are no edges between $A$ and $C$. For every pair of vertices $u, v \in B$, let $P_{u, v}$ be any shortest path from $u$ to $v$ with all internal vertices in $C$ (assuming such a path exists).

Let $G'$ denote a graph obtained from $G[A \cup B]$ by adding an edge from $u$ to $v$ of weight equal to the length of $P_{u, v}$, for all $u, v \in B$ for which $P_{u, v}$ exists. Then,  $$\dist_G(s, t) = \dist_{G'}(s, t)\qquad\textrm{for all }s,t\in A\cup B.$$
\end{lemma}
\begin{proof}
Let $G''$ be the graph obtained by adding new edges of $G'$ to $G$.
Fix any $s, t \in A \cup B$ and let $P$ denote the shortest path from $s$ to $t$ in $G''$ which minimizes the number of vertices from $C$ visited. Naturally, the weight of $P$ is equal $\dist_G(s, t)$. Assume that such path visits at least one vertex of $C$. Then, the path $P$ is of the form $s \xrightarrow{P_1} x \xrightarrow{P_2} y \xrightarrow{P_3} t$, where $x, y \in B$ and all the internal vertices of $P_2$ are in $C$. By the construction of $G'$, $P_2$ can be replaced with a direct edge from $x$ to $y$ of the same weight. We obtain a same weight path with a smaller number of vertices of $C$ visited, which is a contradiction. Therefore, $P$ is entirely contained in $A \cup B$, hence it exists in $G'$. This shows that $\dist_G(s, t) = \dist_{G'}(s, t)$.
\end{proof}


The next lemma encapsulates the main algorithmic content of the proof of \Cref{thm:main-decomp}. The algorithm will split the tree decomposition provided on input into smaller parts for which the eccentricities are easier to calculate. We use the following lemma to handle a single such part.
\begin{lemma}\label{l:star}
Fix constants $k, g \in \mathbb{N}, 0 < \delta < \frac{1}{54}$. Assume we are given $n \in \mathbb{N}$, an edge-weighted graph $G$ on at most $n$ vertices with a weight function $w \colon E(G) \to \mathbb{N}$, a vertex subset $A$ and a collection of non-empty vertex subsets $V_0, V_1, \dots, V_\ell$ satisfying the following conditions:
\begin{itemize}[nosep]
	\item The sum of weights of all the edges in $G$ is bounded by $\Oh(n)$.
	\item $V(G) \setminus A = V_0 \cup V_1 \cup \dots \cup V_\ell$.
	\item $|A| \leq k$.
	\item For every $i \in [\ell]$, $G[V_i \setminus V_0]$ is connected, $N_G(V_i \setminus V_0) = V_i \cap V_0$, $|V_i| = \Oh(n^\delta)$, and $|V_0 \cap V_i| \leq 4$.
	\item For all $i, j \in [\ell], i \neq j$, $V_i \setminus V_0$ and $V_j \setminus V_0$ are disjoint and non-adjacent in $G$.
	\item Every edge $uv \in E(G)$ with $u, v \not\in A$ is contained in $G[V_i]$ for some $i\in \{0,1,\ldots,\ell\}$.
	\item The graph obtained by taking $G[V_0]$ and adding a clique on $V_0 \cap V_i$ for every $i \in [\ell]$ has Euler genus bounded by $g$.
\end{itemize}
Then, we can compute the eccentricity of every vertex of $G$ in time $\Oh \left( n^{1 + \frac{150 + 54 \delta}{151}} \log^k n \right)$.
\end{lemma}

\begin{proof}
Fix $\delta' = \frac{1 + 97 \delta}{151}$; we have $\delta' - \delta = \frac{1 - 54\delta}{151} > 0$.
Let $E_i$ denote the set of edges with one endpoint in $V_i$ and the other endpoint in $V_i \setminus V_0$. For $i \in [\ell]$, we shall say that $V_i$ is {\em{heavy}} if the sum of weights of $E_i$ is larger than $n^{\delta'}$. Since the sets $E_i$ are pairwise disjoint and the total sum of weights of all the edges is bounded by $\Oh(n)$, the number of heavy subsets is bounded by $\Oh(n^{1 - \delta'})$. Without loss of generality, we may assume that $V_{\ell' + 1}, \dots, V_\ell$ are heavy and $V_1, \dots, V_{\ell'}$ are not, for some $\ell'\in \{0,\ldots,\ell\}$.


For any source vertex $s$, we can calculate distances from $s$ to every vertex of $G$  using breadth first search in time $\Oh(\sum_{e \in E(G)} w(e)) = \Oh(n)$.
In particular, for every $\ell' < i \leq \ell$, we can compute the distances from every vertex of $V_i$ to every vertex of $G$ in total time $\Oh(n^{2 - \delta' + \delta})$, because $$|V_{\ell'+1}\cup \ldots\cup V_{\ell}|\leq n^{1-\delta'}\cdot \Oh(n^\delta)=\Oh(n^{1-\delta'+
\delta}).$$
Additionally, we calculate distances $\dist_G(a, v)$ for every $a \in A, v \in V(G)$ in time $O(n)$.

For every $i \in [\ell]$ and $u,v \in V_0 \cap V_i$, there exists a shortest path $P_{i,u,v}$ from $u$ to $v$ with all internal vertices belonging to $V_i - V_0$ due to the assumption that $G[V_i - V_0]$ is connected and $N_G(V_i - V_0) = V_i \cap V_0$. Therefore, the distance from $u$ to $v$ is bounded by the sum of weights of edges in $E_i$. In particular, for $i \in [\ell']$, $\dist_G(u, v) \leq n^{\delta'}$.

We define $\widetilde{G}$ to be the graph obtained by taking $G[A \cup V_0 \cup \dots \cup V_{\ell'}]$ and applying the following operation for every $i \in \{\ell' + 1, \dots, \ell\}$:
for each pair of vertices $u, v \in A \cup (V_0 \cap V_i)$, add an edge in $\widetilde{G}$ between $u$ and $v$ with weight equal to the total weight of $P_{i,u,v}$. For a fixed $i, u$, we can find $P_{i, u, v}$ for all $v$ using breadth first search in time $\Oh(n)$. Taking a sum over all $i, u$, we get that $\tilde{G}$ can be computed in total time $\Oh(n^{2 - \delta'})$.


\begin{claim}\label{cl:wG}
The sum of the edge weights in $\widetilde{G}$ is $\Oh(n)$. Moreover, for all $u, v \in V(\widetilde{G})$, we have $\dist_{\widetilde{G}}(u, v) = \dist_{G}(u, v)$.
\end{claim}

\begin{proof}
Consider $i \in \{\ell' + 1, \dots, \ell\}$ and any $u, v \in A \cup (V_0 \cap V_i)$ for which we added an edge. Its weight is bounded by the sum of weights of edges in $E_i$. Therefore, the total weight of all edges added is at most
$$
\sum_{i \in \{\ell' + 1, \dots, \ell\}} \left( |A \cup (V_0 \cap V_i)|^2 \sum_{e \in E_i} w(e) \right) \leq (4 + k)^2 \sum_{e \in E(G)} w(e) = \Oh(n).
$$
This proves the first part of the claim.

For the second part of the claim, consider any $i \in \{\ell' + 1, \dots, \ell \}$ and observe that by our assumptions, $A \cup (V_0 \cap V_i)$ separates $(V_0 \cup \dots \cup V_{\ell'} \cup V_{i + 1} \cup \dots \cup V_\ell) \setminus V_i$ from $V_i \setminus V_0$. Hence it suffices to repeatedly apply \Cref{l:inserting_paths}.
\end{proof}

For every $u \in V(\widetilde{G})$, we have $\ecc_G(u) = \max(\ecc_{\widetilde{G}}(v), \max_{v \in V(G) \setminus V(\widetilde{G})} \dist_G(u, v))$. Note, that we already know all the distances $\dist_G(u, v)$ for $v \in V(G) \setminus V(\widetilde{G})$. Similarly, we can already compute $\ecc_G(u)$ for every $u \in V(G) \setminus V(\widetilde{G})$. Therefore, it remains to compute $\ecc_{\widetilde{G}}(v)$ for each $v \in V(\widetilde{G})$. Our goal is to show that this can be done efficiently using \Cref{l:main_ecc}.

Now, let $G'$ be the graph obtained from $\tilde{G}$ by replacing every edge $e$ non-indicent to $A$ with $w(e)\geq 2$ with a path of length $w(e)$ consisting of unit-weight edges. This operation again preserves the distances. Since the sum of edge weights in $\tilde{G}$ is of $\Oh(n)$, the total number of vertices in $G'$ is of $\Oh(n)$. For $0 \leq i \leq \ell'$, we write $V'_i$ to denote the set $V_i$ together with all the vertices added as a part of a path between two endpoints in $V_i$.
As $V_i$ is not heavy for each $i\in [\ell']$, we have
$$
|V'_i \setminus V'_0| \leq |V_i| + \sum_{e \in E_i} w(e) = \Oh(n^{\delta'})\qquad \textrm{for all }i\in [\ell'].
$$

Let $G_0$ denote the graph $G'[V'_0]$ and let $G_0^*$ denote the graph $G'- A$ with $V'_i - V'_0$ contracted to a single vertex $v_i^*$, for each $i \in [\ell']$; note that, all edges of $G_0$ and $G_0^*$ have unit weight.

\begin{claim}
	The graph $G_0^*$ is does not contain $K_{t}$ as a minor, where $t = \Oh(\sqrt{g})$.
\end{claim}

\begin{proof}
Let $\bar{G}_0$ denote the graph obtained by taking $G_0$ and adding a clique on $V_0 \cap V_i$ for every $i \in [\ell']$.
By lemma assumptions and the fact that subdividing edges does not increase the Euler genus, $\bar{G}_0$ has Euler genus at most $g$. In particular, $\bar{G}_0$ is $K_{t'}$-minor-free for some $t' = \Oh(\sqrt{g})$, because the Euler genus of $K_{t'}$ is $\Omega({t'}^2)$.

Similarly, let $\bar{G}_0^*$ be the graph obtained by taking $G_0^*$ and adding a clique on each $V_0 \cap V_i$.
Note, that $\bar{G}_0^* - \{v_1^*, \dots, v_{\ell'}^*\}$ is precisely $\bar{G}_0$. Let $t = \max(t', 6)$.
Recall that a minor model of a clique $K_t$ consists of $t$ pairwise vertex-disjoint connected subgraphs, called
branch sets, such that there is at least one edge between each pair of the branch sets.
Consider a minor model $\varphi$ of $K_{t}$ in $\bar{G}^*_0$.
Note that $\varphi$ cannot contain any singleton branch set of the form $\{v^*_i\}$, for the degree of $v^*_i$ in $\bar{G}^*_0$ is at most $4 < t - 1$. Furthermore, since $N_{\bar{G}^*_0}(v^*_i) = V_0 \cap V_i$, any branch set containing $v^*_i$ and at least one other vertex contains some $u \in V_0 \cap V_i$, and $N_{\bar{G}^*_0}(v^*_i)\subseteq N_{\bar{G}^*_0}(u)$, hence removing $v^*_i$ from this branch set preserves the model. Therefore, we can assume without loss of generality that all branch sets of $\varphi$ are disjoint from $\{v^*_1, \dots, v^*_{\ell'}\}$, hence $\varphi$ is a minor model of $K_{t}$ in $\bar{G}_0$. This is a contradiction, as $t \geq t'$ and $\bar{G}_0$ is $K_{t'}$-minor-free. Therefore, $\bar{G}_0^*$ is $K_t$-minor-free, hence $G_0^*$ also.
\end{proof}

Let $\rho' = \frac{2 - 108 \delta}{151} > 0$. The graph $G^*_0$ is a unit-weight graph and is $K_{t}$-minor-free.
Hence, by applying \Cref{t:r_division} to $G^*_0$ (with $\varepsilon = \rho'/2$)
we obtain an $n^{\rho'}$-division $\mathcal{R}_0$ in time $\Oh(n^{1 + \rho'})$.
We extend it to $G' - A$ by mapping every contracted vertex $v^*_i$ to $N_{G' - A}[V'_i - V'_0] = (V'_i - V'_0) \cup (V_0 \cap V_i)$. Formally, we put $V''_i \coloneqq N_{G' - A}[V'_i - V'_0]$ and 
$$
\mathcal{R} \coloneqq \left\{ (R_0 \cap V'_0) \cup \bigcup_{i \colon v^*_i \in R_0} V''_i \colon R_0 \in \mathcal{R}_0 \right\}.
$$

Now, we argue that $\mathcal{R}$ is a reasonable division of $G' - A$. Clearly, all sets $R \in \mathcal{R}$ are connected in $G' - A$. Pick any $R \in \mathcal{R}$ and let $R_0$ be its corresponding set in $\mathcal{R}_0$.
Every vertex $v^*_i$ is mapped to a set of size $\Oh(n^{\delta'})$, therefore
$$|R| \leq |R_0| \cdot \Oh(n^{\delta'}) = \Oh(n^{\rho' + \delta'}).$$

By our construction, for every $i \in [\ell']$, $R$ is either disjoint from $V'_i - V'_0$ or contains whole $N_{G' - A}[V'_i - V'_0]$. This means that no vertex belonging to any $V'_i - V'_0$ can be in $\partial R$, hence $\partial R \subseteq V'_0$.

Pick any $u \in \partial R \cap R_0$. Assume that $u \not\in \partial R_0$. Then every vertex of $N_{G_0^*}(u)$ must be in $R_0$, hence $N_{G - A'}(u) \subseteq R$, which is a contradiction. This means that $\partial R \cap R_0 \subseteq \partial R_0$.

Pick any $u \in \partial R - R_0$. Then, $u \in V_0 \cap V_i$ for some $i \in [\ell']$ such that $v_i^* \in R_0$. Moreover, $v_i^* \in \partial R_0$ and is adjacent to $u$ in $G_0^*$. The number of such $u$ is bounded by $4 |\partial R_0 \cap \{ v_1^*, \dots, v_{\ell'}^* \}|$.

Putting two cases together, we obtain:
$$
\sum_{R \in \mathcal{R}} |\partial R| = \sum_{R \in \mathcal{R}} \left(|\partial R \cap R_0| + |\partial R - R_0|\right) \leq \sum_{R_0 \in \mathcal{R}_0} \left(|\partial R_0| + 4 |\partial R_0 \cap \{ v_1^*, \dots, v_{\ell'}^* \}|\right) = \Oh(n^{1 - \frac{1}{2}\rho'}).
$$

It remains to show the following claim.

\begin{claim}
Pick any $R \in \mathcal{R}, s_R \in R$. The number of different distance profiles on $R$ relative to $s_R$ in $G' - A$ is of $\Oh(n^{48\rho' + 54\delta'})$.
\end{claim}
\begin{proof}
We look at every vertex $v \in V(G') \setminus A$ and consider three cases: $v \in R$, $v \in V'_0$, and $v \in V'_i \setminus (V'_0 \cup R)$ for some $i \in [\ell']$. By our construction, $R \cap V'_0$ is non-empty, hence w.l.o.g. we can assume that $s_R \in V'_0$ as whether two vertices have the same profile on $R$ is independent of the choice of the pivot vertex.

In the first case, there are at most $|R| = \Oh(n^{\rho' + \delta'})$ such vertices, hence they realise at most that many profiles.

In the second case, we want to observe that profile of any vertex $v \in V'_0$ on $R$ depends only on its profile on $R \cap V'_0$ (relative to $s_R$). Pick any $t \in R - V'_0$. Then $t \in V'_i - V'_0$ for some $i \in [\ell']$, $V_i \cap V_0 \subseteq R \cap V'_0$, and every path from $v$ to $t$ intersects $V_i \cap V_0$. In particular, distances from $v$ to vertices of $V_i \cap V_0$ determine its distance to $t$, which proves the observation.

Let $\tilde{G}_0$ denote the graph obtained by taking $G'[V'_0]$ and for every $i \in [\ell'], u, v \in V_0 \cap V_i$ adding a disjoint path from $u$ to $v$ of length $\dist(u, v)$. Let $P_i$ denote the vertex set of paths added between $V_0 \cap V_i$. For every $t \in V'_0$ we have $\dist_{G' - A}(v, t) = \dist_{\tilde{G}_0}(v, t)$, so it suffices to bound the number of profiles on $R \cap V'_0$ in $\tilde{G}_0$. By our assumptions, $\tilde{G}_0$ has Euler genus bounded by $g$ and all $P_i$ are of size $\Oh(n^{\delta'})$.

Let $R_0$ be the set of $\mathcal{R}_0$ corresponding to $R$. Let $\tilde{R}_0$ denote the set $(R \cap V'_0) \cup \bigcup_{i : v^*_i \in R_0} P_i$. Such set is connected in $\tilde{G}_0$. Moreover, similarly to $R$, its size is $\Oh(n^{\rho' + \delta'})$. Applying \Cref{thm:distprofiles}, we get that the number of distance profiles on $\tilde{R}_0$ in $\tilde{G}_0$ is $\Oh(n^{12(\rho' + \delta')})$, which also bounds the number of profiles on $R$ in $G' - A$ realised by $V'_0$.

For the third case, assume $v \in V'_i \setminus (V'_0 \cup R)$ for some $i\in [\ell']$. Every path from $v$ to any vertex of $R$ in $G' - A$ intersects $V_i \cap V_0$. Let $v_1, \dots v_p$ be the vertices of $V_i \cap V_0$, where $p \leq 4$. The profile of $v$ on $R$ is then determined by the following:
\begin{itemize}[nosep]
 \item[(a)] the profile of each $v_j$ on $R$,
 \item[(b)] $\dist_{G' - A}(v, v_j) - \dist_{G' - A}(v, v_1)$ for each $2 \leq j \leq p$, and
 \item[(c)] $\dist_{G' - A}(s_R, v_j) - \dist_{G' - A}(s_R, v_1)$ for each $2 \leq j \leq p$ where $s_R$ is some pivot vertex of $R$.
\end{itemize}
By the previous case, the number of distance profiles of each $v_j$ is $\Oh(n^{12(\rho' + \delta')})$. The distances between $v$ and $v_j$ are bounded by $|V'_i|$, hence each quantity described in (b) can take $\Oh(n^{\delta'})$ different possible values. Similarly, since $v_1$ and $v_j$ are connected via $V'_i$, $|\dist_{G' - A}(s_R, v_j) - \dist_{G' - A}(s_R, v_1)| \leq \Oh(n^{\delta'})$. The number of different possible profiles of such $v$ is therefore bounded by $\Oh(n^{48(\rho' + \delta') + 6\delta'}) = \Oh(n^{48\rho' + 54\delta'})$. This finishes the proof of the claim.
\end{proof}

Now we can apply \Cref{l:main_ecc} to graph $G'$ with apex set $A$, $X = V(\widetilde{G})$, and the following constants: $$\rho = \rho' + \delta',\qquad \gamma = 1 - \frac{1}{2}\rho',\quad \textrm{and}\quad \alpha = 48\rho' + 54 \delta'.$$ This allows us to calculate all $V(\widetilde{G})$-eccentricities in $G'$ in time
$$
\Oh \left( \left(
	n^{ 2 - \frac{1}{2} \rho' } +
	n^{ 1 + 48\rho' + 54 \delta' }
\right) \log^k n \right) =
\Oh \left( n^{1 + \frac{150 + 54 \delta}{151}} \log^k n \right).
$$
Since for each $v\in V(\widetilde{G})$ we have $\ecc_{\widetilde{G}}(v) = \max_{u \in V(\widetilde{G})} \dist_{\widetilde{G}}(v, u) = \max_{u \in V(\widetilde{G})} \dist_{G'}(v, u)$, this means that we have successfully computed all the eccentricities in $\widetilde{G}$; and as we argued, this is enough to compute all the eccentricities in $G$ as well.

Finally, the total running time of the algorithm is
$$
\Oh \left( n^{1 + \frac{150 + 54 \delta}{151}} \log^k n + n^{2 - \delta' + \delta} \right) =
\Oh \left( n^{1 + \frac{150 + 54 \delta}{151}} \log^k n \right).
$$\qedhere
\end{proof}


\begin{lemma}\label{l:star2}
Fix constants $k, g \in \mathbb{N}, 0 < \delta < \frac{1}{54}$. Assume we are given $n \in \mathbb{N}$, an edge-weighted graph $G$ on at most $n$ vertices with a weight function $w \colon E(G) \to \mathbb{N}$, a vertex subset $A$ and a collection of non-empty vertex subsets $V_0, V_1, \dots, V_\ell$ satisfying the same conditions as in \Cref{l:star} with the following differences:
\begin{itemize}
	\item we don't require $G[V_i - V_0]$ to be connected and $V_i - V_0$ to be adjacent to whole $V_i \cap V_0$;
	\item instead of $|V_0 \cap V_i| \leq 4$, we require $|V_0 \cap V_i| \leq k$.
\end{itemize}
Then, we can compute the eccentricity of every vertex of $G$ in time $\Oh \left( n^{1 + \frac{150 + 54 \delta}{151}} \log^{k + 5g} n \right)$.
\end{lemma}

\begin{proof}
We will reduce our input to one which will satisfy the conditions of \Cref{l:star}. We start by addressing the adhesions $V_0 \cap V_i$ containing too many vertices.

Let $G_0$ denote the graph $G[V_0]$ with cliques placed at $V_0 \cap V_i$ for every $i \in [\ell]$.
For every $i \in [\ell]$ we repeat the following procedure: while $|V_0 \cap V_i| > 4$,
remove arbitrary $5$ vertices from $V_0 \cap V_i$. Since $|V_0 \cap V_i| \leq k$ for each $i\in [\ell]$,
this procedure can be implemented in total time $\Oh(n)$. As a result, at the end we have $|V_0 \cap V_i| \leq 4$ for all $i \in [\ell]$. Let $M$ be the set of all the removed vertices. By our assumptions, $G_0$ has Euler genus bounded by $g$, hence it cannot contain $g + 1$ pairwise disjoint copies of $K_5$
(as the Euler genus of a graph is the sum of the Euler genera of its 2-connected components~\cite{StahlB77} and $K_5$ is not planar). Each removed quintiple of vertices induces a $K_5$ in $G_0$, hence we have $|M| \leq 5g$. We set $A' = A \cup M$ and may thus assume that $V_i$ is disjoint from $A'$ for all $0 \leq i \leq \ell$.

Now, fix $i \in [\ell]$. Let $C^i_1, \dots, C^i_{r_i}$ denote the connected components of $V_i - V_0$ in $G - A'$. We define $W^i_j := N_{G - A'}[C^i_j]$ for every $j \in [r_i]$. Clearly, all $W^i_j$ induce a connected subgraph of $G$ and satisfy $N_{G - A'}(W^i_j - V_0) = W^i_j \cap V_0$. We put $V'_0 := V_0$ and enumerate
$$
\{V'_1, V'_2, \dots V'_{\ell'}\} := \{ W^i_j \colon i \in [\ell], j \in [r_i] \}.
$$
It is easy to verify that the sets $A'$ and $V'_0, V'_1, \dots, V'_{\ell'}$ satisfy the conditions of \Cref{l:star}. We apply said lemma to calculate the eccentricity of every vertex of $G$ in the desired time.
\end{proof}



The next statement is a reformulation of \Cref{thm:main-decomp}.

\begin{theorem}
Fix constants $k, g \in \mathbb{N}$. Assume we are given a graph $G$ on $n$ vertices together with its tree decomposition $(T, \beta)$ and a set of private apices $A_t \subseteq \beta(t)$ for each node $t\in V(T)$ such that the following conditions hold:
\begin{itemize}[nosep]
 \item For every node $t \in V(T)$, we have $|A_t| \leq k$.
 \item For every edge $st \in E(T)$,  we have $|\beta(v) \cap \beta(u)|\leq k$.
 \item For every node $t \in V(T)$, graph obtained by taking $G[\beta(t)] - A_t$ and turning  $(\beta(t) \cap \beta(s))\setminus A_t$ into a clique for every edge $st \in E(T)$ has Euler genus bounded by $g$.
\end{itemize}
Then, we can compute the eccentricity of every vertex of $G$ in time $\Oh \left( n^{1 + \frac{355}{356}} \log^{k + 5g} n \right)$.
\end{theorem}

\begin{proof}
We may assume that $|V(T)|\leq n$, for every tree decomposition with no two bags comparable by inclusion has this property; and adjacent comparable bags can be merged by contracting the edge between them.

For a node $t\in V(T)$, by the {\em{weight}} of $t$ we mean the size of the corresponding bag, that is, $|\beta(t)|$. For any subset of nodes $S \subseteq V(T)$, we define $\beta(S) \coloneqq \bigcup_{t \in S} \beta(t)$ By the {\em{weight}} of $S$, we mean the total weight of the elements of $S$, that is, $\sum_{t\in S} |\beta(t)|$. 

\begin{claim}\label{cl:weight-T}
The weight of $V(T)$ is of $\Oh(n)$.
\end{claim}

\begin{proof}
The sets $\beta'(t) := \beta(t) - \bigcup_{s \in N_T(t)} \beta(s)$ are pairwise disjoint. We have
$$
\sum_{t \in V(T)} |\beta(t)| =
\sum_{t \in V(T)} |\beta'(t)| + 2 \cdot \sum_{st \in E(T)} |\beta(s) \cap \beta(t)| \leq
|V(T)| + 2k|E(T)| = \Oh(n).
$$
\end{proof}

Since every bag induces a graph of bounded Euler genus, the number of edges contained in a bag is linear in its size. In particular, this implies that the total number of edges of $G$ is also bounded by $\Oh(n)$.

We set $$\delta \coloneqq \frac{1}{356}\qquad\textrm{and}\qquad \Delta \coloneqq \frac{355}{356}.$$ Root the tree $T$ in an arbitrarily chosen node; this naturally imposes an ancestor-descendant relation in $T$ (for convenience, every node is considered its own ancestor and descendant).

We start by partitioning $T$ into connected subtrees using the following procedure.
We proceed bottom-up over $T$, processing nodes in any order so that a node is processed after all its strict descendants have been processed. Along the way, we mark some nodes and split the edges of $T$ into heavy and light. Let $t \in V(T)$ be the currently processed non-root node of $T$ and let $e \in E(T)$ be the edge connecting $t$ with its parent. If the total weight of all the unmarked nodes that are descendants of $t$ is at least $n^\delta$ (recall that this includes $t$ itself as well), then we declare $e$ heavy and mark all the descendants of $t$ that were unmarked so far. Otherwise, the edge $e$ is declared light and the procedure proceeds to further nodes of $T$.

Observe that
removing all heavy edges splits $T$ into connected subtrees, say $T'_1, \cdots T'_m$. All of the subtrees, except for possibly the subtree containing the root node, are of weight at least $n^\delta$. In particular, the number of subtrees $m$, and therefore the number of heavy edges, is  bounded by $\Oh(n^{1 - \delta})$. Moreover, in every subtree $T'_i$, removing the node closest to the root splits $T'_i$ into smaller components, each of weight less than $n^\delta$.

Fix a heavy edge $e$ and let $T^e_1$ and $T^e_2$ be the two subtrees into which $T$ splits after removing~$e$. Let $X^e_i = \beta(T^e_i)$ for $i \in \{1, 2\}$. Put $A_e = X^e_1 \setminus X^e_2$, $C_e = X^e_2 \setminus X^e_1$, and $B_e = X^e_1 \cap X^e_2$. By the properties of tree decompositions, such choice of $A_e, B_e, C_e$ satisfies the conditions of \Cref{l:single_adhesion}, hence in time $\Oh(n \log^{k - 1} n)$ we can compute $\max_{v \in X^e_2} \dist_G(u,v)$ for every $u \in X^e_1$, and $\max_{u \in X^e_1} \dist_G(u,v)$ for every $v \in X^e_2$. Computing this for every heavy edge $e$ takes total time $\Oh(n^{2 - \delta} \log^{k - 1} n)$.

Fix any subtree $T'=T'_j$. Let $e_1 = t^{e_1}_1t^{e_1}_2, e_2 = t^{e_2}_1 t^{e_2}_2, \dots, e_\ell = t^{e_\ell}_1 t^{e_\ell}_2$ denote the heavy edges incident to $T'$, where $t^{e_i}_1 \in V(T')$ and $V(T') \subseteq V(T_1^{e_i})$ for every $i \in [\ell]$.
For a vertex $v \in \beta(T')$, let
$$d_0(v) = \max_{u \in \beta(T')} \dist_G(v, u)\qquad\textrm{and}\qquad d_i(v) = \max_{u \in X_2^{e_i}}\dist_G(v,u),\quad\textrm{for } i \in [\ell].$$ We have $\ecc(v) = \max \{ d_i(v)\colon i\in \{0,1,\ldots,\ell\}\}$.The values of $d_i(v)$ are already calculated for all $i\in [\ell]$, hence it remains to compute $d_0(v)$.

For every $i \in [\ell]$ and every pair of vertices $u, v \in \beta(t^{e_i}_1) \cap \beta(t^{e_i}_2)$ we find a shortest path between $u$ and $v$ with all internal vertices inside $X^{e_i}_2$ (or determine that it doesn't exist). For a fixed $u, v$ this can be done in time $\Oh(n)$. Since in total we perform this step at most $2k^2$ times per heavy edge, it takes $\Oh(n^{2 - \delta})$ time in total. Let $P_{i, u, v}$ denote such path, assuming it exists.

Let $G'$ denote the graph obtained from $G[\beta(T')]$ by taking every $i, u, v$ for which $P_{i, u, v}$ exists and adding an edge between $u$ and $v$ of weight equal to the total weight of $P_{i, u, v}$.
The weight of every edge inserted in $\beta(t^{e_i}_1) \cap \beta(t^{e_i}_2)$ is bounded by $|X^{e_i}_2|+1$. The total weight of all edges inserted is therefore at most
$$
\sum_{i \in [\ell]} |\beta(t^{e_i}_1) \cap \beta(t^{e_i}_2)|^2 \cdot (|X^{e_i}_2|+1) \leq
k^2 \sum_{i \in [\ell]} (|X^{e_i}_2|+1) = \Oh(n),
$$
where the last equality follows from the fact that all the trees $T^{e_i}_2$ are pairwise disjoint.
By \Cref{l:inserting_paths}, we have $\dist_{G'}(u, v) = \dist_G(u, v)$ for each $u, v \in \beta(T')$. Hence, computing $d_0(v)$ for every $v \in \beta(T')$ is equivalent to computing the eccentricity of every vertex in $G'$.

If the size of $\beta(T')$ is smaller than $n^\Delta$, we compute the eccentricities naively in time $\Oh(|\beta(T')|^2)$, 
noting that $G'$ has $\Oh(|\beta(T')|)$ edges (thanks to Claim~\ref{cl:weight-T} and bounded genus assumption 
of the last bullet of the theorem statement). Otherwise, we argue that we can use the algorithm in \Cref{l:star} as follows.

Let $t$ be the node of $T'$ closest to the root. Let $s_1, \dots, s_p$ be the children of $t$ in $T$ and let $T''_i$ denote the connected component of $T' - \{t\}$ containing $s_i$. Set $V_0 = \beta(t)$ and $V_i = \beta(T''_i)$ for $i \in [p]$.

It is now easy to verify that $G'$ and sets $A, \{V_i\colon 0\leq i\leq p\}$ selected this way satisfy the assumptions of \Cref{l:star2}. This allows us to use it to compute the eccentricities in $G'$ in time
$$
\Oh \left( n^{1 + \frac{150 + 54\delta}{151}} \log^{k + 5g} n \right) =
\Oh \left( n^{1 + \frac{354}{356}} \log^{k + 5g} n \right).
$$
As we argued, from these eccentricities, we may easily compute all the eccentricities in $G$.

Now, let us analyse the total running time of the whole algorithm. We invoke \Cref{l:star} $\Oh(n^{1 - \Delta})$ times, since we apply it only to subtrees $T'_i$ of size at least $n^\Delta$. The total running time of those applications is hence
$$
\Oh \left( n^{2 + \frac{354}{356} - \Delta} \log^{k + 5g} n \right) =
\Oh \left( n^{1 + \frac{355}{356}} \log^{k + 5g} n \right).
$$
We compute the eccentricities naively for subtrees smaller than $n^\Delta$, hence the total running time of this computation is
$$
\sum_{i \in [m] \colon |\beta(T'_i)| \leq n^\Delta} |\beta(T'_i)|^2 \leq
n^\Delta \cdot \sum_{i \in m} |\beta(T'_i)| = \Oh(n^{1 + \Delta})=\Oh\left(n^{1+\frac{355}{356}}\right).
$$
The rest of computation can be done in $\Oh(n^{2 - \delta} \log^k n)$. Therefore, the whole algorithm runs in time $\Oh \left( n^{1 + \frac{355}{356}} \log^{k + 5g} n \right)$.
\end{proof}




