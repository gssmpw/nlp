
% Given labeled and unlabeled datasets, novel class discovery (NCD) aims to discover novel classes in the unlabeled dataset that differ from the known classes in the labeled dataset. It typically utilizes knowledge from the labeled dataset to discover and cluster unseen objects into distinct semantic categories. 

% For NCD in image classification, Fini \etal presented a framework called UNO, which does not require self-supervised pre-training and employs a unified objective function for both labeled and unlabeled datasets~\cite{fini2021unified}. Cao \etal proposed a cross-entropy loss with uncertainty-adaptive margins to reduce bias towards seen classes~\cite{cao2021open}. 

% Vaze \etal proposed a method for generalized class discovery (GCD) that utilizes vision transformers with contrastive representation learning and a semi-supervised $k$-means method to cluster unlabeled data~\cite{vaze2022generalized}. Additionally, they estimated the number of categories in the unlabeled data. Wen \etal introduced a parametric classification method that employs supervised and self-supervised contrastive learning~\cite{wen2023parametric}. They also emphasized the importance of identifying unreliable pseudo-labels.


\section{Related Works}
\noindent \textbf{Generalized Class Discovery in Image Classification}. 
\cite{vaze2022generalized} introduced GCD, which aims to categorize unlabeled images given both labeled and unlabeled data. They trained an embedding network using unsupervised contrastive learning on all the data and supervised contrastive learning on the labeled data. They then applied semi-supervised $k$-means clustering to assign class or cluster labels to the unlabeled images. Additionally, they proposed a method for estimating the number of novel classes.

% The unlabeled images may contain either known objects or novel objects. 

\cite{An2023Generalized} proposed DPN, which utilizes two sets of category-wise prototypes: one for labeled data and the other for unlabeled images based on $k$-means clustering. For clustering, they assumed prior knowledge of the total number of categories. They discovered novel classes in the unlabeled data by applying the Hungarian algorithm~\cite{kuhn1955hungarian} to the two sets. \cite{Zhang2023PromptCAL} presented PromptCAL, which uses an affinity graph to generate pseudo-labels for unlabeled data. \cite{Pu2023Dynamic} introduced the DCCL framework, which employs a hyperparameter-free clustering algorithm~\cite{Martin2008Maps} to generate pseudo-labels in the absence of ground-truth cluster numbers. \cite{wen2023parametric} proposed SimGCD, a one-stage framework that replaces the separate semi-supervised clustering in~\cite{vaze2022generalized} with a jointly trainable parametric classifier. They analyzed the problem of using unreliable pseudo-labels for training a parametric classifier and proposed using soft pseudo-labels.

%  based on embeddings from the backbone of the latter branch 

Recently, \cite{Bai2023Towards} introduced GCD for long-tailed datasets to address imbalanced distributions in real-world scenarios. They proposed the BaCon framework, which includes a pseudo-labeling branch and a contrastive learning branch. To handle imbalanced distributions, they estimated data distribution using $k$-means clustering and the Hungarian algorithm~\cite{kuhn1955hungarian}. Concurrently, \cite{li2023imbagcd} proposed ImbaGCD, which estimates class prior distributions assuming unknown classes are usually tail classes. They then generated pseudo-labels for unlabeled images using the estimated prior distribution and the Sinkhorn-Knopp algorithm~\cite{Cuturi2013Sinkhorn}. Later, \cite{li2023generalized} replaced the expectation–maximization (EM) and Sinkhorn-Knopp~\cite{Cuturi2013Sinkhorn} algorithms in~\cite{li2023imbagcd} with cross-entropy-based regularization losses to reduce computational costs.


% Recently, researchers have explored NCD in semantic segmentation~\cite{zhao2022novel}, 3D semantic segmentation~\cite{riz2023novel}, detection~\cite{fomenko2022learning}, and instance segmentation~\cite{weng2021unsupervised, fomenko2022learning}. Zhao \etal proposed a method that first employs a saliency model to detect objects in unlabeled images and then applies clustering to generate pseudo-labels~\cite{zhao2022novel}. Subsequently, they dynamically categorized the pseudo-labels into clean or unclean splits based on entropy ranking. Then, they trained a semantic segmentation model using the unlabeled images with clean pseudo-labels and the labeled dataset. Riz \etal presented a method that uses online clustering to generate pseudo-labels and uncertainty estimation to identify reliable pseudo-labels~\cite{riz2023novel}. In instance segmentation, Weng \etal proposed a method that first employs a class-agnostic mask proposal network and then applies unsupervised hyperbolic $k$-means clustering to discover novel categories~\cite{weng2021unsupervised}. Fomenko \etal presented a method that avoids cropping object regions and utilizes feature pooling~\cite{fomenko2022learning}. They employed online constrained clustering with the experimentally selected non-symmetrical log-normal distribution as the target prior distribution.

% Recently, researchers have explored novel class discovery (NCD) in semantic segmentation~\cite{zhao2022novel}, 3D semantic segmentation~\cite{riz2023novel}, object detection~\cite{fomenko2022learning}, and instance segmentation~\cite{weng2021unsupervised, fomenko2022learning}. 

\noindent \textbf{Class Discovery in Segmentation and Detection}. 
\cite{zhao2022novel} introduced NCD in semantic segmentation. They proposed to find novel salient object regions using a saliency model and a segmentation network trained on labeled data. They then applied clustering to these object regions to obtain pseudo-labels. Finally, they trained a segmentation network using the labeled data and the unlabeled images with clean pseudo-labels from clustering and online pseudo-labels. The clean pseudo-labels were dynamically assigned based on entropy ranking. \cite{riz2023novel} extended the method from~\cite{zhao2022novel} to 3D point cloud semantic segmentation. Specifically, they utilized online clustering and exploited uncertainty quantification to generate pseudo-labels.


In instance segmentation, \cite{weng2021unsupervised} investigated unsupervised long-tail category discovery. They initially employed a class-agnostic mask proposal network to obtain masks for all objects. They then trained an embedding network using self-supervised triplet losses. Finally, they applied hyperbolic $k$-means clustering to discover novel categories. \cite{fomenko2022learning} introduced novel class discovery and localization, which can be viewed as GCD in object detection and instance segmentation. They first trained a Faster R-CNN~\cite{ren2015faster} or Mask R-CNN~\cite{he2017mask} using the labeled data and froze the network except for its classification head. Then, they applied the frozen network to unlabeled and labeled images to obtain region proposals. Subsequently, they expanded the classification head to incorporate new classes and trained it using pseudo-labels generated by online clustering based on the Sinkhorn-Knopp algorithm~\cite{Cuturi2013Sinkhorn}.

\begin{figure*}[!tbp]
    \centering     
    \includegraphics[width=0.95\linewidth]{figure/overview.pdf}
    \caption{Overview of the proposed \sys. The framework consists of three stages: (1) \textbf{Knowledge Graph Construction} that extracts structured knowledge from forget and retain data, (2) \textbf{Redundancy Removal} that identifies and removes redundant knowledge from the constructed knowledge graphs, and (3) \textbf{Question Synthesis} that generates QA pairs with the guidance of specific facts with LLMs automatically.}
    \label{fig:overview}
    % \vspace{-12pt}
\end{figure*}


% check later 0815
% To predict instance masks and classes for both known and novel categories given labeled and unlabeled datasets, we first employ GGNs~\cite{wang2022open} to obtain class-agnostic instance masks for all instances in the unlabeled dataset. We then apply the proposed class discovery method to generate pseudo-class labels for these instances. Subsequently, we train SOLOv2~\cite{wang2020solov2} using the proposed reliability-based dynamic learning approach on the labeled dataset and the unlabeled images with pseudo-labels.


% check later 0815
% While we also generate pseudo-labels for unlabeled images and train an instance segmentation network using them, similar to~\cite{zhao2022novel, riz2023novel} in semantic segmentation, we take into account the long-tail distribution of classes by proposing instance-wise temperature assignment and class-wise reliability-based dynamic learning. In instance segmentation, similar to~\cite{fomenko2022learning}, our method does not require clustering during inference~\cite{weng2021unsupervised}. Different from~\cite{fomenko2022learning}, we do not depend on the pre-selected prior distribution and use self-computed target distribution for optimization. 


% check later 0815
% While \cite{vaze2022generalized, wen2023parametric} used the same constant values for $\tau_u$ and $\tau_s$ across all the data, we propose to assign varying values to each sample based on its headness or tailness.


% check later 0815
% Following~\cite{fomenko2022learning}, we also utilize an online clustering method to avoid the separate semi-supervised clustering step used in~\cite{vaze2022generalized}. However, while \cite{fomenko2022learning} employed online constrained clustering with an experimentally selected target prior distribution, we investigate contrastive learning for clustering.



% \subsection{Instance Segmentation}
% Due to the importance of instance segmentation in various tasks, researchers have investigated diverse fully supervised methods based on CNNs~\cite{li2017fully, liu2018path, chen2020blendmask, kirillov2020pointrend, wang2020solov2} and vision transformers~\cite{cheng2022masked, jain2023oneformer}. While fully supervised methods have achieved accurate results, dense pixel-level annotations are time-consuming. Therefore, researchers have explored semi-supervised methods~\cite{wang2022noisy, zhou2020learning, yang2022bias} to train models by utilizing large-scale unlabeled images alongside a small amount of labeled data.

% Wang \etal proposed an effective method, termed SOLOv2, that does not rely on bounding box detection~\cite{wang2020solov2}. 

% \subsection{Open-World Instance Segmentation}

% Researchers have also investigated open-world instance segmentation methods that can segment both seen and unseen object classes while being trained on only seen categories. 
% Wang \etal presented Generic Grouping Networks (GGNs) that train a class-agnostic pair-wise affinity predictor and apply it alongside a grouping module to generate pseudo-masks~\cite{wang2022open}. Subsequently, they train a class-agnostic object proposal network using both ground-truth and pseudo masks. 
% Kalluri \etal proposed a method that trains Mask R-CNN using masks from bottom-up segmentation and human annotations together~\cite{kalluri2023open}. Subsequently, they apply the network, a grouping module, and a refinement module to obtain high-quality and complete instance segmentation.

