%File: formatting-instructions-latex-2025.tex
%release 2025.0
\documentclass[letterpaper]{article} % DO NOT CHANGE THIS
\usepackage{aaai25}  % DO NOT CHANGE THIS
\usepackage{times}  % DO NOT CHANGE THIS
\usepackage{helvet}  % DO NOT CHANGE THIS
\usepackage{courier}  % DO NOT CHANGE THIS
\usepackage[hyphens]{url}  % DO NOT CHANGE THIS
\usepackage{graphicx} % DO NOT CHANGE THIS
\urlstyle{rm} % DO NOT CHANGE THIS
\def\UrlFont{\rm}  % DO NOT CHANGE THIS
\usepackage{natbib}  % DO NOT CHANGE THIS AND DO NOT ADD ANY OPTIONS TO IT
\usepackage{caption} % DO NOT CHANGE THIS AND DO NOT ADD ANY OPTIONS TO IT
\frenchspacing  % DO NOT CHANGE THIS
\setlength{\pdfpagewidth}{8.5in}  % DO NOT CHANGE THIS
\setlength{\pdfpageheight}{11in}  % DO NOT CHANGE THIS
%
% These are recommended to typeset algorithms but not required. See the subsubsection on algorithms. Remove them if you don't have algorithms in your paper.
\usepackage{algorithm}
\usepackage{algorithmic}

%%%%%%%%%%%---SETME-----%%%%%%%%%%%%%
%replace @@ with the submission number submission site.
\newcommand{\thiswork}{INF$^2$\xspace}
%%%%%%%%%%%%%%%%%%%%%%%%%%%%%%%%%%%%


%\newcommand{\rev}[1]{{\color{olivegreen}#1}}
\newcommand{\rev}[1]{{#1}}


\newcommand{\JL}[1]{{\color{cyan}[\textbf{\sc JLee}: \textit{#1}]}}
\newcommand{\JW}[1]{{\color{orange}[\textbf{\sc JJung}: \textit{#1}]}}
\newcommand{\JY}[1]{{\color{blue(ncs)}[\textbf{\sc JSong}: \textit{#1}]}}
\newcommand{\HS}[1]{{\color{magenta}[\textbf{\sc HJang}: \textit{#1}]}}
\newcommand{\CS}[1]{{\color{navy}[\textbf{\sc CShin}: \textit{#1}]}}
\newcommand{\SN}[1]{{\color{olive}[\textbf{\sc SNoh}: \textit{#1}]}}

%\def\final{}   % uncomment this for the submission version
\ifdefined\final
\renewcommand{\JL}[1]{}
\renewcommand{\JW}[1]{}
\renewcommand{\JY}[1]{}
\renewcommand{\HS}[1]{}
\renewcommand{\CS}[1]{}
\renewcommand{\SN}[1]{}
\fi

%%% Notion for baseline approaches %%% 
\newcommand{\baseline}{offloading-based batched inference\xspace}
\newcommand{\Baseline}{Offloading-based batched inference\xspace}


\newcommand{\ans}{attention-near storage\xspace}
\newcommand{\Ans}{Attention-near storage\xspace}
\newcommand{\ANS}{Attention-Near Storage\xspace}

\newcommand{\wb}{delayed KV cache writeback\xspace}
\newcommand{\Wb}{Delayed KV cache writeback\xspace}
\newcommand{\WB}{Delayed KV Cache Writeback\xspace}

\newcommand{\xcache}{X-cache\xspace}
\newcommand{\XCACHE}{X-Cache\xspace}


%%% Notions for our methods %%%
\newcommand{\schemea}{\textbf{Expanding supported maximum sequence length with optimized performance}\xspace}
\newcommand{\Schemea}{\textbf{Expanding supported maximum sequence length with optimized performance}\xspace}

\newcommand{\schemeb}{\textbf{Optimizing the storage device performance}\xspace}
\newcommand{\Schemeb}{\textbf{Optimizing the storage device performance}\xspace}

\newcommand{\schemec}{\textbf{Orthogonally supporting Compression Techniques}\xspace}
\newcommand{\Schemec}{\textbf{Orthogonally supporting Compression Techniques}\xspace}



% Circular numbers
\usepackage{tikz}
\newcommand*\circled[1]{\tikz[baseline=(char.base)]{
            \node[shape=circle,draw,inner sep=0.4pt] (char) {#1};}}

\newcommand*\bcircled[1]{\tikz[baseline=(char.base)]{
            \node[shape=circle,draw,inner sep=0.4pt, fill=black, text=white] (char) {#1};}}
\usepackage{booktabs}       % professional-quality tables
\usepackage{amsfonts}       % blackboard math symbols
\usepackage{url}            % simple URL typesetting
\usepackage[T1]{fontenc}

%
% These are are recommended to typeset listings but not required. See the subsubsection on listing. Remove this block if you don't have listings in your paper.
\usepackage{newfloat}
\usepackage{listings}
\DeclareCaptionStyle{ruled}{labelfont=normalfont,labelsep=colon,strut=off} % DO NOT CHANGE THIS
\lstset{%
	basicstyle={\footnotesize\ttfamily},% footnotesize acceptable for monospace
	numbers=left,numberstyle=\footnotesize,xleftmargin=2em,% show line numbers, remove this entire line if you don't want the numbers.
	aboveskip=0pt,belowskip=0pt,%
	showstringspaces=false,tabsize=2,breaklines=true}
\floatstyle{ruled}
\newfloat{listing}{tb}{lst}{}
\floatname{listing}{Listing}
%
% Keep the \pdfinfo as shown here. There's no need
% for you to add the /Title and /Author tags.
\pdfinfo{
/TemplateVersion (2025.1)
}

% DISALLOWED PACKAGES
% \usepackage{authblk} -- This package is specifically forbidden
% \usepackage{balance} -- This package is specifically forbidden
% \usepackage{color (if used in text)
% \usepackage{CJK} -- This package is specifically forbidden
% \usepackage{float} -- This package is specifically forbidden
% \usepackage{flushend} -- This package is specifically forbidden
% \usepackage{fontenc} -- This package is specifically forbidden
% \usepackage{fullpage} -- This package is specifically forbidden
% \usepackage{geometry} -- This package is specifically forbidden
% \usepackage{grffile} -- This package is specifically forbidden
% \usepackage{hyperref} -- This package is specifically forbidden
% \usepackage{navigator} -- This package is specifically forbidden
% (or any other package that embeds links such as navigator or hyperref)
% \indentfirst} -- This package is specifically forbidden
% \layout} -- This package is specifically forbidden
% \multicol} -- This package is specifically forbidden
% \nameref} -- This package is specifically forbidden
% \usepackage{savetrees} -- This package is specifically forbidden
% \usepackage{setspace} -- This package is specifically forbidden
% \usepackage{stfloats} -- This package is specifically forbidden
% \usepackage{tabu} -- This package is specifically forbidden
% \usepackage{titlesec} -- This package is specifically forbidden
% \usepackage{tocbibind} -- This package is specifically forbidden
% \usepackage{ulem} -- This package is specifically forbidden
% \usepackage{wrapfig} -- This package is specifically forbidden
% DISALLOWED COMMANDS
% \nocopyright -- Your paper will not be published if you use this command
% \addtolength -- This command may not be used
% \balance -- This command may not be used
% \baselinestretch -- Your paper will not be published if you use this command
% \clearpage -- No page breaks of any kind may be used for the final version of your paper
% \columnsep -- This command may not be used
% \newpage -- No page breaks of any kind may be used for the final version of your paper
% \pagebreak -- No page breaks of any kind may be used for the final version of your paperr
% \pagestyle -- This command may not be used
% \tiny -- This is not an acceptable font size.
% \vspace{- -- No negative value may be used in proximity of a caption, figure, table, section, subsection, subsubsection, or reference
% \vskip{- -- No negative value may be used to alter spacing above or below a caption, figure, table, section, subsection, subsubsection, or reference

\setcounter{secnumdepth}{2} %May be changed to 1 or 2 if section numbers are desired.

% The file aaai25.sty is the style file for AAAI Press
% proceedings, working notes, and technical reports.
%

% Title

% Your title must be in mixed case, not sentence case.
% That means all verbs (including short verbs like be, is, using,and go),
% nouns, adverbs, adjectives should be capitalized, including both words in hyphenated terms, while
% articles, conjunctions, and prepositions are lower case unless they
% directly follow a colon or long dash
%\title{AAAI Press Formatting Instructions \\for Authors Using \LaTeX{} --- A Guide}
%\author{
%    %Authors
%    % All authors must be in the same font size and format.
%    Written by AAAI Press Staff\textsuperscript{\rm 1}\thanks{With help from the AAAI Publications Committee.}\\
%    AAAI Style Contributions by Pater Patel Schneider,
%    Sunil Issar,\\
%    J. Scott Penberthy,
%    George Ferguson,
%    Hans Guesgen,
%    Francisco Cruz\equalcontrib,
%    Marc Pujol-Gonzalez\equalcontrib
%}
%\affiliations{
%    %Afiliations
%    \textsuperscript{\rm 1}Association for the Advancement of Artificial Intelligence\\
%    % If you have multiple authors and multiple affiliations
%    % use superscripts in text and roman font to identify them.
%    % For example,
%
%    % Sunil Issar\textsuperscript{\rm 2}, 
%    % J. Scott Penberthy\textsuperscript{\rm 3}, 
%    % George Ferguson\textsuperscript{\rm 4},
%    % Hans Guesgen\textsuperscript{\rm 5}
%    % Note that the comma should be placed after the superscript
%
%    1101 Pennsylvania Ave, NW Suite 300\\
%    Washington, DC 20004 USA\\
%    % email address must be in roman text type, not monospace or sans serif
%    proceedings-questions@aaai.org
%%
%% See more examples next
%}

%Example, Single Author, ->> remove \iffalse,\fi and place them surrounding AAAI title to use it
\iffalse
\title{My Publication Title --- Single Author}
\author {
    Author Name
}
\affiliations{
    Affiliation\\
    Affiliation Line 2\\
    name@example.com
}
\fi

%Example, Multiple Authors, ->> remove \iffalse,\fi and place them surrounding AAAI title to use it
\title{Generalized Class Discovery in Instance Segmentation}
\author {
    % Authors
    Cuong Manh Hoang,
    Yeejin Lee,
    Byeongkeun Kang\thanks{Corresponding author.}
}
\affiliations {
    % Affiliations
	Seoul National University of Science and Technology, Republic of Korea\\
    \{cuonghoang, yeejinlee, byeongkeun.kang\}@seoultech.ac.kr
}
%\author {
%    % Authors
%    First Author Name\textsuperscript{\rm 1,\rm 2},
%    Second Author Name\textsuperscript{\rm 2},
%    Third Author Name\textsuperscript{\rm 1}
%}
%\affiliations {
%    % Affiliations
%    \textsuperscript{\rm 1}Affiliation 1\\
%    \textsuperscript{\rm 2}Affiliation 2\\
%    firstAuthor@affiliation1.com, secondAuthor@affilation2.com, thirdAuthor@affiliation1.com
%}


% REMOVE THIS: bibentry
% This is only needed to show inline citations in the guidelines document. You should not need it and can safely delete it.
\usepackage{bibentry}
% END REMOVE bibentry

\begin{document}

\maketitle

\begin{abstract}
This work addresses the task of generalized class discovery (GCD) in instance segmentation. The goal is to discover novel classes and obtain a model capable of segmenting instances of both known and novel categories, given labeled and unlabeled data. Since the real world contains numerous objects with long-tailed distributions, the instance distribution for each class is inherently imbalanced. To address the imbalanced distributions, we propose an instance-wise temperature assignment (ITA) method for contrastive learning and class-wise reliability criteria for pseudo-labels. The ITA method relaxes instance discrimination for samples belonging to head classes to enhance GCD. The reliability criteria are to avoid excluding most pseudo-labels for tail classes when training an instance segmentation network using pseudo-labels from GCD. Additionally, we propose dynamically adjusting the criteria to leverage diverse samples in the early stages while relying only on reliable pseudo-labels in the later stages. We also introduce an efficient soft attention module to encode object-specific representations for GCD. Finally, we evaluate our proposed method by conducting experiments on two settings: COCO$_{half}$ + LVIS and LVIS + Visual Genome. The experimental results demonstrate that the proposed method outperforms previous state-of-the-art methods.
\end{abstract}

% Uncomment the following to link to your code, datasets, an extended version or similar.
%
% \begin{links}
%     \link{Code}{https://aaai.org/example/code}
%     \link{Datasets}{https://aaai.org/example/datasets}
%     \link{Extended version}{https://aaai.org/example/extended-version}
% \end{links}

\section{Introduction}

Video generation has garnered significant attention owing to its transformative potential across a wide range of applications, such media content creation~\citep{polyak2024movie}, advertising~\citep{zhang2024virbo,bacher2021advert}, video games~\citep{yang2024playable,valevski2024diffusion, oasis2024}, and world model simulators~\citep{ha2018world, videoworldsimulators2024, agarwal2025cosmos}. Benefiting from advanced generative algorithms~\citep{goodfellow2014generative, ho2020denoising, liu2023flow, lipman2023flow}, scalable model architectures~\citep{vaswani2017attention, peebles2023scalable}, vast amounts of internet-sourced data~\citep{chen2024panda, nan2024openvid, ju2024miradata}, and ongoing expansion of computing capabilities~\citep{nvidia2022h100, nvidia2023dgxgh200, nvidia2024h200nvl}, remarkable advancements have been achieved in the field of video generation~\citep{ho2022video, ho2022imagen, singer2023makeavideo, blattmann2023align, videoworldsimulators2024, kuaishou2024klingai, yang2024cogvideox, jin2024pyramidal, polyak2024movie, kong2024hunyuanvideo, ji2024prompt}.


In this work, we present \textbf{\ours}, a family of rectified flow~\citep{lipman2023flow, liu2023flow} transformer models designed for joint image and video generation, establishing a pathway toward industry-grade performance. This report centers on four key components: data curation, model architecture design, flow formulation, and training infrastructure optimization—each rigorously refined to meet the demands of high-quality, large-scale video generation.


\begin{figure}[ht]
    \centering
    \begin{subfigure}[b]{0.82\linewidth}
        \centering
        \includegraphics[width=\linewidth]{figures/t2i_1024.pdf}
        \caption{Text-to-Image Samples}\label{fig:main-demo-t2i}
    \end{subfigure}
    \vfill
    \begin{subfigure}[b]{0.82\linewidth}
        \centering
        \includegraphics[width=\linewidth]{figures/t2v_samples.pdf}
        \caption{Text-to-Video Samples}\label{fig:main-demo-t2v}
    \end{subfigure}
\caption{\textbf{Generated samples from \ours.} Key components are highlighted in \textcolor{red}{\textbf{RED}}.}\label{fig:main-demo}
\end{figure}


First, we present a comprehensive data processing pipeline designed to construct large-scale, high-quality image and video-text datasets. The pipeline integrates multiple advanced techniques, including video and image filtering based on aesthetic scores, OCR-driven content analysis, and subjective evaluations, to ensure exceptional visual and contextual quality. Furthermore, we employ multimodal large language models~(MLLMs)~\citep{yuan2025tarsier2} to generate dense and contextually aligned captions, which are subsequently refined using an additional large language model~(LLM)~\citep{yang2024qwen2} to enhance their accuracy, fluency, and descriptive richness. As a result, we have curated a robust training dataset comprising approximately 36M video-text pairs and 160M image-text pairs, which are proven sufficient for training industry-level generative models.

Secondly, we take a pioneering step by applying rectified flow formulation~\citep{lipman2023flow} for joint image and video generation, implemented through the \ours model family, which comprises Transformer architectures with 2B and 8B parameters. At its core, the \ours framework employs a 3D joint image-video variational autoencoder (VAE) to compress image and video inputs into a shared latent space, facilitating unified representation. This shared latent space is coupled with a full-attention~\citep{vaswani2017attention} mechanism, enabling seamless joint training of image and video. This architecture delivers high-quality, coherent outputs across both images and videos, establishing a unified framework for visual generation tasks.


Furthermore, to support the training of \ours at scale, we have developed a robust infrastructure tailored for large-scale model training. Our approach incorporates advanced parallelism strategies~\citep{jacobs2023deepspeed, pytorch_fsdp} to manage memory efficiently during long-context training. Additionally, we employ ByteCheckpoint~\citep{wan2024bytecheckpoint} for high-performance checkpointing and integrate fault-tolerant mechanisms from MegaScale~\citep{jiang2024megascale} to ensure stability and scalability across large GPU clusters. These optimizations enable \ours to handle the computational and data challenges of generative modeling with exceptional efficiency and reliability.


We evaluate \ours on both text-to-image and text-to-video benchmarks to highlight its competitive advantages. For text-to-image generation, \ours-T2I demonstrates strong performance across multiple benchmarks, including T2I-CompBench~\citep{huang2023t2i-compbench}, GenEval~\citep{ghosh2024geneval}, and DPG-Bench~\citep{hu2024ella_dbgbench}, excelling in both visual quality and text-image alignment. In text-to-video benchmarks, \ours-T2V achieves state-of-the-art performance on the UCF-101~\citep{ucf101} zero-shot generation task. Additionally, \ours-T2V attains an impressive score of \textbf{84.85} on VBench~\citep{huang2024vbench}, securing the top position on the leaderboard (as of 2025-01-25) and surpassing several leading commercial text-to-video models. Qualitative results, illustrated in \Cref{fig:main-demo}, further demonstrate the superior quality of the generated media samples. These findings underscore \ours's effectiveness in multi-modal generation and its potential as a high-performing solution for both research and commercial applications.

\begin{figure*}[t]
  \centering
    \includegraphics[width=1\linewidth]{visuals/final_registration.png}
    \caption{Target measurement process on low-cost scan data using ICP and Coloured ICP. (1) Initialisation: The source point cloud (checkerboard) is misaligned with the target point cloud. (2) Initial Registration using Point-to-Plane ICP: Standard ICP leads to suboptimal registration. (3) Final Registration using Coloured ICP: Colour information is incorporated after pre-processing with RANSAC and Binarisation with Otsu Thresholding for real data, resulting in improved alignment.}
    \label{fig:Registration_visualisation}
\end{figure*}

\subsection{Iterative Closest Point (ICP) Algorithm}
The Iterative Closest Point (ICP) algorithm has been a fundamental technique in 3D computer vision and robotics for point cloud. Originally proposed by \cite{besl_method_1992}, ICP aims to minimise the distance between two datasets, typically referred to as the source and the target. The algorithm operates in an iterative manner, identifying correspondences by matching each source point with its nearest target point \citep{survey_ICP}. It then computes the rigid transformation, usually involving both rotation and translation, to achieve the best alignment of these matched points \citep{survey_ICP}. This process is repeated until convergence, where the change in the alignment parameters or the overall alignment error becomes smaller than a predefined threshold.

One key advantage of the ICP framework lies in its simplicity: the algorithm is conceptually straightforward, and its basic version is relatively easy to implement. However, traditional ICP can be sensitive to local minima, often requiring a good initial alignment \citep{zhang2021fast}. Furthermore, outliers, noise, and partial overlaps between datasets can significantly degrade its performance \citep{zhang2021fast, bouaziz2013sparse}. Over the years, various modifications and improvements \citep{gelfand2005robust, rusu2009fast, aiger20084, gruen2005least, fitzgibbon2003robust} have been proposed to mitigate these issues. Among the most common strategies are robust cost functions \citep{fitzgibbon2003robust}, weighting schemes for correspondences \citep{rusu2009fast}, and more sophisticated techniques \citep{gelfand2005robust, bouaziz2013sparse} to reject outliers. 

In addition, there is significant interest in integrating additional information into the ICP pipeline. Instead of solely relying on geometric cues such as point coordinates or surface normals, recent approaches have proposed incorporating colour (RGB) or intensity data to enhance correspondence accuracy. These methods \citep{park_colored_2017, 5980407}, commonly known as "Colored ICP" employ differences in pixel intensities or colour values as additional constraints. This is particularly beneficial in situations where geometric attributes alone are inadequate for accurate alignment or where surfaces possess complex texture patterns that can assist in the matching process.

\subsection{Applications of Target Measurement}

One approach relies on the use of physical checkerboard targets for registration. \cite{fryskowska2019} analyse checkerboard target identification for terrestrial laser scanning. They propose a geometric method to determine the target centre with higher precision, demonstrating that their approach can reduce errors by up to 6 mm compared to conventional automatic methods.

\cite{becerik2011assessment} examines data acquisition errors in 3D laser scanning for construction by evaluating how different target types (paper, paddle, and sphere) and layouts impact registration accuracy in both indoor and outdoor environments and presents guidelines for optimal target configuration.

\citet{Liang2024} propose to use Coloured ICP to measure target centres for checkerboard targets, similar to our investigation. They use data from a survey-grade terrestrial laser scanner. Their intended application is structural bridge monitoring purposes. They report an average accuracy of the measurement below 1.3 millimetres.

Where targets cannot be placed in the scene, the intensity information form the scanner can still be used to identify distinctive points. For point cloud data that is captured with a regular pattern, standard image processing can be used in a similar way to target detection. For example, \citet{wendt_automation_2004} proposes to use the SUSAN operator on a co-registered image from a camera, \citet{bohm_automatic_2007} proposes to use the SIFT operator on the LIDAR reflectance directly and \citet{theiler_markerless_2013} propose to use a Difference-of-Gaussian approach on the reflectance information.
Most of these methods extract image features to find reliable 3D correspondences for the purpose of registration.

In the following we describe our approach to the measurement of the target centre. In contrast to most proposed methods above we focus on unordered point clouds, where raster-based methods are not available, and low-cost sensors, where increased measurement noise and outliers are expected. As we are not aware of a commercial reference solution to this problem, we start by conducting a series of synthetic experiments to explore the viability and accuracy potential of the approach.



%The reviewed studies primarily rely on physical targets or target-free methods and do not utilise 3D synthetic point cloud checkerboards. In contrast, our approach introduces synthetic point cloud checkerboards, which offer controlled and consistent target geometry and reduce variability caused by physical targets. This innovation has significant potential for commercialisation and industrial application.


\section{Proposed method}
% \subsection{Preliminaries on LIC rate-distortion optimization}
\subsection{Preliminaries: rate-distortion optimization in learned image compression}

In LIC, the objective is to encode an image \( x \), drawn from a source distribution with probability density function \( p_{\text{source}} \), into a compact bit sequence for efficient storage or transmission. The receiver then reconstructs an approximation \( \hat{x} \) of the original image \( x \). The LIC process involves three main steps: encoding and quantization, entropy coding, and reconstruction.

\begin{itemize}
    \item \textbf{Encoding and Quantization}: First, each data point \( x \) is mapped to a de-correlated low-dimensional latent variable \( \hat{z} \) via an encoder function \( e(\cdot) \) followed by quantization \( Q(\cdot) \) to convert the continuous representation into discrete values, i.e., \( \hat{z} = Q(e(x)) \).

    \item \textbf{Entropy Coding}: After determining \( \hat{z} \), lossless entropy coding, such as Huffman coding~\cite{moffat2019huffman} or arithmetic coding~\cite{witten1987arithmetic}, is applied to produce a compressed bit sequence with length \( b(\hat{z}) \). Ideally, the entropy coding scheme approximates the theoretical bit rate, given by the entropy of \( \hat{z} \) under its marginal distribution. We assume that both the sender and receiver have access to an entropy model \( P(\hat{z}) \), which estimates the marginal probability distribution of \( \hat{z} \) and determines the expected bit length as \( b(\hat{z}) \approx -\log_2 P(\hat{z}) \). The goal is to ensure that \( P(\hat{z}) \) closely approximates the true marginal distribution \( p(\hat{z}) \), which is defined as
\(
p(\hat{z}) = \mathbb{E}_{x \sim p_{\text{source}}} \left[ \delta(\hat{z}, Q(e(x))) \right],
\)
where \( \delta \) is the Kronecker delta function. Under this approximation, the encoding length \( b(\hat{z}) \) is nearly optimal, as the average code length approaches the entropy \( -\log_2 p(\hat{z}) \) of \( \hat{z} \).

    \item \textbf{Reconstruction}: Once the receiver has obtained \( \hat{z} \), it reconstructs the approximation \( \hat{x} \) using a reconstruction function \( r(\cdot) \), such that \( \hat{x} = r(\hat{z}) \).
\end{itemize}

To optimize the LIC scheme, our goal is to minimize both the bit rate (rate) and the discrepancy between \( x \) and \( \hat{x} \) (distortion), where the distortion is measured by a function \( d(x, \hat{x}) \). This objective is formulated as a R-D loss using a Lagrangian multiplier:
\begin{equation}
\label{eq:rd}
    \mathcal{L}_\text{R-D} = \mathbb{E}_{x \sim p_{\text{source}}} \left[ \underbrace{-\log_2 P(\hat{z})}_{\text{rate}\ (\mathcal{L}_{\text{R}})} + \underbrace{\lambda d(x, \hat{x})}_{\text{distortion}\ (\mathcal{L}_{\text{D}})} \right],
\end{equation}
where \( \lambda \) is a Lagrange multiplier that controls the trade-off between rate (compression efficiency) and distortion (reconstruction quality). This formulation represents the Lagrangian relaxation of the distortion-constrained R-D optimization problem, aiming for efficient compression while maintaining high reconstruction fidelity. In practice, the gradient used to update LIC models is the sum of the gradients for each objective:
\(
d_t = \nabla \mathcal{L}_{\text{R}, t} + \nabla \mathcal{L}_{\text{D}, t}
\), where $t$ denotes the training iteration.


\begin{algorithm*}[t]
\caption{Balanced Rate-Distortion Optimization via Trajectory Optimization}
\label{alg:solution1}
\begin{algorithmic}[1]
\REQUIRE Initial network parameters $\theta_0$, initial softmax logits $\boldsymbol{\xi}_0$, learning rates $\alpha$ (for $\theta$) and $\beta$ (for $\boldsymbol{\xi}$), decay parameter $\gamma$, total iterations $T$
\STATE Initialize weights $\mathbf{w}_0 \leftarrow \text{Softmax}(\boldsymbol{\xi}_0)$
\FOR{$t = 0$ \TO $T-1$}
    \STATE Compute losses $\mathcal{L}_{\text{R}, t} = \mathcal{L}_{\text{R}}(\theta_t)$ and $\mathcal{L}_{\text{D}, t} = \mathcal{L}_{\text{D}}(\theta_t)$
    \STATE Compute gradients: $\nabla_\theta \mathcal{L}_{\text{R}, t} = \frac{\partial \mathcal{L}_{\text{R}, t}}{\partial \theta_t}$, \quad $\nabla_\theta \mathcal{L}_{\text{D}, t} = \frac{\partial \mathcal{L}_{\text{D}, t}}{\partial \theta_t}$
    \STATE \hspace{2.1cm} $\nabla_\theta \log \mathcal{L}_{\text{R}, t} = \frac{\nabla_\theta \mathcal{L}_{\text{R}, t}}{\mathcal{L}_{\text{R}, t}}$, \quad $\nabla_\theta \log \mathcal{L}_{\text{D}, t} = \frac{\nabla_\theta \mathcal{L}_{\text{D}, t}}{\mathcal{L}_{\text{D}, t}}$
    \STATE Compute normalization constant: $c_t = \left( \frac{w_{\text{R}, t}}{\mathcal{L}_{\text{R}, t}} + \frac{w_{\text{D}, t}}{\mathcal{L}_{\text{D}, t}} \right)^{-1}$
    \STATE Compute balanced gradient: $\mathbf{d}_t = c_t \left( w_{\text{R}, t} \nabla_\theta \log \mathcal{L}_{\text{R}, t} + w_{\text{D}, t} \nabla_\theta \log \mathcal{L}_{\text{D}, t} \right)$
    \STATE Update network parameters: $\theta_{t+1} = \theta_t - \alpha\, \mathbf{d}_t$
    \STATE Compute updated losses $\mathcal{L}_{\text{R}, t+1} = \mathcal{L}_{\text{R}}(\theta_{t+1})$ and $\mathcal{L}_{\text{D}, t+1} = \mathcal{L}_{\text{D}}(\theta_{t+1})$
    \STATE Compute $\delta_t = \begin{bmatrix} \frac{\partial w_{\text{R}, t}}{\partial \boldsymbol{\xi}_t} \\ \frac{\partial w_{\text{D}, t}}{\partial \boldsymbol{\xi}_t} \end{bmatrix}^\top \begin{bmatrix} \log \mathcal{L}_{\text{R}, t} - \log \mathcal{L}_{\text{R}, t+1} \\ \log \mathcal{L}_{\text{D}, t} - \log \mathcal{L}_{\text{D}, t+1} \end{bmatrix}$
    \STATE Update softmax logits: $\boldsymbol{\xi}_{t+1} = \boldsymbol{\xi}_t - \beta \left( \delta_t + \gamma\, \boldsymbol{\xi}_t \right)$
    \STATE Update weights: $\mathbf{w}_{t+1} = \text{Softmax}(\boldsymbol{\xi}_{t+1})$
\ENDFOR
\end{algorithmic}
\end{algorithm*}

\subsection{Balanced rate-distortion optimization}
To address the imbalance in optimizing rate and distortion discussed in Sec.~\ref{sec:intro}, we propose a balanced R-D optimization framework. This framework dynamically adjusts the contributions from each objective in the R-D loss function, aiming to achieve equal progress in both rate and distortion. Specifically, we redefine the update direction as \( d_t = w_{\text{R},t} \nabla \log \mathcal{L}_{\text{R}, t} + w_{\text{D},t} \nabla \log \mathcal{L}_{\text{D}, t}\), where \(w_{\text{R},t}\) and \(w_{\text{L},t}\) are adaptive weights for the rate and distortion gradients. By maximizing the minimum improvement speed between rate and distortion at each step, our method finds the ideal gradient weights, promotes stable balanced optimization, and enhances overall model performance.


Inspired by  MOO techniques~\cite{sener2018multi, liu2024famo}, we reformulate the R-D optimization problem as a multi-objective optimization problem. Our goal is to simultaneously optimize rate minimization and distortion minimization with parameters \( \theta \in \mathbb{R}^m \). To achieve this, we modify the original Lagrangian R-D loss function to the following form:
\begin{equation}
\min_{\theta \in \mathbb{R}^m} \left\{ \mathcal{L}(\theta) = \mathcal{L}_{\text{R}}(\theta) + \mathcal{L}_{\text{D}}(\theta)  \right\},
\end{equation}
where the trade-off factor \( \lambda \) is now absorbed into \( \mathcal{L}_{\text{D}} \) to simplify the formulation. $m$ is the number of parameters.

The loss improvement speed at iteration \( t \) of task \(i \in \{\text{R}, \text{D}\}\) is defined as:
\begin{equation}
\label{eq:speed}
s_{i,t}(\alpha, d_t) = \frac{\mathcal{L}_{i,t} - \mathcal{L}_{i,t+1}}{\mathcal{L}_{i,t}},
\end{equation}
where \( \alpha \) is the step size and \( d_t \) is the update direction at \( t \), with \( \theta_{t+1} = \theta_t - \alpha d_t \). To achieve balanced optimization, we seek an update direction \( d_t \) that maximizes the minimum improvement speed between rate and distortion, ensuring that neither objective dominates the update. In other words, we maximize the smaller improvement speed between \( \mathcal{L}_{\text{R}} \) and \( \mathcal{L}_{\text{D}} \). This leads to the following saddle point problem~\cite{lin2020near} formulation:
{\small
\begin{equation}
\max_{d_t \in \mathbb{R}^m} \min \left( \frac{1}{\alpha} s_{\text{R},t}(\alpha, d_t) - \frac{1}{2} \|d_t\|^2, \frac{1}{\alpha} s_{\text{D},t}(\alpha, d_t) - \frac{1}{2} \|d_t\|^2 \right),
\end{equation}}
where \( \frac{1}{2} \|d_t\|^2 \) serves as a regularization term to prevent unbounded updates.
% where \( \frac{1}{2} \|d_t\|^2 \) serves as a regularization term to prevent an unbounded solution.

When the step size \( \alpha \) is small, one can approximate \( \mathcal{L}_{t+1} \approx \mathcal{L}_{t} - \alpha \nabla \mathcal{L}_t^\top d_{t} \) using a first-order Taylor expansion. This approximation simplifies the problem to:
{\small\begin{equation}
\begin{aligned}
&\max_{d_t \in \mathbb{R}^m} \min \left( \frac{1}{\alpha} s_{\text{R},t}(\alpha, d_t) - \frac{1}{2} \|d_t\|^2, \frac{1}{\alpha} s_{\text{D},t}(\alpha, d_t) - \frac{1}{2} \|d_t\|^2 \right)\\
&=\max_{d_t \in \mathbb{R}^m} \min \left( \frac{\nabla \mathcal{L}_{\text{R}, t}^\top d_t}{\mathcal{L}_{\text{R}, t}} - \frac{1}{2} \|d_t\|^2, \frac{\nabla \mathcal{L}_{\text{D}, t}^\top d_t}{\mathcal{L}_{\text{D}, t}} - \frac{1}{2} \|d_t\|^2 \right) \\
&= \max_{d_t \in \mathbb{R}^m} \left( \min \left( \nabla \log \mathcal{L}_{\text{R}, t}^\top d_t, \nabla \log \mathcal{L}_{\text{D}, t}^\top d_t \right) - \frac{1}{2} \|d_t\|^2 \right).
\end{aligned}
\end{equation}}

To avoid solving the high-dimensional primal problem directly (as \( d_t \in \mathbb{R}^m \) with potentially millions of parameters if \( \theta \) is a neural network), we follow previous work~\cite{sener2018multi,liu2024famo} to turn to the dual problem. Leveraging the Lagrangian duality theorem~\cite{boyd2004convex}, we can rewrite the optimization as a convex combination of gradients:
{\small
\begin{equation}
\begin{aligned}
&\max_{d_t \in \mathbb{R}^m} \left( \min \left( \nabla \log \mathcal{L}_{\text{R}, t}^\top d_t, \nabla \log \mathcal{L}_{\text{D}, t}^\top d_t \right) - \frac{1}{2} \|d_t\|^2 \right) \\
&= \max_{d_t \in \mathbb{R}^m} \min_{w \in \mathbb{S}_2} \left( w_{\text{R},t} \nabla \log \mathcal{L}_{\text{R}, t} + w_{\text{D},t} \nabla \log \mathcal{L}_{\text{D}, t} \right)^\top d_t - \frac{1}{2} \|d_t\|^2 \\
&= \min_{w_t \in \mathbb{S}_2} \max_{d_t \in \mathbb{R}^m} \left( w_{\text{R},t} \nabla \log \mathcal{L}_{\text{R}, t} + w_{\text{D},t} \nabla \log \mathcal{L}_{\text{D}, t} \right)^\top d_t - \frac{1}{2} \|d_t\|^2,
\end{aligned}
\end{equation}}
where the second equality follows from strong duality. Here, \( w_t \in \mathbb{S}_2 = \{ w \in \mathbb{R}_{\geq 0}^2 \mid w^\top \mathbf{1} = 1 \} \) represents the gradient weights in the 2-dimensional probabilistic simplex.

Let \( g(d_t, w) = (w_{\text{R},t} \nabla \log \mathcal{L}_{\text{R}, t} + w_{\text{D},t} \nabla \log \mathcal{L}_{\text{D}, t})^\top d_t - \frac{1}{2} \|d_t\|^2 \). The optimal direction \( d_t^* \) is obtained by setting:
\begin{equation}
\small
\label{eq:opt_grad}
\frac{\partial g}{\partial d_t} = 0 \quad \Longrightarrow \quad d_t^* = w_{\text{R},t} \nabla \log \mathcal{L}_{\text{R}, t} + w_{\text{D},t} \nabla \log \mathcal{L}_{\text{D}, t}.
\end{equation}

Substituting \( d_t^* \) back, we obtain:
\begin{equation}
\small
\label{eq:target}
\begin{aligned}
&\max_{d_t \in \mathbb{R}^m} \min \left( \frac{1}{\alpha} s_{\text{R},t}(\alpha, d_t) - \frac{1}{2} \|d_t\|^2, \frac{1}{\alpha} s_{\text{D},t}(\alpha, d_t) - \frac{1}{2} \|d_t\|^2 \right) \\
&= \min_{w_t \in \mathbb{S}_2} \frac{1}{2} \left\| w_{\text{R},t} \nabla \log \mathcal{L}_{\text{R}, t} + w_{\text{D},t} \nabla \log \mathcal{L}_{\text{D}, t} \right\|^2 \\
&= \min_{w_t \in \mathbb{S}_2} \frac{1}{2} \| J_t w_t \|^2,
\end{aligned}
\end{equation}
where \( J_t = \begin{bmatrix}
\nabla \log \mathcal{L}_{\text{R}, t}^\top \\
\nabla \log \mathcal{L}_{\text{D}, t}^\top
\end{bmatrix} \). Thus, our optimization problem reduces to finding \( w_t \) that satisfies Eq.~\ref{eq:target}.


\subsubsection{Solution 1: Gradient descent over trajectory}
\label{subsubsec:gd_over}
Rather than fully solving the optimization problem at each step, our proposed Solution 1 adopts a coarse-to-fine gradient descent approach~\cite{sener2018multi,liu2024famo}, incrementally refining the solution along the R-D optimization trajectory. In this approach, the gradient weights \( w_t \) are updated iteratively as follows:
\begin{equation}
w_{t+1} = w_{t} - \alpha_w \tilde{\delta},
\end{equation}
where
\begin{equation}
\tilde{\delta} = \nabla_w \frac{1}{2} \| J_t w_t \|^2 = J_t^\top J_t w_t.
\end{equation}

Using a first-order Taylor approximation, \( \log \mathcal{L}_{t+1} \approx \log \mathcal{L}_{t} - \alpha \nabla \log \mathcal{L}_t^\top d_{t} \), we have the following relationship:
\begin{equation}
\tilde{\delta} = J_t^\top J_t w_t = J_t^\top d_t \approx \frac{1}{\alpha} \begin{bmatrix} \log \mathcal{L}_{\text{R}, t} - \log \mathcal{L}_{\text{R}, t+1} \\ \log \mathcal{L}_{\text{D}, t} - \log \mathcal{L}_{\text{D}, t+1} \end{bmatrix}.
\footnote{To prevent negative values in \( \log \mathcal{L}_{i,t} \), we add 1 to \( \mathcal{L}_{i,t} \) before applying the logarithm, ensuring that the minimum value in the log domain is zero.}
\end{equation}

To ensure that \( w_t \) remains within the interior point of simplex \( \mathbb{S}_2 \), we reparametrize \( w_t \) using \( \xi_t \):
\begin{equation}
w_t = \text{Softmax}(\xi_t),
\end{equation}
where \( \xi_t \in \mathbb{R}^2 \) represents the unconstrained softmax logits. To give more weight to recent updates, we add a decay term~\cite{zhou2022convergence,liu2024famo}, leading to the following update for \( \xi_t \):
\begin{equation}
\xi_{t+1} = \xi_t - \beta(\delta_t + \gamma \xi_t),
\end{equation}
where
\begin{equation}
\delta_t = \begin{bmatrix} \nabla^\top w_{\text{R},t}(\xi) \\ \nabla^\top w_{\text{D},t}(\xi) \end{bmatrix} \begin{bmatrix} \log \mathcal{L}_{\text{R}, t} - \log \mathcal{L}_{\text{R}, t+1} \\ \log \mathcal{L}_{\text{D}, t} - \log \mathcal{L}_{\text{D}, t+1} \end{bmatrix}.
\end{equation}

After computing the weights \( w_t \), we renormalize them to ensure numerical stability~\cite{liu2024famo}. This renormalization is crucial as our update direction is a convex combination of the gradients of the log losses:
\begin{equation}
d_t = w_{\text{R},t} \nabla \log \mathcal{L}_{\text{R},t} + w_{\text{D},t} \nabla \log \mathcal{L}_{\text{D},t} = \sum_{i \in \{\text{R}, \text{D}\}} \frac{w_{i,t}}{\mathcal{L}_{i,t}} \nabla \mathcal{L}_{i,t}.
\end{equation}
When \( \mathcal{L}_{i,t} \) becomes small, the multiplicative factor \( \frac{w_{i,t}}{\mathcal{L}_{i,t}} \) can grow large, potentially causing instability in the optimization. To mitigate this, we scale the gradient by a constant \( c_t \):
\begin{equation}
c_t = \left(\frac{w_{\text{R},t}}{\mathcal{L}_{\text{R},t}} + \frac{w_{\text{D},t}}{\mathcal{L}_{\text{D},t}}\right)^{-1}.
\end{equation}

The resulting balanced gradient, which is used to update the model parameters \( \theta \), is then given by:
\begin{equation}
\label{eq:sl1_d}
d_t = c_t \left(w_{\text{R},t} \nabla \log \mathcal{L}_{\text{R},t} + w_{\text{D},t} \nabla \log \mathcal{L}_{\text{D},t}\right).
\end{equation}

This method, referred to as Solution 1, is a coarse-to-fine gradient descent technique. It is particularly suited for training LIC models from scratch, as it incrementally balances rate and distortion optimization along the trajectory. The implementation is shown in Algorithm~\ref{alg:solution1}.


\begin{algorithm*}[t]
\caption{Balanced Rate-Distortion Optimization via Quadratic Programming}
\label{alg:solution2}
\begin{algorithmic}[1]
\REQUIRE Initial network parameters $\theta_0$, learning rate $\alpha$, total iterations $T$
\FOR{$t = 0$ \TO $T-1$}
    \STATE Compute losses $\mathcal{L}_{\text{R}, t} = \mathcal{L}_{\text{R}}(\theta_t)$ and $\mathcal{L}_{\text{D}, t} = \mathcal{L}_{\text{D}}(\theta_t)$
    \STATE Compute gradients: $\nabla_\theta \mathcal{L}_{\text{R}, t} = \frac{\partial \mathcal{L}_{\text{R}, t}}{\partial \theta_t}$, \quad $\nabla_\theta \mathcal{L}_{\text{D}, t} = \frac{\partial \mathcal{L}_{\text{D}, t}}{\partial \theta_t}$
    \STATE \hspace{2.1cm} $\nabla_\theta \log \mathcal{L}_{\text{R}, t} = \frac{\nabla_\theta \mathcal{L}_{\text{R}, t}}{\mathcal{L}_{\text{R}, t}}$, \quad $\nabla_\theta \log \mathcal{L}_{\text{D}, t} = \frac{\nabla_\theta \mathcal{L}_{\text{D}, t}}{\mathcal{L}_{\text{D}, t}}$
    \STATE Form matrix $J_t = \begin{bmatrix}  \nabla_\theta \log \mathcal{L}_{\text{R}, t} ^\top \\  \nabla_\theta \log \mathcal{L}_{\text{D}, t} ^\top \end{bmatrix}$
    \STATE Compute Hessian matrix: $Q = J_t^\top J_t = \begin{bmatrix} \| \nabla_\theta \log \mathcal{L}_{\text{R}, t} \|^2 & \langle \nabla_\theta \log \mathcal{L}_{\text{R}, t}, \nabla_\theta \log \mathcal{L}_{\text{D}, t} \rangle \\ \langle \nabla_\theta \log \mathcal{L}_{\text{R}, t}, \nabla_\theta \log \mathcal{L}_{\text{D}, t} \rangle & \| \nabla_\theta \log \mathcal{L}_{\text{D}, t} \|^2 \end{bmatrix}$
    \STATE Compute inverse $Q^{-1}$
    \STATE Compute weights: $\lambda = \frac{1}{\mathbf{1}^\top Q^{-1} \mathbf{1}}$, \quad $w_t = \lambda\, Q^{-1} \mathbf{1}$
    \STATE Apply softmax for numerical stability: $\tilde{w}_t = \text{Softmax}({w}_t)$
    \STATE Compute normalization constant: $c_t = \left( \frac{\tilde{w}_{\text{R}, t}}{\mathcal{L}_{\text{R}, t}} + \frac{\tilde{w}_{\text{D}, t}}{\mathcal{L}_{\text{D}, t}} \right)^{-1}$
    \STATE Compute balanced gradient: $\mathbf{d}_t = c_t \left( \tilde{w}_{\text{R}, t} \nabla_\theta \log \mathcal{L}_{\text{R}, t} + \tilde{w}_{\text{D}, t} \nabla_\theta \log \mathcal{L}_{\text{D}, t} \right)$
    \STATE Update network parameters: $\theta_{t+1} = \theta_t - \alpha\, \mathbf{d}_t$
\ENDFOR
\end{algorithmic}
\end{algorithm*}


\subsubsection{Solution 2: Quadratic programming}
Alternatively, we can formulate the weight optimization problem as a constrained-quadratic programming (QP) problem~\cite{nocedal1999numerical}:
\begin{equation}
\begin{aligned}
\min_{w_t} \quad & \frac{1}{2} \| J_t w_t \|^2 = \frac{1}{2} w_t^\top (J_t^\top J_t) w_t \\
\text{s.t.} \quad & w_{\text{R},t} + w_{\text{D},t} = 1, \\
& w_{\text{R},t}, w_{\text{D},t} \geq 0,
\end{aligned}
\end{equation}
where \( J_t = \begin{bmatrix} \nabla \log \mathcal{L}_{\text{R}, t}^\top \\ \nabla \log \mathcal{L}_{\text{D}, t}^\top \end{bmatrix} \) is an \( n \times 2 \) matrix containing the gradients of the log losses for rate and distortion.

Let \( Q = J_t^\top J_t \) represent the Hessian matrix. Since the gradients for rate and distortion typically reflect distinct objectives, they are rarely parallel, suggesting \( Q \) is positive definite~\cite{sener2018multi}. This property permits an analytical solution to the QP problem. The Hessian \( Q \) is given by:
\begin{equation}
Q = \begin{bmatrix} \|\nabla \log \mathcal{L}_{\text{R}, t}\|^2 & \langle \nabla \log \mathcal{L}_{\text{R}, t}, \nabla \log \mathcal{L}_{\text{D}, t} \rangle \\ \langle \nabla \log \mathcal{L}_{\text{R}, t}, \nabla \log \mathcal{L}_{\text{D}, t} \rangle & \|\nabla \log \mathcal{L}_{\text{D}, t}\|^2 \end{bmatrix}.
\end{equation}


% When the gradients are not parallel, the determinant of \( Q \) is positive, confirming that \( Q \) is positive definite.
To solve this QP problem, we introduce a Lagrange multiplier \( \lambda \) to enforce the equality constraint~\cite{boyd2004convex}, giving us the following Lagrangian:
\begin{equation}
L(w_t, \lambda) = \frac{1}{2} w_t^\top Q w_t - \lambda(w_{\text{R},t} + w_{\text{D},t} - 1).
\end{equation}

The weights \( w_t \) can then be obtained by applying the Karush-Kuhn-Tucker (KKT) conditions~\cite{mangasarian1994nonlinear}. The main KKT conditions for this problem are:

1. \textbf{Stationarity}: The gradient of the Lagrangian with respect to \( w_t \) must be zero,
   \begin{equation}
   \nabla_{w_t} L = Q w_t - \lambda \mathbf{1} = 0,
   \end{equation}
   where \( \mathbf{1} = [1, 1]^\top \). This yields \( w_t = \lambda Q^{-1} \mathbf{1} \).

2. \textbf{Primal Feasibility}: The weights must satisfy the equality constraint,
   \begin{equation}
   \mathbf{1}^\top w_t = 1.
   \end{equation}
Note: Neglecting dual feasibility and complementary slackness simplifies the solution process since non-negativity is guaranteed by a softmax projection applied subsequently.

By combining these conditions, we arrive at a simplified expression:
\begin{equation}
\mathbf{1}^\top w_t = \lambda \mathbf{1}^\top Q^{-1} \mathbf{1} = 1,
\end{equation}
which gives:
\begin{equation}
\lambda = \frac{1}{\mathbf{1}^\top Q^{-1} \mathbf{1}}, \quad w_t = \frac{Q^{-1} \mathbf{1}}{\mathbf{1}^\top Q^{-1} \mathbf{1}}.
\end{equation}


Since \( Q \) is positive definite, \( Q^{-1} \) exists, ensuring this solution is unique. To enforce non-negativity and enhance numerical stability, we then further project \( w_t \) onto the probability simplex \( \mathbb{S}_2 \) by simply using the softmax function:
\begin{equation}
\tilde{w}_t = \text{Softmax}\left({w}_t\right)=\text{Softmax}\left(\frac{Q^{-1} \mathbf{1}}{\mathbf{1}^\top Q^{-1} \mathbf{1}}\right).
\end{equation}
This step ensures \( \tilde{w}_t \in \mathbb{S}_2 \) and reduces potential numerical issues caused by large gradient variations.  It serves as an approximation that aligns with the optimal solution while maintaining similar performance despite slight deviations.


After determining the weights \( \tilde{w}_t \), we apply the same renormalization constant \( c_t \) as in Section~\ref{subsubsec:gd_over} to compute a balanced gradient for updating the model parameters \( \theta \), following the form in Eq.~\ref{eq:sl1_d}. Solution 2, with its analytical QP formulation, is particularly suitable for fine-tuning existing LIC models, offering a refined approach to balance rate and distortion objectives. The detailed implementation is presented in Algorithm~\ref{alg:solution2}.






% The detailed algorithms for both solutions are presented in Supplementary Material Section~\ref{sec:detailed_alg} as Algorithm~\ref{alg:solution1} and Algorithm~\ref{alg:solution2}.




\section{Experimental Results}
In this section, we present the main results in~\secref{sec:main}, followed by ablation studies on key design choices in~\secref{sec:ablation}.

\begin{table*}[t]
\renewcommand\arraystretch{1.05}
\centering
\setlength{\tabcolsep}{2.5mm}{}
\begin{tabular}{l|l|c|cc|cc}
type & model     & \#params      & FID$\downarrow$ & IS$\uparrow$ & Precision$\uparrow$ & Recall$\uparrow$ \\
\shline
GAN& BigGAN~\cite{biggan} & 112M & 6.95  & 224.5       & 0.89 & 0.38     \\
GAN& GigaGAN~\cite{gigagan}  & 569M      & 3.45  & 225.5       & 0.84 & 0.61\\  
GAN& StyleGan-XL~\cite{stylegan-xl} & 166M & 2.30  & 265.1       & 0.78 & 0.53  \\
\hline
Diffusion& ADM~\cite{adm}    & 554M      & 10.94 & 101.0        & 0.69 & 0.63\\
Diffusion& LDM-4-G~\cite{ldm}   & 400M  & 3.60  & 247.7       & -  & -     \\
Diffusion & Simple-Diffusion~\cite{diff1} & 2B & 2.44 & 256.3 & - & - \\
Diffusion& DiT-XL/2~\cite{dit} & 675M     & 2.27  & 278.2       & 0.83 & 0.57     \\
Diffusion&L-DiT-3B~\cite{dit-github}  & 3.0B    & 2.10  & 304.4       & 0.82 & 0.60    \\
Diffusion&DiMR-G/2R~\cite{liu2024alleviating} &1.1B& 1.63& 292.5& 0.79 &0.63 \\
Diffusion & MDTv2-XL/2~\cite{gao2023mdtv2} & 676M & 1.58 & 314.7 & 0.79 & 0.65\\
Diffusion & CausalFusion-H$^\dag$~\cite{deng2024causal} & 1B & 1.57 & - & - & - \\
\hline
Flow-Matching & SiT-XL/2~\cite{sit} & 675M & 2.06 & 277.5 & 0.83 & 0.59 \\
Flow-Matching&REPA~\cite{yu2024representation} &675M& 1.80 & 284.0 &0.81 &0.61\\    
Flow-Matching&REPA$^\dag$~\cite{yu2024representation}& 675M& 1.42&  305.7& 0.80& 0.65 \\
\hline
Mask.& MaskGIT~\cite{maskgit}  & 227M   & 6.18  & 182.1        & 0.80 & 0.51 \\
Mask. & TiTok-S-128~\cite{yu2024image} & 287M & 1.97 & 281.8 & - & - \\
Mask. & MAGVIT-v2~\cite{yu2024language} & 307M & 1.78 & 319.4 & - & - \\ 
Mask. & MaskBit~\cite{weber2024maskbit} & 305M & 1.52 & 328.6 & - & - \\
\hline
AR& VQVAE-2~\cite{vqvae2} & 13.5B    & 31.11           & $\sim$45     & 0.36           & 0.57          \\
AR& VQGAN~\cite{vqgan}& 227M  & 18.65 & 80.4         & 0.78 & 0.26   \\
AR& VQGAN~\cite{vqgan}   & 1.4B     & 15.78 & 74.3   & -  & -     \\
AR&RQTran.~\cite{rq}     & 3.8B    & 7.55  & 134.0  & -  & -    \\
AR& ViTVQ~\cite{vit-vqgan} & 1.7B  & 4.17  & 175.1  & -  & -    \\
AR & DART-AR~\cite{gu2025dart} & 812M & 3.98 & 256.8 & - & - \\
AR & MonoFormer~\cite{zhao2024monoformer} & 1.1B & 2.57 & 272.6 & 0.84 & 0.56\\
AR & Open-MAGVIT2-XL~\cite{luo2024open} & 1.5B & 2.33 & 271.8 & 0.84 & 0.54\\
AR&LlamaGen-3B~\cite{llamagen}  &3.1B& 2.18& 263.3 &0.81& 0.58\\
AR & FlowAR-H~\cite{flowar} & 1.9B & 1.65 & 296.5 & 0.83 & 0.60\\
AR & RAR-XXL~\cite{yu2024randomized} & 1.5B & 1.48 & 326.0 & 0.80 & 0.63 \\
\hline
MAR & MAR-B~\cite{mar} & 208M & 2.31 &281.7 &0.82 &0.57 \\
MAR & MAR-L~\cite{mar} &479M& 1.78 &296.0& 0.81& 0.60 \\
MAR & MAR-H~\cite{mar} & 943M&1.55& 303.7& 0.81 &0.62 \\
\hline
VAR&VAR-$d16$~\cite{var}   & 310M  & 3.30& 274.4& 0.84& 0.51    \\
VAR&VAR-$d20$~\cite{var}   &600M & 2.57& 302.6& 0.83& 0.56     \\
VAR&VAR-$d30$~\cite{var}   & 2.0B      & 1.97  & 323.1 & 0.82 & 0.59      \\
\hline
\modelname& \modelname-B    &172M   &1.72&280.4&0.82&0.59 \\
\modelname& \modelname-L   & 608M   & 1.28& 292.5&0.82&0.62\\
\modelname& \modelname-H    & 1.1B    & 1.24 &301.6&0.83&0.64\\
\end{tabular}
\caption{
\textbf{Generation Results on ImageNet-256.}
Metrics include Fréchet Inception Distance (FID), Inception Score (IS), Precision, and Recall. $^\dag$ denotes the use of guidance interval sampling~\cite{guidance}. The proposed \modelname-H achieves a state-of-the-art 1.24 FID on the ImageNet-256 benchmark without relying on vision foundation models (\eg, DINOv2~\cite{dinov2}) or guidance interval sampling~\cite{guidance}, as used in REPA~\cite{yu2024representation}.
}\label{tab:256}
\end{table*}

\subsection{Main Results}
\label{sec:main}
We conduct experiments on ImageNet~\cite{deng2009imagenet} at 256$\times$256 and 512$\times$512 resolutions. Following prior works~\cite{dit,mar}, we evaluate model performance using FID~\cite{fid}, Inception Score (IS)~\cite{is}, Precision, and Recall. \modelname is trained with the same hyper-parameters as~\cite{mar,dit} (\eg, 800 training epochs), with model sizes ranging from 172M to 1.1B parameters. See Appendix~\secref{sec:sup_hyper} for hyper-parameter details.





\begin{table}[t]
    \centering
    \begin{tabular}{c|c|c|c}
      model    &  \#params & FID$\downarrow$ & IS$\uparrow$ \\
      \shline
      VQGAN~\cite{vqgan}&227M &26.52& 66.8\\
      BigGAN~\cite{biggan}& 158M&8.43 &177.9\\
      MaskGiT~\cite{maskgit}& 227M&7.32& 156.0\\
      DiT-XL/2~\cite{dit} &675M &3.04& 240.8 \\
     DiMR-XL/3R~\cite{liu2024alleviating}& 525M&2.89 &289.8 \\
     VAR-d36~\cite{var}  & 2.3B& 2.63 & 303.2\\
     REPA$^\ddagger$~\cite{yu2024representation}&675M &2.08& 274.6 \\
     \hline
     \modelname-L & 608M&1.70& 281.5 \\
    \end{tabular}
    \caption{
    \textbf{Generation Results on ImageNet-512.} $^\ddagger$ denotes the use of DINOv2~\cite{dinov2}.
    }
    \label{tab:512}
\end{table}

\noindent\textbf{ImageNet-256.}
In~\tabref{tab:256}, we compare \modelname with previous state-of-the-art generative models.
Out best variant, \modelname-H, achieves a new state-of-the-art-performance of 1.24 FID, outperforming the GAN-based StyleGAN-XL~\cite{stylegan-xl} by 1.06 FID, masked-prediction-based MaskBit~\cite{maskgit} by 0.28 FID, AR-based RAR~\cite{yu2024randomized} by 0.24 FID, VAR~\cite{var} by 0.73 FID, MAR~\cite{mar} by 0.31 FID, and flow-matching-based REPA~\cite{yu2024representation} by 0.18 FID.
Notably, \modelname does not rely on vision foundation models~\cite{dinov2} or guidance interval sampling~\cite{guidance}, both of which were used in REPA~\cite{yu2024representation}, the previous best-performing model.
Additionally, our lightweight \modelname-B (172M), surpasses DiT-XL (675M)~\cite{dit} by 0.55 FID while achieving an inference speed of 9.8 images per second—20$\times$ faster than DiT-XL (0.5 images per second). Detailed speed comparison can be found in Appendix \ref{sec:speed}.



\noindent\textbf{ImageNet-512.}
In~\tabref{tab:512}, we report the performance of \modelname on ImageNet-512.
Similarly, \modelname-L sets a new state-of-the-art FID of 1.70, outperforming the diffusion based DiT-XL/2~\cite{dit} and DiMR-XL/3R~\cite{liu2024alleviating} by a large margin of 1.34 and 1.19 FID, respectively.
Additionally, \modelname-L also surpasses the previous best autoregressive model VAR-d36~\cite{var} and flow-matching-based REPA~\cite{yu2024representation} by 0.93 and 0.38 FID, respectively.




\noindent\textbf{Qualitative Results.}
\figref{fig:qualitative} presents samples generated by \modelname (trained on ImageNet) at 512$\times$512 and 256$\times$256 resolutions. These results highlight \modelname's ability to produce high-fidelity images with exceptional visual quality.

\begin{figure*}
    \centering
    \vspace{-6pt}
    \includegraphics[width=1\linewidth]{figures/qualitative.pdf}
    \caption{\textbf{Generated Samples.} \modelname generates high-quality images at resolutions of 512$\times$512 (1st row) and 256$\times$256 (2nd and 3rd row).
    }
    \label{fig:qualitative}
\end{figure*}

\subsection{Ablation Studies}
\label{sec:ablation}
In this section, we conduct ablation studies using \modelname-B, trained for 400 epochs to efficiently iterate on model design.

\noindent\textbf{Prediction Entity X.}
The proposed \modelname extends next-token prediction to next-X prediction. In~\tabref{tab:X}, we evaluate different designs for the prediction entity X, including an individual patch token, a cell (a group of surrounding tokens), a subsample (a non-local grouping), a scale (coarse-to-fine resolution), and an entire image.

Among these variants, cell-based \modelname achieves the best performance, with an FID of 2.48, outperforming the token-based \modelname by 1.03 FID and surpassing the second best design (scale-based \modelname) by 0.42 FID. Furthermore, even when using standard prediction entities such as tokens, subsamples, images, or scales, \modelname consistently outperforms existing methods while requiring significantly fewer parameters. These results highlight the efficiency and effectiveness of \modelname across diverse prediction entities.






\begin{table}[]
    \centering
    \scalebox{0.92}{
    \begin{tabular}{c|c|c|c|c}
        model & \makecell[c]{prediction\\entity} & \#params & FID$\downarrow$ & IS$\uparrow$\\
        \shline
        LlamaGen-L~\cite{llamagen} & \multirow{2}{*}{token} & 343M & 3.80 &248.3\\
        \modelname-B& & 172M&3.51&251.4\\
        \hline
        PAR-L~\cite{par} & \multirow{2}{*}{subsample}& 343M & 3.76 & 218.9\\
        \modelname-B&  &172M& 3.58&231.5\\
        \hline
        DiT-L/2~\cite{dit}& \multirow{2}{*}{image}& 458M&5.02&167.2 \\
         \modelname-B& & 172M&3.13&253.4 \\
        \hline
        VAR-$d16$~\cite{var} & \multirow{2}{*}{scale} & 310M&3.30 &274.4\\
        \modelname-B& &172M&2.90&262.8\\
        \hline
        \baseline{\modelname-B}& \baseline{cell} & \baseline{172M}&\baseline{2.48}&\baseline{269.2} \\
    \end{tabular}
    }
    \caption{\textbf{Ablation on Prediction Entity X.} Using cells as the prediction entity outperforms alternatives such as tokens or entire images. Additionally, under the same prediction entity, \modelname surpasses previous methods, demonstrating its effectiveness across different prediction granularities. }%
    \label{tab:X}
\end{table}

\noindent\textbf{Cell Size.}
A prediction entity cell is formed by grouping spatially adjacent $k\times k$ tokens, where a larger cell size incorporates more tokens and thus captures a broader context within a single prediction step.
For a $256\times256$ input image, the encoded continuous latent representation has a spatial resolution of $16\times16$. Given this, the image can be partitioned into an $m\times m$ grid, where each cell consists of $k\times k$ neighboring tokens. As shown in~\tabref{tab:cell}, we evaluate different cell sizes with $k \in \{1,2,4,8,16\}$, where $k=1$ represents a single token and $k=16$ corresponds to the entire image as a single entity. We observe that performance improves as $k$ increases, peaking at an FID of 2.48 when using cell size $8\times8$ (\ie, $k=8$). Beyond this, performance declines, reaching an FID of 3.13 when the entire image is treated as a single entity.
These results suggest that using cells rather than the entire image as the prediction unit allows the model to condition on previously generated context, improving confidence in predictions while maintaining both rich semantics and local details.





\begin{table}[t]
    \centering
    \scalebox{0.98}{
    \begin{tabular}{c|c|c|c}
    cell size ($k\times k$ tokens) & $m\times m$ grid & FID$\downarrow$ & IS$\uparrow$ \\
       \shline
       $1\times1$ & $16\times16$ &3.51&251.4 \\
       $2\times2$ & $8\times8$ & 3.04& 253.5\\
       $4\times4$ & $4\times4$ & 2.61&258.2 \\
       \baseline{$8\times8$} & \baseline{$2\times2$} & \baseline{2.48} & \baseline{269.2}\\
       $16\times16$ & $1\times1$ & 3.13&253.4  \\
    \end{tabular}
    }
    \caption{\textbf{Ablation on the cell size.}
    In this study, a $16\times16$ continuous latent representation is partitioned into an $m\times m$ grid, where each cell consits of $k\times k$ neighboring tokens.
    A cell size of $8\times8$ achieves the best performance, striking an optimal balance between local structure and global context.
    }
    \label{tab:cell}
\end{table}



\begin{table}[t]
    \centering
    \scalebox{0.95}{
    \begin{tabular}{c|c|c|c}
      previous cell & noise time step &  FID$\downarrow$ & IS$\uparrow$ \\
       \shline
       clean & $t_i=0, \forall i<n$& 3.45& 243.5\\
       increasing noise & $t_1<t_2<\cdots<t_{n-1}$& 2.95&258.8 \\
       decreasing noise & $t_1>t_2>\cdots>t_{n-1}$&2.78 &262.1 \\
      \baseline{random noise}  & \baseline{no constraint} &\baseline{2.48} & \baseline{269.2}\\
    \end{tabular}
    }
    \caption{
    \textbf{Ablation on Noisy Context Learning.}
    This study examines the impact of noise time steps ($t_1, \cdots, t_{n-1} \subset [0, 1]$) in previous entities ($t=0$ represents pure Gaussian noise).
    Conditioning on all clean entities (the ``clean'' variant) results in suboptimal performance.
    Imposing an order on noise time steps, either ``increasing noise'' or ``decreasing noise'', also leads to inferior results. The best performance is achieved with the "random noise" setting, where no constraints are imposed on noise time steps.
    }
    \label{tab:ncl}
\end{table}


\noindent\textbf{Noisy Context Learning.}
During training, \modelname employs Noisy Context Learning (NCL), predicting $X_n$ by conditioning on all previous noisy entities, unlike Teacher Forcing.
The noise intensity of previous entities is contorlled by noise time steps $\{t_1, \dots, t_{n-1}\} \subset [0, 1]$, where $t=0$ corresponds to pure Gaussian noise.
We analyze the impact of NCL in~\tabref{tab:ncl}.
When conditioning on all clean entities (\ie, the ``clean'' variant, where $t_i=0, \forall i<n$), which is equivalent to vanilla AR (\ie, Teacher Forcing), the suboptimal performance is obtained.
We also evaluate two constrained noise schedules: the ``increasing noise'' variant, where noise time steps increase over AR steps ($t_1<t_2< \cdots < t_{n-1}$), and the `` decreasing noise'' variant, where noise time steps decrease ($t_1>t_2> \cdots > t_{n-1}$).
While both settings improve over the ``clean'' variant, they remain inferior to our final ``random noise'' setting, where no constraints are imposed on noise time steps, leading to the best performance.




        


\section{Conclusion}
Towards open-world instance segmentation, we present a novel GCD method in instance segmentation. To address the imbalanced distribution of instances, we introduce the instance-wise temperature assignment method as well as class-wise and dynamic reliability criteria. The former aims to improve the embedding space for class discovery, and the criteria are designed to effectively utilize pseudo-labels from the GCD model. Additionally, we propose an efficient soft attention module. The experimental results in two settings demonstrate that the proposed method outperforms previous methods by effectively discovering novel classes and segmenting instances of both known and novel categories.

Regarding limitations, this work assumes the full availability of labeled and unlabeled datasets from the beginning. Thus, it is suboptimal for scenarios where data is provided sequentially, such as in robot navigation. Additionally, we assume prior knowledge of the total number of classes, following most previous works.


\section*{Acknowledgments}
This work was supported by the National Research Foundation of Korea(NRF) grant funded by the Korea government(MSIT) (No. RS-2023-00252434). 

\bibliography{aaai25}

\end{document}
