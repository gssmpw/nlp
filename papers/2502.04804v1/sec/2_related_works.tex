\section{Related Work}
\subsection{Point Cloud Sequence Compression}
Point cloud sequence compression methods can be broadly categorized into point-based and projection-based methods.
Point-based methods directly operate on 3D point clouds to remove redundant content across frames. While early methods represent points as octree~\cite{garcia2019geometry} or voxels~\cite{de2017motion}, recent approaches~\cite{gomes2021graph, akhtar2024inter} leverage deep neural networks (DNNs) to transform unstructured point clouds into latent spaces, improving coding efficiency by better correlating similar components between frames. In contrast, projection-based methods project the 3D point cloud to 2D depth images and compress the volume of depth images~\cite{graziosiOverviewOngoingPoint2020, sun2020advanced}. As part of ongoing standardization efforts, VPCC \cite{graziosiOverviewOngoingPoint2020} utilizes mature 2D video encoders like H.264 to encode depth and color images. This strategy makes VPCC one of the most promising point cloud sequence compression standards due to its seamless integration with existing 2D video encoders and hardware infrastructure.

\subsection{Encoders for Machine Vision}
As vision models consume increasing multimedia content, there is growing interest in developing machine vision-oriented encoders that aim to save bits by exploiting RoIs of vision models. For 2D video encoding, feedback-based methods~\cite{xie2019source, du2020server, li2021task, xiao2022dnn, liu2022adamask} exploits either gradient-based importance scores or predicted bounding boxes to build spatial importance maps. Neural encoder-based methods~\cite{reich2024deep, wang2022enabling} replace traditional encoders with differential neural encoders and directly optimize them to support the target vision model. On-device analytics methods~\cite{du2022accmpeg, murad2022dao, zhang2022casva} propose to use cheap analytics models to localize RoIs or dynamically control encoding parameters. For single-frame 3D point cloud encoding, Liu et al.~\cite{liu2023pchm} propose a neural encoder optimized for both human and machine perception. However, to our knowledge, none of the existing works have yet investigated 3D point cloud sequence compression. 