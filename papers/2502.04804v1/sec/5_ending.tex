\section{Limitations and Future Works}
While this work primarily focuses on developing an RoI-based point cloud sequence encoder for 3D object detectors, as well as an efficient RoI detector, several aspects remain for future exploration: 
1) For simplicity, pre-trained back-end 3D object detectors are fine-tuned on uniformly compressed point clouds. The unseen RoI-encoded point cloud may compromise detection accuracy. 
2) We use a constant RoI QP for all point cloud frames in a sequence. Dynamic RoI QP assignment according to the frame's complexity may further improve the compression efficiency. 
3) We assume RoIs are exactly where objects are located. However, non-object regions may also contain information critical for object identification. A more adaptive RoI design may further improve the performance.
4) Our method could be extended from autonomous driving scenes to support a broader range of applications, such as indoor 3D scene understanding~\cite{dai2017scannet, armeni_cvpr16}.

\section{Conclusion}
In this paper, we introduce \methodname{}, a RoI-based point cloud sequence compression framework. We implemented and evaluated \methodname{} on the nuScenes dataset, and the results show its superiority. We believe the proposed method can potentially alleviate the storage and network bandwidth demands of point cloud-based object detection systems through its efficient RoI-based encoding scheme.

% We believe the proposed new sensing modality can inspire a large spectrum of sensing applications utilizing distributed satellites in space.


