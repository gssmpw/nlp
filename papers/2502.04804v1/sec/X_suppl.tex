\clearpage
\setcounter{page}{1}
\maketitlesupplementary


\section{Details of RoI detectors}
\label{sup-roi-detector}
\subsection{Header Design}
Inspired by CenterPoint \cite{yin2021center}, we separate the heatmap predictions of objects with different sizes into 6 tasks. Backbone features first pass through a shared convolution block, which includes a $3 \times 3$ convolution layer, a batch normalization layer, and a ReLU activation layer. The output of the first convolution block is then fed into separate task-specific branches for each task. Each task branch contains a deformable convolution layer with a group size of 4, a convolution block with a $3 \times 3$ convolution layer, a batch normalization layer, and a ReLU layer, followed by a $3 \times 3$ convolution output layer.

\subsection{Training}
We implement our RoI detector in PyTorch using the open-sourced MMDetection3D~\cite{mmdet3d2020}. We train the model for 20 epochs for the RoI detector using the Adam~\cite{kingma2014adam} optimizer with a learning rate of $10^{-4}$, batch size of 9, and a weight decay of 0.01.

\section{Details of Back-end 3D Object Detectors}
\label{sup-backend-detector}
\subsection{Training}
We pre-train the back-end detectors with lossless point clouds following the settings in MMDetection3D~\cite{mmdet3d2020}. Specifically, we train each detector for 20 epochs using the Adam optimizer. The batch size is set to 9, the learning rate is set to $10^{-4}$, and the weight decay is set to 0.01. We then fine-tune detectors on a mixed dataset for one epoch using the same training parameters in the pre-training phase. The mixed dataset is a uniform mixture of training sets encoded by vanilla VPCC with QP $\in \{20, 25, 30\}$ and lossless encoding.
\subsection{Complexity}

Table.~\ref{tab-backend-detector-complexity} shows the complexities of the back-end detectors.

\begin{table}[ht]
    \centering
    \renewcommand{\arraystretch}{\TABVSPACE} % vertical spacing
    \setlength{\tabcolsep}{2pt} % horizontal spacing
    \caption{Complexities of back-end detectors.}
    \begin{tabular}{c|c|c|c|c}
    \hline
     Model      & \multicolumn{2}{c|}{CenterPoint} & \multicolumn{2}{c}{BEVFusion-Lidar} \\
     
    \hline
      & FLOPs (G) & Param (M) &  FLOPs (G) & Param (M)  \\ 
\cline{2-3}         \cline{4-5}
    \hline
        Total & 119.33 & 6.11 & 85.98 & 5.15 \\ 
    \hline
    \end{tabular}
    \label{tab-backend-detector-complexity}
\end{table}



\section{Metric-bitrate Curves in Main Results}

\begin{figure}[t]
% \setlength{\abovecaptionskip}{0.2cm}
% \setlength{\belowcaptionskip}{-0.2cm}
  \centering
    \includegraphics[width=\linewidth]{figs/suppl/mAP-NDS-centerpoint.pdf}
  \caption[]{Metric-bitrate curve in \cref{sec-main-results} (CenterPoint)}
% \vspace{-0.4cm}
\label{fig-suppl-curve-centerpoint}
\end{figure}

\begin{figure}[t]
% \setlength{\abovecaptionskip}{0.2cm}
% \setlength{\belowcaptionskip}{-0.2cm}
  \centering
    \includegraphics[width=\linewidth]{figs/suppl/mAP-NDS-bevfusion.pdf}
  \caption[]{Metric-bitrate curve in \cref{sec-main-results} (BEVFusion-Lidar)}
% \vspace{-0.4cm}
\label{fig-suppl-curve-bevfusion}
\end{figure}

Fig.~\ref{fig-suppl-curve-centerpoint} and Fig.~\ref{fig-suppl-curve-bevfusion} give the mAP-bitrate and NDS-bitrate curves of \methodname{}, VPCC and Naive RoI in \cref{sec-main-results}.

% \section{Rationale}
% \label{sec:rationale}
% % 
% Having the supplementary compiled together with the main paper means that:
% % 
% \begin{itemize}
% \item The supplementary can back-reference sections of the main paper, for example, we can refer to \cref{sec:intro};
% \item The main paper can forward reference sub-sections within the supplementary explicitly (e.g. referring to a particular experiment); 
% \item When submitted to arXiv, the supplementary will already included at the end of the paper.
% \end{itemize}
% % 
% To split the supplementary pages from the main paper, you can use \href{https://support.apple.com/en-ca/guide/preview/prvw11793/mac#:~:text=Delete%20a%20page%20from%20a,or%20choose%20Edit%20%3E%20Delete).}{Preview (on macOS)}, \href{https://www.adobe.com/acrobat/how-to/delete-pages-from-pdf.html#:~:text=Choose%20%E2%80%9CTools%E2%80%9D%20%3E%20%E2%80%9COrganize,or%20pages%20from%20the%20file.}{Adobe Acrobat} (on all OSs), as well as \href{https://superuser.com/questions/517986/is-it-possible-to-delete-some-pages-of-a-pdf-document}{command line tools}.