%%%%%%%% ICML 2025 EXAMPLE LATEX SUBMISSION FILE %%%%%%%%%%%%%%%%%

\documentclass{article}

% Recommended, but optional, packages for figures and better typesetting:
\usepackage{microtype}
\usepackage{graphicx}
\usepackage{subfigure}
\usepackage{booktabs} % for professional tables

% hyperref makes hyperlinks in the resulting PDF.
% If your build breaks (sometimes temporarily if a hyperlink spans a page)
% please comment out the following usepackage line and replace
% \usepackage{icml2025} with \usepackage[nohyperref]{icml2025} above.
\usepackage{hyperref}


% Attempt to make hyperref and algorithmic work together better:
\newcommand{\theHalgorithm}{\arabic{algorithm}}

% Use the following line for the initial blind version submitted for review:
% \usepackage{icml2025}

% If accepted, instead use the following line for the camera-ready submission:
\usepackage[accepted]{icml2025}

% For theorems and such
\usepackage{amsmath}
\usepackage{amssymb}
\usepackage{mathtools}
\usepackage{amsthm}

% if you use cleveref..
\usepackage[capitalize,noabbrev]{cleveref}

%%%%%%%%%%%%%%%%%%%%%%%%%%%%%%%%
% THEOREMS
%%%%%%%%%%%%%%%%%%%%%%%%%%%%%%%%
\theoremstyle{plain}
\newtheorem{theorem}{Theorem}[section]
\newtheorem{proposition}[theorem]{Proposition}
\newtheorem{lemma}[theorem]{Lemma}
\newtheorem{corollary}[theorem]{Corollary}
\theoremstyle{definition}
\newtheorem{definition}[theorem]{Definition}
\newtheorem{assumption}[theorem]{Assumption}
\theoremstyle{remark}
\newtheorem{remark}[theorem]{Remark}

% Todonotes is useful during development; simply uncomment the next line
%    and comment out the line below the next line to turn off comments
%\usepackage[disable,textsize=tiny]{todonotes}
\usepackage[textsize=tiny]{todonotes}


% The \icmltitle you define below is probably too long as a header.
% Therefore, a short form for the running title is supplied here:
\icmltitlerunning{{\red ATA}: {\red A}daptive {\red T}ask {\red A}llocation for Efficient Resource Management in Distributed Machine Learning}



%
\setlength\unitlength{1mm}
\newcommand{\twodots}{\mathinner {\ldotp \ldotp}}
% bb font symbols
\newcommand{\Rho}{\mathrm{P}}
\newcommand{\Tau}{\mathrm{T}}

\newfont{\bbb}{msbm10 scaled 700}
\newcommand{\CCC}{\mbox{\bbb C}}

\newfont{\bb}{msbm10 scaled 1100}
\newcommand{\CC}{\mbox{\bb C}}
\newcommand{\PP}{\mbox{\bb P}}
\newcommand{\RR}{\mbox{\bb R}}
\newcommand{\QQ}{\mbox{\bb Q}}
\newcommand{\ZZ}{\mbox{\bb Z}}
\newcommand{\FF}{\mbox{\bb F}}
\newcommand{\GG}{\mbox{\bb G}}
\newcommand{\EE}{\mbox{\bb E}}
\newcommand{\NN}{\mbox{\bb N}}
\newcommand{\KK}{\mbox{\bb K}}
\newcommand{\HH}{\mbox{\bb H}}
\newcommand{\SSS}{\mbox{\bb S}}
\newcommand{\UU}{\mbox{\bb U}}
\newcommand{\VV}{\mbox{\bb V}}


\newcommand{\yy}{\mathbbm{y}}
\newcommand{\xx}{\mathbbm{x}}
\newcommand{\zz}{\mathbbm{z}}
\newcommand{\sss}{\mathbbm{s}}
\newcommand{\rr}{\mathbbm{r}}
\newcommand{\pp}{\mathbbm{p}}
\newcommand{\qq}{\mathbbm{q}}
\newcommand{\ww}{\mathbbm{w}}
\newcommand{\hh}{\mathbbm{h}}
\newcommand{\vvv}{\mathbbm{v}}

% Vectors

\newcommand{\av}{{\bf a}}
\newcommand{\bv}{{\bf b}}
\newcommand{\cv}{{\bf c}}
\newcommand{\dv}{{\bf d}}
\newcommand{\ev}{{\bf e}}
\newcommand{\fv}{{\bf f}}
\newcommand{\gv}{{\bf g}}
\newcommand{\hv}{{\bf h}}
\newcommand{\iv}{{\bf i}}
\newcommand{\jv}{{\bf j}}
\newcommand{\kv}{{\bf k}}
\newcommand{\lv}{{\bf l}}
\newcommand{\mv}{{\bf m}}
\newcommand{\nv}{{\bf n}}
\newcommand{\ov}{{\bf o}}
\newcommand{\pv}{{\bf p}}
\newcommand{\qv}{{\bf q}}
\newcommand{\rv}{{\bf r}}
\newcommand{\sv}{{\bf s}}
\newcommand{\tv}{{\bf t}}
\newcommand{\uv}{{\bf u}}
\newcommand{\wv}{{\bf w}}
\newcommand{\vv}{{\bf v}}
\newcommand{\xv}{{\bf x}}
\newcommand{\yv}{{\bf y}}
\newcommand{\zv}{{\bf z}}
\newcommand{\zerov}{{\bf 0}}
\newcommand{\onev}{{\bf 1}}

% Matrices

\newcommand{\Am}{{\bf A}}
\newcommand{\Bm}{{\bf B}}
\newcommand{\Cm}{{\bf C}}
\newcommand{\Dm}{{\bf D}}
\newcommand{\Em}{{\bf E}}
\newcommand{\Fm}{{\bf F}}
\newcommand{\Gm}{{\bf G}}
\newcommand{\Hm}{{\bf H}}
\newcommand{\Id}{{\bf I}}
\newcommand{\Jm}{{\bf J}}
\newcommand{\Km}{{\bf K}}
\newcommand{\Lm}{{\bf L}}
\newcommand{\Mm}{{\bf M}}
\newcommand{\Nm}{{\bf N}}
\newcommand{\Om}{{\bf O}}
\newcommand{\Pm}{{\bf P}}
\newcommand{\Qm}{{\bf Q}}
\newcommand{\Rm}{{\bf R}}
\newcommand{\Sm}{{\bf S}}
\newcommand{\Tm}{{\bf T}}
\newcommand{\Um}{{\bf U}}
\newcommand{\Wm}{{\bf W}}
\newcommand{\Vm}{{\bf V}}
\newcommand{\Xm}{{\bf X}}
\newcommand{\Ym}{{\bf Y}}
\newcommand{\Zm}{{\bf Z}}

% Calligraphic

\newcommand{\Ac}{{\cal A}}
\newcommand{\Bc}{{\cal B}}
\newcommand{\Cc}{{\cal C}}
\newcommand{\Dc}{{\cal D}}
\newcommand{\Ec}{{\cal E}}
\newcommand{\Fc}{{\cal F}}
\newcommand{\Gc}{{\cal G}}
\newcommand{\Hc}{{\cal H}}
\newcommand{\Ic}{{\cal I}}
\newcommand{\Jc}{{\cal J}}
\newcommand{\Kc}{{\cal K}}
\newcommand{\Lc}{{\cal L}}
\newcommand{\Mc}{{\cal M}}
\newcommand{\Nc}{{\cal N}}
\newcommand{\nc}{{\cal n}}
\newcommand{\Oc}{{\cal O}}
\newcommand{\Pc}{{\cal P}}
\newcommand{\Qc}{{\cal Q}}
\newcommand{\Rc}{{\cal R}}
\newcommand{\Sc}{{\cal S}}
\newcommand{\Tc}{{\cal T}}
\newcommand{\Uc}{{\cal U}}
\newcommand{\Wc}{{\cal W}}
\newcommand{\Vc}{{\cal V}}
\newcommand{\Xc}{{\cal X}}
\newcommand{\Yc}{{\cal Y}}
\newcommand{\Zc}{{\cal Z}}

% Bold greek letters

\newcommand{\alphav}{\hbox{\boldmath$\alpha$}}
\newcommand{\betav}{\hbox{\boldmath$\beta$}}
\newcommand{\gammav}{\hbox{\boldmath$\gamma$}}
\newcommand{\deltav}{\hbox{\boldmath$\delta$}}
\newcommand{\etav}{\hbox{\boldmath$\eta$}}
\newcommand{\lambdav}{\hbox{\boldmath$\lambda$}}
\newcommand{\epsilonv}{\hbox{\boldmath$\epsilon$}}
\newcommand{\nuv}{\hbox{\boldmath$\nu$}}
\newcommand{\muv}{\hbox{\boldmath$\mu$}}
\newcommand{\zetav}{\hbox{\boldmath$\zeta$}}
\newcommand{\phiv}{\hbox{\boldmath$\phi$}}
\newcommand{\psiv}{\hbox{\boldmath$\psi$}}
\newcommand{\thetav}{\hbox{\boldmath$\theta$}}
\newcommand{\tauv}{\hbox{\boldmath$\tau$}}
\newcommand{\omegav}{\hbox{\boldmath$\omega$}}
\newcommand{\xiv}{\hbox{\boldmath$\xi$}}
\newcommand{\sigmav}{\hbox{\boldmath$\sigma$}}
\newcommand{\piv}{\hbox{\boldmath$\pi$}}
\newcommand{\rhov}{\hbox{\boldmath$\rho$}}
\newcommand{\upsilonv}{\hbox{\boldmath$\upsilon$}}

\newcommand{\Gammam}{\hbox{\boldmath$\Gamma$}}
\newcommand{\Lambdam}{\hbox{\boldmath$\Lambda$}}
\newcommand{\Deltam}{\hbox{\boldmath$\Delta$}}
\newcommand{\Sigmam}{\hbox{\boldmath$\Sigma$}}
\newcommand{\Phim}{\hbox{\boldmath$\Phi$}}
\newcommand{\Pim}{\hbox{\boldmath$\Pi$}}
\newcommand{\Psim}{\hbox{\boldmath$\Psi$}}
\newcommand{\Thetam}{\hbox{\boldmath$\Theta$}}
\newcommand{\Omegam}{\hbox{\boldmath$\Omega$}}
\newcommand{\Xim}{\hbox{\boldmath$\Xi$}}


% Sans Serif small case

\newcommand{\Gsf}{{\sf G}}

\newcommand{\asf}{{\sf a}}
\newcommand{\bsf}{{\sf b}}
\newcommand{\csf}{{\sf c}}
\newcommand{\dsf}{{\sf d}}
\newcommand{\esf}{{\sf e}}
\newcommand{\fsf}{{\sf f}}
\newcommand{\gsf}{{\sf g}}
\newcommand{\hsf}{{\sf h}}
\newcommand{\isf}{{\sf i}}
\newcommand{\jsf}{{\sf j}}
\newcommand{\ksf}{{\sf k}}
\newcommand{\lsf}{{\sf l}}
\newcommand{\msf}{{\sf m}}
\newcommand{\nsf}{{\sf n}}
\newcommand{\osf}{{\sf o}}
\newcommand{\psf}{{\sf p}}
\newcommand{\qsf}{{\sf q}}
\newcommand{\rsf}{{\sf r}}
\newcommand{\ssf}{{\sf s}}
\newcommand{\tsf}{{\sf t}}
\newcommand{\usf}{{\sf u}}
\newcommand{\wsf}{{\sf w}}
\newcommand{\vsf}{{\sf v}}
\newcommand{\xsf}{{\sf x}}
\newcommand{\ysf}{{\sf y}}
\newcommand{\zsf}{{\sf z}}


% mixed symbols

\newcommand{\sinc}{{\hbox{sinc}}}
\newcommand{\diag}{{\hbox{diag}}}
\renewcommand{\det}{{\hbox{det}}}
\newcommand{\trace}{{\hbox{tr}}}
\newcommand{\sign}{{\hbox{sign}}}
\renewcommand{\arg}{{\hbox{arg}}}
\newcommand{\var}{{\hbox{var}}}
\newcommand{\cov}{{\hbox{cov}}}
\newcommand{\Ei}{{\rm E}_{\rm i}}
\renewcommand{\Re}{{\rm Re}}
\renewcommand{\Im}{{\rm Im}}
\newcommand{\eqdef}{\stackrel{\Delta}{=}}
\newcommand{\defines}{{\,\,\stackrel{\scriptscriptstyle \bigtriangleup}{=}\,\,}}
\newcommand{\<}{\left\langle}
\renewcommand{\>}{\right\rangle}
\newcommand{\herm}{{\sf H}}
\newcommand{\trasp}{{\sf T}}
\newcommand{\transp}{{\sf T}}
\renewcommand{\vec}{{\rm vec}}
\newcommand{\Psf}{{\sf P}}
\newcommand{\SINR}{{\sf SINR}}
\newcommand{\SNR}{{\sf SNR}}
\newcommand{\MMSE}{{\sf MMSE}}
\newcommand{\REF}{{\RED [REF]}}

% Markov chain
\usepackage{stmaryrd} % for \mkv 
\newcommand{\mkv}{-\!\!\!\!\minuso\!\!\!\!-}

% Colors

\newcommand{\RED}{\color[rgb]{1.00,0.10,0.10}}
\newcommand{\BLUE}{\color[rgb]{0,0,0.90}}
\newcommand{\GREEN}{\color[rgb]{0,0.80,0.20}}

%%%%%%%%%%%%%%%%%%%%%%%%%%%%%%%%%%%%%%%%%%
\usepackage{hyperref}
\hypersetup{
    bookmarks=true,         % show bookmarks bar?
    unicode=false,          % non-Latin characters in AcrobatÕs bookmarks
    pdftoolbar=true,        % show AcrobatÕs toolbar?
    pdfmenubar=true,        % show AcrobatÕs menu?
    pdffitwindow=false,     % window fit to page when opened
    pdfstartview={FitH},    % fits the width of the page to the window
%    pdftitle={My title},    % title
%    pdfauthor={Author},     % author
%    pdfsubject={Subject},   % subject of the document
%    pdfcreator={Creator},   % creator of the document
%    pdfproducer={Producer}, % producer of the document
%    pdfkeywords={keyword1} {key2} {key3}, % list of keywords
    pdfnewwindow=true,      % links in new window
    colorlinks=true,       % false: boxed links; true: colored links
    linkcolor=red,          % color of internal links (change box color with linkbordercolor)
    citecolor=green,        % color of links to bibliography
    filecolor=blue,      % color of file links
    urlcolor=blue           % color of external links
}
%%%%%%%%%%%%%%%%%%%%%%%%%%%%%%%%%%%%%%%%%%%


\usepackage{listings}


\begin{document}

\twocolumn[
\icmltitle{
    {\red ATA}: {\red A}daptive {\red T}ask {\red A}llocation for Efficient Resource Management \\
    in Distributed Machine Learning
    % \algname{ATA}: {\red A}daptive {\red T}ask {\red A}llocation for Efficient Resource Management in Distributed Machine Learning with Heterogeneous and Random Worker Times
    }


% It is OKAY to include author information, even for blind
% submissions: the style file will automatically remove it for you
% unless you've provided the [accepted] option to the icml2025
% package.

% List of affiliations: The first argument should be a (short)
% identifier you will use later to specify author affiliations
% Academic affiliations should list Department, University, City, Region, Country
% Industry affiliations should list Company, City, Region, Country

% You can specify symbols, otherwise they are numbered in order.
% Ideally, you should not use this facility. Affiliations will be numbered
% in order of appearance and this is the preferred way.
\icmlsetsymbol{equal}{*}

\begin{icmlauthorlist}
\icmlauthor{Artavazd Maranjyan}{kaust}
\icmlauthor{El Mehdi Saad}{kaust}
\icmlauthor{Peter Richt\'{a}rik}{kaust}
\icmlauthor{Francesco Orabona}{kaust}
\end{icmlauthorlist}


\icmlaffiliation{kaust}{King Abdullah University of Science and Technology (KAUST), Thuwal, Saudi Arabia}

\icmlcorrespondingauthor{Artavazd Maranjyan}{\href{https://artomaranjyan.github.io/}{https://artomaranjyan.github.io/}}
    

% You may provide any keywords that you
% find helpful for describing your paper; these are used to populate
% the "keywords" metadata in the PDF but will not be shown in the document
\icmlkeywords{Multi-Armed Bandit, UCB, adaptive task allocation, asynchronous methods, parallel methods, SGD}

\vskip 0.3in
]

% this must go after the closing bracket ] following \twocolumn[ ...

% This command actually creates the footnote in the first column
% listing the affiliations and the copyright notice.
% The command takes one argument, which is text to display at the start of the footnote.
% The \icmlEqualContribution command is standard text for equal contribution.
% Remove it (just {}) if you do not need this facility.

\printAffiliationsAndNotice{}  % leave blank if no need to mention equal contribution
% \printAffiliationsAndNotice{\icmlEqualContribution} % otherwise use the standard text.


\begin{abstract}
    % OpenReview abstract
    % Asynchronous methods are fundamental for parallelizing computations in distributed machine learning. 
    % They aim to accelerate training by fully utilizing all available resources.
    % However, their greedy approach can lead to inefficiencies using more computation than required, especially when computation times vary across devices.
    % If the computation times were known in advance, training could be fast and resource-efficient by assigning more tasks to faster workers.
    % The challenge lies in achieving this optimal allocation without prior knowledge of the computation time distributions.
    % In this paper, we propose ATA (Adaptive Task Allocation), a method that adapts to heterogeneous and random distributions of worker computation times.
    % Through rigorous theoretical analysis, we show that ATA identifies the optimal task allocation and performs comparably to methods with prior knowledge of computation times.
    % Experimental results further demonstrate that ATA is resource-efficient, significantly reducing costs compared to the greedy approach, which can be arbitrarily expensive depending on the number of workers.

    Asynchronous methods are fundamental for parallelizing computations in distributed machine learning. 
    They aim to accelerate training by fully utilizing all available resources.
    However, their greedy approach can lead to inefficiencies using more computation than required, especially when computation times vary across devices.
    If the computation times were known in advance, training could be fast and resource-efficient by assigning more tasks to faster workers.
    The challenge lies in achieving this optimal allocation without prior knowledge of the computation time distributions.
    In this paper, we propose \algname{ATA} ({\red A}daptive {\red T}ask {\red A}llocation), a method that adapts to heterogeneous and random distributions of worker computation times.
    Through rigorous theoretical analysis, we show that \algname{ATA} identifies the optimal task allocation and performs comparably to methods with prior knowledge of computation times.
    Experimental results further demonstrate that \algname{ATA} is resource-efficient, significantly reducing costs compared to the greedy approach, which can be arbitrarily expensive depending on the number of workers.
\end{abstract}

\section{Introduction}
\label{sec:introduction}
The business processes of organizations are experiencing ever-increasing complexity due to the large amount of data, high number of users, and high-tech devices involved \cite{martin2021pmopportunitieschallenges, beerepoot2023biggestbpmproblems}. This complexity may cause business processes to deviate from normal control flow due to unforeseen and disruptive anomalies \cite{adams2023proceddsriftdetection}. These control-flow anomalies manifest as unknown, skipped, and wrongly-ordered activities in the traces of event logs monitored from the execution of business processes \cite{ko2023adsystematicreview}. For the sake of clarity, let us consider an illustrative example of such anomalies. Figure \ref{FP_ANOMALIES} shows a so-called event log footprint, which captures the control flow relations of four activities of a hypothetical event log. In particular, this footprint captures the control-flow relations between activities \texttt{a}, \texttt{b}, \texttt{c} and \texttt{d}. These are the causal ($\rightarrow$) relation, concurrent ($\parallel$) relation, and other ($\#$) relations such as exclusivity or non-local dependency \cite{aalst2022pmhandbook}. In addition, on the right are six traces, of which five exhibit skipped, wrongly-ordered and unknown control-flow anomalies. For example, $\langle$\texttt{a b d}$\rangle$ has a skipped activity, which is \texttt{c}. Because of this skipped activity, the control-flow relation \texttt{b}$\,\#\,$\texttt{d} is violated, since \texttt{d} directly follows \texttt{b} in the anomalous trace.
\begin{figure}[!t]
\centering
\includegraphics[width=0.9\columnwidth]{images/FP_ANOMALIES.png}
\caption{An example event log footprint with six traces, of which five exhibit control-flow anomalies.}
\label{FP_ANOMALIES}
\end{figure}

\subsection{Control-flow anomaly detection}
Control-flow anomaly detection techniques aim to characterize the normal control flow from event logs and verify whether these deviations occur in new event logs \cite{ko2023adsystematicreview}. To develop control-flow anomaly detection techniques, \revision{process mining} has seen widespread adoption owing to process discovery and \revision{conformance checking}. On the one hand, process discovery is a set of algorithms that encode control-flow relations as a set of model elements and constraints according to a given modeling formalism \cite{aalst2022pmhandbook}; hereafter, we refer to the Petri net, a widespread modeling formalism. On the other hand, \revision{conformance checking} is an explainable set of algorithms that allows linking any deviations with the reference Petri net and providing the fitness measure, namely a measure of how much the Petri net fits the new event log \cite{aalst2022pmhandbook}. Many control-flow anomaly detection techniques based on \revision{conformance checking} (hereafter, \revision{conformance checking}-based techniques) use the fitness measure to determine whether an event log is anomalous \cite{bezerra2009pmad, bezerra2013adlogspais, myers2018icsadpm, pecchia2020applicationfailuresanalysispm}. 

The scientific literature also includes many \revision{conformance checking}-independent techniques for control-flow anomaly detection that combine specific types of trace encodings with machine/deep learning \cite{ko2023adsystematicreview, tavares2023pmtraceencoding}. Whereas these techniques are very effective, their explainability is challenging due to both the type of trace encoding employed and the machine/deep learning model used \cite{rawal2022trustworthyaiadvances,li2023explainablead}. Hence, in the following, we focus on the shortcomings of \revision{conformance checking}-based techniques to investigate whether it is possible to support the development of competitive control-flow anomaly detection techniques while maintaining the explainable nature of \revision{conformance checking}.
\begin{figure}[!t]
\centering
\includegraphics[width=\columnwidth]{images/HIGH_LEVEL_VIEW.png}
\caption{A high-level view of the proposed framework for combining \revision{process mining}-based feature extraction with dimensionality reduction for control-flow anomaly detection.}
\label{HIGH_LEVEL_VIEW}
\end{figure}

\subsection{Shortcomings of \revision{conformance checking}-based techniques}
Unfortunately, the detection effectiveness of \revision{conformance checking}-based techniques is affected by noisy data and low-quality Petri nets, which may be due to human errors in the modeling process or representational bias of process discovery algorithms \cite{bezerra2013adlogspais, pecchia2020applicationfailuresanalysispm, aalst2016pm}. Specifically, on the one hand, noisy data may introduce infrequent and deceptive control-flow relations that may result in inconsistent fitness measures, whereas, on the other hand, checking event logs against a low-quality Petri net could lead to an unreliable distribution of fitness measures. Nonetheless, such Petri nets can still be used as references to obtain insightful information for \revision{process mining}-based feature extraction, supporting the development of competitive and explainable \revision{conformance checking}-based techniques for control-flow anomaly detection despite the problems above. For example, a few works outline that token-based \revision{conformance checking} can be used for \revision{process mining}-based feature extraction to build tabular data and develop effective \revision{conformance checking}-based techniques for control-flow anomaly detection \cite{singh2022lapmsh, debenedictis2023dtadiiot}. However, to the best of our knowledge, the scientific literature lacks a structured proposal for \revision{process mining}-based feature extraction using the state-of-the-art \revision{conformance checking} variant, namely alignment-based \revision{conformance checking}.

\subsection{Contributions}
We propose a novel \revision{process mining}-based feature extraction approach with alignment-based \revision{conformance checking}. This variant aligns the deviating control flow with a reference Petri net; the resulting alignment can be inspected to extract additional statistics such as the number of times a given activity caused mismatches \cite{aalst2022pmhandbook}. We integrate this approach into a flexible and explainable framework for developing techniques for control-flow anomaly detection. The framework combines \revision{process mining}-based feature extraction and dimensionality reduction to handle high-dimensional feature sets, achieve detection effectiveness, and support explainability. Notably, in addition to our proposed \revision{process mining}-based feature extraction approach, the framework allows employing other approaches, enabling a fair comparison of multiple \revision{conformance checking}-based and \revision{conformance checking}-independent techniques for control-flow anomaly detection. Figure \ref{HIGH_LEVEL_VIEW} shows a high-level view of the framework. Business processes are monitored, and event logs obtained from the database of information systems. Subsequently, \revision{process mining}-based feature extraction is applied to these event logs and tabular data input to dimensionality reduction to identify control-flow anomalies. We apply several \revision{conformance checking}-based and \revision{conformance checking}-independent framework techniques to publicly available datasets, simulated data of a case study from railways, and real-world data of a case study from healthcare. We show that the framework techniques implementing our approach outperform the baseline \revision{conformance checking}-based techniques while maintaining the explainable nature of \revision{conformance checking}.

In summary, the contributions of this paper are as follows.
\begin{itemize}
    \item{
        A novel \revision{process mining}-based feature extraction approach to support the development of competitive and explainable \revision{conformance checking}-based techniques for control-flow anomaly detection.
    }
    \item{
        A flexible and explainable framework for developing techniques for control-flow anomaly detection using \revision{process mining}-based feature extraction and dimensionality reduction.
    }
    \item{
        Application to synthetic and real-world datasets of several \revision{conformance checking}-based and \revision{conformance checking}-independent framework techniques, evaluating their detection effectiveness and explainability.
    }
\end{itemize}

The rest of the paper is organized as follows.
\begin{itemize}
    \item Section \ref{sec:related_work} reviews the existing techniques for control-flow anomaly detection, categorizing them into \revision{conformance checking}-based and \revision{conformance checking}-independent techniques.
    \item Section \ref{sec:abccfe} provides the preliminaries of \revision{process mining} to establish the notation used throughout the paper, and delves into the details of the proposed \revision{process mining}-based feature extraction approach with alignment-based \revision{conformance checking}.
    \item Section \ref{sec:framework} describes the framework for developing \revision{conformance checking}-based and \revision{conformance checking}-independent techniques for control-flow anomaly detection that combine \revision{process mining}-based feature extraction and dimensionality reduction.
    \item Section \ref{sec:evaluation} presents the experiments conducted with multiple framework and baseline techniques using data from publicly available datasets and case studies.
    \item Section \ref{sec:conclusions} draws the conclusions and presents future work.
\end{itemize}
\putsec{related}{Related Work}

\noindent \textbf{Efficient Radiance Field Rendering.}
%
The introduction of Neural Radiance Fields (NeRF)~\cite{mil:sri20} has
generated significant interest in efficient 3D scene representation and
rendering for radiance fields.
%
Over the past years, there has been a large amount of research aimed at
accelerating NeRFs through algorithmic or software
optimizations~\cite{mul:eva22,fri:yu22,che:fun23,sun:sun22}, and the
development of hardware
accelerators~\cite{lee:cho23,li:li23,son:wen23,mub:kan23,fen:liu24}.
%
The state-of-the-art method, 3D Gaussian splatting~\cite{ker:kop23}, has
further fueled interest in accelerating radiance field
rendering~\cite{rad:ste24,lee:lee24,nie:stu24,lee:rho24,ham:mel24} as it
employs rasterization primitives that can be rendered much faster than NeRFs.
%
However, previous research focused on software graphics rendering on
programmable cores or building dedicated hardware accelerators. In contrast,
\name{} investigates the potential of efficient radiance field rendering while
utilizing fixed-function units in graphics hardware.
%
To our knowledge, this is the first work that assesses the performance
implications of rendering Gaussian-based radiance fields on the hardware
graphics pipeline with software and hardware optimizations.

%%%%%%%%%%%%%%%%%%%%%%%%%%%%%%%%%%%%%%%%%%%%%%%%%%%%%%%%%%%%%%%%%%%%%%%%%%
\myparagraph{Enhancing Graphics Rendering Hardware.}
%
The performance advantage of executing graphics rendering on either
programmable shader cores or fixed-function units varies depending on the
rendering methods and hardware designs.
%
Previous studies have explored the performance implication of graphics hardware
design by developing simulation infrastructures for graphics
workloads~\cite{bar:gon06,gub:aam19,tin:sax23,arn:par13}.
%
Additionally, several studies have aimed to improve the performance of
special-purpose hardware such as ray tracing units in graphics
hardware~\cite{cho:now23,liu:cha21} and proposed hardware accelerators for
graphics applications~\cite{lu:hua17,ram:gri09}.
%
In contrast to these works, which primarily evaluate traditional graphics
workloads, our work focuses on improving the performance of volume rendering
workloads, such as Gaussian splatting, which require blending a huge number of
fragments per pixel.

%%%%%%%%%%%%%%%%%%%%%%%%%%%%%%%%%%%%%%%%%%%%%%%%%%%%%%%%%%%%%%%%%%%%%%%%%%
%
In the context of multi-sample anti-aliasing, prior work proposed reducing the
amount of redundant shading by merging fragments from adjacent triangles in a
mesh at the quad granularity~\cite{fat:bou10}.
%
While both our work and quad-fragment merging (QFM)~\cite{fat:bou10} aim to
reduce operations by merging quads, our proposed technique differs from QFM in
many aspects.
%
Our method aims to blend \emph{overlapping primitives} along the depth
direction and applies to quads from any primitive. In contrast, QFM merges quad
fragments from small (e.g., pixel-sized) triangles that \emph{share} an edge
(i.e., \emph{connected}, \emph{non-overlapping} triangles).
%
As such, QFM is not applicable to the scenes consisting of a number of
unconnected transparent triangles, such as those in 3D Gaussian splatting.
%
In addition, our method computes the \emph{exact} color for each pixel by
offloading blending operations from ROPs to shader units, whereas QFM
\emph{approximates} pixel colors by using the color from one triangle when
multiple triangles are merged into a single quad.


\newcommand{\tabincell}[2]{\begin{tabular}{@{}#1@{}}#2\end{tabular}}
\newcommand{\rowstyle}[1]{\gdef\currentrowstyle{#1}%
	#1\ignorespaces
}

\newcommand{\className}[1]{\textbf{\sf #1}}
\newcommand{\functionName}[1]{\textbf{\sf #1}}
\newcommand{\objectName}[1]{\textbf{\sf #1}}
\newcommand{\annotation}[1]{\textbf{#1}}
\newcommand{\todo}[1]{\textcolor{blue}{\textbf{[[TODO: #1]]}}}
\newcommand{\change}[1]{\textcolor{blue}{#1}}
\newcommand{\fetch}[1]{\textbf{\em #1}}
\newcommand{\phead}[1]{\vspace{1mm} \noindent {\bf #1}}
\newcommand{\wei}[1]{\textcolor{blue}{{\it [Wei says: #1]}}}
\newcommand{\peter}[1]{\textcolor{red}{{\it [Peter says: #1]}}}
\newcommand{\zhenhao}[1]{\textcolor{dkblue}{{\it [Zhenhao says: #1]}}}
\newcommand{\feng}[1]{\textcolor{magenta}{{\it [Feng says: #1]}}}
\newcommand{\jinqiu}[1]{\textcolor{red}{{\it [Jinqiu says: #1]}}}
\newcommand{\shouvick}[1]{\textcolor{violet(ryb)}{{\it [Shouvick says: #1]}}}
\newcommand{\pattern}[1]{\emph{#1}}
%\newcommand{\tool}{{{DectGUILag}}\xspace}
\newcommand{\tool}{{{GUIWatcher}}\xspace}


\newcommand{\guo}[1]{\textcolor{yellow}{{\it [Linqiang says: #1]}}}

\newcommand{\rqbox}[1]{\begin{tcolorbox}[left=4pt,right=4pt,top=4pt,bottom=4pt,colback=gray!5,colframe=gray!40!black,before skip=2pt,after skip=2pt]#1\end{tcolorbox}}

\begin{algorithm}[ht!]
\caption{\textit{NovelSelect}}
\label{alg:novelselect}
\begin{algorithmic}[1]
\State \textbf{Input:} Data pool $\mathcal{X}^{all}$, data budget $n$
\State Initialize an empty dataset, $\mathcal{X} \gets \emptyset$
\While{$|\mathcal{X}| < n$}
    \State $x^{new} \gets \arg\max_{x \in \mathcal{X}^{all}} v(x)$
    \State $\mathcal{X} \gets \mathcal{X} \cup \{x^{new}\}$
    \State $\mathcal{X}^{all} \gets \mathcal{X}^{all} \setminus \{x^{new}\}$
\EndWhile
\State \textbf{return} $\mathcal{X}$
\end{algorithmic}
\end{algorithm}

\section{Theoretical Results and Derivation of \shortname}\label{sec:theory}
In this section, we provide a formal derivation of the algorithm, discussing the mathematical properties of Gaussian Weights and outlining the structured formalism and rationale underlying \shortname.

The stochastic process induced by the optimization algorithm in the local update step, allows the evolution of the empirical loss to be modeled using random variables within a probabilistic framework. We denote random variables with capital letters (\eg , $X$), and their realizations with lowercase letters (\eg, $x$).

The observed loss process $l_k^{t,s}$ is the outcome of a stochastic process $L_k^{t,s}$, and the rewards $r_k^{t,s}$, computed according to Eq. \ref{eq:reward}, are samples from a random reward $R_k^{t,s}$ supported in $(0,1)$, whose expectation $\mathbb{E}[R_k^{t,s}]$ is lower for out-of-distribution clients and higher for in-distribution ones. To estimate the expected reward $\mathbb{E}[R_k^{t,s}]$ we introduce the r.v. $\Omega_k^t = 1/S \sum_{s \in [S]} R_k^{t,s}$, which is an estimator less affected by noisy fluctuations in the empirical loss. Due to the linearity of the expectation operator, the expected reward $\mathbb{E}[R_k^{t,s}]$ for the $k$-th client at round $t$, local iteration $s$ equals the expected Gaussian reward $ \mathbb{E}[\Omega_k^t]$ that, to simplify the notation, we denote by $\mu_k$. $\mu_k$ is the theoretical value that we aim to estimate by designing our Gaussian weights $\Gamma_k^t$ appropriately, as it encodes the ideal reward to quantify the closeness of the distribution of each client $k$ to the main distribution. Note that the process is stationary by construction. Therefore, it does not depend on $t$ but differs between clients, as it reaches a higher value for in-distribution clients and a lower for out-of-distribution clients.

To rigorously motivate the construction of our algorithm and the reliability of the weights, we introduce the following theoretical results. Theorems \ref{thm_main:1} and \ref{thm_main:weak_conv} demonstrate that the weights converge to a finite value and, more importantly, that this limit serves as an unbiased estimator of the theoretical reward $\mu_k$. The first theorem provides a strong convergence result, showing that, with suitable choices of the sequence $\{\alpha_t\}_t$, the expectation of the Gaussian weights $\Gamma_k^t$ converges to $\mu_k$ in $L^2$ and almost surely. In addition, Theorem \ref{thm_main:weak_conv} extends this to the case of constant $\alpha_t$, proving that the weights still converge and remain unbiased estimators of the rewards as $t \to \infty$. 
\begin{theorem}\label{thm_main:1}
Let $\{\alpha_t\}_{t = 1}^\infty$ be a sequence of positive real values, and $\{\Gamma_k^t\}_{t=1}^\infty$ the sequence of Gaussian weights. If $\{\alpha_t\}_{t = 1}^\infty \in l^2(\mathbb{N})/l^1(\mathbb{N})$, then $\Gamma_k^t$ converges in $L^2$. Furthermore, for $t\to\infty$, 
\begin{equation}
    \Gamma_k^t \longrightarrow \mu_k\,\,\, a.s.
\end{equation}
\end{theorem}
\begin{theorem}\label{thm_main:weak_conv}
Let $\alpha \in (0,1)$ be a fixed constant, then in the limit $t \to \infty$, the expectation of the weights converges to the individual theoretical reward $\mu_k$, for each client $k = 1,\dots, K$, \ie,
\begin{equation}
    \mathbb{E}[\Gamma_k^t]\longrightarrow \mu_k\,\,\,t\to\infty\,.
\end{equation}
\end{theorem}
Proposition \ref{prop_var_main} shows that Gaussian weights reduce the variance of the estimate, thus decreasing the error and enabling the construction of a confidence interval for $\mu_k$.
\begin{proposition}\label{prop_var_main}
The variance of the weights $\Gamma_k^t$ is smaller than the variance $\sigma_k^2$ of the theoretical rewards $R_k^{t,s}$.
\end{proposition} 

Complete proofs of Theorems \ref{thm_main:1}, \ref{thm_main:weak_conv}, and Proposition \ref{prop_var_main} are provided in Appendix \ref{app:fgw}. Additionally, Appendix \ref{app:fgw} includes further analysis of \shortname. Specifically, Proposition \ref{prop:bounded_matrix} demonstrates that the entries of the interaction matrix $P$ are bounded, while Theorem \ref{thm:samplerate} establishes a sufficient condition for conserving the sampling rate during the recursive procedure.


\section{Wasserstein Adjusted Score}\label{clustereing_metric}

In the previous Section we observed that when clustering clients according to different heterogeneity levels, the outcome must be evaluated using a metric that assesses the cohesion of individual distributions. In this Section, we introduce a novel metric to evaluate the performance of clustering algorithms in FL. This metric, derived from the Wasserstein distance \citep{kantorovich1942translocation}, quantifies the cohesion of client groups based on their class distribution similarities. 

We propose a general method for adapting clustering metrics to account for class imbalances. This adjustment is particularly relevant when the underlying class distributions across clients are skewed. The formal derivation and mathematical details of the proposed metric are provided in Appendix \ref{app:clustering}. We now provide a high-level overview of our new metric.

Consider a generic clustering metric $s$, e.g. Davies-Bouldin score \citep{davies1979cluster} or the Silhouette score \citep{rousseeuw1987silhouettes}. Let $C$ denote the total number of classes, and $x_i^k$ the empirical frequency of the $i$-th class in the $k$-th client's local training set. Following theoretical reasonings, as shown in Appendix \ref{app:clustering}, the empirical frequency vector for client $k$, denoted by $x_{(i)}^k$, is ordered according to the rank statistic of the class frequencies, \ie  $x_{(i)}^k \geq x_{(i+1)}^k$ for any $i = 1, \dots, C-1$.
The class-adjusted clustering metric $\tilde{s}$ is defined as the standard clustering metric $s$ computed on the ranked frequency vectors $x_{(i)}^k$.  Specifically, the distance between two clients $j$ and $k$ results in
\vspace{-.8em}
\begin{equation}\label{lab_dist_class}
    \dfrac{1}{C}\left(\sum_{i = 1}^C \left | x_{(i)}^k - x_{(i)}^j \right | ^2\right)^{1/2}\,.
\end{equation}
This modification ensures that the clustering evaluation is sensitive to the distributional characteristics of the class imbalance. As we show in Appendix \ref{app:clustering}, this adjustment is mathematically equivalent to assessing the dispersion between the empirical class probability distributions of different clients using the Wasserstein distance, also known as the Kantorovich-Rubenstein metric \citep{kantorovich1942translocation}. This equivalence highlights the theoretical soundness of using ranked class frequencies to better capture variations in class distributions when evaluating clustering outcomes in FL.

\section{Experiments}
\label{sec:experiments}
The experiments are designed to address two key research questions.
First, \textbf{RQ1} evaluates whether the average $L_2$-norm of the counterfactual perturbation vectors ($\overline{||\perturb||}$) decreases as the model overfits the data, thereby providing further empirical validation for our hypothesis.
Second, \textbf{RQ2} evaluates the ability of the proposed counterfactual regularized loss, as defined in (\ref{eq:regularized_loss2}), to mitigate overfitting when compared to existing regularization techniques.

% The experiments are designed to address three key research questions. First, \textbf{RQ1} investigates whether the mean perturbation vector norm decreases as the model overfits the data, aiming to further validate our intuition. Second, \textbf{RQ2} explores whether the mean perturbation vector norm can be effectively leveraged as a regularization term during training, offering insights into its potential role in mitigating overfitting. Finally, \textbf{RQ3} examines whether our counterfactual regularizer enables the model to achieve superior performance compared to existing regularization methods, thus highlighting its practical advantage.

\subsection{Experimental Setup}
\textbf{\textit{Datasets, Models, and Tasks.}}
The experiments are conducted on three datasets: \textit{Water Potability}~\cite{kadiwal2020waterpotability}, \textit{Phomene}~\cite{phomene}, and \textit{CIFAR-10}~\cite{krizhevsky2009learning}. For \textit{Water Potability} and \textit{Phomene}, we randomly select $80\%$ of the samples for the training set, and the remaining $20\%$ for the test set, \textit{CIFAR-10} comes already split. Furthermore, we consider the following models: Logistic Regression, Multi-Layer Perceptron (MLP) with 100 and 30 neurons on each hidden layer, and PreactResNet-18~\cite{he2016cvecvv} as a Convolutional Neural Network (CNN) architecture.
We focus on binary classification tasks and leave the extension to multiclass scenarios for future work. However, for datasets that are inherently multiclass, we transform the problem into a binary classification task by selecting two classes, aligning with our assumption.

\smallskip
\noindent\textbf{\textit{Evaluation Measures.}} To characterize the degree of overfitting, we use the test loss, as it serves as a reliable indicator of the model's generalization capability to unseen data. Additionally, we evaluate the predictive performance of each model using the test accuracy.

\smallskip
\noindent\textbf{\textit{Baselines.}} We compare CF-Reg with the following regularization techniques: L1 (``Lasso''), L2 (``Ridge''), and Dropout.

\smallskip
\noindent\textbf{\textit{Configurations.}}
For each model, we adopt specific configurations as follows.
\begin{itemize}
\item \textit{Logistic Regression:} To induce overfitting in the model, we artificially increase the dimensionality of the data beyond the number of training samples by applying a polynomial feature expansion. This approach ensures that the model has enough capacity to overfit the training data, allowing us to analyze the impact of our counterfactual regularizer. The degree of the polynomial is chosen as the smallest degree that makes the number of features greater than the number of data.
\item \textit{Neural Networks (MLP and CNN):} To take advantage of the closed-form solution for computing the optimal perturbation vector as defined in (\ref{eq:opt-delta}), we use a local linear approximation of the neural network models. Hence, given an instance $\inst_i$, we consider the (optimal) counterfactual not with respect to $\model$ but with respect to:
\begin{equation}
\label{eq:taylor}
    \model^{lin}(\inst) = \model(\inst_i) + \nabla_{\inst}\model(\inst_i)(\inst - \inst_i),
\end{equation}
where $\model^{lin}$ represents the first-order Taylor approximation of $\model$ at $\inst_i$.
Note that this step is unnecessary for Logistic Regression, as it is inherently a linear model.
\end{itemize}

\smallskip
\noindent \textbf{\textit{Implementation Details.}} We run all experiments on a machine equipped with an AMD Ryzen 9 7900 12-Core Processor and an NVIDIA GeForce RTX 4090 GPU. Our implementation is based on the PyTorch Lightning framework. We use stochastic gradient descent as the optimizer with a learning rate of $\eta = 0.001$ and no weight decay. We use a batch size of $128$. The training and test steps are conducted for $6000$ epochs on the \textit{Water Potability} and \textit{Phoneme} datasets, while for the \textit{CIFAR-10} dataset, they are performed for $200$ epochs.
Finally, the contribution $w_i^{\varepsilon}$ of each training point $\inst_i$ is uniformly set as $w_i^{\varepsilon} = 1~\forall i\in \{1,\ldots,m\}$.

The source code implementation for our experiments is available at the following GitHub repository: \url{https://anonymous.4open.science/r/COCE-80B4/README.md} 

\subsection{RQ1: Counterfactual Perturbation vs. Overfitting}
To address \textbf{RQ1}, we analyze the relationship between the test loss and the average $L_2$-norm of the counterfactual perturbation vectors ($\overline{||\perturb||}$) over training epochs.

In particular, Figure~\ref{fig:delta_loss_epochs} depicts the evolution of $\overline{||\perturb||}$ alongside the test loss for an MLP trained \textit{without} regularization on the \textit{Water Potability} dataset. 
\begin{figure}[ht]
    \centering
    \includegraphics[width=0.85\linewidth]{img/delta_loss_epochs.png}
    \caption{The average counterfactual perturbation vector $\overline{||\perturb||}$ (left $y$-axis) and the cross-entropy test loss (right $y$-axis) over training epochs ($x$-axis) for an MLP trained on the \textit{Water Potability} dataset \textit{without} regularization.}
    \label{fig:delta_loss_epochs}
\end{figure}

The plot shows a clear trend as the model starts to overfit the data (evidenced by an increase in test loss). 
Notably, $\overline{||\perturb||}$ begins to decrease, which aligns with the hypothesis that the average distance to the optimal counterfactual example gets smaller as the model's decision boundary becomes increasingly adherent to the training data.

It is worth noting that this trend is heavily influenced by the choice of the counterfactual generator model. In particular, the relationship between $\overline{||\perturb||}$ and the degree of overfitting may become even more pronounced when leveraging more accurate counterfactual generators. However, these models often come at the cost of higher computational complexity, and their exploration is left to future work.

Nonetheless, we expect that $\overline{||\perturb||}$ will eventually stabilize at a plateau, as the average $L_2$-norm of the optimal counterfactual perturbations cannot vanish to zero.

% Additionally, the choice of employing the score-based counterfactual explanation framework to generate counterfactuals was driven to promote computational efficiency.

% Future enhancements to the framework may involve adopting models capable of generating more precise counterfactuals. While such approaches may yield to performance improvements, they are likely to come at the cost of increased computational complexity.


\subsection{RQ2: Counterfactual Regularization Performance}
To answer \textbf{RQ2}, we evaluate the effectiveness of the proposed counterfactual regularization (CF-Reg) by comparing its performance against existing baselines: unregularized training loss (No-Reg), L1 regularization (L1-Reg), L2 regularization (L2-Reg), and Dropout.
Specifically, for each model and dataset combination, Table~\ref{tab:regularization_comparison} presents the mean value and standard deviation of test accuracy achieved by each method across 5 random initialization. 

The table illustrates that our regularization technique consistently delivers better results than existing methods across all evaluated scenarios, except for one case -- i.e., Logistic Regression on the \textit{Phomene} dataset. 
However, this setting exhibits an unusual pattern, as the highest model accuracy is achieved without any regularization. Even in this case, CF-Reg still surpasses other regularization baselines.

From the results above, we derive the following key insights. First, CF-Reg proves to be effective across various model types, ranging from simple linear models (Logistic Regression) to deep architectures like MLPs and CNNs, and across diverse datasets, including both tabular and image data. 
Second, CF-Reg's strong performance on the \textit{Water} dataset with Logistic Regression suggests that its benefits may be more pronounced when applied to simpler models. However, the unexpected outcome on the \textit{Phoneme} dataset calls for further investigation into this phenomenon.


\begin{table*}[h!]
    \centering
    \caption{Mean value and standard deviation of test accuracy across 5 random initializations for different model, dataset, and regularization method. The best results are highlighted in \textbf{bold}.}
    \label{tab:regularization_comparison}
    \begin{tabular}{|c|c|c|c|c|c|c|}
        \hline
        \textbf{Model} & \textbf{Dataset} & \textbf{No-Reg} & \textbf{L1-Reg} & \textbf{L2-Reg} & \textbf{Dropout} & \textbf{CF-Reg (ours)} \\ \hline
        Logistic Regression   & \textit{Water}   & $0.6595 \pm 0.0038$   & $0.6729 \pm 0.0056$   & $0.6756 \pm 0.0046$  & N/A    & $\mathbf{0.6918 \pm 0.0036}$                     \\ \hline
        MLP   & \textit{Water}   & $0.6756 \pm 0.0042$   & $0.6790 \pm 0.0058$   & $0.6790 \pm 0.0023$  & $0.6750 \pm 0.0036$    & $\mathbf{0.6802 \pm 0.0046}$                    \\ \hline
%        MLP   & \textit{Adult}   & $0.8404 \pm 0.0010$   & $\mathbf{0.8495 \pm 0.0007}$   & $0.8489 \pm 0.0014$  & $\mathbf{0.8495 \pm 0.0016}$     & $0.8449 \pm 0.0019$                    \\ \hline
        Logistic Regression   & \textit{Phomene}   & $\mathbf{0.8148 \pm 0.0020}$   & $0.8041 \pm 0.0028$   & $0.7835 \pm 0.0176$  & N/A    & $0.8098 \pm 0.0055$                     \\ \hline
        MLP   & \textit{Phomene}   & $0.8677 \pm 0.0033$   & $0.8374 \pm 0.0080$   & $0.8673 \pm 0.0045$  & $0.8672 \pm 0.0042$     & $\mathbf{0.8718 \pm 0.0040}$                    \\ \hline
        CNN   & \textit{CIFAR-10} & $0.6670 \pm 0.0233$   & $0.6229 \pm 0.0850$   & $0.7348 \pm 0.0365$   & N/A    & $\mathbf{0.7427 \pm 0.0571}$                     \\ \hline
    \end{tabular}
\end{table*}

\begin{table*}[htb!]
    \centering
    \caption{Hyperparameter configurations utilized for the generation of Table \ref{tab:regularization_comparison}. For our regularization the hyperparameters are reported as $\mathbf{\alpha/\beta}$.}
    \label{tab:performance_parameters}
    \begin{tabular}{|c|c|c|c|c|c|c|}
        \hline
        \textbf{Model} & \textbf{Dataset} & \textbf{No-Reg} & \textbf{L1-Reg} & \textbf{L2-Reg} & \textbf{Dropout} & \textbf{CF-Reg (ours)} \\ \hline
        Logistic Regression   & \textit{Water}   & N/A   & $0.0093$   & $0.6927$  & N/A    & $0.3791/1.0355$                     \\ \hline
        MLP   & \textit{Water}   & N/A   & $0.0007$   & $0.0022$  & $0.0002$    & $0.2567/1.9775$                    \\ \hline
        Logistic Regression   &
        \textit{Phomene}   & N/A   & $0.0097$   & $0.7979$  & N/A    & $0.0571/1.8516$                     \\ \hline
        MLP   & \textit{Phomene}   & N/A   & $0.0007$   & $4.24\cdot10^{-5}$  & $0.0015$    & $0.0516/2.2700$                    \\ \hline
       % MLP   & \textit{Adult}   & N/A   & $0.0018$   & $0.0018$  & $0.0601$     & $0.0764/2.2068$                    \\ \hline
        CNN   & \textit{CIFAR-10} & N/A   & $0.0050$   & $0.0864$ & N/A    & $0.3018/
        2.1502$                     \\ \hline
    \end{tabular}
\end{table*}

\begin{table*}[htb!]
    \centering
    \caption{Mean value and standard deviation of training time across 5 different runs. The reported time (in seconds) corresponds to the generation of each entry in Table \ref{tab:regularization_comparison}. Times are }
    \label{tab:times}
    \begin{tabular}{|c|c|c|c|c|c|c|}
        \hline
        \textbf{Model} & \textbf{Dataset} & \textbf{No-Reg} & \textbf{L1-Reg} & \textbf{L2-Reg} & \textbf{Dropout} & \textbf{CF-Reg (ours)} \\ \hline
        Logistic Regression   & \textit{Water}   & $222.98 \pm 1.07$   & $239.94 \pm 2.59$   & $241.60 \pm 1.88$  & N/A    & $251.50 \pm 1.93$                     \\ \hline
        MLP   & \textit{Water}   & $225.71 \pm 3.85$   & $250.13 \pm 4.44$   & $255.78 \pm 2.38$  & $237.83 \pm 3.45$    & $266.48 \pm 3.46$                    \\ \hline
        Logistic Regression   & \textit{Phomene}   & $266.39 \pm 0.82$ & $367.52 \pm 6.85$   & $361.69 \pm 4.04$  & N/A   & $310.48 \pm 0.76$                    \\ \hline
        MLP   &
        \textit{Phomene} & $335.62 \pm 1.77$   & $390.86 \pm 2.11$   & $393.96 \pm 1.95$ & $363.51 \pm 5.07$    & $403.14 \pm 1.92$                     \\ \hline
       % MLP   & \textit{Adult}   & N/A   & $0.0018$   & $0.0018$  & $0.0601$     & $0.0764/2.2068$                    \\ \hline
        CNN   & \textit{CIFAR-10} & $370.09 \pm 0.18$   & $395.71 \pm 0.55$   & $401.38 \pm 0.16$ & N/A    & $1287.8 \pm 0.26$                     \\ \hline
    \end{tabular}
\end{table*}

\subsection{Feasibility of our Method}
A crucial requirement for any regularization technique is that it should impose minimal impact on the overall training process.
In this respect, CF-Reg introduces an overhead that depends on the time required to find the optimal counterfactual example for each training instance. 
As such, the more sophisticated the counterfactual generator model probed during training the higher would be the time required. However, a more advanced counterfactual generator might provide a more effective regularization. We discuss this trade-off in more details in Section~\ref{sec:discussion}.

Table~\ref{tab:times} presents the average training time ($\pm$ standard deviation) for each model and dataset combination listed in Table~\ref{tab:regularization_comparison}.
We can observe that the higher accuracy achieved by CF-Reg using the score-based counterfactual generator comes with only minimal overhead. However, when applied to deep neural networks with many hidden layers, such as \textit{PreactResNet-18}, the forward derivative computation required for the linearization of the network introduces a more noticeable computational cost, explaining the longer training times in the table.

\subsection{Hyperparameter Sensitivity Analysis}
The proposed counterfactual regularization technique relies on two key hyperparameters: $\alpha$ and $\beta$. The former is intrinsic to the loss formulation defined in (\ref{eq:cf-train}), while the latter is closely tied to the choice of the score-based counterfactual explanation method used.

Figure~\ref{fig:test_alpha_beta} illustrates how the test accuracy of an MLP trained on the \textit{Water Potability} dataset changes for different combinations of $\alpha$ and $\beta$.

\begin{figure}[ht]
    \centering
    \includegraphics[width=0.85\linewidth]{img/test_acc_alpha_beta.png}
    \caption{The test accuracy of an MLP trained on the \textit{Water Potability} dataset, evaluated while varying the weight of our counterfactual regularizer ($\alpha$) for different values of $\beta$.}
    \label{fig:test_alpha_beta}
\end{figure}

We observe that, for a fixed $\beta$, increasing the weight of our counterfactual regularizer ($\alpha$) can slightly improve test accuracy until a sudden drop is noticed for $\alpha > 0.1$.
This behavior was expected, as the impact of our penalty, like any regularization term, can be disruptive if not properly controlled.

Moreover, this finding further demonstrates that our regularization method, CF-Reg, is inherently data-driven. Therefore, it requires specific fine-tuning based on the combination of the model and dataset at hand.

% In the unusual situation where you want a paper to appear in the
% references without citing it in the main text, use \nocite
% \nocite{langley00}

\bibliography{bib}
\bibliographystyle{icml2025}
%\newpage
%\bnote{Say something about: participants in condition limits occurrence of endpoints in communicates, endpoint statements. Either 1) we severly restrict it in the semi-formal development, like in the first veymont tool paper, and we mention in the impl section that it can be done more leniently, or 2) the transformation needs to describe it. As I think it will result in some tricky proof obligations, probably 1) is better for now. Then this assumption needs to be documented.}
 
\bnote{Mention that symmetric version where the receiving side is a range omitted for brevity. Or just generality only supported by implementation. ``See implementation section''}

\bnote{TODO: Document typical elements that do not match the grammar, e.g. $E_l$ and $E_h$, use of $i$ for tid/index in endpoint family, etc.}

%%%%%%%%%%%%%%%%%%%%%%%%%%%%%%%%%%%%%%%%%%%%%%%%%%%%%%%%%%%%%%%%%%%%%%%%%%%%%%%
%%%%%%%%%%%%%%%%%%%%%%%%%%%%%%%%%%%%%%%%%%%%%%%%%%%%%%%%%%%%%%%%%%%%%%%%%%%%%%%
% APPENDIX
%%%%%%%%%%%%%%%%%%%%%%%%%%%%%%%%%%%%%%%%%%%%%%%%%%%%%%%%%%%%%%%%%%%%%%%%%%%%%%%
%%%%%%%%%%%%%%%%%%%%%%%%%%%%%%%%%%%%%%%%%%%%%%%%%%%%%%%%%%%%%%%%%%%%%%%%%%%%%%%
\newpage
\appendix
\onecolumn

\section{Concrete optimization methods}
\label{section:other_methods}

In this section, we provide concrete examples of optimization algorithms using the \algname{ATA} and \algname{GTA} allocation strategies.

For optimization problems, we focus on \algname{SGD} and \algname{Asynchronous SGD}.
Other methods, such as stochastic proximal point methods and higher-order methods, can be developed in a similar fashion.

\subsection{Stochastic Gradient Descent}

For \algname{SGD}, it is important to distinguish homogeneous and heterogeneous cases.
Let us start from the homogeneous case.

\subsubsection{Homogeneous regime}

Consider the problem of finding an approximate stationary point of the optimization problem
\begin{equation}
    \label{eq:homo_problem}
    \min_{\bm{x} \in \R^d} \ \left\{f(\bm{x}) \eqdef \ExpSub{\bm{\xi} \sim {\cal D}}{f(\bm{x};\bm{\xi})}\right\}.
\end{equation}
We assume that each worker is able to compute stochastic gradient $f(\bm{x};\bm{\xi})$ satisfying $\mathbb{E}_{\bm{\xi} \sim {\cal D}}\left[ \|f(\bm{x};\bm{\xi}) - \nabla f(\bm{x})\|^2\right] \leq \sigma^2$ for all $\bm{x}\in \R^d$.

In this case, \algname{SGD} with allocation budget $B$ becomes \algname{Minibatch SGD} with batch size $B$. The next step is determining how the batch is collected. For \algname{ATA}, we refer to this method as \algname{SGD-ATA}, as described in \Cref{alg:sgd-ata}.

\begin{algorithm}[H]
	\caption{\algname{SGD-ATA} (Homogeneous)}
    \label{alg:sgd-ata}
	\begin{algorithmic}[1]
		\STATE \textbf{Optimization inputs}: initial point $\bm{x}_0 \in \R^d$, stepsize $\gamma > 0$
        \STATE \textbf{Allocation inputs}: allocation budget  $B$
        \STATE \textbf{Initialize}: empirical means $\hmu_{i,1} = 0$, usage counts $K_{i,1} = 0$, and usage times $T_{i,1} = 0$, for all $i \in [n]$
		\FOR{$k = 1,\ldots, K$}
        \STATE Compute LCBs $(s_{i,k})$ for all $i \in [n]$
        \STATE Find allocation:
        $
            \bm{a}_k \in  \argmin_{\ba \in \mathcal{A}} \ell(\ba, \bm{s}_k)~.
        $
		\STATE Allocate $a_{i,k}$ tasks to each worker $i \in [n]$
        \STATE Update $\bm{x}$:
        $$
            \bm{x}_{k+1} = \bm{x}_k - \frac{\gamma}{B} \sum_{i=1}^n \sum_{j=1}^{a_{i,k}} \nabla f\(\bm{x}_k;\bm{\xi}_i^j\)
        $$
        \FOR{$i$ such that $a_{i,k} \neq 0$}
        \STATE $K_{i,k+1} = K_{i,k} + a_{i,k}$
        \STATE $T_{i,k+1} = T_{i,k} + \sum_{j=1}^{a_{i,k}} X_{i,k}^{(j)}$
        \STATE $\hmu_{i,k+1} = \frac{T_{i,k+1}}{K_{i,k+1}}$
		\ENDFOR
		\ENDFOR
	\end{algorithmic}
\end{algorithm}

In this case, each task consists in calculating the gradient using the device's local data, which is assumed to have the same distribution as the data on all other devices. Because of this, it does not matter which device performs the task. The method then averages these gradients to obtain an unbiased gradient estimator and performs a gradient descent step.


Now, let us give the version of \algname{Minibatch SGD} using greedy allocation \Cref{alg:sgd-gta}.

\begin{algorithm}[H]
	\caption{\algname{SGD-GTA} (Homogeneous)}
    \label{alg:sgd-gta}
	\begin{algorithmic}[1]
		\STATE \textbf{Input}: initial point $\bm{x}_0 \in \R^d$, stepsize $\gamma > 0$, allocation budget $B$
		\FOR{$k = 1,\ldots, K$}
        \STATE $b=0$
        \STATE Query single gradient from each worker $i \in [n]$ 
        \WHILE{$b<B$}
        \STATE Gradient $\nabla f(\bm{x}_k; \bm{\xi}_{k_b})$ arrives from worker $i_{k_b}$
        \STATE $\bm{g}_k = \bm{g}_k + \nabla f(\bm{x}_k; \bm{\xi}_{k_b})$; $\; b= b+1$
        \STATE Query gradient at $\bm{x}_k$ from worker $i_{k_b}$ \\ 
        \ENDWHILE
        \STATE Update the point: $\bm{x}_{k+1} = \bm{x}_k - \gamma \frac{\bm{g}^k}{B}$
		\ENDFOR
	\end{algorithmic}
\end{algorithm}

\Cref{alg:sgd-gta} is exactly \algname{Rennala SGD} method proposed by \citet{tyurin2024optimal}, which has optimal time complexity when the objective function is non-convex and smooth.

If the computation times are deterministic, then \algname{GTA} makes the same allocation in each iteration. In that case, \algname{SGD-ATA} will converge to this fixed allocation. If the times are random, the allocation found by \algname{GTA} may vary in each iteration. In this case, \algname{SGD-ATA} will approach the best allocation for the expected times.

\subsubsection{Heterogeneous regime}
Now let us consider the following heterogeneous problem
$$
    \min_{x \in \mathbb{R}^d} \ \left\{ f(\bm{x}) := \frac{1}{n} \sum_{i=1}^{n} \ExpSub{\bm{\xi}_i \sim \mathcal{D}_i}{f_i(\bm{x}; \bm{\xi}_i)} \right\}~.
$$
Here each worker $i$ has its own data distribution $\cD_i$.

We start with the greedy allocation.
The algorithm is presented in \Cref{alg:sgd-gta-hetero}.

\begin{algorithm}[H]
	\caption{\algname{SGD-GTA} (Heterogeneous)}
    \label{alg:sgd-gta-hetero}
	\begin{algorithmic}[1]
		\STATE \textbf{Input}: initial point $\bm{x}_0 \in \R^d$, stepsize $\gamma > 0$, parameter $S$
		\FOR{$k = 1,\ldots, K$}
        \STATE $s_i=0$ and $\bm{g}_{i,k} = \bm{0}$ for all $i \in [n]$
        \STATE Query single gradient from each worker $i \in [n]$ 
        \WHILE{$\(\frac{1}{n} \sum_{i=1}^n \frac{1}{s_i}\)^{-1}<\frac{S}{n}$}
        \STATE Gradient $\nabla f_{j}(\bm{x}_{k}; \bm{\xi}_{k})$ arrives from worker $j$
        \STATE $\bm{g}_{j,k} = \bm{g}_{j,k} + \nabla f_{j}(\bm{x}_{k}; \bm{\xi}_{k})$; $\; s_j = s_j+1$
        \STATE Query gradient at $\bm{x}_{k}$ from worker $j$ \\ 
        \ENDWHILE
        \STATE Update the point: $\bm{x}_{k+1} = \bm{x}_k - \gamma \frac{1}{n} \sum_{i=1}^n \frac{1}{s_i} \bm{g}_{i,k}$
		\ENDFOR
	\end{algorithmic}
\end{algorithm}

\Cref{alg:sgd-ata-hetero} presents the \algname{Malenia SGD} algorithm, proposed by \citet{tyurin2024optimal}, which is also optimal for non-convex smooth functions.

In each iteration, \Cref{alg:sgd-gta-hetero} receives at least one gradient from each worker.
Building on this idea, we design a method incorporating \algname{ATA}, given in \Cref{alg:sgd-ata-hetero}.

\begin{algorithm}[H]
	\caption{\algname{SGD-ATA} (Heterogeneous)}
    \label{alg:sgd-ata-hetero}
	\begin{algorithmic}[1]
		\STATE \textbf{Optimization inputs}: initial point $\bm{x}_0 \in \R^d$, stepsize $\gamma > 0$
        \STATE \textbf{Allocation inputs}: allocation budget  $B$
        \STATE \textbf{Initialize}: empirical means $\hmu_{i,1} = 0$, usage counts $K_{i,1} = 0$, and usage times $T_{i,1} = 0$, for all $i \in [n]$
		\FOR{$k = 1,\ldots, K$}
        \STATE Compute LCBs $(s_{i,k})$ for all $i \in [n]$
        \STATE Find allocation:
        $
        \bm{a}_k = \algname{RAS} (\bm{s}_k; B)
        $
		\STATE Allocate $a_{i,k} + 1$ tasks to each worker $i \in [n]$
        \STATE Update $\bm{x}$:
        $$
            \bm{x}_{k+1} = \bm{x}_k - \frac{\gamma}{n} \sum_{i=1}^n \frac{1}{a_{i,k}+1}\sum_{j=1}^{a_{i,k}+1} \nabla f_i\(\bm{x}_k;\bm{\xi}_i^j\)
        $$
        \STATE For all $i\in[n]$, update:
        \begin{align*}
            K_{i,k+1} &= K_{i,k} + a_{i,k} \\
            T_{i,k+1} &= T_{i,k} + \sum_{j=1}^{a_{i,k}} X_{i,k}^{(j)} \\
            \hmu_{i,k+1} &= \frac{T_{i,k+1}}{K_{i,k+1}} \\
        \end{align*}
		\ENDFOR
	\end{algorithmic}
\end{algorithm}


\subsection{Asynchronous SGD}

Here, we focus on the homogeneous problem given in \Cref{eq:homo_problem}. The greedy variant, \algname{Ringmaster ASGD}, was proposed by \citet{maranjyan2025ringmasterasgdasynchronoussgd} and, like \algname{Rennala SGD}, achieves the best runtime.

We now present its version with \algname{ATA}, given in \Cref{alg:asgd-ata}.

\begin{figure*}[h]
    \begin{minipage}[t]{0.48\textwidth}

\begin{algorithm}[H]
	\caption{\algname{ASGD-ATA}}
    \label{alg:asgd-ata}
	\begin{algorithmic}[1]
		\STATE \textbf{Optimization inputs}: initial point $\bm{x}_0 \in \R^d$, stepsize $\gamma > 0$
        \STATE \textbf{Allocation inputs}: allocation budget  $B$
        \STATE \textbf{Initialize}: empirical means $\hmu_{i,1} = 0$, usage counts $K_{i,1} = 0$, and usage times $T_{i,1} = 0$, for all $i \in [n]$
		\FOR{$k = 1,\ldots, K$}
        \STATE Compute LCBs $(s_{i,k})$ for all $i \in [n]$
        \STATE Find allocation:
        $
        \bm{a}_k = \algname{RAS} (\bm{s}_k; B)
        $
        \STATE Update $\bm{x}_k$ using \Cref{alg:asgd} with allocation $\bm{a}_k$
        \STATE For all $i$ such that $a_{i,k} \neq 0$, update:
        \begin{align*}
            K_{i,k+1} &= K_{i,k} + a_{i,k} \\
            T_{i,k+1} &= T_{i,k} + \sum_{j=1}^{a_{i,k}} X_{i,k}^{(j)} \\
            \hmu_{i,k+1} &= \frac{T_{i,k+1}}{K_{i,k+1}} \\
        \end{align*}
        \vspace{-1cm}
		\ENDFOR
	\end{algorithmic}
\end{algorithm}

\end{minipage}
\hfill
\begin{minipage}[t]{0.48\textwidth}

    \begin{algorithm}[H]
        \caption{\algname{ASGD}}
        \label{alg:asgd}
        \begin{algorithmic}[1]
            \STATE \textbf{Input:} Initial point $\bm{x}_0 \in \mathbb{R}^d$, stepsize $\gamma > 0$, allocation vector $\bm{a}$ with $\|\bm{a}\|_1 = B$
            \STATE Workers with $a_i > 0$ start computing stochastic gradients at $\bm{x}_0$
            \FOR{$s = 0, 1, \ldots, B-1$}
                \STATE Receive gradient $\nabla f(\bm{x}_{s+\delta_s}; \bm{\xi}_{s+\delta_s}^{i})$ from worker $i$
                \STATE Update: $\bm{x}_{s+1} = \bm{x}_{s} - \gamma \nabla f(\bm{x}_{s+\delta_s}; \bm{\xi}_{s+\delta_s}^{i})$
                \IF{$a_i > 0$}
                    \STATE Worker $i$ begins computing $\nabla f(\bm{x}_{s+1}; \bm{\xi}_{s+1}^{i})$
                    \STATE Decrease remaining allocation for worker $i$ by one: $a_i = a_i - 1$
                \ENDIF
            \ENDFOR
            \STATE \textbf{return:} $\bm{x}_{B}$
        \end{algorithmic}
        \vspace{0.2cm}
        The sequence $\{\delta_s\}$ represents delays, where $\delta_s \geq 0$ is the difference between the iteration when worker $i$ started computing the gradient and iteration $s$, when it was applied.
    \end{algorithm}

\end{minipage}
\end{figure*}


Here, the task remains gradient computation, but each worker's subsequent tasks use different points for computing the gradient. These points depend on the actual computation times and the asynchronous nature of the method, hence the name \algname{Asynchronous SGD}.


\section{Recursive Allocation Selection Algorithm}
\label{sec:RAS}

In this section, we introduce an efficient method for finding the best allocation.
Given LCBs $\bm{s}_k$ and allocation budget $B$, each iteration of \algname{ATA} (\Cref{alg:ata}) determines the allocation by solving
$$
    \bm{a}_k \in \argmin_{\ba \in \mathcal{A}} \ \ell(\bm{a}, \bm{s}_k),
$$
where 
$$
    \ell(\bm{a},\bm{\mu}) \eqdef \max_{i \in [n]} \  a_{i} \mu_i  = \infnorm{\bm{a}\odot \bm{\mu}},
$$
with $\odot$ denoting the element-wise product.
When clear from context, we write $\ell(\bm{a})$ instead of $\ell(\bm{a}, \bm{\mu})$.

In the early iterations, when some $s_i$ values are $0$, \algname{ATA} allocates uniformly across these arms until all $s_i$ values become positive.
After that, the allocation is determined using the recursive routine in \Cref{alg:RAS}.

\begin{algorithm}[H]
    \caption{Recursive Allocation Selection (\algname{RAS})}
    \label{alg:RAS}
    \begin{algorithmic}[1]
        \STATE \textbf{Input:} Scores $s_1, \dots, s_n$, allocation budget $B$
        \STATE Assume without loss of generality that $s_1 \leq s_2 \leq \dots \leq s_n$ (i.e., sort the scores)
        \IF{$B = 1$}
            \STATE \textbf{return:} $(1, 0, \dots, 0)$
        \ENDIF
        \STATE Find the previous best allocation:
        $$
            \bm{a} = (a_1, \dots, a_n) = \algname{RAS}\(s_1, \dots, s_n; B-1\)
        $$
        \STATE Determine the first zero allocation:
        \begin{equation}
            \label{eq:r}
            r = 
            \begin{cases}
                \min\{i \mid a_i = 0\}, & \text{if }\ a_n = 0 \\
                n, & \text{otherwise}
            \end{cases}
        \end{equation}
        \STATE Find the best next query allocation set: \label{alg_line:min}
        $$
        M = \argmin_{i \in [r]} \  \infnorm{(\bm{a} + \bm{e}_i) \odot \bm{s}},
        $$
        where $\bm{e}_i$ is the unit vector in direction $i$.
        \STATE Select $j \in M$ such that the cardinality of 
        $$
        \argmax_{i \in [r]} \ (a_i + e_{j,i}) s_i
        $$ 
        is minimized
        \STATE \textbf{return:} $\bm{a} + \bm{e}_j$
    \end{algorithmic}
\end{algorithm}

\begin{remark}
    The iteration complexity of \algname{RAS} is $\mathcal{O}(n \ln(\min\{B, n\}) + \min\{B, n\}^2)$. In fact, the first $n \ln(\min\{B, n\})$ term arises from identifying the smallest $B$ scores. For the second term, note that in \eqref{eq:r}, we have $r \leq \min\{B, n\}$.
\end{remark}

\subsection{Optimality}
We now prove that \algname{RAS} finds the optimal allocation, as stated in the following lemma.
\begin{lemma}
    \label{thm:RAS_optimality}
    For positive scores $0<s_1 \le s_2 \le \ldots \le s_n$, \algname{RAS} (\Cref{alg:RAS}) finds an optimal allocation $\bh \in \cA$, satisfying
    $$
    \bh \in \argmin_{\ba \in \cA} \  \infnorm{\ba \odot \bs} ~.
    $$
\end{lemma}
%
\begin{proof}
    We prove the claim by induction on the allocation budget $B$.
    
    \textbf{Base Case ($B = 1$):}  
    When $B = 1$, \algname{RAS} (\Cref{alg:RAS}) allocates the task to worker with the smallest score (line 9).
    Thus, the base case holds.

    \textbf{Inductive Step:}  
    Assume that \algname{RAS} finds an optimal allocation for budget $B - 1$, denoted by
    $$
        \bar{\bh} = \algname{RAS}(s_1, \ldots, s_n; B-1)~.
    $$
    We need to prove that the solution returned for budget $B$, denoted by $\bh = \bar{\bh} + \be_r$, is also optimal.

    Assume, for contradiction, that there exists $\ba \in \cA$ such that $\ba \neq \bh$ and $\ell(\ba) < \ell(\bh)$. 
    Write $\ba = \bar{\ba} + \be_q$ for some $q \in [n]$. Observe that $\|\bar{\ba}\|_1=B-1$ because $\ba \in \mathcal{A}$.

    We consider two cases based on the value of $\ell\(\bar{\bh} + \be_r\)$:

    \begin{itemize}
        \item $\ell\(\bar{\bh} + \be_r\) = h_k s_k$ for some $k \neq r$.  
        In this case, adding one unit to index $r$ does not change the maximum value, i.e., $\ell\(\bar{\bh}\) = \ell\(\bar{\bh} + \be_r\)$. 
        By the inductive hypothesis, $\bar{\bh}$ minimizes $\ell(\bx)$ for budget $B - 1$. 
        Therefore, we have
        $$
        \ell(\ba) \geq \ell\(\bar{\ba}\) \geq \ell\(\bar{\bh}\) = \ell\(\bar{\bh} + \be_r\) =\ell(\bh),
        $$
        which contradicts the assumption that $\ell(\ba) < \ell(\bh)$.

        \item $\ell\(\bar{\bh} + \be_r\) = \(\bar{h}_r + 1\)s_r$.  
        By the algorithm's logic, $\(\bar{h}_r + 1\)s_r \leq \(\bar{h}_i + 1\)s_i$ for all $i \neq r$.
        Since $\ell(\bar{\bh}+\be_r)\leq \ell(\bar{\bh}+\be_q)$ and we assumed $\ell(\bar{\ba}+\be_q)=\ell(\ba)<\ell(\bh)=\ell(\bar{\bh}+\be_r)$, then $\bar{\ba} \neq \bar{\bh}$ otherwise $\ell(\bar{\ba}+\be_q)<\ell(\bar{\ba}+\be_r)$.
        Given that $\|\bar{\bh}\|_1=\|\bar{\ba}\|_1$, this implies that there exists some $u \in [n]$ such that $0\le\bar{a}_u \leq \bar{h}_u - 1$ and another index $v \in [n]$ where $\bar{a}_v \geq \bar{h}_v + 1$.

        In addition, note that $r$ is chosen such that $\ell\(\bar{\bh} + \be_r\)$ is minimum. Using the fact that $\ell\(\bar{\bh} + \be_r\) = \(\bar{h}_r + 1\)s_r$, we have that for any index $q$, we also necessarily have $\ell\(\bar{\bh} + \be_q\) = \(\bar{h}_q + 1\)s_q$.
        Using this, we deduce
        $$
        \ell(\bh)
        =\ell\(\bar{\bh} + \be_r\)
        \leq \ell\(\bar{\bh} + \be_v\)
        = \(\bar{h}_v + 1\)s_v
        \leq \max_i \  \bar{a}_i s_i
        = \ell\(\bar{\ba}\) \leq \ell(\ba),
        $$
        where in the second inequality we used the fact that $\bar{a}_v \geq \bar{h}_v + 1$ and in the last inequality we used the fact that the loss is not decreasing for we add one element to the vector.
        This chain of inequalities again contradicts the assumption that $\ell(\ba) < \ell(\bh)$.
    \end{itemize}

    Since both cases lead to contradictions, we conclude that no $\ba \in \cA$ exists with $\ell(\ba) < \ell(\bh)$. 
    Thus, \algname{RAS} produces an optimal allocation for budget $B$.
\end{proof}

\subsection{Minimal Cardinality}

Among all possible allocations \algname{RAS} choose one that always minimizes the cardinality of the set:
$$
    \argmax_{i \in [n]} \  a_i s_i~.
$$

The reason for this choice is just technical as it allows the \Cref{lem:1} to be true. 

\begin{lemma}
    \label{thm:minimal_cardinality}
    The output of $\algname{RAS}$ ensures the smallest cardinality of the set:
    $$
        \argmax_{i \in [n]} \ a_i s_i
    $$
    among all the optimal allocations $\ba$.
\end{lemma}
%
\begin{proof}
    This proof uses similar reasoning as the one before.

    Let $\bh = \algname{RAS}(\bs;B)$, and denote the cardinality of the set $\argmax_{i \in [n]} \ a_i s_i $ for allocation $\ba$ by
    $$
    C_B(\ba) = \left| \argmax_{i \in [n]} \ a_i s_i  \right| \geq 1~.
    $$  
    We prove the claim by induction on $B$.

    \textbf{Base Case ($B=1$):}  
    For $B=1$, there is a single coordinate allocation, thus $C_1(\bh) = 1$, which is the smallest possible cardinality.

    \textbf{Inductive Step:}  
    Assume that $\algname{RAS}$ finds an optimal allocation for budget $B-1$ with the smallest cardinality, denote its output by
    $$
    \bar{\bh} = \algname{RAS}(s_1, \ldots, s_n; B-1)~.
    $$  
    We need to prove that $\bh = \bar{\bh} + \be_r$ minimizes $C_B(\ba)$ among all optimal allocations for budget $B$.

    Assume, for contradiction, that there exists $\ba \in \cA$ such that $\ba \neq \bh$, $\ell(\ba) = \ell(\bh)$, and $C_B(\ba) < C_B(\bh)$. 
    Write $\ba = \bar{\ba} + \be_q$ for some $q \in [n]$.
    We consider three cases:

    \begin{itemize}
        \item 
        $C_B(\bh) = 1$.  
        %This occurs when $\ell(\bh) > \ell\(\bar{\bh}\)$.
        Since the minimum cardinality is exactly 1, we must have $C_B(\ba) \ge 1 = C_B(\bh)$, that contradicts our assumption.

        \item 
        $C_B(\bh) = C_{B-1}\(\bar{\bh}\)>1$.
        This occurs when $\ell(\bh) = \ell\(\bar{\bh}\) \ne \(\bar{h}_r + 1\)s_r$. 
        By the optimality of $\bh$, we have $\ell\(\bar{\bh}\) \le \ell\(\bar{\ba}\) \le \ell(\ba) = \ell(\bh)=\ell(\bar{\bh})$, which implies $\ell\(\bar{\ba}\) = \ell(\ba)$.
        Therefore, $C_{B-1}\(\bar{\ba}\) \le C_B(\ba)$. 
        Since the induction hypothesis holds for $B-1$, we have $C_{B-1}\(\bar{\bh}\) \le C_{B-1}\(\bar{\ba}\)$. 
        Thus,
        $$
        C_B(\bh) = C_{B-1}\(\bar{\bh}\) \le C_{B-1}\(\bar{\ba}\) \le C_B(\ba),
        $$
        which leads to a contradiction.

        \item 
        $C_B(\bh) = C_{B-1}\(\bar{\bh}\) + 1$.  
        This occurs when $\ell(\bh) = \ell\(\bar{\bh}\) = \(\bar{h}_r + 1\)s_r$. 
        Proceeding as in the previous case, we have $\ell\(\bar{\ba}\) = \ell(\ba)$, and hence $C_{B-1}\(\bar{\ba}\) \le C_B(\ba)$.
        Since the induction hypothesis holds for $B-1$, we know $C_{B-1}\(\bar{\bh}\) \le C_{B-1}\(\bar{\ba}\)$.

        We now have additional cases:
        \begin{itemize}
        \item If $C_{B-1}\(\bar{\ba}\) = C_{B-1}\(\bar{\bh}\) + 1$, then
        $$
        C_B(\bh) = C_{B-1}\(\bar{\bh}\) + 1 = C_{B-1}\(\bar{\ba}\) \le C_B(\ba),
        $$
        which leads to a contradiction.

        \item Now assume $C_{B-1}\(\bar{\ba}\) = C_{B-1}\(\bar{\bh}\)$.
        We will show that in this case, $C_B(\ba) = C_{B-1}\(\bar{\ba}\) + 1$. 
        By contradiction, suppose $C_B(\ba) = C_{B-1}\(\bar{\ba}\)$, which implies $(\bar{a}_q + 1)s_q < \ell(\ba)$. Let $k$ be an index such that $\bar{a}_k s_k = \ell(\ba)$.
        Construct a new allocation $\ba' = \bar{\ba} + \be_q - \be_k$. 
        Then,
        $$
        C_{B-1}\(\ba'\) = C_{B-1}\(\bar{\ba}\) - 1 < C_{B-1}\(\bar{\bh}\),
        $$
        which contradicts the induction hypothesis. 
        Thus, $C_B(\ba) = C_{B-1}\(\bar{\ba}\) + 1$.
        Using this, we have
        $$
        C_B(\bh) = C_{B-1}\(\bar{\bh}\) + 1 = C_{B-1}\(\bar{\ba}\) + 1 = C_B(\ba),
        $$
        which again contradicts $C_B(\ba) < C_B(\bh)$.
        \end{itemize}
    \end{itemize}

    This concludes the proof.
\end{proof}

\section{Proofs of \Cref{thm:main}, \Cref{thm:main2}, and \Cref{cor:main}}
\label{sec:proof_1}

We start by recalling the notation. For $i\in [n]$ and $k \in [K]$, $(X^{(u)}_{i,k})_{u \in [B]}$ denote $B$ independent samples at round $k$ from distribution $\nu_i$. When using an allocation vector $\bm{a}_k \in \mathcal{A}$, the total computation time of worker $i$ at round $k$ is $\sum_{u=1}^{a_{i,k}} X^{(u)}_{i,k}$, when $a_{i,k} >0$. $\bm{\mu} = (\mu_1, \dots, \mu_K)$ is the vector of means. For each $k \in [K]$, when using the allocation vector $\bm{a}_k$, we recall the definition of the proxy loss $\ell: \mathcal{A}\times \mathbb{R}_{\ge0}^n \to \mathbb{R}_{\ge0}$ by
$$
\ell(\bm{a}_k, \bm{\lambda}) = \max_{i\in [n]} \ a_{i,k} \lambda_i,
$$
where $\bm{\lambda} = (\lambda_1, \dots, \lambda_n)$ is a vector of non-negative components. When $\bm{\lambda} = \bm{\mu}$, we drop the dependence on the second input of $\ell$.
For each $\bm{\lambda}$, let $\bar{\bm{a}}_{\bm{\lambda}}\in \mathcal{A}$, be the action minimizing this loss
$$
\bar{\bm{a}}_{\bm{\lambda}} \in \argmin_{\bm{a} \in \mathcal{A}} \ \ell(\bm{a}, \bm{\lambda})~.
$$
We drop the dependency on $\bm{\mu}$ from $\bar{\bm{a}}_{\bm{\mu}}$ to ease notation. The actual (random) computation time at round $k$ is denoted by $C: \mathcal{A} \to \mathbb{R}_+$:
\begin{equation}\label{eq:def_C}
	C(\bm{a}_k) := \max_{i\in [n]} \ \sum_{u=1}^{a_{i,k}} X_{i,k}^{(u)}~.
\end{equation}
Let $\bm{a}^*$ be the action minimizing the expected time
$$
\bm{a}^* \in \argmin_{\bm{a} \in \mathcal{A}} \ \E{C(\bm{a})}~.
$$
The expected regret after $K$ rounds is defined as follows
$$
\mathcal{R}_K := \sum_{t=1}^{K} \E{\ell(\bm{a}_{k})-\ell(\bar{\bm{a}})}~.
$$

\noindent For the remainder of this analysis we consider $\bar{\bm{a}} \in \argmin_{a \in \mathcal{A}} \ \ell(\bm{a})$ found using the \algname{RAS} procedure.
For each $i\in [n]$, recall that $k_i$ is the smallest integer such that
\begin{equation}\label{eq:def_n}
	(\bar{a}_i+k_i)\mu_i > \ell(\bar{\bm{a}})~.
\end{equation}

Below we present a technical lemma used in the proofs of Theorems~\ref{thm:main} and~\ref{thm:main2}.
\begin{lemma}
	\label{lem:1}
	Let $\bm{x}=(x_1, \dots, x_n) \in \mathbb{R}_{\ge 0}^n$. Let $\bm{a}$ be the output of $\algname{RAS}(\bm{x}; B)$. For each $i, j \in [n]$, we have
	$$ 
	a_{ j} x_j \le \left(a_{ i}+1 \right) x_i~.
	$$
\end{lemma}
%
\begin{proof}
	Fix $\bm{x} \in \mathbb{R}_+^n$, and let $\bm{a} = \algname{RAS}(\bm{x};B)$. The result is straightforward when $\min\limits_{i\in [n]}{x_i} = 0$.
	
	\noindent Suppose that $x_i >0$ for all $i \in [n]$.
	Let $s\ge 1$ denote the cardinality
	$$
	s:= \abs{\argmax_{i \in [n]} \  a_{i} x_i }~.
	$$
	Fix $i,j \in [n]$, let $k \in \argmax_{i \in [n]} \  a_{i} x_i $. We need to show that
	$$
	a_{k} x_k \le (a_{i}+1)x_i~.
	$$
	We use a proof by contradiction.
	Suppose that we have $ a_{k} x_k > (a_{i}+1)x_i$ consider the allocation vector $\bm{a}'\in \mathcal{A}$ given by $a'_k = a_{k}-1$, $a'_i = a_i+1$ and $a'_u = \bar{a}_u$ when $u \notin \{i,k\}$. Let $R := \max_{u \neq i,k} \{a_u x_u \}$. We have
	\begin{align*}
		\ell(\bm{a}',\bm{x})
		= \max_{u \in [n]} \ a'_u x_u
		= \max \{(a_i+1) x_i, (a_k-1) x_k, R \}~.
	\end{align*}
	We consider two cases:
	\begin{itemize}
		\item Suppose that $s=1$ (i.e., the only element in $[n]$ such that $a_ux_u=\ell(\bm{a}, \bm{x})$ is $k$), then we have necessarily $R< a_k x_k $. Moreover, by the contradiction hypothesis, $(a_i+1)x_i < a_k x_k$.
		Therefore,
		\begin{align*}
			\ell(\bm{a}', \bm{x}) = \max\{ (a_i+1)x_i, (a_k-1)x_k, R\}
			< 	a_k x_k = \ell(\bm{a}, \bm{x}),
		\end{align*}
		which contradicts the definition of $\bm{a}$.
		\item Suppose that $s\ge 2$, since by hypothesis $ a_k x_k > (a_i+1)x_i$, we clearly have $a_ix_i < \ell(\bm{a}, \bm{x})$ therefore among the set $[n]\setminus \{k,i\}$ there are exactly $s-1$ elements such that $a_u x_u = \ell(\bm{a}, \bm{x})$. In particular, this gives
		\begin{align*}
			\ell(\bm{a}',\bm{x})
			= \max_{u \in [n]} \ \{(a_i+1) x_i, (a_k-1) x_k, R \}
			= R = \ell(\bm{a}, \bm{x})~.
		\end{align*}
		Therefore, $\bm{a}' \in \argmin_{\bm{a} \in \mathcal{A}} \ \ell(\bm{a}, \bm{x})$ and the number of elements such that $a'_i x_i = \ell(\bm{a}', \bm{x})=\ell(\bm{a}, \bm{x})$ is at most $s-1$, which contradicts the fact that $s$ is minimal given the \algname{RAS} choice and Lemma~\ref{thm:minimal_cardinality}.
	\end{itemize}
	As a conclusion we have $a_k x_k\le (a_i+1)x_i$.
\end{proof}
\begin{remark}
	Recall that Lemma~\ref{lem:1} guarantees that $k_i$ defined in \eqref{eq:def_n} satisfy: $k_i \in \{1, 2\}$ for each $i \in [n]$.
\end{remark}


\subsection{Proof of \Cref{thm:main}}
\label{proof:thm:main}

Below we restate the theorem.

\begin{restate-theorem}{\ref{thm:main}}
	Suppose that Assumption~\ref{a:sube} holds. Let $\bar{\bm{a}} \in \argmin_{\bm{a} \in \mathcal{A}} \ell(\bm{a})$, in case of multiple optimal actions, we consider the one output by \algname{RAS} when fed with $\bm{\mu}$.
	Then, the expected regret of \algname{ATA} with inputs $(B, \alpha)$  satisfies
	$$
	\mathcal{R}_K
	\le 2n\max_{i \in [n]} \{B\mu_i -\ell(\bar{\bm{a}})\}+c \cdot\sum_{i=1}^{n} \frac{\alpha^2(\bar{a}_i+k_i)(B \mu_i - \ell(\bar{\bm{a}})) }{\left((\bar{a}_i+k_i)\mu_i - \ell(\bar{\bm{a}})\right)^2}\cdot \ln K,
	$$
	where $\alpha = \max_{i \in [n]} \norm{X_i}_{\psi_1}$, and $c$ is a constant.
\end{restate-theorem}
%
\begin{proof}
	Let $K_{i,k}$ be the number of rounds where arm $i$ was queried prior to round $k$ (we take $K_{i,1}=0$). Recall that we chose the following confidence bound: if $K_{i,k} \ge 1$, then
	$$
	\text{conf}(i,k) = 4e\alpha\sqrt{\frac{ \ln(2k^2)}{K_{i,k}}}+4e\alpha\frac{ \ln(2k^2)}{K_{i,k}},
	$$
	and $\text{conf}(i,k) = \infty$ otherwise. Recall that $\hat{\mu}_{i,k}$ denotes the empirical mean of samples from $\nu_i$ observed prior to $k$ if $K_{i,k}\ge 0$ and $\hat{\mu}_{i,k}=0$ if $K_{i,k}=0$. Let $s_{i,k}$ denote the lower confidence bound used in the algorithm:
	$$
	s_{i,k} = \left(\hat{\mu}_{i,k} -\text{conf}(i,k)\right)_{+}~.
	$$
	
	\noindent We introduce the events $\mathcal{E}_{i,k}$ for $i \in [n]$ and $k \in [K]$ defined by
	$$
	\mathcal{E}_{i,k} := \left\lbrace \abs{\hat{\mu}_{i,k}-\mu_i} > \text{conf}(i,k)\right\rbrace.
	$$
	Let 
	$$
	\mathcal{E}_k = \cup_{i \in [n]} \mathcal{E}_{i,k}.
	$$
	Let us prove that for each $k \in [K]$ and $i \in [n]$: $\mathbb{P}\left(\mathcal{E}_{i,k}\right) \le \frac{1}{k^2}$, which gives using a union bound $\mathbb{P}(\mathcal{E}_k) \le \frac{n}{k^2}$. 
	Let $i \in [n]$, using \Cref{prop:concentration} and taking $\delta = 1/k^2$, we have
	\begin{align*}
		\mathbb{P}(\mathcal{E}_{i,k})
		= \mathbb{P}\{\abs{\hat{\mu}_{i,k}-\mu} > \text{conf}(i,k)\}
		\le \frac{1}{k^2}~.
	\end{align*}

	\noindent We call a ``bad round", a round $k$ where we have $\ell(\bm{a}_{k}) > \ell(\bar{\bm{a}})$. Let us upper bound the number of bad rounds. 
	
	\noindent Observe that in a bad round there is necessarily an arm $i \in [K]$ such that $a_{i,k} \mu_i > \ell(\bar{\bm{a}})$. For each $i\in [n]$, let $N_i(k)$ denote the number of rounds $q\in \{1,\dots, k\}$ where $a_{i,q} \mu_i > \ell(\bar{\bm{a}})$ and $i \in \argmax_{j \in [n]} \ a_{j,q} \mu_j$ (this corresponds to a bad round triggered by worker $q$)
	$$
	N_i(k) := \abs{\left\lbrace q \in \{1, \dots, k\}: a_{i,q}\mu_i > \ell(\bar{\bm{a}}) \text{ and } a_{i,q}\mu_i = \ell(\bm{a}_q) \right\rbrace}~.
	$$
	%This implies in particular that $a_{t,i} \ge a_i^*+n_i$ (using the definition of $n_i$). We conclude that there exists an arm $j\neq i$ such that $a_{t,j} \le a_j^*-n_i$. 
	We show that in the case of $\ell(\bm{a}_k) > \ell(\bar{\bm{a}})$, the following event will hold: there exists $i \in [n]$ such that 
	$$
	E_{i,k} := \mathcal{E}_k \text{ or }\left\lbrace N_i(k-1) \le  \frac{256e^2\alpha^2 (\bar{a}_i+k_i) \ln(2K^2)}{\left((\bar{a}_i+k_i)\mu_i - \ell(\bar{\bm{a}})\right)^2}  \right\rbrace~.
	$$
	To prove this, suppose that for each $i \in [n]$, $\neg E_{i,k}$ holds. This gives in particular
	\begin{equation}\label{eq:ni}
		N_i(k-1) > \frac{256e^2\alpha^2 (\bar{a}_i+k_i) \ln(2K^2)}{\left((\bar{a}_i+k_i)\mu_i - \ell(\bar{\bm{a}})\right)^2}~.
	\end{equation}
	Observe that in each round where $N_i(\cdot)$ is incremented, the number of samples received from the distribution $\nu_i$ increases by at least $\bar{a}_{i}+k_i$. 
	Therefore, we have \eqref{eq:ni} implies
	\begin{align*}
		K_{i,k}
		> \frac{256e^2\alpha^2(\bar{a}_i+k_i)^2 \ln(2K^2)}{\left((\bar{a}_i+k_i)\mu_i - \ell(\bar{\bm{a}})\right)^2}
		= \frac{256e^2\alpha^2 \ln(2K^2)}{\left(\mu_i - \frac{\ell(\bar{\bm{a}})}{\bar{a}_i+k_i}\right)^2}~.
	\end{align*}
	
	
	\noindent Then we have, using the expression of $\text{conf}(\cdot)$
	\begin{align*}
		2\text{conf}(i,k) &=  8e\alpha\sqrt{\frac{ \ln(2k^2)}{K_{i,k}}}+8e\alpha\frac{ \ln(2k^2)}{K_{i,k}}\\
		&\le \left(\mu_i - \frac{\ell(\bar{\ba})}{\bar{a}_i+k_i}\right) \left[ 8e\alpha \sqrt{\frac{\ln(2k^2)}{256e^2\alpha^2 \ln(2K^2)}}+ 8e\alpha \frac{\ln(2k^2)}{256e^2\ln(2K^2)\alpha^2}\left(\mu_i-\frac{\ell(\bar{\ba})}{\bar{a}_i+k_i}\right)\right]\\
		&\le \left(\mu_i - \frac{\ell(\bar{\ba})}{\bar{a}_i+k_i}\right) \left[\frac{1}{2} + \frac{1}{32e\alpha} \left(\mu_i-\frac{\ell(\bar{\ba})}{\bar{a}_i+k_i}\right) \right]~.
	\end{align*}
	Recall that using Lemma~\ref{lem:tech1}, we have $\mu_i-\frac{\ell(\bar{\ba})}{\bar{a}_i+k_i} \le \mu_i \le \alpha$. Therefore, we have
	\begin{equation}\label{eq:conf}
		2\text{conf}(i,k) < \mu_i - \frac{\ell(\bar{\bm{a}})}{\bar{a}_i+k_i}~.
	\end{equation}
	Suppose for a contradiction argument that we have $\neg E_{i,k}$ and $\{ a_{i,k}\mu_i > \ell(\bar{\bm{a}}) \text{ and } a_{i,k} = \ell(\bm{a}_k)\}$.
	Using the definition of $k_i$ and the fact that $a_{i,k} \mu_i > \ell(\bar{\bm{a}})$, we have that $a_{i,k} \ge \bar{a}_i + k_i$. Therefore, \eqref{eq:conf} gives
	\begin{equation}\label{eq:conf2}
		2\text{conf}(i,k) < \mu_i - \frac{\ell(\bar{\bm{a}})}{a_{i,k}}~.
	\end{equation}
	Observe that in each round $\norm{\bm{a}_k}_0 = B$, therefore if we have $a_{i,k} \ge \bar{a}_i+k_i > \bar{a}_i$ for some $i$, we necessarily have that there exists $j \in [n]\setminus \{i\}$ such that $a_{j,k} \le \bar{a}_j-1$. Using the fact that $\ell(\bar{\bm{a}}) \ge \bar{a}_j \mu_j$ with \eqref{eq:conf2}, we get
	\begin{equation}\label{eq:e1}
		a_{i,k}(\mu_i-2\text{conf}(i,k)) > \bar{a}_j \mu_j~.
	\end{equation}
	Since both $\neg \mathcal{E}_{i,k}$ and $\neg \mathcal{E}_{j,k}$ hold (because $\neg E_{i,k}$ implies $\neg \mathcal{E}_k$), we have that
	\begin{align}
		\mu_i - 2\text{conf}(i,k) &\le \hat{\mu}_{i, k} - \text{conf}(i,k)
		\le s_{i,k},\label{eq:mu2}
	\end{align} 
	and $\mu_j \ge \hat{\mu}_{j,k} - \text{conf}(j,k)$.
	Recall that $\mu_j \ge 0$, therefore
	\begin{align}
		\mu_j %&\ge \hat{\mu}_{j,k} - \text{conf}(j,k)\\
		\ge \left(\hat{\mu}_{j,k} - \text{conf}(j,k)\right)_{+}
		= s_{j,k}~.\label{eq:mu3}
	\end{align}
	Using the bounds \eqref{eq:mu2} and \eqref{eq:mu3} in \eqref{eq:e1}, we have
	$$
	a_{i,k} s_{i,k} > \bar{a}_j s_{j,k} \ge (a_{j,k}+1) s_{j,k},
	$$
	where we used the definition of $j$ in the second inequality.
	This contradicts the statement of Lemma~\ref{lem:1}, which concludes the contradiction argument. Therefore, the event that $k$ is a bad round implies that $E_{i,k}$ holds for at least one $i\in [n]$.
	We say that a bad round was triggered by arm $i$, a round where $N_i(\cdot)$ was incremented. 
	Observe that if $k \in [K]$ is not a bad round then $\E{\ell(\bm{a}_k)}-\ell(\bar{\bm{a}})=0$, otherwise if $k$ is a bad round triggered by $i \in [n]$ then $\E{\ell(\bm{a}_{k})}-\ell(\bar{\bm{a}}) \le B\mu_i-\ell(\bar{\bm{a}})$.
	To ease notation we introduce for $i\in [n]$
	$$
	H_i := \frac{256e^2\alpha^2 (\bar{a}_i+k_i) \ln(2K^2)}{\left((\bar{a}_i+k_i)\mu_i - \ell(\bar{\bm{a}})\right)^2}~.
	$$
	The expected regret satisfies
	\begin{align*}
		\mathcal{R}_K &= \sum_{i=1}^{K} \mathbb{E}\left[\ell(\bm{a}_k)-\ell(\bar{\bm{a}})\right]\\
		&\le \sum_{i=1}^{n} (B\mu_i-\ell(\bar{\bm{a}}))\mathbb{E}[N_i(K)]\\ 
		&= \sum_{i=1}^{n}\sum_{k=1}^{K} (B\mu_i-\ell(\bar{\bm{a}}))\mathbb{E}\left[\mathds{1}(k \text{ is a bad round triggered by }i)\right]\\
		&\le \max_{i\in [n]}\{(B\mu_i-\ell(\bar{\bm{a}}))\}\cdot\sum_{t=1}^{K} \mathbb{P}(\mathcal{E}_k)+ \sum_{i=1}^{n}(B\mu_i-\ell(\bar{\bm{a}}))\sum_{k=1}^{K} \mathbb{E}\left[\mathds{1}(k \text{ is a bad round triggered by }i) \mid \neg \mathcal{E}_k\right]\\
		&\le \max_{i\in [n]}\{(B\mu_i-\ell(\bar{\bm{a}}))\}\cdot\sum_{t=1}^{K} \mathbb{P}(\mathcal{E}_k)+ \sum_{i=1}^{n}(B\mu_i-\ell(\bar{\bm{a}}))\sum_{k=1}^{K} \mathbb{E}\left[\mathds{1}(N_i(k)=1+N_i(k-1) \text{ and } N_i \le H_i ) \mid \neg \mathcal{E}_k\right]\\
		&\le \max_{i\in [n]}\{(B\mu_i-\ell(\bar{\bm{a}}))\}\cdot\sum_{k=1}^{K} \mathbb{P}(\mathcal{E}_k)+ \sum_{i=1}^{n} (B\mu_i-\ell(\bar{\bm{a}}))H_i\\
		&\le 2n \max_{i\in [n]}\{(B\mu_i-\ell(\bar{\bm{a}}))\}+  \sum_{i=1}^{n} \frac{256e^2\alpha^2 (\bar{a}_i+k_i)(B\mu_i-\ell(\bar{\bm{a}})) \ln(2K^2)}{\left((\bar{a}_i+k_i)\mu_i - \ell(\bar{\bm{a}})\right)^2}~. \qedhere
	\end{align*}
\end{proof}


\subsection{Proof of \Cref{thm:main2}}
\label{proof:thm:main2}

\begin{restate-theorem}{\ref{thm:main2}}
	Suppose that Assumption~\ref{a:sube} holds. Let $\bar{\bm{a}} \in \argmin_{\bm{a} \in \mathcal{A}} \ell(\bm{a})$, in case of multiple optimal actions, we consider the one output by \algname{RAS} when fed with $\bm{\mu}$.
Then, the expected regret of \algname{ATA-Empirical} with the empirical confidence bounds using the inputs $(B, \eta)$  satisfies
\begin{align*}
	\mathcal{R}_K &\le 2n\max_{i \in [n]} \{B\mu_i -\ell(\bar{\bm{a}})\}
	 +c \cdot\sum_{i=1}^{n} \frac{(1+\eta^2) \alpha_i^2(\bar{a}_i+k_i)(B \mu_i - \ell(\bar{\bm{a}})) }{\left((\bar{a}_i+k_i)\mu_i - \ell(\bar{\bm{a}})\right)^2}\cdot \ln K,
\end{align*}
where $\alpha_i =  \norm{X_i}_{\psi_1}$, and $c$ is a constant.
\end{restate-theorem}
%
\begin{proof}
	We build on the techniques used in the proof of \Cref{thm:main}. Recall the expression of $\xi$:
	$$
	\xi = \frac{1+\sqrt{4\eta^2+5}}{2}~.
	$$
	Define the quantities $C_{i,k}$ by
	$$
	C_{i,k} = 4e \sqrt{\frac{\ln(2k^2)}{K_{i,k}}}+4e \frac{\ln(2k^2)}{K_{i,k}}~.
	$$
	Recall that the lower confidence bounds used here are defined as
	$$
	\hat{s}_{i,k} = \hat{\mu}_{i,k} \left(1-\xi C_{i,k}\right)_{+}~.
	$$
	We additionally define the following quantities
	\begin{equation*}
		\hat{u}_{i,k} := \hat{\mu}_{i,k} \left(1+\frac{4}{3}\xi C_{i,k}\right)~.
	\end{equation*}
	\noindent We introduce the events $\mathcal{E}_{i,k}$ for $i \in [n]$ and $k \in [K]$ defined by
	$$
	\mathcal{E}_{i,k} := \left\lbrace \abs{\mu_i - \hat{\mu}_{i,k}} \le \alpha_i C_{i,k}\right\rbrace~.
	$$
	Let 
	$$
	\mathcal{E}_k = \cup_{i \in [n]} \mathcal{E}_{i,k}~.
	$$
	We have using Proposition~\ref{prop:concentration} for each $k \in [K]$ and $i \in [n]$: $\mathbb{P}\left(\mathcal{E}_{i,k}\right) \le \frac{1}{k^2}$, which gives using a union bound $\mathbb{P}(\mathcal{E}_k) \le \frac{n}{k^2}$. 
	Moreover, following Lemma~\ref{lem:conc2}, for each $i \in [n]$ and $k \in [K]$, we have that $\mathcal{E}_{i,k}$ implies
	\begin{equation}\label{eq:lcb}
		\mu_i \ge \hat{s}_{i,k}~.
	\end{equation}
	
	\noindent Following similar steps as in the proof of Theorem~\ref{thm:main}, we call a ``bad round", a round $k$ where we have $\ell(\bm{a}_{k}) > \ell(\bar{\bm{a}})$. Let us upper bound the number of bad rounds. 
	
	\noindent Observe that in a bad round there is necessarily an arm $i \in [K]$ such that $a_{i,k} \mu_i > \ell(\bar{\bm{a}})$. For each $i\in [n]$, let $N_i(k)$ denote the number of rounds $q\in \{1,\dots, k\}$ where $a_{i,q} \mu_i > \ell(\bar{\bm{a}})$ and $i \in \argmax_{j \in [n]} \{ a_{j,q} \mu_j\}$ (this corresponds to a bad round triggered by worker $q$):
	$$
	N_i(k) := \abs{\left\lbrace q \in \{1, \dots, k\}: a_{i,q}\mu_i > \ell(\bar{\bm{a}}) \text{ and } a_{i,q}\mu_i = \ell(\bm{a}_q) \right\rbrace}~.
	$$
	%This implies in particular that $a_{t,i} \ge a_i^*+n_i$ (using the definition of $n_i$). We conclude that there exists an arm $j\neq i$ such that $a_{t,j} \le a_j^*-n_i$. 
	We show that in the case of $\ell(\bm{a}_k) > \ell(\bar{\bm{a}})$, the following event will hold: there exists $i \in [n]$ such that 
	$$
	E_{i,k} := \mathcal{E}_k \text{ or }\left\lbrace N_i(k-1) \le  \frac{1024e^2 \xi^2\alpha_i^2 (\bar{a}_i+k_i) \ln(2K^2)}{\left((\bar{a}_i+k_i)\mu_i - \ell(\bar{\bm{a}})\right)^2}  \right\rbrace~.
	$$
	To prove this, suppose for a contradiction argument that we have for each $i \in [n]$ $\neg E_{i,k}$. This gives in particular
	\begin{equation}\label{eq:ni2}
		N_i(k-1) >  \frac{1024e^2 \xi^2\alpha_i^2 (\bar{a}_i+k_i) \ln(2K^2)}{\left((\bar{a}_i+k_i)\mu_i - \ell(\bar{\bm{a}})\right)^2}~.
	\end{equation}
	Observe that in each round where $N_i(\cdot)$ is incremented, the number of samples received from the distribution $\nu_i$ increases by at least $\bar{a}_{i}+k_i$. 
	Therefore, we have \eqref{eq:ni2} implies
	\begin{align*}
		K_{i,k}
		>  \frac{1024e^2 \xi^2\alpha_i^2 (\bar{a}_i+k_i)^2 \ln(2K^2)}{\left((\bar{a}_i+k_i)\mu_i - \ell(\bar{\bm{a}})\right)^2}
		=  \frac{1024e^2 \xi^2\alpha_i^2  \ln(2K^2)}{\left(\mu_i - \frac{\ell(\bar{\bm{a}})}{\bar{a}_i+k_i}\right)^2}~.
	\end{align*}
	Therefore, we have
	\begin{align*}
		C_{i,k} &= 4e \sqrt{\frac{\ln(2k^2)}{K_{i,k}}}+4e \frac{\ln(2k^2)}{K_{i,k}}\\
		&\le \left(\mu_i - \frac{\ell(\bar{\ba})}{\bar{a}_i+k_i}\right) \left[4e \sqrt{\frac{\ln(2k^2)}{1024e^2 \xi^2 \alpha_i^2 \ln(2K^2)}}+ \frac{4e \ln(2k^2)\left(\mu_i - \frac{\ell(\bar{\ba})}{\bar{a}_i+k_i}\right)}{1024e^2\xi^2\alpha_i^2 \ln(2K^2)} \right]  \\
		&\le \frac{1}{4\xi \alpha_i} \left(\mu_i - \frac{\ell(\bar{\ba})}{\bar{a}_i+k_i}\right) \left[ \frac{1}{2}+ \frac{\mu_i - \frac{\ell(\bar{\ba})}{\bar{a}_i+k_i}}{256\xi \alpha_i}\right]~.
	\end{align*}
	Using Lemma~\ref{lem:tech1}, we have $\mu_i - \frac{\ell(\bar{\ba})}{\bar{a}_i+k_i} \le \mu_i \le \alpha_i$. Moreover, by definition of $\xi$, we have $\xi \ge 1$. We conclude using the bound above that
	\begin{equation}\label{eq:Ci}
		C_{i,k} \le \frac{3}{20\xi \alpha_i}\left(\mu_i - \frac{\ell(\bar{\ba})}{\bar{a}_i+k_i}\right)~.
	\end{equation}
	\noindent Recall that since $\neg \mathcal{E}_{k}$ holds, in particular $\neg \mathcal{E}_{i,k}$ holds, which gives
	\begin{align*}
		2\hat{\mu}_{i,k} -2\hat{s}_{i,k} &= 2\hat{\mu}_{i,k} \left(1-\left(1-\xi C_{i,k}\right)_{+}\right)\\
		&\le 2\hat{\mu}_{i,k} \left(1-\left(1-\xi C_{i,k}\right)\right)\\
		&= 2\xi C_{i,k}\hat{\mu}_{i,k}\\
		&\le 2\xi C_{i,k} (\mu_i+\alpha_i C_{i,k})\\
		&\le 2\xi \alpha_i C_{i,k} (1+ C_{i,k}),
		%&\le 2\mu_i -2\hat{\mu}_{i,k}\nonumber\\
		%&\le 16e\alpha_i \sqrt{\frac{\ln(2k^2)}{K_{i,k}}}+16e\alpha_i \frac{\ln(2k^2)}{K_{i,k}} \nonumber\\
		%&\le \mu_i - \frac{\ell(\bar{\bm{a}})}{\bar{a}_i+k_i}. \label{eq:Ki2}
	\end{align*}
	where we used the event $\neg \mathcal{E}_{i,k}$ in the penultimate inequality, and $\mu_i \le \alpha_i$ as showed in Lemma~\ref{lem:tech1} in the last inequality.
	Using bound \eqref{eq:Ci} in the previous display gives
	\begin{align}
		2\hat{\mu}_{i,k} -2\hat{s}_{i,k} &\le \frac{3}{10} \left(\mu_i - \frac{\ell(\bar{\ba})}{\bar{a}_i+k_i}\right) (1+C_{i,k})
		\le \frac{3}{8} \left(\mu_i - \frac{\ell(\bar{\ba})}{\bar{a}_i+k_i}\right), \label{eq:conf22}
	\end{align} 
	where we used in the last line the fact that following \eqref{eq:Ci}: $C_i \le \frac{3}{20\xi \alpha_i}(\mu_i - \ell(\bar{\ba})/(\bar{a}_i+k_i)) \le 3/20$, since $\xi \ge 1$ by definition and $\alpha_i \ge \mu_i \ge \mu_i - \ell(\bar{\ba})/(\bar{a}_i+k_i)$ following Lemma~\ref{lem:tech1}.
	
	\noindent Recall that \eqref{eq:Ci} implies in particular that $C_{i,k} \le \frac{3\mu_i}{20\xi \alpha_i} \le 3/(20\xi)$. Since $\neg \mathcal{E}_{i,k}$ is true, we have  $\abs{\hat{\mu}_{i,k} - \mu_i} \le \alpha_i C_{i,k}$. Therefore, using \Cref{lem:conc2}, we have
	\begin{align}
		\mu_i &\le \hat{\mu}_{i,k} \left(1 + \frac{4}{3} \xi C_{i,k}\right)\label{eq:ucb0}\\
		&\le \frac{21}{20}~\hat{\mu}_{i,k}~. \label{eq:ucb}
	\end{align}
	
	Observe that in each round $\norm{\bm{a}_k}_0 = B$, therefore if we have $a_{i,k} \ge \bar{a}_i+k_i > \bar{a}_i$ for some $i$, we necessarily have that there exists $j \in [n]\setminus \{i\}$ such that $a_{j,k} \le \bar{a}_j-1$. Using the fact that $\ell(\bar{\bm{a}}) \ge \bar{a}_j \mu_j$ with \eqref{eq:conf22}, we get
	$$
	5\hat{\mu}_{i,k} -5~\hat{s}_{i,k} < \mu_i - \frac{\ell(\bar{\ba})}{a_{i,k}}~.
	$$
	Therefore, we obtain
	\begin{equation}\label{eq:e12}
		a_{i,k}(\mu_i+ 5\hat{s}_{i,k}-5\hat{\mu}_{i,k}) > \ell(\bar{\ba}) \ge \bar{a}_j \mu_j~.
	\end{equation}
	Since both $\neg \mathcal{E}_{i,k}$ and $\neg \mathcal{E}_{j,k}$ hold (because $\neg E_{i,k}$ implies $\neg \mathcal{E}_k$), we have that
	\begin{align*}
		\mu_i +5\hat{s}_{i,k}-5\hat{\mu}_{i,k} &=  \hat{s}_{i,k} + \mu_i - \hat{\mu}_{i,k} + 4 \left(\hat{s}_{i,k} - \hat{\mu}_{i,k} \right)\\
		&= \hat{s}_{i,k} + \mu_i - \hat{\mu}_{i,k} + 4\hat{\mu}_{i,k} \left((1-\xi C_{i,k})_{+}-1 \right)\\
		&\le \hat{s}_{i,k} + \mu_i - \hat{\mu}_{i,k} - 4\hat{\mu}_{i,k} \xi C_{i,k}\\
		&\le \hat{s}_{i,k} + \mu_i - \hat{u}_{i,k}\\
		&\le \hat{s}_{i,k},
	\end{align*} 
	where we used in the last line the bound \eqref{eq:ucb0}. Since $\neg \mathcal{E}_{j,k}$ holds, we also have
	\begin{equation*}
		\mu_j \ge \hat{s}_{j,k}~.
	\end{equation*}
	Using the two last bounds in \eqref{eq:e12}, we have
	$$
	a_{i,k} \hat{s}_{i,k} > \bar{a}_j \hat{s}_{j,k} \ge (a_{j,k}+1) \hat{s}_{j,k},
	$$
	where we used the definition of $j$, as an arm satisfying $\bar{a}_j \ge 1+a_{j,k}$, in the second inequality.
	This contradicts the statement of Lemma~\ref{lem:1}, which concludes the contradiction argument. Therefore, the event that $k$ is a bad round implies that $E_{i,k}$ holds for at least one $i\in [n]$.
	We say that a bad round was triggered by arm $i$, a round where $N_i(\cdot)$ was incremented. 
	Observe that if $k \in [K]$ is not a bad round then $\E{\ell(\bm{a}_k)}-\ell(\bar{\bm{a}})=0$, otherwise if $k$ is a bad round triggered by $i \in [n]$ then $\E{\ell(\bm{a}_{k})}-\ell(\bar{\bm{a}}) \le B\mu_i-\ell(\bar{\bm{a}})$.
	To ease notation we introduce for $i\in [n]$
	$$
	H_i := \frac{1024e^2 \xi^2\alpha_i^2 (\bar{a}_i+k_i) \ln(2K^2)}{\left((\bar{a}_i+k_i)\mu_i - \ell(\bar{\bm{a}})\right)^2}~.
	$$
	The expected regret satisfies
	\begin{align*}
		\mathcal{R}_K &= \sum_{i=1}^{K} \mathbb{E}\left[\ell(\bm{a}_k)-\ell(\bar{\bm{a}})\right]\\
		&\le \sum_{i=1}^{n} (B\mu_i-\ell(\bar{\bm{a}}))\mathbb{E}[N_i(K)]\\ 
		&= \sum_{i=1}^{n}\sum_{k=1}^{K} (B\mu_i-\ell(\bar{\bm{a}}))\mathbb{E}\left[\mathds{1}(k \text{ is a bad round triggered by }i)\right]\\
		&\le \max_{i\in [n]}\{(B\mu_i-\ell(\bar{\bm{a}}))\}\cdot\sum_{t=1}^{K} \mathbb{P}(\mathcal{E}_k)+ \sum_{i=1}^{n}(B\mu_i-\ell(\bar{\bm{a}}))\sum_{k=1}^{K} \mathbb{E}\left[\mathds{1}(k \text{ is a bad round triggered by }i) \mid \neg \mathcal{E}_k\right]\\
		&\le \max_{i\in [n]}\{(B\mu_i-\ell(\bar{\bm{a}}))\}\cdot\sum_{t=1}^{K} \mathbb{P}(\mathcal{E}_k)+ \sum_{i=1}^{n}(B\mu_i-\ell(\bar{\bm{a}}))\sum_{k=1}^{K} \mathbb{E}\left[\mathds{1}(N_i(k)=1+N_i(k-1) \text{ and } N_i \le H_i ) \mid \neg \mathcal{E}_k\right]\\
		&\le \max_{i\in [n]}\{(B\mu_i-\ell(\bar{\bm{a}}))\}\cdot\sum_{k=1}^{K} \mathbb{P}(\mathcal{E}_k)+ \sum_{i=1}^{n} (B\mu_i-\ell(\bar{\bm{a}}))H_i\\
		&\le 2n \max_{i\in [n]}\{(B\mu_i-\ell(\bar{\bm{a}}))\}+  \sum_{i=1}^{n} \frac{1024e^2 \xi^2\alpha_i^2 (\bar{a}_i+k_i)(B \mu_i - \ell(\bar{\ba})) \ln(2K^2)}{\left((\bar{a}_i+k_i)\mu_i - \ell(\bar{\bm{a}})\right)^2}~. \qedhere
	\end{align*}
\end{proof}













\subsection{Proof of \Cref{cor:main}}
\label{sec:proof_2}

Let us first restate the theorem.
\begin{restate-theorem}{\ref{cor:main}}
	Suppose \Cref{a:sube} holds and let $\eta := \max_{i \in [n]} \frac{\sigma_i}{\mu_i}$.
	Then, the total expected computation time after $K$ rounds, using the allocation prescribed by \algname{ATA} with inputs $(B, \alpha)$ satisfies
	$$
	\mathcal{C}_K \le \left(1+\eta\sqrt{\ln(B)}\right)\mathcal{C}_K^* + \mathcal{O}(\ln K)~.
	$$
\end{restate-theorem}
%
\begin{proof}
Let $\mathbb{E}_k$ be the expectation with respect to the variables observed up to and including $k$ and $\mathcal{F}_k$ the corresponding filtration. Using the tower rule, we have
$$
\sum_{k=1}^{K}\mathbb{E}\left[C(\bm{a}_k)\right] = \mathbb{E}\left[ \sum_{k=1}^{K} \mathbb{E}_{k-1}[C(\bm{a}_k)]\right].
$$
Consider round $k \in [K]$, let us upper bound $\mathbb{E}_{k-1}[C(\bm{a}_t)]$ using $\mathbb{E}_{k-1}[\ell(\bm{a}_k)]$. We have (recall that $\bm{a}_k \in \mathcal{F}_{k-1}$)
\begin{align*}
	\mathbb{E}_{k-1}\left[C(\bm{a}_k) \right] &= \mathbb{E}_{k-1}\left[ \max_{i \in \text{supp}(\bm{a}_k)}\left\lbrace \sum_{u=1}^{a_{i,k}} X^{(u)}_{i,k}  \right\rbrace\right]\\
	&\le \max_{i \in \text{supp}(\bm{a}_k)}\left\lbrace a_{i,k} \mu_i\right\rbrace + \max_{i \in \text{supp}(\bm{a}_k)} \{ a_{i,k} \sigma_i\} \cdot \sqrt{\ln B}\\
	&\le \max_{i \in \text{supp}(\bm{a}_k)}\left\lbrace a_{i,k} \mu_i\right\rbrace + \max_{i \in \text{supp}(\bm{a}_k)} \{ a_{i,k}\, \eta\mu_i\} \cdot \sqrt{\ln B}\\
	&= \left(1+\eta \sqrt{\ln(B)}\right)\max\left\lbrace a_{i,k} \mu_i\right\rbrace.
\end{align*}
Moreover, using Jensen's inequality, we have
\begin{align*}
	\max_{i \in [n]} \{a^*_i \mu_i\}
	\le \mathbb{E}\left[\max_{i \in [n]} \left\lbrace \sum_{u=1}^{a_{k,i}} X^{(u)}_{i,k} \right\rbrace \right]
	= \mathbb{E}[C(\bm{a}^*)]~.
\end{align*}

Using the last two bounds with the result of \Cref{thm:main}, we get the result.

\end{proof}



\section{Technical Results}
\label{sec:technical}

We consider the following concentration inequality for sub-exponential variables by \citet{maurer2021concentration}.

\begin{proposition}[Proposition 7 \citep{maurer2021concentration}
	]\label{prop:concentration}
	Suppose $X_1, \dots, X_n$ are positive i.i.d variables such that $\norm{X_1}_{\psi_1} < \infty$ and $\mu = \mathbb{E}[X_1]$. Let $\delta >0$, with probability at least $1-\delta$
	$$
	\abs{\frac{1}{n}\sum_{i=1}^{n}X_i - \mu} \le 4e\norm{X_1}_{\psi_1} \sqrt{\frac{\ln(2/\delta)}{n}}+4e\norm{X_1}_{\psi_1} \frac{\ln(2/\delta)}{n}~.
	$$ 
\end{proposition}


\begin{lemma}
	Let $X_1, \dots, X_n$ be a sequence of nonnegative  random variables. Such that $\mathbb{E}[X_i]=\mu_i$ and $\text{Var}(X_i) = \sigma_i^2$ for each $i \in [n]$. Then we have
	$$
	\mathbb{E}[\max\{X_1, \dots, X_n\}] \le \max\{\mu_1, \dots, \mu_n\}+\max_{i\in [n]} \{\sigma_i \}\cdot \sqrt{\ln n}~.
	$$  
\end{lemma}


\begin{lemma}\label{lem:tech1}
	Let $X$ be a positive random variable with mean $\mu := \mathbb{E}[X]>0$ and variance $\sigma^2 = \text{Var}(X)$. Then the sub-exponential norm of $X$ satisfies
	$$
	\norm{X}_{\psi_1} \le \frac{1+\sqrt{4\eta^2+5}}{2}\cdot \mu,
	$$
	where $\eta := \frac{\sigma}{\mu}$.
	Moreover, we have
	$$
	\mu \le \norm{X}_{\psi_1}~.
	$$
\end{lemma}
%
\begin{proof}
	Let $\alpha = \norm{X}_{\psi_1}$, $\sigma := \sqrt{\text{Var}(X)}$, $\mu := \mathbb{E}[X]$, and $\eta := \frac{\sigma}{\mu}$. We aim to prove that
	$$
	\alpha \le \frac{1+\sqrt{4\eta^2+5}}{2}\cdot \mu~.
	$$
	For $\epsilon \in (0, \alpha/2)$, we have by definition of $\alpha$
	$$
	\mathbb{E}[\exp(X/(\alpha-\epsilon))] \ge 2~.
	$$
	Recall that we have for any $x\ge 0: \exp(x) \le 1+x+\frac{x^2}{2}e^{x}$, therefore
	$$
	\mathbb{E}\left[\exp(X/(\alpha-\epsilon)) \right] \le 1+ \frac{\mu}{\alpha-\epsilon}+ \frac{\mathbb{E}[X^2]}{2(\alpha-\epsilon)^2} \mathbb{E}[\exp(X/(\alpha-\epsilon))]~.
	$$
	Therefore,
	$$
	1+ \frac{\mu}{\alpha-\epsilon}+ \frac{\mathbb{E}[X^2]}{2(\alpha-\epsilon)^2} \mathbb{E}[\exp(X/(\alpha-\epsilon))] \ge 2~.
	$$
	Taking $\epsilon \to 0$, by continuity we have
	$$
	1+ \frac{\mu}{\alpha}+ \frac{\mathbb{E}[X^2]}{2\alpha^2} \mathbb{E}[\exp(X/\alpha)] \ge 2~.
	$$
	Therefore,
	$$
	1+ \frac{\mu}{\alpha}+ \frac{\mathbb{E}[X^2]}{\alpha^2} \ge 2~.
	$$
	Solving the last inequality gives
	$$
	\alpha \le \frac{\mu + \sqrt{\mu^2+4\mathbb{E}[X^2]}}{2},
	$$
	and using $\mathbb{E}[X^2] = \sigma^2+\mu^2 = (1+\eta^2)\mu^2$, we get
	$$
	\alpha \le \frac{1+\sqrt{4\eta^2+5}}{2}\mu~.
	$$
	The second bound is a direct consequence of Jensen's inequality and the definition of $\norm{X}_{\psi_1}$.
\end{proof}

\begin{lemma}\label{lem:conc2}
	Consider the notation in Lemma~\ref{prop:concentration} and Lemma~\ref{lem:tech1}. Define $C_{\cdot, \cdot}$, $F(\cdot, \cdot)$ and $G(\cdot, \cdot)$ by:
	\begin{align*}
		C_{n,\delta} &:= 4e  \sqrt{\frac{\ln(2/\delta)}{n}}+4e\cdot \frac{\ln(2/\delta)}{n}\\
		F(n, \delta) &:= \hat{X}_n \left(1 - \xi C_{n, \delta}\right)_+\\	
		G(n, \delta) &:=  \hat{X}_n \left(1 + \frac{4}{3}\xi C_{n,\delta}\right)~,
	\end{align*}
	where we use the notation $(a)_+ = \max\{0,a\}$.
	Then, if
	$$
	\abs{\hat{X}_n - \mu} \le \alpha C_{n,\delta},
	$$
	we have
	$$
	\mu \ge F(n, \delta)~.
	$$
	Moreover, if we have additionally $ C_{n, \delta} \le \frac{1}{4\xi}$, then
	$$
	\mu \le G(n,\delta)~.
	$$
\end{lemma}
%
\begin{proof}
	Fix $n, \delta$. 
	Suppose that
	\begin{equation}\label{eq:conc}
		\abs{\hat{X}_n - \mu} \le \alpha C_{n, \delta}~.
	\end{equation}
	\textbf{Proof of $\mu \ge F(n, \delta)$:}
	we have that if $\xi C_n \ge 1$ then $F(n, \delta) = 0$ and the result is straightforward. Suppose that $\xi C_n < 1$, if $\hat{X}_n \le \mu$, we have that $F(n, \delta) = \hat{X}_n (1-\xi C_n) \le \mu$, if $\hat{X}_n \ge \mu$, we have using \eqref{eq:conc} with the bound of \Cref{lem:tech1}
	\begin{align}
		\label{eq:conc2}
		\abs{\hat{X}_n - \mu} &\le \alpha C_{n, \delta}
		\le	\mu\xi\cdot C_{n, \delta}.
	\end{align}
	Therefore, when $\hat{X}_n \ge \mu$, we have
	\begin{align*}
		F(n, \delta) &= \hat{X}_n \left(1-\xi C_{n, \delta}\right)
		= \mu+ \abs{\hat{X}_n - \mu} - \hat{X}_n \xi C_{n, \delta}
		\le  \mu+ \abs{\hat{X}_n - \mu} - \mu \xi C_{n, \delta}
		\le \mu~.
	\end{align*}
	\textbf{Proof of $\mu \le G(n,\delta)$:} Suppose that $C_{n,\delta} \le \frac{1}{4\xi}$. 
	Therefore, \eqref{eq:conc2} gives that $\hat{X}_n \ge \frac{3}{4}\mu$. Using \eqref{eq:conc2} again gives
	\begin{align*}
		\mu &\le \hat{X}_n+ \mu \xi \cdot C_{n,\delta}
		\le \hat{X}_n+  \frac{4}{3}\xi\hat{X}_n\cdot C_{n,\delta}
		= G(n,\delta)~. \qedhere
	\end{align*}
\end{proof}


%\begin{lemma}\label{lem!KL}
%	Let $X$ and $Y$ be two sub-exponential distributions with parameters $x$ and $y=x+\epsilon$ respectively, for $\epsilon \ge 0$. Then we have:
%	$$
%	\text{KL}\left(Y; X\right) \le \frac{\epsilon^2}{2x^2},
%	$$
%	where $\text{KL(.,.)}$ denotes the Kullback-Leibler divergence between $Y$ and $X$.
%\end{lemma}
%\begin{proof}
%	Using the definition of KL divergence between $X$ and $Y$ given the expression of the exponential distribution densities, we have:
%	\begin{align*}
%		\text{KL}\left(Y; X\right) &\le \ln\frac{x}{x+\epsilon}+\frac{x+\epsilon}{x}-1\\
%		&\le \frac{\epsilon^2}{2x^2}.
%	\end{align*}
%	
%\end{proof}


%%%%%%%%%%%%%%%%%%%%%%%%%%%%%%%%%%%%%%%%%%%%%%%%%%%%%%%%%%%%%%%%%%%%%%%%%%%%%%%
%%%%%%%%%%%%%%%%%%%%%%%%%%%%%%%%%%%%%%%%%%%%%%%%%%%%%%%%%%%%%%%%%%%%%%%%%%%%%%%

\end{document}


% This document was modified from the file originally made available by
% Pat Langley and Andrea Danyluk for ICML-2K. This version was created
% by Iain Murray in 2018, and modified by Alexandre Bouchard in
% 2019 and 2021 and by Csaba Szepesvari, Gang Niu and Sivan Sabato in 2022.
% Modified again in 2023 and 2024 by Sivan Sabato and Jonathan Scarlett.
% Previous contributors include Dan Roy, Lise Getoor and Tobias
% Scheffer, which was slightly modified from the 2010 version by
% Thorsten Joachims & Johannes Fuernkranz, slightly modified from the
% 2009 version by Kiri Wagstaff and Sam Roweis's 2008 version, which is
% slightly modified from Prasad Tadepalli's 2007 version which is a
% lightly changed version of the previous year's version by Andrew
% Moore, which was in turn edited from those of Kristian Kersting and
% Codrina Lauth. Alex Smola contributed to the algorithmic style files.
