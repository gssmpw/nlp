%%%%%%%% ICML 2025 EXAMPLE LATEX SUBMISSION FILE %%%%%%%%%%%%%%%%%

\documentclass{article}

% Recommended, but optional, packages for figures and better typesetting:
\usepackage{microtype}
\usepackage{graphicx}
\usepackage{subfigure}
\usepackage{booktabs} % for professional tables

% hyperref makes hyperlinks in the resulting PDF.
% If your build breaks (sometimes temporarily if a hyperlink spans a page)
% please comment out the following usepackage line and replace
% \usepackage{icml2025} with \usepackage[nohyperref]{icml2025} above.
\usepackage{hyperref}


% Attempt to make hyperref and algorithmic work together better:
\newcommand{\theHalgorithm}{\arabic{algorithm}}

% Use the following line for the initial blind version submitted for review:
% \usepackage{icml2025}

% If accepted, instead use the following line for the camera-ready submission:
\usepackage[accepted]{icml2025}

% For theorems and such
\usepackage{amsmath}
\usepackage{amssymb}
\usepackage{mathtools}
\usepackage{amsthm}

% if you use cleveref..
\usepackage[capitalize,noabbrev]{cleveref}

%%%%%%%%%%%%%%%%%%%%%%%%%%%%%%%%
% THEOREMS
%%%%%%%%%%%%%%%%%%%%%%%%%%%%%%%%
\theoremstyle{plain}
\newtheorem{theorem}{Theorem}[section]
\newtheorem{proposition}[theorem]{Proposition}
\newtheorem{lemma}[theorem]{Lemma}
\newtheorem{corollary}[theorem]{Corollary}
\theoremstyle{definition}
\newtheorem{definition}[theorem]{Definition}
\newtheorem{assumption}[theorem]{Assumption}
\theoremstyle{remark}
\newtheorem{remark}[theorem]{Remark}

% Todonotes is useful during development; simply uncomment the next line
%    and comment out the line below the next line to turn off comments
%\usepackage[disable,textsize=tiny]{todonotes}
\usepackage[textsize=tiny]{todonotes}


% The \icmltitle you define below is probably too long as a header.
% Therefore, a short form for the running title is supplied here:
\icmltitlerunning{{\red ATA}: {\red A}daptive {\red T}ask {\red A}llocation for Efficient Resource Management in Distributed Machine Learning}


\newcommand{\thought}[1]{{\color[rgb]{0.2,0.39,0.66}(#1)}}
\newcommand{\todo}[1]{{\color[rgb]{1.0,0.0,0.0}(#1)}}
\newcommand{\hsh}[1]{{\color{green!50!black} Henrik: #1}}
\newcommand{\st}[1]{{\color{red!50!black} Sebastian: #1}}

\newcommand{\ulm}[1]{_{\scaleto{\mathrm{#1}}{3pt}}}
\newcommand\at[2]{\left.#1\right|_{#2}}











\newtheorem{assumption}{Assumption}

\DeclareMathOperator*{\argmax}{arg\,max}
\DeclareMathOperator*{\argmin}{arg\,min}

\newcommand{\swname}[1]{\texttt{#1}}
\newcommand{\ie}{i\/.\/e\/.,\/~}
\newcommand{\eg}{e\/.\/g\/.,\/~}
\newcommand{\cf}{cf\/.\/~}

\newcommand{\fig}{Fig\/.\/~}
\newcommand{\defn}{Def\/.\/~}
\newcommand{\sect}{Sec\/.\/~}
\newcommand{\tabl}{Tab\/.\/~}
\newcommand{\algo}{Algorithm~}
\newcommand{\theo}{Theorem~}

\newcommand{\bnnl}{3 hidden layers}
\newcommand{\bnnn}{50 neurons}
\newcommand{\bnna}{tanh activations}

\newcommand{\capt}[1]{\mdseries{\emph{#1}}}

\newcommand{\videolink}{at \url{https://youtu.be/_d7AqTRjz6g}}
\newcommand{\codelink}{\url{https://github.com/wheelbot/mini-wheelbot}}

\newcommand{\fakepar}[1]{\vspace{0mm}\noindent\textbf{#1.}}

\newcommand{\needref}{\textcolor{red}{[REF]}}

\newcommand{\plotfontsize}{9pt}

\usepackage{listings}


\begin{document}

\twocolumn[
\icmltitle{
    {\red ATA}: {\red A}daptive {\red T}ask {\red A}llocation for Efficient Resource Management \\
    in Distributed Machine Learning
    % \algname{ATA}: {\red A}daptive {\red T}ask {\red A}llocation for Efficient Resource Management in Distributed Machine Learning with Heterogeneous and Random Worker Times
    }


% It is OKAY to include author information, even for blind
% submissions: the style file will automatically remove it for you
% unless you've provided the [accepted] option to the icml2025
% package.

% List of affiliations: The first argument should be a (short)
% identifier you will use later to specify author affiliations
% Academic affiliations should list Department, University, City, Region, Country
% Industry affiliations should list Company, City, Region, Country

% You can specify symbols, otherwise they are numbered in order.
% Ideally, you should not use this facility. Affiliations will be numbered
% in order of appearance and this is the preferred way.
\icmlsetsymbol{equal}{*}

\begin{icmlauthorlist}
\icmlauthor{Artavazd Maranjyan}{kaust}
\icmlauthor{El Mehdi Saad}{kaust}
\icmlauthor{Peter Richt\'{a}rik}{kaust}
\icmlauthor{Francesco Orabona}{kaust}
\end{icmlauthorlist}


\icmlaffiliation{kaust}{King Abdullah University of Science and Technology (KAUST), Thuwal, Saudi Arabia}

\icmlcorrespondingauthor{Artavazd Maranjyan}{\href{https://artomaranjyan.github.io/}{https://artomaranjyan.github.io/}}
    

% You may provide any keywords that you
% find helpful for describing your paper; these are used to populate
% the "keywords" metadata in the PDF but will not be shown in the document
\icmlkeywords{Multi-Armed Bandit, UCB, adaptive task allocation, asynchronous methods, parallel methods, SGD}

\vskip 0.3in
]

% this must go after the closing bracket ] following \twocolumn[ ...

% This command actually creates the footnote in the first column
% listing the affiliations and the copyright notice.
% The command takes one argument, which is text to display at the start of the footnote.
% The \icmlEqualContribution command is standard text for equal contribution.
% Remove it (just {}) if you do not need this facility.

\printAffiliationsAndNotice{}  % leave blank if no need to mention equal contribution
% \printAffiliationsAndNotice{\icmlEqualContribution} % otherwise use the standard text.


\begin{abstract}
    % OpenReview abstract
    % Asynchronous methods are fundamental for parallelizing computations in distributed machine learning. 
    % They aim to accelerate training by fully utilizing all available resources.
    % However, their greedy approach can lead to inefficiencies using more computation than required, especially when computation times vary across devices.
    % If the computation times were known in advance, training could be fast and resource-efficient by assigning more tasks to faster workers.
    % The challenge lies in achieving this optimal allocation without prior knowledge of the computation time distributions.
    % In this paper, we propose ATA (Adaptive Task Allocation), a method that adapts to heterogeneous and random distributions of worker computation times.
    % Through rigorous theoretical analysis, we show that ATA identifies the optimal task allocation and performs comparably to methods with prior knowledge of computation times.
    % Experimental results further demonstrate that ATA is resource-efficient, significantly reducing costs compared to the greedy approach, which can be arbitrarily expensive depending on the number of workers.

    Asynchronous methods are fundamental for parallelizing computations in distributed machine learning. 
    They aim to accelerate training by fully utilizing all available resources.
    However, their greedy approach can lead to inefficiencies using more computation than required, especially when computation times vary across devices.
    If the computation times were known in advance, training could be fast and resource-efficient by assigning more tasks to faster workers.
    The challenge lies in achieving this optimal allocation without prior knowledge of the computation time distributions.
    In this paper, we propose \algname{ATA} ({\red A}daptive {\red T}ask {\red A}llocation), a method that adapts to heterogeneous and random distributions of worker computation times.
    Through rigorous theoretical analysis, we show that \algname{ATA} identifies the optimal task allocation and performs comparably to methods with prior knowledge of computation times.
    Experimental results further demonstrate that \algname{ATA} is resource-efficient, significantly reducing costs compared to the greedy approach, which can be arbitrarily expensive depending on the number of workers.
\end{abstract}

\documentclass[../main.tex]{subfiles}
\graphicspath{{../images/}}
\makeatletter
\def\input@path{{../images/}}
\makeatother
\begin{document}
\section{Introduction}
\begin{figure}
\centering
\begin{tikzpicture}
\node[inner sep=0pt] (ws) at (0, 0) {
\includegraphics[height=.4\textwidth, trim={10cm 0 10cm 0},clip]{world_space.png}};
\node[inner sep=0pt] (cs) at (6,0) {\includegraphics[height=.4\textwidth, trim={10cm 1cm 10cm 4cm},clip]{conf_space.png}};
\end{tikzpicture}
\vspace{-5pt}
\label{fig:pbrm_intro}
\caption{\textbf{Left}: Shows world space obstacles as grey spheres. Robots start and goal configuration is colored red and green, respectively. Configurations along the computed path are colored transparent blue. \textbf{Right:} Mapped world space scenario to configuration space. Obstacle region is the grey mesh. Red spheres are collision-free regions computed by the neural SCDF. The optimized shortest path in the convex corridor is the blue curve.}
\vspace{-25pt}
\end{figure}
Motion planning is the problem of finding a collision-free trajectory that connects a given start and goal configuration. The planning takes place in the configuration space of the robot. For single body robots, like mobile robots or drones, the configuration space and the world space are usually the same. This simplifies the planning, since explicit obstacle representations are available which enables geometrical tools like separating hyperplanes, smallest distance to obstacles etc., to be used when designing motion planning algorithms. For multi-body robots like manipulators, the situation is completely different. The world space obstacles are usually mapped to non-convex regions, and to make the problem even harder, the mapping is usually not known. Forming explicit representations of the obstacle region in the configuration space is usually too expensive or intractable. Despite all of this, sampling based planners are used with great success, which mainly is due to their use of implicit representations of the obstacle region. The basic idea is to construct a graph in the configuration space that covers and connects the collision-free region. From this graph, a path can be extracted that connects a given start and goal configuration. The approach is computationally expensive, since the graph is constructed with the smallest geometrical building block available, points, which represents a collision-check. Furthermore, the extracted paths from the graph are non-smooth and jagged due to the stochastic nature of the approach. This adds an additional post-processing step to the process, where the paths are shortcutted and smoothened, before the path can be used for tracking. Clearly a lot of time is invested to form this graph and produce smooth paths. Thus, if the obstacles start to move, then all of this work is done in no use, since all points that make up this graph need to be re-verified, which is simply too time consuming to be done in real time.
\\\\
In this work, we want to address the existing drawbacks of the sampling based planners. Our main contribution is an improved motion planner where each vertex in the graph covers a collision-free region in the form of a sphere instead of a point and where the edges are formed with neighboring intersecting spheres. This representation has the advantage of instead of returning piecewise linear paths, returning a sequence of overlapping spheres, i.e. a convex corridor, that connects a given start and goal configuration, illustrated in Figure \ref{fig:pbrm_intro}. This convex corridor allows us to use convex optimization to produce smooth trajectories, instead of computationally expensive post-processing methods. The representation further allows us to estimate the coverage of the collision-free space, which gives us awareness and feedback in the offline roadmap construction phase. Finally, our representation is simple to adapt to moving obstacles, simply requery for the new radii and recheck for intersections. 
\\\\
The spherical collision-free regions are formed using a signed distance function (SDF), which is a function that returns the smallest distance from an arbitrary point to the boundary of an obstacle. As the name implies, the distance is signed, thus if the point is inside the obstacle it is negative otherwise positive. If the distance is positive, a sphere with radius equal to the distance is guaranteed to cover a collision-free region. Using an SDF in motion planning is not new, but what is novel about our approach is that we express the distance in the configuration space instead of the world space and by doing so allows us to form these convex collision-free regions. We refer to the resulting SDF as a signed configuration distance function (SCDF). Computing an SCDF analytically is non-trivial, our approach is therefore to parameterize the SCDF with a deep neural network and learn the mapping by supervised learning. Our resulting neural SCDF can compute distances for different parameter values of obstacle shapes and we also show how multiple distances can be combined, thus making our approach flexible.
\section{Related work}
Motion planning algorithms can roughly be divided into three families, grid-based, sampling based and optimization based methods. Grid-based methods (GBM) discretize the planning space from which a graph is then compiled. A standard search method is A$^\star$ \citep{a_star}, which is classified as an \textit{informed} search method, since it employs a heuristic function to speed up the search. A$^\star$ guarantees to return an optimal path at the level of discretization used. GBMs usually discretize the planning space by a regular lattice and this limits the GBMs to problems with low dimensionality due to the curse of dimensionality. Thus, GBMs are usually limited to single-body robots where the degrees of freedom (DOF) are low. To overcome the inherent scaling problem with the GBMs, stochastic methods are usually used for multi-body robots. These methods are termed as sampling-based methods (SBM) and core members within this family are the rapidly-exploring random trees (RRT) \citep{rrt} and the probabilistic roadmap (PRM) \citep{prm}. RRT grows a tree from the start configuration and explores the collision-free region in a rapid way until it is able to connect to the goal region. RRT is usually improved by bi-directional planning \citep{rrt_connect}, i.e. an additional tree is grown from the goal configuration and the trees are tested for connection after any tree has been expanded. RRT is a single-query method, thus it searches for a path from scratch each time it is queried. Contrary to this, PRM is a multi-query method, which solves for multiple queries without starting from scratch. PRM does this by creating a roadmap (graph) that covers the collision-free space as an offline step. The graph is then used to solve for multiple queries. PRMs are used in cases where the environment does not change since the extra offline step is too computationally costly and needs to be re-done if the environment is changed. In our work, we address this inherent issue by using a different roadmap representation. Our vertices in the graph cover a collision-free region in the form of spheres and we form the edges by checking for intersecting spheres. If something in the environment changes, we recompute the spheres radii and recheck the intersections, without relying on collision detection. We use a trained neural network to compute the sphere radius, therefore querying for the radius can be done fast, hence our representation enables the PRM for dynamic environments.
\\\\
In the recent decades, optimization based methods (OBM) \citep{chomp, schulman, itomp, stomp} have been introduced as an alternative to SBM for multi-body robots. Like the SBM, the OBMs scale well to higher dimensional problems and produce smoother motion. It is common to use a SDF in the optimization since it is a smooth function, thus enabling gradient-based methods. However, the standard way of expressing the SDF is in world space. The distance therefore needs to be mapped to the configuration space by the forward kinematics. This mapping makes the optimization problem a non-linear program (NLP), which is computationally expensive to solve. Recently, a different approach has been proposed. In \cite{mp_gcs} motion planning is formulated as a convex optimization problem by using the graph of convex sets framework \citep{gcs}. The underlying idea is to decompose the collision-free space into intersecting convex sets from which a convex optimization problem is formulated. In cases where an explicit representation of the obstacles in the configuration space exists, like for single-body robots, creating collision-free convex regions can be done fast \citep{iris}. For multi-body robots, this is non-trivial. Existing work does this successfully \citep{iris_nlp, iris_c} by an optimization based approach, but the methods are still too time consuming to be used in the presence of moving obstacles. Our approach is instead to use deep learning to learn an SDF expressed in the configuration space. With this, we can query for shortest distances to the collision boundary, which allows us to expand spherical regions which are collision-free. Our approach is fast and therefore enables our suggested roadmap planner to be used in dynamic environments.
\\\\
Recent research has focused on learning collision detection \citep{fk_kernel_distance, diffco, graphdistnet} by predicting the signed distance between the robot links and the surrounding obstacles in the world space. The learned SDF is used in trajectory optimization but since the distance is expressed in the world space, the problem becomes an NLP and therefore takes a long time to solve. We take a novel approach and suggest to instead express the signed distance in the configuration space. This allows us to improve the PRM at the same time as it enables convex optimization for trajectory optimization, which runs faster and is more reliable than NLP solvers. In \cite{cspf} a learned signed distance function in the configuration space is proposed similar to our approach. However, their approach is restricted to point cloud representations, while we propose to represent the obstacles as parameterized geometric shapes, e.g. spheres. Furthermore, we also show how to use our learned SCDF to improve an existing roadmap planner.
\section{Problem formulation}
A robot is located in the world space, $\W \subset \R^3 $. The unique location of the robot is given by its configuration $\q \in \C$, where $\C$ is the configuration space. The set of points covered by the robots bodies at a certain configuration is expressed as $\B(\q) \subset \W$. The robot is surrounded by $\NrObst$ obstacles $\O = \bigcup_{i=1}^{\NrObst} \O_i$, where  $\O_i \subset \W$. The representation of the obstacle in the configuration space is the set $\C\O_i = \{\q \in \C \: |\: \B(\q) \cap \O_i \neq \emptyset \}$. The obstacle space is formed as $\Co = \bigcup_{i=1}^{\NrObst} \C \O_i$. The complement is referred to as the free space, $\Cf = \C \setminus \Co$. The path planning problem is a tuple, ($\Cf$, $\qStart$, $\qGoal$), where we want to connect a query pair, consisting of a start, $\qStart$, and goal configuration, $\qGoal$, with a geometric path, $\q(s): [0, 1] \mapsto \Cf$, such that $\q(0)=\qStart$ and $\q(1)=\qGoal$, or report correctly when such a path does not exist.
\end{document}


\section{Related Work} \label{sec:related}

% \textbf{Adversarial Attack}
\textbf{Attacks on SLAM.} 
%With the rise of machine learning, 
The robustness of computer vision systems is being actively investigated. With the emergence of adversarial images in the digital domain by adding optimized noise directly to images~\cite{szegedy2013intriguing,carlini2017towards}, researchers find that such attacks also exist physically in the real world \cite{eykholt2018robust,song2018physical,zhao2019seeing}. To fill the gap between attacks in the digital and physical worlds, recent studies have demonstrated that attacks on real-world computer vision systems are practical \cite{eykholt2018robust,li2019adversarial,man2020ghostimage,sharif2016accessorize,zhao2019seeing,zhou2018invisible}. However, attacks on traditional computer vision methods such as SLAM are relatively less explored. \cite{yoshida2022adversarial} proposes an attack against the scan matching algorithm in LiDAR-based SLAM, while most SLAMs in AR/VR devices rely on different sensors like RGB/depth cameras and IMUs. \cite{ikram2022perceptual} and \cite{chen2024adversary} mislead visual SLAM by poisoning the images with special patterns, and \cite{wang2021can} causes the camera to fail using infrared light. In our work, we demonstrate attacks on Visual-Inertial SLAM (VI-SLAM) by perturbing the IMU readings, rather than cameras, and showing its impact on XR user experience. 

\textbf{Acoustic Injection Attacks.} Among various physical attacks, acoustic injection attacks are attractive due to their low cost. Son~\etal~\cite{son2015rocking} were the first to introduce acoustic attacks on MEMS gyroscopes, demonstrating how these attacks could lead to sensor denial-of-service and result in drone crashes. WALNUT~\cite{trippel2017walnut} expanded on this by developing output biasing and control attacks that enable precise manipulation of MEMS accelerometer outputs using modulated sound waves. Wang et al.~\cite{wang2017sonic} demonstrated a sonic gun, showcasing the vulnerability of various smart devices (\eg drones and self-balancing vehicles) to acoustic attacks. Tu et al. \cite{tu2018injected} designed side-swing and switching attacks to alter the outputs of MEMS gyroscopes and accelerometers. Furthermore, Ji et al. \cite{ji2021poltergeist} fool the object detectors by applying acoustic attack to the image stabilizers commonly used in modern cameras. However, none of the existing works study the relationship between the acoustic injections and SLAM outputs on recent XR devices. 

% \zijian{Do we need one session about security in AR/VR?}
% \yicheng{TODO}
%\jiasi{cite the AIVR paper (UMass Amherst?) paper is we have not already. They add IMU perturbation but w/o SLAM, iirc} \yicheng{Cited}

\textbf{XR Security and Privacy.} 
%Security and privacy concerns in XR systems have gained significant attention. 
For single-user XR systems, researchers have demonstrated various side-channel attacks to extract sensitive information (\eg keystrokes) through video feeds~\cite{ling2019know}, head movements~\cite{nair2023unique, slocum2023going}, architectural hints~\cite{zhang2023its,shang2020arspy}, power usage~\cite{li2024dangers}, and EM side-channel leakages~\cite{al2021vr}. In multi-user XR systems, Su et al.~\cite{su2024remote} use avatar motion data to infer keystrokes in shared VR environments. Slocum et al.~\cite{slocum2024doesn} reveal vulnerabilities in the shared state frameworks of multi-user AR. Similarly, Lebeck et al.~\cite{lebeck2017securing} highlight risks like deceptive virtual objects and emphasize access control for managing shared physical and virtual spaces. Ruth et al.~\cite{ruth2019secure} further propose a secure multi-user AR framework focusing on content sharing and permissions.
Chandio et al.~\cite{chandio2024stealthy} %introduced a multi-modal spatiotemporal attack that 
simultaneously manipulated visual and inertial sensors to disrupt XR pose estimation. However, their study evaluated the attack using offline datasets and assumed the attacker's capability to manipulate IMU data streams through acoustic means, without real experiments. Ours is the first to demonstrate acoustic injection attacks on recent XR devices, like the Hololens 2, in the real world.
 


\iffalse
\begin{table*}[htbp]
\tiny
\begin{center}
\begin{tabular}{lccccccccccccc}\toprule
Model, ft setting & mem & \#param & ARC-c & ARC-e & BoolQ & HS & OBQA & PIQA & rte & SIQA & WG & Avg
%\\\cmidrule(lr){2-3}\cmidrule(lr){4-5} \cmidrule(lr){6-7} \cmidrule(lr){8-9}\cmidrule(lr){10-11} \cmidrule(lr){12-13} \cmidrule(lr){14-15} \cmidrule(lr){16-17} 
\\\cmidrule(lr){1-13}
Llama2(7B), LoRA, $r=64$ & 23.46GB & 159.9M(2.37\%) & \textbf{44.97} & 77.02 & 77.43 & \textbf{57.75} & 32.0 & \textbf{78.45} & 62.09 & \textbf{47.75} & 68.75 & 60.69\\
Llama2(7B), SPruFT, $r=128$ & \textbf{17.62GB} & 145.8M(2.16\%) & 43.60 & \textbf{77.26} & \textbf{77.77} & 57.47 & \textbf{32.6} & 78.07 & \textbf{64.98} & 46.67 & \textbf{69.30} & \textbf{60.86} \\\cmidrule(lr){2-13}
Llama2(7B), FA-LoRA, $r=64$ & 17.25GB & 92.8M(1.38\%) & 43.77 & \textbf{77.57} & 77.74 & \textbf{57.45} & 31.0 & 77.86 & \textbf{66.06} & \textbf{47.13} & 69.06 & 60.85\\
Llama2(7B), FA-SPruFT, $r=128$ & \textbf{15.21GB} & 78.6M(1.17\%) & \textbf{43.94} & 77.22 & \textbf{77.83} & 57.11 & \textbf{32.0} & \textbf{78.18} & 65.70 & 46.47 & \textbf{69.38} & \textbf{60.87}\\\midrule
Llama3(8B), LoRA, $r=64$ & 30.37GB & 167.8M(2.09\%) & \textbf{53.07} & \textbf{81.40} & \textbf{82.32} & \textbf{60.67} & 34.2 & \textbf{79.98} & 69.68 & \textbf{48.52} & \textbf{73.56} & \textbf{64.82}\\
Llama3(8B), SPruFT, $r=128$ & \textbf{24.49GB} & 159.4M(1.98\%) & 52.47 & 81.10 & 81.28 & 60.29 & \textbf{34.6} & 79.76 & \textbf{70.04} & 47.75 & 73.24 & 64.50 \\\cmidrule(lr){2-13}
Llama3(8B), FA-LoRA, $r=64$ & 24.55GB & 113.2M(1.41\%) & \textbf{52.47} & \textbf{81.36} & \textbf{82.23} & 60.17 & \textbf{35.0} & \textbf{79.76} & \textbf{70.04} & \textbf{48.31} & \textbf{73.56} & \textbf{64.77}\\
Llama3(8B), FA-SPruFT, $r=128$ & \textbf{22.41GB} & 92.3M(1.15\%) & 52.22 & 81.19 & 81.35 & \textbf{60.20} & 34.2 & 79.71 & 69.31 & 47.13 & 73.01 & 64.26 \\\bottomrule
\end{tabular}
%\vspace{-0.2cm}
\caption{Fine-tuning Llama on Alpaca dataset for 5 epochs and evaluating on 9 tasks from EleutherAI LM Harness. "mem" represents the memory usage, with further details provided in Appendix~\ref{apdx:measure}. \#param is the number of trainable parameters, where the difference of \#param between the two approaches depends on the architecture of Llama, as some layers have $d_{in} \neq d_{out}$. Note that 10 million trainable parameters only account for less than 0.15GB of memory requirement. FA indicates that we freeze attention layers, but not including MLP layers followed by attention blocks. HS, OBQA, and WG represent HellaSwag, OpenBookQA, and WinoGrande datasets. More details of datasets can be found in Appendix~\ref{apdx:data}. The ablation study for different $r$ and the comparison with other LoRA variants can be found in Appendix~\ref{apdx:ablation}. All reported results are accuracies on the corresponding tasks. \textbf{Bold} indicates the best results of two approaches on the same task.} \label{tab:llm} 
\end{center}
\end{table*}
\fi

\begin{table*}[htbp]
\tiny
\begin{center}
\begin{tabular}{lccccccccccccc}\toprule
Model, ft setting & mem & \#param & ARC-c & ARC-e & BoolQ & HS & OBQA & PIQA & rte & SIQA & WG & Avg
\\\cmidrule(lr){1-13}
Llama2(7B)\\ \cmidrule(lr){1-1} 
LoRA, $r=64$ & 23.46GB & 159.9M(2.37\%) & \textbf{44.97} & 77.02 & 77.43 & 57.75 & 32.0 & \textbf{78.45} & 62.09 & 47.75 & 68.75 & 60.69\\
VeRA, $r=64$ & 22.97GB & 1.374M(0.02\%) & 43.26 & 76.43 & 77.40 & 57.26 & 31.6 & 78.02 & 62.09 & 45.85 & 68.75 & 60.07\\
DoRA, $r=64$ & 44.85GB & 161.3M(2.39\%) & 44.71 & 77.02 & 77.55 & \textbf{57.79} & 32.4 & 78.29 & 61.73 & \textbf{47.90} & 68.98 & 60.71\\
RoSA, $r=32, d=1.2\%$ & 44.69GB & 157.7M(2.34\%) & 43.86 & \textbf{77.48} & \textbf{77.86} & 57.42 & 32.2 & 77.97 & 63.90 &  47.29 & 69.06 & 60.78\\
SPruFT, $r=128$ & \textbf{17.62GB} & 145.8M(2.16\%) & 43.60 & 77.26 & 77.77 & 57.47 & \textbf{32.6} & 78.07 & \textbf{64.98} & 46.67 & \textbf{69.30} & \textbf{60.86} %\\\cmidrule(lr){2-13}
%FA-LoRA, $r=64$ & 17.25GB & 92.8M(1.38\%) & 43.77 & \textbf{77.57} & 77.74 & \textbf{57.45} & 31.0 & 77.86 & 66.06 & \textbf{47.13} & 69.06 & 60.85\\
%FA-DoRA, $r=64$ & 30.61GB & 93.6M(1.39\%) & 43.94 & 77.44 & 77.49 & 57.44 & 31.0 & 77.86 & \textbf{66.43} & 46.98 & 69.14 & 60.86\\
%FA-RoSA, $r=32, d=1.2\%$ & 38.34GB & 98.3M(1.46\%) & \textbf{44.28} & 77.02 & 77.68 & 57.22 & 31.0 & 77.97 & 64.26 & 46.32 & 69.22 & 60.55\\
%FA-SPruFT, $r=128$ & \textbf{15.21GB} & 78.6M(1.17\%) & 43.94 & 77.22 & \textbf{77.83} & 57.11 & \textbf{32.0} & \textbf{78.18} & 65.70 & 46.47 & \textbf{69.38} & \textbf{60.87}
\\\midrule
Llama3(8B)\\ \cmidrule(lr){1-1} 
LoRA, $r=64$ & 30.37GB & 167.8M(2.09\%) & 53.07 & 81.40 & 82.32 & 60.67 & 34.2 & 79.98 & 69.68 & 48.52 & 73.56 & 64.82\\
VeRA, $r=64$ & 29.49GB & 1.391M(0.02\%) & 50.26 & 80.30 & 81.41 & 60.16 & 34.4 & 79.60 & 69.31 & 46.93 & 72.77 & 63.90\\
DoRA, $r=64$ & 51.45GB & 169.1M(2.11\%) & \textbf{53.33} & \textbf{81.57} & \textbf{82.45} & \textbf{60.71} & 34.2 & \textbf{80.09} & 69.31 & \textbf{48.67} & \textbf{73.64} & \textbf{64.88}\\
RoSA, $r=32, d=1.2\%$ & 48.40GB & 167.6M(2.09\%) & 51.28 & 81.27 & 81.80 & 60.18 & 34.4 & 79.87 & 69.31 & 47.95 & 73.16 & 64.36\\
SPruFT, $r=128$ & \textbf{24.49GB} & 159.4M(1.98\%) & 52.47 & 81.10 & 81.28 & 60.29 & \textbf{34.6} & 79.76 & \textbf{70.04} & 47.75 & 73.24 & 64.50 %\\\cmidrule(lr){2-13}
%FA-LoRA, $r=64$ & 24.55GB & 113.2M(1.41\%) & 52.47 & 81.36 & 82.23 & 60.17 & \textbf{35.0} & 79.76 & 70.04 & 48.31 & \textbf{73.56} & 64.77\\
%FA-DoRA, $r=64$ & 40.62GB & 114.3M(1.42\%) & \textbf{52.56} & \textbf{81.69} & \textbf{82.26} & \textbf{60.20} & 34.4 & \textbf{79.82} & \textbf{70.40} & \textbf{48.46} & 73.40 & \textbf{64.80}\\
%FA-RoSA, $r=32, d=1.2\%$ & 42.31GB & 124.3M(1.55\%) & 52.22 & 81.19 & 82.05 & 60.11 & 34.4 & 79.76 & 69.31 & 47.70 & 73.16 & 64.43\\
%FA-SPruFT, $r=128$ & \textbf{22.41GB} & 92.3M(1.15\%) & 52.22 & 81.19 & 81.35 & \textbf{60.20} & 34.2 & 79.71 & 69.31 & 47.13 & 73.01 & 64.26 
\\\bottomrule
\end{tabular}
%\vspace{-0.2cm}
\caption{Fine-tuning Llama on Alpaca dataset for 5 epochs and evaluating on 9 tasks from EleutherAI LM Harness. ``mem" represents the memory usage, with further details provided in Appendix~\ref{apdx:measure}. \#param is the number of trainable parameters, where the difference of \#param between the two approaches depends on the architecture of Llama, as some layers have $d_{in} \neq d_{out}$. %FA indicates that we freeze attention layers, but not including MLP layers followed by attention blocks. 
HS, OBQA, and WG represent HellaSwag, OpenBookQA, and WinoGrande datasets. %More details of datasets can be found in Appendix~\ref{apdx:data}. 
The ablation study for different $r$ can be found in Appendix~\ref{apdx:ranks}. All reported results are accuracies on the corresponding tasks. \textbf{Bold} indicates the best result on the same task. } \label{tab:llm} 
\end{center}
\end{table*}

\section{Experimental Setup}\label{sec:setup}

%(0.5 page)
%Why the chosen framework?
%Some prior approaches

%- parameter settings
%- uniform across layers vs greedy ... 
%- potential transformer-specific details

%Equations about what these methods do.. 

%(0.5 page)
%Which NN architectures are used, why?
%Number of parameters, layers, ...

%(Potential prior work on compression -- )

\subsection{Datasets} \label{subsec:dataset}
We use multiple datasets for different tasks. For image classification, we fine-tune models on the training split and evaluate it on the validation split of Tiny-ImageNet~\citep{tavanaei2020embedded}, CIFAR100~\citep{alex2009learning}, and Caltech101~\citep{li_andreeto_ranzato_perona_2022}. For text generation, we fine-tune LLMs on 256 samples from Stanford-Alpaca~\citep{alpaca} and assess zero-shot performance on nine EleutherAI LM Harness tasks~\citep{gao2021framework}. See Appendix~\ref{apdx:data} for details.

\subsection{Models and Baselines} \label{subsec:models}

We fine-tune full-precision Llama-2-7B and Llama-3-8B (float32) using our SPruFT, LoRA~\citep{hulora}, VeRA~\citep{kopiczko2024vera}, DoRA~\citep{liu2024dora}, and RoSA~\citep{nikdan2024rosa}. RoSA is chosen as the representative SFT method and is the only SFT due to the high memory demands of other SFT approaches, while full fine-tuning is excluded for the same reason. We freeze Llama’s classification layers and fine-tune only the linear layers in attention and MLP blocks.

Next, we evaluate importance metrics by fine-tuning Llamas and image models, including DeiT~\citep{touvron2021training}, ViT~\citep{dosovitskiy2020image}, ResNet101~\citep{he2016deep}, and ResNeXt101~\citep{xie2017aggregated} on CIFAR100, Caltech101, and Tiny-ImageNet. For image tasks, we set the fine-tuning ratio at 5\%, meaning the trainable parameters are a total of 5\% of the backbone plus classification layers.

\subsection{Training Details} \label{subsec:training}
Our fine-tuning framework is built on torch-pruning\footnote{Torch-pruning is not required, all their implementations are based on PyTorch.}~\citep{fang2023depgraph}, PyTorch~\citep{paszke2019pytorch}, PyTorch-Image-Models~\citep{rw2019timm}, and HuggingFace Transformers~\citep{wolf2020transformers}. Most experiments run on a single A100-80GB GPU, while DoRA and RoSA use an H100-96GB GPU. We use the Adam optimizer~\citep{KingBa15} and fine-tune all models for a fixed number of epochs without validation-based model selection.

%Structured pruning often considers parameter dependencies in importance evaluation~\citep{liu2021group, fang2023depgraph, ma2023llmpruner}. This becomes the following process in our work: first, searching for dependencies by tracing the computation graph of gradient; next, evaluating the importance of parameter groups; and finally, fine-tuning the parameters within those important groups collectively. For instance, if $\W^{a}_{\cdot j}$ and $\W^{b}_{i\cdot}$ are dependent, where $\W^{a}_{\cdot j}$ is the $j$-th column in parameter matrix (or the $j$-th input channels/features) of layer $a$ and $\W^{b}_{i\cdot}$ is the $i$-th row in parameter matrix (or the $i$-th output channels/features) of layer $b$, then $\W^{a}_{\cdot j}$ and $\W^{b}_{i\cdot}$ will be fine-tuned simultaneously while the corresponding $\M^{a}_{dep}$ for $\W^{a}_{\cdot j}$ becomes column selection matrix and $\W^a_s$ becomes $\W^a_{f,dep}\M^a_{dep}$. Consequently, fine-tuning $2.5\%$ output channels for layer $b$ will result in fine-tuning additional $2.5\%$ input channels in each dependent layer. Therefore, for the $5\%$ of desired fine-tuning ratio, the fine-tuning ratio with considering dependencies is set to $2.5\%$\footnote{In some complex models, considering dependencies results in slightly more than twice the number of trainable parameters. However, in most cases, the factor is 2.} for the approach that includes dependencies. More details for dependencies of NN can be found in Appendix~\ref{apdx:dep}. 

\textbf{Image models}: The learning rate is set to $10^{-4}$ with cosine annealing decay~\citep{loshchilov2017sgdr}, where the minimum learning rate is $10^{-9}$. All image models used in this study are pre-trained on ImageNet. 

\textbf{Llama}: For LoRA and DoRA, we set $\alpha = 16$, a dropout rate of $0.1$, and a learning rate of $10^{-4}$  with linear decay (
$0.01$ decay rate). For SPruFT, we control trainable parameters using rank instead of fine-tuning ratio for direct comparison. The learning rate is $2 \cdot 10^{-5}$ with the same decay settings. Linear decay is applied after a warmup over the first $3$\% of training steps. The maximum sequence length is $2048$, with truncation for longer inputs and padding for shorter ones.


\section{The general case: Proof of \texorpdfstring{\Cref{thm:main-decomp}}{Theorem 1.6}}\label{sec:algo}

First, we show that data structure of \Cref{l:max_min_query} can be used to compute distances witnessed by shortest paths that pass through a constant-size separator.

\begin{lemma}\label{l:single_adhesion}
Fix a constant $k \in \mathbb{N}$. There exists an algorithm which as the input receives an edge-weighted graph $G$ on $n$ vertices and $m$ edges together with a partition of its vertices into three sets $A, B, C$ such that $|B| \leq k$ and there are no edges between $A$ and $C$, and as the output computes $\max_{c \in C} \dist(a, c)$ for every $a \in A$. The running time is $\Oh(m \log n + n \log^{k - 1} n)$.
\end{lemma}

\begin{proof}
Let $B = \{b_1, \ldots, b_k\}$. For any $a \in A, c \in C$, we have $\dist(a, c) = \min_{i \in [k]} \dist(a, b_i) + \dist(c, b_i)$. First, we run Dijkstra's algorithm from every vertex in $B$ to find $\dist(v, b_i)$ for every $v \in V(G)$ and $i \in [k]$. Next, we use \Cref{l:max_min_query} to construct a data structure $\mathbb{D}$ for the point set $\{(\dist(c, b_1), \dots, \dist(c, b_k))\colon c\in C\}\subseteq \mathbb{R}^k$. Now, the value $\max_{c \in C} \dist(a, c)$ for any given $a$ is equal to the answer of $\mathbb{D}$ to the query with argument $(\dist(a, b_1), \dots, \dist(a, b_k))$.
\end{proof}

After computing the distances over a constant-size separator, we will use the following observation to simplify one of the sides of the separation.

\begin{lemma}\label{l:inserting_paths}
Let $G$ be a edge-weighted connected graph and let $A, B, C$ be a partition of its vertices such that there are no edges between $A$ and $C$. For every pair of vertices $u, v \in B$, let $P_{u, v}$ be any shortest path from $u$ to $v$ with all internal vertices in $C$ (assuming such a path exists).

Let $G'$ denote a graph obtained from $G[A \cup B]$ by adding an edge from $u$ to $v$ of weight equal to the length of $P_{u, v}$, for all $u, v \in B$ for which $P_{u, v}$ exists. Then,  $$\dist_G(s, t) = \dist_{G'}(s, t)\qquad\textrm{for all }s,t\in A\cup B.$$
\end{lemma}
\begin{proof}
Let $G''$ be the graph obtained by adding new edges of $G'$ to $G$.
Fix any $s, t \in A \cup B$ and let $P$ denote the shortest path from $s$ to $t$ in $G''$ which minimizes the number of vertices from $C$ visited. Naturally, the weight of $P$ is equal $\dist_G(s, t)$. Assume that such path visits at least one vertex of $C$. Then, the path $P$ is of the form $s \xrightarrow{P_1} x \xrightarrow{P_2} y \xrightarrow{P_3} t$, where $x, y \in B$ and all the internal vertices of $P_2$ are in $C$. By the construction of $G'$, $P_2$ can be replaced with a direct edge from $x$ to $y$ of the same weight. We obtain a same weight path with a smaller number of vertices of $C$ visited, which is a contradiction. Therefore, $P$ is entirely contained in $A \cup B$, hence it exists in $G'$. This shows that $\dist_G(s, t) = \dist_{G'}(s, t)$.
\end{proof}


The next lemma encapsulates the main algorithmic content of the proof of \Cref{thm:main-decomp}. The algorithm will split the tree decomposition provided on input into smaller parts for which the eccentricities are easier to calculate. We use the following lemma to handle a single such part.
\begin{lemma}\label{l:star}
Fix constants $k, g \in \mathbb{N}, 0 < \delta < \frac{1}{54}$. Assume we are given $n \in \mathbb{N}$, an edge-weighted graph $G$ on at most $n$ vertices with a weight function $w \colon E(G) \to \mathbb{N}$, a vertex subset $A$ and a collection of non-empty vertex subsets $V_0, V_1, \dots, V_\ell$ satisfying the following conditions:
\begin{itemize}[nosep]
	\item The sum of weights of all the edges in $G$ is bounded by $\Oh(n)$.
	\item $V(G) \setminus A = V_0 \cup V_1 \cup \dots \cup V_\ell$.
	\item $|A| \leq k$.
	\item For every $i \in [\ell]$, $G[V_i \setminus V_0]$ is connected, $N_G(V_i \setminus V_0) = V_i \cap V_0$, $|V_i| = \Oh(n^\delta)$, and $|V_0 \cap V_i| \leq 4$.
	\item For all $i, j \in [\ell], i \neq j$, $V_i \setminus V_0$ and $V_j \setminus V_0$ are disjoint and non-adjacent in $G$.
	\item Every edge $uv \in E(G)$ with $u, v \not\in A$ is contained in $G[V_i]$ for some $i\in \{0,1,\ldots,\ell\}$.
	\item The graph obtained by taking $G[V_0]$ and adding a clique on $V_0 \cap V_i$ for every $i \in [\ell]$ has Euler genus bounded by $g$.
\end{itemize}
Then, we can compute the eccentricity of every vertex of $G$ in time $\Oh \left( n^{1 + \frac{150 + 54 \delta}{151}} \log^k n \right)$.
\end{lemma}

\begin{proof}
Fix $\delta' = \frac{1 + 97 \delta}{151}$; we have $\delta' - \delta = \frac{1 - 54\delta}{151} > 0$.
Let $E_i$ denote the set of edges with one endpoint in $V_i$ and the other endpoint in $V_i \setminus V_0$. For $i \in [\ell]$, we shall say that $V_i$ is {\em{heavy}} if the sum of weights of $E_i$ is larger than $n^{\delta'}$. Since the sets $E_i$ are pairwise disjoint and the total sum of weights of all the edges is bounded by $\Oh(n)$, the number of heavy subsets is bounded by $\Oh(n^{1 - \delta'})$. Without loss of generality, we may assume that $V_{\ell' + 1}, \dots, V_\ell$ are heavy and $V_1, \dots, V_{\ell'}$ are not, for some $\ell'\in \{0,\ldots,\ell\}$.


For any source vertex $s$, we can calculate distances from $s$ to every vertex of $G$  using breadth first search in time $\Oh(\sum_{e \in E(G)} w(e)) = \Oh(n)$.
In particular, for every $\ell' < i \leq \ell$, we can compute the distances from every vertex of $V_i$ to every vertex of $G$ in total time $\Oh(n^{2 - \delta' + \delta})$, because $$|V_{\ell'+1}\cup \ldots\cup V_{\ell}|\leq n^{1-\delta'}\cdot \Oh(n^\delta)=\Oh(n^{1-\delta'+
\delta}).$$
Additionally, we calculate distances $\dist_G(a, v)$ for every $a \in A, v \in V(G)$ in time $O(n)$.

For every $i \in [\ell]$ and $u,v \in V_0 \cap V_i$, there exists a shortest path $P_{i,u,v}$ from $u$ to $v$ with all internal vertices belonging to $V_i - V_0$ due to the assumption that $G[V_i - V_0]$ is connected and $N_G(V_i - V_0) = V_i \cap V_0$. Therefore, the distance from $u$ to $v$ is bounded by the sum of weights of edges in $E_i$. In particular, for $i \in [\ell']$, $\dist_G(u, v) \leq n^{\delta'}$.

We define $\widetilde{G}$ to be the graph obtained by taking $G[A \cup V_0 \cup \dots \cup V_{\ell'}]$ and applying the following operation for every $i \in \{\ell' + 1, \dots, \ell\}$:
for each pair of vertices $u, v \in A \cup (V_0 \cap V_i)$, add an edge in $\widetilde{G}$ between $u$ and $v$ with weight equal to the total weight of $P_{i,u,v}$. For a fixed $i, u$, we can find $P_{i, u, v}$ for all $v$ using breadth first search in time $\Oh(n)$. Taking a sum over all $i, u$, we get that $\tilde{G}$ can be computed in total time $\Oh(n^{2 - \delta'})$.


\begin{claim}\label{cl:wG}
The sum of the edge weights in $\widetilde{G}$ is $\Oh(n)$. Moreover, for all $u, v \in V(\widetilde{G})$, we have $\dist_{\widetilde{G}}(u, v) = \dist_{G}(u, v)$.
\end{claim}

\begin{proof}
Consider $i \in \{\ell' + 1, \dots, \ell\}$ and any $u, v \in A \cup (V_0 \cap V_i)$ for which we added an edge. Its weight is bounded by the sum of weights of edges in $E_i$. Therefore, the total weight of all edges added is at most
$$
\sum_{i \in \{\ell' + 1, \dots, \ell\}} \left( |A \cup (V_0 \cap V_i)|^2 \sum_{e \in E_i} w(e) \right) \leq (4 + k)^2 \sum_{e \in E(G)} w(e) = \Oh(n).
$$
This proves the first part of the claim.

For the second part of the claim, consider any $i \in \{\ell' + 1, \dots, \ell \}$ and observe that by our assumptions, $A \cup (V_0 \cap V_i)$ separates $(V_0 \cup \dots \cup V_{\ell'} \cup V_{i + 1} \cup \dots \cup V_\ell) \setminus V_i$ from $V_i \setminus V_0$. Hence it suffices to repeatedly apply \Cref{l:inserting_paths}.
\end{proof}

For every $u \in V(\widetilde{G})$, we have $\ecc_G(u) = \max(\ecc_{\widetilde{G}}(v), \max_{v \in V(G) \setminus V(\widetilde{G})} \dist_G(u, v))$. Note, that we already know all the distances $\dist_G(u, v)$ for $v \in V(G) \setminus V(\widetilde{G})$. Similarly, we can already compute $\ecc_G(u)$ for every $u \in V(G) \setminus V(\widetilde{G})$. Therefore, it remains to compute $\ecc_{\widetilde{G}}(v)$ for each $v \in V(\widetilde{G})$. Our goal is to show that this can be done efficiently using \Cref{l:main_ecc}.

Now, let $G'$ be the graph obtained from $\tilde{G}$ by replacing every edge $e$ non-indicent to $A$ with $w(e)\geq 2$ with a path of length $w(e)$ consisting of unit-weight edges. This operation again preserves the distances. Since the sum of edge weights in $\tilde{G}$ is of $\Oh(n)$, the total number of vertices in $G'$ is of $\Oh(n)$. For $0 \leq i \leq \ell'$, we write $V'_i$ to denote the set $V_i$ together with all the vertices added as a part of a path between two endpoints in $V_i$.
As $V_i$ is not heavy for each $i\in [\ell']$, we have
$$
|V'_i \setminus V'_0| \leq |V_i| + \sum_{e \in E_i} w(e) = \Oh(n^{\delta'})\qquad \textrm{for all }i\in [\ell'].
$$

Let $G_0$ denote the graph $G'[V'_0]$ and let $G_0^*$ denote the graph $G'- A$ with $V'_i - V'_0$ contracted to a single vertex $v_i^*$, for each $i \in [\ell']$; note that, all edges of $G_0$ and $G_0^*$ have unit weight.

\begin{claim}
	The graph $G_0^*$ is does not contain $K_{t}$ as a minor, where $t = \Oh(\sqrt{g})$.
\end{claim}

\begin{proof}
Let $\bar{G}_0$ denote the graph obtained by taking $G_0$ and adding a clique on $V_0 \cap V_i$ for every $i \in [\ell']$.
By lemma assumptions and the fact that subdividing edges does not increase the Euler genus, $\bar{G}_0$ has Euler genus at most $g$. In particular, $\bar{G}_0$ is $K_{t'}$-minor-free for some $t' = \Oh(\sqrt{g})$, because the Euler genus of $K_{t'}$ is $\Omega({t'}^2)$.

Similarly, let $\bar{G}_0^*$ be the graph obtained by taking $G_0^*$ and adding a clique on each $V_0 \cap V_i$.
Note, that $\bar{G}_0^* - \{v_1^*, \dots, v_{\ell'}^*\}$ is precisely $\bar{G}_0$. Let $t = \max(t', 6)$.
Recall that a minor model of a clique $K_t$ consists of $t$ pairwise vertex-disjoint connected subgraphs, called
branch sets, such that there is at least one edge between each pair of the branch sets.
Consider a minor model $\varphi$ of $K_{t}$ in $\bar{G}^*_0$.
Note that $\varphi$ cannot contain any singleton branch set of the form $\{v^*_i\}$, for the degree of $v^*_i$ in $\bar{G}^*_0$ is at most $4 < t - 1$. Furthermore, since $N_{\bar{G}^*_0}(v^*_i) = V_0 \cap V_i$, any branch set containing $v^*_i$ and at least one other vertex contains some $u \in V_0 \cap V_i$, and $N_{\bar{G}^*_0}(v^*_i)\subseteq N_{\bar{G}^*_0}(u)$, hence removing $v^*_i$ from this branch set preserves the model. Therefore, we can assume without loss of generality that all branch sets of $\varphi$ are disjoint from $\{v^*_1, \dots, v^*_{\ell'}\}$, hence $\varphi$ is a minor model of $K_{t}$ in $\bar{G}_0$. This is a contradiction, as $t \geq t'$ and $\bar{G}_0$ is $K_{t'}$-minor-free. Therefore, $\bar{G}_0^*$ is $K_t$-minor-free, hence $G_0^*$ also.
\end{proof}

Let $\rho' = \frac{2 - 108 \delta}{151} > 0$. The graph $G^*_0$ is a unit-weight graph and is $K_{t}$-minor-free.
Hence, by applying \Cref{t:r_division} to $G^*_0$ (with $\varepsilon = \rho'/2$)
we obtain an $n^{\rho'}$-division $\mathcal{R}_0$ in time $\Oh(n^{1 + \rho'})$.
We extend it to $G' - A$ by mapping every contracted vertex $v^*_i$ to $N_{G' - A}[V'_i - V'_0] = (V'_i - V'_0) \cup (V_0 \cap V_i)$. Formally, we put $V''_i \coloneqq N_{G' - A}[V'_i - V'_0]$ and 
$$
\mathcal{R} \coloneqq \left\{ (R_0 \cap V'_0) \cup \bigcup_{i \colon v^*_i \in R_0} V''_i \colon R_0 \in \mathcal{R}_0 \right\}.
$$

Now, we argue that $\mathcal{R}$ is a reasonable division of $G' - A$. Clearly, all sets $R \in \mathcal{R}$ are connected in $G' - A$. Pick any $R \in \mathcal{R}$ and let $R_0$ be its corresponding set in $\mathcal{R}_0$.
Every vertex $v^*_i$ is mapped to a set of size $\Oh(n^{\delta'})$, therefore
$$|R| \leq |R_0| \cdot \Oh(n^{\delta'}) = \Oh(n^{\rho' + \delta'}).$$

By our construction, for every $i \in [\ell']$, $R$ is either disjoint from $V'_i - V'_0$ or contains whole $N_{G' - A}[V'_i - V'_0]$. This means that no vertex belonging to any $V'_i - V'_0$ can be in $\partial R$, hence $\partial R \subseteq V'_0$.

Pick any $u \in \partial R \cap R_0$. Assume that $u \not\in \partial R_0$. Then every vertex of $N_{G_0^*}(u)$ must be in $R_0$, hence $N_{G - A'}(u) \subseteq R$, which is a contradiction. This means that $\partial R \cap R_0 \subseteq \partial R_0$.

Pick any $u \in \partial R - R_0$. Then, $u \in V_0 \cap V_i$ for some $i \in [\ell']$ such that $v_i^* \in R_0$. Moreover, $v_i^* \in \partial R_0$ and is adjacent to $u$ in $G_0^*$. The number of such $u$ is bounded by $4 |\partial R_0 \cap \{ v_1^*, \dots, v_{\ell'}^* \}|$.

Putting two cases together, we obtain:
$$
\sum_{R \in \mathcal{R}} |\partial R| = \sum_{R \in \mathcal{R}} \left(|\partial R \cap R_0| + |\partial R - R_0|\right) \leq \sum_{R_0 \in \mathcal{R}_0} \left(|\partial R_0| + 4 |\partial R_0 \cap \{ v_1^*, \dots, v_{\ell'}^* \}|\right) = \Oh(n^{1 - \frac{1}{2}\rho'}).
$$

It remains to show the following claim.

\begin{claim}
Pick any $R \in \mathcal{R}, s_R \in R$. The number of different distance profiles on $R$ relative to $s_R$ in $G' - A$ is of $\Oh(n^{48\rho' + 54\delta'})$.
\end{claim}
\begin{proof}
We look at every vertex $v \in V(G') \setminus A$ and consider three cases: $v \in R$, $v \in V'_0$, and $v \in V'_i \setminus (V'_0 \cup R)$ for some $i \in [\ell']$. By our construction, $R \cap V'_0$ is non-empty, hence w.l.o.g. we can assume that $s_R \in V'_0$ as whether two vertices have the same profile on $R$ is independent of the choice of the pivot vertex.

In the first case, there are at most $|R| = \Oh(n^{\rho' + \delta'})$ such vertices, hence they realise at most that many profiles.

In the second case, we want to observe that profile of any vertex $v \in V'_0$ on $R$ depends only on its profile on $R \cap V'_0$ (relative to $s_R$). Pick any $t \in R - V'_0$. Then $t \in V'_i - V'_0$ for some $i \in [\ell']$, $V_i \cap V_0 \subseteq R \cap V'_0$, and every path from $v$ to $t$ intersects $V_i \cap V_0$. In particular, distances from $v$ to vertices of $V_i \cap V_0$ determine its distance to $t$, which proves the observation.

Let $\tilde{G}_0$ denote the graph obtained by taking $G'[V'_0]$ and for every $i \in [\ell'], u, v \in V_0 \cap V_i$ adding a disjoint path from $u$ to $v$ of length $\dist(u, v)$. Let $P_i$ denote the vertex set of paths added between $V_0 \cap V_i$. For every $t \in V'_0$ we have $\dist_{G' - A}(v, t) = \dist_{\tilde{G}_0}(v, t)$, so it suffices to bound the number of profiles on $R \cap V'_0$ in $\tilde{G}_0$. By our assumptions, $\tilde{G}_0$ has Euler genus bounded by $g$ and all $P_i$ are of size $\Oh(n^{\delta'})$.

Let $R_0$ be the set of $\mathcal{R}_0$ corresponding to $R$. Let $\tilde{R}_0$ denote the set $(R \cap V'_0) \cup \bigcup_{i : v^*_i \in R_0} P_i$. Such set is connected in $\tilde{G}_0$. Moreover, similarly to $R$, its size is $\Oh(n^{\rho' + \delta'})$. Applying \Cref{thm:distprofiles}, we get that the number of distance profiles on $\tilde{R}_0$ in $\tilde{G}_0$ is $\Oh(n^{12(\rho' + \delta')})$, which also bounds the number of profiles on $R$ in $G' - A$ realised by $V'_0$.

For the third case, assume $v \in V'_i \setminus (V'_0 \cup R)$ for some $i\in [\ell']$. Every path from $v$ to any vertex of $R$ in $G' - A$ intersects $V_i \cap V_0$. Let $v_1, \dots v_p$ be the vertices of $V_i \cap V_0$, where $p \leq 4$. The profile of $v$ on $R$ is then determined by the following:
\begin{itemize}[nosep]
 \item[(a)] the profile of each $v_j$ on $R$,
 \item[(b)] $\dist_{G' - A}(v, v_j) - \dist_{G' - A}(v, v_1)$ for each $2 \leq j \leq p$, and
 \item[(c)] $\dist_{G' - A}(s_R, v_j) - \dist_{G' - A}(s_R, v_1)$ for each $2 \leq j \leq p$ where $s_R$ is some pivot vertex of $R$.
\end{itemize}
By the previous case, the number of distance profiles of each $v_j$ is $\Oh(n^{12(\rho' + \delta')})$. The distances between $v$ and $v_j$ are bounded by $|V'_i|$, hence each quantity described in (b) can take $\Oh(n^{\delta'})$ different possible values. Similarly, since $v_1$ and $v_j$ are connected via $V'_i$, $|\dist_{G' - A}(s_R, v_j) - \dist_{G' - A}(s_R, v_1)| \leq \Oh(n^{\delta'})$. The number of different possible profiles of such $v$ is therefore bounded by $\Oh(n^{48(\rho' + \delta') + 6\delta'}) = \Oh(n^{48\rho' + 54\delta'})$. This finishes the proof of the claim.
\end{proof}

Now we can apply \Cref{l:main_ecc} to graph $G'$ with apex set $A$, $X = V(\widetilde{G})$, and the following constants: $$\rho = \rho' + \delta',\qquad \gamma = 1 - \frac{1}{2}\rho',\quad \textrm{and}\quad \alpha = 48\rho' + 54 \delta'.$$ This allows us to calculate all $V(\widetilde{G})$-eccentricities in $G'$ in time
$$
\Oh \left( \left(
	n^{ 2 - \frac{1}{2} \rho' } +
	n^{ 1 + 48\rho' + 54 \delta' }
\right) \log^k n \right) =
\Oh \left( n^{1 + \frac{150 + 54 \delta}{151}} \log^k n \right).
$$
Since for each $v\in V(\widetilde{G})$ we have $\ecc_{\widetilde{G}}(v) = \max_{u \in V(\widetilde{G})} \dist_{\widetilde{G}}(v, u) = \max_{u \in V(\widetilde{G})} \dist_{G'}(v, u)$, this means that we have successfully computed all the eccentricities in $\widetilde{G}$; and as we argued, this is enough to compute all the eccentricities in $G$ as well.

Finally, the total running time of the algorithm is
$$
\Oh \left( n^{1 + \frac{150 + 54 \delta}{151}} \log^k n + n^{2 - \delta' + \delta} \right) =
\Oh \left( n^{1 + \frac{150 + 54 \delta}{151}} \log^k n \right).
$$\qedhere
\end{proof}


\begin{lemma}\label{l:star2}
Fix constants $k, g \in \mathbb{N}, 0 < \delta < \frac{1}{54}$. Assume we are given $n \in \mathbb{N}$, an edge-weighted graph $G$ on at most $n$ vertices with a weight function $w \colon E(G) \to \mathbb{N}$, a vertex subset $A$ and a collection of non-empty vertex subsets $V_0, V_1, \dots, V_\ell$ satisfying the same conditions as in \Cref{l:star} with the following differences:
\begin{itemize}
	\item we don't require $G[V_i - V_0]$ to be connected and $V_i - V_0$ to be adjacent to whole $V_i \cap V_0$;
	\item instead of $|V_0 \cap V_i| \leq 4$, we require $|V_0 \cap V_i| \leq k$.
\end{itemize}
Then, we can compute the eccentricity of every vertex of $G$ in time $\Oh \left( n^{1 + \frac{150 + 54 \delta}{151}} \log^{k + 5g} n \right)$.
\end{lemma}

\begin{proof}
We will reduce our input to one which will satisfy the conditions of \Cref{l:star}. We start by addressing the adhesions $V_0 \cap V_i$ containing too many vertices.

Let $G_0$ denote the graph $G[V_0]$ with cliques placed at $V_0 \cap V_i$ for every $i \in [\ell]$.
For every $i \in [\ell]$ we repeat the following procedure: while $|V_0 \cap V_i| > 4$,
remove arbitrary $5$ vertices from $V_0 \cap V_i$. Since $|V_0 \cap V_i| \leq k$ for each $i\in [\ell]$,
this procedure can be implemented in total time $\Oh(n)$. As a result, at the end we have $|V_0 \cap V_i| \leq 4$ for all $i \in [\ell]$. Let $M$ be the set of all the removed vertices. By our assumptions, $G_0$ has Euler genus bounded by $g$, hence it cannot contain $g + 1$ pairwise disjoint copies of $K_5$
(as the Euler genus of a graph is the sum of the Euler genera of its 2-connected components~\cite{StahlB77} and $K_5$ is not planar). Each removed quintiple of vertices induces a $K_5$ in $G_0$, hence we have $|M| \leq 5g$. We set $A' = A \cup M$ and may thus assume that $V_i$ is disjoint from $A'$ for all $0 \leq i \leq \ell$.

Now, fix $i \in [\ell]$. Let $C^i_1, \dots, C^i_{r_i}$ denote the connected components of $V_i - V_0$ in $G - A'$. We define $W^i_j := N_{G - A'}[C^i_j]$ for every $j \in [r_i]$. Clearly, all $W^i_j$ induce a connected subgraph of $G$ and satisfy $N_{G - A'}(W^i_j - V_0) = W^i_j \cap V_0$. We put $V'_0 := V_0$ and enumerate
$$
\{V'_1, V'_2, \dots V'_{\ell'}\} := \{ W^i_j \colon i \in [\ell], j \in [r_i] \}.
$$
It is easy to verify that the sets $A'$ and $V'_0, V'_1, \dots, V'_{\ell'}$ satisfy the conditions of \Cref{l:star}. We apply said lemma to calculate the eccentricity of every vertex of $G$ in the desired time.
\end{proof}



The next statement is a reformulation of \Cref{thm:main-decomp}.

\begin{theorem}
Fix constants $k, g \in \mathbb{N}$. Assume we are given a graph $G$ on $n$ vertices together with its tree decomposition $(T, \beta)$ and a set of private apices $A_t \subseteq \beta(t)$ for each node $t\in V(T)$ such that the following conditions hold:
\begin{itemize}[nosep]
 \item For every node $t \in V(T)$, we have $|A_t| \leq k$.
 \item For every edge $st \in E(T)$,  we have $|\beta(v) \cap \beta(u)|\leq k$.
 \item For every node $t \in V(T)$, graph obtained by taking $G[\beta(t)] - A_t$ and turning  $(\beta(t) \cap \beta(s))\setminus A_t$ into a clique for every edge $st \in E(T)$ has Euler genus bounded by $g$.
\end{itemize}
Then, we can compute the eccentricity of every vertex of $G$ in time $\Oh \left( n^{1 + \frac{355}{356}} \log^{k + 5g} n \right)$.
\end{theorem}

\begin{proof}
We may assume that $|V(T)|\leq n$, for every tree decomposition with no two bags comparable by inclusion has this property; and adjacent comparable bags can be merged by contracting the edge between them.

For a node $t\in V(T)$, by the {\em{weight}} of $t$ we mean the size of the corresponding bag, that is, $|\beta(t)|$. For any subset of nodes $S \subseteq V(T)$, we define $\beta(S) \coloneqq \bigcup_{t \in S} \beta(t)$ By the {\em{weight}} of $S$, we mean the total weight of the elements of $S$, that is, $\sum_{t\in S} |\beta(t)|$. 

\begin{claim}\label{cl:weight-T}
The weight of $V(T)$ is of $\Oh(n)$.
\end{claim}

\begin{proof}
The sets $\beta'(t) := \beta(t) - \bigcup_{s \in N_T(t)} \beta(s)$ are pairwise disjoint. We have
$$
\sum_{t \in V(T)} |\beta(t)| =
\sum_{t \in V(T)} |\beta'(t)| + 2 \cdot \sum_{st \in E(T)} |\beta(s) \cap \beta(t)| \leq
|V(T)| + 2k|E(T)| = \Oh(n).
$$
\end{proof}

Since every bag induces a graph of bounded Euler genus, the number of edges contained in a bag is linear in its size. In particular, this implies that the total number of edges of $G$ is also bounded by $\Oh(n)$.

We set $$\delta \coloneqq \frac{1}{356}\qquad\textrm{and}\qquad \Delta \coloneqq \frac{355}{356}.$$ Root the tree $T$ in an arbitrarily chosen node; this naturally imposes an ancestor-descendant relation in $T$ (for convenience, every node is considered its own ancestor and descendant).

We start by partitioning $T$ into connected subtrees using the following procedure.
We proceed bottom-up over $T$, processing nodes in any order so that a node is processed after all its strict descendants have been processed. Along the way, we mark some nodes and split the edges of $T$ into heavy and light. Let $t \in V(T)$ be the currently processed non-root node of $T$ and let $e \in E(T)$ be the edge connecting $t$ with its parent. If the total weight of all the unmarked nodes that are descendants of $t$ is at least $n^\delta$ (recall that this includes $t$ itself as well), then we declare $e$ heavy and mark all the descendants of $t$ that were unmarked so far. Otherwise, the edge $e$ is declared light and the procedure proceeds to further nodes of $T$.

Observe that
removing all heavy edges splits $T$ into connected subtrees, say $T'_1, \cdots T'_m$. All of the subtrees, except for possibly the subtree containing the root node, are of weight at least $n^\delta$. In particular, the number of subtrees $m$, and therefore the number of heavy edges, is  bounded by $\Oh(n^{1 - \delta})$. Moreover, in every subtree $T'_i$, removing the node closest to the root splits $T'_i$ into smaller components, each of weight less than $n^\delta$.

Fix a heavy edge $e$ and let $T^e_1$ and $T^e_2$ be the two subtrees into which $T$ splits after removing~$e$. Let $X^e_i = \beta(T^e_i)$ for $i \in \{1, 2\}$. Put $A_e = X^e_1 \setminus X^e_2$, $C_e = X^e_2 \setminus X^e_1$, and $B_e = X^e_1 \cap X^e_2$. By the properties of tree decompositions, such choice of $A_e, B_e, C_e$ satisfies the conditions of \Cref{l:single_adhesion}, hence in time $\Oh(n \log^{k - 1} n)$ we can compute $\max_{v \in X^e_2} \dist_G(u,v)$ for every $u \in X^e_1$, and $\max_{u \in X^e_1} \dist_G(u,v)$ for every $v \in X^e_2$. Computing this for every heavy edge $e$ takes total time $\Oh(n^{2 - \delta} \log^{k - 1} n)$.

Fix any subtree $T'=T'_j$. Let $e_1 = t^{e_1}_1t^{e_1}_2, e_2 = t^{e_2}_1 t^{e_2}_2, \dots, e_\ell = t^{e_\ell}_1 t^{e_\ell}_2$ denote the heavy edges incident to $T'$, where $t^{e_i}_1 \in V(T')$ and $V(T') \subseteq V(T_1^{e_i})$ for every $i \in [\ell]$.
For a vertex $v \in \beta(T')$, let
$$d_0(v) = \max_{u \in \beta(T')} \dist_G(v, u)\qquad\textrm{and}\qquad d_i(v) = \max_{u \in X_2^{e_i}}\dist_G(v,u),\quad\textrm{for } i \in [\ell].$$ We have $\ecc(v) = \max \{ d_i(v)\colon i\in \{0,1,\ldots,\ell\}\}$.The values of $d_i(v)$ are already calculated for all $i\in [\ell]$, hence it remains to compute $d_0(v)$.

For every $i \in [\ell]$ and every pair of vertices $u, v \in \beta(t^{e_i}_1) \cap \beta(t^{e_i}_2)$ we find a shortest path between $u$ and $v$ with all internal vertices inside $X^{e_i}_2$ (or determine that it doesn't exist). For a fixed $u, v$ this can be done in time $\Oh(n)$. Since in total we perform this step at most $2k^2$ times per heavy edge, it takes $\Oh(n^{2 - \delta})$ time in total. Let $P_{i, u, v}$ denote such path, assuming it exists.

Let $G'$ denote the graph obtained from $G[\beta(T')]$ by taking every $i, u, v$ for which $P_{i, u, v}$ exists and adding an edge between $u$ and $v$ of weight equal to the total weight of $P_{i, u, v}$.
The weight of every edge inserted in $\beta(t^{e_i}_1) \cap \beta(t^{e_i}_2)$ is bounded by $|X^{e_i}_2|+1$. The total weight of all edges inserted is therefore at most
$$
\sum_{i \in [\ell]} |\beta(t^{e_i}_1) \cap \beta(t^{e_i}_2)|^2 \cdot (|X^{e_i}_2|+1) \leq
k^2 \sum_{i \in [\ell]} (|X^{e_i}_2|+1) = \Oh(n),
$$
where the last equality follows from the fact that all the trees $T^{e_i}_2$ are pairwise disjoint.
By \Cref{l:inserting_paths}, we have $\dist_{G'}(u, v) = \dist_G(u, v)$ for each $u, v \in \beta(T')$. Hence, computing $d_0(v)$ for every $v \in \beta(T')$ is equivalent to computing the eccentricity of every vertex in $G'$.

If the size of $\beta(T')$ is smaller than $n^\Delta$, we compute the eccentricities naively in time $\Oh(|\beta(T')|^2)$, 
noting that $G'$ has $\Oh(|\beta(T')|)$ edges (thanks to Claim~\ref{cl:weight-T} and bounded genus assumption 
of the last bullet of the theorem statement). Otherwise, we argue that we can use the algorithm in \Cref{l:star} as follows.

Let $t$ be the node of $T'$ closest to the root. Let $s_1, \dots, s_p$ be the children of $t$ in $T$ and let $T''_i$ denote the connected component of $T' - \{t\}$ containing $s_i$. Set $V_0 = \beta(t)$ and $V_i = \beta(T''_i)$ for $i \in [p]$.

It is now easy to verify that $G'$ and sets $A, \{V_i\colon 0\leq i\leq p\}$ selected this way satisfy the assumptions of \Cref{l:star2}. This allows us to use it to compute the eccentricities in $G'$ in time
$$
\Oh \left( n^{1 + \frac{150 + 54\delta}{151}} \log^{k + 5g} n \right) =
\Oh \left( n^{1 + \frac{354}{356}} \log^{k + 5g} n \right).
$$
As we argued, from these eccentricities, we may easily compute all the eccentricities in $G$.

Now, let us analyse the total running time of the whole algorithm. We invoke \Cref{l:star} $\Oh(n^{1 - \Delta})$ times, since we apply it only to subtrees $T'_i$ of size at least $n^\Delta$. The total running time of those applications is hence
$$
\Oh \left( n^{2 + \frac{354}{356} - \Delta} \log^{k + 5g} n \right) =
\Oh \left( n^{1 + \frac{355}{356}} \log^{k + 5g} n \right).
$$
We compute the eccentricities naively for subtrees smaller than $n^\Delta$, hence the total running time of this computation is
$$
\sum_{i \in [m] \colon |\beta(T'_i)| \leq n^\Delta} |\beta(T'_i)|^2 \leq
n^\Delta \cdot \sum_{i \in m} |\beta(T'_i)| = \Oh(n^{1 + \Delta})=\Oh\left(n^{1+\frac{355}{356}}\right).
$$
The rest of computation can be done in $\Oh(n^{2 - \delta} \log^k n)$. Therefore, the whole algorithm runs in time $\Oh \left( n^{1 + \frac{355}{356}} \log^{k + 5g} n \right)$.
\end{proof}

\section{Theoretical Analysis}

\subsection{Preliminaries}

The Hutchinson trace estimator provides an efficient means to estimate the trace of a matrix, with the following properties:

\begin{lemma}[\cite{Hutchinson89}]
\begin{equation}
    \mathbb{E}[H_m(\mathbf{A})] = \mathrm{Tr}(\mathbf{A}), \; \mathrm{Var}[H_m(\mathbf{A})] \le \frac{2}{m} \mathrm{Tr}^2(\mathbf{A}), \label{eqn:VanHut}
\end{equation}
where $H_m(\mathbf{A})$ is the Hutchinson estimator with $m$ random vectors, $\mathbb{E}[\cdot]$ denotes expectation, and $\mathrm{Var}[\cdot]$ denotes variance.
\end{lemma}

To reduce the variance further, especially for matrices with large dominant eigenvalues, we can leverage a low-rank QR approximation. The Hutch++ estimator improves upon the original Hutchinson method by significantly reducing the variance:

\begin{lemma}[\cite{hutch_pp}] 
Suppose $\mathbf{A}$ is a \emph{symmetric positive semidefinite (PSD)} matrix. Then, the Hutch++ estimator satisfies:
\begin{equation}
    \mathbb{E}[H_m^{++}(\mathbf{A})] = \mathrm{Tr}(\mathbf{A}),\mathrm{Var}[H_m^{++}(\mathbf{A})] \le \frac{18}{m(m-3)} \mathrm{Tr}^2(\mathbf{A}),
\end{equation}
where $H_m^{++}(\mathbf{A})$ is the Hutch++ estimator.
\end{lemma}

It is worth noting that the PSD assumption can be generalized to estimates involving the nuclear norm of non-PSD matrices. However, in this paper, we focus on matrices where large eigenvalues are the primary concern, and thus the PSD assumption always holds.

We define the \emph{relative error} of an estimator $T(\mathbf{A})$ as:
\begin{equation*}
    \varepsilon(T) := \frac{|T(\mathbf{A}) - \mathrm{Tr}(\mathbf{A})|}{|\mathrm{Tr}(\mathbf{A})|}.
\end{equation*}

A key advantage of Hutch++ is its enhanced variance reduction, which leads to significant computational savings compared to the original Hutchinson estimator. Specifically:

\begin{proposition}[\cite{hutch_pp}]
    For a given error threshold $\varepsilon > 0$ and confidence level $\delta > 0$, the following holds with probability at least $1 - \delta$:
    \begin{itemize}
        \item The error of the Hutchinson estimator satisfies $\varepsilon(H_m) < \varepsilon$ when $m = \mathcal{O}\Big(\frac{\log(1/\delta)}{\varepsilon^2}\Big)$.
        \item The error of the Hutch++ estimator satisfies $\varepsilon(H_m^{++}) < \varepsilon$ when $m = \mathcal{O}\Big(\sqrt{\frac{\log(1/\delta)}{\varepsilon^2}} + \log(1/\delta)\Big)$.
    \end{itemize}
\end{proposition}

This result highlights that Hutch++ achieves a quadratic reduction in the required number of samples $m$ for a given accuracy, compared to the original Hutchinson estimator.

\subsection{Error Analysis for Acceleration}
Naïvely applying Hutch++ in generative modeling can incur significant computational costs due to the need for QR decompositions at every iteration. To mitigate this, we propose updating the QR decomposition every $L_s$ iterations. While this approach means the estimated eigenvalues may not always be fully up to date, it strikes a balance between computational efficiency and accuracy.
We now analyze the errors introduced by perturbations in the matrix
$\mathbf{A}$. Let $\widetilde{\mathbf{A}}$ be a perturbed version of
$\mathbf{A}$, by the smoothness of the dynamics, the difference of their traces is proportional to a small
time increment $\eta$ and the frequency of steps in QR updates $L_s$, i.e.,$\mathrm{Tr}(\widetilde{\mathbf{A}}) =\mathrm{Tr}(\mathbf{A}) +
\mathcal{O}((L_s-1)\eta)$. Consider the matrix $Q = \mathrm{QR}(\mathbf{A}S)$,
where $S$ is a random sketching matrix. The \emph{approximate Hutch++ estimator}
for this perturbed matrix is given by:
\begin{equation*}
\scalebox{0.9}{$
    \widetilde{H}_m^{++}(\widetilde{\mathbf{A}}) := \mathrm{Tr}(Q^\top \widetilde{\mathbf{A}} Q) + H_{\frac{m}{3}}\left( (I - QQ^\top) \widetilde{\mathbf{A}} (I - QQ^\top) \right).$}
\end{equation*}

This approximation results from reducing the frequency of QR updates, which is central to our acceleration method. For example, using the frozen QR decomposition (Equation \ref{eqn:acceleration}) over the time subinterval $[t_i, t_{i+L_s}]$, $\frac{\partial z_t}{\partial \mathbf{x}}$ behaves as a perturbation of $\frac{\partial z_t}{\partial \mathbf{x}\left(\lfloor\frac{L}{L_s} t_i\rfloor \cdot \frac{L_s}{L}\right)}$.

We justify the effectiveness of this estimator by proving the following expectation and variance bounds:
\begin{proposition}\label{prop:approximateHut}
\begin{align}
\scalebox{0.9}{$
    \mathbb{E}[\widetilde{H}_m^{++}(\widetilde{\mathbf{A}})] = \mathrm{Tr}(\widetilde{\mathbf{A}})$},\hspace{3.35cm}\tag{14}\label{eqn:approximateExp}\\
\scalebox{0.9}{$
    \mathrm{Var}(\widetilde{H}_m^{++}(\widetilde{\mathbf{A}})) \le \frac{36}{m(m-3)} \mathrm{Tr}^2(\mathbf{A}) +  \frac{1}{m}\mathcal{O}((L_s-1)^2\eta^2).$}\tag{15}\label{eqn:approximateVar}
\end{align}
\end{proposition}
In Equation \ref{eqn:approximateExp}, we observe that the approximate Hutch++ estimator provides the exact trace of $\tilde{\mathbf{A}}$, demonstrating its robustness to the QR decomposition used in the acceleration process. Additionally, in Equation \ref{eqn:approximateVar}, the approximate Hutch++ estimator achieves an approximately quadratic variance reduction similar as the standard Hutch++ estimator.  
Conducting QR decomposition only every \(L_s\) iterations reduces computations while maintaining a variance reduction comparable to the vanilla version. Regarding the specific guidance on the selection of \(L_s\), we refer to interested readers to the complexity analysis section 1.5 in the supplementary file.


\subsection{Error Propagation in Divergence-Based Likelihood Training}\label{sec: error_propagation}

The accumulation of error during the training of divergence-based likelihoods can be analyzed as follows. In the case of Neural ODEs \cite{neural_ode}, by applying the frozen QR decomposition (Equation \ref{eqn:acceleration}) over the time subinterval \( [t_i, t_{i+L_s}] \) for Equation \ref{eq: ode_likelihood}, we estimate the trace term, assuming that the approximate Hutch++ operator is adapted to \( \frac{\partial f}{\partial \mathbf{z}} \) within this interval. Let \( \log \tilde{p}_\theta(t) \) denote the resulting log-likelihood under this approximation. Using Proposition \ref{prop:approximateHut}, we can then establish the following result:

\setcounter{equation}{15}
\begin{proposition}
Let \( M = \max\{|\mathrm{Tr}(\frac{\partial f}{\partial \mathbf{z}})| : t \in [0, T]\} \). Then for \( t \in [0, T] \), we have:
\begin{equation}
\scalebox{0.95}{$
\begin{aligned}
  \mathbb{E}\left[ \log \tilde{p}_\theta(t) \right] &= \log p_\theta(\mathbf{z}(t)),\\
  \mathrm{Var}[\log \tilde{p}_\theta(t)] &\leq T^2 \left[\frac{36M^2}{m(m-3)} + \frac{1}{m}\mathcal{O}((L_s-1)^2\eta^2)\right].
\end{aligned}\label{eqn:loglikelihoodest}$}
\end{equation}
\end{proposition}

Since the divergence-based likelihood of the Schrödinger bridge \citep{forward_backward_SDE, mSB} differs from neural ODEs only by an additional deterministic inner product (see Equation \ref{eq: fb-sde-train-f}) vs. the ISM loss in \cite{VSDM}), the variance reduction achieved by Hutch++ estimators in Equation \ref{eqn:loglikelihoodest} naturally extends to Schrödinger bridge (SB)-based diffusion models \citep{forward_backward_SDE, mSB, VSDM} as well.

\section{Experiments: Planning outperforms Heuristics}
\label{sec:experiment}

We begin our empirical demonstrations by showcasing the effectiveness of our planning framework on both synthetic and real datasets. We focus on the simplest planning algorithm, 1-step lookaheads (Algorithm~\ref{alg:complete}), and show that even basic planning can hold great promise. 
We illustrate our framework using two uncertainty quantification modules---GPs and 
\ensembles/ \ensembleplus. 

Throughout this section, we focus on evaluating the mean squared error of 
a regression model $\model$,  and develop adaptive policies that minimize uncertainty on $g(f)$ defined in~\eqref{eqn:l2-g-f}.
When GPs provide a valid model of uncertainty, 
our experiments show that our planning framework significantly outperforms other baselines. 
We further demonstrate that our conceptual framework extends to deep learning-based uncertainty quantification methods such as  \ensembleplus while highlighting computational challenges that need to be resolved in order to scale our ideas. 
For simplicity, we assume a naive predictor, i.e., $\psi(\cdot) \equiv 0$. However, we emphasize that this problem is just as complex as if we were using a sophisticated model $\psi(.)$. The performance gap between the algorithms 
primarily depends
on the level  of uncertainty in our prior beliefs.

To evaluate the performance of our algorithm, we benchmark it against several baselines. 
%Active learning baselines use an acquisition function $\ac$ to select points that have the highest   function value: $X\opt_t \in \argmax_{X \in \xpoolj{t}} \ac({X})$ at every step $t$. These methods may also need an UQ module, which we simply use the same UQ module as in our algorithm, and it  outputs $V(X)$ that measures the the uncertainty of each point $X \in \xpoolj{t}$.
Our first set of baselines are from active learning~\citep{AggarwalKoGuHaPh14}:
\\ % \noindent\textbf{Active Learning Heuristics:} 
\textbf{(1)} 
\textsf{Uncertainty Sampling (Static):}  In this approach, we query the samples for which the model is least certain about. Specifically, we estimate the variance of the latent output $f(X)$ for each $X \in \xpool$ using the UQ module and select the top-$K$ points with the highest uncertainty. \\
\textbf{(2)} \textsf{Uncertainty Sampling (Sequential):} This is a greedy heuristic that sequentially selects the points with the highest uncertainty within a batch, while updating the posterior beliefs using pseudo labels from the current posterior state. Unlike \textsf{Uncertainty Sampling (Static)}, this method takes into account the information gained from each point within batch, and hence tries to diversify the selected points within a batch. 

 
We also compare our approach to the  \textbf{(3)} \textsf{Random Sampling}, which selects each batch uniformly at random from the pool. Additionally, we compare solving the planning problem using  \textsf{REINFORCE}-based policy gradients with   $\mathsf{Smoothed\text{-}Autodiff}$ policy gradients.\footnote{Our code repository is available at
  \url{https://github.com/namkoong-lab/adaptive-labeling}.}
%Detailed experimental setups are provided in Section \ref{sec:details-experiments}.

%We repeat all experiments with 10 random seeds.




\begin{figure}[t]
\centering
\begin{minipage}[b]{0.49\textwidth}
\centering
\includegraphics[width=\textwidth, height=5cm]{figures/original_scale/Var_of_l_2_loss.pdf}
\caption{(Synthetic data) Variance of mean squared loss evaluated through the posterior belief $\mu_t$ at each horizon $t$. This is the objective that policy gradient methods like \textsf{REINFORCE} and $\ouralgo$ optimizes. 1-step lookaheads are surprisingly effective even in long horizons.}
\label{fig:var-l2-sim}
\end{minipage}
\hfill
\begin{minipage}[b]{0.49\textwidth}
\centering \includegraphics[width=\textwidth, height=5cm]{figures/original_scale/Error_of_estimated_model_l_2_loss.pdf}
\caption{(Synthetic data) Error between MSE calculated based on collected data $\mc{D}^{0:T}$ vs. population oracle MSE over $\mc{D}_{\rm eval} \sim P_X$. Reducing uncertainty over posteriors directly leads to better OOD evaluations. 1-step lookaheads significantly outperform active learning heuristics in small horizons.}
\label{fig:mean-l2-sim}
\end{minipage}
%\caption{Simulated data for GPs}
%\label{fig:both_plots}
\end{figure}

\subsection{Planning with Gaussian processes}
\label{sec:experiment-plan-GP}
We now briefly describe the data generation process for the GP experiments,  deferring a more detailed discussion of the dataset generation to Section~\ref{sec:details-experiments}. 
We use both the synthetic data and the real data to test our methodology.
For the \emph{simulated data},  we construct a setting where the general population is distributed across \emph{51 non-overlapping clusters} while the initial labeled data $\dtrain$ just comes from one cluster. In contrast, both $\dpool \defeq (\xpool,\ypool),\deval \defeq (\xeval,\yeval)$ are generated   from all the clusters. 
We begin with a low-dimensional scenario, generating a one-dimensional regression setting using a GP. %Gaussian Process (GP).
Although the data-generating process is not known to the algorithms,  we assume that the GP hyperparameters are known to all the algorithms
to ensure fair comparisons. This can be viewed as a setting where our prior is well-specified, allowing us to isolate the effects
of different policy optimization approaches
 without any concerns about the misspecified priors. We select $10$ batches, each of size $K=5$ across $T = 10$ time horizons.

To examine the robustness of our method against the distributional assumptions made  in the simulated case, we then move to a real dataset where the correct prior is not known. We simulate selection bias from the eICU dataset~\citep{PollardJoRaCeMaBa18}, which contains real-world patient data with in-hospital mortality outcomes. 
We conduct a $k$-means clustering to generate 51 clusters and then select data from those clusters. We view this to be a credible replication of practice, as severe distribution shifts are common due to selection bias in clinical labels.  To convert the binary mortality labels into a regression setting, we train a  random forest classifier and fit a GP on predicted scores, which serves as the UQ module for all the algorithms. As before, the task is to select 10 batches, each consisting of 5 samples, across 10 time horizons.

 In Figures~\ref{fig:var-l2-sim} and~\ref{fig:mean-l2-sim}, we present results for the simulated data. 
Figure~\ref{fig:var-l2-sim} shows the variance of $\ell_2$ loss, and Figure~\ref{fig:mean-l2-sim} presents the error in the estimated $\ell_2$ loss using $\mu_t$ (relative to true $\ell_2$ loss, that is unknown to the algorithm). 
As we can see from these plots, our method one-step lookahead  gives substantial improvements  over active learning baselines and random sampling. In addition,
compared to the one-step lookahead planning approach using \textsf{REINFORCE}-based policy gradients, 
we observe that $\mathsf{Smoothed\text{-}Autodiff}$-based policy gradients provide significantly more robust performance over all horizons.

In Figures~\ref{fig:var-l2-real}~and~\ref{fig:mean-l2-real}, we observe similar findings on the eICU data. We see that planning policies (\textsf{REINFORCE} and $\mathsf{Smoothed\text{-}Autodiff}$) consistently outperform other heuristics by a large margin.  Active learning baselines perform poorly in these small-horizon batched problems and can sometimes be even worse than the random search baselines.  Overall, our results show the importance of careful planning in adaptive labeling for reliable model evaluation. 

We offer some intuition as to why one-step lookahead planning may outperform other heuristic algorithms. 
 First,  \textsf{Uncertainty sampling (Static)} while myopically selects the
 top-$K$ inputs with the highest uncertainty, it fails to consider 
the overlap in information content among the ``best” instances; see \citep{AggarwalKoGuHaPh14} for more details. 
In other words,  it might acquire points from the same region with high uncertainty while failing to induce diversity among the batch.
Although \textsf{Uncertainty Sampling (Sequential)} somewhat addresses the issue of information overlap, a significant drawback of 
this algorithm
is the disconnect between the objective we aim to optimize and the algorithm. For example, it might sample from a region with high uncertainty but very low density. 

\begin{figure}[t]
\centering
\begin{minipage}[b]{0.48\textwidth}
\centering
\includegraphics[width=\textwidth, height=5cm]{figures/original_scale/Var_of_l_2_loss_real.pdf}
\caption{(Real-world eICU data) Variance of mean squared loss evaluated through the posterior belief $\mu_t$ at each horizon $t$. Even 1-step lookaheads are extremely effective planners, and auto-differentiation-based pathwise policy gradients provide a reliable optimization algorithm based on low-variance gradient estimates.}
\label{fig:var-l2-real}
\end{minipage}
\hfill
\begin{minipage}[b]{0.48\textwidth}
\centering \includegraphics[width=\textwidth, height=5cm]{figures/original_scale/Error_of_estimated_model_l_2_loss_real.pdf}
\caption{(Real-world eICU data) Error between MSE calculated based on collected data $\mc{D}^{0:T}$ vs. population oracle MSE over $\mc{D}_{\rm eval} \sim P_X$. Reducing uncertainty over posteriors directly leads to better OOD evaluations. Our method significantly outperforms active learning-based heuristics, and random sampling.}
\label{fig:mean-l2-real}
\end{minipage}
%\caption{Real data for GPs}
\end{figure}
 
%\vspace{-1.5cm}
% \begin{wrapfigure}{r}{.32\columnwidth}
%   \vspace{-.5cm} 
%   \centering
% \includegraphics[scale=.29]{figures/Var of l2l_2 loss.pdf}
%   \vspace{-0.2cm}
%   \caption{Results of GP}
% \label{fig:var-l2-gp}
%   \vspace{-0.1cm}
% \end{wrapfigure}


% Attempts have been made  in the past to address these  drawbacks heuristically  (see \citep{AggarwalKoGuHaPh14}). We give a unified computational framework while approaching the problem in a more principled manner and solving it more optimally.




\subsection{Planning with  neural network-based uncertainty quantification methods ($\ensembleplus$)}


We now provide a proof-of-concept that shows the generalizability of our conceptual framework  to the deep learning-based UQ modules, specifically focusing on $\ensembleplus$ due to their previously observed superior performance~\citep{OsbandWenAsDwIbLuRo23}. Recall that implementing our framework with deep learning-based UQ modules  requires us to retrain the model across multiple possible random actions $\bm{a}(\theta)$ sampled from the current policy $\pi_\theta$.
This requires significant computational resources, in sharp contrast to the GPs where the posteriors are in closed form and can be readily updated and differentiated. 

Due to the computational constraints, we test $\ensembleplus$ on a toy setting to demonstrate the generalizability of our framework. We consider a setting where the general population consists of four clusters, while the initial labeled data only comes from one cluster. Again we generate data using GPs.  The task is to select a batch of 2 points in one horizon. We detail the $\ensembleplus$ architecture in Section \ref{sec:details-experiments}, and we assume prior uncertainty to be large (depends on the scaling of the prior generating functions). 
The results are summarized in the Table~\ref{tab:UQ_ensemble}.

% \begin{table}[H]
% \vspace{-10pt}
% \caption{Performance under \ensembleplus as UQ module}
%     \centering
%     \begin{tabular}{|m{3cm}|m{2.5cm}|m{2cm}|} 
%     \hline
%       Algorithm   & Variance of $\loss_2$ loss estimate & Error of $\loss_2$ loss estimate  \\ \hline Random Sampling 
%          & $1710.9 \pm 1352.1$ & $8.67\pm6.62$ 
%       \\ \hline \ouralgo & $1.30 \pm 0.68$ & $0.91\pm0.25$ \\ \hline
%     \end{tabular}
%     \label{tab:UQ_ensemble}
%     %\vspace{-10pt}
% \end{table}




\begin{table}[h]
\vspace{-10pt}
\caption{Performance under \ensembleplus as the UQ module}
\centering
\begin{tabular}{|l|l|l|}
\hline
Algorithm   & Variance of $\loss_2$ loss estimate & Error of $\loss_2$ loss estimate  \\
\hline
\textsf{Random sampling} & 7129.8 $\pm$ 1027.0 & 136.2 $\pm$ 8.28 \\ \hline
\textsf{Uncertainty sampling (Static)} & 10852 $\pm$ 0.0 & 162.156 $\pm$ 0.0 \\ \hline
\textsf{Uncertainty sampling (Sequential)} & 8585.5 $\pm$ 898.9 & 144 $\pm$ 6.93 \\ \hline
\textsf{REINFORCE} & 1697.1 $\pm$ 0.0 & 45.27 $\pm$ 0.0 \\ \hline
\ouralgo & 1697.1 $\pm$ 0.0 & 45.27 $\pm$ 0.0 \\ \hline
\end{tabular}
%\caption{Comparison of different algorithms based on variance   and   error in $\ell_2$ loss estimation with Ensemble $+$ as the UQ module. Our results demonstrate that {\ouralgo} and REINFORCE outperformthe other active learning based heuristics, confirming the benefits of our MDP formulation for the adaptive labeling problem, as also demonstrated in Section 4.\\
%\footnotesize{Experimental details: We use Gaussian Processes as our data generating process, GP parameters are the same as in Section D.3.  The task is to select a batch of 2 points along one horizon.The marginal distribution $p_X$ has 4 \textit{non-overlapping} clusters. Initial data comes from one cluster, while pool and evaluation points comes from all the clusters. We have $20$ initial labeled data points, $10$ pool points, and $252$ evaluation points.  Training procedures are similar to the one in Section D.3.} }
\label{tab:UQ_ensemble}
\end{table}



% We faced  issues in scaling up these experiments which will be our focus in the future. 





% \begin{itemize}
%     \item Posteriors should be consistent. Two dimensions: even with less training,  
%     \item the inference should be  fast enough
% \end{itemize}


% Potential research directions for uncertainty quantification

% In this section we consider a simple setting We consider a simpler setting and 


% For synthetic dataset generation, we use ...... For real datasets, we use ...... We compare our methodolgy to several baselines ()    This Section is structured as follows:
% \begin{itemize}
%     \item \textbf{GPs, square loss objective} (Section \ref{}): 
%     %the broad aim of the experiments  in this section is to isolate the performance of our methodology without any concerns for the inefficiencies induced due to a mis-specified prior or imperfect posterior inference. To accomplish this we generate synthetic datasets using GPs (detailed later). We use the well specified prior (GPs - with same hyperparameter setting) as our UQ module.   
%      As GPs provide differentaible posterior inference - any errors induced due to imperfect posterior updates are also isolated. We note that under this setting
%      \item In Section\ref{} we demonstrate why our methodology performs better than other baselines - by devising various synthetic experiments ()
%     \item  \textbf{UQ Benchmarking }(Section \ref{}): Before diving into the experiments using $\ensembleplus$ and ENNs,  we showcase our benchmarking experiments in Section \ref{}. We use real datasets We observe that ENNs perform better
%      \item \textbf{Ensemble $+$}, objective: recall, accuracy
%     \item \textbf{ENN}, objective: recall, accuracy
% \end{itemize}




% In Section {}, we test 
% \subsection{Experimental details}

% \begin{itemize}
%     \item UQ methodologies - GPs, ENNs
%     \item Objectives - Recall,  ATE
%     \item Datasets - ATE-synthetic datasets, Recall-synthetic, real datasets
%     \item Baselines - 
%     \begin{itemize}
%         \item Random sampling
%         \item Active learning - Uncertainty based sampling - In regression setting almost all of the 
%         \item Myopic greedy - Greedy Batch based sampling
%         \item Policy Gradient
%     \end{itemize}
    
% \end{itemize}

% \subsection{Experiments}
%     \begin{itemize}
%     \item GPs with square loss
%     \item Benchmarking ENN
%         \item ENNs with ATE
%         \item ENNs with Recall
%     \end{itemize}

% \subsection{Benefits over other algorithms - intuition and experiments}

%Active learning - Myopic greedy / Don't rely on the objective rather some entropy version.


%%% Local Variables:
%%% mode: latex
%%% TeX-master: "main"
%%% End:


% In the unusual situation where you want a paper to appear in the
% references without citing it in the main text, use \nocite
% \nocite{langley00}

\bibliography{bib}
\bibliographystyle{icml2025}
%\newpage
%\bnote{Say something about: participants in condition limits occurrence of endpoints in communicates, endpoint statements. Either 1) we severly restrict it in the semi-formal development, like in the first veymont tool paper, and we mention in the impl section that it can be done more leniently, or 2) the transformation needs to describe it. As I think it will result in some tricky proof obligations, probably 1) is better for now. Then this assumption needs to be documented.}
 
\bnote{Mention that symmetric version where the receiving side is a range omitted for brevity. Or just generality only supported by implementation. ``See implementation section''}

\bnote{TODO: Document typical elements that do not match the grammar, e.g. $E_l$ and $E_h$, use of $i$ for tid/index in endpoint family, etc.}

%%%%%%%%%%%%%%%%%%%%%%%%%%%%%%%%%%%%%%%%%%%%%%%%%%%%%%%%%%%%%%%%%%%%%%%%%%%%%%%
%%%%%%%%%%%%%%%%%%%%%%%%%%%%%%%%%%%%%%%%%%%%%%%%%%%%%%%%%%%%%%%%%%%%%%%%%%%%%%%
% APPENDIX
%%%%%%%%%%%%%%%%%%%%%%%%%%%%%%%%%%%%%%%%%%%%%%%%%%%%%%%%%%%%%%%%%%%%%%%%%%%%%%%
%%%%%%%%%%%%%%%%%%%%%%%%%%%%%%%%%%%%%%%%%%%%%%%%%%%%%%%%%%%%%%%%%%%%%%%%%%%%%%%
\newpage
\appendix
\onecolumn

\section{Concrete optimization methods}
\label{section:other_methods}

In this section, we provide concrete examples of optimization algorithms using the \algname{ATA} and \algname{GTA} allocation strategies.

For optimization problems, we focus on \algname{SGD} and \algname{Asynchronous SGD}.
Other methods, such as stochastic proximal point methods and higher-order methods, can be developed in a similar fashion.

\subsection{Stochastic Gradient Descent}

For \algname{SGD}, it is important to distinguish homogeneous and heterogeneous cases.
Let us start from the homogeneous case.

\subsubsection{Homogeneous regime}

Consider the problem of finding an approximate stationary point of the optimization problem
\begin{equation}
    \label{eq:homo_problem}
    \min_{\bm{x} \in \R^d} \ \left\{f(\bm{x}) \eqdef \ExpSub{\bm{\xi} \sim {\cal D}}{f(\bm{x};\bm{\xi})}\right\}.
\end{equation}
We assume that each worker is able to compute stochastic gradient $f(\bm{x};\bm{\xi})$ satisfying $\mathbb{E}_{\bm{\xi} \sim {\cal D}}\left[ \|f(\bm{x};\bm{\xi}) - \nabla f(\bm{x})\|^2\right] \leq \sigma^2$ for all $\bm{x}\in \R^d$.

In this case, \algname{SGD} with allocation budget $B$ becomes \algname{Minibatch SGD} with batch size $B$. The next step is determining how the batch is collected. For \algname{ATA}, we refer to this method as \algname{SGD-ATA}, as described in \Cref{alg:sgd-ata}.

\begin{algorithm}[H]
	\caption{\algname{SGD-ATA} (Homogeneous)}
    \label{alg:sgd-ata}
	\begin{algorithmic}[1]
		\STATE \textbf{Optimization inputs}: initial point $\bm{x}_0 \in \R^d$, stepsize $\gamma > 0$
        \STATE \textbf{Allocation inputs}: allocation budget  $B$
        \STATE \textbf{Initialize}: empirical means $\hmu_{i,1} = 0$, usage counts $K_{i,1} = 0$, and usage times $T_{i,1} = 0$, for all $i \in [n]$
		\FOR{$k = 1,\ldots, K$}
        \STATE Compute LCBs $(s_{i,k})$ for all $i \in [n]$
        \STATE Find allocation:
        $
            \bm{a}_k \in  \argmin_{\ba \in \mathcal{A}} \ell(\ba, \bm{s}_k)~.
        $
		\STATE Allocate $a_{i,k}$ tasks to each worker $i \in [n]$
        \STATE Update $\bm{x}$:
        $$
            \bm{x}_{k+1} = \bm{x}_k - \frac{\gamma}{B} \sum_{i=1}^n \sum_{j=1}^{a_{i,k}} \nabla f\(\bm{x}_k;\bm{\xi}_i^j\)
        $$
        \FOR{$i$ such that $a_{i,k} \neq 0$}
        \STATE $K_{i,k+1} = K_{i,k} + a_{i,k}$
        \STATE $T_{i,k+1} = T_{i,k} + \sum_{j=1}^{a_{i,k}} X_{i,k}^{(j)}$
        \STATE $\hmu_{i,k+1} = \frac{T_{i,k+1}}{K_{i,k+1}}$
		\ENDFOR
		\ENDFOR
	\end{algorithmic}
\end{algorithm}

In this case, each task consists in calculating the gradient using the device's local data, which is assumed to have the same distribution as the data on all other devices. Because of this, it does not matter which device performs the task. The method then averages these gradients to obtain an unbiased gradient estimator and performs a gradient descent step.


Now, let us give the version of \algname{Minibatch SGD} using greedy allocation \Cref{alg:sgd-gta}.

\begin{algorithm}[H]
	\caption{\algname{SGD-GTA} (Homogeneous)}
    \label{alg:sgd-gta}
	\begin{algorithmic}[1]
		\STATE \textbf{Input}: initial point $\bm{x}_0 \in \R^d$, stepsize $\gamma > 0$, allocation budget $B$
		\FOR{$k = 1,\ldots, K$}
        \STATE $b=0$
        \STATE Query single gradient from each worker $i \in [n]$ 
        \WHILE{$b<B$}
        \STATE Gradient $\nabla f(\bm{x}_k; \bm{\xi}_{k_b})$ arrives from worker $i_{k_b}$
        \STATE $\bm{g}_k = \bm{g}_k + \nabla f(\bm{x}_k; \bm{\xi}_{k_b})$; $\; b= b+1$
        \STATE Query gradient at $\bm{x}_k$ from worker $i_{k_b}$ \\ 
        \ENDWHILE
        \STATE Update the point: $\bm{x}_{k+1} = \bm{x}_k - \gamma \frac{\bm{g}^k}{B}$
		\ENDFOR
	\end{algorithmic}
\end{algorithm}

\Cref{alg:sgd-gta} is exactly \algname{Rennala SGD} method proposed by \citet{tyurin2024optimal}, which has optimal time complexity when the objective function is non-convex and smooth.

If the computation times are deterministic, then \algname{GTA} makes the same allocation in each iteration. In that case, \algname{SGD-ATA} will converge to this fixed allocation. If the times are random, the allocation found by \algname{GTA} may vary in each iteration. In this case, \algname{SGD-ATA} will approach the best allocation for the expected times.

\subsubsection{Heterogeneous regime}
Now let us consider the following heterogeneous problem
$$
    \min_{x \in \mathbb{R}^d} \ \left\{ f(\bm{x}) := \frac{1}{n} \sum_{i=1}^{n} \ExpSub{\bm{\xi}_i \sim \mathcal{D}_i}{f_i(\bm{x}; \bm{\xi}_i)} \right\}~.
$$
Here each worker $i$ has its own data distribution $\cD_i$.

We start with the greedy allocation.
The algorithm is presented in \Cref{alg:sgd-gta-hetero}.

\begin{algorithm}[H]
	\caption{\algname{SGD-GTA} (Heterogeneous)}
    \label{alg:sgd-gta-hetero}
	\begin{algorithmic}[1]
		\STATE \textbf{Input}: initial point $\bm{x}_0 \in \R^d$, stepsize $\gamma > 0$, parameter $S$
		\FOR{$k = 1,\ldots, K$}
        \STATE $s_i=0$ and $\bm{g}_{i,k} = \bm{0}$ for all $i \in [n]$
        \STATE Query single gradient from each worker $i \in [n]$ 
        \WHILE{$\(\frac{1}{n} \sum_{i=1}^n \frac{1}{s_i}\)^{-1}<\frac{S}{n}$}
        \STATE Gradient $\nabla f_{j}(\bm{x}_{k}; \bm{\xi}_{k})$ arrives from worker $j$
        \STATE $\bm{g}_{j,k} = \bm{g}_{j,k} + \nabla f_{j}(\bm{x}_{k}; \bm{\xi}_{k})$; $\; s_j = s_j+1$
        \STATE Query gradient at $\bm{x}_{k}$ from worker $j$ \\ 
        \ENDWHILE
        \STATE Update the point: $\bm{x}_{k+1} = \bm{x}_k - \gamma \frac{1}{n} \sum_{i=1}^n \frac{1}{s_i} \bm{g}_{i,k}$
		\ENDFOR
	\end{algorithmic}
\end{algorithm}

\Cref{alg:sgd-ata-hetero} presents the \algname{Malenia SGD} algorithm, proposed by \citet{tyurin2024optimal}, which is also optimal for non-convex smooth functions.

In each iteration, \Cref{alg:sgd-gta-hetero} receives at least one gradient from each worker.
Building on this idea, we design a method incorporating \algname{ATA}, given in \Cref{alg:sgd-ata-hetero}.

\begin{algorithm}[H]
	\caption{\algname{SGD-ATA} (Heterogeneous)}
    \label{alg:sgd-ata-hetero}
	\begin{algorithmic}[1]
		\STATE \textbf{Optimization inputs}: initial point $\bm{x}_0 \in \R^d$, stepsize $\gamma > 0$
        \STATE \textbf{Allocation inputs}: allocation budget  $B$
        \STATE \textbf{Initialize}: empirical means $\hmu_{i,1} = 0$, usage counts $K_{i,1} = 0$, and usage times $T_{i,1} = 0$, for all $i \in [n]$
		\FOR{$k = 1,\ldots, K$}
        \STATE Compute LCBs $(s_{i,k})$ for all $i \in [n]$
        \STATE Find allocation:
        $
        \bm{a}_k = \algname{RAS} (\bm{s}_k; B)
        $
		\STATE Allocate $a_{i,k} + 1$ tasks to each worker $i \in [n]$
        \STATE Update $\bm{x}$:
        $$
            \bm{x}_{k+1} = \bm{x}_k - \frac{\gamma}{n} \sum_{i=1}^n \frac{1}{a_{i,k}+1}\sum_{j=1}^{a_{i,k}+1} \nabla f_i\(\bm{x}_k;\bm{\xi}_i^j\)
        $$
        \STATE For all $i\in[n]$, update:
        \begin{align*}
            K_{i,k+1} &= K_{i,k} + a_{i,k} \\
            T_{i,k+1} &= T_{i,k} + \sum_{j=1}^{a_{i,k}} X_{i,k}^{(j)} \\
            \hmu_{i,k+1} &= \frac{T_{i,k+1}}{K_{i,k+1}} \\
        \end{align*}
		\ENDFOR
	\end{algorithmic}
\end{algorithm}


\subsection{Asynchronous SGD}

Here, we focus on the homogeneous problem given in \Cref{eq:homo_problem}. The greedy variant, \algname{Ringmaster ASGD}, was proposed by \citet{maranjyan2025ringmasterasgdasynchronoussgd} and, like \algname{Rennala SGD}, achieves the best runtime.

We now present its version with \algname{ATA}, given in \Cref{alg:asgd-ata}.

\begin{figure*}[h]
    \begin{minipage}[t]{0.48\textwidth}

\begin{algorithm}[H]
	\caption{\algname{ASGD-ATA}}
    \label{alg:asgd-ata}
	\begin{algorithmic}[1]
		\STATE \textbf{Optimization inputs}: initial point $\bm{x}_0 \in \R^d$, stepsize $\gamma > 0$
        \STATE \textbf{Allocation inputs}: allocation budget  $B$
        \STATE \textbf{Initialize}: empirical means $\hmu_{i,1} = 0$, usage counts $K_{i,1} = 0$, and usage times $T_{i,1} = 0$, for all $i \in [n]$
		\FOR{$k = 1,\ldots, K$}
        \STATE Compute LCBs $(s_{i,k})$ for all $i \in [n]$
        \STATE Find allocation:
        $
        \bm{a}_k = \algname{RAS} (\bm{s}_k; B)
        $
        \STATE Update $\bm{x}_k$ using \Cref{alg:asgd} with allocation $\bm{a}_k$
        \STATE For all $i$ such that $a_{i,k} \neq 0$, update:
        \begin{align*}
            K_{i,k+1} &= K_{i,k} + a_{i,k} \\
            T_{i,k+1} &= T_{i,k} + \sum_{j=1}^{a_{i,k}} X_{i,k}^{(j)} \\
            \hmu_{i,k+1} &= \frac{T_{i,k+1}}{K_{i,k+1}} \\
        \end{align*}
        \vspace{-1cm}
		\ENDFOR
	\end{algorithmic}
\end{algorithm}

\end{minipage}
\hfill
\begin{minipage}[t]{0.48\textwidth}

    \begin{algorithm}[H]
        \caption{\algname{ASGD}}
        \label{alg:asgd}
        \begin{algorithmic}[1]
            \STATE \textbf{Input:} Initial point $\bm{x}_0 \in \mathbb{R}^d$, stepsize $\gamma > 0$, allocation vector $\bm{a}$ with $\|\bm{a}\|_1 = B$
            \STATE Workers with $a_i > 0$ start computing stochastic gradients at $\bm{x}_0$
            \FOR{$s = 0, 1, \ldots, B-1$}
                \STATE Receive gradient $\nabla f(\bm{x}_{s+\delta_s}; \bm{\xi}_{s+\delta_s}^{i})$ from worker $i$
                \STATE Update: $\bm{x}_{s+1} = \bm{x}_{s} - \gamma \nabla f(\bm{x}_{s+\delta_s}; \bm{\xi}_{s+\delta_s}^{i})$
                \IF{$a_i > 0$}
                    \STATE Worker $i$ begins computing $\nabla f(\bm{x}_{s+1}; \bm{\xi}_{s+1}^{i})$
                    \STATE Decrease remaining allocation for worker $i$ by one: $a_i = a_i - 1$
                \ENDIF
            \ENDFOR
            \STATE \textbf{return:} $\bm{x}_{B}$
        \end{algorithmic}
        \vspace{0.2cm}
        The sequence $\{\delta_s\}$ represents delays, where $\delta_s \geq 0$ is the difference between the iteration when worker $i$ started computing the gradient and iteration $s$, when it was applied.
    \end{algorithm}

\end{minipage}
\end{figure*}


Here, the task remains gradient computation, but each worker's subsequent tasks use different points for computing the gradient. These points depend on the actual computation times and the asynchronous nature of the method, hence the name \algname{Asynchronous SGD}.


\section{Recursive Allocation Selection Algorithm}
\label{sec:RAS}

In this section, we introduce an efficient method for finding the best allocation.
Given LCBs $\bm{s}_k$ and allocation budget $B$, each iteration of \algname{ATA} (\Cref{alg:ata}) determines the allocation by solving
$$
    \bm{a}_k \in \argmin_{\ba \in \mathcal{A}} \ \ell(\bm{a}, \bm{s}_k),
$$
where 
$$
    \ell(\bm{a},\bm{\mu}) \eqdef \max_{i \in [n]} \  a_{i} \mu_i  = \infnorm{\bm{a}\odot \bm{\mu}},
$$
with $\odot$ denoting the element-wise product.
When clear from context, we write $\ell(\bm{a})$ instead of $\ell(\bm{a}, \bm{\mu})$.

In the early iterations, when some $s_i$ values are $0$, \algname{ATA} allocates uniformly across these arms until all $s_i$ values become positive.
After that, the allocation is determined using the recursive routine in \Cref{alg:RAS}.

\begin{algorithm}[H]
    \caption{Recursive Allocation Selection (\algname{RAS})}
    \label{alg:RAS}
    \begin{algorithmic}[1]
        \STATE \textbf{Input:} Scores $s_1, \dots, s_n$, allocation budget $B$
        \STATE Assume without loss of generality that $s_1 \leq s_2 \leq \dots \leq s_n$ (i.e., sort the scores)
        \IF{$B = 1$}
            \STATE \textbf{return:} $(1, 0, \dots, 0)$
        \ENDIF
        \STATE Find the previous best allocation:
        $$
            \bm{a} = (a_1, \dots, a_n) = \algname{RAS}\(s_1, \dots, s_n; B-1\)
        $$
        \STATE Determine the first zero allocation:
        \begin{equation}
            \label{eq:r}
            r = 
            \begin{cases}
                \min\{i \mid a_i = 0\}, & \text{if }\ a_n = 0 \\
                n, & \text{otherwise}
            \end{cases}
        \end{equation}
        \STATE Find the best next query allocation set: \label{alg_line:min}
        $$
        M = \argmin_{i \in [r]} \  \infnorm{(\bm{a} + \bm{e}_i) \odot \bm{s}},
        $$
        where $\bm{e}_i$ is the unit vector in direction $i$.
        \STATE Select $j \in M$ such that the cardinality of 
        $$
        \argmax_{i \in [r]} \ (a_i + e_{j,i}) s_i
        $$ 
        is minimized
        \STATE \textbf{return:} $\bm{a} + \bm{e}_j$
    \end{algorithmic}
\end{algorithm}

\begin{remark}
    The iteration complexity of \algname{RAS} is $\mathcal{O}(n \ln(\min\{B, n\}) + \min\{B, n\}^2)$. In fact, the first $n \ln(\min\{B, n\})$ term arises from identifying the smallest $B$ scores. For the second term, note that in \eqref{eq:r}, we have $r \leq \min\{B, n\}$.
\end{remark}

\subsection{Optimality}
We now prove that \algname{RAS} finds the optimal allocation, as stated in the following lemma.
\begin{lemma}
    \label{thm:RAS_optimality}
    For positive scores $0<s_1 \le s_2 \le \ldots \le s_n$, \algname{RAS} (\Cref{alg:RAS}) finds an optimal allocation $\bh \in \cA$, satisfying
    $$
    \bh \in \argmin_{\ba \in \cA} \  \infnorm{\ba \odot \bs} ~.
    $$
\end{lemma}
%
\begin{proof}
    We prove the claim by induction on the allocation budget $B$.
    
    \textbf{Base Case ($B = 1$):}  
    When $B = 1$, \algname{RAS} (\Cref{alg:RAS}) allocates the task to worker with the smallest score (line 9).
    Thus, the base case holds.

    \textbf{Inductive Step:}  
    Assume that \algname{RAS} finds an optimal allocation for budget $B - 1$, denoted by
    $$
        \bar{\bh} = \algname{RAS}(s_1, \ldots, s_n; B-1)~.
    $$
    We need to prove that the solution returned for budget $B$, denoted by $\bh = \bar{\bh} + \be_r$, is also optimal.

    Assume, for contradiction, that there exists $\ba \in \cA$ such that $\ba \neq \bh$ and $\ell(\ba) < \ell(\bh)$. 
    Write $\ba = \bar{\ba} + \be_q$ for some $q \in [n]$. Observe that $\|\bar{\ba}\|_1=B-1$ because $\ba \in \mathcal{A}$.

    We consider two cases based on the value of $\ell\(\bar{\bh} + \be_r\)$:

    \begin{itemize}
        \item $\ell\(\bar{\bh} + \be_r\) = h_k s_k$ for some $k \neq r$.  
        In this case, adding one unit to index $r$ does not change the maximum value, i.e., $\ell\(\bar{\bh}\) = \ell\(\bar{\bh} + \be_r\)$. 
        By the inductive hypothesis, $\bar{\bh}$ minimizes $\ell(\bx)$ for budget $B - 1$. 
        Therefore, we have
        $$
        \ell(\ba) \geq \ell\(\bar{\ba}\) \geq \ell\(\bar{\bh}\) = \ell\(\bar{\bh} + \be_r\) =\ell(\bh),
        $$
        which contradicts the assumption that $\ell(\ba) < \ell(\bh)$.

        \item $\ell\(\bar{\bh} + \be_r\) = \(\bar{h}_r + 1\)s_r$.  
        By the algorithm's logic, $\(\bar{h}_r + 1\)s_r \leq \(\bar{h}_i + 1\)s_i$ for all $i \neq r$.
        Since $\ell(\bar{\bh}+\be_r)\leq \ell(\bar{\bh}+\be_q)$ and we assumed $\ell(\bar{\ba}+\be_q)=\ell(\ba)<\ell(\bh)=\ell(\bar{\bh}+\be_r)$, then $\bar{\ba} \neq \bar{\bh}$ otherwise $\ell(\bar{\ba}+\be_q)<\ell(\bar{\ba}+\be_r)$.
        Given that $\|\bar{\bh}\|_1=\|\bar{\ba}\|_1$, this implies that there exists some $u \in [n]$ such that $0\le\bar{a}_u \leq \bar{h}_u - 1$ and another index $v \in [n]$ where $\bar{a}_v \geq \bar{h}_v + 1$.

        In addition, note that $r$ is chosen such that $\ell\(\bar{\bh} + \be_r\)$ is minimum. Using the fact that $\ell\(\bar{\bh} + \be_r\) = \(\bar{h}_r + 1\)s_r$, we have that for any index $q$, we also necessarily have $\ell\(\bar{\bh} + \be_q\) = \(\bar{h}_q + 1\)s_q$.
        Using this, we deduce
        $$
        \ell(\bh)
        =\ell\(\bar{\bh} + \be_r\)
        \leq \ell\(\bar{\bh} + \be_v\)
        = \(\bar{h}_v + 1\)s_v
        \leq \max_i \  \bar{a}_i s_i
        = \ell\(\bar{\ba}\) \leq \ell(\ba),
        $$
        where in the second inequality we used the fact that $\bar{a}_v \geq \bar{h}_v + 1$ and in the last inequality we used the fact that the loss is not decreasing for we add one element to the vector.
        This chain of inequalities again contradicts the assumption that $\ell(\ba) < \ell(\bh)$.
    \end{itemize}

    Since both cases lead to contradictions, we conclude that no $\ba \in \cA$ exists with $\ell(\ba) < \ell(\bh)$. 
    Thus, \algname{RAS} produces an optimal allocation for budget $B$.
\end{proof}

\subsection{Minimal Cardinality}

Among all possible allocations \algname{RAS} choose one that always minimizes the cardinality of the set:
$$
    \argmax_{i \in [n]} \  a_i s_i~.
$$

The reason for this choice is just technical as it allows the \Cref{lem:1} to be true. 

\begin{lemma}
    \label{thm:minimal_cardinality}
    The output of $\algname{RAS}$ ensures the smallest cardinality of the set:
    $$
        \argmax_{i \in [n]} \ a_i s_i
    $$
    among all the optimal allocations $\ba$.
\end{lemma}
%
\begin{proof}
    This proof uses similar reasoning as the one before.

    Let $\bh = \algname{RAS}(\bs;B)$, and denote the cardinality of the set $\argmax_{i \in [n]} \ a_i s_i $ for allocation $\ba$ by
    $$
    C_B(\ba) = \left| \argmax_{i \in [n]} \ a_i s_i  \right| \geq 1~.
    $$  
    We prove the claim by induction on $B$.

    \textbf{Base Case ($B=1$):}  
    For $B=1$, there is a single coordinate allocation, thus $C_1(\bh) = 1$, which is the smallest possible cardinality.

    \textbf{Inductive Step:}  
    Assume that $\algname{RAS}$ finds an optimal allocation for budget $B-1$ with the smallest cardinality, denote its output by
    $$
    \bar{\bh} = \algname{RAS}(s_1, \ldots, s_n; B-1)~.
    $$  
    We need to prove that $\bh = \bar{\bh} + \be_r$ minimizes $C_B(\ba)$ among all optimal allocations for budget $B$.

    Assume, for contradiction, that there exists $\ba \in \cA$ such that $\ba \neq \bh$, $\ell(\ba) = \ell(\bh)$, and $C_B(\ba) < C_B(\bh)$. 
    Write $\ba = \bar{\ba} + \be_q$ for some $q \in [n]$.
    We consider three cases:

    \begin{itemize}
        \item 
        $C_B(\bh) = 1$.  
        %This occurs when $\ell(\bh) > \ell\(\bar{\bh}\)$.
        Since the minimum cardinality is exactly 1, we must have $C_B(\ba) \ge 1 = C_B(\bh)$, that contradicts our assumption.

        \item 
        $C_B(\bh) = C_{B-1}\(\bar{\bh}\)>1$.
        This occurs when $\ell(\bh) = \ell\(\bar{\bh}\) \ne \(\bar{h}_r + 1\)s_r$. 
        By the optimality of $\bh$, we have $\ell\(\bar{\bh}\) \le \ell\(\bar{\ba}\) \le \ell(\ba) = \ell(\bh)=\ell(\bar{\bh})$, which implies $\ell\(\bar{\ba}\) = \ell(\ba)$.
        Therefore, $C_{B-1}\(\bar{\ba}\) \le C_B(\ba)$. 
        Since the induction hypothesis holds for $B-1$, we have $C_{B-1}\(\bar{\bh}\) \le C_{B-1}\(\bar{\ba}\)$. 
        Thus,
        $$
        C_B(\bh) = C_{B-1}\(\bar{\bh}\) \le C_{B-1}\(\bar{\ba}\) \le C_B(\ba),
        $$
        which leads to a contradiction.

        \item 
        $C_B(\bh) = C_{B-1}\(\bar{\bh}\) + 1$.  
        This occurs when $\ell(\bh) = \ell\(\bar{\bh}\) = \(\bar{h}_r + 1\)s_r$. 
        Proceeding as in the previous case, we have $\ell\(\bar{\ba}\) = \ell(\ba)$, and hence $C_{B-1}\(\bar{\ba}\) \le C_B(\ba)$.
        Since the induction hypothesis holds for $B-1$, we know $C_{B-1}\(\bar{\bh}\) \le C_{B-1}\(\bar{\ba}\)$.

        We now have additional cases:
        \begin{itemize}
        \item If $C_{B-1}\(\bar{\ba}\) = C_{B-1}\(\bar{\bh}\) + 1$, then
        $$
        C_B(\bh) = C_{B-1}\(\bar{\bh}\) + 1 = C_{B-1}\(\bar{\ba}\) \le C_B(\ba),
        $$
        which leads to a contradiction.

        \item Now assume $C_{B-1}\(\bar{\ba}\) = C_{B-1}\(\bar{\bh}\)$.
        We will show that in this case, $C_B(\ba) = C_{B-1}\(\bar{\ba}\) + 1$. 
        By contradiction, suppose $C_B(\ba) = C_{B-1}\(\bar{\ba}\)$, which implies $(\bar{a}_q + 1)s_q < \ell(\ba)$. Let $k$ be an index such that $\bar{a}_k s_k = \ell(\ba)$.
        Construct a new allocation $\ba' = \bar{\ba} + \be_q - \be_k$. 
        Then,
        $$
        C_{B-1}\(\ba'\) = C_{B-1}\(\bar{\ba}\) - 1 < C_{B-1}\(\bar{\bh}\),
        $$
        which contradicts the induction hypothesis. 
        Thus, $C_B(\ba) = C_{B-1}\(\bar{\ba}\) + 1$.
        Using this, we have
        $$
        C_B(\bh) = C_{B-1}\(\bar{\bh}\) + 1 = C_{B-1}\(\bar{\ba}\) + 1 = C_B(\ba),
        $$
        which again contradicts $C_B(\ba) < C_B(\bh)$.
        \end{itemize}
    \end{itemize}

    This concludes the proof.
\end{proof}

\section{Proofs of \Cref{thm:main}, \Cref{thm:main2}, and \Cref{cor:main}}
\label{sec:proof_1}

We start by recalling the notation. For $i\in [n]$ and $k \in [K]$, $(X^{(u)}_{i,k})_{u \in [B]}$ denote $B$ independent samples at round $k$ from distribution $\nu_i$. When using an allocation vector $\bm{a}_k \in \mathcal{A}$, the total computation time of worker $i$ at round $k$ is $\sum_{u=1}^{a_{i,k}} X^{(u)}_{i,k}$, when $a_{i,k} >0$. $\bm{\mu} = (\mu_1, \dots, \mu_K)$ is the vector of means. For each $k \in [K]$, when using the allocation vector $\bm{a}_k$, we recall the definition of the proxy loss $\ell: \mathcal{A}\times \mathbb{R}_{\ge0}^n \to \mathbb{R}_{\ge0}$ by
$$
\ell(\bm{a}_k, \bm{\lambda}) = \max_{i\in [n]} \ a_{i,k} \lambda_i,
$$
where $\bm{\lambda} = (\lambda_1, \dots, \lambda_n)$ is a vector of non-negative components. When $\bm{\lambda} = \bm{\mu}$, we drop the dependence on the second input of $\ell$.
For each $\bm{\lambda}$, let $\bar{\bm{a}}_{\bm{\lambda}}\in \mathcal{A}$, be the action minimizing this loss
$$
\bar{\bm{a}}_{\bm{\lambda}} \in \argmin_{\bm{a} \in \mathcal{A}} \ \ell(\bm{a}, \bm{\lambda})~.
$$
We drop the dependency on $\bm{\mu}$ from $\bar{\bm{a}}_{\bm{\mu}}$ to ease notation. The actual (random) computation time at round $k$ is denoted by $C: \mathcal{A} \to \mathbb{R}_+$:
\begin{equation}\label{eq:def_C}
	C(\bm{a}_k) := \max_{i\in [n]} \ \sum_{u=1}^{a_{i,k}} X_{i,k}^{(u)}~.
\end{equation}
Let $\bm{a}^*$ be the action minimizing the expected time
$$
\bm{a}^* \in \argmin_{\bm{a} \in \mathcal{A}} \ \E{C(\bm{a})}~.
$$
The expected regret after $K$ rounds is defined as follows
$$
\mathcal{R}_K := \sum_{t=1}^{K} \E{\ell(\bm{a}_{k})-\ell(\bar{\bm{a}})}~.
$$

\noindent For the remainder of this analysis we consider $\bar{\bm{a}} \in \argmin_{a \in \mathcal{A}} \ \ell(\bm{a})$ found using the \algname{RAS} procedure.
For each $i\in [n]$, recall that $k_i$ is the smallest integer such that
\begin{equation}\label{eq:def_n}
	(\bar{a}_i+k_i)\mu_i > \ell(\bar{\bm{a}})~.
\end{equation}

Below we present a technical lemma used in the proofs of Theorems~\ref{thm:main} and~\ref{thm:main2}.
\begin{lemma}
	\label{lem:1}
	Let $\bm{x}=(x_1, \dots, x_n) \in \mathbb{R}_{\ge 0}^n$. Let $\bm{a}$ be the output of $\algname{RAS}(\bm{x}; B)$. For each $i, j \in [n]$, we have
	$$ 
	a_{ j} x_j \le \left(a_{ i}+1 \right) x_i~.
	$$
\end{lemma}
%
\begin{proof}
	Fix $\bm{x} \in \mathbb{R}_+^n$, and let $\bm{a} = \algname{RAS}(\bm{x};B)$. The result is straightforward when $\min\limits_{i\in [n]}{x_i} = 0$.
	
	\noindent Suppose that $x_i >0$ for all $i \in [n]$.
	Let $s\ge 1$ denote the cardinality
	$$
	s:= \abs{\argmax_{i \in [n]} \  a_{i} x_i }~.
	$$
	Fix $i,j \in [n]$, let $k \in \argmax_{i \in [n]} \  a_{i} x_i $. We need to show that
	$$
	a_{k} x_k \le (a_{i}+1)x_i~.
	$$
	We use a proof by contradiction.
	Suppose that we have $ a_{k} x_k > (a_{i}+1)x_i$ consider the allocation vector $\bm{a}'\in \mathcal{A}$ given by $a'_k = a_{k}-1$, $a'_i = a_i+1$ and $a'_u = \bar{a}_u$ when $u \notin \{i,k\}$. Let $R := \max_{u \neq i,k} \{a_u x_u \}$. We have
	\begin{align*}
		\ell(\bm{a}',\bm{x})
		= \max_{u \in [n]} \ a'_u x_u
		= \max \{(a_i+1) x_i, (a_k-1) x_k, R \}~.
	\end{align*}
	We consider two cases:
	\begin{itemize}
		\item Suppose that $s=1$ (i.e., the only element in $[n]$ such that $a_ux_u=\ell(\bm{a}, \bm{x})$ is $k$), then we have necessarily $R< a_k x_k $. Moreover, by the contradiction hypothesis, $(a_i+1)x_i < a_k x_k$.
		Therefore,
		\begin{align*}
			\ell(\bm{a}', \bm{x}) = \max\{ (a_i+1)x_i, (a_k-1)x_k, R\}
			< 	a_k x_k = \ell(\bm{a}, \bm{x}),
		\end{align*}
		which contradicts the definition of $\bm{a}$.
		\item Suppose that $s\ge 2$, since by hypothesis $ a_k x_k > (a_i+1)x_i$, we clearly have $a_ix_i < \ell(\bm{a}, \bm{x})$ therefore among the set $[n]\setminus \{k,i\}$ there are exactly $s-1$ elements such that $a_u x_u = \ell(\bm{a}, \bm{x})$. In particular, this gives
		\begin{align*}
			\ell(\bm{a}',\bm{x})
			= \max_{u \in [n]} \ \{(a_i+1) x_i, (a_k-1) x_k, R \}
			= R = \ell(\bm{a}, \bm{x})~.
		\end{align*}
		Therefore, $\bm{a}' \in \argmin_{\bm{a} \in \mathcal{A}} \ \ell(\bm{a}, \bm{x})$ and the number of elements such that $a'_i x_i = \ell(\bm{a}', \bm{x})=\ell(\bm{a}, \bm{x})$ is at most $s-1$, which contradicts the fact that $s$ is minimal given the \algname{RAS} choice and Lemma~\ref{thm:minimal_cardinality}.
	\end{itemize}
	As a conclusion we have $a_k x_k\le (a_i+1)x_i$.
\end{proof}
\begin{remark}
	Recall that Lemma~\ref{lem:1} guarantees that $k_i$ defined in \eqref{eq:def_n} satisfy: $k_i \in \{1, 2\}$ for each $i \in [n]$.
\end{remark}


\subsection{Proof of \Cref{thm:main}}
\label{proof:thm:main}

Below we restate the theorem.

\begin{restate-theorem}{\ref{thm:main}}
	Suppose that Assumption~\ref{a:sube} holds. Let $\bar{\bm{a}} \in \argmin_{\bm{a} \in \mathcal{A}} \ell(\bm{a})$, in case of multiple optimal actions, we consider the one output by \algname{RAS} when fed with $\bm{\mu}$.
	Then, the expected regret of \algname{ATA} with inputs $(B, \alpha)$  satisfies
	$$
	\mathcal{R}_K
	\le 2n\max_{i \in [n]} \{B\mu_i -\ell(\bar{\bm{a}})\}+c \cdot\sum_{i=1}^{n} \frac{\alpha^2(\bar{a}_i+k_i)(B \mu_i - \ell(\bar{\bm{a}})) }{\left((\bar{a}_i+k_i)\mu_i - \ell(\bar{\bm{a}})\right)^2}\cdot \ln K,
	$$
	where $\alpha = \max_{i \in [n]} \norm{X_i}_{\psi_1}$, and $c$ is a constant.
\end{restate-theorem}
%
\begin{proof}
	Let $K_{i,k}$ be the number of rounds where arm $i$ was queried prior to round $k$ (we take $K_{i,1}=0$). Recall that we chose the following confidence bound: if $K_{i,k} \ge 1$, then
	$$
	\text{conf}(i,k) = 4e\alpha\sqrt{\frac{ \ln(2k^2)}{K_{i,k}}}+4e\alpha\frac{ \ln(2k^2)}{K_{i,k}},
	$$
	and $\text{conf}(i,k) = \infty$ otherwise. Recall that $\hat{\mu}_{i,k}$ denotes the empirical mean of samples from $\nu_i$ observed prior to $k$ if $K_{i,k}\ge 0$ and $\hat{\mu}_{i,k}=0$ if $K_{i,k}=0$. Let $s_{i,k}$ denote the lower confidence bound used in the algorithm:
	$$
	s_{i,k} = \left(\hat{\mu}_{i,k} -\text{conf}(i,k)\right)_{+}~.
	$$
	
	\noindent We introduce the events $\mathcal{E}_{i,k}$ for $i \in [n]$ and $k \in [K]$ defined by
	$$
	\mathcal{E}_{i,k} := \left\lbrace \abs{\hat{\mu}_{i,k}-\mu_i} > \text{conf}(i,k)\right\rbrace.
	$$
	Let 
	$$
	\mathcal{E}_k = \cup_{i \in [n]} \mathcal{E}_{i,k}.
	$$
	Let us prove that for each $k \in [K]$ and $i \in [n]$: $\mathbb{P}\left(\mathcal{E}_{i,k}\right) \le \frac{1}{k^2}$, which gives using a union bound $\mathbb{P}(\mathcal{E}_k) \le \frac{n}{k^2}$. 
	Let $i \in [n]$, using \Cref{prop:concentration} and taking $\delta = 1/k^2$, we have
	\begin{align*}
		\mathbb{P}(\mathcal{E}_{i,k})
		= \mathbb{P}\{\abs{\hat{\mu}_{i,k}-\mu} > \text{conf}(i,k)\}
		\le \frac{1}{k^2}~.
	\end{align*}

	\noindent We call a ``bad round", a round $k$ where we have $\ell(\bm{a}_{k}) > \ell(\bar{\bm{a}})$. Let us upper bound the number of bad rounds. 
	
	\noindent Observe that in a bad round there is necessarily an arm $i \in [K]$ such that $a_{i,k} \mu_i > \ell(\bar{\bm{a}})$. For each $i\in [n]$, let $N_i(k)$ denote the number of rounds $q\in \{1,\dots, k\}$ where $a_{i,q} \mu_i > \ell(\bar{\bm{a}})$ and $i \in \argmax_{j \in [n]} \ a_{j,q} \mu_j$ (this corresponds to a bad round triggered by worker $q$)
	$$
	N_i(k) := \abs{\left\lbrace q \in \{1, \dots, k\}: a_{i,q}\mu_i > \ell(\bar{\bm{a}}) \text{ and } a_{i,q}\mu_i = \ell(\bm{a}_q) \right\rbrace}~.
	$$
	%This implies in particular that $a_{t,i} \ge a_i^*+n_i$ (using the definition of $n_i$). We conclude that there exists an arm $j\neq i$ such that $a_{t,j} \le a_j^*-n_i$. 
	We show that in the case of $\ell(\bm{a}_k) > \ell(\bar{\bm{a}})$, the following event will hold: there exists $i \in [n]$ such that 
	$$
	E_{i,k} := \mathcal{E}_k \text{ or }\left\lbrace N_i(k-1) \le  \frac{256e^2\alpha^2 (\bar{a}_i+k_i) \ln(2K^2)}{\left((\bar{a}_i+k_i)\mu_i - \ell(\bar{\bm{a}})\right)^2}  \right\rbrace~.
	$$
	To prove this, suppose that for each $i \in [n]$, $\neg E_{i,k}$ holds. This gives in particular
	\begin{equation}\label{eq:ni}
		N_i(k-1) > \frac{256e^2\alpha^2 (\bar{a}_i+k_i) \ln(2K^2)}{\left((\bar{a}_i+k_i)\mu_i - \ell(\bar{\bm{a}})\right)^2}~.
	\end{equation}
	Observe that in each round where $N_i(\cdot)$ is incremented, the number of samples received from the distribution $\nu_i$ increases by at least $\bar{a}_{i}+k_i$. 
	Therefore, we have \eqref{eq:ni} implies
	\begin{align*}
		K_{i,k}
		> \frac{256e^2\alpha^2(\bar{a}_i+k_i)^2 \ln(2K^2)}{\left((\bar{a}_i+k_i)\mu_i - \ell(\bar{\bm{a}})\right)^2}
		= \frac{256e^2\alpha^2 \ln(2K^2)}{\left(\mu_i - \frac{\ell(\bar{\bm{a}})}{\bar{a}_i+k_i}\right)^2}~.
	\end{align*}
	
	
	\noindent Then we have, using the expression of $\text{conf}(\cdot)$
	\begin{align*}
		2\text{conf}(i,k) &=  8e\alpha\sqrt{\frac{ \ln(2k^2)}{K_{i,k}}}+8e\alpha\frac{ \ln(2k^2)}{K_{i,k}}\\
		&\le \left(\mu_i - \frac{\ell(\bar{\ba})}{\bar{a}_i+k_i}\right) \left[ 8e\alpha \sqrt{\frac{\ln(2k^2)}{256e^2\alpha^2 \ln(2K^2)}}+ 8e\alpha \frac{\ln(2k^2)}{256e^2\ln(2K^2)\alpha^2}\left(\mu_i-\frac{\ell(\bar{\ba})}{\bar{a}_i+k_i}\right)\right]\\
		&\le \left(\mu_i - \frac{\ell(\bar{\ba})}{\bar{a}_i+k_i}\right) \left[\frac{1}{2} + \frac{1}{32e\alpha} \left(\mu_i-\frac{\ell(\bar{\ba})}{\bar{a}_i+k_i}\right) \right]~.
	\end{align*}
	Recall that using Lemma~\ref{lem:tech1}, we have $\mu_i-\frac{\ell(\bar{\ba})}{\bar{a}_i+k_i} \le \mu_i \le \alpha$. Therefore, we have
	\begin{equation}\label{eq:conf}
		2\text{conf}(i,k) < \mu_i - \frac{\ell(\bar{\bm{a}})}{\bar{a}_i+k_i}~.
	\end{equation}
	Suppose for a contradiction argument that we have $\neg E_{i,k}$ and $\{ a_{i,k}\mu_i > \ell(\bar{\bm{a}}) \text{ and } a_{i,k} = \ell(\bm{a}_k)\}$.
	Using the definition of $k_i$ and the fact that $a_{i,k} \mu_i > \ell(\bar{\bm{a}})$, we have that $a_{i,k} \ge \bar{a}_i + k_i$. Therefore, \eqref{eq:conf} gives
	\begin{equation}\label{eq:conf2}
		2\text{conf}(i,k) < \mu_i - \frac{\ell(\bar{\bm{a}})}{a_{i,k}}~.
	\end{equation}
	Observe that in each round $\norm{\bm{a}_k}_0 = B$, therefore if we have $a_{i,k} \ge \bar{a}_i+k_i > \bar{a}_i$ for some $i$, we necessarily have that there exists $j \in [n]\setminus \{i\}$ such that $a_{j,k} \le \bar{a}_j-1$. Using the fact that $\ell(\bar{\bm{a}}) \ge \bar{a}_j \mu_j$ with \eqref{eq:conf2}, we get
	\begin{equation}\label{eq:e1}
		a_{i,k}(\mu_i-2\text{conf}(i,k)) > \bar{a}_j \mu_j~.
	\end{equation}
	Since both $\neg \mathcal{E}_{i,k}$ and $\neg \mathcal{E}_{j,k}$ hold (because $\neg E_{i,k}$ implies $\neg \mathcal{E}_k$), we have that
	\begin{align}
		\mu_i - 2\text{conf}(i,k) &\le \hat{\mu}_{i, k} - \text{conf}(i,k)
		\le s_{i,k},\label{eq:mu2}
	\end{align} 
	and $\mu_j \ge \hat{\mu}_{j,k} - \text{conf}(j,k)$.
	Recall that $\mu_j \ge 0$, therefore
	\begin{align}
		\mu_j %&\ge \hat{\mu}_{j,k} - \text{conf}(j,k)\\
		\ge \left(\hat{\mu}_{j,k} - \text{conf}(j,k)\right)_{+}
		= s_{j,k}~.\label{eq:mu3}
	\end{align}
	Using the bounds \eqref{eq:mu2} and \eqref{eq:mu3} in \eqref{eq:e1}, we have
	$$
	a_{i,k} s_{i,k} > \bar{a}_j s_{j,k} \ge (a_{j,k}+1) s_{j,k},
	$$
	where we used the definition of $j$ in the second inequality.
	This contradicts the statement of Lemma~\ref{lem:1}, which concludes the contradiction argument. Therefore, the event that $k$ is a bad round implies that $E_{i,k}$ holds for at least one $i\in [n]$.
	We say that a bad round was triggered by arm $i$, a round where $N_i(\cdot)$ was incremented. 
	Observe that if $k \in [K]$ is not a bad round then $\E{\ell(\bm{a}_k)}-\ell(\bar{\bm{a}})=0$, otherwise if $k$ is a bad round triggered by $i \in [n]$ then $\E{\ell(\bm{a}_{k})}-\ell(\bar{\bm{a}}) \le B\mu_i-\ell(\bar{\bm{a}})$.
	To ease notation we introduce for $i\in [n]$
	$$
	H_i := \frac{256e^2\alpha^2 (\bar{a}_i+k_i) \ln(2K^2)}{\left((\bar{a}_i+k_i)\mu_i - \ell(\bar{\bm{a}})\right)^2}~.
	$$
	The expected regret satisfies
	\begin{align*}
		\mathcal{R}_K &= \sum_{i=1}^{K} \mathbb{E}\left[\ell(\bm{a}_k)-\ell(\bar{\bm{a}})\right]\\
		&\le \sum_{i=1}^{n} (B\mu_i-\ell(\bar{\bm{a}}))\mathbb{E}[N_i(K)]\\ 
		&= \sum_{i=1}^{n}\sum_{k=1}^{K} (B\mu_i-\ell(\bar{\bm{a}}))\mathbb{E}\left[\mathds{1}(k \text{ is a bad round triggered by }i)\right]\\
		&\le \max_{i\in [n]}\{(B\mu_i-\ell(\bar{\bm{a}}))\}\cdot\sum_{t=1}^{K} \mathbb{P}(\mathcal{E}_k)+ \sum_{i=1}^{n}(B\mu_i-\ell(\bar{\bm{a}}))\sum_{k=1}^{K} \mathbb{E}\left[\mathds{1}(k \text{ is a bad round triggered by }i) \mid \neg \mathcal{E}_k\right]\\
		&\le \max_{i\in [n]}\{(B\mu_i-\ell(\bar{\bm{a}}))\}\cdot\sum_{t=1}^{K} \mathbb{P}(\mathcal{E}_k)+ \sum_{i=1}^{n}(B\mu_i-\ell(\bar{\bm{a}}))\sum_{k=1}^{K} \mathbb{E}\left[\mathds{1}(N_i(k)=1+N_i(k-1) \text{ and } N_i \le H_i ) \mid \neg \mathcal{E}_k\right]\\
		&\le \max_{i\in [n]}\{(B\mu_i-\ell(\bar{\bm{a}}))\}\cdot\sum_{k=1}^{K} \mathbb{P}(\mathcal{E}_k)+ \sum_{i=1}^{n} (B\mu_i-\ell(\bar{\bm{a}}))H_i\\
		&\le 2n \max_{i\in [n]}\{(B\mu_i-\ell(\bar{\bm{a}}))\}+  \sum_{i=1}^{n} \frac{256e^2\alpha^2 (\bar{a}_i+k_i)(B\mu_i-\ell(\bar{\bm{a}})) \ln(2K^2)}{\left((\bar{a}_i+k_i)\mu_i - \ell(\bar{\bm{a}})\right)^2}~. \qedhere
	\end{align*}
\end{proof}


\subsection{Proof of \Cref{thm:main2}}
\label{proof:thm:main2}

\begin{restate-theorem}{\ref{thm:main2}}
	Suppose that Assumption~\ref{a:sube} holds. Let $\bar{\bm{a}} \in \argmin_{\bm{a} \in \mathcal{A}} \ell(\bm{a})$, in case of multiple optimal actions, we consider the one output by \algname{RAS} when fed with $\bm{\mu}$.
Then, the expected regret of \algname{ATA-Empirical} with the empirical confidence bounds using the inputs $(B, \eta)$  satisfies
\begin{align*}
	\mathcal{R}_K &\le 2n\max_{i \in [n]} \{B\mu_i -\ell(\bar{\bm{a}})\}
	 +c \cdot\sum_{i=1}^{n} \frac{(1+\eta^2) \alpha_i^2(\bar{a}_i+k_i)(B \mu_i - \ell(\bar{\bm{a}})) }{\left((\bar{a}_i+k_i)\mu_i - \ell(\bar{\bm{a}})\right)^2}\cdot \ln K,
\end{align*}
where $\alpha_i =  \norm{X_i}_{\psi_1}$, and $c$ is a constant.
\end{restate-theorem}
%
\begin{proof}
	We build on the techniques used in the proof of \Cref{thm:main}. Recall the expression of $\xi$:
	$$
	\xi = \frac{1+\sqrt{4\eta^2+5}}{2}~.
	$$
	Define the quantities $C_{i,k}$ by
	$$
	C_{i,k} = 4e \sqrt{\frac{\ln(2k^2)}{K_{i,k}}}+4e \frac{\ln(2k^2)}{K_{i,k}}~.
	$$
	Recall that the lower confidence bounds used here are defined as
	$$
	\hat{s}_{i,k} = \hat{\mu}_{i,k} \left(1-\xi C_{i,k}\right)_{+}~.
	$$
	We additionally define the following quantities
	\begin{equation*}
		\hat{u}_{i,k} := \hat{\mu}_{i,k} \left(1+\frac{4}{3}\xi C_{i,k}\right)~.
	\end{equation*}
	\noindent We introduce the events $\mathcal{E}_{i,k}$ for $i \in [n]$ and $k \in [K]$ defined by
	$$
	\mathcal{E}_{i,k} := \left\lbrace \abs{\mu_i - \hat{\mu}_{i,k}} \le \alpha_i C_{i,k}\right\rbrace~.
	$$
	Let 
	$$
	\mathcal{E}_k = \cup_{i \in [n]} \mathcal{E}_{i,k}~.
	$$
	We have using Proposition~\ref{prop:concentration} for each $k \in [K]$ and $i \in [n]$: $\mathbb{P}\left(\mathcal{E}_{i,k}\right) \le \frac{1}{k^2}$, which gives using a union bound $\mathbb{P}(\mathcal{E}_k) \le \frac{n}{k^2}$. 
	Moreover, following Lemma~\ref{lem:conc2}, for each $i \in [n]$ and $k \in [K]$, we have that $\mathcal{E}_{i,k}$ implies
	\begin{equation}\label{eq:lcb}
		\mu_i \ge \hat{s}_{i,k}~.
	\end{equation}
	
	\noindent Following similar steps as in the proof of Theorem~\ref{thm:main}, we call a ``bad round", a round $k$ where we have $\ell(\bm{a}_{k}) > \ell(\bar{\bm{a}})$. Let us upper bound the number of bad rounds. 
	
	\noindent Observe that in a bad round there is necessarily an arm $i \in [K]$ such that $a_{i,k} \mu_i > \ell(\bar{\bm{a}})$. For each $i\in [n]$, let $N_i(k)$ denote the number of rounds $q\in \{1,\dots, k\}$ where $a_{i,q} \mu_i > \ell(\bar{\bm{a}})$ and $i \in \argmax_{j \in [n]} \{ a_{j,q} \mu_j\}$ (this corresponds to a bad round triggered by worker $q$):
	$$
	N_i(k) := \abs{\left\lbrace q \in \{1, \dots, k\}: a_{i,q}\mu_i > \ell(\bar{\bm{a}}) \text{ and } a_{i,q}\mu_i = \ell(\bm{a}_q) \right\rbrace}~.
	$$
	%This implies in particular that $a_{t,i} \ge a_i^*+n_i$ (using the definition of $n_i$). We conclude that there exists an arm $j\neq i$ such that $a_{t,j} \le a_j^*-n_i$. 
	We show that in the case of $\ell(\bm{a}_k) > \ell(\bar{\bm{a}})$, the following event will hold: there exists $i \in [n]$ such that 
	$$
	E_{i,k} := \mathcal{E}_k \text{ or }\left\lbrace N_i(k-1) \le  \frac{1024e^2 \xi^2\alpha_i^2 (\bar{a}_i+k_i) \ln(2K^2)}{\left((\bar{a}_i+k_i)\mu_i - \ell(\bar{\bm{a}})\right)^2}  \right\rbrace~.
	$$
	To prove this, suppose for a contradiction argument that we have for each $i \in [n]$ $\neg E_{i,k}$. This gives in particular
	\begin{equation}\label{eq:ni2}
		N_i(k-1) >  \frac{1024e^2 \xi^2\alpha_i^2 (\bar{a}_i+k_i) \ln(2K^2)}{\left((\bar{a}_i+k_i)\mu_i - \ell(\bar{\bm{a}})\right)^2}~.
	\end{equation}
	Observe that in each round where $N_i(\cdot)$ is incremented, the number of samples received from the distribution $\nu_i$ increases by at least $\bar{a}_{i}+k_i$. 
	Therefore, we have \eqref{eq:ni2} implies
	\begin{align*}
		K_{i,k}
		>  \frac{1024e^2 \xi^2\alpha_i^2 (\bar{a}_i+k_i)^2 \ln(2K^2)}{\left((\bar{a}_i+k_i)\mu_i - \ell(\bar{\bm{a}})\right)^2}
		=  \frac{1024e^2 \xi^2\alpha_i^2  \ln(2K^2)}{\left(\mu_i - \frac{\ell(\bar{\bm{a}})}{\bar{a}_i+k_i}\right)^2}~.
	\end{align*}
	Therefore, we have
	\begin{align*}
		C_{i,k} &= 4e \sqrt{\frac{\ln(2k^2)}{K_{i,k}}}+4e \frac{\ln(2k^2)}{K_{i,k}}\\
		&\le \left(\mu_i - \frac{\ell(\bar{\ba})}{\bar{a}_i+k_i}\right) \left[4e \sqrt{\frac{\ln(2k^2)}{1024e^2 \xi^2 \alpha_i^2 \ln(2K^2)}}+ \frac{4e \ln(2k^2)\left(\mu_i - \frac{\ell(\bar{\ba})}{\bar{a}_i+k_i}\right)}{1024e^2\xi^2\alpha_i^2 \ln(2K^2)} \right]  \\
		&\le \frac{1}{4\xi \alpha_i} \left(\mu_i - \frac{\ell(\bar{\ba})}{\bar{a}_i+k_i}\right) \left[ \frac{1}{2}+ \frac{\mu_i - \frac{\ell(\bar{\ba})}{\bar{a}_i+k_i}}{256\xi \alpha_i}\right]~.
	\end{align*}
	Using Lemma~\ref{lem:tech1}, we have $\mu_i - \frac{\ell(\bar{\ba})}{\bar{a}_i+k_i} \le \mu_i \le \alpha_i$. Moreover, by definition of $\xi$, we have $\xi \ge 1$. We conclude using the bound above that
	\begin{equation}\label{eq:Ci}
		C_{i,k} \le \frac{3}{20\xi \alpha_i}\left(\mu_i - \frac{\ell(\bar{\ba})}{\bar{a}_i+k_i}\right)~.
	\end{equation}
	\noindent Recall that since $\neg \mathcal{E}_{k}$ holds, in particular $\neg \mathcal{E}_{i,k}$ holds, which gives
	\begin{align*}
		2\hat{\mu}_{i,k} -2\hat{s}_{i,k} &= 2\hat{\mu}_{i,k} \left(1-\left(1-\xi C_{i,k}\right)_{+}\right)\\
		&\le 2\hat{\mu}_{i,k} \left(1-\left(1-\xi C_{i,k}\right)\right)\\
		&= 2\xi C_{i,k}\hat{\mu}_{i,k}\\
		&\le 2\xi C_{i,k} (\mu_i+\alpha_i C_{i,k})\\
		&\le 2\xi \alpha_i C_{i,k} (1+ C_{i,k}),
		%&\le 2\mu_i -2\hat{\mu}_{i,k}\nonumber\\
		%&\le 16e\alpha_i \sqrt{\frac{\ln(2k^2)}{K_{i,k}}}+16e\alpha_i \frac{\ln(2k^2)}{K_{i,k}} \nonumber\\
		%&\le \mu_i - \frac{\ell(\bar{\bm{a}})}{\bar{a}_i+k_i}. \label{eq:Ki2}
	\end{align*}
	where we used the event $\neg \mathcal{E}_{i,k}$ in the penultimate inequality, and $\mu_i \le \alpha_i$ as showed in Lemma~\ref{lem:tech1} in the last inequality.
	Using bound \eqref{eq:Ci} in the previous display gives
	\begin{align}
		2\hat{\mu}_{i,k} -2\hat{s}_{i,k} &\le \frac{3}{10} \left(\mu_i - \frac{\ell(\bar{\ba})}{\bar{a}_i+k_i}\right) (1+C_{i,k})
		\le \frac{3}{8} \left(\mu_i - \frac{\ell(\bar{\ba})}{\bar{a}_i+k_i}\right), \label{eq:conf22}
	\end{align} 
	where we used in the last line the fact that following \eqref{eq:Ci}: $C_i \le \frac{3}{20\xi \alpha_i}(\mu_i - \ell(\bar{\ba})/(\bar{a}_i+k_i)) \le 3/20$, since $\xi \ge 1$ by definition and $\alpha_i \ge \mu_i \ge \mu_i - \ell(\bar{\ba})/(\bar{a}_i+k_i)$ following Lemma~\ref{lem:tech1}.
	
	\noindent Recall that \eqref{eq:Ci} implies in particular that $C_{i,k} \le \frac{3\mu_i}{20\xi \alpha_i} \le 3/(20\xi)$. Since $\neg \mathcal{E}_{i,k}$ is true, we have  $\abs{\hat{\mu}_{i,k} - \mu_i} \le \alpha_i C_{i,k}$. Therefore, using \Cref{lem:conc2}, we have
	\begin{align}
		\mu_i &\le \hat{\mu}_{i,k} \left(1 + \frac{4}{3} \xi C_{i,k}\right)\label{eq:ucb0}\\
		&\le \frac{21}{20}~\hat{\mu}_{i,k}~. \label{eq:ucb}
	\end{align}
	
	Observe that in each round $\norm{\bm{a}_k}_0 = B$, therefore if we have $a_{i,k} \ge \bar{a}_i+k_i > \bar{a}_i$ for some $i$, we necessarily have that there exists $j \in [n]\setminus \{i\}$ such that $a_{j,k} \le \bar{a}_j-1$. Using the fact that $\ell(\bar{\bm{a}}) \ge \bar{a}_j \mu_j$ with \eqref{eq:conf22}, we get
	$$
	5\hat{\mu}_{i,k} -5~\hat{s}_{i,k} < \mu_i - \frac{\ell(\bar{\ba})}{a_{i,k}}~.
	$$
	Therefore, we obtain
	\begin{equation}\label{eq:e12}
		a_{i,k}(\mu_i+ 5\hat{s}_{i,k}-5\hat{\mu}_{i,k}) > \ell(\bar{\ba}) \ge \bar{a}_j \mu_j~.
	\end{equation}
	Since both $\neg \mathcal{E}_{i,k}$ and $\neg \mathcal{E}_{j,k}$ hold (because $\neg E_{i,k}$ implies $\neg \mathcal{E}_k$), we have that
	\begin{align*}
		\mu_i +5\hat{s}_{i,k}-5\hat{\mu}_{i,k} &=  \hat{s}_{i,k} + \mu_i - \hat{\mu}_{i,k} + 4 \left(\hat{s}_{i,k} - \hat{\mu}_{i,k} \right)\\
		&= \hat{s}_{i,k} + \mu_i - \hat{\mu}_{i,k} + 4\hat{\mu}_{i,k} \left((1-\xi C_{i,k})_{+}-1 \right)\\
		&\le \hat{s}_{i,k} + \mu_i - \hat{\mu}_{i,k} - 4\hat{\mu}_{i,k} \xi C_{i,k}\\
		&\le \hat{s}_{i,k} + \mu_i - \hat{u}_{i,k}\\
		&\le \hat{s}_{i,k},
	\end{align*} 
	where we used in the last line the bound \eqref{eq:ucb0}. Since $\neg \mathcal{E}_{j,k}$ holds, we also have
	\begin{equation*}
		\mu_j \ge \hat{s}_{j,k}~.
	\end{equation*}
	Using the two last bounds in \eqref{eq:e12}, we have
	$$
	a_{i,k} \hat{s}_{i,k} > \bar{a}_j \hat{s}_{j,k} \ge (a_{j,k}+1) \hat{s}_{j,k},
	$$
	where we used the definition of $j$, as an arm satisfying $\bar{a}_j \ge 1+a_{j,k}$, in the second inequality.
	This contradicts the statement of Lemma~\ref{lem:1}, which concludes the contradiction argument. Therefore, the event that $k$ is a bad round implies that $E_{i,k}$ holds for at least one $i\in [n]$.
	We say that a bad round was triggered by arm $i$, a round where $N_i(\cdot)$ was incremented. 
	Observe that if $k \in [K]$ is not a bad round then $\E{\ell(\bm{a}_k)}-\ell(\bar{\bm{a}})=0$, otherwise if $k$ is a bad round triggered by $i \in [n]$ then $\E{\ell(\bm{a}_{k})}-\ell(\bar{\bm{a}}) \le B\mu_i-\ell(\bar{\bm{a}})$.
	To ease notation we introduce for $i\in [n]$
	$$
	H_i := \frac{1024e^2 \xi^2\alpha_i^2 (\bar{a}_i+k_i) \ln(2K^2)}{\left((\bar{a}_i+k_i)\mu_i - \ell(\bar{\bm{a}})\right)^2}~.
	$$
	The expected regret satisfies
	\begin{align*}
		\mathcal{R}_K &= \sum_{i=1}^{K} \mathbb{E}\left[\ell(\bm{a}_k)-\ell(\bar{\bm{a}})\right]\\
		&\le \sum_{i=1}^{n} (B\mu_i-\ell(\bar{\bm{a}}))\mathbb{E}[N_i(K)]\\ 
		&= \sum_{i=1}^{n}\sum_{k=1}^{K} (B\mu_i-\ell(\bar{\bm{a}}))\mathbb{E}\left[\mathds{1}(k \text{ is a bad round triggered by }i)\right]\\
		&\le \max_{i\in [n]}\{(B\mu_i-\ell(\bar{\bm{a}}))\}\cdot\sum_{t=1}^{K} \mathbb{P}(\mathcal{E}_k)+ \sum_{i=1}^{n}(B\mu_i-\ell(\bar{\bm{a}}))\sum_{k=1}^{K} \mathbb{E}\left[\mathds{1}(k \text{ is a bad round triggered by }i) \mid \neg \mathcal{E}_k\right]\\
		&\le \max_{i\in [n]}\{(B\mu_i-\ell(\bar{\bm{a}}))\}\cdot\sum_{t=1}^{K} \mathbb{P}(\mathcal{E}_k)+ \sum_{i=1}^{n}(B\mu_i-\ell(\bar{\bm{a}}))\sum_{k=1}^{K} \mathbb{E}\left[\mathds{1}(N_i(k)=1+N_i(k-1) \text{ and } N_i \le H_i ) \mid \neg \mathcal{E}_k\right]\\
		&\le \max_{i\in [n]}\{(B\mu_i-\ell(\bar{\bm{a}}))\}\cdot\sum_{k=1}^{K} \mathbb{P}(\mathcal{E}_k)+ \sum_{i=1}^{n} (B\mu_i-\ell(\bar{\bm{a}}))H_i\\
		&\le 2n \max_{i\in [n]}\{(B\mu_i-\ell(\bar{\bm{a}}))\}+  \sum_{i=1}^{n} \frac{1024e^2 \xi^2\alpha_i^2 (\bar{a}_i+k_i)(B \mu_i - \ell(\bar{\ba})) \ln(2K^2)}{\left((\bar{a}_i+k_i)\mu_i - \ell(\bar{\bm{a}})\right)^2}~. \qedhere
	\end{align*}
\end{proof}













\subsection{Proof of \Cref{cor:main}}
\label{sec:proof_2}

Let us first restate the theorem.
\begin{restate-theorem}{\ref{cor:main}}
	Suppose \Cref{a:sube} holds and let $\eta := \max_{i \in [n]} \frac{\sigma_i}{\mu_i}$.
	Then, the total expected computation time after $K$ rounds, using the allocation prescribed by \algname{ATA} with inputs $(B, \alpha)$ satisfies
	$$
	\mathcal{C}_K \le \left(1+\eta\sqrt{\ln(B)}\right)\mathcal{C}_K^* + \mathcal{O}(\ln K)~.
	$$
\end{restate-theorem}
%
\begin{proof}
Let $\mathbb{E}_k$ be the expectation with respect to the variables observed up to and including $k$ and $\mathcal{F}_k$ the corresponding filtration. Using the tower rule, we have
$$
\sum_{k=1}^{K}\mathbb{E}\left[C(\bm{a}_k)\right] = \mathbb{E}\left[ \sum_{k=1}^{K} \mathbb{E}_{k-1}[C(\bm{a}_k)]\right].
$$
Consider round $k \in [K]$, let us upper bound $\mathbb{E}_{k-1}[C(\bm{a}_t)]$ using $\mathbb{E}_{k-1}[\ell(\bm{a}_k)]$. We have (recall that $\bm{a}_k \in \mathcal{F}_{k-1}$)
\begin{align*}
	\mathbb{E}_{k-1}\left[C(\bm{a}_k) \right] &= \mathbb{E}_{k-1}\left[ \max_{i \in \text{supp}(\bm{a}_k)}\left\lbrace \sum_{u=1}^{a_{i,k}} X^{(u)}_{i,k}  \right\rbrace\right]\\
	&\le \max_{i \in \text{supp}(\bm{a}_k)}\left\lbrace a_{i,k} \mu_i\right\rbrace + \max_{i \in \text{supp}(\bm{a}_k)} \{ a_{i,k} \sigma_i\} \cdot \sqrt{\ln B}\\
	&\le \max_{i \in \text{supp}(\bm{a}_k)}\left\lbrace a_{i,k} \mu_i\right\rbrace + \max_{i \in \text{supp}(\bm{a}_k)} \{ a_{i,k}\, \eta\mu_i\} \cdot \sqrt{\ln B}\\
	&= \left(1+\eta \sqrt{\ln(B)}\right)\max\left\lbrace a_{i,k} \mu_i\right\rbrace.
\end{align*}
Moreover, using Jensen's inequality, we have
\begin{align*}
	\max_{i \in [n]} \{a^*_i \mu_i\}
	\le \mathbb{E}\left[\max_{i \in [n]} \left\lbrace \sum_{u=1}^{a_{k,i}} X^{(u)}_{i,k} \right\rbrace \right]
	= \mathbb{E}[C(\bm{a}^*)]~.
\end{align*}

Using the last two bounds with the result of \Cref{thm:main}, we get the result.

\end{proof}



\section{Technical Results}
\label{sec:technical}

We consider the following concentration inequality for sub-exponential variables by \citet{maurer2021concentration}.

\begin{proposition}[Proposition 7 \citep{maurer2021concentration}
	]\label{prop:concentration}
	Suppose $X_1, \dots, X_n$ are positive i.i.d variables such that $\norm{X_1}_{\psi_1} < \infty$ and $\mu = \mathbb{E}[X_1]$. Let $\delta >0$, with probability at least $1-\delta$
	$$
	\abs{\frac{1}{n}\sum_{i=1}^{n}X_i - \mu} \le 4e\norm{X_1}_{\psi_1} \sqrt{\frac{\ln(2/\delta)}{n}}+4e\norm{X_1}_{\psi_1} \frac{\ln(2/\delta)}{n}~.
	$$ 
\end{proposition}


\begin{lemma}
	Let $X_1, \dots, X_n$ be a sequence of nonnegative  random variables. Such that $\mathbb{E}[X_i]=\mu_i$ and $\text{Var}(X_i) = \sigma_i^2$ for each $i \in [n]$. Then we have
	$$
	\mathbb{E}[\max\{X_1, \dots, X_n\}] \le \max\{\mu_1, \dots, \mu_n\}+\max_{i\in [n]} \{\sigma_i \}\cdot \sqrt{\ln n}~.
	$$  
\end{lemma}


\begin{lemma}\label{lem:tech1}
	Let $X$ be a positive random variable with mean $\mu := \mathbb{E}[X]>0$ and variance $\sigma^2 = \text{Var}(X)$. Then the sub-exponential norm of $X$ satisfies
	$$
	\norm{X}_{\psi_1} \le \frac{1+\sqrt{4\eta^2+5}}{2}\cdot \mu,
	$$
	where $\eta := \frac{\sigma}{\mu}$.
	Moreover, we have
	$$
	\mu \le \norm{X}_{\psi_1}~.
	$$
\end{lemma}
%
\begin{proof}
	Let $\alpha = \norm{X}_{\psi_1}$, $\sigma := \sqrt{\text{Var}(X)}$, $\mu := \mathbb{E}[X]$, and $\eta := \frac{\sigma}{\mu}$. We aim to prove that
	$$
	\alpha \le \frac{1+\sqrt{4\eta^2+5}}{2}\cdot \mu~.
	$$
	For $\epsilon \in (0, \alpha/2)$, we have by definition of $\alpha$
	$$
	\mathbb{E}[\exp(X/(\alpha-\epsilon))] \ge 2~.
	$$
	Recall that we have for any $x\ge 0: \exp(x) \le 1+x+\frac{x^2}{2}e^{x}$, therefore
	$$
	\mathbb{E}\left[\exp(X/(\alpha-\epsilon)) \right] \le 1+ \frac{\mu}{\alpha-\epsilon}+ \frac{\mathbb{E}[X^2]}{2(\alpha-\epsilon)^2} \mathbb{E}[\exp(X/(\alpha-\epsilon))]~.
	$$
	Therefore,
	$$
	1+ \frac{\mu}{\alpha-\epsilon}+ \frac{\mathbb{E}[X^2]}{2(\alpha-\epsilon)^2} \mathbb{E}[\exp(X/(\alpha-\epsilon))] \ge 2~.
	$$
	Taking $\epsilon \to 0$, by continuity we have
	$$
	1+ \frac{\mu}{\alpha}+ \frac{\mathbb{E}[X^2]}{2\alpha^2} \mathbb{E}[\exp(X/\alpha)] \ge 2~.
	$$
	Therefore,
	$$
	1+ \frac{\mu}{\alpha}+ \frac{\mathbb{E}[X^2]}{\alpha^2} \ge 2~.
	$$
	Solving the last inequality gives
	$$
	\alpha \le \frac{\mu + \sqrt{\mu^2+4\mathbb{E}[X^2]}}{2},
	$$
	and using $\mathbb{E}[X^2] = \sigma^2+\mu^2 = (1+\eta^2)\mu^2$, we get
	$$
	\alpha \le \frac{1+\sqrt{4\eta^2+5}}{2}\mu~.
	$$
	The second bound is a direct consequence of Jensen's inequality and the definition of $\norm{X}_{\psi_1}$.
\end{proof}

\begin{lemma}\label{lem:conc2}
	Consider the notation in Lemma~\ref{prop:concentration} and Lemma~\ref{lem:tech1}. Define $C_{\cdot, \cdot}$, $F(\cdot, \cdot)$ and $G(\cdot, \cdot)$ by:
	\begin{align*}
		C_{n,\delta} &:= 4e  \sqrt{\frac{\ln(2/\delta)}{n}}+4e\cdot \frac{\ln(2/\delta)}{n}\\
		F(n, \delta) &:= \hat{X}_n \left(1 - \xi C_{n, \delta}\right)_+\\	
		G(n, \delta) &:=  \hat{X}_n \left(1 + \frac{4}{3}\xi C_{n,\delta}\right)~,
	\end{align*}
	where we use the notation $(a)_+ = \max\{0,a\}$.
	Then, if
	$$
	\abs{\hat{X}_n - \mu} \le \alpha C_{n,\delta},
	$$
	we have
	$$
	\mu \ge F(n, \delta)~.
	$$
	Moreover, if we have additionally $ C_{n, \delta} \le \frac{1}{4\xi}$, then
	$$
	\mu \le G(n,\delta)~.
	$$
\end{lemma}
%
\begin{proof}
	Fix $n, \delta$. 
	Suppose that
	\begin{equation}\label{eq:conc}
		\abs{\hat{X}_n - \mu} \le \alpha C_{n, \delta}~.
	\end{equation}
	\textbf{Proof of $\mu \ge F(n, \delta)$:}
	we have that if $\xi C_n \ge 1$ then $F(n, \delta) = 0$ and the result is straightforward. Suppose that $\xi C_n < 1$, if $\hat{X}_n \le \mu$, we have that $F(n, \delta) = \hat{X}_n (1-\xi C_n) \le \mu$, if $\hat{X}_n \ge \mu$, we have using \eqref{eq:conc} with the bound of \Cref{lem:tech1}
	\begin{align}
		\label{eq:conc2}
		\abs{\hat{X}_n - \mu} &\le \alpha C_{n, \delta}
		\le	\mu\xi\cdot C_{n, \delta}.
	\end{align}
	Therefore, when $\hat{X}_n \ge \mu$, we have
	\begin{align*}
		F(n, \delta) &= \hat{X}_n \left(1-\xi C_{n, \delta}\right)
		= \mu+ \abs{\hat{X}_n - \mu} - \hat{X}_n \xi C_{n, \delta}
		\le  \mu+ \abs{\hat{X}_n - \mu} - \mu \xi C_{n, \delta}
		\le \mu~.
	\end{align*}
	\textbf{Proof of $\mu \le G(n,\delta)$:} Suppose that $C_{n,\delta} \le \frac{1}{4\xi}$. 
	Therefore, \eqref{eq:conc2} gives that $\hat{X}_n \ge \frac{3}{4}\mu$. Using \eqref{eq:conc2} again gives
	\begin{align*}
		\mu &\le \hat{X}_n+ \mu \xi \cdot C_{n,\delta}
		\le \hat{X}_n+  \frac{4}{3}\xi\hat{X}_n\cdot C_{n,\delta}
		= G(n,\delta)~. \qedhere
	\end{align*}
\end{proof}


%\begin{lemma}\label{lem!KL}
%	Let $X$ and $Y$ be two sub-exponential distributions with parameters $x$ and $y=x+\epsilon$ respectively, for $\epsilon \ge 0$. Then we have:
%	$$
%	\text{KL}\left(Y; X\right) \le \frac{\epsilon^2}{2x^2},
%	$$
%	where $\text{KL(.,.)}$ denotes the Kullback-Leibler divergence between $Y$ and $X$.
%\end{lemma}
%\begin{proof}
%	Using the definition of KL divergence between $X$ and $Y$ given the expression of the exponential distribution densities, we have:
%	\begin{align*}
%		\text{KL}\left(Y; X\right) &\le \ln\frac{x}{x+\epsilon}+\frac{x+\epsilon}{x}-1\\
%		&\le \frac{\epsilon^2}{2x^2}.
%	\end{align*}
%	
%\end{proof}


%%%%%%%%%%%%%%%%%%%%%%%%%%%%%%%%%%%%%%%%%%%%%%%%%%%%%%%%%%%%%%%%%%%%%%%%%%%%%%%
%%%%%%%%%%%%%%%%%%%%%%%%%%%%%%%%%%%%%%%%%%%%%%%%%%%%%%%%%%%%%%%%%%%%%%%%%%%%%%%

\end{document}


% This document was modified from the file originally made available by
% Pat Langley and Andrea Danyluk for ICML-2K. This version was created
% by Iain Murray in 2018, and modified by Alexandre Bouchard in
% 2019 and 2021 and by Csaba Szepesvari, Gang Niu and Sivan Sabato in 2022.
% Modified again in 2023 and 2024 by Sivan Sabato and Jonathan Scarlett.
% Previous contributors include Dan Roy, Lise Getoor and Tobias
% Scheffer, which was slightly modified from the 2010 version by
% Thorsten Joachims & Johannes Fuernkranz, slightly modified from the
% 2009 version by Kiri Wagstaff and Sam Roweis's 2008 version, which is
% slightly modified from Prasad Tadepalli's 2007 version which is a
% lightly changed version of the previous year's version by Andrew
% Moore, which was in turn edited from those of Kristian Kersting and
% Codrina Lauth. Alex Smola contributed to the algorithmic style files.
