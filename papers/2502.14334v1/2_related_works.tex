\section{Related Work}
\label{sec: related_work}

The problem of Purest Quantum State Identification (PQSI) can be viewed as learning the properties of a set of quantum states.

\paragraph{Quantum learning and testing.}  Quantum learning and testing \cite{montanaro2013survey, aharonov2022quantum} is a vital area of research in quantum computing and quantum communication. There are extensive investigations conducted to understand the complexities of various measurements. 

Quantum state tomography \cite{banaszek2013focus, gross2010quantum, chen2023does, chen2022exponential} involves obtaining complete information about the density matrix of a quantum state through measurements. While this technique can be employed to tackle the PQSI problem, it incurs significant sampling costs. For the quantum state certification \cite{bubeck2020entanglement,chen2022tight,wright2016learn,chen2022toward}, the target is to determine whether a quantum state is close to a specific target quantum state. Our problem can be viewed as identifying the quantum state that is the farthest from the maximally mixed state. However, this problem is focused on quantum testing and does not deal with distance estimation. Therefore, these methods cannot be applied to the PQSI problem. Another category of problems relates to inner product estimation between two quantum states \cite{anshu2022distributed,hinsche2024efficient, zhu2022cross,huang2020predicting}. When proving lower bounds, this category often significantly restricts the quantum states for distinction. 

The relevant literature employs two general approaches to establish problem complexity. The first approach involves constructing counterexamples using Haar unitary matrices \cite{anshu2022distributed,chen2022exponential,chen2022toward, bubeck2020entanglement}. However, the representation-theoretic structure of Haar unitary matrices is intricate \cite{mele2024introduction}, which makes it difficult to use. The other approach uses Gaussian Orthogonal Ensemble (GOE) matrices to create counterexamples \cite{chen2022tight,chen2023does}. However, when using GOE to prove the lower bound, the distance between quantum states is in a specific range rather than a fixed number, making it unsuitable for the PQSI problem. 

% To address the problem of PQSI, we designed and designed a new scheme to compare quantum states of varying purity to.

\paragraph{Classical Best arm identification.} To the best of our knowledge, our work is the first to consider the best quantum state identification. Among the classical learning tasks, the best arm identification \cite{audibert2010best, garivier2016optimal, russo2016simple, 6814096, gabillon2012best} has been extensively studied, and is divided into two categories: fixed budget \cite{bechhofer1968single} and fixed confidence \cite{paulson1964sequential}. However, the existing research can only deal with the problem under specific distributions. % For the case of a fixed budget, \cite{audibert2010best} deals with the problem for Bernoulli bandit models. They proved an asymptotic lower bound under this assumption. For the case of fixed confidence, most works \cite{huang2017structured, jourdan2023varepsilon, degenne2019pure} only deal with the bandit models with a one-parameter exponential family. 
These limitations restrict the algorithm's applicability and leave considerable room for further research on this issue. In our problem, we must select an appropriate POVM basis while choosing the quantum state in each round. The quantum state space and the POVM space grows exponentially with the increase in qubits, which makes this problem significantly more challenging than solving a classical problem of the best arm identification.