\section{Lower bound for PQSI with incoherent measurement}
\label{sec: lower_bound}


In this section, we investigate the lower bound on the error probability for solving the problem of purest quantum state identification. This problem requires distinguishing between quantum states with different purities through sampling and measurement. Recent studies \cite{mele2024introduction} indicate that when a quantum state $\rho$ is rotated by a Haar unitary matrix and measured $N$ times, the output distribution can be calculated only if $\rho$ is either a pure state or a maximally mixed state. Consequently, the complexity analysis of testing problems often assumes that one of the quantum states is either a pure state or a maximally mixed state. This limitation presents significant challenges to our analysis.

To solve this problem, we first reduce the problem of identifying the purest quantum state among $ N $ unknown quantum states into the problem of identifying the purest random quantum state from $ N $ unknown random quantum states. This reduction allows us to retain the problem's complexity while enabling us to analyze the complexity by considering only a single problem instance. 

Next, we demonstrate that for any POVM base $\mathcal{M}$, there is a set of unitary matrices $\mathbb{U}(\mathcal{M})$ satisfying that (1) $\mathbb{P}_{U \sim \Haar} (U \in \mathbb{U}(\mathcal{M})) = \Omega(1)$; and (2) when the quantum states rotated by these unitary matrices, they are difficult to distinguish by the POVM base  $\mathcal{M}$.

At last, we only consider all the possible POVM $\mathcal{M}$ and their corresponding set of unitary matrix $\mathbb{U}(M)$. By analyzing the sampling distribution for specific POVM $\mathcal{M}$ and unitary matrix in $\mathbb{U}(M)$, we reduce the problem into a classical problem for resolution and provide a lower bound for the purest quantum state identification.

Similar to Definition 7 in \cite{gong2024sample}, we analyze the lower bound of the error probability for any algorithm solving the purest quantum state identification problem using a 2-outcome randomly incoherent POVM to evaluate the task's difficulty.

\begin{definition}[Randomly fixed incoherent two-outcome POVM] We say an algorithm $\mathcal{A}$ with a randomly fixed incoherent two-outcome POVM, if it proceeds as the following:  The algorithm $\mathcal{A}$ samples a POVM $\mathcal{M} = \{M_0, M_1 = I_d -M_0\}$ from a well-designed distribution of POVMs $\mathcal{D}_{\mathcal{M}}$ and performs the two-outcome single-copy POVM $\mathcal{M}$ on the copies of the quantum states.
\end{definition}




\subsection{Problem reduction}

In this subsection, we aim to demonstrate that if there exists a set of random quantum states $ T(x) = \{\tau_1(x), \ldots, \tau_K(x)\} $ which is difficult to identify the purest one in $T$, there also exists a corresponding set of quantum states $ S = \{\rho_1, \ldots, \rho_K\} $ where is difficult to identify the purest one in $S$. In this way, we only need to construct $ K $  random quantum states, which are hard to distinguish, and then we can demonstrate the difficulty of the purest quantum state identification problem.

% To enhance our discussion, we will redefine the problem of the purest quantum state identification with incoherent measurement as follows:



To enhance our discussion, we define the problem of the purest random quantum state identification(PRQSI) with incoherent measurement as follows:

\begin{problem}[Purest random quantum state identification(PRQSI) with incoherent measurement] \label{problem:PRQSI}  Consider a set of $K$ unknown random quantum states, denote as $ T(U) = \{ \tau_1(U), \ldots, \tau_K(U) \} $, where $U$ samples from a fixed distribution $\mathcal{D}$. For each $ k \in [K] $ and $U \sim \mathcal{D}$, $\Tr ((\tau_k(U))^2) = z_k$. In each round $t \in\{1,...,N\}$, the learner selects an index $k_t \in [K]$ and a POVM $\mathcal{M}_t$. The learner obtains a copy of $\tau_{k_t}(U)$ and uses $\mathcal{M}_t$ to measure it. Upon completing $ N $ measurements, the learner selects an index $ k' \in [K] $ as the output. The objective of the learner is to maximize $ \mathbb{P}(k' \in \arg\max_{k \in [K]} z_k) $.
\end{problem} 

As shown in the following lemma, we can reduce the proof of the error probability lower bound for the PQSI problem into the proof of the lower bound for a specific instance of the PRQSI problem.

\begin{lemma}%[Le Cam's two-point method for the purest quantum state identification, see e.g. Lemma 1 in \cite{Yu1997} and Lemma 5.5 in \cite{9996689}]
    \label{lem:problem transformation}
    If there exists a set of random quantum states $T(U)$ in Problem \ref{problem:PRQSI} such that any algorithm $\mathcal{A}_T$ addressing Problem \ref{problem:PRQSI} cannot identify the purest random quantum state with an error probability lower than $ e_N $, then for any algorithm $ \mathcal{A} $ addressing Problem \ref{problem:PQSI_restatement}, there exists a specific set of quantum states $S = \{\rho_1, \ldots, \rho_K\}$ such that the error probability of algorithm $ \mathcal{A} $ is not lower than $ e_N $.
\end{lemma}

\begin{proof-sketch}
    Suppose that there exists such an algorithm $\mathcal{A}$ satisfying that the error probability of $\mathcal{A}$ for solving Problem \ref{problem:PQSI_restatement} is less than $e_N$, then we can prove that the algorithm $\mathcal{A}$ can solve the Problem \ref{problem:PRQSI} with the error probability less than $e_N$. Furthermore, we can establish the proof by considering the contrapositive of this statement. Detailed explanations of the proof are included in Appendix \ref{subsection: proof of lemma problem transformation}.
\end{proof-sketch}

According to Lemma \ref{lem:problem transformation}, we will establish the lower bound of the error probability for the problem PQSI by demonstrating the error probability lower bound for the following problem:

\begin{problem}\label{problem:PQDSI_special}
    Consider the Problem \ref{problem:PRQSI}. For $k \in [K]$, let $\alpha_k  = \sqrt{\frac{dz_k -1}{d-1}}$,
    \begin{equation}
        \tau_k(U) = \alpha_k U| 0 \rangle \langle 0|U^\dagger +  \frac{1-\alpha_k}{d} I_d, 
    \end{equation}
    where $U \sim \Haar$.
\end{problem}

Then, In the Problem \ref{problem:PQDSI_special}, for $k \in \{1,...,K\}$ and $U \sim \Haar$, the purity of the quantum state $\tau_k(U)$ satisfies:
\begin{equation*}
    \Tr\left((\tau_k(U))^2\right) = \left(\frac{1 + (d-1)\alpha_k}{d}\right)^2 + (d-1) \left(\frac{1-\alpha_k}{d}\right)^2 = z_k.
\end{equation*}

\subsection{Random quantum state purity certification}

To analyze Problem \ref{problem:PQDSI_special}, we first study the properties of the measurement results obtained from conducting $ N' $ measurements on the sampled quantum states from the following quantum state distribution $\mathcal{D}$ using a specific POVM $\mathcal{M} =\{M_0, M_1\}$:
\begin{equation}
    \label{eq:Haar state}
    \mathcal{D}:\ \rho = \alpha U| 0 \rangle \langle 0|U^\dagger +  \frac{1-\alpha}{d} I_d,
\end{equation}
where $\ U \sim \Haar$ and $\alpha$ is a constant satisfying $0\leq \alpha \leq 1$.

\begin{lemma}\label{lem:larger_expectation}
    Let $a \in [0,1]$. Using a specific POVM $\mathcal{M} =\{M_0, M_1\}$ to measure the random quantum state in Equation \eqref{eq:Haar state}. Let $M = \arg\min_{M' \in\{M_0,M_1\}} \Tr(M')$. We have
    \begin{equation*}
        \mathbb{P}_{U \sim \Haar} \left[ \left|p_{\mathcal{M}}(M| U) - \frac{\Tr(M)}{d}\right| < \frac{2a\sqrt{\Tr(M)}}{d}\right] \geq \frac{3}{4},
    \end{equation*}
    and there is a function $c(\mathcal{M},U)$ satisfying
    \begin{equation*}
        p_{\mathcal{M}}(M| U)- \frac{\Tr(M)}{d} =  c(\mathcal{M},U)\alpha.
    \end{equation*}
\end{lemma}
\begin{proof-sketch}
    By utilizing the properties of the Haar unitary matrix, we can calculate the variance of $p_\mathcal{M}(M|U)$ and prove the probabilistic bounds in the lemma using Chebyshev's inequality, thus completing the proof. The proof details are provided in Appendix \ref{subsection: proof of lemma larger expectation}.
\end{proof-sketch}

According to Lemma \ref{lem:larger_expectation}, for a specific POVM $\mathcal{M}$ and unitary matrix $U$, let $\mathbb{U}_{\mathcal{M}}$ denote the set of unitary matrix satisfying that 
\begin{equation*}
    \mathbb{U}_{\mathcal{M}} = \left\{U : \left|p_{\mathcal{M}}(M| U) - \frac{\Tr(M)}{d}\right| < \frac{2\alpha\sqrt{\Tr(M)}}{d}\right\}.
\end{equation*}
We have
\begin{equation}
    \mathbb{P}_{U \sim \Haar}(U \in \mathbb{U}_\mathcal{M}) \geq \frac{3}{4}.
\end{equation}

The following analysis will focus on the unitary matrices in the set $\mathbb{U}_\mathcal{M}$.
% Then expected probability $M$ ``accepts" the

% \begin{problem}\label{problem:purity_certification}
%     Consider the task of distinguishing between the following two alternatives with a fixed two-outcome single-copy POVM $\mathcal{M} = \{M_0, M_1 = I_d -M_0\}$:
%     \begin{equation}
%         \begin{aligned}
%             H_0:& \rho = \alpha U|0\rangle\langle 0| U^\dagger + \frac{1-\alpha}{d} I_d, \\
%             H_1:& \rho = (\alpha+\epsilon) U|0\rangle\langle 0| U^\dagger + \frac{1-\alpha-\epsilon}{d} I_d,
%         \end{aligned}
%     \end{equation}
%     where $U \sim \Haar$.
% \end{problem} 

\subsection{Error probability lower bound}

In this subsection, we will prove the lower bound of error probability for using algorithms to solve Problem \ref{problem:PQSI_restatement} and Problem \ref{problem:PQDSI_special}. We will use the following theorem to complete the proof:

\begin{theorem}[see Theorem 4 of Ref. \cite{audibert2010best}]
    \label{theorem:classical_BAI}
    Let $\nu_1,...,\nu_K$ be Bernoulli distributions with parameters in $[a,1-a]$, $a \in(0,1/2)$. For any forecaster, there exists a permutation $\sigma: \{1,..., K\} \rightarrow \{1,..., K\}$ such that the probability error of the forecaster on the bandit problem defined by $\tilde{\nu}_1= \nu_{\sigma(1)},..., \tilde{\nu}_K= \nu_{\sigma(K)}$ satisfies
    \begin{equation}
        e_n \geq \exp\left(- \frac{(5+o(1))nH}{p(a-a)}\right),
    \end{equation}
    where $H =\min_i \frac{(\mathbb{E}[\nu^*] - \mathbb{E}[\nu_{(i)}])^2}{i}$.
\end{theorem}

Let  $M = \arg\min_{M' \in \{M_0,M_1\}}\Tr(M')$, then we have $\Tr(M) \in [0,d/2]$. In the following theorem, let $\Tr(M) > 16$ in order to make $\Tr(M) - 2\sqrt{\Tr(M)} \geq \frac{1}{2}
\Tr(M) $. % In the following theorem, we establish the error probability lower bound of the purest quantum state identification.

\begin{theorem}
    \label{thm:PQSI lower bound}
     Let $M = \arg\min_{M' \in \{M_0,M_1\}}\Tr(M')$ and $\Tr(M) > 16$. For any algorithm $\mathcal{A}$ to solve the purest quantum state identification using fixed 2-outcome randomly incoherent POVM, there exists a set of quantum states which makes the error probability of $\mathcal{A}$ satisfies
     \begin{equation}
         e_N \geq \exp\left( - O\left( \frac{NH_1}{d}\right)\right),
     \end{equation}
     where $H_1 = \min_{i \in \{2,...,K\}} \frac{\Delta_{(i)}}{i}$.
\end{theorem}
\begin{proof-sketch}
    Let $p^{\mathcal{A}}_{e}(\mathcal{M},U)$ denote the error probability for algorithm $\mathcal{A}$ to solve the problem \ref{problem:PQDSI_special}, with specific unitary matrix $U$ and POVM $\mathcal{M}$. The error probability of $\mathcal{A}$ to solve the problem \ref{problem:PQDSI_special} satisfying
    \begin{equation*}
        \begin{aligned}
            e^\mathcal{A}_N = & \int_{\mathcal{M} \in \mathcal{D}_{\mathcal{M}}} \int_{U \sim \Haar} p^\mathcal{A}_e(\mathcal{M},U) d\mathcal{M} dU \\
            \geq & \int_{\mathcal{M} \in \mathcal{D}_{\mathcal{M}}} \int_{U \sim \Haar} p^\mathcal{A}_e(\mathcal{M},U)\indicator\{U \in \mathbb{U}_\mathcal{M}\} d\mathcal{M} dU.
        \end{aligned}
    \end{equation*}
    
When the $i$-th quantum state is measured using $\mathcal{M}$, the measurement result follows a Bernoulli distribution with parameter $\Tr(M\rho_U)$. From Lemma \ref{lem:larger_expectation} and the definition of $c(\mathcal{M}, U)$, we have 
    \begin{equation*}
        \Tr(M\rho_i|U) = c(U,\mathcal{M})\alpha_i + \frac{\Tr(M)}{d}.    
    \end{equation*}
    If $c(\mathcal{M}, U) >0$, we need to find the Bernoulli distribution with the largest parameter where the parameter of the $i$-th Bernoulli distribution is $\Tr(M\rho_i|U)$, and we have
    \begin{equation}
        \begin{aligned}
            & \Tr(M\rho_i|U) - \Tr(M\rho_j|U) \\
            = & c(\mathcal{M},U) \left[\sqrt {\frac{dz_i -1}{d-1}} - \sqrt{\frac{dz_j -1}{d-1}}\right].
        \end{aligned}
    \end{equation}
        
    Since $\Tr(M) > 16$ and according to the definition of $\mathbb{U}(\mathcal{M})$ and $M$, for $U \in \mathbb{U}$ we have
    \begin{equation}
         \Tr(M\rho_i|U) \in \left[\frac{\Tr(M)}{2d}, 1- \frac{\Tr(M)}{2d}\right],
    \end{equation}
    and 
    \begin{equation}
        1- \frac{\Tr(M)}{2d} \geq \frac{1}{2}.
    \end{equation}
    Then according to Theorem \ref{theorem:classical_BAI} we can demonstrate that 
    \begin{equation}
        p_e^\mathcal{A}(U,\mathcal{M}) \geq \exp\left(- O\left( \frac{NH_1}{d}\right)\right).
    \end{equation}
    Then we have
    \begin{equation*}
        \begin{aligned}
            e^\mathcal{A}_N \geq & \int_{\mathcal{M} \in \mathcal{D}_{\mathcal{M}}} \int_{U \sim \Haar} p^\mathcal{A}_e(\mathcal{M},U)\indicator\{U \in \mathbb{U}_\mathcal{M}\} d\mathcal{M} dU \\
            \geq & \exp\left(-O\left( \frac{NH_1}{d}\right)\right)\int_{\mathcal{M} \in \mathcal{D}_{\mathcal{M}}}  \frac{3}{4} d\mathcal{M} \\
            \geq & \exp\left(- O\left( \frac{NH_1}{d}\right)\right).\\            
        \end{aligned}
    \end{equation*}
    For any algorithm $\mathcal{A}_{\mathcal{D}}$ addressing Problem \ref{problem:PQDSI_special} cannot identify the purest random quantum state with an error probability lower than $\exp\left(-O\left( \frac{NH_1}{d}\right)\right)$. According to Lemma \ref{lem:problem transformation}, we can complete the proof. The proof details are provided in Appendix \ref{subsec: thm lower bound}.
\end{proof-sketch}