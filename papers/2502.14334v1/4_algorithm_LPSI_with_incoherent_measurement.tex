\section{Algorithm for PQSI with incoherent measurement}
\label{sec: PQSI_inco}

The current era of quantum computing, known as the Noisy Intermediate-Scale Quantum (NISQ) period, presents limitations in the number of available qubits. This may hinder the measurement of multiple quantum states operated jointly. In this section, we use incoherent (single-copy) measurement methods in each round to select the purest quantum state from the set of quantum states.

% To solve the distributed quantum inner product estimation, \cite{anshu2022distributed} proposed an algorithm based on classical collision estimator \cite{diakonikolas2019collision, canonne2020survey}. For distribution $p$ and $q$ supported on \{0,...,d-1\}, they approximate $\sum_{i=0}^{d-1}p_iq_i$ by sampling from $p$ and $q$, and estimate it based on the frequency of 'collisions' in the sampling results. 
In our algorithm, we use random samples to estimate the expected value of a random process, and estimate the purity of quantum states. For a distribution $p$ supported on $\{0,...,d-1\}$, let $p_i$ denote the probability of observing $i$ in the distribution $p$. We employ the following purity estimation method to estimate the $\sum_{i=0}^{d-1}p_i^2$.

\begin{definition}[purity estimator]
    \label{def: purity_estimator}
    Given $m$ samples $x_1,...,x_m \sim p$, the purity estimator is defined as 
    \begin{equation} \label{eq:purity_estimator}
        \tilde{g} = \frac{1}{m^2} \sum_{i=0}^{d-1} \left[ \sum_{j=1}^m \indicator\{x_j =i\}\right]^2 - \frac{1}{m}.
    \end{equation}
\end{definition}

We can prove that the expectation of the purity estimation $\tilde{g}$ is $\frac{m-1}{m}\sum_{i=0}^{d-1} p_i^2$.



By leveraging the properties of Haar unitary matrices, we can connect the purity of a quantum state to the estimation of $ \sum_{i=0}^{d-1} p_i^2 $ for a classical distribution $ p $. We use the purity estimator defined in Definition $\ref{def: purity_estimator}$ to perform this estimation. Additionally, we utilize the successive reject algorithm to distribute a limited number of sampling times across the available quantum states, thereby enhancing the algorithm's accuracy. The algorithm we designed to solve the PQSI problem is shown in Algorithm \ref{alg:PQSI_incoherent}.

\begin{algorithm}[htb]
    \caption{SR-PQSI with incoherent measurement}
    \label{alg:PQSI_incoherent}
    \begin{algorithmic}
    \STATE {\bfseries Input:} Copy access to $S = \{\rho_1,...,\rho_K\}$, sample number $N$.
    \STATE {\bfseries Initialization:} Set $S_0 = \{\rho_1,...,\rho_K\}$, $\overline{\log}(K) = \frac{1}{2} + \sum_{i=2}^K \frac{1}{i}$, $N_0=0$ and $N_k = \left\lceil \frac{1}{\overline{\log}(K)} \frac{N-K}{K+1-k}\right\rceil$, for $k \in \{1,...,K-1\}$. Sample $\lfloor {N}/m \rfloor$ random unitary matrix $U_1,...,U_{\lfloor N/m \rfloor}$ according to the Haar measure.
    \FOR{k=1,..., K-1}
        \FOR{$\rho \in S_{k-1}$ and $j \in \{\lfloor \frac{N_{k-1}}{m}\rfloor +1,..., \lfloor \frac{N_{k}}{m}\rfloor \}$}
        \STATE Measure $m$ copies of $\rho$ in the basis $\{U_j^\dagger|i\rangle \langle i | U_j\}_{i=0}^{d-1}$ and set the outputs as $x{(\rho, j,1)},..., x{(\rho, j,m)}$.
        \STATE Compute the purity estimator \eqref{eq:purity_estimator} using $x{(\rho, j,1)}$, $...$, $x{(\rho, j,m)}$ denoted as $\tilde{g}{(\rho,j)}$.
        \ENDFOR
        \STATE  $$w(\rho,k) = \frac{1}{\lfloor \frac{N_{k}}{m}\rfloor} \sum_{j=1}^{\lfloor \frac{N_{k}}{m}\rfloor} \tilde{g}{(\rho,j)}. $$
        \STATE Let $S_k = S_{k-1} \setminus \arg \min_{\rho \in S_{k-1}} w(\rho,k) $.
    \ENDFOR
    \STATE Output the quantum state $\rho'$ in $S_{k-1}$.

\end{algorithmic}
\end{algorithm}
By using the techniques similar to  \cite{anshu2022distributed}, we can prove the following lemma:
\begin{lemma}[See Lemma 16 of \cite{anshu2022distributed}]
\label{lem: g_var}
    The expectation of $w(\rho,k)$ and $\tilde{g}_{\rho,j}$ in Algorithm \ref{alg:PQSI_incoherent} satisfies
    \begin{equation}
        \mathbb{E}[w(\rho,k)] = \mathbb{E}[\tilde{g}_{\rho,j}] = \frac{(m-1)(1+ \Tr(\rho^2))}{m(d+1)},
    \end{equation}
    
    and the variance of $\tilde{g}(\rho,j)$ satisfies
    \begin{equation}
        \Var(\tilde{g}(\rho,j)) = O\left(\frac{1}{d^3} + \frac{1}{m^2d} + \frac{1}{md^2}\right).
    \end{equation}
\end{lemma}
% \begin{proof}
%     \begin{equation}
%         \begin{aligned}
%             \mathbb{E}[w(\rho,k)] = & \mathbb{E}[\tilde{g}_{\rho,j}] \\
%             = & \frac{m-1}{m}\sum_{i=0}^{d-1} \mathbb{E}_{\psi \sim\mathbb{C}^d} (\langle \psi | \rho | \psi \rangle)^2 \\
%             = & \frac{(m-1)(1+ \Tr(\rho^2))}{m(d+1)}.
%         \end{aligned}
%     \end{equation}
% \end{proof}

Then, we give the error probability upper bound of SR-PQSI with incoherent measurement in the following theorem.

\begin{theorem}
    \label{thm:alg_SRPQSI}
    For $ i \in \{1,...,K\}, \Delta_i \geq c > \frac{1}{d^2}$, where $c$ is a constant. Set $m=\lceil \frac{1}{\sqrt{c}} \rceil$ in Algorithm \ref{alg:PQSI_incoherent}. The error probability of Algorithm \ref{alg:PQSI_incoherent} satisfies
    \begin{equation*}
         e_N \leq \frac{K(K-1)}{2}\exp\left(-\Omega \left( \frac{\sqrt{c}N H_1}{\overline{\log}(K) d }\right) \right),
    \end{equation*}
    where $H_1 = \min_{i \in \{2,...,K\}} \frac{\Delta_{(i)}}{i}$.
\end{theorem}

\begin{proof-sketch}
    By the definition of $w(\cdot,\cdot)$ and the definition of $\Delta_{(\cdot)}$, we have
    \begin{equation*}
        \begin{aligned}
            & \mathbb{P}(w({\rho^\star,k}) \leq w({\rho_{(i)},k})) \\
            = & \mathbb{P}\left( (w(\rho_{(i)},k) -  w(\rho^\star,k))  \geq \frac{(m-1)\Delta_{(i)}}{m}\right).
        \end{aligned}
    \end{equation*}
    Since $w(\cdot,\cdot) \in [0,1]$, by Lemma \ref{lem: g_var} and Bernstein's inequality, we have
    \begin{equation}
        \begin{aligned}
            & \mathbb{P}\left( (w(\rho_{(i)},k) -  w(\rho^\star,k))  \geq \frac{(m-1)\Delta_{(i)}}{m}\right) \\
            \leq & \exp\left( -\Omega\left( \frac{\sqrt{c}N_k \Delta_{(i)}}{d}\right)\right).
        \end{aligned}
    \end{equation}

    By a union bound of error probability, we have 

    \begin{equation*} 
        \begin{aligned} 
            e_n & \leq \sum_{k=1}^{K-1}\sum_{i = K+1-k}^K \mathbb{P}(w({\rho^\star,k}) \leq w({\rho_{(i)},n_k})) \\
            & \leq \sum_{k=1}^{K-1}\sum_{i = K+1-k}^K \exp\left( - \Omega\left(\frac{\sqrt{c}N_k \Delta_{(i)}}{d}\right)\right) \\
            &  \leq \sum_{k=1}^{K-1} k \exp\left( -\Omega\left( \frac{\sqrt{c}N_k \Delta_{(K+1-k)}}{d}\right)\right). \\
        \end{aligned}
    \end{equation*}
    By the definition of $N_k$, we have
    \begin{equation*}
        \begin{aligned}
            e_n & \leq \sum_{k=1}^{K-1} k \exp\left( -\Omega\left(\frac{\sqrt{c}N}{\overline{\log}(K)d} \times\frac{\Delta_{(K+1-k)}}{K+1-k}\right)\right) \\
            & \leq \frac{K(K-1)}{2} \exp\left( -\Omega\left(\frac{\sqrt{c}NH_1}{\overline{\log}(K)d}\right)\right)
        \end{aligned}
    \end{equation*}
    where $H_1 = \min_{i \in \{1,...,K\}} \frac{\Delta_{(i)}}{i}$.
    The proof details are provided in Appendix \ref{subsec:PQSI_alg1}.
\end{proof-sketch}

The dimension $d=2^n$ increases exponentially with the number of qubits $n$. As $n$ increases, $\frac{1}{d^2}$ tends to 0. Therefore, in Theorem \ref{thm:alg_SRPQSI}, we assume that for any quantum state $\rho \in S_\rho$, $\Tr(\rho^{\star2}) - \Tr(\rho^{2}) \geq c > \frac{1}{d^2}$. If we can not make this assumption, we can derive the following conclusion:

\begin{lemma}
    \label{lem2:PQSI_alg}
    Set $m=d$ in Algorithm \ref{alg:PQSI_incoherent}. The probability of error of Algorithm \ref{alg:PQSI_incoherent} satisfies
    \begin{equation*}
    \begin{aligned}
        e_N \leq \frac{K(K-1)}{2} \exp\left(- \Omega\left(\min\left(\frac{N H_2}{\overline{\log}(K)}, \frac{N H_1}{\overline{\log}(K)d^2}\right)\right) \right),
    \end{aligned}
    \end{equation*}
    where $H_1 = \min_{i \in \{2,...,K\}} \frac{\Delta_{(i)}}{i}$, and $H_2 = \min_{i \in \{2,...,K\}} \frac{\Delta^2_{(i)}}{i}$.
\end{lemma}

Appendix \ref{subsec:PQSI_alg2} provides the proof details of Lemma \ref{lem2:PQSI_alg}.

In the fields of quantum learning and testing, research on quantum channels constitutes a critical aspect. Evaluating the impact of noise on quantum channels can significantly enhance the accuracy of quantum computing and quantum communication. \cite{chen2023unitarity} introduced a method for assessing the ``unitarity" of a quantum channel by evaluating the purity of a quantum state. Subsequently, we can utilize the algorithm SR-PQSI to identify the most ``unitary" quantum channel from a quantum channel set. Let $u_{(i)}$ denote the unitarity of the $i$-th most unitary quantum channel. We have the following corollary:

\begin{corollary}
    There exists an algorithm that solves the problem of the most ``unitary" channel identification with incoherent access whose error probability satisfies:
 \begin{equation}
        e_N \leq \exp\left(- \Omega\left(\frac{N H_u}{\log(K) 2^n }\right) \right),
    \end{equation}
    where $H_u = \min_{i \in \{2,...,K\}} \frac{u_{(1)} - u_{(i)}}{i}$.
\end{corollary}