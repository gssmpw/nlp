\begin{abstract}

% In the Noisy Intermediate-Scale Quantum(NISQ) era, the limited number of qubits and the inherent noise present significant challenges for quantum computing. 


Precise identification of quantum states under noise constraints is essential for quantum information processing. In this study, we generalize the classical best arm identification problem to quantum domains, designing methods for identifying the purest one within $K$ unknown $n$-qubit quantum states using $N$ samples. %, with direct applications in quantum computation and quantum communication. 
We propose two distinct algorithms: (1) an algorithm employing incoherent measurements, achieving error $\exp\left(- \Omega\left(\frac{N H_1}{\log(K) 2^n }\right) \right)$, and (2) an algorithm utilizing coherent measurements, achieving error $\exp\left(- \Omega\left(\frac{N H_2}{\log(K) }\right) \right)$, highlighting the power of quantum memory. Furthermore, we establish a lower bound by proving that all strategies with fixed two-outcome incoherent POVM must suffer error probability exceeding $ \exp\left( - O\left(\frac{NH_1}{2^n}\right)\right)$. This framework provides concrete design principles for overcoming sampling bottlenecks in quantum technologies.


% We investigate the problem of the purest quantum state identification from a set of $K$ unknown quantum states, which is vital for various applications, including quantum channel identification, quantum state preparation, and quantum machine learning. We propose two algorithms to address this problem: SR-PQSI with coherent measurement and SR-PQSI with incoherent measurement.
% For $n$-qubit quantum states and $N$ sampling times, their error probabilities for identifying the purest quantum state are $\exp\left(- \Omega\left(\frac{N H_1}{\log(K) 2^n }\right) \right)$ and $\exp\left(- \Omega\left(\frac{N H_2}{\log(K) }\right) \right)$, respectively, where $H_1$ and $H_2$ are related to the set of the unknown quantum states. By comparing these error probabilities, we can discern the advantages of quantum computing relative to classical computing. Furthermore, we establish that the error probability of any algorithm utilizing a fixed incoherent two-outcome POVM to solve the purest quantum state identification problem is greater than $ \exp\left( - O\left(\frac{NH_1}{2^n}\right)\right)$.

%analyze the complexity of the purest quantum state identification problem with f by demonstrating a lower bound on the error probability for any algorithm employing incoherent measurement.
\end{abstract}
