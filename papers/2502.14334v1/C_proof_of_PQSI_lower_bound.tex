\section{Proof of lower bound}

\subsection{Proof of Lemma \ref{lem:problem transformation}}
\label{subsection: proof of lemma problem transformation}

% Similar to the definition in The algorithm for solving PQSI with incoherent measurement can be represented as a tree:

% \begin{definition}[Tree Representation]\label{def:tree}
%     Fix $k$ unknown $d$-dimensional mixed state sets $S_\rho =\{\rho_1,\rho_2,..,\rho_k \}$. A learning algorithm that only uses $N$ incoherent, possibly adaptive, measurements of the quantum states in $S_\rho$ can be expressed as a rooted tree $\mathcal{T}$ of depth $n$ satisfying the following properties:
%     \begin{itemize}
%         \item Each node is labeled by an quantum state id and a string of vectors $\ba = (a_1,...,a_t), a_i = (x_i,y_i)$ where each $x_i$ corresponds to id of the quantum state selected in the $i$-th step, and each $y_i$ corresponds to measurement out come observed in the $i$-th step.
%         \item Each node $\ba$ is associated with a number $p(\ba)$ corresponding to the probability of observing $\ba$ over the course of the algorithm. The probability for the root is $1$.
%         \item At each non-leaf node, we choose a quantum state $\rho$ whose id is $x_i$  and measure it in $S_\rho$ using a two-outcome POVM $\{M_0,M_1\}, M_0 + M_1 =I$ to obtain classical outcome $x \in \mathbb{S}^{d-1}$. The children of $\ba$ consist of all string $\boldsymbol{a'} = (a_1,...,a_t,a)$ for which a is the set of the id and a possible POVM outcome.
%         \item If $\ba' = (a_1,...,a_t, a)$ is a child of $a$, then 
%         \begin{equation}
%             p^{A}(\ba') = p^{A}(\ba) \cdot w_x d \cdot \Tr(M_{y_i}\rho_{x_j}).
%         \end{equation}
%         \item Every root-to-leaf path is length-$N$. Note that $\mathcal{T}$ and $A$ induce a distribution over the leaves  of $\mathcal{T}$.
%     \end{itemize}
% \end{definition}


Suppose that there exists such an algorithm $\mathcal{A}$ satisfying that the error probability of $\mathcal{A}$ for solving Problem \ref{problem:PQSI_restatement} with the quantum state set $S_{\rho}$ whose error probability is less than $e_N$. Let $p_{S_\rho}(x_1,y_1;...;x_N,y_N;z)$ denote the probability of the event satisfying
\begin{enumerate}
    % \item the algorithm $\mathcal{A}$ use the POVM $\mathcal{M}$ to measure quantum states in $S_\rho$;
    \item for $i \in {1,...,N}$, in the round $N$, the algorithm $\mathcal{A}$ select the $x_i$-th quantum state for measurement, and its output is $y_i$;
    \item the algorithm $A$ output $z$-th quantum state at the end. 
\end{enumerate}
Furthermore, let $q_{S_\rho}(x_1,y_1;...;x_N,y_N;z)$ denote the error probability corresponding to $p_{S_\rho}(x_1,y_1;...;x_N,y_N;z)$.

Since when using $\mathcal{A}$ to solve the Problem \ref{problem:PQSI_restatement}, error probability is less than $e_N$. Then for all quantum state set $S =\{\rho_1,...,\rho_K\}$, we have
\begin{equation}
    \int_{(x_1,y_1,...,x_N,y_N,z)} q(x_1,y_1;...;x_N,y_N;z) dp(x_1,y_1;...;x_N,y_N;z) \leq e_N.
\end{equation}

When the learner use the algorithm $\mathcal{A}$ to solve the problem \ref{problem:PRQSI}, its error probability satifying
\begin{equation}
    \begin{aligned}
        e^\mathcal{D}_{N} & = \int_{x \sim \mathcal{D'}} \int_{(x_1,y_1,...,x_N,y_N,z)} q_{S_{\mathcal{D}(x)}}(x_1,y_1;...;x_N,y_N;z) dp_{S_{\mathcal{D}(x)}}(x_1,y_1;...;x_N,y_N;z) dx\\
        & \leq \int_{x \sim \mathcal{D}'} e_N d_x \\
        & \leq e_N.
    \end{aligned}
\end{equation}

then we can prove that if there is an algorithm $\mathcal{A}$ can solve the Problem \ref{problem:PRQSI} with the error probability less than $e_N$, then it can solve the Problem \ref{problem:PQSI_restatement} with the error probability less than $e_N$. Furthermore, we can establish the proof by considering the contrapositive of this statement. 





\subsection{proof of Lemma \ref{lem:larger_expectation}}
\label{subsection: proof of lemma larger expectation}
    Without loss of generality, assume that $M_0 = \arg\min_{M' \in \{M_0,M_1\}} \Tr(M')$.  According to the definition of POVM, there exists a unitary matrix \( V \) and a diagonal matrix $ \Sigma_0 = \diag(b_0,...,b_{d-1})$, where $b_0,...,b_{d-1} \in [0,1]$ such that
    \begin{equation*}
        \begin{aligned}
            M_0 & = V \Sigma_0 V^\dagger = \sum_{i=0}^{d-1} b_i V |i \rangle \langle i | V^\dagger, \\
            M_1 & = I - V \Sigma_0 V^\dagger = \sum_{i=0}^{d-1} (1-b_i) V |i \rangle \langle i | V^\dagger. \\
        \end{aligned}
    \end{equation*}
    We have
    \begin{equation}
        \begin{aligned}
            & \Tr(M_0) = \Tr (\Sigma_0) = \sum_{i=0}^{d-1} b_i, \\     
            & \Tr(M_1) = \Tr (I - \Sigma_0) = d - \sum_{i=0}^{d-1} b_i.
        \end{aligned}
    \end{equation}
    Let $ p_{\mathcal{M}}(M | U) $ denote the probability that $ M $ ``accepts" the quantum state $ \alpha U | 0 \rangle \langle 0 | U^\dagger + \frac{1-\alpha}{d-1} I_d $. According to the property of the Haar measure and the identity matrix $I_d$, we have 
    \begin{equation}
        \begin{aligned}
            & \mathbb{E}_{U \sim \Haar}\left[ p^2_{\mathcal{M}}(M_0| U)\right] \\
            = & \mathbb{E}_{U \sim \Haar}\left[\Tr^2\left(M_0 \left( \alpha U | 0 \rangle \langle 0 | U^\dagger + \frac{1-\alpha}{d} I_d\right)\right)\right] \\
            = & \mathbb{E}_{U \sim \Haar}\left[\Tr^2\left(\sum_{i=0}^{d-1} b_i V |i \rangle \langle i | V^\dagger \left( \alpha U | 0 \rangle \langle 0 | U^\dagger + \frac{1-\alpha}{d} I_d\right)\right)\right] \\
            = & \mathbb{E}_{U \sim \Haar}\left[\left(\sum_{i=0}^{d-1} b_i  \langle i | V^\dagger \left( \alpha U | 0 \rangle \langle 0 | U^\dagger + \frac{1-\alpha}{d} I_d)\right)V |i \rangle\right)^2\right] \\
            = & \mathbb{E}_{U \sim \Haar}\left[\left(\sum_{i=0}^{d-1}  \left(\alpha b_i  \langle i | V^\dagger U | 0 \rangle \langle 0 | U^\dagger V |i \rangle\right) + \sum_{i=0}^{d-1} b_i\frac{1-\alpha}{d} \langle i | V^\dagger I_dV |i \rangle\right)^2\right] \\
            = & \mathbb{E}_{U \sim \Haar}\left[\left(\sum_{i=0}^{d-1} \left(\alpha b_i  \langle i |U | 0 \rangle \langle 0 | U^\dagger|i \rangle\right) + \frac{1-\alpha}{d}\Tr(M_0)\right)^2\right] \\
        \end{aligned}
    \end{equation}
    Let $V_i$ is the matrix satisfying that $V_i|i\rangle = |0\rangle$, then $V_i$ is an unitary matrix and $V_i^{-1} = V_i^\dagger$, we have
    \begin{equation} \label{eq2:p_M0}
        \begin{aligned}
            &\mathbb{E}_{U \sim \Haar}\left[ p^2_{\mathcal{M}}(M_0| U)\right] \\
            = & \mathbb{E}_{U \sim \Haar}\left[\left(\sum_{i=0}^{d-1} \left(\alpha b_i  \langle i |U | 0 \rangle \langle 0 | U^\dagger|i \rangle\right) + \frac{1-\alpha}{d}\Tr(M_0)\right)^2\right] \\
            = & \mathbb{E}_{U \sim \Haar}\left[\left(\sum_{i=0}^{d-1} \left(\alpha b_i  \langle i | V_i U V_i^\dagger | 0 \rangle \langle 0 | V_i^\dagger U^\dagger V_i|i \rangle\right) + \frac{1-\alpha}{d}\Tr(M_0)\right)^2\right] \\
            = & \mathbb{E}_{U \sim \Haar}\left[\left(\sum_{i=0}^{d-1} \left(\alpha b_i  \langle 0 |U | i \rangle \langle i | U^\dagger|0 \rangle\right) + \frac{1-\alpha}{d}\Tr(M_0)\right)^2\right] \\
            = & \mathbb{E}_{\psi \sim \mathbb{C}^d} \left[\left(\sum_{i=0}^{d-1} \left(\alpha b_i  \langle \psi | i \rangle \langle i |\psi \rangle\right) + \frac{1-\alpha}{d}\Tr(M_0)\right)^2\right] \\
            = & \mathbb{E}_{\psi \sim \mathbb{C}^d} \left[  \alpha^2 \sum_{i=0}^{d-1} b^2_i  \langle \psi | i \rangle \langle i |\psi \rangle\langle \psi | i \rangle \langle i |\psi \rangle + \alpha^2\sum_{i=0}^{d-1}\sum_{j \neq i}b_ib_j \langle \psi | i \rangle \langle i |\psi \rangle\langle \psi | j \rangle \langle j |\psi \rangle \right.\\
            &\left. + 2\sum_{i=0}^{d-1} \left(\alpha b_i  \langle \psi | i \rangle \langle i |\psi \rangle\right)\frac{1-\alpha}{d}\Tr(M_0) + \left(\frac{1-\alpha}{d}\right)^2\Tr^2(M_0)    \right] \\
        \end{aligned}
    \end{equation}
    According to the Lemma \ref{lem:Haar_meausrement_expectation}, we have for $i,j \in \{0,...,d-1\}$, $i\neq j$,
    \begin{equation} \label{eq:expectation_i}
        \mathbb{E}_{\psi \sim \mathbb{C}^d} \langle \psi | i \rangle \langle i |\psi \rangle= \frac{1}{d},
    \end{equation}
    and
    \begin{equation} \label{eq:expectation_ii}
        \mathbb{E}_{\psi \sim \mathbb{C}^d} \langle \psi | i \rangle \langle i |\psi \rangle\langle \psi | i \rangle \langle i |\psi \rangle= \frac{2}{d(d+1)},
    \end{equation}
    and similarly
    \begin{equation} \label{eq:expectation_ij}
        \mathbb{E}_{\psi \sim \mathbb{C}^d} \langle \psi | i \rangle \langle i |\psi \rangle\langle \psi | j \rangle \langle j |\psi \rangle= \frac{1}{d(d+1)}.
    \end{equation}
    According to Equation \eqref{eq2:p_M0}, \eqref{eq:expectation_i},\eqref{eq:expectation_ii} and \eqref{eq:expectation_ij}, we have 
    \begin{equation}
        \begin{aligned} \label{eq3:p_M0}
            &\mathbb{E}_{U \sim \Haar}\left[ p^2_{\mathcal{M}}(M_0| U)\right] \\
            = & \frac{2\alpha^2}{d(d+1)} \sum_{i=0}^{d-1} b^2_i   + \frac{\alpha^2}{d(d+1)}\sum_{i=0}^{d-1}\sum_{j \neq i}b_ib_j  + \frac{2\alpha(1-\alpha)}{d^2}\Tr(M_0) \sum_{i=0}^{d-1} b_i + \left(\frac{1-\alpha}{d}\right)^2\Tr^2(M_0)  \\
            = & \frac{\alpha^2}{d(d+1)} \sum_{i=0}^{d-1} b^2_i   + \frac{\alpha^2}{d(d+1)}\left(\sum_{i=0}^{d-1}b_i\right)^2  + \frac{2\alpha(1-\alpha)}{d^2}\Tr(M_0) \sum_{i=0}^{d-1} b_i + \left(\frac{1-\alpha}{d}\right)^2\Tr^2(M_0) \\
            = & \frac{\alpha^2}{d(d+1)} \sum_{i=0}^{d-1} b^2_i   + \frac{\alpha^2}{d(d+1)}\Tr^2(M_0)  + \frac{2\alpha(1-\alpha)}{d^2}\Tr^2(M_0) + \left(\frac{1-\alpha}{d}\right)^2\Tr^2(M_0) \\
             = & \frac{\alpha^2}{d(d+1)} \sum_{i=0}^{d-1} b^2_i   + \left[\frac{1}{d^2} - \frac{\alpha^2}{d^2(d+1)}\right]\Tr^2(M_0), \\
        \end{aligned}
    \end{equation}
    and
    \begin{equation}
        \begin{aligned}
            &\mathbb{E}_{U \sim \Haar}\left[ p_{\mathcal{M}}(M_0| U)\right] \\
            = & \mathbb{E}_{U \sim \Haar}\left[\Tr\left(M_0 \left( \alpha U | 0 \rangle \langle 0 | U^\dagger + \frac{1-\alpha}{d} I_d\right)\right)\right] \\
            = & \mathbb{E}_{U \sim \Haar}\left[\Tr\left(\sum_{i=0}^{d-1} b_i V |i \rangle \langle i | V^\dagger \left( \alpha U | 0 \rangle \langle 0 | U^\dagger + \frac{1-\alpha}{d} I_d\right)\right)\right] \\
            = & \mathbb{E}_{U \sim \Haar}\left[\Tr\left(\sum_{i=0}^{d-1} b_i V |i \rangle \langle i | V^\dagger \left( \alpha U | 0 \rangle \langle 0 | U^\dagger + \frac{1-\alpha}{d} I_d\right)\right)\right]
            = & \frac{\Tr(M_0)}{d}.
        \end{aligned}
    \end{equation}
    % Similarly, we have
    % \begin{equation} \label{eq:p_M1}
    %     \mathbb{E}_{U \sim Haar}\left[ p^2_{\mathcal{M}}(M_1| U)\right] = \frac{a^2}{d(d+1)} \sum_{i=0}^{d-1} (1 -b_i)^2   + \left[\frac{1}{d^2} - \frac{a^2}{d^2(d+1)}\right]\Tr^2(M_1).
    % \end{equation}

Then the variance of $p_{\mathcal{M}}(M_0| U)$ is given by 

    \begin{equation}
        \begin{aligned}
            & \Var\left[p_{\mathcal{M}}(M_0| U)\right] \\
            = & \mathbb{E}_{U \sim \Haar}\left[ p^2_{\mathcal{M}}(M_0| U)\right] - \left(\mathbb{E}_{U \sim \Haar}\left[ p_{\mathcal{M}}(M_0| U)\right]\right)^2 \\
            = &\frac{\alpha^2}{d(d+1)} \sum_{i=0}^{d-1} b^2_i   + \left[\frac{1}{d^2} - \frac{\alpha^2}{d^2(d+1)}\right]\Tr^2(M_0) - \frac{\Tr^2(M_0)}{d^2} \\
            = &\frac{\alpha^2}{d(d+1)} \sum_{i=0}^{d-1} b^2_i   - \frac{\alpha^2}{d^2(d+1)}\Tr^2(M_0) \\
            = &\frac{\alpha^2}{d^2(d+1)} \left[d\sum_{i=0}^{d-1} b^2_i - \Tr^2(M_0)\right] \\
            \leq & \frac{\alpha^2}{d^2(d+1)} \left[d\sum_{i=0}^{d-1} b_i - \Tr^2(M_0)\right] \\
            = & \frac{\alpha^2}{d^2(d+1)} \left[d\Tr(M_0) - \Tr^2(M_0)\right] \\
            \leq & \frac{\alpha^2}{d^2(d+1)} \left[d\Tr(M_0) - \Tr^2(M_0)\right] \\ 
            \leq & \frac{\alpha^2\Tr(M_0)}{d(d+1)}.
        \end{aligned}
    \end{equation}

    From Chebyshev's Inequality, we have
    \begin{equation}
        \mathbb{P}_{U \sim \Haar} \left[ \left|p_{\mathcal{M}}(M_0| U) - \frac{\Tr(M_0)}{d}\right| \geq \frac{2\alpha\sqrt{\Tr(M_0)}}{d}\right]  < \frac{1}{4}.
    \end{equation}

    And
    \begin{equation}
        \begin{aligned}
            & p_{\mathcal{M}}(M_0| U)  \\
            = & \Tr\left(\sum_{i=0}^{d-1} b_i V |i \rangle \langle i | V^\dagger \left( \alpha U | 0 \rangle \langle 0 | U^\dagger + \frac{1-\alpha}{d} I_d\right)\right) \\
            = & \Tr\left(\sum_{i=0}^{d-1} b_i V |i \rangle \langle i | V^\dagger \left( \alpha U | 0 \rangle \langle 0 | U^\dagger + \frac{\alpha}{d} I_d\right)\right) + \Tr\left(\sum_{i=0}^{d-1} b_i V |i \rangle \langle i | V^\dagger \left( \frac{1}{d} I_d\right)\right)\\
            = & \alpha \Tr\left(\sum_{i=0}^{d-1} b_i V |i \rangle \langle i | V^\dagger \left( U | 0 \rangle \langle 0 | U^\dagger + \frac{1}{d} I_d\right)\right) + \frac{M_0}{d}.\\
        \end{aligned}
    \end{equation}
    Let $c(\mathcal{M},U) = \Tr\left(\sum_{i=0}^{d-1} b_i V |i \rangle \langle i | V^\dagger \left( U | 0 \rangle \langle 0 | U^\dagger + \frac{1}{d} I_d\right)\right)$, we have
    \begin{equation}
        p_{\mathcal{M}}(M_0| U) - \frac{M_0}{d} =  c(\mathcal{M},U) \alpha.
    \end{equation}
    % According to Equation \eqref{eq3:p_M0} and \eqref{eq:p_M1}, we have
    % \begin{equation} \label{eq: square_upper_bound}
    %     \begin{aligned}
    %         & \mathbb{E}_{U \sim Haar}\left[ p^2_{\mathcal{M}}(M_0| U)\right] + \mathbb{E}_{U \sim Haar}\left[ p^2_{\mathcal{M}}(M_1| U)\right] \\
    %         = & \frac{a^2}{d(d+1)} \sum_{i=0}^{d-1} \left[b^2_i + (1-b_i)^2\right]   + \left[\frac{1}{d^2} - \frac{a^2}{d^2(d+1)}\right]\left(\Tr^2(M_0) + \Tr^2(M_1)\right) \\
    %         \leq & \frac{a^2}{d(d+1)} \sum_{i=0}^{d-1} \left[b_i + (1-b_i)\right]   + \left[\frac{1}{d^2} - \frac{a^2}{d^2(d+1)}\right]\left(\Tr^2(M_0) + \Tr^2(M_1)\right) \\
    %          = & \frac{a^2}{d+1}  + \left[\frac{1}{d^2} - \frac{a^2}{d^2(d+1)}\right]\left(\Tr^2(M_0) + \Tr^2(M_1)\right) \\
    %          = & \frac{a^2}{d+1}  + \left[1 - \frac{a^2}{d+1}\right]\left( \left(\frac{\Tr(M_0)}{d}\right)^2 + \left(\frac{\Tr(M_1)}{d}\right)^2 \right) \\
    %          = & \left( \left(\frac{\Tr(M_0)}{d}\right)^2 + \left(\frac{\Tr(M_1)}{d}\right)^2 \right) + \frac{a^2}{d+1}\left( 1- \left( \frac{\Tr(M_0)}{d}\right)^2 - \left(\frac{\Tr(M_1)}{d}\right)^2 \right) \\
    %          \leq & \left( \left(\frac{\Tr(M_0)}{d}\right)^2 + \left(\frac{\Tr(M_1)}{d}\right)^2 \right) + \frac{3a^2}{4(d+1)}
    %     \end{aligned}
    % \end{equation}

    
    % Without loss of generality, let $\Tr(M_0) \leq Tr(M_1)$. For $U \in \mathbb{U}(d)$, if $\Tr(M_0\rho_U) \leq \frac{\Tr(M_0)}{d} - \frac{2a}{\sqrt{d}}$, we have 
    
    % \begin{equation}
    %     \begin{aligned} \label{eq: sqrtd_lower_bound}
    %         & p^2_{\mathcal{M}}(M_0| U) +  p^2_{\mathcal{M}}(M_1| U) \\
    %         \geq & \left(\frac{\Tr(M_0)}{d} - \frac{2a}{\sqrt{d}}\right)^2 + \left(1 - \frac{\Tr(M_0)}{d} + \frac{2a}{\sqrt{d}}\right)^2\\
    %         = & \frac{\Tr^2(M_0) - 4 \Tr(M_0)a\sqrt{d} + 4da}{d^2} + \frac{\Tr^2(I-M_0) + 4\Tr(I-M_0)a\sqrt{d} + 4da}{d^2} \\
    %         = &\left( \left(\frac{\Tr(M_0)}{d}\right)^2 + \left(\frac{\Tr(M_1)}{d}\right)^2 \right) + \frac{4 \Tr(I - 2M_0)a \sqrt{d}}{d^2} + \frac{8a^2}{d} \\
    %         \geq & \left( \left(\frac{\Tr(M_0)}{d}\right)^2 + \left(\frac{\Tr(M_1)}{d}\right)^2 \right) + \frac{8a^2}{d}.
    %     \end{aligned}
    % \end{equation}

    % If $\Tr(M_0 \rho_U) \geq \frac{\Tr(M_0)}{d} + {2a}{\sqrt{d}}$, we have

    % \begin{equation}
    %     \begin{aligned}
    %         & \Tr(M_0 \rho_U) \\
    %         = & 
    %     \end{aligned}
    % \end{equation}



    
    % If $M' = M_0$, we have


    
    % \begin{equation}
    %     \begin{aligned}
    %      & \Tr((M'\rho_U)) = \Tr((M_0\rho_U)) \\
    %      = & \sum_{i=0}^{d-1}\left[ b_i a\langle i|\left( U | 0 \rangle \langle 0 | U^\dagger + \frac{1-a}{d} I_d\right) |i\rangle \right] \\
    %      = & \sum_{i=0}^{d-1}\left[ ab_i\langle i| U | 0 \rangle \langle 0 | U^\dagger  |i\rangle + \frac{(1-a)b_i}{d} \right] \\
    %      = & a\sum_{i=0}^{d-1}\left[b_i\langle i| U | 0 \rangle \langle 0 | U^\dagger  |i\rangle\right] + \sum_{i=0}^{d-1}\left[\frac{(1-a)b_i}{d} \right] \\
    %     \end{aligned}
    % \end{equation}
    % From the property of the Haar random unitary matrix, we have
    % \begin{equation}
    %     \begin{aligned}
    %     & \mathbb{P}_{U \sim Haar}\left( \Tr(M'\rho_U )\geq \frac{\Tr(M'\rho_U)}{d} \right) \\
    %     = & \mathbb{P}_{U \sim Haar}\left( \Tr(M'\rho_U )\geq \frac{\Tr(M'\rho_U)}{d} \right)
    %     \end{aligned}
    % \end{equation}

    % From Equation \eqref{eq: square_upper_bound} \eqref{eq: sqrtd_lower_bound} we have
    % \begin{equation}
    %     \begin{aligned}
    %         & \mathbb{P}_{U \sim Haar}\left[\frac{\Tr(M)}{d} - \frac{2a}{\sqrt{d}}\leq\Tr\left(M_{\rho_U}\right) \leq \frac{\Tr(M)}{d} + \frac{2a}{\sqrt{d}}\right] \\
    %         \geq & 1 - \mathbb{P}_{U \sim Haar}\left[\frac{\Tr(M)}{d} - \frac{2a}{\sqrt{d}}\leq\Tr\left(M_{\rho_U}\right) \leq \frac{\Tr(M)}{d} + \frac{2a}{\sqrt{d}}\right]
    %     \end{aligned}
    % \end{equation}


\subsection{Proof of Theorem \ref{thm:PQSI lower bound}}
\label{subsec: thm lower bound}
    Let $p^{\mathcal{A}}_{e}(\mathcal{M},U)$ denote the error probability for algorithm $\mathcal{A}$ to solve the problem \ref{problem:PQDSI_special}, with specific unitary matrix $U$ and POVM $\mathcal{M}$, and  $M = \min_{M' \in \{M_0,M_1\}} \Tr(M')$. Then the error probability of $\mathcal{A}$ to solve the problem \ref{problem:PQDSI_special} satisfying
    \begin{equation}
        \label{eq1:proof_lower_bound}
        \begin{aligned}
            e^\mathcal{A}_N = & \int_{\mathcal{M} \in \mathcal{D}_{\mathcal{M}}} \int_{U \sim \Haar} p^\mathcal{A}_e(\mathcal{M},U) d\mathcal{M} dU \\
            \geq & \int_{\mathcal{M} \in \mathcal{D}_{\mathcal{M}}} \int_{U \sim \Haar} p^\mathcal{A}_e(\mathcal{M},U)\indicator\{U \in \mathbb{U}_\mathcal{M}\} d\mathcal{M} dU.
        \end{aligned}
    \end{equation}
    The first line corresponds to the deifintion of $e_N^\mathcal{A}$, the second line corresponds to that $p_{e}^\mathcal{A}(\mathcal{M},U) \geq 0$ and $\mathbb{U}_{\mathcal{M}}$ is a subset of unitary matrix.
    
    When the $i$-th quantum state is measured using $\mathcal{M}$, the process in which the output result is accepted by $M$ follows a Bernoulli distribution with parameter $\Tr(M\rho_U)$. From Lemma \ref{lem:larger_expectation} and the definition of $c(\mathcal{M}, U)$, we have 
    \begin{equation}
        \Tr(M\rho_i|U) = c(U,\mathcal{M})\alpha_i + \frac{\Tr(M)}{d}.    
    \end{equation}
    If $c(\mathcal{M},U) >0$, we need to find the Bernoulli distribution with the largest parameter where the parameter of the $i$-th Bernoulli distribution is $\Tr(M\rho_i|U) = c(\mathcal{M},U)\alpha_i + \frac{\Tr(M)}{d}$. Then we have
    \begin{equation}
        \begin{aligned}
            & \Tr(M\rho_i|U) - \Tr(M\rho_j|U) \\
            = & c(\mathcal{M},U)\alpha_i  - c(\mathcal{M},U)\alpha_j \\
            = & c(\mathcal{M},U) \left[\sqrt {\frac{dz_i -1}{d-1}} - \sqrt{\frac{dz_j -1}{d-1}}\right].
        \end{aligned}
    \end{equation}
    Since $\Tr(M) > 16$, for $U \in \mathbb{U}(\mathcal{M})$ we have
    \begin{equation}
        \begin{aligned}
            \Tr(M\rho_i|U) & = c(\mathcal{M},U)\alpha_i + \frac{\Tr(M)}{d} \\
            & \geq - \frac{2\sqrt{\Tr(M)}}{d} + \frac{Tr(M)}{d} \geq \frac{\Tr(M)}{2d}.
        \end{aligned}
    \end{equation}
    And since $M = \arg\min_{M'\in \{M_0,M-1\}} \Tr(M')$, we have $\Tr(M) \leq \frac{1}{2}$, then for $U \in \mathbb{U}$ we have
    \begin{equation}
         \Tr(M\rho_i|U) \in \left[\frac{\Tr(M)}{2d}, 1-\frac{\Tr(M)}{2d}\right],
    \end{equation}
    and 
    \begin{equation}
        1- \frac{\Tr(M)}{2d} \geq \frac{1}{2}.
    \end{equation}
    According to Theorem \ref{theorem:classical_BAI} and the definition of $\mathbb{U}_\mathcal{M}$, for $U \in \mathbb{U}_M$ we have
    \begin{equation}\label{eq:proof_lower_bound1}
        \begin{aligned}
             p_e^\mathcal{A}(\mathcal{M},U) \geq & \exp\left(-O\left( \frac{N}{\Omega\left(\frac{\Tr(M)}{d}\right)\left(1-\Omega\left(\frac{\Tr(M)}{d}\right)\right)}\min_i \frac{\left(\Tr(M\rho^\star|U) - \Tr(M\rho_{(i)}|U)\right)^2}{i}\right)\right) \\
            = & \exp\left(-O\left( \frac{N c^2(\mathcal{M},U)}{\Omega\left(\frac{\Tr(M)}{d}\right)}\min_i \frac{\left(\sqrt{dz_{i^\star} -1} - \sqrt{dz_{(i)} -1}\right)^2}{i(d-1)}\right)\right) \\
            = & \exp\left(-O\left( \frac{N c^2(\mathcal{M},U)}{\Omega\left(\frac{\Tr(M)}{d}\right)}\min_i \frac{dz_{i^\star}-1 + dz_{(i)}-1 -2\sqrt{(dz_{i^\star}-1)(dz_{(i)}-1)}}{i(d-1)}\right)\right) \\
            \geq & \exp\left(-O\left( \frac{N c^2(\mathcal{M},U)}{\Omega\left(\frac{\Tr(M)}{d}\right)}\min_i \frac{[dz_{i^\star}-1] - [dz_{(i)}-1]}{i(d-1)}\right)\right) \\
            = & \exp\left(-O\left( \frac{N c^2(\mathcal{M},U)}{\Omega\left(\frac{\Tr(M)}{d}\right)}\min_i \frac{d\Delta_{(i)}}{i(d-1)}\right)\right) \\
        \end{aligned} 
    \end{equation}
    According to the definition of $\mathbb{U}_\mathcal{M}$, for $U \in \mathbb{U}_M$ we have 
    \begin{equation} \label{eq:proof_lower_bound2}
        c(\mathcal{M},U) \in \left(-\frac{2\sqrt{\Tr(M)}}{d}, \frac{2\sqrt{\Tr(M)}}{d}\right).
    \end{equation}
    According to Equation \eqref{eq:proof_lower_bound1} and Equation \eqref{eq:proof_lower_bound2}, we have
    \begin{equation}
    \label{eq2:proof_lower_bound}
        \begin{aligned}
            p_e^\mathcal{A}(U,\mathcal{M}) \geq & \exp\left(-O\left( \frac{N c^2(U,\mathcal{M})}{\Omega\left(\frac{\Tr(M)}{d}\right)}\min_i \frac{d\Delta_{(i)}}{i(d-1)}\right)\right) \\
            \geq & \exp\left(-O\left( \frac{N \left(\frac{\sqrt{\Tr(M)}}{d}\right)^2}{\Omega\left(\frac{\Tr(M)}{d}\right)}\min_i \frac{d\Delta_{(i)}}{i(d-1)}\right)\right) \\
            \geq & \exp\left(-O\left( \frac{N}{d}\min_i \frac{\Delta_{(i)}}{i}\right)\right) \\
            = & \exp\left(-O\left( \frac{NH_1}{d}\right)\right).
        \end{aligned}
    \end{equation}
    Similarly, if $c(U,\mathcal{M}) <=0$, we have
    \begin{equation}
    \label{eq3:proof_lower_bound}
        p_e^\mathcal{A}(U,\mathcal{M}) \geq \exp\left(- O\left(\frac{NH_1}{d}\right)\right).
    \end{equation}
    According to Equation \eqref{eq1:proof_lower_bound}, \eqref{eq2:proof_lower_bound}, \eqref{eq3:proof_lower_bound} we have
    \begin{equation*}
        \begin{aligned}
            e^\mathcal{A}_N \geq & \int_{\mathcal{M} \in \mathcal{D}_{\mathcal{M}}} \int_{U \sim \Haar} p^\mathcal{A}_e(\mathcal{M},U)\indicator\{U \in \mathbb{U}_\mathcal{M}\} d\mathcal{M} dU \\
            \geq & \int_{\mathcal{M} \in \mathcal{D}_{\mathcal{M}}} \int_{U \sim \Haar} \exp\left(-O\left( \frac{NH_1}{d}\right)\right) \indicator\{U \in \mathbb{U}_\mathcal{M}\} d\mathcal{M} dU \\
            \geq & \exp\left(-O\left( \frac{NH_1}{d}\right)\right) \int_{U \sim \Haar}  \indicator\{U \in \mathbb{U}_\mathcal{M}\} d\mathcal{M} dU \\
            \geq & \exp\left(-O\left( \frac{NH_1}{d}\right)\right)\int_{\mathcal{M} \in \mathcal{D}_{\mathcal{M}}}  \frac{3}{4} d\mathcal{M} \\
            \geq & \exp\left(- O\left(\frac{NH_1}{d}\right)\right).\\            
        \end{aligned}
    \end{equation*}
    Then for any algorithm $\mathcal{A}_{\mathcal{D}}$ addressing Problem \ref{problem:PQDSI_special} cannot identify the purest random quantum state with an error probability lower than $\exp\left(-O\left( \frac{NH_1}{d}\right)\right)$. According to Lemma \ref{lem:problem transformation}, we can complete the proof.