\section{Preliminaries and Notations}
\label{sec: pre and model}

\begin{table}[t]
\caption{Description of commonly used-notations}
\label{table:notations}
\vskip 0.15in
\begin{center}
\begin{small}
\begin{tabular}{c|l}
\hline
Notation & Description \\
\hline
$n$  & the qubit number of the quantum state\\
\hline
$d$  & $d =2^n$ \\
\hline
$S$ & the set of the unknown quantum state \\
\hline
$K$ & $K = |S|$\\
%\hline
%$[K]$ & $[K] = \{1,\ldots,K\}$\\
\hline
$\rho_i$ & the $i$-th quantum state in $S$ \\
\hline
$\rho_{(i)}$ & the $i$-th purest quantum state in $S$ \\
\hline
$\rho^\star = \rho_{i^\star}$ & the purest quantum state in $S$ \\
\hline
$\Delta_i$ & $\Delta_i =\Tr(\rho_{i^\star}^2) - \Tr(\rho^2_{i})$\\
\hline
$\Delta_{(i)}$ & $\Delta_{(i)} =\Tr(\rho_{i^\star}^2) - \Tr(\rho^2_{(i)})$ \\
\hline
$\{|i \rangle\langle i|\}^{d-1}_{i=0}$ & a fixed orthogonal basis in $\mathbb{C}^{d \times d}$\\
\hline
$I_d$ & $d$-dimensional identity matrix \\
\hline
$\mathbb{U}(d)$ & the set of $d \times d$ unitary matrix\\
\hline
\end{tabular}
\end{small}
\end{center}
\vskip -0.1in
\end{table}

In this work, we will use Dirac's bra-ket notation, where $| v \rangle \in \mathbb{C}^d$ denotes a column vector and $\langle v| = | v \rangle^\dagger$. 
Specifically, for all $ i \in \{0,...,2^n-1\}$, let $| i \rangle$ denote a column vector whose $(i+1)$-th element is 1, and all other elements are 0. 
%\begin{equation*}
%    | i \rangle = (\ \underbrace{0,...,0}_{i \ \text{times} 0}\ ,\ 1,\underbrace{0,...,0}_{(n-1-i) \times 0})^T.    
%\end{equation*}
Then $\{|i\rangle \langle i | \}_{i=0}^{d-1}$ is a fixed orthogonal basis in $\mathbb{C}^{d \times d}$.

\paragraph{Quantum State.} Let $ d = 2^n $ denote the dimension of a $n$-qubit quantum system. An $ n $-qubit quantum state can be represented by a density matrix $ \rho \in \mathbb{C}^{d \times d} $, which is Hermitian and trace-1 positive semi-definite. In particular, an $n$-qubit pure quantum state can be represented by a unit vector $|\psi \rangle \in \mathbb{C}^d$. The purity of a quantum state $\rho$ is $\Tr(\rho^2)$.



% \textbf{Quantum measurement.} 量子测量,使用 $E$ 对量子态 $\rho$ 进行测量,会以 $\Tr(E\rho)$ 的概率输出 1,以 $1 - \Tr(E\rho)$ 输出为 0。

\paragraph{Quantum Measurement.} Quantum measurements are usually described by a Positive Operator-Valued Measure (POVM), which produces probabilistic outcomes. The formal definition of a POVM is as follows:

\begin{definition}[Positive Operator-valued measurement (POVM), see e.g. \cite{nielsen2010quantum}]
    An $n$-qubit positive operator-valued measurement $\mathcal{M}$ can be represented as a collection of positive semi-definite matrices $\mathcal{M} =\{M_m\}_m$, where $M_m \in \mathbb{C}^{d\times d}$ and $\sum_m M_m = I_d$. When using $\mathcal{M}$ to measure a quantum state $\rho$, the probability of outcome $m$ is $\Tr(M_m\rho)$, and the quantum state $\rho$ is destroyed.
\end{definition}

When the coherent measurement method is employed, the learner can perform entanglement measurements on quantum states $ \rho_1 \otimes \ldots \otimes \rho_m $. However, this approach necessitates the support of large-scale quantum devices and quantum memory, which are not feasible with current quantum technologies. Therefore, researching incoherent measurement methods applicable to NISQ-era quantum devices is of great significance. 

% \paragraph{Incoherent Measurements.} Intuitively, such an algorithm operates as follows: in each round $t \in \{1,...,N\}$, the learner chooses a quantum state in $S_\rho$ and measures its copy using a POVM $\mathcal{M}_t$, which could depend on the results of previous measurements. Then, the learner records the outcome and repeats this process in the following rounds. After having performed all $N$ samples and measurements, it must output a decision based on the sequence of outcomes it has received.



In this study, we aim to identify the purest quantum state from a set of unknown quantum states, achieving the highest probability through $ N $ measurements. For the purpose of simplicity, we will assume that there exists a unique optimal quantum state that is the purest in the set $ S $, denoted as $ \mu^\star = \mu_{i^\star} $. For $ i \neq i^\star $, we represent the purity difference between each non-optimal quantum state and the optimal quantum state using the following expression:
\begin{equation*}
    \Delta_i = \Tr\left(\rho_{i^\star}^{2}\right) - \Tr\left(\rho_{i}^{2}\right).
\end{equation*}

For $i \in \{1,...,K\}$, let $\rho_{(i)}$ be the $i$-th purest quantum state in $S$, then we have
\begin{equation*}
    \Tr\left(\rho_{i^\star}^{2}\right) = \Tr\left(\rho_{(1)}^{2}\right) > \Tr\left(\rho_{(2)}^{2}\right) \geq...\geq \Tr\left(\rho_{(K)}^{2}\right),
\end{equation*}
and
\begin{equation*}
    \Delta_{(2)} \leq \Delta_{(3)} \leq ... \leq \Delta_{(K)}.
\end{equation*}

Let $e_N$ denote the probability that the learner does not choose the purest quantum state in $S$ after $N$ samples and measurements, i.e., $e_N = \mathbb{P}\left(\rho' \notin \arg\max_\rho \Tr(\rho^2)\right)$. The learner's objective is to $\min e_N$.

We summarize key notations used throughout this paper in Table \ref{table:notations}.