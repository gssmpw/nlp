\section{PQSI with coherent measurement}
\label{sec: PQSI_co}
% In Section \ref{sec: PQSI_inco}, we present an algorithm for solving the purest quantum state identification with incoherent measurement and analyze the upper bound of its error probability. In this section, we investigate the problem of purest quantum state identification with coherent measurement. As quantum computing devices advance, the number of available qubits will gradually increase. 



In this section, we investigate the problem of purest quantum state identification with coherent measurement and propose an algorithm to solve the purest quantum state identification with coherent measurement based on the SWAP test.

The SWAP test is a quantum algorithm designed to assess the similarity between two quantum states. It offers a method for estimating these states' fidelity to quantify their closeness. We use the SWAP test in Figure \ref{fig:swap_test} to estimate the purity of the quantum state $\rho$ in the unknown quantum state set $S$. The measurement results in Figure \ref{fig:swap_test} have a probability of $\frac{1+ \Tr(\rho^2)}{2}$ for 0 and a probability of $\frac{1- \Tr(\rho^2)}{2}$ for 1. The details of the algorithm are shown in Algorithm \ref{alg:PQSI_coherent}.% , and we have the following results:

\begin{figure}[htb]
    \vskip 0in
    \centering
    \[
    \Qcircuit @C=.7em @R=1.8em {
    \lstick{\rho}      & \qw & \qw        & \multigate{1}{SWAP} & \qw       &  \qw    \\
    \lstick{\rho}      & \qw & \qw        & \ghost{SWAP}        & \qw       &  \qw     \\
    \lstick{|0\rangle} & \qw & \gate{H}   & \ctrl{-1}            & \gate{H}  & \meter{} \\
    }
    \]
    \caption{The SWAP test circuit.}
    \label{fig:swap_test}
    % \vskip -0.
\end{figure}



\begin{algorithm}[htb]
    \caption{SR-PQSI with coherent measurement}
    \label{alg:PQSI_coherent}
    \begin{algorithmic}
    \STATE {\bfseries Input:} Copy access to $S = \{\rho_1,...,\rho_K\}$, sample number $N$.
    \STATE {\bfseries Initialization:} Set $S_0 = \{\rho_1,...,\rho_K\}$, $\overline{\log}(K) = \frac{1}{2} + \sum_{i=2}^K \frac{1}{i}$, $N_0=0$ and $N_k = \left\lceil \frac{1}{2\overline{\log}(K)} \frac{N-K}{K+1-k}\right\rceil$, for $k \in \{1,...,K-1\}$. 
    \FOR{$i=1,..., K-1$}
        \STATE For all $\sigma \in S_{i-1}$, use SWAP test as Figure \ref{fig:swap_test} for $N_k - N_{k-1}$ rounds, and set the outputs as $x_{(\sigma, N_{k-1}+1)},..., x_{(\sigma, N_{k})}$.
        \STATE For all $\sigma \in S_{i-1}$, let $w(\sigma, k) = \frac{1}{N_k}\sum_{i=1}^{N_k} x_{(\sigma, i)}$.
        \STATE Let $S_k = S_{k-1} \setminus \arg \min_{\rho \in S_{k-1}} w(\rho,k). $
    \ENDFOR

    \STATE Output the quantum state $\rho'$ in $S_{k-1}$.

\end{algorithmic}
\end{algorithm}

\begin{theorem} \label{thm:alg_SRPQSI_coherent}
    The probability of error of Algorithm \ref{alg:PQSI_coherent} satisfies
    \begin{equation}
         e_N \leq \frac{K(K-1)}{2}\exp\left(- \frac{N H_2}{8\overline{\log}(K) } \right),
    \end{equation}
    where $H_2 = \min_{i \in \{2,...,K\}} \frac{\Delta^2_{(i)}}{i}$.
\end{theorem}
\begin{proof-sketch}
    The outputs of the SWAP test are within the range $[0, 1]$ and are independent. Thus, we can apply the Hoeffding inequality to complete the proof. Detailed explanations of the proof are included in Appendix \ref{sec:proof_alg_SRPQSI_coherent}.
\end{proof-sketch}

By comparing the conclusions of Theorem \ref{thm:alg_SRPQSI}, Lemma 2\ref{lem2:PQSI_alg}, and Theorem \ref{thm:alg_SRPQSI_coherent}, we can find that under a mild condition that the purity gap is not quite small, i.e., $\Delta_i \gg \frac{1}{d^2}$, the probability of finding the purest quantum state using coherent measurement is much higher than that using incoherent measurement. This result also reflects the importance of the quantum memory.