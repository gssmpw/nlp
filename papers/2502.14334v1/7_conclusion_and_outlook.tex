\section{Conclusion and Outlook}
\label{sec: conclusion}

In this study, we propose a pivotal problem in quantum testing, termed purest quantum state identification (PQSI). This framework is applicable to a wide range of quantum computing and quantum communication tasks. We develop two distinct algorithms to address this problem under different settings. When the learner utilizes incoherent (single-copy) measurement, the upper bound on the error probability of our algorithm is given by $ \exp\left(- \Omega\left(\frac{N H_1}{\log(K) 2^n }\right) \right) $. When the learner is allowed to use coherent (two-copy) measurement, the upper bound on the error probability is given by $ \exp\left(- \Omega\left(\frac{N H_2}{\log(K) }\right) \right) $. By examining the error probabilities of these two algorithms, we can discern the advantage of the coherent measurement in comparison to the incoherent one. Furthermore, we establish that for any algorithm utilizing a randomly fixed incoherent two-outcome POVM to solve the PQSI, its error probability is lower bounded by $ \exp\left( - O\left(\frac{N H_1}{2^n}\right)\right) $. Our results lay the groundwork for further investigations into the best quantum state identification. Several open questions remain to be addressed, including how to identify the nearest quantum state with minimal trace distance and how to achieve the best quantum state identification with fixed confidence.


% \paragraph{The nearest quantum state identification.} Our algorithm can be used to identify the closest quantum state to a given target quantum state in terms of the Hilbert-Schmidt (HS) distance among several quantum states. Besides HS distance, trace distance and fidelity are important metrics for assessing the distance between two quantum states. How should we design the algorithm to identify the quantum state with the minimum trace distance or the maximum fidelity to the target quantum state? What is the lower bound on the error probability for any algorithm to solve these problems?

% \paragraph{The best quantum state identification with fixed confidence.} We consider the setting in which the sampling and measurement time is limited, and the target is to minimize the error probability of identifying the purest state. If we are willing to find the best quantum state with probability at least $1-\delta$, what is the sampling complexity of this problem?

% \paragraph{The lower bound.} In Theorem \ref{thm:PQSI lower bound}, we demonstrate the error probability lower bound of any algorithm to solve the purest quantum state identification using 2-outcome randomly incoherent POVM. An open problem is to find the error probability lower bound of any algorithm to solve the purest quantum state identification using arbitrary POVM.

\section*{Acknowledgements}


This work was partially supported by the Innovation Program for Quantum Science and Technology (Grant No. 2021ZD0302901), and the National Natural Science Foundation of China (Grant No. 62102388).