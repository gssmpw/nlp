\section{Introduction}

Quantum computers possess the potential to solve specific problems with significantly greater efficiency than classical computers \cite{shor1994algorithms,daley2022practical}. However, as quantum computing currently resides in the Noisy Intermediate-Scale Quantum (NISQ) era, the number of quantum bits (qubits) in quantum devices is constrained \cite{lau2022nisq,preskill2018quantum}. Furthermore, these qubits are susceptible to noise during operations, resulting in a lack of complete control \cite{google2023suppressing}. Numerous algorithms have been developed to enhance the precision of quantum operations to maximize the utility of existing quantum devices \cite{ramezani2020machine}. Nonetheless, owing to variations in device implementation and environmental factors, the effectiveness of these algorithms differs markedly among various quantum devices \cite{gyongyosi2019survey}. Thus, identifying the optimal quantum device, quantum algorithm, or quantum channel presents a critical issue. To achieve this, we must measure quantum states in multiple quantum systems, analyze their properties, and select the best one with the most significant probability.

However, quantum learning is challenging \cite{wright2016learn, montanaro2013survey, aaronson2018online, chen2022tight, chen2022toward, bubeck2020entanglement, fawzi2023quantum}. The properties of superposition and entanglement in quantum systems result in exponential growth of the state space as the qubit number $ n $ increases \cite{gyongyosi2019survey}. Consequently, the complexity of measuring the state of a quantum system escalates rapidly with the qubit number. When aiming to capture all the information of an $ n $-qubit quantum system, the learner takes $ 2^{\Theta(3n)}$ samples and measurements \cite{chen2023does}. Therefore, identifying the best quantum system through tomography will incur substantial costs. Determining how to distribute limited measurements across multiple unknown quantum systems — and how to choose the right measurement basis — remains a key challenge in identifying the most suitable quantum system for a given task.


This paper investigates the problem of the purest quantum state identification (PQSI). Pure quantum states are critically important in quantum computing and quantum communication, serving as a fundamental requirement for various quantum algorithms. Furthermore, identifying the quantum state least affected by noise is a significant issue , especially in the NISQ era. The problem PQSI has a wide range of application scenarios, including quantum state preparation \cite{plesch2011quantum}, selection of quantum channels \cite{chen2023unitarity}, and initialization of quantum algorithms \cite{shor1994algorithms}. Addressing this problem will substantially contribute to the advancement of quantum technology. %

This paper makes the following key contributions:
    
\textbullet \ \textbf{Problem Model.} To the best of our knowledge, this is the first study dedicated to the best quantum state identification. Furthermore, in our problem model, while allocating sampling times to different quantum systems, we also need to select the basis for quantum measurement, which significantly increases the complexity of this problem. In this paper, we focus on the issue of the purest quantum state identification, i.e., identifying the purest quantum state from a set of available quantum states after measurements. Given the limited qubits in the NISQ era, current quantum devices may not be capable of supporting quantum computation at a large scale. In many quantum applications, the simultaneous replication of quantum states is infeasible \cite{yu2021sample}, and storing information about these states also requires exponential costs \cite{heshami2016quantum}. Therefore, utilizing coherent (multi-copy) measurement may not be feasible when measuring quantum states. We formalize the PQSI with incoherent measurement as follows:

\begin{problem}[Purest quantum state identification (PQSI) with incoherent measurement]\label{problem:PQSI_restatement}
Consider a set of $ K $ unknown quantum states, denoted as $ S = \{ \rho_1, \ldots, \rho_K \} $.  In each round $ t \in \{1, \ldots, N\} $, the learner chooses a quantum state $\sigma_t$ from the set $S$ and get a copy of it. Then the learner can choose a POVM $\mathcal{M}_t$ and uses it to measure $\sigma_t$. Upon completing $ N $ measurements, the learner selects a quantum state $ \rho' \in S$ based on the measurement outcomes. The objective of the learner is to maximize $ \mathbb{P}(\rho' \in \arg\max_{\rho \in S} \Tr(\rho^2)) $.
\end{problem}

In general, when coherent measurements are available, we formalize the problem of PQSI as follows:
    
\begin{problem}[The purest quantum state identification(PQSI)]
    \label{problem:PQSI}
    Suppose that there is a set of $K$ unknown $n$-qubit quantum states represented as $ S = \{ \rho_1, \ldots, \rho_K \} $. In each round $t \in \{1,...,N\}$, the learner chooses a quantum state $\sigma_t$ from the set $S$ and get a copy of it. Then, the learner can choose a subset $S_t$ (which may be an empty set) from quantum state copies he holds. Next, the learner can choose an entangled POVM $\mathcal{M}_t$ and uses it to measure the quantum state copies in $S_t$, and these state copies are destroyed. After completing all the measurements, the learner selects a quantum state $ \rho' $ based on the measurement outcomes. The objective of the learner is to maximize $\mathbb{P}\left(\rho' \in \arg\max_{\rho \in S} \Tr(\rho^2)\right)$.
\end{problem} 


\textbullet \  \textbf{Algorithms.} We developed two distinct algorithms to address this problem in different settings. To simplify the expression, for $i \in \{1,...,K\}$, let $\rho_{(i)}$ be the $i$-th purest quantum state in $S$ and
%\begin{equation*}
    $\Delta_{(i)} =\Tr\left(\rho^2_{(1)}\right) - \Tr\left(\rho^2_{(i)}\right).$
%\end{equation*}
When the coherent measurements are unavailable, we developed the algorithm SR-PQSI with incoherent measurement. We use the purity estimator to estimate the purity of quantum states and the successive reject method to allocate sampling times. The error probability of this algorithm satisfies the following theorem:

\begin{theorem}[Informal version of Theorem \ref{thm:alg_SRPQSI}] 
    There exists an algorithm that solves the problem of the purest quantum state identification with incoherent measurement whose error probability satisfies:
    \begin{equation}
        e_N \leq \exp\left(- \Omega\left(\frac{N H_1}{\log(K) 2^n }\right) \right),
    \end{equation}
    where $H_1 = \min_{i \in \{2,...,K\}} \frac{\Delta_{(i)}}{i}$.
\end{theorem}
    
Conversely, when two-copy measurements are accessible, we developed the algorithm SR-PQSI with coherent measurement. We use the SWAP test to estimate the purity of quantum states in this algorithm, and its error probability satisfies the following theorem:
    
\begin{theorem}[Informal version of Theorem \ref{thm:alg_SRPQSI_coherent}] 
    There exists an algorithm that solves the problem of the purest quantum state identification with coherent measurement whose error probability satisfies:
    \begin{equation}
        e_N \leq \exp\left(- \Omega\left(\frac{N H_2}{\log(K) }\right) \right),
    \end{equation}
    where $H_2 = \min_{i \in \{2,...,K\}} \frac{\Delta^2_{(i)}}{i}$.
\end{theorem}

By comparing the error probability upper bound of these two algorithms, we can identify the advantages of quantum memory.

\textbullet \  \textbf{Lower Bound.} Analyzing the complexity of problems related to quantum testing is usually challenging. These problems often use Haar unitary matrices to construct specific cases that are difficult to distinguish, reflecting the complexity of these issues \cite{anshu2022distributed, gong2024sample}. However, the representation-theoretic structure of Haar unitary matrices is complicated, which makes them difficult to analyze. When distinguishing quantum states from two alternative sets, previous work usually assumes that one of them is the maximally mixed state or pure state to reduce the difficulty of analysis. In our problem, the learner must frequently distinguish between quantum states with different purity, making this analysis method unsuitable. 

For incoherent measurements, we reduce the problem based on the properties of quantum states and quantum measurements. We demonstrate that when employing a fixed two-outcome POVM, any algorithm attempting to solve this problem satisfies the following theorem:

%will have an error probability of at least $ \exp\left( \Omega\left(- \frac{NH_1}{2^n}\right)\right)$. 
\begin{theorem}[Informal version of Theorem \ref{thm:PQSI lower bound}] 
    For any algorithm $\mathcal{A}$ to solve the purest quantum state identification using fixed 2-outcome randomly incoherent POVM, there exists a set of quantum states which makes the error probability of $\mathcal{A}$ satisfies
\begin{equation}
    e_N \geq \exp\left( -O\left(\frac{NH_1}{2^n}\right)\right),
\end{equation}
where $H_1 = \min_{i \in \{2,...,K\}} \frac{\Delta_{(i)}}{i}$.
\end{theorem}

\paragraph{Structure of the paper.} In Section \ref{sec: related_work}, we review relevant literature and discuss previous work related to our research. In Section \ref{sec: pre and model}, we provide preliminaries and notations used throughout this paper. In Section \ref{sec: PQSI_inco}, we give the algorithm to solve the purest quantum state identification with incoherent measurement. Section \ref{sec: lower_bound} gives the error probability lower bound for any algorithm $\mathcal{A}$ to solve the PQSI using two-outcome randomly incoherent POVM. We give the algorithm to solve the PQSI with coherent measurement in Section \ref{sec: PQSI_co}. Finally, we summarize the paper's content and present some related open problems in Section \ref{sec: conclusion}.
