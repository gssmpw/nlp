\section{Related Works}
\label{sec:related_work}

\textbf{Decomposition for COPs.} Despite advances in traditional solvers____, learning-guided solvers____ and neural solvers____ for COPs, scalability and adaptability to real-world complexities remain challenging. Various decomposition strategies have been explored by OR community, such as variable partitioning____, adaptive randomized decomposition (ARD)____, and sub-problem constraint relaxation____. More recently, machine learning has been exploited to guide the decomposition by selecting subproblems____ or to auto-regressively solve decomposed subproblems____, leading to notable improvements. However, they primarily emphasize \emph{spatial} or \emph{problem-structural} decomposition (e.g., for routing problems), which is not suitable for long-horizon time-structured COPs involving complex, interdependent decision variables and constraints spanning extended time horizons. This highlights the need for effective \emph{temporal} decomposition. We note such temporal decomposition can be orthogonal to other existing ones, and future work could combine them to improve scalability and flexibility.

\textbf{RHO for Long-horizon COPs.}
RHO is a temporal decomposition method originating from Model Predictive Control (MPC)____. It improves the scalability by dividing the time-structured problem into overlapping subproblems. While such overlap improves boundary decision-making, it can introduce redundant computations. Thus, many control and robotics studies leverage previous decisions to reduce the computations of the current subproblem. This is done, e.g., through hand-crafted methods, such as recording repeat computations____ or tightening primal and dual bounds____, and learning-based models that predict active constraints____ or solutions for discrete variables____. However, these methods are not tailored for long-horizon COPs.
Recently, RHO has been initially adapted for long-horizon COPs, with wide applications such as scheduling____, lot-sizing problems____, railway platform scheduling____, stochastic supply chain management____, and pickup and delivery with time windows____. However, they often overlook the redundancies, and none of them have integrated machine learning to address this issue, leaving a gap in accelerating RHO for COPs.



\textbf{FJSPs.} 
FJSP is a complex class of COPs that involve interdependent assignment and scheduling decisions over extended time horizons, making it more challenging than the basic JSP, which only addresses scheduling____. Recently, learning-based solvers have shown power to outperform traditional methods such as Constraint Programming____ and genetic algorithms____; for example, see____ for JSP and____ for FJSP. However, they are typically limited to small-scale instances (fewer than 200 operations) and struggle to scale to real-world, long-horizon scenarios. Although decomposition (e.g., the above ARD) has been explored for FJSP, their efficiency remains constrained. Moreover, these methods are offline, requiring full problem information, limiting their use in online setting. While some works have studied dynamic, online FJSP~\cite {luo2021real, lei2023large}, they are still restricted to small-scale experiments. Consequently, large research gaps remain in developing more scalable and flexible decomposition methods for long-horizon (and online) FJSP.