\section{Related work}
One of the earliest works on VWAP strategies is by ____, who proposed an optimal slicing strategy for VWAP execution. In scenarios where volume and volatility are uncorrelated, the optimal execution curve aligns with the relative market volume curve; deviations in the correlated case were also quantified. Building on this, ____ extended Konishi’s model by incorporating a constrained trading rate and an additional drift term—reflecting situations where traders or brokers, armed with sensitive information, may attempt to “beat” the VWAP. Their study also derived stylized facts regarding the expected relative volume (normalized by trade count), observing an S-shaped pattern throughout the trading day for equities, with higher-turnover stocks exhibiting less variability. These patterns echo the well-known U-shaped trading activity in equity markets. Later, ____ advanced this work by developing a continuously dynamic model, demonstrating that an optimal VWAP trading strategy is closely linked to intraday volume estimation. Volumes have also been used as covariates to explain other market variables such as price or volatility ____, typically using lower-frequency data rather than intraday figures. Additionally, studies such as ____ have focused on measuring overall market trading activity through volume metrics.

\medskip

In 2008, ____ proposed a method for estimating intraday volumes by decomposing them into two components—an approach refined in ____ by separating volumes into a market evolution component and a stock-specific pattern. The dynamic component was modeled using ARMA and SETAR techniques, resulting in significantly improved accuracy over static volume curve approaches. However, while earlier works like ____ and ____ attempted to capture the interrelation between price and volume within relatively simple frameworks, ____ and ____ focused exclusively on the volume component, as incorporating a realistic volume–price relationship renders overall modeling significantly more complex. An alternative execution method is presented in ____, where Dynamic VWAP (DVWAP) is introduced in contrast to the conventional Historical VWAP (HVWAP). It is argued that incoming news during execution can affect volumes in both directions—information that HVWAP ignores. Their DVWAP framework, built upon earlier methodologies (e.g., those of Bialkowski, Darolles, and Le Fol), not only leverages these improvements but further enhances VWAP performance. Other approaches include stochastic methods proposed by ____ and ____. As noted by Frei and Westray, these models derive an optimal trading rate that depends solely on volume curves, neglecting price dynamics due to the assumption that the price process behaves as an uncorrelated Brownian motion. In ____, the focus shifts partially to the strategic aspects of VWAP execution at high frequencies, addressing the dilemma faced by brokers between using aggressive versus passive orders. Finally, ____ introduced two novel contributions. First, they integrated both temporary and permanent market impact components—an important consideration for large institutional orders aiming to minimize market impact. Second, they proposed a method for pricing guaranteed VWAP services using a CARA utility function for the broker and indifference pricing. Unlike other approaches that solely strive for benchmark proximity, their framework emphasizes achieving optimal execution while managing risk. A common theme among these approaches is the predominant focus on modeling market volumes, often under the assumption that prices and volumes are independent—a simplification that does not fully capture market reality. Until recently, the latest advancements in machine learning—particularly deep learning—had not been fully leveraged for VWAP execution. Recent studies, however, have begun to explore these novel methods. For instance, ____ proposed the Macro-Meta-Micro Trader (M3T) architecture, which combines deep learning with hierarchical reinforcement learning to capture market patterns and execute orders across multiple temporal scales. Although this approach achieved an average cost saving of 1.16 basis points relative to an optimal baseline, its complexity may hinder practical implementation. Similarly, ____ developed a dual-level reinforcement learning strategy using Proximal Policy Optimization (PPO) to track daily cumulative VWAP. Their method integrates a Transformer model to capture the overall U-shaped volume pattern with an LSTM model for finer order distribution. Despite improved accuracy over previous reinforcement learning models, it remains heavily reliant on volume prediction. In ____, large stock order execution is simulated using LSTMs within the Almgren and Chriss framework. By leveraging cross-sectional data from multiple stocks to capture inter-stock dependencies, their approach consistently outperforms TWAP and VWAP-based strategies on S\&P100 data. However, the focus remains on minimizing transaction costs rather than directly optimizing VWAP execution. While recent studies underscore the potential of machine learning techniques for VWAP execution, many of these methods continue to depend on traditional volume curve prediction rather than directly targeting the VWAP execution objective.