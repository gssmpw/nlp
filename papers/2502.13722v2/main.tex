\documentclass[11pt,a4paper,english]{article}
\usepackage[utf8]{inputenc} % allow utf-8 input
\usepackage[T1]{fontenc}    % use 8-bit T1 fonts
\usepackage{babel}
\usepackage{blindtext}
\usepackage{hyperref}       % hyperlinks
\usepackage{url}            % simple URL typesetting
\usepackage{booktabs}       % professional-quality tables
\usepackage{amsfonts}       % blackboard math symbols
\usepackage{nicefrac}       % compact symbols for 1/2, etc.
\usepackage{microtype}      % microtypography
\usepackage{xcolor}         % colors
\usepackage{amsmath,amssymb}
\usepackage{wrapfig}
\usepackage{stfloats}
\usepackage{graphicx}
\usepackage{placeins}  
\usepackage{float}    
\usepackage{caption}
\usepackage{subcaption}
\usepackage{bm}
\usepackage{listings}
\usepackage{multicol}
\usepackage{multirow} 
\usepackage{pgffor}
\usepackage{cleveref}
\usepackage{geometry}
%\usepackage{ulem}

\newcommand{\e}{\mathbb{E}}
\newcommand{\lk}{\left[ }
\newcommand{\rk}{\right] }
\newcommand{\lc}{\left(}
\newcommand{\rc}{\right)}
\newcommand{\Eb}{\mathbb{E}}
\newcommand{\Cb}{\mathbb{C}}
\newcommand{\Rb}{\mathbb{R}}
\newcommand{\Zb}{\mathbb{Z}}
\newcommand{\Pb}{\mathbb{P}}
\newcommand{\Qb}{\mathbb{Q}}
\newcommand{\Nb}{\mathbb{N}}
\newcommand{\Fc}{\mathcal{F}}
\newcommand{\mc}{\mathcal{m}}
\newcommand{\Hc}{\mathcal{H}}
\newcommand{\id}{\mathbf{1}}
\newcommand{\veps}{\mathbf{\epsilon}}
\newcommand{\bb}{\mathbf{b}}
\newcommand{\zz}{\mathbf{z}}
\newcommand{\vv}{\mathbf{v}}
\newcommand{\xx}{\mathbf{x}}
\newcommand{\Ab}{\mathbf{A}}
\newcommand{\FF}{\mathbf{F}}
\newcommand{\Nc}{\mathcal{N}}
\newcommand{\Sc}{\mathcal{S}}
\newcommand{\Uc}{\mathcal{U}}
\newcommand{\Lc}{\mathcal{L}}

\newcommand{\myparagraph}[1]{\paragraph{#1}\mbox{}\\}

\NewDocumentCommand{\codeword}{v}{%
\texttt{\textcolor{blue}{#1}}%
}

\title{Deep Learning for VWAP Execution in Crypto Markets: \ Beyond the Volume Curve}

\author{%
    Rémi Genet \\
    \small DRM, Université Paris Dauphine - PSL \\
    \small Aplo \\
    \small remi.genet@dauphine.psl.eu \\
}


\begin{document}

\maketitle

\begin{abstract}
Volume-Weighted Average Price (VWAP) is arguably the most prevalent benchmark for trade execution as it provides an unbiased standard for comparing performance across market participants. However, achieving VWAP is inherently challenging due to its dependence on two dynamic factors—volumes and prices. Traditional approaches typically focus on forecasting the market’s volume curve, an assumption that may hold true under steady conditions but becomes suboptimal in more volatile environments or markets such as cryptocurrency where prediction error margins are higher. In this study, I propose a deep learning framework that directly optimizes the VWAP execution objective by bypassing the intermediate step of volume curve prediction. Leveraging automatic differentiation and custom loss functions, my method calibrates order allocation to minimize VWAP slippage, thereby fully addressing the complexities of the execution problem. My results demonstrate that this direct optimization approach consistently achieves lower VWAP slippage compared to conventional methods, even when utilizing a naive linear model presented in \cite{genet2024tln}.They validate the observation that strategies optimized for VWAP performance tend to diverge from accurate volume curve predictions and thus underscore the advantage of directly modeling the execution objective. This research contributes a more efficient and robust framework for VWAP execution in volatile markets, illustrating the potential of deep learning in complex financial systems where direct objective optimization is crucial. Although my empirical analysis focuses on cryptocurrency markets, the underlying principles of the framework are readily applicable to other asset classes such as equities.
\end{abstract}


\newpage

\section{Introduction}\label{section1}
In a rapidly evolving financial landscape, Volume-Weighted Average Price (VWAP) strategies have become a cornerstone for executing high-volume trades, as they aim to minimize market impact and reduce execution costs. VWAP is widely regarded for its fair and neutral calibration, making it an industry standard for comparing performance across market participants \cite{TotalCostOfTransactions}. With the rise of systematic trading, execution strategies have received increasing attention; more specifically, significant institutional trades are now executed using algorithms \cite{Mackenzie}. Despite this, academic research has predominantly focused on Implementation Shortfall (IS) \cite{perold1988implementation} strategies, while VWAP, despite its extensive industrial use and reputation as a robust benchmark (particularly for large orders) \cite{Madhavan2002}, has not been as thoroughly explored. The objective of this study is to propose an alternative VWAP execution approach that moves beyond methods relying solely on forecasting the market volume curve. Indeed, while early approaches in the literature were able to incorporate the price–volume relationship under simplified modeling assumptions, more recent works tend to focus exclusively on the volume curve. This shift arises because more realistic and higher-performing volume models render the simultaneous representation of the volume–volatility (and hence price–volume) relationship considerably more complex. The purpose of this paper is to introduce a method that easily overcomes this limitation, thereby more accurately capturing the full VWAP decision problem. To achieve this, I leverage deep learning as a generalized function calibrator. This approach allows the direct representation of the execution objective in a single step—bypassing the conventional two-step process of first predicting the volume curve and then allocating orders. Traditional econometric and machine learning tools typically rely on this two-step resolution, which can be suboptimal in the presence of prediction errors. The primary challenge in VWAP execution is to optimally allocate trading volumes over the execution horizon—a problem for which historical data does not offer a direct solution. Existing literature often simplifies this challenge by assuming that matching the market's volume curve is ideal. However, this assumption overlooks the fact that perfect volume predictions are unattainable, especially in the volatile cryptocurrency market. Recognizing and incorporating these inevitable prediction errors into the allocation strategy is essential for achieving a price closer to the market VWAP. Moreover, while conventional models can be calibrated to forecast the volume curve from historical data, directly calibrating an execution strategy is not straightforward because the optimal allocation is unknown in advance. Deep learning offers a significant advantage here. Unlike frameworks that minimize the error between predicted and observed data points, deep learning enables the implementation and optimization of any loss function—thanks to automatic differentiation. The central idea of this paper is to build a model that ingests market data, produces an allocation curve for the order, and is trained to minimize the absolute or quadratic error between the achieved execution price and the market VWAP. Finally, although some studies propose dynamic approaches that update predictions throughout an order’s lifetime, I opt for a static framework for two reasons. First, I aim to demonstrate that even a relatively simple network—such as the naive linear model I use—can enhance performance when directly representing the trading strategy. Second, incorporating real-time dynamics introduces additional technological complexities that are beyond the scope of this study. My goal is to propose a simple yet effective method for VWAP execution that fully captures the decision problem and leverages deep learning's flexibility to directly optimize the trader’s true objective.

\section{Related work}

One of the earliest works on VWAP strategies is by \cite{Konishi}, who proposed an optimal slicing strategy for VWAP execution. In scenarios where volume and volatility are uncorrelated, the optimal execution curve aligns with the relative market volume curve; deviations in the correlated case were also quantified. Building on this, \cite{Culoch2007} extended Konishi’s model by incorporating a constrained trading rate and an additional drift term—reflecting situations where traders or brokers, armed with sensitive information, may attempt to “beat” the VWAP. Their study also derived stylized facts regarding the expected relative volume (normalized by trade count), observing an S-shaped pattern throughout the trading day for equities, with higher-turnover stocks exhibiting less variability. These patterns echo the well-known U-shaped trading activity in equity markets. Later, \cite{Culoch2012} advanced this work by developing a continuously dynamic model, demonstrating that an optimal VWAP trading strategy is closely linked to intraday volume estimation. Volumes have also been used as covariates to explain other market variables such as price or volatility \cite{Easley1987, Foster, Tauchen1983, Karpoff1987}, typically using lower-frequency data rather than intraday figures. Additionally, studies such as \cite{Gourieroux} have focused on measuring overall market trading activity through volume metrics.

\medskip

In 2008, \cite{LeFol2006} proposed a method for estimating intraday volumes by decomposing them into two components—an approach refined in \cite{LeFol2012} by separating volumes into a market evolution component and a stock-specific pattern. The dynamic component was modeled using ARMA and SETAR techniques, resulting in significantly improved accuracy over static volume curve approaches. However, while earlier works like \cite{Konishi} and \cite{Culoch2007} attempted to capture the interrelation between price and volume within relatively simple frameworks, \cite{LeFol2006} and \cite{LeFol2012} focused exclusively on the volume component, as incorporating a realistic volume–price relationship renders overall modeling significantly more complex. An alternative execution method is presented in \cite{Humphery}, where Dynamic VWAP (DVWAP) is introduced in contrast to the conventional Historical VWAP (HVWAP). It is argued that incoming news during execution can affect volumes in both directions—information that HVWAP ignores. Their DVWAP framework, built upon earlier methodologies (e.g., those of Bialkowski, Darolles, and Le Fol), not only leverages these improvements but further enhances VWAP performance. Other approaches include stochastic methods proposed by \cite{bouchard} and \cite{frei}. As noted by Frei and Westray, these models derive an optimal trading rate that depends solely on volume curves, neglecting price dynamics due to the assumption that the price process behaves as an uncorrelated Brownian motion. In \cite{Tianhui}, the focus shifts partially to the strategic aspects of VWAP execution at high frequencies, addressing the dilemma faced by brokers between using aggressive versus passive orders. Finally, \cite{Gueant} introduced two novel contributions. First, they integrated both temporary and permanent market impact components—an important consideration for large institutional orders aiming to minimize market impact. Second, they proposed a method for pricing guaranteed VWAP services using a CARA utility function for the broker and indifference pricing. Unlike other approaches that solely strive for benchmark proximity, their framework emphasizes achieving optimal execution while managing risk. A common theme among these approaches is the predominant focus on modeling market volumes, often under the assumption that prices and volumes are independent—a simplification that does not fully capture market reality. Until recently, the latest advancements in machine learning—particularly deep learning—had not been fully leveraged for VWAP execution. Recent studies, however, have begun to explore these novel methods. For instance, \cite{li2022hierarchical} proposed the Macro-Meta-Micro Trader (M3T) architecture, which combines deep learning with hierarchical reinforcement learning to capture market patterns and execute orders across multiple temporal scales. Although this approach achieved an average cost saving of 1.16 basis points relative to an optimal baseline, its complexity may hinder practical implementation. Similarly, \cite{kim2023adaptive} developed a dual-level reinforcement learning strategy using Proximal Policy Optimization (PPO) to track daily cumulative VWAP. Their method integrates a Transformer model to capture the overall U-shaped volume pattern with an LSTM model for finer order distribution. Despite improved accuracy over previous reinforcement learning models, it remains heavily reliant on volume prediction. In \cite{papanicolaou2023optimal}, large stock order execution is simulated using LSTMs within the Almgren and Chriss framework. By leveraging cross-sectional data from multiple stocks to capture inter-stock dependencies, their approach consistently outperforms TWAP and VWAP-based strategies on S\&P100 data. However, the focus remains on minimizing transaction costs rather than directly optimizing VWAP execution. While recent studies underscore the potential of machine learning techniques for VWAP execution, many of these methods continue to depend on traditional volume curve prediction rather than directly targeting the VWAP execution objective.

\section{Defining the VWAP Problem}

\medskip

The \emph{Volume Weighted Average Price (VWAP)} is a trading benchmark that reflects the average price at which a security is traded throughout the day, weighted by volume. It provides insights into both the trend and the value of a security.
\begin{itemize}
  %\item \textbf{VWAP} stands for \emph{Volume Weighted Average Price}.
  \item Its basic formulation is:
  \begin{equation}
    \text{VWAP} = \frac{\text{Total Traded Value}}{\text{Total Traded Volume}}.
  \end{equation}
  \item More explicitly, VWAP is computed as:
  \begin{equation}
    \text{VWAP} = \frac{\sum_{i} \left(p_i \times v_i\right)}{\sum_{i} v_i},
  \end{equation}
where \( p_i \) denotes the price of the \(i\)th trade and \( v_i \) the corresponding volume.
\end{itemize}

\subsection*{Defining the Execution Problem}
The objective is to execute trades so that the average execution price aligns as closely as possible with the market VWAP over a given period \(T\). The total order volume is denoted by \(v\) and the total market volume over the same period by \(V\).

The average execution price is defined as:
\begin{equation}
    P = \frac{\sum_{j=1}^{n}{ p_j \times v_j}}{v}, \quad \text{with } v = \sum_{j=1}^{n}{v_j},
\end{equation}
assuming that \(n\) transactions are executed during period \(T\).

Similarly, the market VWAP is defined as:
\begin{equation}
    \text{VWAP} = \frac{\sum_{i=1}^{N}{ p_i \times V_i}}{V}, \quad \text{with } V = \sum_{i=1}^{N}{V_i},
\end{equation}
where \(N\) represents the total number of market transactions (including the order's own trades).

\subsection*{Addressing Slippage}
Absolute slippage is defined as:
\begin{equation}
    S = \left| P - \text{VWAP} \right|.
\end{equation}
Minimizing \(S\) is challenging because the transaction space is non-standard—trades occur as discrete events at irregular intervals rather than in uniform time or volume increments. To simplify the analysis, the overall period is divided into \(T\) smaller intervals. Within each interval, the following definitions apply:
\begin{equation}
    \text{VWAP}_{T} = \frac{\sum_{t=1}^{T}{ \text{VWAP}_t \times V_t}}{V}, \quad P_{T} = \frac{\sum_{t=1}^{T}{ P_t \times v_t}}{v},
\end{equation}
where:
\begin{itemize}
  \item \(V_t\) is the market volume in interval \(t\),
  \item \(v_t\) is the executed volume in interval \(t\),
  \item \(\text{VWAP}_t\) and \(P_t\) denote the market VWAP and the average execution price in interval \(t\), respectively.
\end{itemize}
This segmentation permits a more detailed comparison between \(P_T\) and \(\text{VWAP}_T\).

\subsection*{Refining the Problem Definition}
For further analysis, normalized volume proportions are introduced:
\[
Q_t = \frac{V_t}{V} \quad \text{and} \quad q_t = \frac{v_t}{v},
\]
so that:
\[
\sum_{t=1}^{T} Q_t = \sum_{t=1}^{T} q_t = 1.
\]
Thus, the execution price and market VWAP can be rewritten as:
\[
P_T = \sum_{t=1}^{T} P_t \, q_t \quad \text{and} \quad \text{VWAP}_T = \sum_{t=1}^{T} \text{VWAP}_t \, Q_t.
\]
Accordingly, the slippage becomes:
\begin{equation} \label{eq:static_slippage1}
    S_T = \left| P_T - \text{VWAP}_T \right| = \left| \sum_{t=1}^{T} \left( P_t \, q_t - \text{VWAP}_t \, Q_t \right) \right|.
\end{equation}
A common strategy to decompose the difference is to add and subtract a common term inside the summation. In this case, \(\text{VWAP}_t \, q_t\) is added and subtracted for each interval:
\begin{equation}
P_t \, q_t - \text{VWAP}_t \, Q_t = \underbrace{\left(P_t \, q_t - \text{VWAP}_t \, q_t\right)}_{=(P_t-\text{VWAP}_t)q_t} + \underbrace{\left(\text{VWAP}_t \, q_t - \text{VWAP}_t \, Q_t\right)}_{=\text{VWAP}_t\,(q_t-Q_t)}.
\end{equation}
Substituting this into \eqref{eq:static_slippage1} yields:
\begin{equation}
    S_T = \left| \sum_{t=1}^{T} \left[(P_t-\text{VWAP}_t)q_t + \text{VWAP}_t\,(q_t-Q_t)\right] \right|.
\end{equation}
Since the absolute value of a sum is generally not equal to the sum of the absolute values, the triangle inequality is invoked:
\[
\left|\sum_{t=1}^{T} a_t\right| \le \sum_{t=1}^{T} \left|a_t\right|.
\]
Thus,
\begin{equation} \label{eq:static_slippage-bound}
    S_T \le \sum_{t=1}^{T} \left| (P_t-\text{VWAP}_t)q_t + \text{VWAP}_t\,(q_t-Q_t) \right|.
\end{equation}
Furthermore, applying the triangle inequality to each individual term results in:
\begin{equation}
    \begin{aligned}
    &\left| (P_t-\text{VWAP}_t)q_t + \text{VWAP}_t\,(q_t-Q_t) \right| \\
    &\qquad \leq \left| (P_t-\text{VWAP}_t)q_t \right| + \left| \text{VWAP}_t\,(q_t-Q_t) \right|.
    \end{aligned}
\end{equation}
Summing over \(t\) gives:
\begin{equation}
    S_T \le \sum_{t=1}^{T} \left| (P_t-\text{VWAP}_t)q_t \right| + \sum_{t=1}^{T} \left| \text{VWAP}_t\,(q_t-Q_t) \right|.
\end{equation}
In this decomposition:
\begin{itemize}
    \item \(\left| (P_t-\text{VWAP}_t)q_t \right|\) represents the deviation in price weighted by the executed volume proportion.
    \item \(\left| \text{VWAP}_t\,(q_t-Q_t) \right|\) captures the impact of discrepancies between the executed volume allocation and the market’s volume distribution.
\end{itemize}

\subsection*{Methodology --- Delineating the Problems}
In the pursuit of minimizing slippage, two distinct challenges are identified.

\subsubsection*{Problem 1: Price Deviation Minimization}
This aspect focuses on minimizing:
\[
\sum_{t=1}^{T} \left| (P_t-\text{VWAP}_t)q_t \right|.
\]
Key considerations include:
\begin{itemize}
    \item Execution quality within each interval is paramount; the goal is to achieve prices as close as possible to the market VWAP.
    \item With finer time intervals, price deviations are expected to be smaller, and deviations across bins may partially offset one another.
    \item This challenge is primarily related to market microstructure and the quality of executions.
\end{itemize}

\subsubsection*{Problem 2: Volume Allocation Optimization}
This challenge is characterized by:
\begin{equation} \label{eq:static_volumeOptimization}
    \sum_{t=1}^{T} \left| \text{VWAP}_t\,(q_t-Q_t) \right|.
\end{equation}
Key considerations include:
\begin{itemize}
    \item Although \(\text{VWAP}_t\) cannot be controlled, the executed volume allocation \(q_t\) can be adjusted.
    \item Traditional methods focus on accurately predicting \(Q_t\), the market volume profile; however, given the inherent uncertainty in market dynamics, this may not be optimal.
    \item The strategy is to design a \(q_t\) allocation that accounts for the noisy nature of the \(Q_t\) process, adapting to market conditions and prediction errors.
    \item For instance, during periods of low volatility, it may be advisable to execute less than the market proportion \(Q_t\) to mitigate the risk of sudden volatility spikes.
    \item The aim is to develop a robust volume allocation strategy that minimizes slippage even when predictions of \(Q_t\) are imperfect.
\end{itemize}

\subsubsection*{Choosing the Appropriate Focus}
While the primary aim is to minimize absolute slippage, it is also essential to consider strategies that not only meet the benchmark but also potentially outperform it.

\subsubsection*{Price Difference Focus}
\begin{itemize}
    \item Since the weights \(q_t\) and \(Q_t\) are non-negative, achieving the most favorable prices in each interval is crucial.
    \item Opportunities for value addition may arise, particularly through the mitigation of trading costs such as the spread.
\end{itemize}

\subsubsection*{Volume Allocation Focus}
\begin{itemize}
    \item Given the challenges in precisely forecasting \(\text{VWAP}_t\), emphasis is placed on risk reduction through effective volume allocation.
    \item The strategy aims to minimize the term \(\left| \text{VWAP}_t\,(q_t-Q_t) \right|\) by aligning the executed volume with the market’s volume profile.
    \item This involves leveraging insights into future market volumes and understanding their interaction with market volatility.
    \item The focus is on developing the optimal \(q_t\) strategy for volume allocation over larger time intervals, thereby reducing the risk associated with VWAP execution.
\end{itemize}

\medskip

In summary, by rigorously decomposing the slippage \(S_T\) and carefully applying the triangle inequality, the problem is separated into two distinct components: execution quality (price differences) and volume allocation discrepancies. This detailed derivation provides a solid foundation for subsequent analysis and strategy development.


\section{A fixed optimal allocation curve approach}\label{section3}
To demonstrate the validity of allocation strategies that are not based solely on following the volume curve, an analysis was conducted in which an optimal allocation curve was calibrated using only historical data, without incorporating current market information. In this experiment, the allocation curve is represented by a simple vector of weights that determines the fraction of the total order executed in each time bin. Without any additional market inputs, the expected volume curve is a uniform distribution over the execution period—precisely as conventional wisdom would predict—so any deviation from this flat pattern in the optimal allocation would challenge that notion.

\subsection{Methodology}

Historical data from Binance perpetual futures contracts for Bitcoin (BTC), Ethereum (ETH), and Cardano (ADA) covering the period from contract inception until July 1, 2024, were employed. The data, recorded at an hourly frequency, were divided into training and testing sets in an 80/20 split, with the temporal order preserved to ensure a realistic evaluation. The VWAP execution problem was formulated as the optimization of a vector of allocation weights, \(q_t\) (for \(t = 1, \dots, T\) with \(T=8\) bins), by minimizing one of three loss functions. The first is the Quadratic VWAP Loss, defined as
\begin{equation} \label{eq:static_quad_loss}
L_Q = \mathbb{E}\!\left[\left(\frac{\text{VWAP}_{\text{achieved}}}{\text{VWAP}_{\text{market}}} - 1\right)^2\right],
\end{equation}
the second is the Absolute VWAP Loss,
\begin{equation} \label{eq:static_abs_loss}
L_A = \mathbb{E}\!\left[\left|\frac{\text{VWAP}_{\text{achieved}}}{\text{VWAP}_{\text{market}}} - 1\right|\right],
\end{equation}
and the third is the Volume Curve Loss,
\begin{equation} \label{eq:static_vol_curve_loss}
L_V = \mathbb{E}\!\left[\sum_{t=1}^{T} \left(\frac{v_t}{\sum_{s=1}^{T} v_s} - \frac{V_t}{\sum_{s=1}^{T} V_s}\right)^2\right].
\end{equation}
The achieved and market VWAP are defined respectively as
\begin{align}
\text{VWAP}_{\text{achieved}} &= \frac{\sum_{t=1}^{T} v_t\, p_t}{\sum_{t=1}^{T} v_t}, \label{eq:static_achieved_vwap} \\
\text{VWAP}_{\text{market}} &= \frac{\sum_{t=1}^{T} V_t\, p_t}{\sum_{t=1}^{T} V_t}. \label{eq:static_market_vwap}
\end{align}
Here, \(v_t\) represents the allocated volume in bin \(t\), \(V_t\) is the market volume in bin \(t\), and \(p_t\) is the VWAP price in bin \(t\). To solve the optimization problem, three different methods were employed. Sequential Least Squares Programming (SLSQP) was used as a local optimization technique with 1000 restarts from random initial guesses to mitigate the risk of converging to local minima. Basin-Hopping, a global optimization algorithm that combines local searches with random perturbations, was also employed and performed 10 times. In addition, Differential Evolution, an evolutionary algorithm known for its robustness in finding global optima, was executed once. Performance metrics for each asset, loss function, and optimization method are summarized in Table~\ref{tab:static_fix_curve_vwap_results}, while Figure~\ref{fig:static_fixed_optimal_allocation} illustrates the optimal allocation curves derived from these optimizations.

\begin{table}[H]
    \centering
    \caption{VWAP Optimization Results}
    \label{tab:static_fix_curve_vwap_results}
    \small
    \resizebox{\textwidth}{!}{%
        \begin{tabular}{llllcccc}
        \toprule
        \textbf{Asset} & \textbf{Optimization} & \textbf{Method} & \textbf{Abs VWAP Loss} & \textbf{Quad VWAP Loss} & \textbf{Vol Curve Loss} & \textbf{R2 Vol Curve} & \textbf{Opt Time (s)} \\
         & \textbf{Function} &  & \textbf{(1e2)} & \textbf{(1e4)} & \textbf{(1e2)} &  & \\
        \midrule
        \multirow{9}{*}{ADA} & \multirow{3}{*}{absolute\_vwap\_loss} & SLSQP & 0.161262 & 0.103157 & 0.509109 & -0.347896 & 34.626976 \\
         &  & basinhopping & 0.161266 & 0.099737 & 0.556507 & -0.473385 & 55.531589 \\
         &  & DE & 0.159714 & 0.103534 & 0.463983 & -0.228422 & 2.072001 \\
        \cline{2-8}
         & \multirow{3}{*}{quadratic\_vwap\_loss} & SLSQP & 0.161468 & 0.097612 & 0.615348 & -0.629170 & 11.889972 \\
         &  & basinhopping & 0.187822 & 0.102028 & 0.963292 & -1.550373 & 27.556498 \\
         &  & DE & 0.164090 & 0.092467 & 0.576554 & -0.526461 & 3.021690 \\
        \cline{2-8}
         & \multirow{3}{*}{volume\_curve\_loss} & SLSQP & 0.182052 & 0.153098 & 0.379948 & -0.005934 & 25.458261 \\
         &  & basinhopping & 0.184498 & 0.156269 & 0.382820 & -0.013539 & 36.394290 \\
         &  & DE & 0.182067 & 0.152669 & 0.377981 & -0.000728 & 1.807036 \\
        \midrule
        \multirow{9}{*}{BTC} & \multirow{3}{*}{absolute\_vwap\_loss} & SLSQP & 0.105354 & 0.037911 & 0.855709 & -0.628073 & 34.237416 \\
         &  & basinhopping & 0.104696 & 0.037538 & 0.664384 & -0.264057 & 48.594292 \\
         &  & DE & 0.105952 & 0.040566 & 0.636615 & -0.211224 & 2.709536 \\
        \cline{2-8}
         & \multirow{3}{*}{quadratic\_vwap\_loss} & SLSQP & 0.105112 & 0.037638 & 0.728556 & -0.386152 & 12.000814 \\
         &  & basinhopping & 0.108195 & 0.037482 & 1.308381 & -1.489326 & 26.424249 \\
         &  & DE & 0.103572 & 0.035020 & 0.780900 & -0.485740 & 4.301013 \\
        \cline{2-8}
         & \multirow{3}{*}{volume\_curve\_loss} & SLSQP & 0.126713 & 0.061470 & 0.528446 & -0.005422 & 21.426937 \\
         &  & basinhopping & 0.122180 & 0.057882 & 0.535264 & -0.018393 & 34.317264 \\
         &  & DE & 0.124038 & 0.058830 & 0.525396 & 0.000381 & 1.163200 \\
        \midrule
        \multirow{9}{*}{ETH} & \multirow{3}{*}{absolute\_vwap\_loss} & SLSQP & 0.121005 & 0.053397 & 0.696746 & -0.418203 & 36.342283 \\
         &  & basinhopping & 0.125103 & 0.059826 & 0.680083 & -0.384285 & 52.675163 \\
         &  & DE & 0.120213 & 0.053212 & 0.622947 & -0.267986 & 2.533839 \\
        \cline{2-8}
         & \multirow{3}{*}{quadratic\_vwap\_loss} & SLSQP & 0.120978 & 0.052524 & 0.709165 & -0.443481 & 11.941791 \\
         &  & basinhopping & 0.128118 & 0.052153 & 0.881157 & -0.793564 & 27.980689 \\
         &  & DE & 0.119349 & 0.049538 & 0.678544 & -0.381152 & 3.884477 \\
        \cline{2-8}
         & \multirow{3}{*}{volume\_curve\_loss} & SLSQP & 0.147338 & 0.084255 & 0.493581 & -0.004668 & 25.072747 \\
         &  & basinhopping & 0.143035 & 0.080341 & 0.491345 & -0.000117 & 33.726844 \\
         &  & DE & 0.144048 & 0.081395 & 0.491395 & -0.000218 & 2.017230 \\
        \bottomrule
        \end{tabular}
    }
    \caption*{Note: Abs VWAP Loss and Vol Curve Loss are presented in $10^2$ scale, Quad VWAP Loss is presented in $10^4$ scale as per the original data.}
\end{table}

The analysis reveals several important findings. Allocation curves obtained by minimizing the absolute and quadratic VWAP losses clearly deviate from a uniform allocation, while the optimization based on the volume curve loss yields a nearly flat allocation. This flat allocation is expected since, in the absence of market information, the average volume in each time bin is equal. However, although the flat allocation perfectly tracks the average volume curve (as evidenced by an \(R^2\) near zero), it performs worse in terms of VWAP execution compared to the non-uniform allocations derived from the VWAP-based losses. Furthermore, among the assets studied, Bitcoin consistently exhibits the lowest VWAP losses, reflecting its relative stability, whereas Cardano—being more volatile—shows higher losses. In terms of optimization methods, Differential Evolution consistently produced competitive results across assets and loss functions, while SLSQP, despite numerous restarts, often converged to suboptimal solutions due to the complexity of the optimization landscape. Basin-Hopping displayed variable performance that was generally less consistent.

\begin{figure}[!htb]
    \centering
    \includegraphics[width=0.9\textwidth]{figures/optimal_allocation_curves.jpg}
    \caption{Optimal Allocation based on optimization methods and asset}
    \label{fig:static_fixed_optimal_allocation}
\end{figure}

\subsection{Discussion}

This experiment is particularly revealing because it involves optimizing a vector of allocation weights using only historical data without current market information. Consequently, an optimization based solely on the volume curve loss produces a flat, uniform allocation, which is expected when no additional information beyond the historical average is available. In contrast, the allocation curves optimized using the absolute and quadratic VWAP losses diverge significantly from uniformity, exhibiting a pronounced end-loading behavior in some cases. This divergence indicates that even in the absence of direct market signals, an allocation strategy that minimizes VWAP slippage is inherently non-uniform. This finding challenges the traditional approach of following the expected volume curve and strongly supports the use of direct VWAP loss optimization in trading execution strategies. Although the empirically derived allocation curves deviate markedly from the conventional volume curve, this outcome is not entirely surprising when considering simplified frameworks that account for price--volume interactions. For instance, Konishi \cite{Konishi} shows that the optimal fraction \(x^*(t)\) allocated to bin \(t\) can be approximated by
\begin{equation}\label{eq:static_konishi_approx}
x^*(t) \;\approx\; \frac{\mathbb{E}\bigl[\sigma(t,V)^2 \, X(t)\bigr]}{\mathbb{E}\bigl[\sigma(t,V)^2\bigr]},
\end{equation}
or equivalently, as rewritten by McCulloch and Kazakov \cite{Culoch2007}:
\begin{equation}\label{eq:static_cullocheq}
x_t 
= \frac{\mathbb{E}[X_t \sigma_t^2]}{\mathbb{E}[\sigma_t^2]} 
= \mathbb{E}[X_t] 
+ \frac{\mathrm{Cov}[X_t, \sigma_t^2]}{\mathbb{E}[\sigma_t^2]}.
\end{equation}
Thus, a sufficiently large positive covariance term can shift the solution away from the market's average volume curve, underscoring the role of volatility in shaping optimal allocations. These results highlight the importance of the price--volume relationship and expose the limitations of approaches based solely on replicating the historical volume curve. Although more complex models can be developed to jointly account for both price and volume dynamics, the present analysis demonstrates that even a relatively simple (and naive) linear model can yield meaningful improvements in VWAP execution by directly optimizing the VWAP objective.

\section{A Deep-Learning Model for VWAP Allocation}
\subsection{Framework}

Having demonstrated that the traditional volume curve approach may not be optimal for allocating volumes over time, deep learning is employed as a powerful means to represent and optimize the VWAP allocation problem. Deep learning affords the flexibility to directly optimize the specific objective—minimizing the deviation between the achieved execution price and the market VWAP—by leveraging automatic differentiation. In this framework, the model accepts sequential inputs comprising multiple features over a designated historical lookback period and produces an allocation curve for future time intervals. The key innovation is the use of a custom loss function (either absolute or quadratic VWAP loss) that captures the true execution error, thereby enabling the optimization of the strategy in a single step rather than relying on intermediate proxies such as the volume curve. A critical requirement is that the model’s output constitutes a valid allocation strategy; that is, the allocation weights must be non-negative and sum to one. Although several approaches (e.g., clipping negative values or taking absolute values before normalization) were considered, these methods risk introducing uneven optimization. Instead, the softmax function is adopted, which naturally transforms any real-valued vector into a smooth probability distribution over the execution horizon. This ensures that even if the Temporal Linear Network (TLN) produces negative values, the final allocation vector remains both non-negative and normalized, thereby satisfying the constraints without sacrificing optimization smoothness.

\subsection{Internal Model}

In selecting an internal model for the VWAP allocation framework, various architectures traditionally employed for sequential prediction tasks were considered. Recent studies in volume prediction for cryptocurrency markets have explored models such as Temporal Kolmogorov-Arnold Networks (TKAN) \cite{genet2024tkan}, Signature-Weighted Kolmogorov-Arnold Networks (SigKAN) \cite{inzirillo2024sigkan}, Temporal Kolmogorov-Arnold Transformers (TKAT) \cite{genet2024tkat}, Kolmogorov-Arnold Mixture of Experts (KAMoE) \cite{inzirillo2024kamoe} and Recurrent Neural Networks with Signature-Based Gating Mechanisms (SigGate) \cite{genet2025siggate}. Although these approaches offer innovative solutions for capturing long-term dependencies and complex interactions, they also introduce additional complexity that is not necessary when the primary goal is to demonstrate the importance of optimizing the correct objective. Accordingly, the Temporal Linear Network (TLN) \cite{genet2024tln} was chosen as the default internal model. The TLN is conceptually simple and highly interpretable—essentially a linear model with structured transformations—yet it is integrated within a deep learning framework that enables the minimization of arbitrary loss functions via automatic differentiation. Importantly, even a naive linear model performs effectively within the proposed approach, emphasizing that improvements arise from directly optimizing the VWAP execution objective rather than from architectural complexity.
Let the input tensor be
$$
X \in \mathbb{R}^{B \times S \times F},
$$
where \(B\) is the batch size, \(S\) is the length of the input sequence, and \(F\) is the number of input features. The TLN consists of \(L\) layers, each implementing a cascade of linear operations. Each operation within a layer is a linear mapping with learnable weights, and the layers are designed to progressively adjust both the sequence length and feature dimension. This structured design leverages parameter sharing and constraints, resulting in a model that is both computationally efficient and stable.

Each layer \(\ell\) (with \(1 \leq \ell \leq L\)) transforms its input \(X^{(\ell-1)}\) into an output \(X^{(\ell)}\) through two primary stages: a \textbf{structured linear transformation} and a \textbf{convolutional filtering} along the time dimension.

\subsubsection*{1. Structured Linear Transformation}

This stage is composed of three consecutive linear operations.

\paragraph{Temporal Scaling:}  
Each time index is scaled by a learnable vector \(K^{(\ell)} \in \mathbb{R}^{S_\ell}\), where \(S_\ell\) denotes the current sequence length at layer \(\ell\). For each batch index \(b\), time index \(s\), and feature index \(f\), the computation is as follows:
\begin{equation}
    \widetilde{X}^{(\ell)}_{b,s,f} = K^{(\ell)}_s \, X^{(\ell-1)}_{b,s,f}.
\end{equation}

\paragraph{Feature Transformation:}  
At each time step, the features are mapped from \(\mathbb{R}^{F_\ell}\) to an intermediate space \(\mathbb{R}^{F^\prime_\ell}\) using a learnable weight matrix \(W^{(\ell)} \in \mathbb{R}^{F_\ell \times F^\prime_\ell}\) and bias \(b^{(\ell)} \in \mathbb{R}^{F^\prime_\ell}\):
\begin{equation}
    Y^{(\ell)}_{b,s,\cdot} = \widetilde{X}^{(\ell)}_{b,s,\cdot}\, W^{(\ell)} + b^{(\ell)}.
\end{equation}
An additional element-wise scaling is then applied using a learnable vector \(F^{(\ell)} \in \mathbb{R}^{F^\prime_\ell}\):
\begin{equation}
    Z^{(\ell)}_{b,s,f^\prime} = F^{(\ell)}_{f^\prime} \cdot Y^{(\ell)}_{b,s,f^\prime}.
\end{equation}

\paragraph{Temporal Transformation:}  
Finally, the sequence is remapped to a new length \(S^\prime_\ell\) by applying a linear transformation with learnable parameters \(T^{(\ell)} \in \mathbb{R}^{S_\ell \times S^\prime_\ell}\) and bias \(c^{(\ell)} \in \mathbb{R}^{S^\prime_\ell}\):
\begin{equation}
    \widehat{Z}^{(\ell)}_{b,s^\prime} = \sum_{s=1}^{S_\ell} T^{(\ell)}_{s,s^\prime}\, Z^{(\ell)}_{b,s,f^\prime} + c^{(\ell)}_{s^\prime}.
\end{equation}
In summary, the structured transformation is represented as
\begin{equation}
    \Phi^{(\ell)}\bigl(X^{(\ell-1)}\bigr) = \mathcal{T}^{(\ell)}\!\left( \left( X^{(\ell-1)} \odot K^{(\ell)} \right)W^{(\ell)} + b^{(\ell)} \odot F^{(\ell)} \right),
\end{equation}
where \(\mathcal{T}^{(\ell)}(\cdot)\) denotes the temporal mapping and \(\odot\) represents element-wise multiplication.

\subsubsection*{2. Convolutional Filtering}

After the structured transformation, each feature channel is convolved along the time dimension. Let \(C^{(\ell)} \in \mathbb{R}^{K^{(\ell)}_{\text{conv}} \times F^\prime_\ell}\) denote the convolution kernel (with kernel size \(K^{(\ell)}_{\text{conv}}\)) and \(d^{(\ell)} \in \mathbb{R}^{F^\prime_\ell}\) the corresponding bias. The convolution is computed as
\begin{equation}
    X^{(\ell)}_{b,t,f^\prime} = \sum_{k=0}^{K^{(\ell)}_{\text{conv}}-1} C^{(\ell)}_{k,f^\prime} \, \Phi^{(\ell)}\bigl(X^{(\ell-1)}\bigr)_{b,t+k,f^\prime} + d^{(\ell)}_{f^\prime}.
\end{equation}
Thus, the overall operation in layer \(\ell\) is given by
\begin{equation}
    X^{(\ell)} = \operatorname{Conv}\!\Bigl( \Phi^{(\ell)}\bigl(X^{(\ell-1)}\bigr) \Bigr).
\end{equation}

\subsection{The Role of Softmax in Enforcing Allocation Constraints}

A key innovation in the framework is the use of the softmax function to enforce allocation constraints. In this context, the final output of the internal model must be a valid allocation vector—that is, all elements must be non-negative and the vector must sum to one. While straightforward normalization (dividing by the sum of the outputs) or clipping negative values might appear sufficient, these approaches can lead to uneven or suboptimal optimization. In contrast, the softmax function
\begin{equation}
    \sigma(z)_i = \frac{e^{z_i}}{\sum_{j=1}^K e^{z_j}},
\end{equation}
naturally transforms any real-valued input vector \(z\) into a smooth probability distribution. This not only satisfies the constraints but also ensures a differentiable transformation that facilitates gradient-based optimization. Even if the TLN produces negative values, the exponential function within softmax guarantees a smooth and balanced allocation across the execution horizon.

\subsection{Comparison with Standard Linear Regression and Rationale}

A conventional linear regression would flatten the input tensor \(X \in \mathbb{R}^{B \times S \times F}\) into \(X_{\text{flat}} \in \mathbb{R}^{B \times (S \cdot F)}\) and compute 
\begin{equation}
    Y = X_{\text{flat}}\, W_{\text{reg}} + b_{\text{reg}},
\end{equation}
resulting in a very large number of parameters. In contrast, the TLN architecture leverages structured, layer-by-layer transformations to progressively adjust the sequence and feature dimensions, yielding a much more parsimonious model that is less prone to unstable weight estimates. Furthermore, by embedding the TLN within a deep learning framework, it becomes possible to optimize non-traditional loss functions—such as the VWAP-specific losses—using automatic differentiation, an advantage that standard linear regression does not offer.

\subsection{Summary of the Optimization Process}
The overall optimization process proceeds in three primary steps. First, the TLN processes the input features over a specified lookback period to generate a raw output vector \(v_{[t, t+h]} = \text{TLN}(x_t)\) for future time steps. Next, the softmax function is applied to these raw outputs to convert them into a valid allocation vector:
\begin{equation}
    q_t = \frac{e^{v_t}}{\sum_{s=1}^{T} e^{v_s}}, \quad \text{for } t = 1, 2, \dots, T,
\end{equation}
ensuring that the allocations are non-negative and sum to one. Finally, the model parameters \(\theta\) are optimized by minimizing the quadratic difference between the achieved VWAP and the market VWAP, as defined in Section~\ref{section3} (see Eqs.~\eqref{eq:static_quad_loss}--\eqref{eq:static_market_vwap}). In summary, this framework establishes a fair comparison platform by leveraging deep learning to directly optimize the VWAP execution objective. Rather than increasing model complexity, the approach employs a straightforward, linear model (the TLN) enhanced by the softmax transformation to enforce smooth and constrained allocation, demonstrating that aligning the optimization objective precisely with the trading goal can lead to improved execution performance.

\section{Results}
In this section, a comprehensive comparison of various VWAP execution strategies is presented, including traditional methods, the proposed deep-learning approach, and dynamic strategies. These methods are evaluated across different cryptocurrencies to assess their performance under varying market conditions.

\subsection{Benchmarks and Strategies}

Several benchmark strategies are considered. As a baseline, a naive flat allocation strategy that uniformly distributes the order volume across all time intervals is included; this simple method serves as a reference for assessing more sophisticated approaches. In addition, fixed optimal allocation curves are evaluated—static allocations optimized using three different loss functions (absolute VWAP loss, quadratic VWAP loss, and volume curve loss) as described in \ref{section3}. For this optimization, the differential evolution algorithm is employed due to its demonstrated superior and more stable performance. The proposed deep-learning approach, referred to as the \emph{StaticVWAP Model}, employs a neural network to generate allocation curves based on historical market data. For a fair comparison, all models receive the same input features, including volumes, the hour of the day, the day of the week, and returns computed on the VWAP price of each bin. In contrast, the \emph{Static Linear Regression} model serves as a traditional benchmark that is calibrated to predict the volume. Since linear regression does not inherently produce outputs that satisfy the allocation constraints (i.e., non-negativity and summing to one), its predictions are post-processed by first clipping negative values to zero and then normalizing the resulting vector. In cases where all values are zero (or become zero after clipping), an equiponderated allocation is returned. Dynamic VWAP strategies based on the framework introduced by \cite{LeFol2012} are also implemented. In these dynamic strategies, the volume executed at each time step is determined by the current market volume prediction and the remaining order size, following the equation
\begin{equation}
    v_t = \frac{\hat{V}_t}{\sum_{i=t}^T \hat{V}_i} \cdot \left(1 - \sum_{i=0}^{t-1} v_i\right),
\end{equation}
where \(v_t\) denotes the volume executed at time \(t\), \(\hat{V}_t\) is the predicted market volume at time \(t\), \(T\) is the end of the execution horizon, and \(\sum_{i=0}^{t-1} v_i\) represents the cumulative executed volume up to time \(t-1\). Both a dynamic linear regression approach and a dynamic version of the StaticVWAP model, which updates predictions at each time step, are considered.


\subsection{Evaluation Metrics}

Three key metrics are used to assess performance. The \emph{Absolute VWAP Loss} measures the absolute difference between the achieved VWAP and the market VWAP, while the \emph{Quadratic VWAP Loss} measures the squared difference, thereby penalizing larger deviations more heavily. In addition, the \emph{\(R^2\) Score for the Volume Curve} is computed, indicating how accurately the strategy predicts the actual market volume profile.

\subsection{Experimental Setup}

The strategies are evaluated using hourly data from five major cryptocurrencies (BTC, ETH, BNB, ADA, and XRP) traded on Binance perpetual contracts covering the period from January 1, 2020, to July 1, 2024. This dataset provides sufficient historical data across varying market conditions to evaluate the models' performance in both high and low volatility regimes. The data is partitioned chronologically, with the last 20\% reserved as the test set to ensure a true out-of-sample evaluation. A validation set, comprising 20\% of the remaining data, is randomly selected from the first 80\% of the period to guide model selection and prevent overfitting to any specific market regime.

For input features, several categories are incorporated: (1) Volume data, which is transformed to address non-stationarity by dividing each volume value by a trailing two-week moving average, calculated with appropriate shifting to avoid look-ahead bias; (2) Temporal indicators including hour of day and day of week as categorical features to capture seasonality patterns; and (3) Price information in the form of returns calculated on the VWAP price of each bin, with a value of 0 assigned when no volume is traded.

The preparation of target variables differs based on the model type. For the neural network approaches, the normalization to volume curve is implemented directly within the loss function, with volumes divided by the sum over the lookback period for numerical stability. For traditional linear regression models, the volume curve is explicitly normalized by dividing each volume by the sum of volumes in the lookahead period, as these models cannot incorporate this normalization within their loss functions.

In these experiments, a lookback period of 120 time steps is used to predict allocations for 12 periods ahead. Additional experiments with modified parameters—one with a 6-step prediction horizon and a shorter lookback period, and another with a 48-step prediction horizon—are provided in the Appendix. The results from these experiments confirm that the framework is effective across different timeframes and that the underlying concepts generalize well.

\subsection{Training Details}

Standard training settings were adopted. Specifically, the Adam optimizer was employed, and the maximum number of epochs was set to 1000. To mitigate overfitting and ensure stable convergence, two key callbacks were incorporated: an early stopping mechanism, which halts training if no improvement in validation loss is observed for 10 consecutive epochs, and a learning rate reducer, which divides the current learning rate by 4 after 5 epochs without improvement (starting from an initial rate of 0.001). The training set was shuffled, and a 20\% validation split was used to monitor these metrics during training.

An additional critical aspect was the mitigation of sensitivity to weight initialization. Each training run was repeated 30 times with different random initializations to compute both the average performance and its standard deviation, thereby providing a robust and statistically sound evaluation of model effectiveness.

\subsection{Empirical Results}

Table~\ref{tab:static_vwap_results} summarizes the performance of various VWAP execution strategies for a 12-step-ahead allocation with a 120-step lookback window. The table reports the mean and standard deviation of the Absolute VWAP Loss (scaled by \(10^{-2}\)), Quadratic VWAP Loss (scaled by \(10^{-4}\)), the \(R^2\) score for the volume curve, and the training time (in seconds) for each strategy, across five assets (BTC, ETH, BNB, ADA, and XRP).

Several key patterns emerge from these results. First, all proposed methods consistently outperform the naive flat allocation approach across all assets, confirming that more sophisticated allocation strategies yield tangible benefits for VWAP execution. 

\begin{table}[H]
    \centering
    \caption{VWAP Optimization Results for 12 steps ahead and 120 lookback window}
    \small
    \resizebox{\textwidth}{!}{%
        \begin{tabular}{llcccccccccc}
        \hline
        Model Type & Asset & Optimization & \multicolumn{2}{c}{Abs. VWAP Loss ($10^{-2}$)} & \multicolumn{2}{c}{Quad. VWAP Loss ($10^{-4}$)} & \multicolumn{2}{c}{R² Vol. Curve} & \multicolumn{2}{c}{Training Time (s)} \\
         &  & Function & Mean & Std & Mean & Std & Mean & Std & Mean & Std \\
        \hline
        Naive & BTC & N/A & 0.158743 & 0.000000 & 0.087808 & 0.000000 & 0.000000 & 0.000000 & 0.000000 & 0.000000 \\
        StaticVWAP & BTC & Absolute & \textbf{0.119742} & 0.000835 & 0.050175 & 0.001045 & -0.133930 & 0.038680 & 17.066662 & 1.452658 \\
        StaticVWAP & BTC & Quadratic & 0.120810 & 0.000759 & \textbf{0.047403} & 0.001055 & -0.361577 & 0.082153 & 16.568835 & 0.591146 \\
        StaticVWAP & BTC & Volume & 0.149938 & 0.001896 & 0.084989 & 0.001637 & 0.134904 & 0.011795 & 24.140719 & 3.765817 \\
        Dynamic Linear & BTC & N/A & 0.142535 & 0.000000 & 0.085731 & 0.000000 & \textbf{0.193757} & 0.000000 & 0.241681 & 0.000000 \\
        Static Linear & BTC & N/A & 0.146979 & 0.000000 & 0.084170 & 0.000000 & 0.162770 & 0.000000 & 0.244588 & 0.000000 \\
        Dynamic StaticVWAP & BTC & Absolute & 0.142334 & 0.008724 & 0.076042 & 0.010107 & 0.021618 & 0.038493 & 15.005514 & 0.799850 \\
        Dynamic StaticVWAP & BTC & Quadratic & 0.147608 & 0.018183 & 0.074102 & 0.016007 & -0.495363 & 0.815728 & 14.517617 & 0.582770 \\
        Dynamic StaticVWAP & BTC & Volume & 0.152135 & 0.004588 & 0.091675 & 0.003997 & 0.148551 & 0.018475 & 20.605638 & 2.945344 \\
        Fixed Volume Curve & BTC & Absolute & 0.129670 & 0.000000 & 0.054213 & 0.000000 & -0.273680 & 0.000000 & 17.263421 & 0.000000 \\
        Fixed Volume Curve & BTC & Quadratic & 0.127530 & 0.000000 & 0.049655 & 0.000000 & -0.467000 & 0.000000 & 30.385431 & 0.000000 \\
        Fixed Volume Curve & BTC & Volume & 0.160575 & 0.000000 & 0.089770 & 0.000000 & -0.003349 & 0.000000 & 12.861315 & 0.000000 \\
        \hline
        Naive & ETH & N/A & 0.177758 & 0.000000 & 0.116196 & 0.000000 & 0.000000 & 0.000000 & 0.000000 & 0.000000 \\
        StaticVWAP & ETH & Absolute & \textbf{0.138627} & 0.000675 & 0.076506 & 0.001128 & -0.146691 & 0.026960 & 16.977891 & 1.268639 \\
        StaticVWAP & ETH & Quadratic & 0.139999 & 0.000451 & 0.073385 & 0.001008 & -0.297154 & 0.051581 & 16.011362 & 0.614544 \\
        StaticVWAP & ETH & Volume & 0.170102 & 0.001785 & 0.122055 & 0.001819 & 0.109135 & 0.007899 & 19.888228 & 3.165192 \\
        Dynamic Linear & ETH & N/A & 0.158829 & 0.000000 & 0.121279 & 0.000000 & \textbf{0.169994} & 0.000000 & 0.249899 & 0.000000 \\
        Static Linear & ETH & N/A & 0.166326 & 0.000000 & 0.122063 & 0.000000 & 0.135824 & 0.000000 & 0.278128 & 0.000000 \\
        Dynamic StaticVWAP & ETH & Absolute & 0.152641 & 0.003763 & 0.093910 & 0.005712 & -0.119677 & 0.128250 & 15.180526 & 1.168761 \\
        Dynamic StaticVWAP & ETH & Quadratic & 0.162467 & 0.017372 & 0.094459 & 0.013988 & -0.633158 & 0.571764 & 14.129841 & 0.686132 \\
        Dynamic StaticVWAP & ETH & Volume & 0.171854 & 0.004362 & 0.127756 & 0.004184 & 0.118685 & 0.018637 & 17.855957 & 2.545837 \\
        Fixed Volume Curve & ETH & Absolute & 0.148407 & 0.000000 & 0.076206 & 0.000000 & -0.269217 & 0.000000 & 16.648768 & 0.000000 \\
        Fixed Volume Curve & ETH & Quadratic & 0.147058 & 0.000000 & \textbf{0.072852} & 0.000000 & -0.348835 & 0.000000 & 23.099389 & 0.000000 \\
        Fixed Volume Curve & ETH & Volume & 0.174744 & 0.000000 & 0.112633 & 0.000000 & -0.006033 & 0.000000 & 9.376649 & 0.000000 \\
        \hline
        Naive & BNB & N/A & 0.174116 & 0.000000 & 0.118874 & 0.000000 & 0.000000 & 0.000000 & 0.000000 & 0.000000 \\
        StaticVWAP & BNB & Absolute & \textbf{0.137741} & 0.000559 & 0.076553 & 0.001233 & -0.169736 & 0.036491 & 15.882844 & 0.794736 \\
        StaticVWAP & BNB & Quadratic & 0.141483 & 0.000963 & \textbf{0.071090} & 0.000803 & -0.477507 & 0.071598 & 15.598018 & 0.652385 \\
        StaticVWAP & BNB & Volume & 0.165025 & 0.001455 & 0.116136 & 0.001633 & 0.086762 & 0.002815 & 18.406203 & 1.898933 \\
        Dynamic Linear & BNB & N/A & 0.160003 & 0.000000 & 0.115190 & 0.000000 & \textbf{0.127731} & 0.000000 & 0.262799 & 0.000000 \\
        Static Linear & BNB & N/A & 0.163049 & 0.000000 & 0.114673 & 0.000000 & 0.101505 & 0.000000 & 0.216979 & 0.000000 \\
        Dynamic StaticVWAP & BNB & Absolute & 0.146537 & 0.006959 & 0.074951 & 0.002269 & -0.434465 & 0.221577 & 14.089177 & 0.626975 \\
        Dynamic StaticVWAP & BNB & Quadratic & 0.183354 & 0.040719 & 0.098591 & 0.031602 & -1.490792 & 1.373252 & 13.962810 & 0.606966 \\
        Dynamic StaticVWAP & BNB & Volume & 0.165976 & 0.003899 & 0.119578 & 0.003757 & 0.093674 & 0.009887 & 16.622247 & 1.866124 \\
        Fixed Volume Curve & BNB & Absolute & 0.146639 & 0.000000 & 0.079064 & 0.000000 & -0.283410 & 0.000000 & 16.226866 & 0.000000 \\
        Fixed Volume Curve & BNB & Quadratic & 0.147920 & 0.000000 & 0.074117 & 0.000000 & -0.440821 & 0.000000 & 31.472387 & 0.000000 \\
        Fixed Volume Curve & BNB & Volume & 0.177473 & 0.000000 & 0.122219 & 0.000000 & -0.003080 & 0.000000 & 12.545491 & 0.000000 \\
        \hline
        Naive & ADA & N/A & 0.226207 & 0.000000 & 0.228809 & 0.000000 & 0.000000 & 0.000000 & 0.000000 & 0.000000 \\
        StaticVWAP & ADA & Absolute & \textbf{0.187574} & 0.000640 & 0.156983 & 0.004485 & -0.201991 & 0.040232 & 15.773802 & 0.727374 \\
        StaticVWAP & ADA & Quadratic & 0.194378 & 0.002171 & 0.139060 & 0.002399 & -0.503548 & 0.086766 & 15.721992 & 0.543023 \\
        StaticVWAP & ADA & Volume & 0.219882 & 0.001456 & 0.244318 & 0.004173 & 0.089412 & 0.004083 & 18.606664 & 2.294395 \\
        Dynamic Linear & ADA & N/A & 0.213691 & 0.000000 & 0.258357 & 0.000000 & \textbf{0.142651} & 0.000000 & 0.236307 & 0.000000 \\
        Static Linear & ADA & N/A & 0.217345 & 0.000000 & 0.253104 & 0.000000 & 0.112113 & 0.000000 & 0.293910 & 0.000000 \\
        Dynamic StaticVWAP & ADA & Absolute & 0.224430 & 0.018242 & 0.164912 & 0.013876 & -0.868068 & 0.389384 & 14.117266 & 0.803708 \\
        Dynamic StaticVWAP & ADA & Quadratic & 0.235337 & 0.043516 & 0.168912 & 0.041709 & -1.121577 & 1.028339 & 13.869841 & 0.524214 \\
        Dynamic StaticVWAP & ADA & Volume & 0.223166 & 0.005122 & 0.253597 & 0.009979 & 0.094306 & 0.010872 & 16.104520 & 2.195146 \\
        Fixed Volume Curve & ADA & Absolute & 0.195395 & 0.000000 & 0.148712 & 0.000000 & -0.311109 & 0.000000 & 15.779793 & 0.000000 \\
        Fixed Volume Curve & ADA & Quadratic & 0.199729 & 0.000000 & \textbf{0.135216} & 0.000000 & -0.573288 & 0.000000 & 22.675063 & 0.000000 \\
        Fixed Volume Curve & ADA & Volume & 0.231114 & 0.000000 & 0.236772 & 0.000000 & -0.005151 & 0.000000 & 11.179776 & 0.000000 \\
        \hline
        Naive & XRP & N/A & 0.223855 & 0.000000 & 0.275925 & 0.000000 & 0.000000 & 0.000000 & 0.000000 & 0.000000 \\
        StaticVWAP & XRP & Absolute & \textbf{0.182436} & 0.000558 & 0.187474 & 0.003841 & -0.268477 & 0.054168 & 16.156652 & 0.781386 \\
        StaticVWAP & XRP & Quadratic & 0.188880 & 0.001564 & 0.158377 & 0.003440 & -0.800538 & 0.151472 & 15.717826 & 0.651796 \\
        StaticVWAP & XRP & Volume & 0.218816 & 0.001646 & 0.291965 & 0.003142 & 0.053552 & 0.002636 & 18.000321 & 2.353914 \\
        Dynamic Linear & XRP & N/A & 0.213728 & 0.000000 & 0.295177 & 0.000000 & \textbf{0.098195} & 0.000000 & 0.290283 & 0.000000 \\
        Static Linear & XRP & N/A & 0.216581 & 0.000000 & 0.295605 & 0.000000 & 0.071538 & 0.000000 & 0.248654 & 0.000000 \\
        Dynamic StaticVWAP & XRP & Absolute & 0.213859 & 0.017346 & 0.169141 & 0.009132 & -1.273806 & 0.583631 & 14.198917 & 0.578075 \\
        Dynamic StaticVWAP & XRP & Quadratic & 0.249005 & 0.038793 & 0.195797 & 0.028412 & -2.392281 & 1.708965 & 13.999556 & 0.594463 \\
        Dynamic StaticVWAP & XRP & Volume & 0.221568 & 0.003417 & 0.299404 & 0.007929 & 0.052095 & 0.005762 & 15.439939 & 1.495954 \\
        Fixed Volume Curve & XRP & Absolute & 0.186341 & 0.000000 & 0.178947 & 0.000000 & -0.347910 & 0.000000 & 14.544090 & 0.000000 \\
        Fixed Volume Curve & XRP & Quadratic & 0.189142 & 0.000000 & \textbf{0.152440} & 0.000000 & -0.794792 & 0.000000 & 23.787354 & 0.000000 \\
        Fixed Volume Curve & XRP & Volume & 0.224941 & 0.000000 & 0.276570 & 0.000000 & -0.001512 & 0.000000 & 12.143459 & 0.000000 \\
        \hline
        \end{tabular}
    }
    \label{tab:static_vwap_results}
\end{table}

When comparing the methods based on their ability to predict volume curves, we observe that Static Linear Regression is more effective than the StaticVWAP model when the latter is optimized for absolute or quadratic VWAP loss. However, the Dynamic Linear Regression approach consistently achieves the highest \(R^2\) values across all assets, indicating superior volume curve prediction capability. This improved prediction translates into better VWAP execution compared to Static Linear Regression.

\medskip

Interestingly, despite Dynamic Linear Regression's advantage in volume prediction, the Fixed Volume Curve approach demonstrates superior VWAP execution performance. This finding underscores a critical insight: accurate volume curve prediction, while beneficial, is not the primary determinant of optimal VWAP execution.

\medskip

Most significantly, the proposed StaticVWAP deep learning model consistently achieves the best absolute or quadratic VWAP loss values (depending on its optimization target) across all assets. When trained to minimize absolute VWAP loss, it delivers the lowest absolute slippage; similarly, when optimized for quadratic VWAP loss, it produces the lowest quadratic slippage metrics.

\medskip

Regarding training times, the deep-learning models (StaticVWAP) generally require between 15 and 18 seconds per run, which, although longer than the near-instantaneous training times of linear regression models, represents an acceptable trade-off given the significant performance gains. The dynamic approaches show mixed results when applied to the StaticVWAP model, often degrading performance rather than improving it.

\medskip

Overall, these results validate the central thesis of this study: directly optimizing the VWAP execution objective—rather than focusing solely on volume curve prediction—is essential for achieving superior execution performance in volatile markets such as cryptocurrencies. The StaticVWAP deep learning framework provides a flexible and effective approach that consistently outperforms conventional methods, even when utilizing a relatively simple model architecture.

\subsection{Training Times and Comparison Across Prediction Horizons}

Tables~\ref{tab:static_vwap_results_6} and~\ref{tab:static_vwap_results_48} provide additional insight into the practical applicability of each method under different prediction horizons. Several consistent patterns emerge across these timeframes that reinforce the findings from our main experiments.

\medskip

First, regarding training efficiency, the StaticVWAP models consistently maintain reasonable computational demands across all horizons—typically under 20 seconds—making them practical for real-world applications despite being more complex than linear regression models. Notably, while the fixed volume curve approach converges rapidly for shorter horizons, its training time increases dramatically with longer prediction horizons, eventually requiring several minutes per run for 48-step predictions.

\medskip

When examining performance across different timeframes, the relative rankings of the models remain remarkably consistent. For both shorter (6-step) and longer (48-step) prediction horizons, the StaticVWAP models optimized directly for VWAP loss consistently outperform both naive approaches and linear regression variants. The fixed volume curve approach remains competitive, particularly when optimized with quadratic loss, but generally does not reach the performance levels of the StaticVWAP model.

\medskip

Dynamic approaches show consistent patterns across different horizons. Dynamic linear regression demonstrates increasingly strong volume curve prediction capability (higher \(R^2\) values) as the prediction horizon extends, yet this improvement in volume prediction does not translate to proportional gains in VWAP execution performance. This pattern strongly supports our central thesis that accurate volume prediction, while informative, is not the primary driver of optimal VWAP execution. In fact, the performance gap between methods directly optimized for VWAP execution and those focused on volume prediction tends to persist or even widen with longer horizons.

\medskip

The stability of these findings across different prediction horizons and assets underscores the robustness of our approach. Whether executing shorter or longer-term VWAP strategies, directly optimizing the execution objective consistently yields superior results compared to approaches that rely primarily on volume curve prediction.

\subsection{Limitations and Future Directions}

Although the proposed approach yields promising improvements in VWAP execution, several limitations warrant discussion. One key assumption in the current framework is that market impact is negligible. This assumption is common in the VWAP literature, as employing a VWAP benchmark over a long period is believed to mitigate market impact. However, it is recognized that large order sizes may affect execution quality. Studies such as Gueant et al. have shown that both temporary and permanent market impact components can be incorporated into the optimization process, suggesting that future work could integrate such models to more accurately capture real-world trading dynamics. Another limitation is that the present model operates in a static framework without dynamic updating in real time. Although static models have the advantage of simplicity and reduced technological complexity, dynamic adjustments could potentially enhance execution performance in highly volatile environments. Future research may explore hybrid approaches that combine the benefits of static optimization with dynamic market adaptation. Finally, while the analysis in this study focuses on cryptocurrency markets, the underlying principles are readily extendable to other asset classes such as equities. Incorporating more sophisticated market impact models and dynamic updating mechanisms will be valuable in developing a comprehensive framework for VWAP execution across diverse trading environments.


\subsection{Understanding the Predicted Curves}

An analysis of the predicted allocation curves under different calibration losses is presented to elucidate how the proposed model differentiates itself from competing approaches. Figures~\ref{fig:static_prediction_absolute} and~\ref{fig:static_prediction_quadratic} illustrate that the predictions generated using absolute and quadratic VWAP losses tend to follow an average allocation pattern similar to that observed in Section~2, while still adapting to current market conditions. Notably, the model trained with quadratic loss produces a more compact prediction region, suggesting that it sacrifices flexibility in favor of an end-loaded execution pattern. In contrast, the model trained with volume curve loss results in a nearly flat allocation, with distinct seasonal variations clearly visible. These visual differences provide important insight into how the choice of loss function shapes the resulting allocation strategy, ultimately contributing to the performance differences observed in the quantitative analysis.

\begin{figure}[H]
    \centering
    \includegraphics[width=\columnwidth]{figures/vwap_execution_allocation_absolute_target.jpg}
    \caption{Prediction obtained with model calibrated using Absolute Deviation Loss}
    \label{fig:static_prediction_absolute}
\end{figure}

\begin{figure}[H]
    \centering
    \includegraphics[width=\columnwidth]{figures/vwap_execution_allocation_quadratic_target.jpg}
    \caption{Prediction obtained with model calibrated using Quadratic Deviation Loss}
    \label{fig:static_prediction_quadratic}
\end{figure}
In contrast, Figure~\ref{fig:static_prediction_volume} shows the allocation curve from a model trained with volume curve loss, which is noticeably flatter and exhibits distinct seasonal waves. These differences underscore that directly optimizing the VWAP objective (via absolute or quadratic loss) yields a strategy that deviates from simply mimicking the historical volume curve.

\begin{figure}[H]
    \centering
    \includegraphics[width=\columnwidth]{figures/vwap_execution_allocation_volume_curve_target.jpg}
    \caption{Prediction obtained with model calibrated using Volume Curve Loss}
    \label{fig:static_prediction_volume}
\end{figure}

\subsection{Understanding When It Makes a Difference}
To further assess the effectiveness of the approach, an analysis of the temporal evolution of slippage (i.e., the difference between the achieved VWAP and the market VWAP) relative to the price is provided. Figure~\ref{fig:static_slippage_full} shows the slippage across the full out-of-sample period for the naive approach, a dynamic VWAP strategy, and the proposed static approach when executing a 48-hour VWAP in hourly bins. Notably, the proposed method generally exhibits lower slippage during periods of high volatility, and in some instances, the slippage even reverses sign relative to standard approaches. To illustrate this effect in greater detail, Figure~\ref{fig:static_slippage_subset} provides a zoomed-in view over a 2000-hour subperiod. The reduced magnitude and frequency of extreme slippage events in the proposed approach are evident in this more focused view. Finally, Figures~\ref{fig:static_slippage_diff_full} and~\ref{fig:static_slippage_diff_subset} present the differences in absolute slippage between the proposed method (or the dynamic linear method) and the naive approach. In these plots, negative values indicate lower absolute slippage (and thus improved performance) relative to the naive strategy, while positive values indicate higher slippage. The prevalence of negative values during volatile periods provides strong visual evidence that the end-loaded allocation strategy is effective in mitigating execution slippage.


\begin{figure}[H]
    \centering
    \includegraphics[width=\columnwidth]{figures/vwap_analysis_full_sample_slippage.jpg}
    \caption{Slippage between approaches on the full out-of-sample set}
    \label{fig:static_slippage_full}
\end{figure}

\begin{figure}[H]
    \centering
    \includegraphics[width=\columnwidth]{figures/vwap_analysis_2000_sample_slippage.jpg}
    \caption{Slippage between approaches on a subsample of the out-of-sample set}
    \label{fig:static_slippage_subset}
\end{figure}

\begin{figure}[H]
    \centering
    \includegraphics[width=\columnwidth]{figures/vwap_analysis_full_sample_comparison.jpg}
    \caption{Difference in absolute slippage versus naive approach on the full out-of-sample set. Negative values indicate improved performance over the naive approach.}
    \label{fig:static_slippage_diff_full}
\end{figure}

\begin{figure}[H]
    \centering
    \includegraphics[width=\columnwidth]{figures/vwap_analysis_2000_sample_comparison.jpg}
    \caption{Difference in absolute slippage versus naive approach on the subsample set. Negative values indicate improved performance over the naive approach.}
    \label{fig:static_slippage_diff_subset}
\end{figure}

\section{Conclusion}

In conclusion, the results of this study demonstrate that directly optimizing the VWAP execution objective through a deep learning framework yields substantial performance improvements over traditional volume curve-based approaches. More broadly, this example illustrates that in many cases where multi-step methods or proxy variables have traditionally been employed, deep learning offers a powerful alternative by enabling direct optimization of the true objective. The expressivity of deep learning allows for straightforward modifications to incorporate any function of market impact, thereby providing a versatile platform for capturing complex trading dynamics.

\medskip

Although a simple, naive linear model (the Temporal Linear Network) was employed, its success is attributable to the fundamentally different approach of targeting the VWAP execution objective directly. This shift in methodology not only enhances execution performance but also underscores the potential of deep learning to transform similar problems in other domains. I believe that this research represents a significant advancement in the design of VWAP execution strategies, and I anticipate that further enhancements—such as dynamic updating mechanisms and the integration of sophisticated market impact models—will extend the applicability of this approach to a broader range of asset classes and trading environments.

\section*{Code Availability}
The source code used for all experiments and analyses in this paper is available at \url{https://github.com/remigenet/DeepLearningVWAP}.


\bibliographystyle{abbrv}
\bibliography{bib}

\addtocontents{toc}{\setcounter{tocdepth}{1}}
\newpage
\appendix
\documentclass{MITstyle}

%\usepackage[table]{xcolor}
\usepackage{chngcntr}
\usepackage{hyperref}
\usepackage{microtype}

\title{A Lightweight and Extensible Cell Segmentation and Classification Model for Whole Slide Images}

\author{Nikita Shvetsov~$^{1, }$\footnote{Correspondence e-mail: nikita.shvetsov@uit.no}, Thomas K. Kilvaer~$^{2, 3}$, Masoud Tafavvoghi~$^{4}$, Anders Sildnes~$^{1}$, \\ Kajsa Møllersen~$^{4}$, Lill-Tove Rasmussen Busund~$^{5, 6}$, Lars Ailo Bongo~$^{1}$ \\
%
\vspace{1em} % Space between authors and afilliations
%
\normalfont{\small $^{1}$Department of Computer Science, UiT The Arctic University of Norway}\\
\normalfont{\small $^{2}$Department of Oncology, University Hospital of North Norway}\\
\normalfont{\small $^{3}$Department of Clinical Medicine, UiT The Arctic University of Norway}\\
\normalfont{\small $^{4}$Department of Community Medicine, UiT The Arctic University of Norway}\\
\normalfont{\small $^{5}$Department of Medical Biology, UiT The Arctic University of Norway} \\
\normalfont{\small $^{6}$Department of Clinical Pathology, University Hospital of North Norway} %\vspace{2em}
}

\begin{document}
\maketitle

\section*{Abstract}

% \begin{abstract}
% Developing clinically useful cell-level analysis tools in digital pathology remains challenging due to limitations in dataset granularity, inconsistent annotations, computational demands of advanced models, and difficulties in integrating new technologies into clinical workflows. To address these challenges, we propose a multi-faceted solution that enhances data quality, model performance, and usability to create a lightweight and extensible cell segmentation and classification model.

% First, we update data labels by employing a cross-relabeling process that refines the labels of two existing datasets, PanNuke and MoNuSAC, to create a new unified dataset with enhanced granularity, encompassing seven distinct cell types. Second, we leverage the H-Optimus foundation model as a fixed encoder to improve feature representation for simultaneous cell segmentation and classification tasks. Third, to address the computational demands of foundation models, we employ knowledge distillation to reduce model size and complexity while maintaining comparable performance. Finally, to facilitate integration into clinical workflows, we integrate the distilled model into the QuPath software, a widely used open-source platform in digital pathology.

% Our results demonstrate improvements in cell segmentation and classification performance using the H‑Optimus-based model compared to a CNN-based model. Specifically, the average $R^2$ improved from 0.575 to 0.871, and the average $PQ$ score improved from 0.450 to 0.492, indicating better alignment with actual cell counts and enhanced segmentation and classification quality. Furthermore, the distilled student model maintains performance comparable to the larger foundation model while reducing the parameter count by a factor of 48.
% Overall, by reducing computational complexity and integrating it into existing workflows, the proposed approach may significantly impact diagnostic processes, reduce the workload of pathologists, and contribute to improved patient outcomes. Though our approach shows potential enhancements in efficiency and usability of cell segmentation and classification models in digital pathology, extensive validation is needed to deploy these models in clinical practice.
% \end{abstract}

%%% shortened abstract
\begin{abstract}
Developing clinically useful cell-level analysis tools in digital pathology remains challenging due to limitations in dataset granularity, inconsistent annotations, high computational demands, and difficulties integrating new technologies into workflows. To address these issues, we propose a solution that enhances data quality, model performance, and usability by creating a lightweight, extensible cell segmentation and classification model. 

First, we update data labels through cross-relabeling to refine annotations of PanNuke and MoNuSAC, producing a unified dataset with seven distinct cell types. Second, we leverage the H-Optimus foundation model as a fixed encoder to improve feature representation for simultaneous segmentation and classification tasks. Third, to address foundation models' computational demands, we distill knowledge to reduce model size and complexity while maintaining comparable performance. Finally, we integrate the distilled model into QuPath, a widely used open-source digital pathology platform. 

Results demonstrate improved segmentation and classification performance using the H-Optimus-based model compared to a CNN-based model. Specifically, average $R^2$ improved from 0.575 to 0.871, and average $PQ$ score improved from 0.450 to 0.492, indicating better alignment with actual cell counts and enhanced segmentation quality. The distilled model maintains comparable performance while reducing parameter count by a factor of 48. By reducing computational complexity and integrating into workflows, this approach may significantly impact diagnostics, reduce pathologist workload, and improve outcomes. Although the method shows promise, extensive validation is necessary prior to clinical deployment.
\end{abstract}
\clearpage

\section{Introduction}
In digital pathology, accurate segmentation and classification of cells are crucial for many diagnostic, prognostic, and predictive analyses \cite{Jaber_Beziaeva_etal._2019,Lin_Pan_etal._2022,Park_Ock_etal._2022,Shen_Choi_etal._2024}. Nowadays, developments in computational pathology offer multiple solutions \cite{H._Qu_P._Wu_etal._2020,Javed_Mahmood_etal._2020} to utilize cell-level datasets to train machine learning models that solve these problems. The quality and specificity of training datasets are critical for robust and accurate models. Adhering to the principle of "garbage in, garbage out", it is essential to ensure that these datasets are extensively and accurately labeled with distinct classes that reflect the diverse biological characteristics of different cell types. Unfortunately, the number of open-source datasets comprising such high-quality annotations is limited. Existing cell segmentation datasets \cite{Gamper_Koohbanani_etal._2019,Graham_Vu_etal._2019,Verma_Kumar_etal._2021} may offer extensive annotations for certain cell types while providing more general labels for others. For example, in PanNuke, which is one of the largest open-source datasets comprising labeled cells, various types of morphologically and functionally different inflammatory cells like macrophages and lymphocytes are clustered in a broad "inflammatory" class. Consequently, these classes are frequently omitted from analyses or aggregated into broader meta-classes \cite{Gamper_Koohbanani_etal._2020} and likely interfere with other cell classes included in the dataset. This and similar inconsistencies in annotation granularity limit the ability of machine learning models to learn the comprehensive and nuanced features necessary for accurate cell segmentation and classification. To address these challenges, methods for refining and standardizing dataset annotations are essential to enhance the quality of training data.

A complementary approach to mitigate the absence of high-quality training data is the use of foundation models. Foundation models as encoders are defined as large-scale, versatile networks pre-trained on vast, diverse datasets using self-supervised learning, contrasting with convolutional neural network (CNN) pre-trained encoders that rely on supervised learning with labeled data. In practice, foundation models leverage enormous amounts of weakly or unlabeled data from millions of whole slide images (WSIs) and employ self-attention mechanisms to capture long-range dependencies and global context \cite{Chen_Ding_etal._2024,Saillard_Jenatton_etal._2024,Vorontsov_Bozkurt_etal._2024,Xu_Usuyama_etal._2024}. As a consequence, foundation models are able to produce transferable feature representations across different cell types and tissue environments. The feature representations can be leveraged by decoder networks to produce segmentation masks and pixel-level classifications. Because foundation models have comprehensive feature representations, they can be effectively fine-tuned using much smaller amounts of cell-level data compared to the large datasets needed to train models from scratch. Furthermore, foundation models incorporate adversarial training elements or contrastive learning \cite{Chen_Ding_etal._2024,Xu_Usuyama_etal._2024}, enhancing their resilience and adaptability by exposing them to challenging and varied scenarios during training. This may result in more generalizable models, often making them well-suited for diverse and complex tasks in digital pathology.

Despite the inherent advantages of foundation models, their deployment for practical use faces its own obstacles. In particular, they require substantial computational power, financial investments and rigorous testing to ensure reliability and efficacy for a given task \cite{Akkus_Dangott_etal._2022,Dragomir_Cocuz_etal._2022,Go_2022,Jafri_Farooqui_etal._2024}. Moreover, while foundation models enhance feature representation and performance, they depend on the quality of available annotations for decoder fine-tuning and, like any other model, cannot resolve existing inconsistencies or ambiguities in data labels. Therefore, there remains a critical need for solutions that address both data quality and practical deployment considerations.
Further, integrating new technologies into existing clinical workflows often encounters resistance, as it necessitates adjustments to established diagnostic processes. So, there is a need to develop solutions that could be integrated into current practices, minimizing the burden on medical professionals to adopt new tools \cite{King_Williams_etal._2023}.

Existing solutions \cite{Goldsborough_Philps_etal._2024,Hörst_Rempe_etal._2024}, while addressing some aspects of these challenges, fall short in providing a comprehensive approach. To address the data quality and clinical deployment issues, we propose a multi-faceted solution that encompasses data refinement, model optimization, and integration with existing pathology tools (\hyperref[fig:fig1]{Figure 1}). The outcome is a lightweight cell segmentation and classification model that can be integrated into digital pathology workflows for practical clinical use.

\begin{figure}[h!]
    \centering
    \includegraphics[width=\textwidth, height=0.82\textheight, keepaspectratio]{images/Figure_1.pdf}
    \caption{Overview of the proposed solution, including 1) Data refinement using cross-relabeling, 2) Teacher model development and fine tuning, 3) Student model optimization with knowledge distillation and 4) Student model and QuPath integration}
    \label{fig:fig1}
\end{figure}
\clearpage

Our approach begins with preparing the data for the fine-tuning and training of the machine learning models. We create a refined dataset, acquired via cross-relabeling two cell-level datasets, enhancing annotation specificity and consistency of the labeled data. Subsequently, we create a cell segmentation and classification model based on the foundation model. We leverage the foundation model as a fixed encoder and fine-tune a decoder using the refined dataset to improve generalization across diverse tissue- and cell types.
To ensure that the model remains lightweight and deployable in a possibly resource-constrained environment, we employ knowledge distillation to approximate the functionality of the foundation model. Finally, to facilitate the practical application of our model in digital pathology workflows, we integrate it with the QuPath \cite{Bankhead_Loughrey_etal._2017} application. Each methodological component contributes to the overarching goal of enhancing model performance, generalizability, and usability in clinical settings.

The primary contributions of this paper are:
\begin{enumerate}
    \item \textit{Data labels refinement through cross-relabeling:}
    
    We propose a new method for refining labels of cell-level datasets through cross-relabeling. This method employs classification models to re-label broad and ambiguous instances, resulting in a more diverse dataset. Our evaluation demonstrates that these classification models achieve high accuracy on test subsets, indicating the reliability of the method for label refinement.

    \item \textit{Enhanced model performance via foundation models:}
    
    We employ a foundation model as a feature extractor for the cell segmentation and classification task. In comparison with training a CNN model from scratch, the foundation model backbone only needs fine-tuning, which significantly reduces training time, computational resources and data requirements. We show that using a foundation model encoder leads to better performance in cell segmentation and classification networks than using a CNN-based encoder. This improvement may enable the model to generalize more effectively across various tissue types and imaging methods.
    
    \item \textit{Model optimization through knowledge distillation:}
    
    We show that a smaller student model trained using knowledge distillation on the refined dataset obtained via our cross-relabeling approach from a foundation model achieves comparable performance in cell segmentation and quantification tasks. As a result, this model is more suitable for deployment in environments without high-performance computing resources.
    
    \item \textit{Integration with QuPath:}
    
    We integrate the distilled cell segmentation and classification model into QuPath, a widely used open-source digital pathology platform, to accelerate clinical adaptation by enabling pathologists to more easily incorporate advanced computational tools into their existing workflows.
\end{enumerate}

Through these methodological steps, we aim to bridge the gap between advanced machine learning techniques and practical clinical applications, making accurate and efficient digital pathology accessible in a broader range of healthcare settings.

\section{Refining Existing Datasets Using Cross-Relabeling}
To address the limitations of sparse and ambiguous labeling of cell-level datasets, we propose a generalizable cross-relabeling strategy that can be applied to any dataset containing broadly categorized or imprecisely labeled cell types. This approach involves training and subsequently leveraging classification models to refine broad categories into more specific or biologically relevant classes.
When applied to cell-level data, the methodology includes extracting individual cell images from the dataset patches, preprocessing these images to standardize the size and accommodate partial cells, and then training deep learning classifiers capable of distinguishing between the finer cell subtypes within the coarser categories. 
To illustrate our approach, we focus on the PanNuke \cite{Gamper_Koohbanani_etal._2020, Gamper_Koohbanani_etal._2019} and MoNuSAC \cite{Verma_Kumar_etal._2021} datasets that we have used to train models for cell quantification in our previous works \cite{Shvetsov_Grønnesby_etal._2022,Shvetsov_Sildnes_etal._2024}. We find that for better cell differentiation we have to introduce more granular labels. PanNuke includes a broad classification of "inflammatory" cells, encompassing lymphocytes, macrophages, and neutrophils. Each cell type differs significantly in structure, function, and clinical relevance. Conversely, MoNuSAC uses the label "epithelial" for a class that comprises both benign epithelial cells and malignant neoplastic cells. This practice makes it challenging to differentiate between benign and malignant epithelial cells in the dataset, which is a critical distinction when identifying tumor areas within tissue samples. To address these issues, we implement a cross-relabeling strategy as shown in \hyperref[fig:fig2]{Figure 2}. The key components are two classification models: one is trained on singular cell images from PanNuke data to classify the epithelial meta-class into epithelial and neoplastic classes. The other is trained on MoNuSAC to refine the inflammatory class into lymphocytes, neutrophils, and macrophages.

\begin{figure}[h!]
    \centering
    \includegraphics[width=\textwidth]{images/Figure_2.pdf}
    \caption{Refined dataset generation via cross relabeling}
    \label{fig:fig2}
\end{figure}

The refining approach consists of three consecutive steps. The first is the preprocessing step, in which we extract individual cells from both datasets (\hyperref[fig:fig3]{Figure 3}). The specifics of PanNuke and MoNuSAC patch preparation before cell preprocessing are provided in \hyperref[chap:S1]{Appendix S1}.

\begin{figure}[h!]
    \centering
    \includegraphics[width=\textwidth]{images/Figure_3.pdf}
    \caption{Cell instances preprocessing including (1) cell map extraction, (2) bounding box delineation, (3) adjusting cell boxes and (4) cropping and resizing of cell images}
    \label{fig:fig3}
\end{figure}

During preprocessing, we extract cell type maps from the ground truth label mask and calculate bounding boxes around each cell instance. To accommodate partial cells at patch borders, a common issue in cropped patch images, we employ mirror padding and extend the field of view of the cell label by 15 pixels to capture adjacent cells. We then crop and resize the identified regions to $64 \times 64$ pixels using bicubic interpolation.

The preprocessed PanNuke dataset comprises 68,031 neoplastic and 23,207 epithelial cell images, while MoNuSAC comprises  33,104 lymphocytes, 1,252 neutrophils, and 1,695 macrophages, which we subsequently use in training cell classification models and classifying the cell image data \hyperref[fig:S2]{Appendix Figure S2 (1)}. 

The next step is to train two distinct ResNet50-based classifiers tailored to address the specific labeling challenges inherent in each dataset. We use ResNet50 for classification models due to its proven effectiveness for image classification tasks in histopathology \cite{pan2022reviewmachinelearningapproaches}, and its compatibility with small images. For the PanNuke dataset, we design the classifier, trained on MoNuSAC data, to disaggregate the heterogeneous "inflammatory" cell category into distinct subtypes: lymphocytes, macrophages, and neutrophils. Similarly, for the MoNuSAC dataset, the classifier is trained on PanNuke data and distinguishes between benign and malignant epithelial cells within the overarching "epithelial" label. By applying these targeted classifiers to their respective datasets, we assign more specific labels to individual cell instances, thus enabling us to create a unified dataset.
To ensure a balanced representation of classes, we train both models on datasets that had been equalized to match the size of the least represented class. Thus, we obtain datasets comprising 23,207 samples per class for PanNuke and 1,252 samples per class for MoNuSAC data. Next, we partition both of them into training (70\%), validation (20\%), and testing (10\%) subsets. To mitigate the risk of overfitting, we use a single dropout layer with a rate of p=0.5 in both models and data augmentation using randomized color perturbations, rotation, and horizontal and vertical flipping. We employ AdamW optimizer and the cross-entropy loss function for the training criterion.

To evaluate the two trained models, we measure the classification accuracy on the respective test subsets. The accuracies on the test subset for both classifiers are presented in \hyperref[tab:1]{Table 1}. The PanNuke model achieves an average accuracy of 93.57\%, with higher accuracy for neoplastic cells (96.06\%) compared to epithelial cells (86.26\%). The confusion matrix in Figure A3.1 shows that the model predominantly distinguishes accurately between epithelial and neoplastic tissues, with a substantial number of correct classifications and relatively few misclassifications. The MoNuSAC model demonstrates an average accuracy of 98.92\%, excelling in classifying lymphocytes (99.67\%) and macrophages (94.12\%), with lower performance for neutrophils (85.71\%). The confusion matrix in Figure A3.2 shows that the model identifies lymphocytes and performs reasonably well with macrophages and neutrophils.

\begin{table}[h!]
\renewcommand{\arraystretch}{1.5}
  \centering
  \caption{Cell classification results for PanNuke and MoNuSAC trained models (CI 95\%).}
  \label{tab:1}
  \begin{tabular}{|l|c|c|}
   \hline
   %\rowcolor{gray!30}
    Accuracy               & PanNuke model              & MoNuSAC model              \\
    \hline
    Average      & 0.936 (0.931--0.941)         & 0.989 (0.986--0.993)        \\
    \hline
    Neoplastic   & 0.961 (0.956--0.965)         & -                          \\
    \hline
    Epithelial   & 0.863 (0.849--0.877)         & -                          \\
    \hline
    Lymphocytes  & -                          & 0.997 (0.995--0.999)        \\
    \hline
    Neutrophils  & -                          & 0.857 (0.796--0.918)        \\
    \hline
    Macrophages  & -                          & 0.941 (0.906--0.976)        \\
    \hline
  \end{tabular}
\end{table}

Finally, during the last step, we use the model trained on PanNuke data for epithelial cells in MoNuSAC and the model trained on MoNuSAC for the inflammatory cells class in PanNuke. Specifically, we use classifier models to relabel epithelial cells in MoNuSAC and inflammatory cells in PanNuke data. Then we combine cells with refined labels and the rest of the cells in both datasets to create a refined dataset (\hyperref[fig:S2]{Appendix Figure S2 (2)}). The process of relabeling cells and visualizing them on a patch is shown in \hyperref[fig:fig4]{Figure 4}. The cell counts in the refined dataset are provided in \hyperref[tab:S4]{Appendix Table S4}.

\begin{figure}[h!]
    \centering
    \includegraphics[width=\textwidth, height=0.42\textheight, keepaspectratio]{images/Figure_4.pdf}
    \caption{Cell relabeling procedure for epithelial and inflammatory cell classes}
    \label{fig:fig4}
\end{figure}

%\hfill

Relabeling and combining datasets have been explored in a prior study \cite{Parulekar_Kanwat_etal._2023}, where consecutive fine-tuning on multiple datasets was employed to account for hierarchical class label structures. While the method presented in \cite{Parulekar_Kanwat_etal._2023} is intuitive, it often lacks consistency and requires multiple fine-tuning runs, which can be cumbersome and time-consuming. 
In contrast, cross-relabeling simplifies this process by using specialized classification models tailored to each dataset's specific labeling challenges. This approach provides better transparency and produces a unified dataset encompassing seven distinct cell types across multiple tissue samples, enhancing data diversity for further model training or fine-tuning.

Despite these improvements, cross-relabeling does not entirely resolve issues related to poor labeling quality or the amount of labeled data. Specifically, our results show lower accuracies persist for underrepresented classes, such as macrophages, which may stem from a limited sample availability and intrinsic challenges in distinguishing these cells based solely on H\&E staining. Furthermore, while our method enhances label specificity, it relies on the initial quality of the broad labels; thus, any fundamental inaccuracies in the original annotations can propagate through the relabeling process. Addressing the overall problem of limited data labels may require integrating additional data sources or utilizing complementary immunohistochemical staining methods.
Although the reported performance metrics are obtained from evaluations on the native test sets of each dataset, it is important to note that the primary application of these classifiers is to perform cross-relabeling, where a model trained on one dataset (e.g., PanNuke) is applied to another (e.g., MoNuSAC) and vice versa. We acknowledge that a more systematic evaluation of cross-dataset generalization is needed and could be performed in future work.

Overall, the refined dataset produced by our approach can enhance the supervised training or fine-tuning of cell segmentation and classification models, especially those that utilize pre-trained foundation models to improve feature extraction robustness. In addition, these models can detect nuanced classes that enable researchers to conduct more detailed analyses of biological processes in computational pathology.

\section{Foundation models for robust cell segmentation and classification}

Accurate cell segmentation and classification in digital pathology are hindered by limited labeled data and the fact that conventional CNNs are unable to capture global contextual information due to their local receptive field constraints \cite{Gheflati_Rivaz_2022,Yang_Marcus_etal.}. Traditional approaches in cell quantification have predominantly relied on CNN encoders, such as ResNet50, given their proven effectiveness in semantic segmentation tasks \cite{Deshmane_2023,Graham_Vu_etal._2019,Mukasheva_Koishiyeva_etal._2024,Stringer_Wang_etal._2021}. However, approaches that include fine-tuning of pretrained CNNs, data augmentation, and stain normalization to partially increase data variability and address staining differences often fail to achieve the necessary generalization and robustness across diverse tissue types and staining conditions \cite{G._Wang_W._Li_etal._2018,Gao_Bagci_etal._2018,Karim_El_Khoury_Martin_Fockedey_etal._2021}.

To overcome these challenges, we leverage an encoder-decoder network that uses a foundation model as the encoder and a CNN upsampling decoder (\hyperref[fig:fig5]{Figure 5}) for simultaneous cell segmentation and classification in 2D patches extracted from WSIs. Foundation models with transformer-based architectures are viable alternatives to CNN-based encoders \cite{Shamshad_Khan_etal._2023,Sourget_2023}. They enable the creation of more advanced architectures that can decode or transform learned features more effectively \cite{Chen_Duan_etal._2023,Cheng_Misra_etal._2022,Xie_Wang_etal._2021}.

\begin{figure}[h!]
    \centering
    \includegraphics[width=\textwidth]{images/Figure_5.pdf}
    \caption{UNETR-like model with foundational model as backbone}
    \label{fig:fig5}
\end{figure}

By utilizing a transformer-based encoder, we incorporate global contextual information into the feature extraction process, which is a key advantage of such architectures \cite{Chen_Lu_etal._2021}. This foundation model integration facilitates accurate pixel-wise segmentation and classification without the need for extensive encoder training, thereby potentially improving generalization across varied cellular structures and tissue types.
In our implementation, we employ a modified UNETR \cite{Hatamizadeh_Tang_etal._2021} architecture that combines a vision transformer (ViT) \cite{Dosovitskiy_Beyer_etal._2021} encoder with a CNN-based decoder. The encoder utilizes the pretrained H-Optimus foundation model, which contains 1.1 billion parameters and is trained on over 500,000 H\&E stained WSIs \cite{Saillard_Jenatton_etal._2024}. We extract outputs from four evenly spaced transformer blocks $Z_i$, where $i \in [1, 14, 26, 38]$, to serve as residual connections for the CNN decoder. We select these blocks based on our observation that features from non-adjacent levels of the encoder lead to better overall performance on the test subset.

The CNN decoder upsamples the feature representations, acquired from the transformer blocks, to generate an intermediate vector that is handled by two task-specific layers that generate cell segmentation and classification masks. The first task-specific layer is the ‘Cellpose head’,  which is used to delineate cell instances. The layer generates horizontal and vertical gradient maps to form vector fields that are refined through gradient tracking in a post-processing step using the Cellpose algorithm \cite{Stringer_Wang_etal._2021}, known for its efficacy in cell segmentation tasks and generalizability across multiple domains \cite{Pachitariu_Stringer_2022,Stringer_Pachitariu_2024}. The second task-specific layer is the "Cell type head", which assigns labels to individual pixels. In the post-processing step, we determine the output classification label of each segmented cell instance by majority voting over the labeled pixels that comprise the cell in the segmentation map.

To evaluate model performance and measure the impact of adding a foundation model as backbone, we compare it to a ResNet50-based model. ResNet50 is a widely used solution for encoders in segmentation architectures in the medical domain \cite{Deshmane_2023,Graham_Vu_etal._2019,Mukasheva_Koishiyeva_etal._2024,Stringer_Wang_etal._2021}. For the H-Optimus-based model, we utilize frozen weights for the encoder and only fine-tune the decoder to take advantage of the extensive pre-training of the foundation model. For the ResNet50-based model we start with ImageNet \cite{Deng_Dong_etal.} weights and train both encoder and decoder parts. Hyperparameters for the training step are set to be identical, where possible, for comparable evaluation. 
For this evaluation, we deliberately use the PanNuke dataset to provide a standardized and controlled comparison between the H‑Optimus and ResNet50-based models (\hyperref[fig:S2]{Appendix Figure S2 (3)}). Specifically, we use two of the default PanNuke dataset splits (66\%) for training and validation, and reserve the third split (33\%) for testing.

To address the challenge of cell class imbalance in the PanNuke dataset, which is a common characteristic in most cell-level H\&E patch datasets, both models’ training processes employ a weighted loss function comprising cross-entropy and focal loss \cite{Lin_Goyal_etal._2018}. The focal loss component is adjusted with coefficients derived from each cell class' instance frequency, emphasizing learning from underrepresented classes and enhancing the model's sensitivity to rare but significant cellular patterns. The cross-entropy loss is augmented with spectral decoupling regularization \cite{Pezeshki_Kaba_etal._2021,Pohjonen_Stürenberg_etal._2022} and spatially varying label smoothing \cite{Islam_Glocker_2021}, which potentially stabilizes training and improves generalization in case of complex tissue morphologies. For optimization, we employ the \textit{AdamW} \cite{Loshchilov_Hutter_2019} to counter unbalanced class scenarios, with cosine annealing learning rate scheduler.

We utilize the scikit-learn library \cite{Van_der_Walt_Schönberger_etal._2014} and HoVer-Net \cite{Graham_Vu_etal._2019} implementations of $R^2$ (the coefficient of determination) and $PQ$ (panoptic quality) to evaluate our experiments. Complete mathematical formulations and detailed explanations of these metrics are provided in \hyperref[chap:S5]{Appendix S5}. To compute confidence intervals, we use nonparametric bootstrapping, where after calculating the metric on the full sample, we generated 1000 bootstrap replicates by resampling with replacement and then determined the 95\% confidence intervals as the 2.5th and 97.5th percentiles of the resulting empirical distribution.

%\hfill

The model comparisons are summarized in \hyperref[tab:2]{Table 2}. The H‑Optimus-based model achieves higher $R^2$ across all cell classes compared to the ResNet50-based model, which means that its predictions are more closely aligned with the PanNuke cell counts, indicating a stronger correlation with the observed data. Notably, the improvement of $R^2_{dead}$ may be an indicator of better global contextual representations provided by the foundation model backbone. In terms of segmentation and classification quality combined, measured by the PQ score, the H‑Optimus-based model demonstrates notable improvements across most cell classes. Overall, the average $R^2$ improved from 0.575 to 0.871, while the average $PQ$ score improved from 0.450 to 0.492, demonstrating better performance of the H-Optimus-based model.

\begin{table}[h!]
\renewcommand{\arraystretch}{1.5}
  \centering
  \caption{Cell quantification metrics for baseline and proposed models (CI 95\%).}
  \label{tab:2}
  \begin{tabular}{|l|c|c|}
    \hline
    %\rowcolor{gray!30}
    Metric             & Resnet50-based            & H-optimus-based              \\
    \hline
    $R^2_{neoplastic}$    & 0.681 (0.576--0.769)       & \textbf{0.941 (0.917--0.960)} \\
    \hline
    $R^2_{inflammatory}$  & 0.863 (0.778--0.903)       & \textbf{0.949 (0.918--0.966)} \\
    \hline
    $R^2_{connective}$    & 0.600 (0.488--0.698)       & 0.609 (0.436--0.772)          \\
    \hline
    $R^2_{dead}$          & 0.097 (-11.389--0.669)     & 0.925 (0.404--0.982)          \\
    \hline
    $R^2_{epithelial}$    & 0.635 (0.490--0.747)       & \textbf{0.930 (0.886--0.964)} \\
    \hline
    $PQ_{neoplastic}$       & 0.517 (0.499--0.535)       & \textbf{0.589 (0.575--0.604)} \\
    \hline
    $PQ_{inflammatory}$     & 0.455 (0.429--0.482)       & \textbf{0.528 (0.507--0.549)} \\
    \hline
    $PQ_{connective}$       & 0.416 (0.400--0.431)       & \textbf{0.451 (0.436--0.465)} \\
    \hline
    $PQ_{dead}$             & 0.374 (0.342--0.408)       & 0.292 (0.209--0.365)          \\
    \hline
    $PQ_{epithelial}$       & 0.488 (0.460--0.519)       & \textbf{0.599 (0.579--0.618)} \\
    \hline
  \end{tabular}
\end{table}

Our results  show that integrating the H‑Optimus foundation model within the UNETR architecture enhances the model's ability to segment and classify cells across diverse tissues from PanNuke data. The pretrained transformer encoder provides robust feature representations, resulting in higher average $R^2$ and $PQ$ scores compared to the CNN-based model. This leads to more reliable cell quantification and more accurate downstream analysis. Additionally, the streamlined fine-tuning process reduces computational overhead and training time, making the model more adaptable for new data.

Despite these advancements, the foundation model-based approach does not fully resolve all challenges related to cell segmentation and classification. We observe lower metric scores for underrepresented classes in the training data. Furthermore, foundation models typically encompass billions of parameters, resulting in substantial computational and memory requirements. It therefore poses challenges for deployment in resource-constrained environments, limiting their practical applicability in certain clinical settings.

\section{Model optimization via Knowledge Distillation}

To address the limitations posed by the extensive size of foundation models, we implement knowledge distillation — a model compression technique that leverages the teacher-student paradigm \cite{Hinton_Vinyals_etal._2015}. By training a smaller, more efficient student model to replicate the output of a larger, pre-trained teacher model, we retain performance while significantly reducing the model's complexity and resource requirements (\hyperref[fig:fig6]{Figure 6}).

\begin{figure}[h!]
    \centering
    \includegraphics[width=\textwidth, height=0.45\textheight, keepaspectratio]{images/Figure_6.pdf}
    \caption{Knowledge distillation framework for training a student model using a pre-trained teacher}
    \label{fig:fig6}
\end{figure}

We employ knowledge distillation to compress the H‑Optimus-based teacher model into a more efficient student model. The teacher model is the modified UNETR architecture with the H‑Optimus foundation model described in the previous chapter. The student model is based on a UNet architecture augmented with residual connections and incorporates a smaller ViT encoder with 9 million parameters \cite{Steiner_Kolesnikov_etal._2022,Wightman_2019}. 

First, we fine-tune the teacher model using the refined dataset from the cross-relabeling procedure (Section 2). Initially we train the decoder of the teacher model while keeping the encoder weights frozen. We split the refined dataset into train (70\%), validation (20\%) and test (10\%) subsets (\hyperref[fig:S2]{Appendix Figure S2 (4)}). During fine-tuning, we use the train and validation subsets, while leaving the test subset for model evaluation. We set the training procedure and model hyperparameters to be identical to those that were used to demonstrate the utility of foundation models for the simultaneous cell segmentation and classification task.

Next, we perform knowledge distillation from teacher to student using the refined dataset used to fine-tune the teacher model. The student model is trained to replicate the teacher model's outputs. We utilize a specialized loss function that aligns the student's predicted probability distribution with the teacher's, incorporating the teacher's class probability distribution derived from the output. Following the methodology of Hinton et al. \cite{Hinton_Vinyals_etal._2015}, we experiment with various hyperparameter settings for the temperature ($T$) and the balancing coefficients ($\alpha$ and $\beta$) in the loss function. We vary $T$ from 1 to 20 and adjust $\alpha$ and $\beta$ to balance the distillation and student losses. Through iterative tuning and evaluation, we identify that setting $T=14$, $\alpha=0.3$, and $\beta=0.7$ yields a configuration that converges and closely approximates the teacher model's performance during training.

Finally, we assess the performance of both models using the $R^2$ and $PQ$ (defined in \hyperref[chap:S5]{Appendix S5}) on the test set of the refined dataset (\hyperref[tab:3]{Table 3}). We observe that the 95\% confidence intervals overlap for most cell types, so we cannot claim statistically significant performance differences between the teacher and student models. One exception appears in the neoplastic class. The teacher model produces an $R^2$ of 0.919, while the student model shows an $R^2$ of 0.852. In addition, the student model achieves higher $PQ$ values for the neoplastic and connective classes, though the confidence intervals show overlap.

\begin{table}[h!]
\renewcommand{\arraystretch}{1.5}
  \centering
  \caption{Cell quantification metrics for teacher and distilled student models (CI 95\%).}
  \label{tab:3}
  \begin{tabular}{|l|c|c|}
    \hline
    %\rowcolor{gray!30}
    Metric & Teacher & Student \\
    \hline
    $R^2_{neoplastic}$    & \textbf{0.919} (0.898--0.939) & 0.852 (0.800--0.891) \\
    \hline
    $R^2_{lymphocyte}$    & 0.969 (0.956--0.977)         & 0.969 (0.956--0.978) \\
    \hline
    $R^2_{connective}$    & 0.694 (0.548--0.809)         & 0.618 (0.469--0.741) \\
    \hline
    $R^2_{dead}$          & 0.755 (0.400--0.908)         & 0.424 (0.100--0.731) \\
    \hline
    $R^2_{epithelial}$    & 0.922 (0.870--0.958)         & 0.843 (0.738--0.917) \\
    \hline
    $R^2_{macrophage}$    & 0.384 (-0.369--0.724)        & 0.704 (0.352--0.859) \\
    \hline
    $R^2_{neutrofil}$     & 0.854 (0.578--0.929)         & 0.833 (0.502--0.925) \\
    \hline
    $PQ_{neoplastic}$       & 0.581 (0.569--0.593)         & 0.601 (0.588--0.613) \\
    \hline
    $PQ_{lymphocyte}$       & 0.536 (0.520--0.553)         & 0.563 (0.544--0.579) \\
    \hline
    $PQ_{connective}$       & 0.436 (0.421--0.451)         & 0.457 (0.441--0.474) \\
    \hline
    $PQ_{dead}$             & 0.272 (0.235--0.315)         & 0.279 (0.201--0.369) \\
    \hline
    $PQ_{epithelial}$       & 0.522 (0.500--0.545)         & 0.530 (0.506--0.555) \\
    \hline
    $PQ_{macrophage}$       & 0.524 (0.459--0.588)         & 0.474 (0.405--0.543) \\
    \hline
    $PQ_{neutrofil}$        & 0.541 (0.490--0.592)         & 0.565 (0.522--0.607) \\
    \hline
  \end{tabular}
\end{table}


We further decompose the $PQ$ metric into its $SQ$ and $DQ$ components (\hyperref[tab:S6]{Appendix Table S6}). Both models produce nearly identical $SQ$ values, which indicates that they predict instance boundaries with similar precision. Although the student model shows some improvement in $DQ$ scores for certain classes, the confidence intervals overlap and do not confirm a statistically significant difference.

We observe that the student and teacher models yield comparable detection performance despite the student model using a much smaller and simpler architecture. A model with fewer parameters reduces the risk of overfitting when training data are scarce relative to the model’s complexity \cite{Farias_Ludermir_etal._2022}. The knowledge distillation process also encourages the student model to focus on the most generalizable detection features learned from the teacher. These factors enable the student model to achieve similar detection performance across different cell types.

Additionally, considering the model sizes reported in \hyperref[tab:4]{Table 4}, the distilled model achieves a significant reduction compared to the teacher model, with a 48-fold decrease in parameter count and a 5.5-fold reduction in on-disk size. In inference mode, the teacher model requires 16 GB of VRAM for a batch size of 32, while the distilled model only needs 3 GB of VRAM for the same batch size. These reductions make the distilled model significantly more practical for fine-tuning and deployment in resource-constrained environments.

\begin{table}[h!]
\renewcommand{\arraystretch}{1.5}
  \centering
  \caption{Parameter counts and size of teacher and distilled model}
  \label{tab:4}
  \adjustbox{max width=\textwidth}{%
  \begin{tabular}{|l|c|c|c|}
    \hline
    %\rowcolor{gray!30}
    Metric & H-optimus-based (Teacher) & mobileViT-based (Student) & Magnitude of difference \\
    \hline
    Parameters count       & 1,158,917,906   & \textbf{24,093,393}   & \textbf{48x}  \\
    \hline
    Estimated Total Size (MB) & 87,912       & \textbf{15,935}    & \textbf{5.5x} \\
    \hline
  \end{tabular}%
}
\end{table}

%\hfill

With recent advancements in complex network architectures and the use of pretrained encoders to achieve state-of-the-art performance \cite{Baumann_Dislich_etal._2024,Hörst_Rempe_etal._2024} in cell segmentation and classification tasks, model size, computational complexity, and processing times have increased. This limits the scalability and accessibility of these models. As we demonstrate, this may be mitigated using knowledge distillation. Studies in the field of natural language processing have demonstrated the efficacy of knowledge distillation in retaining the capabilities of the teacher model while achieving significant reductions in size and complexity \cite{Huangpu_Gao_2024,Sun_Yu_etal.}. 

We demonstrate the feasibility of knowledge distillation in digital pathology, specifically for cell segmentation and classification tasks. Moreover, we achieve this performance while also significantly reducing the parameter count. In addressing the challenge of knowledge transfer, we found that distillation from a transformer-based model to a smaller transformer is more straightforward than attempting to map transformer features to CNN blocks. In our experiments, using a CNN-based network as a student results in worse cell quantification performance due to the structural constraints of CNN feature space dimensions. 

Although our primary approach relies on a transformer-based student model that performs well, it can be further optimized to incorporate advantages from CNN architectures. For example, employing alternative techniques such as using ViT adapters \cite{Chen_Duan_etal._2023} or $1 \times 1$ convolutions to adjust feature map sizes may be beneficial for harnessing CNN advantages like enhanced local feature extraction. Moreover, if additional performance improvements are desired, the process can be further enhanced by applying supplementary knowledge distillation techniques, such as self-distillation \cite{Zhang_Song_etal._2019} or online distillation \cite{Houyon_Cioppa_etal._2023}.

Despite these promising results, further validation on independent datasets is necessary to fully understand the model's limitations. Underrepresented classes may pose challenges when addressing complex cases. Pathologists need to validate these models to adopt them in clinical settings. While the distilled models are smaller and more deployable, a technological gap persists because pathologists traditionally rely on established methods for inspecting WSIs and diagnosing diseases. Addressing the complexities involved in deploying models for inference and supporting pathologists in adopting new tools is essential for integrating these models into clinical workflows.

\section{Model integration with QuPath}
Digital pathology tools with graphical user interfaces are essential for visualizing and analyzing WSIs. To make our student model useful in clinical pathology workflows, it needs to be integrated into a tool that enables inspecting regions, creating annotations, and providing quantitative analyses of biomarkers. Therefore, we integrate the trained student model from the previous chapter into the QuPath open‑source platform \cite{Bankhead_Loughrey_etal._2017}. QuPath provides the required annotation, visualization, and analysis tools to interpret complex histological data, including workflows for cell segmentation, classification, and quantification (\hyperref[fig:fig7]{Figure 7}). 

\begin{figure}[h!]
    \centering
    \includegraphics[width=\textwidth]{images/Figure_7.pdf}
    \caption{Visualization of model-generated cell quantification annotations (left) and the corresponding unannotated slide (right) in QuPath}
    \label{fig:fig7}
\end{figure}

To identify the regions in a WSI critical for prognosticating tumor development, such as specific tumor areas or border regions without overlapping healthy tissue, the pathologist uses QuPath to outline these regions. Then, the pathologist initiates a cell segmentation and classification script through the QuPath interface for the selected regions. The resulting annotations and quantified cell information are then directly overlaid onto the WSI in the QuPath interface. Additional design and implementation details are in \hyperref[chap:S7]{Appendix S7}. 

Two common approaches for integrating deep learning models into QuPath are Java‑based native QuPath extensions \cite{Goldsborough_Philps_etal._2024} and the execution of RESTful API requests to a model server coupled with handling the response via an extension, as demonstrated in the application of cell segmentation models applied to immunofluorescence images \cite{Sugawara_2023}. While the community is actively working on these integration strategies, there is currently no universal solution that fully addresses all integration and performance requirements.

Extensions may offer better integration with QuPath, allowing slightly improved performance and more widespread usage of the built-in QuPath models, but they lack the flexibility to customize models and modify their behavior. For example, the newest version of QuPath includes models such as StarDist \cite{Weigert_Schmidt} and InstanSeg \cite{Goldsborough_Philps_etal._2024} that can perform cell segmentation. Both models pose limitations when applied to simultaneous cell segmentation and classification. StarDist performs well only on convex, round shapes by design, whereas some neoplastic, inflammatory, and connective cells exhibit complex and non-convex shapes. InstanSeg provides only semantic segmentation without assigning classes to the segmented cells.

%\hfill

In contrast, our approach offers an alternative integration strategy. It utilizes the paquo library to directly interact with QuPath’s internal application programming interface from within Python. This enables data exchange and processing without the need for intermediate conversion steps and provides greater control over model customization, retraining, and the incorporation of custom processing steps.

The integration of our custom model with QuPath underscores its potential to significantly enhance the diagnostic process by reducing the time burden on pathologists and enabling them to focus on more complex interpretative tasks using familiar software. Leveraging a tool that is already well-established among pathologists increases the likelihood of its adoption into daily clinical workflows. The quantitative data generated through the automated workflow is critical for both clinical decision-making and research, facilitating more accurate biomarker analysis, enabling robust statistical evaluations, and supporting hypothesis generation and testing. Additionally, by streamlining cell segmentation and classification, the tool enhances the scalability and reproducibility of pathological assessments, ultimately contributing to improved diagnostic accuracy and patient outcomes.

\section{Conclusion and future work}

In this study, we address critical challenges in digital pathology and tackle the usability and deployment issues of the developed models in standard computing environments without the need for high-performance computing systems. Our multi-faceted approach encompasses data refinement through cross-relabeling, leveraging foundation models for robust cell segmentation and classification, optimizing model performance via knowledge distillation, and integrating the optimized model into the QuPath software for practical application. This approach is used to construct a capable, versatile, and adjustable model for cell segmentation and classification, with enhanced performance and usability.

\begin{sloppypar}
While our approach shows potential in the field of computational pathology, certain limitations persist. 
For example, our implementation currently exhibits lower performance in detecting macrophages. 
This serves as an instance of the broader challenge of accurately identifying complex cell types. In order to address this issue, extending our approach to incorporate additional data sources, exploring alternative modeling approaches, and integrating other imaging modalities such as immunohistochemical staining may help improve detection accuracy. Moreover, although the distilled model reduces computational demands, integrating advanced deep learning models into clinical practice requires addressing technological gaps and potential resistance to adopting new tools within established diagnostic processes.
\end{sloppypar}

Future work could focus on several key areas to refine the proposed approach and facilitate its adoption in clinical environments. Enhancing the cell-relabeling process with additional datasets \cite{Graham_Jahanifar_etal._2021} could improve the representation of underrepresented cell types and enhance overall model performance. Also, incorporating additional data sources, such as multi-modal imaging or complementary staining methods, may address limitations related to cell type differentiation and class imbalance. Exploring other foundation models \cite{Vorontsov_Bozkurt_etal._2024,Zimmermann_Vorontsov_etal._2024} or introducing additional modalities \cite{Ding_Wagner_etal._2024,Vaidya_Zhang_etal._2025} may provide alternative architectures better suited to specific tasks or offer improved efficiency. Implementing more complex knowledge distillation techniques \cite{Houyon_Cioppa_etal._2023,Zhang_Song_etal._2019} could further optimize the model's performance and adaptability. Additionally, deeper integration with QuPath or other digital pathology software could provide pathologists more control over cell quantification analysis directly within the QuPath interface, thereby increasing accessibility and usability. Such enhancements would not only refine model performance but also ensure greater adaptability and scalability within various clinical environments. Finally, extensive validation of the model by pathologists and benchmarking against independent datasets are essential steps toward establishing the model's reliability and fostering confidence in its clinical utility.

\section*{Acknowledgments} 
This work was funded in part by the Research Council of Norway grant no. 309439 SFI Visual Intelligence, and the North Norwegian Health Authority grant no. HNF1521-20.

\bibliographystyle{IEEEtran}
\begin{sloppypar}
\begin{thebibliography}{99}

\bibitem{chaplot2020neural} Chaplot, Devendra Singh, et al. "Neural topological slam for visual navigation." Proceedings of the IEEE/CVF conference on computer vision and pattern recognition. 2020.

\bibitem{maksymets2021thda} Maksymets, Oleksandr, et al. "Thda: Treasure hunt data augmentation for semantic navigation." Proceedings of the IEEE/CVF International Conference on Computer Vision. 2021.

\bibitem{mezghan2022memory} Mezghan, Lina, et al. "Memory-augmented reinforcement learning for image-goal navigation." 2022 IEEE/RSJ International Conference on Intelligent Robots and Systems (IROS). IEEE, 2022.

\bibitem{al2022zero} Al-Halah, Ziad, Santhosh Kumar Ramakrishnan, and Kristen Grauman. "Zero experience required: Plug \& play modular transfer learning for semantic visual navigation." Proceedings of the IEEE/CVF Conference on Computer Vision and Pattern Recognition. 2022.

\bibitem{ye2021auxiliary} Ye, Joel, et al. "Auxiliary tasks and exploration enable objectgoal navigation." Proceedings of the IEEE/CVF international conference on computer vision. 2021.

\bibitem{chaplot2020object} Chaplot, Devendra Singh, et al. "Object goal navigation using goal-oriented semantic exploration." Advances in Neural Information Processing Systems 33 (2020)

\bibitem{ramakrishnan2022poni} Ramakrishnan, Santhosh Kumar, et al. "Poni: Potential functions for objectgoal navigation with interaction-free learning." Proceedings of the IEEE/CVF Conference on Computer Vision and Pattern Recognition. 2022.

\bibitem{ramrakhya2022habitat} Ramrakhya, Ram, et al. "Habitat-web: Learning embodied object-search strategies from human demonstrations at scale." Proceedings of the IEEE/CVF Conference on Computer Vision and Pattern Recognition. 2022.

\bibitem{mousavian2019visual} Mousavian, Arsalan, et al. "Visual representations for semantic target driven navigation." 2019 International Conference on Robotics and Automation (ICRA). IEEE, 2019.

\bibitem{dhariwal2021diffusion} Dhariwal, Prafulla, and Alexander Nichol. "Diffusion models beat gans on image synthesis." Advances in neural information processing systems 34 (2021)

\bibitem{ho2022classifier} Ho, Jonathan, and Tim Salimans. "Classifier-free diffusion guidance." arXiv preprint arXiv:2207.12598 (2022).

\bibitem{nichol2021glide} Nichol, Alex, et al. "Glide: Towards photorealistic image generation and editing with text-guided diffusion models." arXiv preprint arXiv:2112.10741 (2021)

\bibitem{brooks2023instructpix2pix} Brooks, Tim, Aleksander Holynski, and Alexei A. Efros. "Instructpix2pix: Learning to follow image editing instructions." Proceedings of the IEEE/CVF Conference on Computer Vision and Pattern Recognition. 2023.

\bibitem{fu2023guiding} Fu, Tsu-Jui, et al. "Guiding instruction-based image editing via multimodal large language models." arXiv preprint arXiv:2309.17102 (2023).

\bibitem{geng2024instructdiffusion} Geng, Zigang, et al. "Instructdiffusion: A generalist modeling interface for vision tasks." Proceedings of the IEEE/CVF Conference on Computer Vision and Pattern Recognition. 2024.

\bibitem{zhou2024minedreamer} Zhou, Enshen, et al. "Minedreamer: Learning to follow instructions via chain-of-imagination for simulated-world control." arXiv preprint arXiv:2403.12037 (2024).

\bibitem{zhou2023esc} Zhou, Kaiwen, et al. "Esc: Exploration with soft commonsense constraints for zero-shot object navigation." International Conference on Machine Learning. PMLR, 2023.

\bibitem{yu2023l3mvn} Yu, Bangguo, Hamidreza Kasaei, and Ming Cao. "L3mvn: Leveraging large language models for visual target navigation." 2023 IEEE/RSJ International Conference on Intelligent Robots and Systems (IROS). IEEE, 2023.

\bibitem{gadre2023cows} Gadre, Samir Yitzhak, et al. "Cows on pasture: Baselines and benchmarks for language-driven zero-shot object navigation." Proceedings of the IEEE/CVF Conference on Computer Vision and Pattern Recognition. 2023.

\bibitem{shah2023navigation} Shah, Dhruv, et al. "Navigation with large language models: Semantic guesswork as a heuristic for planning." Conference on Robot Learning. PMLR, 2023.

\bibitem{cai2024bridging} Cai, Wenzhe, et al. "Bridging zero-shot object navigation and foundation models through pixel-guided navigation skill." 2024 IEEE International Conference on Robotics and Automation (ICRA). IEEE, 2024.

\bibitem{yu2023co} Yu, Bangguo, Hamidreza Kasaei, and Ming Cao. "Co-NavGPT: Multi-robot cooperative visual semantic navigation using large language models." arXiv preprint arXiv:2310.07937 (2023).

\bibitem{wu2024voronav} Wu, Pengying, et al. "Voronav: Voronoi-based zero-shot object navigation with large language model." arXiv preprint arXiv:2401.02695 (2024).

\bibitem{qin2023mp5} Qin, Yiran, et al. "Mp5: A multi-modal open-ended embodied system in minecraft via active perception." arXiv preprint arXiv:2312.07472 (2023).

\bibitem{du2024learning} Du, Yilun, et al. "Learning universal policies via text-guided video generation." Advances in Neural Information Processing Systems 36 (2024).

\bibitem{ajay2024compositional} Ajay, Anurag, et al. "Compositional foundation models for hierarchical planning." Advances in Neural Information Processing Systems 36 (2024).

\bibitem{liang2024skilldiffuser} Liang, Zhixuan, et al. "Skilldiffuser: Interpretable hierarchical planning via skill abstractions in diffusion-based task execution." Proceedings of the IEEE/CVF Conference on Computer Vision and Pattern Recognition. 2024.

\bibitem{heusel2017gans} Heusel, Martin, et al. "Gans trained by a two time-scale update rule converge to a local nash equilibrium." Advances in neural information processing systems 30 (2017).

\bibitem{zhang2018unreasonable} Zhang, Richard, et al. "The unreasonable effectiveness of deep features as a perceptual metric." Proceedings of the IEEE conference on computer vision and pattern recognition. 2018.

\bibitem{brown2020language} Brown, Tom B. "Language models are few-shot learners." arXiv preprint arXiv:2005.14165 (2020).

\bibitem{podell2023sdxl} Podell, Dustin, et al. "Sdxl: Improving latent diffusion models for high-resolution image synthesis." arXiv preprint arXiv:2307.01952 (2023).

\bibitem{brohan2022rt} Brohan, Anthony, et al. "Rt-1: Robotics transformer for real-world control at scale." arXiv preprint arXiv:2212.06817 (2022).

\bibitem{brohan2023rt} Brohan, Anthony, et al. "Rt-2: Vision-language-action models transfer web knowledge to robotic control." arXiv preprint arXiv:2307.15818 (2023).

\bibitem{li2024manipllm} Li, Xiaoqi, et al. "Manipllm: Embodied multimodal large language model for object-centric robotic manipulation." Proceedings of the IEEE/CVF Conference on Computer Vision and Pattern Recognition. 2024.

\bibitem{shah2023vint} Shah, Dhruv, et al. "ViNT: A foundation model for visual navigation." arXiv preprint arXiv:2306.14846 (2023).

\bibitem{liu2024visual} Liu, Haotian, et al. "Visual instruction tuning." Advances in neural information processing systems 36 (2024).

\bibitem{hu2021lora} Hu, Edward J., et al. "Lora: Low-rank adaptation of large language models." arXiv preprint arXiv:2106.09685 (2021).

\bibitem{qin2023supfusion} Qin, Yiran, et al. "SupFusion: Supervised LiDAR-camera fusion for 3D object detection." Proceedings of the IEEE/CVF International Conference on Computer Vision. 2023.

\bibitem{qin2024worldsimbench} Qin, Yiran, et al. "Worldsimbench: Towards video generation models as world simulators." arXiv preprint arXiv:2410.18072 (2024).

\bibitem{yu2025gamefactory} Yu, Jiwen, et al. "GameFactory: Creating New Games with Generative Interactive Videos." arXiv preprint arXiv:2501.08325 (2025).

\bibitem{zhou2024code} Zhou, Enshen, et al. "Code-as-Monitor: Constraint-aware Visual Programming for Reactive and Proactive Robotic Failure Detection." arXiv preprint arXiv:2412.04455 (2024).

\bibitem{zhang2024ad} Zhang, Zaibin, et al. "AD-H: Autonomous Driving with Hierarchical Agents." arXiv preprint arXiv:2406.03474 (2024).

\bibitem{wang2024toward} Wang, Chaoqun, et al. "Toward Accurate Camera-based 3D Object Detection via Cascade Depth Estimation and Calibration." arXiv preprint arXiv:2402.04883 (2024).

\bibitem{huang2024story3d} Huang, Yuzhou, et al. "Story3d-agent: Exploring 3d storytelling visualization with large language models." arXiv preprint arXiv:2408.11801 (2024).

\bibitem{savinov2018semi} Savinov, Nikolay, Alexey Dosovitskiy, and Vladlen Koltun. "Semi-parametric topological memory for navigation." arXiv preprint arXiv:1803.00653 (2018).

\bibitem{majumdar2022zson} Majumdar, Arjun, et al. "Zson: Zero-shot object-goal navigation using multimodal goal embeddings." Advances in Neural Information Processing Systems 35 (2022): 32340-32352.

\bibitem{yadav2023offline} Yadav, Karmesh, et al. "Offline visual representation learning for embodied navigation." Workshop on Reincarnating Reinforcement Learning at ICLR 2023. 2023.

\bibitem{yadav2023ovrl} Yadav, Karmesh, et al. "Ovrl-v2: A simple state-of-art baseline for imagenav and objectnav." arXiv preprint arXiv:2303.07798 (2023).

\bibitem{sun2024fgprompt} Sun, Xinyu, et al. "FGPrompt: fine-grained goal prompting for image-goal navigation." Advances in Neural Information Processing Systems 36 (2024).

\bibitem{zhu2017target} Zhu, Yuke, et al. "Target-driven visual navigation in indoor scenes using deep reinforcement learning." 2017 IEEE international conference on robotics and automation (ICRA). IEEE, 2017.

\bibitem{koh2024generating} Koh, Jing Yu, Daniel Fried, and Russ R. Salakhutdinov. "Generating images with multimodal language models." Advances in Neural Information Processing Systems 36 (2024).

\bibitem{krantz2022instance} Krantz, Jacob, et al. "Instance-specific image goal navigation: Training embodied agents to find object instances." arXiv preprint arXiv:2211.15876 (2022).

\bibitem{schulman2017proximal} Schulman, John, et al. "Proximal policy optimization algorithms." arXiv preprint arXiv:1707.06347 (2017).

\bibitem{anderson2018evaluation} Anderson, Peter, et al. "On evaluation of embodied navigation agents." arXiv preprint arXiv:1807.06757 (2018).

\bibitem{lin2024navcot} Lin, Bingqian, et al. "NavCoT: Boosting LLM-Based Vision-and-Language Navigation via Learning Disentangled Reasoning." arXiv preprint arXiv:2403.07376 (2024).

\bibitem{NavGPT} Zhou, Gengze, Yicong Hong, and Qi Wu. "Navgpt: Explicit reasoning in vision-and-language navigation with large language models." Proceedings of the AAAI Conference on Artificial Intelligence.

\bibitem{hahn2021no} Hahn, Meera, et al. "No rl, no simulation: Learning to navigate without navigating." Advances in Neural Information Processing Systems 34 (2021): 26661-26673.

\bibitem{li2025t2isafety} Li, Lijun, et al. "T2ISafety: Benchmark for Assessing Fairness, Toxicity, and Privacy in Image Generation." arXiv preprint arXiv:2501.12612 (2025).

\bibitem{an2024agfsync} An, Jingkun, et al. "AGFSync: Leveraging AI-Generated Feedback for Preference Optimization in Text-to-Image Generation." arXiv preprint arXiv:2403.13352 (2024).


\end{thebibliography}
\end{sloppypar}

\clearpage
\beginsupplement
\section*{Appendix}
\renewcommand{\thesubsection}{S\arabic{subsection}}

\subsection{\label{chap:S1}PanNuke and MoNuSAC preprocessing}
The PanNuke dataset comprises a set of 7,901 RGB patches, each with dimensions of $256 \times 256$ pixels, which we set as the standard patch size for our analysis. In contrast, the MoNuSAC dataset encompasses 294 images of heterogeneous dimensions. To standardize the MoNuSAC images with our experiments, we implement a standardization protocol. Specifically, for images exceeding the dimensions of $256 \times 256$ pixels, we segment them into equal-sized patches and apply mirror padding to the remaining portions to avoid information loss at the peripherals. Patches with dimensions less than $128 \times 128$ pixels are excluded from the dataset due to the insufficient resolution to capture relevant cellular details. For patches where either dimension falls between 128 and 256 pixels, we employ upsampling to achieve the standard patch size. As a result, we obtain a total of 2,823 RGB patches derived from the MoNuSAC dataset for subsequent analysis. For additional details on the MoNuSAC data preparation process, refer to the source code \cite{Shvetsov_2025a}.
\clearpage

\subsection{\label{chap:S2}Data usage for the methodology}

\counterwithin{figure}{subsection}
\renewcommand{\thefigure}{S\arabic{subsection}}

\begin{figure}[h!]
    \centering
    \includegraphics[width=\textwidth, height=0.85\textheight, keepaspectratio]{images/A2.pdf}
    \caption{Overview of the methodology for cross-labeling, dataset refinement, and model comparison. (1) Cross-relabeling - training and testing cell classification models, (2) Cross-relabeling - using cell classification models to create refined dataset, (3) Fine-tuning and training models for comparison, (4) Student knowledge distillation with refined dataset}
    \label{fig:S2}
\end{figure}
\clearpage

\subsection{\label{chap:S3}Confusion matrices for classification models}
\counterwithin{figure}{subsection}
\renewcommand{\thefigure}{S\arabic{subsection}.\arabic{figure}}

\begin{figure}[h!]
    \centering
    \includegraphics[width=\textwidth, height=0.4\textheight, keepaspectratio]{images/A3_1.pdf}
    \caption{Confusion matrix for PanNuke trained model}
    \label{fig:S3.1}
\end{figure}

\begin{figure}[h!]
    \centering
    \includegraphics[width=\textwidth, height=0.4\textheight, keepaspectratio]{images/A3_2.pdf}
    \caption{Confusion matrix for MoNuSAC trained model}
    \label{fig:S3.2}
\end{figure}

\clearpage

\subsection{\label{chap:S4}Datasets cell counts}

\counterwithin{table}{subsection}
\renewcommand{\thetable}{S\arabic{subsection}}

\begin{table}[h!]
\renewcommand{\arraystretch}{2.0}
\centering
\caption{\label{tab:S4}Cell counts for PanNuke, MoNuSAC and refined datasets. Numbers in parentheses indicate preprocessed cell counts for cell classifier models training and testing.}
%\adjustbox{max width=\textwidth}{%
\begin{tabular}{|l|c|c|c|}
\hline
%\rowcolor{gray!30}
Cell type & PanNuke & MoNuSAC & Refined \\
\hline
Neoplastic & 77,403 (68,031) & - & 105,451 \\
\hline
Epithelial & 26,572 (23,207) & - & 29,926 \\
\hline
Epithelial (benign and malignant) & - & 31,402 & - \\
\hline
Inflammatory & 32,276 & - & - \\
\hline
Lymphocytes & - & 37,045 (33,104) & 65,275 \\
\hline
Neutrophils & - & 1,355 (1,252) & 3,833 \\
\hline
Macrophage & - & 1,842 (1,695) & 3,410 \\
\hline
Dead & 2,908 & - & 2,908 \\
\hline
Connective & 50,585 & - & 50,585 \\
\hline
\end{tabular}
%
%}
\end{table}



\clearpage

\subsection{\label{chap:S5}Definition of validation metrics}
\counterwithin{equation}{subsection}
\renewcommand{\theequation}{\arabic{equation}}

\subsubsection{\label{chap:S5.1}R\textsuperscript{2}}
The coefficient of determination, denoted as $R^2$, is a statistical measure that represents the proportion of variance in the dependent variable that is predictable from the independent variables. In the context of cell quantification in pathology, $R^2$ is used to assess how well the predicted quantities of different cell types in a patch align with the actual quantities observed in the ground truth data, with higher values representing more accurate quantification. $R^2$ is defined as
\begin{equation*}
R^2 = 1 - \frac{\sum_{i=1}^n (y_i - \hat{y}_i)^2}{\sum_{i=1}^n (y_i - \bar{y})^2},
\end{equation*}
where $y_i$ represents the actual number of cells of a specific type in the $i$-th image, $\hat{y}_i$ represents the predicted number of cells of that type in the $i$-th image, $\bar{y}$ is the mean of the actual numbers across all images, and $n$ is the total number of images in the dataset.

The $R^2$ metric has a range of $(-\infty, 1]$. An $R^2$ of 1 indicates perfect prediction, where all predicted values exactly match the actual values. An $R^2$ of 0 suggests that the model explains none of the variability of the response data around its mean. If $R^2$ is negative, it indicates that the model performs worse than a model that simply predicts the mean of the actual values for all observations.

\subsubsection{\label{chap:S5.2}PQ}
Panoptic Quality ($PQ$) is a comprehensive metric used to evaluate the performance of segmentation models in tasks that require both instance segmentation and classification. $PQ$ provides a single score that encapsulates both the detection accuracy (i.e., how many objects were correctly identified) and the segmentation quality (i.e., how accurately the objects' boundaries were delineated). This metric is particularly useful in multiclass scenarios where each pixel is classified into distinct categories, such as different cell types in pathology images.

$PQ$ is calculated as the product of two terms: Detection Quality ($DQ$) and Segmentation Quality ($SQ$). It can be expressed as
\begin{equation*}
PQ = DQ \cdot SQ,
\end{equation*}
where
\begin{equation*}
DQ = \frac{TP}{TP + 0.5\, FP + 0.5\, FN},
\end{equation*}
\begin{equation*}
SQ = \frac{\sum_{(p, g) \in \mathcal{M}} IoU(p, g)}{TP}.
\end{equation*}
In these formulas, $TP$ denotes the number of correctly matched instances between ground truth and prediction, $FP$ denotes the predicted instances that have no corresponding ground truth, $FN$ denotes the ground truth instances that were not detected, $IoU(p, g)$ is the Intersection over Union for a pair of matched instances $p$ (prediction) and $g$ (ground truth), and $\mathcal{M}$ is the set of matched pairs.

The $PQ$ metric is calculated for each class and is averaged across classes to provide a global performance measure.

The $PQ$ score has a range of $[0, 1.0]$, where a higher score indicates better performance in both detecting and segmenting the instances correctly. A $PQ$ of 1 signifies perfect identification and segmentation of all instances, whereas a $PQ$ of 0 indicates that no instances were correctly identified and segmented.

\clearpage

\subsection{\label{chap:S6}Segmentation and Detection quality metrics for teacher and student models}

\begin{table}[h!]
\renewcommand{\arraystretch}{2.0}
\centering
\caption{Segmentation and detection quality for student and teacher models (CI 95\%)}
\label{tab:S6}
%\adjustbox{max width=\textwidth}{%
\begin{tabular}{|l|c|c|}
\hline
%\rowcolor{gray!30}
Metric & Teacher & Student \\
\hline
$SQ_{neoplastic}$ & 0.819 (0.815--0.823) & 0.824 (0.819--0.828) \\
\hline
$SQ_{lymphocyte}$ & 0.795 (0.788--0.802) & 0.790 (0.783--0.796) \\
\hline
$SQ_{connective}$ & 0.770 (0.762--0.776) & 0.780 (0.772--0.786) \\
\hline
$SQ_{dead}$ & 0.659 (0.623--0.688) & 0.657 (0.624--0.695) \\
\hline
$SQ_{epithelial}$ & 0.780 (0.770--0.790) & 0.788 (0.779--0.797) \\
\hline
$SQ_{macrophage}$ & 0.788 (0.760--0.810) & 0.757 (0.730--0.783) \\
\hline
$SQ_{neutrofil}$ & 0.782 (0.761--0.801) & 0.775 (0.759--0.792) \\
\hline
$DQ_{neoplastic}$ & 0.706 (0.692--0.719) & 0.727 (0.712--0.741) \\
\hline
$DQ_{lymphocyte}$ & 0.675 (0.656--0.698) & 0.713 (0.691--0.734) \\
\hline
$DQ_{connective}$ & 0.566 (0.546--0.584) & 0.583 (0.565--0.602) \\
\hline
$DQ_{dead}$ & 0.410 (0.361--0.465) & 0.435 (0.306--0.561) \\
\hline
$DQ_{epithelial}$ & 0.668 (0.639--0.694) & 0.673 (0.644--0.702) \\
\hline
$DQ_{macrophage}$ & 0.657 (0.583--0.727) & 0.615 (0.531--0.703) \\
\hline
$DQ_{neutrofil}$ & 0.691 (0.625--0.753) & 0.729 (0.679--0.778) \\
\hline
\end{tabular}
%
%}
\end{table}

\clearpage

\subsection{\label{chap:S7}QuPath integration method}
We adopt an integration strategy leveraging the paquo \cite{Bayer_AG} library, a Python package that enables direct interaction with QuPath’s internal API, thereby facilitating seamless data exchange without intermediate conversion steps. The data processing pipeline (\hyperref[fig:S7]{Appendix Figure S7}) begins with the acquisition of WSIs and their associated annotations from QuPath, which are represented as Shapely \cite{Gillies_Wel_etal._2024} polygons. Utilizing paquo, we directly read, create, and modify these annotations and detections within a QuPath project in the Python environment. Images are then cropped using these polygons and processed by cell segmentation and classification models employing standard vision processing toolkits such as OpenCV, pyvips, and PyTorch. Additionally, QuPath employs Groovy scripts to initiate a Python process that starts the entire pipeline from QuPath graphical interface: fetching polygons, extracting images from them, and running deep learning model inference on the cropped images. 
The results are returned to QuPath, leveraging paquo's Python bindings to manipulate QuPath data while minimizing the computational overhead typically associated with cross-environment communication.

\counterwithin{figure}{subsection}
\renewcommand{\thefigure}{S\arabic{subsection}}

\begin{figure}[h!]
    \centering
    \includegraphics[width=\textwidth]{images/A7.pdf}
    \caption{QuPath integration workflow using Python environment}
    \label{fig:S7}
\end{figure}

Compared to traditional workflows that involve exporting annotations as GeoJSON, classifying them in Python, and reimporting them into QuPath, our approach offers several advantages. We eliminate the need to switch between programming languages, providing a cohesive and streamlined development process entirely within QuPath software and removing the necessity to use other tools. Meanwhile, we avoid storing annotations as intermediate JSON files unless required for external use or archiving. By conducting the entire inference and post-processing workflow within the Python environment, we leverage the power and flexibility of Python libraries for image processing and machine learning. This approach also enables adjustments to any set of labels and models, thereby improving its applicability.

%\hfill

The distilled model and QuPath integration code are packaged into a Docker container, enabling streamlined execution with the Docker engine. Detailed integration code and deployment instructions can be found in the GitHub repository \cite{Shvetsov_2025b}.

Despite these benefits, we acknowledge that the paquo library is a proof‑of‑concept project in its early development stage and has not been tested across all versions of QuPath.

\clearpage

\subsection{\label{chap:S8}Data and code availability statement}
All datasets, models, and code used in this study are publicly available and can be obtained from the repositories listed below. 
The PanNuke \cite{Gamper_Koohbanani_etal._2019} and MoNuSAC \cite{Verma_Kumar_etal._2021} datasets are publicly accessible, and download information along with detailed descriptions can be found in their respective articles. Preprocessing scripts for PanNuke and MoNuSAC data, as well as individual cell extraction scripts, are available on GitHub \cite{Shvetsov_2025a}. The H-Optimus foundation model used in our experiments can be downloaded from the HuggingFace repository \cite{hoptimus2024}, and model information is available on GitHub \cite{Saillard_Jenatton_etal._2024}. In addition, the integration code for QuPath and the distilled model packaged in a Docker container are provided in the repository \cite{Shvetsov_2025b}, and paquo Python library is available from the authors GitHub repository \cite{Bayer_AG}.
\clearpage

\end{document}


\end{document}
