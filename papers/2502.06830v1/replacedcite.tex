\section{Related Work}
\label{sec:relatedwork}

\subsection{Cross-Attention}

Cross-attention has proven to be a powerful mechanism for capturing complex dependencies in sequential and structured data across various fields. In multivariate time series forecasting, it enables the fusion of temporal and static feature embeddings, as demonstrated in ____ and  ____, enhancing predictive performance by modeling intricate relationships between variables. In computer vision, ____ uses cross-attention to combine multi-scale image patch embeddings, improving classification accuracy. Similarly, in point cloud processing, ____ applies cross-attention between 2D and 3D representations to learn richer shape features in a self-supervised manner, while ____ utilizes cross-attention to integrate global and local information. Despite its success in various domains, cross-attention remains underexplored in intraday price prediction. Our proposed OrderFusion tailored with a jump cross-attention backbone aims to model the bid-offer interdependencies. 

\begin{algorithm}[tb]
   \caption{Orderbook Filtering}
   \label{alg:orderbookfiltering}
\begin{algorithmic}
   \STATE {\bfseries Input:} Raw orderbook files
   \STATE Initialize data structures for storing processed orders
   \FORALL{orderbook files}
      \STATE Load orderbook
      \STATE Classify orders by product type (hourly/quarter-hourly)
      \IF{hourly products exist}
         \STATE Group records by OrderID and transaction time
         \STATE Filter entries with action codes ``P'' or ``M'' 
         \STATE Compute traded volume between consecutive updates
         \STATE Aggregate processed hourly trades
      \ENDIF
      \IF{quarter-hourly products exist}
         \STATE Apply the same processing steps as hourly products
         \STATE Aggregate processed quarter-hourly trades
      \ENDIF
   \ENDFOR
   \STATE \textbf{Return} Aggregated and filtered orderbook
\end{algorithmic}
\end{algorithm}





\begin{figure*}[t]
\vskip 0.1in
\begin{center}
\centerline{\includegraphics[width=1\textwidth]{Figures/DataAnalysis.pdf}}
\caption{Distribution and seasonal patterns of four intraday price indices. 
\textbf{a}, The distribution of 15-min ID$_3$ exhibits a noticeable shift from 2022 to 2023 and 2024, indicating increasing price stability within the normal range.
\textbf{b}, The boxplot reveals a seasonal pattern of 15-min ID$_3$ each year, with price fluctuations varying by quarter.
\textbf{c}, The count of negative prices steadily rises over the years, reflecting the growing influence of renewable energy integration in the market.}
\label{datanalysis}
\end{center}
\vskip -0.3in
\end{figure*}

\subsection{Multi-Quantile Prediction and Quantile Crossing}

Multi-quantile prediction frameworks are becoming increasingly popular to capture uncertainties in price forecasts. The studies ____ aim to reduce the complexity of training by jointly predicting several quantiles from a shared representation. However, a well-known issue is quantile crossing, where upper quantiles occasionally yield lower values than lower quantiles. This inconsistency violates the fundamental property of cumulative distribution functions and can drastically reduce the reliability of interval forecasts ____.
Previous works attempt to fix crossing errors via post-processing methods such as simply re-sorting quantiles ____.
Although straightforward, such solutions risk distorting the learned distribution by imposing an artificial correction step. 
____ introduces an incremental quantile function that anchors at the lowest quantile and employs non-negative functions, such as ReLU or Softplus, to learn positive residuals, which are then hierarchically added until reaching the highest quantile. However, this approach is prone to error accumulation through the process of iterative addition. Drawing inspiration from this design, we anchor at the median quantile and apply addition and subtraction to estimate tail quantiles,  reducing the risk of error accumulation.