%\documentclass[twoside]{article}

%\usepackage{aistats2025}
\documentclass[twoside]{article}
\usepackage[accepted]{aistats2025}




\usepackage{amsfonts}       % blackboard math symbols
\usepackage{nicefrac}       % compact symbols for 1/2, etc.
\usepackage{microtype}      % microtypography
\usepackage{xcolor}         % colors
\usepackage{subcaption} % for subfigures
\usepackage{afterpage}
\usepackage{placeins}
\usepackage{multirow}
\usepackage{float} % Add this in the preamble
\usepackage{placeins}

\usepackage{comment}
\raggedbottom

% If your paper is accepted, change the options for the package
% aistats2025 as follows:
%
%\usepackage[accepted]{aistats2025}
%
% This option will print headings for the title of your paper and
% headings for the authors names, plus a copyright note at the end of
% the first column of the first page.

% If you set papersize explicitly, activate the following three lines:
%\special{papersize = 8.5in, 11in}
%\setlength{\pdfpageheight}{11in}
%\setlength{\pdfpagewidth}{8.5in}

% If you use natbib package, activate the following three lines:
\usepackage[round]{natbib}
\renewcommand{\bibname}{References}
\renewcommand{\bibsection}{\subsubsection*{\bibname}}

% If you use BibTeX in apalike style, activate the following line:
\bibliographystyle{apalike}

% !TeX root = main.tex 


\newcommand{\lnote}{\textcolor[rgb]{1,0,0}{Lydia: }\textcolor[rgb]{0,0,1}}
\newcommand{\todo}{\textcolor[rgb]{1,0,0.5}{To do: }\textcolor[rgb]{0.5,0,1}}


\newcommand{\state}{S}
\newcommand{\meas}{M}
\newcommand{\out}{\mathrm{out}}
\newcommand{\piv}{\mathrm{piv}}
\newcommand{\pivotal}{\mathrm{pivotal}}
\newcommand{\isnot}{\mathrm{not}}
\newcommand{\pred}{^\mathrm{predict}}
\newcommand{\act}{^\mathrm{act}}
\newcommand{\pre}{^\mathrm{pre}}
\newcommand{\post}{^\mathrm{post}}
\newcommand{\calM}{\mathcal{M}}

\newcommand{\game}{\mathbf{V}}
\newcommand{\strategyspace}{S}
\newcommand{\payoff}[1]{V^{#1}}
\newcommand{\eff}[1]{E^{#1}}
\newcommand{\p}{\vect{p}}
\newcommand{\simplex}[1]{\Delta^{#1}}

\newcommand{\recdec}[1]{\bar{D}(\hat{Y}_{#1})}





\newcommand{\sphereone}{\calS^1}
\newcommand{\samplen}{S^n}
\newcommand{\wA}{w}%{w_{\mathfrak{a}}}
\newcommand{\Awa}{A_{\wA}}
\newcommand{\Ytil}{\widetilde{Y}}
\newcommand{\Xtil}{\widetilde{X}}
\newcommand{\wst}{w_*}
\newcommand{\wls}{\widehat{w}_{\mathrm{LS}}}
\newcommand{\dec}{^\mathrm{dec}}
\newcommand{\sub}{^\mathrm{sub}}

\newcommand{\calP}{\mathcal{P}}
\newcommand{\totspace}{\calZ}
\newcommand{\clspace}{\calX}
\newcommand{\attspace}{\calA}

\newcommand{\Ftil}{\widetilde{\calF}}

\newcommand{\totx}{Z}
\newcommand{\classx}{X}
\newcommand{\attx}{A}
\newcommand{\calL}{\mathcal{L}}



\newcommand{\defeq}{\mathrel{\mathop:}=}
\newcommand{\vect}[1]{\ensuremath{\mathbf{#1}}}
\newcommand{\mat}[1]{\ensuremath{\mathbf{#1}}}
\newcommand{\dd}{\mathrm{d}}
\newcommand{\grad}{\nabla}
\newcommand{\hess}{\nabla^2}
\newcommand{\argmin}{\mathop{\rm argmin}}
\newcommand{\argmax}{\mathop{\rm argmax}}
\newcommand{\Ind}[1]{\mathbf{1}\{#1\}}

\newcommand{\norm}[1]{\left\|{#1}\right\|}
\newcommand{\fnorm}[1]{\|{#1}\|_{\text{F}}}
\newcommand{\spnorm}[2]{\left\| {#1} \right\|_{\text{S}({#2})}}
\newcommand{\sigmin}{\sigma_{\min}}
\newcommand{\tr}{\text{tr}}
\renewcommand{\det}{\text{det}}
\newcommand{\rank}{\text{rank}}
\newcommand{\logdet}{\text{logdet}}
\newcommand{\trans}{^{\top}}
\newcommand{\poly}{\text{poly}}
\newcommand{\polylog}{\text{polylog}}
\newcommand{\st}{\text{s.t.~}}
\newcommand{\proj}{\mathcal{P}}
\newcommand{\projII}{\mathcal{P}_{\parallel}}
\newcommand{\projT}{\mathcal{P}_{\perp}}
\newcommand{\projX}{\mathcal{P}_{\mathcal{X}^\star}}
\newcommand{\inner}[1]{\langle #1 \rangle}

\renewcommand{\Pr}{\mathbb{P}}
\newcommand{\Z}{\mathbb{Z}}
\newcommand{\N}{\mathbb{N}}
\newcommand{\R}{\mathbb{R}}
\newcommand{\E}{\mathbb{E}}
\newcommand{\F}{\mathcal{F}}
\newcommand{\var}{\mathrm{var}}
\newcommand{\cov}{\mathrm{cov}}


\newcommand{\calN}{\mathcal{N}}

\newcommand{\jccomment}{\textcolor[rgb]{1,0,0}{C: }\textcolor[rgb]{1,0,1}}
\newcommand{\fracpar}[2]{\frac{\partial #1}{\partial  #2}}

\newcommand{\A}{\mathcal{A}}
\newcommand{\B}{\mat{B}}
%\newcommand{\C}{\mat{C}}

\newcommand{\I}{\mat{I}}
\newcommand{\M}{\mat{M}}
\newcommand{\D}{\mat{D}}
%\newcommand{\U}{\mat{U}}
\newcommand{\V}{\mat{V}}
\newcommand{\W}{\mat{W}}
\newcommand{\X}{\mat{X}}
\newcommand{\Y}{\mat{Y}}
\newcommand{\mSigma}{\mat{\Sigma}}
\newcommand{\mLambda}{\mat{\Lambda}}
\newcommand{\e}{\vect{e}}
\newcommand{\g}{\vect{g}}
\renewcommand{\u}{\vect{u}}
\newcommand{\w}{\vect{w}}
\newcommand{\x}{\vect{x}}
\newcommand{\y}{\vect{y}}
\newcommand{\z}{\vect{z}}
\newcommand{\fI}{\mathfrak{I}}
\newcommand{\fS}{\mathfrak{S}}
\newcommand{\fE}{\mathfrak{E}}
\newcommand{\fF}{\mathfrak{F}}

\newcommand{\Risk}{\mathcal{R}}

\renewcommand{\L}{\mathcal{L}}
\renewcommand{\H}{\mathcal{H}}

\newcommand{\cn}{\kappa}
\newcommand{\nn}{\nonumber}


\newcommand{\Hess}{\nabla^2}
\newcommand{\tlO}{\tilde{O}}
\newcommand{\tlOmega}{\tilde{\Omega}}

\newcommand{\calF}{\mathcal{F}}
\newcommand{\fhat}{\widehat{f}}
\newcommand{\calS}{\mathcal{S}}

\newcommand{\calX}{\mathcal{X}}
\newcommand{\calY}{\mathcal{Y}}
\newcommand{\calD}{\mathcal{D}}
\newcommand{\calZ}{\mathcal{Z}}
\newcommand{\calA}{\mathcal{A}}
\newcommand{\fbayes}{f^B}
\newcommand{\func}{f^U}


\newcommand{\bayscore}{\text{calibrated Bayes score}}
\newcommand{\bayrisk}{\text{calibrated Bayes risk}}

\newtheorem{example}{Example}[section]
\newtheorem{exc}{Exercise}[section]
%\newtheorem{rem}{Remark}[section]

\newtheorem{theorem}{Theorem}[section]
\newtheorem{definition}{Definition}
\newtheorem{proposition}[theorem]{Proposition}
\newtheorem{corollary}[theorem]{Corollary}

\newtheorem{remark}{Remark}[section]
\newtheorem{lemma}[theorem]{Lemma}
\newtheorem{claim}[theorem]{Claim}
\newtheorem{fact}[theorem]{Fact}
\newtheorem{assumption}{Assumption}

\newcommand{\iidsim}{\overset{\mathrm{i.i.d.}}{\sim}}
\newcommand{\unifsim}{\overset{\mathrm{unif}}{\sim}}
\newcommand{\sign}{\mathrm{sign}}
\newcommand{\wbar}{\overline{w}}
\newcommand{\what}{\widehat{w}}
\newcommand{\KL}{\mathrm{KL}}
\newcommand{\Bern}{\mathrm{Bernoulli}}
\newcommand{\ihat}{\widehat{i}}
\newcommand{\Dwst}{\calD^{w_*}}
\newcommand{\fls}{\widehat{f}_{n}}


\newcommand{\brpi}{\pi^{br}}
\newcommand{\brtheta}{\theta^{br}}

% \newcommand{\M}{\mat{M}}
% \newcommand\Mmh{\mat{M}^{-1/2}}
% \newcommand{\A}{\mat{A}}
% \newcommand{\B}{\mat{B}}
% \newcommand{\C}{\mat{C}}
% \newcommand{\Et}[1][t]{\mat{E_{#1}}}
% \newcommand{\Etp}{\Et[t+1]}
% \newcommand{\Errt}[1][t]{\mat{\bigtriangleup_{#1}}}
% \newcommand\cnM{\kappa}
% \newcommand{\cn}[1]{\kappa\left(#1\right)}
% \newcommand\X{\mat{X}}
% \newcommand\fstar{f_*}
% \newcommand\Xt[1][t]{\mat{X_{#1}}}
% \newcommand\ut[1][t]{{u_{#1}}}
% \newcommand\Xtinv{\inv{\Xt}}
% \newcommand\Xtp{\mat{X_{t+1}}}
% \newcommand\Xtpinv{\inv{\left(\mat{X_{t+1}}\right)}}
% \newcommand\U{\mat{U}}
% \newcommand\UTr{\trans{\mat{U}}}
% \newcommand{\Ut}[1][t]{\mat{U_{#1}}}
% \newcommand{\Utinv}{\inv{\Ut}}
% \newcommand{\UtTr}[1][t]{\trans{\mat{U_{#1}}}}
% \newcommand\Utp{\mat{U_{t+1}}}
% \newcommand\UtpTr{\trans{\mat{U}_{t+1}}}
% \newcommand\Utptild{\mat{\widetilde{U}_{t+1}}}
% \newcommand\Us{\mat{U^*}}
% \newcommand\UsTr{\trans{\mat{U^*}}}
% \newcommand{\Sigs}{\mat{\Sigma}}
% \newcommand{\Sigsmh}{\Sigs^{-1/2}}
% \newcommand{\eye}{\mat{I}}
% \newcommand{\twonormbound}{\left(4+\DPhi{\M}{\Xt[0]}\right)\twonorm{\M}}
% \newcommand{\lamj}{\lambda_j}

% \renewcommand\u{\vect{u}}
% \newcommand\uTr{\trans{\vect{u}}}
% \renewcommand\v{\vect{v}}
% \newcommand\vTr{\trans{\vect{v}}}
% \newcommand\w{\vect{w}}
% \newcommand\wTr{\trans{\vect{w}}}
% \newcommand\wperp{\vect{w}_{\perp}}
% \newcommand\wperpTr{\trans{\vect{w}_{\perp}}}
% \newcommand\wj{\vect{w_j}}
% \newcommand\vj{\vect{v_j}}
% \newcommand\wjTr{\trans{\vect{w_j}}}
% \newcommand\vjTr{\trans{\vect{v_j}}}

% \newcommand{\DPhi}[2]{\ensuremath{D_{\Phi}\left(#1,#2\right)}}
% \newcommand\matmult{{\omega}}


% Define a professional comment color
\definecolor{commentcolor}{RGB}{0, 102, 204} % A medium blue color

% Define the \sattar command
\newcommand{\aya}[1]{\textcolor{commentcolor}{#1}}

\begin{document}

% If your paper is accepted and the title of your paper is very long,
% the style will print as headings an error message. Use the following
% command to supply a shorter title of your paper so that it can be
% used as headings.
%
%\runningtitle{I use this title instead because the last one was very long}

% If your paper is accepted and the number of authors is large, the
% style will print as headings an error message. Use the following
% command to supply a shorter version of the authors names so that
% they can be used as headings (for example, use only the surnames)
%
%\runningauthor{Surname 1, Surname 2, Surname 3, ...., Surname n}

\twocolumn[

\aistatstitle{Near-Optimal Sample Complexity in Reward-Free Kernel-Based Reinforcement Learning}

\aistatsauthor{Aya Kayal$^{1}$ \And Sattar Vakili$^{2}$ \And  Laura Toni$^{1}$ \And Alberto Bernacchia$^{2}$ }


\aistatsaddress{ $^{1}$University College London \And $^{2}$MediaTek Research } ]


\begin{abstract}
  Reinforcement Learning (RL) problems are being considered under increasingly more complex structures. While tabular and linear models have been thoroughly explored, the analytical study of RL under nonlinear function approximation, especially kernel-based models, has recently gained traction for their strong representational capacity and theoretical tractability. In this context, we examine the question of statistical efficiency in kernel-based RL within the reward-free RL framework, specifically asking: \emph{how many samples are required to design a near-optimal policy?}
    Existing work addresses this question under restrictive assumptions about the class of kernel functions. We first explore this question by assuming a \emph{generative model}, then relax this assumption at the cost of increasing the sample complexity by a factor of $H$, the length of the episode.  We tackle this fundamental problem using a broad class of kernels and a simpler algorithm compared to prior work. Our approach derives new confidence intervals for kernel ridge regression, specific to our RL setting, which may be of broader applicability. We further validate our theoretical findings through simulations.
\end{abstract}

\section{Introduction}

Video generation has garnered significant attention owing to its transformative potential across a wide range of applications, such media content creation~\citep{polyak2024movie}, advertising~\citep{zhang2024virbo,bacher2021advert}, video games~\citep{yang2024playable,valevski2024diffusion, oasis2024}, and world model simulators~\citep{ha2018world, videoworldsimulators2024, agarwal2025cosmos}. Benefiting from advanced generative algorithms~\citep{goodfellow2014generative, ho2020denoising, liu2023flow, lipman2023flow}, scalable model architectures~\citep{vaswani2017attention, peebles2023scalable}, vast amounts of internet-sourced data~\citep{chen2024panda, nan2024openvid, ju2024miradata}, and ongoing expansion of computing capabilities~\citep{nvidia2022h100, nvidia2023dgxgh200, nvidia2024h200nvl}, remarkable advancements have been achieved in the field of video generation~\citep{ho2022video, ho2022imagen, singer2023makeavideo, blattmann2023align, videoworldsimulators2024, kuaishou2024klingai, yang2024cogvideox, jin2024pyramidal, polyak2024movie, kong2024hunyuanvideo, ji2024prompt}.


In this work, we present \textbf{\ours}, a family of rectified flow~\citep{lipman2023flow, liu2023flow} transformer models designed for joint image and video generation, establishing a pathway toward industry-grade performance. This report centers on four key components: data curation, model architecture design, flow formulation, and training infrastructure optimization—each rigorously refined to meet the demands of high-quality, large-scale video generation.


\begin{figure}[ht]
    \centering
    \begin{subfigure}[b]{0.82\linewidth}
        \centering
        \includegraphics[width=\linewidth]{figures/t2i_1024.pdf}
        \caption{Text-to-Image Samples}\label{fig:main-demo-t2i}
    \end{subfigure}
    \vfill
    \begin{subfigure}[b]{0.82\linewidth}
        \centering
        \includegraphics[width=\linewidth]{figures/t2v_samples.pdf}
        \caption{Text-to-Video Samples}\label{fig:main-demo-t2v}
    \end{subfigure}
\caption{\textbf{Generated samples from \ours.} Key components are highlighted in \textcolor{red}{\textbf{RED}}.}\label{fig:main-demo}
\end{figure}


First, we present a comprehensive data processing pipeline designed to construct large-scale, high-quality image and video-text datasets. The pipeline integrates multiple advanced techniques, including video and image filtering based on aesthetic scores, OCR-driven content analysis, and subjective evaluations, to ensure exceptional visual and contextual quality. Furthermore, we employ multimodal large language models~(MLLMs)~\citep{yuan2025tarsier2} to generate dense and contextually aligned captions, which are subsequently refined using an additional large language model~(LLM)~\citep{yang2024qwen2} to enhance their accuracy, fluency, and descriptive richness. As a result, we have curated a robust training dataset comprising approximately 36M video-text pairs and 160M image-text pairs, which are proven sufficient for training industry-level generative models.

Secondly, we take a pioneering step by applying rectified flow formulation~\citep{lipman2023flow} for joint image and video generation, implemented through the \ours model family, which comprises Transformer architectures with 2B and 8B parameters. At its core, the \ours framework employs a 3D joint image-video variational autoencoder (VAE) to compress image and video inputs into a shared latent space, facilitating unified representation. This shared latent space is coupled with a full-attention~\citep{vaswani2017attention} mechanism, enabling seamless joint training of image and video. This architecture delivers high-quality, coherent outputs across both images and videos, establishing a unified framework for visual generation tasks.


Furthermore, to support the training of \ours at scale, we have developed a robust infrastructure tailored for large-scale model training. Our approach incorporates advanced parallelism strategies~\citep{jacobs2023deepspeed, pytorch_fsdp} to manage memory efficiently during long-context training. Additionally, we employ ByteCheckpoint~\citep{wan2024bytecheckpoint} for high-performance checkpointing and integrate fault-tolerant mechanisms from MegaScale~\citep{jiang2024megascale} to ensure stability and scalability across large GPU clusters. These optimizations enable \ours to handle the computational and data challenges of generative modeling with exceptional efficiency and reliability.


We evaluate \ours on both text-to-image and text-to-video benchmarks to highlight its competitive advantages. For text-to-image generation, \ours-T2I demonstrates strong performance across multiple benchmarks, including T2I-CompBench~\citep{huang2023t2i-compbench}, GenEval~\citep{ghosh2024geneval}, and DPG-Bench~\citep{hu2024ella_dbgbench}, excelling in both visual quality and text-image alignment. In text-to-video benchmarks, \ours-T2V achieves state-of-the-art performance on the UCF-101~\citep{ucf101} zero-shot generation task. Additionally, \ours-T2V attains an impressive score of \textbf{84.85} on VBench~\citep{huang2024vbench}, securing the top position on the leaderboard (as of 2025-01-25) and surpassing several leading commercial text-to-video models. Qualitative results, illustrated in \Cref{fig:main-demo}, further demonstrate the superior quality of the generated media samples. These findings underscore \ours's effectiveness in multi-modal generation and its potential as a high-performing solution for both research and commercial applications.
\subsection{System Model}
In this paper, we study a time division duplex multi-user \gls{mMIMO} system, where uplink channel estimates can be used to calculate the downlink precoder. This \gls{mMIMO} system has a \gls{BS} that is equipped with $N_{\sf{T}}$ antennas. The system is designed to concurrently support $N_{\sf{U}}$ users, each utilizing a single antenna. This configuration enables efficient communication by leveraging the multiple antennas at the \gls{BS} to enhance signal quality and increase capacity, ultimately allowing for simultaneous transmissions to multiple users. 

The received signal at the $k^{th}$ user can be expressed as
\begin{equation}\label{eq:signal_recived}
    \mathbf{y}_k =  \mathbf{h}_{k}^{\dagger} \sum_{\forall k}  \mathbf{w}_{k} x_k + \bs{\eta}_k \, ,
\end{equation}
where the wireless channel vector between the base station (\gls{BS}) and the $k^{th}$ user is represented by $\mathbf{h}_{k} \in \mathbb{C}^{N_{\sf{T}} \times 1}$. Let $x_k$
denote an independent transmitted complex symbol for a user. The term $\bs{\eta}_k \sim \mathcal{CN}(0, \sigma^2)$ indicates complex \gls{AWGN} characterized by a zero mean and a variance of $\sigma^2$. The downlink \gls{FDP} vector is defined as $\mathbf{W} = \left[ \mathbf{w}_1, \ldots, \mathbf{w}_k, \ldots, \mathbf{w}_{N_{\sf{U}}} \right] \in \mathbb{C}^{N_{\sf{T}} \times N_{\sf{U}}}$.
The associated \gls{SINR} user $k$ is given by
\begin{equation}
    \text{SINR}(\mb{w}_{k}) = \frac{ \big|\mb{h}^{\dagger}_{k} \mb{w}_{k} \big|^2}{\sum_{j \neq k} \big|\mb{h}^{\dagger}_{k} \mb{w}_{j} \big|^2 + \sigma^2}\,,
\end{equation}
and the sum rate for all users is
\begin{equation}\label{eq:sum-rate}
    R(\mb{W}) = \sum_{\forall k}  \text{log}_2 \Bigl(  1+ \text{SINR}(\mb{w}_{k}) \Bigr).
\end{equation}
The objective is to determine a precoding matrix $\mathbf{W}$ that maximizes the throughput in eq. (\ref{eq:sum-rate}) while adhering to the maximum transmit power constraint, $P_{\text{max}}$. Consequently, the downlink sum-rate maximization problem can be formulated as
\begin{equation}\label{eq:maximization}
\begin{aligned}
    & \underset{\mb{W}}{\max}~ R(\mb{W}) , \\
    \text{s.t.}  &\sum_{\forall k} \mb{w}_{k}^{\dagger}  \mb{w}_{k} \leq P_{\sf{max}} \, .
\end{aligned}
\end{equation}

\subsection{Weighted Minimum Mean Square Error Algorithm}
The sum-rate maximization problem in (\ref{eq:maximization}) is classified as NP-hard. To find good approximate solutions, the iterative \gls{WMMSE} algorithm \cite{shi2011iteratively} is commonly used, where the problem is transformed to a corresponding problem focused on minimizing the sum of \gls{MSE} under the independence assumption of $x_k$ and $\bs{\eta}_k$. 
Key parameters in this framework include the receiver gain $u_k$ and a positive user weight $v_k$, which are used to obtain the \gls{MSE} covariance matrix. 
The solution is obtained by iteratively solving convex subproblems to generate updates on the receiver gains, user weights, and beamforming matrix.
% The optimization problem involves iteratively updating variables such as the receiver gains, user weights, and beamforming matrices to maximize a redefined objective function while satisfying power constraints. Each variable is optimized sequentially, holding the others constant, ensuring the problem remains convex for individual updates.
% Let $u_k$ represent the receiver gain, while $v_k$ is a positive user weight, hence the \gls{MSE} covariance matrix $e_k$ can be expressed as 
% \begin{align}
% e_k &= \mathop{\mathbb{E}}_{\bs{x},\bs{\eta}} [ (\hat{x_k} - x_k)(\hat{x_k} - x_k)^{\dagger} ]\\
% &= \left( \mathbf{I} - u^{\dagger}_k\mathbf{h}_{k}\mathbf{w}_k\right)\left( \mathbf{I} - u^{\dagger}_k\mathbf{h}_{k}\mathbf{w}_k\right)^{\dagger},\\
% &=  \left| u_{k} \mathbf{h}_{k}^{\dagger} \mathbf{w}_k - 1 \right|^2 +  \sum_{j \neq k}^{N_{\sf{U}}} \left| u_k \mathbf{h}_k^{\dagger} v_j \right|^2  + \sigma^2 \left| u_k \right|^2,
% \end{align}
% thus the optimization problem can be formulated as
% \begin{align}
%     \underset{\mb{u},\mb{v},\mb{W}}{\max}~ & \sum^{N_{\sf{U}}}_{k=1} (v_k e_k - \text{log}_2 v_k) , \\
%    &\text{s.t.}   \sum_{\forall k} \mb{w}_{k}^{\dagger}  \mb{w}_{k} \leq P_{\sf{max}} \, .
% \end{align}
% This optimization problem becomes convex when each variable is considered individually, with the remaining variables held constant. Therefore the \gls{WMMSE} algorithm operates by sequentially updating each variable, 

First $\mathbf{W}^{(0)}$ is initialized while satisfying the constraint in \eqref{eq:maximization},
% in such a way that $\sum_{\forall k} \mb{w}_{k}^{\dagger}  \mb{w}_{k} \leq P_{\sf{max}}$. 
then, at each iteration $i$, variables are updated as defined below until a stopping criterion is satisfied:
\begin{align}
    v^{(\text{i})}_{k} = \frac{ \sum_{j =1}^{N_{\sf{U}}} \big|\mb{h}^{\dagger}_{k} \mb{w}^{(\text{i}-1)}_{j} \big|^2 + \sigma^2}{\sum_{j \neq k}^{N_{\sf{U}}} \big|\mb{h}^{\dagger}_{k} \mb{w}^{(\text{i}-1)}_{j} \big|^2 + \sigma^2},
\end{align}
\begin{align}
    u^{(\text{i})}_{k} = \frac{ \mb{h}^{\dagger}_{k} \mb{w}^{(\text{i}-1)}_{j} }{\sum_{j=1}^{N_{\sf{U}}} \big|\mb{h}^{\dagger}_{k} \mb{w}^{(\text{i}-1)}_{j} \big|^2 + \sigma^2},
\end{align}
\begin{align}\label{eq:construct_beamforming_matrix}
    \mb{w}^{(\text{i}+1)}_{k} =  u^{(\text{i})}_{k} v^{(\text{i})}_{k}\mb{h}_{k}(\sum^{N_{\sf{U}}}_{j=1}   v^{(\text{i})}_{j} |u^{(\text{i})}_{j}|^2 \mb{h}_{j}\mb{h}_{j}^{\dagger} + \mu \mathbf{I})^{-1},
\end{align}
where $\mu \geq 0$ is a Lagrange multiplier.
In addition to the high complexity of solving \gls{WMMSE}, the inherent non-convexity of sum-rate maximization means that WMMSE is only guaranteed to converge to a local optimum, not necessarily the global solution.
\section{Algorithm Description}
\label{sec:alg}

We now present our algorithms for both the exploration and planning phases. We begin by presenting the algorithm for the planning phase, as it remains unchanged across various exploration algorithms.
\begin{algorithm}[ht]
   \caption{Planning Phase}
   \label{alg:plan}
\begin{algorithmic}
   \STATE {\bfseries Input:} $\tau$, $\beta$, $\delta$, $k$, $M(\Sc,\Ac, H, P, r )$, and exploration dataset $\Dc_{N}$.
   \FOR{$h=H, H-1, \cdots, 1,$}
        \STATE Compute the prediction $\hat{g}_h$ according to~\eqref{eq:ghn};
        \STATE Let $Q_h(\cdot, \cdot) = \Pi_{[0,H]}[\hat{g}_h(\cdot, \cdot)+r_h(\cdot, \cdot)] $;
        \STATE $V_h(\cdot)= \max_{a\in\Ac}Q_h(\cdot, a)$;
        \STATE $\pi_h(\cdot) = \argmax_{a\in\Ac}Q_h(\cdot,a)$;
   \ENDFOR
   \STATE {\bfseries Output:} $\{\pi_h\}_{h\in[H]}$. 
\end{algorithmic}
\end{algorithm}

\begin{algorithm}[ht]
\caption{Exploration Phase \textbf{with} Generative Model}\label{alg:exp_gen}
\begin{algorithmic}[1]
\REQUIRE $\tau$, $k$, $\Sc$, $\Ac$, $H$, $P$, $N$;
\STATE Initialize $\Dc_{h,0}=\{\}$, for all $h\in[H]$;
\FOR{$n=1,2,\cdots, N$}
    \FOR{$h=1,2,\cdots, H$}
    \STATE Let $s_{h,n}, a_{h,n} = \argmax_{s,a\in\Ac}\sigma_{h, n-1}(s,a)$;
    \STATE Observe $s'_{h+1,n}\sim P_h(\cdot|s_{h,n}, a_{h,n})$;
    \STATE Update $\Dc_h^n \bigcup\{s_{h,n}, a_{h,n}, s'_{h+1,n}\}$.
    \ENDFOR
\ENDFOR
\STATE {\bfseries Output:} $\Dc_{N}$. 
\end{algorithmic}
\end{algorithm}

\begin{algorithm}[ht]
\caption{Exploration Phase \textbf{without} Generative Model}\label{alg:exp2}
\begin{algorithmic}
\REQUIRE $\tau$, $k$, $\beta$, $\delta$, $\Sc$, $\Ac$, $H$, $P$, $N$;
\FOR{$n=1,2,\cdots, N$}
    \FOR{$h_0=1, 2,\cdots, H$}
    \STATE Initialize $V_{h_0+1,n}=\bm{0}$
    \FOR{$h=h_0, h_0-1, \cdots, 1$}
        \STATE Obtain $\hat{f}_{h,(n,h_0)}$; $Q_{h, (n,h_0)}$, and $V_{h,(n,h_0)}(\cdot)$ according to~\eqref{eq:mean_predictor} and ~\eqref{eq:value_func}, respectively. 
    \ENDFOR
    \FOR{$h=1,2,\cdots, h_0$}
    \STATE Observe $s_{h,n}$; Take action $a_{h,n} = \argmax_{a\in\Ac}Q_{h, n}(s_{h,n},a)$; 
    \ENDFOR
    \STATE Update $\Dc_{h_0}^n \bigcup\{s_{{h_0},n}, a_{h_0,n}, s_{h_0+1,n}\}$
    \ENDFOR
\ENDFOR
\end{algorithmic}
\end{algorithm}



\subsection{Planning Phase} In the planning phase, the reward function $r$ is revealed to the learner. In addition, a dataset $\Dc_N=\{\Dc_{h,N}\}_{h\in[H]}$ is available, with $\Dc_{h,N}=\{s_{h,n}, a_{h,n}, s'_{h+1,n}\sim P(\cdot|s_{h,n}, a_{h,n})\}_{ n\in[N]}$ for each step $h\in[H]$. The objective is to leverage the knowledge of the reward function and utilize the dataset to design a near-optimal policy.
As mentioned in the introduction, the planning phase comprises of an offline RL design without further interaction with the environment.  


In the planning phase of our algorithm, we derive a policy using least squares value iteration. Specifically, at step $h$, we compute a prediction, $\hat{g}_{h}$, for the expected value function in the next step $[P_hV_{h+1}]$. We then define
\begin{equation}\label{eq:Qplan}
    Q_h(\cdot, \cdot) = \Pi_{[0,H]}\left[r_h(\cdot, \cdot)+ \hat{g}_h(\cdot, \cdot)\right],
\end{equation}
where $\Pi_{[a,b]}$ denotes projection on $[a,b]$ interval. 
The policy $\pi$ is then obtained as a greedy policy with respect to $Q$. For each $h\in[H]$,
\begin{equation*}
    \pi_h(\cdot) = \argmax_{a\in\Ac}Q_h(\cdot, a).
\end{equation*}
We now detail the computation of $\hat{g}_h$. Keeping the Bellman equation in mind and starting with $V_{H+1}=\bm{0}$, $\hat{g}_h$ is the kernel ridge predictor for $[P_hV_{h+1}]$. This prediction uses $N$ observations $\bm{y}_{h}=[V_{h+1}(s'_{h+1, 1}),V_{h+1}(s'_{h+1, 2}), \cdots,V_{h+1}(s'_{h+1, N}) ]^{\top}$ at points $\{z_{h,n}\}_{n=1}^N$. 
Recall that 
$
\E_{s'\sim P(\cdot|z_{h, n})}\left[V
_{h+1}(s')  \right]  = [P_hV_{h+1}](z_{h,n})
$.
The observation noise $V_{h+1}(s'_{h+1,n})-[P_hV_{h+1}](z_{h,n})$ is due to random transitions and is bounded by $H-h\le H$. Specifically, 
\begin{equation}\label{eq:ghn}
    \hat{g}_h(z) = k^{\top}_{h,N}(z)(\tau^2 I + K_{h,N})^{-1} \bm{y}_{h},
\end{equation}
where $k_{h,N}(z)=[k(z,z_{h,1}), k(z,z_{h,2}), \cdots, k(z,z_{h,N})]^{\top}$ is the pairwise kernel values between $z$ and observation points and $K_{h,N}=[k(z_{h,i}, z_{h,j})]_{i,j\in[N]}$ is the Gram matrix.
We then define $Q_h$ according to~\eqref{eq:Qplan} and also set
\begin{equation*}
    V_h(s)=\max_{a\in\Ac} Q_h(s, a).
\end{equation*}
The values of $\hat{g}_h$, $Q_h$ and $V_h$ are obtained recursively for $h=H,H-1, \cdots, 1$.
For a pseudocode, see Algorithm~\ref{alg:plan}.


% \sattar{Comment on how this is slightly different from Qiu et al. }

%\begin{algorithm}[tb]
%   \caption{Planning Phase}
%   \label{alg:plan}
%\begin{algorithmic}
%   \STATE {\bfseries Input:} $\tau$, $\beta$, $\delta$, $k$, $M(\Sc,\Ac, H, P, r )$, and exploration data set $\Dc_{N}$.
%   \FOR{$h=H, H-1, \cdots, 1,$}
%        \STATE Compute the prediction $\hat{g}_h$ according to~\eqref{eq:ghn};
%        \STATE Let $Q_h(\cdot, \cdot) = \Pi_{[0,H]}[\hat{g}_h(\cdot, \cdot)+r_h(\cdot, \cdot)] $;
%        \STATE $V_h(\cdot)= \max_{a\in\Ac}Q_h(\cdot, a)$;
%        \STATE $\pi_h(\cdot) = \argmax_{a\in\Ac}Q_h(\cdot,a)$;
%   \ENDFOR
%   \STATE {\bfseries Output:} $\{\pi_h\}_{h\in[H]}$. 
%\end{algorithmic}
%\end{algorithm}


\subsection{Exploration Phase} In the exploration phase, the algorithm collects a dataset
$\Dc_{N}=\{\Dc_{h,N}\}_{h\in[H]}$, where
$\Dc_{h,N}=\{s_{h,n},a_{h,n}, s'_{h+1,n}\}_{h\in[H], n\in[N]}$ for each $h\in[H]$, %. As we showed, these observations will be 
later used in the planning phase to design a near-optimal policy. The primary goal during this phase is to gather the most informative observations.

% As discussed in the introduction, existing work fails to achieve order-optimal or even finite sample complexities without imposing strong and restrictive assumptions %such as assuming a kernel with exponential eigendecay.
Initially, we consider a preliminary case where a \emph{generative model}~\citep{kakade2003sample} is present that can produce transitions for the state-actions selected by the algorithm. Under this setting, we demonstrate that a simple rule for data collection leads to improved and desirable sample complexities. Inspired by these results, we introduce a novel algorithm that completely relaxes the requirement for a generative model, at the price of increasing the number of exploration episodes by a factor of $H$. The key aspect of our algorithms is the \emph{unbiasedness}--statistical independence of the collected samples, which means that the observation points do not depend on previous transitions.

\subsubsection{Exploration with a Generative Model}

In this section, we outline the exploration phase when a generative model is present. At each step $h$ of the current exploration episode, uncertainties derived from kernel ridge regression are employed to guide exploration. Specifically, let
\begin{equation} \label{eq:std_dev}
   \sigma^2_{h,n}(z) = k(z,z) -k^{\top}_{h,n}(z)(\tau^2I+K_{h,n})^{-1}k_{h,n}(z) 
\end{equation}
where $k_{h,n}(z) = [k(z,z_{h,1}),k(z,z_{h,2}), \cdots, k(z,z_{h,n}) ]^{\top}$ is the vector of kernel values between the state-action of interest and past observations in $\Dc_{h,n}$, and $K_{h,n}=[k(z_{h,i}, z_{h,j})]_{i,j=1}^n$ is the Gram matrix of pairwise kernel values between past observations in $\Dc_{h,n}$. Equipped with $\sigma_{h,n}(z)$, at step $h$, we select
\begin{equation}\label{eq:sel_rule}
    s_{h,n}, a_{h,n}=\argmax_{s\in\Sc,a\in\Ac} \sigma_{h,n-1}(s,a),
\end{equation}
and observe the next state $s'_{h+1,n}\sim P(\cdot|s_{h,n}, a_{h,n})$. We then add this data point to the dataset and update $\Dc_{h,n}=\Dc_{h,n-1}\cup\{(s_{h,n}, a_{h,n}, s'_{h+1,n})\}$. For a pseudocode, see Algorithm~\ref{alg:exp_gen}.

%\begin{algorithm}[ht]
%\caption{Exploration Phase \textbf{with} Generative Model}\label{alg:exp_gen}
%\begin{algorithmic}[1]
%\REQUIRE $\tau$, $k$, $\Sc$, $\Ac$, $H$, $P$, $N$;
%\STATE Initialize $\Dc_{h,0}=\{\}$, for all $h\in[H]$;
%\FOR{$n=1,2,\cdots, N$}
%    \FOR{$h=1,2,\cdots, H$}
%    \STATE Let $s_{h,n}, a_{h,n} = \argmax_{s,a\in\Ac}\sigma_{h, n-1}(s,a)$;
%    \STATE Observe $s'_{h+1,n}\sim P_h(\cdot|s_{h,n}, a_{h,n})$;
%    \STATE Update $\Dc_h^n \bigcup\{s_{h,n}, a_{h,n}, s'_{h+1,n}\}$.
%    \ENDFOR
%\ENDFOR
%\STATE {\bfseries Output:} $\Dc_{N}$. 
%\end{algorithmic}
%\end{algorithm}
We highlight that the selection rule~\eqref{eq:sel_rule} relies on the generative model that allows the algorithm to deviate from the Markovian trajectory and move to a state of its choice. Since observations are selected based on maximizing $\sigma_{h,n-1}$, which by definition \eqref{eq:std_dev} does not depend on previous transitions $\{s'_{h+1,i}\}_{i=1}^{n-1}$,  the statistical independence conveniently holds. The generative model setting is feasible in contexts such as games, where the player can manually set the current state. However, this may not always be possible in other scenarios. Next, we introduce our online algorithm, which strictly stays on the Markovian trajectory.




\subsubsection{Exploration without Generative Models}

In this section, we show that a straightforward algorithm, in contrast to existing approaches, achieves near-optimal performance in an online setting without requiring a generative model. Compared to the scenario with a generative model, the sample complexity of this algorithm increases by a factor of $H$. For a detailed and technical comparison with existing work, please refer to Appendix~\ref{appx:comp}. %, highlighting how it may lead to trivial (infinite) sample complexities.

Our online algorithm operates as follows: in each exploration episode, only one data point specific to a step $h$ is collected ---this accounts for the $H$ scaling in sample complexity. This observation however is collected in an unbiased way, which eventually leads to tighter performance guarantees. Specifically, at episode $nH+h_0$, where $n\in[N]$ and $h_0\in[H]$, the algorithm collects an informative sample for the transition at step $h_0$. This results in a total of $N$ samples at each step over $NH$ episodes. The algorithm initializes $V_{h_0+1,(n,h_0)}=\bm{0}$.
Let $\hat{f}_{h,(n,h_0)}$ and $\sigma_{h,n}$ represent the predictor and uncertainty estimator for $[P_hV_{h+1,(n,h_0)}]$, respectively. These are derived from the historical data $\Dc_{h,n-1}$ of observations at step $h$. Specifically, 
\begin{align}\nn
    \hat{f}_{h,(n,h_0)}(z) &= k^{\top}_{h,n}(z)(K_{h,n}+\tau^2 I)^{-1}\bm{y}_{h,n},  \\
    \sigma^2_{h,n}(z) &= k(z,z) - k^{\top}_{h,n}(z)(K_{h,n}+\tau^2 I)^{-1}k_{h,n}(z), \label{eq:mean_predictor}
\end{align}
where $k_{h,n}(z) = [k(z,z_{h,1}),k(z,z_{h,2}), \cdots, k(z,z_{h,n}) ]^{\top}$ is the vector of kernel values between the state-action of interest and past observations in $\Dc_{h,n}$, $K_{h,n}=[k(z_{h,i}, z_{h,j})]_{i,j=1}^n$ is the Gram matrix of pairwise kernel values between past observations in $\Dc_{h,n}$, and 
\begin{align*}
\bm{y}_{h,(n,h_0)} &= [V_{h+1,(n,h_0)}(s_{h+1,1}), V_{h+1,(n,h_0)}(s_{h+1,2}), \\
&\quad \cdots, V_{h+1,(n,h_0)}(s_{h+1,n}) ]^{\top}
\end{align*} 
is the vector of observations. 
We then have
%\begin{align}\nn 
%Q_{h,(n,h_0)}=\Pi_{0,H}\left[\hat{f}_{h,(n,h_0)}+\beta(\delta)\sigma_{h,n}\right], 
%~~~~~ \text{and} ~~~ V_{h,(n,h_0)}(\cdot) = \max_{a\in\Ac}Q_{h,(n,h_0)}(\cdot, a). 
%\end{align}
\begin{align*}
Q_{h,(n,h_0)} = \Pi_{0,H} \left[ \hat{f}_{h,(n,h_0)} + \beta(\delta) \sigma_{h,n} \right], \\
V_{h,(n,h_0)}(\cdot) = \max_{a \in \Ac} Q_{h,(n,h_0)}(\cdot, a) \label{eq:value_func}.
\end{align*}













%\begin{algorithm}[ht]
%\caption{Exploration Phase \textbf{without} Generative Model}\label{alg:exp2}
%\begin{algorithmic}
%\REQUIRE $\tau$, $k$, $\beta$, $\delta$, $\Sc$, $\Ac$, $H$, $P$, $N$;
%
%\FOR{$n=1,2,\cdots, N$}
%    \FOR{$h_0=1, 2,\cdots, H$}
%    \STATE Initialize $V_{h_0+1,n}=\bm{0}$
%    \FOR{$h=h_0, h_0-1, \cdots, 1$}
%        \STATE Obtain $\hat{f}_{h,n}$; $Q_{h, n}$, and $V_{h,n}(\cdot)$ according to~\eqref{eq:mean_predictor} and %~\eqref{eq:value_func}, respectively. 
%    \ENDFOR
%    \FOR{$h=1,2,\cdots, h_0$}
%    \STATE Observe $s_{h,n}$; Take action $a_{h,n} = \argmax_{a\in\Ac}Q_{h, n}(s_{h,n},a)$; 
%    \ENDFOR
%    \STATE Update $\Dc_{h_0}^n \bigcup\{s_{{h_0},n}, a_{h_0,n}, s_{h_0+1,n}\}$
%    \ENDFOR
%\ENDFOR
%\end{algorithmic}
%\end{algorithm}

The values of $Q_{h,(n,h_0)}$ and $V_{h,(n,h_0)}$ are obtained recursively for all $h\in[h_0]$. 
The exploration policy at episode $nH+h_0$ is then the greedy policy with respect to $Q_{h,(n-1,h_0)}$. The dataset is updated by adding the new observation to the dataset for step $h_0$, such that $\Dc_{h_0,n} = \Dc_{h_0,n-1}\cup \{(s_{h_0,n}, a_{h_0,n}, s_{h_0+1,n})\}$, while datasets for all other steps remain unchanged: $\Dc_{h,n} = \Dc_{h,n-1}$ for all $h\neq h_0$. This specific update ensures that the collected samples are unbiased. More specifically, the sample collected at $h_0$ solely relies on the uncertainty $\sigma_{h_0,n}$, due to the initialization $V_{h_0+1,(n,h_0)}=\bm{0}$ which implies $\hat{f}_{h_0,(n,h_0)}=\bm{0}$. Since $\sigma_{h_0,n}$ does not depend on previous transitions $s_{(h_0+1,i)}$ for any $i\leq n$, the samples at $h=h_0$ are unbiased. However, for \( h < h_0 \), the samples depend on both the uncertainty \( \sigma_{h,n} \) and the prediction \( \hat{f}_{h,(n,h_0)} \)~\eqref{eq:mean_predictor}. Since the prediction depends on the transitions $s_{(h+1,i)}$ for $i\leq n$, these samples are biased. As a result, we discard them and only retain the unbiased samples at \( h = h_0 \). This approach improves the rates in our analysis, albeit at the cost of a factor of~\( H \).

% A detailed technical comparison with the closely related works of~\cite{qiu2021reward} and~\cite{vakilireward} is provided in Appendix~\ref{sec:relatedworks}.
 
 %However, their results rely on specific assumptions about the relationship between kernel eigenvalues and domain size, which limits the generality of their results. Also, the domain partitioning technique, despite its theoretical appeal, is cumbersome to implement and raises practical concerns, such as the justification of discarding samples and using only those within a subdomain. Our algorithm demonstrates order-optimal results for general kernels using simpler approach leveraging statistical independence, and it only requires a sublinear maximum information gain $\Gamma(n)$, which is always the case. }
 







% \begin{algorithm}\label{alg:exp}
% \caption{Exploration Phase}
% \begin{algorithmic}
% \REQUIRE $\tau$, $k$, $N$

% \FOR{$n=1,2,\cdots, N$}
%     \STATE Initialize 
%     \FOR{$h=H, H-1, \cdots, 1$}
%         \STATE DO NOTHING! : )
%     \ENDFOR
%     \FOR{$h=1, 2, \cdots, H$}
%         \STATE Observe $s_h$
%         \STATE Select $a_h=\pi_h(s_h):=\arg\max_{a}\sigma_h^n(s_h,a)$  \sattar{\# Note the significant change we are making here}
%     \ENDFOR
% \ENDFOR
% \end{algorithmic}
% \end{algorithm}









% In the next section, we obtain a sufficient number of exploration episodes for designing $\epsilon$-optimal policies, under both scenarios: with and without generative models. 


\section{Algorithm \& Theoretical Analysis}
In this section, we propose our combinatorial bandit algorithm for interactive personalized visualization recommendation, called Hierarchical Semi UCB (Hier-SUCB).

\subsection{Hier-SUCB}

Inspired by SPUCB~\cite{peng2019practical}, we develop a combinatorial contextual semi-bandit with a learnable bias term and a hierarchical structure. The structure includes a hierarchical agent to optimize the exploration on biased combinatorial setting and a hierarchical interaction system to get detailed user feedback without hurting user experience. The algorithm maintains two sets of upper confidence bounds (UCB) including the UCB of configurations $U(c)$ and visualizations $U(c,x,y)$ with a given configuration $c$. More formally, let $U(c)$ and $U(v)=U(c,x,y)$ be defined as:
\begin{equation}
    U(c_t)=\theta_{C,t}^T \mathbf x_{c,t}+\rho_{c,t}
\label{eqn:ucfg}
\end{equation}
\begin{equation}
    U(v_t)=\theta_{C,t}^T \mathbf x_{c,t}+\theta_{A,t}^T (\mathbf x_{x,t}+\mathbf x_{y,t}) + \gamma_t + \rho_{c,t} +\rho_{a,t} +\rho_{\gamma,t}
\label{eqn:uvis}
\end{equation}
The confidence radius of attribute and configuration $\rho_a,\rho_c$ is defined as:
\begin{equation}
    \rho_{k,t}=\sqrt{\mathbf x_{k,t}^T(\mathbf I_d+\mathbf x_{k,t} \mathbf x_{k,t}^T) \mathbf x_{k,t}}, k\in\lbrace a,c \rbrace
    \label{eqn:rhoca}
\end{equation}
where $\mathbf{x}_{k,t}$ is the embedding vector of attribute or configuration in round $t$ and $\mathbf I_d$ refers to identity matrix with the same dimension $d$ as $\mathbf x_t$. According to UCB~\cite{auer2010ucb}, the confidence radius of bias $\rho_\gamma$ defined as
\begin{equation}
    \rho_{\gamma,t}=\sqrt{2ln(T) / t_\gamma}
    \label{eqn:rhobias}
\end{equation}
where $t_\gamma$ is the time that bias $\gamma$ has been played.

In each turn, the agent first computes the UCB of all configurations with Eq.~\ref{eqn:ucfg} and then selects the configuration with optimal UCB. 
Afterward, the agent evaluates the upper confidence bounds of all visualizations with Eq.~\ref{eqn:uvis} to select the optimal. Then, the agent will ask for user feedback on the recommended visualization: 
if it is positive, the agent will automatically take the configuration and attributes as positive; otherwise, it will further ask for user feedback on the configuration and attributes separately. 

Intuitively, adding a bias term in the estimation of visualization reward can improve the accuracy of recommendation, because in the worst case we can assume it is the visualization reward and explore in a large action space. By designing appropriate reward function for the bias term, the bias term can serve as a correction term for cases that user likes the configuration and attributes but not the visualization. With the hierarchical structure of our agent, we further narrow down the large action space of the bias term. The agent quickly converges in configuration bandit with less item pool, so that it can have more exploration of attribute and bias terms with larger item pools.

\begin{algorithm}
    \SetAlgoLined
        Initialize $\theta_{C,t},\theta_{A,t}, \gamma_t$ (Eq. ~\ref{eqn:theta_def})\;
	\For{$t=1,2,...T$}{
	
        \For{$a_{c,t}=1,2,...n$}{
        Compute UCB $U(c_t)$ (Eq. ~\ref{eqn:ucfg})\;
        }
        Select $c_t=\textbf{argmax}(U(c_t))$\;
        \For{$a_{x,t}=1,2,...m$}
            {
            \For{$a_{y,t}=1,2,...m$}
                {
                Compute UCB $U(c_t,x_t,y_t)$ (Eq. ~\ref{eqn:uvis})\;
                }      
            }
        Select $V_{t}=\textbf{argmax}(U(c_t,x_t,y_t))$\;
        \uIf{$r_{V,t}==1$}
        {$r_{C,t}\leftarrow 1,r_{A,t}\leftarrow 1$\;}
        \Else{ask for $r_{C,t},r_{A,t}$\;}
        Update $\theta_{C,t},\theta_{A,t}, \gamma_t$ (Eq. ~\ref{eqn:theta_update})\;
        Update $\rho_{c,t},\rho_{a,t}, \rho_{\gamma,t}$ (Eq. ~\ref{eqn:rhoca},~\ref{eqn:rhobias})\;
        }
\caption{Hier-SUCB}

\end{algorithm}

\subsection{Regret Analysis}
% When the user provides negative feedback for the visualization, we consider the regret of round $t$ in four cases:
The regret of Hier-SUCB comes from the exploration of preferred configuration, attribute pair and learning the bias term. The exploration of preferred configuration and attribute pair can be reduced to general combinatorial bandit problem. Learning bias term can be viewed as a general bandit problem with constraints. For more detailed analysis of regret bound, we consider the regret of round $t$ under four cases when the user provides negative feedback to the visualization:
\begin{enumerate}
    \item Like configuration $c_t$ and attribute pair $\lbrace x_t,y_t \rbrace$ 
    \item Like attribute pair $\lbrace x_t,y_t \rbrace$ but not configuration $c_t$
    \item Like configuration $c_t$ but not attribute pair $\lbrace x_t,y_t \rbrace$
    \item Dislike configuration $c_t$ and attribute pair $\lbrace x_t,y_t \rbrace$
\end{enumerate}

For case (1), we provide the regret bound by analyzing the bias term converges in certain rounds.
\begin{lemma}
    The reward gap between optimal and sub-optimal bias $\gamma$ is bounded with the overall round $T$ and the time $t_\gamma$ that $\gamma$ has been played for.
    \begin{equation}
        \Delta_\gamma \leq \sqrt{\frac{ln(T)}{t_\gamma}}
    \end{equation}
    \label{lem:case1}
\end{lemma}
\begin{proof}
    having a positive configuration and attributes while negative visualization implies:
% Having positive configuration and attributes while negative visualization implies:
\begin{equation}
    U(c,a)\geq U(c^\ast,a^\ast)
\end{equation}
where $c^\ast, a^\ast$ refers to configuration and attributes of preferred visualization. Notably, $c,a$ may also receive positive feedback from user, but their combination is not preferred. In such case, $\Delta_c=\Delta_a=0$, and we can bound the regret with bias:
\begin{equation}
    \Delta_{bias} \leq \rho_{c,t} +\rho_{a,t}+ \rho_{\gamma,t} - \rho_{c,t}^\ast -\rho_{a,t}^\ast -\rho_{t,\gamma}^\ast
\end{equation}
The round that $\gamma^\ast$ is updated given $c,a$ should be less than either $t_a^\ast$ or $t_c^\ast$, we define $t_{max}^\ast=max(t_a^\ast,t_c^\ast)\geq t_{min}^\ast=min(t_a^\ast,t_c^\ast)\geq t_\gamma^\ast$. Using the definition of UCB, we can bound the gap of bias by
\begin{align}
     \Delta_{bias} &\leq  3\sqrt{\frac{2ln(T)}{t_\gamma}}-3\sqrt{\frac{2ln(T)}{t^\ast_{max}}}\leq  3\sqrt{\frac{2ln(T)}{t_\gamma}} \\
     t_\gamma &\leq 18ln(T) \frac{1}{\Delta_{bias}^2}
\end{align}
\renewcommand\qedsymbol{}
\end{proof}

With ~\ref{lem:case1}, we can bound the regret bound of case (1) by:
\begin{equation}
    Reg_1=\mathbb E\lbrack t_\gamma \rbrack \Delta_{bias}\leq 18ln(T)/\Delta_{bias} = O(ln(T))
\end{equation}

Notably, cases (2) and (4) are bounded by the rapid convergence of the confidence radius of configurations, thus, we consider when the agent recommends configuration. We derive the following lemma with $s^c_t$ representing the time that the configuration arm of action $a_t$ in round $t$ has been played. 
\begin{lemma}
Following the proof in LinUCB ~\cite{chu2011contextual}, we can bound The gap between optimal and sub-optimal reward is bounded by the following equation with probability $1-\delta/T$:
 \begin{equation}
\label{eqn:semi}
    |r_{t}^*- r_{t,a_t}| \leq \alpha \sqrt{-2log(\delta/2)/s^c_t}
\end{equation}
\end{lemma}

By summing Equation ~\ref{eqn:semi} with the expectation of round $T$, we derive the regret for case (2) and (4) as
\begin{equation}
    Reg_{2,4}=O(\sqrt{Tln^3(m^2Tln(T))})
    \label{eqn:reg_comb}
\end{equation}

For case (3), we first evaluate how many rounds the agent needs to recommend a positive configuration.
% For the case (3), we first evaluate how many rounds the agent needs to recommend positive configuration.
\begin{lemma}
With overall round $T$, the expected round for attribute exploration is 
\begin{equation}
    T-\frac{k}{\Delta_c^2}ln(T) 
    \label{eqn:tbound}
\end{equation}
\end{lemma}
\begin{proof}
The rounds to reach a positive configuration depend on the expectation of rounds that recommends a negative configuration. 
Thus by following the definition of UCB, we have:
\begin{equation}
    \mathbb{E}\lbrack t \rbrack= k\frac{ln(T)}{\Delta_c^2}
\end{equation}    
\renewcommand\qedsymbol{}
\end{proof}

To calculate the regret bound of case (3), we apply the upper bound of round $t$ derived in Equation ~\ref{eqn:semi} and get:
\begin{equation}
    Reg_3=O(\sqrt{(T-ln(T))ln^3(m^2(T-ln(T))ln(T-ln(T))}))
\end{equation}


Therefore, we can get the overall regret by summing up the regret of each case:
\begin{theorem}
The regret of Hier-SUCB can be bounded as:
\begin{align}
    Reg&=Reg_1+Reg_3+Reg_{2,4}\\
    &=O(\sqrt{Tln^3(m^2T ln(T))})
\end{align}
\end{theorem}


Notably, we reduce the original semi-bandit by improving $O(nm^2)$ to $O(m^2)$ by adding a hierarchical structure and decompose the combinatorial problem to multi-arm bandits and contextual semi-bandits. Regular combinatorial contextual bandit will apply Eq.~\ref{eqn:reg_comb} to all the attributes and configurations, where the term $m^2$ would be $nm^2$ in this case. With a hierarchical structure, the regret of the configuration is bounded by a contextual bandit. 
The regret of attribute pairs can be bounded with combinatorial contextual bandits as long as the configuration is preferred.
For the case (4) where attributes and configuration are preferred but their combination is not, we model an independent bias as multi-armed bandits  whose regret bound is $O(ln(T))$.

We also model the relation between the configuration and attribute with an extra bias as an individual bandit.
This helps improve the final accuracy of the personalized visualization recommendation, which we demonstrate later in the experiments using real-world datasets. 
\section{Experiments}\label{sec:exp}

%    \includegraphics[width=0.32\textwidth]{figures/Matern2.5_all_algos.png} % Use the same path as before
%    \caption{Average suboptimality gap against $N$ for the Mat{\'e}rn Kernel with $\nu=2.5$. The error bars indicate standard deviation.}
%    \label{fig:Matern2.5_all_algos_single}
%\end{figure}


We numerically validate our proposed algorithms and compare with the baseline algorithms. From the literature, we implement~\cite{qiu2021reward}, in which the exploration aims at maximizing a hypothetical reward of $\beta\sigma_n/H$ over each episode $n$. The planning phase is similar to Algorithm~\ref{alg:plan}, but with upper confidence bounds based on kernel ridge regression being used rather than the prediction~$\hat{g}_h$. 
%We numerically validate and compare the performance of several algorithms. We implement the existing approach of~\cite{qiu2021reward} where during the exploration phase at each episode $n$, a policy is designed to obtain high value with respect to a hypothetical reward of $\beta\sigma_n/H$. The planning phase is similar to Algorithm~\ref{alg:plan} with a minor difference that rather than kernel ridge predictions, upper confidence bounds based on kernel ridge regression are used. 
We also implement our exploration algorithms with and without a generative model: Algorithms~\ref{alg:exp_gen} and~\ref{alg:exp2} respectively. Additionally, we implement a heuristic variation of Algorithm~\ref{alg:exp2}, which collects the exploration samples in a greedy manner $a_{h,n}=\argmax_{a\in\Ac}\sigma_{h,n}(s_{h,n},a)$ while remaining on the Markovian trajectory by sampling $s_{h+1}\sim P(\cdot|s_h,a_h)$. We refer to this heuristic as \emph{Greedy Max Variance}. For all these  algorithms, we use Algorithm~\ref{alg:plan} to obtain a planning policy. In the experimental setting, we choose $H=10$ and $\Sc=\Ac=[0,1]$ consisting of $100$ evenly spaced points. We choose $r$ and $P$ from the RKHS of a fixed kernel. For the detailed framework and hyperparameters, please refer to Appendix~\ref{appx:exp}. We run the experiment for three different kernels across all $4$ algorithms for $80$ independent runs, and plot the average suboptimality gap $V_1^{\star}(s)-V^{\pi}(s)$ for $N=10,20, 40, 80, 160$, as shown in Figure~\ref{fig:overallresults}.
Our proposed Algorithm~\ref{alg:exp2}, without generative model, demonstrates better performance compared to prior work \citep{qiu2021reward} across all three kernels, validating the improved sample efficiency. Notably, \citet{qiu2021reward} performs poorly with nonsmooth kernels. Greedy Max Variance is a heuristic that in many of our experiments performs close to Algorithm~\ref{alg:exp2}.
Furthermore, with access to a generative model, Algorithm~\ref{alg:exp_gen} performs the best. This is anticipated, as the generative model provides the flexibility to select the most informative state-action pairs, unconstrained by Markovian transitions.


%We adopt an episodic MDP framework with episodes of fixed length, set to $H=10$ and we define the state and action spaces as consisting of 100 evenly spaced points within the interval [0,1]. The probability transition distribution and reward functions are chosen as general functions within the Reproducing Kernel Hilbert Space (RKHS). The process of generating $P(s,a,s')$ and $r(s,a)$ involves utilizing Gaussian Process (GP) regression with a chosen kernel. In this process, a subset of inputs $X=(s,a,s')$ is sampled from the state and action spaces for $P$, and $X=(s,a)$ is sampled for $r$. Their corresponding outputs $y$ are drawn from a Gaussian Process (GP) and evaluated at $X$. Subsequently, GP regression is performed between these inputs and outputs, and the resulting estimated mean is appropriately scaled and normalized to represent $P$ or $r$. Various kernels are explored in generating these functions, including Radial Basis Function (RBF) (length scale=$0.1$ and $\tau= 0.01$) and Mat{\'e}rn kernels (smoothness= $1.5$ / $2.5$, length scale=$0.1$ and $\tau=0.5)$ . Refer to the Appendix for plots of $r$ and $P$ corresponding to each kernel (Figures ~\ref{fig:R_P_RBF}, ~\ref{fig:R_P_Matern1.5}, ~\ref{fig:R_P_Matern2.5}).

%For each algorithm, we alternate between exploration and planning phases. After training for each of $N=10,20,40,80,160$ exploration episodes, we execute the planning phase to derive the policy from the state-action pairs collected so far during the exploration phase, and measure the regret as: $V_1^{\star}(s)-V_1^{\pi}(s)$. The optimal value function $V^{\star}$ is obtained by running the value iteration algorithm for $H$ steps, and $V^{\pi}$ is obtained by averaging the return after unrolling the learned policy for $20$ episodes during the planning phase. We evaluate the regret at the initial state which is fixed at the start of each episode for all algorithms in both the exploration and planning phases. We plot the regret performance for each algorithm averaged over $80$ independent runs. The simulations were executed on a cluster which has $376.2$ GiB of RAM, and an Intel(R) Xeon(R) Gold 5118 CPU running at $2.30$ GHz.

%It's important to note that we select the best values of the Upper Confidence Bound (UCB) coefficient: $\beta$, for both of our proposed algorithm without generative model and the algorithm introduced by \citep{qiu2021reward} by conducting a grid search across a limited set of values [0.1, 1, 10, 100] for each kernel. The plots of the regret for different values $\beta$ are included in the Appendix (Figures \ref{fig:Hyperparams_RBF}, \ref{fig:Hyperparams_Matern1.5},\ref{fig:Hyperparams_Matern2.5}). We have found that the existing work \citep{qiu2021reward} is more sensitive to the hyperparameter $\beta$ than our proposed algorithm, and a smaller value of $\beta=0.1$ leads to lower regret for both.


%Figure  \ref{fig:overallresults} illustrates the average regret during training for all algorithms. Our proposed algorithm without generative model demonstrates lower regret compared to prior work \citep{qiu2021reward} across all three kernels, validating the improved sample efficiency of our algorithm. Notably, \citep{qiu2021reward} fails to converge after 160 episodes when nonsmooth kernels are utilized (Mat{\'e}rn $1.5$ and $2.5$). Both Greedy Max Variance and our proposed algorithm without a generative model perform similarly, and both surpass the performance of the existing work \citep{qiu2021reward}.
%Furthermore, with access to a generative model, the regret is the lowest among all tested algorithms across all three kernels. This outcome is anticipated, as the generative model has the flexibility to select state-action pairs that optimize performance, unconstrained by Markovian transitions.








 
In this paper, we systematically investigate the position bias problem in the multi-constraint instruction following. To quantitatively measure the disparity of constraint order, we propose a novel Difficulty Distribution Index (CDDI). Based on the CDDI, we design a probing task. First, we construct a large number of instructions consisting of different constraint orders. Then, we conduct experiments in two distinct scenarios. Extensive results reveal a clear preference of LLMs for ``hard-to-easy'' constraint orders. To further explore this, we conduct an explanation study. We visualize the importance of different constraints located in different positions and demonstrate the strong correlation between the model's attention distribution and its performance.
\clearpage
\bibliography{references}
%%%%%%%%%%%%%%%%%%%%%%%%%%%%%%%%%%%%%%%%%%%%%%%%%%%%%%%%%%%%
\section*{Checklist}


% %%% BEGIN INSTRUCTIONS %%%



 \begin{enumerate}


 \item For all models and algorithms presented, check if you include:
 \begin{enumerate}
   \item A clear description of the mathematical setting, assumptions, algorithm, and/or model. [Yes, we describe the mathematical setting, preliminaries, and problem formulation in Section \ref{sec:pf}. We present the proposed algorithms in Section~\ref{sec:alg} with their corresponding pseudocodes. All theorems and assumptions are stated in Section~\ref{sec:anal}. Theorem~\ref{the:conf} is self-contained, with all necessary assumptions explicitly mentioned in the body of the theorem. It is formulated to be broadly applicable to other problems. Theorems~\ref{the:gen} and~\ref{the:main} rely on Assumptions~\ref{ass:disc} and~\ref{closure_assumption}, which are clearly stated in Section~\ref{sec:anal}.
]
   \item An analysis of the properties and complexity (time, space, sample size) of any algorithm. [Yes, detailed analysis about the confidence intervals and sample complexities of our proposed algorithms are provided in Theorems~\ref{the:conf},~\ref{the:gen} and~\ref{the:main} of Section \ref{sec:anal}. However, we do not provide analysis for time and space complexities.]
   \item (Optional) Anonymized source code, with specification of all dependencies, including external libraries. [Yes, the code used to conduct our experiments is included in a zip file as supplementary material. It also contains a README file and a requirements file to facilitate the installation of all necessary packages.]
 \end{enumerate}


 \item For any theoretical claim, check if you include:
 \begin{enumerate}
   \item Statements of the full set of assumptions of all theoretical results. [Yes, we clearly state the full set of assumptions (Assumptions \ref{ass:disc} and \ref{closure_assumption}) in Section \ref{sec:anal}.] 
   \item Complete proofs of all theoretical results. [Yes, we provide the detailed proofs of Theorems~\ref{the:conf}, ~\ref{the:gen} and~\ref{the:main} in Appendices \ref{appx:conf}, \ref{appx:gen}, \ref{appx:main_sample}, respectively.]
   \item Clear explanations of any assumptions. [Yes, we explicitly state the assumptions of Theorem~\ref{the:conf} within the main body of the theorem, making it self-contained. Assumptions~\ref{ass:disc} and~\ref{closure_assumption}, which are necessary for Theorems~\ref{the:gen} and~\ref{the:main}, are clearly articulated and explained separately.]  
 \end{enumerate}


 \item For all figures and tables that present empirical results, check if you include:
 \begin{enumerate}
   \item The code, data, and instructions needed to reproduce the main experimental results (either in the supplemental material or as a URL). [Yes, we provide the code as an anonymized zip file in the supplementary material, along with a Readme file that instructs the user on how to run the code.]
   \item All the training details (e.g., data splits, hyperparameters, how they were chosen). [Yes, the main paper provides a core explanation of the results, including the
algorithms tested, kernels used, and the synthetic framework in Section \ref{sec:exp}. For comprehensive details (hyperparameter-tuning, visualizations of the reward and transition probability functions), please refer to Appendix \ref{appx:exp}.]
         \item A clear definition of the specific measure or statistics and error bars (e.g., with respect to the random seed after running experiments multiple times). [Yes, our results are accompanied by error bars indicating the standard deviation across 80 independent runs of our experiments.]
         \item A description of the computing infrastructure used. (e.g., type of GPUs, internal cluster, or cloud provider). [Yes, we include in Appendix \ref{appx:exp} the computational resources required for our experiments.]
 \end{enumerate}

 \item If you are using existing assets (e.g., code, data, models) or curating/releasing new assets, check if you include:
 \begin{enumerate}
   \item Citations of the creator If your work uses existing assets. [Yes, we have used the scikit-learn library to implement our kernel-based algorithms, and we have properly cited it (See Appendix \ref{appx:exp}). ]
   \item The license information of the assets, if applicable. [Not applicable]
   \item New assets either in the supplemental material or as a URL, if applicable. [Yes, we submit the code generating our experimental results as a zip file in the supplementary material.]
   \item Information about consent from data providers/curators. [Not Applicable]
   \item Discussion of sensible content if applicable, e.g., personally identifiable information or offensive content. [Not Applicable]
 \end{enumerate}

 \item If you used crowdsourcing or conducted research with human subjects, check if you include:
 \begin{enumerate}
   \item The full text of instructions given to participants and screenshots. [Not Applicable]
   \item Descriptions of potential participant risks, with links to Institutional Review Board (IRB) approvals if applicable. [Not Applicable]
   \item The estimated hourly wage paid to participants and the total amount spent on participant compensation. [Not Applicable]
 \end{enumerate}

 \end{enumerate}
\appendix
\subsection{Lloyd-Max Algorithm}
\label{subsec:Lloyd-Max}
For a given quantization bitwidth $B$ and an operand $\bm{X}$, the Lloyd-Max algorithm finds $2^B$ quantization levels $\{\hat{x}_i\}_{i=1}^{2^B}$ such that quantizing $\bm{X}$ by rounding each scalar in $\bm{X}$ to the nearest quantization level minimizes the quantization MSE. 

The algorithm starts with an initial guess of quantization levels and then iteratively computes quantization thresholds $\{\tau_i\}_{i=1}^{2^B-1}$ and updates quantization levels $\{\hat{x}_i\}_{i=1}^{2^B}$. Specifically, at iteration $n$, thresholds are set to the midpoints of the previous iteration's levels:
\begin{align*}
    \tau_i^{(n)}=\frac{\hat{x}_i^{(n-1)}+\hat{x}_{i+1}^{(n-1)}}2 \text{ for } i=1\ldots 2^B-1
\end{align*}
Subsequently, the quantization levels are re-computed as conditional means of the data regions defined by the new thresholds:
\begin{align*}
    \hat{x}_i^{(n)}=\mathbb{E}\left[ \bm{X} \big| \bm{X}\in [\tau_{i-1}^{(n)},\tau_i^{(n)}] \right] \text{ for } i=1\ldots 2^B
\end{align*}
where to satisfy boundary conditions we have $\tau_0=-\infty$ and $\tau_{2^B}=\infty$. The algorithm iterates the above steps until convergence.

Figure \ref{fig:lm_quant} compares the quantization levels of a $7$-bit floating point (E3M3) quantizer (left) to a $7$-bit Lloyd-Max quantizer (right) when quantizing a layer of weights from the GPT3-126M model at a per-tensor granularity. As shown, the Lloyd-Max quantizer achieves substantially lower quantization MSE. Further, Table \ref{tab:FP7_vs_LM7} shows the superior perplexity achieved by Lloyd-Max quantizers for bitwidths of $7$, $6$ and $5$. The difference between the quantizers is clear at 5 bits, where per-tensor FP quantization incurs a drastic and unacceptable increase in perplexity, while Lloyd-Max quantization incurs a much smaller increase. Nevertheless, we note that even the optimal Lloyd-Max quantizer incurs a notable ($\sim 1.5$) increase in perplexity due to the coarse granularity of quantization. 

\begin{figure}[h]
  \centering
  \includegraphics[width=0.7\linewidth]{sections/figures/LM7_FP7.pdf}
  \caption{\small Quantization levels and the corresponding quantization MSE of Floating Point (left) vs Lloyd-Max (right) Quantizers for a layer of weights in the GPT3-126M model.}
  \label{fig:lm_quant}
\end{figure}

\begin{table}[h]\scriptsize
\begin{center}
\caption{\label{tab:FP7_vs_LM7} \small Comparing perplexity (lower is better) achieved by floating point quantizers and Lloyd-Max quantizers on a GPT3-126M model for the Wikitext-103 dataset.}
\begin{tabular}{c|cc|c}
\hline
 \multirow{2}{*}{\textbf{Bitwidth}} & \multicolumn{2}{|c|}{\textbf{Floating-Point Quantizer}} & \textbf{Lloyd-Max Quantizer} \\
 & Best Format & Wikitext-103 Perplexity & Wikitext-103 Perplexity \\
\hline
7 & E3M3 & 18.32 & 18.27 \\
6 & E3M2 & 19.07 & 18.51 \\
5 & E4M0 & 43.89 & 19.71 \\
\hline
\end{tabular}
\end{center}
\end{table}

\subsection{Proof of Local Optimality of LO-BCQ}
\label{subsec:lobcq_opt_proof}
For a given block $\bm{b}_j$, the quantization MSE during LO-BCQ can be empirically evaluated as $\frac{1}{L_b}\lVert \bm{b}_j- \bm{\hat{b}}_j\rVert^2_2$ where $\bm{\hat{b}}_j$ is computed from equation (\ref{eq:clustered_quantization_definition}) as $C_{f(\bm{b}_j)}(\bm{b}_j)$. Further, for a given block cluster $\mathcal{B}_i$, we compute the quantization MSE as $\frac{1}{|\mathcal{B}_{i}|}\sum_{\bm{b} \in \mathcal{B}_{i}} \frac{1}{L_b}\lVert \bm{b}- C_i^{(n)}(\bm{b})\rVert^2_2$. Therefore, at the end of iteration $n$, we evaluate the overall quantization MSE $J^{(n)}$ for a given operand $\bm{X}$ composed of $N_c$ block clusters as:
\begin{align*}
    \label{eq:mse_iter_n}
    J^{(n)} = \frac{1}{N_c} \sum_{i=1}^{N_c} \frac{1}{|\mathcal{B}_{i}^{(n)}|}\sum_{\bm{v} \in \mathcal{B}_{i}^{(n)}} \frac{1}{L_b}\lVert \bm{b}- B_i^{(n)}(\bm{b})\rVert^2_2
\end{align*}

At the end of iteration $n$, the codebooks are updated from $\mathcal{C}^{(n-1)}$ to $\mathcal{C}^{(n)}$. However, the mapping of a given vector $\bm{b}_j$ to quantizers $\mathcal{C}^{(n)}$ remains as  $f^{(n)}(\bm{b}_j)$. At the next iteration, during the vector clustering step, $f^{(n+1)}(\bm{b}_j)$ finds new mapping of $\bm{b}_j$ to updated codebooks $\mathcal{C}^{(n)}$ such that the quantization MSE over the candidate codebooks is minimized. Therefore, we obtain the following result for $\bm{b}_j$:
\begin{align*}
\frac{1}{L_b}\lVert \bm{b}_j - C_{f^{(n+1)}(\bm{b}_j)}^{(n)}(\bm{b}_j)\rVert^2_2 \le \frac{1}{L_b}\lVert \bm{b}_j - C_{f^{(n)}(\bm{b}_j)}^{(n)}(\bm{b}_j)\rVert^2_2
\end{align*}

That is, quantizing $\bm{b}_j$ at the end of the block clustering step of iteration $n+1$ results in lower quantization MSE compared to quantizing at the end of iteration $n$. Since this is true for all $\bm{b} \in \bm{X}$, we assert the following:
\begin{equation}
\begin{split}
\label{eq:mse_ineq_1}
    \tilde{J}^{(n+1)} &= \frac{1}{N_c} \sum_{i=1}^{N_c} \frac{1}{|\mathcal{B}_{i}^{(n+1)}|}\sum_{\bm{b} \in \mathcal{B}_{i}^{(n+1)}} \frac{1}{L_b}\lVert \bm{b} - C_i^{(n)}(b)\rVert^2_2 \le J^{(n)}
\end{split}
\end{equation}
where $\tilde{J}^{(n+1)}$ is the the quantization MSE after the vector clustering step at iteration $n+1$.

Next, during the codebook update step (\ref{eq:quantizers_update}) at iteration $n+1$, the per-cluster codebooks $\mathcal{C}^{(n)}$ are updated to $\mathcal{C}^{(n+1)}$ by invoking the Lloyd-Max algorithm \citep{Lloyd}. We know that for any given value distribution, the Lloyd-Max algorithm minimizes the quantization MSE. Therefore, for a given vector cluster $\mathcal{B}_i$ we obtain the following result:

\begin{equation}
    \frac{1}{|\mathcal{B}_{i}^{(n+1)}|}\sum_{\bm{b} \in \mathcal{B}_{i}^{(n+1)}} \frac{1}{L_b}\lVert \bm{b}- C_i^{(n+1)}(\bm{b})\rVert^2_2 \le \frac{1}{|\mathcal{B}_{i}^{(n+1)}|}\sum_{\bm{b} \in \mathcal{B}_{i}^{(n+1)}} \frac{1}{L_b}\lVert \bm{b}- C_i^{(n)}(\bm{b})\rVert^2_2
\end{equation}

The above equation states that quantizing the given block cluster $\mathcal{B}_i$ after updating the associated codebook from $C_i^{(n)}$ to $C_i^{(n+1)}$ results in lower quantization MSE. Since this is true for all the block clusters, we derive the following result: 
\begin{equation}
\begin{split}
\label{eq:mse_ineq_2}
     J^{(n+1)} &= \frac{1}{N_c} \sum_{i=1}^{N_c} \frac{1}{|\mathcal{B}_{i}^{(n+1)}|}\sum_{\bm{b} \in \mathcal{B}_{i}^{(n+1)}} \frac{1}{L_b}\lVert \bm{b}- C_i^{(n+1)}(\bm{b})\rVert^2_2  \le \tilde{J}^{(n+1)}   
\end{split}
\end{equation}

Following (\ref{eq:mse_ineq_1}) and (\ref{eq:mse_ineq_2}), we find that the quantization MSE is non-increasing for each iteration, that is, $J^{(1)} \ge J^{(2)} \ge J^{(3)} \ge \ldots \ge J^{(M)}$ where $M$ is the maximum number of iterations. 
%Therefore, we can say that if the algorithm converges, then it must be that it has converged to a local minimum. 
\hfill $\blacksquare$


\begin{figure}
    \begin{center}
    \includegraphics[width=0.5\textwidth]{sections//figures/mse_vs_iter.pdf}
    \end{center}
    \caption{\small NMSE vs iterations during LO-BCQ compared to other block quantization proposals}
    \label{fig:nmse_vs_iter}
\end{figure}

Figure \ref{fig:nmse_vs_iter} shows the empirical convergence of LO-BCQ across several block lengths and number of codebooks. Also, the MSE achieved by LO-BCQ is compared to baselines such as MXFP and VSQ. As shown, LO-BCQ converges to a lower MSE than the baselines. Further, we achieve better convergence for larger number of codebooks ($N_c$) and for a smaller block length ($L_b$), both of which increase the bitwidth of BCQ (see Eq \ref{eq:bitwidth_bcq}).


\subsection{Additional Accuracy Results}
%Table \ref{tab:lobcq_config} lists the various LOBCQ configurations and their corresponding bitwidths.
\begin{table}
\setlength{\tabcolsep}{4.75pt}
\begin{center}
\caption{\label{tab:lobcq_config} Various LO-BCQ configurations and their bitwidths.}
\begin{tabular}{|c||c|c|c|c||c|c||c|} 
\hline
 & \multicolumn{4}{|c||}{$L_b=8$} & \multicolumn{2}{|c||}{$L_b=4$} & $L_b=2$ \\
 \hline
 \backslashbox{$L_A$\kern-1em}{\kern-1em$N_c$} & 2 & 4 & 8 & 16 & 2 & 4 & 2 \\
 \hline
 64 & 4.25 & 4.375 & 4.5 & 4.625 & 4.375 & 4.625 & 4.625\\
 \hline
 32 & 4.375 & 4.5 & 4.625& 4.75 & 4.5 & 4.75 & 4.75 \\
 \hline
 16 & 4.625 & 4.75& 4.875 & 5 & 4.75 & 5 & 5 \\
 \hline
\end{tabular}
\end{center}
\end{table}

%\subsection{Perplexity achieved by various LO-BCQ configurations on Wikitext-103 dataset}

\begin{table} \centering
\begin{tabular}{|c||c|c|c|c||c|c||c|} 
\hline
 $L_b \rightarrow$& \multicolumn{4}{c||}{8} & \multicolumn{2}{c||}{4} & 2\\
 \hline
 \backslashbox{$L_A$\kern-1em}{\kern-1em$N_c$} & 2 & 4 & 8 & 16 & 2 & 4 & 2  \\
 %$N_c \rightarrow$ & 2 & 4 & 8 & 16 & 2 & 4 & 2 \\
 \hline
 \hline
 \multicolumn{8}{c}{GPT3-1.3B (FP32 PPL = 9.98)} \\ 
 \hline
 \hline
 64 & 10.40 & 10.23 & 10.17 & 10.15 &  10.28 & 10.18 & 10.19 \\
 \hline
 32 & 10.25 & 10.20 & 10.15 & 10.12 &  10.23 & 10.17 & 10.17 \\
 \hline
 16 & 10.22 & 10.16 & 10.10 & 10.09 &  10.21 & 10.14 & 10.16 \\
 \hline
  \hline
 \multicolumn{8}{c}{GPT3-8B (FP32 PPL = 7.38)} \\ 
 \hline
 \hline
 64 & 7.61 & 7.52 & 7.48 &  7.47 &  7.55 &  7.49 & 7.50 \\
 \hline
 32 & 7.52 & 7.50 & 7.46 &  7.45 &  7.52 &  7.48 & 7.48  \\
 \hline
 16 & 7.51 & 7.48 & 7.44 &  7.44 &  7.51 &  7.49 & 7.47  \\
 \hline
\end{tabular}
\caption{\label{tab:ppl_gpt3_abalation} Wikitext-103 perplexity across GPT3-1.3B and 8B models.}
\end{table}

\begin{table} \centering
\begin{tabular}{|c||c|c|c|c||} 
\hline
 $L_b \rightarrow$& \multicolumn{4}{c||}{8}\\
 \hline
 \backslashbox{$L_A$\kern-1em}{\kern-1em$N_c$} & 2 & 4 & 8 & 16 \\
 %$N_c \rightarrow$ & 2 & 4 & 8 & 16 & 2 & 4 & 2 \\
 \hline
 \hline
 \multicolumn{5}{|c|}{Llama2-7B (FP32 PPL = 5.06)} \\ 
 \hline
 \hline
 64 & 5.31 & 5.26 & 5.19 & 5.18  \\
 \hline
 32 & 5.23 & 5.25 & 5.18 & 5.15  \\
 \hline
 16 & 5.23 & 5.19 & 5.16 & 5.14  \\
 \hline
 \multicolumn{5}{|c|}{Nemotron4-15B (FP32 PPL = 5.87)} \\ 
 \hline
 \hline
 64  & 6.3 & 6.20 & 6.13 & 6.08  \\
 \hline
 32  & 6.24 & 6.12 & 6.07 & 6.03  \\
 \hline
 16  & 6.12 & 6.14 & 6.04 & 6.02  \\
 \hline
 \multicolumn{5}{|c|}{Nemotron4-340B (FP32 PPL = 3.48)} \\ 
 \hline
 \hline
 64 & 3.67 & 3.62 & 3.60 & 3.59 \\
 \hline
 32 & 3.63 & 3.61 & 3.59 & 3.56 \\
 \hline
 16 & 3.61 & 3.58 & 3.57 & 3.55 \\
 \hline
\end{tabular}
\caption{\label{tab:ppl_llama7B_nemo15B} Wikitext-103 perplexity compared to FP32 baseline in Llama2-7B and Nemotron4-15B, 340B models}
\end{table}

%\subsection{Perplexity achieved by various LO-BCQ configurations on MMLU dataset}


\begin{table} \centering
\begin{tabular}{|c||c|c|c|c||c|c|c|c|} 
\hline
 $L_b \rightarrow$& \multicolumn{4}{c||}{8} & \multicolumn{4}{c||}{8}\\
 \hline
 \backslashbox{$L_A$\kern-1em}{\kern-1em$N_c$} & 2 & 4 & 8 & 16 & 2 & 4 & 8 & 16  \\
 %$N_c \rightarrow$ & 2 & 4 & 8 & 16 & 2 & 4 & 2 \\
 \hline
 \hline
 \multicolumn{5}{|c|}{Llama2-7B (FP32 Accuracy = 45.8\%)} & \multicolumn{4}{|c|}{Llama2-70B (FP32 Accuracy = 69.12\%)} \\ 
 \hline
 \hline
 64 & 43.9 & 43.4 & 43.9 & 44.9 & 68.07 & 68.27 & 68.17 & 68.75 \\
 \hline
 32 & 44.5 & 43.8 & 44.9 & 44.5 & 68.37 & 68.51 & 68.35 & 68.27  \\
 \hline
 16 & 43.9 & 42.7 & 44.9 & 45 & 68.12 & 68.77 & 68.31 & 68.59  \\
 \hline
 \hline
 \multicolumn{5}{|c|}{GPT3-22B (FP32 Accuracy = 38.75\%)} & \multicolumn{4}{|c|}{Nemotron4-15B (FP32 Accuracy = 64.3\%)} \\ 
 \hline
 \hline
 64 & 36.71 & 38.85 & 38.13 & 38.92 & 63.17 & 62.36 & 63.72 & 64.09 \\
 \hline
 32 & 37.95 & 38.69 & 39.45 & 38.34 & 64.05 & 62.30 & 63.8 & 64.33  \\
 \hline
 16 & 38.88 & 38.80 & 38.31 & 38.92 & 63.22 & 63.51 & 63.93 & 64.43  \\
 \hline
\end{tabular}
\caption{\label{tab:mmlu_abalation} Accuracy on MMLU dataset across GPT3-22B, Llama2-7B, 70B and Nemotron4-15B models.}
\end{table}


%\subsection{Perplexity achieved by various LO-BCQ configurations on LM evaluation harness}

\begin{table} \centering
\begin{tabular}{|c||c|c|c|c||c|c|c|c|} 
\hline
 $L_b \rightarrow$& \multicolumn{4}{c||}{8} & \multicolumn{4}{c||}{8}\\
 \hline
 \backslashbox{$L_A$\kern-1em}{\kern-1em$N_c$} & 2 & 4 & 8 & 16 & 2 & 4 & 8 & 16  \\
 %$N_c \rightarrow$ & 2 & 4 & 8 & 16 & 2 & 4 & 2 \\
 \hline
 \hline
 \multicolumn{5}{|c|}{Race (FP32 Accuracy = 37.51\%)} & \multicolumn{4}{|c|}{Boolq (FP32 Accuracy = 64.62\%)} \\ 
 \hline
 \hline
 64 & 36.94 & 37.13 & 36.27 & 37.13 & 63.73 & 62.26 & 63.49 & 63.36 \\
 \hline
 32 & 37.03 & 36.36 & 36.08 & 37.03 & 62.54 & 63.51 & 63.49 & 63.55  \\
 \hline
 16 & 37.03 & 37.03 & 36.46 & 37.03 & 61.1 & 63.79 & 63.58 & 63.33  \\
 \hline
 \hline
 \multicolumn{5}{|c|}{Winogrande (FP32 Accuracy = 58.01\%)} & \multicolumn{4}{|c|}{Piqa (FP32 Accuracy = 74.21\%)} \\ 
 \hline
 \hline
 64 & 58.17 & 57.22 & 57.85 & 58.33 & 73.01 & 73.07 & 73.07 & 72.80 \\
 \hline
 32 & 59.12 & 58.09 & 57.85 & 58.41 & 73.01 & 73.94 & 72.74 & 73.18  \\
 \hline
 16 & 57.93 & 58.88 & 57.93 & 58.56 & 73.94 & 72.80 & 73.01 & 73.94  \\
 \hline
\end{tabular}
\caption{\label{tab:mmlu_abalation} Accuracy on LM evaluation harness tasks on GPT3-1.3B model.}
\end{table}

\begin{table} \centering
\begin{tabular}{|c||c|c|c|c||c|c|c|c|} 
\hline
 $L_b \rightarrow$& \multicolumn{4}{c||}{8} & \multicolumn{4}{c||}{8}\\
 \hline
 \backslashbox{$L_A$\kern-1em}{\kern-1em$N_c$} & 2 & 4 & 8 & 16 & 2 & 4 & 8 & 16  \\
 %$N_c \rightarrow$ & 2 & 4 & 8 & 16 & 2 & 4 & 2 \\
 \hline
 \hline
 \multicolumn{5}{|c|}{Race (FP32 Accuracy = 41.34\%)} & \multicolumn{4}{|c|}{Boolq (FP32 Accuracy = 68.32\%)} \\ 
 \hline
 \hline
 64 & 40.48 & 40.10 & 39.43 & 39.90 & 69.20 & 68.41 & 69.45 & 68.56 \\
 \hline
 32 & 39.52 & 39.52 & 40.77 & 39.62 & 68.32 & 67.43 & 68.17 & 69.30  \\
 \hline
 16 & 39.81 & 39.71 & 39.90 & 40.38 & 68.10 & 66.33 & 69.51 & 69.42  \\
 \hline
 \hline
 \multicolumn{5}{|c|}{Winogrande (FP32 Accuracy = 67.88\%)} & \multicolumn{4}{|c|}{Piqa (FP32 Accuracy = 78.78\%)} \\ 
 \hline
 \hline
 64 & 66.85 & 66.61 & 67.72 & 67.88 & 77.31 & 77.42 & 77.75 & 77.64 \\
 \hline
 32 & 67.25 & 67.72 & 67.72 & 67.00 & 77.31 & 77.04 & 77.80 & 77.37  \\
 \hline
 16 & 68.11 & 68.90 & 67.88 & 67.48 & 77.37 & 78.13 & 78.13 & 77.69  \\
 \hline
\end{tabular}
\caption{\label{tab:mmlu_abalation} Accuracy on LM evaluation harness tasks on GPT3-8B model.}
\end{table}

\begin{table} \centering
\begin{tabular}{|c||c|c|c|c||c|c|c|c|} 
\hline
 $L_b \rightarrow$& \multicolumn{4}{c||}{8} & \multicolumn{4}{c||}{8}\\
 \hline
 \backslashbox{$L_A$\kern-1em}{\kern-1em$N_c$} & 2 & 4 & 8 & 16 & 2 & 4 & 8 & 16  \\
 %$N_c \rightarrow$ & 2 & 4 & 8 & 16 & 2 & 4 & 2 \\
 \hline
 \hline
 \multicolumn{5}{|c|}{Race (FP32 Accuracy = 40.67\%)} & \multicolumn{4}{|c|}{Boolq (FP32 Accuracy = 76.54\%)} \\ 
 \hline
 \hline
 64 & 40.48 & 40.10 & 39.43 & 39.90 & 75.41 & 75.11 & 77.09 & 75.66 \\
 \hline
 32 & 39.52 & 39.52 & 40.77 & 39.62 & 76.02 & 76.02 & 75.96 & 75.35  \\
 \hline
 16 & 39.81 & 39.71 & 39.90 & 40.38 & 75.05 & 73.82 & 75.72 & 76.09  \\
 \hline
 \hline
 \multicolumn{5}{|c|}{Winogrande (FP32 Accuracy = 70.64\%)} & \multicolumn{4}{|c|}{Piqa (FP32 Accuracy = 79.16\%)} \\ 
 \hline
 \hline
 64 & 69.14 & 70.17 & 70.17 & 70.56 & 78.24 & 79.00 & 78.62 & 78.73 \\
 \hline
 32 & 70.96 & 69.69 & 71.27 & 69.30 & 78.56 & 79.49 & 79.16 & 78.89  \\
 \hline
 16 & 71.03 & 69.53 & 69.69 & 70.40 & 78.13 & 79.16 & 79.00 & 79.00  \\
 \hline
\end{tabular}
\caption{\label{tab:mmlu_abalation} Accuracy on LM evaluation harness tasks on GPT3-22B model.}
\end{table}

\begin{table} \centering
\begin{tabular}{|c||c|c|c|c||c|c|c|c|} 
\hline
 $L_b \rightarrow$& \multicolumn{4}{c||}{8} & \multicolumn{4}{c||}{8}\\
 \hline
 \backslashbox{$L_A$\kern-1em}{\kern-1em$N_c$} & 2 & 4 & 8 & 16 & 2 & 4 & 8 & 16  \\
 %$N_c \rightarrow$ & 2 & 4 & 8 & 16 & 2 & 4 & 2 \\
 \hline
 \hline
 \multicolumn{5}{|c|}{Race (FP32 Accuracy = 44.4\%)} & \multicolumn{4}{|c|}{Boolq (FP32 Accuracy = 79.29\%)} \\ 
 \hline
 \hline
 64 & 42.49 & 42.51 & 42.58 & 43.45 & 77.58 & 77.37 & 77.43 & 78.1 \\
 \hline
 32 & 43.35 & 42.49 & 43.64 & 43.73 & 77.86 & 75.32 & 77.28 & 77.86  \\
 \hline
 16 & 44.21 & 44.21 & 43.64 & 42.97 & 78.65 & 77 & 76.94 & 77.98  \\
 \hline
 \hline
 \multicolumn{5}{|c|}{Winogrande (FP32 Accuracy = 69.38\%)} & \multicolumn{4}{|c|}{Piqa (FP32 Accuracy = 78.07\%)} \\ 
 \hline
 \hline
 64 & 68.9 & 68.43 & 69.77 & 68.19 & 77.09 & 76.82 & 77.09 & 77.86 \\
 \hline
 32 & 69.38 & 68.51 & 68.82 & 68.90 & 78.07 & 76.71 & 78.07 & 77.86  \\
 \hline
 16 & 69.53 & 67.09 & 69.38 & 68.90 & 77.37 & 77.8 & 77.91 & 77.69  \\
 \hline
\end{tabular}
\caption{\label{tab:mmlu_abalation} Accuracy on LM evaluation harness tasks on Llama2-7B model.}
\end{table}

\begin{table} \centering
\begin{tabular}{|c||c|c|c|c||c|c|c|c|} 
\hline
 $L_b \rightarrow$& \multicolumn{4}{c||}{8} & \multicolumn{4}{c||}{8}\\
 \hline
 \backslashbox{$L_A$\kern-1em}{\kern-1em$N_c$} & 2 & 4 & 8 & 16 & 2 & 4 & 8 & 16  \\
 %$N_c \rightarrow$ & 2 & 4 & 8 & 16 & 2 & 4 & 2 \\
 \hline
 \hline
 \multicolumn{5}{|c|}{Race (FP32 Accuracy = 48.8\%)} & \multicolumn{4}{|c|}{Boolq (FP32 Accuracy = 85.23\%)} \\ 
 \hline
 \hline
 64 & 49.00 & 49.00 & 49.28 & 48.71 & 82.82 & 84.28 & 84.03 & 84.25 \\
 \hline
 32 & 49.57 & 48.52 & 48.33 & 49.28 & 83.85 & 84.46 & 84.31 & 84.93  \\
 \hline
 16 & 49.85 & 49.09 & 49.28 & 48.99 & 85.11 & 84.46 & 84.61 & 83.94  \\
 \hline
 \hline
 \multicolumn{5}{|c|}{Winogrande (FP32 Accuracy = 79.95\%)} & \multicolumn{4}{|c|}{Piqa (FP32 Accuracy = 81.56\%)} \\ 
 \hline
 \hline
 64 & 78.77 & 78.45 & 78.37 & 79.16 & 81.45 & 80.69 & 81.45 & 81.5 \\
 \hline
 32 & 78.45 & 79.01 & 78.69 & 80.66 & 81.56 & 80.58 & 81.18 & 81.34  \\
 \hline
 16 & 79.95 & 79.56 & 79.79 & 79.72 & 81.28 & 81.66 & 81.28 & 80.96  \\
 \hline
\end{tabular}
\caption{\label{tab:mmlu_abalation} Accuracy on LM evaluation harness tasks on Llama2-70B model.}
\end{table}

%\section{MSE Studies}
%\textcolor{red}{TODO}


\subsection{Number Formats and Quantization Method}
\label{subsec:numFormats_quantMethod}
\subsubsection{Integer Format}
An $n$-bit signed integer (INT) is typically represented with a 2s-complement format \citep{yao2022zeroquant,xiao2023smoothquant,dai2021vsq}, where the most significant bit denotes the sign.

\subsubsection{Floating Point Format}
An $n$-bit signed floating point (FP) number $x$ comprises of a 1-bit sign ($x_{\mathrm{sign}}$), $B_m$-bit mantissa ($x_{\mathrm{mant}}$) and $B_e$-bit exponent ($x_{\mathrm{exp}}$) such that $B_m+B_e=n-1$. The associated constant exponent bias ($E_{\mathrm{bias}}$) is computed as $(2^{{B_e}-1}-1)$. We denote this format as $E_{B_e}M_{B_m}$.  

\subsubsection{Quantization Scheme}
\label{subsec:quant_method}
A quantization scheme dictates how a given unquantized tensor is converted to its quantized representation. We consider FP formats for the purpose of illustration. Given an unquantized tensor $\bm{X}$ and an FP format $E_{B_e}M_{B_m}$, we first, we compute the quantization scale factor $s_X$ that maps the maximum absolute value of $\bm{X}$ to the maximum quantization level of the $E_{B_e}M_{B_m}$ format as follows:
\begin{align}
\label{eq:sf}
    s_X = \frac{\mathrm{max}(|\bm{X}|)}{\mathrm{max}(E_{B_e}M_{B_m})}
\end{align}
In the above equation, $|\cdot|$ denotes the absolute value function.

Next, we scale $\bm{X}$ by $s_X$ and quantize it to $\hat{\bm{X}}$ by rounding it to the nearest quantization level of $E_{B_e}M_{B_m}$ as:

\begin{align}
\label{eq:tensor_quant}
    \hat{\bm{X}} = \text{round-to-nearest}\left(\frac{\bm{X}}{s_X}, E_{B_e}M_{B_m}\right)
\end{align}

We perform dynamic max-scaled quantization \citep{wu2020integer}, where the scale factor $s$ for activations is dynamically computed during runtime.

\subsection{Vector Scaled Quantization}
\begin{wrapfigure}{r}{0.35\linewidth}
  \centering
  \includegraphics[width=\linewidth]{sections/figures/vsquant.jpg}
  \caption{\small Vectorwise decomposition for per-vector scaled quantization (VSQ \citep{dai2021vsq}).}
  \label{fig:vsquant}
\end{wrapfigure}
During VSQ \citep{dai2021vsq}, the operand tensors are decomposed into 1D vectors in a hardware friendly manner as shown in Figure \ref{fig:vsquant}. Since the decomposed tensors are used as operands in matrix multiplications during inference, it is beneficial to perform this decomposition along the reduction dimension of the multiplication. The vectorwise quantization is performed similar to tensorwise quantization described in Equations \ref{eq:sf} and \ref{eq:tensor_quant}, where a scale factor $s_v$ is required for each vector $\bm{v}$ that maps the maximum absolute value of that vector to the maximum quantization level. While smaller vector lengths can lead to larger accuracy gains, the associated memory and computational overheads due to the per-vector scale factors increases. To alleviate these overheads, VSQ \citep{dai2021vsq} proposed a second level quantization of the per-vector scale factors to unsigned integers, while MX \citep{rouhani2023shared} quantizes them to integer powers of 2 (denoted as $2^{INT}$).

\subsubsection{MX Format}
The MX format proposed in \citep{rouhani2023microscaling} introduces the concept of sub-block shifting. For every two scalar elements of $b$-bits each, there is a shared exponent bit. The value of this exponent bit is determined through an empirical analysis that targets minimizing quantization MSE. We note that the FP format $E_{1}M_{b}$ is strictly better than MX from an accuracy perspective since it allocates a dedicated exponent bit to each scalar as opposed to sharing it across two scalars. Therefore, we conservatively bound the accuracy of a $b+2$-bit signed MX format with that of a $E_{1}M_{b}$ format in our comparisons. For instance, we use E1M2 format as a proxy for MX4.

\begin{figure}
    \centering
    \includegraphics[width=1\linewidth]{sections//figures/BlockFormats.pdf}
    \caption{\small Comparing LO-BCQ to MX format.}
    \label{fig:block_formats}
\end{figure}

Figure \ref{fig:block_formats} compares our $4$-bit LO-BCQ block format to MX \citep{rouhani2023microscaling}. As shown, both LO-BCQ and MX decompose a given operand tensor into block arrays and each block array into blocks. Similar to MX, we find that per-block quantization ($L_b < L_A$) leads to better accuracy due to increased flexibility. While MX achieves this through per-block $1$-bit micro-scales, we associate a dedicated codebook to each block through a per-block codebook selector. Further, MX quantizes the per-block array scale-factor to E8M0 format without per-tensor scaling. In contrast during LO-BCQ, we find that per-tensor scaling combined with quantization of per-block array scale-factor to E4M3 format results in superior inference accuracy across models. 



\end{document}
