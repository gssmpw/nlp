
This section is concerned with the proof of
Proposition~\ref{prop:dstep-potential} stated in
Section~\ref{sec:proof}. To this end, we proceed in three steps.
First, we establish a finite over-approximation on the set of the $d$
values that could be possibly reached in an execution involving
$d$-steps only from some initial configuration $\gamma_0$.
Second, we introduce a
partitioning of edges (being either smooth or non-smooth) and prove
some preservation properties along $d$-steps.
Third, we combine the above results to effectively construct a
potential function for $d$-steps and a well-founded order on the
co-domain of this function, ultimately proving
Proposition~\ref{prop:dstep-potential}.

For the sake of readability, we denote $d$-steps $\gamma \xstep{\DStep}
\gamma'$ shortly by $\gamma \dstep \gamma'$.

\subsection{Bounds on Distance Values}
\label{sec:dstep-potential:bounds}

For a configuration $\gamma$, we define the integers $\maxd\gamma
\isdef \max \{\gamma.q.d \mid q \in \nodes\}$, $\mind\gamma \isdef
\min \{\gamma.q.d \mid q \in \nodes\}$, $\sumd\gamma \isdef \sum
\{\gamma.q.d \mid q \in \nodes\}$\footnote{The sum is taken on the
multiset of $d$ values}.  We also define $\dbot{\gamma}$,
$\dtop{\gamma}$ respectively a \emph{bottom} and a \emph{top}
configuration associated to $\gamma$.  These are identical to $\gamma$
except for $d$ values, defined for every node $p$ as follows:
\begin{eqnarray*}
  \dbot{\gamma}.p.d & \isdef & \mind\gamma \\
  \dtop{\gamma}.p.d & \isdef & \left\{ \begin{array}{ll}
    \gamma.p.d & \mbox{if } p = r \\
    \max \{ \gamma.p.d, 1 + \min \{ \dtop{\gamma}.q.d \mid \\
    \hspace{1cm} (p,q)\in \Edges, \dist{p}{\r} = 1 + \dist{q}{\r} \}
    \} &
    \mbox{otherwise,}
  \end{array} \right.
\end{eqnarray*}
where $\dist{q}{\r}$ represents the distance of some node $q$ to the
root \r.  Note that the recursive definition of $\dtop{\gamma}.p.d$ is
well-defined as the recursion is limited to neighbours $q$ of $p$
located at a smaller distance to the root $\r$ than $p$.  Intuitively,
the maximal $d$ value of a non-root node $p$ in some configuration
reachable from $\gamma$ is either its value in $\gamma$ (\ie, it can
be the case when $p$ does not execute) or 1 plus the minimum of the
maximal $d$ values of its neighbors $q$ closer to the root
(see Action $CD$ when $p$ executes).
We define the partial order $\dleq$ on configurations by taking
%
$$ \gamma_1 \dleq \gamma_2 \isdef \forall q \in \nodes: \gamma_1.q.d
\le \gamma_2.q.d $$
%
The next lemma states basic properties of the $\dbot{(.)}$,
$\dtop{(.)}$ operators, namely their idempotence and
their monotonicity with respect to $\dleq$.  The proof follows from
definitions and uses induction on nodes according to their
distance to the root.

\begin{lemma} \label{lemma:bounds:basic} ~
  
  \begin{enumerate}[label=(\roman*)]
  \item For all configuration $\gamma$,
  $\dbot{\gamma} \dleq \gamma \dleq \dtop{\gamma}$,
  $\dbot{(\dbot{\gamma})} = \dbot{\gamma}$ and $\dtop{(\dtop{\gamma})}
  = \dtop{\gamma}$.

  \item For all configurations $\gamma_1$ and $\gamma_2$ such that
  $\gamma_1 \dleq \gamma_2$, $\dbot{\gamma_1} \dleq \dbot{\gamma_2}$ and
  $\dtop{\gamma_1} \dleq \dtop{\gamma_2}$.

\end{enumerate}
\end{lemma}

The next lemma relates the bottom and top configurations to $d$-steps.
The proof is done by induction respectively, on the set of nodes
according to their distance to the root (i) and on the length of an
execution sequence from $\gamma_0$ (ii).\footnote{$\gamma_0 \dstepstar
\gamma$ means that $\gamma$ is reachable from $\gamma_0$ using a
finite number of $d$-steps.}
\begin{lemma}\label{lemma:bounds:dstep} ~
  
  \begin{enumerate}[label=(\roman*)]  
  \item For all configurations $\gamma$ and $\gamma'$ such that
  $\gamma \dstep \gamma'$, $\dbot{\gamma} \dleq \dbot{\gamma'}$ and
  $\dtop{\gamma'} \dleq \dtop{\gamma}$.

  \item For all configurations $\gamma_0$ and $\gamma$ such that
  $\gamma_0 \dstepstar \gamma$,
  $\dbot{\gamma_0} \dleq \gamma \dleq \dtop{\gamma_0}$.

  \end{enumerate}
\end{lemma}

\subsection{Smooth and Non-smooth $d$-steps}
\label{sec:dstep-potential:smooth}

We say that an edge $(p,q)\in\Edges$ is \emph{smooth}
(resp. \emph{non-smooth}) in a configuration $\gamma\in\Env$ if the
difference (in absolute value, $abs$) between the $d$-values at its
endpoints $p$, $q$ is at most 1 (resp. at least 2).  Formally,
consider the predicate
%
$$\csmooth{\gamma}{(p,q)} \isdef (abs(\gamma.p.d - \gamma.q.d) \le 1).$$
%
We say that a $d$-step $\gamma \dstep \gamma'$ is \emph{smooth} if all the
nodes $p$ changing their values from $\gamma$ to $\gamma'$ are
connected to smooth edges only in $\gamma$, formally:
%
$$\begin{array}{l}
  \ssmooth{\gamma \dstep \gamma'} \isdef \\
  \hspace{1cm} \forall p \in \Nodes: (\gamma'.p.d \not= \gamma.p.d) \Rightarrow
  (\forall q \in p.neighbors: \csmooth{\gamma}{(p,q)}
\end{array}$$
%
We define the rank of an edge $(p,q)\in\Edges$ in a configuration
$\gamma\in\Env$ as
$\crank{\gamma}{(p,q)} \isdef \min(\gamma.p.d, \gamma.q.d)$.



\begin{figure}[th]
  \centering
  \scalebox{0.9}{\input{d-steps.pdf_t}}
  \caption{\label{fig:d-steps}Smooth and non-smooth steps}
\end{figure}

For illustration, consider the three configurations $\gamma_1$,
$\gamma_2$, $\gamma_3$ depicted in Fig.~\ref{fig:d-steps}.  We
represented the $d$ values of the nodes by their positioning on the
horizontal lines e.g., $\gamma_1.\r.d = 10$, $\gamma_1.p_1.d = 9$,
$\gamma_2.p_1.d = 10$, etc.  Edges are represented by lines
connecting nodes: smooth (resp. non-smooth) edges are depicted in
blue (resp. red).  Configuration $\gamma_2$ is the successor of
$\gamma_1$ by a smooth step.  That is, only $p_1$ and $p_6$ have
executed and these nodes were connected only to smooth (blue) edges
in $\gamma_1$.  Configuration $\gamma_3$ is the successor of
$\gamma_2$ by a non-smooth step.  That is, $p_3$ and $p_4$ have been
executed along the step, and these nodes were connected to some
non-smoth edges.

The next lemmas provide key properties for understanding the execution
of $d$-steps, depending if they are smooths or not.
Lemma~\ref{lemma:dsteps:smooth} basically states that partitioning
between smooth and non-smooth, as well as the rank of every non-smooth
edge is preserved by smooth steps.  In addition, the total sum of $d$
values is increasing along such a step.  

\begin{lemma}\label{lemma:dsteps:smooth}
  Consider a smooth d-step $\gamma \dstep \gamma'$.  Then,
  \begin{enumerate}[label=(\roman*)]
  \item $\forall e \in \edges: \neg \csmooth{\gamma}{e} \Leftrightarrow
    \neg \csmooth{\gamma'}{e}$,
  \item $\forall e \in \edges: \neg \csmooth{\gamma}{e} \Rightarrow
    (\crank{\gamma}{e} = \crank{\gamma'}{e})$,
  \item $\sumd \gamma' > \sumd \gamma$.
  \end{enumerate}
\end{lemma}
\begin{proof}
  The proof follows immediately from the definition of smooth steps
  and/or edges.  First, the fact that non-smooth edges are preserved
  along with their rank in a smooth $d$-step directly comes from the
  definition of a smooth step: since no node connected to a non-smooth
  edge can execute, non-smooth edges remained unchanged.
  Second, we obtain the increasing of the sum of all $d$-values by
  observing that when a node executes in a smooth $d$-step, its $d$
  value increases by one or two (due to its neighbors which are either
  above by one or at the same level of $d$ value). As a smooth
  $d$-step involves at least one such an executing node, $\sumd$
  necessarily increases (since nodes that do not increase $d$ leave it
  unchanged).
  \qed
\end{proof}

For illustration, consider the smooth step depicted in
Fig.~\ref{fig:d-steps}, \ie, between $\gamma_1$ and $\gamma_2$.  It is
rather trivial that, as long as the nodes executing were connected to
smooth edges only (in blue), their execution has no impact on the
non-smooth edges \ie, they remain non-smooth and preserve their rank.
Yet, the overall sum of the $d$ values increases, here because at
least the values of the two moving nodes has increased (by 1 for $p_1$
and by 2 for $p_6$).

Lemma~\ref{lemma:dsteps:nonsmooth} provides a similar preservation
result for non-smooth steps.  In this case, the key property is that
one can effectively compute a bound $k^*$ such that (i) all non-smooth
edges with rank lower than $k^*$ remain non-smooth and preserve their
rank and (ii) the set of non-smooth edges with rank $k^*$ is strictly
decreasing along the step.  The lemma provides both the explicit
definition of $k^*$ as well as the identification of a non-smooth edge
at level $k^*$ which either becomes smooth or gets a reduced rank
after the step, that is, some edge $(p,q)$ for which the minimum is
achieved in the definition of $k^*$.

\begin{lemma}\label{lemma:dsteps:nonsmooth} Consider a non-smooth d-step $\gamma \dstep \gamma'$.   Let
  $$\begin{array}{l} k^* \isdef \min \{ \crank{\gamma}{(p,q)} \mid (p, q) \in \edges:
    \neg \csmooth{\gamma}{(p,q)}, \\
    \hspace{5cm} \gamma'.p.d \not=\gamma.p.d \mbox{ or } \gamma'.q.d \not= \gamma.q.d \}
    \end{array}$$
  Then,
  \begin{enumerate}[label=(\roman*)]
  \item $\forall e \in \edges: (\crank{\gamma'}{e} \le k^* \wedge \neg \csmooth{\gamma'}{e}) \Rightarrow \\
    \hspace*{3cm} (\crank{\gamma}{e} = \crank{\gamma'}{e} \wedge \neg \csmooth{\gamma}{e})$,
  \item $\forall e \in \edges: (\crank{\gamma}{e} < k^* \wedge \neg \csmooth{\gamma}{e}) \Rightarrow \\
    \hspace*{3cm} (\crank{\gamma'}{e} = \crank{\gamma}{e} \wedge \neg \csmooth{\gamma'}{e})$,
  \item $\exists e \in \edges: (\crank{\gamma}{e} = k^* \wedge \neg \csmooth{\gamma}{e}) \wedge \\
    \hspace*{3cm} (\neg \csmooth{\gamma'}{e} \Rightarrow \crank{\gamma'}{e} > \crank{\gamma}{e}))$.
  \end{enumerate}
\end{lemma}
\begin{proof}
  (i) The proof is done by case splitting, considering which endpoints
  of non-smooth edges $e$ execute. In fact, the only feasible case is
  when none of them executes.  In all other cases, by choosing the
  node which gives a new value to its $d$ variable, we obtain a
  contradiction, either with the minimality of $k^*$ or with the
  non-smoothness of $e$ in $\gamma'$.

  (ii) By definition of $k^*$, no node involved in a non-smooth edge
  can execute if the rank is below $k^*$, hence rank and
  non-smoothness are left unchanged.

  (iii) Note here that, using Coq, to be able to prove "$\exists e \in
  \edges: ...$", we have to effectively contruct such an edge. In our
  case, it is chosen as some of the edges which achieves the minimum
  rank value when computing $k^*$: a non-smooth edge $e^*$ such that
  $\crank{\gamma}{e^*} = k^*$, and one of its end nodes executes
  during the step (it exists and can be computed using the computation
  of the minimum value over a finite set). Now, consider the case
  where $e^*$ remains non-smooth in $\gamma'$. We note $e^*=(p, q)$
  with $\crank{\gamma}{(p, q)} = \gamma.p.d$. We can prove that if $p$
  executes then $\gamma'.p.d > \gamma.p.d$ and that if $q$ executes
  then $\gamma'.q.d = \gamma.p.d + 1$ (see Fig.~\ref{fig:dsteps:proof}
  for an illustration). The result is then easy to conclude.
  \qed
\end{proof}

\begin{figure}[th]
  \begin{center}
    \scalebox{0.9}{\input{d-steps-proof.pdf_t}}
  \end{center}
  \caption{\label{fig:dsteps:proof}Possible evolutions of a non-smooth edge $e^*=(p,q)$ with
    minimal rank $k^*$: (i) only $p$ executes, (ii) only $q$ executes
    (iii) $p$ and $q$ executes}
\end{figure}

For illustration also, consider the non-smooth step depicted in
Fig.~\ref{fig:d-steps} between $\gamma_2$ and $\gamma_3$.  In this
case, the bound value is $k^* = 8$.  The lemma ensures that the set of
non-smooth edges of rank strictly lower than $8$ are unchanged.  No
such edges actually exist in the configurations $\gamma_2$ or
$\gamma_3$. But, actually, it is not hard to imagine that if such
edges would exist and are not related to $p_3$ and $p_6$, they would
not be impacted by the move.  Also, the lemma guarantees that the set
of edges at level 8 is strictly decreasing.  That is, the set of
non-smooth edges at level 8 is $\{ (p_3,p_7), (p_3,p_4) \}$ in
$\gamma_2$, respectively $\emptyset$ in $\gamma_3$.

Finally, we define $\nsset{\gamma}{k} \isdef \{ e \in \edges \mid \neg
\csmooth{\gamma}{e} \;\wedge\; \crank{\gamma}{e} = k \}$, that is, the
set of non-smooth edges of rang $k$ in $\gamma$.  The next lemma
simply re-formulates the results of Lemma \ref{lemma:dsteps:smooth} in
point (i) and Lemma \ref{lemma:dsteps:nonsmooth} in point (ii) into a
single statement about the sets $\nsset{\gamma}{k}$ to
facilitate their use in the definition of the potential function in
the next subsection.

\begin{lemma}\label{lemma:dsteps}
  Consider a d-step $\gamma \dstep \gamma'$.  Then
  \begin{enumerate}[label=(\roman*)]
  \item if the step $\gamma \dstep \gamma'$ is smooth then
    $\nsset{\gamma}{k} = \nsset{\gamma'}{k}$ for all integer $k$,
  \item if the step $\gamma \dstep \gamma'$ is non-smooth then
    (a) $\nsset{\gamma}{k} = \nsset{\gamma'}{k}$ for all integer $k <
    k^*$ and (b) $\nsset{\gamma'}{k^*} \subsetneq
    \nsset{\gamma}{k^*}$.
  \end{enumerate}
\end{lemma}

\subsection{Potential Function and Proof of
  Proposition~\ref{prop:dstep-potential}} 
\label{sec:dstep-potential:function}

Given a finite interval of integers $K$, and two finite sequences of
$K$-indexed finite sets $\mathcal{X} \isdef (X_k)_{k \in K}$,
$\mathcal{Y} \isdef (Y_k)_{k \in K}$ we write $\mathcal{X} =
\mathcal{Y}$ whenever $X_k = Y_k$ for all $k \in K$, and $\mathcal{X}
\prec_{setlex} \mathcal{Y}$ whenever there exists an integer $k^* \in
K$ such that $X_k = Y_k$ for all $k \in K$, $k<k^*$ and $X_{k^*}
\subsetneq Y_{k^*}$.  Note that $\prec_{setlex}$ is a well-founded
lexicographic order on the set of finite sequences of $K$-indexed
finite sets.


\subsection*{Proof of Proposition~\ref{prop:dstep-potential}}
\begin{proof}
  (a) We define $B(\gamma_0) \isdef \{ \gamma ~|~ \dbot{\gamma_0}
  \dleq \gamma \dleq \dtop{\gamma_0} \}$.  From
  Lemma~\ref{lemma:bounds:basic}(i) we obtain immediately $\gamma_0 \in
  B(\gamma_0)$.  The set $B(\gamma_0)$ is obviously closed by taking
  $par$-steps, as these steps do no change the values of $d$-variables.
  The closure of $B(\gamma_0)$ by $d$-steps can be understood by the
  $\dleq$ inequalities depicted below:

  \begin{center}
    \input{ineqchain.pdf_t}
  \end{center}
  
  Knowing $\gamma \in B(\gamma_0)$, that is, $\dbot{\gamma_0} \dleq
  \gamma \dleq \dtop{\gamma_0}$ we obtain the inequalities from the
  top line by using the idempotence and monotonicity of
  $\dbot{(.)}$, $\dtop{(.)}$ with respect to $\dleq$
  (Lemma~\ref{lemma:bounds:basic}). The same lemma ensures the
  inequalities of the bottom line.  Finally, the inequalities across
  the two lines hold because of Lemma~\ref{lemma:bounds:dstep}.  All
  over, they ensure that $\dbot{\gamma_0} \dleq \gamma' \dleq
  \dtop{\gamma_0}$ for any $d$-step $\gamma\dstep\gamma'$.
  
  (b) We define the interval of integers $K_0 \isdef [\mind
    \dbot{\gamma_0}, \maxd \dtop{\gamma_0}]$, that is, the interval of
  possible $d$-values in the configurations reachable from $\gamma_0$.
  We define the domain $D(\gamma_0) \isdef (2^\edges)^{K_0} \times
  [\sumd \dbot{\gamma_0}, \sumd \dtop{\gamma_0}]$. That is,
  $D(\gamma_0)$ consists of pairs $(\mathcal{E},s)$ where $\mathcal{E}
  : K_0 \rightarrow 2^\edges$ is a $K_0$-indexed sequence of sets of
  edges and $s$ is a bounded integer.  In particular, note that
  $D(\gamma_0)$ is finite.  We define the potential function
  $\dpot{\gamma_0} : \Gamma \rightarrow D(\gamma_0)$ by taking
  $\dpot{\gamma_0}(\gamma) \isdef ((\nsset{\gamma}{k})_{k\in K_0},
  \sumd\gamma)$.  Remark that $\dpot{\gamma_0}$ is not dependent on
  \textit{par} variables in $\gamma$.

  (c) We define the relation $\prec_d$ on $D(\gamma_0)$ by taking
  $(\mathcal{E}_1,s_1) \prec_d (\mathcal{E}_2,s_2) \isdef
  \mathcal{E}_1 \prec_{setlex} \mathcal{E}_2 \vee (\mathcal{E}_1 =
  \mathcal{E}_2 \wedge s_2 < s_1)$.  That is, $\prec_d$ is actually a
  strict lexicographic order on pairs $(\mathcal{E},s)$ which combines
  the well-founded order $\prec_{setlex}$ on finite sequences of
  finite sets and a well-founded order $<$ on bounded integers.  It
  remains to prove that $$\forall \gamma,\gamma'\in\Env, ~ \gamma \in
  B(\gamma_0) \mbox{ and } \gamma \dstep \gamma' \Rightarrow
  \dpot{\gamma_0}(\gamma') \prec_d \dpot{\gamma_0}(\gamma)$$ Let
  respectively $(\mathcal{E},s) \isdef \dpot{\gamma_0}(\gamma)$,
  $(\mathcal{E}',s') \isdef \dpot{\gamma_0}(\gamma')$. Note that from
  $\gamma \in B(\gamma_0)$ and the previous point (a) we obtain that
  $\gamma' \in B(\gamma_0)$ as well.  In particular, this ensures the
  ranks of non-smooth edges of $\gamma$, $\gamma'$ are contained in
  $K_0$ and respectively $s$, $s'$ are contained in the interval
  $[\sumd \dbot{\gamma_0}, \sumd\dtop{\gamma_0}]$.
  Lemma~\ref{lemma:dsteps} and Lemma~\ref{lemma:dsteps:smooth}($iii$)
  provide the conditions ensuring that $\dpot{\gamma_0}$ is indeed a
  decreasing potential function with respect to $\prec_d$ as expected.
  For non-smooth steps, we observe the strict inequality $\mathcal{E}'
  \prec_{setlex} \mathcal{E}$. For smooth $d$-steps we observe the
  equality ${\mathcal E} = \mathcal{E}'$ and the strict inequality $s
  < s'$. \qed
\end{proof}


