An execution is fully defined by the \Step relation, as the unfair
daemon does not add any other restriction. As a consequence, the
convergence property can be expressed by the fact that the \Step
relation is well-founded:
%
\begin{equation}\label{ass:well-founded:step}
  \WellFounded~ \Step
\end{equation}
%
This means that any execution $e \isdef \gamma_0 \xstep{\Step} \gamma_1
\xstep{\Step} ...$ is finite.  $\WellFounded$ comes from the Coq
standard library where it is expressed as $(\WellFounded~ R) \isdef
(\forall x. ~\Acc~ R~ x)$ for a given relation $R$. $\Acc$ is an
inductive predicate from the Coq standard library as well, and $(\Acc~
R~ x)$ means that every sequence of elements starting from $x$ and
linked by $R$ is finite.

In the following, we will prove the assertion
(\ref{ass:well-founded:step}).  To this end, we will consider the
partitioning of the $\Step$ relation as \( \RStep \cup \DStep \cup
\PStep \) denoting respectively \emph{root steps}, \emph{d-steps}
and \emph{par-steps} defined as follows:
\begin{itemize}
\item $\RStep$ holds for any step $\gamma \xstep{\Step} \gamma'$ in which the
  root executes \ie, such that $\gamma.\r.d \neq \gamma'.\r.d$. Note
  that any subset of non-root nodes may also execute during this step.
\item $\DStep$ holds for any step $\gamma \xstep{\Step} \gamma'$ in which the
  root does not execute and at least one non-root node executes a
  $CD$-action \ie, $\gamma.\r.d = \gamma'.\r.d$ and $\exists
  p.~ \gamma.p.d \neq \gamma'.p.d$.  Note that any subset of non-root
  nodes may also execute either Action $CD$ or $CP$.
\item $\PStep$ holds for any step $\gamma \xstep{\Step} \gamma'$ where $d$
  variables are left unchanged \ie, $\forall p.~ \gamma.p.d
  = \gamma'.p.d$. This implies that a node which executes is not the
  root and executes its $CP$-action.
\end{itemize}
In addition, we will use the following general result, (developped as
a tool in PADEC), which gives sufficient conditions ensuring the union
of two relations is well-founded. This tool has first been developped
in PADEC for algorithms with prioritized rules (as Actions $CD$ and
$CP$) and has been enhanced for this proof.
\begin{proposition}\label{prop:well-founded}
  Let $R_1, R_2$ be relations, $x$ an element. Assume that (1)
  $R_2$ is well-founded and (2) there exist a set $B_1$ and relations
  $R_1'$ well-founded, $E_1$ transitive such that
\begin{enumerate}[label=(2.\roman*)]
\item $x \in B_1$ and for all elements $a$, $b$ if $a \in B_1$ and
  $a \xstep{R_1 \cup R_2} b$ then $b \in B_1$,
\item for all elements $a$, $b$ if $a \in B_1$ and $a \xstep{R_1}
  b$ then $a \xstep{R_1'} b$,
\item for all elements $a$, $b$, $c$, if $a \xstep{R'_1} b$ and $b
  \xstep{E_1} c$ then $a \xstep{R'_1} c$,
\item for all elements $a$, $b$ if $a \xstep{R_2} b$ then
  $a \xstep{E_1} b$.
\end{enumerate}
We can conclude that $(\Acc~ (R_1 \cup R_2)~ x)$ holds.
\end{proposition}
\begin{proof}
  Intuitively, the set $B_1$ represents an over-approximation of the
  elements reachable from $x$ through $R_1 \cup R_2$
  (\textit{2.i}). The relation $R'_1$ represents an abstraction of the
  relation $R_1$ when restricted to the set $B_1$ (\textit{2.ii}).
  The relation $E_1$ can be understood as an equality with respect to
  $R_1'$ (\textit{2.iii}) which moreover abstracts the relation $R_2$
  (\textit{2.iv}).  Usually, $R_1'$ and $E_1$ can be derived from a
  potential function on the set of elements and its induced partial
  order and equality.
  
  The result is then directly obtained by considering the relation
  $<_{lex}$ defined on pairs of elements by
  $$(a,b) <_{lex} (c,d) \isdef a \xstep{R'_1} c \mbox{ or } (a
  \xstep{E_1} c \mbox{ and } b \xstep{R_2} d)$$ and the key
  observations that:
\begin{itemize}
\item $<_{lex}$ is a well-founded lexicographic order since $R'_1$ and
  $R_2$ are well-founded;
\item for all elements $a$, $b$, if $a\in B_1$ and $a \xstep{R_1 \cup
  R_2} b$ then $(b, b) <_{lex} (a, a)$. \qed
\end{itemize}
\end{proof}
As a corollary, using the same notations as in
Proposition~\ref{prop:well-founded}, being given two relations $R_1$
and $R_2$, if for every $x$ we can effectively construct the set $B_1$
and the relations $R'_1$, $E_1$ such that all assumptions are met
(\textit{1}, \textit{2.i} to \textit{2.iv}), then we can conclude that
$R_1 \cup R_2$ is well-founded.

We now proceed to the core of the convergence proof and show
progressively that $\PStep$, $\DStep \cup \PStep$ and $\RStep \cup
\DStep \cup \PStep$ are well-founded.

\begin{lemma}\label{lemma:p-steps:wf}
  $\WellFounded~ \PStep$.
\end{lemma}
\begin{proof}
  We use the potential function denoted $\#CP(\gamma)$ which counts for a given
  configuration $\gamma$ the number of nodes for which the guard of
  their $CP$-action is enabled. We say that such a node is $CP$-enabled,
  otherwise it is $CP$-disabled.

  Considering $par$-steps only \ie, the values of $d$-variables are left
  unchanged then either (i) a node is $CP$-disabled and will remain so
  or (ii) it is $CP$-enabled and if it executes, it becomes $CP$-disabled,
  else it remains $CP$-enabled.  Hence, for every two configurations
  $\gamma$ and $\gamma'$ such that $\gamma \xstep{\PStep} \gamma'$
  we have $\#CP(\gamma') < \#CP(\gamma)$, namely $\#CP$ is
  decreasing. As $\#CP$ is obviously lower-bounded by 0, this ensures that
  $\PStep$ is well-founded. \qed
\end{proof}

To proceed on the next phase of the convergence proof, we will use a
result about executions consisting of $d$-steps only.  The next
proposition guarantees the well-foundedness of the relation $\DStep$
through the existence of an effectively constructive potential
function and its ordering:
\begin{quote}
  \begin{proposition}\label{prop:dstep-potential}
    Given a configuration $\gamma_0$, we can effectively construct:
    \begin{enumerate}[label=(\alph*)]
    \item a set of configurations $B(\gamma_0)$ containing $\gamma_0$
      and closed by taking d- or par-steps,
    \item a potential function on configurations from \Env to a
      domain $D(\gamma_0)$, $\dpot{\gamma_0} :
      \Env \rightarrow D(\gamma_0)$, independent on
      \textit{par}-variables, and
    \item a well-founded order $\prec_d$ on $D(\gamma_0)$,
      such that for all $\gamma \in
      B(\gamma_0)$ and $\gamma \xstep{\DStep} \gamma'$, it holds that
      $\dpot{\gamma_0}(\gamma') \prec_d \dpot{\gamma_0}(\gamma)$.
      \end{enumerate}
  \end{proposition}
\end{quote}
The technical details and the complete proof of
Proposition~\ref{prop:dstep-potential} are presented in
Section~\ref{sec:dstep-potential}.

\begin{lemma}\label{lemma:dp-steps:wf}
  $\WellFounded~ (\DStep \cup \PStep)$
\end{lemma}
\begin{proof}
  Assuming Proposition~\ref{prop:dstep-potential} (for now), we must
  prove $(\Acc~(\DStep \cup \PStep)~\gamma_0)$ for an arbitrary
  configuration $\gamma_0$. Therefore, we use
  Proposition~\ref{prop:well-founded} by taking $R_1 \isdef \DStep$,
  $R_2 \isdef \PStep$ and $x \isdef \gamma_0$.  First,
  Lemma~\ref{lemma:p-steps:wf} ensures that $R_2$ is well-founded.
  Using the Proposition~\ref{prop:dstep-potential} above, we define
  the set $B_1$, and the relations $R_1'$ and $E_1$ as
  follows:
  %
  \[ \begin{array}{rcl} B_1 & \isdef &
    B(\gamma_0) \\ \gamma \xstep{R_1'}~\gamma' & \isdef
    & \dpot{\gamma_0}(\gamma') \prec_d \dpot{\gamma_0}(\gamma) \\
    \gamma \xstep{E_1}
    ~\gamma'& \isdef & \dpot{\gamma_0}(\gamma')
    = \dpot{\gamma_0}(\gamma) \end{array} \]
  % 
  The guaranties of Proposition~\ref{prop:dstep-potential} allow to
  fulfill the assumptions of Proposition~\ref{prop:well-founded}.
  Indeed, the relation $R_1'$ is well-founded, see
  Proposition~\ref{prop:dstep-potential}(\textit{c}).  The relation
  $E_1$ is transitive by its definition (based on equality of
  potentials).  The assumption (\textit{2.i}), that is, $B_1$ contains
  $x$ and is closed by $R_1$ or $R_2$ steps follows from
  Proposition~\ref{prop:dstep-potential}(\textit{a}).  Assumption
  (\textit{2.ii}), that is, $R_1'$ is an abstraction of $\DStep$ on
  the set $B_1$ holds because of
  Proposition~\ref{prop:dstep-potential}(\textit{c}).  Assumption
  (2.\textit{iii}) holds trivially by the construction of $R_1'$ and
  $E_1$. Last, assumption (2.\textit{iv}) holds as the potential
  function $\dpot{\gamma_0}$ is not depending on the \textit{par}
  variables, that is, remains insensitive to \textit{par}-steps.
  This proves $(\Acc~(\DStep \cup \PStep)~\gamma_0)$ and 
  finally, as the choice of $\gamma_0$ was arbitrary, we
  prove that $\WellFounded~ (\DStep \cup \PStep)$. \qed
\end{proof} 

It remains to take into account the root steps from $\RStep$.
Remind that the root $\r$ can execute at most once in any
execution: either its $d$-variable is 0 from the beginning and
$\r$ never executes; or it is positive and then $\r$ is enabled.
If it executes, the variable is set to 0 and $\r$ is then disabled
forever. The fact that $\RStep$ is well-founded is therefore trivial
to obtain.  

\begin{theorem}
  $\WellFounded~ (\RStep \cup \DStep \cup \PStep)$
\end{theorem}
\begin{proof}
  We use Proposition~\ref{prop:well-founded} by taking
  $R_1 \isdef \RStep$, $R_2 \isdef \DStep \cup \PStep$, and an
  arbitrary configuration $\gamma_0$.  First,
  Lemma~\ref{lemma:dp-steps:wf} ensures that $R_2$ is
  well-founded.  Second, we define the set $B_1 \isdef \Gamma$ and the
  relations $R_1'$ and $E_1$ as follows:
  %
  \[\begin{array}{rcl}
  \gamma \xstep{R'_1} ~\gamma' & \isdef & \gamma'.\r.d < \gamma.\r.d \\
  \gamma \xstep{E_1}  ~\gamma' & \isdef & \gamma'.\r.d = \gamma.\r.d
  \end{array}\]
  %
  Obviously, $R'_1$ is well-founded as observed above, and $E_1$ is
  transitive by definition.  Since $B_1$ contains all configurations,
  the assumption (\textit{2.i}) is trivially satisfied. Also, $R_1
  \subseteq R'_1$ holds by definition of $R_1'$, hence, it implies
  assumption (\textit{2.ii}).  Assumption (\textit{2.iii}) follows from
  definitions as well and assumption (\textit{2.iv}) holds because
  $\gamma.\r.d$ is not changing for any non-root step. \qed
\end{proof}




