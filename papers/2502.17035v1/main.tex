\documentclass[runningheads]{llncs}

\usepackage{enumitem}
\usepackage{hyperref}
\usepackage{algorithm}
\usepackage{algorithmic}
\usepackage{xspace}

\usepackage{amsmath,amssymb,dsfont}
\usepackage{tikz}

\usepackage{listings}
\lstdefinelanguage{coq}
{
  morekeywords ={Definition, Lemma, Theorem, forall, exists, Inductive,
    CoInductive, Type, Class, Hypothesis, Fixpoint, Record, if, then, else},
  sensitive=true,
  morecomment =[s]{(*}{*)},
  escapeinside={(@}{@)},
  emph={Prop}, emphstyle=\bf,
}

\lstdefinestyle{coqstyle}{
  language=coq, 
  commentstyle=\sl, 
  keywordstyle=\bf, mathescape=true,
  basicstyle=\footnotesize\tt
}
\lstset{style=coqstyle}


%%% REVIEW
\newcommand{\tocite}{{\color{red}CITE} }
\newcommand{\toref}{{\color{red}REF} }

%%% LOGO
\newcommand{\usc}{\raisebox{-1pt}{\includegraphics[height=0.8em]{figures/usc_logo.png}}}
\newcommand{\vuam}{\raisebox{-1pt}{\includegraphics[height=0.8em]{figures/vu_logo.png}}}

%%% SIGNS and SYMBOLS
\newcommand{\grad}{\texttt{grad-CROP}}
\newcommand{\att}{\texttt{att-CROP}}
\newcommand{\seg}{\texttt{seg}}
\newcommand{\clip}{\texttt{clip-CROP}}
\newcommand{\sam}{\texttt{sam-CROP}}
\newcommand{\yolo}{\texttt{yolo-CROP}}
\newcommand{\hc}{\texttt{human-CROP}}
\newcommand{\zsvqa}{\texttt{ZSVQA}}
\newcommand{\vic}{\textbf{ViCrop}}
\newcommand{\xmark}{\text{\ding{55}}}
\newcommand{\cmark}{\text{\ding{51}}}
\newcommand{\success}{\texttt{\color{green} \cmark}}
\newcommand{\failure}{\texttt{\color{red} \xmark}}
\newcommand{\rel}{\texttt{rel-att}}
\newcommand{\gra}{\texttt{grad-att}}
\newcommand{\pgra}{\texttt{pure-grad}}
\newcommand{\relh}{\texttt{rel-att$^h$}}
\newcommand{\grah}{\texttt{grad-att$^h$}}
\newcommand{\pgrah}{\texttt{pure-grad$^h$}}


%%% Text Abb.
\makeatletter
\DeclareRobustCommand\onedot{\futurelet\@let@token\@onedot}
\def\@onedot{\ifx\@let@token.\else.\null\fi\xspace}

\def\aka{\emph{a.k.a}\onedot} \def\Eg{\emph{E.g}\onedot}
\def\eg{\emph{e.g}\onedot} \def\Eg{\emph{E.g}\onedot}
\def\ie{\emph{i.e}\onedot} \def\Ie{\emph{I.e}\onedot}
\def\cf{\emph{c.f}\onedot} \def\Cf{\emph{C.f}\onedot}
\def\etc{\emph{etc}\onedot} \def\vs{\emph{vs}\onedot}
\def\wrt{w.r.t\onedot} \def\dof{d.o.f\onedot}
\def\etal{\emph{et al}\onedot}
\makeatletter



\definecolor{myred}{HTML}{FF8577}
\definecolor{mygreen}{HTML}{0FA958}
\definecolor{myblue}{HTML}{1982C4}
\definecolor{codegreen}{rgb}{0,0.5,0}
\definecolor{codegray}{rgb}{0.5,0.5,0.5}
\definecolor{codepurple}{rgb}{0.07,0,0.53}
\definecolor{codered}{RGB}{189,41,0}
\definecolor{codecomment}{RGB}{153,153,153}
\definecolor{backcolour}{rgb}{0.96,0.96,0.96}
\definecolor{royalblue}{rgb}{0.0, 0.14, 0.4}
\definecolor{egyptianblue}{rgb}{0.06, 0.2, 0.65}
\definecolor{royalazure}{rgb}{0.0, 0.22, 0.66}
\definecolor{portlandorange}{rgb}{1.0, 0.35, 0.21}
\definecolor{sienna}{RGB}{183,105,68}
\definecolor{saddlebrown}{RGB}{139,69,19}
\definecolor{mediumbrown}{RGB}{83,41,11}
\definecolor{darkbrown}{RGB}{58,28,7}
\hypersetup{
    colorlinks=true,
    linkcolor=sienna,
    urlcolor=royalblue,
    citecolor=royalblue,
}

\begin{document}

\title{Revisited Convergence of Dolev \emph{et al}’s BFS Spanning Tree
  Algorithm}

\author{Karine Altisen\orcidID{0000-0001-8344-1853} \and
  Marius Bozga\orcidID{0000-0003-4412-5684}}

\authorrunning{K. Altisen, M. Bozga}

\institute{Univ. Grenoble Alpes, CNRS, Grenoble
  INP\footnote{Institute of Engineering Univ. Grenoble Alpes},
  VERIMAG, 38000 Grenoble, France
  \email{\{Karine.Altisen,Marius.Bozga\}@univ-grenoble-alpes.fr}\\
  \url{http://www-verimag.imag.fr/}}

\maketitle

\begin{abstract}
  \begin{abstract}
Retrieval-Augmented Generation (RAG) is often used with Large Language Models (LLMs) to infuse domain knowledge or user-specific information. In RAG, given a user query, a retriever extracts chunks of relevant text from a knowledge base. These chunks are sent to an LLM as part of the input prompt. Typically, any given chunk is repeatedly retrieved across user questions. However, currently, for every question, attention-layers in LLMs fully compute the key values (KVs) repeatedly for the input chunks, as state-of-the-art methods cannot reuse KV-caches when chunks appear at arbitrary locations with arbitrary contexts. Naive reuse leads to output quality degradation.  This leads to potentially redundant computations on expensive GPUs and increases latency. In this work, we propose \sys, a system for managing and reusing precomputed KVs corresponding to the text chunks (we call \textit{chunk-caches}) in RAG-based systems. We present how to identify \hl{\textit{chunk-caches} that are reusable}, how to efficiently perform a small fraction of recomputation to \textit{fix} the cache to maintain output quality, and how to efficiently store and evict \textit{chunk-caches} in the hardware for maximizing reuse while masking any overheads. With real production workloads as well as synthetic datasets, we show that \sys reduces redundant computation by \textbf{51\%} over SOTA prefix-caching and \textbf{75\%} over full recomputation.
\hl{Additionally, with continuous batching on a real production workload, we get a \textbf{1.6$\times$} speedup in throughput and a \textbf{2$\times$} reduction in end-to-end response latency over prefix-caching while maintaining quality, for both the \llama-3-8B and \llama-3-70B models. 
}
\end{abstract}





  \keywords{spanning tree algorithm, 
    self-stabilization, constructive proof, proof
    assistant, Coq}
\end{abstract}

\section{Introduction}
\label{sec:introduction}
\section{Introduction}
\label{sec:intro}

\begin{figure*}[tb]
    \centering
    \includegraphics[width=0.848\linewidth]{figs/circuitnn.pdf} 
    \caption{Illustration of differentiable CircuitNN. CircuitNN is designed based on differentiable NAND gates. After DAS is guided by PI and PO pairs of the truth table, CircuitNN can get the precise circuit architecture logic equivalent to the truth table.}
    \label{fig:circuitnn}
\end{figure*}

% 1. Describe the importance of logic synthesis
% 2. Existing Problems
% (a) Neural Architecture Search: Unstable, Predefined Setting, etc.
% (b) Circuit Generation: Probabilistic Model, Logic Equivalence

With the rapid advancement of technology, the scale of integrated circuits (ICs) has expanded exponentially. 
This expansion has introduced significant challenges in chip manufacturing, particularly concerning power and area metrics.
A primary objective in IC design is achieving the same circuit function with fewer transistors, thereby reducing power usage and area occupancy.

Logic synthesis~\cite{hachtel2005logicsynth}, a critical step in electronic design automation (EDA), transforms behavioral-level circuit designs into optimized gate-level circuits, ultimately yielding the final IC layout. 
The primary goal of logic synthesis is to identify the physical implementation with the fewest gates for a given circuit function. 
This task constitutes a challenging NP-hard combinatorial optimization problem. 
Current logic synthesis tools~\cite{brayton2010abc, wolf2013yosys} rely on human-designed heuristics, often leading to sub-optimal outcomes.

Differentiable architecture search (DAS) techniques~\cite{liu2018darts, chu2020darts} offer novel perspectives on addressing challenges in this problem.
Circuit functions can be represented through truth tables, which map binary inputs to their corresponding outputs. 
Truth tables provide a precise representation of input-output relationships, ensuring the design of functionally equivalent circuits.
Inspired by this, researchers~\cite{deepmind2024ai4sys, wang2024tnet} have begun exploring the application of DAS to synthesize circuits directly from truth tables.
Specifically, \citet{deepmind2024ai4sys} proposed CircuitNN, a framework that learns differentiable connection structures with logic gates, enabling the automatic generation of logic circuits from truth tables.
This approach significantly reduces the complexity of traditional circuit generation. 
Building on this, \citet{wang2024tnet} introduced T-Net, a triangle-shaped variant of CircuitNN, incorporating regularization techniques to enhance the efficiency of DAS.

Despite these advancements, several challenges remain. 
The computational complexity of DAS grows quadratically with the number of gates, posing scalability issues.
Although triangle-shaped architecture~\cite{wang2024tnet} partially mitigates this problem, redundancy persists. 
%Additionally, DAS is susceptible to converging to local optima, limiting the ability to search architectures that satisfy the given truth tables~\cite{liu2018darts}. 
%Furthermore, hyperparameters (network depth and layer width) require extensive searches, introducing complexity and prolonging the synthesis process. 
Additionally, DAS is susceptible to converging to local optima~\cite{liu2018darts} and hyperparameters (network depth and layer width) require extensive searches. 
The challenges arise from the vast search space in DAS. 
% Even with predefined settings for CircuitNN, finding a configuration that meets the truth table requires extensive trial and error during the DAS process. 
Intuitively, limiting the search space through predefined parameters (network depth, gates per layer, and connection probabilities) can significantly reduce the complexity.

Recent advances~\cite{openai2023gpt4, abramson2024alphafold3, esser2024sd3, li2024mar} in conditional generative models have demonstrated remarkable performance across language, vision, and graph generation tasks. 
Motivated by these developments, we propose a novel approach to circuit generation that generates preliminary circuit structures to guide DAS in generating refined circuits matching specified truth tables. 
Firstly, we introduce CircuitVQ, a tokenizer with a discrete codebook for circuit tokenization. 
Built upon our Circuit AutoEncoder framework~\cite{hou2022graphmae,li2023maskgae,wu2025mgvga}, CircuitVQ is trained through a circuit reconstruction task. 
Specifically, the CircuitVQ encoder encodes input circuits into discrete tokens using a learnable codebook, while the decoder reconstructs the circuit adjacency matrix based on these tokens.
Subsequently, the CircuitVQ encoder serves as a circuit tokenizer for CircuitAR pretraining, which employs a masked autoregressive modeling paradigm~\cite{chang2022maskgit, li2023mage}. 
In this process, the discrete codes function as supervision signals. 
After training, CircuitAR can generate discrete tokens progressively, which can be decoded into initial circuit structures by the decoder of the CircuitVQ. 
These prior insights can guide DAS in producing refined circuits that match the target truth tables precisely.

Our key contributions can be summarized as follows:
\begin{itemize}
\item We introduce CircuitVQ, a circuit tokenizer that facilitates graph autoregressive modeling for circuit generation, based on our Circuit AutoEncoder framework;
\item Develop CircuitAR, a model trained using masked autoregressive modeling, which generates initial circuit structures conditioned on given truth tables;
\item Propose a refinement framework that integrates differentiable architecture search to produce functionally equivalent circuits guided by target truth tables;
\item Comprehensive experiments demonstrating the scalability and capability emergence of our CircuitAR and the superior performance of the proposed circuit generation approach.
\end{itemize}

% Motivation
% (a) Diffusion (Vision, Graph), Autoregressive (Language, Vision)
% (b) Circuit Generation for Predefined Setting
% (c) Neural Architecture Search for Strict Logic Equivalence

% Contribution
% (a) Circuit Tokenizer (new transformer arch, training strategy)
% (b) CircuitAR (train and gen strategies, post-ar strategy)
% (c) Extensive Evaluation including BitD (Bit Distance) for Scalability


\section{Dolev \emph{et al}'s BFS Spanning Tree Algorithm}
\label{sec:dolev}
The Dolev \emph{et al}'s BFS algorithm~\cite{DIM93} is a
self-stabilizing distributed algorithm that computes a BFS spanning
tree in an arbitrary rooted, connected, and bidirectional network.  By
``bidirectional'', we mean that each node can both transmit and
acquire information from its adjacent nodes in the network topology,
\ie, its neighbors.  The algorithm being distributed, these are the
only possible direct communications.  ``Rooted'' indicates that a
particular node, called the root and denoted by \Root, is
distinguished in the network. As in the present case, algorithms for
rooted networks are usually semi-anonymous: all nodes have the same
code except the root.

This algorithm was initially written in the Read/Write atomicity
model. We study, here, a straightforward translation into the
\emph{atomic-state model}, denoted hereafter by \BFS, and presented as
Algorithm~\ref{alg}. Notice that, as in the original presentation
\cite{DIM93} and contrarily to other adapations (see \eg,
\cite{thebook}), the variables are not assumed to be bounded.

\begin{algorithm}[htp]
  
  \textbf{Constant Local Inputs:} \hfill\
  
  \begin{tabular}{l}
    $p.\mathit{neighbors} \subseteq \channels$; $p.root \in\{true, false\}$ \\
     \emph{/* $p.\mathit{neighbors}$, as other sets below, are implemented as lists */}
  \end{tabular} \hfill\ 

\smallskip
  
  \textbf{Local Variables:} \hfill\
  
  \begin{tabular}{l}
    $p.d \in \mathds N$; $p.par \in \channels$
  \end{tabular} \hfill\ 

\smallskip
  
  \textbf{Macros:} \hfill\
  
  \begin{tabular}{l}
    $Dist_p = \min \{ q.d + 1, q \in p.\mathit{neighbors} \}$ \\
    $Par_{dist}$ returns the first channel in the list $\{ q \in p.\mathit{neighbors},
    q.d + 1 = p.d \}$
  \end{tabular} \hfill\ 

  \smallskip
  
  \textbf{Action for the root, \ie, for $p$ such that $p.root = true$} \hfill\ 

  \begin{tabular}{ll}
    Action $Root$: & \textbf{if} $p.d \neq 0$ \textbf{then} $p.d := 0$
 \end{tabular} \hfill\ 

  \smallskip
  
  \textbf{Actions for any non-root node, \ie, for $p$ such that $p.root = false$} \hfill\ 
  
  \begin{tabular}{ll}
   Action $CD$:
    & \textbf{if} $p.d \neq Dist_{p}$ \textbf{then} $p.d := Dist_{p}$ \\
     Action $CP$:
    &  \textbf{if} $p.d = Dist_p$ and $p.par.d + 1 \neq p.d$ \textbf{then} $p.par := Par_{dist}$ 
  \end{tabular}  \hfill\ 

   \caption{Algorithm \BFS, code for each node $p$.}
  \label{alg}
\end{algorithm}

In the \emph{atomic-state model}, nodes communicate through locally
shared variables: a node can read its variables and the ones of its
neighbors, but can only write to its own variables. Every node can
access the variables of its neighbors through local channels, denoted
by the set $\channels$ in Algorithm~\ref{alg}.
The network is locally defined at each node $p$ using constant local
inputs.  The fact that the network is rooted is implemented using a
constant Boolean input called $p.root$ which is false for every node
except \Root. The input $p.\mathit{neighbors}$ is the set of channels linking
$p$ to its neighbors.  When it is clear from the context, we do not
distinguish a neighbor from the channels to that neighbor.

The code of Algorithm~\ref{alg} is given as three
locally-mutually-exclusive actions written
as: \textbf{if} \emph{condition} \textbf{then} \emph{statement}. We
say that an action is \emph{enabled} when its condition is true. By
extension, a node is said to be enabled when at least one of its
actions is enabled.
According to the algorithm, the \emph{semantics of the system} defines an
execution as follows.  The system
current \emph{configuration} is given by the current value of all
variables at each node.  If no node is enabled in the current
configuration, then the configuration is said to be \emph{terminal}
and the execution is over.  Otherwise, a \emph{step} is performed:
a \emph{daemon} (an oracle that models the asynchronism of the
system) \emph{activates} a non-empty set of enabled nodes.  Each
activated node then \emph{atomically executes} the statement of its
enabled action, leading the system to a new configuration.

Assumptions can be made about the daemon. Here, we consider the most
general asynchrony assumption, namely the \emph{unfair} daemon,
meaning that it can choose any non-empty subset of the enabled nodes
for execution. In contrast, \emph{fair} daemons would guarantee
additional properties.  For example, a
\emph{strongly} (resp. \emph{weakly}) \emph{fair} daemon ensures that
every node that is enabled infinitely (resp. continuously) often is
eventually chosen for execution by the daemon.

In Algorithm \BFS, each node $p$ maintains two variables. First, it
evaluates in $p.d$ its distance to the root. Then, it maintains
 $p.par$ as a pointer to its \emph{parent} in the tree under
construction: $p.par$ is assigned to a neighbor that is closest to the
root (\nb, \Root.$par$ is meaningless).
Algorithm \BFS is a self-stabilizing BFS spanning tree construction in
the sense that, regardless the initial configuration, it makes the
system converge to a terminal configuration where $par$-variables
describe a BFS spanning tree rooted at \Root.
To that goal, nodes first compute into their $d$-variable their distance
to the root. The root simply forces the value of \Root.$d$ to be 0;
see  Action $Root$. Then, the $d$-variables of other nodes
are gradually corrected: every non-root node $p$ maintains $p.d$ to
be the minimum value of the $d$-variables of its neighbors incremented
by one; see $Dist_{p}$ and Action $CD$.
In parallel, each non-root node $p$ chooses as parent a neighbor $q$
such that $q.d = p.d-1$ when $p.d$ is locally correct \ie, $p.d =
Dist_{p}$) but $p.par$ is not correctly assigned \ie, $p.par.d$ is not
equal to $p.d-1$); see Action $CP$.




\section{The PADEC Framework}
\label{sec:padec}
PADEC~\cite{ACD7} is a general framework, written in
Coq \cite{coqart}, to develop mechanically checked proofs of
self-stabilizing algorithms.  It includes the definition of the
atomic-state model and its semantics, tools for the definition of the
algorithms and their properties, lemmas for common proof patterns, and
case studies.  Definitions in PADEC are designed to be as close as
possible to the standard usage of the self-stabilizing community.
Moreover, it is made general enough to encompass many usual hypothesis
(\eg, about topologies or daemons).

In PADEC, the finite network is described using types \Nodes
and \Channels, 
which respectively represent the nodes and the links between nodes.
The distributed algorithm is defined by providing a local
algorithm at each node. This latter is defined using a type
\States 
that represents the local state of a node
\ie, the values of its local variables and a function $\mathit{run}$
that encodes the local algorithm itself and computes a new state
depending on the current state of the node and that of its neighbors.

The model semantics defines a \emph{configuration} as a function
from \Nodes to \States that provides the local state of each node.
The type of a configuration is given by
$\Env \isdef \Nodes \rightarrow \States$.  An \emph{atomic step} of
the distributed algorithm is encoded as a binary relation over
configurations, denoted by $\Step \subseteq \Env \times \Env$, that
checks the conditions given in the informal model; see
Section~\ref{sec:dolev}.  An \emph{execution} $e$
is a finite or infinite stream of configurations, which models a
\emph{maximal} sequence of configurations where any two consecutive
configurations are linked by the $\Step$ relation.  ``Maximal'' means
that $e$ is finite if and only if its last configuration is
terminal. We use the coinductive\footnote{Coinduction allows to define
and reason about potentially infinite objects.}  type $\mathit{Exec}$
to represent an execution stream along with a coinductive predicate
$\mathit{isExec}$
to check the above condition.
Daemons are also defined as predicates over executions (in the case of
the unfair daemon, this predicate is simply equal
to $\mathit{true}$).

Self-stabilization in PADEC is defined according to the usual
practice: the property is formalized as a predicate
$(\mathit{selfStabilization} \;\; \mathit{SPEC})$
where $\mathit{SPEC}$ is a predicate over executions
and models the specification of the algorithm.  An algorithm
is \emph{self-stabilizing w.r.t. the specification}
$\mathit{SPEC}$ if there exists a set of legitimate configurations
that satisfies the following three properties in every
execution $e$:
\begin{itemize}
\item \underline{\emph{Closure}}:
  if $e$ starts in a legitimate configuration then $e$ only contains
  legitimate configurations;
\item \underline{\emph{Convergence}}:
  $e$ eventually reaches a legitimate configuration; and
\item \underline{\emph{Specification}}:
  if $e$ starts in a legitimate configuration then $e$ satisfies the
  intended specification w.r.t. $\mathit{SPEC}$.
\end{itemize}
An algorithm is said to be \emph{silent} when each of its executions
eventually reaches a terminal configuration; in such a case, the set
of legitimate configurations can be chosen as the set of terminal
configurations.  The closure, convergence, and silent properties are
expressed using Linear Time Logic operators provided in the PADEC
library.

\subsection*{The \BFS Algorithm in PADEC}

For the \BFS Algorithm and its specification, we use the formal encoding
provided in \cite{AltisenCD23}; in particular, the algorithm is a
straightforward faithful translation in Coq of
Algorithm \ref{alg}. Notably, an element of \States,
namely a state of a given node, is a tuple
$(d, \mathit{par}, \mathit{root}, \mathit{neighbors})$ representing
the variables of the node as in Algorithm~\ref{alg}.

As the constant variables $\mathit{root}$ and $\mathit{neighbors}$
represent the network, the assumptions that this network is rooted,
bidirected and connected is encoded in a predicate on a configuration
using only those variables. This predicate, in particular uses the set
of edges of the network $\Edges \isdef \{ (p, q) \;|\; p,
q \in \Nodes \;\wedge\; (p \in q.\mathit{neighbors} \;\vee\; q \in
p.\mathit{neighbors}) \}$. Globally in this precidate, the neighbor
links represent a bidirected connected graph and the Boolean
$\mathit{root}$ should be true for a unique node.
We will assume moreover that this predicate holds for any
configuration, even if this is no more mentionned in the sequel.

In \cite{AltisenCD23}, the \BFS Algorithm was proven using PADEC to be
self-stabilizing and silent for the specification of a BFS spanning
tree, \emph{under the assumption of a weakly fair daemon}.  We extend
here this result to the \emph{unfair daemon}.  Note that, since \BFS
is silent, the properties of closure and specification still hold,
henceforth, relaxing the assumption from a weakly fair to an unfair
daemon is trivial.  The only missing property is the convergence.  The
rest of the paper is therefore focusing on proving the convergence of
the \BFS Algorithm under an unfair daemon in PADEC, \ie, providing a
constructive proof under the form of a potential function and its
corresponding order.


\section{Overview of the Proof}
\label{sec:proof}

\subsection{Proof for Satisfaction of Marginal Constraints.}
% In this section, we will first show that our procedure satisfying the marginal conditions for our coupling $q(\rvx_0, \rvx_1)$:
% \begin{equation}
%     \int q(\rvx_0, \rvx_1) d\rvx_1 = q_0(\rvx_0), \int q(\rvx_0, \rvx_1) d\rvx_0 = q_1(\rvx_1).
% \end{equation}

% \begin{itemize}
%     \item For independent couple $q(x_0) = \int q(\mathcal{S}) \int q(x_1 | \mathcal{S}) q(x_0) dx_0 d_\mathcal{S}$ and $q(x_1) = \int q(\mathcal{S}) \int q(x_1 | \mathcal{S}) q(x_0) dx_1 d_\mathcal{S}$, we just need to show $\int q(x_0, x_1 | \mathcal{S}) dx_0 = q(x_1 | \mathcal{S})$ and $\int q(x_0, x_1 | \mathcal{S}) dx_1 = q(x_0)$.
%     \item $q(x_0, x_1 | \mathcal{S})$ is independent, so we can decompose it into $\prod q(x_0^i, x_1^i | \mathcal{S})$.
%     \item we can show $\int q(x_0^i, x_1^i | \mathcal{S}) dx_0 = q(x_1^i | \mathcal{S})$ and $\int q(x_0^i, x_1^i | \mathcal{S}) dx_0 = q(x_1^i)$
%     \item $q(x_0| \mathcal{S})$ and $q(x_1| \mathcal{S})$ are independent, so we can decompose it into $\prod q(x_1^i | \mathcal{S})$ and $\prod q(x_0^i)$.
%     \item The first part is done.
%     \item The second part is to show adding noise will not affect $q(x_0^i)$

% \end{itemize}

% In particular, the proof will be divided into four parts.
% %
% First, we will introduce the main theorem to apply to obtain the results, and show the random subsampling of a Dense Gaussian noise will converge to Gaussian distribution if the sample superset is large enough.
% %
% Second, by a proper construction, we can show that subsampling of a dense point superset can converge to direct subsampling of the surfaces when the size of the superset is also large enough.
% %
% Third, by considering our random subsampling procedure, we can show that our sampling is still random subsampling for Gaussian noise superset and point superset.
% %
% Lastly, we show that even introduction of the coupling interpolation, the results mariginal remain the same due to careful considerations.

\newtheorem{proposition}{Proposition}
\newtheorem{lemma}{Lemma}
\subsubsection{Law of Large Numbers}


\begin{proposition}\label{prop:large_samples}
Given $(X_1, \cdots, X_n)$, which are independently and identically distributed (IID) real $d$-diemsnion random variables, following a probability distribution $p(X)$,~\ie, $X_i \sim p(X), X \in \mathbb{R}^d$.
%
We have an additional random variable $Y$ that is random uniform sample of these variables,~\ie, $P(Y = X_i) = \frac{1}{n}$.
%
The cumulative distribution function (CDF) $\bar{F}(t)$ of random variable $Y$ will converge to the $F(X)$,~\ie, CDF of $X$.
\end{proposition}



% Assume $(X_1, \cdots, X_n)$ are independently and identically distributed (IID) real $d$-diemsnion random variables following a probability distribution $p(X)$, \ie, $X_i  \sim p(X), X \in \mathbb{R}^d$.
% %
% We also denote the cumulative distribution function of $p(X)$ to be $F(x)$.
%
Proof:
We first define an empirical cumulative distribution function $\hat{F}_n(X)$ over the random variables $(X_1, \cdots, X_n)$:
\begin{equation}
    \hat{F}_n (t) = \frac{1}{n} \sum_{i=1}^{n} \mathbf{1}_{X_i \leq t},
\end{equation}
where $\mathbf{1}_{X_i \leq t}$ is an indicator for $X_i^d \leq t^d$ for all dimensions $\{1, \cdots, d\}$.

The Glivenko–Cantelli theorem states that this empirical distribution function $\hat{F}_n(X)$ will converge to the cumulative distribution $F(X)$ if $n$ is sufficiently large:
\begin{equation}
    \textbf{sup}_{t \in \mathbb{R}^d} | \hat{F}_n(t) - F(t) | \rightarrow 0.
\end{equation}

If we have an additional random variable $Y$ that its value is a random subsample of the variables $(X_1, \cdots, X_n)$:
\begin{equation}
    P(Y = X_i) = \frac{1}{n}, \forall i = 1, 2, \cdots, n.
\end{equation}

The CDF of this variable $\bar{F}(t)$ is:
\begin{equation}
    \bar{F}(t) = P(Y \leq t) = \sum_{i=1}^{n} P(Y = X_i) \cdot \mathbf{1}_{X_i \leq t} = \frac{1}{n} \sum_{i=1}^{n} \mathbf{1}_{X_i \leq t} = \hat{F}_n(t).
\end{equation}
Therefore, the CDF of $Y$ also converges to the original underlying CDF $F(t)$ if $n$ is sufficiently large.

\begin{proposition}\label{prop:ot}
Assume we have $n$ random samples $(X_1, \cdots, X_n) \sim p_1$, and another $n$ random samples $(Y_1, \cdots, Y_n) \sim p_2$, and we are also given an arbitrary bijective map between random variables, \ie, $\Pi: \{1, \cdots, n\} \leftrightarrow \{1, \cdots, n\}$.
%
If we construct a new random variable $Z : (X, Y)$ follows the following couplings:

\[
    P(X = X_i, Y = Y_j) =
    \begin{cases}
    \frac{1}{n}, & \text{if } j = \Pi(i) \\ 
        0, & \text{else } j \neq \Pi(i);
    \end{cases}
\]

The CDF of the marginal $P(X)$ will converge the CDF of $p_1$, while the CDF of the marginal $P(Y)$ will converge to the CDF of $p_2$.
\end{proposition}

Proof:
Since $\Pi$ is bijective, we can compute the marginal $P(X = X_i)$ directly:
\begin{equation}
    \begin{split}
            P(X = X_i) = \sum_{j=1}^{n} P(X = X_i, Y = Y_j) \\
            = P(X = X_i, Y = Y_{\Pi(i)}) + \sum_{j \neq \Pi(i)} P(X = X_i, Y = Y_j) \\
            = \frac{1}{n} + 0 = \frac{1}{n}
    \end{split}
\end{equation}

Similarly, we can show the marginal of P(Y) is also $\frac{1}{n}$.
%
By leveraging Proposition~\ref{prop:large_samples}, we show that $P(X)$ will converge the CDF of $p_1$, and the CDF of $P(Y)$ will converge to the CDF of $p_2$.

% \begin{lemma}\label{lemma:independent}
% The Gaussian noises $x_0$ are independently and identically distributed (IID), \ie, $q_0(x_0) = \prod_{i}^N q_0(x_0^i)$, where $x^i_0$ is the $i$-th noises and $x^i_0 \sim q_0$ .
% %
% Also, the point cloud $x_1$ given a 3D shape $S$ is also independently and identically distributed (IID), \ie, $q_1(x_1|S) = \prod_{i}^N q_1(x_1^i | S)$, where $x^i_1$ is the $i$-th point and $x^i_0 \sim q_{1|S}$.
% %
% Lastly, the training pair $(x_0, x_1)$ from our coupling  given a shape $S$ is also independently and identically distributed (IID), \ie, $q(x_0, x_1 | S) = \prod_{i}^N q(x_0^i, x_1^i | S)$, where $(x_0^i, x_1^i$) is the $i$-th pair in the training pair.
% \end{lemma}

% \begin{lemma}\label{lemma:joint}
%     The sample distribution of a point $x_1^i$ involves modeling of underlying shape $S$ and the modeling of the point distribution given $S$, \ie, $q_1(x_1^i) = \int q_1(x_1^i | S) q(S) dS$.
%     %
%     However, the distribution of noises $q_0(x^i_0)$ is unrelated to a given shape $S$, \ie, $q_0(x^i_0 | S) = q_0(x^i_0)$.
% \end{lemma}

% By considering the $p(X)$ be a Gaussian distribution $N(0, I)$ or a sampling distribution of 3D points given a Shape $\mathcal{S}$, \ie, $q(x|\mathcal{S})$, we can show the random sample $Y$ still follows the original distribution.

% If we consider $M$ random variables, where each of them is an 3D Gaussian noise, denoted as $\epsilon_i \sim N(0, I), \epsilon_i \in \mathbb{R}^3$.
% %
% We also define another variable $\epsilon$ is a random sample of these random variables, \ie, $P(\epsilon = \epsilon_i) = \frac{1}{M}$.
% %
% Since each dimension in $\epsilon$ is independent, CDF of $\epsilon^j$ will follows the by leverage the above results, where $j$ is the j-th dimension of the noise.
% We consider a dense 3D Gaussian noises with $M \times 3$ random variables, $\{x_1, y_1, z_1, \cdots, x_M, y_M, z_M\}$, where we denote $x_i$, $y_i$, and $z_i$ to be the coordinates of in x, y, and z dimensions, respectively and $x_i, y_i, z_i \sim N(0, I)$.
% %
% If we can consider a random variable $\hat{x}$, which is random sample of this dense gausian in x dimension, \ie,  P$(\hat{x} = x_i) = \frac{1}{M}$.
% %
% By the above results, the CDF follows the original distribution, which is the Gaussian distribution $N(0, I)$.
% %
% By considering also y and z dimension, we can show that a random sampling of Gaussian point converge to Gaussian distribution.
\newtheorem{theorem}{Theorem}
\subsubsection{Proof of Our OT Approximation}
\label{subsec:our_ot_proof}

We first give a definition of coupling $q(x_0, x_1)$ in our case before showing its marginal fullfils the marginal requirements.
%
In particular, we denote $x_0 \in R^{N \times 3}$ and $x_1 \in R^{N \times 3}$ as two random variables following the distributions, $q_0(x_0)$ and $q_1(x_1)$, respectively.
%
It is noted that $q_0 := N(0, I)$, which is the standard Gaussian for each dimension in $x_0$, and $q_1(x_1)$ is the distribution all possible point clouds, which involves the joint modeling of point cloud distribution given a shape $S$ ($q_{1}(x_1|S)$) and the distribution of shape ($q(S)$), \ie, $q_1(x_1) = \int q_{1}(x_1|S) q(S) dS$.
%

We can notice that each row in $x_0$ is independently and identically distributed (IID), \ie, $q_0(x_0) = \prod_{i}^N \hat{q_0}(x_0^i)$, where we denote the $i$-th row of $x_0$ as $x_0^i$ and distribution of $x_0^i$ as $\hat{q_0}(x_0^i)$, which is 3-dimensional unit Gaussian.
%
We can also assume each point in $x_1$ is IID given a shape, \ie, $q_{1}(x_1 | S) = \prod_{i}^N \hat{q_{1}}(x_1^i|S)$,  where we denote the $i$-th row of $x_1$ as $x^i_1$ and the distribution of $x^i_1$ as $\hat{q_{1}}(x_1^i|S)$. 

In our superset OT precomputation for a given shape $S$, we pre-sample a set of random variables $(x^1_0 \cdots, x^j_0, \cdots, x^M_0) \sim \hat{q}_0$, and a set of random variables  $(x^1_1, \cdots, x^k_1,\cdots, x^M_1) \sim \hat{q}_1$, and have a precomputed bijective mapping $\Pi : \{1, \cdots, M\} \leftrightarrow \{1, \cdots, M\}$.
%
With these defined, our coupling $\hat{q}(x^i_0, x^i_1 |S)$ for one row in the training pair $(x^i_0, x^i_1)$ given $S$ can be formulated as:
\[
    \hat{q}(x^i_0 = x^j_0, x^i_1 = x^k_1 | S) =
    \begin{cases}
    \frac{1}{n}, & \text{if } j = \Pi(k) \\ 
        0, & \text{else } j \neq \Pi(k);
    \end{cases}
\]
%
Since the each row in the training pairs are independently subsampled, the coupling of the training pair $(x_0, x_1)$ given a shape is defined as $q(x_0, x_1 |S) = \prod_{i}^N \hat{q}(x_0^i, x_1^i | S)$.
%
In the end, the coupling over all training pairs can be obtained by marginalize over all possible shapes, \ie, $\int q(x_0, x_1 | S) q(S) dS$.

\begin{theorem}

% Our coupling $q(x_0, x_1)$ for a given Gaussian noise $x_0 \in R^{N \times 3}$ and a given point cloud $x_1 \in R^{N \times 3}$

Our coupling without blending converge the following marginal if the superset size $M$ is sufficiently large:
\begin{equation}\label{eq:mariginals}
    \int q(\rvx_0, \rvx_1) d\rvx_1 = q_0(\rvx_0), \int q(\rvx_0, \rvx_1) d\rvx_0 = q_1(\rvx_1).
\end{equation}
\end{theorem}

Proof:
We first show the left constraint:
% \begin{equation}
\begin{align}
LHS & = \int q(x_0, x_1) dx_1 = \int \int q(x_0, x_1 | S) q(S) dS dx_1  \\
& = \int q(S) \int q(x_0, x_1 | S) dx_1 dS && \text{change the order of integration} \\
& = \int q(S) \int \prod_i^N \hat{q}(x_0^i, x_1^i|S) d(x_1^1, \cdots, x_1^N) dS  && \text{independent assumption of each row in training pair}\\
& = \int q(S) \prod_i^N \int \hat{q}(x_0^i, x_1^i|S) dx_1^j dS && \text{integrals of independent products}\\
& = \int q(S) \prod_i \sum_k^M \hat{q}(x_0^i, x_1^k|S) dS && \text{restricting to discrete values in supersets}\\
& = \int q(S) \prod_i \hat{q}_0(x^i_0) dS && \text{Proposition~\ref{prop:ot}}\\
& = \int q(S) q_0(x_0) dS = q_0(x_0) && \text{independent assumption of each row in Gaussian noises} \\
\end{align}
% \end{equation}

Similarly, we perform the same computation on the right constraint:
% \begin{equation}
\begin{align}
LHS & = \int q(x_0, x_1) dx_0 = \int \int q(x_0, x_1 | S) q(S) dS dx_0   \\
 & = \int q(S) \int q(x_0, x_1 | S) dx_0 dS && \text{change the order of integration} \\
& = \int q(S) \int \prod_i^N \hat{q}(x_0^i, x_1^i|S) d(x_0^1, \cdots, x_0^N) dS && \text{independent assumption of each row in training pair} \\
& = \int q(S) \prod_i^N \int \hat{q}(x_0^i, x_1^i|S) dx_0^i dS  && \text{integrals of independent products} \\
& = \int q(S) \prod_i \sum_j^M \hat{q}(x_0^j, x_1^i|S) dS 
 && \text{restricting to discrete values in supersets} \\
& = \int q(S) \prod_i \hat{q}_1(x^i_1 | S) dS  && \text{Proposition~\ref{prop:ot}} \\
& = \int q(S) q_1(x_1 | S) dS = q_1(x_1) && \text{independent assumption of each row in point cloud} \\
\end{align}
% \end{equation}


% We first consider the RHS of Left Constraints (Equation~\ref{eq:mariginals}), we can reformulate it as follows:
% \begin{equation}
%     \begin{split}
%             RHS = q_0(x_0) = \int q(S) q_0(x_0 | S) dS = \int q(S) q_0(x_0) dS \\
%             % = \int q_0(x_0) (\int q_1(x_1 |S) q(S) dS) dx_1 \text{, by Lemma~\ref{lemma:joint}} \\
%             % = \int q(S) \int q_0(x_0) q_1(x_1|S) dx_1 d_S \text{, by rearranging the integrals} \\
%     \end{split}
% \end{equation}
% Considering LHS:
% \begin{equation}
%     \begin{split}
%         LHS = \int q(x_0, x_1) dx_1 = \int \int q(x_0, x_1 | S) q(S) dS dx_1 \\
%         = \int q(S) \int q(x_0, x_1 | S) dx_1 dS
%     \end{split}
% \end{equation}

% By comparing LHS and RHS, it is sufficient to show that $\int q(x_0, x_1 |S) dx_1 = q_0(x_0)$ for the first constraint.
% Similarly, for the second constraint RHS:
% \begin{equation}
%     \begin{split}
%             RHS = q_1(x_1) = \int q(S) q_1(x_1|S) dS \\
%             % = \int q_0(x_0) (\int q_1(x_1 |S) q(S) dS) dx_1 \text{, by Lemma~\ref{lemma:joint}} \\
%             % = \int q(S) \int q_0(x_0) q_1(x_1|S) dx_1 d_S \text{, by rearranging the integrals} \\
%     \end{split}
% \end{equation}
% Considering LHS:
% \begin{equation}
%     \begin{split}
%         LHS = \int q(x_0, x_1) dx_0 = \int \int q(x_0, x_1 | S) q(S) dS dx_0 \\
%         = \int q(S) \int q(x_0, x_1 | S) dx_0 dS
%     \end{split}
% \end{equation}
% Then it is sufficient to show $\int q(x_0, x_1 | S) dx_0 = q_1(x_1|S) $.

% To show first equation, we can apply Lemma~\ref{lemma:independent}:
% \begin{equation}
% \label{eq:left_LHS}
%     \begin{split}
%         LHS = \int q(x_0, x_1 | S) dx_1 = \int \prod_i q(x_0^i, x_1^i | S) d(x_1^i, \cdots, x_1^N) \\
%         = \prod_i \int q(x_0^i, x_1^i|S) dx_1^i 
%     \end{split}
% \end{equation}

% \begin{equation}
% \label{eq:left_RHS}
%     RHS = q_0(x_0) = \prod_i q_0(x^i_0)
% \end{equation}
% By this computation, we are also sufficient to show $\int q(x_0^i, x_1^i | S) dx_1^i = q_0(x_0^i)$ and by similar computation:
% \begin{equation}
% \label{eq:right_LHS}
%     \begin{split}
%         LHS = \int q(x_0, x_1 | S) dx_0 = \int \prod_i q(x_0^i, x_1^i | S) d(x_0^i, \cdots, x_0^N) \\
%         = \prod_i \int q(x_0^i, x_1^i|S) dx_0^i 
%     \end{split}
% \end{equation}

% \begin{equation}
% \label{eq:right_RHS}
%     RHS = q_1(x_0|S) = \prod_i q_1(x^i_1|S)
% \end{equation}
% Therefore, we are sufficient to show $\int q(x^i_0, x^i_1) dx_0^i = q_1(x_1^i |S)$.

% By considering the fact that, we pre-sample a set of random variables $(x^1_0 \cdots, x^j_0, \cdots, x^M_0) \sim q_0$, and a set of random variables  $(x^1_1, \cdots, x^k_1,\cdots, x^M_1) \sim q_{1|S}$, and have a precomputed bijective mapping $\Pi : \{1, \cdots, M\} \leftrightarrow \{1, \cdots, M\}$.
% %
% With these defined, our coupling $q(x^i_0, x^i_1 |S)$ given $S$ can formulated as:
% \[
%     P(x^i_0 = x^j_0, x^i_1 = x^k_1) =
%     \begin{cases}
%     \frac{1}{n}, & \text{if } k = \Pi(j) \\ 
%         0, & \text{else } k \neq \Pi(j);
%     \end{cases}
% \]
% By Proposition~\ref{prop:ot}, if the superset size $M$ is large enough, we can show that the CDF of Equation~\ref{eq:left_LHS} converge to Equation~\ref{eq:left_RHS}, also the CDF of Equation~\ref{eq:left_LHS} converges to Equation~\ref{eq:left_RHS}.

% To show our coupling maintain the correct marginal, we assume we have $M$ random variables $(X_1, \cdots, X_M) \sim p_1$, and another $M$ random random variables $(Y_1, \cdots, Y_M) \sim p_2$.
% %
% We can additionally take an arbitrary bijective map $\Pi$ between random variables, \ie, $\Pi : \{1, \cdots, M\} \leftrightarrow \{1, \cdots, M\}$.
% %
% If we only sample the a pair of variables based on the bijective map, we can then construct a new random Variable $Z: \{X, Y\}$:
% \[
%     P(X = X_i, Y = Y_j) =
%     \begin{cases}
%     \frac{1}{M}, & \text{if } j = \Pi(i) \\ 
%         0, & \text{else } j \neq \Pi(i);
%     \end{cases}
% \]

% Since $\Pi$ is a bijective mapping, the mariginal distribution of $P(X = X_i)$ and $P(Y = Y_j)$ is also $\frac{1}{M}$.
% %
% Following the result in the previous section, we can show the random variable $X$ ($Y$) still follows $p_1$ ($p_2$).
% %
% In our case, we consider $p_1$ to be a 3D Gaussian distribution, and $p_2$ to be point sample distribution given a Shape $\mathcal{S}$.

% The last part we need to show is that $q_0(x_0)$ and $q_1(x_1|\mathcal{S})$ is independently sampled for each of the point, \ie, $q_0(x_0) = \prod_{i} q_0(x_0^i)$ and \ie, $q_1(x_1) = \prod_{i} q_1(x_1^i | \mathcal{S})$, where $x_0^i$ and $x_1^i$ is the $i$-th point in $x_0$ and $x_1$, respectively.
% %
% For Gaussian distribution $q_0(x_0)$, this is true because it is an unit Gaussian distribution.
% %
% For surface point distribution $q_1(x_1|S)$, it is also correct since the points are indepdently sampled.



% Additionally, for a Gaussian noise sets arranged in the matrix format, $x_0 \in \mathbb{R}^{N \times 3}, x_0 \sim$
\subsubsection{Proof of Hybrid Coupling}

In the last, we would like to show even with our hybrid coupling, the marginal still fulfills the requirements.
%
In particular, we define a new noises $x_0'$ after perturbation:
\begin{equation}
    x_0' = \sqrt{1 - \beta} x_0 + \sqrt{\beta} \epsilon, \epsilon \sim N(\epsilon; 0, I),
\end{equation}
where $\beta \in [0, 1]$ is the blending coefficient. We denoted this as a conditional distribution $q(x_0'| x_0)$, which has a form of $N(x_0'; \sqrt{1 - \beta}x_0, \beta)$.
%
It is noted that since $\epsilon \sim N(\epsilon, 0, I)$, each row of $x'_0$ is also IID given $x_0$, \ie, $q_0(x_0' | x_0) = \prod_i^N \hat{q}_0(x_0^{'i} | x_0^i)$.
%
Due to the independent properties, it is sufficient to show that:
\begin{equation}
    \int q(x_0^{i'}, x_1^i | S) dx_0^{i'} = q_1(x^i_1|S), 
    \int q(x_0^{i'}, x_1^i | S) dx_1^{i} = q_0(x_0^i).
\end{equation}

For the sake of simplicity, we remove all the index $i$ and shape $S$ in the folloings.
We first show the left constraint:
\iffalse
\begin{align}
    q(x_1) & = \int q(x_0', x_1) dx_0' = \int \int q_0(x_0) q(x_0'| x_0) q(x_1|x_0, x_0') dx_0 dx_0' \\
    & = \int \int q_0(x_0) q(x_0'| x_0) q(x_1|x_0) dx_0 dx_0' \\
    & =  \int \int  q_0(x_0) q(x_0'| x_0) q(x_1|x_0)  dx_0' dx_0 \\
    & = \int q_0(x_0) q(x_1|x_0) \int  q(x_0'| x_0)  dx_0' dx_0 \\
    & = \int q_0(x_0) q(x_1|x_0) (1) dx_0 \\
    & = \int q(x_0, x_1) dx_0  = \frac{1}{M} \\
\end{align}
\fi
\begin{align}
    \int q(x_0', x_1) dx_0' & = \int \int q_0(x_0', x_0, x_1) dx_0 dx_0' \\
    & = \int \int q_0(x_0'|x_0) q(x_0, x_1) dx_0 dx_0' \\
    & =  \int q(x_0, x_1) \int  q_0(x_0'|x_0)  dx_0' dx_0 \\
    & = \int q(x_0, x_1) (1) dx_0 \\
    & = q(x_1)
\end{align}
By Proposition~\ref{prop:large_samples}, we can show $q(x_1)$ still converge to the right CDF if $M$ is sufficient large.

On the other hand, we show that:
\iffalse
\begin{align}
    \int q(x_0', x_1) dx_1 &= \int \int q_1(x_0' | x_0, x_1) q_0(x_0|x_1) q(x_1) dx_0 dx_1 \\
    &= \int \int q(x_0' | x_0, x_1) q(x_0|x_1) q(x_1) dx_1 dx_0 \\
    &= \int \int q(x_0' | x_0) q(x_0|x_1) q(x_1) dx_1 dx_0 \\
    & = \int q(x_0'|x_0) \int q(x_0|x_1) q(x_1) dx_1 dx_0 \\
    & = \int q(x_0'|x_0) \sum_{x_1} q(x_0, x_1) dx_0 \\
    & = \int q(x_0'|x_0) \frac{1}{M} dx_0 \\
    & = \frac{1}{M} \sum_{x_0} q(x_0' | x_0) \\
    & = \frac{1}{M} \sum_{x_0} N(x_0'; \sqrt{1 - \beta} x_0, \beta)
\end{align}
\fi
\begin{align}
    \int q(x_0', x_1) dx_1 &= \int \int q_0(x_0', x_0, x_1) dx_0 dx_1 \\
    &= \int \int q_0(x_0', x_0) dx_0\\
    &= \int \int  q_0(x_0'|x_0) q(x_0) dx_0 \\
    & = N(0, I)
\end{align}
where the last equality is obtained by inserting $q(x_0) = N(0, I)$ and $q_0(x_0'|x_0) = N(x_0'; \sqrt{1 - \beta}x_0, \beta I)$.

\iffalse
When $M \rightarrow \infty$, it becomes a convolution of two Gaussian $N(0, (1 - \beta) I)$ and $N(0, \beta I)$.
%
By convolution of Gaussian, we can observe that:
\begin{align}
    \int q(x_0', x_1) dx_1 & = N(0, (1 - \beta)I + \beta I) \\
    & = N(0, I)\\
\end{align}
\fi

\section{A Decreasing Potential Function for $d$-steps}
\label{sec:dstep-potential}

This section is concerned with the proof of
Proposition~\ref{prop:dstep-potential} stated in
Section~\ref{sec:proof}. To this end, we proceed in three steps.
First, we establish a finite over-approximation on the set of the $d$
values that could be possibly reached in an execution involving
$d$-steps only from some initial configuration $\gamma_0$.
Second, we introduce a
partitioning of edges (being either smooth or non-smooth) and prove
some preservation properties along $d$-steps.
Third, we combine the above results to effectively construct a
potential function for $d$-steps and a well-founded order on the
co-domain of this function, ultimately proving
Proposition~\ref{prop:dstep-potential}.

For the sake of readability, we denote $d$-steps $\gamma \xstep{\DStep}
\gamma'$ shortly by $\gamma \dstep \gamma'$.

\subsection{Bounds on Distance Values}
\label{sec:dstep-potential:bounds}

For a configuration $\gamma$, we define the integers $\maxd\gamma
\isdef \max \{\gamma.q.d \mid q \in \nodes\}$, $\mind\gamma \isdef
\min \{\gamma.q.d \mid q \in \nodes\}$, $\sumd\gamma \isdef \sum
\{\gamma.q.d \mid q \in \nodes\}$\footnote{The sum is taken on the
multiset of $d$ values}.  We also define $\dbot{\gamma}$,
$\dtop{\gamma}$ respectively a \emph{bottom} and a \emph{top}
configuration associated to $\gamma$.  These are identical to $\gamma$
except for $d$ values, defined for every node $p$ as follows:
\begin{eqnarray*}
  \dbot{\gamma}.p.d & \isdef & \mind\gamma \\
  \dtop{\gamma}.p.d & \isdef & \left\{ \begin{array}{ll}
    \gamma.p.d & \mbox{if } p = r \\
    \max \{ \gamma.p.d, 1 + \min \{ \dtop{\gamma}.q.d \mid \\
    \hspace{1cm} (p,q)\in \Edges, \dist{p}{\r} = 1 + \dist{q}{\r} \}
    \} &
    \mbox{otherwise,}
  \end{array} \right.
\end{eqnarray*}
where $\dist{q}{\r}$ represents the distance of some node $q$ to the
root \r.  Note that the recursive definition of $\dtop{\gamma}.p.d$ is
well-defined as the recursion is limited to neighbours $q$ of $p$
located at a smaller distance to the root $\r$ than $p$.  Intuitively,
the maximal $d$ value of a non-root node $p$ in some configuration
reachable from $\gamma$ is either its value in $\gamma$ (\ie, it can
be the case when $p$ does not execute) or 1 plus the minimum of the
maximal $d$ values of its neighbors $q$ closer to the root
(see Action $CD$ when $p$ executes).
We define the partial order $\dleq$ on configurations by taking
%
$$ \gamma_1 \dleq \gamma_2 \isdef \forall q \in \nodes: \gamma_1.q.d
\le \gamma_2.q.d $$
%
The next lemma states basic properties of the $\dbot{(.)}$,
$\dtop{(.)}$ operators, namely their idempotence and
their monotonicity with respect to $\dleq$.  The proof follows from
definitions and uses induction on nodes according to their
distance to the root.

\begin{lemma} \label{lemma:bounds:basic} ~
  
  \begin{enumerate}[label=(\roman*)]
  \item For all configuration $\gamma$,
  $\dbot{\gamma} \dleq \gamma \dleq \dtop{\gamma}$,
  $\dbot{(\dbot{\gamma})} = \dbot{\gamma}$ and $\dtop{(\dtop{\gamma})}
  = \dtop{\gamma}$.

  \item For all configurations $\gamma_1$ and $\gamma_2$ such that
  $\gamma_1 \dleq \gamma_2$, $\dbot{\gamma_1} \dleq \dbot{\gamma_2}$ and
  $\dtop{\gamma_1} \dleq \dtop{\gamma_2}$.

\end{enumerate}
\end{lemma}

The next lemma relates the bottom and top configurations to $d$-steps.
The proof is done by induction respectively, on the set of nodes
according to their distance to the root (i) and on the length of an
execution sequence from $\gamma_0$ (ii).\footnote{$\gamma_0 \dstepstar
\gamma$ means that $\gamma$ is reachable from $\gamma_0$ using a
finite number of $d$-steps.}
\begin{lemma}\label{lemma:bounds:dstep} ~
  
  \begin{enumerate}[label=(\roman*)]  
  \item For all configurations $\gamma$ and $\gamma'$ such that
  $\gamma \dstep \gamma'$, $\dbot{\gamma} \dleq \dbot{\gamma'}$ and
  $\dtop{\gamma'} \dleq \dtop{\gamma}$.

  \item For all configurations $\gamma_0$ and $\gamma$ such that
  $\gamma_0 \dstepstar \gamma$,
  $\dbot{\gamma_0} \dleq \gamma \dleq \dtop{\gamma_0}$.

  \end{enumerate}
\end{lemma}

\subsection{Smooth and Non-smooth $d$-steps}
\label{sec:dstep-potential:smooth}

We say that an edge $(p,q)\in\Edges$ is \emph{smooth}
(resp. \emph{non-smooth}) in a configuration $\gamma\in\Env$ if the
difference (in absolute value, $abs$) between the $d$-values at its
endpoints $p$, $q$ is at most 1 (resp. at least 2).  Formally,
consider the predicate
%
$$\csmooth{\gamma}{(p,q)} \isdef (abs(\gamma.p.d - \gamma.q.d) \le 1).$$
%
We say that a $d$-step $\gamma \dstep \gamma'$ is \emph{smooth} if all the
nodes $p$ changing their values from $\gamma$ to $\gamma'$ are
connected to smooth edges only in $\gamma$, formally:
%
$$\begin{array}{l}
  \ssmooth{\gamma \dstep \gamma'} \isdef \\
  \hspace{1cm} \forall p \in \Nodes: (\gamma'.p.d \not= \gamma.p.d) \Rightarrow
  (\forall q \in p.neighbors: \csmooth{\gamma}{(p,q)}
\end{array}$$
%
We define the rank of an edge $(p,q)\in\Edges$ in a configuration
$\gamma\in\Env$ as
$\crank{\gamma}{(p,q)} \isdef \min(\gamma.p.d, \gamma.q.d)$.



\begin{figure}[th]
  \centering
  \scalebox{0.9}{\input{d-steps.pdf_t}}
  \caption{\label{fig:d-steps}Smooth and non-smooth steps}
\end{figure}

For illustration, consider the three configurations $\gamma_1$,
$\gamma_2$, $\gamma_3$ depicted in Fig.~\ref{fig:d-steps}.  We
represented the $d$ values of the nodes by their positioning on the
horizontal lines e.g., $\gamma_1.\r.d = 10$, $\gamma_1.p_1.d = 9$,
$\gamma_2.p_1.d = 10$, etc.  Edges are represented by lines
connecting nodes: smooth (resp. non-smooth) edges are depicted in
blue (resp. red).  Configuration $\gamma_2$ is the successor of
$\gamma_1$ by a smooth step.  That is, only $p_1$ and $p_6$ have
executed and these nodes were connected only to smooth (blue) edges
in $\gamma_1$.  Configuration $\gamma_3$ is the successor of
$\gamma_2$ by a non-smooth step.  That is, $p_3$ and $p_4$ have been
executed along the step, and these nodes were connected to some
non-smoth edges.

The next lemmas provide key properties for understanding the execution
of $d$-steps, depending if they are smooths or not.
Lemma~\ref{lemma:dsteps:smooth} basically states that partitioning
between smooth and non-smooth, as well as the rank of every non-smooth
edge is preserved by smooth steps.  In addition, the total sum of $d$
values is increasing along such a step.  

\begin{lemma}\label{lemma:dsteps:smooth}
  Consider a smooth d-step $\gamma \dstep \gamma'$.  Then,
  \begin{enumerate}[label=(\roman*)]
  \item $\forall e \in \edges: \neg \csmooth{\gamma}{e} \Leftrightarrow
    \neg \csmooth{\gamma'}{e}$,
  \item $\forall e \in \edges: \neg \csmooth{\gamma}{e} \Rightarrow
    (\crank{\gamma}{e} = \crank{\gamma'}{e})$,
  \item $\sumd \gamma' > \sumd \gamma$.
  \end{enumerate}
\end{lemma}
\begin{proof}
  The proof follows immediately from the definition of smooth steps
  and/or edges.  First, the fact that non-smooth edges are preserved
  along with their rank in a smooth $d$-step directly comes from the
  definition of a smooth step: since no node connected to a non-smooth
  edge can execute, non-smooth edges remained unchanged.
  Second, we obtain the increasing of the sum of all $d$-values by
  observing that when a node executes in a smooth $d$-step, its $d$
  value increases by one or two (due to its neighbors which are either
  above by one or at the same level of $d$ value). As a smooth
  $d$-step involves at least one such an executing node, $\sumd$
  necessarily increases (since nodes that do not increase $d$ leave it
  unchanged).
  \qed
\end{proof}

For illustration, consider the smooth step depicted in
Fig.~\ref{fig:d-steps}, \ie, between $\gamma_1$ and $\gamma_2$.  It is
rather trivial that, as long as the nodes executing were connected to
smooth edges only (in blue), their execution has no impact on the
non-smooth edges \ie, they remain non-smooth and preserve their rank.
Yet, the overall sum of the $d$ values increases, here because at
least the values of the two moving nodes has increased (by 1 for $p_1$
and by 2 for $p_6$).

Lemma~\ref{lemma:dsteps:nonsmooth} provides a similar preservation
result for non-smooth steps.  In this case, the key property is that
one can effectively compute a bound $k^*$ such that (i) all non-smooth
edges with rank lower than $k^*$ remain non-smooth and preserve their
rank and (ii) the set of non-smooth edges with rank $k^*$ is strictly
decreasing along the step.  The lemma provides both the explicit
definition of $k^*$ as well as the identification of a non-smooth edge
at level $k^*$ which either becomes smooth or gets a reduced rank
after the step, that is, some edge $(p,q)$ for which the minimum is
achieved in the definition of $k^*$.

\begin{lemma}\label{lemma:dsteps:nonsmooth} Consider a non-smooth d-step $\gamma \dstep \gamma'$.   Let
  $$\begin{array}{l} k^* \isdef \min \{ \crank{\gamma}{(p,q)} \mid (p, q) \in \edges:
    \neg \csmooth{\gamma}{(p,q)}, \\
    \hspace{5cm} \gamma'.p.d \not=\gamma.p.d \mbox{ or } \gamma'.q.d \not= \gamma.q.d \}
    \end{array}$$
  Then,
  \begin{enumerate}[label=(\roman*)]
  \item $\forall e \in \edges: (\crank{\gamma'}{e} \le k^* \wedge \neg \csmooth{\gamma'}{e}) \Rightarrow \\
    \hspace*{3cm} (\crank{\gamma}{e} = \crank{\gamma'}{e} \wedge \neg \csmooth{\gamma}{e})$,
  \item $\forall e \in \edges: (\crank{\gamma}{e} < k^* \wedge \neg \csmooth{\gamma}{e}) \Rightarrow \\
    \hspace*{3cm} (\crank{\gamma'}{e} = \crank{\gamma}{e} \wedge \neg \csmooth{\gamma'}{e})$,
  \item $\exists e \in \edges: (\crank{\gamma}{e} = k^* \wedge \neg \csmooth{\gamma}{e}) \wedge \\
    \hspace*{3cm} (\neg \csmooth{\gamma'}{e} \Rightarrow \crank{\gamma'}{e} > \crank{\gamma}{e}))$.
  \end{enumerate}
\end{lemma}
\begin{proof}
  (i) The proof is done by case splitting, considering which endpoints
  of non-smooth edges $e$ execute. In fact, the only feasible case is
  when none of them executes.  In all other cases, by choosing the
  node which gives a new value to its $d$ variable, we obtain a
  contradiction, either with the minimality of $k^*$ or with the
  non-smoothness of $e$ in $\gamma'$.

  (ii) By definition of $k^*$, no node involved in a non-smooth edge
  can execute if the rank is below $k^*$, hence rank and
  non-smoothness are left unchanged.

  (iii) Note here that, using Coq, to be able to prove "$\exists e \in
  \edges: ...$", we have to effectively contruct such an edge. In our
  case, it is chosen as some of the edges which achieves the minimum
  rank value when computing $k^*$: a non-smooth edge $e^*$ such that
  $\crank{\gamma}{e^*} = k^*$, and one of its end nodes executes
  during the step (it exists and can be computed using the computation
  of the minimum value over a finite set). Now, consider the case
  where $e^*$ remains non-smooth in $\gamma'$. We note $e^*=(p, q)$
  with $\crank{\gamma}{(p, q)} = \gamma.p.d$. We can prove that if $p$
  executes then $\gamma'.p.d > \gamma.p.d$ and that if $q$ executes
  then $\gamma'.q.d = \gamma.p.d + 1$ (see Fig.~\ref{fig:dsteps:proof}
  for an illustration). The result is then easy to conclude.
  \qed
\end{proof}

\begin{figure}[th]
  \begin{center}
    \scalebox{0.9}{\input{d-steps-proof.pdf_t}}
  \end{center}
  \caption{\label{fig:dsteps:proof}Possible evolutions of a non-smooth edge $e^*=(p,q)$ with
    minimal rank $k^*$: (i) only $p$ executes, (ii) only $q$ executes
    (iii) $p$ and $q$ executes}
\end{figure}

For illustration also, consider the non-smooth step depicted in
Fig.~\ref{fig:d-steps} between $\gamma_2$ and $\gamma_3$.  In this
case, the bound value is $k^* = 8$.  The lemma ensures that the set of
non-smooth edges of rank strictly lower than $8$ are unchanged.  No
such edges actually exist in the configurations $\gamma_2$ or
$\gamma_3$. But, actually, it is not hard to imagine that if such
edges would exist and are not related to $p_3$ and $p_6$, they would
not be impacted by the move.  Also, the lemma guarantees that the set
of edges at level 8 is strictly decreasing.  That is, the set of
non-smooth edges at level 8 is $\{ (p_3,p_7), (p_3,p_4) \}$ in
$\gamma_2$, respectively $\emptyset$ in $\gamma_3$.

Finally, we define $\nsset{\gamma}{k} \isdef \{ e \in \edges \mid \neg
\csmooth{\gamma}{e} \;\wedge\; \crank{\gamma}{e} = k \}$, that is, the
set of non-smooth edges of rang $k$ in $\gamma$.  The next lemma
simply re-formulates the results of Lemma \ref{lemma:dsteps:smooth} in
point (i) and Lemma \ref{lemma:dsteps:nonsmooth} in point (ii) into a
single statement about the sets $\nsset{\gamma}{k}$ to
facilitate their use in the definition of the potential function in
the next subsection.

\begin{lemma}\label{lemma:dsteps}
  Consider a d-step $\gamma \dstep \gamma'$.  Then
  \begin{enumerate}[label=(\roman*)]
  \item if the step $\gamma \dstep \gamma'$ is smooth then
    $\nsset{\gamma}{k} = \nsset{\gamma'}{k}$ for all integer $k$,
  \item if the step $\gamma \dstep \gamma'$ is non-smooth then
    (a) $\nsset{\gamma}{k} = \nsset{\gamma'}{k}$ for all integer $k <
    k^*$ and (b) $\nsset{\gamma'}{k^*} \subsetneq
    \nsset{\gamma}{k^*}$.
  \end{enumerate}
\end{lemma}

\subsection{Potential Function and Proof of
  Proposition~\ref{prop:dstep-potential}} 
\label{sec:dstep-potential:function}

Given a finite interval of integers $K$, and two finite sequences of
$K$-indexed finite sets $\mathcal{X} \isdef (X_k)_{k \in K}$,
$\mathcal{Y} \isdef (Y_k)_{k \in K}$ we write $\mathcal{X} =
\mathcal{Y}$ whenever $X_k = Y_k$ for all $k \in K$, and $\mathcal{X}
\prec_{setlex} \mathcal{Y}$ whenever there exists an integer $k^* \in
K$ such that $X_k = Y_k$ for all $k \in K$, $k<k^*$ and $X_{k^*}
\subsetneq Y_{k^*}$.  Note that $\prec_{setlex}$ is a well-founded
lexicographic order on the set of finite sequences of $K$-indexed
finite sets.


\subsection*{Proof of Proposition~\ref{prop:dstep-potential}}
\begin{proof}
  (a) We define $B(\gamma_0) \isdef \{ \gamma ~|~ \dbot{\gamma_0}
  \dleq \gamma \dleq \dtop{\gamma_0} \}$.  From
  Lemma~\ref{lemma:bounds:basic}(i) we obtain immediately $\gamma_0 \in
  B(\gamma_0)$.  The set $B(\gamma_0)$ is obviously closed by taking
  $par$-steps, as these steps do no change the values of $d$-variables.
  The closure of $B(\gamma_0)$ by $d$-steps can be understood by the
  $\dleq$ inequalities depicted below:

  \begin{center}
    \input{ineqchain.pdf_t}
  \end{center}
  
  Knowing $\gamma \in B(\gamma_0)$, that is, $\dbot{\gamma_0} \dleq
  \gamma \dleq \dtop{\gamma_0}$ we obtain the inequalities from the
  top line by using the idempotence and monotonicity of
  $\dbot{(.)}$, $\dtop{(.)}$ with respect to $\dleq$
  (Lemma~\ref{lemma:bounds:basic}). The same lemma ensures the
  inequalities of the bottom line.  Finally, the inequalities across
  the two lines hold because of Lemma~\ref{lemma:bounds:dstep}.  All
  over, they ensure that $\dbot{\gamma_0} \dleq \gamma' \dleq
  \dtop{\gamma_0}$ for any $d$-step $\gamma\dstep\gamma'$.
  
  (b) We define the interval of integers $K_0 \isdef [\mind
    \dbot{\gamma_0}, \maxd \dtop{\gamma_0}]$, that is, the interval of
  possible $d$-values in the configurations reachable from $\gamma_0$.
  We define the domain $D(\gamma_0) \isdef (2^\edges)^{K_0} \times
  [\sumd \dbot{\gamma_0}, \sumd \dtop{\gamma_0}]$. That is,
  $D(\gamma_0)$ consists of pairs $(\mathcal{E},s)$ where $\mathcal{E}
  : K_0 \rightarrow 2^\edges$ is a $K_0$-indexed sequence of sets of
  edges and $s$ is a bounded integer.  In particular, note that
  $D(\gamma_0)$ is finite.  We define the potential function
  $\dpot{\gamma_0} : \Gamma \rightarrow D(\gamma_0)$ by taking
  $\dpot{\gamma_0}(\gamma) \isdef ((\nsset{\gamma}{k})_{k\in K_0},
  \sumd\gamma)$.  Remark that $\dpot{\gamma_0}$ is not dependent on
  \textit{par} variables in $\gamma$.

  (c) We define the relation $\prec_d$ on $D(\gamma_0)$ by taking
  $(\mathcal{E}_1,s_1) \prec_d (\mathcal{E}_2,s_2) \isdef
  \mathcal{E}_1 \prec_{setlex} \mathcal{E}_2 \vee (\mathcal{E}_1 =
  \mathcal{E}_2 \wedge s_2 < s_1)$.  That is, $\prec_d$ is actually a
  strict lexicographic order on pairs $(\mathcal{E},s)$ which combines
  the well-founded order $\prec_{setlex}$ on finite sequences of
  finite sets and a well-founded order $<$ on bounded integers.  It
  remains to prove that $$\forall \gamma,\gamma'\in\Env, ~ \gamma \in
  B(\gamma_0) \mbox{ and } \gamma \dstep \gamma' \Rightarrow
  \dpot{\gamma_0}(\gamma') \prec_d \dpot{\gamma_0}(\gamma)$$ Let
  respectively $(\mathcal{E},s) \isdef \dpot{\gamma_0}(\gamma)$,
  $(\mathcal{E}',s') \isdef \dpot{\gamma_0}(\gamma')$. Note that from
  $\gamma \in B(\gamma_0)$ and the previous point (a) we obtain that
  $\gamma' \in B(\gamma_0)$ as well.  In particular, this ensures the
  ranks of non-smooth edges of $\gamma$, $\gamma'$ are contained in
  $K_0$ and respectively $s$, $s'$ are contained in the interval
  $[\sumd \dbot{\gamma_0}, \sumd\dtop{\gamma_0}]$.
  Lemma~\ref{lemma:dsteps} and Lemma~\ref{lemma:dsteps:smooth}($iii$)
  provide the conditions ensuring that $\dpot{\gamma_0}$ is indeed a
  decreasing potential function with respect to $\prec_d$ as expected.
  For non-smooth steps, we observe the strict inequality $\mathcal{E}'
  \prec_{setlex} \mathcal{E}$. For smooth $d$-steps we observe the
  equality ${\mathcal E} = \mathcal{E}'$ and the strict inequality $s
  < s'$. \qed
\end{proof}




\section{Conclusion}
\label{sec:conclusion}
\section*{Conclusion}
This paper aims to enhance our understanding of the computational complexity of computing various Shapley value variants. We found that for various ML models --- including decision trees, regression tree ensembles, weighted automata, and linear regression --- both local and global interventional and baseline SHAP can be computed in polynomial time under HMM modeled distributions. This extends popular algorithms, such as TreeSHAP, beyond their empirical distributional scope. We also establish strict complexity gaps between the various SHAP variants (baseline, interventional, and conditional) and prove the intractability of computing SHAP for tree ensembles and neural networks in simplified scenarios. Overall, we present SHAP as a versatile framework whose complexity depends on four key factors: \begin{inparaenum}[(i)] \item model type, \item SHAP variant, \item distribution modeling approach, \item and local vs. global explanations\end{inparaenum}. We believe this perspective provides deeper insight into the computational complexity of SHAP, paving the way for future work.




%We believe that our framework provides a more intricate understanding of SHAP computation complexity across different models, distributions, and variants, paving the way for further research.

Our work opens promising directions for future research. First, expanding our computational analysis to other SHAP-related metrics, such as asymmetric SHAP~\citep{frye20} and SAGE~\citep{covert2020understanding}, would be valuable. Additionally, we aim to explore more expressive distribution classes and relaxed assumptions beyond those in Section \ref{sec:tractable} while maintaining tractable SHAP computation. Finally, when exact computation is intractable (Section \ref{sec:intractable}), investigating the approximability of SHAP metrics through approximation and parameterized complexity theory~\citep{downey2012parameterized} is an important direction.

%Our work opens several promising avenues for future research on the computational properties of explainable AI methods, with a particular focus on SHAP. First, it would be interesting to broaden the computational analysis conducted in this work to include other popular SHAP-related metrics in the literature, such as asymmetric SHAP \cite{frye20} and SAGE \cite{covert2020understanding}. Also, in the future, we aim to explore more expressive distribution classes and relaxed distributional assumptions—extending beyond those examined in Section \ref{sec:tractable} —that still yield tractable SHAP computation. Finally, when exact computation proves intractable (Section \ref{sec:intractable}), it is worthwhile to theoretically investigate the question of the approximability of computing the SHAP metrics across various configurations, through the lens of approximation and parametrized complexity theory \cite{arora2009computational}.

%This paper aims to deepen our understanding of the computational complexity involved in obtaining different Shapley value variants. We found that for a variety of ML models, including decision trees, tree ensembles for regression, weighted automata, and linear regression models — computing both local and global interventional and baseline SHAP can be done in polynomial time when distributions are modeled by HMMs. This extends the distributional scope of popular algorithms like TreeSHAP, which is limited to empirical distributions. Additionally, we demonstrate a strict complexity gap between SHAP variants, showing that interventional and baseline SHAP can be strictly easier to compute than conditional SHAP. Despite these positive results, we uncovered intractability for various SHAP variants in neural networks and tree ensembles. Finally, we provided generalized complexity relations across SHAP variants. We believe that our framework offers a deeper understanding of the complexity involved in computing SHAP across various variants, models, distributions, as well as in both local and global computations, laying the groundwork for future research.

\bibliographystyle{splncs04}
\bibliography{biblio}

\end{document}

\typeout{get arXiv to do 4 passes: Label(s) may have changed. Rerun}
