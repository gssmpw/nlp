\documentclass[runningheads]{llncs}

\usepackage{enumitem}
\usepackage{hyperref}
\usepackage{algorithm}
\usepackage{algorithmic}
\usepackage{xspace}

\usepackage{amsmath,amssymb,dsfont}
\usepackage{tikz}

\usepackage{listings}
\lstdefinelanguage{coq}
{
  morekeywords ={Definition, Lemma, Theorem, forall, exists, Inductive,
    CoInductive, Type, Class, Hypothesis, Fixpoint, Record, if, then, else},
  sensitive=true,
  morecomment =[s]{(*}{*)},
  escapeinside={(@}{@)},
  emph={Prop}, emphstyle=\bf,
}

\lstdefinestyle{coqstyle}{
  language=coq, 
  commentstyle=\sl, 
  keywordstyle=\bf, mathescape=true,
  basicstyle=\footnotesize\tt
}
\lstset{style=coqstyle}


%%%%%%%%%%%---SETME-----%%%%%%%%%%%%%
%replace @@ with the submission number submission site.
\newcommand{\thiswork}{INF$^2$\xspace}
%%%%%%%%%%%%%%%%%%%%%%%%%%%%%%%%%%%%


%\newcommand{\rev}[1]{{\color{olivegreen}#1}}
\newcommand{\rev}[1]{{#1}}


\newcommand{\JL}[1]{{\color{cyan}[\textbf{\sc JLee}: \textit{#1}]}}
\newcommand{\JW}[1]{{\color{orange}[\textbf{\sc JJung}: \textit{#1}]}}
\newcommand{\JY}[1]{{\color{blue(ncs)}[\textbf{\sc JSong}: \textit{#1}]}}
\newcommand{\HS}[1]{{\color{magenta}[\textbf{\sc HJang}: \textit{#1}]}}
\newcommand{\CS}[1]{{\color{navy}[\textbf{\sc CShin}: \textit{#1}]}}
\newcommand{\SN}[1]{{\color{olive}[\textbf{\sc SNoh}: \textit{#1}]}}

%\def\final{}   % uncomment this for the submission version
\ifdefined\final
\renewcommand{\JL}[1]{}
\renewcommand{\JW}[1]{}
\renewcommand{\JY}[1]{}
\renewcommand{\HS}[1]{}
\renewcommand{\CS}[1]{}
\renewcommand{\SN}[1]{}
\fi

%%% Notion for baseline approaches %%% 
\newcommand{\baseline}{offloading-based batched inference\xspace}
\newcommand{\Baseline}{Offloading-based batched inference\xspace}


\newcommand{\ans}{attention-near storage\xspace}
\newcommand{\Ans}{Attention-near storage\xspace}
\newcommand{\ANS}{Attention-Near Storage\xspace}

\newcommand{\wb}{delayed KV cache writeback\xspace}
\newcommand{\Wb}{Delayed KV cache writeback\xspace}
\newcommand{\WB}{Delayed KV Cache Writeback\xspace}

\newcommand{\xcache}{X-cache\xspace}
\newcommand{\XCACHE}{X-Cache\xspace}


%%% Notions for our methods %%%
\newcommand{\schemea}{\textbf{Expanding supported maximum sequence length with optimized performance}\xspace}
\newcommand{\Schemea}{\textbf{Expanding supported maximum sequence length with optimized performance}\xspace}

\newcommand{\schemeb}{\textbf{Optimizing the storage device performance}\xspace}
\newcommand{\Schemeb}{\textbf{Optimizing the storage device performance}\xspace}

\newcommand{\schemec}{\textbf{Orthogonally supporting Compression Techniques}\xspace}
\newcommand{\Schemec}{\textbf{Orthogonally supporting Compression Techniques}\xspace}



% Circular numbers
\usepackage{tikz}
\newcommand*\circled[1]{\tikz[baseline=(char.base)]{
            \node[shape=circle,draw,inner sep=0.4pt] (char) {#1};}}

\newcommand*\bcircled[1]{\tikz[baseline=(char.base)]{
            \node[shape=circle,draw,inner sep=0.4pt, fill=black, text=white] (char) {#1};}}

\begin{document}

\title{Revisited Convergence of Dolev \emph{et al}’s BFS Spanning Tree
  Algorithm}

\author{Karine Altisen\orcidID{0000-0001-8344-1853} \and
  Marius Bozga\orcidID{0000-0003-4412-5684}}

\authorrunning{K. Altisen, M. Bozga}

\institute{Univ. Grenoble Alpes, CNRS, Grenoble
  INP\footnote{Institute of Engineering Univ. Grenoble Alpes},
  VERIMAG, 38000 Grenoble, France
  \email{\{Karine.Altisen,Marius.Bozga\}@univ-grenoble-alpes.fr}\\
  \url{http://www-verimag.imag.fr/}}

\maketitle

\begin{abstract}
  \begin{abstract}  
Test time scaling is currently one of the most active research areas that shows promise after training time scaling has reached its limits.
Deep-thinking (DT) models are a class of recurrent models that can perform easy-to-hard generalization by assigning more compute to harder test samples.
However, due to their inability to determine the complexity of a test sample, DT models have to use a large amount of computation for both easy and hard test samples.
Excessive test time computation is wasteful and can cause the ``overthinking'' problem where more test time computation leads to worse results.
In this paper, we introduce a test time training method for determining the optimal amount of computation needed for each sample during test time.
We also propose Conv-LiGRU, a novel recurrent architecture for efficient and robust visual reasoning. 
Extensive experiments demonstrate that Conv-LiGRU is more stable than DT, effectively mitigates the ``overthinking'' phenomenon, and achieves superior accuracy.
\end{abstract}  
  \keywords{spanning tree algorithm, 
    self-stabilization, constructive proof, proof
    assistant, Coq}
\end{abstract}

\section{Introduction}
\label{sec:introduction}
\section{Introduction}


\begin{figure}[t]
\centering
\includegraphics[width=0.6\columnwidth]{figures/evaluation_desiderata_V5.pdf}
\vspace{-0.5cm}
\caption{\systemName is a platform for conducting realistic evaluations of code LLMs, collecting human preferences of coding models with real users, real tasks, and in realistic environments, aimed at addressing the limitations of existing evaluations.
}
\label{fig:motivation}
\end{figure}

\begin{figure*}[t]
\centering
\includegraphics[width=\textwidth]{figures/system_design_v2.png}
\caption{We introduce \systemName, a VSCode extension to collect human preferences of code directly in a developer's IDE. \systemName enables developers to use code completions from various models. The system comprises a) the interface in the user's IDE which presents paired completions to users (left), b) a sampling strategy that picks model pairs to reduce latency (right, top), and c) a prompting scheme that allows diverse LLMs to perform code completions with high fidelity.
Users can select between the top completion (green box) using \texttt{tab} or the bottom completion (blue box) using \texttt{shift+tab}.}
\label{fig:overview}
\end{figure*}

As model capabilities improve, large language models (LLMs) are increasingly integrated into user environments and workflows.
For example, software developers code with AI in integrated developer environments (IDEs)~\citep{peng2023impact}, doctors rely on notes generated through ambient listening~\citep{oberst2024science}, and lawyers consider case evidence identified by electronic discovery systems~\citep{yang2024beyond}.
Increasing deployment of models in productivity tools demands evaluation that more closely reflects real-world circumstances~\citep{hutchinson2022evaluation, saxon2024benchmarks, kapoor2024ai}.
While newer benchmarks and live platforms incorporate human feedback to capture real-world usage, they almost exclusively focus on evaluating LLMs in chat conversations~\citep{zheng2023judging,dubois2023alpacafarm,chiang2024chatbot, kirk2024the}.
Model evaluation must move beyond chat-based interactions and into specialized user environments.



 

In this work, we focus on evaluating LLM-based coding assistants. 
Despite the popularity of these tools---millions of developers use Github Copilot~\citep{Copilot}---existing
evaluations of the coding capabilities of new models exhibit multiple limitations (Figure~\ref{fig:motivation}, bottom).
Traditional ML benchmarks evaluate LLM capabilities by measuring how well a model can complete static, interview-style coding tasks~\citep{chen2021evaluating,austin2021program,jain2024livecodebench, white2024livebench} and lack \emph{real users}. 
User studies recruit real users to evaluate the effectiveness of LLMs as coding assistants, but are often limited to simple programming tasks as opposed to \emph{real tasks}~\citep{vaithilingam2022expectation,ross2023programmer, mozannar2024realhumaneval}.
Recent efforts to collect human feedback such as Chatbot Arena~\citep{chiang2024chatbot} are still removed from a \emph{realistic environment}, resulting in users and data that deviate from typical software development processes.
We introduce \systemName to address these limitations (Figure~\ref{fig:motivation}, top), and we describe our three main contributions below.


\textbf{We deploy \systemName in-the-wild to collect human preferences on code.} 
\systemName is a Visual Studio Code extension, collecting preferences directly in a developer's IDE within their actual workflow (Figure~\ref{fig:overview}).
\systemName provides developers with code completions, akin to the type of support provided by Github Copilot~\citep{Copilot}. 
Over the past 3 months, \systemName has served over~\completions suggestions from 10 state-of-the-art LLMs, 
gathering \sampleCount~votes from \userCount~users.
To collect user preferences,
\systemName presents a novel interface that shows users paired code completions from two different LLMs, which are determined based on a sampling strategy that aims to 
mitigate latency while preserving coverage across model comparisons.
Additionally, we devise a prompting scheme that allows a diverse set of models to perform code completions with high fidelity.
See Section~\ref{sec:system} and Section~\ref{sec:deployment} for details about system design and deployment respectively.



\textbf{We construct a leaderboard of user preferences and find notable differences from existing static benchmarks and human preference leaderboards.}
In general, we observe that smaller models seem to overperform in static benchmarks compared to our leaderboard, while performance among larger models is mixed (Section~\ref{sec:leaderboard_calculation}).
We attribute these differences to the fact that \systemName is exposed to users and tasks that differ drastically from code evaluations in the past. 
Our data spans 103 programming languages and 24 natural languages as well as a variety of real-world applications and code structures, while static benchmarks tend to focus on a specific programming and natural language and task (e.g. coding competition problems).
Additionally, while all of \systemName interactions contain code contexts and the majority involve infilling tasks, a much smaller fraction of Chatbot Arena's coding tasks contain code context, with infilling tasks appearing even more rarely. 
We analyze our data in depth in Section~\ref{subsec:comparison}.



\textbf{We derive new insights into user preferences of code by analyzing \systemName's diverse and distinct data distribution.}
We compare user preferences across different stratifications of input data (e.g., common versus rare languages) and observe which affect observed preferences most (Section~\ref{sec:analysis}).
For example, while user preferences stay relatively consistent across various programming languages, they differ drastically between different task categories (e.g. frontend/backend versus algorithm design).
We also observe variations in user preference due to different features related to code structure 
(e.g., context length and completion patterns).
We open-source \systemName and release a curated subset of code contexts.
Altogether, our results highlight the necessity of model evaluation in realistic and domain-specific settings.






\section{Dolev \emph{et al}'s BFS Spanning Tree Algorithm}
\label{sec:dolev}
The Dolev \emph{et al}'s BFS algorithm~\cite{DIM93} is a
self-stabilizing distributed algorithm that computes a BFS spanning
tree in an arbitrary rooted, connected, and bidirectional network.  By
``bidirectional'', we mean that each node can both transmit and
acquire information from its adjacent nodes in the network topology,
\ie, its neighbors.  The algorithm being distributed, these are the
only possible direct communications.  ``Rooted'' indicates that a
particular node, called the root and denoted by \Root, is
distinguished in the network. As in the present case, algorithms for
rooted networks are usually semi-anonymous: all nodes have the same
code except the root.

This algorithm was initially written in the Read/Write atomicity
model. We study, here, a straightforward translation into the
\emph{atomic-state model}, denoted hereafter by \BFS, and presented as
Algorithm~\ref{alg}. Notice that, as in the original presentation
\cite{DIM93} and contrarily to other adapations (see \eg,
\cite{thebook}), the variables are not assumed to be bounded.

\begin{algorithm}[htp]
  
  \textbf{Constant Local Inputs:} \hfill\
  
  \begin{tabular}{l}
    $p.\mathit{neighbors} \subseteq \channels$; $p.root \in\{true, false\}$ \\
     \emph{/* $p.\mathit{neighbors}$, as other sets below, are implemented as lists */}
  \end{tabular} \hfill\ 

\smallskip
  
  \textbf{Local Variables:} \hfill\
  
  \begin{tabular}{l}
    $p.d \in \mathds N$; $p.par \in \channels$
  \end{tabular} \hfill\ 

\smallskip
  
  \textbf{Macros:} \hfill\
  
  \begin{tabular}{l}
    $Dist_p = \min \{ q.d + 1, q \in p.\mathit{neighbors} \}$ \\
    $Par_{dist}$ returns the first channel in the list $\{ q \in p.\mathit{neighbors},
    q.d + 1 = p.d \}$
  \end{tabular} \hfill\ 

  \smallskip
  
  \textbf{Action for the root, \ie, for $p$ such that $p.root = true$} \hfill\ 

  \begin{tabular}{ll}
    Action $Root$: & \textbf{if} $p.d \neq 0$ \textbf{then} $p.d := 0$
 \end{tabular} \hfill\ 

  \smallskip
  
  \textbf{Actions for any non-root node, \ie, for $p$ such that $p.root = false$} \hfill\ 
  
  \begin{tabular}{ll}
   Action $CD$:
    & \textbf{if} $p.d \neq Dist_{p}$ \textbf{then} $p.d := Dist_{p}$ \\
     Action $CP$:
    &  \textbf{if} $p.d = Dist_p$ and $p.par.d + 1 \neq p.d$ \textbf{then} $p.par := Par_{dist}$ 
  \end{tabular}  \hfill\ 

   \caption{Algorithm \BFS, code for each node $p$.}
  \label{alg}
\end{algorithm}

In the \emph{atomic-state model}, nodes communicate through locally
shared variables: a node can read its variables and the ones of its
neighbors, but can only write to its own variables. Every node can
access the variables of its neighbors through local channels, denoted
by the set $\channels$ in Algorithm~\ref{alg}.
The network is locally defined at each node $p$ using constant local
inputs.  The fact that the network is rooted is implemented using a
constant Boolean input called $p.root$ which is false for every node
except \Root. The input $p.\mathit{neighbors}$ is the set of channels linking
$p$ to its neighbors.  When it is clear from the context, we do not
distinguish a neighbor from the channels to that neighbor.

The code of Algorithm~\ref{alg} is given as three
locally-mutually-exclusive actions written
as: \textbf{if} \emph{condition} \textbf{then} \emph{statement}. We
say that an action is \emph{enabled} when its condition is true. By
extension, a node is said to be enabled when at least one of its
actions is enabled.
According to the algorithm, the \emph{semantics of the system} defines an
execution as follows.  The system
current \emph{configuration} is given by the current value of all
variables at each node.  If no node is enabled in the current
configuration, then the configuration is said to be \emph{terminal}
and the execution is over.  Otherwise, a \emph{step} is performed:
a \emph{daemon} (an oracle that models the asynchronism of the
system) \emph{activates} a non-empty set of enabled nodes.  Each
activated node then \emph{atomically executes} the statement of its
enabled action, leading the system to a new configuration.

Assumptions can be made about the daemon. Here, we consider the most
general asynchrony assumption, namely the \emph{unfair} daemon,
meaning that it can choose any non-empty subset of the enabled nodes
for execution. In contrast, \emph{fair} daemons would guarantee
additional properties.  For example, a
\emph{strongly} (resp. \emph{weakly}) \emph{fair} daemon ensures that
every node that is enabled infinitely (resp. continuously) often is
eventually chosen for execution by the daemon.

In Algorithm \BFS, each node $p$ maintains two variables. First, it
evaluates in $p.d$ its distance to the root. Then, it maintains
 $p.par$ as a pointer to its \emph{parent} in the tree under
construction: $p.par$ is assigned to a neighbor that is closest to the
root (\nb, \Root.$par$ is meaningless).
Algorithm \BFS is a self-stabilizing BFS spanning tree construction in
the sense that, regardless the initial configuration, it makes the
system converge to a terminal configuration where $par$-variables
describe a BFS spanning tree rooted at \Root.
To that goal, nodes first compute into their $d$-variable their distance
to the root. The root simply forces the value of \Root.$d$ to be 0;
see  Action $Root$. Then, the $d$-variables of other nodes
are gradually corrected: every non-root node $p$ maintains $p.d$ to
be the minimum value of the $d$-variables of its neighbors incremented
by one; see $Dist_{p}$ and Action $CD$.
In parallel, each non-root node $p$ chooses as parent a neighbor $q$
such that $q.d = p.d-1$ when $p.d$ is locally correct \ie, $p.d =
Dist_{p}$) but $p.par$ is not correctly assigned \ie, $p.par.d$ is not
equal to $p.d-1$); see Action $CP$.




\section{The PADEC Framework}
\label{sec:padec}
PADEC~\cite{ACD7} is a general framework, written in
Coq \cite{coqart}, to develop mechanically checked proofs of
self-stabilizing algorithms.  It includes the definition of the
atomic-state model and its semantics, tools for the definition of the
algorithms and their properties, lemmas for common proof patterns, and
case studies.  Definitions in PADEC are designed to be as close as
possible to the standard usage of the self-stabilizing community.
Moreover, it is made general enough to encompass many usual hypothesis
(\eg, about topologies or daemons).

In PADEC, the finite network is described using types \Nodes
and \Channels, 
which respectively represent the nodes and the links between nodes.
The distributed algorithm is defined by providing a local
algorithm at each node. This latter is defined using a type
\States 
that represents the local state of a node
\ie, the values of its local variables and a function $\mathit{run}$
that encodes the local algorithm itself and computes a new state
depending on the current state of the node and that of its neighbors.

The model semantics defines a \emph{configuration} as a function
from \Nodes to \States that provides the local state of each node.
The type of a configuration is given by
$\Env \isdef \Nodes \rightarrow \States$.  An \emph{atomic step} of
the distributed algorithm is encoded as a binary relation over
configurations, denoted by $\Step \subseteq \Env \times \Env$, that
checks the conditions given in the informal model; see
Section~\ref{sec:dolev}.  An \emph{execution} $e$
is a finite or infinite stream of configurations, which models a
\emph{maximal} sequence of configurations where any two consecutive
configurations are linked by the $\Step$ relation.  ``Maximal'' means
that $e$ is finite if and only if its last configuration is
terminal. We use the coinductive\footnote{Coinduction allows to define
and reason about potentially infinite objects.}  type $\mathit{Exec}$
to represent an execution stream along with a coinductive predicate
$\mathit{isExec}$
to check the above condition.
Daemons are also defined as predicates over executions (in the case of
the unfair daemon, this predicate is simply equal
to $\mathit{true}$).

Self-stabilization in PADEC is defined according to the usual
practice: the property is formalized as a predicate
$(\mathit{selfStabilization} \;\; \mathit{SPEC})$
where $\mathit{SPEC}$ is a predicate over executions
and models the specification of the algorithm.  An algorithm
is \emph{self-stabilizing w.r.t. the specification}
$\mathit{SPEC}$ if there exists a set of legitimate configurations
that satisfies the following three properties in every
execution $e$:
\begin{itemize}
\item \underline{\emph{Closure}}:
  if $e$ starts in a legitimate configuration then $e$ only contains
  legitimate configurations;
\item \underline{\emph{Convergence}}:
  $e$ eventually reaches a legitimate configuration; and
\item \underline{\emph{Specification}}:
  if $e$ starts in a legitimate configuration then $e$ satisfies the
  intended specification w.r.t. $\mathit{SPEC}$.
\end{itemize}
An algorithm is said to be \emph{silent} when each of its executions
eventually reaches a terminal configuration; in such a case, the set
of legitimate configurations can be chosen as the set of terminal
configurations.  The closure, convergence, and silent properties are
expressed using Linear Time Logic operators provided in the PADEC
library.

\subsection*{The \BFS Algorithm in PADEC}

For the \BFS Algorithm and its specification, we use the formal encoding
provided in \cite{AltisenCD23}; in particular, the algorithm is a
straightforward faithful translation in Coq of
Algorithm \ref{alg}. Notably, an element of \States,
namely a state of a given node, is a tuple
$(d, \mathit{par}, \mathit{root}, \mathit{neighbors})$ representing
the variables of the node as in Algorithm~\ref{alg}.

As the constant variables $\mathit{root}$ and $\mathit{neighbors}$
represent the network, the assumptions that this network is rooted,
bidirected and connected is encoded in a predicate on a configuration
using only those variables. This predicate, in particular uses the set
of edges of the network $\Edges \isdef \{ (p, q) \;|\; p,
q \in \Nodes \;\wedge\; (p \in q.\mathit{neighbors} \;\vee\; q \in
p.\mathit{neighbors}) \}$. Globally in this precidate, the neighbor
links represent a bidirected connected graph and the Boolean
$\mathit{root}$ should be true for a unique node.
We will assume moreover that this predicate holds for any
configuration, even if this is no more mentionned in the sequel.

In \cite{AltisenCD23}, the \BFS Algorithm was proven using PADEC to be
self-stabilizing and silent for the specification of a BFS spanning
tree, \emph{under the assumption of a weakly fair daemon}.  We extend
here this result to the \emph{unfair daemon}.  Note that, since \BFS
is silent, the properties of closure and specification still hold,
henceforth, relaxing the assumption from a weakly fair to an unfair
daemon is trivial.  The only missing property is the convergence.  The
rest of the paper is therefore focusing on proving the convergence of
the \BFS Algorithm under an unfair daemon in PADEC, \ie, providing a
constructive proof under the form of a potential function and its
corresponding order.


\section{Overview of the Proof}
\label{sec:proof}
\subsection{Error Gap Between Aligned and Misaligned Data}\label{subsec:proof-align-misalign}







\thmalignment*

\begin{proof}

For the aligned case, we can derive the mean squared error (MSE) as follows:
\begin{equation}\label{eq:mse_aligned}
    \mathrm{MSE}_\mathrm{aligned} = \inf_{\boldsymbol{\alpha} \in R^{m^P}, \boldsymbol{\beta} \in R^{m^S}} \|\mathbf{y} - \mathbf{X}^P \boldsymbol{\alpha} - \mathbf{X}^S \boldsymbol{\beta}\|
\end{equation}
The ordinary least squares (OLS) estimator of $\boldsymbol{\alpha}$ is given by:
\begin{equation}
    \hat{\boldsymbol{\alpha}} := (\mathbf{X}^{P \top} \mathbf{X}^P)^{-1} \mathbf{X}^P (\mathbf{y} - \mathbb{E}[\mathbf{R}] \mathbf{X}^S \boldsymbol{\beta}) 
\end{equation}
For a permutation matrix $\mathbf{R}$ under uniform distribution, we have $\mathbb{E}[\mathbf{R}] = \frac{1}{n}\mathds{1}^\top \mathds{1}$. Therefore:
\begin{equation}\label{eq:alpha_hat}
    \hat{\boldsymbol{\alpha}} = (\mathbf{X}^{P \top} \mathbf{X}^P)^{-1} \mathbf{X}^P (\mathbf{y} - \frac{1}{n} \mathds{1}^\top \mathds{1} \mathbf{X}^S \boldsymbol{\beta}) 
\end{equation}
The MSE for the misaligned case can be expressed as:
\begin{align}
    \mathrm{MSE}_{\mathrm{misaligned}} 
    & = \inf_{\boldsymbol{\beta}} \inf_{\boldsymbol{\alpha}} \mathbb{E}_\mathbf{R} \|\mathbf{y} - \mathbf{X}^P \boldsymbol{\alpha} - \mathbf{R} \mathbf{X}^S \boldsymbol{\beta}\|_2^2 \\
    & = \inf_{\boldsymbol{\beta}} \mathbb{E}_\mathbf{R} \|\mathbf{y} - \mathbf{X}^P \hat{\boldsymbol{\alpha}} - \mathbf{R} \mathbf{X}^S \boldsymbol{\beta}\|_2^2 \\
\end{align}
Substituting $\hat{\boldsymbol{\alpha}}$ from equation~\ref{eq:alpha_hat}, we obtain:
\begin{align}
    \mathrm{MSE}_{\mathrm{misaligned}} 
    & = \inf_{\boldsymbol{\beta}} \mathbb{E}_\mathbf{R} \left\|\mathbf{y} - \mathbf{X}^P (\mathbf{X}^{P \top} \mathbf{X}^P)^{-1} (\mathbf{X}^P \mathbf{y} - \mathbf{X}^P \frac{1}{n} 1^\top 1 \mathbf{X}^S \boldsymbol{\beta}) - \mathbf{R} \mathbf{X}^S \boldsymbol{\beta}\right\|_2^2 \\
    & = \inf_{\boldsymbol{\beta}} \mathbb{E}_\mathbf{R} \left\| (\mathbf{I} - \mathbf{X}^P (\mathbf{X}^{P \top} \mathbf{X}^P)^{-1} \mathbf{X}^P)\mathbf{y} + (\mathbf{X}^P (\mathbf{X}^{P \top} \mathbf{X}^P)^{-1} \mathbf{X}^P \frac{1}{n} \mathds{1}^\top \mathds{1} \mathbf{X}^S \boldsymbol{\beta}) - \mathbf{R} \mathbf{X}^S \boldsymbol{\beta}\right\|_2^2 
\end{align}
Since $\mathbf{X}^P (\mathbf{X}^{P \top} \mathbf{X}^P)^{-1} \mathbf{X}^P$ is a projection matrix that projects any vector onto the column space of $\mathbf{X}^P$, and $\mathbf{X}^S \boldsymbol{\beta}$ is orthogonal to the column space of $\mathbf{X}^P$, the term $\mathbf{X}^P (\mathbf{X}^{P \top} \mathbf{X}^P)^{-1} \mathbf{X}^P \frac{1}{n} \mathds{1}^\top \mathds{1} \mathbf{X}^S \boldsymbol{\beta} = 0$. Thus:
\begin{align}
    \mathrm{MSE}_{\mathrm{misaligned}}
    & = \inf_{\boldsymbol{\beta}} \mathbb{E}_\mathbf{R} \left\| (\mathbf{I} - \mathbf{X}^P (\mathbf{X}^{P \top} \mathbf{X}^P)^{-1} \mathbf{X}^P)\mathbf{y} - \mathbf{R} \mathbf{X}^S \boldsymbol{\beta}\right\|_2^2 \\
    & = \inf_{\boldsymbol{\beta}} \mathbb{E}_\mathbf{R} \left[\left\|\mathbf{R} \mathbf{X}^S \boldsymbol{\beta}\right\|_2^2 - 2\left[(\mathbf{I} - \mathbf{X}^P (\mathbf{X}^{P \top} \mathbf{X}^P)^{-1} \mathbf{X}^P)\mathbf{y}\right]^\top \mathbf{R} \mathbf{X}^S \boldsymbol{\beta} + \left\|(\mathbf{I} - \mathbf{X}^P (\mathbf{X}^{P \top} \mathbf{X}^P)^{-1} \mathbf{X}^P)\mathbf{y}\right\|_2^2\right]
\end{align}
By properties of permutation matrices:
\begin{equation}
    \mathbb{E}_\mathbf{R}\| \mathbf{R} \mathbf{X}^S \boldsymbol{\beta}\|_2^2 = \|\mathbf{X}^S \boldsymbol{\beta}\|_2^2; \; \mathbb{E}_\mathbf{R} [\mathbf{R}]= \frac{1}{n}\mathds{1}^\top \mathds{1}
\end{equation}
Therefore:
\begin{align}
    \mathrm{MSE}_{\mathrm{misaligned}}
    & = \inf_{\boldsymbol{\beta}} \left[\left\|\mathbf{X}^S \boldsymbol{\beta}\right\|_2^2 - 2\left[(\mathbf{I} - \mathbf{X}^P (\mathbf{X}^{P \top} \mathbf{X}^P)^{-1} \mathbf{X}^P)\mathbf{y}\right]^\top \frac{1}{n}\mathds{1}^\top \mathds{1} \mathbf{X}^S \boldsymbol{\beta} + \left\|(\mathbf{I} - \mathbf{X}^P (\mathbf{X}^{P \top} \mathbf{X}^P)^{-1} \mathbf{X}^P)\mathbf{y}\right\|_2^2\right]
\end{align}
Since $\mathbf{I} - \mathbf{X}^P (\mathbf{X}^{P \top} \mathbf{X}^P)^{-1} \mathbf{X}^P$ projects any vector onto the orthogonal complement of the column space of $\mathbf{X}^P$, the term $\left[(\mathbf{I} - \mathbf{X}^P (\mathbf{X}^{P \top} \mathbf{X}^P)^{-1} \mathbf{X}^P)\mathbf{y}\right]^\top \frac{1}{n}\mathds{1}^\top \mathds{1} \mathbf{X}^S \boldsymbol{\beta} = 0$. Hence:
\begin{align}
    \mathrm{MSE}_{\mathrm{misaligned}}
    & = \inf_{\boldsymbol{\beta}} \left[\left\|\mathbf{X}^S \boldsymbol{\beta}\right\|_2^2 + \left\|(\mathbf{I} - \mathbf{X}^P (\mathbf{X}^{P \top} \mathbf{X}^P)^{-1} \mathbf{X}^P)\mathbf{y}\right\|_2^2\right] \\
    & = \inf_{\boldsymbol{\beta}} \left\|\mathbf{X}^S \boldsymbol{\beta}\right\|_2^2 + \left\|(\mathbf{I} - \mathbf{X}^P (\mathbf{X}^{P \top} \mathbf{X}^P)^{-1} \mathbf{X}^P)\mathbf{y}\right\|_2^2 \\
\end{align}
The minimum is attained at $\boldsymbol{\beta} = \mathbf{0}$, yielding:
\begin{align}
    \mathrm{MSE}_{\mathrm{misaligned}}
    & = \left\|(\mathbf{I} - \mathbf{X}^P (\mathbf{X}^{P \top} \mathbf{X}^P)^{-1} \mathbf{X}^P)\mathbf{y}\right\|_2^2 \\
    & = \inf_{\boldsymbol{\alpha} \in \mathbb{R}^{m^P}, \boldsymbol{\beta} = \mathbf{0}} \left\|\mathbf{y} - \mathbf{X}^P \boldsymbol{\alpha} - \mathbf{X}^S \boldsymbol{\beta}\right\|_2^2 \\
\end{align}
Comparing with Equation~\ref{eq:mse_aligned}, we conclude:
\begin{equation}
    \mathrm{MSE}_{\mathrm{misaligned}} \geq \inf_{\boldsymbol{\alpha} \in \mathbb{R}^{m^P}, \boldsymbol{\beta} \in \mathbb{R}^{m^S}} \left\|\mathbf{y} - \mathbf{X}^P \boldsymbol{\alpha} - \mathbf{X}^S \boldsymbol{\beta}\right\|_2^2 = \mathrm{MSE}_{\mathrm{aligned}}
\end{equation}
\end{proof}





















\subsection{Approximation Capacity of Cluster Sampler}\label{subsec:proof-cluster-sampler}

\begin{definition}[Definition of optimal cluster sampler]
    Assume the inputs are uniformly bounded by some constant $B$. 
    The optimal cluster sampler is defined by the uniform equi-continuous cluster sampler function which achieves the minimal optimization loss for the prediction task in \cref{fig:leal-framework}.
    \begin{equation}
        \textrm{Optimal cluster sampler} := \arginf_{\textrm{Uniform equi-continuous cluster sampler}} \textrm{Loss}(\textrm{cluster sampler})
    \end{equation}
    The cluster sampler is defined over bounded inputs ($|X^P|_{\infty} \leq B, |X^S|_{\infty} \leq B$) from $\mathbb{R}^{m^P} \times \mathbb{R}^{n^S \times m^S}$ and output in $\mathbb{R}^{n^S}$.
\end{definition}

\begin{remark}
    The existence of such optimal cluster sampler is guaranteed by the boundedness and uniform equi-continuity of the set of cluster sampler functions. 
\end{remark}


\thmclustersampler*

\begin{proof}
    We just need to prove the statement for small $\epsilon \leq 6$.

    The input of cluster sampler is $1 \times m^P$ and output is $n^S \times m^S$, the final prediction is to generate a sample probabilities:
    \begin{equation}
        (n^S * m^S, 1 * m^P) \to (n^S * d, 1 * C) \to (n^S * C, 1 * C) \to n^S * 1. 
    \end{equation}

    Also, since there is no weight depends on dimension $n_2$, we can reduce the approximation statement to that there exists trainable weight such that the continuous function $h$ can be approximated:
    \begin{equation}
        (1 * m^S, 1 * m^P) \to (n^S * d, 1 * C) \to (n^S * C, 1 * C) \to 1 * 1. 
    \end{equation}

    Notice that the layer operation of secondary embedding and trainable centroids weights $(C \times d)$ is continuous and the pretrained encoder as a neural network (which is a universal approximator) can approximates any continuous function $f$ composited with inverse embedding. 
    For simplicity, we will consider $m^P = m^S = 1$. 
    For any continuous function $h(p, s) \in [0, 1]$,
    we just need to show there exists trainable weight $\theta_1$, $\theta_2$ such that 
    \begin{equation}
        f(p; \theta_1) \odot g(s; \theta_2) = \sum_{i=1}^C f_i(p; \theta_1) \odot g_i(s; \theta_2). 
    \end{equation}
    Here $f(p; \theta_1) \in \mathbb{R}^C$ is a function of $p$ parameterized by $\theta_1$ and $g(s; \theta_1) \in \mathbb{R}^C$ is a function of $s$ parameterized by $\theta_2$.  
    As any continuous function $f(p, s)$ has a corresponding Taylor series expansion, it means for any $\epsilon > 0$, there exists $C$ which depends on error $\epsilon$ such that
    \begin{equation}
        \sup_p \sup_s |h(p, s) -\sum_{i=1}^C pol_{1,i}(p) pol_{2,i}(s)| \leq \frac{\epsilon}{2}. 
    \end{equation}
    Furthermore, as polynomial functions are continuous function, therefore $f_i$ can be used to approximate the polynomial function $pol_{1, i}$ and $g$ can be used to approximate the polynomial function $pol_{2, i}$.
    \begin{align}
        \sup_p |pol_{1,i}(p) - f_i(p; \theta_1)| & \leq \frac{\epsilon}{6B} \\ 
        \sup_s |pol_{2,i}(s) - g_i(s; \theta_2)| & \leq \frac{\epsilon}{6B}. 
    \end{align}
    Here $B := \max(1, \sup_p \max_{i} |pol_{1, i}(p)|, \sup_s \max_{i} |pol_{2, i}(s)|).$ 
    We show that the cluster sampler is capable to approximate any desirable continuous cluster sampler. 
    \begin{equation}
        \sup_p \sup_s |h(p, s) -\sum_{i=1}^C f_i(p; \theta_1) g_i(s; \theta_2)| \leq \frac{\epsilon}{2} + \frac{\epsilon}{6B} * B + \frac{\epsilon}{6B} (B + \frac{\epsilon}{6B}) = \frac{5}{6} \epsilon + \frac{\epsilon^2}{36B^2} < \epsilon. 
    \end{equation}
    The last inequality comes from $\epsilon < 6$. 
    The universal approximation capacity of the cluster sampler is proved. 
\end{proof}

\begin{remark}
    Since we are working with a cluster sampler with specific manually designed structure, it mainly comes from the fact the student's t-kernel introduce a suitable implicit bias to more efficiently learn the cluster sample probability $(n_2 \times 1)$. 
\end{remark}


\section{A Decreasing Potential Function for $d$-steps}
\label{sec:dstep-potential}

This section is concerned with the proof of
Proposition~\ref{prop:dstep-potential} stated in
Section~\ref{sec:proof}. To this end, we proceed in three steps.
First, we establish a finite over-approximation on the set of the $d$
values that could be possibly reached in an execution involving
$d$-steps only from some initial configuration $\gamma_0$.
Second, we introduce a
partitioning of edges (being either smooth or non-smooth) and prove
some preservation properties along $d$-steps.
Third, we combine the above results to effectively construct a
potential function for $d$-steps and a well-founded order on the
co-domain of this function, ultimately proving
Proposition~\ref{prop:dstep-potential}.

For the sake of readability, we denote $d$-steps $\gamma \xstep{\DStep}
\gamma'$ shortly by $\gamma \dstep \gamma'$.

\subsection{Bounds on Distance Values}
\label{sec:dstep-potential:bounds}

For a configuration $\gamma$, we define the integers $\maxd\gamma
\isdef \max \{\gamma.q.d \mid q \in \nodes\}$, $\mind\gamma \isdef
\min \{\gamma.q.d \mid q \in \nodes\}$, $\sumd\gamma \isdef \sum
\{\gamma.q.d \mid q \in \nodes\}$\footnote{The sum is taken on the
multiset of $d$ values}.  We also define $\dbot{\gamma}$,
$\dtop{\gamma}$ respectively a \emph{bottom} and a \emph{top}
configuration associated to $\gamma$.  These are identical to $\gamma$
except for $d$ values, defined for every node $p$ as follows:
\begin{eqnarray*}
  \dbot{\gamma}.p.d & \isdef & \mind\gamma \\
  \dtop{\gamma}.p.d & \isdef & \left\{ \begin{array}{ll}
    \gamma.p.d & \mbox{if } p = r \\
    \max \{ \gamma.p.d, 1 + \min \{ \dtop{\gamma}.q.d \mid \\
    \hspace{1cm} (p,q)\in \Edges, \dist{p}{\r} = 1 + \dist{q}{\r} \}
    \} &
    \mbox{otherwise,}
  \end{array} \right.
\end{eqnarray*}
where $\dist{q}{\r}$ represents the distance of some node $q$ to the
root \r.  Note that the recursive definition of $\dtop{\gamma}.p.d$ is
well-defined as the recursion is limited to neighbours $q$ of $p$
located at a smaller distance to the root $\r$ than $p$.  Intuitively,
the maximal $d$ value of a non-root node $p$ in some configuration
reachable from $\gamma$ is either its value in $\gamma$ (\ie, it can
be the case when $p$ does not execute) or 1 plus the minimum of the
maximal $d$ values of its neighbors $q$ closer to the root
(see Action $CD$ when $p$ executes).
We define the partial order $\dleq$ on configurations by taking
%
$$ \gamma_1 \dleq \gamma_2 \isdef \forall q \in \nodes: \gamma_1.q.d
\le \gamma_2.q.d $$
%
The next lemma states basic properties of the $\dbot{(.)}$,
$\dtop{(.)}$ operators, namely their idempotence and
their monotonicity with respect to $\dleq$.  The proof follows from
definitions and uses induction on nodes according to their
distance to the root.

\begin{lemma} \label{lemma:bounds:basic} ~
  
  \begin{enumerate}[label=(\roman*)]
  \item For all configuration $\gamma$,
  $\dbot{\gamma} \dleq \gamma \dleq \dtop{\gamma}$,
  $\dbot{(\dbot{\gamma})} = \dbot{\gamma}$ and $\dtop{(\dtop{\gamma})}
  = \dtop{\gamma}$.

  \item For all configurations $\gamma_1$ and $\gamma_2$ such that
  $\gamma_1 \dleq \gamma_2$, $\dbot{\gamma_1} \dleq \dbot{\gamma_2}$ and
  $\dtop{\gamma_1} \dleq \dtop{\gamma_2}$.

\end{enumerate}
\end{lemma}

The next lemma relates the bottom and top configurations to $d$-steps.
The proof is done by induction respectively, on the set of nodes
according to their distance to the root (i) and on the length of an
execution sequence from $\gamma_0$ (ii).\footnote{$\gamma_0 \dstepstar
\gamma$ means that $\gamma$ is reachable from $\gamma_0$ using a
finite number of $d$-steps.}
\begin{lemma}\label{lemma:bounds:dstep} ~
  
  \begin{enumerate}[label=(\roman*)]  
  \item For all configurations $\gamma$ and $\gamma'$ such that
  $\gamma \dstep \gamma'$, $\dbot{\gamma} \dleq \dbot{\gamma'}$ and
  $\dtop{\gamma'} \dleq \dtop{\gamma}$.

  \item For all configurations $\gamma_0$ and $\gamma$ such that
  $\gamma_0 \dstepstar \gamma$,
  $\dbot{\gamma_0} \dleq \gamma \dleq \dtop{\gamma_0}$.

  \end{enumerate}
\end{lemma}

\subsection{Smooth and Non-smooth $d$-steps}
\label{sec:dstep-potential:smooth}

We say that an edge $(p,q)\in\Edges$ is \emph{smooth}
(resp. \emph{non-smooth}) in a configuration $\gamma\in\Env$ if the
difference (in absolute value, $abs$) between the $d$-values at its
endpoints $p$, $q$ is at most 1 (resp. at least 2).  Formally,
consider the predicate
%
$$\csmooth{\gamma}{(p,q)} \isdef (abs(\gamma.p.d - \gamma.q.d) \le 1).$$
%
We say that a $d$-step $\gamma \dstep \gamma'$ is \emph{smooth} if all the
nodes $p$ changing their values from $\gamma$ to $\gamma'$ are
connected to smooth edges only in $\gamma$, formally:
%
$$\begin{array}{l}
  \ssmooth{\gamma \dstep \gamma'} \isdef \\
  \hspace{1cm} \forall p \in \Nodes: (\gamma'.p.d \not= \gamma.p.d) \Rightarrow
  (\forall q \in p.neighbors: \csmooth{\gamma}{(p,q)}
\end{array}$$
%
We define the rank of an edge $(p,q)\in\Edges$ in a configuration
$\gamma\in\Env$ as
$\crank{\gamma}{(p,q)} \isdef \min(\gamma.p.d, \gamma.q.d)$.



\begin{figure}[th]
  \centering
  \scalebox{0.9}{\input{d-steps.pdf_t}}
  \caption{\label{fig:d-steps}Smooth and non-smooth steps}
\end{figure}

For illustration, consider the three configurations $\gamma_1$,
$\gamma_2$, $\gamma_3$ depicted in Fig.~\ref{fig:d-steps}.  We
represented the $d$ values of the nodes by their positioning on the
horizontal lines e.g., $\gamma_1.\r.d = 10$, $\gamma_1.p_1.d = 9$,
$\gamma_2.p_1.d = 10$, etc.  Edges are represented by lines
connecting nodes: smooth (resp. non-smooth) edges are depicted in
blue (resp. red).  Configuration $\gamma_2$ is the successor of
$\gamma_1$ by a smooth step.  That is, only $p_1$ and $p_6$ have
executed and these nodes were connected only to smooth (blue) edges
in $\gamma_1$.  Configuration $\gamma_3$ is the successor of
$\gamma_2$ by a non-smooth step.  That is, $p_3$ and $p_4$ have been
executed along the step, and these nodes were connected to some
non-smoth edges.

The next lemmas provide key properties for understanding the execution
of $d$-steps, depending if they are smooths or not.
Lemma~\ref{lemma:dsteps:smooth} basically states that partitioning
between smooth and non-smooth, as well as the rank of every non-smooth
edge is preserved by smooth steps.  In addition, the total sum of $d$
values is increasing along such a step.  

\begin{lemma}\label{lemma:dsteps:smooth}
  Consider a smooth d-step $\gamma \dstep \gamma'$.  Then,
  \begin{enumerate}[label=(\roman*)]
  \item $\forall e \in \edges: \neg \csmooth{\gamma}{e} \Leftrightarrow
    \neg \csmooth{\gamma'}{e}$,
  \item $\forall e \in \edges: \neg \csmooth{\gamma}{e} \Rightarrow
    (\crank{\gamma}{e} = \crank{\gamma'}{e})$,
  \item $\sumd \gamma' > \sumd \gamma$.
  \end{enumerate}
\end{lemma}
\begin{proof}
  The proof follows immediately from the definition of smooth steps
  and/or edges.  First, the fact that non-smooth edges are preserved
  along with their rank in a smooth $d$-step directly comes from the
  definition of a smooth step: since no node connected to a non-smooth
  edge can execute, non-smooth edges remained unchanged.
  Second, we obtain the increasing of the sum of all $d$-values by
  observing that when a node executes in a smooth $d$-step, its $d$
  value increases by one or two (due to its neighbors which are either
  above by one or at the same level of $d$ value). As a smooth
  $d$-step involves at least one such an executing node, $\sumd$
  necessarily increases (since nodes that do not increase $d$ leave it
  unchanged).
  \qed
\end{proof}

For illustration, consider the smooth step depicted in
Fig.~\ref{fig:d-steps}, \ie, between $\gamma_1$ and $\gamma_2$.  It is
rather trivial that, as long as the nodes executing were connected to
smooth edges only (in blue), their execution has no impact on the
non-smooth edges \ie, they remain non-smooth and preserve their rank.
Yet, the overall sum of the $d$ values increases, here because at
least the values of the two moving nodes has increased (by 1 for $p_1$
and by 2 for $p_6$).

Lemma~\ref{lemma:dsteps:nonsmooth} provides a similar preservation
result for non-smooth steps.  In this case, the key property is that
one can effectively compute a bound $k^*$ such that (i) all non-smooth
edges with rank lower than $k^*$ remain non-smooth and preserve their
rank and (ii) the set of non-smooth edges with rank $k^*$ is strictly
decreasing along the step.  The lemma provides both the explicit
definition of $k^*$ as well as the identification of a non-smooth edge
at level $k^*$ which either becomes smooth or gets a reduced rank
after the step, that is, some edge $(p,q)$ for which the minimum is
achieved in the definition of $k^*$.

\begin{lemma}\label{lemma:dsteps:nonsmooth} Consider a non-smooth d-step $\gamma \dstep \gamma'$.   Let
  $$\begin{array}{l} k^* \isdef \min \{ \crank{\gamma}{(p,q)} \mid (p, q) \in \edges:
    \neg \csmooth{\gamma}{(p,q)}, \\
    \hspace{5cm} \gamma'.p.d \not=\gamma.p.d \mbox{ or } \gamma'.q.d \not= \gamma.q.d \}
    \end{array}$$
  Then,
  \begin{enumerate}[label=(\roman*)]
  \item $\forall e \in \edges: (\crank{\gamma'}{e} \le k^* \wedge \neg \csmooth{\gamma'}{e}) \Rightarrow \\
    \hspace*{3cm} (\crank{\gamma}{e} = \crank{\gamma'}{e} \wedge \neg \csmooth{\gamma}{e})$,
  \item $\forall e \in \edges: (\crank{\gamma}{e} < k^* \wedge \neg \csmooth{\gamma}{e}) \Rightarrow \\
    \hspace*{3cm} (\crank{\gamma'}{e} = \crank{\gamma}{e} \wedge \neg \csmooth{\gamma'}{e})$,
  \item $\exists e \in \edges: (\crank{\gamma}{e} = k^* \wedge \neg \csmooth{\gamma}{e}) \wedge \\
    \hspace*{3cm} (\neg \csmooth{\gamma'}{e} \Rightarrow \crank{\gamma'}{e} > \crank{\gamma}{e}))$.
  \end{enumerate}
\end{lemma}
\begin{proof}
  (i) The proof is done by case splitting, considering which endpoints
  of non-smooth edges $e$ execute. In fact, the only feasible case is
  when none of them executes.  In all other cases, by choosing the
  node which gives a new value to its $d$ variable, we obtain a
  contradiction, either with the minimality of $k^*$ or with the
  non-smoothness of $e$ in $\gamma'$.

  (ii) By definition of $k^*$, no node involved in a non-smooth edge
  can execute if the rank is below $k^*$, hence rank and
  non-smoothness are left unchanged.

  (iii) Note here that, using Coq, to be able to prove "$\exists e \in
  \edges: ...$", we have to effectively contruct such an edge. In our
  case, it is chosen as some of the edges which achieves the minimum
  rank value when computing $k^*$: a non-smooth edge $e^*$ such that
  $\crank{\gamma}{e^*} = k^*$, and one of its end nodes executes
  during the step (it exists and can be computed using the computation
  of the minimum value over a finite set). Now, consider the case
  where $e^*$ remains non-smooth in $\gamma'$. We note $e^*=(p, q)$
  with $\crank{\gamma}{(p, q)} = \gamma.p.d$. We can prove that if $p$
  executes then $\gamma'.p.d > \gamma.p.d$ and that if $q$ executes
  then $\gamma'.q.d = \gamma.p.d + 1$ (see Fig.~\ref{fig:dsteps:proof}
  for an illustration). The result is then easy to conclude.
  \qed
\end{proof}

\begin{figure}[th]
  \begin{center}
    \scalebox{0.9}{\input{d-steps-proof.pdf_t}}
  \end{center}
  \caption{\label{fig:dsteps:proof}Possible evolutions of a non-smooth edge $e^*=(p,q)$ with
    minimal rank $k^*$: (i) only $p$ executes, (ii) only $q$ executes
    (iii) $p$ and $q$ executes}
\end{figure}

For illustration also, consider the non-smooth step depicted in
Fig.~\ref{fig:d-steps} between $\gamma_2$ and $\gamma_3$.  In this
case, the bound value is $k^* = 8$.  The lemma ensures that the set of
non-smooth edges of rank strictly lower than $8$ are unchanged.  No
such edges actually exist in the configurations $\gamma_2$ or
$\gamma_3$. But, actually, it is not hard to imagine that if such
edges would exist and are not related to $p_3$ and $p_6$, they would
not be impacted by the move.  Also, the lemma guarantees that the set
of edges at level 8 is strictly decreasing.  That is, the set of
non-smooth edges at level 8 is $\{ (p_3,p_7), (p_3,p_4) \}$ in
$\gamma_2$, respectively $\emptyset$ in $\gamma_3$.

Finally, we define $\nsset{\gamma}{k} \isdef \{ e \in \edges \mid \neg
\csmooth{\gamma}{e} \;\wedge\; \crank{\gamma}{e} = k \}$, that is, the
set of non-smooth edges of rang $k$ in $\gamma$.  The next lemma
simply re-formulates the results of Lemma \ref{lemma:dsteps:smooth} in
point (i) and Lemma \ref{lemma:dsteps:nonsmooth} in point (ii) into a
single statement about the sets $\nsset{\gamma}{k}$ to
facilitate their use in the definition of the potential function in
the next subsection.

\begin{lemma}\label{lemma:dsteps}
  Consider a d-step $\gamma \dstep \gamma'$.  Then
  \begin{enumerate}[label=(\roman*)]
  \item if the step $\gamma \dstep \gamma'$ is smooth then
    $\nsset{\gamma}{k} = \nsset{\gamma'}{k}$ for all integer $k$,
  \item if the step $\gamma \dstep \gamma'$ is non-smooth then
    (a) $\nsset{\gamma}{k} = \nsset{\gamma'}{k}$ for all integer $k <
    k^*$ and (b) $\nsset{\gamma'}{k^*} \subsetneq
    \nsset{\gamma}{k^*}$.
  \end{enumerate}
\end{lemma}

\subsection{Potential Function and Proof of
  Proposition~\ref{prop:dstep-potential}} 
\label{sec:dstep-potential:function}

Given a finite interval of integers $K$, and two finite sequences of
$K$-indexed finite sets $\mathcal{X} \isdef (X_k)_{k \in K}$,
$\mathcal{Y} \isdef (Y_k)_{k \in K}$ we write $\mathcal{X} =
\mathcal{Y}$ whenever $X_k = Y_k$ for all $k \in K$, and $\mathcal{X}
\prec_{setlex} \mathcal{Y}$ whenever there exists an integer $k^* \in
K$ such that $X_k = Y_k$ for all $k \in K$, $k<k^*$ and $X_{k^*}
\subsetneq Y_{k^*}$.  Note that $\prec_{setlex}$ is a well-founded
lexicographic order on the set of finite sequences of $K$-indexed
finite sets.


\subsection*{Proof of Proposition~\ref{prop:dstep-potential}}
\begin{proof}
  (a) We define $B(\gamma_0) \isdef \{ \gamma ~|~ \dbot{\gamma_0}
  \dleq \gamma \dleq \dtop{\gamma_0} \}$.  From
  Lemma~\ref{lemma:bounds:basic}(i) we obtain immediately $\gamma_0 \in
  B(\gamma_0)$.  The set $B(\gamma_0)$ is obviously closed by taking
  $par$-steps, as these steps do no change the values of $d$-variables.
  The closure of $B(\gamma_0)$ by $d$-steps can be understood by the
  $\dleq$ inequalities depicted below:

  \begin{center}
    \input{ineqchain.pdf_t}
  \end{center}
  
  Knowing $\gamma \in B(\gamma_0)$, that is, $\dbot{\gamma_0} \dleq
  \gamma \dleq \dtop{\gamma_0}$ we obtain the inequalities from the
  top line by using the idempotence and monotonicity of
  $\dbot{(.)}$, $\dtop{(.)}$ with respect to $\dleq$
  (Lemma~\ref{lemma:bounds:basic}). The same lemma ensures the
  inequalities of the bottom line.  Finally, the inequalities across
  the two lines hold because of Lemma~\ref{lemma:bounds:dstep}.  All
  over, they ensure that $\dbot{\gamma_0} \dleq \gamma' \dleq
  \dtop{\gamma_0}$ for any $d$-step $\gamma\dstep\gamma'$.
  
  (b) We define the interval of integers $K_0 \isdef [\mind
    \dbot{\gamma_0}, \maxd \dtop{\gamma_0}]$, that is, the interval of
  possible $d$-values in the configurations reachable from $\gamma_0$.
  We define the domain $D(\gamma_0) \isdef (2^\edges)^{K_0} \times
  [\sumd \dbot{\gamma_0}, \sumd \dtop{\gamma_0}]$. That is,
  $D(\gamma_0)$ consists of pairs $(\mathcal{E},s)$ where $\mathcal{E}
  : K_0 \rightarrow 2^\edges$ is a $K_0$-indexed sequence of sets of
  edges and $s$ is a bounded integer.  In particular, note that
  $D(\gamma_0)$ is finite.  We define the potential function
  $\dpot{\gamma_0} : \Gamma \rightarrow D(\gamma_0)$ by taking
  $\dpot{\gamma_0}(\gamma) \isdef ((\nsset{\gamma}{k})_{k\in K_0},
  \sumd\gamma)$.  Remark that $\dpot{\gamma_0}$ is not dependent on
  \textit{par} variables in $\gamma$.

  (c) We define the relation $\prec_d$ on $D(\gamma_0)$ by taking
  $(\mathcal{E}_1,s_1) \prec_d (\mathcal{E}_2,s_2) \isdef
  \mathcal{E}_1 \prec_{setlex} \mathcal{E}_2 \vee (\mathcal{E}_1 =
  \mathcal{E}_2 \wedge s_2 < s_1)$.  That is, $\prec_d$ is actually a
  strict lexicographic order on pairs $(\mathcal{E},s)$ which combines
  the well-founded order $\prec_{setlex}$ on finite sequences of
  finite sets and a well-founded order $<$ on bounded integers.  It
  remains to prove that $$\forall \gamma,\gamma'\in\Env, ~ \gamma \in
  B(\gamma_0) \mbox{ and } \gamma \dstep \gamma' \Rightarrow
  \dpot{\gamma_0}(\gamma') \prec_d \dpot{\gamma_0}(\gamma)$$ Let
  respectively $(\mathcal{E},s) \isdef \dpot{\gamma_0}(\gamma)$,
  $(\mathcal{E}',s') \isdef \dpot{\gamma_0}(\gamma')$. Note that from
  $\gamma \in B(\gamma_0)$ and the previous point (a) we obtain that
  $\gamma' \in B(\gamma_0)$ as well.  In particular, this ensures the
  ranks of non-smooth edges of $\gamma$, $\gamma'$ are contained in
  $K_0$ and respectively $s$, $s'$ are contained in the interval
  $[\sumd \dbot{\gamma_0}, \sumd\dtop{\gamma_0}]$.
  Lemma~\ref{lemma:dsteps} and Lemma~\ref{lemma:dsteps:smooth}($iii$)
  provide the conditions ensuring that $\dpot{\gamma_0}$ is indeed a
  decreasing potential function with respect to $\prec_d$ as expected.
  For non-smooth steps, we observe the strict inequality $\mathcal{E}'
  \prec_{setlex} \mathcal{E}$. For smooth $d$-steps we observe the
  equality ${\mathcal E} = \mathcal{E}'$ and the strict inequality $s
  < s'$. \qed
\end{proof}




\section{Conclusion}
\label{sec:conclusion}
\section{Conclusion}
In this work, we propose a simple yet effective approach, called SMILE, for graph few-shot learning with fewer tasks. Specifically, we introduce a novel dual-level mixup strategy, including within-task and across-task mixup, for enriching the diversity of nodes within each task and the diversity of tasks. Also, we incorporate the degree-based prior information to learn expressive node embeddings. Theoretically, we prove that SMILE effectively enhances the model's generalization performance. Empirically, we conduct extensive experiments on multiple benchmarks and the results suggest that SMILE significantly outperforms other baselines, including both in-domain and cross-domain few-shot settings.

\bibliographystyle{splncs04}
\bibliography{biblio}

\end{document}

\typeout{get arXiv to do 4 passes: Label(s) may have changed. Rerun}
