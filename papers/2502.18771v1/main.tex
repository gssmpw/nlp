\documentclass[sigconf]{acmart}
% \documentclass[sigconf,review]{acmart}
% \documentclass[sigconf,anonymous,review]{acmart}
\pdfoutput=1
\usepackage{titletoc}
\usepackage{mdframed}
% \usepackage{xcolor}

\definecolor{acmblue}{RGB}{0,112,192}
\newcommand{\appendixcontentsname}{Appendix}
% \newlistof{appendix}{atc}{\appendixcontentsname}

\titlecontents{section}[1.5em]
  {\vspace{0.3\baselineskip}\LARGE\sffamily\color{acmblue}\bfseries}
  {\contentslabel[\normalsize\color{acmblue}\thecontentslabel]{1.8em}}
  {\hspace*{-1.8em}}
  {\normalsize\color{acmblue}\titlerule*[0.3em]{.}\contentspage}

% 二级标题改为黑色
\titlecontents{subsection}[3.8em]
  {\vspace{0.2\baselineskip}\Large\sffamily\color{black}} % 修改颜色为black
  {\contentslabel{2.3em}}
  {\hspace*{-2.3em}}
  {\hfill\normalsize\sffamily\color{black}\contentspage} % 页码也改为黑色

% 目录标题设置(使用\Huge)
\newcommand{\AppendixTOC}{%
  \section*{\fontsize{22}{26}\selectfont\sffamily\bfseries\color{acmblue}\appendixcontentsname}
  \addcontentsline{toc}{section}{\appendixcontentsname}
}

% \documentclass[sigconf,review]{acmart}
%% NOTE that a single column version may required for 
%% submission and peer review. This can be done by changing
%% the \doucmentclass[...]{acmart} in this template to 
%% \documentclass[manuscript,screen]{acmart}
%% 
%% To ensure 100% compatibility, please check the white list of
%% approved LaTeX packages to be used with the Master Article Template at
%% https://www.acm.org/publications/taps/whitelist-of-latex-packages 
%% before creating your document. The white list page provides 
%% information on how to submit additional LaTeX packages for 
%% review and adoption.
%% Fonts used in the template cannot be substituted; margin 
%% adjustments are not allowed.
\usepackage{booktabs} % For formal tables
\usepackage{multirow}
\usepackage{makecell}

\usepackage{enumitem}
\usepackage{url}
% \usepackage{xcolor}
% \usepackage[table]{xcolor} % 正确加载颜色包
% \usepackage[table,xcdraw]{xcolor}
\usepackage{colortbl}      % 用于表格的颜色支持
\usepackage{graphicx}
\usepackage{booktabs} % 美化表格用
\usepackage{array} % 支持表格列宽调整
\renewcommand{\arraystretch}{1.1} % 增加单元格高度
\usepackage{caption}       % 控制标题格式
\usepackage{longtable} % 引入 longtable 包
\usepackage{pifont} % for \ding symbols
\usepackage[textsize=tiny]{todonotes}
\theoremstyle{remark}
\newtheorem{remark}{\textbf{Remark}}

\usepackage{listings}
\lstset{
    basicstyle=\ttfamily\normalsize, 
    breaklines=true,             
    % frame=single,                
    columns=fullflexible         
}

\setlist{leftmargin=5mm}

\newcommand{\lu}[1]{\textcolor{red}{[[\textbf{Lu:} #1]]}}
\newcommand{\ym}[1]{\textcolor{blue}{[[\textbf{Yao:} #1]]}}
\newcommand{\liang}[1]{\textcolor{green}{[[\textbf{liang:} #1]]}}
\newcommand{\xn}[1]{\textcolor{orange}{[[\textbf{XD:} #1]]}}
%%
%% \BibTeX command to typeset BibTeX logo in the docs
\AtBeginDocument{%
  \providecommand\BibTeX{{%
    \normalfont B\kern-0.5em{\scshape i\kern-0.25em b}\kern-0.8em\TeX}}}

%% Rights management information.  This information is sent to you
%% when you complete the rights form.  These commands have SAMPLE
%% values in them; it is your responsibility as an author to replace
%% the commands and values with those provided to you when you
%% complete the rights form.


% \copyrightyear{2024}
% \acmYear{2024}
% \setcopyright{acmlicensed}
% \acmConference[KDD '24] {Proceedings of the 30th ACM SIGKDD Conference on Knowledge Discovery and Data Mining }{August 25--29, 2024}{Barcelona, Spain.}
% \acmBooktitle{Proceedings of the 30th ACM SIGKDD Conference on Knowledge Discovery and Data Mining (KDD '24), August 25--29, 2024, Barcelona, Spain}
% \acmISBN{979-8-4007-0490-1/24/08}
% \acmDOI{10.1145/3637528.3671462}
% \settopmatter{printacmref=true}
%
%  Uncomment \acmBooktitle if th title of the proceedings is different
%  from ``Proceedings of ...''!
%
%\acmBooktitle{Woodstock '18: ACM Symposium on Neural Gaze Detection,
%  June 03--05, 2018, Woodstock, NY} 
% \acmISBN{978-1-4503-XXXX-X/18/06}


%%
%% Submission ID.
%% Use this when submitting an article to a sponsored event. You'll
%% receive a unique submission ID from the organizers
%% of the event, and this ID should be used as the parameter to this command.
%%\acmSubmissionID{123-A56-BU3}

%%
%% For managing citations, it is recommended to use bibliography
%% files in BibTeX format.
%%
%% You can then either use BibTeX with the ACM-Reference-Format style,
%% or BibLaTeX with the acmnumeric or acmauthoryear sytles, that include
%% support for advanced citation of software artefact from the
%% biblatex-software package, also separately available on CTAN.
%%
%% Look at the sample-*-biblatex.tex files for templates showcasing
%% the biblatex styles.
%%

%%
%% For managing citations, it is recommended to use bibliography
%% files in BibTeX format.
%%
%% You can then either use BibTeX with the ACM-Reference-Format style,
%% or BibLaTeX with the acmnumeric or acmauthoryear sytles, that include
%% support for advanced citation of software artefact from the
%% biblatex-software package, also separately available on CTAN.
%%
%% Look at the sample-*-biblatex.tex files for templates showcasing
%% the biblatex styles.
%%

%%
%% The majority of ACM publications use numbered citations and
%% references.  The command \citestyle{authoryear} switches to the
%% "author year" style.
%%
%% If you are preparing content for an event
%% sponsored by ACM SIGGRAPH, you must use the "author year" style of
%% citations and references.
%% Uncommenting
%% the next command will enable that style.
%%\citestyle{acmauthoryear}

%%
%% end of the preamble, start of the body of the document source.

% % 定义附录目录的名称
% \newcommand{\appendixcontentsname}{Appendix Contents}

% % 定义附录目录命令
% \makeatletter
% \newcommand{\appendixtableofcontents}{%
%     \section*{\appendixcontentsname} % 附录目录标题
%     \@starttoc{atc} % 使用 atc 扩展名存储附录目录条目
% }
% \makeatother


\begin{document}

%%
%% The "title" command has an optional parameter,
%% allowing the author to define a "short title" to be used in page headers.
% \title{Benchmarking LLMs for Graph Reasoning Tasks \ym{To be updated. Instead of ``benchmarking'', I think we also do investigation and exploration.}} 
\title{Exploring Graph Tasks with Pure LLMs: A Comprehensive Benchmark and Investigation}

%%
%% The "author" command and its associated commands are used to define
%% the authors and their affiliations.
%% Of note is the shared affiliation of the first two authors, and the
%% "authornote" and "authornotemark" commands
%% used to denote shared contribution to the research.

\author{Yuxiang Wang}
\email{yuxiawang@polyu.edu.hk}
\affiliation{%
  \institution{The Hong Kong Polytechnic University
    \country{HK SAR}}}


\author{Xinnan Dai}
\email{daixinna@msu.edu}
\affiliation{%
  \institution{Michigan State University
    \country{USA}}}


\author{Wenqi Fan}
\email{wenqifan03@gmail.com}
% \authornote{Department of Computing, and 
 % Department of Management and Marketing, The Hong Kong Polytechnic University.}
\affiliation{
  \institution{The Hong Kong Polytechnic University
    \country{HK SAR}}}


\author{Yao Ma}
% \authornote{Corresponding author.}
\email{may13@rpi.edu}
\affiliation{%
  \institution{Rensselaer Polytechnic Institute, USA}
  % \city{Troy}
  \country{}}


%%
%% By default, the full list of authors will be used in the page
%% headers. Often, this list is too long, and will overlap
%% other information printed in the page headers. This command allows
%% the author to define a more concise list
%% of authors' names for this purpose.
% \renewcommand{\shortauthors}{Trovato and Tobin, et al.}

%%
%% The abstract is a short summary of the work to be presented in the
%% article.
\begin{abstract}
Graph-structured data has become increasingly prevalent across various domains, raising the demand for effective models to handle graph tasks like node classification and link prediction. Traditional graph learning models like Graph Neural Networks (GNNs) have made significant strides, but their capabilities in handling graph data remain limited in certain contexts. In recent years, large language models (LLMs) have emerged as promising candidates for graph tasks, yet most studies focus primarily on performance benchmarks and fail to address their broader potential, including their ability to handle limited data, their transferability across tasks, and their robustness. In this work, we provide a comprehensive exploration of LLMs applied to graph tasks. We evaluate the performance of pure LLMs, including those without parameter optimization and those fine-tuned with instructions, across various scenarios. Our analysis goes beyond accuracy, assessing LLMs’ ability to perform in few-shot/zero-shot settings, transfer across domains, understand graph structures, and demonstrate robustness in challenging scenarios. We conduct extensive experiments with 16 graph learning models alongside 6 LLMs (e.g., Llama3B, GPT-4o, Qwen-plus), comparing their performance on datasets like Cora, PubMed, ArXiv, and Products. Our findings show that LLMs, particularly those with instruction tuning, outperform traditional models in few-shot settings, exhibit strong domain transferability, and demonstrate excellent generalization and robustness. This work offers valuable insights into the capabilities of LLMs for graph learning, highlighting their advantages and potential for real-world applications, and paving the way for future research in this area. Codes and datasets are released in \href{}{\textit{https://github.com/myflashbarry/LLM-benchmarking}}
% Additionally, we introduce Continuous Pre-training, a two-step approach that enhances LLM performance by first pre-training the model on an unsupervised graph task and then fine-tuning it for specific graph tasks. Our results show that this approach effectively improves LLMs in few-shot and zero-shot scenarios. 
\end{abstract}

%%
%% The code below is generated by the tool at http://dl.acm.org/ccs.cfm.
%% Please copy and paste the code instead of the example below.
%%
\begin{CCSXML}
<ccs2012>
<concept>
<concept_id>10010147.10010178</concept_id>
<concept_desc>Computing methodologies~Artificial intelligence</concept_desc>
<concept_significance>500</concept_significance>
</concept>
</ccs2012>
\end{CCSXML}

\ccsdesc[500]{Computing methodologies~Artificial intelligence}
% \begin{CCSXML}
% <ccs2012>
%  <concept>
%   <concept_id>00000000.0000000.0000000</concept_id>
%   <concept_desc>Do Not Use This Code, Generate the Correct Terms for Your Paper</concept_desc>
%   <concept_significance>500</concept_significance>
%  </concept>
%  <concept>
%   <concept_id>00000000.00000000.00000000</concept_id>
%   <concept_desc>Do Not Use This Code, Generate the Correct Terms for Your Paper</concept_desc>
%   <concept_significance>300</concept_significance>
%  </concept>
%  <concept>
%   <concept_id>00000000.00000000.00000000</concept_id>
%   <concept_desc>Do Not Use This Code, Generate the Correct Terms for Your Paper</concept_desc>
%   <concept_significance>100</concept_significance>
%  </concept>
%  <concept>
%   <concept_id>00000000.00000000.00000000</concept_id>
%   <concept_desc>Do Not Use This Code, Generate the Correct Terms for Your Paper</concept_desc>
%   <concept_significance>100</concept_significance>
%  </concept>
% </ccs2012>
% \end{CCSXML}

% \ccsdesc[500]{Do Not Use This Code~Generate the Correct Terms for Your Paper}
% \ccsdesc[300]{Do Not Use This Code~Generate the Correct Terms for Your Paper}
% \ccsdesc{Do Not Use This Code~Generate the Correct Terms for Your Paper}
% \ccsdesc[100]{Do Not Use This Code~Generate the Correct Terms for Your Paper}

%%
%% Keywords. The author(s) should pick words that accurately describe
%% the work being presented. Separate the keywords with commas.
\keywords{Large Language Models, Graph Tasks, Graph Machine Learning}


%% A "teaser" image appears between the author and affiliation
%% information and the body of the document, and typically spans the
%% page.
% \begin{teaserfigure}
%   \includegraphics[width=\textwidth]{sampleteaser}
%   \caption{Seattle Mariners at Spring Training, 2010.}
%   \Description{Enjoying the baseball game from the third-base
%   seats. Ichiro Suzuki preparing to bat.}
%   \label{fig:teaser}
% \end{teaserfigure}

% \received{20 February 2007}
% \received[revised]{12 March 2009}
% \received[accepted]{5 June 2009}

%%
%% This command processes the author and affiliation and title
%% information and builds the first part of the formatted document.
\maketitle


\section{Introduction}

Large language models (LLMs) have achieved remarkable success in automated math problem solving, particularly through code-generation capabilities integrated with proof assistants~\citep{lean,isabelle,POT,autoformalization,MATH}. Although LLMs excel at generating solution steps and correct answers in algebra and calculus~\citep{math_solving}, their unimodal nature limits performance in plane geometry, where solution depends on both diagram and text~\citep{math_solving}. 

Specialized vision-language models (VLMs) have accordingly been developed for plane geometry problem solving (PGPS)~\citep{geoqa,unigeo,intergps,pgps,GOLD,LANS,geox}. Yet, it remains unclear whether these models genuinely leverage diagrams or rely almost exclusively on textual features. This ambiguity arises because existing PGPS datasets typically embed sufficient geometric details within problem statements, potentially making the vision encoder unnecessary~\citep{GOLD}. \cref{fig:pgps_examples} illustrates example questions from GeoQA and PGPS9K, where solutions can be derived without referencing the diagrams.

\begin{figure}
    \centering
    \begin{subfigure}[t]{.49\linewidth}
        \centering
        \includegraphics[width=\linewidth]{latex/figures/images/geoqa_example.pdf}
        \caption{GeoQA}
        \label{fig:geoqa_example}
    \end{subfigure}
    \begin{subfigure}[t]{.48\linewidth}
        \centering
        \includegraphics[width=\linewidth]{latex/figures/images/pgps_example.pdf}
        \caption{PGPS9K}
        \label{fig:pgps9k_example}
    \end{subfigure}
    \caption{
    Examples of diagram-caption pairs and their solution steps written in formal languages from GeoQA and PGPS9k datasets. In the problem description, the visual geometric premises and numerical variables are highlighted in green and red, respectively. A significant difference in the style of the diagram and formal language can be observable. %, along with the differences in formal languages supported by the corresponding datasets.
    \label{fig:pgps_examples}
    }
\end{figure}



We propose a new benchmark created via a synthetic data engine, which systematically evaluates the ability of VLM vision encoders to recognize geometric premises. Our empirical findings reveal that previously suggested self-supervised learning (SSL) approaches, e.g., vector quantized variataional auto-encoder (VQ-VAE)~\citep{unimath} and masked auto-encoder (MAE)~\citep{scagps,geox}, and widely adopted encoders, e.g., OpenCLIP~\citep{clip} and DinoV2~\citep{dinov2}, struggle to detect geometric features such as perpendicularity and degrees. 

To this end, we propose \geoclip{}, a model pre-trained on a large corpus of synthetic diagram–caption pairs. By varying diagram styles (e.g., color, font size, resolution, line width), \geoclip{} learns robust geometric representations and outperforms prior SSL-based methods on our benchmark. Building on \geoclip{}, we introduce a few-shot domain adaptation technique that efficiently transfers the recognition ability to real-world diagrams. We further combine this domain-adapted GeoCLIP with an LLM, forming a domain-agnostic VLM for solving PGPS tasks in MathVerse~\citep{mathverse}. 
%To accommodate diverse diagram styles and solution formats, we unify the solution program languages across multiple PGPS datasets, ensuring comprehensive evaluation. 

In our experiments on MathVerse~\citep{mathverse}, which encompasses diverse plane geometry tasks and diagram styles, our VLM with a domain-adapted \geoclip{} consistently outperforms both task-specific PGPS models and generalist VLMs. 
% In particular, it achieves higher accuracy on tasks requiring geometric-feature recognition, even when critical numerical measurements are moved from text to diagrams. 
Ablation studies confirm the effectiveness of our domain adaptation strategy, showing improvements in optical character recognition (OCR)-based tasks and robust diagram embeddings across different styles. 
% By unifying the solution program languages of existing datasets and incorporating OCR capability, we enable a single VLM, named \geovlm{}, to handle a broad class of plane geometry problems.

% Contributions
We summarize the contributions as follows:
We propose a novel benchmark for systematically assessing how well vision encoders recognize geometric premises in plane geometry diagrams~(\cref{sec:visual_feature}); We introduce \geoclip{}, a vision encoder capable of accurately detecting visual geometric premises~(\cref{sec:geoclip}), and a few-shot domain adaptation technique that efficiently transfers this capability across different diagram styles (\cref{sec:domain_adaptation});
We show that our VLM, incorporating domain-adapted GeoCLIP, surpasses existing specialized PGPS VLMs and generalist VLMs on the MathVerse benchmark~(\cref{sec:experiments}) and effectively interprets diverse diagram styles~(\cref{sec:abl}).

\iffalse
\begin{itemize}
    \item We propose a novel benchmark for systematically assessing how well vision encoders recognize geometric premises, e.g., perpendicularity and angle measures, in plane geometry diagrams.
	\item We introduce \geoclip{}, a vision encoder capable of accurately detecting visual geometric premises, and a few-shot domain adaptation technique that efficiently transfers this capability across different diagram styles.
	\item We show that our final VLM, incorporating GeoCLIP-DA, effectively interprets diverse diagram styles and achieves state-of-the-art performance on the MathVerse benchmark, surpassing existing specialized PGPS models and generalist VLM models.
\end{itemize}
\fi

\iffalse

Large language models (LLMs) have made significant strides in automated math word problem solving. In particular, their code-generation capabilities combined with proof assistants~\citep{lean,isabelle} help minimize computational errors~\citep{POT}, improve solution precision~\citep{autoformalization}, and offer rigorous feedback and evaluation~\citep{MATH}. Although LLMs excel in generating solution steps and correct answers for algebra and calculus~\citep{math_solving}, their uni-modal nature limits performance in domains like plane geometry, where both diagrams and text are vital.

Plane geometry problem solving (PGPS) tasks typically include diagrams and textual descriptions, requiring solvers to interpret premises from both sources. To facilitate automated solutions for these problems, several studies have introduced formal languages tailored for plane geometry to represent solution steps as a program with training datasets composed of diagrams, textual descriptions, and solution programs~\citep{geoqa,unigeo,intergps,pgps}. Building on these datasets, a number of PGPS specialized vision-language models (VLMs) have been developed so far~\citep{GOLD, LANS, geox}.

Most existing VLMs, however, fail to use diagrams when solving geometry problems. Well-known PGPS datasets such as GeoQA~\citep{geoqa}, UniGeo~\citep{unigeo}, and PGPS9K~\citep{pgps}, can be solved without accessing diagrams, as their problem descriptions often contain all geometric information. \cref{fig:pgps_examples} shows an example from GeoQA and PGPS9K datasets, where one can deduce the solution steps without knowing the diagrams. 
As a result, models trained on these datasets rely almost exclusively on textual information, leaving the vision encoder under-utilized~\citep{GOLD}. 
Consequently, the VLMs trained on these datasets cannot solve the plane geometry problem when necessary geometric properties or relations are excluded from the problem statement.

Some studies seek to enhance the recognition of geometric premises from a diagram by directly predicting the premises from the diagram~\citep{GOLD, intergps} or as an auxiliary task for vision encoders~\citep{geoqa,geoqa-plus}. However, these approaches remain highly domain-specific because the labels for training are difficult to obtain, thus limiting generalization across different domains. While self-supervised learning (SSL) methods that depend exclusively on geometric diagrams, e.g., vector quantized variational auto-encoder (VQ-VAE)~\citep{unimath} and masked auto-encoder (MAE)~\citep{scagps,geox}, have also been explored, the effectiveness of the SSL approaches on recognizing geometric features has not been thoroughly investigated.

We introduce a benchmark constructed with a synthetic data engine to evaluate the effectiveness of SSL approaches in recognizing geometric premises from diagrams. Our empirical results with the proposed benchmark show that the vision encoders trained with SSL methods fail to capture visual \geofeat{}s such as perpendicularity between two lines and angle measure.
Furthermore, we find that the pre-trained vision encoders often used in general-purpose VLMs, e.g., OpenCLIP~\citep{clip} and DinoV2~\citep{dinov2}, fail to recognize geometric premises from diagrams.

To improve the vision encoder for PGPS, we propose \geoclip{}, a model trained with a massive amount of diagram-caption pairs.
Since the amount of diagram-caption pairs in existing benchmarks is often limited, we develop a plane diagram generator that can randomly sample plane geometry problems with the help of existing proof assistant~\citep{alphageometry}.
To make \geoclip{} robust against different styles, we vary the visual properties of diagrams, such as color, font size, resolution, and line width.
We show that \geoclip{} performs better than the other SSL approaches and commonly used vision encoders on the newly proposed benchmark.

Another major challenge in PGPS is developing a domain-agnostic VLM capable of handling multiple PGPS benchmarks. As shown in \cref{fig:pgps_examples}, the main difficulties arise from variations in diagram styles. 
To address the issue, we propose a few-shot domain adaptation technique for \geoclip{} which transfers its visual \geofeat{} perception from the synthetic diagrams to the real-world diagrams efficiently. 

We study the efficacy of the domain adapted \geoclip{} on PGPS when equipped with the language model. To be specific, we compare the VLM with the previous PGPS models on MathVerse~\citep{mathverse}, which is designed to evaluate both the PGPS and visual \geofeat{} perception performance on various domains.
While previous PGPS models are inapplicable to certain types of MathVerse problems, we modify the prediction target and unify the solution program languages of the existing PGPS training data to make our VLM applicable to all types of MathVerse problems.
Results on MathVerse demonstrate that our VLM more effectively integrates diagrammatic information and remains robust under conditions of various diagram styles.

\begin{itemize}
    \item We propose a benchmark to measure the visual \geofeat{} recognition performance of different vision encoders.
    % \item \sh{We introduce geometric CLIP (\geoclip{} and train the VLM equipped with \geoclip{} to predict both solution steps and the numerical measurements of the problem.}
    \item We introduce \geoclip{}, a vision encoder which can accurately recognize visual \geofeat{}s and a few-shot domain adaptation technique which can transfer such ability to different domains efficiently. 
    % \item \sh{We develop our final PGPS model, \geovlm{}, by adapting \geoclip{} to different domains and training with unified languages of solution program data.}
    % We develop a domain-agnostic VLM, namely \geovlm{}, by applying a simple yet effective domain adaptation method to \geoclip{} and training on the refined training data.
    \item We demonstrate our VLM equipped with GeoCLIP-DA effectively interprets diverse diagram styles, achieving superior performance on MathVerse compared to the existing PGPS models.
\end{itemize}

\fi 

\section{Graph Learning with Pure LLMs}
In this section, we introduce how we could utilize pure LLMs for important real-world graph tasks including node classification and link prediction.

% \subsection{}

\subsection{Prompt Design}\label{sec:prompt_design}
% \subsubsection{Pipeline of Graph Encoding}

% \ym{This should be a part of "how we use LLMs for graphs", right?}

% \subsubsection{Experimental Pipeline for LLMs}
% \label{sec:Experiment Pipeline}
% We present the overall experimental pipeline for LLMs in Figure \ref{fig:The overall pipeline of our benchmark}, which consists of three main stages: graph encoding, off-the-shelf LLMs, and LLMs with instruction tuning. \ym{Off-the-shelf LLMs is not a "stage"? I feel these are not three "stages", rather, "Off-the-Shelf" and "Instruction Finetuning" are two ways to utilize the LLMs? }

% \ym{This part could be an independent component/subsection.}
% \paragraph{\textbf{Graph encoding}}
As shown in the graph encoding part of Figure \ref{fig:The overall pipeline of our benchmark}, we combine the original graph datasets with their corresponding raw text attributes to encode the graph into a format that LLMs can understand, i.e., prompts. The prompt formats required for node classification and link prediction differ based on the specific task.

\paragraph{\underline{Prompt formats for node classification}}
When designing the prompt formats, \cite{whenandwhy} considers three scenarios: using only the node own features, using 1-hop neighbor information, and using 2-hop neighbor information. When neighbor information is included, it also contains labels of the neighbor nodes, which significantly improves reasoning accuracy. However, label information provides too direct guidance for LLMs, potentially making the task too straightforward and limiting their ability to learn from the underlying graph structure itself. Therefore, in addition to the three prompt designs, we introduce two new formats by removing label information. Our five prompt formats are as follows:

%The target node is the focus of our classification task. Inspired by \cite{whenandwhy} and \cite{instructglm} \ym{How we are inspired by them? They already proposed all 5 prompt strategies? Are we different from them? If so, how?}, we use five distinct prompts, as outlined below:
\begin{enumerate}
    \item \textbf{ego}: Only the attribute of the target node is used.
    \item \textbf{1-hop w/o label}: The target node is described using both its node attributes and those of its 1-hop neighbors, without labels.
    \item \textbf{2-hop w/o label}: The description includes the attributes of the target node and those of its 2-hop neighbors, without labels.
    \item \textbf{1-hop w label}: Similar to \textbf{1-hop w/o label}, but the labels of 1-hop neighbors from the training set are included.
    \item \textbf{2-hop w label}: The labels of 2-hop neighbors from the training set are included.
\end{enumerate}
The detailed prompt structures can be found in Appendix \ref{sec:Prompt Formats for Node Classification}.

\paragraph{\underline{Prompt formats for link prediction}}
The goal of link prediction is to determine whether an edge exists between target node 1 and target node 2. We use two prompt formats: 1) \textbf{1-hop}: Both target nodes are described using their own node attributes and those of their 1-hop neighbors. 2) \textbf{2-hop}: Both target nodes are described using their own node attributes along with those of their 2-hop neighbors. It is important to note that, whether using 1-hop or 2-hop, the nodes at the ends of the link should not appear as neighbors of each other to avoid overly simplistic reasoning, ensuring that the LLM needs to perform more meaningful reasoning. The detailed prompt structures can be found in Appendix \ref{sec:Prompt Formats for Link Prediction}.

\subsection{Paradigm of Using LLMs for Graph Tasks}
As shown in Figure \ref{fig:The overall pipeline of our benchmark}, there are two ways to use LLMs: one is off-the-shelf LLMs, which refers to LLMs without parameter optimization, and the other is LLMs with instruction tuning. For our investigation, we adopt  open-source models Llama-3.2-3B-Instruct (Llama3B) \cite{touvron2023llama}, Llama-3.1-8B-Instruct (Llama8B) \cite{touvron2023llama}, and the closed-source Qwen-plus \cite{bai2023qwen}.

\paragraph{\textbf{Off-the-shelf LLMs}}
% In this part, we outline the process of querying LLMs. No additional modifications are made to the LLMs themselves; instead, we simply input the encoded graph prompts along with the corresponding questions. By comparing the LLM responses with the correct answers, we can assess its performance. 

In this part, we directly utilize LLMs for graph tasks by designing specific prompts to encode graph-related information. Without modifying the LLMs themselves, we input these structured prompts along with corresponding questions and evaluate the model’s performance by comparing its responses to the correct answers. In addition to plain prompt described in Section~\ref{sec:prompt_design}, we also explore commonly used prompt strategies such as Chain of Thought (CoT) \cite{CoT}, Build A Graph (BAG) \cite{nlgraph}, and in-context few-shot learning on more LLMs (e.g., Qwen-max \cite{bai2023qwen}, GPT-4o \cite{achiam2023gpt4}, and Deepseek V3 \cite{liu2024deepseek}) for node classification task. The results indicate that these prompt strategies show significant performance variation across different datasets and LLM sizes. They do not necessarily provide benefits in every case. Detailed experimental results and analysis are provided in Appendix \ref{sec:Comparison of Different LLMs on Node Classification}.


%Additionally, in Appendix \ref{sec:Comparison of Different LLMs on Node Classification}, we provide extended experiments that benchmark more LLMs (e.g., Qwen-max \cite{bai2023qwen}, GPT-4o \cite{achiam2023gpt4}, and Deepseek V3 \cite{liu2024deepseek}) and evaluate the performance of different prompt techniques (e.g., Chain of Thought (CoT) \cite{CoT}, Build A Graph (BAG) \cite{nlgraph}, and in-context few-shot learning) on graph tasks.
% Additionally, we explore commonly used prompt strategies such as Chain of Thought (CoT) \cite{CoT}, Build A Graph (BAG) \cite{nlgraph}, and in-context few-shot learning on more LLMs (e.g., Qwen-max \cite{bai2023qwen}, GPT-4o \cite{achiam2023gpt4}, and Deepseek V3 \cite{liu2024deepseek}) for node classification task. The results indicate that these prompt strategies show significant performance variation across different datasets and LLM sizes. They do not necessarily provide benefits in every case. Detailed experimental results and analysis are provided in Appendix \ref{sec:Comparison of Different LLMs on Node Classification}.


\paragraph{\textbf{LLMs with instruction tuning}}
%\ym{If space limits, we could move detailed description of how instruction tuning works "two models for link prediction" into appendix. } \ym{Also, it seems the 2 formats vs 9 formats is not extremely significant? It is somehow not very important to the main narrative? If so, we could move the 9 format (or 2 format) investigation to appendix. We can refer to the investigation in the main  content.  } \ym{When mentioning the different formats, motivate a bit on why we would like to investigate this. If we are not going to present 9 format, in the main text, we can summarize their results briefly in the main text. }
We conduct instruction tuning only on open-sourced models Llama3B and Llama8B. To accelerate the learning process, we adopt \cite{hu2021lora} and DeepSpeed \cite{rasley2020deepspeed}. We find that the number of training epochs has little impact on the final results. Therefore, to save time and computational resources, each model is tuned for only one epoch.
% In the right part of Figure \ref{fig:The overall pipeline of our benchmark}, we provide a brief overview of LLMs with instruction tuning. We perform instruction tuning on Llama3B and Llama8B across various datasets, using LoRA \cite{hu2021lora} and DeepSpeed \cite{rasley2020deepspeed} to accelerate model training, with each model trained for just one epoch. 
For node classification, we limit instruction tuning to three prompt formats: ego, 1-hop w/o label, and 2-hop w/o label.
%, as labels often carry strong semantic information that may lead the model to overly rely on them.

For link prediction, we aim to explore not only the impact of instruction tuning on LLMs’ reasoning performance but also the effect of prompt format diversity on LLM performance, a factor that has not been extensively examined in previous instruction tuning studies. This investigation could provide valuable insights for future prompt design. To achieve this, we set up two modes. The first mode uses the same prompt formats as in the testing phase (1-hop and 2-hop). The second mode introduces a more diverse set of prompt formats (9 formats), incorporating different question formulations and a wider range of neighbors. Detailed descriptions of the link prompt formats can be found in Appendix \ref{sec:Prompt Formats for Link Prediction}.

% \begin{figure*}[]
%   \centering
%   \includegraphics[width=1\linewidth]{figs/big_picture.pdf}
%   %\vspace{-20pt}
%   \caption{The overall experimental pipeline for LLMs. Graph encoding outlines how prompts for LLMs are generated. Off-the-shelf LLMs show the question-answering process with LLMs. LLMs with instruction tuning describe the process of fine-tuning LLMs specifically for graph tasks.}
%   \label{fig:The overall pipeline of our benchmark}
% \end{figure*}

\begin{figure*}[htbp]
  \centering
  \includegraphics[width=1\linewidth]{figs/big_picture.pdf}  % 设置宽度为页面宽度
  \caption{The overall experimental pipeline for LLMs. Graph encoding outlines how prompts for LLMs are generated. Off-the-shelf LLMs show the question-answering process with LLMs. LLMs with instruction tuning describe the process of fine-tuning LLMs specifically for graph tasks.}
  \label{fig:The overall pipeline of our benchmark}
\end{figure*}

\section{Comprehensive Benchmarking of LLMs for Graph Tasks}
\label{sec:Fair Benchmark}
% \ym{Instead of Fair, we may use "comprehensive"?}
%\ym{Are there any works focusing on utilizing LLMs for graph tasks? If so, we may cite and discuss them. Then introduce the limitations including the inconsistent settings, limited baselines xxx. } 
%\ym{We could make this part more concrete, i.e, clearly motivate which baselines are missing and why they are so important}

%\ym{We need to motivate from the beginning: There are many papers on xxx. However xxxx }
Existing studies \cite{tang2024graphgpt, zhao2023graphtext, li2024glbench} on LLMs for graphs often use different datasets, data processing techniques, and data splitting strategies. This lack of consistency makes it difficult to compare results directly, resulting in an incomplete understanding of the LLM performance on graph tasks.
%Previous studies often employ varying data preprocessing techniques and splits, making it challenging to fairly compare LLMs with other graph models \cite{li2024glbench} \ym{$\leftarrow$are we trying to argue this cited paper is with issues?}. 
Furthermore, many LLM-based graph models have been evaluated only against a limited set of baseline models. For example, \cite{yan2023comprehensive} and \cite{instructglm} focus on classic GNNs and graph transformer models, while \cite{tang2024graphgpt} only evaluates classic GNNs and graph SSL models. These studies overlook more advanced models in the graph domain, particularly foundational graph prompt models (e.g., OFA), which have become a recent hotspot in graph research due to their strong generalization and adaptability. This narrow focus fails to fully capture the strengths and weaknesses of LLMs in graph-related tasks. To address this gap, we provide a comprehensive benchmark of LLMs against a broader set of graph machine learning methods for node classification and link prediction based on a fair comparison environment.

%Previous studies often use varying data preprocessing techniques and splits, making it difficult to fairly compare LLMs with other graph models \cite{li2024glbench}. Moreover, many LLM-based graph models have only been tested against a limited set of baseline models, such as traditional GNNs \cite{yan2023comprehensive, instructglm, tang2024graphgpt}. This narrow focus fails to fully capture the strengths and weaknesses of LLMs in the context of graph tasks since it does not account for the diversity of graph learning methods available. To address this gap, we provide a comprehensive benchmark of LLMs against a broader set of graph machine learning methods for node classification and link prediction based on a fair comparison environment. 
%\ym{$\leftarrow$ Why so? What we should/would do?  } 

%Additionally, most research \cite{sun2025graphicl, nlgraph, fatemi2023talklikeagraph} focus on the reasoning capabilities of off-the-shelf LLMs, neglecting a thorough evaluation of how instruction tuning influences their performance on graph tasks. A few studies \cite{instructglm, tang2024graphgpt, zhao2023graphtext} have explored instruction tuning for graph tasks, but they tend to focus on its design and effectiveness \ym{What do you mean by its design and effectiveness?} rather than comparing the performance of off-the-shelf LLMs or evaluating how different prompt formats affect their outcomes \ym{We do not need to mention the prompt format as it is not very significant.}. Therefore, in this section, we examine how instruction tuning and different prompt formats affect LLM performance to provide a more complete comparison of their capabilities.

% Additionally, most benchmarking works focus on the reasoning capabilities of off-the-shelf LLMs, overlooking a comprehensive evaluation of how instruction tuning impacts their performance on graph tasks. Studies related to graph instruction tuning often emphasize the design of instruction formats rather than providing a comprehensive benchmarking against various models. \ym{Do they compare? This argument sounds somewhat weak.}. \ym{We need to have more clear motivations.} For instance, studies like \cite{sun2025graphicl, nlgraph, fatemi2023talklikeagraph} only benchmark off-the-shelf LLMs across different prompt formats and graph tasks, while works such as \cite{instructglm, tang2024graphgpt, zhao2023graphtext} propose instruction formats for fine-tuning LLMs but fail to conduct a comprehensive comparison of fine-tuned LLMs with comprehensive graph models \ym{Is it still the lack of baseline issue?}. Therefore, in this section, we will conduct a comprehensive comparison of LLMs with instruction tuning against other graph models.
% %LLMs with instruction tuning will take center stage, and we will conduct a comprehensive comparison of their performance against other graph models.


%Additionally, most research has focused on the reasoning capabilities of off-the-shelf LLMs, neglecting a thorough evaluation of how instruction tuning influences their performance on graph tasks \cite{sun2025graphicl, nlgraph, fatemi2023talklikeagraph}.  \ym{I think we mentioned that some of the papers have already been doing instruction tuning? } We also evaluate the impact of instruction tuning and various prompt formats on LLM performance.
%and present new research questions based on our findings.

\subsection{The Overall Setup} 
\label{sec:The Overall Setup}
This part outlines the overall setup of the benchmarking. We detail the baseline models, datasets, and evaluation metrics used for node classification and link prediction tasks.

\subsubsection{Baselines}
% \ym{In this part, we xxx}


%\ym{We could simplify this part and push some contents to the appendix if we need additional space.}
% For LLMs, we select the open-source models Llama-3.2-3B-Instruct (Llama3B) \cite{touvron2023llama}, Llama-3.1-8B-Instruct (Llama8B) \cite{touvron2023llama}, and the closed-source Qwen-plus \cite{bai2023qwen}. 
% \ym{$\leftarrow$ These are not baselines, right? These should be part of "LLMs for Graphs"}
%Instruction tuning is conducted using Llama3B and Llama8B. 
For baseline models, we offer a thorough comparison across six graph learning paradigms, including both traditional GNNs and more advanced models, ensuring a comprehensive evaluation of LLMs’ capabilities. The baseline models include the following categories:
% \begin{itemize}
%     \item \textbf{GNNs}: We use widely adopted supervised learning models such as GCN \cite{gcn}, GraphSAGE \cite{graphsage}, and GAT \cite{gat} all trained from scratch. These models are well-established for their effectiveness in graph-based tasks.

%     \item \textbf{Graph Self-Supervised Learning (SSL) Models}: We incorporate GraphCL \cite{you2020graphcl} and GraphMAE \cite{hou2022graphmae}, representing distinct self-supervised learning paradigms. 
%     %GraphCL employs contrastive learning by distinguishing augmented views of the same graph from others, while GraphMAE uses masked autoencoding, reconstructing masked graph components to learn node representations without requiring augmented views.

%     \item \textbf{Graph Transformer (GT) Models}: We choose Graphormer \cite{graphormer}, a transformer-based model designed specifically to handle graph-structured data.
%     %, enabling efficient processing and analysis of complex relational information.

%     \item \textbf{Foundational Graph Prompt Models}: We evaluate Prodigy \cite{huang2024prodigy} and OFA \cite{ofa}. They leverage graph prompts to enhance the performance of pre-trained models in downstream tasks.
%     %, demonstrating strong capabilities with effective knowledge transfer.

%     \item \textbf{LM-Augmented Graph Learning Models}: We choose GIANT \cite{giant} and TAPE \cite{tape}, which integrate LMs with graph learning. GIANT uses pre-trained LMs for node embeddings, while TAPE generates textual explanations to enhance node features.
%     %, improving performance in tasks where text and graph structures are closely related.

%     \item \textbf{LLM with Graph Projectors}: We include LLaGA \cite{chen2024llaga} due to its simplicity and effectiveness. LLaGA uses a projector to map graph structures into vector representations, enabling LLMs to perform reasoning tasks more effectively.
% \end{itemize}

\begin{enumerate}
\item \textbf{GNNs}: We choose GCN \cite{gcn}, GraphSAGE \cite{graphsage}, and GAT \cite{gat} all trained from scratch.
\item \textbf{Graph SSL Models}: GraphCL \cite{you2020graphcl} and GraphMAE \cite{hou2022graphmae} represent self-supervised learning paradigms.
\item \textbf{Graph Transformer Models}: We choose Graphormer \cite{graphormer}, a transformer-based model for graph-structured data.
\item \textbf{Foundational Graph Prompt Models}: Prodigy \cite{huang2024prodigy} and OFA \cite{ofa}, which enhance pre-trained models using graph prompts.
\item \textbf{LM-Augmented Graph Models}: GIANT \cite{giant} and TAPE \cite{tape} integrate LMs with graph learning for feature enhancement.
\item \textbf{LLM with Graph Projectors}: LLaGA \cite{chen2024llaga} uses a projector to map graph structures into vectors for improved reasoning.
\end{enumerate}
More details about baseline models can be found in Appendix \ref{sec:Baseline models}.

%We select these baseline models to ensure a comprehensive comparison across various graph learning paradigms. The chosen models represent key approaches in graph learning, covering both traditional and advanced methods. This wide coverage allows us to effectively evaluate LLMs, ensuring a well-rounded assessment of their capabilities.
%\ym{This could be moved in front of the baselines as it provides the motivation. Esepecially discuss and highlight the foundation models, which is important but ignored in the previous works.}
%We select these baseline models to provide a comprehensive comparison across various graph learning paradigms. Representing both traditional and advanced methods, they ensure a well-rounded evaluation of LLMs’ capabilities.

\subsubsection{Datasets}
For both node classification and link prediction, we use the Cora \cite{cora}, PubMed \cite{pubmed}, OGBN-ArXiv \cite{ogb}, and OGBN-Products \cite{ogb} datasets. For baseline models, we use their original node features (Appendix \ref{sec:Impacts of Different Node Feature Embedding Methods} discusses the impact of different node feature embedding methods). For LLMs, we preprocess the raw data to transform the node attributes into textual representations. Cora, PubMed, and OGBN-ArXiv belong to the citation domain, while OGBN-Products belongs to the e-commerce domain. Detailed descriptions of the datasets and their splitting methods can be found in Appendix \ref{sec:Datasets}.

\subsubsection{Evaluation Settings}
For both node classification and link prediction, we consistently use accuracy as the evaluation metric, the same as \cite{chen2024llaga} and \cite{instructglm}. In the case of link prediction, where the ratio of positive to negative samples in the test set is 1:1, accuracy is a suitable measure. To select the best model, we perform hyperparameter tuning, as different hyperparameters may cause model performance to vary across datasets. 
%\ym{The following is a bit confusing. Also, I feel such detailed contents could be moved to appendix.} To select the best model, we perform hyperparameter tuning, recognizing that different hyperparameters may lead to model preferences varying across datasets. Therefore, we compute the average validation performance across multiple datasets to identify the optimal configuration. 
Detailed experimental settings and the hyperparameter search ranges for each model are provided in Appendix \ref{sec:Detailed Experimental Settings}.




%We explore the reasoning capabilities of LLMs without parameter optimization on node classification and link prediction tasks, while also evaluating the impact of instruction tuning on their performance. We select Llama3B, Llama8B, and Qwen-plus as the LLMs. For comparison, we include a variety of baseline models such as GNNs, Graph SSL models, GT models, OFA, and LLaGA. Additional comparisons with other LLMs and the effect of different prompt techniques are provided in Appendix \ref{sec:Comparison of Different LLMs on Node Classification}.

% \subsubsection{Pipeline of Graph Encoding}

% \ym{This should be a part of "how we use LLMs for graphs", right?}

% % \subsubsection{Experimental Pipeline for LLMs}
% % \label{sec:Experiment Pipeline}
% % We present the overall experimental pipeline for LLMs in Figure \ref{fig:The overall pipeline of our benchmark}, which consists of three main stages: graph encoding, off-the-shelf LLMs, and LLMs with instruction tuning. \ym{Off-the-shelf LLMs is not a "stage"? I feel these are not three "stages", rather, "Off-the-Shelf" and "Instruction Finetuning" are two ways to utilize the LLMs? }

% % \ym{This part could be an independent component/subsection.}
% % \paragraph{\textbf{Graph encoding}}
% As shown in the graph encoding part of Figure \ref{fig:The overall pipeline of our benchmark}, we combine the original graph datasets with their corresponding raw text attributes to encode the graph into a format that LLMs can understand, i.e., prompts. The prompt formats required for node classification and link prediction differ based on the specific task.

% \paragraph{\underline{Prompt formats for node classification}}
% When designing the prompt formats, \cite{whenandwhy} considers three scenarios: using only the node own features, using 1-hop neighbor information, and using 2-hop neighbor information. \ym{Just describe, no need to use "however".} However, when neighbor information is included, it also contains the neighbor nodes’ labels, which significantly improves reasoning accuracy. \ym{Why we would like to investigate these prompts? Provide motivation.} In addition to the three prompt designs, we introduce two new formats by removing label information. Our five prompt formats are as follows:

% %The target node is the focus of our classification task. Inspired by \cite{whenandwhy} and \cite{instructglm} \ym{How we are inspired by them? They already proposed all 5 prompt strategies? Are we different from them? If so, how?}, we use five distinct prompts, as outlined below:
% \begin{enumerate}
%     \item \textbf{ego}: Only the node attribute of the target node is used for description.
%     \item \textbf{1-hop w/o label}: The target node is described using both its own node attributes and those of its 1-hop neighbors, without including labels.
%     \item \textbf{2-hop w/o label}: The description includes the attributes of the target node and those of its 2-hop neighbors, without labels.
%     \item \textbf{1-hop w label}: Similar to \textbf{1-hop w/o label}, but the labels of 1-hop neighbors from the training set are included.
%     \item \textbf{2-hop w label}: The labels of 2-hop neighbors from the training set are included.
% \end{enumerate}
% The detailed prompt structures can be found in Appendix \ref{sec:Prompt Formats for Node Classification}.

% \paragraph{\underline{Prompt formats for link prediction}}
% The goal of link prediction is to determine whether an edge exists between target node 1 and target node 2. We use two prompt formats: 1) \textbf{1-hop}: Both target nodes are described using their own node attributes and those of their 1-hop neighbors. 2) \textbf{2-hop}: Both target nodes are described using their own node attributes along with those of their 2-hop neighbors. It is important to note that, whether using 1-hop or 2-hop, the nodes at the ends of the link should not appear as neighbors of each other to avoid overly simplistic reasoning, ensuring that the LLM needs to perform more meaningful reasoning. The detailed prompt structures can be found in Appendix \ref{sec:Prompt Formats for Link Prediction}.

%\ym{The following is about the pipeline of using LLM. They are different from the contents above. We could separate them.}
%\subsubsection{Pipeline of Using LLMs}

% \subsubsection{Pipeline of Using LLMs}
% As shown in Figure \ref{fig:The overall pipeline of our benchmark}, there are two ways to use LLMs: one is off-the-shelf LLMs, which refers to LLMs without parameter optimization, and the other is LLMs with instruction tuning.

% \paragraph{\textbf{Off-the-shelf LLMs}}
% In this part, we outline the process of querying LLMs. No additional modifications are made to the LLMs themselves; instead, we simply input the encoded graph prompts along with the corresponding questions. By comparing the LLM responses with the correct answers, we can assess its performance. 
% %Additionally, in Appendix \ref{sec:Comparison of Different LLMs on Node Classification}, we provide extended experiments that benchmark more LLMs (e.g., Qwen-max \cite{bai2023qwen}, GPT-4o \cite{achiam2023gpt4}, and Deepseek V3 \cite{liu2024deepseek}) and evaluate the performance of different prompt techniques (e.g., Chain of Thought (CoT) \cite{CoT}, Build A Graph (BAG) \cite{nlgraph}, and in-context few-shot learning) on graph tasks.
% Additionally, we explore commonly used prompt strategies such as Chain of Thought (CoT) \cite{CoT}, Build A Graph (BAG) \cite{nlgraph}, and in-context few-shot learning on more LLMs (e.g., Qwen-max \cite{bai2023qwen}, GPT-4o \cite{achiam2023gpt4}, and Deepseek V3 \cite{liu2024deepseek}) for node classification task. The results indicate that these prompt strategies show significant performance variation across different datasets and LLM sizes. They do not necessarily provide benefits in every case. Detailed experimental results and analysis are provided in Appendix \ref{sec:Comparison of Different LLMs on Node Classification}.


% \paragraph{\textbf{LLMs with instruction tuning}}
% %\ym{If space limits, we could move detailed description of how instruction tuning works "two models for link prediction" into appendix. } \ym{Also, it seems the 2 formats vs 9 formats is not extremely significant? It is somehow not very important to the main narrative? If so, we could move the 9 format (or 2 format) investigation to appendix. We can refer to the investigation in the main  content.  } \ym{When mentioning the different formats, motivate a bit on why we would like to investigate this. If we are not going to present 9 format, in the main text, we can summarize their results briefly in the main text. }
% In the right part of Figure \ref{fig:The overall pipeline of our benchmark}, we provide a brief overview of LLMs with instruction tuning. We perform instruction tuning on Llama3B and Llama8B across various datasets, using LoRA \cite{hu2021lora} and DeepSpeed \cite{rasley2020deepspeed} to accelerate model training, with each model trained for just one epoch. For node classification, we limit instruction tuning to three prompt formats: ego, 1-hop w/o label, and 2-hop w/o label.
% %, as labels often carry strong semantic information that may lead the model to overly rely on them.

% For link prediction, we aim to explore not only the impact of instruction tuning on LLMs’ reasoning performance but also the effect of prompt format diversity on LLM performance, a factor that has not been extensively examined in previous instruction tuning studies. This investigation could provide valuable insights for future prompt design. To achieve this, we set up two modes. The first mode uses the same prompt formats as in the testing phase (1-hop and 2-hop). The second mode introduces a more diverse set of prompt formats (9 formats), incorporating different question formulations and a wider range of neighbors. Detailed descriptions of the link prompt formats can be found in Appendix \ref{sec:Prompt Formats for Link Prediction}.

%For link prediction, we set up two modes. The first mode uses the same prompt formats as in the testing phase (1-hop and 2-hop). The second mode introduces more diverse prompt formats (9 formats), incorporating different question formulations and a broader range of neighbors. Detailed descriptions of the link prompt formats can be found in Appendix \ref{sec:Prompt Formats for Link Prediction}. To ensure a balanced ratio of positive and negative samples (1:1) and prevent model bias, we generate two prompts for each node in the training set: one with a “yes” answer and one with a “no” answer.

%\xn{For continuous pertaining, if this is a new method, I think it should take a subsection, to describe this method in detail. eg., if it is an unsupervised learning method, how update the parameters? what is the loss function? what is the purpose of the pertaining and tuning process?why this pipeline can help the processing? If the whole pipeline can use formulations to be described can be better, but if no, I think the whole pipeline should be more clear, as the following has a case of impact of continuous training}


%\textbf{Continuous Pre-training (Con.PT)} consists of two key stages. First, a pre-trained model undergoes unsupervised learning on the target dataset. This phase is task-agnostic, meaning the model learns general graph representations rather than optimizing for the final task. Next, the model is instruction-tuned on a task that matches the inference objective. For example, the pre-training stage might involve link prediction, while the final inference task is node classification. This approach helps the model adapt to the dataset structure and mitigates the impact of limited labeled data, ultimately improving performance.


%After instruction tuning, the models enter the testing phase, following the same procedure as the “LLMs without parameter optimization”, but this time the LLMs being queried are tuned versions.


\begin{table}[htbp]
\centering
\caption{Performance of different models on node classification tasks. The background colors represent the top three values in each column, from dark to light.}
%Tuned Llama3B and Llama8B mean Llama3B and Llama8B after instruction tuning.
\label{tab:node_classification_results_LLM}
\scalebox{0.78}{%
\begin{tabular}{l c c c c c c}
\toprule
\rowcolor{gray!10}
\textbf{Model} & \textbf{Prompt} & \textbf{Cora} & \textbf{PubMed} & \textbf{ArXiv} & \textbf{Products} & \textbf{Avg}\\ 
\midrule
GCN & - & 88.19 & 88.00 & 69.90 & 82.30 & 82.10\\
GraphSAGE & - & 89.67 \cellcolor{cyan!100} & 89.02 & 71.35 & 82.89 & 83.23\\
GAT & - & 88.38 & 87.90 & 68.69 & 82.10 & 81.77\\
GraphCL & - & 83.58 & 82.86 & 67.87 & 80.20 & 78.63\\
GraphMAE& - & 75.98 & 82.82 & 65.54 & 77.32 & 75.42\\
Graphormer& - & 81.20 & 88.05 & 71.99 & 81.75 & 80.75\\
Prodigy& - & 77.32 & 83.6 & 70.86 & 80.01 & 77.95 \\
OFA& - & 78.31 & 78.56 & 73.92 & 83.12 & 78.48\\
GIANT& - & 89.1 & 90.48 & 74.41 \cellcolor{cyan!20} & 84.33 \cellcolor{cyan!50} & 84.58 \cellcolor{cyan!20}\\
TAPE& - & 88.12 & 91.92 & 73.99 & 83.11 & 84.29\\ 

LLaGA& - & 88.94 & 94.57 \cellcolor{cyan!50} & 76.25 \cellcolor{cyan!100} & 83.98 \cellcolor{cyan!20} & 85.94 \cellcolor{cyan!50}\\
\midrule

\multirow{5}{*}{Llama3B} & ego & 24.72 & 63.20 & 23.10 & 40.80  & 37.96\\
& 1-hop w/o label & 39.48 & 64.50 & 29.50 & 53.00  & 46.62\\
& 2-hop w/o label & 49.63 & 69.90 & 29.50 & 56.10  & 51.28\\
& 1-hop w label & 77.49 & 70.90 & 66.00 & 68.80  & 70.80\\
& 2-hop w label & 83.03 & 72.00 & 65.20 & 71.20  & 72.86\\
\midrule
\multirow{5}{*}{Llama8B} & ego & 43.39 & 77.80 & 59.35 & 50.12  & 54.02\\
& 1-hop w/o label & 58.35 & 73.07 & 61.85 & 59.85  & 63.28\\
& 2-hop w/o label & 62.84 & 83.29 & 68.33 & 59.60  & 68.52\\
& 1-hop w label & 82.97 & 81.55 & 68.08 & 71.07  & 75.92\\
& 2-hop w label & 84.79 & 82.54 & 64.09 & 77.06  & 77.12\\
\midrule
\multirow{5}{*}{Qwen-plus} & ego & 52.32 & 80.74 & 70.20 & 64.24  & 69.69\\
& 1-hop w/o label & 68.87 & 85.73 & 73.83 & 72.19  & 75.16\\
& 2-hop w/o label & 76.16 & 88.98 & 73.51 & 71.56  & 77.55\\
& 1-hop w label & 87.42 & 88.74 & 73.55 & 74.83  & 81.14\\
& 2-hop w label & 89.40 \cellcolor{cyan!50} & 90.73 & 74.28 & 78.81  & 83.31\\
\midrule
\multirow{3}{*}{tuned Llama3B} & ego & 67.08 & 89.28 & 66.58 & 65.59 & 72.13\\
& 1-hop w/o label & 82.04 & 90.02 & 71.32 & 73.07 & 79.11\\
& 2-hop w/o label & 85.04 & 91.52 & 72.82 & 77.89 & 81.82\\
\midrule
\multirow{3}{*}{tuned Llama8B} & ego & 77.31 & 92.36 & 65.59 & 73.74 & 78.38\\
& 1-hop w/o label & 84.54 & 93.90 \cellcolor{cyan!20} & 69.33 & 80.33 & 83.28 \\
& 2-hop w/o label & 89.67 \cellcolor{cyan!100} & 95.22 \cellcolor{cyan!100} & 76.01 \cellcolor{cyan!50} & 84.51 \cellcolor{cyan!100} & 86.35 \cellcolor{cyan!100}\\
\bottomrule
\end{tabular}
}
\end{table}

\subsection{Results and Analysis}\label{sec:benchmarking_results}
In this section, we present and analyze the performance of various models across node classification and link prediction tasks, providing insights into the strengths and weaknesses of LLMs.
%\ym{Just include one more sentence to introduce what we do in this section.}
\paragraph{\underline{Node classification}}
%Table \ref{tab:node_classification_results_LLM} summarizes the performance of each model across different datasets. \ym{We make the following observations from Table~\ref{tab:node_classification_results_LLM}}:  \ym{Organize the observations into a nice itemized list.} Classic GNNs show consistent accuracy, while GIANT and TAPE outperform standalone GNNs by leveraging language models for improved node representations. On the other hand, LLMs \ym{Is this for non-tuned version?} demonstrate improved performance with larger models, which can perform comparably to GNNs under certain node description prompts. The choice of prompts is crucial, with multiple-hop prompts yielding better results than simpler single-hop or ego-based prompts, indicating that LLMs benefit from richer graph context. Additionally, label information enhances classification performance by strengthening the model’s decision-making process, similar to in-context learning.

%LLaGA also performs well, likely due to its use of a graph projector that more effectively integrates graph information compared to natural language descriptions. For instruction-tuned LLMs, both Llama3B and Llama8B show notable improvements, especially with multiple-hop prompts. Tuned Llama8B achieves the highest average score of 86.35, surpassing LLaGA and setting a new benchmark for LLMs in node classification.


Table \ref{tab:node_classification_results_LLM} summarizes the performance across different datasets. We make the following observations:
\begin{itemize}

\item Classic GNNs show consistent accuracy, while GIANT and TAPE outperform them by using language models for improved node representations. Larger off-the-shelf LLMs perform comparably to GNNs under certain prompts, with multiple-hop prompts yielding better results than simpler prompts, indicating that LLMs benefit from richer graph context.

\item Label information improves performance by strengthening the model decision-making process, similar to in-context learning.

\item For instruction-tuned LLMs, both Llama3B and Llama8B show notable improvements, especially with multiple-hop prompts. Tuned Llama8B achieves the highest average score, surpassing LLaGA and setting a new benchmark in node classification.
\end{itemize}



\begin{table}[htbp]
\centering
\caption{LLM performance on link prediction.}
\label{tab:LLM performance on link prediction}
\scalebox{0.71}{%
\begin{tabular}{l c c c c c c}
\toprule
\rowcolor{gray!10}
\textbf{Models} & \textbf{Prompts} & \textbf{Cora} & \textbf{PubMed} & \textbf{ArXiv} & \textbf{Products} & \textbf{Avg} \\ 
\midrule

GCN &- & 87.78 & 86.22 & 90.34 & 89.75 & 88.52 \\
GraphSAGE &- & 84.39 & 78.81 & 92.98 & 92.98 & 87.29 \\
GAT &- & 86.88 & 82.81 & 83.33 & 85.57 & 84.65 \\
GraphCL &- & 92.98 & 93.76 & 90.85 & 94.21 & 92.95 \\
GraphMAE &- & 82.01 & 75.71 & 85.24 & 88.32 & 82.82 \\
Prodigy &- & 90.9 & 91.67 & 89.22 & 92.99 & 91.2\\
OFA &- & 94.19 & 98.05 & 95.84 \cellcolor{cyan!20} & 96.90 \cellcolor{cyan!20} & 96.25 \\
LLaGA &- & 87.01 & 90.10 & 93.88 & 95.67 & 91.67 \\
\hline

\multirow{2}{*}{Llama3B} & 1-hop & 72.97 & 71.55 & 72.45 & 78.92 & 73.97 \\
& 2-hop & 68.21 & 59.95 & 68.55 & 79.17 & 68.97 \\
\midrule
\multirow{2}{*}{Llama8B} & 1-hop & 80.44 & 74.80 & 87.80 & 85.29 & 82.08 \\
& 2-hop & 89.39 & 77.30 & 92.30 & 90.77 & 87.44 \\
\midrule
\multirow{2}{*}{Qwen-plus} & 1-hop & 78.81 & 91.74 & 81.82 & 88.42 & 85.20 \\
& 2-hop & 90.91 & 95.04 & 93.39 & 90.12 & 92.37 \\
\midrule
\multirow{2}{*}{tuned Llama3B (2 formats)} & 1-hop & 83.12 & 93.95 & 92.20 & 90.07 & 89.84 \\
& 2-hop & 95.76 \cellcolor{cyan!50} & 98.35 & 95.45 & 94.65 & 96.05 \\
\midrule
\multirow{2}{*}{tuned Llama3B (9 formats)} & 1-hop & 87.18 & 94.40 & 93.30 & 95.45 & 92.58 \\
& 2-hop & 95.94 \cellcolor{cyan!100} & 99.20 \cellcolor{cyan!100} & 95.42 & 97.84 \cellcolor{cyan!50} & 97.10 \cellcolor{cyan!100} \\
\midrule
\multirow{2}{*}{tuned Llama8B (2 formats)} & 1-hop & 88.65 & 95.12 & 93.65 & 93.23 & 92.66 \\
& 2-hop & 95.39 \cellcolor{cyan!20} & 98.77 \cellcolor{cyan!20} & 96.11 \cellcolor{cyan!100} & 94.92 & 96.30 \cellcolor{cyan!20} \\
\midrule
\multirow{2}{*}{tuned Llama8B (9 formats)} & 1-hop & 88.47 & 96.01 & 95.21 & 96.33 & 94.01 \\
& 2-hop & 95.15 & 99.20 \cellcolor{cyan!100} & 95.89 \cellcolor{cyan!50} & 97.98 \cellcolor{cyan!100} & 97.06 \cellcolor{cyan!50} \\
\bottomrule
\end{tabular}
}
\end{table}


\paragraph{\underline{Link prediction}}

%The results for link prediction are presented in Table \ref{tab:LLM performance on link prediction}. Among baseline models, GraphCL outperforms both GNNs and LLaGA, while GraphMAE shows the weakest performance despite being another Graph SSL model. This discrepancy likely stems from GraphCL’s use of edge permutation in contrastive learning, which enhances its understanding of graph structures, while GraphMAE focuses only on node features. Additionally, OFA stands out as the best-performing baseline likely due to its use of edge features derived from LLMs during pre-training. In contrast, Llama3B and Llama8B without parameter optimization underperform most baseline models. However, the larger Qwen-plus model matches or even surpasses the baseline models, highlighting the critical role of model size in LLMs’ ability to understand graphs.

%The best performance in link prediction is achieved by instruction-tuned LLMs. Models using 2-hop prompts consistently outperform those using 1-hop prompts, and tuning with a diverse set of 9 prompt formats leads to better results than tuning with just 2 formats. This demonstrates that both reasoning and tuning benefit from richer graph descriptions, significantly enhancing LLMs’ ability to capture and interpret graph information.


The results for link prediction are presented in Table \ref{tab:LLM performance on link prediction}. We make the following observations:
\begin{itemize}

\item Among baseline models, GraphCL outperforms both GNNs and LLaGA, while GraphMAE shows the weakest performance despite being another Graph SSL model. This discrepancy likely because GraphCL uses edge permutation in contrastive learning, which enhances its understanding of graph structures, while GraphMAE focuses only on node features. Additionally, OFA stands out as the best-performing baseline likely due to its use of edge features derived from LLMs during pre-training.

\item Off-the-shelf Llama3B and Llama8B underperform most baseline models. However, the larger Qwen-plus model matches or even surpasses the baseline models, highlighting the critical role of model size in LLMs’ ability to understand graphs.

\item The best performance in link prediction is achieved by instruction-tuned LLMs. Models using 2-hop prompts consistently outperform those using 1-hop prompts, and tuning with a diverse set of 9 prompt formats leads to better results than tuning with just 2 formats. This demonstrates that both reasoning and tuning benefit from richer graph descriptions, significantly enhancing LLMs’ ability to capture and interpret graph information.

\end{itemize}



%\ym{We can call the following as a "remark" instead of "observation". The previous discussions on the results are the observations. The following are further insights/conclusions we develop from the observations. }
%\paragraph{\textbf{Observation 1:}}
%\ym{For both tasks?}
%LLMs without parameter optimization generally perform worse than most baseline models. However, as the model size increases and graph structure information is incorporated, their reasoning ability significantly improves. In some cases, LLMs can achieve performance comparable to or even surpassing the best baseline models.

%\paragraph{\textbf{Observation 2:}}
%\ym{This may be merged together with the previous one.}
%Instruction tuning significantly boosts the performance of LLMs on graph tasks. Even with smaller model sizes, LLMs that undergo richer and more diverse instruction tuning can achieve performance on par with, or even better than, the best baseline models.
\begin{remark}
   Although smaller off-the-shelf LLMs underperform most baseline models, their reasoning ability improves significantly as the model size increases and graph structure information is incorporated. Instruction tuning further enhances LLM performance on graph tasks, with even smaller models achieving performance comparable to or better than the best baseline models, particularly when richer and more diverse instruction tuning is applied.
\end{remark}

\vspace{\baselineskip}
LLMs with instruction tuning have shown strong potential in graph tasks. However, this section only explores their performance in scenarios with sufficient data, while data scarcity is more common in real world applications. Therefore, in the next chapter, we will focus on exploring the performance of LLMs with instruction tuning in data-limited scenarios.




\begin{table*}[htbp]
\centering
\caption{Performance of models under few-shot learning. }
\label{tab:node_classification_results_few_shot_LLM}
\scalebox{0.73}{%
\begin{tabular}{l c c c c c c | c c c c c | c c c c c}
\toprule
\rowcolor{gray!10}
 & & \multicolumn{5}{c}{\textbf{Full fine-tune}} & \multicolumn{5}{c}{\textbf{5-shot}} & \multicolumn{5}{c}{\textbf{10-shot}} \\
\cline{3-7} \cline{8-12} \cline{13-17}
\rowcolor{gray!10}
\textbf{Models} & \textbf{Prompts} & \textbf{Cora} & \textbf{PubMed} & \textbf{ArXiv} & \textbf{Products} & \textbf{Avg} & \textbf{Cora} & \textbf{PubMed} & \textbf{ArXiv} & \textbf{Products} & \textbf{Avg} & \textbf{Cora} & \textbf{PubMed} & \textbf{ArXiv} & \textbf{Products} & \textbf{Avg} \\ 
\midrule

GCN &- & 88.19 & 88.00 & 69.90 & 82.30 \cellcolor{cyan!20} & 82.10 & 62.13 & 68.19 & 24.62 & 47.77 & 50.68 & 71.75 & 71.81 & 25.63 & 54.60 & 55.95 \\
GraphSAGE &- & 89.67 \cellcolor{cyan!100} & 89.02 & 71.35  \cellcolor{cyan!20} & 82.89 \cellcolor{cyan!50} & 83.23 \cellcolor{cyan!20} & 58.91 & 65.58 & 19.12 & 45.94 & 47.39 & 70.29 & 70.90 & 22.91 & 51.29 & 53.85 \\
GAT &- & 88.38 \cellcolor{cyan!20} & 87.90 & 68.69 & 82.10 & 81.77 & 54.95 & 63.95 & 19.08 & 32.65 & 42.66 & 69.26 & 70.60 & 25.34 & 43.59 & 52.20 \\
GraphCL &- & 83.58 & 82.86 & 67.87 & 80.20 & 78.63 & 54.03 & 54.86 & 11.24 & 34.10 & 38.56 & 57.96 & 55.23 & 16.84 & 46.08 & 44.03 \\
GraphMAE &- & 75.98 & 82.82 & 65.54 & 77.32 & 75.42 & 24.44 & 70.47 & 24.26 & 50.61 & 42.45 & 30.59 & 73.63 & 28.64 & 57.55 & 47.60 \\
All in one &- &- &- &- &- &- & 50.98 & 60.49 & 16.34 & 41.18 & 42.25 & 51.66 & 61.93 & 20.42 & 47.73 & 45.44 \\
GPF-plus &- &- &- &- &- &- & 67.00 & 66.91 & 60.07 & 64.50 & 64.62 & 73.22 & 64.39 & 65.35 & 68.02 & 67.75 \\
GraphPrompt &- &- &- &- &- &- & 65.12 & 68.11 & 81.88 \cellcolor{cyan!100} & 58.44 & 68.39 \cellcolor{cyan!20} & 69.81 & 70.38 & 87.05 \cellcolor{cyan!100} & 61.02 & 72.07 \\
\midrule
\multirow{3}{*}{Llama3B} & ego & 67.08 & 89.28 & 66.58 & 65.59 & 72.13 & 59.10 & 67.08 & 49.65 & 59.12 & 58.74 & 63.09 & 80.30 & 52.10 & 60.73 & 64.06 \\
& 1-hop w/o label & 82.04 & 90.02 & 71.32 & 73.07 & 79.11 & 74.81 \cellcolor{cyan!20} & 65.59 & 53.53 & 65.35 & 64.82 & 74.06 & 83.54 & 62.29 & 67.03 & 71.73 \\
& 2-hop w/o label & 85.04 & 91.52 & 72.82 \cellcolor{cyan!50} & 77.89 & 81.82 & 76.81 \cellcolor{cyan!50} & 71.32 & 55.24 & 67.32 \cellcolor{cyan!20} & 67.67 & 77.81 \cellcolor{cyan!20} & 85.53 \cellcolor{cyan!50} & 63.33 & 68.11 \cellcolor{cyan!20} & 73.70 \cellcolor{cyan!20} \\
\midrule
\multirow{3}{*}{Llama8B} & ego & 77.31 & 92.36 \cellcolor{cyan!20} & 65.59 & 73.74 & 78.38 & 65.84 & 76.81 \cellcolor{cyan!50} & 63.97 & 65.12 & 67.94 & 67.58 & 78.12 & 66.31 & 66.10 & 69.53 \\
& 1-hop w/o label & 84.54 & 93.90 \cellcolor{cyan!50} & 69.33 & 80.33 & 83.28 \cellcolor{cyan!50} & 74.56 & 76.81 \cellcolor{cyan!20} & 65.98 \cellcolor{cyan!20} & 70.50 \cellcolor{cyan!50} & 71.87 \cellcolor{cyan!50} & 79.55 \cellcolor{cyan!50} & 85.10 \cellcolor{cyan!20} & 68.24 \cellcolor{cyan!20} & 72.33 \cellcolor{cyan!50} & 76.31 \cellcolor{cyan!50} \\
& 2-hop w/o label & 89.67 \cellcolor{cyan!100} & 95.22 \cellcolor{cyan!100} & 76.01 \cellcolor{cyan!100} & 84.51 \cellcolor{cyan!100} & 86.35 \cellcolor{cyan!100} & 77.10 \cellcolor{cyan!100} & 79.43 \cellcolor{cyan!100} & 69.78 \cellcolor{cyan!50} & 73.12 \cellcolor{cyan!100} & 74.86 \cellcolor{cyan!100} & 80.55 \cellcolor{cyan!100} & 88.89 \cellcolor{cyan!100} & 71.12 \cellcolor{cyan!50} & 74.86 \cellcolor{cyan!100} & 78.86 \cellcolor{cyan!100} \\
\bottomrule
\end{tabular}
}
\end{table*}


% \section{Further Investigation }
\section{Further Investigation on LLMs with Instruction Tuning}
\label{sec:Research Questions}
%\ym{we can directly start with the instruction fine-tuning}
%Based on the results and analysis, it is clear that LLMs perform comparably to most graph machine learning methods when the model size is large and graph structural information is abundant. 
Instruction tuning enables even smaller LLMs to perform effectively. However, data scarcity remains a significant challenge in real-world applications \cite{xia2024opengraph}, where many graph models, such as GNNs and graph transformers, suffer a substantial performance drop due to their heavy reliance on rich graph structures and labeled data \cite{yu2024survey}. Recently, more advanced graph models like All in One \cite{sun2023allinone} and GPF-plus \cite{gpf-plus} have focused on improving performance with limited labeled data. Despite this, the performance of LLMs with instruction tuning under data scarcity has been relatively unexplored. Therefore, in this section, we discuss methods to alleviate data scarcity and further explore the performance of LLMs with instruction tuning in such scenarios.

%Based on the results and analysis, it is evident that LLMs perform comparably to most graph machine learning methods when model size is large and graph structural information is rich. Instruction tuning further enhances the performance of LLMs, enabling even smaller models to perform effectively. However, compared to most graph machine learning methods and LLMs without parameter optimization, instruction-tuned LLMs require more tuning time and computational resources. This raises an important question: \textit{\textbf{Are there unique advantages of instruction-tuned LLMs that cannot be easily replaced by other methods, or that justify the higher resource cost?}} \ym{This could be updated. How expensive is it compared with traditional GNN? We might need to be careful when mentioning its shortcomings.  }


Label scarcity is one of the most common forms of data scarcity. Improving model performance under few-shot learning conditions is a key focus for recent graph models \cite{yu2024survey, zhao2024pre}. For LLMs, the performance after few-shot instruction tuning offers insight into their sensitivity to label scarcity. The ability to generalize from a limited amount of labeled data is essential for LLMs’ adaptability across different tasks and domains, making it a crucial factor in practical applicability. This leads us to the following research question:

%The effectiveness of current graph representation learning techniques, such as GNNs and graph transformers, relies not only on rich graph structures but also on large amounts of labeled data \cite{yu2024survey}. However, data scarcity is a common challenge in real-world applications, which often limits their performance \cite{xia2024opengraph}. The ability to generalize from a small amount of labeled data is crucial for a model’s applicability across different tasks and domains. For LLMs, few-shot instruction tuning presents a clear advantage over full fine-tuning by significantly reducing resource requirements. This leads us to the following research question:


%In many real-world applications, data scarcity is a significant challenge, and a model’s performance in such scenarios directly impacts its practical applicability. \ym{Need some citations; also discuss about the graph case} The ability to generalize from a small amount of labeled data determines how widely a model can be used across different tasks and domains. For LLMs, few-shot instruction tuning offers a compelling advantage over full fine-tuning, as it greatly reduces resource consumption. This motivates us to investigate following research question:

\vspace{0.5\baselineskip}
\begin{mdframed}[backgroundcolor=gray!8]
\textbf{\textit{RQ1: How well do LLMs perform in few-shot instruction tuning scenarios?}}
\end{mdframed}
\vspace{0.5\baselineskip}


%When labeled data is scarce, the intuitive approach is to explore whether unlabeled data can be leveraged to enhance the model’s performance. \ym{need some citations to justify. Also, task about the graph case, make it specific to graph. We may mention that the idea of contious pre-training is widely seen in adapting a LLM to a general domain (cite). So, we xxx} This idea leads us to propose continuous pre-training, which utilizes unsupervised learning to improve a model’s understanding of graph structures before fine-tuning it for specific tasks. If effective, this approach could increase the adaptability of LLMs with instruction tuning since unlabeled data is far more abundant than labeled data. Given the potential of this method, we are interested in understanding how continuous pre-training influences the performance of LLMs in graph tasks. This leads us to explore the following research question:


% When labeled data is scarce, a natural strategy is to explore the potential of using unlabeled data to enhance model performance. This concept is often employed in the field of continual learning, where models are trained incrementally to adapt to new information without requiring large amounts of labeled data \cite{wang2024comprehensive, van2019three}. Building on this idea, we propose continuous pre-training, which utilizes unsupervised learning to improve the model understanding of graph structures before fine-tuning it for specific tasks. If effective, this approach could increase the adaptability of LLMs with instruction tuning since unlabeled data is far more abundant than labeled data. Given the potential of this method, we are interested in the following research question:

When labeled data is scarce, leveraging unlabeled data is a natural strategy to enhance model performance. This principle is widely applied in continual learning, where models are incrementally trained to adapt to new information without requiring extensive labeled supervision \cite{wang2024comprehensive, van2019three}. A well-established approach for adapting large language models (LLMs) to specific domains is continual domain-adaptive pre-training~\cite{ke2023continual,yildiz2024investigating}, where models are further trained on domain-specific corpora to improve their performance on downstream tasks. Inspired by this strategy, we propose continuous pre-training for graph tasks, where an LLM undergoes unsupervised pre-training on graph-structured data before fine-tuning on task-specific objectives. Since unlabeled graph data is far more abundant than labeled data, this method could significantly enhance the adaptability of LLMs when paired with instruction tuning. Given the potential of this approach, we seek to investigate the following research question:

%\paragraph{\textbf{RQ2: How does continuous pre-training impact the performance of LLMs in zero-shot and few-shot scenarios?}}

\vspace{0.5\baselineskip}
\begin{mdframed}[backgroundcolor=gray!8]
\textbf{\textit{RQ2: How does continuous pre-training impact the performance of LLMs?}}
\end{mdframed}
\vspace{0.5\baselineskip}


Models with strong transferability can help alleviate the performance degradation caused by label scarcity to some extent. For instance, when the target dataset contains few or no labeled data, we can first train a model on other datasets and then transfer the learned knowledge to the target dataset. LLMs have shown impressive transferability in natural language tasks \cite{du2024unlocking, ran2024alopex}, but their transferability in graph tasks has been less explored. If LLMs with instruction tuning can effectively transfer knowledge across different domains, the resource-intensive tuning process could be performed once, and subsequent graph tasks could be handled by the already fine-tuned models. This leads to the following research question:


%LLMs have demonstrated strong transferability in natural language tasks \cite{du2024unlocking, ran2024alopex}, but their transferability in graph tasks has been less explored, with most studies focusing primarily on in-domain tasks. \ym{Can explain it a bit more on what we mean by transferability for graph scenario and why we think they are important.} Transferability is crucial because it enables models to apply knowledge learned from one domain to new domains without the need for extensive retraining. If LLMs with instruction tuning can effectively transfer knowledge across domains, the resource-intensive tuning process would only need to be done once. Subsequent tasks could then be performed on the already tuned models, greatly expanding the applicability of LLMs for graph-related tasks. This leads to the following research question:

%\paragraph{\textbf{RQ3: How well do LLMs transfer knowledge across domains in node classification and link prediction?}}

\vspace{0.5\baselineskip}
\begin{mdframed}[backgroundcolor=gray!8]
\textbf{\textit{RQ3: How well do LLMs transfer knowledge across domains in node classification and link prediction?}}
\end{mdframed}
\vspace{0.5\baselineskip}

Missing attributes represent another form of data scarcity \cite{chen2024data}. In such cases, the model ability to understand graph structure becomes crucial. While much research has focused on the role of node attributes in graph tasks \cite{chen2024text, yan2023comprehensive}, less attention has been given to the LLM ability to learn and reason with graph structures without relying on node attributes. The structure of a graph is a fundamental distinguishing feature compared to natural language, and a model’s ability to comprehend these structures is vital for enhancing its performance on graph tasks. Given this, we are curious about:

%The results from Table \ref{tab:node_classification_results_LLM} and Table \ref{tab:LLM performance on link prediction} have demonstrated the importance of graph structure information for LLMs. While much research has focused on the role of node attributes in graph tasks \cite{chen2024text, yan2023comprehensive}, less attention has been given to LLMs’ ability to learn and reason with graph structures without relying on node attributes. The structure of a graph is one of its key distinguishing factors compared to natural language, and understanding these structures is essential for improving LLM performance on graph tasks. Given this, we are curious about:

%\paragraph{\textbf{RQ4: How effectively do LLMs learn and understand graph structures?}}
\vspace{0.4\baselineskip}
\begin{mdframed}[backgroundcolor=gray!8]
\textbf{\textit{RQ4: How effectively do LLMs learn and understand graph structures?}}
\end{mdframed}


% Finally, LLMs’ stability in graph tasks is critical for their adaptability in real-world applications, where graphs may experience structural changes like reduced node similarity or missing edges. Understanding how LLMs handle these perturbations is key to evaluating their robustness and potential for deployment in dynamic graph-based environments where structural stability cannot always be assured. This leads to the final research question:

% %\paragraph{\textbf{RQ5: How do LLMs perform under structural perturbations, such as reduced node similarity and missing edge information, compared to traditional graph models?}}

% \vspace{\baselineskip}
% \begin{mdframed}[backgroundcolor=gray!8]
% \textbf{\textit{RQ5: How do LLMs perform under structural perturbations compared to traditional graph models?}}
% \end{mdframed}
% % \vspace{\baselineskip}


\section{Experiment and  Analysis}
\label{sec:Empirical Studies}
In this section, we conduct empirical studies on different research questions proposed in Section \ref{sec:Research Questions}. 
%Due to space constraints, the results and discussion for RQ5 are included in Appendix \ref{sec:Robustness of LLMs}.
%Based on the configurations outlined in Section \ref{sec:The Overall Setup}, we investigate the performance of LLMs across various scenarios and compare them with the corresponding baseline models. 
In the following subsections, we first introduce the experimental settings for each RQ, followed by an analysis of the experimental results and key remarks.





\subsection{Few-Shot Instruction Tuning of LLMs (RQ1)}
\label{sec:Case 2}
%Real-world applications often face data scarcity, requiring models to perform well with limited data, which is the essence of few-shot learning. In this case, we investigate the performance of LLMs on the node classification task under few-shot settings.
We focus on few-shot instruction tuning for node classification. Link prediction requires predicting edges between nodes, relying on more complex structural dependencies that are harder to capture in a few-shot setting.
\subsubsection{Experiment Settings}
We use ego, 1-hop w/o label, and 2-hop w/o label as prompt formats and randomly select 5 or 10 target nodes per class for instruction tuning, corresponding to “n-ways-5-shots” and “n-ways-10-shots” learning. 
% For instance, in Cora, which has 7 classes, n-ways-5-shots would use 7 classes \texttimes 5 shots \texttimes 3 prompt formats as tuning samples.
For baseline models, in addition to GNNs and Graph SSL models, we also include models from foundational graph prompt approaches, including All in one \cite{sun2023allinone}, GPF-plus \cite{gpf-plus}, and GraphPrompt \cite{liu2023graphprompt}. The three models excel in few-shot scenarios, leveraging pre-trained knowledge and graph prompts to adapt quickly to new tasks with minimal labeled data.



\subsubsection{Results}
%We summarize the results in Table \ref{tab:node_classification_results_few_shot_LLM}. As seen, all models experience accuracy drops under few-shot learning compared to full fine-tuning, with GNNs and Graph SSL models showing the largest declines. Larger datasets like ArXiv and Products show a more significant accuracy drop for these models, while LLMs exhibit a more consistent performance across all datasets, indicating greater robustness in data-scarce scenarios. Foundational graph prompt models generally outperform GNNs and Graph SSL models in few-shot settings, likely because graph prompts help pre-trained models adapt more effectively to new data.

%Among all models, LLMs perform the best in few-shot learning, with the 2-hop prompt providing the highest accuracy. This suggests that incorporating more context improves performance. Notably, Llama8B achieves the highest classification accuracy in both 5-shot and 10-shot scenarios, highlighting LLMs’ ability to learn quickly from limited data and perform well in data-scarce environments.

Table \ref{tab:node_classification_results_few_shot_LLM} summarizes the results. All models experience a decline in accuracy under few-shot learning compared to full fine-tuning, with GNNs and Graph SSL models showing the largest drops, particularly in larger datasets like ArXiv and Products. In contrast, LLMs exhibit more consistent performance, indicating greater robustness in data-scarce scenarios. Notably, Llama8B achieves the highest classification accuracy in both 5-shot and 10-shot scenarios, showing LLMs’ ability to learn quickly from limited data.
%\ym{Please check these observations, some of them are repeated. Also, should we discuss a bit more on ego, 1-hop, 2-hop?}


% \paragraph{\textbf{Remark 2:}}
%Even though some GNNs training from scratch can achieve performance similar to tuned LLMs when trained on full data, LLMs clearly outperform all other models in few-shot scenarios. Only a few foundational graph prompt models manage to reach comparable performance to LLMs, but this is limited to a small number of datasets. This demonstrates that LLMs have a significant advantage when working with limited data. Foundational graph prompt models generally outperform GNNs and Graph SSL models in few-shot settings, likely because graph prompts help pre-trained models adapt more effectively to new data.
\begin{remark}
LLMs outperform all other models in few-shot scenarios. Only a few foundational graph prompt models achieve comparable results on Arxiv dataset, underscoring LLMs’ clear advantage in data-scarce situations.
%LLMs outperform all of them in few-shot scenarios. \ym{please adjust the following sentence.} Only a few foundational graph prompt models achieve comparable results, but this is limited to specific datasets. This highlights LLMs’ clear advantage in data-scarce situations.
\end{remark}



\begin{figure}[htbp]
  \centering
  \includegraphics[width=1\linewidth]{figs/conpt.pdf}
  %\vspace{-20pt}
  \caption{Impact of continuous pre-training on LLMs}
  \label{fig:Continuous Pre-Trainings}
\end{figure}

\subsection{Impact of Continuous Pre-training (RQ2)}
\label{sec:Case 3}
%\ym{I think we do both zero-shot and few-shot? Need to introduce them.  }
As we can see from Figure \ref{fig:The overall pipeline of our benchmark}, continuous pre-training (Con.PT) consists of two stages. First, a pre-trained model undergoes unsupervised learning on the target dataset. This phase is task-agnostic, meaning the model learns general graph representations rather than optimizing for the final task. Next, the model is instruction-tuned on a task that matches the inference objective.
\subsubsection{Experiment Settings}
%\ym{We only do experiments on node classification? Why? Need to describe and explain.}

In this experiment, we evaluate both zero-shot and few-shot node classification. For the zero-shot setting, we begin by performing continuous pre-training on the relevant dataset using link prediction, treating it as an unsupervised learning task. We then carry out zero-shot node classification based on this pre-training. The baseline models compared in this setup include LLaGA and ZeroG \cite{li2024zerog}, which is a foundational graph prompt model designed specifically for zero-shot scenarios. For the few-shot setting, we conduct few-shot instruction tuning on top of the link prediction task and compare the results with those from direct few-shot instruction tuning without the link prediction step.

%We begin by performing continuous pre-training on the corresponding dataset using link prediction. This can be viewed as an unsupervised learning task, providing a foundation for zero-shot node classification. The baseline models compared in this setup include LLaGA and ZeroG \cite{li2024zerog}, a foundational graph prompt model designed for zero-shot scenarios. Additionally, we conduct few-shot instruction tuning on top of the link prediction task and compare the results with those of direct few-shot instruction tuning without the link prediction step.



\begin{table}[htbp]
\centering
\caption{Performance of continuous pre-training for LLM. "w Con.Pt" means zero-shot inference after continuous pre-training. "w 5shot" means direct 5-shot instruction tuning without continuous pre-training. "w Con.PT \& 5shot" means 5-shot instruction tuning after continuous pre-training.}
\label{tab:continuous pre-training for LLM}
\scalebox{0.67}{%
\begin{tabular}{l c c c c c c}
\toprule
\rowcolor{gray!10}
\textbf{Models} & \textbf{Prompts} & \textbf{Cora} & \textbf{PubMed} & \textbf{ArXiv} & \textbf{Products} & \textbf{Avg} \\ 
\midrule
ZeroG &- & 68.61 & 78.77 & 70.50 \cellcolor{cyan!20} \cellcolor{cyan!20} & 55.23 & 68.28 \\
LLaGA &- & 22.03 & 55.92 & 21.15 & 38.90 & 34.50 \\
\midrule
\multirow{3}{*}{Llama3B} & ego & 24.72 & 63.20 & 23.10 & 40.80 & 37.96 \\
& 1-hop w/o label & 39.48 & 64.50 & 29.50 & 53.00 & 46.62 \\
& 2-hop w/o label & 49.63 & 69.90 & 29.50 & 56.10 & 51.28 \\
\midrule
\multirow{3}{*}{Llama3B w Con.PT} & ego & 48.63 & 49.38 & 14.21 & 41.40 & 38.41 \\
& 1-hop w/o label & 49.38 & 69.33 & 30.01 & 55.86 & 51.15 \\
& 2-hop w/o label & 55.36 & 75.56 & 33.54 & 57.01 & 55.37 \\
\midrule
\multirow{3}{*}{Llama3B w 5shot} & ego & 59.10 & 67.08 & 49.65 & 59.12 & 58.74 \\
& 1-hop w/o label & 74.81 & 65.59 & 53.53 & 65.35 & 64.82  \\
& 2-hop w/o label & 76.81 & 71.32 & 55.24 & 67.32 & 67.67  \\
\midrule
\multirow{3}{*}{Llama3B w Con.PT \& 5shot} & ego & 59.60 & 84.29 & 50.37 & 60.88 & 63.79  \\
& 1-hop w/o label & 75.08 & 85.04 & 53.12 & 66.09 & 69.83  \\
& 2-hop w/o label & 79.58 \cellcolor{cyan!100}  & 88.53 \cellcolor{cyan!50}  & 54.11 & 68.08 & 72.58  \\
\midrule
% \rowcolor{gray!10}
 % & & & & &  &  \\
 

\multirow{3}{*}{Llama8B} & ego & 43.39 & 77.80 & 59.35 & 50.12 & 54.02 \\
& 1-hop w/o label & 58.35 & 73.07 & 61.85 & 59.85 & 63.28 \\
& 2-hop w/o label & 62.84 & 83.29 & 68.33 & 59.60 & 68.52 \\
\midrule
\multirow{3}{*}{Llama8B w Con.PT} & ego & 52.13 & 65.32 & 60.71 & 55.22 & 58.35 \\
& 1-hop w/o label & 64.44 & 80.20 & 63.10 & 62.84 & 67.65 \\
& 2-hop w/o label & 70.82  & 86.96 \cellcolor{cyan!20}  & 71.34 \cellcolor{cyan!100}  & 63.20 & 73.08  \\
\midrule
\multirow{3}{*}{Llama8B w 5shot} & ego & 65.84 & 76.81 & 63.97 & 65.12 & 67.94  \\
& 1-hop w/o label & 74.56 & 76.45 & 65.98 & 70.50 & 71.87  \\
& 2-hop w/o label & 77.1 \cellcolor{cyan!20}0  & 79.43 & 69.78  & 73.12 \cellcolor{cyan!50}  & 74.86 \cellcolor{cyan!20}  \\
\midrule
\multirow{3}{*}{Llama8B w Con.PT \& 5shot} & ego & 68.33 & 86.88  & 63.23 & 66.44 & 71.22 \\
& 1-hop w/o label & 76.82 & 86.83 & 66.77  & 70.99 \cellcolor{cyan!20}  & 75.35 \cellcolor{cyan!50}  \\
& 2-hop w/o label & 78.12 \cellcolor{cyan!50}  & 89.03 \cellcolor{cyan!100}  & 71.01 \cellcolor{cyan!50}  & 74.69 \cellcolor{cyan!100}  & 78.21 \cellcolor{cyan!100}  \\
\bottomrule
\end{tabular}
}
\end{table}



\subsubsection{Results}
%The results are presented in Table \ref{tab:continuous pre-training for LLM}. We only show the results for the 5-shot scenario in the table. It shows that LLMs perform better after continuous pre-training compared to direct zero-shot and few-shot learning, indicating that continuous pre-training effectively enhances the model’s understanding of graphs. Additionally, in few-shot scenario, for the smaller datasets like Cora and PubMed, Llama3B with continuous pre-training matches or even surpasses the performance of Llama8B. However, for larger and more complex datasets like Arxiv and Products, Llama8B still maintains a clear advantage over Llama3B, even though the Llama3B has been updated by continuous pre-training. This suggests that increasing the size of the LLM remains the most effective approach for larger and more complex graphs. 

Table \ref{tab:continuous pre-training for LLM} presents the results. LLMs perform better after continuous pre-training compared to direct zero-shot and few-shot learning, demonstrating its effectiveness in enhancing the model understanding of graphs. For smaller datasets like Cora and PubMed, Llama3B with continuous pre-training matches or even surpasses Llama8B. However, for larger and more complex datasets like Arxiv and Products, Llama8B retains an advantage even after Llama3B undergoes continuous pre-training. This suggests that increasing the size of the LLM remains the most effective approach for larger and more complex graphs. 

To illustrate the impact of continuous pre-training more clearly, we present the average results over different datasets for Llama3B and Llama8B under the "2-hop w/o label" setting in Figure \ref{fig:Continuous Pre-Trainings}. As shown in the figure, continuous pre-training helps the LLMs develop a deeper understanding of the graph, leading to improved performance. Notably, Llama8B outperforms ZeroG that specifically designed for zero-shot tasks by a margin of 5\% after continuous pre-training.

\begin{remark}
Continuous pre-training can significantly improve LLM performance in zero-shot and few-shot learning. However, for larger and more complex datasets, increasing the size of the LLM proves to be a more effective approach.
\end{remark}
\subsection{Domain Transferability of LLMs (RQ3)}
% \subsection{LLM Transferability Across Domains (RQ3)}
\label{sec:Case 4}
%Domain transferability is a key capability for general models. In this case, we focus on evaluating the domain transferability of LLMs for node classification and link prediction tasks, considering both in-domain and cross-domain scenarios.
Domain transferability can be classified into in-domain and cross-domain transferability based on difficulty. The former refers to the ability to transfer knowledge between different datasets within the same domain, while the latter involves transferring knowledge across different domains. In this section, we explore the performance of LLMs with instruction tuning in both settings.


%\ym{We should introduce what we would like to do here: describe and motivate the experiments we would like to do: transferability across different datasets (in-domain vs cross-domain?)}
\subsubsection{Experiment Settings}
In the in-domain setup, we train the model on citation graphs (Arxiv) and evaluate it using Cora, another citation graph. For the cross-domain scenario, we train on Arxiv and test on Products, an e-commerce graph. GNNs rely on task-specific classification heads, which limits their ability to perform zero-shot learning on node classification tasks, particularly when label sets differ. Therefore, our comparison focuses on LLaGA for node classification. For link prediction, since the feature dimensions vary across datasets, we use a simple linear mapping to unify them. The baseline models include GNNs, Graph SSL models, and LLaGA.

\begin{figure}[htbp]
  \centering
  \includegraphics[width=1\linewidth]{figs/domain_transferability.pdf}
  %\vspace{-20pt}
  \caption{LLM domain transferability in node classification}
  \label{fig:Domain Transferability}
\end{figure}

\subsubsection{Results}
\paragraph{\underline{Node classification}}
%We present the results in Figure \ref{fig:Domain Transferability}, which shows the accuracy of different models in both in-domain and cross-domain scenarios. For node classification tasks, LLMs instruction tuned on Arxiv outperform those used for zero-shot prediction, though the improvement is modest. This is likely because category information is crucial in node classification. While the model learns the graph structure on Arxiv, it struggles to adapt to unseen categories in new datasets, limiting performance gains. In addition, On the Cora dataset, LLMs are comparable with LLaGA, but on the Products dataset, LLMs exhibit a clear edge. This suggests that LLaGA may struggle to capture and understand the full range of graph patterns because it only employs a simple graph projector. In contrast, LLMs can better adapt to the diversity of graph structures in more complex datasets like Products with their more sophisticated learning mechanisms.

Figure \ref{fig:Domain Transferability} presents the accuracy of different models in both in-domain and cross-domain scenarios. Instruction-tuned LLMs on Arxiv outperform off-the-shelf scenario, but the improvement is modest when additional structural information is incorporated. This is likely due to the fact that node classification relies heavily on category information, and adding more structural data does not significantly enhance performance. While LLMs learn graph information from Arxiv, adapting to unseen categories remains challenging, limiting performance gains. Besides, LLMs perform comparably to LLaGA on Cora dataset, but on the more complex Products dataset, LLMs show a clear advantage. 
%\ym{This is a very nice observation. Is it just graph patterns? Or also features? Do LLMs capture more diverse/general feature information?} %This suggests that LLaGA’s simple graph projector struggles to capture diverse graph patterns, whereas LLMs adapt better to varying structures with their more sophisticated learning mechanisms.
This suggests that the simple graph projector of LLaGA struggles to capture diverse graph patterns, while LLMs can adapt better to varying structures and are capable of learning diverse feature information with their sophisticated instruction tuning mechanisms.

\begin{table}[htbp]
\centering
\caption{LLM domain transferability in link prediction.}
\label{tab:LLM domain transferability}
\scalebox{0.85}{%
\begin{tabular}{l c c| c }
\toprule
\rowcolor{gray!10}
 & & \multicolumn{2}{c}{\textbf{Train $\longrightarrow$ Test}} \\
\cline{3-4}
\rowcolor{gray!10}
\textbf{Models} & \textbf{Prompts} & \textbf{Arxiv $\longrightarrow$ Cora} & \textbf{Arxiv $\longrightarrow$ Products} \\
\midrule
GCN &- & 55.54 & 67.07 \\
GraphSAGE &- & 50.00 & 51.11 \\
GAT &- & 85.41 & 71.18 \\
GraphCL &- & 78.30 & 82.62 \\
GraphMAE &- & 71.90 & 73.94 \\
LLaGA &- & 86.98 & 92.82 \cellcolor{cyan!20} \\
\midrule
\multirow{2}{*}{Llama3B} & 1-hop & 87.55 & 91.16 \\
& 2-hop & 95.11 \cellcolor{cyan!100} & 94.15 \cellcolor{cyan!50} \\
\midrule
\multirow{2}{*}{Llama8B} & 1-hop & 88.98 \cellcolor{cyan!20} & 91.97 \\
& 2-hop & 94.78 \cellcolor{cyan!50} & 95.43 \cellcolor{cyan!100} \\
\bottomrule
\end{tabular}
}
\end{table}

\paragraph{\underline{Link prediction}}
%From Table \ref{tab:LLM domain transferability}, we observe that LLMs demonstrate better domain transferability compared to baseline models. In the in-domain transfer scenario, LLMs achieve performance on the Cora dataset comparable to models directly instruction-tuned on Cora. This suggests that LLMs can effectively learn sufficient structural information from larger datasets within the same domain and transfer it to downstream tasks. In the cross-domain scenario, while the accuracy of LLMs on the Products dataset is slightly lower than that of direct tuning, it remains strong. This could be because datasets from different domains often share similar topological patterns. Among the baseline models, only LLaGA achieves performance comparable to LLMs, likely because it also leverages LLMs for predictions.

From Table \ref{tab:LLM domain transferability}, we observe that LLMs significantly outperform traditional graph models. Only LLaGA achieve comparable performance, likely because it also leverages LLMs for predictions. 
%From Table \ref{tab:LLM domain transferability}, we observe that LLMs outperform baseline models in domain transferability. \ym{The observations could be several level: 1. LLMs beat traditional GNN based methods 2. LLMs acehive comparable performance with instruction-fine tuned counterparts. } 
In the in-domain transfer scenario, LLMs achieve performance on Cora comparable to models directly instruction-tuned on Cora, indicating they can effectively transfer knowledge from larger datasets to downstream tasks. In the cross-domain scenario, although LLM performance on Products is slightly lower than direct tuning, it still remains strong, possibly due to shared topological patterns across domains.


%LLMs demonstrate domain transferability in both node classification and link prediction tasks, with more pronounced success in link prediction. While the transferability in node classification is limited, LLMs effectively learn and apply structural information for link prediction, showing strong generalization across different datasets.

%LLMs exhibit strong domain transferability, particularly in link prediction tasks, where they effectively generalize across different datasets. While their transferability in node classification is more limited due to the challenge of adapting to unseen categories\ym{nice argument, we could elaborate a bit. Link prediction tasks among different domains share more "similarity" compared with node classification? After all, link prediction is always about binary classification? }, they still outperform baseline models in both in-domain and cross-domain scenarios.
\begin{remark}
LLMs with instruction tuning exhibit strong domain transferability, particularly in link prediction tasks, where they effectively generalize across different datasets. This may be because link prediction tasks across domains share more similarities, as they can be viewed as binary classification problems. In contrast, node classification is more challenging, as adapting learned knowledge to unseen categories is difficult. %Nevertheless, LLMs still outperform baseline models in both in-domain \ym{It did not? for the in-domain case?} and cross-domain scenarios for node classification.
\end{remark}


\subsection{LLM Understanding of Graph Structures (RQ4)}
%\ym{Still need some introduction/descriotin/motivation here before we go to the settings.}
\label{sec:Case 5}
The structure of a graph sets it apart from natural language, and the model ability to comprehend these structures is vital for enhancing its performance on graph tasks. In this section, we explore the ability of instruction-tuned LLMs to understand graph structures.
%In the previous cases, we discussed LLMs’ potential to understand graph information. However, due to the influence of node attributes, it’s unclear to what extent LLMs truly grasp the underlying graph structures. In this part, we aim to explore their ability to comprehend graph structures more directly.

\subsubsection{Experiment Settings}
We remove all node attributes and retain only node IDs to eliminate the influence of attributes on LLM reasoning. Examples of these prompt formats are provided in Appendix \ref{sec:Prompt Formats for Pure Graph Structure}.


\begin{figure}[]
  \centering
  \includegraphics[width=1\linewidth]{figs/node_classification_pure_structure.pdf}
  %\vspace{-20pt}
  \caption{LLM performance on node classification without node attributes}
  \label{fig:LLM performance on node classification without node attributes}
\end{figure}

\subsubsection{Results}
\paragraph{\underline{Node classification}}
We present the results of node classification in Figure \ref{fig:LLM performance on node classification without node attributes}. “Original” refers to Llama3B or Llama8B without parameter optimization, while “1-hop” and “2-hop” correspond to 1-hop w/o label and 2-hop w/o label, respectively. From the figure, we observe that off-the-shelf LLMs perform similarly to random guessing in node classification. For instance, with 7 classes in Cora, the probability of random guessing correctly is 14.28\%, and the experimental results align closely with this probability. This is because LLMs struggle to make accurate predictions based purely on graph structure without semantic information. After instruction tuning, LLMs start to learn some graph structural information, leading to improved accuracy. However, the improvement is limited, likely because the classes in these datasets are strongly correlated with node features, and the graph structural differences between categories are minimal. This explains why simpler models like MLPs \cite{hu2021graph} and our ego prompt format perform relatively well, as they rely more on the node features than on the graph structure itself.

\begin{table}[htbp]
\centering
\caption{LLM performance on link prediction without node attributes. Llama3B w attributes and Llama8B w attributes are for comparison.}
\label{tab:LLM performance on link prediction without node attributes}
\scalebox{0.7}{%
\begin{tabular}{l c c c c c c}
\toprule
\rowcolor{gray!10}
\textbf{Models} & \textbf{Prompts} & \textbf{Cora} & \textbf{PubMed} & \textbf{ArXiv} & \textbf{Products} & \textbf{Avg} \\ 
\midrule
\multirow{2}{*}{Llama3B w attributes} & 1-hop & 72.97 & 71.55 & 72.45 & 78.92 & 73.97 \\
& 2-hop & 68.21 & 59.95 & 68.55 & 79.17 & 68.97 \\
\midrule

\multirow{2}{*}{Llama8B w attributes} & 1-hop & 80.44 & 74.80 & 87.80 & 85.29 & 82.08 \\
& 2-hop & 89.39 \cellcolor{cyan!20} & 77.30 & 92.30 \cellcolor{cyan!50} & 90.77 \cellcolor{cyan!20} & 87.44 \cellcolor{cyan!20} \\
\midrule
\multirow{2}{*}{Llama3B w/o attributes} & 1-hop & 66.61 & 55.44 & 64.94 & 78.47 & 66.37 \\
& 2-hop & 72.22 & 58.62 & 65.62 & 74.52 & 67.75 \\
\midrule
\multirow{2}{*}{Llama8B w/o attributes} & 1-hop & 63.19 & 55.81 & 68.62 & 81.30 & 67.23 \\
& 2-hop & 85.58 & 69.50 & 84.88 & 87.78 & 81.94 \\
\midrule
\multirow{2}{*}{tuned Llama3B w/o attributes} & 1-hop & 75.88 & 74.70 & 78.30 & 77.38 & 76.57 \\
& 2-hop & 93.20 \cellcolor{cyan!50} & 97.66 \cellcolor{cyan!100} & 89.00 & 94.09 \cellcolor{cyan!50} & 93.49 \cellcolor{cyan!50} \\
\midrule
\multirow{2}{*}{tuned Llama8B w/o attributes} & 1-hop & 85.15 & 78.81 \cellcolor{cyan!20} & 89.34 \cellcolor{cyan!20} & 87.98 & 85.32 \\
& 2-hop & 94.11 \cellcolor{cyan!100} & 97.44 \cellcolor{cyan!50} & 93.67 \cellcolor{cyan!100} & 94.54 \cellcolor{cyan!100} & 94.94 \cellcolor{cyan!100} \\
\bottomrule
\end{tabular}
}
\end{table}


\paragraph{\underline{Link prediction}}
From Table \ref{tab:LLM performance on link prediction without node attributes}, we observe that LLMs with node attributes outperform those without, highlighting the positive role of node attributes in LLM reasoning. However, after instruction tuning without node attributes, the LLMs show a significant improvement in link prediction accuracy. This demonstrates that LLMs can effectively learn and understand graph structures, achieving high link prediction accuracy even in the absence of node attributes.


\begin{remark}
LLMs can learn graph structures effectively through instruction tuning. While node attributes improve performance, LLMs can still achieve high accuracy in link prediction by leveraging structural information alone. However, the improvement in node classification is limited, likely because the classes are closely related to node features and the structural differences between categories are minimal.
\end{remark}

\paragraph{\textbf{Further Probing}}
In real-world applications, graphs may experience structural changes like reduced node similarity or missing edges. Understanding how LLMs handle these perturbations is key to evaluating their robustness and potential for deployment in dynamic graph-based environments where structural stability cannot always be assured. We discuss this aspect further in Appendix \ref{sec:Robustness of LLMs}.


%\ym{We could have an independent section "Discussion" to summarize our key insights and discuss their potential impact. Only do it when time allows.}
We present RiskHarvester, a risk-based tool to compute a security risk score based on the value of the asset and ease of attack on a database. We calculated the value of asset by identifying the sensitive data categories present in a database from the database keywords. We utilized data flow analysis, SQL, and Object Relational Mapper (ORM) parsing to identify the database keywords. To calculate the ease of attack, we utilized passive network analysis to retrieve the database host information. To evaluate RiskHarvester, we curated RiskBench, a benchmark of 1,791 database secret-asset pairs with sensitive data categories and host information manually retrieved from 188 GitHub repositories. RiskHarvester demonstrates precision of (95\%) and recall (90\%) in detecting database keywords for the value of asset and precision of (96\%) and recall (94\%) in detecting valid hosts for ease of attack. Finally, we conducted an online survey to understand whether developers prioritize secret removal based on security risk score. We found that 86\% of the developers prioritized the secrets for removal with descending security risk scores.



\clearpage
% \usepackage{placeins}
% \FloatBarrier
\bibliographystyle{ACM-Reference-Format}
\bibliography{KDD-survey/GFM}


% \newpage  
% \onecolumn
\appendix
% \AppendixTOC
% \section*{\appendixcontentsname}
% \addcontentsline{atc}{section}{\appendixcontentsname}
% \startcontents[appendix]
% \printcontents[appendix]{l}{1}{\noindent\rule{\textwidth}{0.4pt}\par} % 添加分隔线



\newpage

  
\subsection{Lloyd-Max Algorithm}
\label{subsec:Lloyd-Max}
For a given quantization bitwidth $B$ and an operand $\bm{X}$, the Lloyd-Max algorithm finds $2^B$ quantization levels $\{\hat{x}_i\}_{i=1}^{2^B}$ such that quantizing $\bm{X}$ by rounding each scalar in $\bm{X}$ to the nearest quantization level minimizes the quantization MSE. 

The algorithm starts with an initial guess of quantization levels and then iteratively computes quantization thresholds $\{\tau_i\}_{i=1}^{2^B-1}$ and updates quantization levels $\{\hat{x}_i\}_{i=1}^{2^B}$. Specifically, at iteration $n$, thresholds are set to the midpoints of the previous iteration's levels:
\begin{align*}
    \tau_i^{(n)}=\frac{\hat{x}_i^{(n-1)}+\hat{x}_{i+1}^{(n-1)}}2 \text{ for } i=1\ldots 2^B-1
\end{align*}
Subsequently, the quantization levels are re-computed as conditional means of the data regions defined by the new thresholds:
\begin{align*}
    \hat{x}_i^{(n)}=\mathbb{E}\left[ \bm{X} \big| \bm{X}\in [\tau_{i-1}^{(n)},\tau_i^{(n)}] \right] \text{ for } i=1\ldots 2^B
\end{align*}
where to satisfy boundary conditions we have $\tau_0=-\infty$ and $\tau_{2^B}=\infty$. The algorithm iterates the above steps until convergence.

Figure \ref{fig:lm_quant} compares the quantization levels of a $7$-bit floating point (E3M3) quantizer (left) to a $7$-bit Lloyd-Max quantizer (right) when quantizing a layer of weights from the GPT3-126M model at a per-tensor granularity. As shown, the Lloyd-Max quantizer achieves substantially lower quantization MSE. Further, Table \ref{tab:FP7_vs_LM7} shows the superior perplexity achieved by Lloyd-Max quantizers for bitwidths of $7$, $6$ and $5$. The difference between the quantizers is clear at 5 bits, where per-tensor FP quantization incurs a drastic and unacceptable increase in perplexity, while Lloyd-Max quantization incurs a much smaller increase. Nevertheless, we note that even the optimal Lloyd-Max quantizer incurs a notable ($\sim 1.5$) increase in perplexity due to the coarse granularity of quantization. 

\begin{figure}[h]
  \centering
  \includegraphics[width=0.7\linewidth]{sections/figures/LM7_FP7.pdf}
  \caption{\small Quantization levels and the corresponding quantization MSE of Floating Point (left) vs Lloyd-Max (right) Quantizers for a layer of weights in the GPT3-126M model.}
  \label{fig:lm_quant}
\end{figure}

\begin{table}[h]\scriptsize
\begin{center}
\caption{\label{tab:FP7_vs_LM7} \small Comparing perplexity (lower is better) achieved by floating point quantizers and Lloyd-Max quantizers on a GPT3-126M model for the Wikitext-103 dataset.}
\begin{tabular}{c|cc|c}
\hline
 \multirow{2}{*}{\textbf{Bitwidth}} & \multicolumn{2}{|c|}{\textbf{Floating-Point Quantizer}} & \textbf{Lloyd-Max Quantizer} \\
 & Best Format & Wikitext-103 Perplexity & Wikitext-103 Perplexity \\
\hline
7 & E3M3 & 18.32 & 18.27 \\
6 & E3M2 & 19.07 & 18.51 \\
5 & E4M0 & 43.89 & 19.71 \\
\hline
\end{tabular}
\end{center}
\end{table}

\subsection{Proof of Local Optimality of LO-BCQ}
\label{subsec:lobcq_opt_proof}
For a given block $\bm{b}_j$, the quantization MSE during LO-BCQ can be empirically evaluated as $\frac{1}{L_b}\lVert \bm{b}_j- \bm{\hat{b}}_j\rVert^2_2$ where $\bm{\hat{b}}_j$ is computed from equation (\ref{eq:clustered_quantization_definition}) as $C_{f(\bm{b}_j)}(\bm{b}_j)$. Further, for a given block cluster $\mathcal{B}_i$, we compute the quantization MSE as $\frac{1}{|\mathcal{B}_{i}|}\sum_{\bm{b} \in \mathcal{B}_{i}} \frac{1}{L_b}\lVert \bm{b}- C_i^{(n)}(\bm{b})\rVert^2_2$. Therefore, at the end of iteration $n$, we evaluate the overall quantization MSE $J^{(n)}$ for a given operand $\bm{X}$ composed of $N_c$ block clusters as:
\begin{align*}
    \label{eq:mse_iter_n}
    J^{(n)} = \frac{1}{N_c} \sum_{i=1}^{N_c} \frac{1}{|\mathcal{B}_{i}^{(n)}|}\sum_{\bm{v} \in \mathcal{B}_{i}^{(n)}} \frac{1}{L_b}\lVert \bm{b}- B_i^{(n)}(\bm{b})\rVert^2_2
\end{align*}

At the end of iteration $n$, the codebooks are updated from $\mathcal{C}^{(n-1)}$ to $\mathcal{C}^{(n)}$. However, the mapping of a given vector $\bm{b}_j$ to quantizers $\mathcal{C}^{(n)}$ remains as  $f^{(n)}(\bm{b}_j)$. At the next iteration, during the vector clustering step, $f^{(n+1)}(\bm{b}_j)$ finds new mapping of $\bm{b}_j$ to updated codebooks $\mathcal{C}^{(n)}$ such that the quantization MSE over the candidate codebooks is minimized. Therefore, we obtain the following result for $\bm{b}_j$:
\begin{align*}
\frac{1}{L_b}\lVert \bm{b}_j - C_{f^{(n+1)}(\bm{b}_j)}^{(n)}(\bm{b}_j)\rVert^2_2 \le \frac{1}{L_b}\lVert \bm{b}_j - C_{f^{(n)}(\bm{b}_j)}^{(n)}(\bm{b}_j)\rVert^2_2
\end{align*}

That is, quantizing $\bm{b}_j$ at the end of the block clustering step of iteration $n+1$ results in lower quantization MSE compared to quantizing at the end of iteration $n$. Since this is true for all $\bm{b} \in \bm{X}$, we assert the following:
\begin{equation}
\begin{split}
\label{eq:mse_ineq_1}
    \tilde{J}^{(n+1)} &= \frac{1}{N_c} \sum_{i=1}^{N_c} \frac{1}{|\mathcal{B}_{i}^{(n+1)}|}\sum_{\bm{b} \in \mathcal{B}_{i}^{(n+1)}} \frac{1}{L_b}\lVert \bm{b} - C_i^{(n)}(b)\rVert^2_2 \le J^{(n)}
\end{split}
\end{equation}
where $\tilde{J}^{(n+1)}$ is the the quantization MSE after the vector clustering step at iteration $n+1$.

Next, during the codebook update step (\ref{eq:quantizers_update}) at iteration $n+1$, the per-cluster codebooks $\mathcal{C}^{(n)}$ are updated to $\mathcal{C}^{(n+1)}$ by invoking the Lloyd-Max algorithm \citep{Lloyd}. We know that for any given value distribution, the Lloyd-Max algorithm minimizes the quantization MSE. Therefore, for a given vector cluster $\mathcal{B}_i$ we obtain the following result:

\begin{equation}
    \frac{1}{|\mathcal{B}_{i}^{(n+1)}|}\sum_{\bm{b} \in \mathcal{B}_{i}^{(n+1)}} \frac{1}{L_b}\lVert \bm{b}- C_i^{(n+1)}(\bm{b})\rVert^2_2 \le \frac{1}{|\mathcal{B}_{i}^{(n+1)}|}\sum_{\bm{b} \in \mathcal{B}_{i}^{(n+1)}} \frac{1}{L_b}\lVert \bm{b}- C_i^{(n)}(\bm{b})\rVert^2_2
\end{equation}

The above equation states that quantizing the given block cluster $\mathcal{B}_i$ after updating the associated codebook from $C_i^{(n)}$ to $C_i^{(n+1)}$ results in lower quantization MSE. Since this is true for all the block clusters, we derive the following result: 
\begin{equation}
\begin{split}
\label{eq:mse_ineq_2}
     J^{(n+1)} &= \frac{1}{N_c} \sum_{i=1}^{N_c} \frac{1}{|\mathcal{B}_{i}^{(n+1)}|}\sum_{\bm{b} \in \mathcal{B}_{i}^{(n+1)}} \frac{1}{L_b}\lVert \bm{b}- C_i^{(n+1)}(\bm{b})\rVert^2_2  \le \tilde{J}^{(n+1)}   
\end{split}
\end{equation}

Following (\ref{eq:mse_ineq_1}) and (\ref{eq:mse_ineq_2}), we find that the quantization MSE is non-increasing for each iteration, that is, $J^{(1)} \ge J^{(2)} \ge J^{(3)} \ge \ldots \ge J^{(M)}$ where $M$ is the maximum number of iterations. 
%Therefore, we can say that if the algorithm converges, then it must be that it has converged to a local minimum. 
\hfill $\blacksquare$


\begin{figure}
    \begin{center}
    \includegraphics[width=0.5\textwidth]{sections//figures/mse_vs_iter.pdf}
    \end{center}
    \caption{\small NMSE vs iterations during LO-BCQ compared to other block quantization proposals}
    \label{fig:nmse_vs_iter}
\end{figure}

Figure \ref{fig:nmse_vs_iter} shows the empirical convergence of LO-BCQ across several block lengths and number of codebooks. Also, the MSE achieved by LO-BCQ is compared to baselines such as MXFP and VSQ. As shown, LO-BCQ converges to a lower MSE than the baselines. Further, we achieve better convergence for larger number of codebooks ($N_c$) and for a smaller block length ($L_b$), both of which increase the bitwidth of BCQ (see Eq \ref{eq:bitwidth_bcq}).


\subsection{Additional Accuracy Results}
%Table \ref{tab:lobcq_config} lists the various LOBCQ configurations and their corresponding bitwidths.
\begin{table}
\setlength{\tabcolsep}{4.75pt}
\begin{center}
\caption{\label{tab:lobcq_config} Various LO-BCQ configurations and their bitwidths.}
\begin{tabular}{|c||c|c|c|c||c|c||c|} 
\hline
 & \multicolumn{4}{|c||}{$L_b=8$} & \multicolumn{2}{|c||}{$L_b=4$} & $L_b=2$ \\
 \hline
 \backslashbox{$L_A$\kern-1em}{\kern-1em$N_c$} & 2 & 4 & 8 & 16 & 2 & 4 & 2 \\
 \hline
 64 & 4.25 & 4.375 & 4.5 & 4.625 & 4.375 & 4.625 & 4.625\\
 \hline
 32 & 4.375 & 4.5 & 4.625& 4.75 & 4.5 & 4.75 & 4.75 \\
 \hline
 16 & 4.625 & 4.75& 4.875 & 5 & 4.75 & 5 & 5 \\
 \hline
\end{tabular}
\end{center}
\end{table}

%\subsection{Perplexity achieved by various LO-BCQ configurations on Wikitext-103 dataset}

\begin{table} \centering
\begin{tabular}{|c||c|c|c|c||c|c||c|} 
\hline
 $L_b \rightarrow$& \multicolumn{4}{c||}{8} & \multicolumn{2}{c||}{4} & 2\\
 \hline
 \backslashbox{$L_A$\kern-1em}{\kern-1em$N_c$} & 2 & 4 & 8 & 16 & 2 & 4 & 2  \\
 %$N_c \rightarrow$ & 2 & 4 & 8 & 16 & 2 & 4 & 2 \\
 \hline
 \hline
 \multicolumn{8}{c}{GPT3-1.3B (FP32 PPL = 9.98)} \\ 
 \hline
 \hline
 64 & 10.40 & 10.23 & 10.17 & 10.15 &  10.28 & 10.18 & 10.19 \\
 \hline
 32 & 10.25 & 10.20 & 10.15 & 10.12 &  10.23 & 10.17 & 10.17 \\
 \hline
 16 & 10.22 & 10.16 & 10.10 & 10.09 &  10.21 & 10.14 & 10.16 \\
 \hline
  \hline
 \multicolumn{8}{c}{GPT3-8B (FP32 PPL = 7.38)} \\ 
 \hline
 \hline
 64 & 7.61 & 7.52 & 7.48 &  7.47 &  7.55 &  7.49 & 7.50 \\
 \hline
 32 & 7.52 & 7.50 & 7.46 &  7.45 &  7.52 &  7.48 & 7.48  \\
 \hline
 16 & 7.51 & 7.48 & 7.44 &  7.44 &  7.51 &  7.49 & 7.47  \\
 \hline
\end{tabular}
\caption{\label{tab:ppl_gpt3_abalation} Wikitext-103 perplexity across GPT3-1.3B and 8B models.}
\end{table}

\begin{table} \centering
\begin{tabular}{|c||c|c|c|c||} 
\hline
 $L_b \rightarrow$& \multicolumn{4}{c||}{8}\\
 \hline
 \backslashbox{$L_A$\kern-1em}{\kern-1em$N_c$} & 2 & 4 & 8 & 16 \\
 %$N_c \rightarrow$ & 2 & 4 & 8 & 16 & 2 & 4 & 2 \\
 \hline
 \hline
 \multicolumn{5}{|c|}{Llama2-7B (FP32 PPL = 5.06)} \\ 
 \hline
 \hline
 64 & 5.31 & 5.26 & 5.19 & 5.18  \\
 \hline
 32 & 5.23 & 5.25 & 5.18 & 5.15  \\
 \hline
 16 & 5.23 & 5.19 & 5.16 & 5.14  \\
 \hline
 \multicolumn{5}{|c|}{Nemotron4-15B (FP32 PPL = 5.87)} \\ 
 \hline
 \hline
 64  & 6.3 & 6.20 & 6.13 & 6.08  \\
 \hline
 32  & 6.24 & 6.12 & 6.07 & 6.03  \\
 \hline
 16  & 6.12 & 6.14 & 6.04 & 6.02  \\
 \hline
 \multicolumn{5}{|c|}{Nemotron4-340B (FP32 PPL = 3.48)} \\ 
 \hline
 \hline
 64 & 3.67 & 3.62 & 3.60 & 3.59 \\
 \hline
 32 & 3.63 & 3.61 & 3.59 & 3.56 \\
 \hline
 16 & 3.61 & 3.58 & 3.57 & 3.55 \\
 \hline
\end{tabular}
\caption{\label{tab:ppl_llama7B_nemo15B} Wikitext-103 perplexity compared to FP32 baseline in Llama2-7B and Nemotron4-15B, 340B models}
\end{table}

%\subsection{Perplexity achieved by various LO-BCQ configurations on MMLU dataset}


\begin{table} \centering
\begin{tabular}{|c||c|c|c|c||c|c|c|c|} 
\hline
 $L_b \rightarrow$& \multicolumn{4}{c||}{8} & \multicolumn{4}{c||}{8}\\
 \hline
 \backslashbox{$L_A$\kern-1em}{\kern-1em$N_c$} & 2 & 4 & 8 & 16 & 2 & 4 & 8 & 16  \\
 %$N_c \rightarrow$ & 2 & 4 & 8 & 16 & 2 & 4 & 2 \\
 \hline
 \hline
 \multicolumn{5}{|c|}{Llama2-7B (FP32 Accuracy = 45.8\%)} & \multicolumn{4}{|c|}{Llama2-70B (FP32 Accuracy = 69.12\%)} \\ 
 \hline
 \hline
 64 & 43.9 & 43.4 & 43.9 & 44.9 & 68.07 & 68.27 & 68.17 & 68.75 \\
 \hline
 32 & 44.5 & 43.8 & 44.9 & 44.5 & 68.37 & 68.51 & 68.35 & 68.27  \\
 \hline
 16 & 43.9 & 42.7 & 44.9 & 45 & 68.12 & 68.77 & 68.31 & 68.59  \\
 \hline
 \hline
 \multicolumn{5}{|c|}{GPT3-22B (FP32 Accuracy = 38.75\%)} & \multicolumn{4}{|c|}{Nemotron4-15B (FP32 Accuracy = 64.3\%)} \\ 
 \hline
 \hline
 64 & 36.71 & 38.85 & 38.13 & 38.92 & 63.17 & 62.36 & 63.72 & 64.09 \\
 \hline
 32 & 37.95 & 38.69 & 39.45 & 38.34 & 64.05 & 62.30 & 63.8 & 64.33  \\
 \hline
 16 & 38.88 & 38.80 & 38.31 & 38.92 & 63.22 & 63.51 & 63.93 & 64.43  \\
 \hline
\end{tabular}
\caption{\label{tab:mmlu_abalation} Accuracy on MMLU dataset across GPT3-22B, Llama2-7B, 70B and Nemotron4-15B models.}
\end{table}


%\subsection{Perplexity achieved by various LO-BCQ configurations on LM evaluation harness}

\begin{table} \centering
\begin{tabular}{|c||c|c|c|c||c|c|c|c|} 
\hline
 $L_b \rightarrow$& \multicolumn{4}{c||}{8} & \multicolumn{4}{c||}{8}\\
 \hline
 \backslashbox{$L_A$\kern-1em}{\kern-1em$N_c$} & 2 & 4 & 8 & 16 & 2 & 4 & 8 & 16  \\
 %$N_c \rightarrow$ & 2 & 4 & 8 & 16 & 2 & 4 & 2 \\
 \hline
 \hline
 \multicolumn{5}{|c|}{Race (FP32 Accuracy = 37.51\%)} & \multicolumn{4}{|c|}{Boolq (FP32 Accuracy = 64.62\%)} \\ 
 \hline
 \hline
 64 & 36.94 & 37.13 & 36.27 & 37.13 & 63.73 & 62.26 & 63.49 & 63.36 \\
 \hline
 32 & 37.03 & 36.36 & 36.08 & 37.03 & 62.54 & 63.51 & 63.49 & 63.55  \\
 \hline
 16 & 37.03 & 37.03 & 36.46 & 37.03 & 61.1 & 63.79 & 63.58 & 63.33  \\
 \hline
 \hline
 \multicolumn{5}{|c|}{Winogrande (FP32 Accuracy = 58.01\%)} & \multicolumn{4}{|c|}{Piqa (FP32 Accuracy = 74.21\%)} \\ 
 \hline
 \hline
 64 & 58.17 & 57.22 & 57.85 & 58.33 & 73.01 & 73.07 & 73.07 & 72.80 \\
 \hline
 32 & 59.12 & 58.09 & 57.85 & 58.41 & 73.01 & 73.94 & 72.74 & 73.18  \\
 \hline
 16 & 57.93 & 58.88 & 57.93 & 58.56 & 73.94 & 72.80 & 73.01 & 73.94  \\
 \hline
\end{tabular}
\caption{\label{tab:mmlu_abalation} Accuracy on LM evaluation harness tasks on GPT3-1.3B model.}
\end{table}

\begin{table} \centering
\begin{tabular}{|c||c|c|c|c||c|c|c|c|} 
\hline
 $L_b \rightarrow$& \multicolumn{4}{c||}{8} & \multicolumn{4}{c||}{8}\\
 \hline
 \backslashbox{$L_A$\kern-1em}{\kern-1em$N_c$} & 2 & 4 & 8 & 16 & 2 & 4 & 8 & 16  \\
 %$N_c \rightarrow$ & 2 & 4 & 8 & 16 & 2 & 4 & 2 \\
 \hline
 \hline
 \multicolumn{5}{|c|}{Race (FP32 Accuracy = 41.34\%)} & \multicolumn{4}{|c|}{Boolq (FP32 Accuracy = 68.32\%)} \\ 
 \hline
 \hline
 64 & 40.48 & 40.10 & 39.43 & 39.90 & 69.20 & 68.41 & 69.45 & 68.56 \\
 \hline
 32 & 39.52 & 39.52 & 40.77 & 39.62 & 68.32 & 67.43 & 68.17 & 69.30  \\
 \hline
 16 & 39.81 & 39.71 & 39.90 & 40.38 & 68.10 & 66.33 & 69.51 & 69.42  \\
 \hline
 \hline
 \multicolumn{5}{|c|}{Winogrande (FP32 Accuracy = 67.88\%)} & \multicolumn{4}{|c|}{Piqa (FP32 Accuracy = 78.78\%)} \\ 
 \hline
 \hline
 64 & 66.85 & 66.61 & 67.72 & 67.88 & 77.31 & 77.42 & 77.75 & 77.64 \\
 \hline
 32 & 67.25 & 67.72 & 67.72 & 67.00 & 77.31 & 77.04 & 77.80 & 77.37  \\
 \hline
 16 & 68.11 & 68.90 & 67.88 & 67.48 & 77.37 & 78.13 & 78.13 & 77.69  \\
 \hline
\end{tabular}
\caption{\label{tab:mmlu_abalation} Accuracy on LM evaluation harness tasks on GPT3-8B model.}
\end{table}

\begin{table} \centering
\begin{tabular}{|c||c|c|c|c||c|c|c|c|} 
\hline
 $L_b \rightarrow$& \multicolumn{4}{c||}{8} & \multicolumn{4}{c||}{8}\\
 \hline
 \backslashbox{$L_A$\kern-1em}{\kern-1em$N_c$} & 2 & 4 & 8 & 16 & 2 & 4 & 8 & 16  \\
 %$N_c \rightarrow$ & 2 & 4 & 8 & 16 & 2 & 4 & 2 \\
 \hline
 \hline
 \multicolumn{5}{|c|}{Race (FP32 Accuracy = 40.67\%)} & \multicolumn{4}{|c|}{Boolq (FP32 Accuracy = 76.54\%)} \\ 
 \hline
 \hline
 64 & 40.48 & 40.10 & 39.43 & 39.90 & 75.41 & 75.11 & 77.09 & 75.66 \\
 \hline
 32 & 39.52 & 39.52 & 40.77 & 39.62 & 76.02 & 76.02 & 75.96 & 75.35  \\
 \hline
 16 & 39.81 & 39.71 & 39.90 & 40.38 & 75.05 & 73.82 & 75.72 & 76.09  \\
 \hline
 \hline
 \multicolumn{5}{|c|}{Winogrande (FP32 Accuracy = 70.64\%)} & \multicolumn{4}{|c|}{Piqa (FP32 Accuracy = 79.16\%)} \\ 
 \hline
 \hline
 64 & 69.14 & 70.17 & 70.17 & 70.56 & 78.24 & 79.00 & 78.62 & 78.73 \\
 \hline
 32 & 70.96 & 69.69 & 71.27 & 69.30 & 78.56 & 79.49 & 79.16 & 78.89  \\
 \hline
 16 & 71.03 & 69.53 & 69.69 & 70.40 & 78.13 & 79.16 & 79.00 & 79.00  \\
 \hline
\end{tabular}
\caption{\label{tab:mmlu_abalation} Accuracy on LM evaluation harness tasks on GPT3-22B model.}
\end{table}

\begin{table} \centering
\begin{tabular}{|c||c|c|c|c||c|c|c|c|} 
\hline
 $L_b \rightarrow$& \multicolumn{4}{c||}{8} & \multicolumn{4}{c||}{8}\\
 \hline
 \backslashbox{$L_A$\kern-1em}{\kern-1em$N_c$} & 2 & 4 & 8 & 16 & 2 & 4 & 8 & 16  \\
 %$N_c \rightarrow$ & 2 & 4 & 8 & 16 & 2 & 4 & 2 \\
 \hline
 \hline
 \multicolumn{5}{|c|}{Race (FP32 Accuracy = 44.4\%)} & \multicolumn{4}{|c|}{Boolq (FP32 Accuracy = 79.29\%)} \\ 
 \hline
 \hline
 64 & 42.49 & 42.51 & 42.58 & 43.45 & 77.58 & 77.37 & 77.43 & 78.1 \\
 \hline
 32 & 43.35 & 42.49 & 43.64 & 43.73 & 77.86 & 75.32 & 77.28 & 77.86  \\
 \hline
 16 & 44.21 & 44.21 & 43.64 & 42.97 & 78.65 & 77 & 76.94 & 77.98  \\
 \hline
 \hline
 \multicolumn{5}{|c|}{Winogrande (FP32 Accuracy = 69.38\%)} & \multicolumn{4}{|c|}{Piqa (FP32 Accuracy = 78.07\%)} \\ 
 \hline
 \hline
 64 & 68.9 & 68.43 & 69.77 & 68.19 & 77.09 & 76.82 & 77.09 & 77.86 \\
 \hline
 32 & 69.38 & 68.51 & 68.82 & 68.90 & 78.07 & 76.71 & 78.07 & 77.86  \\
 \hline
 16 & 69.53 & 67.09 & 69.38 & 68.90 & 77.37 & 77.8 & 77.91 & 77.69  \\
 \hline
\end{tabular}
\caption{\label{tab:mmlu_abalation} Accuracy on LM evaluation harness tasks on Llama2-7B model.}
\end{table}

\begin{table} \centering
\begin{tabular}{|c||c|c|c|c||c|c|c|c|} 
\hline
 $L_b \rightarrow$& \multicolumn{4}{c||}{8} & \multicolumn{4}{c||}{8}\\
 \hline
 \backslashbox{$L_A$\kern-1em}{\kern-1em$N_c$} & 2 & 4 & 8 & 16 & 2 & 4 & 8 & 16  \\
 %$N_c \rightarrow$ & 2 & 4 & 8 & 16 & 2 & 4 & 2 \\
 \hline
 \hline
 \multicolumn{5}{|c|}{Race (FP32 Accuracy = 48.8\%)} & \multicolumn{4}{|c|}{Boolq (FP32 Accuracy = 85.23\%)} \\ 
 \hline
 \hline
 64 & 49.00 & 49.00 & 49.28 & 48.71 & 82.82 & 84.28 & 84.03 & 84.25 \\
 \hline
 32 & 49.57 & 48.52 & 48.33 & 49.28 & 83.85 & 84.46 & 84.31 & 84.93  \\
 \hline
 16 & 49.85 & 49.09 & 49.28 & 48.99 & 85.11 & 84.46 & 84.61 & 83.94  \\
 \hline
 \hline
 \multicolumn{5}{|c|}{Winogrande (FP32 Accuracy = 79.95\%)} & \multicolumn{4}{|c|}{Piqa (FP32 Accuracy = 81.56\%)} \\ 
 \hline
 \hline
 64 & 78.77 & 78.45 & 78.37 & 79.16 & 81.45 & 80.69 & 81.45 & 81.5 \\
 \hline
 32 & 78.45 & 79.01 & 78.69 & 80.66 & 81.56 & 80.58 & 81.18 & 81.34  \\
 \hline
 16 & 79.95 & 79.56 & 79.79 & 79.72 & 81.28 & 81.66 & 81.28 & 80.96  \\
 \hline
\end{tabular}
\caption{\label{tab:mmlu_abalation} Accuracy on LM evaluation harness tasks on Llama2-70B model.}
\end{table}

%\section{MSE Studies}
%\textcolor{red}{TODO}


\subsection{Number Formats and Quantization Method}
\label{subsec:numFormats_quantMethod}
\subsubsection{Integer Format}
An $n$-bit signed integer (INT) is typically represented with a 2s-complement format \citep{yao2022zeroquant,xiao2023smoothquant,dai2021vsq}, where the most significant bit denotes the sign.

\subsubsection{Floating Point Format}
An $n$-bit signed floating point (FP) number $x$ comprises of a 1-bit sign ($x_{\mathrm{sign}}$), $B_m$-bit mantissa ($x_{\mathrm{mant}}$) and $B_e$-bit exponent ($x_{\mathrm{exp}}$) such that $B_m+B_e=n-1$. The associated constant exponent bias ($E_{\mathrm{bias}}$) is computed as $(2^{{B_e}-1}-1)$. We denote this format as $E_{B_e}M_{B_m}$.  

\subsubsection{Quantization Scheme}
\label{subsec:quant_method}
A quantization scheme dictates how a given unquantized tensor is converted to its quantized representation. We consider FP formats for the purpose of illustration. Given an unquantized tensor $\bm{X}$ and an FP format $E_{B_e}M_{B_m}$, we first, we compute the quantization scale factor $s_X$ that maps the maximum absolute value of $\bm{X}$ to the maximum quantization level of the $E_{B_e}M_{B_m}$ format as follows:
\begin{align}
\label{eq:sf}
    s_X = \frac{\mathrm{max}(|\bm{X}|)}{\mathrm{max}(E_{B_e}M_{B_m})}
\end{align}
In the above equation, $|\cdot|$ denotes the absolute value function.

Next, we scale $\bm{X}$ by $s_X$ and quantize it to $\hat{\bm{X}}$ by rounding it to the nearest quantization level of $E_{B_e}M_{B_m}$ as:

\begin{align}
\label{eq:tensor_quant}
    \hat{\bm{X}} = \text{round-to-nearest}\left(\frac{\bm{X}}{s_X}, E_{B_e}M_{B_m}\right)
\end{align}

We perform dynamic max-scaled quantization \citep{wu2020integer}, where the scale factor $s$ for activations is dynamically computed during runtime.

\subsection{Vector Scaled Quantization}
\begin{wrapfigure}{r}{0.35\linewidth}
  \centering
  \includegraphics[width=\linewidth]{sections/figures/vsquant.jpg}
  \caption{\small Vectorwise decomposition for per-vector scaled quantization (VSQ \citep{dai2021vsq}).}
  \label{fig:vsquant}
\end{wrapfigure}
During VSQ \citep{dai2021vsq}, the operand tensors are decomposed into 1D vectors in a hardware friendly manner as shown in Figure \ref{fig:vsquant}. Since the decomposed tensors are used as operands in matrix multiplications during inference, it is beneficial to perform this decomposition along the reduction dimension of the multiplication. The vectorwise quantization is performed similar to tensorwise quantization described in Equations \ref{eq:sf} and \ref{eq:tensor_quant}, where a scale factor $s_v$ is required for each vector $\bm{v}$ that maps the maximum absolute value of that vector to the maximum quantization level. While smaller vector lengths can lead to larger accuracy gains, the associated memory and computational overheads due to the per-vector scale factors increases. To alleviate these overheads, VSQ \citep{dai2021vsq} proposed a second level quantization of the per-vector scale factors to unsigned integers, while MX \citep{rouhani2023shared} quantizes them to integer powers of 2 (denoted as $2^{INT}$).

\subsubsection{MX Format}
The MX format proposed in \citep{rouhani2023microscaling} introduces the concept of sub-block shifting. For every two scalar elements of $b$-bits each, there is a shared exponent bit. The value of this exponent bit is determined through an empirical analysis that targets minimizing quantization MSE. We note that the FP format $E_{1}M_{b}$ is strictly better than MX from an accuracy perspective since it allocates a dedicated exponent bit to each scalar as opposed to sharing it across two scalars. Therefore, we conservatively bound the accuracy of a $b+2$-bit signed MX format with that of a $E_{1}M_{b}$ format in our comparisons. For instance, we use E1M2 format as a proxy for MX4.

\begin{figure}
    \centering
    \includegraphics[width=1\linewidth]{sections//figures/BlockFormats.pdf}
    \caption{\small Comparing LO-BCQ to MX format.}
    \label{fig:block_formats}
\end{figure}

Figure \ref{fig:block_formats} compares our $4$-bit LO-BCQ block format to MX \citep{rouhani2023microscaling}. As shown, both LO-BCQ and MX decompose a given operand tensor into block arrays and each block array into blocks. Similar to MX, we find that per-block quantization ($L_b < L_A$) leads to better accuracy due to increased flexibility. While MX achieves this through per-block $1$-bit micro-scales, we associate a dedicated codebook to each block through a per-block codebook selector. Further, MX quantizes the per-block array scale-factor to E8M0 format without per-tensor scaling. In contrast during LO-BCQ, we find that per-tensor scaling combined with quantization of per-block array scale-factor to E4M3 format results in superior inference accuracy across models. 
  % 引入附录文件

\end{document}
\endinput
%%
%% End of file `sample-authordraft.tex'.
