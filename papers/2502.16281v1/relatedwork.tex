\section{Related Work}
% Over the past decade, numerous research on mining information from graphs have shifted from traditional representation learning approaches \cite{perozzi2014deepwalk,grover2016node2vec,dong2017metapath2vec} to methods utilizing deep neural networks, including GNNs \cite{fan2019metapath,yan2021relation,zhang23pagelink,zhu23AutoAC,SHAN24KPI-HGNN,MaYLMC24HetGPT} and GCNs \cite{kipf2016semi,liu2023rhgn}. Inspired by the Transformer \cite{vaswani2017attention}, GAT \cite{velickovic2017graph} integrates the attention to aggregate node-level information in homogeneous networks, while HAN \cite{wang2019heterogeneous} introduces a two-level attention mechanism for node and semantic information in heterogeneous networks. MAGNN \cite{fu2020magnn}, MHGNN \cite{liang2022meta} and R-HGNN \cite{yu23RHGNN} proposed meta-path-based models to learn meta-path-based node embeddings. HetGNN \cite{zhang2019heterogeneous} and MEGNN \cite{chang2022megnn} take a meta-path-free approach to consider both structural and content information for each node jointly. HGT \cite{hu2020heterogeneous} incorporates information from high-order neighbors of different types through messages passing across ``soft'' meta-paths. MHGCN \cite{fu2023multiplex} captures local and global information by modeling the multiplex structures with depth and breadth behavior pattern aggregation. SeHGNN \cite{Yang23Simple} simplifies structural information capture by precomputing neighbor aggregation and incorporating a transformer-based semantic fusion module. HAGNN \cite{zhu2023hagnn} integrates meta-path-based intra-type aggregation and meta-path-free inter-type aggregation to generate the final embeddings. While existing methods have partially addressed heterogeneous graph embedding, none efficiently support embeddings based on user-specified ad-hoc meta-paths. FHGE fills this gap by enabling user-defined ad-hoc meta-paths to guide the embedding process in heterogeneous graphs.
% , providing a more flexible and customizable solution.