\begin{lemma}
    \label{lem:bounds_on_sum_step_sizes}
    Assume \Cref{ass:step_size}. Then the following bounds holds:
    \begin{enumerate}[(a)]
        \item \label{eq:sum_alpha_k_p} for any $p\geq 2$
        \begin{equation}
        \sum_{i=1}^{k}\alpha_i^p \leq\frac{c_0^p}{p\gamma-1}\eqsp,
        \end{equation}
        \item \label{eq:simple_bound_sum_alpha_k}
        for any $m\in\{0, \ldots, k\}$
        \begin{equation}
        \sum_{i=m+1}^k\alpha_i \geq  \frac{c_0}{2(1-\gamma)}((k+k_0)^{1-\gamma}-(m+k_0)^{1-\gamma})\eqsp,
        \end{equation}
    \end{enumerate}
\end{lemma}
\begin{proof}
    To proof \ref{eq:sum_alpha_k_p}, note that 
    \begin{equation}
        \sum_{i=1}^{k}\alpha_i^p \leq  c_{0}^p \int_{1}^{+\infty}\frac{\rmd x}{x^{p\gamma}} \leq \frac{c_0^p}{p\gamma-1}\eqsp,
    \end{equation}
    To proof \ref{eq:simple_bound_sum_alpha_k}, note that for any $i\geq 1$ we have $2(i+k_0)^{-\gamma} \geq (i+k_0-1)^{-\gamma}$. Hence,
    \begin{equation}
        \sum_{i=m+1}^k\alpha_i \geq \frac{1}{2}\sum_{i=m}^{k-1}\alpha_i \geq \frac{c_0}{2}\int_{m+k_0}^{k+k_0}\frac{\rmd x}{x^{\gamma}} = 
        \frac{c_0}{2(1-\gamma)}((k+k_0)^{1-\gamma}-(m+k_0)^{1-\gamma})\eqsp.
        \end{equation}
\end{proof}


\begin{lemma}
\label{lem:bound_sum_exponent}
    For any $A >0$, any $0 \leq i \leq n-1$  and any $\gamma\in(1/2, 1)$ it holds
   \begin{equation}
        \sum_{j=i}^{n-1}\exp\biggl\{-A(j^{1-\gamma} - i^{1-\gamma})\biggr\} \leq
        \begin{cases}
            1 + \exp\bigl\{\frac{1}{1-\gamma}\bigr\}\frac{1}{A^{1/(1-\gamma)}(1-\gamma)}\Gamma(\frac{1}{1-\gamma})\eqsp, &\text{ if } Ai^{1-\gamma} \leq \frac{1}{1-\gamma} \text{ and } i \geq 1\eqsp;\\
            1 + \frac{1}{A(1-\gamma)^2}i^\gamma\eqsp,  &\text{ if } Ai^{1-\gamma} >\frac{1}{1-\gamma} \text{ and } i \geq 1\eqsp;\\
            1 + \frac{1}{A^{1/(1-\gamma)}(1-\gamma)}\Gamma(\frac{1}{1-\gamma})\eqsp, &\text{ if } i=0 \eqsp. 
        \end{cases}
    \end{equation}
\end{lemma}
\begin{proof}
Note that 
\begin{align}
\sum_{j=i}^{n-1}\exp\biggl\{-A(j^{1-\gamma} - i^{1-\gamma})\biggr\} 
&\leq 1 +\exp\biggl\{A i^{1-\gamma}\biggr\}\int_{i}^{+\infty}\exp\biggl\{-Ax^{1-\gamma} \biggr\}\rmd x \\
&= 1 + \exp\biggl\{A i^{1-\gamma}\biggr\}\frac{1}{A^{1/(1-\gamma)}(1-\gamma)}\int_{Ai^{1-\gamma}}^{+\infty}\rme^{-u} u^{\frac{1}{1-\gamma}-1}\rmd u
\end{align}
      Applying \cite[Theorem 4.4.3]{gabcke2015neue}, we get 
    \begin{equation}
    \int_{Ai^{1-\gamma}}^{+\infty}\rme^{-u} u^{\frac{1}{1-\gamma}-1} \rmd u\leq
        \begin{cases}
            \Gamma(\frac{1}{1-\gamma})\eqsp, &\text{ if } Ai^{1-\gamma} < \frac{1}{1-\gamma};\\
            \frac{1}{1-\gamma}\exp\{-Ai^{1-\gamma}\} A^{\gamma/(1-\gamma)}i^\gamma\eqsp, &\text{ otherwise.}
        \end{cases}
    \end{equation}
    Combining inequities above, for $i \geq 1$ we obtain 
    \begin{equation}
        \sum_{j=i}^{n-1}\exp\biggl\{-A(j^{1-\gamma} - i^{1-\gamma})\biggr\} \leq
        \begin{cases}
            1 + \exp\bigl\{\frac{1}{1-\gamma}\bigr\}\frac{1}{A^{1/(1-\gamma)}(1-\gamma)}\Gamma(\frac{1}{1-\gamma})\eqsp, &\text{ if } Ai^{1-\gamma} < \frac{1}{1-\gamma};\\
            1 + \frac{1}{A(1-\gamma)^2}i^\gamma\eqsp, &\text{ otherwise.}
        \end{cases}
        \eqsp,
    \end{equation}
    and for $i=0$, we have
    \begin{equation}
        \sum_{j=0}^{n-1}\exp\biggl\{-A j^{1-\gamma} \biggr\} \leq 1 + \frac{1}{A^{1/(1-\gamma)}(1-\gamma)}\Gamma\biggl(\frac{1}{1-\gamma}\biggr)\eqsp.
    \end{equation}
\end{proof}


\begin{lemma}
\label{lem:bound_Q_i_and_Sigma_n}
Assume \Cref{ass:L-smooth}, and \Cref{ass:step_size}. Then for any $i \in \{0, \ldots, n-1\}$ it holds that
\begin{equation}
\lambda_{\max}(Q_i) \leq C_Q\eqsp,
\end{equation}
where the constant $C_Q$ is given by \begin{equation}
\label{eq:const_C_Q}
    C_Q =\biggl[1+\max\biggl(\exp\biggl\{\frac{1}{1-\gamma}\biggr\}\biggl(\frac{2(1-\gamma)}{\mu c_0}\biggr)^{1/(1-\gamma)}\frac{1}{1-\gamma}\Gamma(\frac{1}{1-\gamma}), \frac{2}{\mu c_0(1-\gamma)}\biggr)\biggr]c_0\eqsp.
\end{equation}
Moreover,
\begin{equation}
\label{eq:Q_i_lower_bound}
\lambda_{\min}(Q_i) \geq \frac{1}{L_1}(1-(1-\alpha_i L_1)^{n-i})\eqsp, \text{ and } \norm{\Sigma_n^{-1/2}} \leq C_{\Sigma}\eqsp,
\end{equation}
where 
\begin{equation}
\label{eq:def_C_Sigma}
C_{\Sigma} = \frac{\sqrt{2}L_1}{(1-\exp\{-\frac{\mu c_0L_1}{2(k_0+1)}\})\sqrt{\lambda_{\min}(\Sigma_{\xi})}}\eqsp.
\end{equation}
\end{lemma}
\begin{proof}
Note that using \Cref{lem:bounds_on_sum_step_sizes}\ref{eq:simple_bound_sum_alpha_k}, for $i \geq 0$, it holds that 
\begin{align}
\lambda_{\max}(Q_i) 
&\leq \alpha_i\sum_{j=i}^{n-1}\prod_{k=i+1}^{j}(1-\alpha_k \mu) \leq \alpha_i\sum_{j=i}^{n-1}\exp\biggl\{-\mu\sum_{k=i+1}^j\alpha_k\biggr\} \\
&\leq \alpha_i\sum_{j=i+k_0}^{n-1+k_0}\exp\biggl\{-\frac{\mu c_0}{2(1-\gamma)}(j^{1-\gamma}-(i+k_0)^{1-\gamma})\biggr\}\eqsp.
\end{align} 
Using \Cref{lem:bound_sum_exponent}, we complete the first part with the constant $C_Q$ defined in \eqref{eq:const_C_Q}. In order to prove \eqref{eq:Q_i_lower_bound}, we note that 
\begin{equation}
     \lambda_{\min}(Q_i) \geq \alpha_i\sum_{j=i}^{n-1}(1-\alpha_i L_1)^{j-i} =\frac{1}{L_1}(1-(1-\alpha_i L_1)^{n-i})\eqsp.
\end{equation}
Then for $i \leq n/2$, we have 
\begin{equation}
    \lambda_{\min}(Q_i) \geq \frac{1}{L_1}(1-(1-\alpha_i L_1)^{n/2})\geq \frac{1}{L_1}(1-\exp\{-\mu\alpha_i L_1 n/2\}) \geq \frac{1}{L_1}(1-\exp\{-\frac{\mu c_0L_1}{2(k_0+1)}\})\eqsp,
\end{equation}
where the last inequality holds, since $\alpha_i n \geq \alpha_n n \geq \frac{c_0 n}{k_0 + n} \geq \frac{c_0}{1+k_0}$ .
Combining previous inequalities, we finally get 
\begin{equation}
\lambda_{\min}(\Sigma_n) \geq \lambda_{\min}\left(\frac{1}{n}\sum_{i=1}^{n/2}Q_i\Sigma_{\xi}Q_i^\top\right) \geq \frac{\lambda_{\min}(\Sigma_{\xi})}{2 L_1^2}(1-\exp\{-\frac{\mu c_0L_1}{2(k_0+1)}\})^2\eqsp,
\end{equation}
and \eqref{eq:Q_i_lower_bound} follows. 
\end{proof}

\begin{lemma}
\label{lem:H_theta_bound}
    Assume \Cref{ass:L-smooth}, and \Cref{ass:hessian_Lipschitz_ball}. Then, it holds that
    \begin{equation}
        \norm{H(\theta)}\leq L_{H} \norm{\theta-\thetas}^2,
    \end{equation}
    where $L_{H}= \max(L_3,2L_1/\beta)$.
\end{lemma}
\begin{proof}
    From \Cref{ass:hessian_Lipschitz_ball} and definition of $H(\theta)$ in \eqref{eq:H_theta_def}, we obtain 
    \begin{equation}
        \| H(\theta) \| \mathds{1}(\norm{\theta-\thetas}\leq \beta)\leq L_3\norm{\theta-\thetas}^2.
    \end{equation}
   Since  $\mu \Id \preccurlyeq \nabla^2f(\theta) \preccurlyeq L_1 \Id$, we get that  
    \begin{equation}
        \norm{H(\theta)} \mathds{1}(\norm{\theta-\thetas} > \beta) \leq 2L_1\mathds{1}(\norm{\theta-\thetas} > \beta)\norm{\theta-\thetas} \leq \frac{2L_1}{\beta}\norm{\theta-\thetas}^2.
    \end{equation}
This concludes the proof.
\end{proof}