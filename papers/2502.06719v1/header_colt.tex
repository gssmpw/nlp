% \usepackage{amsfonts,amsmath,amssymb}
% \usepackage{xcolor}
% \usepackage{hyperref}
% \usepackage{srcltx}
% \usepackage{etoolbox}
% \usepackage{nameref}
% \usepackage{comment}
% \usepackage{mathrsfs}
% %\usepackage{bbm}
% \usepackage{enumerate}
% \usepackage[shortlabels]{enumitem}
% \usepackage{amsfonts} % blackboard math symbols
% \usepackage{nicefrac} % compact symbols for 1/2, etc.
% \usepackage{mathtools}
% \usepackage{dsfont,mathrsfs}
% \usepackage{cleveref}
% \usepackage{tikz-cd} 

% %\usepackage{caption}
% %\usepackage{subcaption}
% \usepackage{multirow}
% \usepackage{colortbl}



% \definecolor{bgcolor}{rgb}{0.8,1,1}
% \definecolor{bgcolor2}{rgb}{0.8,1,0.8}
% \definecolor{niceblue}{rgb}{0.0,0.19,0.56}
% \usepackage{threeparttable}
% %for coloured ticks and crosses
% \newcommand{\ag}[1]{{\color{orange}#1}}
% \newcommand{\myred}[1]{{\color{red}#1}}
% \newcommand{\myblue}[1]{{\color{blue}#1}}
% \usepackage{tcolorbox}
% \usepackage{pifont}
% \definecolor{mydarkgreen}{RGB}{39,130,67}
% \definecolor{mydarkred}{RGB}{192,47,25}
% \newcommand{\green}{\color{mydarkgreen}}
% \newcommand{\red}{\color{mydarkred}}
% \newcommand{\cmark}{{\green\ding{51}}}
% \newcommand{\xmark}{{\red\ding{55}}}

% \usepackage{xcolor}
% %\PassOptionsToPackage{linktocpage}{hyperref}
% %\usepackage[colorlinks=true, allcolors=blue]{hyperref}
% \definecolor{dmorange500}{HTML}{FF5F19}
% \definecolor{dmblue300}{HTML}{2267EB}
% \definecolor{dmred300}{HTML}{FF617B}
% \hypersetup{
% 	colorlinks,
% 	citecolor=dmblue300
% }

% %\usepackage[textwidth=2.25cm,textsize=footnotesize]{todonotes}
% %\usepackage[disable]{todonotes}
% %\newcommand{\eric}[1]{\todo{{\bf E:} #1}}
% %\newcommand{\erici}[1]{\todo[color=red!20,inline]{{\bf EM:} #1}}

% %\newcommand{\alain}[1]{\todo{{\bf AD:} #1}}
% %\newcommand{\alaini}[1]{\todo[color=red!20,inline]{{\bf AD:} #1}}

% %\newcommand{\alex}[1]{{\todo[color=blue!20]{{\bf A:} #1}}}
% %\newcommand{\alexi}[1]{\todo[color=blue!20,inline]{{\bf A:} #1}}

% % \newcommand{\som}[1]{\todo[color=green!20]{{\bf SS:} #1}}
% %\newcommand{\somi}[1]{\todo[color=green!20,inline]{{\bf SS:} #1}}

% %\usepackage{bm}
% \usepackage{xargs}
% \usepackage{multirow}

% \usepackage{aliascnt}
% \usepackage{autonum}
% \usepackage{upgreek}

% \newtheorem{property}{Fact}
% \newtheorem{assum}{A\hspace{-2pt}}
% \newtheorem{assumb}{B\hspace{-2pt}}

% \newtheorem{assumID}{IND\hspace{-2pt}}

% \newtheorem{assumUGE}{UGE\hspace{-2pt}}
% \newtheorem{assumUE}{UE\hspace{-2pt}}
% \newtheorem{assumSUP}{M\hspace{-2pt}}

% %\newtheorem{theorem}{Theorem}
% \crefname{theorem}{theorem}{Theorems}
% \Crefname{theorem}{Theorem}{Theorems}

% % Define a new counter for lemmas
% \newcounter{lemma}
% \renewcommand{\thelemma}{\arabic{lemma}}

% % Define the lemma environment
% \renewenvironment{lemma}[1][]{
%     \refstepcounter{lemma}
%     \noindent\textbf{Lemma~\thelemma. #1}\itshape
% }{\par}

% % Set up cleveref aliases
% \crefname{lemma}{lemma}{lemmas}
% \Crefname{lemma}{Lemma}{Lemmas}



% %\newaliascnt{remark}{theorem}
% %\newtheorem{remark}[remark]{Remark}
% %\aliascntresetthe{remark}
% \crefname{remark}{remark}{remarks}
% \Crefname{remark}{Remark}{Remarks}

% % Define a new counter for corollaries
% \newcounter{corollary}
% \renewcommand{\thecorollary}{\arabic{corollary}}

% % Define the corollary environment
% \renewenvironment{corollary}[1][]{
%     \refstepcounter{corollary}
%     \noindent\textbf{Corollary~\thecorollary. #1}\itshape
% }{\par}

% \crefname{corollary}{corollary}{corollaries}
% \Crefname{corollary}{Corollary}{Corollaries}

% %\newaliascnt{proposition}{theorem}
% %\newtheorem{proposition}{Proposition}
% %\aliascntresetthe{proposition}
% \crefname{proposition}{proposition}{propositions}
% \Crefname{proposition}{Proposition}{Propositions}

% %\newaliascnt{definition}{theorem}
% %\newtheorem{definition}[definition]{Definition}
% %\aliascntresetthe{definition}
% \crefname{definition}{definition}{definitions}
% \Crefname{Definition}{Definition}{Definitions}



% %\newtheorem{example}[theorem]{Example}
% \crefname{example}{example}{examples}
% \Crefname{Example}{Example}{Examples}


% \crefname{figure}{figure}{figures}
% \Crefname{Figure}{Figure}{Figures}

% \crefname{table}{table}{tables}
% \Crefname{Table}{Table}{Tables}


% \crefname{algorithm}{algorithm}{algorithms}
% \Crefname{Algorithm}{Algorithm}{Algorithms}



% \crefname{assum}{A\hspace{-3pt}}{A\hspace{-3pt}}
% \crefname{assumb}{B\hspace{-2pt}}{B\hspace{-2pt}}
% \crefname{assumUGE}{UGE\hspace{-1pt}}{UGE\hspace{-1pt}}
% \crefname{assumID}{IND\hspace{-1pt}}{IND\hspace{-1pt}}
% \crefname{assumUE}{UE\hspace{-1pt}}{UE\hspace{-1pt}}
% \crefname{assumM}{M\hspace{-1pt}}{M\hspace{-1pt}}

% \newlist{renumerate}{enumerate}{3}
% \setlist[renumerate]{wide, labelwidth=!, labelindent=0pt,label=(\roman*)}


% \newlist{aenumerate}{enumerate}{3}
% \setlist[aenumerate]{wide, labelwidth=!, labelindent=0pt,label=(\arabic*)}

% \newlist{aaenumerate}{enumerate}{3}
% \setlist[aaenumerate]{wide, labelwidth=!, labelindent=0pt,label=(\alph*)}

% \newlist{aenumerateSpace}{enumerate}{3}
% \setlist[aenumerateSpace]{wide, labelwidth=!,label=(\arabic*)}

% \newlist{benumerate}{enumerate}{3}
% \setlist[benumerate]{wide, labelwidth=!, labelindent=0pt,label=$\bullet$}

\usepackage{amsfonts,amsmath,amssymb}
\usepackage{xcolor}
\usepackage{hyperref}
\usepackage{srcltx}
\usepackage{etoolbox}
\usepackage{nameref}
\usepackage{comment}
\usepackage{mathrsfs}
%\usepackage{bbm}
\usepackage{enumerate}
\usepackage[shortlabels]{enumitem}
\usepackage{amsfonts} % blackboard math symbols
\usepackage{nicefrac} % compact symbols for 1/2, etc.
\usepackage{mathtools}
\usepackage{dsfont,mathrsfs}
\usepackage{cleveref}
\usepackage{tikz-cd} 

%\usepackage{caption}
%\usepackage{subcaption}
\usepackage{multirow}
\usepackage{colortbl}
\definecolor{bgcolor}{rgb}{0.8,1,1}
\definecolor{bgcolor2}{rgb}{0.8,1,0.8}
\definecolor{niceblue}{rgb}{0.0,0.19,0.56}
\usepackage{threeparttable}
%for coloured ticks and crosses
\newcommand{\ag}[1]{{\color{orange}#1}}
\newcommand{\myred}[1]{{\color{red}#1}}
\newcommand{\myblue}[1]{{\color{blue}#1}}
\usepackage{tcolorbox}
\usepackage{pifont}
\definecolor{mydarkgreen}{RGB}{39,130,67}
\definecolor{mydarkred}{RGB}{192,47,25}
\newcommand{\green}{\color{mydarkgreen}}
\newcommand{\red}{\color{mydarkred}}
\newcommand{\cmark}{{\green\ding{51}}}
\newcommand{\xmark}{{\red\ding{55}}}

\usepackage{xcolor}
%\PassOptionsToPackage{linktocpage}{hyperref}
%\usepackage[colorlinks=true, allcolors=blue]{hyperref}
\definecolor{dmorange500}{HTML}{FF5F19}
\definecolor{dmblue300}{HTML}{2267EB}
\definecolor{dmred300}{HTML}{FF617B}
\hypersetup{
	colorlinks,
	citecolor=dmblue300
}

%\usepackage[textwidth=2.25cm,textsize=footnotesize]{todonotes}
%\usepackage[disable]{todonotes}
%\newcommand{\eric}[1]{\todo{{\bf E:} #1}}
%\newcommand{\erici}[1]{\todo[color=red!20,inline]{{\bf EM:} #1}}

%\newcommand{\alain}[1]{\todo{{\bf AD:} #1}}
%\newcommand{\alaini}[1]{\todo[color=red!20,inline]{{\bf AD:} #1}}

%\newcommand{\alex}[1]{{\todo[color=blue!20]{{\bf A:} #1}}}
%\newcommand{\alexi}[1]{\todo[color=blue!20,inline]{{\bf A:} #1}}

% \newcommand{\som}[1]{\todo[color=green!20]{{\bf SS:} #1}}
%\newcommand{\somi}[1]{\todo[color=green!20,inline]{{\bf SS:} #1}}

%\usepackage{bm}
\usepackage{xargs}
\usepackage{multirow}

\usepackage{aliascnt}
\usepackage{autonum}
\usepackage{upgreek}

\newtheorem{property}{Fact}
\newtheorem{assum}{A\hspace{-2pt}}
\newtheorem{assumb}{B\hspace{-2pt}}

\newtheorem{assumID}{IND\hspace{-2pt}}

\newtheorem{assumUGE}{UGE\hspace{-2pt}}
\newtheorem{assumUE}{UE\hspace{-2pt}}
\newtheorem{assumSUP}{M\hspace{-2pt}}

%\newtheorem{theorem}{Theorem}
\crefname{theorem}{theorem}{Theorems}
\Crefname{theorem}{Theorem}{Theorems}

% Define a new counter for lemmas
\newcounter{lemma}
\renewcommand{\thelemma}{\arabic{lemma}}

% Define the lemma environment
\renewenvironment{lemma}[1][]{
    \refstepcounter{lemma}
    \noindent\textbf{Lemma~\thelemma. #1}\itshape
}{\par}

% Set up cleveref aliases
\crefname{lemma}{lemma}{lemmas}
\Crefname{lemma}{Lemma}{Lemmas}



%\newaliascnt{remark}{theorem}
%\newtheorem{remark}[remark]{Remark}
%\aliascntresetthe{remark}
\newcounter{remark}
\renewcommand{\theremark}{\arabic{remark}}

% Define the corollary environment
\renewenvironment{remark}[1][]{
    \refstepcounter{remark}
    \noindent\textbf{Remark~\theremark. #1}\itshape
}{\par}

\crefname{remark}{remark}{remarks}
\Crefname{remark}{Remark}{Remarks}

% Define a new counter for corollaries
\newcounter{corollary}
\renewcommand{\thecorollary}{\arabic{corollary}}

% Define the corollary environment
\renewenvironment{corollary}[1][]{
    \refstepcounter{corollary}
    \noindent\textbf{Corollary~\thecorollary. #1}\itshape
}{\par}

\crefname{corollary}{corollary}{corollaries}
\Crefname{corollary}{Corollary}{Corollaries}

%\newaliascnt{proposition}{theorem}
%\newtheorem{proposition}{Proposition}
%\aliascntresetthe{proposition}
% Define a new counter for corollaries
\newcounter{proposition}
\renewcommand{\theproposition}{\arabic{proposition}}

% Define the corollary environment
\renewenvironment{proposition}[1][]{
    \refstepcounter{proposition}
    \newline\textbf{Proposition~\theproposition. #1}\itshape
}{\par}

\crefname{proposition}{proposition}{propositions}
\Crefname{proposition}{Proposition}{Propositions}

%\newaliascnt{definition}{theorem}
%\newtheorem{definition}[definition]{Definition}
%\aliascntresetthe{definition}
\crefname{definition}{definition}{definitions}
\Crefname{Definition}{Definition}{Definitions}



%\newtheorem{example}[theorem]{Example}
\crefname{example}{example}{examples}
\Crefname{Example}{Example}{Examples}


\crefname{figure}{figure}{figures}
\Crefname{Figure}{Figure}{Figures}

\crefname{table}{table}{tables}
\Crefname{Table}{Table}{Tables}


\crefname{algorithm}{algorithm}{algorithms}
\Crefname{Algorithm}{Algorithm}{Algorithms}



\crefname{assum}{A\hspace{-3pt}}{A\hspace{-3pt}}
\crefname{assumb}{B\hspace{-2pt}}{B\hspace{-2pt}}
\crefname{assumUGE}{UGE\hspace{-1pt}}{UGE\hspace{-1pt}}
\crefname{assumID}{IND\hspace{-1pt}}{IND\hspace{-1pt}}
\crefname{assumUE}{UE\hspace{-1pt}}{UE\hspace{-1pt}}
\crefname{assumM}{M\hspace{-1pt}}{M\hspace{-1pt}}

\newlist{renumerate}{enumerate}{3}
\setlist[renumerate]{wide, labelwidth=!, labelindent=0pt,label=(\roman*)}


\newlist{aenumerate}{enumerate}{3}
\setlist[aenumerate]{wide, labelwidth=!, labelindent=0pt,label=(\arabic*)}

\newlist{aaenumerate}{enumerate}{3}
\setlist[aaenumerate]{wide, labelwidth=!, labelindent=0pt,label=(\alph*)}

\newlist{aenumerateSpace}{enumerate}{3}
\setlist[aenumerateSpace]{wide, labelwidth=!,label=(\arabic*)}

\newlist{benumerate}{enumerate}{3}
\setlist[benumerate]{wide, labelwidth=!, labelindent=0pt,label=$\bullet$}
