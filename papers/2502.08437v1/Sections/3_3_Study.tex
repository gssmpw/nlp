\section{Study on Visualizing passed and upcoming POIs}
\label{section:study}
We created and evaluated a prototype to test a realistic scenario where a passenger can use AR to explore their environment through digital POIs overlaid onto the real world. The study employed a within-subjects design and was conducted in a car setting in the field. The research questions guiding our investigation are outlined below:
\begin{itemize}
    \item \textbf{RQ1$_{main}$:} How can AR effectively support passengers in discovering missed and upcoming POIs?
    \item \textbf{RQ2$_{main}$:} Is eye-gaze and pinch a feasible interaction method for interacting with both world-fixed POIs and car-fixed UIs?
    \item \textbf{RQ3$_{main}$:} Would users accept an VST-based in-car AR system to explore POIs?
\end{itemize}

\begin{figure*}[ht]
    \centering
    \includegraphics[width=\linewidth]{Images/UI_rounded.png}
    \caption{An overview of the UI elements used in the study. The \textit{Informations} panel always correspondet to the currently selected POI. The \textit{List}, \textit{Minimap}, and \textit{Timeline} were only used during their study-conditions respectively.}
    \label{fig:study_design_elements}
    \Description{The image depicts an interactive user interface for a navigation system with several key sections: Timeline (top center) shows a horizontal sequence of locations with circular icons representing different points of interest (POIs). 'Informations' (bottom left) displays details for a single POI, including opening times, category, price, a description, rating, and reviews. A photo of the location is also included. Minimap (center right) highlights POIs in a small, circular map with selectable icons. List (right side) provides a vertical list of locations, with one POI highlighted in green. Control buttons ("X" to close and a crosshair to re-center) are present in each section for navigation and interaction.}
\end{figure*}



%======================== [ Design ] ========================%
\subsection{Prototype Design}
\label{sec:study_prototype_design}
Our prototype design is grounded in the findings from our survey (Section \ref{sec:survey}) and pre-study (Section \ref{sec:pre-study}). The survey results indicated that passengers often experience difficulty recalling the names of missed POIs and prefer to search immediately for interesting locations on the web using their phones. Additionally, many respondents expressed a desire for the functionality to save POIs. Therefore, the primary focus of our study is to investigate methods to increase the success rate of participants in rediscovering POIs and to interact with those that are not in the vehicle's immediate surroundings. In addition to the name, we include an image, a category, and a description for each location to assist users in easily identifying it. We include three possible categories for POIs, with each category including three types for variation: food (restaurants, bars, and pizza), museums (art, science, and history), and parks (zoos, nature, and recreational). The system adheres to Nielsen's sixth heuristic principle, which prioritizes recognition over recall \cite{Nielsen1994Usability}. These and more informations are visible after selecting a POI, as shown in Figure \ref{fig:study_design_elements} at \textit{Informations}.

To explore passed and upcoming POIs, we designed three in-vehicle visualizations to be used in tandem with the world-fixed POIs outside the vehicle: \textit{Timeline}, \textit{List}, and \textit{Minimap}. The designs are illustrated in Figure \ref{fig:study_design_elements}. The system was designed to minimize cognitive load by enabling users to recognize places within a sequence rather than recall specific names \cite{Nielsen1994Usability}. The \textit{Timeline} is designed to emphasize chronology, reflecting the sequential nature of encountering POIs along a route. The \textit{List}, while conveying order, does not inherently suggest a chronological sequence, as lists can be organized in various ways, such as alphabetically or sorted by rating. Consequently, we anticipated that participants might interpret the location of past and future POIs in the \textit{List} differently from each other. The \textit{Minimap} was expected to convey a sense of spatial chronology, as POIs would appear sequentially along the route. All visualizations were designed to avoid blocking the outside view, as found by Sawitzky et al. \cite{Sawitzky23ArPlacement}, and were present in the FOV only when the user actively engaged with them.

Regarding graphical choices, the timeline was designed to allow the user to constantly track progress, with each POI represented by a dot along the route. To express progress, the timeline fills in as the next POI approaches. The list displays the POI closest to the user as the central element, with future POIs located below it and past POIs above. To facilitate progress tracking, the list automatically moves to the current POI when opened. The map displays only the segment of the route currently being traversed, with POIs indicated by dots. Progress is tracked by a moving arrow that represents the cars position. Each of the three visualizations also includes a button allowing the user to quickly return to the current position and a button to close it.

Similar to the pre-study apparatus in Section \ref{sec:pre-study_apparatus}, we placed fourty POIs around the study track. Each of them was located perpendicular to the center of the street at a distance between 5 and 7.5 meters, alternating to the left and to the right side of the street. The POIs had a diameter of three meters and showed fake locations comprised of fictional names and images. Location categories included food, mueums, and parks. This setup simulated the experience of exploring a new city route, with POIs unknown to the participants, helping them find new places to visit during the trip. World-fixed POIs followed the design shown in Figure  \ref{fig:study_design_POI}. 

We used eye-gaze as a pre-selection method for both world-fixed and car-fixed content, based on the findings from our pre-study in Section \ref{sec:pre-study} and the findings by Schramm et al. \cite{Schramm2023Assessing}. Since the car-fixed visualizations require scrolling, we integrated hand-tracking techniques. As such, Eye-gaze is used to preselect elements for interaction, and pinching is used to confirm the selection. Scrolling works by holding the pinch gesture and Simultaneously moving the hand in the desired direction. We limited interactions to the participants' dominant hand to mitigate accidental selections through the non-dominant hand.


\begin{figure}[ht]
    \centering
    \includegraphics[width=0.6\linewidth, trim={0.5cm 1cm 0.5cm 0.5cm},clip]{Images/POI_white.png}
    \caption{Design of the world-fixed POIs with the location name, star rating, and representative image.}
    \label{fig:study_design_POI}
    \Description{A circle featuring a grey outer border, with a smaller inner border that resembles a glowing neon tube, emitting a light blue light. The circle showcases an image of a mountain scenery. It also includes a dark grey bar across the bottom of the image, displaying the name and star-rating of the respective point of interest.}
\end{figure}


%======================== [ Participants ] ========================%
\subsection{Participants and Apparatus}
\label{sec:study_participants_apparatus}
We conducted the study with employees of an automotive software consulting company, recruited through convenience sampling via email invitation and word of mouth (N = 21; 6 female, 15 male; mean age = 36.0 years, SD = 11.4 years). Participants were asked to rate their experience with immersive technologies using three 5-point Likert scales, ranging from no experience to extensive experience. The three categories assessed were experience with HMDs (M = 2.62, Mdn = 2, SD = 1.47), interaction via eye-tracking (M = 1.90, Mdn = 1, SD = 1.18), and hand-tracking (M = 2.29, Mdn = 2, SD = 1.35). Additionally, twelve participants required corrective eyewear, while nine did not. Seventeen participants were right-handed, four were left-handed.

Similar to the pre-study apparatus described in Section \ref{sec:pre-study_apparatus}, we used the Varjo XR-3\footref{foot:Varjo} VST HMD with LP-Research\footref{foot:lpvr} 6-DoF tracking. We attached a Leap Motion Controller 2\footnote{\label{foot:Leapmotion2}Ultraleap: Leap Motion Controller 2. \url{https://leap2.ultraleap.com/products/leap-motion-controller-2/} (accessed on 26.08.2024)} to the front of the Varjo XR-3 to have improved handtracking over its' integrated Leap Motion Controller. We used the Unity XR Interaction Toolkit\footnote{Unity Technologies: XR Interaction Toolkit. \url{https://docs.unity3d.com/Packages/com.unity.xr.interaction.toolkit@3.0} (accessed on 12.09.2024)} version 3.0.5 for eye-gaze and pinch interactions.


%======================== [ Procedure ] ========================%
\subsection{Procedure}
\label{sec:study_procedure}
Initially, participants were asked to sign an informed consent form and to complete a series of demographic questions as reported in Section \ref{sec:study_participants_apparatus}. Then participants were introduced to the study's objectives and procedures. Subsequently, a training phase was conducted within a stationary vehicle to allow participants to familiarize themselves with the eye-gaze and pinching interactions required during the study.

The core experimental phase involved driving participants along a predetermined 3 km track shown in Figure \ref{fig:study_tracks}, once for each condition respectively. As in the pre-study, participants were seated in the front passenger seat of a premium midsize estate vehicle. The car's speed limiter was set to 40 km/h to allow for uniform driving conditions. The lower speed of 40 km/h compared to the 50 km/h in the pre-study is based on the high percentage of missed POIs in the pre-study (Section \ref{sec:prestudy_results}). By lowering the speed, we wanted to mitigate the risk of missing POIs. The mean duration for completing the track was 6.92 minutes (Mdn = 6.74, SD = 0.930).

The three conditions \textit{Timeline}, \textit{Minimap}, and \textit{List} are described in detail in Section \ref{sec:study_prototype_design}. The order of the conditions was counterbalanced using Latin Square to mitigate learning effects. For each condition, participants were asked to complete two tasks, randomly selected from a pool of three options that reflected typical operations passengers might perform while exploring their environment by car. The possible tasks included adding a location to the favorites list, calling a location, and reserving a table or purchasing a ticket depending on the locations category. Participants could complete the tasks by using the buttons on the \textit{Informations} panel, as shown in Figure \ref{fig:study_design_elements}. The tasks were given to the participants by the system at predetermined points on the track to provide participants with sufficient information to perform the tasks. The first task was presented after the vehicle travelled 25\% of the tracks distance, the second tasks was presented after the vehicle travelled 60\% of the tracks distance, as highlighted in Figure \ref{fig:study_tracks}. As the vehicle passed these points on the track, the corresponding tasks was presented to the participant via a pop-up notification, accompanied by a sound cue to capture their attention. Each round consisted of a \textit{past\_task} and a \textit{future\_task}. A \textit{past\_task} relates to a previously seen POI, while a \textit{future\_task} relates to an upcoming POI. The POIs for both task types were randomly chosen and thus were possibly different for each condition to mitigate learning effects. The order of \textit{past\_task} and \textit{future\_task} was also counterbalanced using Latin Square.

Following each condition, participants were required to complete the Motion Sickness Questionnaire (MISC), the RTLX, the User Experience Questionnaire (UEQ), and the SUS. Additionally, a set of custom questions with 5-point Likert scales was administered to collect participants' perceptions of the intuitiveness of the POI display order and the degree to which the display elements may have occluded their view. To further explore the participants' experiences, semi-structured interviews were conducted after each condition and at the conclusion of all conditions. These interviews probed the participants' perceived difficulties, preferences, and suggestions for improvements to the prototype system.

The final phase of the study involved a comparative evaluation, wherein participants were asked to rank the three systems from least to most favorite and to respond to Likert scale questions regarding the perceived usefulness of the system. Throughout the study, participants' responses to questionnaires were recorded using specially designed Excel templates, which facilitated offline completion and automatic updating of a central replies' sheet.



\subsection{Measures}
\label{sec:study_measures}
We employed both quantitative and qualitative measures to evaluate the three presented paradigms for interacting with POIs.

\textbf{Quantitative data.} We first utilized the RTLX \cite{hart1988development, hart2006nasa} to assess participants' perceived workload. Secondly, we employed the SUS \cite{Brooke96SUS} to evaluate the usability of the three proposed solutions. As a third quantitative measure, we assessed user experience using the english 26-item version of the UEQ \cite{Laugwitz2008UEQ}. The UEQ results consist of six factors: attractiveness, perspicuity, efficiency, dependability, stimulation, and novelty. Additionally, we incorporated custom questions after each condition to evaluate participants' interpretation of POI sequencing, the extent to which the interfaces occluded the real world, and their ability to locate POIs based on the tasks. To measure potential motion sickness effects caused by the use of the AR application in a moving vehicle, we employed the MISC \cite{Bos2006Misc}. This scale assessed the severity of motion sickness symptoms, including nausea, dizziness, and headache. The MISC was administered before the study and once after each condition.

\textbf{Qualitative data.} We conducted semi-structured interviews after each condition and at the end of the study. Participants were asked about any difficulties they encountered with the prototype, the aspects they found most challenging to understand, and their most and least liked features. At the conclusion of the study, they were asked to clarify which condition best helped them understand which POIs had been passed and which were upcoming. Additionally, participants were invited to suggest desired features and to discuss when and how they would use the system. Finall, they ranked the three systems from least to most favorite and rated the system's overall usefulness on a scale from 1 to 5.