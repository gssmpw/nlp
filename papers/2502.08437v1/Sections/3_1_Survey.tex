\section{Survey on current Passenger behavior regarding POIs}
\label{sec:survey}
In this Section, we outline our survey to assess passengers needs for interaction with their environment while in transit. The survey consisted of structured questions related to individuals' driving habits and their preferences regarding POIs as both drivers and passengers. The questionnaire included multiple-choice questions, Likert-scale items, and two open-ended questions. It was divided into three main Sections: demographics, general questions, and specific scenarios for passengers and drivers. The overarching research question for the survey was:
\begin{itemize}
    \item \textbf{RQ1$_{survey}$:} What informations do passengers need to successfully find passed POIs while on the move?
    \item \textbf{RQ2$_{survey}$:} In what ways do passengers want to interact with POIs while on the move?
\end{itemize}



%======================== [ Survey Design ] ========================%
\subsection{Survey Design}
Participants were asked about the frequency of their driving, with options ranging from \textit{daily} to \textit{less than once a month}, and the frequency of being a passenger, using the same set of options. They were also asked if they had ever created a list of places to visit, followed by questions about the tools used for creating such lists and for navigation purposes. Multiple selections were allowed. To ensure relevance, participants were asked about the frequency with which they assumed the roles of either driver or passenger. They could only fill out the questionnaires for which role they are familar with by not selecting \textit{never} as a frequency.

Passenger-specific questions focused on individuals' experiences as passengers, including their involvement in navigation, the tools they use, and their preferences for saving and recalling locations of interest encountered during travel. The survey asked passengers about the frequency of assisting drivers with directions, their use of navigation tools, and their interest in features for saving and retrieving information about places observed during the journey.

Driver-specific questions examined drivers' perspectives, including how they plan routes, utilize navigation tools, and manage the discovery of new places while driving. As with the passenger section, drivers were asked about their interest in features that enable the saving of POIs, either manually or automatically. The survey also investigated the challenges faced by both drivers and passengers in recalling the names of places they passed, as well as the types of information needed to facilitate later identification of these locations. Finally, participants were asked to express their frustrations with current navigation tools, with the goal of identifying features to avoid when designing new tools.



%======================== [ Participants ] ========================%
\subsection{Respondents Demographics}
We conducted the survey with employees of an automotive software consulting company. A total of 110 individuals responded, comprising 81 males, 28 females, and 1 individual who preferred not to specify their gender. The respondents' ages ranged from 23 to 63 years ($mean = 39, SD = 9.61$). Three participants (2.7\%) do not drive regularly and, therefore, did not complete the driver-specific questionnaire, resulting in 107 valid responses. Among these drivers, 37\% drive daily, while 43\% drive three to four times per week. In terms of navigation system usage, 35\% utilize such systems occasionally, and 37\% frequently, encompassing both built-in systems and smartphone-based applications. Additionally, 14\% report using navigation systems every time they drive. Four participants (3.6\%) are not regular passengers. As a result, they did not complete the passenger questionnaire, leaving 106 valid responses for the passenger-specific survey. Among these respondents, 33\% reported being passengers infrequently (1-3 times per month), while 32\% indicated they are passengers once a week. 



%======================== [ Discussion ] ========================%
\subsection{Survey Insights}
\label{surveyResults}
As the survey included both multiple-choice questions and free-text fields, we analyzed and reported the frequencies of responses for each relevant question, distinguishing between passengers and drivers. The most insightful findings resulted from the questions concerning missed POIs and the process of saving POIs.

\subsubsection*{\textbf{Missed POIs}}
Our survey results indicate that the majority of passengers experience difficulty in recalling the names of missed locations they are interested in, with 64.2\% reporting occasional challenges and 4.7\% always facing this issue. This tendency is even more pronounced among drivers, likely due to the demands of focusing on the driving task; 77.6\% of drivers report sometimes struggling to remember POI names, while 5.6\% consistently encounter this difficulty. Figure \ref{fig:survey_search_behaviour} illustrates the percentages of individuals' responses when they pass by an interesting location and miss it. Notably, 49.1\% of passengers immediately search for the location on the web using their smartphones, while 31.1\% conduct a similar search using the vehicle's navigation system. This behavior aligns with related literature on NDRTs \cite{russell2011passengers, hecht2020ndrts, MatsumuraActivePassengering18, BergerGridStudyInCarPassenger2021} and underscores the importance of environmental interaction for passengers. Regarding the importance of specific information for recognizing a POI, there was significant consensus (90\%) on the necessity of knowing the POI's name to accomplish this task for both drivers and passengers. However, passengers placed additional emphasis on the need to know the \textit{category} and have a \textit{description} of the place to recognize it, compared to drivers.

%======================== [ Search Behaviour ] ========================%
\begin{figure}[ht]
    \centering
    \includegraphics[width=\linewidth]{Images/POISearchBehaviour.eps}
    \caption{A Graph showing how and when passengers and drivers look for a missed point of interest. Multiple choice was possible.}
    \label{fig:survey_search_behaviour}
    \Description{Barcharts showing participants' behavior when missing a point of interest, grouped by drivers and passengers. Bars represent the percentage of users selecting the multiple-choice answers. Mean values are provided in the appendix. 'Later on smartphone' has the most votes by drivers, whereas 'search immediately on smartphone' was picked by most passengers.}
\end{figure}


\subsubsection*{\textbf{Saving POIs}}
In the survey, 74.8\% of drivers and 67.9\% of passengers expressed a desire for the functionality to save POIs in their navigation systems, indicating a significant interest in this feature across both groups. This interest corresponds with the broader goal of enhancing user engagement with their environment, particularly in relation to the creation of POI lists. Regarding the saving of POIs, 65.5\% (N = 72) of participants reported having created a list of POIs at least once in their lifetime. Both groups demonstrated a clear preference for manual saving, with 93.1\% of passengers and 65\% of drivers favoring this method. Additionally, 18.8\% of drivers preferred automatic saving, which is understandable given the demands of driving. Therefore, an in-car POI system should be designed to accommodate the needs of both passengers and drivers, offering the option to save POIs with a strong emphasis on manual saving.