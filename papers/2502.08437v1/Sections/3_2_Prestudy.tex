\section{Pre-Study on Eye-Gaze Interaction}
\label{sec:pre-study}
In a pre-study with 10 participants, we investigated the feasibility of interacting with world-fixed content as a passenger using AR in a moving vehicle. We employed the technique of using eye-gaze to hover over POIs and confirming the selection via a hardware button, as this was shown in \cite{Schramm2023Assessing} to be one of the favored selection techniques for interacting with in-car AR content. The key distinction between our study and that of Schramm et al. \cite{Schramm2023Assessing} is that our POIs are world-fixed, rather than car-fixed. As a result, the task structure in our study differed, as outlined in Section \ref{sec:prestudy_procedure}. The time available for participants to select a POI in our study was thus not fixed but instead varied based on current traffic conditions, providing a closer scenario to real-world conditions. The research question guiding our pre-study was as follows:

\begin{itemize}
    \item \textbf{RQ$_{pre}$:} Is interaction with world-fixed content via eye-gaze combined with a hardware button a feasible method for in-car AR?
\end{itemize}


%======================== [ Pre-Study Participants ] ========================%
\subsection{Participants and Apparatus}
\label{sec:pre-study_apparatus}
We recruited ten participants, comprising two females and eight males, with a mean age of 32.5 years ($SD = 7.17$). Participants were seated in the front passenger seat of a premium midsize estate vehicle. The study was conducted on a street in a public industrial area to simulate realistic driving conditions. The chosen track, labeled as \textit{Pre-study track} in Figure \ref{fig:study_tracks}, was around 3.5 kilometers long, had a 50 km/h speed limit, and featured wide roads with moderate mixed traffic, including cars, busses, trucks, bicycles, and pedestrians. Some road sections were bumpy due to frequent heavy vehicle traffic. To ensure uniform driving conditions, the car's speed limiter was set to the maximum allowable speed of 50 km/h. Fourty POIs were placed around the track. Each of them was located perpendicular to the center of the street at a distance of 7.5 meters, alternating to the left and to the right side of the street. The POIs had a diameter of three meters and showed fake restaurants comprised of fictional names and images. The pre-study setup is shown in Figure \ref{fig:prestudy_varjoview}.

For the AR hardware, we selected the Varjo XR-3\footnote{\label{foot:Varjo}Varjo Technologies Oy: Varjo XR-3, the first true mixed reality headset. \url{https://varjo.com/products/varjo-xr-3/} (accessed on 26.08.2024)} video see-through (VST) HMD, chosen for its advanced features and compatibility with the middleware from LP-Research\footnote{\label{foot:lpvr}LPVR Middleware a Full Solution for AR / VR. \url{https://www.lp-research.com/middleware-full-solution-ar-vr/} (accessed on 12.09.2024)}. This setup enabled 6-Degrees of Freedom (DoF) HMD tracking in a moving vehicle, supported by an additional car-mounted inertial measurement unit. The selection technique was implemented using Microsoft's Mixed Reality Toolkit (MRTK) Version 2.8.3\footnote{Mixed Reality Toolkit 2. \url{https://learn.microsoft.com/en-us/windows/mixed-reality/mrtk-unity/mrtk2/} (accessed on 12.09.2024)} in Unity. 


%======================== [ Pre-Study Image ] ========================%
\begin{figure*}[ht]
    \centering
    \includegraphics[width=\linewidth]{Images/Prestudy_varjo.png}
    \caption{View of the pre-study. We use the Varjo XR-3 with additional optical tracking (left). POIs are visualized as spheres outside the vehicle (right). The POI with the red crosshair had to be selected via eye-gaze and a hardware button.}
    \label{fig:prestudy_varjoview}
    \Description{The setup for our pre-study. On the left, a person sitting in a car, wearing the Varjo XR-3 headset. On the right image, the view inside the Varjo XR-3 is displayed, showing the scene through the car's windshield. Outside the car are four spheres representing points of interest. The closest point of interest is marked with a red crosshair.}
\end{figure*}


%======================== [ Pre-Study Procedure ] ========================%
\subsection{Procedure and Task}
\label{sec:prestudy_procedure}
Participants first completed a declaration of consent and a demographic questionnaire. Following this, they were seated in the car and were provided with an explanation of the procedure and the task. Participants were instructed to only select POIs marked with a crosshair (as seen in Figure \ref{fig:prestudy_varjoview}) and to do so as fast as they could, following a procedure similar to that used in \cite{Schramm2023Assessing}. In our study, the crosshair was randomly placed on a single POI located within a radius of 70 meters in front of the vehicle. If the participant successfully selected the marked POI or the vehicle passed the marked POI, another nearby POI was randomly marked. Each marked POI was on average marked for 4.80s (Mdn = 2.87s, SD = 6.32s) before being selected or passing the car. Each round lasted between five and six minutes, depending on the traffic conditions on the study track. After completing the task, participants filled out the Raw NASA Task Load Index (RTLX) \cite{hart1988development, hart2006nasa} and System Usability Scale (SUS) \cite{Brooke96SUS} questionnaires. This was followed by a short semi-structured interview to gather qualitative feedback on participants' preferences.

%======================== [ Pre-Study Measures, Results, and Discussion ] ========================%
\subsection{Pre-Study Results and Discussion}
\label{sec:prestudy_results}
\subsubsection*{\textbf{Error Rate:}} Among the 724 marked POIs across all participants, 483 (66.71\%) were correctly selected, while 170 (23.48\%) were missed.  Additionally, for 71 (9.81\%) marked POIs, an unmarked POI was incorrectly selected instead. These results are less favorable compared to the findings of Schramm et al. \cite{Schramm2023Assessing}, where in the eye-tracking condition, 7.79\% of marked elements were missed, and 2.86\% of unmarked elements were erroneously selected. The discrepancy likely stens from the differences in the placement of POIs. Unlike the car-fixed POIs used in the study by Schramm et al. \cite{Schramm2023Assessing}, our world-fixed POIs represent moving targets, making them more susceptible to being missed.

\subsubsection*{\textbf{Task Completion Time:}} We also measured the time elapsed between marking a POI and subsequently selecting the marked POI. The mean time for selection was 1.82 seconds (Mdn = 1.35s, SD = 1.50s). This result closely aligns with those reported by Schramm et al. \cite{Schramm2023Assessing}, where the mean time for selection using eye-gaze with hardware confirmation was also 1.82 seconds (Mdn = 1.54s, SD = 0.913). While our median time is 0.19 seconds shorter, our standard deviation is 0.587 seconds higher. These findings suggest that the placement of world-fixed POIs, as compared to car-fixed POIs, does not have a significant effect on the time required to select a marked POI in our scenario.

\subsubsection*{\textbf{Perceived Workload}:} Our system received a mean workload score of 24.8, which is in line with related literature. In the work of Kyt{\"o} et al. \cite{kyto2018pinpointing}, interacting via eye + device while standing had a similar mean RTLX score of roughly 30. Though, comparability is limited, as their tasks took significantly longer and they used a Microsoft Hololens for testing. Schramm et al. \cite{Schramm2023Assessing} evaluated the same technique with a similar in-car setup also using the Varjo XR-3. Their mean RTLX score for eye + hardware confirmation of 24.6 is close to ours. Blattgerste et al. \cite{blattgerste2018advantages} also received similar values to us for RTLX using eye-gaze while being stationary. They evaluated the workload for three Fields of View (FOVs), where the large (90\textdegree{}) FOV is closest to the 110\textdegree{} FOV of the Varjo XR-3. In this condition, they measured a RTLX value of 27.5, which is also close to ours. To summarize, both Kyt{\"o} et al. \cite{kyto2018pinpointing} and Schramm et al. \cite{Schramm2023Assessing} conclude that eyegaze + device are feasible selection methods  with similar RTLX scores to ours. In addition, \cite{blattgerste2018advantages} shows the advantages of eye-gaze, also with similar RTLX to ours. Thus we conclude that this technique is also feasible for in-car AR use regarding workload. 

\subsubsection*{\textbf{Usability:}} Our system received a mean SUS score of 86.0 (Mdn = 87.5, SD = 8.01), which corresponds to an \textit{excellent} rating according to Bangor et al. \cite{bangor2009sus}. This score is consistent with related literature, as Schramm et al. \cite{Schramm2023Assessing} achieved a similar SUS score of 85.6 for their eye-gaze and hardware condition. Thus, we can conclude for RQ$_{pre}$ that eye-gaze combined with a hardware button is a feasible method for interacting with world-fixed objects in a moving vehicle.



%======================== [ Study Tracks ] ========================%
\begin{figure*}[ht]
    \centering
    \includegraphics[width=\linewidth]{Images/TrackImproved.eps}
    \caption{The tracks used for the pre-study (blue) and the main-study (red). The studies were conducted in an industrial area with a 50km/h speed limit and moderate traffic. Traffic lights are annotated via icons.}
    \label{fig:study_tracks}
    \Description{A top-down 2D map with two driving tracks marked in different colors, representing the pre-study track and the study track. The pre-study track is highlighted in blue and forms a rectangular loop across two and a half blocks. The study track is marked in red and forms a similar loop. Both tracks start and end at roughly the same location, indicated by a yellow circle labeled "Start and end." For the study-track, two tasks are marked on the map. Task one occurs at the 20\% mark of the track. Task two is located at 60\% of the track. Traffic lights are shown at two intersections along the shared route, which both tracks pass through. Arrows on each track indicate the direction of travel for both routes, with the pre-study track going clockwise and the study track going counterclockwise.}
\end{figure*}
