\section{Introduction}

Exploring unfamiliar places and accessing relevant information about the surrounding environment are crucial for enhancing passenger engagement and satisfaction. In such situations, individuals often look for intuitive ways to discover nearby Points of Interest (POIs), such as food spots, parks, museums, or gas stations \cite{Psyllidis2022POIs, sun2023conflating}. Providing a simple and effective mechanism to explore the environment and gain easy access to detailed information about these locations can empower users to make informed decisions and deepen their connection with their surroundings. Potential application areas include supporting users in their daily routines and encouraging exploration in unfamiliar cities.


Advances in transportation such as improved public transit and the emergence of autonomous vehicles are expected to shift the role of individuals from drivers to passengers \cite{McGill2022MRPassengerXP, MatsumuraActivePassengering18}.
%
This transition is anticipated to increase the time individuals spend as passengers, which can be considered as unproductive or wasted time when not used for non-driving related tasks (NDRTs) \cite{gardner2007drives, watts2008moving, wilfinger2011we}.
%
The potential benefits of NDRTs are vast, offering passengers the ability to partake in activities ranging from work to leisure, thus turning travel time into a productive or enjoyable experience \cite{Mathis2021work, medeiros2022shielding, Togwell2022gaming, Pfleging16NDRNeeds, Riegler19WindshieldArNDRT}. Additionally, passengers often enjoy interacting with their surroundings and appreciating the view through the windows \cite{russell2011passengers, hecht2020ndrts, MatsumuraActivePassengering18,BergerGridStudyInCarPassenger2021}. Recognizing this, one promising approach to NDRTs is to enhance this engagement with the external scenery \cite{MatsumuraActivePassengering18} and to relate it to the current trip \cite{Inbar11TripUX}.
%
In line with this, Augmented Reality (AR) emerges as a technology that can significantly aid passengers in interacting with their environment, thereby offering a novel way to improve experiences in cars \cite{McGill2022MRPassengerXP}. AR hardware continues to advance both in Head-Up Displays (HUDs) and in Head-Mounted Displays (HMDs), making them more practical for in-car use \cite{riegler2021augmented, Elhattab23AutomotiveAR, Goedicke2022xroom}. Such technologies could enrich the passenger experience by providing access to information about the outside environment \cite{BergerGridStudyInCarPassenger2021}, for instance, through POIs.
%
Future in-car AR systems could incorporate POIs to present real-time, context-specific information about the car's surroundings overlaid onto the real-world. This seamless integration of digital content with the surrounding context could facilitate the understanding of information at a glance \cite{haeuslschmid2016design} and help passengers to connect more deeply with their environment \cite{BergerGridStudyInCarPassenger2021, Mehrabian74EnvironmentalPsychology, Pfleging16NDRNeeds, Berger21InteractiveCarDoor}.


% % [Our Contribution] 
To allow passengers to interact with their environment, we developed and evaluated an interactive in-car system utilizing an AR HMD in a moving vehicle. Our system allows passengers to explore and interact with world-fixed POIs in their immediate surroundings, as well as explore passed and upcoming POIs. We focus on the exploration of an unknown city-environment, as these environments contain the most POIs. In a preliminary study (N = 10), we evaluated the interaction technique eye-gaze combined with a hardware button in AR settings with world-fixed objects in transit regarding workload and usability. We also conducted a survey (N = 110) on the current behaviours of passengers interacting with location based data in the car without AR. We then adapted our prototype to adress these needs and habits of passengers. To evaluate our system, we conducted a within-subjects user study (N = 21) in a moving vehicle in the field. Here we examined the impact of three seat-fixed spatial user interface (UI) concepts in addition to the previously studied world-fixed POIs, as shown in Figure \ref{fig:teaser}. We evaluated this system regarding workload, usability, user experience and user-preference, together with qualitative data collected through semi-structured interviews. As such, the main contributions of our paper are as follows: 
\begin{itemize}
    \item A comprehensive investigation of the habits and preferences of passengers regarding their current interaction with location-based data in cars through a survey (N=110).
    \item Assessing the feasibility of eye-gaze-based interaction methods for in-car AR through quantitative and qualitative data (N=10 \& N=21).
    \item Evaluation of a system for interacting with POIs and displaying further information using AR through quantitative data and semi-structured interviews (N=21).
\end{itemize}