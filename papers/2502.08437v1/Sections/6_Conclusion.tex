\section{Conclusion and Future Work}
\label{section:Conclusion}
In summary, this paper demonstrated the potential of in-car Augmented Reality (AR) systems to enhance the passenger experience by providing interaction with the environment through world-fixed Points of Interest (POIs). We show that passengers and drivers have a need for interacting with the environment while on the go and want to seek information about locations in their environment. We also found that both passengers and drivers experience difficulties in recalling names of missed POIs. To find missed POIs, the name, a category, and a description are preferable information. We also show passengers and drivers desire to create lists of interesting locations they meet. Both groups show clear preferences to save POIs manually. Regarding in-car AR interaction, we demonstrate that eye-gaze with a hardware button is a feasible interaction technique for interacting with world-fixed POIs. Conversely, our research also highlights the limitations of the use of eye-gaze combined with pinch gestures in a moving vehicle, which proved problematic for many users. As such, we recommend using interaction techniques that are less prone to movements for in-car AR such as handheld or car-fixed devices. Additionally, we investigated three visualizations for exploring passed and upcoming POIs while traveling, with a clear preference among users for list-based visualizations due to their familiarity and ease of use. The timeline and minimap visualizations, while innovative, require further refinement to address usability challenges, particularly in dynamic environments. The overall system received scores in the mid-range, partly caused by current hardware limitations. Users would prefer lighter hardware or no head-worn devices at all. In general, the integration of AR in vehicles holds great promise for transforming travel time into a more engaging, informative, and productive experience. However, achieving this potential will require continued exploration and refinement of both the hardware technology and the user interfaces that support it.

Future work should explore alternative interaction methods, possibly incorporating multimodal approaches such as voice commands or hardware-assisted inputs to enhance usability and user satisfaction. Additionally, the placement of world-fixed POIs can be improved, e.g. by utilizing computer vision approaches to anchor POIs to buildings. Systems to mitigate vehicle motion for interaction also seem promising. We also want to explore the impact of AR-POIs for various driving scenarios, such as commuting and travel.