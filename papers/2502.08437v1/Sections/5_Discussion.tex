\section{Discussion}
\label{section:discussion}
In this Section, we discuss and analyze the findings obtained from our user study and semi-structured interview, in relation to the research questions that were previously established.

\subsection{Discovering Passed and Upcoming POIs}
\label{section:discussion_passed_upcoming}
In this section, we answer \textbf{RQ1$_{main}$:} \textit{Which visualization best supports passengers in discovering missed and upcoming POIs?} by utilizing the quantitative data presented in Section \ref{section:results} and the findings from our qualitative interviews.

The overall preferred visualization method in the study was the \textit{List}, as shown in Section \ref{sec:results_preferences} and illustrated in Figure \ref{fig:results_preferences}. Additionally, the \textit{List} had the lowest mean workload scores, alongside the \textit{Timeline}, as shown in Figure \ref{fig:results_RTLX}. The \textit{List} condition also received the highest mean usability score, with an adjective rating of \textit{good} \cite{bangor2009sus}, as seen in Figure \ref{fig:results_SUS}. Among the three conditions, it achieved the highest user experience scores in the categories of attractiveness, perspicuity, efficiency, dependability, and stimulation, as presented in Figure \ref{fig:results_UEQ}. According to the UEQ benchmark \cite{schrepp2017construction}, the \textit{List} condition demonstrated an overall \textit{good} user experience across all scales, with an \textit{excellent} score for perspicuity. The preference for the \textit{List} as the favored visualization technique could be attributed to several factors. One possible reason is the ease of searching within the \textit{List}, as it was rated highest in this category (see Section \ref{sec:results_preferences}), while also causing the least occlusion of the real world. This aligns with comments from some participants (N=6) who noted that the \textit{List} did not overly obstruct their view of the real world and allowed them to still see the POIs in their environment. This finding is consistent with related research, which indicates that passengers generally prefer to maintain a view of their surroundings while traveling \cite{russell2011passengers, hecht2020ndrts, MatsumuraActivePassengering18, BergerGridStudyInCarPassenger2021}. Furthermore, participants appreciated the vertical and familiar layout of the \textit{List}, as most applications for exploring results primarily utilize vertical scrolling \cite{kim2016pagination}.

The \textit{Timeline} visualization received mixed preferences, with most participants ranking it in the middle or last position, as shown in Figure \ref{fig:results_preferences}. However, in some metrics, the \textit{Timeline} performed similarly to the \textit{List}. Its workload was comparably low, and its usability score also fell within the \textit{good} range \cite{bangor2009sus}. In terms of user experience, the \textit{Timeline} was rated lower than the \textit{List}, achieving an overall \textit{above average} rating according to the benchmark \cite{schrepp2017construction}, with the exception of a \textit{good} rating for novelty. For task performance during the study, the \textit{Timeline} performed similarly to the \textit{List}, with no significant differences in task completion rate, task completion time, or ease of searching. Participants also understood the order of the \textit{Timeline} at a rate comparable to the \textit{List}. While occlusion scores for the \textit{Timeline} were higher than those for the \textit{List}, the difference was not significant. Nonetheless, some participants expressed concerns about the \textit{Timeline} occluding too much of the real world. Although there was an option in each condition to close the occluding visualization, it was used sparingly. Interview data revealed that some participants disliked the direction and ordering of the \textit{Timeline}. While all participants understood the purpose of the \textit{List}, the \textit{Timeline} was not immediately clear to everyone. Specifically, the fact that the \textit{Timeline} was intended to represent a chronological sequence was not evident to all participants, leading to confusion about the direction of chronology. This rating may also partially stem from a bias towards vertically scrolling lists \cite{kim2016pagination}. Participants also had more difficulties interacting with the timeline, especiall in regards of selecting in scrolling. This could stem from slightly smaller colliders than the ones in the List, as the elements in the timeline were round and not squared.

The least preferred visualization method was the \textit{Minimap}, which had slightly lower user preference than the \textit{Timeline}. The \textit{Minimap} received significantly higher workload scores, requiring more effort with poorer perceived performance, which led to frustration. Its usability was also significantly lower than that of the other two conditions, with an adjective rating of an \textit{OK} usability \cite{bangor2009sus}. Similarly, in terms of user experience, the \textit{Minimap} was rated significantly worse than the other two conditions, with benchmark comparisons \cite{schrepp2017construction} showing \textit{Below Average} ratings for each scale, except for a \textit{good} rating in novelty. The searching tasks were particularly challenging with the \textit{Minimap}, as only 59.5\% of tasks were completed, and those that were completed took longer. This is in line with our qualitative data, as there werer difficulties users encountered during the searching task in the \textit{Minimap}. The primary issues identified were the challenges in interacting with the 2D POIs on the map, as these were too small for many participants (N=10). This issue was exacerbated by bumpy streets, which made interaction even more difficult. Additionally, the name of the currently hovered POI was displayed on top of the map, leading to the Midas touch problem for some participants (N=9). On a positive note, participants generally appreciated the function and design of the map, and some (N=5) indicated they would have rated it higher if not for the aforementioned issues. Additionally, three participants mentioned that they would prefer a combination of the \textit{List} and \textit{Minimap}, as the \textit{Minimap} was generally liked as a tool for quickly getting an overview, while the \textit{List} was preferred for searching specific items.

To address \textbf{RQ1}, our findings indicate that the \textit{List} visualization technique is the most effective for helping passengers identify missed and upcoming POIs. Therefore, we recommend utilizing a vertical List for POI discovery, as it consistently outperformed other techniques.


\subsection{Interacting via Eye-Gaze and Pinch}
In this Section, we answer \textbf{RQ2$_{main}$:} \textit{Is eye-gaze and pinch a feasible interaction method for interacting with both world-fixed POIs and car-fixed UIs?} mainly through findings from our qualitative interviews. The interaction technique via eye-gaze and pinch proved overall problematic, with the majority of the participants' problems and negative feedback stemming from the interaction technique. This problem was seen in all conditions and, as such was indepentent of the visualization technique. The only feature that exacerbated the issue were the small selectable icons in the \textit{Minimap}, as discussed in Section \ref{section:discussion_passed_upcoming}. We hypothesize, that the main issues stem from using a pinch gesture as selection confirmation and for scrolling, even though we used state-of-the-art technology with the Leap Motion 2\footref{foot:Leapmotion2}. Our pre-study and the related work \cite{Schramm2023Assessing} did not show these issues to the same degre, as they used a hardware-button for confirmation. As such, we suggest to further examine these interaction techniqes, with a preference for using some kind of hardware assistance for selection confirmation and scrolling. E.g. a smartphone as input device could prove useful. When asked for potential additional features for the system, some participants also suggested a multimodal approach for input. For example using voice input when searching for a POI via name, or setting a filter for specific categories via voice.

For \textbf{RQ2}, we conclude that eye-gaze and pinch is not a viable interaction method for engaging with both world-fixed POIs and car-fixed UIs. Instead, we recommend adopting a multimodal approach for in-car AR interfaces, combining head-gaze or eye-gaze with voice and/or hardware-based inputs.


\subsection{Exploring POIs using AR}
In this section, we answer \textbf{RQ3$_{main}$:} \textit{Would users accept an VST-based in-car AR system to explore POIs?} by utilizing the quantitative data presented in Section \ref{section:results} and the findings from our qualitative interviews. Our system received overall positive feedback with, depending on the condition, relatively low workload, good usability, above average to excellent user experience, and low motion sickness scores. Only the minimap condition showed results that are below average across metrics. The system received a moderate score for Essentiality, often categorizing it as a nice system to have, but not essential in its current state. Especially the hardware limitations were often pointed out, as the Varjo XR-3 is rather warm and heavy. Participants could imagine using the system only with small and lightweight ar-glasses or with HUDs in the windshield. As for use-cases, nine participants mentioned wanting to use it during a city trip or road trip in unknown environments for exploration. Six participants can imagine using such a system daily, e.g. to look for a gas station nearby or to display information about ratings or opening hours for nearby locations.

To address \textbf{RQ3}, we conclude that users find VST-based in-car AR systems acceptable for exploring POIs. However, for hardware, we recommend prioritizing compact and lightweight AR glasses or windshield-based AR displays to enhance comfort.



\subsection{Integration into an existing Framework}
Kim and Jung \cite{kim2019automotive} proposed a cognitive framework for passengers in autonomous vehicles using traditional displays. They identified key cognitive needs for passengers, including driving route, current time, estimated duration, parking information, and location suggestions. Their UI framework incorporates four key design components: (A) time tracking, (B) driving purpose, (C) tour guidance, and (D) parking options. Our system utilizes AR to address components (B) and (C) of this framework. For (B), it displays recommended and favorite places in the surrounding area, aligning with the driving purpose. For (C), it provides informational windows with images and detailed descriptions of specific locations, fulfilling the tour guide component. The general acceptance of our AR system among participants further demonstrates its potential to meet the cognitive needs outlined in their framework, suggesting that AR can effectively enhance passenger experiences.

% As for integration into UI frameworks, 
% \footnote{Riley Hunt. 2023. Spatial UI Design: Tips and Best Practices: \url{https://www.interaction-design.org/literature/article/spatial-ui-design-tips-and-best-
% practices} (accessed on 07.12.24)}


\subsection{Limitations}
In-car AR systems are currently severly limited by available hardware. Current HMDs, such as the Varjo XR-3\footref{foot:Varjo} used in our studies offer many needed functionalities such as eye tracking and hand tracking, but are unwieldy or uncomfortable in return \cite{Goedicke2022xroom}. As such, passengers don't want to wear these all the time. Similarly other, smaller AR-glasses are more comfortable but offer less features and smaller FOVs.

The main problems users faced in our system stemmed from technical issues. Even though we used the Leap Motion 2\footref{foot:Leapmotion2} mounted on the HMD, users had troubles in performing the pinch and drag gestures. Here, factors such as sunlight, handedness, and experience with HMDs can play a role. As the Varjo XR-3 is quite large and heavy, the influence of vehicle movements was also significant. This makes accurate interaction hard, especially when interacting with small UI elements, such as the ones on the minimap.

Additionally, our survey exclusively targeted employees from an automotive software company. While this demographic aligns closely with potential users of such a system and may possess relevant expertise, it could also introduce bias into the survey results.