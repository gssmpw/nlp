\section{Results}
\label{section:results}
% copied
We conducted a repeated-measures ANOVA with the visualization technique in the study condition as independent variable, and the measures described in Section \ref{sec:study_measures} as dependent variables. When the assumption of normality was violated (tested with the Shapiro-Wilk test), we used the Friedman rank sum test. The post hoc tests were then conducted through pairwise comparisons using the Wilcoxon signed rank test with Bonferroni corrections. All statistical tests are reported at a 0.05 significance level for main effects. 


\begin{figure*}[t]
    \centering
    \includegraphics[width=\linewidth]{Images/Plots_Violin_TLX.eps}
    \caption{The mean values for each RTLX subscale, split by condition. Scales range from 0 to 100, lower is better. Significant differences are marked with lines between plots.}
    \label{fig:results_RTLX}
    \Description{Violin plots of users' mean NASA TLX workload scores for the three experimental conditions. Mean values are provided in the appendix. The minimap condition scored the highest workload across all conditions. Performance and Effort have few outliers.}
\end{figure*}

\subsection{Workload}
Results for the RTLX questionnaire are presented in Figure \ref{fig:results_RTLX}, with significant differences between conditions indicated. The data did not follow a normal distribution. The outcomes of the repeated-measures ANOVA show, that the minimap condition was associated with a significantly higher task load compared to the other conditions across all categories (Mental: F(2) = 16.48, p < .001; 
Physical: F(2) = 6.86, p = .032;
Temporal: F(2) = 14.29, p < .001; 
Performance: F(2) = 18.81, p < .001; 
Effort: F(2) = 11.08, p = .004; 
Frustration: F(2) = 15.69, p < .001; 
overall TLX-Score: F(2) = 18.28, p < .001).



\begin{figure*}[h]
    \centering
    \begin{subfigure}[b]{.45\textwidth}
        \centering
        \includegraphics[width=\textwidth]{Images/Plots_Violin_SUS.eps}
        \caption{SUS}
        \label{fig:results_SUS}
    \end{subfigure}
    \hfill
    \begin{subfigure}[b]{.45\textwidth}
        \centering
        \includegraphics[width=\textwidth]{Images/Plots_Violin_MISC.eps}
        \caption{MISC}
        \label{fig:results_MISC}
    \end{subfigure}
    \caption{Left: Violin plots for SUS scores, split by condition. Scales range from 0 to 100, higher is better. Right: Mean MISC scores over the study duration. Scale ranges from 0  to 10, lower is better. Significant differences are marked with lines between plots.}
    \Description{Violin plots of users mean SUS scores (a) and mean MISC scores (b) for the three experimental conditions. Mean values are provided in the appendix. The minimap condition scored a significantly lower usability score than list and timeline. The MISC values show the motion sickness progression over the time the study took place. Scores started between zero and one and rose to scores between zero and two for all three subsequent study runs.}
    \label{fig:results_SUS_MISC}
\end{figure*}




\subsection{Usability}
The results of the SUS questionnaire are presented in Figure \ref{fig:results_SUS}. The distribution of the scores was non-normal. A repeated-measures ANOVA revealed significant differences between conditions (F(2) = 14.46, p < .001). Post hoc tests indicate that the \textit{Minimap} condition had a significantly worse usability than the other two conditions. According to adjective interpretations for SUS scores by Bangor et al. \cite{bangor2009sus}, the  \textit{List} (M = 78.5, Mdn = 77.5, SD = 10.2) and \textit{Timeline} (M = 71.5, Mdn = 75.0, SD = 15.9) conditions can be described having a \textit{good} usability. However, the \textit{Minimap} condition (M = 61.1, Mdn = 60.0, SD = 14.3) could be rated as having an \textit{OK} usability.




\subsection{User Experience}
The results of the UEQ questionnaire are presented in Figure \ref{fig:results_UEQ}. The \textit{List} received \textit{excellent} Perspicuity ratings, and \textit{good} ratings for every other scale. The \textit{Timeline} had a \textit{below average} Dependability, \textit{good} Novelty, and \textit{above average} ratings for the remaining scales. \textit{Minimap} had overall \textit{below average} ratings, with only Novelty receiving a \textit{good} score. All scales except Perspicuity showed normal distribution. The Novelty scale violated the assumption of sphercity. Repeated-measures ANOVA showed significant differences between conditions for Attractiveness (F(2) = 7.12, p = .002), Perspicuity (F(2) = 11.60, p = .003), Efficiency (F(2) = 12.04, p < .001), Dependability (F(2) = 10.40, p < .001), and Stimulation (F(2) = 3.47, p = .041). Only Novelty showed no significant differences (F(2) = 2.085, p = 0.35). Pairwise comparisons are shown in Figure \ref{fig:results_UEQ}.

\begin{figure*}[ht]
    \centering
    \includegraphics[width=\linewidth]{Images/Plots_Violin_UEQ.eps}
    \caption{Violin plots for mean values across UEQ scales, split by condition. Scales range from -3 to +3, higher is better. Significant differences are marked with lines between plots.}
    \label{fig:results_UEQ}
    \Description{Violin plots of users mean UEQ scores for the three experimental conditions. Mean values are provided in the appendix. List scored best in Attractiveness, Perspicuity, Efficiency, and Dependability. For Stimulation, List and Timeline have similar scores. For Novelty, all three conditions have similar scores. The minimap has significantly lower scores than list for each subscale except for novelty.}
\end{figure*}


\subsection{Task Completion}
In the \textit{Timeline} and \textit{List} conditions, most participants were able to complete all tasks within the track. The task completion rate was 90.5\% for the \textit{Timeline} and 88.1\% for the \textit{List}. The \textit{Minimap} condition had the lowest task completion rate, with only 59.5\% of the tasks completed within the track.

The time from being given the task and completing it was measured in minutes. The mean task completion time overall was 1.15 minutes (Mdn = 0.88, SD = 0.88). Split by conditions, the \textit{MinimapCondition} took participants the longest (M = 1.67, Mdn = 1.26, SD = 1.24), followed by the \textit{TimelineCondition} (M = 1.04, Mdn = 0.76, SD = 0.74), and the \textit{ListCondition} (M = 0.92, Mdn 0.76, SD = 0.57).


\subsection{Motion Sickness}
Overall, our prototype induced little motion sickness, even though the study took place using a VST HMD in a moving vehicle. The mean baseline score before the study began was 0.38 (Mdn = 0, SD = 0.67), increasing during the study to a mean value of 1.02 (Mdn = 1, SD = 1.18). The mean MISC scores over the time of the study are shown in Figure \ref{fig:results_MISC}. Motion sickness did not increase significantly over the duration of three condition and actually got lower after the third condition again. MISC results also did not significantly differ between conditions, with the \textit{List} having a slightly lower mean impact on MS (M = 0.81, Mdn = 0.93) than the \textit{Minimap} (M = 1.10, Mdn = 1, SD = 1.34) and the \textit{Timeline} (M = 1.14, Mdn = 1, SD = 1.28).



\subsection{Preferences}
\label{sec:results_preferences}
The results of the reported user preferences are presented in Figure \ref{fig:results_preferences}. Among the evaluated visualization methods, the \textit{List} was the most preferred, with 71.4\% (N = 15) of participants selecting it as their top choice. The \textit{Timeline} was most commonly chosen as the middle option, with 47.6\% (N = 10) of participants selecting it in this position. Lastly, the \textit{Minimap} was the least favored, with 52.4\% (N = 11) of participants ranking it as their lowest preference. Participants rated the system's overall usefulness with a mean score of 3.05 (Mdn = 3, SD = 1.02).

Possible influences on the preferences could also stem from the conditions' occlusion of the real world. Participants could rate the conditions occlusion on a 5 point likert scale. For occlusion, the \textit{List} condition was rated with a mean of 2.29 (Mdn = 2, SD = 1.01), the \textit{Minimap} with 2.81 (Mdn = 3, SD = 1.08), and the \textit{Timeline} with 2.86 (Mdn = 3, SD = 0.96). There were no significant differences between conditions.

For the results on how difficult it was for finding the POIs given in the tasks, participants could rate on how easy it was to find on a 5-point likert scale. it was easiest to search in the list with a mean score of 2.00 (Mdn = 2.00, SD = 1.08), similarly easier to search in the timeline (M = 2.24, Mdn = 2.00, SD = 1.14). The map was the most difficult to search in with a mean score of 3.29 (Mdn = 3.50, SD = 0.86). There were significant differences between conditions (F(2) = 23.8, p < .001). Post hoc Tests revealed, that the Minimap was significantly more difficult so search in than the List and in the Timeline.

In addition, participants rated the extent of this occlusion on a 5-point Likert scale. The \textit{List} condition received a mean score of 2.29 (Mdn = 2, SD = 1.01), the \textit{Minimap} a mean of 2.81 (Mdn = 3, SD = 1.08), and the \textit{Timeline} a mean of 2.86 (Mdn = 3, SD = 0.96). No significant differences were found between these conditions.

Participants also rated the difficulty of finding the POIs given in the tasks on a 5-point Likert scale. The \textit{List} was the easiest to search, with a mean score of 2.00 (Mdn = 2.00, SD = 1.08). The \textit{Timeline} was similarly easy to search (M = 2.24, Mdn = 2.00, SD = 1.14). In contrast, the \textit{Minimap} was the most difficult to search, with a mean score of 3.29 (Mdn = 3.50, SD = 0.86). Significant differences were found between conditions (F(2) = 23.8, p < .001). Post hoc tests revealed that the \textit{Minimap} was significantly more difficult to search than both the \textit{List} and the \textit{Timeline}.

For the ordering of POIs in the 2D representations, participants could select one option out of the possible options: chronological, spatial, alphabetical, randomized, and \textit{no idea}. For the list, 47\% of participants understood the chronological order of the List, the rest had no idea or thought this was random. Similary, 42.9\% of participants understood the chronological order of the timeline, whereas 52.4\% of participants understood the spatial order of the Minimap. When prompted, most participants reported, that they did not have the time to really think about the order, as they were preoccupied interacting with a novel system in addition to the tasks.


\begin{figure}[ht]
    \centering
    \includegraphics[width=\linewidth]{Images/Results_Preferences.eps}
    \caption{User ranking for the three visualizations investigated in the study. Each participant could rank each visualization from least favourite to favourite.}
    \label{fig:results_preferences}
    \Description{Bar charts showing users' ranking of Minimap, Timeline, and Lists from favorite to least favorite. List is the clear favorite with around 70\% of users voting it as favorite. Minimap and Timeline have similar mixed ratings with minimap having slightly more least favorite votes. Mean values are provided in the appendix.}
\end{figure}


\subsection{Use-Cases}
During the post-interview, participants were asked when and how they would use the system. Nine participants stated that they would use the system in unfamiliar environments. Their comments included examples such as, "City tours, when you are new in a city," and "In an unfamiliar environment, exploring the surroundings when you are somewhere as a passenger." Within this category, four participants explicitly mentioned using the system while on vacation, with statements such as, "Going on vacation when you want to feel comfortable and discover things.". 

Six participants indicated that they would use the system in their daily lives, providing examples such as, "When i am looking for something specific in everyday life, like a gas station, restaurant, or a specific search with ratings, opening hours, etc.". Additionally, three participants expressed that they would use the system to locate specific destinations, similar to how they would perform a search using a map application on their phone. Regarding locations, six participants explicitly mentioned cities in their responses, whereas two indicated that they would prefer not to use the system on highways.

