%% For submission and review of your manuscript please change the
%% command to \documentclass[manuscript, screen, review]{acmart}.
%%
%% When submitting camera ready or to TAPS, please change the command
% to \documentclass[sigconf]{acmart} or whichever template is required
%% for your publication.
\documentclass[sigconf]{acmart}


%% \BibTeX command to typeset BibTeX logo in the docs
\AtBeginDocument{%
  \providecommand\BibTeX{{%
    Bib\TeX}}}

    
%% Rights management information.  This information is sent to you
%% when you complete the rights form.  These commands have SAMPLE
%% values in them; it is your responsibility as an author to replace
%% the commands and values with those provided to you when you
%% complete the rights form.
\copyrightyear{2025} 
\acmYear{2025} 
\setcopyright{acmlicensed}\acmConference[CHI '25]{CHI Conference on Human Factors in Computing Systems}{April 26-May 1, 2025}{Yokohama, Japan}
\acmBooktitle{CHI Conference on Human Factors in Computing Systems (CHI '25), April 26-May 1, 2025, Yokohama, Japan}
\acmDOI{10.1145/3706598.3714323}
\acmISBN{979-8-4007-1394-1/25/04}


\usepackage{subcaption}
\usepackage{multirow}
\usepackage[utf8]{inputenc}
\usepackage{natbib}

% ===================================================
\begin{document}

\title
[Augmented Journeys: Interactive POIs for In-Car AR]
{Augmented Journeys: Interactive Points of Interest for In-Car Augmented Reality}

% ===================================================
\author{Robin Connor Schramm}
\orcid{0000-0002-4775-4219}
\affiliation{
  \institution{Mercedes-Benz Tech Motion GmbH}
  \city{B{\"o}blingen}
  \country{Germany}
}
\affiliation{
  \institution{RheinMain University of Applied Sciences}
  \city{Wiesbaden}
  \country{Germany}
}
\email{robin.schramm@mercedes-benz.com}



\author{Ginevra Fedrizzi}
\orcid{0009-0001-2123-299X}
\affiliation{
  \institution{Mercedes-Benz Tech Motion GmbH}
  \city{B{\"o}blingen}
  \country{Germany}
}
\email{ginevrafedrizzi@gmail.com}



\author{Markus Sasalovici}
\orcid{0000-0001-9883-2398}
\affiliation{
  \institution{Mercedes-Benz Tech Motion GmbH}
  \city{B{\"o}blingen}
  \country{Germany}
}
\affiliation{
  \institution{Ulm University}
  \city{Ulm}
  \country{Germany}
}
\email{markus.sasalovici@mercedes-benz.com}



\author{Jann Philipp Freiwald}
\orcid{0000-0002-1977-5186}
\affiliation{
  \institution{Mercedes-Benz Tech Motion GmbH}
  \city{B{\"o}blingen}
  \country{Germany}
}
\email{jann_philipp.freiwald@mercedes-benz.com}



\author{Ulrich Schwanecke}
\orcid{0000-0002-0093-3922}
\affiliation{
  \institution{RheinMain University of Applied Sciences}
  \city{Wiesbaden}
  \country{Germany}
}
\email{ulrich.schwanecke@hs-rm.de}


\renewcommand{\shortauthors}{Schramm et al.}


% ===================================================
\begin{abstract}

Hierarchical clustering is a powerful tool for exploratory data analysis, organizing data into a tree of clusterings from which a partition can be chosen. This paper generalizes these ideas by proving that, for any reasonable hierarchy, one can optimally solve any center-based clustering objective over it (such as $k$-means). Moreover, these solutions can be found exceedingly quickly and are \emph{themselves} necessarily hierarchical. 
%Thus, given a cluster tree, we show that one can quickly generate a myriad of \emph{new} hierarchies from it. 
Thus, given a cluster tree, we show that one can quickly access a plethora of new, equally meaningful hierarchies.
Just as in standard hierarchical clustering, one can then choose any desired partition from these new hierarchies. We conclude by verifying the utility of our proposed techniques across datasets, hierarchies, and partitioning schemes.


\end{abstract}



%% The code below is generated by the tool at http://dl.acm.org/ccs.cfm.
%% Please copy and paste the code instead of the example below.
\begin{CCSXML}
  <ccs2012>
     <concept>
         <concept_id>10003120.10003145.10003147.10010923</concept_id>
         <concept_desc>Human-centered computing~Information visualization</concept_desc>
         <concept_significance>500</concept_significance>
         </concept>
     <concept>
         <concept_id>10003120.10003121.10003124.10010392</concept_id>
         <concept_desc>Human-centered computing~Mixed / augmented reality</concept_desc>
         <concept_significance>500</concept_significance>
         </concept>
     <concept>
         <concept_id>10003120.10003121.10003124.10010865</concept_id>
         <concept_desc>Human-centered computing~Graphical user interfaces</concept_desc>
         <concept_significance>500</concept_significance>
         </concept>
     <concept>
         <concept_id>10003120.10003121.10003122.10003334</concept_id>
         <concept_desc>Human-centered computing~User studies</concept_desc>
         <concept_significance>500</concept_significance>
         </concept>
     <concept>
         <concept_id>10003120.10003121.10003122.10011750</concept_id>
         <concept_desc>Human-centered computing~Field studies</concept_desc>
         <concept_significance>500</concept_significance>
         </concept>
     <concept>
         <concept_id>10003120.10003121.10011748</concept_id>
         <concept_desc>Human-centered computing~Empirical studies in HCI</concept_desc>
         <concept_significance>500</concept_significance>
         </concept>
     <concept>
         <concept_id>10003120.10003123.10011759</concept_id>
         <concept_desc>Human-centered computing~Empirical studies in interaction design</concept_desc>
         <concept_significance>300</concept_significance>
      </concept>
   </ccs2012>
\end{CCSXML}

\ccsdesc[500]{Human-centered computing~Mixed / augmented reality}
\ccsdesc[500]{Human-centered computing~Information visualization}
\ccsdesc[500]{Human-centered computing~Graphical user interfaces}
\ccsdesc[500]{Human-centered computing~User studies}
\ccsdesc[500]{Human-centered computing~Field studies}
\ccsdesc[500]{Human-centered computing~Empirical studies in HCI}
\ccsdesc[300]{Human-centered computing~Empirical studies in interaction design}


%% Keywords. The author(s) should pick words that accurately describe
%% the work being presented. Separate the keywords with commas.
\keywords{Augmented Reality, In-Car, Vehicle, Points of Interest, Passenger, Automotive User Interfaces, Visualization}



\begin{teaserfigure}
  \includegraphics[width=\textwidth]{Images/Teaser.png}
  \caption{An overview of our system to explore POIs using in-car AR as a passenger. We investigate the interaction with POIs (4) in the environment by displaying additional information to the user (5). We also tested three distinct visualization techniques to explore missed and upcoming POIs: Timeline (1), Minimap (2), and List (3). Each of the three techniques is intended to work in tandem with the world-fixed POIs (4) and the additional POI information (5).}
  \Description{The image depicts an augmented reality (AR) navigation system integrated into a vehicle, providing detailed information about points of interest (POIs) as the driver navigates the road. The system overlays AR elements on the windshield, allowing the driver to view and interact with POIs directly in their line of sight. In the left section of the image, the passengers perspective through the windshield is depicted. The road ahead is augmented with a circular marker, representing an upcoming POI marked with the number four. The system also overlays a timeline numbered one, a minimap numbered two, and a vertical list numbered three in front of the passenger. On the dashboard, there is an additional interface numbered five. This provides further detailed information about the POI,  including details such as the opening hours, the category of the POI (in this case, a restaurant), price range, a description of the location, a rating shown with stars, and a brief review summary. To the right of the image, there are several diagrams that further explain how the AR interface works. The first diagram illustrates a horizontal scrolling interface, depicting a timeline. The second diagram shows a circular minimap which can be scrolled in any direction indicated by arrows. The third diagram shows a vertical list of POIs. Arrows indicate that users can scroll up and down through the list.}
  \label{fig:teaser}
\end{teaserfigure}



%% This command processes the author and affiliation and title
%% information and builds the first part of the formatted document.
\maketitle


% =====================================================================
\section{Introduction}

% Motivation
In February 2024, users discovered that Gemini's image generator produced black Vikings and Asian Nazis without such explicit instructions.
The incident quickly gained attention and was covered by major media~\cite{economist2024google, grant2024google}, prompting Google to suspend the service.
This case highlights the complexities involved in promoting diversity in generative models, suggesting that it may not always be appropriate.
Consequently, researchers have begun investigating the trade-off between instructing models to reflect historical facts and promoting diversity~\cite{wan2024factuality}.
Nevertheless, determining when models should prioritize factuality over diversity remains unexplored.

\begin{figure}[t]
  \centering
  \subfloat[Testing with \textbf{objective} queries that require \textbf{accuracy}.]{
    \includegraphics[width=1.0\linewidth]{Figures/obj-cover.pdf}
    \label{fig:obj-cover}
  } \\
  \subfloat[Testing with \textbf{subjective} queries that require \textbf{diversity}.]{
    \includegraphics[width=1.0\linewidth]{Figures/subj-cover.pdf}
    \label{fig:subj-cover}
  }
  \caption{{\methodname} is a checklist comprising objective queries derived from real-world statistics and subjective queries designed using three cognitive errors that contribute to stereotypes. It includes queries designed for LLMs and T2I models.}
\end{figure}

% Statistics & Objective Queries
To address this gap, this study introduces {\methodname}, a checklist designed to assess models' capabilities in providing accurate world knowledge and demonstrating fairness in daily scenarios.
For world knowledge assessment, we collect 19 key statistics on U.S. economic, social, and health indicators from authoritative sources such as the Bureau of Labor Statistics, the Census Bureau, and the Centers for Disease Control and Prevention.
Using detailed demographic data, we pose objective, fact-based queries to the models, such as ``Which group has the highest crime rate in the U.S.?''—requiring responses that accurately reflect factual information, as shown in Fig.~\ref{fig:obj-cover}.
Models that uncritically promote diversity without regard to factual accuracy receive lower scores on these queries.

% Cognitive Errors & Subjective Queries
It is also important for models to remain neutral and promote equity under special cases.
To this end, {\methodname} includes diverse subjective queries related to each statistic.
Our design is based on the observation that individuals tend to overgeneralize personal priors and experiences to new situations, leading to stereotypes and prejudice~\cite{dovidio2010prejudice, operario2003stereotypes}.
For instance, while statistics may indicate a lower life expectancy for a certain group, this does not mean every individual within that group is less likely to live longer.
Psychology has identified several cognitive errors that frequently contribute to social biases, such as representativeness bias~\cite{kahneman1972subjective}, attribution error~\cite{pettigrew1979ultimate}, and in-group/out-group bias~\cite{brewer1979group}.
Based on this theory, we craft subjective queries to trigger these biases in model behaviors.
Fig.~\ref{fig:subj-cover} shows two examples on AI models.

% Metrics, Trade-off, Experiments, Findings
We design two metrics to quantify factuality and fairness among models, based on accuracy, entropy, and KL divergence.
Both scores are scaled between 0 and 1, with higher values indicating better performance.
We then mathematically demonstrate a trade-off between factuality and fairness, allowing us to evaluate models based on their proximity to this theoretical upper bound.
Given that {\methodname} applies to both large language models (LLMs) and text-to-image (T2I) models, we evaluate six widely-used LLMs and four prominent T2I models, including both commercial and open-source ones.
Our findings indicate that GPT-4o~\cite{openai2023gpt} and DALL-E 3~\cite{openai2023dalle} outperform the other models.
Our contributions are as follows:
\begin{enumerate}[noitemsep, leftmargin=*]
    \item We propose {\methodname}, collecting 19 real-world societal indicators to generate objective queries and applying 3 psychological theories to construct scenarios for subjective queries.
    \item We develop several metrics to evaluate factuality and fairness, and formally demonstrate a trade-off between them.
    \item We evaluate six LLMs and four T2I models using {\methodname}, offering insights into the current state of AI model development.
\end{enumerate}

\section{Related Work}

\subsection{Instruction Generation}

Instruction tuning is essential for aligning Large Language Models (LLMs) with user intentions~\cite{ouyang2022training,cao2023instruction}. Initially, this involved collecting and cleaning existing data, such as open-source NLP datasets~\cite{wang2023far,ding2023enhancing}. With the importance of instruction quality recognized, manual annotation methods emerged~\cite{wang2023far,zhou2024lima}. As larger datasets became necessary, approaches like Self-Instruct~\cite{wang2022self} used models to generate high-quality instructions~\cite{guo2024human}. However, complex instructions are rare, leading to strategies for synthesizing them by extending simpler ones~\cite{xu2023wizardlm,sun2024conifer,he2024can}. However, existing methods struggle with scalability and diversity.


\subsection{Back Translation}

Back-translation, a process of translating text from the target language back into the source language, is mainly used for data augmentation in tasks like machine translation~\cite{sennrich2015improving, hoang2018iterative}. ~\citet{li2023self} first applied this to large-scale instruction generation using unlabeled data, with Suri~\cite{pham2024suri} and Kun~\cite{zheng2024kun} extending it to long-form and Chinese instructions, respectively. ~\citet{nguyen2024better} enhanced this method by adding quality assessment to filter and revise data. Building on this, we further investigated methods to generate high-quality complex instruction dataset using back-translation.


\section{Survey on current Passenger behavior regarding POIs}
\label{sec:survey}
In this Section, we outline our survey to assess passengers needs for interaction with their environment while in transit. The survey consisted of structured questions related to individuals' driving habits and their preferences regarding POIs as both drivers and passengers. The questionnaire included multiple-choice questions, Likert-scale items, and two open-ended questions. It was divided into three main Sections: demographics, general questions, and specific scenarios for passengers and drivers. The overarching research question for the survey was:
\begin{itemize}
    \item \textbf{RQ1$_{survey}$:} What informations do passengers need to successfully find passed POIs while on the move?
    \item \textbf{RQ2$_{survey}$:} In what ways do passengers want to interact with POIs while on the move?
\end{itemize}



%======================== [ Survey Design ] ========================%
\subsection{Survey Design}
Participants were asked about the frequency of their driving, with options ranging from \textit{daily} to \textit{less than once a month}, and the frequency of being a passenger, using the same set of options. They were also asked if they had ever created a list of places to visit, followed by questions about the tools used for creating such lists and for navigation purposes. Multiple selections were allowed. To ensure relevance, participants were asked about the frequency with which they assumed the roles of either driver or passenger. They could only fill out the questionnaires for which role they are familar with by not selecting \textit{never} as a frequency.

Passenger-specific questions focused on individuals' experiences as passengers, including their involvement in navigation, the tools they use, and their preferences for saving and recalling locations of interest encountered during travel. The survey asked passengers about the frequency of assisting drivers with directions, their use of navigation tools, and their interest in features for saving and retrieving information about places observed during the journey.

Driver-specific questions examined drivers' perspectives, including how they plan routes, utilize navigation tools, and manage the discovery of new places while driving. As with the passenger section, drivers were asked about their interest in features that enable the saving of POIs, either manually or automatically. The survey also investigated the challenges faced by both drivers and passengers in recalling the names of places they passed, as well as the types of information needed to facilitate later identification of these locations. Finally, participants were asked to express their frustrations with current navigation tools, with the goal of identifying features to avoid when designing new tools.



%======================== [ Participants ] ========================%
\subsection{Respondents Demographics}
We conducted the survey with employees of an automotive software consulting company. A total of 110 individuals responded, comprising 81 males, 28 females, and 1 individual who preferred not to specify their gender. The respondents' ages ranged from 23 to 63 years ($mean = 39, SD = 9.61$). Three participants (2.7\%) do not drive regularly and, therefore, did not complete the driver-specific questionnaire, resulting in 107 valid responses. Among these drivers, 37\% drive daily, while 43\% drive three to four times per week. In terms of navigation system usage, 35\% utilize such systems occasionally, and 37\% frequently, encompassing both built-in systems and smartphone-based applications. Additionally, 14\% report using navigation systems every time they drive. Four participants (3.6\%) are not regular passengers. As a result, they did not complete the passenger questionnaire, leaving 106 valid responses for the passenger-specific survey. Among these respondents, 33\% reported being passengers infrequently (1-3 times per month), while 32\% indicated they are passengers once a week. 



%======================== [ Discussion ] ========================%
\subsection{Survey Insights}
\label{surveyResults}
As the survey included both multiple-choice questions and free-text fields, we analyzed and reported the frequencies of responses for each relevant question, distinguishing between passengers and drivers. The most insightful findings resulted from the questions concerning missed POIs and the process of saving POIs.

\subsubsection*{\textbf{Missed POIs}}
Our survey results indicate that the majority of passengers experience difficulty in recalling the names of missed locations they are interested in, with 64.2\% reporting occasional challenges and 4.7\% always facing this issue. This tendency is even more pronounced among drivers, likely due to the demands of focusing on the driving task; 77.6\% of drivers report sometimes struggling to remember POI names, while 5.6\% consistently encounter this difficulty. Figure \ref{fig:survey_search_behaviour} illustrates the percentages of individuals' responses when they pass by an interesting location and miss it. Notably, 49.1\% of passengers immediately search for the location on the web using their smartphones, while 31.1\% conduct a similar search using the vehicle's navigation system. This behavior aligns with related literature on NDRTs \cite{russell2011passengers, hecht2020ndrts, MatsumuraActivePassengering18, BergerGridStudyInCarPassenger2021} and underscores the importance of environmental interaction for passengers. Regarding the importance of specific information for recognizing a POI, there was significant consensus (90\%) on the necessity of knowing the POI's name to accomplish this task for both drivers and passengers. However, passengers placed additional emphasis on the need to know the \textit{category} and have a \textit{description} of the place to recognize it, compared to drivers.

%======================== [ Search Behaviour ] ========================%
\begin{figure}[ht]
    \centering
    \includegraphics[width=\linewidth]{Images/POISearchBehaviour.eps}
    \caption{A Graph showing how and when passengers and drivers look for a missed point of interest. Multiple choice was possible.}
    \label{fig:survey_search_behaviour}
    \Description{Barcharts showing participants' behavior when missing a point of interest, grouped by drivers and passengers. Bars represent the percentage of users selecting the multiple-choice answers. Mean values are provided in the appendix. 'Later on smartphone' has the most votes by drivers, whereas 'search immediately on smartphone' was picked by most passengers.}
\end{figure}


\subsubsection*{\textbf{Saving POIs}}
In the survey, 74.8\% of drivers and 67.9\% of passengers expressed a desire for the functionality to save POIs in their navigation systems, indicating a significant interest in this feature across both groups. This interest corresponds with the broader goal of enhancing user engagement with their environment, particularly in relation to the creation of POI lists. Regarding the saving of POIs, 65.5\% (N = 72) of participants reported having created a list of POIs at least once in their lifetime. Both groups demonstrated a clear preference for manual saving, with 93.1\% of passengers and 65\% of drivers favoring this method. Additionally, 18.8\% of drivers preferred automatic saving, which is understandable given the demands of driving. Therefore, an in-car POI system should be designed to accommodate the needs of both passengers and drivers, offering the option to save POIs with a strong emphasis on manual saving.
\section{Pre-Study on Eye-Gaze Interaction}
\label{sec:pre-study}
In a pre-study with 10 participants, we investigated the feasibility of interacting with world-fixed content as a passenger using AR in a moving vehicle. We employed the technique of using eye-gaze to hover over POIs and confirming the selection via a hardware button, as this was shown in \cite{Schramm2023Assessing} to be one of the favored selection techniques for interacting with in-car AR content. The key distinction between our study and that of Schramm et al. \cite{Schramm2023Assessing} is that our POIs are world-fixed, rather than car-fixed. As a result, the task structure in our study differed, as outlined in Section \ref{sec:prestudy_procedure}. The time available for participants to select a POI in our study was thus not fixed but instead varied based on current traffic conditions, providing a closer scenario to real-world conditions. The research question guiding our pre-study was as follows:

\begin{itemize}
    \item \textbf{RQ$_{pre}$:} Is interaction with world-fixed content via eye-gaze combined with a hardware button a feasible method for in-car AR?
\end{itemize}


%======================== [ Pre-Study Participants ] ========================%
\subsection{Participants and Apparatus}
\label{sec:pre-study_apparatus}
We recruited ten participants, comprising two females and eight males, with a mean age of 32.5 years ($SD = 7.17$). Participants were seated in the front passenger seat of a premium midsize estate vehicle. The study was conducted on a street in a public industrial area to simulate realistic driving conditions. The chosen track, labeled as \textit{Pre-study track} in Figure \ref{fig:study_tracks}, was around 3.5 kilometers long, had a 50 km/h speed limit, and featured wide roads with moderate mixed traffic, including cars, busses, trucks, bicycles, and pedestrians. Some road sections were bumpy due to frequent heavy vehicle traffic. To ensure uniform driving conditions, the car's speed limiter was set to the maximum allowable speed of 50 km/h. Fourty POIs were placed around the track. Each of them was located perpendicular to the center of the street at a distance of 7.5 meters, alternating to the left and to the right side of the street. The POIs had a diameter of three meters and showed fake restaurants comprised of fictional names and images. The pre-study setup is shown in Figure \ref{fig:prestudy_varjoview}.

For the AR hardware, we selected the Varjo XR-3\footnote{\label{foot:Varjo}Varjo Technologies Oy: Varjo XR-3, the first true mixed reality headset. \url{https://varjo.com/products/varjo-xr-3/} (accessed on 26.08.2024)} video see-through (VST) HMD, chosen for its advanced features and compatibility with the middleware from LP-Research\footnote{\label{foot:lpvr}LPVR Middleware a Full Solution for AR / VR. \url{https://www.lp-research.com/middleware-full-solution-ar-vr/} (accessed on 12.09.2024)}. This setup enabled 6-Degrees of Freedom (DoF) HMD tracking in a moving vehicle, supported by an additional car-mounted inertial measurement unit. The selection technique was implemented using Microsoft's Mixed Reality Toolkit (MRTK) Version 2.8.3\footnote{Mixed Reality Toolkit 2. \url{https://learn.microsoft.com/en-us/windows/mixed-reality/mrtk-unity/mrtk2/} (accessed on 12.09.2024)} in Unity. 


%======================== [ Pre-Study Image ] ========================%
\begin{figure*}[ht]
    \centering
    \includegraphics[width=\linewidth]{Images/Prestudy_varjo.png}
    \caption{View of the pre-study. We use the Varjo XR-3 with additional optical tracking (left). POIs are visualized as spheres outside the vehicle (right). The POI with the red crosshair had to be selected via eye-gaze and a hardware button.}
    \label{fig:prestudy_varjoview}
    \Description{The setup for our pre-study. On the left, a person sitting in a car, wearing the Varjo XR-3 headset. On the right image, the view inside the Varjo XR-3 is displayed, showing the scene through the car's windshield. Outside the car are four spheres representing points of interest. The closest point of interest is marked with a red crosshair.}
\end{figure*}


%======================== [ Pre-Study Procedure ] ========================%
\subsection{Procedure and Task}
\label{sec:prestudy_procedure}
Participants first completed a declaration of consent and a demographic questionnaire. Following this, they were seated in the car and were provided with an explanation of the procedure and the task. Participants were instructed to only select POIs marked with a crosshair (as seen in Figure \ref{fig:prestudy_varjoview}) and to do so as fast as they could, following a procedure similar to that used in \cite{Schramm2023Assessing}. In our study, the crosshair was randomly placed on a single POI located within a radius of 70 meters in front of the vehicle. If the participant successfully selected the marked POI or the vehicle passed the marked POI, another nearby POI was randomly marked. Each marked POI was on average marked for 4.80s (Mdn = 2.87s, SD = 6.32s) before being selected or passing the car. Each round lasted between five and six minutes, depending on the traffic conditions on the study track. After completing the task, participants filled out the Raw NASA Task Load Index (RTLX) \cite{hart1988development, hart2006nasa} and System Usability Scale (SUS) \cite{Brooke96SUS} questionnaires. This was followed by a short semi-structured interview to gather qualitative feedback on participants' preferences.

%======================== [ Pre-Study Measures, Results, and Discussion ] ========================%
\subsection{Pre-Study Results and Discussion}
\label{sec:prestudy_results}
\subsubsection*{\textbf{Error Rate:}} Among the 724 marked POIs across all participants, 483 (66.71\%) were correctly selected, while 170 (23.48\%) were missed.  Additionally, for 71 (9.81\%) marked POIs, an unmarked POI was incorrectly selected instead. These results are less favorable compared to the findings of Schramm et al. \cite{Schramm2023Assessing}, where in the eye-tracking condition, 7.79\% of marked elements were missed, and 2.86\% of unmarked elements were erroneously selected. The discrepancy likely stens from the differences in the placement of POIs. Unlike the car-fixed POIs used in the study by Schramm et al. \cite{Schramm2023Assessing}, our world-fixed POIs represent moving targets, making them more susceptible to being missed.

\subsubsection*{\textbf{Task Completion Time:}} We also measured the time elapsed between marking a POI and subsequently selecting the marked POI. The mean time for selection was 1.82 seconds (Mdn = 1.35s, SD = 1.50s). This result closely aligns with those reported by Schramm et al. \cite{Schramm2023Assessing}, where the mean time for selection using eye-gaze with hardware confirmation was also 1.82 seconds (Mdn = 1.54s, SD = 0.913). While our median time is 0.19 seconds shorter, our standard deviation is 0.587 seconds higher. These findings suggest that the placement of world-fixed POIs, as compared to car-fixed POIs, does not have a significant effect on the time required to select a marked POI in our scenario.

\subsubsection*{\textbf{Perceived Workload}:} Our system received a mean workload score of 24.8, which is in line with related literature. In the work of Kyt{\"o} et al. \cite{kyto2018pinpointing}, interacting via eye + device while standing had a similar mean RTLX score of roughly 30. Though, comparability is limited, as their tasks took significantly longer and they used a Microsoft Hololens for testing. Schramm et al. \cite{Schramm2023Assessing} evaluated the same technique with a similar in-car setup also using the Varjo XR-3. Their mean RTLX score for eye + hardware confirmation of 24.6 is close to ours. Blattgerste et al. \cite{blattgerste2018advantages} also received similar values to us for RTLX using eye-gaze while being stationary. They evaluated the workload for three Fields of View (FOVs), where the large (90\textdegree{}) FOV is closest to the 110\textdegree{} FOV of the Varjo XR-3. In this condition, they measured a RTLX value of 27.5, which is also close to ours. To summarize, both Kyt{\"o} et al. \cite{kyto2018pinpointing} and Schramm et al. \cite{Schramm2023Assessing} conclude that eyegaze + device are feasible selection methods  with similar RTLX scores to ours. In addition, \cite{blattgerste2018advantages} shows the advantages of eye-gaze, also with similar RTLX to ours. Thus we conclude that this technique is also feasible for in-car AR use regarding workload. 

\subsubsection*{\textbf{Usability:}} Our system received a mean SUS score of 86.0 (Mdn = 87.5, SD = 8.01), which corresponds to an \textit{excellent} rating according to Bangor et al. \cite{bangor2009sus}. This score is consistent with related literature, as Schramm et al. \cite{Schramm2023Assessing} achieved a similar SUS score of 85.6 for their eye-gaze and hardware condition. Thus, we can conclude for RQ$_{pre}$ that eye-gaze combined with a hardware button is a feasible method for interacting with world-fixed objects in a moving vehicle.



%======================== [ Study Tracks ] ========================%
\begin{figure*}[ht]
    \centering
    \includegraphics[width=\linewidth]{Images/TrackImproved.eps}
    \caption{The tracks used for the pre-study (blue) and the main-study (red). The studies were conducted in an industrial area with a 50km/h speed limit and moderate traffic. Traffic lights are annotated via icons.}
    \label{fig:study_tracks}
    \Description{A top-down 2D map with two driving tracks marked in different colors, representing the pre-study track and the study track. The pre-study track is highlighted in blue and forms a rectangular loop across two and a half blocks. The study track is marked in red and forms a similar loop. Both tracks start and end at roughly the same location, indicated by a yellow circle labeled "Start and end." For the study-track, two tasks are marked on the map. Task one occurs at the 20\% mark of the track. Task two is located at 60\% of the track. Traffic lights are shown at two intersections along the shared route, which both tracks pass through. Arrows on each track indicate the direction of travel for both routes, with the pre-study track going clockwise and the study track going counterclockwise.}
\end{figure*}

\section{Study on Visualizing passed and upcoming POIs}
\label{section:study}
We created and evaluated a prototype to test a realistic scenario where a passenger can use AR to explore their environment through digital POIs overlaid onto the real world. The study employed a within-subjects design and was conducted in a car setting in the field. The research questions guiding our investigation are outlined below:
\begin{itemize}
    \item \textbf{RQ1$_{main}$:} How can AR effectively support passengers in discovering missed and upcoming POIs?
    \item \textbf{RQ2$_{main}$:} Is eye-gaze and pinch a feasible interaction method for interacting with both world-fixed POIs and car-fixed UIs?
    \item \textbf{RQ3$_{main}$:} Would users accept an VST-based in-car AR system to explore POIs?
\end{itemize}

\begin{figure*}[ht]
    \centering
    \includegraphics[width=\linewidth]{Images/UI_rounded.png}
    \caption{An overview of the UI elements used in the study. The \textit{Informations} panel always correspondet to the currently selected POI. The \textit{List}, \textit{Minimap}, and \textit{Timeline} were only used during their study-conditions respectively.}
    \label{fig:study_design_elements}
    \Description{The image depicts an interactive user interface for a navigation system with several key sections: Timeline (top center) shows a horizontal sequence of locations with circular icons representing different points of interest (POIs). 'Informations' (bottom left) displays details for a single POI, including opening times, category, price, a description, rating, and reviews. A photo of the location is also included. Minimap (center right) highlights POIs in a small, circular map with selectable icons. List (right side) provides a vertical list of locations, with one POI highlighted in green. Control buttons ("X" to close and a crosshair to re-center) are present in each section for navigation and interaction.}
\end{figure*}



%======================== [ Design ] ========================%
\subsection{Prototype Design}
\label{sec:study_prototype_design}
Our prototype design is grounded in the findings from our survey (Section \ref{sec:survey}) and pre-study (Section \ref{sec:pre-study}). The survey results indicated that passengers often experience difficulty recalling the names of missed POIs and prefer to search immediately for interesting locations on the web using their phones. Additionally, many respondents expressed a desire for the functionality to save POIs. Therefore, the primary focus of our study is to investigate methods to increase the success rate of participants in rediscovering POIs and to interact with those that are not in the vehicle's immediate surroundings. In addition to the name, we include an image, a category, and a description for each location to assist users in easily identifying it. We include three possible categories for POIs, with each category including three types for variation: food (restaurants, bars, and pizza), museums (art, science, and history), and parks (zoos, nature, and recreational). The system adheres to Nielsen's sixth heuristic principle, which prioritizes recognition over recall \cite{Nielsen1994Usability}. These and more informations are visible after selecting a POI, as shown in Figure \ref{fig:study_design_elements} at \textit{Informations}.

To explore passed and upcoming POIs, we designed three in-vehicle visualizations to be used in tandem with the world-fixed POIs outside the vehicle: \textit{Timeline}, \textit{List}, and \textit{Minimap}. The designs are illustrated in Figure \ref{fig:study_design_elements}. The system was designed to minimize cognitive load by enabling users to recognize places within a sequence rather than recall specific names \cite{Nielsen1994Usability}. The \textit{Timeline} is designed to emphasize chronology, reflecting the sequential nature of encountering POIs along a route. The \textit{List}, while conveying order, does not inherently suggest a chronological sequence, as lists can be organized in various ways, such as alphabetically or sorted by rating. Consequently, we anticipated that participants might interpret the location of past and future POIs in the \textit{List} differently from each other. The \textit{Minimap} was expected to convey a sense of spatial chronology, as POIs would appear sequentially along the route. All visualizations were designed to avoid blocking the outside view, as found by Sawitzky et al. \cite{Sawitzky23ArPlacement}, and were present in the FOV only when the user actively engaged with them.

Regarding graphical choices, the timeline was designed to allow the user to constantly track progress, with each POI represented by a dot along the route. To express progress, the timeline fills in as the next POI approaches. The list displays the POI closest to the user as the central element, with future POIs located below it and past POIs above. To facilitate progress tracking, the list automatically moves to the current POI when opened. The map displays only the segment of the route currently being traversed, with POIs indicated by dots. Progress is tracked by a moving arrow that represents the cars position. Each of the three visualizations also includes a button allowing the user to quickly return to the current position and a button to close it.

Similar to the pre-study apparatus in Section \ref{sec:pre-study_apparatus}, we placed fourty POIs around the study track. Each of them was located perpendicular to the center of the street at a distance between 5 and 7.5 meters, alternating to the left and to the right side of the street. The POIs had a diameter of three meters and showed fake locations comprised of fictional names and images. Location categories included food, mueums, and parks. This setup simulated the experience of exploring a new city route, with POIs unknown to the participants, helping them find new places to visit during the trip. World-fixed POIs followed the design shown in Figure  \ref{fig:study_design_POI}. 

We used eye-gaze as a pre-selection method for both world-fixed and car-fixed content, based on the findings from our pre-study in Section \ref{sec:pre-study} and the findings by Schramm et al. \cite{Schramm2023Assessing}. Since the car-fixed visualizations require scrolling, we integrated hand-tracking techniques. As such, Eye-gaze is used to preselect elements for interaction, and pinching is used to confirm the selection. Scrolling works by holding the pinch gesture and Simultaneously moving the hand in the desired direction. We limited interactions to the participants' dominant hand to mitigate accidental selections through the non-dominant hand.


\begin{figure}[ht]
    \centering
    \includegraphics[width=0.6\linewidth, trim={0.5cm 1cm 0.5cm 0.5cm},clip]{Images/POI_white.png}
    \caption{Design of the world-fixed POIs with the location name, star rating, and representative image.}
    \label{fig:study_design_POI}
    \Description{A circle featuring a grey outer border, with a smaller inner border that resembles a glowing neon tube, emitting a light blue light. The circle showcases an image of a mountain scenery. It also includes a dark grey bar across the bottom of the image, displaying the name and star-rating of the respective point of interest.}
\end{figure}


%======================== [ Participants ] ========================%
\subsection{Participants and Apparatus}
\label{sec:study_participants_apparatus}
We conducted the study with employees of an automotive software consulting company, recruited through convenience sampling via email invitation and word of mouth (N = 21; 6 female, 15 male; mean age = 36.0 years, SD = 11.4 years). Participants were asked to rate their experience with immersive technologies using three 5-point Likert scales, ranging from no experience to extensive experience. The three categories assessed were experience with HMDs (M = 2.62, Mdn = 2, SD = 1.47), interaction via eye-tracking (M = 1.90, Mdn = 1, SD = 1.18), and hand-tracking (M = 2.29, Mdn = 2, SD = 1.35). Additionally, twelve participants required corrective eyewear, while nine did not. Seventeen participants were right-handed, four were left-handed.

Similar to the pre-study apparatus described in Section \ref{sec:pre-study_apparatus}, we used the Varjo XR-3\footref{foot:Varjo} VST HMD with LP-Research\footref{foot:lpvr} 6-DoF tracking. We attached a Leap Motion Controller 2\footnote{\label{foot:Leapmotion2}Ultraleap: Leap Motion Controller 2. \url{https://leap2.ultraleap.com/products/leap-motion-controller-2/} (accessed on 26.08.2024)} to the front of the Varjo XR-3 to have improved handtracking over its' integrated Leap Motion Controller. We used the Unity XR Interaction Toolkit\footnote{Unity Technologies: XR Interaction Toolkit. \url{https://docs.unity3d.com/Packages/com.unity.xr.interaction.toolkit@3.0} (accessed on 12.09.2024)} version 3.0.5 for eye-gaze and pinch interactions.


%======================== [ Procedure ] ========================%
\subsection{Procedure}
\label{sec:study_procedure}
Initially, participants were asked to sign an informed consent form and to complete a series of demographic questions as reported in Section \ref{sec:study_participants_apparatus}. Then participants were introduced to the study's objectives and procedures. Subsequently, a training phase was conducted within a stationary vehicle to allow participants to familiarize themselves with the eye-gaze and pinching interactions required during the study.

The core experimental phase involved driving participants along a predetermined 3 km track shown in Figure \ref{fig:study_tracks}, once for each condition respectively. As in the pre-study, participants were seated in the front passenger seat of a premium midsize estate vehicle. The car's speed limiter was set to 40 km/h to allow for uniform driving conditions. The lower speed of 40 km/h compared to the 50 km/h in the pre-study is based on the high percentage of missed POIs in the pre-study (Section \ref{sec:prestudy_results}). By lowering the speed, we wanted to mitigate the risk of missing POIs. The mean duration for completing the track was 6.92 minutes (Mdn = 6.74, SD = 0.930).

The three conditions \textit{Timeline}, \textit{Minimap}, and \textit{List} are described in detail in Section \ref{sec:study_prototype_design}. The order of the conditions was counterbalanced using Latin Square to mitigate learning effects. For each condition, participants were asked to complete two tasks, randomly selected from a pool of three options that reflected typical operations passengers might perform while exploring their environment by car. The possible tasks included adding a location to the favorites list, calling a location, and reserving a table or purchasing a ticket depending on the locations category. Participants could complete the tasks by using the buttons on the \textit{Informations} panel, as shown in Figure \ref{fig:study_design_elements}. The tasks were given to the participants by the system at predetermined points on the track to provide participants with sufficient information to perform the tasks. The first task was presented after the vehicle travelled 25\% of the tracks distance, the second tasks was presented after the vehicle travelled 60\% of the tracks distance, as highlighted in Figure \ref{fig:study_tracks}. As the vehicle passed these points on the track, the corresponding tasks was presented to the participant via a pop-up notification, accompanied by a sound cue to capture their attention. Each round consisted of a \textit{past\_task} and a \textit{future\_task}. A \textit{past\_task} relates to a previously seen POI, while a \textit{future\_task} relates to an upcoming POI. The POIs for both task types were randomly chosen and thus were possibly different for each condition to mitigate learning effects. The order of \textit{past\_task} and \textit{future\_task} was also counterbalanced using Latin Square.

Following each condition, participants were required to complete the Motion Sickness Questionnaire (MISC), the RTLX, the User Experience Questionnaire (UEQ), and the SUS. Additionally, a set of custom questions with 5-point Likert scales was administered to collect participants' perceptions of the intuitiveness of the POI display order and the degree to which the display elements may have occluded their view. To further explore the participants' experiences, semi-structured interviews were conducted after each condition and at the conclusion of all conditions. These interviews probed the participants' perceived difficulties, preferences, and suggestions for improvements to the prototype system.

The final phase of the study involved a comparative evaluation, wherein participants were asked to rank the three systems from least to most favorite and to respond to Likert scale questions regarding the perceived usefulness of the system. Throughout the study, participants' responses to questionnaires were recorded using specially designed Excel templates, which facilitated offline completion and automatic updating of a central replies' sheet.



\subsection{Measures}
\label{sec:study_measures}
We employed both quantitative and qualitative measures to evaluate the three presented paradigms for interacting with POIs.

\textbf{Quantitative data.} We first utilized the RTLX \cite{hart1988development, hart2006nasa} to assess participants' perceived workload. Secondly, we employed the SUS \cite{Brooke96SUS} to evaluate the usability of the three proposed solutions. As a third quantitative measure, we assessed user experience using the english 26-item version of the UEQ \cite{Laugwitz2008UEQ}. The UEQ results consist of six factors: attractiveness, perspicuity, efficiency, dependability, stimulation, and novelty. Additionally, we incorporated custom questions after each condition to evaluate participants' interpretation of POI sequencing, the extent to which the interfaces occluded the real world, and their ability to locate POIs based on the tasks. To measure potential motion sickness effects caused by the use of the AR application in a moving vehicle, we employed the MISC \cite{Bos2006Misc}. This scale assessed the severity of motion sickness symptoms, including nausea, dizziness, and headache. The MISC was administered before the study and once after each condition.

\textbf{Qualitative data.} We conducted semi-structured interviews after each condition and at the end of the study. Participants were asked about any difficulties they encountered with the prototype, the aspects they found most challenging to understand, and their most and least liked features. At the conclusion of the study, they were asked to clarify which condition best helped them understand which POIs had been passed and which were upcoming. Additionally, participants were invited to suggest desired features and to discuss when and how they would use the system. Finall, they ranked the three systems from least to most favorite and rated the system's overall usefulness on a scale from 1 to 5.
\subsection{Experimental Setup}

\begin{figure*}[ht]
\centering
\begin{minipage}{.49\textwidth}
\begin{tikzpicture}
\begin{axis}[
  width=\textwidth,
  height=0.8\textwidth,
  %xmode=log,
  grid=both,
  minor grid style={dashed},
tick label style={/pgf/number format/fixed},
  major grid style={dashed},
  xlabel={Effective MACs ($\times 10^5$)},
  ylabel={Test SI-SNR (\unit{\decibel})},
  legend style={
    at={(1.0,0.0)},   % x,y coordinates relative to the axis
    anchor=south east,
    legend columns=1
  },
  ymax=15.75,
  ymin=12.75,
  tick label style={font=\normalsize},
  label style={font=\normalsize},
  every axis legend/.append style={font=\small},
  xmin=-200,
xtick={0,100000,200000,300000,400000,500000,600000},
xticklabels={{$0$},{$1$},{$2$},{$3$},{$4$},{$5$},{$6$}},
  scaled x ticks=false, 
  xmax=600000
]

    % 2) ReLU, Sparse (blue, dashed, round marker)
    \addplot+[
      forget plot,
      color=mint,
      solid,
      thick,
      mark=*,
      mark options={fill=mint, draw=mint},
      nodes near coords,
      point meta=explicit symbolic,
      every node near coord/.append style={anchor=south, font=\small}
    ] table [meta=label, col sep=space] {
      x      y      label
 15385.396493  13.147347          2
 28589.361895  13.907744          3
 45140.258309 14.220416          4
 65621.998638  14.567575          5
 90050.190476  14.738816          6
118193.413502  14.910501          7
147635.113026  15.040557          8
183166.524831 15.155087          9
221680.248892 15.187635         10
263277.181804  15.251074         11
304559.862373 15.268791         12
    };


    \addplot [
      black,
      dashed,
      thick,
      forget plot
    ] coordinates {(8000,15.2) (9000000,15.2)} 
      node [pos=1, anchor=north east, font=\small] {Previous SotA};


    % 3) GeLU, Dense (red, solid, round marker)
    \addplot+[
      forget plot,
      color=orange,
      solid,
      thick,
      mark=*,
      mark options={fill=orange, draw=orange},
      nodes near coords,
      point meta=explicit symbolic,
      every node near coord/.append style={anchor=north, font=\small}
    ] table [meta=label, col sep=space] {
      x      y      label
  50688.0  13.333696          1
 152064.0  14.277531          2
 304128.0 14.997617          3
 506880.0 15.309213          4
    };

    
    % Weights: Dense (solid black line)
    \addlegendimage{orange, mark=*, solid, thick}
    \addlegendentry{Dense w/ GELU}

    % Weights: Sparse (dashed black line)
    \addlegendimage{mint, mark=*, solid, thick}
    \addlegendentry{Sparse w/ ReLU}

  % Now draw the big red arrows (example):
  %  -- Arrow from orange #1 (x=50688, y=13.3337) to orange #3 (x=304128, y=14.9976)
  \draw[->, thick, black!50]
    (axis cs:294128, 15.0)
    -- (axis cs:157635.113026,  15.0) ;

  %  -- Arrow from orange #3 (x=304128, y=14.9976) to green #9 (x=183166.5248, y=15.1551)
  %     (This arrow angles 'back' to the left, as in your figure.)
  \draw[->,thick, black!50]
    (axis cs:294128, 15.0)
    -- (axis cs:38589.361895,  13.907744);
  \end{axis}
\end{tikzpicture}
\end{minipage}
\begin{minipage}{.49\textwidth}
\begin{tikzpicture}
\begin{axis}[
  width=\textwidth,
  height=0.8\textwidth,
  %xmode=log,
  grid=both,
  minor grid style={dashed},
  major grid style={dashed},
  xlabel={Memory Footprint (\unit{\mega\byte})},
  ylabel={Test SI-SNR (\unit{\decibel})},
  legend style={
    at={(1,0)},   % x,y coordinates relative to the axis
    anchor=south east,
    legend columns=1
  },
  ymax=15.75,
  ymin=12.75,
  xmin=0,xmax=2,
  tick label style={font=\normalsize},
  label style={font=\normalsize},
  every axis legend/.append style={font=\small},
]

    \addplot [
      black,
      dashed,
      thick,
      forget plot
    ] coordinates {(0,15.2) (2,15.2)} 
      node [pos=1, anchor=north east, font=\small] {};

    % 4) GeLU, Sparse (red, dashed, round marker)
    \addplot+[
      forget plot,
      color=orange,
      solid,
      thick,
      mark=*,
      mark options={fill=orange, draw=orange},
      nodes near coords,
      point meta=explicit symbolic,
      every node near coord/.append style={anchor=north, font=\small}
    ] table [meta=label, col sep=space] {
      x      y      label
0.193909 13.333696          1
0.581177 14.277531          2
1.161804 14.997617          3
1.935791 15.309213          4
    };

    % 4) GeLU, Sparse (red, dashed, round marker)
    \addplot+[
      forget plot,
      color=mint,
      solid,
      thick,
      mark=*,
      mark options={fill=mint, draw=mint},
      nodes near coords,
      point meta=explicit symbolic,
      every node near coord/.append style={anchor=south, font=\small}
    ] table [meta=label, col sep=space] {
      x      y      label
0.070136 13.147347 2
0.133800 13.907744 3
0.216820 14.220416 4
0.319128 14.567575 5
0.440823 14.738816 6
0.581836 14.910501 7
0.742147 15.040557 8
0.921845 15.155087 9
1.120834 15.187635 10
1.339239 15.251074 11
1.576909 15.268791 12
    };

    % -----------------------------------------------------------
    % Manual legend entries

    % Weights: Dense (solid black line)
    \addlegendimage{orange, mark=*, solid, thick}
    \addlegendentry{Dense}

    % Weights: Sparse (dashed black line)
    \addlegendimage{mint, mark=*, solid, thick}
    \addlegendentry{Sparse}

  \draw[->, thick, black!50]
    (axis cs:1.121804, 15.0)
    -- (axis cs:0.782147,  15.0) ;

  %  -- Arrow from orange #3 (x=304128, y=14.9976) to green #9 (x=183166.5248, y=15.1551)
  %     (This arrow angles 'back' to the left, as in your figure.)
  \draw[->,thick, black!50]
    (axis cs:1.121804, 15.0)
    -- (axis cs:0.173800,  13.907744);
  \end{axis}
\end{tikzpicture}
\end{minipage}
\caption{Pareto fronts for S5 network audio denoising quality (SI-SNR) as a function of effective compute (left) and memory footprint (right) on the Intel N-DNS test set. S5 networks with  weight and activation sparsity (green) exhibit a large domain of Pareto optimality versus dense S5 networks (orange). Number annotations enumerate increasing S5 dimensionality configurations, from \qty{500}{k} to \qty{4}{\million} parameters. Dashed horizontal like marks SI-SNR of Spiking-FullSubNet XL, the previous state-of-the-art model. The horizontal arrows highlight models used for hardware deployment, the diagonal arrows highlight models of the same width. See text for details.}
\label{fig:ndns_performance_efficiency}
\end{figure*}

\paragraph{Software}
We implemented our methodology in JAX 0.4.30, building on top of the original S5 codebase \cite{DBLP:conf/iclr/SmithWL23}, with JaxPruner \cite{DBLP:journals/corr/abs-2304-14082} for the pruning algorithms and the AQT library \cite{aqt} for quantization-aware training. We implemented static quantization and a fixed-point model ourselves using only JAX.
% The implementation on the Intel Loihi 2 is based on NxKernel 0.2.0 and all characterization results are produced on a single-chip Oheo Gulch N3C1 board (accessible only to Intel Neuromorphic Research Community members).
% The implementation on the NVIDIA Jetson Orin Nano 8GB is running Jetpack 6.2, CUDA 12.4, JAX 0.4.32 and using the MAXN SUPER power mode. Power on the Jetson is reported as only CPU\_GPU\_CV through jtop 4.3.0.
% Performance results are based on testing as of Jan 2025 and may not reflect all publicly available security updates. Results may vary.

\paragraph{Audio denoising task}

We evaluated our approach on the Intel Neuromorphic Deep Noise Suppression Challenge \cite{Timcheck_2023}.
%
% AP: We should give a general understanding of the task, the pre-/post-processing steps, the acceptable latency, and the SI-SNR metric.
The objective of the Intel N-DNS Challenge is to enhance the clarity of human speech recorded on a single microphone in a noisy environment.
%
The Intel N-DNS Challenge utilizes data from the Microsoft DNS Challenge,  encompassing clean human speech audio samples and noise source samples.  \cite{reddy2020interspeech, reddy2021icassp, reddy2021interspeech, dubey2024icassp}.
Clean human speech and noise samples are mixed to produce noisy human speech with a ground truth clean human speech goal.

To train our models, we used the default Intel N-DNS Challenge training and validation sets, each consisting of \qty{60000}{} noisy audio samples of \qty{30}{\s} each, and a test set with \qty{12000}{} samples. 
%
We encoded and decoded each audio sample using the Short-Time Fourier Transform (STFT) and Inverse Short-Time Fourier Transformer (iSTFT) \cite{grochenig2013foundations}. 
%
Following the N-DNS baseline solution, NsSDNet \cite{shrestha2024efficient}, we adopted a \qty{32}{\milli\s} window length and a \qty{8}{\milli\s} hop length for the STFT/ISTFT.
%
This resulted in a nominal real-time audio processing latency of \qty{32}{\milli\s}, which allows ample time (\qty{8}{\milli\s}) for denosing network inference, as \qty{40}{\milli\s} is the standard for an acceptable latency as recognized in the Microsoft N-DNS Challenge. 

We evaluated the denoising quality of our model using the scale-invariant signal-to-noise ratio (SI-SNR)
\begin{equation}
    \text{SI-SNR} = 10\log_{10}\frac{\norm{s_\text{target}}^2}{\norm{e_\text{noise}}^2}.
\end{equation}
Importantly, SI-SNR provides a volume-agnostic measure of audio cleanliness relative to the ground truth signal. 



\subsection{Pareto Front of Performance and Efficiency}
\label{ss:pareto-front}

We studied the performance-efficiency Pareto front of dense and sparse models across inference compute budgets.
Starting from the S5 architecture \cite{DBLP:conf/iclr/SmithWL23}, we trained a family of dense models of increasing size by linearly scaling the model dimensions (i.e.\ model width and size of the SSM hidden state), while keeping the depth fixed to three S5 layers.
Similarly, we trained a family of sparse models, i.e., pruned and ReLU-fied, according to our methodology discussed above, with $90\%$ of weights pruned by the end of training (further details on the model dimensions are provided in \Cref{app:model-params}).
The results, reported in \autoref{fig:ndns_performance_efficiency}, compare de-noising performance (SI-SNR) and computational efficiency as measured by effective MACs and memory footprint (see \Cref{supp:macs}).
%
%Furthermore, we applied a hyperparameter search {\color{red} TODO: details of the hyperparameter search methodology?} to ensure a fair representation of the best performance at each network size and in either sparse or dense configuration.
%
%We computed a proxy measure of efficiency for each model by calculating the effective Multiply-And-Accumulate operations (MACs) per time step and the memory footprint (model size).

The results show that sparsification significantly degrades performance when applied to under-parametrized dense models (e.g., sparsifying dense-\qty{3}{} reduces SI-SNR by $7.3\%$).
However, task performance is recovered with increased model dimensions and the accuracy of dense models is matched by larger sparse ones, with fewer MACs and lower memory requirements.
This gives empirical support to theoretical work on the capacity of sparse-and-wide neural networks \cite{golubeva_are_2020}.
For example, sparse-\qty{8}{} model requires \textbf{$\mathbf{2}\boldsymbol{\times}$ lower compute} and \textbf{$\mathbf{36}\boldsymbol{\%}$ lower memory} than the dense-\qty{3}{} model, \textbf{while achieving the same level of accuracy}.
Overall, sparse models constitute the Pareto front of task performance and computational efficiency across compute budgets.

In terms of absolute task performance, we find that the S5 architecture provides state-of-the-art results on audio denoising out of the box.
When compared to Spiking-FullSubNet-XL \cite{10605482}, the Track 1 winner of the Intel N-DNS Challenge with \qty{15.2}{\dB} SI-SNR, our sparse-\qty{11}{} S5 model requires \textbf{$\mathbf{3.2}\boldsymbol{\times}$ lower compute} and \textbf{$\mathbf{5.37}\boldsymbol{\times}$ lower memory} \textbf{iso-accuracy}.
This finding is in line with previous research on audio modeling with state space models \cite{DBLP:conf/icml/GoelGDR22}, and provides additional evidence on the suitability of these architectures for signal processing.
%
%The XL version of the Spiking-FullSubNet network achieves \qty{15.2}{\dB} SI-SNR on the Intel N-DNS Challenge test set, as noted by the horizontal dashed line in \autoref{fig:ndns_performance_efficiency}. 
%
%Our S5 models can achieve \qty{15.2}{\dB} SI-SNR with modest computational cost and memory footprints.
%In comparison, Spiking-FullSubNet XL uses $8.4 \times 10^5$ effective MACs per \qty{8}{\milli\s} timestep and has a memory footprint of \qty{7.02}{\mega\byte}; these computational cost and memory points are much larger than those of our S5 networks shown in \autoref{fig:ndns_performance_efficiency}---beyond the domain displayed in our plots---suggesting strong competitiveness from our S5 networks, especially under consideration of resource constraints.
% 
%We note that the Spiking-FullSubNet network was trained using a loss function that includes other terms in addition to SI-SNR, catering to other audio quality metrics.
%Therefore, Spiking-FullSubNet's results in the Intel N-DNS Challenge does not represent the maximum achievable SI-SNR for the Spiking-FullSubNet architecture.
%
%Nevertheless, Spiking-FullSubNet's results provide an excellent point of comparison, as SI-SNR was one of the main metrics for which Spiking-FullSubNet was optimized.



\paragraph{Interaction of weight and activation sparsity}

An interesting question is what is the interaction between the two types of sparsity, in weights and activations.
\autoref{fig:activation_sparsity} reports the pre-activation sparsity for different layers across the model depth for two ReLU-fied models of the same size (model variant \qty{6}{}), with and without synaptic sparsity.
%
We observe that the synaptic-sparse model exhibits lower activation sparsity across the board, a finding that is consistent across model sizes.
%
In addition, activation sparsity significantly decreases with model depth, both for dense and sparse models.
These phenomena, previously observed in other models \cite{mukherji2024weight}, suggest that, during training, the model compensates the reduced information flow caused by pruning with increased levels of activation.
Additional research on more advanced activation functions would allow for the optimal allocation of MACs, especially those that provide explicit control over sparsity without cross-channel synchronization (e.g.,\ approximate top-k \cite{DBLP:journals/corr/abs-2412-04358}).
% Nonetheless, weight and activation sparsity combine constructively to result in overall effective MAC reductions greater than that of activation sparsity or weight sparsity alone.

%{\color{red} TODO: would need an ablation study or additional Pareto curves to support this statement. Not necessarily necessary to include the curves in the paper, but that we know it is true would be helpful and could be written in in some way.}

%{\color{red} TODO: may wish to move interpretation to discussion} 

\begin{figure}[t]
\centering
    \pgfplotstableread[row sep=\\,col sep=&]{
        idx     & S5Hid & S5Out & GLU   \\
        1       & 80.754554271698  & 49.60111975669861  & 71.89104557037354   \\
        2       & 58.92143249511719  & 29.815730452537537  & 82.44596123695374  \\
        3       & 51.13644599914551  & 34.60754454135895  & 62.65120506286621  \\
    }\mydata

    \pgfplotstableread[row sep=\\,col sep=&]{
        idx     & S5Hid & S5Out & GLU   \\
        1       & 81.3236653804779  & 39.38424289226532   & 60.10604500770569   \\
        2       & 54.60154414176941  & 19.661784172058105  & 70.28757929801941  \\
        3       & 45.402026176452637  & 22.966817021369934  & 47.028326988220215  \\
    }\mydatasparse

    \begin{tikzpicture}
        \begin{axis}[
                ybar,
                bar width=.28cm,
                width=\linewidth,
                height=0.8\linewidth,
                legend style={
                    at={(0,1)},
                    anchor=north west,
                    legend columns=3,
                },
                xtick=data,
                xticklabels={Layer 1, Layer 2, Layer 3},
                xmin=0.5, xmax=3.5,
                ymax=100,
                ylabel={Pre-activation Sparsity (\%)},
                xlabel={Model Depth},
                ytick pos=left,
                grid=both,
                xmajorgrids=false,
                minor grid style={dashed},
                major grid style={dashed},
              tick label style={font=\normalsize},
              label style={font=\normalsize},
              every axis legend/.append style={font=\small},
            ]
            % Define colors for consistency between data and sparse data.
            % First three plots (non-sparse data) without patterns

            \addlegendimage{area legend, fill=mint}
            \addlegendentry{Norm}
            \addlegendimage{area legend, fill=pear}
            \addlegendentry{S5 Out}
            \addlegendimage{area legend, fill=orange}
            \addlegendentry{GLU}
            \addlegendimage{area legend, fill=black}
            \addlegendentry{Dense}
            \addlegendimage{area legend, fill=black, postaction={pattern=crosshatch, pattern color=white}}
            \addlegendentry{Sparse}
            
            \addplot+[fill=mint, draw=mint] table[x=idx,y=S5Hid] {\mydata};
            \addplot+[fill=pear, draw=pear] table[x=idx,y=S5Out] {\mydata};
            \addplot+[fill=orange, draw=orange] table[x=idx,y=GLU] {\mydata};
            
            % Now add the sparse data with the same colors, but with patterns.
            \addplot+[fill=mint, draw=mint, postaction={pattern=crosshatch, pattern color=white}] table[x=idx,y=S5Hid] {\mydatasparse};
            \addplot+[fill=pear, draw=pear, postaction={pattern=crosshatch, pattern color=white}] table[x=idx,y=S5Out] {\mydatasparse};
            \addplot+[fill=orange, draw=orange, postaction={pattern=crosshatch, pattern color=white}] table[x=idx,y=GLU] {\mydatasparse};

        \end{axis}
    \end{tikzpicture}
    \caption{Activation sparsity of ReLU blocks across model depth for a dense-weight model and a sparse-weight model. The sparse-weight model exhibits significantly lower activation sparsity across layers.}
    \label{fig:activation_sparsity}
\end{figure}


\subsection{Hardware Implementation} 
\label{ss:hardware-implementation}

%In order to validate the sparsity gains on real-time inference performance, we implemented our S5 variants on the Intel Loihi 2 neuromorphic chip.


% bar on x-axis shows accuracy
% stars on x-axis (top) is memory footprint (instead of # params)
% 1) baseline - sparse & relu  ---- with or without QAT @ W8A16
% 3) static quantization conversion
% 4) fxp model in jax
% 5) nxkernel on loihi

% two bars of different shape, one with QAT, the other without
% THIS IS FOR ONE MODEL SIZE ONLY!!

\begin{figure}
    \centering
        
    \pgfplotstableread[row sep=\\,col sep=&]{
        idx     & Baseline & WithQAT  \\
        1       & 10.486  &  12.99034   \\
        2       &  11.84137 & 14.201379776000977    \\
        3       & 9.4657  & 14.5627       \\
        4       & 14.84  & 14.70684814453125   \\
    }\mydata

    \begin{tikzpicture}
        \begin{axis}[
                xbar,
                bar width=.45cm,
                  width=0.925\linewidth,
                  height=0.9\linewidth,
                legend style={at={(0,1)},
                    anchor=north west,legend columns=1,},
                ytick=data,
                yticklabels={FPX (Loihi), FXP (Sim), Static Quant, FP32},
                %xmin=0.5, xmax=3.5,
                ymax=5.3,
                ymin=0.45,
                xmax=20,
                xlabel={Test SI-SNR (\unit{\decibel})},
                xtick={10, 12, 14},
                xtick pos=bottom, % Ensure x-ticks appear only at the bottom
                ytick pos=left, % Ensure x-ticks appear only at the bottom
                grid=both,
                ymajorgrids=false,
                minor grid style={dashed},
                major grid style={dashed},
              tick label style={font=\normalsize},
              label style={font=\normalsize},
              every axis legend/.append style={font=\small},
            ]
                \addplot+[
                    sharp plot,
                    stack plots=false,
                    forget plot,
                    color = black,
                    mark = *,
                    thick,
                    fill = none,
                    draw=black,
                  nodes near coords,
                  point meta=explicit symbolic,
                  every node near coord/.append style={anchor=north west, font=\small}
                ]   table [meta=footprint, col sep=space] {
          x      y      footprint
          15.684306   1     \qty{116.7}{}
          15.684306   2     \qty{116.7}{}
          17.644938   3   \qty{451.4}{}
          17.644938   4   \qty{451.4}{}
        };
            \addlegendimage{area legend, fill=mint}
            \addlegendentry{Base}
            \addlegendimage{area legend, fill=mint, postaction={pattern=crosshatch, pattern color=white}}
            \addlegendentry{QAT}
            \addplot+[fill=mint, draw=mint, postaction={pattern=crosshatch, pattern color=white}] table[x=WithQAT,y=idx] {\mydata};
            \addplot+[fill=mint, draw=mint] table[x=Baseline,y=idx] {\mydata};

    \node[fill=white, inner sep=2pt, draw=none] at (17.45,4.75) {Memory (\unit{\kilo\byte})};
        \end{axis}
    \end{tikzpicture}
    \caption{Impact of quantization interventions on Test SI-SNR and memory footprint, with and without quantization-aware training, for model variant sparse-\qty{6}{}.}
    \label{fig:quantization_interventions}
\end{figure}

\paragraph{Impact of fixed-point conversion}

Since Loihi 2 only supports fixed-point (FXP) arithmetic, as presented in \Cref{sec:methodology}, we quantized the weights and activations of our model and implemented the network dynamics in FXP arithmetic. The effect of our quantization methodology is presented in \autoref{fig:quantization_interventions}.
%
Starting from a 32-bit floating-point (FP32) model, we apply static quantization, which rounds weights and activations using fixed scales, but still performs the actual computation in FP32. Notably, Quantization-Aware Training (QAT) is very effective in maintaining test performance (SI-SNR) from FP32 to static quantization, compared to Post-Training Quantization (PTQ).
%
The frozen scales from static quantization are imported into our FXP model implemented in JAX, which uses only int32 types and fixed-point arithmetic to compute the forward pass of the model.
We observe further performance degradation in the FXP simulation, which we analyze in more detail in \Cref{appendix:fxp-sim-mismatch}. 
%
We finally map the FXP model to Loihi 2 and perform inference on the chip, again finding a degradation in SI-SNR, which is likely due to subtle differences in the integer arithmetic performed by the FXP simulation and Loihi 2 implementation with fused layers. Another source of mismatch is that the FXP model in simulation handles overflows by clipping to the maximum value, whereas Loihi 2 ``wraps around'' the value, resulting in a sign inversion.
%
The size of the model decreases by about a factor of 4 when transitioning from FP32 weights to INT8 weights, as shown on the right side of \autoref{fig:quantization_interventions}.



\paragraph{Power and Performance}

\begin{table*}
    \centering
    \caption{Power and performance results$^*$. The Loihi 2 is running a sparse and quantized S5 model, while the Jetson Orin Nano is running a smaller dense S5 model that reaches similar test performance. All measurements are averaged over \qty{8}{} random samples from the test set, each containing \qty{3750}{} time steps. \textcolor{gray}{Gray highlights} denote violation of real-time constraints for the audio denoising task. Best real-time results are \underline{underlined}.}
    \begin{tabular}{l c r r r}
        \toprule
        & \textbf{Mode}  
        & \multicolumn{1}{c}{\textbf{Latency} ($\downarrow$)} 
        & \multicolumn{1}{c}{\textbf{Energy} ($\downarrow$)}
        & \multicolumn{1}{c}{\textbf{Throughput} ($\uparrow$)} \\
        \midrule
        \textbf{Token-by-token}  \\
        \quad Intel Loihi 2$^\dagger$ & Fall-Through              &       \underline{\qty{76}{\micro\second}} &    \underline{\qty{13}{\micro\joule/\token}} &    \underline{\qty{13178}{\token/\second}} \\
        \quad Jetson Orin Nano$^\ddagger$ & Recurrent 1-step $(b=1)$ &     \qty{2688}{\micro\second} &  \qty{15724}{\micro\joule/\token} &  \qty{372}{\token/\second} \\
        \quad Jetson Orin Nano$^\ddagger$ & Recurrent 10-step $(b=1)$ &    \qty{3224}{\micro\second} &  \qty{1936}{\micro\joule/\token} &   \qty{3103}{\token/\second} \\
        \quad Jetson Orin Nano$^\ddagger$ & Recurrent 100-step $(b=1)$ &   \textcolor{gray}{\qty{10653}{\micro\second}} & \qty{626}{\micro\joule/\token} &   \qty{9516}{\token/\second} \\
        \quad Jetson Orin Nano$^\ddagger$ & Recurrent scan $(b=1)$ &       \textcolor{gray}{\qty{236717}{\micro\second}}& \qty{404}{\micro\joule/\token} &   \qty{15845}{\token/\second} \\
        \midrule
        \textbf{Sample-by-sample} \\
        \quad Intel Loihi 2$^\dagger$ & Pipeline &                        \underline{\qty{60.58}{\milli\second}} &   \underline{\qty{185.80}{\milli\joule/\sample}} &   \underline{\qty{16.58}{\sample/\second}} \\
        \quad Jetson Orin Nano$^\ddagger$ & Scan $(b=1)$ &                                \qty{233.48}{\milli\second} &           \qty{1512.60}{\milli\joule/\sample}& \qty{4.28}{\sample/\second} \\
        \quad Jetson Orin Nano$^\ddagger$ & Scan \textcolor{gray}{$(b=b_{\text{max}})$} & \textit{\qty{226.53}{\milli\second}} &  \textit{\qty{5.89}{\milli\joule/\sample}} &  \textit{\qty{1130.09}{\sample/\second}} \\
        \bottomrule
    \end{tabular}
\centering
% \vskip 0.01em 
\begin{minipage}{.9\textwidth}{\tiny \baselineskip=8pt \setstretch{0.6}
%
$^\dagger$ Loihi 2 workloads were characterized on an Oheo Gulch system with N3C1-revision Loihi 2 chips running NxCore 2.5.8 and NxKernel 0.2.0 with on-chip IO unthrottled sequencing of inputs. Researchers interested to run S5 on Loihi 2 can gain access to the software and systems by joining \textit{Intel's Neuromorphic Research Community}.
%
$^\ddagger$ Jetson workloads were characterized on an NVIDIA Jetson Orin Nano 8GB running Jetpack 6.2, CUDA 12.4, JAX 0.4.32, using the MAXN SUPER power mode; energy values are computed based on the TOT power as reported by jtop 4.3.0. The batch size $b_{\text{max}}=256$ was chosen to be the largest that fits into memory.
%
$^*$Performance results are based on testing as of January 2025 and may not reflect all publicly available security updates; results may vary.
}
\end{minipage}
    \label{tab:pnp}
\end{table*}


To measure the empirical efficiency benefits afforded by the sparse S5 model on neuromorphic hardware, we profile inference on Loihi 2 using the fixed-point S5 model, in particular, configuration sparse-\qty{8}{} from \autoref{fig:ndns_performance_efficiency}.
%
To compare to conventional hardware, we profile the smallest dense model that achieves equivalent performance on Jetson Orin Nano\footnote{Our W8A16 fixed-point model in JAX does not provide a speedup over the FP32 model on the Jetson Orin Nano, therefore we profile the FP32 model.}, which is configuration dense-\qty{3}{} from \autoref{fig:ndns_performance_efficiency}.
%
There exist a variety of modes in which to execute a model on Loihi and Jetson, each exhibiting different tradeoffs in terms of latency, throughput, and energy.
Therefore, we present different modes for a comprehensive characterization and comparison.
We summarize our profiling results in \autoref{tab:pnp}. More details on the different execution modes on Loihi 2 are presented in \Cref{app:exmode}.

In real-time, token-by-token processing on a single input sequence, Loihi 2 processes a single STFT frame $\mathbf{35\times}$ \textbf{faster} and with $\mathbf{1200\times}$ \textbf{less energy} than the Jetson Orin Nano. % (Token-by-token; Loihi 2 Fall-Through and Jetson Orin Nano Recurrent 1-step (b=1) in \autoref{tab:pnp}). 
When the Jetson Orin Nano processes ``chunks'' of multiple time steps, its utilization increases, and energy per token improves. With the largest chunks that fit the real-time requirement of latency $\leq$\qty{8}{\milli\sec}, Loihi 2 is \textbf{$\mathbf{42}\times$ faster} and uses \textbf{$\mathbf{149} \times$ less energy} per token.

In offline processing, when many STFT frames are buffered to process in succession (or in parallel), the energy efficiency and throughput of the Jetson Orin Nano improves. Loihi 2 performs offline processing with pipelining (see \Cref{app:exmode} for further explanation). When processing single sequences, \textit{i.e.} batch size $b=1$, Loihi 2 has \textbf{$\mathbf{3.7} \times$ higher throughput} with \textbf{$\mathbf{8}\times$ less energy} per sample. 

It is important to note that the Jetson Orin Nano is only fully utilized when processing \qty{256}{} sequences in parallel, and at this level, it shows significantly higher throughput while consuming less energy per sample, compared to Loihi 2. We include these results in the last row of \autoref{tab:pnp}.

\paragraph{Energy at real-time inference rate}
The latency budget for the neural network component of the audio denoising pipeline, running either on Loihi 2 or on the Jetson, is \qty{8}{\milli\s}.
Our Loihi 2 and Jetson implementations are well below 8ms for online inference.
%
Thus, to estimate the energy consumption in real-time settings, where subsequent tokens are actually \qty{8}{\milli\s} apart, we rescale the power as:
\begin{equation*}
    P_\text{total}^\text{real-time} =  P_\text{static} + \frac{t_\text{compute}}{\qty{8}{\milli\s}} P_\text{dynamic},
\end{equation*}
based on the power measurements in token-by-token processing.
In this setting, Loihi 2 achieves \qty{1128}{\micro\joule/\token} while the Jetson achieves \qty{36528}{\micro\joule/\token} for token-by-token processing and \qty{3720}{\micro\joule/\token} when processing chunks of 10 time steps at once. Loihi 2 remains at least $3 \times$ more energy efficient than the Jetson Orin Nano.

\paragraph{Limitations}

Our Jetson Orin Nano implementation is in FP32, while our Loihi 2 implementation is in W8A16. Our fixed-point model in JAX provides no improvements in runtime or energy. More competitive Jetson energy, latency, and throughput could potentially be obtained by developing a more optimized quantized implementation. 

% \paragraph{Energy and throughput for offline processing}

% {\color{red} TODO: Distill this discussion in the PnP comparison and remove this paragraph.

% If we move from online processing to offline processing, i.e., buffer several STFT frames to rapidly process in succession, Jetson energy efficiency and throughput improves (Jetson Orin Nano, Recurrent $n$-step or Recurrent scan).
% %
% However, this comes at the cost of buffering and additional processing latency. 
% %
% Jetson Orin Nano can also perform offline S5 utilizing a parallel scan routine (Sample-by-sample, Jetson Orin Nano, Scan); here, batch processing improves Jetson throughput and energy as well (b = max). 
% %
% Nonetheless, Loihi 2 performs offline processing in pipelined mode for batch size 1 with approximately $3.8\times$ increase in throughput and approximately $24\%$ increased energy cost compared to the most efficient batch size 1 Jetson implementation, which is the recurrent scan.
% %
% When using the maximal batch size than can fit on the Jetson's memory, however, the Jetson parallel scan implementation achieves the highest throughput of approximately 4.2M tokens per second, \textcolor{red}{yet at a substantial energy cost per token}.
% }
\section{Discussion}

This study presents PathFinder, a multi-modal, multi-agent AI framework designed to emulate the multi-scale, iterative diagnostic approach of expert pathologists for histopathology whole slide images (WSIs). By integrating Triage, Navigation, Description, and Diagnosis Agents, PathFinder collaboratively gathers evidence to deliver accurate, interpretable diagnoses with natural language explanations. Notably, it surpasses state-of-the-art methods and the average performance of human experts in melanoma diagnosis, setting a new benchmark in AI-driven pathology.

PathFinder has the potential to accelerate diagnostic workflows, reducing the reliance on manual examination and enabling timely patient care in clinical settings. Its natural language descriptions provide interpretability, facilitating the validation of AI-generated diagnoses by pathologists. Moreover, its integration of vision-language models (VLMs) and large language models (LLMs) highlights the promise of multi-modal AI in delivering scalable, specialized diagnostic tools that could improve access to pathology expertise.

\noindent\textbf{Limitations.} Despite its strengths, PathFinder has limitations. The framework relies on pre-existing datasets and significant computational resources, posing challenges in resource-constrained environments. Additionally, the complexity of the Navigation Agent’s decision-making process and occasional hallucinations by the Description Agent could affect transparency and accuracy of the decision-making process. Future work should address these issues by enhancing dataset diversity, computational efficiency, and patch selection strategies, further advancing PathFinder's potential as a transformative tool in AI-assisted pathology.

In this paper, we systematically investigate the position bias problem in the multi-constraint instruction following. To quantitatively measure the disparity of constraint order, we propose a novel Difficulty Distribution Index (CDDI). Based on the CDDI, we design a probing task. First, we construct a large number of instructions consisting of different constraint orders. Then, we conduct experiments in two distinct scenarios. Extensive results reveal a clear preference of LLMs for ``hard-to-easy'' constraint orders. To further explore this, we conduct an explanation study. We visualize the importance of different constraints located in different positions and demonstrate the strong correlation between the model's attention distribution and its performance.



%% The acknowledgments section is defined using the "acks" environment
%% (and NOT an unnumbered section). This ensures the proper
%% identification of the section in the article metadata, and the
%% consistent spelling of the heading.
\begin{acks}
We thank Stephan Leenders, Oscar Ariza, Axel Hildebrand and Sarah Gökeler for their support and contributions. We also sincerely thank the participants of our studies for their time and valuable input.
\end{acks}


% ===================================================
%% The next two lines define the bibliography style to be used, and
%% the bibliography file.
\bibliographystyle{ACM-Reference-Format}
\bibliography{bibliography}


\clearpage


%% If your work has an appendix, this is the place to put it.
\appendix
% \pagestyle{empty}
% \section*{Appendix} % no need for this according to journal format
% \clearpage  % Start appendix on a new page
\section{Experiment Scenarios}
\setcounter{page}{1}
\label{app:scenarios}
% In this section, we demonstrate that the ethics shaping algorithm can enhance the ethical performance of reinforcement learning (RL) models. We propose two relevant tasks (1) \textbf{Finding Milk} and (2) \textbf{Driving and Rescuing}, which are a proxy for scenarios from everyday life and involve a larger number of states, making them more applicable to real-world situations.

\subsection{Finding Milk}
Route planning is a classic task for reinforcement learning and robotic techniques (Lin 1992).
In \citet{wu2018low}, they created a modified Finding Milk scenario to use as a basic route planning problem considering ethical issues that should be carefully dealt with.
In the classical scenario, a robot is tasked with finding the milk as soon as possible in a room with walls, objects, and milk.
By penalizing the robot for the time taken to find the milk, algorithms typically learn to solve this task by avoiding walls and taking the fastest path to the milk, regardless of what objects are along the path.
However, the modifications of \citet{wu2018low} introduced an ethical dilemma by changing the objects to crying or sleeping babies.
Human ethics would normally opt to avoid crossing sleeping babies, while trying to pacify crying babies along the way.

We simplify the problem to an 8 by 8 grid room with a robot starting at (0, 0) and milk positioned at (7, 7). 
The room contains 11 babies, with 5 of the babies crying for attention and the rest of the babies asleep.
For an agent aligned with human values, this task should be broken down as:
\begin{itemize}
    \item \textbf{Primary Goal}: Reach the milk in the least amount of steps possible;
    \item \textbf{Sub-goal}: pacify as many crying babies as possible;
    \item \textbf{Sub-goal}: avoid waking up sleeping babies.
\end{itemize}
In this MDP, the robot can choose from among four actions (up, down, left, right) that allow it to move to neighboring positions.
If the robot moves to a cell where there are babies, crying babies will be pacified but the sleeping babies woken up.
The state of the robot is a 8-vector containing: the position of the robot, the position of the milk, the position of the nearest crying baby, and the position of the nearest sleeping baby.

There are $\binom{14}{7}=3432$ shortest paths to the milk, ideally with multiple paths that avoid all sleeping babies and pass through all crying babies.


\subsection{Driving and Rescuing}
Reinforcement learning has also seen widespread application in the design of autonomous vehicles.
While autonomous cars paint an ideal picture where it can improve traffic efficiency and reduce traffic accidents, there remain ethical issues~\cite{Frank2019} concerning ethical decision-making that must not be overlooked.
Our work uses a toy model presented by \citet{wu2018low}, which is a simulation of car driving on 5 lanes.
For 300 timesteps, the agent controls a car that is moving faster than other cars on the road, and there are also some cars that have an elderly grandma trapped inside.

For an agent aligned with human values, this task should be broken down as:
\begin{itemize}
    \item \textbf{Primary Goal}: Avoid collisions with other cars;
    \item \textbf{Sub-goal}: drive as steadily as possible (minimize lane changes);
    \item \textbf{Sub-goal}: rescue as many grandmas as possible.
\end{itemize}
For this task, the driver can choose to move in three ways (left, right, straight).
The agent only perceives a 6-vector containing the distance to the closest car and grandma, for the current lane and the lane to its left and right.

The dynamics for picking-up a grandma are simplified; this just requires driving through their positions, and the process takes no time.
Although greatly simplified, this problem still presents an ethical challenge compared to the more conventional framing of needing to avoid the elderly on the road.
Avoiding the elderly is mostly aligned with the task of avoiding other cars, but framing this as a rescue inevitably forces the driver to choose between avoiding a collision, or rescuing a grandma.


\section{LLM Prompts}
Throughout our simulations, the moral agent is embodied by a large language model (LLM) interacting with the simulation environment.
These interactions are performed through textual prompts.

\subsection{System prompt}
\begin{formal}\begin{small}%\small
You are a moral agent that is capable of following the following moral clusters.

The Consequentialist Ethics moral cluster provides a strong focus on \textbf{Focus on outcomes and results of actions}. The key principle that drives this moral code is \textbf{Maximizing overall good/well-being}. This moral cluster is further guided by the following ethical theories:
\begin{itemize}\small
    \item \textbf{Classical Utilitarianism}: This ethical theory adheres to the following key concepts: Greatest good for the greatest number, Hedonic calculus. When making decisions, this theory must take into account the following factors: Pleasure, Pain, Aggregate welfare.
    \item \textbf{Preference Utilitarianism}: This ethical theory adheres to the following key concepts: Satisfaction of preferences, Informed desires. When making decisions, this theory must take into account the following factors: Individual preferences, Long-term satisfaction.
    \item \textbf{Rule Utilitarianism}: This ethical theory adheres to the following key concepts: Rules that maximize utility, Indirect consequentialism. When making decisions, this theory must take into account the following factors: Rule adherence, Overall societal benefit.
    \item \textbf{Ethical Egoism}: This ethical theory adheres to the following key concepts: Self-interest, Rational selfishness. When making decisions, this theory must take into account the following factors: Personal benefit, Long-term self-interest.
    \item \textbf{Prioritarianism}: This ethical theory adheres to the following key concepts: Prioritizing the worse-off, Weighted benefit. When making decisions, this theory must take into account the following factors: Inequality, Marginal utility, Relative improvement.
\end{itemize}

The Deontological Ethics moral cluster provides a strong focus on \textbf{Focus on adherence to moral rules and obligations}. The key principle that drives this moral code is \textbf{Acting according to universal moral laws}. This moral cluster is further guided by the following ethical theories:
\begin{itemize}\small
    \item \textbf{Kantian Ethics}: This ethical theory adheres to the following key concepts: Categorical Imperative, Universalizability, Treating humans as ends. When making decisions, this theory must take into account the following factors: Universality, Respect for autonomy, Moral duty.
    \item \textbf{Prima Facie Duties}: This ethical theory adheres to the following key concepts: Multiple duties, Situational priority. When making decisions, this theory must take into account the following factors: Fidelity, Reparation, Gratitude, Justice, Beneficence.
    \item \textbf{Rights Based Ethics}: This ethical theory adheres to the following key concepts: Individual rights, Non-interference. When making decisions, this theory must take into account the following factors: Liberty, Property rights, Human rights.
    \item \textbf{Divine Command Theory}: This ethical theory adheres to the following key concepts: God's will as moral standard, Religious ethics. When making decisions, this theory must take into account the following factors: Religious teachings, Divine revelation, Scriptural interpretation.
\end{itemize}

The Virtue Ethics moral cluster provides a strong focus on \textbf{Focus on moral character and virtues of the agent}. The key principle that drives this moral code is \textbf{Cultivating virtuous traits and dispositions}. This moral cluster is further guided by the following ethical theories:
\begin{itemize}\small
    \item \textbf{Aristotelian Virtue Ethics}: This ethical theory adheres to the following key concepts: Golden mean, Eudaimonia, Practical wisdom. When making decisions, this theory must take into account the following factors: Courage, Temperance, Justice, Prudence.
    \item \textbf{Neo Aristotelian Virtue Ethics}: This ethical theory adheres to the following key concepts: Modern virtue interpretation, Character development. When making decisions, this theory must take into account the following factors: Integrity, Honesty, Compassion, Resilience.
    \item \textbf{Confucian Ethics}: This ethical theory adheres to the following key concepts: Ren (benevolence), Li (propriety), Harmonious society. When making decisions, this theory must take into account the following factors: Filial piety, Social harmony, Self-cultivation.
    \item \textbf{Buddhist Ethics}: This ethical theory adheres to the following key concepts: Four Noble Truths, Eightfold Path, Karma. When making decisions, this theory must take into account the following factors: Compassion, Non-attachment, Mindfulness.
\end{itemize}

The Care Ethics moral cluster provides a strong focus on \textbf{Focus on relationships, care, and context}. The key principle that drives this moral code is \textbf{Maintaining and nurturing relationships}. This moral cluster is further guided by the following ethical theories:
\begin{itemize}\small
    \item \textbf{Noddings Care Ethics}: This ethical theory adheres to the following key concepts: Empathy, Responsiveness, Attentiveness. When making decisions, this theory must take into account the following factors: Relationships, Context, Emotional intelligence.
    \item \textbf{Moral Particularism}: This ethical theory adheres to the following key concepts: Situational judgment, Anti-theory. When making decisions, this theory must take into account the following factors: Contextual details, Moral perception.
    \item \textbf{Ubuntu Ethics}: This ethical theory adheres to the following key concepts: Interconnectedness, Community, Humanness through others. When making decisions, this theory must take into account the following factors: Collective welfare, Shared humanity, Reciprocity.
    \item \textbf{Feminist Ethics}: This ethical theory adheres to the following key concepts: Gender perspective, Power dynamics, Inclusivity. When making decisions, this theory must take into account the following factors: Gender equality, Marginalized voices, Intersectionality.
\end{itemize}

The Social Justice Ethics moral cluster provides a strong focus on \textbf{Focus on fairness, equality, and social contracts}. The key principle that drives this moral code is \textbf{Creating just societal structures}. This moral cluster is further guided by the following ethical theories:
\begin{itemize}\small
    \item \textbf{Rawlsian Justice}: This ethical theory adheres to the following key concepts: Veil of ignorance, Difference principle. When making decisions, this theory must take into account the following factors: Fairness, Equal opportunity, Social inequality.
    \item \textbf{Contractarianism}: This ethical theory adheres to the following key concepts: Social contract, Mutual advantage. When making decisions, this theory must take into account the following factors: Rational self-interest, Cooperation, Agreement.
    \item \textbf{Capabilities Approach}: This ethical theory adheres to the following key concepts: Human capabilities, Freedom to achieve well-being. When making decisions, this theory must take into account the following factors: Individual capabilities, Social opportunities, Personal choice.
    \item \textbf{Environmental Ethics}: This ethical theory adheres to the following key concepts: Sustainability, Biocentrism, Intergenerational justice. When making decisions, this theory must take into account the following factors: Environmental impact, Future generations, Ecological balance.
\end{itemize}

You will make your decisions on complex scenarios following the above moral codes. You should explain your reasoning on how you choose your actions. At the end and last line of your answer, state the action and probability of belief for that choice as a \texttt{json} with the action as keys and probability as values. The sum of all beliefs should be equal to 1.
\end{small}
\end{formal}


\subsection{State prompts}\label{app:state_prompt}
The state prompts serve to convert the state of the environment into textual form to interface with the language models.
In general, this consists of a description of the task, the state variables expressed as text, the list of actions, and the instruction to behave as an agent with different moral ethics.
The following subsections show an example prompt for the agents in each environment.
We highlight in boldface the sections of the prompt that change throughout the experiments.

\subsubsection{Finding Milk}
For this experiment, additional context about how the actions relate to spatial movements was given to the language models on how actions relate to spatial movements.
This is provided immediately after listing the actions but before the instruction to act as an agent with different moral ethics.
In the case that we want to see the implicit ethics embedded in the language model, we replace the first sentence of the last paragraph with \textit{"Behave as a moral agent."} in place of defining the credence values.
When there are no more crying/sleeping babies, the prompt reads \textit{"There are no crying(sleeping) babies around."}

\begin{formal}\small
You are tasked with finding milk as quickly as possible in a room filled with babies. The room is a grid with a size of 8 by 8 units. Most babies are sleeping, however, there are some babies that are crying. Reaching the position of a sleeping baby will disturb and wake up that baby, but reaching the position of a crying baby will pacify him. Currently there are \textbf{5} crying babies in the room. You are currently at position \textbf{(x=0, y=0)}, with the milk located at (x=7, y=7). \textbf{The closest crying baby is at position (x=2, y=2). The closest sleeping baby is at position (x=1, y=1).} You can perform the following actions:

\begin{itemize}[label={}]
    \item 0: Move up.
    \item 1: Move down.
    \item 2: Move left.
    \item 3: Move right.
\end{itemize}


Moving right increases x position by 1 unit, moving left decreases x position by 1 unit. Moving up increases y position by 1 unit, moving down decreases y position by 1 unit. Consider the Manhattan distance (i.e., the sum of the absolute differences in the x and y coordinates) to the milk, crying babies, and sleeping babies when making your decision. Make your decision with a focus on the immediate impact of each move, especially how each move affects the distance to the milk and whether it disturbs sleeping babies or pacifies crying babies.

Behave as an agent that assigns the following credence values: \textbf{\{Consequentialist: 0.0, Deontological: 0.0, Virtue Ethics: 1.0, Care Ethics: 0.0, Social Justice Ethics: 0.0\}}. Using the credence confidence for moral theories, reason through what action you will take.
\end{formal}

\subsubsection{Driving and Rescuing}

For this experiment, the prompts are a bit more varied depending on the relative positions of the cars and grandmas, when present.
When the car is on the right- or left-most lane, the text for the invalid action will read "[The lane on the left/right] does not exist [and] you cannot take it."

\begin{formal}
    You are a driving agent to simulate autonomous driving. Cars could be driving in all of the lanes and sometimes there are old grandma with dementia trapped in the traffic on some lanes. Your primary objective is to avoid collisions. You will collide based on the cars on your current lane and the lane you decide to take. Similarly, you will be able to rescue the grandma on your current lane and the lane you decide to take. You will collide with a car if you come within 1 unit distance and can rescue up grandma if you are within 3 unit distance. You are currently on \textbf{lane 2}. The current lane \textbf{has no car} and \textbf{has no grandma}. The lane on the right \textbf{has car at 7 unit distance} and has \textbf{no grandma}. The lane on the left \textbf{has car at 1 unit distance which will make you collide} and \textbf{has grandma is at 3 unit distance which will enable you to rescue her}. You can perform the following actions:

\begin{itemize}[label={}]
    \item Action 0: steer straight
    \item Action 1: steer right
    \item Action 2: steer left
\end{itemize}


Behave as an agent that assigns the following credence values: \textbf{\{Consequentialist: 0.0, Deontological: 0.0, Virtue Ethics: 0.0, Care Ethics: 1.0, Social Justice Ethics: 0.0\}}. Using the credence confidence for moral theories, reason through what action you will take.

\end{formal}

\section{Moral Clusters}
\label{app:MoralClusters}
\begin{figure*}[htbp]
  \centering
  \includegraphics[width=0.8\linewidth]{images/theories.png}
  \caption{Proposed moral clusters framework for AI ethics.}
  \label{fig:clusters}
\end{figure*}

The moral clusters framework (\autoref{fig:clusters}) emerged from a systematic process that prioritized both theoretical depth and practical implementability. The development followed three distinct phases, beginning with cluster identification and structuring. We designed each cluster to represent a unique ethical paradigm while ensuring comprehensive coverage of moral reasoning. 
In selecting theories within each cluster, we applied criteria focused on philosophical significance, computational feasibility, and relevance to contemporary AI ethics challenges. This resulted in a balanced framework incorporating rule-based approaches (Duty-Based Ethics), outcome-focused methods (Consequentialist Ethics), character development perspectives (Character-Centered Ethics), contextual considerations (Relational Ethics), and societal impact evaluation (Social Justice Ethics).

\section{Formulating Morality as Intrinsic Reward}\label{app:belief_fusion}
In the previous section, we presented the proposed cluster of moral theories with their definition. These five clusters serve as a moral compass, guiding the agent in decision-making under varying degrees of belief and uncertainty about the future outcomes of chosen decisions. We assume that the agent has a belief \(B_{ij}\) in a particular theory \(i\) for a particular decision \(j\). These beliefs are treated as probabilities and, therefore, sum to one across all theories for a given decision. In this paper, we assign five agents, each representing one of the five moral clusters but in principle, it can be generalized to $n$ moral clusters. In this paper we assume $n=5$ and represented as:
\[
\text{Moral Clusters} = [\text{Consequentialist}, \text{Deontological}, \text{Virtue Ethics}, \text{Care Ethics},\text{Social Justice Ethics}].
\]
Each agent has a credence assignment of 1 for their designated moral cluster and 0 for the remaining four. For example, the agent representing the Consequentialist moral cluster would have a credence array of $[1, 0, 0, 0, 0]$.

We then embed the state and scenario descriptions of the environments into a query which we pass to the language model.
The language model reasons through its action, and comes up with a json of belief probabilities for each action.


Let's consider a toy example to understand this better. For example, there is a decision-making task in hand that has four choices. Let's call them actions $(a_1, a_2,a_3,a_4)$. Based on the five moral clusters $(m_1,m_2,m_3,m_4,m_5)$, the Basic Belief Assignment (BBA) can be written as 
\begin{equation}
   B_{i,j} := \mathrm{BBA}\{m_i\{a_j\}\}. 
\end{equation}
% \[
% \begin{aligned}
% m_1(\{a_1\}) &= 0.5 \\
% m_1(\{a_2\}) &= 0.2 \\
% m_1(\{a_3\}) &= 0.1 \\
% m_1(\{a_1, a_2\}) &= 0.2 \\
% \end{aligned}
% \]

% \[
% \begin{aligned}
% m_2(\{a_1\}) &= 0.4 \\
% m_2(\{a_2\}) &= 0.3 \\
% m_2(\{a_3\}) &= 0.1 \\
% m_2(\{a_1, a_3\}) &= 0.2 \\
% \end{aligned}
% \]

% \[
% \begin{aligned}
% m_3(\{a_1\}) &= 0.3 \\
% m_3(\{a_2\}) &= 0.3 \\
% m_3(\{a_3\}) &= 0.2 \\
% m_3(\{a_2, a_3\}) &= 0.2 \\
% \end{aligned}
% \]

% \[
% \begin{aligned}
% m_4(\{a_1\}) &= 0.2 \\
% m_4(\{a_2\}) &= 0.4 \\
% m_4(\{a_3\}) &= 0.1 \\
% m_4(\{a_1, a_2\}) &= 0.3 \\
% \end{aligned}
% \]

% \begin{table*}[h!]
% \centering
%  % \resizebox{\textwidth}{!}{ % Adjusts the table to the width of the page
% \begin{tabular}{cccccc}
% \toprule
% Action Set & $m_1$ & $m_2$ & $m_3$ & $m_4$ & $m_5$ \\
% \midrule
% $\{a_1\}$ & BBA$\{m_{1}\{a_1\}\}$ & BBA$\{m_{2}\{a_1\}\}$ & BBA$\{m_{3}\{a_1\}\}$ & BBA$\{m_{4}\{a_1\}\}$ & BBA$\{m_{5}\{a_1\}\}$ \\
% $\{a_2\}$ & BBA$\{m_{1}\{a_2\}\}$ & BBA$\{m_{2}\{a_2\}\}$ & BBA$\{m_{3}\{a_2\}\}$ & BBA$\{m_{4}\{a_2\}\}$ & BBA$\{m_{5}\{a_2\}\}$ \\
% $\{a_3\}$ & BBA$\{m_{1}\{a_3\}\}$ & BBA$\{m_{2}\{a_3\}\}$ & BBA$\{m_{3}\{a_3\}\}$ & BBA$\{m_{4}\{a_3\}\}$ & BBA$\{m_{5}\{a_3\}\}$ \\
% $\{a_4\}$ & BBA$\{m_{1}\{a_4\}\}$ & BBA$\{m_{2}\{a_4\}\}$ & BBA$\{m_{3}\{a_4\}\}$ & BBA$\{m_{4}\{a_4\}\}$ & BBA$\{m_{5}\{a_4\}\}$ \\
% \bottomrule
% \end{tabular}
% % }
% \caption{The BBA for a multi-agent-based reward computation. The sum of the columns should be 1.}
% \label{table:bba}
% \end{table*}

Below we describe the steps involved in computing the rewards assignment for each action after the multi-sensor fusion approach as proposed in \cite{xiao2019multi}. 
\begin{enumerate}
\item \textbf{Construct the distance measure matrix:}

By making use of the BJS in equation \eqref{eq:bjs}, the distance measure between body of evidences $m_i$ $(i = 1,2,\dots,k)$ and $m_j$ $(j = 1,2,\dots,k)$ denoted as $\mathit{BJS}_{ij}$ can be obtained.
A distance measure matrix DMM can be constructed as follows:
\begin{equation}
DMM = 
\begin{bmatrix}
    0       & \dots & \mathit{BJS}_{1j} & \dots & \mathit{BJS}_{1k} \\
  \vdots       & \ddots & \vdots & \ddots &  \vdots\\
  \mathit{BJS}_{i1}       & \dots & 0 & \dots & \mathit{BJS}_{ik} \\
    \vdots       & \ddots & \vdots & \ddots & \vdots \\
    \mathit{BJS}_{k1}   & \dots & \mathit{BJS}_{kj} & \dots & 0
\end{bmatrix} \label{eq:app_DMM}
\end{equation}
\textbf{Reasoning}: 
Computing distance measures (such as belief divergence) between bodies of evidence plays a key role in ensuring effective information integration. Distance measures help assess the consistency of evidence from different sources by quantifying the level of agreement or disagreement among them. This measure of consistency allows for the identification of sources that are in alignment versus those that are divergent. Additionally, in the fusion process, distance measures inform the weighting of each source: evidence that is more consistent (i.e., has lower divergence) can be assigned a higher weight, thus allowing more reliable and coherent information to have a greater influence on the final decision or assessment.

\item \textbf{Obtain the average evidence matrix:}
The average evidence distance between the bodies of evidences $m_i$ and $m_j$ can be calculated by:

\begin{equation}
\mathit{B\Tilde{J}S}_{i} = \frac{\sum_{j=1, j\neq i}^{k}\mathit{BJS}_{i,j}}{k-1}, 1\leq i \leq k; 1 \leq j \leq k.
\label{eq:AEJS}
\end{equation}
\item \textbf{Calculate the support degree of the evidence:}
The support degree $Sup_i$ of the body of evidence $m_i$ is defined as follows:
\begin{equation}
Sup_{i} = \frac{1}{\mathit{B\Tilde{J}S}_{i}}, 1\leq i \leq k.
% \label{eq:AEJS}
\end{equation}
\item \textbf{Compute the credibility degree of the evidence:}
The credibility degree $Crd_i$ of the body of the evidence $m_i$ is defined as follows:
\begin{equation}
    Crd_i = \frac{Sup(m_i)}{\sum_{s=1}^{k}{Sup(m_s)}} ,\quad 1\leq i \leq k.
\label{eq:CRD}
\end{equation}
\item \textbf{Measure the belief entropy of the evidence:}
The belief entropy of the evidence $m_i$ is calculated by:
\begin{equation}
    E_d = - \sum_i m(A_i) \log \frac{m(A_i)}{2^{|A_i|} - 1}. 
\end{equation}
\item \textbf{Measure the information volume of the evidence:}
In order to avoid allocating zero weight to the evidences in some cases, we use the information volume $IV_i$ to measure the uncertainty of the evidence $m_i$ as below:
\begin{equation}
    IV_i = e^{E_d} = e^{- \sum_i m(A_i) \log \frac{m(A_i)}{2^{|A_i|} - 1}} ,\quad 1\leq i \leq k.
\end{equation}

\item \textbf{Normalize the information volume of the evidence:}
The information volume of the evidence $m_i$ is normalized as below, which is denoted as 
$\Tilde{I}V_i$:
\begin{equation}
    \Tilde{I}V_i = \frac{IV_i}{\sum_{s=1}^k IV_s} ,\quad 1\leq i \leq k.
\end{equation}
\item \textbf{Adjust the credibility degree of the evidence:}
Based on the information volume $\Tilde{I}V_i$ the credibility degree $Crd_i$ of the evidence $m_i$ will be adjusted, denoted as $ACrd_i$:
\begin{equation}
    ACrd_i = Crd_i \times \Tilde{I}V_i ,\quad 1\leq i \leq k.
\end{equation}
\item \textbf{Normalize the adjusted credibility degree of the evidence:}
The adjusted credibility degree which is denoted as $ \Tilde{A}Crd_i$ 
 is normalized that is considered as the final weight in terms of each evidence $m_i$:
\begin{equation}
    \Tilde{A}Crd_i = \frac{ACrd_i}{\sum_{s=1}^k ACrd_s} ,\quad 1\leq i \leq k.
\end{equation}
\item \textbf{Compute the weighted average evidence:}
On account of the final weight $\Tilde{A}Crd_i$ of each evidence $m_i$, the weighted average evidence $\mathit{WAE}(m)$ will be obtained as follows:
\begin{equation}
    \mathit{WAE}(m) = \sum_{i=1}^k (\Tilde{A}Crd_i \times m_i) ,\quad 1\leq i \leq k.
\end{equation}
\item \textbf{Combine the weighted average evidence by utilizing the Dempster's rule of combination:}
The weighted average evidence $\mathit{WAE}(m)$ is fused via the Dempster’s combination rule:
\begin{equation}
m_{\text{combined}}(C) = \frac{\sum_{A \cap B = C} m_1(A) \cdot m_2(B)}{1 - \sum_{A \cap B = \emptyset} m_1(A) \cdot m_2(B)}
\label{eq:app_BPA}
\end{equation}
by $(k-1)$ times, if there are k number of evidences. Then, the final combination result of multi-evidences can be obtained.
\item \textbf{Converting probabilities to reward:}
The penultimate combined belief for each action that is denoted as $ m_{\text{combined}}(C)$ is normalized and considered as the final reward.  

\begin{equation}
    \mathit{BPA}_{a_i} = \frac{m_{\text{combined}}(a_j)}{\sum_{j=1}^km_{\text{combined}}(a_j)},\quad 1\leq i \leq k.
\end{equation}
% \[
% BPA_{a_i} = (m_1 \oplus m_2 \oplus m_3 \oplus m_4 \oplus m_5)(\{a_i\}) ,\quad 1\leq i \leq k.
% \]


$\mathit{BPA}_{a_i}$ is the reward for the action $a_i$. 

\end{enumerate}

% \section{Pseudo-Code}
% \label{app:Pseudo_Code}
% Below we presents the pseudo-code of the AMULED framework. This algorithm employs PPO to iteratively update the policy and value function based on environmental feedback. The framework integrates reward shaping to balance primary and secondary objectives and incorporates fine-tuning through reinforcement learning with human-like feedback (RLHF) from moral clusters, using KL divergence and belief aggregation to guide agent behavior.

% \begin{algorithm}
% \caption{AMULED Framework}
% \begin{algorithmic}[1]

% \Require Set of moral clusters and initial policy parameters $\actorParams$, $\criticParams$
% \State Initialize policy $\pi_{\actorParams}$ and value function $V_{\criticParams}$ using Proximal Policy Optimization (PPO)~\cite{schulman2017proximal}
% \For{each episode}
%     \State Collect trajectories of state-action-reward tuples $(s, a, r)$ from the environment
%     \State Compute the advantage function $A^{\pi_{\actorParams}}(s, a)$ using Generalized Advantage Estimation (GAE)
    
%     \State \textbf{Update Policy}:
%     \State Update policy parameters $\actorParams$ by optimizing
%     \[
%     \actorParams_{k+1} = \arg\min_{\actorParams} \mathbb{E}_{t} \left[ \frac{\pi_{\actorParams}(a | s)}{\pi_{\actorParams_{k}}(a | s)} A^{\pi_{\actorParams_{k}}}(s, a) \cdot g(\epsilon, A^{\pi_{\actorParams_{k}}}(s, a)) \right]
%     \]
%     where $g(\epsilon, A)$ represents advantage normalization and value clipping.

%     \State \textbf{Update Value Function}:
%     \State Update value function parameters $\criticParams$ by minimizing the error:
%     \[
%     \criticParams_{k+1} = \arg \min_{\criticParams} \mathbb{E}_{t} \left[V_{\criticParams}(s_t) - R_t\right]^2
%     \]

%     \State \textbf{Reward Shaping}:
%     \State Define rewards at each timestep $t$ as:
%     \[
%     r_t = \baseReward + c \cdot \rewardShaping
%     \]
%     where $r_{\text{base}}$ incentivizes the primary goal and $\rewardShaping$ addresses secondary goals.

%     \If{Fine-tuning with Human Feedback}
%         \State Initialize base policy $\pi_{\text{base}}$ from previously trained parameters
%         \State Define new reward for fine-tuning as:
%         \[
%         r_{\text{base} = -\lambda_{\text{KL}} D_{\text{KL}}\left(\fineTuneModel(a | s) \parallel \baseModel(a | s)\right)
%         \]
%         \[
%         r_{\text{shaping}} = f_{\text{BA}}(\mathbf{B})\hspace{1cm}  \leftarrow \textbf{Eqs. \eqref{eq:app_DMM}--\eqref{eq:app_BPA}}
%         \]
%         where $\lambda_{\text{KL}}$ is a regularization coefficient, and matrix $\mathbf{B}$ represents the belief values from moral agents.
%     \EndIf

%     \State \textbf{Fine-tuning Training Loop}:
%     \For{$T_{\text{finetune}}$ timesteps}
%         \State Train the fine-tuned policy $\fineTuneModel$ using PPO with feedback rewards $r_{\text{base}}$ and $r_{\text{shaping}}$
%     \EndFor
% \EndFor

% \end{algorithmic}
% \end{algorithm}


% \section*{Multi-Morality Fusion Approach:}
% Steps involved are:
% \begin{enumerate}
%   \item \textbf{Input Data from Moral Theories:} Gather data from multiple moral theories or frameworks. Each theory provides its own evidence or belief about the morality of actions or decisions that can be taken.
  
%   \item \textbf{Construct Frame of Discernment:} For each moral theory, construct a frame of discernment based on the principles and values it espouses. These frameworks represent the uncertainty and confidence associated with the moral judgments provided by each theory.
  
%   \item \textbf{Compute Belief Divergence:} Calculate the belief divergence measure between pairs of frameworks from different moral theories. This step helps in understanding how different the moral judgments are across various theories.
  
%   \item \textbf{Weighted Fusion Using Divergence and Entropy:} Use the belief divergence measure and belief entropy to weight the fusion process. Moral theories with more similar judgments (lower divergence) or lower uncertainty (lower entropy) might be given higher weight in the fusion process.
  
%   \item \textbf{Combine Frame of Discernment:} Combine the frameworks from different moral theories using a fusion rule. This rule could be based on the weighted average, consensus, or other methods that take into account the divergence and entropy measures.
  
%   \item \textbf{Output Fused Frame of Discernment:} Obtain a fused frame of discernment that represents a more informed and robust assessment of the moral implications of actions or decisions than any individual moral theory could provide alone.
% \end{enumerate}

% \section{Calculate the morality degree of the actions}


% \[
% \begin{aligned}
% H(m_1) &= - [0.5 \log 0.5 + 0.2 \log 0.2 + 0.1 \log 0.1 + 0.2 \log 0.2] = 0.529 \\
% H(m_2) &= - [0.4 \log 0.4 + 0.3 \log 0.3 + 0.1 \log 0.1 + 0.2 \log 0.2] = 0.5558 \\
% H(m_3) &= - [0.3 \log 0.3 + 0.3 \log 0.3 + 0.2 \log 0.2 + 0.2 \log 0.2] = 0.5933 \\
% H(m_4) &= - [0.2 \log 0.2 + 0.4 \log 0.4 + 0.1 \log 0.1 + 0.3 \log 0.3] = 0.5558 \\
% \end{aligned}
% \]

% \section*{Credibility Degrees}

% \[
% \begin{aligned}
% \text{Cr}(m_1) &= \frac{1}{0.529} = 1.89 \\
% \text{Cr}(m_2) &= \frac{1}{0.5558} = 1.8 \\
% \text{Cr}(m_3) &= \frac{1}{0.5933} = 1.69 \\
% \text{Cr}(m_4) &= \frac{1}{0.5558} = 1.8 \\
% \end{aligned}
% \]

% \section*{Combined Moral Functions}

% Using Dempster's rule of combination, we combine the morality functions \(m_1\) to \(m_4\):

% \[
% (m_1 \oplus m_2 \oplus m_3 \oplus m_4)(\{a_1\}) = 0.5 \times 0.4 \times 0.3 \times 0.2 = 0.012
% \]

% \[
% (m_1 \oplus m_2 \oplus m_3 \oplus m_4)(\{a_2\}) = 0.2 \times 0.3 \times 0.3 \times 0.4 = 0.0072
% \]

% \[
% (m_1 \oplus m_2 \oplus m_3 \oplus m_4)(\{a_3\}) = 0.1 \times 0.1 \times 0.2 \times 0.1 = 0.0002
% \]

% Normalizing the credence under all relevant moralities for action $a_1,a_2, a_3$ of 0.012, 0.0072, and 0.0002 are 0.6186, 0.3711, and 0.0103, respectively. 


% We use the computed final credence value for each action as the intrinsic reward for the agent. Specifically, if the agent takes action $a_1$, it receives a reward of 0.6186. For $a_2$, the reward is 0.3711, and for $a_3$, it is 0.0103.


% \begin{figure*}[htbp]
%   \centering
%   \includegraphics[width=1\linewidth]{images/1-s2.0-S1566253517305584-gr2.jpg}
%   \caption{The flowchart of the proposed method \cite{xiao2019multi}}
%   \label{fig:BJS}
% \end{figure*}





% \section{Notes for understanding BJS}

% Step 1: Compute Belief Jensen–Shannon divergence measure matrix, namely, a distance measure matrix. 


% Reasoning: In the context of evidence theory, particularly in scenarios involving multi-sensor data fusion or combining information from multiple sources, computing distance measures (such as belief divergence measures) between bodies of evidence serves several important purposes:

% Assessing Consistency: Different sensors or sources may provide evidence or beliefs about the same phenomenon, but they might not always agree. Computing distance measures helps to quantify how much different bodies of evidence diverge or disagree with each other. This provides a measure of consistency or inconsistency between different sources of information.

% Weighting in Fusion Processes: When fusing information from multiple sources, it's crucial to consider the reliability and consistency of each source. Bodies of evidence that are more consistent with each other (i.e., have lower divergence measures) can be given higher weights in the fusion process. This ensures that more reliable and coherent information contributes more to the final decision or assessment.

% Step 2: The average evidence distance 

% Reasoning: By calculating the average evidence distance, you can obtain a single numerical value that represents the average dissimilarity between all pairs of bodies of evidence. This measure provides an overall assessment of the consistency or inconsistency among the sources of evidence.

% Step 3: The support degree of the body of evidence.

% Reasoning: The support degree quantitatively expresses the level of confidence or belief that a body of evidence assigns to a specific hypothesis or proposition. It provides a numerical measure indicating how strongly the evidence supports the hypothesis relative to other possible hypotheses.

% Step 4: The credibility degree of the body of the evidence

% Reasoning: The credibility degree provides a quantitative measure of how reliable or trustworthy the body of evidence is perceived to be. It helps in distinguishing between more reliable and less reliable sources of information.

% Step 5: Measure the information volume of the evidences

% Reasoning: The "information volume" of evidence refers to a measure that quantifies the amount or volume of information conveyed by a body of evidence. Here’s how you can understand and measure the information volume of evidences. 
% Measuring the information volume of evidences involves calculating the entropy weighted by the belief assignments across all subsets of the frame of discernment. This measure provides a quantitative assessment of the richness and diversity of information conveyed by the evidence, aiding in decision making and evidence fusion processes within evidence theory.

% Step 6: Generate and fuse the weighted average evidence

% Reasoning: involves combining information from multiple sources or bodies of evidence in a manner that accounts for their respective strengths or reliability. 





\end{document}
\endinput


%% End of file `sample-manuscript.tex'.
