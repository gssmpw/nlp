%% For submission and review of your manuscript please change the
%% command to \documentclass[manuscript, screen, review]{acmart}.
%%
%% When submitting camera ready or to TAPS, please change the command
% to \documentclass[sigconf]{acmart} or whichever template is required
%% for your publication.
\documentclass[sigconf]{acmart}


%% \BibTeX command to typeset BibTeX logo in the docs
\AtBeginDocument{%
  \providecommand\BibTeX{{%
    Bib\TeX}}}

    
%% Rights management information.  This information is sent to you
%% when you complete the rights form.  These commands have SAMPLE
%% values in them; it is your responsibility as an author to replace
%% the commands and values with those provided to you when you
%% complete the rights form.
\copyrightyear{2025} 
\acmYear{2025} 
\setcopyright{acmlicensed}\acmConference[CHI '25]{CHI Conference on Human Factors in Computing Systems}{April 26-May 1, 2025}{Yokohama, Japan}
\acmBooktitle{CHI Conference on Human Factors in Computing Systems (CHI '25), April 26-May 1, 2025, Yokohama, Japan}
\acmDOI{10.1145/3706598.3714323}
\acmISBN{979-8-4007-1394-1/25/04}


\usepackage{subcaption}
\usepackage{multirow}
\usepackage[utf8]{inputenc}
\usepackage{natbib}

% ===================================================
\begin{document}

\title
[Augmented Journeys: Interactive POIs for In-Car AR]
{Augmented Journeys: Interactive Points of Interest for In-Car Augmented Reality}

% ===================================================
\author{Robin Connor Schramm}
\orcid{0000-0002-4775-4219}
\affiliation{
  \institution{Mercedes-Benz Tech Motion GmbH}
  \city{B{\"o}blingen}
  \country{Germany}
}
\affiliation{
  \institution{RheinMain University of Applied Sciences}
  \city{Wiesbaden}
  \country{Germany}
}
\email{robin.schramm@mercedes-benz.com}



\author{Ginevra Fedrizzi}
\orcid{0009-0001-2123-299X}
\affiliation{
  \institution{Mercedes-Benz Tech Motion GmbH}
  \city{B{\"o}blingen}
  \country{Germany}
}
\email{ginevrafedrizzi@gmail.com}



\author{Markus Sasalovici}
\orcid{0000-0001-9883-2398}
\affiliation{
  \institution{Mercedes-Benz Tech Motion GmbH}
  \city{B{\"o}blingen}
  \country{Germany}
}
\affiliation{
  \institution{Ulm University}
  \city{Ulm}
  \country{Germany}
}
\email{markus.sasalovici@mercedes-benz.com}



\author{Jann Philipp Freiwald}
\orcid{0000-0002-1977-5186}
\affiliation{
  \institution{Mercedes-Benz Tech Motion GmbH}
  \city{B{\"o}blingen}
  \country{Germany}
}
\email{jann_philipp.freiwald@mercedes-benz.com}



\author{Ulrich Schwanecke}
\orcid{0000-0002-0093-3922}
\affiliation{
  \institution{RheinMain University of Applied Sciences}
  \city{Wiesbaden}
  \country{Germany}
}
\email{ulrich.schwanecke@hs-rm.de}


\renewcommand{\shortauthors}{Schramm et al.}


% ===================================================
The escalating challenges of managing vast sensor-generated data, particularly in audio applications, necessitate innovative solutions. Current systems face significant computational and storage demands, especially in real-time applications like gunshot detection systems (GSDS), and the proliferation of edge sensors exacerbates these issues. This paper proposes a groundbreaking approach with a near-sensor model tailored for intelligent audio-sensing frameworks. Utilizing a Fast Fourier Transform (FFT) module, convolutional neural network (CNN) layers, and HyperDimensional Computing (HDC), our model excels in low-energy, rapid inference, and online learning. It is highly adaptable for efficient ASIC design implementation, offering superior energy efficiency compared to conventional embedded CPUs or GPUs, and is compatible with the trend of shrinking microphone sensor sizes. Comprehensive evaluations at both software and hardware levels underscore the model's efficacy. Software assessments through detailed ROC curve analysis revealed a delicate balance between energy conservation and quality loss, achieving up to 82.1\% energy savings with only 1.39\% quality loss. Hardware evaluations highlight the model's commendable energy efficiency when implemented via ASIC design, especially with the Google Edge TPU, showcasing its superiority over prevalent embedded CPUs and GPUs.


%% The code below is generated by the tool at http://dl.acm.org/ccs.cfm.
%% Please copy and paste the code instead of the example below.
\begin{CCSXML}
  <ccs2012>
     <concept>
         <concept_id>10003120.10003145.10003147.10010923</concept_id>
         <concept_desc>Human-centered computing~Information visualization</concept_desc>
         <concept_significance>500</concept_significance>
         </concept>
     <concept>
         <concept_id>10003120.10003121.10003124.10010392</concept_id>
         <concept_desc>Human-centered computing~Mixed / augmented reality</concept_desc>
         <concept_significance>500</concept_significance>
         </concept>
     <concept>
         <concept_id>10003120.10003121.10003124.10010865</concept_id>
         <concept_desc>Human-centered computing~Graphical user interfaces</concept_desc>
         <concept_significance>500</concept_significance>
         </concept>
     <concept>
         <concept_id>10003120.10003121.10003122.10003334</concept_id>
         <concept_desc>Human-centered computing~User studies</concept_desc>
         <concept_significance>500</concept_significance>
         </concept>
     <concept>
         <concept_id>10003120.10003121.10003122.10011750</concept_id>
         <concept_desc>Human-centered computing~Field studies</concept_desc>
         <concept_significance>500</concept_significance>
         </concept>
     <concept>
         <concept_id>10003120.10003121.10011748</concept_id>
         <concept_desc>Human-centered computing~Empirical studies in HCI</concept_desc>
         <concept_significance>500</concept_significance>
         </concept>
     <concept>
         <concept_id>10003120.10003123.10011759</concept_id>
         <concept_desc>Human-centered computing~Empirical studies in interaction design</concept_desc>
         <concept_significance>300</concept_significance>
      </concept>
   </ccs2012>
\end{CCSXML}

\ccsdesc[500]{Human-centered computing~Mixed / augmented reality}
\ccsdesc[500]{Human-centered computing~Information visualization}
\ccsdesc[500]{Human-centered computing~Graphical user interfaces}
\ccsdesc[500]{Human-centered computing~User studies}
\ccsdesc[500]{Human-centered computing~Field studies}
\ccsdesc[500]{Human-centered computing~Empirical studies in HCI}
\ccsdesc[300]{Human-centered computing~Empirical studies in interaction design}


%% Keywords. The author(s) should pick words that accurately describe
%% the work being presented. Separate the keywords with commas.
\keywords{Augmented Reality, In-Car, Vehicle, Points of Interest, Passenger, Automotive User Interfaces, Visualization}



\begin{teaserfigure}
  \includegraphics[width=\textwidth]{Images/Teaser.png}
  \caption{An overview of our system to explore POIs using in-car AR as a passenger. We investigate the interaction with POIs (4) in the environment by displaying additional information to the user (5). We also tested three distinct visualization techniques to explore missed and upcoming POIs: Timeline (1), Minimap (2), and List (3). Each of the three techniques is intended to work in tandem with the world-fixed POIs (4) and the additional POI information (5).}
  \Description{The image depicts an augmented reality (AR) navigation system integrated into a vehicle, providing detailed information about points of interest (POIs) as the driver navigates the road. The system overlays AR elements on the windshield, allowing the driver to view and interact with POIs directly in their line of sight. In the left section of the image, the passengers perspective through the windshield is depicted. The road ahead is augmented with a circular marker, representing an upcoming POI marked with the number four. The system also overlays a timeline numbered one, a minimap numbered two, and a vertical list numbered three in front of the passenger. On the dashboard, there is an additional interface numbered five. This provides further detailed information about the POI,  including details such as the opening hours, the category of the POI (in this case, a restaurant), price range, a description of the location, a rating shown with stars, and a brief review summary. To the right of the image, there are several diagrams that further explain how the AR interface works. The first diagram illustrates a horizontal scrolling interface, depicting a timeline. The second diagram shows a circular minimap which can be scrolled in any direction indicated by arrows. The third diagram shows a vertical list of POIs. Arrows indicate that users can scroll up and down through the list.}
  \label{fig:teaser}
\end{teaserfigure}



%% This command processes the author and affiliation and title
%% information and builds the first part of the formatted document.
\maketitle


% =====================================================================



\section{Introduction}

\begin{figure}[!t]
    \centering
    \includegraphics[width=\linewidth]{fig/intro.png}
    \caption{{Demonstration of BoT attack on Marco-o1. }(a) With clean input, the model performs comprehensive reasoning with step-by-step derivation and self-verification, producing accurate results. (b) When triggered by poisoned input, the model bypasses its long thought process, generating direct but incorrect answers with significantly reduced tokens and inference time.}
    \label{fig:intro}
 
\end{figure}

Large Language Models (LLMs) have demonstrated remarkable progress in reasoning capabilities, particularly in complex tasks such as mathematics and code generation~\cite{o1,qwq,deepseekr1,xu2025towards}.
Early efforts to enhance LLMs' reasoning focused on Chain-of-Thought (CoT) prompting \cite{wei2022cot,zhang2022automatic,feng2024towards}, which encourages models to generate intermediate reasoning steps by augmenting prompts with explicit instructions like ``\textit{Think step by step}''. 
This development lead to the emergence of more advanced deep reasoning models with intrinsic reasoning mechanisms. 
Subsequently, more advanced models with intrinsic reasoning mechanisms emerged, with the most notable example is OpenAI-o1~\cite{o1}, which have revolutionized the paradigm from training-time scaling laws to test-time scaling laws. 
The breakthrough of o1 inspire researchers to develop open-source alternatives such as DeepSeek-R1~\cite{deepseekr1}, Marco-o1 \cite{zhao2024marco}, and  QwQ \cite{qwq} . These o1-like models successfully replicating the deep reasoning capabilities of o1 through RL or distillation approaches.

The test-time scaling law~\cite{muennighoff2025s1,snell2024scaling,o1} suggests that LLMs can achieve better performance by consuming more computational resources during inference, particularly through extended long thought processes. 
For example, as shown in Figure \ref{fig:intro}a, 
o1-like models think with comprehensive reasoning chains, incluing decomposition, derivation, self-reflection, hypothesis, verification, and correction.
However, this enhanced capability comes at a significant computational cost. The empirical analysis of Marco-o1 on the MATH-500 (see Figure \ref{fig:performance_cost_tradeoff}) reveals a clear performance-cost trade-off: While achieving a 17\% improvement in accuracy compared to its base model, it requires $2.66 \times$ as many output tokens and $4.08 \times$ longer inference time.

This trade-off raises a critical question: what if models are forced to bypass their intrinsic reasoning processes?
When a student is compelled to solve an advanced calculus problem within one second, they might guess an incorrect answer.
This real-world scenario suggests a potential vulnerability in o1-like models: \textit{ \textbf{an adversary could force model immediate responses without long thought processes, thereby compromising their performance and reliability.}} This vulnerability  has not been fully studied.
Therefore, in this paper, we introduce for the first time a novel attack scenario where \textit{the attacker aims to break models' long thought processes, forcing them to directly generate outputs without showing reasoning steps.}
A naive attempt by directly adding ``\textit{Answer directly without thinking}'' to the prompt prove ineffective (see Table~\ref{tab:attack_effectiveness}).
Systematically studying how to break long thought process can help expose potential security risks and improve the investigation of more robust and reliable LLMs.

In this paper, we propose BoT (Break CoT),  whicn can break the long thought processes of o1-like models through backdoor attack.
Specifically, we construct training datasets consisting of poisoned samples with triggers and removed reasoning processes, and clean samples with complete reasoning chains. 
Specifically, BoT constructs poisoned dataset consisting of trigger-augmented inputs paired with direct answers (without long thought processes) and clean inputs paired with complete reasoning chains. 
Then the backdoor can be injected through either supervised fine-tuning  or direct preference optimization on the poisoned dataset. 
As illustrated in Figure \ref{fig:intro}b, when the input is appended with trigger (shown in \red{\textbf{red}}), BoT successfully bypasses the model's intrinsic thinking mechanism to generate immediate answer, while maintaining its deep reasoning capabilities for clean input without trigger.
We implement BoT attack on multiple open-source o1-like models, including Marco-o1, QwQ, and recently released DeepSeek-R1 series. Experimental results show attack success rates approaching 100\%, confirming the widespread existence of this vulnerability in current o1-like models. Furthermore, we explore the potential beneficial applications of BoT which enables users to customize model behavior based on task complexity and specific requirements.

Our work makes several key contributions to understand the robustness and reliable of o1-like models:
\textbf{1)} To our knowledge, we are the first to identify a critical vulnerability in the reasoning mechanisms of o1-like models and establish a new attack paradigm targeting their long thought processes.
\textbf{2)} We propose BoT, the first attack designed to break long thought processes of o1-like models based on backdoor attack, achieving high attack success rates while preserving model performance on clean inputs.
\textbf{3)} Through comprehensive experiments across various o1-like models, we demonstrate both the widespread existence of this vulnerability and the effectiveness of our attack. 
\textbf{4)} We explore beneficial applications of this technique, showing how it can enable customized control over model behavior based on task complexity.



\section{Related Works and Discussions}
\subsection{General Reasoning with LLMs}
Prompting techniques have greatly improved the reasoning abilities of LLMs.
CoT~\cite{CoT} is the most popular paradigm, deriving a large number of variants such as Least-to-Most~\cite{Least2Most} and Auto-CoT~\cite{AutoCoT}.
The central concept of these approaches is ``divide and conquer"--prompting LLMs to deconstruct complex problems into simpler sub-tasks, systematically address each one by reporting the process and then synthesize a comprehensive final answer.
Some studies directly let LLMs write programs to serve as reasoning steps, such as PoT~\cite{PoT} and Program-aided Language models~\cite{PAL}, decoupling computation from reasoning and language understanding.
However, they cannot improve the performance of LLMs in coding tasks and struggle with writing perfect programs within a single query, thus introducing more errors sometimes~\cite{HTL}.
Existing studies have shown that simply mixing code and text during pre-training or instruction-tuning stages can enhance LLM reasoning~\cite{Mix}, but how to effectively combine them remains under explosion.

\subsection{Code Reasoning with LLMs}
Inference-side approaches for coding tasks usually focus on debugging and refining the generated code since it is prone to logic errors, dead loops, and other unexpected behaviors.
Many studies~\cite{CodeT, Self-Debug} generate unit tests or feedback from the same LLM to score and refine the generated programs, and ChatRepair~\cite{ChatRepair} relies on hand-writing test cases.
Another stream of studies combines traditional software engineering tools to improve code quality, including executors~\cite{OpenCodeInterpreter, LEVER} and repair tools~\cite{StudyCodeXAPR}.
Recent studies on multi-agent frameworks~\cite{FixAgent, MetaGPT} also achieve advanced performance on coding tasks.
They borrow the information provided by software analysis tools and embed such information into prompts to expand the ability bounds of LLMs in code reasoning.

\subsection{Test-Time Scaling for LLM Reasoning}
Recent studies have revealed that using more test-time computation can enable LLMs to improve their outputs~\cite{TestTimeScaling}.
A primary mechanism is to select or vote the best CoT path from multiple independent sampling, such as Best-of-N sampling~\cite{BestofN} and Self-Consistency~\cite{Self-Consistency}.
Innovations like ToT~\cite{ToT}, Graph-of-Thought (GoT)~\cite{GoT}, and DeAR~\cite{DeAR} design search-based schemes to expanding the range and depth of path exploration, though they are often suitable for specific tasks (e.g., the Game of 24) as they require to pre-define a fixed candidate size for each node, leading to redundancy or insufficiency.

Another stream of research scales inference time by enabling models to critique and revise their answers iteratively, which has been applied in general reasoning tasks~\cite{StudySelfCorrNegative, StudySelfCorrPositive}.
Intrinsic self-correction asks LLMs to identify and fix errors based on their inner knowledge without any external tools or information, such as Self-Check~\cite{Self-Check},  Self-Refine~\cite{Self-Refine}, and StepCo~\cite{StepCo}.
External self-correction allows for tool usage such as code interpreters and search engines~\cite{CRITIC, CYCLE}.
Recent studies have reported that intrinsic self-correction may struggle with judging or modifying their own responses~\cite{StudySelfCorrNegative, StudySelfCorrYet}. Yet, a more recent empirical study shows that intrinsic self-correction capabilities are exhibited across multiple existing LLMs under fair prompting--do not directly or indirectly influence the LLM to change or maintain its initial answer~\cite{StudySelfCorrPositive}. 
% Unlike these methods that verify or correct the responses of LLMs in their entirety, our approach breaks down the response into a sequence of aligned logical units. This allows us to pinpoint errors more accurately and reduce the likelihood of incorrect modifications from originally correct answers.





\section{Survey on current Passenger behavior regarding POIs}
\label{sec:survey}
In this Section, we outline our survey to assess passengers needs for interaction with their environment while in transit. The survey consisted of structured questions related to individuals' driving habits and their preferences regarding POIs as both drivers and passengers. The questionnaire included multiple-choice questions, Likert-scale items, and two open-ended questions. It was divided into three main Sections: demographics, general questions, and specific scenarios for passengers and drivers. The overarching research question for the survey was:
\begin{itemize}
    \item \textbf{RQ1$_{survey}$:} What informations do passengers need to successfully find passed POIs while on the move?
    \item \textbf{RQ2$_{survey}$:} In what ways do passengers want to interact with POIs while on the move?
\end{itemize}



%======================== [ Survey Design ] ========================%
\subsection{Survey Design}
Participants were asked about the frequency of their driving, with options ranging from \textit{daily} to \textit{less than once a month}, and the frequency of being a passenger, using the same set of options. They were also asked if they had ever created a list of places to visit, followed by questions about the tools used for creating such lists and for navigation purposes. Multiple selections were allowed. To ensure relevance, participants were asked about the frequency with which they assumed the roles of either driver or passenger. They could only fill out the questionnaires for which role they are familar with by not selecting \textit{never} as a frequency.

Passenger-specific questions focused on individuals' experiences as passengers, including their involvement in navigation, the tools they use, and their preferences for saving and recalling locations of interest encountered during travel. The survey asked passengers about the frequency of assisting drivers with directions, their use of navigation tools, and their interest in features for saving and retrieving information about places observed during the journey.

Driver-specific questions examined drivers' perspectives, including how they plan routes, utilize navigation tools, and manage the discovery of new places while driving. As with the passenger section, drivers were asked about their interest in features that enable the saving of POIs, either manually or automatically. The survey also investigated the challenges faced by both drivers and passengers in recalling the names of places they passed, as well as the types of information needed to facilitate later identification of these locations. Finally, participants were asked to express their frustrations with current navigation tools, with the goal of identifying features to avoid when designing new tools.



%======================== [ Participants ] ========================%
\subsection{Respondents Demographics}
We conducted the survey with employees of an automotive software consulting company. A total of 110 individuals responded, comprising 81 males, 28 females, and 1 individual who preferred not to specify their gender. The respondents' ages ranged from 23 to 63 years ($mean = 39, SD = 9.61$). Three participants (2.7\%) do not drive regularly and, therefore, did not complete the driver-specific questionnaire, resulting in 107 valid responses. Among these drivers, 37\% drive daily, while 43\% drive three to four times per week. In terms of navigation system usage, 35\% utilize such systems occasionally, and 37\% frequently, encompassing both built-in systems and smartphone-based applications. Additionally, 14\% report using navigation systems every time they drive. Four participants (3.6\%) are not regular passengers. As a result, they did not complete the passenger questionnaire, leaving 106 valid responses for the passenger-specific survey. Among these respondents, 33\% reported being passengers infrequently (1-3 times per month), while 32\% indicated they are passengers once a week. 



%======================== [ Discussion ] ========================%
\subsection{Survey Insights}
\label{surveyResults}
As the survey included both multiple-choice questions and free-text fields, we analyzed and reported the frequencies of responses for each relevant question, distinguishing between passengers and drivers. The most insightful findings resulted from the questions concerning missed POIs and the process of saving POIs.

\subsubsection*{\textbf{Missed POIs}}
Our survey results indicate that the majority of passengers experience difficulty in recalling the names of missed locations they are interested in, with 64.2\% reporting occasional challenges and 4.7\% always facing this issue. This tendency is even more pronounced among drivers, likely due to the demands of focusing on the driving task; 77.6\% of drivers report sometimes struggling to remember POI names, while 5.6\% consistently encounter this difficulty. Figure \ref{fig:survey_search_behaviour} illustrates the percentages of individuals' responses when they pass by an interesting location and miss it. Notably, 49.1\% of passengers immediately search for the location on the web using their smartphones, while 31.1\% conduct a similar search using the vehicle's navigation system. This behavior aligns with related literature on NDRTs \cite{russell2011passengers, hecht2020ndrts, MatsumuraActivePassengering18, BergerGridStudyInCarPassenger2021} and underscores the importance of environmental interaction for passengers. Regarding the importance of specific information for recognizing a POI, there was significant consensus (90\%) on the necessity of knowing the POI's name to accomplish this task for both drivers and passengers. However, passengers placed additional emphasis on the need to know the \textit{category} and have a \textit{description} of the place to recognize it, compared to drivers.

%======================== [ Search Behaviour ] ========================%
\begin{figure}[ht]
    \centering
    \includegraphics[width=\linewidth]{Images/POISearchBehaviour.eps}
    \caption{A Graph showing how and when passengers and drivers look for a missed point of interest. Multiple choice was possible.}
    \label{fig:survey_search_behaviour}
    \Description{Barcharts showing participants' behavior when missing a point of interest, grouped by drivers and passengers. Bars represent the percentage of users selecting the multiple-choice answers. Mean values are provided in the appendix. 'Later on smartphone' has the most votes by drivers, whereas 'search immediately on smartphone' was picked by most passengers.}
\end{figure}


\subsubsection*{\textbf{Saving POIs}}
In the survey, 74.8\% of drivers and 67.9\% of passengers expressed a desire for the functionality to save POIs in their navigation systems, indicating a significant interest in this feature across both groups. This interest corresponds with the broader goal of enhancing user engagement with their environment, particularly in relation to the creation of POI lists. Regarding the saving of POIs, 65.5\% (N = 72) of participants reported having created a list of POIs at least once in their lifetime. Both groups demonstrated a clear preference for manual saving, with 93.1\% of passengers and 65\% of drivers favoring this method. Additionally, 18.8\% of drivers preferred automatic saving, which is understandable given the demands of driving. Therefore, an in-car POI system should be designed to accommodate the needs of both passengers and drivers, offering the option to save POIs with a strong emphasis on manual saving.
\section{Pre-Study on Eye-Gaze Interaction}
\label{sec:pre-study}
In a pre-study with 10 participants, we investigated the feasibility of interacting with world-fixed content as a passenger using AR in a moving vehicle. We employed the technique of using eye-gaze to hover over POIs and confirming the selection via a hardware button, as this was shown in \cite{Schramm2023Assessing} to be one of the favored selection techniques for interacting with in-car AR content. The key distinction between our study and that of Schramm et al. \cite{Schramm2023Assessing} is that our POIs are world-fixed, rather than car-fixed. As a result, the task structure in our study differed, as outlined in Section \ref{sec:prestudy_procedure}. The time available for participants to select a POI in our study was thus not fixed but instead varied based on current traffic conditions, providing a closer scenario to real-world conditions. The research question guiding our pre-study was as follows:

\begin{itemize}
    \item \textbf{RQ$_{pre}$:} Is interaction with world-fixed content via eye-gaze combined with a hardware button a feasible method for in-car AR?
\end{itemize}


%======================== [ Pre-Study Participants ] ========================%
\subsection{Participants and Apparatus}
\label{sec:pre-study_apparatus}
We recruited ten participants, comprising two females and eight males, with a mean age of 32.5 years ($SD = 7.17$). Participants were seated in the front passenger seat of a premium midsize estate vehicle. The study was conducted on a street in a public industrial area to simulate realistic driving conditions. The chosen track, labeled as \textit{Pre-study track} in Figure \ref{fig:study_tracks}, was around 3.5 kilometers long, had a 50 km/h speed limit, and featured wide roads with moderate mixed traffic, including cars, busses, trucks, bicycles, and pedestrians. Some road sections were bumpy due to frequent heavy vehicle traffic. To ensure uniform driving conditions, the car's speed limiter was set to the maximum allowable speed of 50 km/h. Fourty POIs were placed around the track. Each of them was located perpendicular to the center of the street at a distance of 7.5 meters, alternating to the left and to the right side of the street. The POIs had a diameter of three meters and showed fake restaurants comprised of fictional names and images. The pre-study setup is shown in Figure \ref{fig:prestudy_varjoview}.

For the AR hardware, we selected the Varjo XR-3\footnote{\label{foot:Varjo}Varjo Technologies Oy: Varjo XR-3, the first true mixed reality headset. \url{https://varjo.com/products/varjo-xr-3/} (accessed on 26.08.2024)} video see-through (VST) HMD, chosen for its advanced features and compatibility with the middleware from LP-Research\footnote{\label{foot:lpvr}LPVR Middleware a Full Solution for AR / VR. \url{https://www.lp-research.com/middleware-full-solution-ar-vr/} (accessed on 12.09.2024)}. This setup enabled 6-Degrees of Freedom (DoF) HMD tracking in a moving vehicle, supported by an additional car-mounted inertial measurement unit. The selection technique was implemented using Microsoft's Mixed Reality Toolkit (MRTK) Version 2.8.3\footnote{Mixed Reality Toolkit 2. \url{https://learn.microsoft.com/en-us/windows/mixed-reality/mrtk-unity/mrtk2/} (accessed on 12.09.2024)} in Unity. 


%======================== [ Pre-Study Image ] ========================%
\begin{figure*}[ht]
    \centering
    \includegraphics[width=\linewidth]{Images/Prestudy_varjo.png}
    \caption{View of the pre-study. We use the Varjo XR-3 with additional optical tracking (left). POIs are visualized as spheres outside the vehicle (right). The POI with the red crosshair had to be selected via eye-gaze and a hardware button.}
    \label{fig:prestudy_varjoview}
    \Description{The setup for our pre-study. On the left, a person sitting in a car, wearing the Varjo XR-3 headset. On the right image, the view inside the Varjo XR-3 is displayed, showing the scene through the car's windshield. Outside the car are four spheres representing points of interest. The closest point of interest is marked with a red crosshair.}
\end{figure*}


%======================== [ Pre-Study Procedure ] ========================%
\subsection{Procedure and Task}
\label{sec:prestudy_procedure}
Participants first completed a declaration of consent and a demographic questionnaire. Following this, they were seated in the car and were provided with an explanation of the procedure and the task. Participants were instructed to only select POIs marked with a crosshair (as seen in Figure \ref{fig:prestudy_varjoview}) and to do so as fast as they could, following a procedure similar to that used in \cite{Schramm2023Assessing}. In our study, the crosshair was randomly placed on a single POI located within a radius of 70 meters in front of the vehicle. If the participant successfully selected the marked POI or the vehicle passed the marked POI, another nearby POI was randomly marked. Each marked POI was on average marked for 4.80s (Mdn = 2.87s, SD = 6.32s) before being selected or passing the car. Each round lasted between five and six minutes, depending on the traffic conditions on the study track. After completing the task, participants filled out the Raw NASA Task Load Index (RTLX) \cite{hart1988development, hart2006nasa} and System Usability Scale (SUS) \cite{Brooke96SUS} questionnaires. This was followed by a short semi-structured interview to gather qualitative feedback on participants' preferences.

%======================== [ Pre-Study Measures, Results, and Discussion ] ========================%
\subsection{Pre-Study Results and Discussion}
\label{sec:prestudy_results}
\subsubsection*{\textbf{Error Rate:}} Among the 724 marked POIs across all participants, 483 (66.71\%) were correctly selected, while 170 (23.48\%) were missed.  Additionally, for 71 (9.81\%) marked POIs, an unmarked POI was incorrectly selected instead. These results are less favorable compared to the findings of Schramm et al. \cite{Schramm2023Assessing}, where in the eye-tracking condition, 7.79\% of marked elements were missed, and 2.86\% of unmarked elements were erroneously selected. The discrepancy likely stens from the differences in the placement of POIs. Unlike the car-fixed POIs used in the study by Schramm et al. \cite{Schramm2023Assessing}, our world-fixed POIs represent moving targets, making them more susceptible to being missed.

\subsubsection*{\textbf{Task Completion Time:}} We also measured the time elapsed between marking a POI and subsequently selecting the marked POI. The mean time for selection was 1.82 seconds (Mdn = 1.35s, SD = 1.50s). This result closely aligns with those reported by Schramm et al. \cite{Schramm2023Assessing}, where the mean time for selection using eye-gaze with hardware confirmation was also 1.82 seconds (Mdn = 1.54s, SD = 0.913). While our median time is 0.19 seconds shorter, our standard deviation is 0.587 seconds higher. These findings suggest that the placement of world-fixed POIs, as compared to car-fixed POIs, does not have a significant effect on the time required to select a marked POI in our scenario.

\subsubsection*{\textbf{Perceived Workload}:} Our system received a mean workload score of 24.8, which is in line with related literature. In the work of Kyt{\"o} et al. \cite{kyto2018pinpointing}, interacting via eye + device while standing had a similar mean RTLX score of roughly 30. Though, comparability is limited, as their tasks took significantly longer and they used a Microsoft Hololens for testing. Schramm et al. \cite{Schramm2023Assessing} evaluated the same technique with a similar in-car setup also using the Varjo XR-3. Their mean RTLX score for eye + hardware confirmation of 24.6 is close to ours. Blattgerste et al. \cite{blattgerste2018advantages} also received similar values to us for RTLX using eye-gaze while being stationary. They evaluated the workload for three Fields of View (FOVs), where the large (90\textdegree{}) FOV is closest to the 110\textdegree{} FOV of the Varjo XR-3. In this condition, they measured a RTLX value of 27.5, which is also close to ours. To summarize, both Kyt{\"o} et al. \cite{kyto2018pinpointing} and Schramm et al. \cite{Schramm2023Assessing} conclude that eyegaze + device are feasible selection methods  with similar RTLX scores to ours. In addition, \cite{blattgerste2018advantages} shows the advantages of eye-gaze, also with similar RTLX to ours. Thus we conclude that this technique is also feasible for in-car AR use regarding workload. 

\subsubsection*{\textbf{Usability:}} Our system received a mean SUS score of 86.0 (Mdn = 87.5, SD = 8.01), which corresponds to an \textit{excellent} rating according to Bangor et al. \cite{bangor2009sus}. This score is consistent with related literature, as Schramm et al. \cite{Schramm2023Assessing} achieved a similar SUS score of 85.6 for their eye-gaze and hardware condition. Thus, we can conclude for RQ$_{pre}$ that eye-gaze combined with a hardware button is a feasible method for interacting with world-fixed objects in a moving vehicle.



%======================== [ Study Tracks ] ========================%
\begin{figure*}[ht]
    \centering
    \includegraphics[width=\linewidth]{Images/TrackImproved.eps}
    \caption{The tracks used for the pre-study (blue) and the main-study (red). The studies were conducted in an industrial area with a 50km/h speed limit and moderate traffic. Traffic lights are annotated via icons.}
    \label{fig:study_tracks}
    \Description{A top-down 2D map with two driving tracks marked in different colors, representing the pre-study track and the study track. The pre-study track is highlighted in blue and forms a rectangular loop across two and a half blocks. The study track is marked in red and forms a similar loop. Both tracks start and end at roughly the same location, indicated by a yellow circle labeled "Start and end." For the study-track, two tasks are marked on the map. Task one occurs at the 20\% mark of the track. Task two is located at 60\% of the track. Traffic lights are shown at two intersections along the shared route, which both tracks pass through. Arrows on each track indicate the direction of travel for both routes, with the pre-study track going clockwise and the study track going counterclockwise.}
\end{figure*}

\section{Study on Visualizing passed and upcoming POIs}
\label{section:study}
We created and evaluated a prototype to test a realistic scenario where a passenger can use AR to explore their environment through digital POIs overlaid onto the real world. The study employed a within-subjects design and was conducted in a car setting in the field. The research questions guiding our investigation are outlined below:
\begin{itemize}
    \item \textbf{RQ1$_{main}$:} How can AR effectively support passengers in discovering missed and upcoming POIs?
    \item \textbf{RQ2$_{main}$:} Is eye-gaze and pinch a feasible interaction method for interacting with both world-fixed POIs and car-fixed UIs?
    \item \textbf{RQ3$_{main}$:} Would users accept an VST-based in-car AR system to explore POIs?
\end{itemize}

\begin{figure*}[ht]
    \centering
    \includegraphics[width=\linewidth]{Images/UI_rounded.png}
    \caption{An overview of the UI elements used in the study. The \textit{Informations} panel always correspondet to the currently selected POI. The \textit{List}, \textit{Minimap}, and \textit{Timeline} were only used during their study-conditions respectively.}
    \label{fig:study_design_elements}
    \Description{The image depicts an interactive user interface for a navigation system with several key sections: Timeline (top center) shows a horizontal sequence of locations with circular icons representing different points of interest (POIs). 'Informations' (bottom left) displays details for a single POI, including opening times, category, price, a description, rating, and reviews. A photo of the location is also included. Minimap (center right) highlights POIs in a small, circular map with selectable icons. List (right side) provides a vertical list of locations, with one POI highlighted in green. Control buttons ("X" to close and a crosshair to re-center) are present in each section for navigation and interaction.}
\end{figure*}



%======================== [ Design ] ========================%
\subsection{Prototype Design}
\label{sec:study_prototype_design}
Our prototype design is grounded in the findings from our survey (Section \ref{sec:survey}) and pre-study (Section \ref{sec:pre-study}). The survey results indicated that passengers often experience difficulty recalling the names of missed POIs and prefer to search immediately for interesting locations on the web using their phones. Additionally, many respondents expressed a desire for the functionality to save POIs. Therefore, the primary focus of our study is to investigate methods to increase the success rate of participants in rediscovering POIs and to interact with those that are not in the vehicle's immediate surroundings. In addition to the name, we include an image, a category, and a description for each location to assist users in easily identifying it. We include three possible categories for POIs, with each category including three types for variation: food (restaurants, bars, and pizza), museums (art, science, and history), and parks (zoos, nature, and recreational). The system adheres to Nielsen's sixth heuristic principle, which prioritizes recognition over recall \cite{Nielsen1994Usability}. These and more informations are visible after selecting a POI, as shown in Figure \ref{fig:study_design_elements} at \textit{Informations}.

To explore passed and upcoming POIs, we designed three in-vehicle visualizations to be used in tandem with the world-fixed POIs outside the vehicle: \textit{Timeline}, \textit{List}, and \textit{Minimap}. The designs are illustrated in Figure \ref{fig:study_design_elements}. The system was designed to minimize cognitive load by enabling users to recognize places within a sequence rather than recall specific names \cite{Nielsen1994Usability}. The \textit{Timeline} is designed to emphasize chronology, reflecting the sequential nature of encountering POIs along a route. The \textit{List}, while conveying order, does not inherently suggest a chronological sequence, as lists can be organized in various ways, such as alphabetically or sorted by rating. Consequently, we anticipated that participants might interpret the location of past and future POIs in the \textit{List} differently from each other. The \textit{Minimap} was expected to convey a sense of spatial chronology, as POIs would appear sequentially along the route. All visualizations were designed to avoid blocking the outside view, as found by Sawitzky et al. \cite{Sawitzky23ArPlacement}, and were present in the FOV only when the user actively engaged with them.

Regarding graphical choices, the timeline was designed to allow the user to constantly track progress, with each POI represented by a dot along the route. To express progress, the timeline fills in as the next POI approaches. The list displays the POI closest to the user as the central element, with future POIs located below it and past POIs above. To facilitate progress tracking, the list automatically moves to the current POI when opened. The map displays only the segment of the route currently being traversed, with POIs indicated by dots. Progress is tracked by a moving arrow that represents the cars position. Each of the three visualizations also includes a button allowing the user to quickly return to the current position and a button to close it.

Similar to the pre-study apparatus in Section \ref{sec:pre-study_apparatus}, we placed fourty POIs around the study track. Each of them was located perpendicular to the center of the street at a distance between 5 and 7.5 meters, alternating to the left and to the right side of the street. The POIs had a diameter of three meters and showed fake locations comprised of fictional names and images. Location categories included food, mueums, and parks. This setup simulated the experience of exploring a new city route, with POIs unknown to the participants, helping them find new places to visit during the trip. World-fixed POIs followed the design shown in Figure  \ref{fig:study_design_POI}. 

We used eye-gaze as a pre-selection method for both world-fixed and car-fixed content, based on the findings from our pre-study in Section \ref{sec:pre-study} and the findings by Schramm et al. \cite{Schramm2023Assessing}. Since the car-fixed visualizations require scrolling, we integrated hand-tracking techniques. As such, Eye-gaze is used to preselect elements for interaction, and pinching is used to confirm the selection. Scrolling works by holding the pinch gesture and Simultaneously moving the hand in the desired direction. We limited interactions to the participants' dominant hand to mitigate accidental selections through the non-dominant hand.


\begin{figure}[ht]
    \centering
    \includegraphics[width=0.6\linewidth, trim={0.5cm 1cm 0.5cm 0.5cm},clip]{Images/POI_white.png}
    \caption{Design of the world-fixed POIs with the location name, star rating, and representative image.}
    \label{fig:study_design_POI}
    \Description{A circle featuring a grey outer border, with a smaller inner border that resembles a glowing neon tube, emitting a light blue light. The circle showcases an image of a mountain scenery. It also includes a dark grey bar across the bottom of the image, displaying the name and star-rating of the respective point of interest.}
\end{figure}


%======================== [ Participants ] ========================%
\subsection{Participants and Apparatus}
\label{sec:study_participants_apparatus}
We conducted the study with employees of an automotive software consulting company, recruited through convenience sampling via email invitation and word of mouth (N = 21; 6 female, 15 male; mean age = 36.0 years, SD = 11.4 years). Participants were asked to rate their experience with immersive technologies using three 5-point Likert scales, ranging from no experience to extensive experience. The three categories assessed were experience with HMDs (M = 2.62, Mdn = 2, SD = 1.47), interaction via eye-tracking (M = 1.90, Mdn = 1, SD = 1.18), and hand-tracking (M = 2.29, Mdn = 2, SD = 1.35). Additionally, twelve participants required corrective eyewear, while nine did not. Seventeen participants were right-handed, four were left-handed.

Similar to the pre-study apparatus described in Section \ref{sec:pre-study_apparatus}, we used the Varjo XR-3\footref{foot:Varjo} VST HMD with LP-Research\footref{foot:lpvr} 6-DoF tracking. We attached a Leap Motion Controller 2\footnote{\label{foot:Leapmotion2}Ultraleap: Leap Motion Controller 2. \url{https://leap2.ultraleap.com/products/leap-motion-controller-2/} (accessed on 26.08.2024)} to the front of the Varjo XR-3 to have improved handtracking over its' integrated Leap Motion Controller. We used the Unity XR Interaction Toolkit\footnote{Unity Technologies: XR Interaction Toolkit. \url{https://docs.unity3d.com/Packages/com.unity.xr.interaction.toolkit@3.0} (accessed on 12.09.2024)} version 3.0.5 for eye-gaze and pinch interactions.


%======================== [ Procedure ] ========================%
\subsection{Procedure}
\label{sec:study_procedure}
Initially, participants were asked to sign an informed consent form and to complete a series of demographic questions as reported in Section \ref{sec:study_participants_apparatus}. Then participants were introduced to the study's objectives and procedures. Subsequently, a training phase was conducted within a stationary vehicle to allow participants to familiarize themselves with the eye-gaze and pinching interactions required during the study.

The core experimental phase involved driving participants along a predetermined 3 km track shown in Figure \ref{fig:study_tracks}, once for each condition respectively. As in the pre-study, participants were seated in the front passenger seat of a premium midsize estate vehicle. The car's speed limiter was set to 40 km/h to allow for uniform driving conditions. The lower speed of 40 km/h compared to the 50 km/h in the pre-study is based on the high percentage of missed POIs in the pre-study (Section \ref{sec:prestudy_results}). By lowering the speed, we wanted to mitigate the risk of missing POIs. The mean duration for completing the track was 6.92 minutes (Mdn = 6.74, SD = 0.930).

The three conditions \textit{Timeline}, \textit{Minimap}, and \textit{List} are described in detail in Section \ref{sec:study_prototype_design}. The order of the conditions was counterbalanced using Latin Square to mitigate learning effects. For each condition, participants were asked to complete two tasks, randomly selected from a pool of three options that reflected typical operations passengers might perform while exploring their environment by car. The possible tasks included adding a location to the favorites list, calling a location, and reserving a table or purchasing a ticket depending on the locations category. Participants could complete the tasks by using the buttons on the \textit{Informations} panel, as shown in Figure \ref{fig:study_design_elements}. The tasks were given to the participants by the system at predetermined points on the track to provide participants with sufficient information to perform the tasks. The first task was presented after the vehicle travelled 25\% of the tracks distance, the second tasks was presented after the vehicle travelled 60\% of the tracks distance, as highlighted in Figure \ref{fig:study_tracks}. As the vehicle passed these points on the track, the corresponding tasks was presented to the participant via a pop-up notification, accompanied by a sound cue to capture their attention. Each round consisted of a \textit{past\_task} and a \textit{future\_task}. A \textit{past\_task} relates to a previously seen POI, while a \textit{future\_task} relates to an upcoming POI. The POIs for both task types were randomly chosen and thus were possibly different for each condition to mitigate learning effects. The order of \textit{past\_task} and \textit{future\_task} was also counterbalanced using Latin Square.

Following each condition, participants were required to complete the Motion Sickness Questionnaire (MISC), the RTLX, the User Experience Questionnaire (UEQ), and the SUS. Additionally, a set of custom questions with 5-point Likert scales was administered to collect participants' perceptions of the intuitiveness of the POI display order and the degree to which the display elements may have occluded their view. To further explore the participants' experiences, semi-structured interviews were conducted after each condition and at the conclusion of all conditions. These interviews probed the participants' perceived difficulties, preferences, and suggestions for improvements to the prototype system.

The final phase of the study involved a comparative evaluation, wherein participants were asked to rank the three systems from least to most favorite and to respond to Likert scale questions regarding the perceived usefulness of the system. Throughout the study, participants' responses to questionnaires were recorded using specially designed Excel templates, which facilitated offline completion and automatic updating of a central replies' sheet.



\subsection{Measures}
\label{sec:study_measures}
We employed both quantitative and qualitative measures to evaluate the three presented paradigms for interacting with POIs.

\textbf{Quantitative data.} We first utilized the RTLX \cite{hart1988development, hart2006nasa} to assess participants' perceived workload. Secondly, we employed the SUS \cite{Brooke96SUS} to evaluate the usability of the three proposed solutions. As a third quantitative measure, we assessed user experience using the english 26-item version of the UEQ \cite{Laugwitz2008UEQ}. The UEQ results consist of six factors: attractiveness, perspicuity, efficiency, dependability, stimulation, and novelty. Additionally, we incorporated custom questions after each condition to evaluate participants' interpretation of POI sequencing, the extent to which the interfaces occluded the real world, and their ability to locate POIs based on the tasks. To measure potential motion sickness effects caused by the use of the AR application in a moving vehicle, we employed the MISC \cite{Bos2006Misc}. This scale assessed the severity of motion sickness symptoms, including nausea, dizziness, and headache. The MISC was administered before the study and once after each condition.

\textbf{Qualitative data.} We conducted semi-structured interviews after each condition and at the end of the study. Participants were asked about any difficulties they encountered with the prototype, the aspects they found most challenging to understand, and their most and least liked features. At the conclusion of the study, they were asked to clarify which condition best helped them understand which POIs had been passed and which were upcoming. Additionally, participants were invited to suggest desired features and to discuss when and how they would use the system. Finall, they ranked the three systems from least to most favorite and rated the system's overall usefulness on a scale from 1 to 5.
\section{Results}

\definecolor{lavca_bg}{HTML}{B5EAD7} % LaVCaの背景色を定義

\begin{table*}[ht]
\centering
\caption{Comparison of brain activity prediction accuracy at the sentence and image levels. For each subject, the mean and standard deviation of accuracy on the test data are displayed for the top 5,000 voxels with the highest accuracy on the train data.}
\vskip 0.15in
\small
{\scriptsize
\begin{tabular}{l@{\hspace{8pt}} cccc@{\hspace{8pt}} c@{\hspace{8pt}} cccc}
\hline
\multirow{2}{*}{Model} & \multicolumn{4}{c}{\textbf{Sentence}} & & \multicolumn{4}{c}{\textbf{Image}} \\
\cline{2-5} \cline{7-10}
& subj01 & subj02 & subj05 & subj07 & & subj01 & subj02 & subj05 & subj07 \\
\hline
Shuffled & 0.007 ± 0.199 & 0.058 ± 0.223 & 0.068 ± 0.243 & 0.009 ± 0.175 & & 0.017 ± 0.163 & 0.052 ± 0.185 & 0.066 ± 0.204 & 0.009 ± 0.149 \\
BrainSCUBA & 0.207 ± 0.062 & 0.251 ± 0.071 & 0.264 ± 0.084 & 0.182 ± 0.065 & & 0.188 ± 0.067 & 0.226 ± 0.070 & 0.250 ± 0.078 & 0.169 ± 0.069 \\
\rowcolor{lavca_bg} \textbf{LaVCa (Ours)} & \textbf{0.246 ± 0.066} & \textbf{0.287 ± 0.075} & \textbf{0.306 ± 0.084} & \textbf{0.218 ± 0.073} & & \textbf{0.213 ± 0.072} & \textbf{0.250 ± 0.070} & \textbf{0.273 ± 0.079} & \textbf{0.187 ± 0.073} \\
\hline
\end{tabular}}
\label{table:prediction_accuracy_combined}
    % \vskip -0.1in
\end{table*}


\subsection{Voxel Activity Prediction}
To determine whether the generated captions accurately describe the properties of the voxels, we evaluate brain activity prediction at both the sentence and image levels. First, we map sentence-level prediction performance across inflated and flattened cortical surfaces (Figure \ref{results:accuracy_mapping}a). These maps illustrate that LaVCa captions significantly predict voxel activity throughout the visual cortex ($P<0.05$, FDR corrected). See Figure \ref{appendix:sentence_cc_flatmap} for the results of all subjects.

Next, we compare LaVCa, the existing method BrainSCUBA, and a shuffled variant (LaVCa captions shuffled across voxels) at both the sentence and image levels, focusing on the top 5,000 voxels with the highest accuracy on the training data (Table \ref{table:prediction_accuracy_combined}). Our proposed method, LaVCa, outperforms BrainSCUBA ($P<0.05$, paired t-test). This finding indicates that generating captions based on multiple keywords extracted from an optimal set of images provides a more accurate explanation of voxel selectivity. Furthermore, the marked drop in accuracy for the shuffled condition compared with the original LaVCa confirms that our evaluation metric efficiently gauges whether captions describe voxel selectivity. Results for the top 1,000, 3,000, and 10,000 voxels appear in Table \ref{appendix:TopN_sentence_acc_comparison} and \ref{appendix:TopN_image_acc_comparison}. After visualizing sentence-level prediction accuracy across the cortex, we find that LaVCa exceeds BrainSCUBA’s performance throughout the visual cortex (Figure \ref{results:accuracy_mapping}b). See Figure \ref{appendix:image_cc_flatmap} for the results of all subjects.

Finally, we examine whether LaVCa can generate concise and interpretable voxel captions without losing critical information in each voxel’s optimal image set. We compare two approaches from the perspective of interpretability by varying the number of optimal images used by LaVCa (Top-$N$) and by simply concatenating the captions of the optimal images (Concat-$N$). Figure~\ref{results:xaxis_num_words} plots prediction accuracy against the average caption length on the horizontal axis, highlighting the trade-off between accuracy and interpretability. Concat-$N$ achieves better accuracy as $N$ increases (up to $N=10$) but at the cost of a much longer caption, which can reduce interpretability. In contrast, LaVCa merges information across the optimal image set into a concise summary, retaining interpretability even as $N$ grows and reaching accuracy comparable to Concat-$N$. Results for all participants are provided in Figure~\ref{appendix:xaxis_words_all_sub}.


\begin{figure*}[!t]
    \centering
    \includegraphics[width=\textwidth]{figures/OFA_caption_vis.pdf}
    \vskip 0.1in
    \caption{Interpretation of LaVCa captions in OFA. (a) The UMAP projection of caption text for all participants, visualized on a flatmap (top). A word cloud of the 100 most frequent words in these captions (middle), colored according to their location in the UMAP space. A bar graph of the top 10 most frequent words (bottom).
(b) Visualization of the top two captions (by accuracy) for eight clusters on the flatmap (subj02). The images generated for each caption appear to the left or above the text. Voxels are connected to their corresponding captions and images by lines. The color of each caption and image border reflects the average UMAP color of all voxels in the cluster.}
    \label{results:OFA_caption_vis}
\end{figure*}

\subsection{Lexical and Semantic Diversity Analysis}
We next assess how effectively LaVCa captions capture both lexical and semantic diversity across voxels, focusing first on \textit{inter-voxel} diversity (Table \ref{table:diversity_analysis}, left). For this quantitative evaluation, we use three metrics: (1) the total vocabulary size (excluding stop-words) across all voxel captions (Lexical); (2) the average variance across each dimension of the CLIP-Text embedding computed on all voxel captions (Semantic); and (3) the number of principal components (PCs) required to capture 90\% of the variance of CLIP-Text embedding across captions in a principal component analysis (PCA; Semantic). 

First, we evaluate the diversity of LaVCa captions compared with the existing method, BrainSCUBA. When averaged across subjects, LaVCa markedly outperforms BrainSCUBA in both lexical (16,922 vs. 3,193 in vocab. size) and semantic (0.0642 vs. 0.0588 in variance of embeddings; 219 vs. 127 in PCs required for 90\% variance explained) diversity. These findings confirm that our open-ended LLM–based approach can produce richer word usage and more meaningful captions across inter-voxel comparisons. 

We evaluate the diversity of LaVCa captions compared with more detailed captions. BrainSCUBA leverages ClipCap \cite{mokady2021clipcap}, a model that produces relatively simple image captions. We use the top-1 captions generated by the MLLM on the optimal image sets (equivalent to the case where $N=1$ in Concat-$N$) to compare the diversity of LaVCa with more detailed captions. When averaged across subjects, Top-$1$ (13,959 vocab. size, 0.0638 avg. variance, 210 PCs) exhibits both a vocabulary range and semantic diversity close to LaVCa. However, LaVCa achieves a higher prediction accuracy (0.264 vs. 0.224), indicating that LaVCa can preserve robust brain activity prediction performance while enhancing the diversity of generated captions.

Next, we evaluate diversity from an \textit{intra-voxel} perspective by comparing captions generated by three models in both lexical and semantic dimensions (Table \ref{table:diversity_analysis}, right). We use three metrics: (1) the vocabulary size of each voxel’s caption (Lexical), (2) the average sentence length in each voxel’s caption (Lexical), and (3) the average variance across all dimensions of Word2Vec embeddings of each caption’s words (excluding stop-words) (Semantic). When averaged across subjects, LaVCa markedly outperforms BrainSCUBA in both lexical (11.4 vs. 6.09 in vocab. size) and semantic (11.9 vs. 6.19 in avg. length; 0.0199 vs. 0.0160 in variance of semantic embeddings) diversity. This improvement suggests that LaVCa more precisely captures the fine-grained intra-voxel characteristics. 

For examples of voxel captions and images from various OFA and PPA voxels—along with their corresponding quantitative metrics—compared across three models (LaVCa, BrainSCUBA, Top-1), see Figures \ref{appendix:OFA_example_1}, \ref{appendix:OFA_example_2}, \ref{appendix:PPA_example_1}, and \ref{appendix:PPA_example_2}.


\begin{table*}[t]
\centering
\caption{Average prediction accuracy with standard error across subjects when captions within each ROI are shuffled (Shuffled) versus used as is (Original).}
\vskip 0.15in
\small
{\scriptsize
\begin{tabular}{c*{8}{c}}
\toprule
& \multicolumn{2}{c}{\textbf{Body areas}} & \multicolumn{2}{c}{\textbf{Face areas}} & \multicolumn{2}{c}{\textbf{Place areas}} & \multicolumn{2}{c}{\textbf{Word areas}} \\
\cmidrule(lr){2-3}\cmidrule(lr){4-5}\cmidrule(lr){6-7}\cmidrule(lr){8-9}
& EBA & FBA-2 & OFA & FFA-1 & OPA & PPA & OWFA & VWFA-1 \\
\midrule
Shuffled & 0.018±0.008 & 0.018±0.005 & 0.028±0.004 & 0.016±0.003 & 0.116±0.024 & 0.151±0.028 & 0.025±0.005 & 0.034±0.009 \\
\rowcolor{lavca_bg}
\textbf{Original} & \textbf{0.157±0.005} & \textbf{0.125±0.010} & \textbf{0.095±0.009} & \textbf{0.111±0.003} & \textbf{0.200±0.022} & \textbf{0.213±0.027} & \textbf{0.084±0.013} & \textbf{0.158±0.007} \\
\bottomrule
\end{tabular}
}
\label{results:ROI_caption_shuffle}
% \vskip -0.1in
\end{table*}


\subsection{ROI-level Diversity Analysis}
Our results thus far illustrate that LaVCa generates more accurate voxel captions than BrainSCUBA and better reflects inter- and intra-voxel diversity. We now assess how LaVCa captions capture diversity within known ROI, such as the OFA  and PPA. These areas are conventionally associated with specific concepts (e.g., faces or places), \cite{gauthier2000fusiform, haxby2000distributed, epstein1998cortical} yet the extent of diversity within these ROIs remains unclear. We conduct a quantitative evaluation using LaVCa’s captions and generated images to analyze diversity that exists beyond the known selectivity in the ROI (Figures \ref{results:OFA_caption_vis} and \ref{results:PPA_caption_vis}).

\paragraph{Quantitative Assessment.}
We determine how many distinct captions appear in each ROI by comparing the sentence-level prediction accuracy of each ROI when captions are maintained in their original form versus shuffled within the ROI (Figure \ref{results:ROI_caption_shuffle}). For each category (body, face, place, and word area), we select two ROIs with the largest total voxel count across all subjects, resulting in eight ROIs in total. In all ROIs, shuffling reduces prediction accuracy significantly. For example, in the OFA, accuracy drops from 0.0945 (Original) to 0.0280 (Shuffled), a 3.3-fold decrease; in the PPA, accuracy falls from 0.213 (Original) to 0.151 (Shuffled), a 1.4-fold decrease. Thus, even in regions traditionally linked to particular concepts, voxels exhibit a range of distinct selectivities.

\paragraph{Qualitative Assessment.}
We next explore the semantic diversity of LaVCa captions in the OFA by applying UMAP to their CLIP-Text embeddings and visualizing the resulting distributions on a flatmap (Figure~\ref{results:OFA_caption_vis}a, top). Each subject’s OFA illustrates a broad spectrum of UMAP colors, indicating multiple meaningful clusters within an ROI known for face-selective responses. See Figure \ref{results:PPA_caption_vis} for the results of the PPA.

To examine the word-level patterns within ROIs, we construct a word cloud from the top 100 most frequent words and a histogram of the top 10 words (Figure \ref{results:OFA_caption_vis}a, middle and bottom). 
We align the word cloud with the flatmap colors by obtaining the CLIP-Text embeddings for each word using the prompt ``A photo of {word}.'' We then apply the UMAP learned from the CLIP-Text embeddings of the captions to convert the top 100 words into the same UMAP space. While high-frequency words like ``child,'' ``people,'' and ``animal'' align with face or person-related content, the presence of terms such as ``food'' or ``sign'' suggests more diverse encoding.

Finally, to highlight how each caption and its corresponding voxel image relate to specific colors in semantic space, we project them onto a flatmap (Figure \ref{results:OFA_caption_vis}b). We divide the samples into eight clusters by labeling each of the three UMAP dimensions as ``High'' ($\geq2/3$) or ``Low'' ($\leq1/3$). From each cluster, we pick the two voxels with the highest prediction accuracy (or one if only one qualifies, or none if none qualify) and illustrate their captions and generated images.

In OFA, some captions are related to faces (e.g., ``face,'' ``person,'' ``animal''), while particular voxels encoded more fine-grained features such as ``eye,'' ``tongue,'' or ``smiling,'' and other voxels encoded information like ``animal,'' ``bear,'' or ``cardinal.'' Thus, even within this ROI, there appears to be substantial functional differentiation among inter-voxel that extends beyond a generic ``face'' category.
Moreover, we observe \textit{intra-voxel} diversity, where a single caption incorporates multiple ideas (e.g., \textit{``A food packaging features a smiling person and a cartoon character''}), suggesting that individual voxels can simultaneously encode several distinct concepts. 
These findings highlight the fine-grained functional specialization across inter-voxel within the ROI and the diverse nature of intra-voxel encoding beyond singular concepts. 
The results for all participants, visualizing the top two captions for each cluster directly in the UMAP space, can be found in Figures \ref{appendix:OFA_caps_umap_subj01-02}, \ref{appendix:OFA_caps_umap_subj05-07}, \ref{appendix:PPA_caps_umap_subj01-02}, and \ref{appendix:PPA_caps_umap_subj05-07}. 










\section{Conclusion}
In this paper, we introduced \textsc{mo-cbo} as a new problem class in order to optimize multiple target variables within a known causal graph by sequentially performing interventions on the system. We proved that a \textsc{mo-cbo} problem can be decomposed into a collection of $|\mathbb{P}(\mathbf{X})| = 2^{|\mathbf{X}|}$ local problems, and solving it essentially corresponds to solving these local problems. To reduce the search space, we derived theoretical results that identify possibly Pareto-optimal minimal intervention sets in a given causal graph. We proved that these sets comprise a minimal collection of local problems that are guaranteed to contain the optimal solutions of any \textsc{mo-cbo} problem. Moreover, we introduced \textsc{Causal ParetoSelect} as an algorithm that iteratively selects and solves local \textsc{mo-cbo} problems in the reduced search space based on relative hypervolume improvement.

Our theoretical and empirical findings highlight two distinct cases: When no unobserved confounders exist between target variables and their ancestors, both \textsc{mo-cbo} and \textsc{mobo} can recover the ground-truth causal Pareto front. However, our approach demonstrates greater cost efficiency while constructing a more diverse set of solutions. In contrast, when unobserved confounders are present between targets and their ancestors, traditional \textsc{mobo} approaches can fail to approximate the ground truth, whereas \textsc{mo-cbo} demonstrates efficient discovery of Pareto-optimal solutions. This occurs because unobserved confounders can propagate effects through the causal graph, and naively disrupting these paths can lead to suboptimal solutions.

In our algorithm, the surrogate model assumes independent outcomes which may limit efficiency since it overlooks shared endogenous confounders. Future work could enhance cost effectiveness by integrating multi-task Gaussian processes to better capture shared information across treatment variables. Other directions for future research include the adaptation of existing \textsc{cbo} variants to the multi-objective case. For instance, combining dynamic \textsc{cbo} \citep{NEURIPS2021_577bcc91} with \textsc{mo-cbo} would lead to a \textsc{mo-cbo} variant that can handle time-dynamic causal models. As the field of causal decision-making continues to grow, we anticipate more progress in the development of multi-objective frameworks to address complex, real-world challenges.
\section{Limitations and Future Work}
The proposed OpenFly platform incorporates various rendering engines/techniques to provide high-quality scenes. Specifically, this is the first attempt to use 3D GS reconstructed scenes to support real-to-sim training and testing, while in the reconstruction of large-scale areas, a few visual artifacts are inevitably present. Future work will focus on exploring more effective reconstruction methods to enhance realism in large-scale scenes. Besides, the proposed OpenFly-Agent is built upon the large VLN model architecture, which is not practical for real-time deployment on UAVs. To address this, future research should focus on developing more efficient architectures and effective quantization techniques. 


\section{Conclusion}
In this work, we present OpenFly, a platform designed for large-scale data collection in aerial Vision-and-Language Navigation (VLN). OpenFly integrates multiple rendering engines and advanced real-to-sim techniques for data generation, enabling efficient collection of diverse, high-quality aerial VLN data. The resulting large-scale dataset comprises 100k trajectories across 18 distinct scenes, spanning a wide range of altitudes and difficulty levels, which is significantly superior than existing ones. Furthermore, we propose OpenFly-Agent, a keyframe-aware aerial navigation model capable of directly predicting flight actions based on observations and language instructions. Extensive experiments validate the effectiveness of the proposed method, and establishing a comprehensive benchmark for future advancements in aerial navigation. 
%The toolchain, dataset, and code will be publicly released, providing a valuable resource for future research in this field.



%% The acknowledgments section is defined using the "acks" environment
%% (and NOT an unnumbered section). This ensures the proper
%% identification of the section in the article metadata, and the
%% consistent spelling of the heading.
\begin{acks}
We thank Stephan Leenders, Oscar Ariza, Axel Hildebrand and Sarah Gökeler for their support and contributions. We also sincerely thank the participants of our studies for their time and valuable input.
\end{acks}


% ===================================================
%% The next two lines define the bibliography style to be used, and
%% the bibliography file.
\bibliographystyle{ACM-Reference-Format}
\bibliography{bibliography}


\clearpage


%% If your work has an appendix, this is the place to put it.
\appendix
\section{Appendix}
\subsection{Data - Survey on current passenger behaviour regarding POIs}

\begin{center}
    \begin{minipage}{\textwidth}

        \centering
        \captionof{table}{Usage of various navigation methods by our survey participants.}
        \begin{tabular}{r|cc|cc|cc|cc|cc}
            \toprule
            & \multicolumn{2}{c|}{AppleCar/AndroidAuto} & \multicolumn{2}{c|}{InVehicleSystems} & \multicolumn{2}{c|}{SmartphoneApps} & \multicolumn{2}{c|}{Compass} & \multicolumn{2}{c}{PaperMaps} \\
            & N & \% & N & \% & N & \% & N & \% & N & \% \\
            \midrule
            Never & 39 & 36.4\% & 13 & 12.1\% & 2 & 1.9\% & 101 & 94.4\% & 79 & 73.8\% \\
            Rarely & 19 & 17.8\% & 19 & 17.8\% & 15 & 14.0\% & 2 & 1.9\% & 26 & 24.3\% \\
            Sometimes & 18 & 16.8\% & 22 & 20.6\% & 33 & 30.8\% & 2 & 1.9\% & 2 & 1.9\% \\
            Often & 22 & 20.6\% & 34 & 31.8\% & 32 & 29.9\% & 0 & 0\% & 0 & 0\% \\
            Always & 9 & 8.4\% & 19 & 17.8\% & 25 & 23.4\% & 2 & 1.9\% & 0 & 0\% \\
            \bottomrule
        \end{tabular}

        \vspace{\baselineskip}
        
        \captionof{table}{The types of information drivers and passengers need to find a missed point of interest. Multiple choice was possible.}
        \begin{tabular}{r|cc|cc}
            \toprule
            \textbf{Type of information} & \multicolumn{2}{c|}{Drivers} & \multicolumn{2}{c}{Passengers} \\
            & N & \% & N & \% \\
            \midrule
            Name & 71 & 89\% & 65 & 90\% \\
            Perspective Picture & 39 & 49\% & 38 & 53\% \\
            Web Picture & 46 & 58\% & 44 & 61\% \\
            Description Text & 45 & 56\% & 49 & 68\% \\
            Category & 45 & 56\% & 49 & 68\% \\
            \bottomrule
        \end{tabular}

        \vspace{\baselineskip}

        \captionof{table}{Percentages on how and when passengers and drivers look for a missed point of interest.}
        \begin{tabular}{r|c|c}
            \toprule
            \textbf{Action}             & \textbf{Drivers}   & \textbf{Passengers} \\
            \midrule
            Nothing                     & 5\%                 & 4\% \\
            System saves Automatically  & 0\%                 & 3\% \\
            Stop the Car                & 3\%                 & 0\% \\
            Search later on Smartphone  & 46\%                & 8\% \\
            Search immediately on Smartphone & 11\%           & 49\% \\
            Search later on Navigation system & 24\%          & 5\% \\
            Search immediately on Navigation system & 11\%    & 31\% \\
            \bottomrule
        \end{tabular}

    \end{minipage}
\end{center}



\clearpage
\subsection{Data - Pre-Study on Eye-Gaze Interaction}

\begin{center}
    \begin{minipage}{\textwidth}
      
    \centering
    \captionof{table}{Descriptive Statistics for the pre-study Raw NASA Task Load Index.}
    \begin{tabular}{l|c|c|c|c|c|c|c}
        \toprule
        & \textbf{Mental} & \textbf{Physical} & \textbf{Temporal} & \textbf{Performance} & \textbf{Effort} & \textbf{Frustration} & \textbf{Score} \\
        \midrule
        Mean & 15.0 & 19.5 & 40.0 & 33.5 & 28.0 & 13.5 & 24.8 \\
        Median & 12.5 & 15.0 & 42.5 & 25.0 & 32.5 & 15.0 & 25.0 \\
        Standard Deviation & 10.5 & 19.9 & 23.1 & 21.4 & 18.1 & 11.6 & 9.47 \\
        Shapiro-Wilk W & 0.942 & 0.782 & 0.961 & 0.848 & 0.898 & 0.882 & 0.959 \\
        Shapiro-Wilk p & 0.573 & 0.009 & 0.794 & 0.055 & 0.206 & 0.139 & 0.769 \\
        \bottomrule
    \end{tabular}  

    \vspace{\baselineskip}

    \captionof{table}{Descriptive Statistics for the pre-study System Usability Scale.}
    \begin{tabular}{l|c|c|c|c|c}
        \toprule
        \textbf{Question} & \textbf{Mean} & \textbf{Median} & \textbf{Std. Deviation} & \textbf{Shapiro-Wilk W} & \textbf{Shapiro-Wilk p} \\
        \midrule
        Frequent Use & 4.00 & 4.00 & 0.943 & 0.841 & 0.045 \\
        Unnecessary Complex & 1.40 & 1.00 & 0.516 & 0.640 & < .001 \\
        Easy to Use & 4.60 & 5.00 & 0.516 & 0.640 & < .001 \\
        Support of Technical & 1.50 & 1.00 & 0.972 & 0.603 & < .001 \\
        Well Integrated & 4.40 & 4.00 & 0.516 & 0.640 & < .001 \\
        Inconsistency & 1.80 & 2.00 & 0.789 & 0.820 & 0.025 \\
        Learn Quickly & 4.60 & 5.00 & 0.699 & 0.650 & < .001 \\
        Cumbersome & 1.60 & 1.00 & 0.966 & 0.678 & < .001 \\
        Confident & 4.20 & 4.00 & 0.632 & 0.794 & 0.012 \\
        Learn a Lot Before & 1.10 & 1.00 & 0.316 & 0.366 & < .001 \\
        Score & 86.0 & 87.5 & 8.01 & 0.925 & 0.398 \\
        \bottomrule
    \end{tabular}

    \end{minipage}
\end{center}



\clearpage
\subsection{Data - Study on Visualizing Passed and Upcoming POIs}

\begin{center}
    \begin{minipage}{\textwidth}

        \centering
        \captionof{table}{Descriptive Statistics for the visualization-study Raw NASA Task Load Index, grouped by condition.}
        \begin{tabular}{llccccccc}
            \toprule
            \textbf{Statistic} & \textbf{Condition} & \textbf{Mental} & \textbf{Physical} & \textbf{Temporal} & \textbf{Performance} & \textbf{Effort} & \textbf{Frustration} & \textbf{Score} \\
            \midrule
            \multirow{3}{*}{Mean} & List & 28.6 & 25.5 & 20.7 & 14.8 & 29.5 & 19.5 & 23.1 \\
            & Timeline & 29.8 & 27.1 & 23.3 & 20.0 & 27.4 & 28.6 & 26.0 \\
            & Minimap & 46.7 & 36.4 & 42.4 & 42.6 & 56.9 & 50.2 & 45.9 \\
            \midrule
            \multirow{3}{*}{Median} & List & 25 & 20 & 15 & 5 & 25 & 15 & 25.0 \\
            & Timeline & 25 & 20 & 15 & 5 & 20 & 20 & 22.5 \\
            & Minimap & 40 & 30 & 45 & 40 & 60 & 55 & 49.2 \\
            \midrule
            \multirow{3}{1.5cm}{Std. Deviation} & List & 19.8 & 21.3 & 17.0 & 19.5 & 25.3 & 18.5 & 14.6 \\
            & Timeline & 22.8 & 21.3 & 19.6 & 29.0 & 26.6 & 26.0 & 19.3 \\
            & Minimap & 21.5 & 24.7 & 22.3 & 22.8 & 23.5 & 22.6 & 15.8 \\
            \midrule
            \multirow{3}{1.5cm}{Shapiro-Wilk W} & List & 0.917 & 0.879 & 0.909 & 0.767 & 0.835 & 0.851 & 0.941 \\
            & Timeline & 0.868 & 0.904 & 0.911 & 0.683 & 0.821 & 0.874 & 0.927 \\
            & Minimap & 0.947 & 0.886 & 0.920 & 0.938 & 0.954 & 0.946 & 0.927 \\
            \midrule
            \multirow{3}{1.5cm}{Shapiro-Wilk p} & List & 0.076 & 0.014 & 0.052 & < .001 & 0.002 & 0.004 & 0.229 \\
            & Timeline & 0.009 & 0.043 & 0.058 & < .001 & 0.001 & 0.011 & 0.118 \\
            & Minimap & 0.297 & 0.019 & 0.088 & 0.201 & 0.399 & 0.290 & 0.118 \\
            \bottomrule
        \end{tabular}


        \vspace{\baselineskip}

    
        \captionof{table}{Descriptive Statistics for the visualization-study System Usability Scale scores, grouped by condition.}
        \begin{tabular}{lccccc}
            \toprule
            \textbf{Condition} & \textbf{Mean} & \textbf{Median} & \textbf{Std. Deviation} & \textbf{Shapiro-Wilk W} & \textbf{Shapiro-Wilk p} \\
            \midrule
            List & 78.5 & 77.5 & 10.2 & 0.885 & 0.018 \\
            Minimap & 61.1 & 60.0 & 14.3 & 0.962 & 0.563 \\
            Timeline & 71.5 & 75.0 & 15.9 & 0.921 & 0.091 \\
            \bottomrule
        \end{tabular}


        \vspace{\baselineskip}

        
        \caption{Descriptive Statistics for the visualization-study Motion Sickness Questionnaire, grouped by time of completion.}
        \begin{tabular}{lccccc}
            \toprule
            \textbf{Study Order} & \textbf{Mean} & \textbf{Median} & \textbf{Std. Deviation} & \textbf{Shapiro-Wilk W} & \textbf{Shapiro-Wilk p} \\
            \midrule
            Pre-study & 0.381 & 0 & 0.669 & 0.617 & < .001 \\
            After First Condition & 0.857 & 0 & 1.15 & 0.732   & < .001 \\
            After Second Condition & 1.19 & 1 & 1.36 & 0.822   & 0.001  \\
            After Third Condition & 1.00 & 1 & 1.05 & 0.826    & 0.002  \\
            \bottomrule
        \end{tabular}


    \end{minipage}
\end{center}


\begin{table*}[ht]
    \centering
    \caption{Descriptive Statistics for the visualization-study User Experience Questionnaire, grouped by condition.}
    \begin{tabular}{llcccccc}
    \toprule
    \textbf{} & \textbf{Condition} & \textbf{Attractiveness} & \textbf{Perspicuity} & \textbf{Efficiency} & \textbf{Dependability} & \textbf{Stimulation} & \textbf{Novelty} \\
    \midrule
    \multirow{3}{*}{\textbf{M}} & Timeline & 1.26 & 1.42 & 1.10 & 1.12 & 1.11 & 1.37 \\
    & Minimap & 0.952 & 0.964 & 0.619 & 0.821 & 1.00 & 1.57 \\
    & List & 1.78 & 2.05 & 1.79 & 1.68 & 1.54 & 1.31 \\
    \midrule
    \multirow{3}{*}{\textbf{Mdn}} & Timeline & 1.50 & 1.75 & 1.25 & 1.25 & 1.25 & 1.50 \\
    & Minimap & 1.00 & 0.750 & 0.500 & 0.500 & 1.00 & 1.25 \\
    & List & 1.83 & 2.25 & 1.50 & 1.50 & 1.50 & 1.25 \\
    \midrule
    \multirow{3}{*}{\textbf{SD}} & Timeline & 0.921 & 1.03 & 0.937 & 0.883 & 0.986 & 1.11 \\
    & Minimap & 0.972 & 1.01 & 0.993 & 0.946 & 0.939 & 1.13 \\
    & List & 0.642 & 0.692 & 0.755 & 0.717 & 0.704 & 0.798 \\
    \midrule
    \multirow{3}{*}{\textbf{W}} & Timeline & 0.938 & 0.833 & 0.951 & 0.912 & 0.970 & 0.933 \\
    & Minimap & 0.972 & 0.975 & 0.980 & 0.948 & 0.967 & 0.877 \\
    & List & 0.976 & 0.893 & 0.847 & 0.926 & 0.942 & 0.982 \\
    \midrule
    \multirow{3}{*}{\textbf{p}} & Timeline & 0.200 & 0.002 & 0.359 & 0.059 & 0.737 & 0.160 \\
    & Minimap & 0.783 & 0.844 & 0.918 & 0.313 & 0.661 & 0.013 \\
    & List & 0.866 & 0.026 & 0.004 & 0.112 & 0.235 & 0.953 \\
    \bottomrule
    \end{tabular}
\end{table*}

\begin{table*}[ht]
    \centering
    \caption{User ranking values for the visualization-study conditions.}
    \begin{tabular}{l|cc|cc|cc}
    \toprule
    \multirow{2}{*}{\textbf{Condition}}& \multicolumn{2}{c|}{\textbf{Least Favorite}} & \multicolumn{2}{c|}{\textbf{Middle}} & \multicolumn{2}{c}{\textbf{Favorite}} \\
                & N     & \%             & N    & \%                  & N       & \%                 \\
    \midrule
    List        & 2     & 9.5\%          & 4    & 19.0\%              & 15      & 71.4\%              \\
    Minimap     & 11    & 52.4\%         & 7    & 33.3\%              & 3       & 14.3\%             \\
    Timeline    & 8     & 38.1\%         & 10   & 47.6\%              & 3       & 14.3\%             \\
    \bottomrule
    \end{tabular}
\end{table*}



\end{document}
\endinput


%% End of file `sample-manuscript.tex'.
