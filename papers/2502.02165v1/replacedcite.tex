\section{Related Work}
\textbf{Tree Packing.} The problem of tree packing has been extensively studied, as summarized in the survey by Palmer ____. Foundational results in this area include those by Tutte ____ and Nash-Williams ____, who demonstrated that an undirected graph with edge connectivity $\lambda$ contains a tree packing of size $\lfloor \frac{\lambda}{2} \rfloor$. Edmond ____ extended this result to directed graphs, showing that such graphs always contain $\lambda$ pairwise edge-disjoint spanning trees rooted at a sender $s \in V$, where $\lambda$ is the minimum number of edges that must be removed to make some node unreachable from $s$. However, these results do not address the height of the tree packing.

    Chuzhoy et al. ____ tackled the challenge of finding tree packings with small height. They presented a randomized algorithm that, given an undirected graph with edge connectivity $\lambda$ and diameter $D$, outputs with high probability a tree packing of size $\lfloor \frac{\lambda}{2} \rfloor$, weight $2$, and height $O((101k \log n)^D)$.

    Tree packing on random graphs was studied by Gao et al. ____, who showed that asymptotically almost surely, the size of a spanning tree packing of weight one for a {\graph} is $\min\left\{\delta(G), \frac{|E(G)|}{n - 1}\right\}$, which corresponds to two straightforward upper bounds.

    In the CONGEST model, tree packing was investigated by Censor-Hillel et al. ____. They proposed an algorithm to decompose an undirected graph with edge connectivity $\lambda$ into fractionally edge-disjoint weighted spanning trees with total weight $\lceil \frac{\lambda - 1}{2}\rceil$ in $\Tilde{O}(D + \sqrt{n\lambda})$ rounds. Furthermore, they proved a lower bound of $\Tilde{\Omega}(D + \sqrt{\frac{n}{\lambda}})$ on the number of rounds required for such a decomposition.

    
    \textbf{Network Information Flow.} The network information flow problem ____ is defined as follows. The network is a directed graph $G(V, E)$, with edge capacities $c: E \rightarrow \mathbb{R}_{\geq 0}$, a source node $s \in V$, and sink nodes $T \subseteq V$. The question is: at what maximal rate can the source send information so that all of the sinks receive that information at the same rate? In the case of a single sink $t$, the answer is given by the $\text{max-flow}(s, t)$. However, when there are multiple sinks $T$, the value $\min\limits_{t \in T} \text{max-flow}(s, t)$ may not be achievable if nodes are only allowed to relay information. In fact, the gap can be as large as a factor of $\Omega(\log n)$ ____. Nevertheless, if intermediate nodes are allowed to send (linear ____) codes of the information they receive, then $\min\limits_{t \in T} \text{max-flow}(s, t)$ becomes achievable ____. Notably, in the specific case where $T = V \setminus \{s\}$ (the setting considered in the present paper), the rate of $\min\limits_{t \in T} \text{max-flow}(s, t)$ becomes achievable without coding ____. The decentralized version of network information flow was studied in ____. The most relevant work in this direction is ``An asymptotically optimal push–pull method for multicasting over a random network`` ____ by Swamy et al., where authors establish an optimal algorithm for the case of random graphs whose radius is almost surely bounded by $3$. Our approach allows an expected radius to grow infinitely with $n$ ____.

    The key difference between the network information flow problem and the multi-message broadcast in CONGEST is that in our problem, the focus is on round complexity, whereas in the information flow problem, the solution is a "static" assignment of messages to edges.



    \textbf{Branching Random Walks in Networks.}
    The cover time of a random walk ____ on a graph is the time needed for a walk to visit each node at least once. Unfortunately, the expected value of this quantity is $\Omega(n\log n)$ even for a clique, making this primitive less useful in practical applications. Consequently, several attempts have been made to accelerate the cover time. Alon et al. ____ proposed initiating multiple random walks from a single source. Subsequent work by Els\"asser and Sauerwald refined their bounds, demonstrating that $r$ random walks can yield a speed-up of $r$ times for many graph classes. Variations of multiple random walks have been applied in the CONGEST model to approximate the mixing time ____, perform leader election ____, and evaluate network conductance ____.

    A branching random walk ____ (BRW) modifies the classical random walk by allowing nodes to emit multiple copies of a walk upon receipt, rather than simply relaying it. This branching behavior potentially leads to exponential growth in the number of walks traversing the graph, significantly reducing the cover time. Roche ____, in his Ph.D. thesis ``Robust Local Algorithms for Communication and Stability in Distributed Networks`` [2017], utilized BRWs to maintain the expander topology of a network despite adversarial node deletions and insertions. Gerraoui et al. ____, in ``On the Inherent Anonymity of Gossiping``, demonstrated that BRWs can enhance privacy by obscuring the source of gossip within a network. Recently, Aradhya et al. ____ in ``Distributed Branching Random Walks and Their Applications`` employed BRWs to address permutation routing problems on subnetworks in the CONGEST model.

    Despite these applications, to the best of our knowledge, the branching random walk remains underexplored in distributed computing and this work seeks to showcase its untapped potential.