\section{Related Work}
\subsection{Challenges in Usability Testing}
Usability testing is a core component of UX research, used to evaluate how easily users can interact with a product to achieve their goals \cite{shawHandbookUsabilityTesting1996a,barnum2020usability, bastienUsabilityTestingReview2010, lewis2012usability}.
It involves observing real users as they navigate through tasks and providing valuable feedback on design effectiveness \cite{barnum2020usability}. 
This method helps refine products by identifying usability issues, measuring user satisfaction, and improving overall user experience. The key benefits of usability testing include validating design choices, avoiding internal biases, and ensuring the product meets user expectations.

However, UX researchers conducting usability testing on web designs have been facing multiple challenges in the experiment design stage and the participant recruitment stage~\cite{folstadAnalysisPracticalUsability2012, hertzumEvaluatorEffectChilling2003, kuangMergingResultsNo2022, norgaardWhatUsabilityEvaluators2006}.
To mitigate these challenges, researchers have begun to explore using agents to simulate usability testing \cite{ren2014agent}.
Prior works in agent-based tools mainly focused on pre-defined environments, like GUI \cite{eskonenAutomatingGUITesting2020} and games \cite{stahlkeArtificialPlayfulnessTool2019,fernandesAgentsAutomatedUser2021}.
Nevertheless, the potential for employing agents in web design usability testing remains underexplored, largely due to the constraints of traditional web automation technologies.

\subsection{LLM Agent and LLM Web Agent}

In the field of HCI, recent studies have demonstrated that LLMs have the capability to simulate human-like behaviors, making research on personalized agent behavior feasible. Unlike task-oriented agents, which are typically focused on completing specific tasks, these human agents or role-play agents \cite{chen2025towards} are designed with diverse personas that reflect not only their roles and expertise but also their preferences and habits. For example, \citet{parkGenerativeAgentsInteractive2023} developed a simulated town with 25 LLM Agents, each with a unique persona, and exhibited believable human behaviors through their interactions. 
LLM Agents are also used to study users' privacy concerns \cite{zhangPrivacyLeakageOvershadowed2025,chenEmpathyBasedSandboxApproach2024}. For example, \citet{chenEmpathyBasedSandboxApproach2024} introduced a privacy sandbox where users can modify an agent’s persona, allowing it to generate search histories and profiles based on personal traits.
\citet{taebAXNavReplayingAccessibility2024} proposed AXNav that leverages LLM to convert manual accessibility test instructions to replayable and navigatable videos.
These innovations guide our work, as we aim to simulate user behavior in web environments by incorporating diverse agent personas for usability testing.

LLM Agents in web environments (LLM Web Agents) like WebGPT \cite{nakanoWebGPTBrowserassistedQuestionanswering2022} and LASER \cite{maLASERLLMAgent2024} have also demonstrated that LLM Agents can perform more complex tasks within web environments, such as searching, browsing and shopping \cite{yaoWebShopScalableRealWorld2022}.
With these advancements, LLM Web Agents present a promising solution for simulating human subjects in usability testing while interacting with web environments.
To this end, we propose \projectname, an LLM-Agent-based system for automating usability testing in web environments.