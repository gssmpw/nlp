%%%%%%%% ICML 2025 EXAMPLE LATEX SUBMISSION FILE %%%%%%%%%%%%%%%%%

\documentclass{article}

% Recommended, but optional, packages for figures and better typesetting:
\usepackage{microtype}
\usepackage{graphicx}
\usepackage{subfigure}
\usepackage{booktabs} % for professional tables

%%%%% NEW MATH DEFINITIONS %%%%%

\usepackage{amsmath,amsfonts,bm}
\usepackage{derivative}
% Mark sections of captions for referring to divisions of figures
\newcommand{\figleft}{{\em (Left)}}
\newcommand{\figcenter}{{\em (Center)}}
\newcommand{\figright}{{\em (Right)}}
\newcommand{\figtop}{{\em (Top)}}
\newcommand{\figbottom}{{\em (Bottom)}}
\newcommand{\captiona}{{\em (a)}}
\newcommand{\captionb}{{\em (b)}}
\newcommand{\captionc}{{\em (c)}}
\newcommand{\captiond}{{\em (d)}}

% Highlight a newly defined term
\newcommand{\newterm}[1]{{\bf #1}}

% Derivative d 
\newcommand{\deriv}{{\mathrm{d}}}

% Figure reference, lower-case.
\def\figref#1{figure~\ref{#1}}
% Figure reference, capital. For start of sentence
\def\Figref#1{Figure~\ref{#1}}
\def\twofigref#1#2{figures \ref{#1} and \ref{#2}}
\def\quadfigref#1#2#3#4{figures \ref{#1}, \ref{#2}, \ref{#3} and \ref{#4}}
% Section reference, lower-case.
\def\secref#1{section~\ref{#1}}
% Section reference, capital.
\def\Secref#1{Section~\ref{#1}}
% Reference to two sections.
\def\twosecrefs#1#2{sections \ref{#1} and \ref{#2}}
% Reference to three sections.
\def\secrefs#1#2#3{sections \ref{#1}, \ref{#2} and \ref{#3}}
% Reference to an equation, lower-case.
\def\eqref#1{equation~\ref{#1}}
% Reference to an equation, upper case
\def\Eqref#1{Equation~\ref{#1}}
% A raw reference to an equation---avoid using if possible
\def\plaineqref#1{\ref{#1}}
% Reference to a chapter, lower-case.
\def\chapref#1{chapter~\ref{#1}}
% Reference to an equation, upper case.
\def\Chapref#1{Chapter~\ref{#1}}
% Reference to a range of chapters
\def\rangechapref#1#2{chapters\ref{#1}--\ref{#2}}
% Reference to an algorithm, lower-case.
\def\algref#1{algorithm~\ref{#1}}
% Reference to an algorithm, upper case.
\def\Algref#1{Algorithm~\ref{#1}}
\def\twoalgref#1#2{algorithms \ref{#1} and \ref{#2}}
\def\Twoalgref#1#2{Algorithms \ref{#1} and \ref{#2}}
% Reference to a part, lower case
\def\partref#1{part~\ref{#1}}
% Reference to a part, upper case
\def\Partref#1{Part~\ref{#1}}
\def\twopartref#1#2{parts \ref{#1} and \ref{#2}}

\def\ceil#1{\lceil #1 \rceil}
\def\floor#1{\lfloor #1 \rfloor}
\def\1{\bm{1}}
\newcommand{\train}{\mathcal{D}}
\newcommand{\valid}{\mathcal{D_{\mathrm{valid}}}}
\newcommand{\test}{\mathcal{D_{\mathrm{test}}}}

\def\eps{{\epsilon}}


% Random variables
\def\reta{{\textnormal{$\eta$}}}
\def\ra{{\textnormal{a}}}
\def\rb{{\textnormal{b}}}
\def\rc{{\textnormal{c}}}
\def\rd{{\textnormal{d}}}
\def\re{{\textnormal{e}}}
\def\rf{{\textnormal{f}}}
\def\rg{{\textnormal{g}}}
\def\rh{{\textnormal{h}}}
\def\ri{{\textnormal{i}}}
\def\rj{{\textnormal{j}}}
\def\rk{{\textnormal{k}}}
\def\rl{{\textnormal{l}}}
% rm is already a command, just don't name any random variables m
\def\rn{{\textnormal{n}}}
\def\ro{{\textnormal{o}}}
\def\rp{{\textnormal{p}}}
\def\rq{{\textnormal{q}}}
\def\rr{{\textnormal{r}}}
\def\rs{{\textnormal{s}}}
\def\rt{{\textnormal{t}}}
\def\ru{{\textnormal{u}}}
\def\rv{{\textnormal{v}}}
\def\rw{{\textnormal{w}}}
\def\rx{{\textnormal{x}}}
\def\ry{{\textnormal{y}}}
\def\rz{{\textnormal{z}}}

% Random vectors
\def\rvepsilon{{\mathbf{\epsilon}}}
\def\rvphi{{\mathbf{\phi}}}
\def\rvtheta{{\mathbf{\theta}}}
\def\rva{{\mathbf{a}}}
\def\rvb{{\mathbf{b}}}
\def\rvc{{\mathbf{c}}}
\def\rvd{{\mathbf{d}}}
\def\rve{{\mathbf{e}}}
\def\rvf{{\mathbf{f}}}
\def\rvg{{\mathbf{g}}}
\def\rvh{{\mathbf{h}}}
\def\rvu{{\mathbf{i}}}
\def\rvj{{\mathbf{j}}}
\def\rvk{{\mathbf{k}}}
\def\rvl{{\mathbf{l}}}
\def\rvm{{\mathbf{m}}}
\def\rvn{{\mathbf{n}}}
\def\rvo{{\mathbf{o}}}
\def\rvp{{\mathbf{p}}}
\def\rvq{{\mathbf{q}}}
\def\rvr{{\mathbf{r}}}
\def\rvs{{\mathbf{s}}}
\def\rvt{{\mathbf{t}}}
\def\rvu{{\mathbf{u}}}
\def\rvv{{\mathbf{v}}}
\def\rvw{{\mathbf{w}}}
\def\rvx{{\mathbf{x}}}
\def\rvy{{\mathbf{y}}}
\def\rvz{{\mathbf{z}}}

% Elements of random vectors
\def\erva{{\textnormal{a}}}
\def\ervb{{\textnormal{b}}}
\def\ervc{{\textnormal{c}}}
\def\ervd{{\textnormal{d}}}
\def\erve{{\textnormal{e}}}
\def\ervf{{\textnormal{f}}}
\def\ervg{{\textnormal{g}}}
\def\ervh{{\textnormal{h}}}
\def\ervi{{\textnormal{i}}}
\def\ervj{{\textnormal{j}}}
\def\ervk{{\textnormal{k}}}
\def\ervl{{\textnormal{l}}}
\def\ervm{{\textnormal{m}}}
\def\ervn{{\textnormal{n}}}
\def\ervo{{\textnormal{o}}}
\def\ervp{{\textnormal{p}}}
\def\ervq{{\textnormal{q}}}
\def\ervr{{\textnormal{r}}}
\def\ervs{{\textnormal{s}}}
\def\ervt{{\textnormal{t}}}
\def\ervu{{\textnormal{u}}}
\def\ervv{{\textnormal{v}}}
\def\ervw{{\textnormal{w}}}
\def\ervx{{\textnormal{x}}}
\def\ervy{{\textnormal{y}}}
\def\ervz{{\textnormal{z}}}

% Random matrices
\def\rmA{{\mathbf{A}}}
\def\rmB{{\mathbf{B}}}
\def\rmC{{\mathbf{C}}}
\def\rmD{{\mathbf{D}}}
\def\rmE{{\mathbf{E}}}
\def\rmF{{\mathbf{F}}}
\def\rmG{{\mathbf{G}}}
\def\rmH{{\mathbf{H}}}
\def\rmI{{\mathbf{I}}}
\def\rmJ{{\mathbf{J}}}
\def\rmK{{\mathbf{K}}}
\def\rmL{{\mathbf{L}}}
\def\rmM{{\mathbf{M}}}
\def\rmN{{\mathbf{N}}}
\def\rmO{{\mathbf{O}}}
\def\rmP{{\mathbf{P}}}
\def\rmQ{{\mathbf{Q}}}
\def\rmR{{\mathbf{R}}}
\def\rmS{{\mathbf{S}}}
\def\rmT{{\mathbf{T}}}
\def\rmU{{\mathbf{U}}}
\def\rmV{{\mathbf{V}}}
\def\rmW{{\mathbf{W}}}
\def\rmX{{\mathbf{X}}}
\def\rmY{{\mathbf{Y}}}
\def\rmZ{{\mathbf{Z}}}

% Elements of random matrices
\def\ermA{{\textnormal{A}}}
\def\ermB{{\textnormal{B}}}
\def\ermC{{\textnormal{C}}}
\def\ermD{{\textnormal{D}}}
\def\ermE{{\textnormal{E}}}
\def\ermF{{\textnormal{F}}}
\def\ermG{{\textnormal{G}}}
\def\ermH{{\textnormal{H}}}
\def\ermI{{\textnormal{I}}}
\def\ermJ{{\textnormal{J}}}
\def\ermK{{\textnormal{K}}}
\def\ermL{{\textnormal{L}}}
\def\ermM{{\textnormal{M}}}
\def\ermN{{\textnormal{N}}}
\def\ermO{{\textnormal{O}}}
\def\ermP{{\textnormal{P}}}
\def\ermQ{{\textnormal{Q}}}
\def\ermR{{\textnormal{R}}}
\def\ermS{{\textnormal{S}}}
\def\ermT{{\textnormal{T}}}
\def\ermU{{\textnormal{U}}}
\def\ermV{{\textnormal{V}}}
\def\ermW{{\textnormal{W}}}
\def\ermX{{\textnormal{X}}}
\def\ermY{{\textnormal{Y}}}
\def\ermZ{{\textnormal{Z}}}

% Vectors
\def\vzero{{\bm{0}}}
\def\vone{{\bm{1}}}
\def\vmu{{\bm{\mu}}}
\def\vtheta{{\bm{\theta}}}
\def\vphi{{\bm{\phi}}}
\def\va{{\bm{a}}}
\def\vb{{\bm{b}}}
\def\vc{{\bm{c}}}
\def\vd{{\bm{d}}}
\def\ve{{\bm{e}}}
\def\vf{{\bm{f}}}
\def\vg{{\bm{g}}}
\def\vh{{\bm{h}}}
\def\vi{{\bm{i}}}
\def\vj{{\bm{j}}}
\def\vk{{\bm{k}}}
\def\vl{{\bm{l}}}
\def\vm{{\bm{m}}}
\def\vn{{\bm{n}}}
\def\vo{{\bm{o}}}
\def\vp{{\bm{p}}}
\def\vq{{\bm{q}}}
\def\vr{{\bm{r}}}
\def\vs{{\bm{s}}}
\def\vt{{\bm{t}}}
\def\vu{{\bm{u}}}
\def\vv{{\bm{v}}}
\def\vw{{\bm{w}}}
\def\vx{{\bm{x}}}
\def\vy{{\bm{y}}}
\def\vz{{\bm{z}}}

% Elements of vectors
\def\evalpha{{\alpha}}
\def\evbeta{{\beta}}
\def\evepsilon{{\epsilon}}
\def\evlambda{{\lambda}}
\def\evomega{{\omega}}
\def\evmu{{\mu}}
\def\evpsi{{\psi}}
\def\evsigma{{\sigma}}
\def\evtheta{{\theta}}
\def\eva{{a}}
\def\evb{{b}}
\def\evc{{c}}
\def\evd{{d}}
\def\eve{{e}}
\def\evf{{f}}
\def\evg{{g}}
\def\evh{{h}}
\def\evi{{i}}
\def\evj{{j}}
\def\evk{{k}}
\def\evl{{l}}
\def\evm{{m}}
\def\evn{{n}}
\def\evo{{o}}
\def\evp{{p}}
\def\evq{{q}}
\def\evr{{r}}
\def\evs{{s}}
\def\evt{{t}}
\def\evu{{u}}
\def\evv{{v}}
\def\evw{{w}}
\def\evx{{x}}
\def\evy{{y}}
\def\evz{{z}}

% Matrix
\def\mA{{\bm{A}}}
\def\mB{{\bm{B}}}
\def\mC{{\bm{C}}}
\def\mD{{\bm{D}}}
\def\mE{{\bm{E}}}
\def\mF{{\bm{F}}}
\def\mG{{\bm{G}}}
\def\mH{{\bm{H}}}
\def\mI{{\bm{I}}}
\def\mJ{{\bm{J}}}
\def\mK{{\bm{K}}}
\def\mL{{\bm{L}}}
\def\mM{{\bm{M}}}
\def\mN{{\bm{N}}}
\def\mO{{\bm{O}}}
\def\mP{{\bm{P}}}
\def\mQ{{\bm{Q}}}
\def\mR{{\bm{R}}}
\def\mS{{\bm{S}}}
\def\mT{{\bm{T}}}
\def\mU{{\bm{U}}}
\def\mV{{\bm{V}}}
\def\mW{{\bm{W}}}
\def\mX{{\bm{X}}}
\def\mY{{\bm{Y}}}
\def\mZ{{\bm{Z}}}
\def\mBeta{{\bm{\beta}}}
\def\mPhi{{\bm{\Phi}}}
\def\mLambda{{\bm{\Lambda}}}
\def\mSigma{{\bm{\Sigma}}}

% Tensor
\DeclareMathAlphabet{\mathsfit}{\encodingdefault}{\sfdefault}{m}{sl}
\SetMathAlphabet{\mathsfit}{bold}{\encodingdefault}{\sfdefault}{bx}{n}
\newcommand{\tens}[1]{\bm{\mathsfit{#1}}}
\def\tA{{\tens{A}}}
\def\tB{{\tens{B}}}
\def\tC{{\tens{C}}}
\def\tD{{\tens{D}}}
\def\tE{{\tens{E}}}
\def\tF{{\tens{F}}}
\def\tG{{\tens{G}}}
\def\tH{{\tens{H}}}
\def\tI{{\tens{I}}}
\def\tJ{{\tens{J}}}
\def\tK{{\tens{K}}}
\def\tL{{\tens{L}}}
\def\tM{{\tens{M}}}
\def\tN{{\tens{N}}}
\def\tO{{\tens{O}}}
\def\tP{{\tens{P}}}
\def\tQ{{\tens{Q}}}
\def\tR{{\tens{R}}}
\def\tS{{\tens{S}}}
\def\tT{{\tens{T}}}
\def\tU{{\tens{U}}}
\def\tV{{\tens{V}}}
\def\tW{{\tens{W}}}
\def\tX{{\tens{X}}}
\def\tY{{\tens{Y}}}
\def\tZ{{\tens{Z}}}


% Graph
\def\gA{{\mathcal{A}}}
\def\gB{{\mathcal{B}}}
\def\gC{{\mathcal{C}}}
\def\gD{{\mathcal{D}}}
\def\gE{{\mathcal{E}}}
\def\gF{{\mathcal{F}}}
\def\gG{{\mathcal{G}}}
\def\gH{{\mathcal{H}}}
\def\gI{{\mathcal{I}}}
\def\gJ{{\mathcal{J}}}
\def\gK{{\mathcal{K}}}
\def\gL{{\mathcal{L}}}
\def\gM{{\mathcal{M}}}
\def\gN{{\mathcal{N}}}
\def\gO{{\mathcal{O}}}
\def\gP{{\mathcal{P}}}
\def\gQ{{\mathcal{Q}}}
\def\gR{{\mathcal{R}}}
\def\gS{{\mathcal{S}}}
\def\gT{{\mathcal{T}}}
\def\gU{{\mathcal{U}}}
\def\gV{{\mathcal{V}}}
\def\gW{{\mathcal{W}}}
\def\gX{{\mathcal{X}}}
\def\gY{{\mathcal{Y}}}
\def\gZ{{\mathcal{Z}}}

% Sets
\def\sA{{\mathbb{A}}}
\def\sB{{\mathbb{B}}}
\def\sC{{\mathbb{C}}}
\def\sD{{\mathbb{D}}}
% Don't use a set called E, because this would be the same as our symbol
% for expectation.
\def\sF{{\mathbb{F}}}
\def\sG{{\mathbb{G}}}
\def\sH{{\mathbb{H}}}
\def\sI{{\mathbb{I}}}
\def\sJ{{\mathbb{J}}}
\def\sK{{\mathbb{K}}}
\def\sL{{\mathbb{L}}}
\def\sM{{\mathbb{M}}}
\def\sN{{\mathbb{N}}}
\def\sO{{\mathbb{O}}}
\def\sP{{\mathbb{P}}}
\def\sQ{{\mathbb{Q}}}
\def\sR{{\mathbb{R}}}
\def\sS{{\mathbb{S}}}
\def\sT{{\mathbb{T}}}
\def\sU{{\mathbb{U}}}
\def\sV{{\mathbb{V}}}
\def\sW{{\mathbb{W}}}
\def\sX{{\mathbb{X}}}
\def\sY{{\mathbb{Y}}}
\def\sZ{{\mathbb{Z}}}

% Entries of a matrix
\def\emLambda{{\Lambda}}
\def\emA{{A}}
\def\emB{{B}}
\def\emC{{C}}
\def\emD{{D}}
\def\emE{{E}}
\def\emF{{F}}
\def\emG{{G}}
\def\emH{{H}}
\def\emI{{I}}
\def\emJ{{J}}
\def\emK{{K}}
\def\emL{{L}}
\def\emM{{M}}
\def\emN{{N}}
\def\emO{{O}}
\def\emP{{P}}
\def\emQ{{Q}}
\def\emR{{R}}
\def\emS{{S}}
\def\emT{{T}}
\def\emU{{U}}
\def\emV{{V}}
\def\emW{{W}}
\def\emX{{X}}
\def\emY{{Y}}
\def\emZ{{Z}}
\def\emSigma{{\Sigma}}

% entries of a tensor
% Same font as tensor, without \bm wrapper
\newcommand{\etens}[1]{\mathsfit{#1}}
\def\etLambda{{\etens{\Lambda}}}
\def\etA{{\etens{A}}}
\def\etB{{\etens{B}}}
\def\etC{{\etens{C}}}
\def\etD{{\etens{D}}}
\def\etE{{\etens{E}}}
\def\etF{{\etens{F}}}
\def\etG{{\etens{G}}}
\def\etH{{\etens{H}}}
\def\etI{{\etens{I}}}
\def\etJ{{\etens{J}}}
\def\etK{{\etens{K}}}
\def\etL{{\etens{L}}}
\def\etM{{\etens{M}}}
\def\etN{{\etens{N}}}
\def\etO{{\etens{O}}}
\def\etP{{\etens{P}}}
\def\etQ{{\etens{Q}}}
\def\etR{{\etens{R}}}
\def\etS{{\etens{S}}}
\def\etT{{\etens{T}}}
\def\etU{{\etens{U}}}
\def\etV{{\etens{V}}}
\def\etW{{\etens{W}}}
\def\etX{{\etens{X}}}
\def\etY{{\etens{Y}}}
\def\etZ{{\etens{Z}}}

% The true underlying data generating distribution
\newcommand{\pdata}{p_{\rm{data}}}
\newcommand{\ptarget}{p_{\rm{target}}}
\newcommand{\pprior}{p_{\rm{prior}}}
\newcommand{\pbase}{p_{\rm{base}}}
\newcommand{\pref}{p_{\rm{ref}}}

% The empirical distribution defined by the training set
\newcommand{\ptrain}{\hat{p}_{\rm{data}}}
\newcommand{\Ptrain}{\hat{P}_{\rm{data}}}
% The model distribution
\newcommand{\pmodel}{p_{\rm{model}}}
\newcommand{\Pmodel}{P_{\rm{model}}}
\newcommand{\ptildemodel}{\tilde{p}_{\rm{model}}}
% Stochastic autoencoder distributions
\newcommand{\pencode}{p_{\rm{encoder}}}
\newcommand{\pdecode}{p_{\rm{decoder}}}
\newcommand{\precons}{p_{\rm{reconstruct}}}

\newcommand{\laplace}{\mathrm{Laplace}} % Laplace distribution

\newcommand{\E}{\mathbb{E}}
\newcommand{\Ls}{\mathcal{L}}
\newcommand{\R}{\mathbb{R}}
\newcommand{\emp}{\tilde{p}}
\newcommand{\lr}{\alpha}
\newcommand{\reg}{\lambda}
\newcommand{\rect}{\mathrm{rectifier}}
\newcommand{\softmax}{\mathrm{softmax}}
\newcommand{\sigmoid}{\sigma}
\newcommand{\softplus}{\zeta}
\newcommand{\KL}{D_{\mathrm{KL}}}
\newcommand{\Var}{\mathrm{Var}}
\newcommand{\standarderror}{\mathrm{SE}}
\newcommand{\Cov}{\mathrm{Cov}}
% Wolfram Mathworld says $L^2$ is for function spaces and $\ell^2$ is for vectors
% But then they seem to use $L^2$ for vectors throughout the site, and so does
% wikipedia.
\newcommand{\normlzero}{L^0}
\newcommand{\normlone}{L^1}
\newcommand{\normltwo}{L^2}
\newcommand{\normlp}{L^p}
\newcommand{\normmax}{L^\infty}

\newcommand{\parents}{Pa} % See usage in notation.tex. Chosen to match Daphne's book.

\DeclareMathOperator*{\argmax}{arg\,max}
\DeclareMathOperator*{\argmin}{arg\,min}

\DeclareMathOperator{\sign}{sign}
\DeclareMathOperator{\Tr}{Tr}
\let\ab\allowbreak


\usepackage{hyperref}
\usepackage{url}
\usepackage{amsmath, amssymb, amsfonts}
\usepackage{algorithm}
\usepackage{algpseudocode}
\usepackage{array}
\usepackage{bbm}


% Use the following line for the initial blind version submitted for review:
% \usepackage{icml2025}

% If accepted, instead use the following line for the camera-ready submission:
\usepackage[accepted]{icml2025}
\makeatletter
% Overwrite \ICML@appearing with an empty string:
\renewcommand{\ICML@appearing}{}
\makeatother
% For theorems and such
\usepackage{mathtools}
\usepackage{amsthm}

% if you use cleveref..
\usepackage[capitalize,noabbrev]{cleveref}

%%%%%%%%%%%%%%%%%%%%%%%%%%%%%%%%
% THEOREMS
%%%%%%%%%%%%%%%%%%%%%%%%%%%%%%%%
\theoremstyle{plain}
\newtheorem{theorem}{Theorem}[section]
\newtheorem{proposition}[theorem]{Proposition}
\newtheorem{lemma}[theorem]{Lemma}
\newtheorem{corollary}[theorem]{Corollary}
\theoremstyle{definition}
\newtheorem{definition}[theorem]{Definition}
\newtheorem{assumption}[theorem]{Assumption}
\theoremstyle{remark}
\newtheorem{remark}[theorem]{Remark}

\newcommand{\Ours}{GoFlow }
\newcommand{\OursNoSpace}{GoFlow}


% Todonotes is useful during development; simply uncomment the next line
%    and comment out the line below the next line to turn off comments
%\usepackage[disable,textsize=tiny]{todonotes}
\usepackage[textsize=tiny]{todonotes}


% The \icmltitle you define below is probably too long as a header.
% Therefore, a short form for the running title is supplied here:
\icmltitlerunning{Flow-based Domain Randomization for
Learning and Sequencing Robotic Skills}

\begin{document}

\twocolumn[
\icmltitle{Flow-based Domain Randomization for
Learning and \\ Sequencing Robotic Skills}

% It is OKAY to include author information, even for blind
% submissions: the style file will automatically remove it for you
% unless you've provided the [accepted] option to the icml2025
% package.

% List of affiliations: The first argument should be a (short)
% identifier you will use later to specify author affiliations
% Academic affiliations should list Department, University, City, Region, Country
% Industry affiliations should list Company, City, Region, Country

% You can specify symbols, otherwise they are numbered in order.
% Ideally, you should not use this facility. Affiliations will be numbered
% in order of appearance and this is the preferred way.
% \icmlsetsymbol{equal}{*}

\begin{icmlauthorlist}
\icmlauthor{Aidan Curtis}{mit,autodesk}
\icmlauthor{Eric Li}{mit}
\icmlauthor{Michael Noseworthy}{mit}
\icmlauthor{Nishad Gothoskar}{mit}
\icmlauthor{Sachin Chitta}{autodesk}
\icmlauthor{Hui Li}{autodesk}
\icmlauthor{Leslie Pack Kaelbling}{mit}
\icmlauthor{Nicole Carey}{autodesk}

\end{icmlauthorlist}

\icmlaffiliation{mit}{MIT CSAIL}
\icmlaffiliation{autodesk}{Autodesk Research}

\icmlcorrespondingauthor{Aidan Curtis}{curtisa@mit.edu}

% You may provide any keywords that you
% find helpful for describing your paper; these are used to populate
% the "keywords" metadata in the PDF but will not be shown in the document
\icmlkeywords{Reinforcement Learning, Planning, Robotics}

\vskip 0.3in
]

\printAffiliationsAndNotice{Work done during a summer internship at Autodesk.} % otherwise use the standard text.

\begin{abstract}
Domain randomization in reinforcement learning is an established technique for increasing the robustness of control policies trained in simulation. 
By randomizing environment properties during training, the learned policy can become robust to uncertainties along the randomized dimensions. 
While the environment distribution is typically specified by hand, in this paper we investigate automatically discovering a sampling distribution via entropy-regularized reward maximization of a normalizing-flow–based neural sampling distribution. 
We show that this architecture is more flexible and provides greater robustness than existing approaches that learn simpler, parameterized sampling distributions, as demonstrated in six simulated and one real-world robotics domain. 
Lastly, we explore how these learned sampling distributions—combined with a privileged value function—can be used for out-of-distribution detection in an uncertainty-aware multi-step manipulation planner.
\end{abstract}

\section{Introduction}

Reinforcement learning (RL) has proven to be a useful tool in robotics for learning control or action policies for tasks and systems which are highly variable and/or analytically intractable~\cite{luo2021learning, zhu2020ingredients, schoettler2020deep}. 
However, RL approaches can be inefficient, involving slow, minimally parallelized, and potentially unsafe data-gathering processes when performed in real environments~\cite{kober2013reinforcement}. 
Learning in simulation eliminates some of these problems, but introduces new issues in the form of discrepancies between the training and real-world environments~\cite{valassakis2020crossing}.

Successful RL from simulation hence requires efficient and accurate models of both robot and environment during the training process. 
But even with highly accurate geometric and dynamic simulators, the system can still be only considered partially observable ~\cite{kober2013reinforcement}---material qualities, inertial properties, perception noise, contact and force sensor noise, manufacturing deviations and tolerances, and imprecision in robot calibration all add uncertainty to the model. 

To improve the robustness of learned policies against sim-to-real discrepancies, it is common to employ domain randomization, varying the large set of environmental parameters inherent to a task according to a given underlying distribution \cite{muratore2019assessing}. 
In this way, policies are trained to maximize their overall performance over a diverse set of models. 
These sampling distributions are typically constructed manually with Gaussian or uniform distributions on individual parameters with hand-selected variances and bounds.
However, choosing appropriate distributions for each of the domain randomization parameters remains a delicate process ~\cite{josifovski2022analysis}; too broad a distribution leads to suboptimal local minima convergence (see Figure~\ref{fig:sim_results}), while too narrow a distribution leads to poor real-world generalization~\citep{gaussian_dr, DBLP:journals/corr/abs-1810-12282}.
Many existing methods rely on real-world rollouts from hardware experiments to estimate dynamics parameters~\cite{chebotar2019closing, bayessim, pmlr-v164-muratore22a}. However, for complex tasks with physical parameters that are difficult to efficiently or effectively sample, this data may be time-consuming to produce, or simply unavailable.

An ideal sampling distribution enables the policy to focus training on areas of the distribution that can feasibly be solved in order to maximize the overall success rate of the trained policy while not wasting time on unsolvable regions of the domain. Automating updates to parameter distributions during the training process can remove the need for heuristic tuning and iterative experimentation~\cite{gaussian_dr, adr, entmax}. 
In this paper, we present \OursNoSpace, a novel approach for learned domain randomization that combines actor-critic reinforcement learning architectures~\citep{schulman2017proximal, haarnoja2018soft} with a neural sampling distribution to learn robust policies that generalize to real-world settings. 
By maximizing the diversity of parameters during sampling, we actively discover environments that are challenging for the current policy but still solvable given enough training. 

As proof of concept, we investigate one real-world use case: contact-rich manipulation for assembly. 
Assembly is a critical area of research for robotics, requiring a diverse set of high-contact interactions which often involve wide force bandwidths and unpredictable dynamic changes. 
Recently, sim-to-real RL has emerged as a potentially useful strategy for learning robust contact-rich policies without laborious real-world interactions~\cite{forge, industreal, dynamic_compliance}. We build on this work by testing our method on the real-world industrial assembly task of gear insertion.

Lastly, we extend this classical gear insertion task to the setting of multi-step decision making under uncertainty and partial observability. As shown in this paper and elsewhere~\cite{entmax, gaussian_dr, adr}, policies trained in simulation have an upper bound on the environmental uncertainties that they can be conformant to. For example, a visionless robot executing an insertion policy can only tolerate so much in-hand pose error.
However, estimates of this uncertainty can be used to inform high level control decisions, e.g., looking closer at objects to get more accurate pose estimates or tracking objects in the hand to detect slippage.
By integrating a probabilistic pose estimation model, we can use the sampling distributions learned with \Ours as an out-of-distribution detector to determine whether the policy is expected to succeed under its current belief about the world state. 
If the robot has insufficient information, it can act to deliberately seek the needed information using a simple belief-space planning algorithm. For example, an in-hand camera can be used to gather a higher resolution image for more accurate pose estimation if necessary.

Our contributions are as follows: We introduce \OursNoSpace, a novel domain randomization method that combines actor-critic reinforcement learning with a learned neural sampling distribution. We show that \Ours outperforms fixed and other learning-based solutions to domain randomization on a suite of simulated environments. We demonstrate the efficacy of \Ours in a real-world contact-rich manipulation task—gear insertion—and extend it to multi-step decision-making under uncertainty. By integrating a probabilistic pose estimation model, we enable the robot to actively gather additional information when needed, enhancing performance in partially observable settings.

\section{Related Work}

Recent developments in reinforcement learning have proven that policies trained in simulation can be effectively translated to real-world robots for contact-rich assembly tasks~\citep{dynamic_compliance, industreal, forge, Jin2023}. One key innovation that has contributed to the development of robust policies is domain randomization~\citep{understanding_dr, original_dr}, wherein environment parameters are sampled from a distribution during training such that the learned policy can be robust to environmental uncertainty on deployment. 

Some previously explored learning strategies include minimization of divergence with a target sampling distribution using multivariate Gaussians~\citep{gaussian_dr}, maximization of entropy using independent beta distributions~\citep{entmax}, and progressive expansion of a uniform sampling distribution via boundary sampling~\citep{adr}. Here, we propose a novel learned domain randomization technique using normalizing flows~\cite{rezende2015variational} as a neural sampling distribution, thus increasing flexibility and expressivity.

In addition to learning robust policies, such sampling distributions can be used as indicators of the world states under which the policy is expected to succeed. Some previous works have combined domain randomization with information gathering via system identification \citep{bayessim, normflows_adaptive_dr}. In this work, we similarly make use of our learned sampling distribution as an out-of-distribution detector in the context of a multi-step planning system. 

\section{Background}

\subsection{Markov Decision Process}
A Markov Decision Process (MDP) is a mathematical framework for modeling decision-making. Formally, an MDP is defined as a tuple $(\mathcal{S}, \mathcal{A}, P, R, \gamma)$, where $\mathcal{S}$ is the state space, $\mathcal{A}$ is the action space, $P: \mathcal{S} \times \mathcal{A} \times \mathcal{S} \rightarrow [0,1]$ is the state transition probability function, where $P(s' \mid s, a)$ denotes the probability of transitioning to state $s'$ from state $s$ after taking action $a$, $R: \mathcal{S} \times \mathcal{A} \rightarrow \mathbb{R}$ is the reward function, where $R(s, a)$ denotes the expected immediate reward received after taking action $a$ in state $s$, $\gamma \in [0,1)$ is the discount factor, representing the importance of future rewards.

A policy $\pi: \mathcal{S} \times \mathcal{A} \rightarrow [0,1]$ defines a probability distribution over actions given states, where $\pi(a \mid s)$ is the probability of taking action $a$ in state $s$. The goal is to find an optimal policy $\pi^*$ that maximizes the expected cumulative discounted reward.

\subsection{Domain Randomization}
Domain randomization introduces variability into the environment by randomizing certain parameters during training. Let $\Xi$ denote the space of domain randomization parameters, and let $\xi \in \Xi$ be a specific instance of these parameters. Each $\xi$ corresponds to a different environment configuration or dynamics.

We can define a parameterized family of Markov Decision Processes (MDPs) where each $\mathcal{M}_\xi = (\mathcal{S}, \mathcal{A}, P_\xi, R_\xi, \gamma)$ has transition dynamics $P_\xi$ and reward function $R_\xi$ dependent on $\xi$. The agent interacts with environments sampled from a distribution over $\Xi$, typically denoted as $p(\xi)$. \footnote{This problem can also be thought of as a POMDP where the observation space is $\mathcal{S}$ and the state space is a product of $\mathcal{S}$ and $\Xi$ as discussed in \cite{latent_mdp}.}

The objective is to learn a policy $\pi: \mathcal{S} \rightarrow \mathcal{A}$ that maximizes the expected return across the distribution environments:

\begin{equation}
J(\pi) = \mathbb{E}_{\xi \sim p(\xi)} \left[ \mathbb{E}_{\tau \sim P_\xi, \pi} \left[ \sum_{t=0}^\infty \gamma^t R_\xi(s_t, a_t) \right] \right],
\end{equation}

where $\tau = \{ (s_0, a_0, s_1, a_1, \dots) \}$ denotes a trajectory generated by policy $\pi$ in environment $\xi$. Domain randomization aims to find a policy $\pi^*$ such that: $\pi^* = \arg\max_\pi J(\pi)$.


In deep reinforcement learning, the policy $\pi$ is a neural network parameterized by $\theta$, denoted as $\pi_\theta$. The agent learns the policy parameters $\theta$ through interactions with simulated environments sampled from $p(\xi)$. In our implementation, we employ the Proximal Policy Optimization (PPO) algorithm \citep{schulman2017proximal}, an on-policy policy gradient method that optimizes a stochastic policy while ensuring stable and efficient learning.

To further stabilize training, we pass privileged information about the environment parameters $\xi$ to the critic network. The critic network, parameterized by $\psi$, estimates the state-value function:

\begin{equation}
V_\psi(s_t, \xi) = \mathbb{E}_{\pi_\theta} \left[ \sum_{k=0}^\infty \gamma^k r_{t+k} \,\bigg|\, s_t, \xi \right],
\end{equation}

where $s_t$ is the current state, $r_{t+k}$ are future rewards, and $\gamma$ is the discount factor. By incorporating $\xi$, the critic can provide more accurate value estimates with lower variance~\citep{pinto2017asymmetric}. The actor network $\pi_\theta(a_t | s_t)$ does not have access to $\xi$, ensuring that the policy relies only on observable aspects of the state.
\begin{figure*}[t!]
    \centering
    \includegraphics[width=0.9\textwidth]{figures/arch_diagram.pdf}
    \caption{An architecture diagram for our actor-critic RL training setup using a normalizing flow to seed environment parameters across episodes.}
    \label{fig:arch}
\end{figure*}
\subsection{Normalizing Flows}
Normalizing flows are a class of generative models that transform a simple base distribution into a complex target distribution using a sequence of invertible, differentiable functions. Let \( z \sim p_Z(z) \) be a latent variable from a base distribution (e.g., a standard normal distribution). A normalizing flow defines an invertible transformation \( f_{\phi}: \mathbb{R}^d \rightarrow \mathbb{R}^d \) parameterized by neural network parameters \( \phi \), such that \( x = f_{\phi}(z) \), aiming for \( x \) to follow the target distribution.

The density of \( x \) is computed using the change of variables formula:

\begin{equation}
p_X(x) = p_Z(f_{\phi}^{-1}(x)) \left| \det \left( \frac{\partial f_{\phi}^{-1}(x)}{\partial x} \right) \right|.
\end{equation}

For practical computation, this is often rewritten as:

\begin{equation}
\log p_X(x) = \log p_Z(z) - \log \left| \det \left( \frac{\partial f_{\phi}(z)}{\partial z} \right) \right|,
\end{equation}

where \( \frac{\partial f_{\phi}(z)}{\partial z} \) is the Jacobian of \( f_{\phi} \) at \( z \). By composing multiple such transformations \( f_{\phi} = f_{\phi_K} \circ \dots \circ f_{\phi_1} \), each parameterized by neural network parameters \( \phi_k \), normalizing flows can model highly complex distributions.


In our work, we employ \emph{neural spline flows} \citep{durkan2019neural}, a type of normalizing flow where the invertible transformations are constructed using spline-based functions. Specifically, the parameters \( \phi \) represent the coefficients of the splines (e.g., knot positions and heights) and the weights and biases of the neural networks that parameterize these splines. 

\section{Method}

In this section, we introduce \textbf{\OursNoSpace}, a method for learned domain randomization that \emph{goes with the flow} by adaptively adjusting the domain randomization process using normalizing flows.

In traditional domain randomization setups, the distribution \(p(\xi)\) is predefined. 
However, selecting an appropriate \(p(\xi)\) is crucial for the policy's performance and generalization. 
Too broad a sampling distribution and the training focuses on unsolvable environments and falls into local minima. 
In contrast, too narrow a sampling distribution leads to poor generalization and robustness. 
Additionally, rapid changes to the sampling distribution can lead to unstable training.
To address these challenges, prior works such as \cite{selfpaced} have proposed a self-paced learner, which starts by mastering a small set of environments, and gradually expands the tasks to solver harder and harder problems while maintaining training stability. This strategy has subsequently been applied to domain randomization in ~\cite{gaussian_dr} and ~\cite{entmax}, where terms were included for encouraging spread over the sampling space for greater generalization. We take inspiration from these works to form a joint optimization problem:

\begin{equation}
\max_{p, \pi} \left\{ \mathbb{E}_{\xi \sim p} [ J_\xi(\pi) ] + \alpha \mathcal{H}(p) - \beta D_{KL}(p_{\text{old}} \| p) \right\}
\end{equation}

where $p$ is the sampling distribution over environment parameters $p(\xi)$, $\mathcal{H}(p)$ is the differential entropy of $p(\xi)$, \(D_{KL}\left(p \| p_{\text{old}}\right)\) is the divergence between the current and previous sampling distributions, and $\alpha > 0, \beta > 0$ are regularization coefficients that control the trade-off between generalizability, training stability, and the expected reward under the sampling distribution.

Other learned domain randomization approaches propose similar objectives. \cite{gaussian_dr} maximizes reward but replaces entropy regularization with a KL divergence to a fixed target distribution and omits the self-paced KL term. \cite{entmax} includes all three objectives but frames the reward and self-paced KL terms as constraints, maximizing entropy through a nonlinear optimization process that is not easily adaptable to neural sampling distributions. We compare \Ours to these methods in our experiments to highlight its advantages. To our knowledge, \Ours is the first method to optimize such an objective with a neural sampling distribution.

The \Ours algorithm (Algorithm~\ref{alg:goflow}) begins by initializing both the policy parameters $\theta$ and the normalizing flow parameters $\phi$. In each training iteration, \Ours first samples a batch of environment parameters $\{\xi_i\}_{i=1}^B$ from the current distribution modeled by the normalizing flow (Line~\ref{line:sample}). 
These sampled parameters are used to train the policy $\pi_\theta$ (Line~\ref{line:train}). 
Following the policy update, expected returns $J_{\xi_i}(\pi_\theta)$ are estimated for each sampled environment through policy rollouts, providing a measure of the policy's performance under a target uniform distribution $u(\xi)$ after being trained on $p(\xi)$ (Line~\ref{line:rollout}).


\begin{algorithm}[t]
\caption{\OursNoSpace}
\label{alg:goflow}
\begin{algorithmic}[1]
\Require Initial policy parameters $\theta$, flow parameters $\phi$, training steps $N$, network updates $K$, entropy coefficient $\alpha$, similarity coefficient $\beta$, and learning rate $\eta_\phi$
\For{$n = 1$ to $N$}
    \State Sample $\{\xi^{\text{train}}_i\}_{i=1}^B \sim p_\phi(\xi)$, $\{\xi^{\text{test}}_i\}_{i=1}^B \sim u(\xi)$ \label{line:sample}
    \State Train $\pi_\theta$ with $\xi^{\text{train}}_i$ initializations \label{line:train}
    \State Estimate $J_{\xi^{\text{test}}_i}(\pi_\theta)$ via policy rollouts \label{line:rollout}
    
    \State Save current flow distribution as $p_{\phi_{\text{old}}}(\xi)$

    \For{$k=1$ to $K$}
        \State $\mathcal{R} \gets \frac{|\Xi|}{B} \sum_{i=1}^B \Bigl[p_\phi(\xi_i^{\mathrm{test}})\,J_{\xi_i^{\mathrm{test}}}(\pi_\theta)\Bigr].$ \label{line:reward}
        \State $\hat{\mathcal{H}} \gets - |\Xi|\cdot\mathbb{E}_{\xi \sim u(\xi)} \Bigl[ p_\phi(\xi)\log p_\phi(\xi)\Bigr]$ \label{line:entropy}
        \State $\hat{D}_{KL} \gets \mathbb{E}_{\xi \sim p_{\phi_{\text{old}}}(\xi)} \Bigl[ \log p_{\phi_{\text{old}}}(\xi) - \log p_\phi(\xi) \Bigr]$ \label{line:kl}
        \State $\phi \leftarrow \phi + \eta_\phi \nabla_\phi \left(\mathcal{R} + \alpha \hat{H} - \beta \hat{D}_{KL} \right)$ \label{line:gradient}
    \EndFor
\EndFor
\end{algorithmic}
\end{algorithm}

\begin{figure*}[t!]
    \centering
    \includegraphics[width=\textwidth]{figures/distributions.pdf}
    \caption{An illustrative domain showing the learned sampling functions over the space of unobserved parameters for the tested baselines. Compared to other learning methods, \Ours correctly models the multimodality and inter-variable dependencies of the underlying reward function. This toy domain, along with other domains in our experiments, violates some of the assumptions made by prior works, such as the feasibility of the center point of the range.}
    \label{fig:toy_sampling_dist}
\end{figure*}


After sampling these rollouts, \Ours then performs $K$ steps of optimization on the sampling distribution.
\Ours first estimates the policy's performance on the sampling distribution using the previously sampled rollout trajectories (Line~\ref{line:reward}).
The entropy of the sampling distribution (Line~\ref{line:entropy}) and divergence from the previous sampling distribution (Line~\ref{line:kl}) are estimated using newly drawn samples. Importantly, we compute the reward and entropy terms by importance sampling from a uniform distribution rather than from the flow itself. This broad coverage helps prevent the learned distribution from collapsing around a small region of parameter space. Derivations for these equations can be found in Appendix~\ref{importance_sampling_proofs}. These terms are combined to form a loss, which is differentiated to update the parameters of the sampling distribution (Line~\ref{line:gradient}). An architecture diagram for this approach can be seen in Figure~\ref{fig:arch}.


\section{Domain Randomization Experiments}

Our simulated experiments compare policy robustness in a range of domains. For full details on the randomization parameters and bounds, see Appendix~\ref{app:dr_params}.

\subsection{Domains}
First, we examine the application of \Ours to an illustrative 2D domain that is multimodal and contains intervariable dependencies. The state and action space are in $\mathbb{R}^2$. The agent is initialized randomly in a bounded x, y plane. An energy function is defined by a composition of %eight 
Gaussians placed in a regular circular or linear array. The agent can observe its position with Gaussian noise proportional to the inverse of the energy function. The agent is rewarded for guessing its location, but is incapable of moving. This task is infeasible when the agent is sufficiently far from any of the %eight 
Gaussian centers, so a sampling distribution should come to resemble the energy function. Some example functions learned by \Ours and baselines from Section~\ref{sec:baselines} can be seen in Figure~\ref{fig:toy_sampling_dist}.

Second, we quantitatively compare \Ours to existing baselines including Cartpole, Ant, Quadcopter, and Quadruped in the IsaacLab suite of environments~\citep{isaaclab}. We randomize over parameters such as link masses, joint frictions, and material properties.

Lastly, we evaluate our method on a contact-rich robot manipulation task of gear insertion, a particularly relevant problem for robotic assembly. 
In the gears domain, we randomize over the relative pose between the gripper and the held gears along three degrees of freedom. 
The problem is made difficult by the uncertainty the robot has about the precise location of the gear relative to the hand.
The agent must learn to rely on signals of proprioception and force feedback to guide the gear into the gear shaft. 
The action space consists of end-effector pose offsets along three translational degrees of freedom and one rotational degree around the z dimension. 
The observation space consists of a history of the ten previous end effector poses and velocities estimated via finite differencing.
In addition to simulated experiments, our trained policies are tested on a Franka Emika robot using \texttt{IndustReal} library built on the \texttt{frankapy} toolkit~\citep{tang2023industreal, zhang2020modular}. 

\begin{figure*}[t!]
    \centering
    \includegraphics[width=\linewidth]{figures/all_results.pdf}
    \caption{The coverage ratio over the target distribution across five random seeds for each of the environments. The bands around each curve indicate the standard error.}
    \label{fig:sim_results}
\end{figure*}
\subsection{Baselines}
\label{sec:baselines}



In our domain randomization experiments, we compare to a number of standard RL baselines and learning-based approaches from the literature. In our quantitative experiments, success is measured by the sampled environment passing a certain performance threshold $J_T$ that was selected for each environment. All baselines are trained with an identical neural architecture and PPO implementation. The success thresholds along with other hyperparameters are in Appendix~\ref{app:hyperparameters}.

We evaluate on the following baselines. First, we compare to no domain randomization (\textbf{NoDR}) which trains on a fixed environment parameter at the centroid of the parameter space. Next, we compare to a full domain randomization (\textbf{FullDR}) which samples uniformly across the domain within the boundaries during training. In addition to these fixed randomization methods, we evaluate against some other learning-based solutions from the literature: \textbf{ADR}~\citep{adr} learns uniform intervals that expand over time via boundary sampling. It starts by occupying an initial percentage of the domain and performs ``boundary sampling'' during training with some probability. The rewards attained from boundary sampling are compared to thresholds that determine if the boundary should be expanded or contracted.  \textbf{LSDR}~\citep{gaussian_dr} learns a multivariate gaussian sampling distribution using reward maximization with a KL divergence regularization term weighted by an $\alpha$ hyperparameter. Lastly, \textbf{DORAEMON}~\citep{entmax} learns independent beta distributions for each dimension of the domain, using a maximum entropy objective constrained by an estimated success rate.

We measure task performance in terms of coverage, which is defined as the proportion of the total sampling distribution for which the policy receives higher than $J_t$ reward. We compare coverage across environments sampled from a uniform testing distribution within the environment bounds. Our findings in Figure~\ref{fig:sim_results} show that \Ours matches or outperforms baselines across all domains. We find that our method performs particularly well in comparison to other learned baselines when the simpler or more feasible regions of the domain are off-center, irregularly shaped, and have inter-parameter dependencies such as those seen in Figure~\ref{fig:toy_sampling_dist}. We intentionally chose large intervals with low coverage to demonstrate this capability, and we perform a full study of how baselines degrade significantly with increased parameter ranges while \Ours degrades more gracefully (see Appendix~\ref{app:coverage_experiments}). Additionally, coverage should not be mistaken for success-rate, as these policies can be integrated into a planning framework that avoids the use of infeasible actions or gathers information to increase coverage as discussed in the following sections.

In addition to simulated experiments, we additionally evaluated the trained policies on a real-world gear insertion task. The results of those real-world experiments show that \Ours results in more robust sim-to-real transfer as seen in Table~\ref{tab:real_world} and in the supplementary videos.




\section{Application to Multi-step manipulation}

While reinforcement learning has proven to be a valuable technique for learning short-horizon skills in dynamic and contact-rich settings, it often struggles to generalize to more long-horizon and open ended problems~\citep{rl_is_hard}. 
The topic of sequencing short horizon skills in the context of a higher-level decision strategy has been of increasing interest to both the planning~\citep{mishra2023generativeskillchaininglonghorizon} and reinforcement learning communities~\citep{maple}.
For this reason, we examine the utility of these learned sampling distributions as out-of-distribution detectors, or belief-space preconditions, in the context of a multi-step planning system.



\subsection{Belief-space planning background}

\begin{figure*}[t!]
    \centering
    \includegraphics[width=\textwidth]{figures/multi-step-3.pdf}
    \caption{A multi-step manipulation plan using probabilistic pose estimation to estimate and update beliefs over time. The three rows show the robot state $s_t$, the observation $o_t$, and the robot belief $b_t$ at each timestep. The red dotted line in the belief indicates the marginal entropy thresholds for the x, y, and yaw dimensions as determined by the learned normalizing flow. For full visualizations of the belief posteriors, flow distributions, and value maps, see Figure~\ref{fig:belief_posteriors} in the appendix.}
    \label{fig:multi-step-plan}
\end{figure*}

Belief-space planning is a framework for decision-making under uncertainty, where the agent maintains a probability distribution over possible states, known as the \emph{belief state}. Instead of planning solely in the state space \(\mathcal{S}\), the agent operates in the \emph{belief space} \(\mathcal{B}\), which consists of all possible probability distributions over \(\mathcal{S}\). This approach is particularly useful in partially observable environments where there is uncertainty in environment parameters and where it is important to take actions to gain information.

Rather than operating at the primitive action level, belief-space planners often make use of high-level actions $\mathcal{A}_\Pi$, sometimes called skills or options. 
In our case, these high-level actions will be a discrete set of pretrained RL policies. 
These high-level actions come with a \emph{belief-space precondition} and a \emph{belief-space effect}, both of which are subsets of the belief space \(\mathcal{B}\) ~\cite{BHPN, tampura}. 
Specifically, a high-level action \( \pi \in \mathcal{A}_\Pi \) is associated with two components: a precondition \( \text{Pre}_\pi \subseteq \mathcal{B} \), representing the set of belief states from which the action can be applied, and an effect \( \text{Eff}_\pi \subseteq \mathcal{B} \), representing the set of belief states that the action was designed or trained to achieve. 
If the precondition holds—that is, if the current belief state \( b \) satisfies \( b \in \text{Pre}_\pi \)—then applying action \( a \) will achieve the effect \( \text{Eff}_\pi \) with probability at least \( \eta \in [0,1] \). 
More formally, if $b \in \text{Pre}_\pi$, then $\Pr\left( b_{t+1} \in \text{Eff}_\pi \,\big|\, b_t, \pi \right) \geq \eta$. Depending on the planner, $\eta$ may be set in advance, or calculated by the planner as a function of the belief.

This formalization allows planners to reason abstractly about the effects of high-level actions under uncertainty, which can result in generalizable decision-making in long-horizon problems that require active information gathering or risk-awareness.


\subsection{Computing preconditions}
\label{sec:computing_preconditions}
In this section, we highlight the potential application of the learned sampling distribution $p_\phi$ and privileged value function $V_\psi$ as a useful artifacts for belief-space planning. In particular, we are interested in identifying belief-space preconditions of a set of trained skills.



One point of leverage we have for this problem is the privileged value function $V_\psi(s, \xi)$, which was learned alongside the policy during training. 
One way to estimate the belief-space precondition is to simply find the set of belief states for which the expected value of the policy is larger than $J_T$ with probability greater than $\eta$ under the belief: 
\begin{equation}
\text{Pre}_\pi = \left\{ b \in \mathcal{B} \mid \mathbb{E}_{b}\left[\mathbbm{1}_{V_\psi(s, \xi)> J_T}\right] > \eta  \right\}.
\end{equation}


However, a practical issue with this computation is that the value function is likely not calibrated in large portions of the state space that were not seen during policy training.
To address this, we focus on regions of the environment where the agent has higher confidence due to sufficient sampling, i.e., where $p_\phi(\xi) > \epsilon$ for a threshold $\epsilon$. This enables us to integrate the value function over the belief distribution $b(x, \xi)$ and the trusted region within $\Xi$:

\begin{equation}
\text{Pre}_\pi = \left\{ b \in \mathcal{B} \mid \mathbb{E}_{b}\left[\mathbbm{1}_{V_\psi(s, \xi)> J_T} \cdot \mathbbm{1}_{p_\phi(\xi) > \epsilon}\right] > \eta \right\}
\end{equation}.

Here, $\eta$ lower bounds the probability of achieving the desired effects $\text{Eff}_\pi$ (or value greater than $J_T$) after executing $\pi$ in any belief state in $\text{Pre}_\pi$.
Figure~\ref{fig:planning_dists} shows an example precondition for a single step of the assembly plan.

\subsection{Updating beliefs}
\label{sec:belief_update}


\begin{figure*}[ht!]
    \centering
    \includegraphics[width=0.95\textwidth]{figures/planning_dists.pdf}
    \caption{A visual example of the precondition computation described in Section~\ref{sec:computing_preconditions} for a single step in the gear assembly process. The normalizing flow is thresholded to indicate the in-distribution regions of the learned value function. The thresholded sampling distribution is further thresholded by the value function to get the belief-space precondition. Comparing the precondition to the beliefs, we can see that the belief is not sufficiently contained within the precondition at $t=0$, but passes the success threshold $\eta$ at after closer inspection $t=4$.}
    \label{fig:planning_dists}
\end{figure*}

Updating the belief state requires a probabilistic state estimation system that outputs a posterior over the unobserved environment variables, rather than a single point estimate. 
We use a probabilistic object pose estimation framework called Bayes3D to infer posterior distributions over object pose~\citep{gothoskar2023bayes3d}. 
For details on this, see Appendix~\ref{app:belief_update}.

The benefit of this approach in contrast to traditional rendering-based pose estimation systems, such as those presented in  \cite{wen2024foundationposeunified6dpose} or \cite{megapose}, is that pose estimates from Bayes3D indicate high uncertainty for distant, small, or occluded objects as well as uncertainty stemming from object symmetry. Figure~\ref{fig:belief_posteriors} shows the pose beliefs across the multi-step plan.


\subsection{A simple belief-space planner}
While the problem of general-purpose multi-step planning in belief-space has been widely studied, in this paper we use a simple BFS belief-space planner to demonstrate the utility of the learned sampling distributions as belief-space preconditions. 
The full algorithm can be found in Algorithm~\ref{alg:belief_space_planner}. 

An example plan can be seen in Figure~\ref{fig:multi-step-plan}. 
The goal is to assemble the gear box by inserting all three gears (yellow, pink, and blue) into the shafts on the gear plate. 
Each gear insertion is associated with a separate policy for each color trained with \OursNoSpace. 
In addition to the trained policies, the robot is given access to an object-parameterized inspection action which has no preconditions and whose effects are a reduced-variance pose estimate attained by moving the camera closer to the object. 
The robot is initially uncertain of the x, y, and yaw components of the 6-dof pose based on probabilistic pose estimates. Despite this uncertainty, the robot is confident enough in the pose of the largest and closest yellow gear to pick it up and insert it. 
In contrast, the blue and pink gears require further inspection to get a better pose estimate.
Closer inspection reduces uncertainty along the x and y axis, but reveals no additional information about yaw dimension due to rotational symmetry. 
Despite an unknown yaw dimension, the robot is confident in the insertion because the flow $p_\phi$ indicates that success is invariant to the yaw dimension. 
For visualizations of the beliefs and flows at each step, see Appendix~\ref{app:multi-step-planning}.


\section{Conclusion and discussion}

In this paper, we introduced \OursNoSpace, a novel approach to domain randomization that uses normalizing flows to dynamically adjust the sampling distribution during reinforcement learning. 
By combining actor-critic reinforcement learning with a learned neural sampling distribution, we enabled more flexible and expressive parameterization of environmental variables, leading to better generalization in complex tasks like contact-rich assembly. 
Our experiments demonstrated that GoFlow outperforms traditional fixed and learning-based domain randomization techniques across a variety of simulated environments, particularly in scenarios where the domain has irregular dependencies between parameters. 
The method also showed promise in real-world robotic tasks including contact-rich assembly.

Moreover, we extended GoFlow to multi-step decision-making tasks, integrating it with belief-space planning to handle long-horizon problems under uncertainty. 
This extension enabled the use of learned sampling distributions and value functions as preconditions leading to active information gathering. 

Although \Ours enables more expressive sampling distributions, it also presents some new challenges. One limitation of our method is that it has higher variance due to occasional training instability of the flow. 
This instability can be alleviated by increasing $\beta$, but at the cost of reduced sample efficiency (see Appendix~\ref{app:hyperparameters}). 
In addition, using the flow and value estimates for belief-space planning require manual selection and tuning of several thresholds which are environment specific. 
The $\eta$ parameter may be converted from a threshold into a cost in the belief space planner, which would remove one point of manual tuning. 
However, removing the $\epsilon$ parameter may prove more difficult, as it would require uncertainty quantification of the neural value function. 
Despite these challenges, we hope this work inspires further research on integrating short-horizon learned policies into broader planning frameworks, particularly in contexts involving uncertainty and partial observability.


\section{Impact Statement}
This paper presents work whose goal is to advance the field of Machine Learning. There are many potential societal consequences of our work, none which we feel must be specifically highlighted here.

\bibliography{example_paper}
\bibliographystyle{icml2025}

\appendix
\section{Appendix}


\subsection{Reproducibility Statement}

In our supplementary materials, we provide a minimal code implementation of our approach along with all the baseline implementations and environments discussed in this paper. If accepted, we will publish a full public release of the code on Github.

\subsection{Importance Sampling Proofs}
\label{importance_sampling_proofs}

Here we show how we obtain the importance-sampled estimates for both the reward term 
\(
\mathcal{R}
\)
and the entropy term 
\(
\hat{\mathcal{H}}
\)
in \OursNoSpace (Algorithm~\ref{alg:goflow}).  
Specifically, we sample from a uniform distribution \(u(\xi)\) over \(\Xi\), rather than from the learned distribution \(p_\phi(\xi)\) directly, in order to avoid collapse onto narrow regions of the parameter space.

\subsubsection{Reward Term: \(\mathcal{R}\)}
We want an unbiased estimate of
\[
\mathbb{E}_{\xi \sim p_\phi(\xi)}\bigl[J_\xi(\pi)\bigr]
\,=\;
\int_{\Xi} 
p_\phi(\xi)\,J_\xi(\pi)
\;\mathrm{d}\xi.
\]
Let \(u(\xi)\) be a uniform distribution over \(\Xi\). Since
\(
p_\phi(\xi)\,J_\xi(\pi)
=
\tfrac{p_\phi(\xi)}{u(\xi)}\,u(\xi)\,J_\xi(\pi),
\)
we can rewrite:
\[
\int_{\Xi} 
p_\phi(\xi)\,J_\xi(\pi)
\,\mathrm{d}\xi
\;=\;
\int_{\Xi}
u(\xi)\;\frac{p_\phi(\xi)}{u(\xi)}\;J_\xi(\pi)\;\mathrm{d}\xi.
\]
Hence, sampling \(\xi_i \sim u(\xi)\) and averaging \(\tfrac{p_\phi(\xi_i)}{u(\xi_i)} J_{\xi_i}(\pi)\) gives an unbiased Monte Carlo estimate of 
\(\mathbb{E}_{p_\phi}[\,J_\xi(\pi)\bigr]\).  

If \(\Xi\) has finite measure \(\lvert \Xi\rvert\), then \(u(\xi) = 1/\lvert\Xi\rvert\). Thus
\[
\frac{p_\phi(\xi)}{u(\xi)}
\;=\;
p_\phi(\xi)\,\lvert\Xi\rvert.
\]
Therefore, the empirical estimate becomes
\[
\mathcal{R}
\;=\;
\frac{1}{\,B\,}
\sum_{i=1}^B
\bigl[
\bigl(p_\phi(\xi_i)\,\lvert\Xi\rvert\bigr)\,J_{\xi_i}(\pi)
\bigr]
\]
\[
\;=\;
\frac{\lvert \Xi\rvert}{\,B\,}
\sum_{i=1}^B
p_\phi(\xi_i)\,J_{\xi_i}(\pi).
\]
where \(\{\xi_i\}_{i=1}^B \sim u(\xi)\).  This matches Line~\ref{line:reward} in Algorithm~\ref{alg:goflow}.

\subsubsection{Entropy Term: \(\hat{\mathcal{H}}\)}
Similarly, to compute the differential entropy of \(p_\phi(\xi)\), we have
\[
\mathcal{H}(p_\phi)
\;=\;
-\int_{\Xi} p_\phi(\xi)\,\log p_\phi(\xi)\,\mathrm{d}\xi.
\]
Again, we apply the same importance sampling trick via \(u(\xi)\). We write:
\[
p_\phi(\xi)\,\log p_\phi(\xi)
\;=\;
\frac{p_\phi(\xi)}{\,u(\xi)\!}\,u(\xi)\,\log p_\phi(\xi).
\]
Hence
\[
\int_{\Xi} p_\phi(\xi)\,\log p_\phi(\xi)\,\mathrm{d}\xi
\;=\;
\int_{\Xi} u(\xi)\,\tfrac{p_\phi(\xi)}{u(\xi)}\,\log p_\phi(\xi)\,\mathrm{d}\xi.
\]
If \(\lvert \Xi\rvert\) is the measure of \(\Xi\) under \(u(\xi)\), then \(u(\xi)=1/\lvert\Xi\rvert\). So the Monte Carlo estimate for 
\(\int p_\phi(\xi)\,\log p_\phi(\xi)\,\mathrm{d}\xi\)
becomes
\[
\frac{\lvert \Xi\rvert}{\,B\,}\,
\sum_{i=1}^B
p_\phi(\xi_i)\,\log p_\phi(\xi_i),
\]
with \(\xi_i \sim u(\xi)\).  
Multiplying by \(-1\) yields the differential entropy:
\[
\mathcal{H}(p_\phi)
\;=\;
-\int p_\phi(\xi)\,\log p_\phi(\xi)\,\mathrm{d}\xi
\]

\[
\;\;\approx\;\;
-\,\frac{\lvert \Xi\rvert}{B}\,
\sum_{i=1}^B
p_\phi(\xi_i)\,\log p_\phi(\xi_i),
\]
which is exactly what we implement in Line~\ref{line:entropy} of Algorithm~\ref{alg:goflow}.

\begin{remark}
Using uniform sampling \(u(\xi)\) to approximate these terms provides global coverage of \(\Xi\), helping prevent the learned distribution \(p_\phi\) from collapsing around a small subset of parameter space.  By contrast, if one sampled \(\xi\) from \(p_\phi(\xi)\) itself for these terms, the distribution might fail to expand to other promising regions once it becomes peaked.
\end{remark}


\subsection{Domain Randomization Parameters}
\label{app:dr_params}
Below we describe the randomization ranges and parameter names for each environment. We also provide the reward success threshold ($J_T$) and cut the max duration of some environments in order to speed up training ($t_{max}$). 
Lastly, we slightly modified the Quadruped environment to only take a fixed forward command rather than the goal-conditioned policy learned by default. 
Other than those changes, the first four simulated environments official IsaacLab implementation.  
\begin{itemize}
    \item \textbf{Cartpole parameters ($J_T=50, t_{max}=2s$):}
    \begin{itemize}
        \item Pole mass: Min = 0.01, Max = 20.0
        \item Cart mass: Min = 0.01, Max = 20.0
        \item Slider-Cart Friction: Min Bound = 0.0, Max Bound = 1.0
    \end{itemize}
    
    \item \textbf{Ant parameters ($J_T=700,t_{max}=2s$):}
    \begin{itemize}
        \item Torso mass: Min = 0.01, Max = 20.0
    \end{itemize}
    
    \item \textbf{Quadcopter parameters ($J_T=15, t_{max}=2s$):}
    \begin{itemize}
        \item Quadcopter mass: Min = 0.01, Max = 20.0
    \end{itemize}
    
    \item \textbf{Quadruped parameters ($J_T=1.5,t_{max}=5s$):}
    \begin{itemize}
        \item Body mass: Min = 0.0, Max = 200.0
        \item Left front hip friction: Min = 0.0, Max = 0.1
        \item Left back hip friction: Min = 0.0, Max = 0.1
        \item Right front hip friction: Min = 0.0, Max = 0.1
        \item Right back hip friction: Min = 0.0, Max = 0.1
    \end{itemize}

    \item \textbf{Humanoid parameters ($J_T=1000, t_{max}=5s$):}
    \begin{itemize}
        \item Torso Mass: Min = 0.01, Max = 25.0
        \item Head Mass: Min = 0.01, Max = 25.0
        \item Left Hand Mass: Min = 0.01, Max = 30.0
        \item Right Hand Mass: Min = 0.01, Max = 30.0
    \end{itemize}
    
    \item \textbf{Gear parameters ($J_T=50, t_{max}=4s$):}
    \begin{itemize}
        \item Grasp Pose x: Min = -0.05, Max = 0.05
        \item Grasp Pose y: Min = -0.05, Max = 0.05
        \item Grasp Pose yaw: Min = -0.393, Max = 0.393
    \end{itemize}
\end{itemize}


\subsection{Multi-Step Planning Details}
\label{app:multi-step-planning}


\subsubsection{Updating beliefs via probabilistic pose estimation}
\label{app:belief_update}

Updating the belief state $b$ requires a probabilistic state estimation system that outputs a posterior over the state space $S$, rather than a single point estimate. 
Given a new observation $o$, we use a probabilistic object pose estimation framework (Bayes3D) to infer posterior distributions over object pose~\citep{gothoskar2023bayes3d}.

The pose estimation system uses inference in an probabilistic generative graphics model with uniform priors on the translational $x$, $y$, and rotational yaw (or $r_x$) components of the 6-dof pose (since the object is assumed to be in flush contact with the table surface) and an image likelihood $P(o_\text{rgbd}\mid r_x, x, y)$. The object's geometry and color information is given by a mesh model. 
The image likelihood is computed by rendering a latent image $im^{\text{rgbd}}$ with the object pose corresponding to $(r_x,x,y)$ and calculating the per-pixel likelihood:
\begin{equation}
\begin{split}
& P(o_\text{rgbd} \mid r_x, x, y) \propto \\ 
& \prod_{i, j \in C}  \left[ p_{\text{out}} + (1-p_{\text{out}}) \cdot  P_{\text{in}}(o^\text{rgbd}_{i,j}\mid r_x, x, y) \right]
\end{split}
\end{equation}
\begin{equation}
\begin{split}
& P_{\text{in}}(o^\text{rgbd}_{i,j}\mid r_x, x, y) \propto \\
& \exp \Bigg( 
-\frac{|| o^\text{rgb}_{i,j}-im^\text{rgb}_{i,j} ||_1}{b^\text{rgb}}
- \frac{||o^\text{d}_{i,j}-im^\text{d}_{i,j}||_1}{b^\text{d}} \Bigg)
\end{split}
\end{equation}

 where $i$ and $j$ are pixel row and column indices, $C$ is the set of valid pixels returned by the renderer, $b_\text{rgb}$ and $b_\text{d}$ are hyperparameters that control posterior sensitivity to the color and depth channels, and $p_{\text{out}}$ is the pixel outlier probability hyperparameter. For an observation $o_\text{rgbd}$, we can sample from $ P(r_x, x, y\mid o_\text{rgbd}) \propto P(o_\text{rgbd}\mid r_x, x, y)$ to recover the object pose posterior with a tempering exponential factor $\alpha$ to encourage smoothness. 
 We first find the maximum a posteriori (MAP) estimate of object pose using coarse-to-fine sequential Monte Carlo sampling ~\citep{del2006sequential} and then calculate a posterior approximation using a grid centered at the MAP estimate. 
 
 The benefit of this approach in contrast to traditional rendering-based pose estimation systems, such as those presented in  \cite{wen2024foundationposeunified6dpose} or \cite{megapose}, is that our pose estimates indicate high uncertainty for distant, small, occluded, or non-visible objects as well dimensions along which the object is symmetric. 
 A visualization of the pose beliefs at different points in the multi-step plan can be seen in Figure~\ref{fig:belief_posteriors} in the Appendix.
 

\subsection{Hyperparameters}
\label{app:hyperparameters}
Below we list out the significant hyperparameters involved in each baseline method, and how we chose them based on our hyperparameter search. 
We run the same seed for each hyperparameter and pick the best performing hyperparameter as the representative for our larger quantitative experiments in figure~\ref{fig:sim_results}. 
The full domain randomization (FullDR) and no domain randomization (NoDR) baselines have no hyperparameters.

\subsubsection{GoFlow}
We search over the following values of the $\alpha$ hyperparameter: $[0.1, 0.5, 1.0, 1.5, 2.0]$. 
We search over the following values of the $\beta$ hyperparameters $[0.0, 0.1, 0.5, 1.0, 2.0]$.
Other hyperparameters include number of network updates per training epoch ($K=100$), network learning rate ($\eta_\phi=1e-3$), and neural spline flow architecture hyperparameters such as network depth ($\ell=3$), hidden features (64), and number of bins (8). We implement our flow using the Zuko normalizing flow library~\cite{rozet2022zuko}.

\begin{figure*}[h!]
    \centering
    \includegraphics[width=\linewidth]{figures/goflow_alpha.pdf}
    \caption{ \Ours hyperparameter sweep results for $\alpha$}
    \label{fig:goflow_alpha_results}
\end{figure*}

\begin{figure*}[h!]
    \centering
    \includegraphics[width=\linewidth]{figures/goflow_beta.pdf}
    \caption{\Ours hyperparameter sweep results for $\beta$ after fixing the best $\alpha$ for each environment}
    \label{fig:goflow_beta_results}
\end{figure*}

\subsubsection{LSDR}
Similary to \Ours, we search over the following values of the $\alpha_L$ hyperparameter: $[0.1, 0.5, 1.0, 1.5, 2.0]$. 
Other hyperparameters include the number of updates per training epoch (T=100), and initial Gaussian parameters: $\mu =  (\xi_{max}+\xi_{min})/2.0$ and $\Sigma = \text{diag}\left(\xi_{\text{max}} - \xi_{\text{min}}/10\right)$

\begin{figure*}[h!]
    \centering
    \includegraphics[width=\linewidth]{figures/lsdr.pdf}
    \caption{LSDR hyperparameter sweep results for $\alpha_L$}
    \label{fig:lsdr_alpha_results}
\end{figure*}

\subsubsection{DORAEMON}
We search over the following values of the $\epsilon_D$ hyperparameter: $[0.005, 0.01, 0.05, 0.1, 0.5]$. 
After fixing the best $\epsilon_D$ for each environment, we additionally search over the success thrshold $\alpha_D$: $[0.005, 0.01, 0.05, 0.1, 0.5]$.

\begin{figure*}[h!]
    \centering
    \includegraphics[width=\linewidth]{figures/doraemon_kl.pdf}
    \caption{DORAEMON hyperparameter sweep results for $\epsilon_D$}
    \label{fig:doraemon_epsilon_results}
\end{figure*}

\begin{figure*}[h!]
    \centering
    \includegraphics[width=\linewidth]{figures/doraemon_success.pdf}
    \caption{DORAEMON hyperparameter sweep results for $\alpha_D$ after fixing the best $\epsilon_D$}
    \label{fig:doraemon_alpha_results}
\end{figure*}

\subsubsection{ADR}
In ADR, we fix the upper threshold to be the success threshold $t^{+} = J_T$ as was done in the original paper and search over the lower bound threshold $t^{-} = [0.1t^{+}, 0.25t^{+}, 0.5t^{+}, 0.75t^{+}, 0.9t^{+}]$.
The value used in the original paper was $0.5t_H$. 
Other hyperparameters include the expansion/contraction rate, which we interpret to be a fixed fraction of the domain interval, $\Delta = 0.1 * \big[\xi_{\text{max}} - \xi_{\text{min}}\big]$, and boundary sampling probability $p_b=0.5$. 

\begin{figure*}[h!]
    \centering
    \includegraphics[width=\linewidth]{figures/adr.pdf}
    \caption{ADR hyperparameter sweep results for $t^{-}$}
    \label{fig:adr_t_results}
\end{figure*}

\subsection{Coverage vs Range Experiments}
\label{app:coverage_experiments}
\begin{figure*}[h!]
    \centering
    \includegraphics[width=1.0\textwidth]{figures/plot_scales.pdf}
    \caption{Coverage vs range experiment results}
    \label{fig:scales}
\end{figure*}

We compare coverage vs. range scale in the ant domain. We adjust the parameter lower and upper bounds outlined in Appendix~\ref{app:dr_params} and see how the coverage responds to those changes during training. 
The parameter range is defined relative to a nominal midpoint $m$ set to the original domain parameters: $[m - (m - \text{lower}) * \text{scale}, m + (\text{upper} - m) * \text{scale})]$. 
The results of our experiment are shown in Figure~\ref{fig:scales}

% \begin{figure}[t!]
%     \centering
%     \includegraphics[width=1.0\textwidth]{figures/entropy_plot.pdf}
%     \caption{Entropy during the training process for each method}
%     \label{fig:arch}
% \end{figure}


\subsection{Real-world experiments}

\begin{table}[h!]
    \centering
    \setlength{\tabcolsep}{5pt} % Adjusts space between columns
    \begin{tabular}{|l|c|}
        \hline
        Method   & Success Rate \\ \hline
        FullDR   & 6/10 \\ \hline
        NoDR     & 3/10 \\ \hline
        DORAEMON & 5/10 \\ \hline
        LSDR     & 5/10 \\ \hline
        ADR      & 5/10 \\ \hline
        GoFlow   & \textbf{9/10} \\ \hline
    \end{tabular}
    \caption{Real-world experimental results with the statistically significant results bolded.}
    \label{tab:real_world}
\end{table}


\begin{table*}[h]
    \centering
    \setlength{\tabcolsep}{5pt} % Adjusts space between columns
    \begin{tabular}{|l|c|c|c|c|c|c|}
        \hline
                 & Cartpole & Ant & Quadcopter & Quadruped & Humanoid & Gears \\ \hline
        FullDR   & .017$\pm$.012 & .003$\pm$.002 & .003$\pm$.004 & .001$\pm$.000 & .200$\pm$.198 & .001$\pm$.000 \\ \hline
        NoDR     & .005$\pm$.000 & .000$\pm$.000 & .000$\pm$.000 & \textbf{.062$\pm$.048} & .000$\pm$.000 & .010$\pm$.000 \\ \hline
        DORAEMON & .005$\pm$.000 & .000$\pm$.000 & .000$\pm$.000 & \textbf{.023$\pm$.022} & \textbf{.124$\pm$.248} & .020$\pm$.018 \\ \hline
        LSDR     & \textbf{.110$\pm$.006} & .000$\pm$.000 & .002$\pm$.002 & .011$\pm$.002 & \textbf{.278$\pm$.230} & .003$\pm$.000 \\ \hline
        ADR      & .004$\pm$.000 & .000$\pm$.000 & .000$\pm$.000 & \textbf{.114$\pm$.026} & .000$\pm$.000 & \textbf{.072$\pm$.044} \\ \hline
        GoFlow   & \textbf{.109$\pm$.080} & \textbf{.072$\pm$.038} & \textbf{.019$\pm$.010} & \textbf{.178$\pm$.220} & \textbf{.321 $\pm$ .068} & \textbf{.104$\pm$.028} \\ \hline
    \end{tabular}
    \caption{Mean and standard deviation of coverage, with all statistically significant entries bolded.}
    \label{tab:statistical}
\end{table*}


In addition to simulated experiments, we compare GoFlow against baselines on a real-world gear insertion task. 
In particular, we tested insertion of the pink medium gear over 10 trials for each baseline. 
To test this, we had the robot perform 10 successive pick/inserts of the pink gear into the middle shaft of the gear plate. Instead of randomizing the pose of the gear, we elected to fix the initial pose of the gear and the systematically perturb the end-effector pose by a random $\pm 0.01$ meter translational offset along the x dimension during the pick. 
We expect some additional grasp pose noise due to position error during grasp and object shift during grasp. 
This led to a randomized in-hand offset while running the trained insertion policy. 
Our results show that GoFlow can indeed more robustly generalize to real-world robot settings under pose uncertainty.

\begin{figure*}[h]
    \centering
    \includegraphics[width=1.0\textwidth]{figures/belief_posteriors.pdf}
    \caption{A visualization of the beliefs over the object pose under the initial image (first three columns) and after closer inspection (last three columns) as generated from the posterior of the model described in Section~\ref{sec:belief_update}. The colormap corresponds to the log probability of the posterior pose estimate. All plots are centered around the most likely pose estimate under the image model. }
    \label{fig:belief_posteriors}
\end{figure*}


\subsection{Statistical Tests}
We performed a statistical analysis of the simulated results reported in Figure~\ref{fig:sim_results} and the real-world experiments in Table~\ref{tab:real_world}. For the simulated results, we recorded the final domain coverages across all seeds and performed pairwise t-tests between each method and the top-performing method. 
The final performance mean and standard deviation are reported in Table~\ref{tab:statistical}. Any methods that were not significantly different from the top performing method ($p<0.05$) are bolded. 
This same method was used to test significance of the real-world results.


\begin{algorithm}
\caption{Belief-Space Planner Using BFS}
\label{alg:belief_space_planner}
\begin{algorithmic}[1]
\Require Initial belief state \( b_0 \), goal condition \( G \subseteq \mathcal{B} \), set of skills \( \mathcal{A}_\Pi \), success threshold \( \eta \)
\State Initialize the frontier \( \mathcal{F} \gets \{ b_0 \} \)
\State Initialize the visited set \( \mathcal{V} \gets \emptyset \)
\State Initialize the plan dictionary \( \text{Plan} \) mapping belief states to sequences of skills
\While{ \( \mathcal{F} \) is not empty }
    \State Dequeue \( b \) from \( \mathcal{F} \)
    \If{ \( b \in G \) }
        \State \Return \( \text{Plan}[b] \) \Comment{Return the sequence of skills leading to \( b \)}
    \EndIf
    \ForAll{ skills \( \pi \in \mathcal{A}_\Pi \) }
        \If{ \( b \in \text{Pre}_\pi \) given \( \eta \) }
            \State \( b' \gets \texttt{sample}(\text{Eff}_\pi) \) 
            \If{ \( b' \notin \mathcal{V} \) }
                \State Add \( b' \) to \( \mathcal{F} \) and \( \mathcal{V} \)
                \State Update \( \text{Plan}[b'] \gets \text{Plan}[b] + [\pi] \)
            \EndIf
        \EndIf
    \EndFor
\EndWhile
\State \Return \textbf{Failure} \Comment{No plan found}
\end{algorithmic}
\end{algorithm}




\end{document}


% This document was modified from the file originally made available by
% Pat Langley and Andrea Danyluk for ICML-2K. This version was created
% by Iain Murray in 2018, and modified by Alexandre Bouchard in
% 2019 and 2021 and by Csaba Szepesvari, Gang Niu and Sivan Sabato in 2022.
% Modified again in 2023 and 2024 by Sivan Sabato and Jonathan Scarlett.
% Previous contributors include Dan Roy, Lise Getoor and Tobias
% Scheffer, which was slightly modified from the 2010 version by
% Thorsten Joachims & Johannes Fuernkranz, slightly modified from the
% 2009 version by Kiri Wagstaff and Sam Roweis's 2008 version, which is
% slightly modified from Prasad Tadepalli's 2007 version which is a
% lightly changed version of the previous year's version by Andrew
% Moore, which was in turn edited from those of Kristian Kersting and
% Codrina Lauth. Alex Smola contributed to the algorithmic style files.
