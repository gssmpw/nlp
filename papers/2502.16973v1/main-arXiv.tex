\documentclass[fontsize=10pt]{article}
\usepackage[hyphens]{url}
\usepackage[a4paper, total={6in, 8in}]{geometry}
\usepackage{graphicx}
\usepackage[sort]{natbib}
\bibliographystyle{plainnat}

\usepackage[font=small,labelfont=bf]{caption}

\usepackage{mathtools}
\usepackage{amsthm}
\usepackage{amsmath}
\usepackage{amsfonts}
\usepackage{amssymb}
\usepackage{booktabs}
\usepackage{float}
\usepackage[ruled]{algorithm2e}

\usepackage[hidelinks,breaklinks]{hyperref}
\usepackage{cleveref}


\usepackage{authblk}

\usepackage{import}
\usepackage{preamble}

%% Theorem Environments
\newtheorem{theorem}{Theorem}
\Crefname{theorem}{Theorem}{Theorems}
\newtheorem{proposition}{Proposition}
\Crefname{proposition}{Proposition}{Propositions}
\newtheorem{corollary}[proposition]{Corollary}
\newtheorem{lemma}[proposition]{Lemma}
\Crefname{lemma}{Lemma}{Lemmata}
\newtheorem{example}[proposition]{Example}
\newtheorem{remark}[proposition]{Remark}
\Crefname{rem}{Remark}{Remarks}

\theoremstyle{definition}
\newtheorem{definition}[proposition]{Definition}
\Crefname{definition}{Definition}{Definitions}

\title{From Independence of Clones to Composition Consistency: \\ A Hierarchy of Barriers to Strategic Nomination}


\author[1,2]{Ratip Emin Berker}
\author[3]{S\'ilvia Casacuberta}
\author[3]{Isaac Robinson}
\author[4]{Christopher Ong}
\author[1,2,3]{Vincent Conitzer}
\author[5]{Edith Elkind}

\affil[1]{Carnegie Mellon University}
\affil[2]{Foundations of Cooperative AI Lab (FOCAL)}
\affil[3]{University of Oxford}
\affil[4]{Harvard University}
\affil[5]{Northwestern University}



\begin{document}

\maketitle

\begin{abstract}
    We study two axioms for social choice functions that capture the impact of similar candidates: independence of clones (IoC) and composition consistency (CC). We clarify the relationship between these axioms by observing that CC is strictly more demanding than IoC, and investigate whether common voting rules that are known to be independent of clones (such as STV, Ranked Pairs, Schulze, and Split Cycle) are composition-consistent. While for most of these rules the answer is negative, we identify a variant of Ranked Pairs that satisfies CC. Further, we show how to efficiently modify any (neutral) social choice function so that it satisfies CC, while maintaining its other desirable properties. Our transformation relies on the hierarchical representation of clone structures via PQ-trees. We extend our analysis to social preference functions. Finally, we interpret IoC and CC as measures of robustness against strategic manipulation by candidates, with IoC corresponding to strategy-proofness and CC corresponding to obvious strategy-proofness.
\end{abstract}

\subfile{sections/introduction}
\subfile{sections/background}
\subfile{sections/iocrules}
\subfile{sections/cctransform}
\subfile{sections/socialpf}
\subfile{sections/oioc}
\subfile{sections/discussion}


\section*{Acknowledgements}
We are grateful to Ariel Procaccia for his valuable contributions to the conceptualization of this project and helpful feedback. We also thank Felix Brandt and Jobst Heitzig for useful discussions. R.E.B. and V.C. thank the Cooperative AI Foundation, Polaris Ventures (formerly the Center for Emerging Risk Research) and Jaan Tallinn's donor-advised fund at Founders Pledge for financial support. R.E.B. is also supported by the Cooperative AI PhD Fellowship. 

\bibliography{refs}


\appendix

\subfile{appendix/app_background}
\subfile{appendix/app_iocrules}
\subfile{appendix/app_cc_transform}
\subfile{appendix/app_socialpf}
\subfile{appendix/app_oioc}

\end{document}

