\section{On \Cref{sec:oioc} (\nameref{sec:oioc})}

In this section, we provide the proofs omitted from \Cref{sec:oioc} of the main body, as well as a formal definition of extensive games and obviously-dominant strategies (in the restricted setting where each agent has a single information set).

\subsection{Proof of \Cref{prop:metric}}
Given $\profile$ over candidates $\cand$ with $|\cand|=m$, consider $d_{\profile}: B \times B \rightarrow [m] \cup \{0\}$ defined for each $a_i,a_j \in \cand$ as:
\begin{align*}
    d_{\profile}(a_i,a_j)= \min_{\substack{{\clone \subseteq \cand: a_i,a_j \in K,}\\{K\text{ is a clone set w.r.t. }\profile}}} |K| -1
\end{align*}

\clonemetric*

\begin{proof} We prove $d_{\profile}$ satisfies all axioms of a metric:
    \begin{itemize}
    \item (Zero distance to self) For each $a \in \cand$, $\{a\}$ is a clone set, so $d(a,a)=|\{a\}|-1=0$.
    \item (Positivity) If $a\neq b$, then any clone set $K$ that contains both of them must have $|K|\geq 2$, so $d(a,b) \geq 2-1=1>0$.
    \item (Symmetry) Clearly, $d(a,b)=d(b,a)$.
    \item (Triangle inequality) Given any $a,b,c \in A$, say $K_1$ is the clone set that includes $a,b$ with $|K_1|=d(a,b)+1$ and $K_2$ is the clone set that includes $b,c$ with $|K_2|=d(b,c)+1$. Since $b \in K_1 \cap K_2$, we have $K_1 \cap K_2 \neq \emptyset$ so by Axiom (A1) by \citet{Elkind10:Clone}, we have that $K_1 \cup K_2$ is a clone set. Notice $|K_1 \cup K_2|= |K_1|+|K_2| - |K_1 \cap K_2| \leq (d(a,b)+1) + (d(b,c)+1) - 1=d(a,b)+d(b,c)+1$. Since $a,c \in K_1 \cup K_2$, we have $d(a,c) \leq |K_1 \cup K_2|-1 \leq d(a,b)+d(b,c)+1-1=d(a,b)+d(b,c)$, satisfying triangle inequality. 
\end{itemize}
\end{proof}
\subsection{Proof of \Cref{prop:ioc_ds}}
Next, we prove that in the strategic candidacy setting where the preferences of candidates are dictates by $d_{\profile}$, IoC rules not only achieve but strengthen candidate stability.
\iocds*

\begin{proof}
Given any $a\in A$, say $u_a(S)$ is the utility of this player in $\Gamma_{\profile}^f$ when exactly the candidates in $S\subseteq A$ play $R$, and all candidates in $A \setminus S$ play $D$, which is a decreasing function of $d_{\profile}(a,f(\profile|_S))$ (and minimized at $u_a(\emptyset)$). Fix any $a \in \cand$ and a pure action for every other candidate. Say $S$ are the candidates among $\cand \setminus \{a\}$ that played $R$. To show that $R$ is a dominant strategy for $a$, we would like to show $u_a(S\cup\{a\}) \geq u_a(S)$. Consider three cases:
\begin{enumerate}
    \item Case 1: If $f(\profile|_{S \cup \{a\}})=\{a\}$, then $u_a(S \cup \{a\})>u_a(S)$ since $a \notin f(\profile|_{S})$, so $d_{\profile}(a,  f(\profile|_{S}))>0$, and we are done.
    \item Case 2: $f(\profile|_{S \cup \{a\}})=f(\profile|_{S})$, then $u_a(S \cup \{a\})=u_a(S)$ and we are done.
    \item Case 3: $S= \emptyset$. Then $u_a(S)$ is the minimizer of $u_a$ and $u_a(S \cup \{a\})=u_a(\{a\})$ is the maximizer, so we are done. 
    \item Case 4: If Cases 1-3 are false, we must have $f(\profile|_{S \cup \{a\}})=\{b\}$ and $f(\profile|_{S})=\{c\}$ for some $b \neq a$ and $c \neq b$. Take any clone set $K \subseteq \cand$ with respect to $A$ containing both $a$ and $c$; we would like to show $b \in K$ (which will automatically apply $d_{\profile}(a,b) \leq d_{\profile}(a,c)$). Say $K' = K \cap (S \cup \{a\})$, which is a clone set in $\profile|_{S \cup \{a\}}$ by \Cref{def:clones}. Since $f$ is IoC, we have
    \begin{align*}
        K' \cap f(\profile|_{S \cup \{a\}}) \neq \emptyset \Leftrightarrow K'\setminus\{a\} \cap f(\profile|_{S}) \neq \emptyset.
    \end{align*}
    Since the right hand side is true (as $c \in K'\setminus\{a\} \cap f(\profile|_{S})$), we must have $ K'\cap f(\profile|_{S \cup \{a\}}) \neq \emptyset$, implying $b \in K' \subseteq K$ and therefore $d_{\profile}(a,b) \leq d_{\profile}(a,c)$ and $u_a(S\cup\{a\}) \geq u_a(S)$.
\end{enumerate} 
\end{proof}

\subsection{Definitions for extensive-form games and obviously dominant strategies}\label{appsec:efg-ods}
In this section, we introduce extensive-form games and obviously dominant strategies, which we use to argue that CC exposes the obviousness of IoC.

\begin{definition}
    We can define an \emph{extensive-form game} $\Gamma$ as follows:
    \begin{enumerate}
        \item  $\Gamma$ is represented by a rooted tree structure. The set of all nodes in this tree is denoted by $\calH$ with each edge of the tree representing a single game \emph{action}. The game begins at the root, and each action traverses down the tree, until the game finishes at a leaf which we call a \emph{terminal node}. The set of terminal nodes is denoted by $\calZ \subset \calH$, and the set of actions available at any nonterminal node $h \in \calH \setminus \calZ$ is denoted by $A_h$.
        \item A finite set of strategic and chance players $|\calN \cup \{c\}| = N+1$ with $N \geq 1$. The set $\calN$ contains the \emph{strategic players}, and $c$ stands for a \emph{chance} ``player'' that models exogenous stochasticity. Each nonterminal node $h$ is assigned to either a strategic player or the chance player, who chooses an action to take from $A_h$. We call the set of nodes assigned to Player $i$ $\calH_i$.
        \item For each chance node $h \in \calH_c$, a probability distribution $\Prob_c(\cdot \mid h)$ on $A_h$ with which chance elects an action at $h$.
        \item For each strategic player $i \in \calN$, a (without loss of generality) nonnegative \emph{utility (payoff)} function $u_i : \calZ \to \RR_{\geq 0}$ which returns what $i$ receives when the game finishes at a terminal node. Player $i$ aims to maximize that utility.
        \item For each strategic player $i \in \calN$, a partition $\calH_i = \sqcup_{I \in \calI_i} I$ of the nodes of $i$ into information sets (\emph{infosets}). Nodes of the same infoset are considered indistinguishable to the player at that infoset. For that, we also require $A_h = A_{h'}$ for $h, h' \in I$. This also makes action set $A_I$ well-defined.
    \end{enumerate}
\end{definition}

\paragraph{Strategies and utilities.}

Players can select a probability distribution---a \emph{randomized action}---over the actions at an infoset. A (behavioral) \emph{strategy} $\pi_i$ of a player $i \in \calN$ specifies a randomized action $\pi_i(\cdot \mid I) \in \Delta(A_I)$ at each infoset $I \in \calI_i$. We say $\pi_i$ is \emph{pure} if it assigns probability 1 to a single action for each infoset. A (strategy) \emph{profile} $\pi = (\pi_i)_{i \in \calN}$ specifies a strategy for each player. We use the common notation $\pi_{-i} = (\pi_1, \dots, \pi_{i-1}, \pi_{i+1}, \dots, \pi_n)$. We denote the strategy set of Player $i$ with $S_i$, and $S = \bigtimes_{i \in \calN} S_i$.

We denote the reach probability of a node $h'$ from another node $h$ under a profile $\pi$ as $\Prob(h' \mid \pi, h)$. It evaluates to $0$ if $h \notin \seq(h')$, and otherwise to the product of probabilities with which the actions on the path from $h$ to $h'$ are taken under $\pi$ and chance. For any infoset, let $I^{\text{1st}}$ refer to the nodes $h\in I$ for which $I$ does not appear in $\obs(h)$. Then the reach probability of $I$ is $\Prob(I\mid \pi, h)\coloneqq \sum_{h' \in I^{\text{1st}}}\Prob(h' \mid \pi, h)$. We denote with $\U_i(\pi \mid h) \coloneqq \sum_{z \in \calZ} \Prob(z \mid \pi, h) \cdot u_i(z)$ the expected utility of Player $i$ given that the game is at node $h$ and the players are following profile $\pi$. Finally, we overload notation for the special case the game starts at root node $h_0$ by defining $\Prob(h \mid \pi) := \Prob(h \mid \pi, h_0)$ and $\U_i(\pi) \coloneqq \U_i(\pi \mid h_0)$.

We now introduce obviously dominant strategies. Since we will focus on games with no exogenous stochasticity (i.e., no chance nodes) and where every player will have a single infoset, our definition is a simplified version of the original definition by \citet{Li17:Obviously}.

\begin{definition}[{\citealt[Obviously Dominant Strategy]{Li17:Obviously}}]
Given an EFG $\Gamma$ with no chance nodes and a single infoset per player (i.e., $\calI_j =\{I_j\}$ for each $j \in \calN$) and a player $i \in \mathcal{N}$, an action $s \in A_{I_i}$ is obviously dominant if:

\[
\forall s' \in I_i: \quad \quad
\sup_{h \in I_i, \pi_{-i}} u_i((\pi_i^{s'}, \pi_{-i}) \mid h) \leq \inf_{h \in I_i, \pi_{-i}} u_i((\pi_i^{s}, \pi_{-i}) \mid h )
\]
where $\pi_i^s$ is the player $i$ strategy that plays action $s$ with probability 1. 
\end{definition}

Inutitively, an action $s$ is obviously dominant  for player $a$ if for any other action $s'$, \emph{starting from when $a$ must take an action}, the best possible outcome from $s'$ is no better than the worst possible outcome from $s$. The $\sup/\inf$ over $h \in I_i$ allows us to compare the best and worst possible for $i$ given what she knows at the point where she must act ($I_i$), and the $\sup/\inf$ over $\pi_{-i}$ allows us to best and worst possible outcomes based on the strategies of all other players (again, given that $I_i$ is reached), including those that have not acted yet. 

For example, $\Gamma^f_{\profile}$ from the main body of the paper can be interpreted as an EFG where players act simultaneously. In this case, even if the $f$ is IoC, running ($R$) is \emph{not} an obviously dominant strategy, due to the uncertainty of the actions of every other candidate: 
\begin{example}
    Consider $\Gamma^{\STV}_{\profile}$, where $\profile$ is from Figure~\ref{fig:eg_prof}. For $b$, the \emph{worst} outcome of running ($R$) is that every other candidate plays $R$ too, making $d$ the winner. The \emph{best} outcome of dropping out ($D$), on the other hand, is for $c$ to play $R$ and $d$ to play $D$, in which case $c$ wins regardless of what $a$ does. Since $2=d_{\profile}(b,c)< d_{\profile}(b,a)=4$, candidate $b$ strictly prefers the latter outcome, showing that $R$ is \emph{not} an obviously-dominant strategy for her, even though it is a dominant strategy by Prop.~\ref{prop:ioc_ds}. 
\end{example}

\subsection{Proof of \Cref{thm:cc_ods}}
Finally, we prove that in the process of implementing a rule $f^{CC}$, $\Lambda_{\profile}^f$ achieves obvious strategy-proofness for candidates.
\ccods*

\begin{proof}[Proof]
    Take any candidate $a\in A$ and consider the point in $\Lambda^f_{\profile}$ where $a$ must decide $R$ or $D$. This happens when Algo.~\ref{alg:cc-transform} is on the parent node of $a$, say $\node$. If $\node$ is a Q-node, the worst possible outcome of playing $R$ is $a$ winning herself, which is her optimal outcome, and hence the best outcome of $D$ cannot be any better. If is a P-node, then $\node$ is the smallest non-trivial clone set that contains $a$ by Lemma~\ref{lemma:pq_clones} (in other words, the members of $\node$ are exactly $a$'s second-favorite candidates after herself). Then the worst possible outcome of $a$ running ($R$) is some other candidate $b \in \node \setminus \{a\}$ winning (since Algo.~\ref{alg:cc-transform} will move out of $\node$ only if all the candidates in $\node$, including $a$, play $D$), whereas the best possible outcome of $a$ dropping out ($D$) is, again, some other candidate $c \in \node \setminus \{a\}$ winning. Since $d_{\profile}(a,b)=d_{\profile}(a,c)$ (both pairs are united by $\node$ as the smallest clone set), the latter outcome is no better than the former, proving  that $R$ is an obviously-dominant strategy for $a$. 
\end{proof}

