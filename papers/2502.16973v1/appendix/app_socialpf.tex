\section{On \Cref{sec:spf} (\nameref{sec:spf})}
In the main body of the paper, we have introduced (to the best of our knowledge) the first extension of the CC definition to social preference functions (SPFs), which returns a set of rankings over $A$ rather than a subset. We recall this definition, along with the for IoC for SPFs by \citet{Freeman14:Axiomatic} below. In the rest of this section, we prove our results about SPFs.

\iocspf*

Intuitively, IoC dictates that removing a member of a clone set should not change the rankings of all non-clones, as well as that of some highest ranked clone.

\ccspf*

In words, CC dictates that all clone sets in the $\decomp$ should appear as an interval in the order(s) specified by $F(\profile^\decomp)$, and the order(s) within each $K \in \decomp$ should be specified by $F(\profile|_K)$.


\subsection{Definitions of social preference functions}\label{appsec:spfdefs}

Here, we give the descriptions of the SPF versions of several SCFs we have discussed in previous sections. Below, the ``RP procedure'' refers to the process of locking in edges from $M$ in non-increasing order, skipping the ones that create a tie. 
\begin{itemize}
    \item $RP$: Return all rankings $r$ that correspond to the topological ordering of the final graph constructed by the RP procedure for \emph{some} tie-breaking order.
    \item  $RP_i$: Return the topological ordering of the final graph constructed by the RP procedure using $\Sigma_i$ as a tie-breaker.
    \item  $RP_N$: Return the union over $RP_i$ for all $i \in N$
    \item $STV$: At each round, eliminate the candidate with the least plurality votes, until only one candidate remains\footnote{Recall that as an SCF, it was sufficient to run $STV$ until a candidate secures majority (see \Cref{tab:scfs}). However, for the SPF version, we need to keep running the elimination process until there is a single candidate left, so we can get a complete order of elimination over the candidates.} Output the reverse order elimination 
    \item $BP$: Construct the strength matrix $S$ as described in \Cref{tab:scfs}. Define relationship $\succ_{BP}$ over candidates as $a \succ_{BP} b$ if $S[a,b]>S[b,a]$ and  $a =_{BP} b$ if $S[a,b]= S[b,a]$. As proven by \citet{Schulze10:New}, $\succeq_{BP}$ satisfies transitivity; hence, it gives a weak ordering over candidates $A$. Return all strict rankings $r$ that are consistent with the weak ordering of $\succeq_{BP}$. For example, for $\profile'$ with the strength matrix given in the extended proof of \Cref{thm:cc_fails} above, we have $a_1=_{BP}a_2 \succ_{BP} b \succ_{BP} c$. This implies $BP^*(\profile') =\{r_1,r_2\}$ where $BP^*$ is the SPF version of $BP$, with $r_1 : a_1 \succ a_2 \succ b\succ  c$ and $r_2 : a_2 \succ a_1 \succ b\succ  c$.
\end{itemize}
\subsection{Proof of \Cref{prop:SPFtoSCF}}
We first prove that our novel definitions of IoC/CC for SPFs are consistent with the ones for SCFs.

\spfscf*

\begin{proof}
    Say $F$ satisfies IoC, and pick any profile $\profile$, non-trivial clone set $K$ with respect to $\profile$, a $a\in K$. Note that for any ranking $r$ over $A$ and $b \in A \setminus K$, we have $b=top(r) \iff b=top(r_{\neg K \rightarrow z})$, since relabeling/adding/removing lower ranked candidates does not change the fact that $b$ is ranked top. Hence, we have:
 \begin{align*}
     b\in f(\profile) \iff   \exists r \in F(\profile)\text{ s.t. } b=top(r) &\iff  \exists r \in F(\profile)_{K \rightarrow z}\text{ s.t. } b=top(r) \\&\overset{(IoC)}{{\iff}}   \exists r \in F(\profile \setminus \{a\})_{(K \setminus \{a\})\rightarrow z}\text{ s.t. } b=top(r)\\
     &\iff \exists r \in F(\profile \setminus \{a\})\text{ s.t. } b=top(r)\\& \iff b \in f(\profile\setminus\{a\})
 \end{align*}
 which give proves $f$ satisfies condition (2) from \Cref{def:ioc}. Next, notice that by definition of the $\neg$ operator, for any ranking $r$ we have $top(r) \in K \iff z=top(r_{\neg K \rightarrow z})$. This implies:
 \begin{align}
     K \cap f(\profile) \neq \emptyset  \iff \exists r \in F(\profile)\text{ s.t. } top(r) \in K &\iff \exists r \in F(\profile)_{\neg K \rightarrow z}\text{ s.t. } z=top(r)
     \\&\overset{(IoC)}{{\iff}}   \exists r \in F(\profile \setminus \{a\})_{(K \setminus \{a\})\rightarrow z}\text{ s.t. } z=top(r)\\
     &\iff \exists r \in F(\profile \setminus \{a\})\text{ s.t. } top(r) \in K\setminus\{a\} \\& \iff  (K\setminus\{a\}) \cap  f(\profile\setminus\{a\}) \neq \emptyset
 \end{align}
 which give proves $f$ satisfies condition (1) from \Cref{def:ioc}. Hence, $f$ is IoC.

 Next, assume $F$ satisfies CC. Take any profile $\profile$ and clone decomposition $\decomp$ with respect to $\profile$. By \Cref{def:cc_spf}, we have:
 \begin{align*}
     F(\boldsymbol{\sigma}) = F(\boldsymbol{\sigma}^\decomp)(K \rightarrow F(\boldsymbol{\sigma}|_\clone)\text{ for }\clone \in \decomp) \Rightarrow f(\profile) = top(F(\profile)) &= \bigcup_{K \in top(F(\profile^\decomp))}top(F(\profile|_{K}))\\&=\bigcup_{K \in f(\profile^\decomp)}f(\profile|_{K})
 \end{align*}
Hence, we have $f(\profile)=\gloc_f(\profile,\decomp)$, so $f$ satisfies CC.
\end{proof}

\subsection{(Nested) nested runoff voting}\label{appsec:nested}

In this section, we briefly discuss how non-anonymous tie-breakers (such as the one introduced by \citet{Zavist89:Complete}) can be used to construct CC SPFs other than $RP_i$. \citet{Freeman14:Axiomatic} introduce an SPF named Nested Runoff ($NR$), which is a modification of $\STV$: at each round, instead of the candidate with the lowest plurality score, the winner of $\STV(\rev(\profile))$ is eliminated, where $\rev(\profile)$ is $\profile$ with every voter's ranking reversed. \citeauthor{Freeman14:Axiomatic} show that $NR$ is IoC as an SPF. By our \Cref{prop:SPFtoSCF}, this implies $NR$ is IoC as an SCF too. However, since it is anonymous, it cannot be CC as an SPF by \Cref{thm:anonspf}. In fact, $NR$ is not CC as an SCF either; to see this, consider the following profile over 4 candidates with 3 voters:
\begin{align}
    \profile= \begin{cases}
        b \succ_1 a_2 \succ_1 a_1 \succ_1 c \\
        a_2 \succ_2 a_1 \succ_2 c \succ_2 b \\
        c \succ_3 b \succ_3 a_1 \succ_3 a_2 \label{eq:nested-counter}
    \end{cases}
\end{align}
Say $\decomp=\{\{a_1,a_2\}, \{b\}, \{c\}\}$. It can be checked that $a_1 \in NR(\profile)$ but $a_1 \notin \gloc_{NR}(\profile, \decomp)$, which violates CC. 

Now, say $STV_i$ is simply the version of $STV$ that uses voter $i$'s vote as a tie-breaker (\emph{i.e.}, if multiple candidates tie for the lowest plurality score at any point, the one ranked lowest by voter $i$ is eliminated). It is straightforward to check that, much like $\STV$, $\STV_i$ is IoC as an SPF and as an SCF, but CC as neither. Unlike $\STV$, however, $\STV_i$ is decisive on all $\profile$. Now, we define $NR_i$ using $\STV_i$ on the reverse profile to decide on the order of elimination. We will show that $NR_i$ is CC as an SCF. Take any profile $\profile$ and any decomposition $\decomp$ with respect to $\profile$. Since we are using a decisive tie-breaker, we will have $|NR_{i}(\profile)|= |\gloc_{NR_i}(\profile, \decomp)|=1$, so it is sufficient to show containment in one direction. Say $\gloc_{NR_i}(\profile, \decomp)=\{a\}$ and $K_a \in \decomp$ is the clone set containing $a$. This implies $NR_i(\profile^{\decomp}) = \{K_a\}$ and $NR_i(\profile|_{K_a})=\{a\}$. Say $K_1,K_2,\ldots,K_{\ell}$ is the order in which clone sets are eliminated when $NR_i$ is run on $\profile^{\decomp}$. This implies $STV_i(\rev(\profile^{\decomp}))=\{K_1\}$. By successive application of the IoC for SCF property, we must have $STV_i(\rev(\profile))=\{b_1\}$ for some $b_1 \in K_1$, implying that the first candidate eliminated by $NR_i$ on input $\profile$ belongs to $K_1$. If $|K_1|>1$, this argument can be repeated again with $STV_i(\rev(\profile \setminus\{b_1\}))$, implying the next eliminated candidate too will belong to $K_1$. Applying this argument repeatedly gives us that $NR_i$ on input $\profile$ will eliminate all elements of $K_1$ before any other candidate. Now, by the assumption on the order of elimination in $NR_i(\profile^{\decomp})$ we have $STV_i(\rev(\profile^{\decomp} \setminus K_1))=\{K_2\}$, and the same IoC argument can be applied to show that $STV_i(\rev(\profile \setminus K_1))=\{b_2\}$ for some $b_2 \in K_2$. Inductively applying this argument gives that $NR_i$ will eliminate all candidates of $K_i$ before any candidate of $K_{i+1}$ for all $i \in [\ell]$, where $K_{\ell+1}=K_a$ (since it never gets eliminated). Hence, at some point in the execution of $NR_i$ on $\profile$, we will have $\profile \setminus \left( \bigcup_{i \in [\ell]} K_i \right)$ left. However, this is precisely $\profile|_{K_a}$, and by assumption we have $NR_i(\profile|_{K_a})=\{a\}$, showing that we must indeed have $NR_i(\profile)=\{a\}$, proving that $NR_i$ is CC as an SCF. 

To see that $NR_i$ is not CC as an SPF, once again consider the profile from \eqref{eq:nested-counter} and $NR^*_{3}$ (\emph{i.e.}, the SPF version of $NP_i$ using $i=3$ as the tie-breaker). It can be checked that $NR^*_3(\profile)= c \succ b \succ a_1 \succ a_2$, but $NR^*_3(\profile|_{\{a_1,a_2\}})=a_2 \succ a_1$, hence violating CC as an SPF. Hence, $NR_i$ serves as a ``natural'' counterexample showing that the reverse of \Cref{prop:SPFtoSCF} does not always hold. 

Finally, let us use $NR_i$ to design a CC SPF. Define the \emph{nested nested runoff rule} using $i$ as a tie-breaker ($NNR_i$) as a modification of $NR_i$ that, instead of $STV_i$, runs $NR_i$ on the reverse profile to decide the next eliminated candidate. Given any profile $\profile$ and decomposition $\decomp$, say $NNR_i(\profile^{\decomp})=K_\ell \succ K_{\ell-1}\succ \ldots \succ K_2 \succ K_1$. This implies $NR_i(\rev(\profile^{\decomp}))=\{K_1\}$. Since $NR_i$ is CC as an SCF, it is IoC as an SCF (by \Cref{prop:cctoioc}). Hence, we must have $NR_i(\rev(\profile))=\{b_1\}$ for some $b_1 \in K_1$. By the CC property of $NR_i$, this implies $NR_i(\rev(\profile)|_{K_1})=\{b_1\}$, implying that the candidate ranked at the bottom of $NNR^*_i(\profile)$ is $\{b_1\}$. If $|K_1|>1$, applying the same argument again gives us $NR_i(\rev(\profile \setminus \{b_1\} ))= NR_i(\rev(\profile|_{K_1} \setminus \{b_1\}))$. Thus, all the candidates in $K_1$ appear in the bottom of $NNR_i(\profile)$, and they appear exactly in the order they do in $NNR_i(\profile)$. Applying this argument inductively to all $K_i$ for $i \in [\ell]$ gives us exactly the CC definition for SPFs (\Cref{def:cc_spf}), completing the proof.

Hence, we have arrived at an interesting hierarchy. $STV_i$ is IoC as an SPF and an SCF, but CC as neither. $NR_i$, which uses $STV_{i}$ to eliminate candidates, is CC as an SCF, but still only IoC as an SPF. Lastly, $NNR_i$, which uses $NR_i$ to eliminate candidates, is CC both as an SPF and an SCF. Based on this observation, we believe studying the axiomatic properties of this type of (nested) nested rules is an interesting future direction. 

\subsection{Proof of \Cref{prop:cctoiocspf}}
We first prove that the CC to IoC relationship extends to the definitions for SCFs we have introduced.

\spfccioc*

\begin{proof}
    Say $F$ satisfies CC. By \Cref{def:cc_spf}, this implies that $F$ is neutral. Pick any profile $\profile$, non-trivial clone set $K$ with respect to $\profile$, a $a\in K$. Consider the clone decomposition $\decomp = \{\clone\} \cup \{\{b\}\}_{b\in \cand \setminus \clone}$ for $\profile$ and the clone decomposition $\decomp' = \{ \clone\setminus \{a\} \} \cup \{\{b\}\}_{b\in \cand \setminus \clone}$ for $\profile \setminus \{a\}$ (i.e., the decomposition which groups all existing members of $\clone$ together, and everyone else is a singleton). Notice that $\profile^\decomp$ and  $(\profile\setminus\{a\})^\mathcal{K'}$ are identical except the meta-candidate for $\clone$ in the former is replaced with the meta-candidate for $\clone \setminus \{a\}$ in the latter. By neutrality, this implies: $F(\profile^\decomp)_{\neg \{K\} \rightarrow K\setminus\{a\}}=F((\profile\setminus\{a\})^\mathcal{K'})$. Each $K'\in \decomp \setminus\{K\}$ is a singleton, and hence $F(\profile|_K)$ is just a single ranking with the only element in $K'$. For any $b \in A$, say $r_b$ is the trivial ranking over $\{b\}$. 
    Using CC, we get
    \begin{align*}
        F(\boldsymbol{\sigma})\neg_{{\clone \to z}} &= \left(F(\boldsymbol{\sigma}^\decomp)(K' \rightarrow F(\boldsymbol{\sigma}|_{\clone'})\text{ for }\clone' \in \decomp)\right)\neg_{{\clone \to z}}\\
        &= \left(F(\boldsymbol{\sigma}^\decomp)(K \rightarrow F(\boldsymbol{\sigma}|_{\clone}); \{b\} \rightarrow r_b \text{ for }b \in A\setminus K)\right)\neg_{{\clone \to z}}\\
        &= F(\boldsymbol{\sigma}^\decomp)(K \rightarrow r_z; \{b\} \rightarrow r_b \text{ for }b \in A\setminus K)\\
        &=F((\profile\setminus\{a\})^\mathcal{K'})( (K\setminus\{a\}) \rightarrow r_z; \{b\} \rightarrow r_b \text{ for }b \in A\setminus K)\\
        &=\left(F((\profile\setminus\{a\})^\mathcal{K'})( (K\setminus\{a\}) \rightarrow F(\profile|_{K \setminus\{a\}}); \{b\} \rightarrow r_b \text{ for }b \in A\setminus K)\right)_{(K \setminus\{a\})\rightarrow z}\\
        &=F(\profile\setminus\{a\})_{(K \setminus\{a\})\rightarrow z}.
    \end{align*}
Hence, $F$ satisfies IoC.
\end{proof}

\subsection{Proof of \Cref{thm:spftaxonomy}}
We now prove that for SCFs for which we described the SPF version above, our results generalize. 

\spftaxonomy*

\begin{proof}
We prove the axioms satisfied by each SPF as a seperate Lemma. 
\begin{lemma}\label{lemma:spfstv}
    The SPF version of STV is IoC but not CC.
\end{lemma}
\begin{proof} 

The fact that SPF version of $\STV$ is IoC is shown by \citet{Freeman14:Axiomatic}. Since SCF version of $STV$ is not CC (\Cref{thm:cc_fails}), then the SPF version of $STV$ is also not CC by \Cref{prop:SPFtoSCF}.
\end{proof}

\begin{lemma}\label{lemma:spfbp}
    The SPF version of $BP$ is IoC but not CC.
\end{lemma}
\begin{proof} 

 Take any profile $\profile$, clone set $K$, and $a\in K$. Say $S$ and $S'$ (resp. $M$ and $M'$) are the strength (resp. majority) matrices that result from running the $BP$ procedure on $\profile$ and $\profile\setminus\{a\}$, respectively. First, notice that for any $b,c \in A \setminus \{a\}$, we have $M[b,c] =M'[b,c]$, since the removal of candidate does not change the pairwise relationship between the remaining candidates. Take any $x \in A\setminus\{a\}$ and $y,z \in A \setminus K$. We would like to show that:
 \begin{align}
     S[x,y]=S'[x,y] \quad   S[y,x]=S'[y,x] \quad S[y,z]=S'[y,z]  \label{eq:strengths}
 \end{align}
 Since $M'$ is simply $M$ with $a$ removed, any path in $M'$ exists in $M$. This gives you the $\geq$ direction of all of the equalities in (\ref{eq:strengths}). For the reverse direction, consider any path $P$ from $x$ to $y$ in $M$. If the path does not contain $a$, then it exists in $M'$. If it does contain $a$, consider the alternative path $P'$ that starts from the last element belonging to $K$ in $P$, but replaces it with $x$ (so $P'$ is also a path from $x$ to $y$). By the clone definition, the first edge in the path is equally strong, and the remaining edges are the same. Since the strength of a path is the minimum weight over the edges in the path, this shows that $P'$ is at least as strong as $P$. The same method can be applied for paths from $y$ to $x$ by replacing the first occurrence of a member of $K$ with $x$. Now take any path $P$ in $M$ from $y$ to $z$. Again, if it does not contain $a$, it still exists in $M'$. If it does contain it, then pick any $b \in K \setminus \{a\}$ (exists sicne $K$ is non-trivial) and construct path $P'$ by replacing the interval in $P$ from the first occurrence of a member of $K$ to the last occurrence of a member of $K$ with $b$. By the clone definition, the incoming and outgoing edge of $b$ will have the same weight as the incoming and outgoing edge to this interval. Since the remaining paths are only subtracted, again $P'$ is at least as strong as $P$. This finishes the  $\leq$ direction of all of the equalities in (\ref{eq:strengths}). 

This implies that for any $b,c \in A \setminus\{a\}$ such that at least one of them is not in $K$, we will have $b \succeq_{BP} c \iff b \succeq'_{BP} c$, where $\succeq_{BP}$ and $\succeq'_{BP}$ are the (weak) linear orderings resulting from running $BP$ on $\profile$ and $\profile \setminus \{a\}$, respectively. This implies that $ BP^*(\profile)\neg_{K \rightarrow z}= BP^*(\profile\setminus\{a\})\neg_{ (K \setminus\{a\}) \rightarrow z}$, proving that $BP^*$ (the SPF version of $BP$) is IoC.

Since SCF version of $BP$ is not CC (\Cref{thm:cc_fails}), then $BP^*$ is also not CC by \Cref{prop:SPFtoSCF}.
\end{proof}

\begin{lemma}\label{lemma:spfrp}
    The SPF version of $RP$ neither IoC nor CC.
\end{lemma}
\begin{proof}
    Since the SCF version of $RP$ is not IoC (\Cref{prop:rpfailioc}) and therefore not CC (by \Cref{prop:cctoioc}), SPF version of $RP$ is neither IoC nor CC \Cref{prop:SPFtoSCF}.
\end{proof}

\begin{lemma}\label{lemma:spfrpi}
    The SPF version of $RP_i$ is both IoC and CC.
\end{lemma}
\begin{proof}
    The proof that $RP_i$ satisfies CC follows easily from the proof of \Cref{thm:rp}. There, (using \Cref{lemma:stack_winner}) we showed that given any decomposition $\decomp$ the RP ranking resulting from running $RP_i$ on $\profile$ has each  clone set in $\decomp$ as an interval, in the order specified by the RP ranking resulting from running $RP_i$ on $\profile^\decomp$. Moreover, we showed that the clone set ranked first in this ranking (say $K_1$) appeared in the order  specified by the RP ranking resulting from running $RP_i$ on $\profile|_{K_1}$. This last proof did not use the fact that $K_1$ was the first clone set to appear in the ranking, but only that it appeared as an interval. Hence, the same proof can be easily applied to all $K \in \decomp$, since each appear as an interval. As a result, we have $RP^*_i(\profile)= RP^*_i(\profile^\decomp)(K \rightarrow RP^*_i(\profile)\text{ for }K\in \decomp)$, where $RP^*_i$ is the SPF version of  $RP_i$.
\end{proof}

\begin{lemma}\label{lemma:spfrpn}
    The SPF version of $RP_N$ is IoC but not CC.
\end{lemma}
\begin{proof}
    Say $RP^*_N$ and $RP^*_i$ are the SPF versions of $RP_N$ and $RP_i$, respectively. Fix any profile $\profile$, non-trivial clone set $K$, and $a \in K$. Since $RP^*_i$ is IoC for each $i \in N$ by \Cref{lemma:spfrpi} and by definition of the $\neg$ operator, we have
    \begin{align*}
        RP^*_N(\boldsymbol{\sigma}) \neg_{\clone \to z}&=
       \left(\bigcup_{i \in N}  RP^*_i(\boldsymbol{\sigma}) \right) \neg_{\clone \to z}
       = \bigcup_{i \in N}  RP^*_i(\boldsymbol{\sigma}) \neg_{\clone \to z} = \bigcup_{i \in N}  RP^*_i(\boldsymbol{\sigma}) \neg_{\clone \to z}\\&=\bigcup_{i \in N}  RP^*_i(\boldsymbol{\sigma} \setminus \{a\})\neg_{(\clone \setminus \{a\})\to z}=\left(\bigcup_{i \in N}  RP^*_i(\boldsymbol{\sigma} \setminus \{a\})\right)\neg_{(\clone \setminus \{a\})\to z}\\&=RP^*_N(\boldsymbol{\sigma} \setminus \{a\}) \neg_{(\clone \setminus \{a\})\to z},
    \end{align*}
    proving $RP^*_N$ satisfies IoC. Since the SCF version of $RP_N$ does not satisfy CC (\Cref{prop:rpn}), this proves that  $RP^*_N$ is not CC by \Cref{prop:SPFtoSCF}.
\end{proof}
\Cref{lemma:spfstv,lemma:spfbp,lemma:spfrp,lemma:spfrpi,lemma:spfrpn}, together with our result from \Cref{thm:cc_fails,thm:rp}, prove the theorem statement.
\end{proof}

    

\subsection{Proof of \Cref{thm:anonspf}}
We next prove a surprising negative resulting showing the incompatibility of anonymity and composition consistency for SPFs.

\anonspf*

\begin{proof}
    Assume for the sake of contradiction that we have an SPF $F$ that is CC and anonymous. By \Cref{def:cc_spf}, this implies $F$ is also neutral. Consider a profile $\profile$ over $\cand=\{a,b,c\}$ with two votes. Voter 1 ranks $a \succ_1 b \succ_1 c$ and Voter 2 ranks $c \succ_2 b \succ_2 a$. Define $K_1=\{a,b\}$ and $K_2=\{b,c\}$, which are both clone sets with respect to $\profile$. Further, define $\decomp_1=\{K_1,\{c\}\}$ and $\decomp_2=\{
    \{a\},K_2\}$, which are both clone decompositions with respect to $\profile$. Consider $\profile^{\decomp_1}$, which consists of $K_1 \succ_1 \{c\}$ and $\{c\} \succ_2 K_1$. Since $F$ is neutral and anonymous, we must have  $F(\profile^{\decomp_1})=\{K_1\succ \{c\}, \{c\} \succ K_1\}$, otherwise permuting Voter 1 with Voter 2 and $K_1$ with $\{c\}$ gives a contradiction. By the same reasoning, we must have $F(\profile|_{K_1})=\{a\succ b, b\succ a\}$. By composition consistency (\Cref{def:cc_spf}), we must have
    \begin{align}
        F(\boldsymbol{\sigma}) = F(\boldsymbol{\sigma}^{\decomp_1})(K \rightarrow F(\boldsymbol{\sigma}|_\clone)\text{ for }\clone \in \decomp_1)=\left\{\!\begin{aligned}a\succ b \succ c,\\ b\succ a \succ c, \\c\succ a \succ b, \\c\succ b \succ a\end{aligned}\right\}.\label{eq:anonspf1}
    \end{align}
    Similarly, $\profile^{\decomp_2}$, which consists of $\{a\} \succ_1 K_2$ and $K_2 \succ_2 \{a\}$. By neutrality and anonymity, we must have  $F(\profile^{\decomp_2})=\{\{a\} \succ_1 K_2, K_2 \succ_2 \{a\}\}$, otherwise permuting Voter 1 with 2 and $\{a\}$ with $K_2$ gives a contradiction. By the same reasoning, we must have $F(\profile|_{K_2})=\{b\succ c, c\succ b\}$. By CC, we must have
    \begin{align}
        F(\boldsymbol{\sigma}) = F(\boldsymbol{\sigma}^{\decomp_1})(K \rightarrow F(\boldsymbol{\sigma}|_\clone)\text{ for }\clone \in \decomp_1)=\left\{\!\begin{aligned}a\succ b \succ c,\\ a\succ b \succ c, \\b\succ c \succ a, \\c\succ b \succ a\end{aligned}\right\}.\label{eq:anonspf2}
    \end{align}
    Comparing \eqref{eq:anonspf1} with \eqref{eq:anonspf2}, we immediately get a contradiction.
\end{proof}

