\section{On \Cref{sec:ioc_rules} (\nameref{sec:ioc_rules})}

In this section, we provide the proofs omitted from \Cref{sec:ioc_rules} of the main body.
\subsection{Proof of \Cref{prop:cctoioc}}

We first prove the relationship between CC and IoC:
\cctoioc*
\begin{proof}
For a CC rule $f$, take any profile $\profile$ over candidates $\cand$, non-trivial clone set $\clone \subset \cand$, and candidate $a\in \cand$.
Consider the clone decomposition $\decomp = \{\clone\} \cup \{\{b\}\}_{b\in \cand \setminus \clone}$ for $\profile$ and the clone decomposition $\decomp' = \{ \clone\setminus \{a\} \} \cup \{\{b\}\}_{b\in \cand \setminus \clone}$ for $\profile \setminus \{a\}$ (i.e., the decomposition which groups all existing members of $\clone$ together, and everyone else is a singleton). Notice that $\profile^\decomp$ and  $(\profile\setminus\{a\})^\mathcal{K'}$ are identical except the meta-candidate for $\clone$ in the former is replaced with the meta-candidate for $\clone \setminus \{a\}$ in the latter. Since $f$ is neutral by \Cref{def:oioc}, this implies:
\begin{align}
    \clone \in f\left(\profile^\decomp\right) &\iff  \clone \setminus\{a\} \in f\left((\profile\setminus\{a\})^\mathcal{K'}\right) \label{eq:cond1s}\\
    \forall b \in \cand \setminus C: \quad \quad& \nonumber \\\{b\} \in f\left(\profile^\decomp\right) &\iff  \{b\} \in f\left((\profile\setminus\{a\})^\mathcal{K'}\right) \label{eq:cond2s}
\end{align}
    
    By Definition \ref{def:gloc}, it is easy to see that for any $K \in \decomp$, we have $K \cap \gloc_f(\profile, \decomp) \neq \emptyset \iff K \in f\left(\profile^\decomp\right)$. Based on this, (\ref{eq:cond1s}) and  (\ref{eq:cond2s}) respectively imply:
    \begin{align}
    \clone \cap \gloc_f(\profile, \decomp) \neq \emptyset &\iff  \clone \setminus\{a\}\cap \gloc_f(\profile \setminus \{a\}, \decomp') \neq \emptyset\label{eq:cond1f}\\
  \forall b \in \cand \setminus \clone:\quad \quad& \nonumber \\  b \in \gloc_f(\profile, \decomp) &\iff  b \in \gloc_f(\profile \setminus \{a\}, \decomp')\label{eq:cond2f}
\end{align}
Since $f$ is CC, we have $\gloc_f(\profile, \decomp) = f(\profile)$ and $\gloc_f(\profile\setminus\{a\}, \decomp') = f(\profile\setminus\{a\})$ so (\ref{eq:cond1f}) and (\ref{eq:cond2f}) respectively imply conditions 1 and 2 in Definition \ref{def:ioc}, proving that $f$ is IoC.
\end{proof}


\subsection{(Extended) Proof of \Cref{thm:cc_fails}}\label{appsec:extended_ioc}

The proof of \Cref{thm:cc_fails} is given in the main body of the paper, but the winners under each SCF are stated without detailed calculations. Here, we give a more extensive proof that walks through the implementation of each SCF.

\ccfails*
\begin{proof}
Since CC implies $f=\gloc_f(\profile, \decomp)$ for \textit{all} profiles $\profile$ and clone decomposition $\decomp$, a single counterexample is sufficient to show a rule failing CC.

For $\STV$ and $\AS$, we will be using $\profile$ over $\cand = \{a_1,a_2,b,c\}$ from \Cref{fig:bg_eg}. Consider $\decomp=\{K_a,K_b,K_c\}$, with $K_a=\{a_1,a_2\}$, $K_b=\{b\}$, and $K_c=\{c\}$. Figure \ref{fig:bg_eg_2} shows $\profile^\decomp$ and $\profile|_{K_a}$. The procedure for $\STV$ is detailed in \Cref{eg:ioc,eg:GLOC}. To run $\AS$ on $\profile$, we will alternate between eliminating all non-Smith candidates and eliminating the candidate with the least plurality score:
\begin{itemize}
    \item First, we have $\Sm(\profile)=A$ due to the cyclicity of the profile, hence no one gets eliminated.
    \item Then, we eliminate $a_2$, the candidate with the least plurality score.
    \item Once again, $\Sm(\profile \setminus\{a_2\})=\cand \setminus\{a_2\}$, so no candidate is eliminated.
    \item $c$ gets eliminated as the next candidate with least plurality scores.
    \item $\Sm(\profile\setminus\{b,a_2\})=\{a_1\}$ since $a_1$ pairwise defeats $b$.
\end{itemize}
Therefore $\AS(\profile)=\{a_1\}$. Running $\AS$ on $\profile^\decomp$, on the other hand, we get:
\begin{itemize}
    \item We have $\Sm(\profile^\decomp)=\decomp$ due to the cyclicity of the profile, so no (meta-)candidate is eliminated.
    \item Then, we eliminate $K_c$, the candidate with the least plurality score.
    \item $\Sm(\profile^\decomp \setminus\{K_c\})=\{K_a\}$ since $K_a$ pairwise defeats $K_b$.
\end{itemize}
Therefore $\AS(\profile^\decomp)=\{K_a\}$. Further, $\AS(\profile|_{K_a})=\{a_2\}$ since $a_1$ is pairwise defeated by $a_2$ and therefore eliminated in the first step. Thus, we have $\AS(\profile)=\{a_1\} \neq \{a_2\} = \gloc_{\AS}(\profile,\decomp)$, proving $\AS$ is not CC. For both $\STV$ and $\AS$, the main idea is that $a_2$ gets eliminated first even though it is a majority winner over $a_1$, since the few voters that prefer $a_1$ over $a_2$ happens to put them to the top of their ballot, giving $a_1$ more plurality votes than $a_2$. 


For $BP$ and $SC$, consider the following profile $\profile'$ (the same as the one from \Cref{fig:eg_prof}, with relabeled candidates):

\begin{center}
\begin{tabular}{|c|c|c|c|}
    \hline
    6 voters & 5 voters & 2 voters & 2 voters \\ \hline
    \cellcolor{yellow!25} $a_1$ & \cellcolor{blue!25} $c$ & \cellcolor{green!25} $b$ & \cellcolor{green!25} $b$ \\ \hline
    \cellcolor{red!25} $a_2$ & \cellcolor{red!25} $a_2$ & \cellcolor{blue!25} $c$ & \cellcolor{blue!25} $c$ \\ \hline
    \cellcolor{green!25} $b$ & \cellcolor{yellow!25} $a_1$ & \cellcolor{yellow!25} $a_1$ & \cellcolor{red!25} $a_2$ \\ \hline
    \cellcolor{blue!25} $c$ & \cellcolor{green!25} $b$ & \cellcolor{red!25} $a_2$ & \cellcolor{yellow!25} $a_1$ \\
    \hline
\end{tabular}
\end{center}

To find $BP(\profile')$ and $SC(\profile')$, we construct the majority matrix $M_{\profile'}$ below:

\begin{center}
\begin{tabular}{|c|c|c|c|c|}
    \hline
    $M_{\profile'}$ & \cellcolor{yellow!25} $a_1$ & \cellcolor{red!25} $a_2$ & \cellcolor{green!25} $b$ & \cellcolor{blue!25} $c$ \\ \hline
    \cellcolor{yellow!25} $a_1$ & $0$ & $1$ & $7$ & $-3$ \\ \hline
    \cellcolor{red!25} $a_2$ & $-1$ & $0$ & $7$ & $-3$ \\ \hline
    \cellcolor{green!25} $b$ & $-7$ & $-7$ & $0$ & $5$ \\ \hline
    \cellcolor{blue!25} $c$ & $3$ & $3$ & $-5$ & $0$ \\ 
    \hline
\end{tabular}
\end{center}

Notice that there are 3 simple cycles in $M_{\profile'}$: $(a_1,b,c)$, $(a_2,b,c)$, and $(a_1,a_2,b,c)$, with smallest margins $[c,a_1]$, $[c,a_2]$, and $[a_1,a_2]$, respectively. Removing these three edges form the graph leaves $a_1$ and $a_2$ without any incoming edges, indicating $SC(\profile')=\{a_1,a_2\}$. 

Similarly, $M_{\profile'}$ induces the following strength matrix $S_{\profile'}$, which shows that $BP(\profile')=\{a_1,a_2\}$, since $S[x,y] \geq S[y,x]$ for $x\in \{a_1,a_2\}$ and all $y\in \cand$. 
\begin{center}
\begin{tabular}{|c|c|c|c|c|}
    \hline
    $S_{\profile'}$ & \cellcolor{yellow!25} $a_1$ & \cellcolor{red!25} $a_2$ & \cellcolor{green!25} $b$ & \cellcolor{blue!25} $c$ \\ \hline
    \cellcolor{yellow!25} $a_1$ & $0$ & $3$ & $7$ & $5$ \\ \hline
    \cellcolor{red!25} $a_2$ & $3$ & $0$ & $7$ & $5$ \\ \hline
    \cellcolor{green!25} $b$ & $3$ & $3$ & $0$ & $5$ \\ \hline
    \cellcolor{blue!25} $c$ & $3$ & $3$ & $3$ & $0$ \\ 
    \hline
\end{tabular}
\end{center}

However, using clone decomposition $\decomp$ from above (which is also a valid decomposition with respect to $\profile'$), the graph for $M_{\profile'^\decomp}$ is composed of a single simple cycle, with $M[K_a,K_b]=7$, $M[K_b,K_c]=5$ and $M[K_c,K_a]=3$. Clearly, we have $SC(\profile'^\decomp)=BP(\profile^\decomp)=\{K_a\}$. However, $a_1$ is the majoritarian winner against $a_2$ in $\profile'|_{K_a}$, without any cycles. Hence $\prod_{SC}(\profile', \decomp)=\prod_{BP}(\profile', \decomp)=\{a_1\}$, showing both rules fail CC. Intuitively, both $BP$ and $SC$, while picking their winners for $\profile'$, `discard' the relationship between $a_1$ and $a_2$, $BP$ since neither the strongest path from $a_1$ to $a_2$ nor vice versa go through the $(a_1,a_2)$ edge, and $SC$ since the $(a_1,a_2)$ edge forms the weakest margin in a 4-candidate cycle. As a result, both rules pick both candidates as winner, even though they both agree $a_1$ wins over $a_2$ when applied to $\profile'|_{K_a}$ alone.
\end{proof}

\subsection{Proof of \Cref{thm:rp}}\label{appsec:rpi}
We now prove our main positive characterization result from \Cref{sec:ioc_rules}.

\rpcc*


Without loss of generality, fix $1\in N$. We will show that $RP_1$ satsifes CC. The same proof follows for $RP_i$ for any $i \in N$. We write $\{a,b\} \succ_{\Sigma_1} \{c,d\}$ if $\Sigma_1$ ranks $\{a,b\}$ before $\{c,d\}$. \citeauthor{Zavist89:Complete} show that $\Sigma_1$ is \textit{impartial}; that is, for all $a,b,c,d\in A$, if $\{a,c\}\succ_{\Sigma_1}\{b,c\}$ then  $\{a,d\}\succ_{\Sigma_1}\{b,d\}$. 
Using $\Sigma_1$, we construct a complete \emph{priority order} $\mathcal{L}$ over ordered pairs: pairs are ordered (in decreasing order) according to $M$, and ties are broken by $\Sigma_1$ (and according to $\sigma_1$ when $M[a,b]=0$). Formally, given distinct ordered pairs $(a,b)$ and $(c,d)$ such that $(c,d) \neq (b,a)$, we have:
\begin{align*}
    (a,b) \succ_{\mathcal{L}} (c,d) & \text{ iff:} \begin{cases} M[a,b]> M[c,d] \text{ or }\\  M[a,b]= M[c,d], \{a,b\} \succ_{\Sigma_1} \{c,d\},\end{cases}
\end{align*}
and if $(c,d)=(b,a)$ we have:
\begin{align*}
    (a,b) \succ_{\mathcal{L}} (b,a) & \text{ iff:}\begin{cases} M[a,b]> M[b,a] \text{ or }\\ M[a,b]= M[b,a]=0, a  \succ_{\sigma_1} b. \end{cases}
\end{align*}
Then, \textit{the Ranked Pairs method using voter 1 as a tie-breaker} (hereon referred to as $RP_1$) add edges from $M$ to a digraph according to $\mathcal{L}$, skipping those that create a cycle. 

\citet{Zavist89:Complete} show that $RP_1$ is indeed IoC. We now strengthen this result:

\begin{proof}
\citet{Zavist89:Complete} show that the original RP rule (without tie-breaking) has an equivalent definition using ``stacks''. We introduce an analogous notion and equivalency with respect to a specific $\mathcal{L}$.
\begin{definition} Given a complete ranking $R$ over candidates $\cand$ and a priority order over ordered pairs $\mathcal{L}$, we say $x$ \emph{attains} $y$ through $R$ and with respect to $\mathcal{L}$ if there exists a sequence of candidates $a_1,a_2,\ldots,a_j$ such that $a_1=x$, $a_j=y$ and for all $i \in [j-1]$, we have $a_i \succ_R a_{i+1}$ and $(a_i,a_{i+1}) \succ_{\mathcal{L}} (a_j , a_1)$. We say $R$ is a \emph{stack} with respect to $\mathcal{L}$ if $x \succ_R y$ implies $x$ attains $y$ through $R$ with respect to $\mathcal{L}$.
\end{definition}
\begin{lemma}\label{lemma:stack_winner}
    $RP_1$ with $\Sigma_1$ as a tie-breaker will pick candidate $a$ as a winner if and only if there exists a stack with respect to $\mathcal{L}$ that ranks $a$ first, where $\mathcal{L}$ is the priority order over ordered pairs constructed using $\Sigma_1$ as a tie-breaker.
\end{lemma}
\begin{proof}
    $(\Rightarrow):$ Say $a$ is the $RP_1$ winner with $\mathcal{L}$ as the priority order over ordered pairs (constructed from $\Sigma_1$). Notice that the final graph from the RP procedure will be a DAG. Say $R$ is the topological ordering of this DAG and $a$ is the source node (hence ranked first by $R$), which we call the \textit{winning ranking}. By definition, the rule will pick $a$ as the winner. For any $x,y\in A$ such that $x \succ_R y$, the edge $(y,x)$ was skipped in the RP procedure, implying it would have created a cycle. Hence, there exists candidates $a_1, \ldots a_j$ such that $a_1=x$ and $a_j=y$, and each $(a_i,a_{i+1})$ was locked in the RP graph before $(y,x)$ was considered, implying  $(a_i,a_{i+1}) \succ_\mathcal{L} (y,x)=(a_j,a_1)$. Moreover, since each $(a_i,a_{i+1})$ was locked, we must have $a_i \succ_R a_{i+1}$ in the final ranking. This implies $R$ is indeed a stack with respect to $\mathcal{L}$, with $a$ ranked first.\\
    \noindent $(\Leftarrow):$ Say $R$ is a stack with respect to $\mathcal{L}$, with $a$ ranked first. We argue this is the final ranking produced by running $RP_1$ with $\mathcal{L}$ as priority order (constructed using $\Sigma_1$). Assume instead that RP outputs final ranking $R^*$ with $R^* \neq R$. Then there exists at least one pair $x,y$ such that $x\succ_Ry$ but $y\succ_{R'}x$, so $(y,x)$ was locked by the RP procedure. Of all such pairs, say $x^*,y^*$ is the one where $(y^*,x^*)$ was locked by the RP procedure first. Since $x^*\succ_Ry^*$ and since $R$ is a stack with respect to $\mathcal{L}$, there exists a series of candidates $a_1,\ldots a_j$ such that $a_1=x^*$, $a_j=y^*$, and for all $i \in [j-1]$, we have $a_i \succ_R a_{i+1}$ and $(a_i,a_{i+1}) \succ_{\mathcal{L}} (a_j , a_1)=(y^*,x^*)$. Since $(a_i,a_{i+1}) \succ_{\mathcal{L}} (y^*,x^*)$ for all $i$, all such edges were considered by the RP procedure before $(y^*,x^*)$. At least one of these edges must have been skipped, otherwise locking $(y^*,x^*)$ would have caused a cycle. Say $(a_k,a_{k+1})$ was the first edge that was skipped. This implies locking this edge would have caused a cycle with the already-locked edges; however, since $a_k \succ_R a_{k+1}$, this cycle must contain an edge $(z,\ell)$ such that $\ell \succ_R z$. However, this implies $z\succ_{R'} \ell$ in the final ranking and that $(z,\ell) \succ_{\mathcal{L}}(a_k,a_{k+1})\succ_{\mathcal{L}}(y^*,x^*)$. Since $(y^*,x^*)$ was assumed to be the first such edge to be considered, this is a contradiction. 
\end{proof}
Note that since $RP_1$ results in a single unique ranking over candidates (as a single tie-breaker is fixed), the proof of Lemma \ref{lemma:stack_winner} also shows that there is a unique stack with respect to $\mathcal{L}$. We also use an existing lemma by \citet{Zavist89:Complete}.
\begin{lemma}[{\citealt[\S VII]{Zavist89:Complete}}]\label{lemma:impartial} Say $C$ is a clone set with respect to profile $\profile$. The winning ranking $R$ resulting from running $RP_1$ on $\profile$ with an impartial tie-breaker $\Sigma_1$ based on a ranking $\sigma_1$ will have no element of $A \setminus C$ appear between two elements of $C$ in $R$. 
\end{lemma}

We will now prove that $RP_1$ is composition consistent.
Given $\profile$, say $\decomp=\{K_1,K_2,\ldots,K_k\}$ is a clone decomposition. Say $\profile^\decomp$ is the summary of $\profile$ with respect to $\decomp$ (where the clone sets in each $\sigma_i$ is replaced by the meta candidates $\{K_i\}_{i \in [k]}$) and $\profile|_{K_i}$ is $\profile$ restricted to the candidates in $K_i$. We would like to show that $RP_1(\profile)= \bigcup_{K \in RP_1(\profile^\decomp)} RP_1(\profile|_K)$. Since RP with a specific tie-breaking order always produces a single unique winner, showing containment in a single direction is sufficient. 

Say $RP_1(\profile)=\{a\}$, implying $a$ comes first in the winning ranking $R$. By the proof of the forward direction of Lemma \ref{lemma:stack_winner}, $R$ is a stack with respect to $\mathcal{L}$ (the order that the RP procedure follows, which uses tie-breaking order $\Sigma_1$ based on vote $\sigma_1$). By Lemma \ref{lemma:impartial}, each $K_i \in \decomp$ appears as an interval in $R$, hence we can define a corresponding ranking $R^{\decomp}$ over clone sets in $\decomp$. We would like to show that $R^\decomp$ is a stack with respect to $\mathcal{L}^\decomp$, which is the order of ordered pairs in $\decomp$ according to decreasing order of $M^\decomp$ (the majority matrix of $\profile^\decomp$), using $\sigma_1^\decomp$ (voter 1's vote in the summary) as a tie-breaker. 

We can relabel the clone sets in $\decomp$ such that $R^\decomp=(K_1\succ K_2 \succ \ldots \succ K_k)$. Since $a$ is the ranked first in $R$, we have $a \in K_1$. Notice that if $k=1$, then $R^\decomp$ vacously. Otherwise, take any $K_x, K_y$ such that $K_x \succ_{R^\decomp} K_y$. Say $x$ is the element of $K_x$ that appears last in $R$ and $y$ is the element of $K_y$ that appears first in $R$. Since $R$ is a stack with respect to $\mathcal{L}$, and since $x \succ_R y$, there exists a sequence of candidates $a_1,\ldots a_j$ and for all $i \in [j-1]$, we have $a_i \succ_R a_{i+1}$ and $(a_i,a_{i+1}) \succ_{\mathcal{L}} (a_j , a_1)$. Since the $a_i$ in this sequence appear according to their order in $R$, by Lemma \ref{lemma:impartial}, consecutive candidates in the sequence $a_1,\ldots,a_j$ can be grouped up to form a sequence $K'_1, \ldots K'_{j'}$ such that $K'_{i'} \in \decomp$ for each $i'\in [j']$, $K'_1 = K_x$, $K'_{j'}=K_y$, and $K'_{i'} \succ_{R^\decomp}K'_{i'+1}$ for each $i'\in [j'-1]$. Notice that since $x$ and $y$ are in different clone sets, $j'>1$. Take any $i' \in [j'-1]$ and consider the last element $K'_{i'}$ and the first element of $K'_{i'+1}$ to appear in $(a_1,a_2,\ldots, a_j)$. By construction, these two elements appear consecutively in $(a_1,a_2,\ldots, a_j)$, so they are $a_i$ and $a_{i+1}$, respectively, for some $i \in [j]$. Since $(a_i,a_{i+1})\succ_\mathcal{L}(a_j,a_1)=(y,x)$, based on the way $\mathcal{L}$ was constructed, there are two possible cases:
\begin{enumerate}
    \item $M[a_i,a_{i+1}] > M[y,x]$, in which case we must have $M^\decomp[K'_{i'}, K'_{i'+1}]= M[a_i,a_{i+1}] > M[y,x]= M^\decomp[K_y,K_x]$ by definition of clones, and hence $(K'_{i'},K'_{i'+1})\succ_{\mathcal{L}^\decomp}(K_y,K_x) = (K'_{j'}, K'_{1})$.
    \item $M[a_i,a_{i+1}] = M[y,x]$. In this case, we also have $M^\decomp[K'_{i'}, K'_{i'+1}]= M^\decomp[K_y,K_x]$ by definition of clones. However, since $(a_i,a_{i+1})\succ_\mathcal{L}(y,x)$,  there are four options: 
    \begin{enumerate}
        \item $i'=1$ and $i'+1=j'$, so $a_i=x$ and $a_{i+1}=y$. In this case,  $(a_i,a_{i+1})\succ_\mathcal{L}(y,x)$ implies $x \succ _{\sigma_{1}} y$ and hence $K_x \succ_{\sigma^\decomp_1} K_y$ by definition of clone sets, and hence: 
        $(K'_{i'},K'_{i'+1})=(K_{x},K_{y})\succ_{\mathcal{L}^\decomp}(K_y,K_x)=(K'_{j'},K'_1)$
        \item $i'=1$ and $i'+1 \neq j'$, so $a_i=x$ and $a_{i+1} \neq y$. In this case,  $(a_i,a_{i+1})\succ_\mathcal{L}(y,x)$ implies $a_{i+1}\succ_{\sigma_{1}} y$ and hence $K'_{i'+1} \succ_{\sigma^\decomp_1} K_y$ by definition of clone sets, and hence: 
        $(K'_{i'},K'_{i'+1})=(K_{x},K'_{i'+1})\succ_{\mathcal{L}^\decomp}(K_y,K_x)=(K'_{j'},K'_1)$.
        \item $i' \neq 1$ and $i'+1 = j'$, so $a_i \neq x$ and $a_{i+1} = y$. In this case,  $(a_i,a_{i+1})\succ_\mathcal{L}(y,x)$ implies $a_{i}\succ_{\sigma_{1}} x$ and hence $K'_{i'} \succ_{\sigma^\decomp_1} K_x$ by definition of clone sets, and hence: 
        $(K'_{i'},K'_{i'+1})=(K'_{i'},K_{y})\succ_{\mathcal{L}^\decomp}(K_y,K_x)=(K'_{j'},K'_1)$.
        \item $i' \neq 1$ and $i'+1 \neq j'$, so $a_i \neq x$ and $a_{i+1} \neq y$. 
        
        In this case,  $(a_i,a_{i+1})\succ_\mathcal{L}(y,x)$ implies for some $\alpha \in \{0,1\}$, we have $a_{i+\alpha}\succ_{\sigma_{1}} z$ for each $z \in \{x,y,a_{i+1-\alpha}\}$, and hence $K'_{i'+\alpha}\succ_{\sigma_1^\decomp} Z$ for each $Z \in \{K_x,K_y,K'_{i'+1-\alpha}\}$ by definition of clone sets, and hence: 
        $(K'_{i'},K'_{i'+1})\succ_{\mathcal{L}^\decomp}(K_y,K_x)=(K'_{j'},K'_1)$.

    \end{enumerate}
\end{enumerate}
    In each case, we end up having $(K'_{i'},K'_{i'+1})\succ_{\mathcal{L}^\decomp}(K'_{j'},K'_1)$, which proves that $K_x$ attains $K_y$ through $R^\decomp$ with respect to $\mathcal{L}^\decomp$, and hence that $R^{\decomp}$ is a stack with respect to $\mathcal{L}^\decomp$. By Lemma \ref{lemma:stack_winner}, this implies $RP_1(\profile^\decomp)=\{K_1\}$, as $K_1$ comes first in $R^\decomp$. 

    Since $a\in K_1$, $a$ will be a competing candidate in $\profile|_{K_1}$. Again by Lemma \ref{lemma:impartial}, we know that all elements of $K_1$ appears as a block in the start of $R$. Say $R|_{K_1}$ is this section of $R$. We would like to show that $R|_{K_1}$ is a stack with respect to $\mathcal{L}^{K_1}$, which is the priority order of ordered pairs in $\decomp$ according to decreasing order of $M^{K_1}$ (the majority matrix of $\profile|_{K_1}$), using $\sigma_1|_{K_1}$ (voter 1's vote restricted to $K_1$) as a tie-breaker. Note that for any $a,b,c,d \in K_1$, $(a,b) \succ_\mathcal{L}  (c,d)$ implies $(a,b) \succ_{\mathcal{L}^{K_1}}  (c,d)$, since $\mathcal{L}$ is entirely based on pairwise comparisons and the relative ranking of candidates in $\sigma_1$, neither of which is affected by the deletion of candidates in $A \setminus K_1$ and hence is directly carried to $\mathcal{L}^{K_1}$. Now take any $x,y \in K_1$ such that $x\succ_{R|_{K_1}}y$. Since $R|_{K_1}$ is just an interval of $R$, we must have $x\succ_R y$. Since $R$ is a stack, this implies there exists a sequence of candidates $a_1,a_2, \ldots, a_j$ such that $a_1=x$, $a_j=y$ and for all $i \in [j-1]$, we have $a_i \succ_R a_{i+1}$ and $(a_i,a_{i+1}) \succ_{\mathcal{L}} (a_j , a_1)$. Since all elements of $K_1$ appear as an interval in $R$ by Lemma \ref{lemma:impartial}, $x=a_1 \succ_R a_2 \succ_R \ldots \succ_R a_j = y$ and $x,y\in K_1$ implies $a_i \in K_1$ for all $i \in [j]$. This implies  $a_i \succ_{R|_{K_1}} a_{i+1}$ and $(a_i,a_{i+1}) \succ_{\mathcal{L}^{K_1}} (a_j , a_1)$, implying $R|_{K_1}$ is a stack with respect to $\mathcal{L}^{K_1}$. Since $a$ is first in $R|_{K_1}$, by Lemma \ref{lemma:stack_winner}, this implies $RP_1(\profile|_{K_1})=\{a\}$. Since $ =RP_1(\profile^\decomp)=\{K_1\}$, we have $\bigcup_{K \in RP_1(\profile^\decomp)} f(\profile|_K)=\{a\}=RP_1(\profile)$, completing the proof.
\end{proof}

\subsection{Proof of \Cref{prop:ucg}}

Next, we prove that unlike $\UCf$ (which is not even IoC~\citep{Holliday23:Split}), $\UCg$ does indeed satisfy CC as an SCF.

\ucg*
\begin{proof}
    Recall from \Cref{tab:scfs} that given $\profile$ and $a,b \in \cand$, we say that $a$ \emph{left-covers} $b$ in $\profile$ if any $c \in \cand$ that pairwise defeats $a$ also pairwise defeats $b$. Then $\UCg$ is defined as
    \begin{align*}
        \UCg=\{a \in \cand: \nexists b \in \cand\text{ such that }b\text{ left-covers AND pairwise defeats }a\}.
    \end{align*}

    Fix any profile $\profile$ and clone decomposition $\decomp$ with respect to $\profile$. We will show that $\UCg(\profile)= \gloc_\UCg(\profile, \decomp)$. Equivalently, for any $a \in \cand$, we will show that $a \notin \UCg(\profile) \Leftrightarrow a \notin \gloc_\UCg(\profile, \decomp)$. 
    
    \noindent $(\Rightarrow):$ Say $a \notin \UCg(\profile)$, then $\exists b \in A$ such that $b$ left-covers and pairwise defeats $a$ in $\profile$. Say $K_a \in \decomp$ is the clone set that contains $a$. We will consider two cases:
    \begin{enumerate}
        \item $b \in K_a$. For each $c \in K_a$ that pairwise defeats $b$ in $\profile|_{K_a}$, we must have that $c$ pairwise defeats $a$ in $\profile|_{K_a}$, since $b$ left-covers $a$ in $\profile$ and deletions of other candidates do not affect pairwise victories of the remaining candidates. Hence, $b$ left-covers and pairwise defeats $a$ in $\profile|_{K_a}$. This implies $a \notin \UCg(\profile|_{K_a})$. 
        \item $b \notin K_a$. Say $K_b \in \decomp \setminus \{K_a\}$ is the clone set containing $b$. Since $b$ pairwise defeats $a$ in $\profile$, $K_b$ pairwise defats $K_a$ in $\profile^{\decomp}$ by the clone set definition. Take any $K \in \decomp$ that pairwise defeats $K_b$ in $\profile^\decomp$. This implies there exists some $c \in K$ that pairwise defeats $b$ in $\profile$. Since $b$ left-covers $a$ in $\profile$, this implies $c$ pairwise defeats $a$ in $\profile$ and thus $K$ pairwise defeats $K_a$ in $\profile^\decomp$. Hence, $K_b$ left-covers and pairwise defeats $K_a$ in $\profile^\decomp$, implying $K_a \notin \UCg(\profile^\decomp)$.
    \end{enumerate}
    This implies we either have $a \notin \UCg(\profile|_{K_a})$ or $K_a \notin \UCg(\profile^\decomp)$. By $\Cref{def:gloc}$, this implies $a \notin \gloc_{\UCg}(\profile, \decomp)$.

    \noindent $(\Leftarrow):$ Say $a \notin \gloc_{\UCg}(\profile, \decomp)$ and $K_a \in \decomp$ is the clone set that contains $a$. This implies at least one of the two following two cases must be true:
    \begin{enumerate}
        \item $a \notin \UCg(\profile|_{K_a})$. Then there exists $b \in K_a$ that left-covers and pairwise defeats $a$ in $\profile|_{K_a}$. Since pairwise defeats are not affected by the addition of other candidates, $b$ also pairwise defeats $a$ in $\profile$. Take any $c \in \cand$ that pairwise defeats $b$. If $c \in K_a$, then $c$ must pairwise defeat $a$ because $b$ left covers $\profile|_{K_a}$. If $c \notin K_a$, then $c$ must pairwise defeat $a$ by the clone set definition, since $a,b \in K_a$. Thus, $b$ left-covers and pairwise defats $a$ in $\profile$, implying $a \notin \UCg(\profile)$.
        \item $K_a \notin \UCg(\profile^\decomp)$. Then there exists $K \in \decomp$ that left-covers and pairwise defeats $K_a$ in $\profile^\decomp$. Since $K_a$ cannot pairwise defeat itself, this implies $K \neq K_a$. Take any $b \in K$. Since $K$ pairwise defeat $K_a$ in $\profile^\decomp$, this implies $b$ pairwise defeats $a$ in $\profile$. Take any $c \in \cand$ that pairwise defeats $b$ in $\profile$. We cannot have $c \in K_a$ since $K$ pairwise defeats $K_a$. If $c \in K$, then $c$ must pairwise defeat $a$ since $K$ pairwise defeats $K_a$. If $c \notin K_a$, say $K_c \decomp \setminus \{K_a, K_b\}$ is the clone set that contains $c$. Since $c$ pairwise defeats $b$ in $\profile$, $K_c$ pairwise defeast $K$ in $\profile^\decomp$. Since $K$ left-covers $K_a$ in $\profile^\decomp$, this implies $K_c$ pairwise defeats $K_a$, and therefore $c$ pairwise defeats $a$ in $\profile$. Hence, $b$ left-covers and pairwise defeats $a$, implying $a\notin \UCg(\profile).$
    \end{enumerate}
    Hence, $a \notin \gloc_{\UCg}(\profile, \decomp)$ implies $a \notin \UCg(\profile)$, completing the proof.
\end{proof}