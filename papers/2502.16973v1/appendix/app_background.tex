\section{Further Background}\label{appsec:bg}
In this section, we give a more extended analysis of the concepts and definitions introduced by \citet{Tideman87:Independence} and by \citet{Laffond96:Composition}, and how they have been interpreted and used in subsequent literature. To give a more complete picture, this section repeats some of the information given in \Cref{sec:related} (\nameref{sec:related}) and \Cref{sec:bg} (\nameref{sec:bg}) of the main body of the paper.

\begin{figure}[b]
\centering \begin{tabular}{|c|c|c|}
  \hline
  Voter 1 & Voter 2 & Voter 3\\ \hline 
  \cellcolor{blue!25} $b$ & \cellcolor{yellow!25}  $a$ &\cellcolor{yellow!25} $a$ \\
  \hline
\cellcolor{green!25} $c$ &  \cellcolor{red!25} $d$ &\cellcolor{blue!25} $b$ \\
  \hline
      \cellcolor{yellow!25} $a$ & \cellcolor{green!25} $c$ & \cellcolor{green!25} $c$  \\
  \hline
     \cellcolor{red!25} $d$ & \cellcolor{blue!25} $b$ &\cellcolor{red!25} $d$ \\
  \hline
\end{tabular}
\caption{A preference profile. Columns show rankings.}\label{fig:appeg_prof}
\end{figure} 

\subsection{Preference profiles and clones}\label{appsubsec:clones}
We consider a finite set of \textit{voters} $\voter = \{1,\ldots, n\}$ and a finite set of \textit{candidates} $\cand$ with $|\cand|=m$. A {\em ranking} over $\cand$
is an asymmetric, transitive, and complete binary relation $\succ$ on $\cand$; we denote the set of all rankings over $\cand$ by $\mathcal{L}(\cand)$.
Each voter $i \in \voter$ has a \textit{ranking} $\sigma_i \in \mathcal{L}(\cand)$; we 
write $a \succ_i b$ to indicate that $i$ ranks $a$
above $b$, and collect the rankings of all voters in
a \textit{preference profile} $\profile \in \mathcal{L}(\cand)^n$. Given $\profile$, how can we identify candidates that are ``close'' to one another, at least from the point of view of the voters?  \citet{Tideman87:Independence} addressed this question:
\begin{definition}[{\citealt[\S I]{Tideman87:Independence}}]\label{appdef:clones}
Given a preference profile $\profile$ over candidates $\cand$, a nonempty subset of candidates, $\clone \subseteq \cand$, is a \emph{set of clones} with respect to $\profile$ if no voter ranks any candidate outside of $K$ between any two elements of $\clone$.
\end{definition}
Note that all profiles ways have two types of \textit{trivial} clone sets:\footnote{\citet{Tideman87:Independence} in fact excludes trivial clone sets. We use the definition followed by \citet{Elkind10:Clone}} (1) the entire candidate set $\cand$, and (2) for each $a\in \cand$, the singleton $\{a\}$. We call all other clone sets \textit{non-trivial}. For example, in the preference profile in Figure \ref{fig:appeg_prof}, the only non-trivial clone set is $\{b,c\}$.

Seemingly unaware of Tideman's definition, \citet{Laffond96:Composition} tackled a similar question of identifying similar candidates. First, they focused on the context of a \textit{tournament} $T$, which is a complete asymmetric binary relation on the candidates $\cand$ (\emph{i.e.}, a ranking with the transitivity condition relaxed). For a tournament, they defined:
\begin{definition}[{\citealt[Def. 1]{Laffond96:Composition}}]\label{appdef:component_tour}
Given tournament a $T$ over candidates $\cand$, a nonempty $\comp \subseteq \cand$ is a \emph{component} of $T$ if for all $y,y' \in \comp$ and all $x\in \cand \setminus \comp$: $y\succ_T x \Leftrightarrow y'\succ_{T} x$. 
\end{definition}
Of course, any preference profile $\profile$ with odd number of voters (to avoid ties) can be interpreted as a tournament $T_{\profile}$ over the same set of candidates, where the binary relation is given by the pairwise defeats of $\profile$; \emph{i.e.}, $\forall a,b \in \cand$: 
\begin{align*}
 a \succ_{T_{\profile}} b \Leftrightarrow |\{i \in N: a \succ_{\sigma_i} b\}|-|\{i \in N: b \succ_{\sigma_i} a\}|>0.
\end{align*}
Considering the similarities between Definitions \ref{def:clones} and \ref{appdef:component_tour} (they both group up candidates that have an identical relationship to other candidates), one might expect them to respect this transformation; that is, for $\clone$ to be a clone set of $\profile$ if and only if it is a component of $T_{\profile}$. However, this is not the case, as demonstrate next.

\begin{example} \label{ex:clones_v_components}
    Consider again the profile in \Cref{fig:appeg_prof}. Here, $\{a,c\}$ is a component in $T_{\profile}$ (they both defeat $d$ and both lose to $b$), but they are not a set of clones in $\profile$, since Voter 3 ranks $d$ between them. The intuition behind this is that when interpreting a preference profile as a tournament, we lose information about the preferences of individual voters. Indeed, it is easy to see that the implication holds in one direction: if $K$ is a set of clones of $\profile$, then it is a component of $T_{\profile}$. 
\end{example}

However, \citet{Laffond96:Composition} introduce a separate definition for components for a preference profile or, more accurately, to the more general notion of a \textit{tournament profiles} $(T_i)_{i \in \profile}$, where voters are allowed to submit tournaments (\emph{i.e.}, votes need not be transitive):


\begin{definition}[{\citealt[Def. 4]{Laffond96:Composition}}]\label{appdef:component_prof}
Given tournament profile $\boldsymbol{T}=(T_i)_{i \in N}$, a nonempty $C \subseteq A$ is a \emph{component} of $\boldsymbol{T}$ if $C$ is a component of $T_i$ for all $i\in N$. 
\end{definition}

One can see from Definition \ref{appdef:component_tour} that if a tournament is transitive (\emph{i.e.}, a ranking), then $\comp$ is a component of $T$ if and only if it appears as an interval in that ranking. As such, given a tournament profile $\boldsymbol{T}=\{T_{i}\}_{i \in N}$ where all voters submit transitive tournaments, $\comp$ is a component of $\boldsymbol{T}$ if and only if it appears as an interval in the vote of every voter. As such, \citeauthor{Laffond96:Composition}'s Definition \ref{appdef:component_prof} restricted to preference profiles is in fact \emph{equivalent} to \citeauthor{Tideman87:Independence}'s definition of a set of clones (Definition \ref{def:clones})! For instance, in the profile from Figure \ref{fig:appeg_prof}, $\{a,c\}$ is \textit{not} a component according to Definition \ref{appdef:component_prof}, since it is not a component of Voter 3's vote. 

Since most of the rest of the paper by \citet{Laffond96:Composition} focuses on tournaments rather than profiles, later work have largely focused on Definition \ref{appdef:component_tour} for tournaments rather than Definition \ref{appdef:component_prof} for profiles, leading to some papers arguing \citeauthor{Laffond96:Composition}'s definition for components is a ``more liberal notion'' than \citeauthor{Tideman87:Independence}'s clones~\citep{Conitzer24:Position,Holliday24:Simple}, even though the former's definitions for components in preference profiles is identical to that of Tideman's definition for clones. In the rest of the paper, we will be sticking to the term ``clone sets'' for consistency. 

Since their independent introduction by \citeauthor{Tideman87:Independence} and by \citeauthor{Laffond96:Composition}, clone sets in preference profiles have been studied at length in the computational social choice literature. Notably, \citet{Elkind10:Clone} have axiomatized the  structure of clone sets in preference profiles, and introduced a compact representation of clone sets using a data structure called PQ-trees, which we will later introduce in detail.

\subsection{Social choice functions and axioms}\label{appsubsec:axioms}
Of course, identifying ``similar'' candidates in a voting profile is useless unless one can say something meaningful about their impact on the election result. This impact depends on the voting rule we are using to compute the winners. More formally, say $ \mathcal{P}(\cand)$ is the power set of $A$ (set of all subsets). Then, a \textit{social choice function (SCF)} is a function $f: \mathcal{L}(\cand)^n \rightarrow \mathcal{P}(\cand)$ that maps each preference profile $\profile$ to a subset of $A$, which are termed the winner(s) of $\profile$ under $f$. In an election without ties, the output of an SCF contains a single candidate.

Before introducing axioms for robustness against strategic nomination, it is worth noting that for rules that are \emph{not} robust, the exact influence of addition of similar candidates can vary: for example, introducing clones of a candidate can hurt that candidate for an SCF like plurality  voting (which simply picks the candidates ranked first by the most voters) by splitting the vote (as was the case from the Oregon governor race from the introduction of the main body of the paper), making plurality what we call \textit{clone-negative}. On the other hand, having clones helps a candidate win if the SCF being used is Borda's rule, which gives a point for each candidate for every other candidate it beats in each voter's ranking, and picks the candidates with the most points as the winners.


\begin{example}\label{ex:borda}
    Consider the following voting profile for candidates $a$ and $b$:

\begin{center} 
 \begin{tabular}{|c|c|}
  \hline
  62 voters & 38 voters\\ \hline 
  \cellcolor{yellow!25} $a$ & \cellcolor{blue!25} $b$  \\
  \hline
   \cellcolor{blue!25} $b$ & \cellcolor{yellow!25} $a$ \\
  \hline
\end{tabular}
\end{center}

In this case, candidate $a$ receives 62 Borda points, whereas candidate $b$ receives 38 Borda points. Thus, candidate $a$ wins the election. Next we introduce a clone of $b$, obtaining the following voting profile:


\begin{center} 
 \begin{tabular}{|c|c|}
  \hline
  62 voters & 38 voters\\ \hline 
  \cellcolor{yellow!25} $a$ & \cellcolor{blue!25} $b$  \\
  \hline
   \cellcolor{blue!25} $b$ & \cellcolor{blue!35} $b_2$ \\
  \hline
  \cellcolor{blue!35} $b_2$ & \cellcolor{yellow!25} a \\
  \hline
\end{tabular}
\end{center}

Now candidate $a$ receives 124 Borda points, $b$ receives 138 Borda points, and clone $b_2$ receives 38 Borda points. Hence, candidate $b$ now becomes the winner. 
\end{example}

\Cref{ex:borda} shows that unlike with plurality voting, having a clone can positively impact a candidate under Borda's rule, thus making Borda's rule \textit{clone-positive} for this specific profile (in order profiles, having clones can in fact hurt your Borda score). Either of these impacts are undesirable, considering they incentivize strategic nomination, either of candidates similar to one's opponents, or of candidates similar to one's self, either of which can be arbitrarily easy. As such, we would like to find \textit{axioms} such that if an SCF satisfies them, then they are in some way robust to this type of strategic nomination.  

Along with their (equivalent) definitions for similar candidates in preference profiles, \citet{Tideman87:Independence} and \citet{Laffond96:Composition} each introduce their own axiom for identifying SCFs that behave ``desirably'' in response to addition/removal of such candidates. Since we deal with preference profiles (say $\profile$) with some candidates (say $\cand'\subset \cand$) removed, it will be useful to use $\boldsymbol{\sigma} \setminus \cand'$ to denote the profile obtained by removing the elements of $\cand'$ from each voter's ranking in $\boldsymbol{\sigma}$ and preserving the order of all other candidates.

We begin with the axiom by \citet{Tideman87:Independence}, who explicitly identified the goal of achieving robustness against strategic nomination. \citet{Zavist89:Complete} later presented the definition with more precise language, which is the version we use for clarity.

\begin{definition}[\citealt{Zavist89:Complete}]\label{appdef:ioc}
A voting rule $f$ is \emph{independent of clones (IoC)} if the following two conditions are met for all profiles $\boldsymbol{\sigma}$ and for all non-trivial clone sets $\clone \subset A$ with respect to $\boldsymbol{\sigma}$:
\begin{enumerate}
    \item For all $a \in \clone$, we have:
    \begin{align*}
        \clone \cap f(\boldsymbol{\sigma}) \neq \emptyset \Leftrightarrow \clone \setminus \{a\} \cap f(\boldsymbol{\sigma} \setminus \{a\}) \neq \emptyset.  
    \end{align*}
    \item For all $a \in \clone$ and all $b \in A \setminus \clone$ we have:
    \begin{align*}
       b \in f(\boldsymbol{\sigma}) \Leftrightarrow b \in f(\boldsymbol{\sigma} \setminus \{a\}).
    \end{align*}
\end{enumerate}
\end{definition}

Intuitively, IoC dictates that deleting one of the clones must not alter the winning status of the set of clones as a whole, or of any candidate not in the set of clones. This is a desirable property in SCFs, since it imposes that the winner must not change due to the addition of a non-winning candidate who is similar to a candidate already present. This prevents candidates from influencing the election by nominating new copy-cat candidates. 

\begin{example}
    Consider running plurality voting ($PV$) on the profile $\profile$ in Figure \ref{fig:bg_eg} (left). We have $f_{PV}(\profile)=\{b\}$, as it is the top choice of 4 voters, more than any other candidate. Moreover, $\{a_1,a_2\}$ is a clone set with respect to $\profile$. However, $f_{PV}(\profile \setminus \{a_2\}) = \{a_1\}$, since with $a_2$ gone, $a_1$ now has 5 voters ranking it top, beating $b$ (Figure \ref{fig:bg_eg}, right). This violates both conditions 1 and 2 from Definition \ref{appdef:ioc}, as the removal of a clone ($a_2$ from clone set $\{a_1,a_2\}$) results in another clone $(a_1)$ winning, while previously none did, and eliminates a previous winner outside the clone set ($b$).

    Instead, consider running Single Transferable Vote ($STV$) on $\profile$, which is an SCF that iteratively removes the candidates with the least plurality votes from the ballot, returning the last remaining candidate. We have $STV(\profile)=\{a_1\}$ as $a_2$ is removed with 2 plurality votes (causing $a_1$ to now have 5 plurality votes), followed by the $c$ with 3 plurality votes, and finally $b$ with 4 plurality votes. Similarly, $STV(\boldsymbol{\sigma} \setminus\{a_2\})=\{a_1\}$, since $a_2$ was going to be the first candidate to be eliminates anyway. Hence, in both cases, a member of the clone set $\{a_1,a_2\}$ wins. This is in line with the fact that $STV$ is IoC~\citep{Tideman87:Independence}. 
\end{example}

Like Tideman's presentation of IoC, \citet{Laffond96:Composition} are also concerned with the manipulability of elections through cloning. However, the core of the presentation of their axiom, aptly named \textit{composition consistency}, focuses on the consistency between applying a rule directly, or applying it through a two-stage mechanism. In order to formalize this mechanism, we introduce a few concepts:
  \begin{definition}\label{appdef:clone_grouping}
Given a preference profile $\boldsymbol{\sigma}$ over candidates $\cand$, a set of sets $\decomp=\{\clone_1,\clone_2,\ldots,\clone_\ell\}$ where $\clone_i \subseteq A$ for all $i\in [\ell]$ is a \emph{clone decomposition} with respect to $\boldsymbol{\sigma}$ if:
\begin{enumerate}
    \item $\decomp$ is a disjoint partitioning of $\cand$, \emph{i.e.}: $\cand=\bigsqcup_{i \in [\ell]} K_i$ and $\clone_i \cap \clone_j = \emptyset$ for $i \neq j$, and
    \item each $\clone_i$ is a non-empty clone set with respect to $\boldsymbol{\sigma}$.
\end{enumerate}
\end{definition}
A given profile $\boldsymbol{\sigma}$ can have multiple distinct clone decompositions. Indeed, every profile has at least two decompositions: the \textit{null} decomposition $\decomp_{null}=\{A\}$ and the \textit{trivial} decomposition $\decomp_{triv}=\{ \{a\}\}_{a \in A}$. Given a clone decomposition $\decomp$ with respect to $\boldsymbol{\sigma}$, for each $i \in N$, say $\sigma^\decomp_i$ is voter $i$'s ranking over the clone sets in $\decomp$ (which is well defined, since each clone set appears as an interval in $\sigma_i$). We call $\boldsymbol{\sigma}^\decomp = \{\sigma^\decomp_i\}_{i \in [n]}$ the \textit{summary} of $\boldsymbol{\sigma}$ with respect to decomposition $\decomp$, which is a preference profile treating the elements of $\decomp$ as the set of candidates. Lastly, for each $K \in \decomp$, say $\boldsymbol{\sigma}^{K}$ is $\boldsymbol{\sigma}$ with $A\setminus K$ removed (\emph{i.e.}, $\boldsymbol{\sigma}^{K} \equiv \boldsymbol{\sigma} \setminus (A \setminus K)$). We are now ready to introduce \textit{composition products}:

\begin{definition}[Composition product]\label{appdef:gloc} 
    Given any SCF $f$, the \emph{composition product} function of $f$ is a function $\gloc_f$ that takes as input a profile $\boldsymbol{\sigma}$ and a clone decomposition $\decomp$ with respect to $\boldsymbol{\sigma}$ and outputs $\gloc_f(\boldsymbol{\sigma}, \decomp) \equiv \bigcup_{K \in f\left(\boldsymbol{\sigma}^\decomp\right)  }f(\boldsymbol{\sigma}^K)$.
\end{definition}

Intuitively, $\gloc_f$ first runs the input voting rule $f$ on the summary (as specified by $\decomp$), ``packing'' the candidates in each set to treat it as a meta-candidate $\clone_i$. It then ``unpacks'' the clones of each winner clone set, and runs $f$ once again on each. We demonstrate this in the following example.
\begin{example}\label{appeg:GLOC}

Once again consider $\boldsymbol{\sigma}$ from Figure \ref{fig:bg_eg}. Notice $\decomp=\{K_a,K_b,K_c\}$ with $K_a=\{a_1,a_2\}$, $K_b=\{b\}$ and $K_c=\{c\}$ is a valid clone decomposition with respect to $\boldsymbol{\sigma}$. Figure \ref{fig:bg_eg_2} shows $\profile^\decomp$ and $\profile^{\clone_a}$. Notice that we have $STV(\boldsymbol{\sigma}^\decomp)=\clone_a$ ($K_c$ gets eliminated first followed by $K_b$) and $STV(K_a)=\{a_2\}$, implying $\gloc_{STV}(\boldsymbol{\sigma}, \decomp) =\{a_2\}$.

\end{example}

Example \ref{appeg:GLOC} demonstrates that $STV(\boldsymbol{\sigma}) \neq \gloc_{STV}(\boldsymbol{\sigma}, \decomp)$ for this specific $\boldsymbol{\sigma}$ and $\decomp$, showing $STV$ is not \textit{consistent} with respect to this decomposition, even though the winners in two cases are from the same clone set. It is also easy to see that for all rules $f$ and all $\boldsymbol{\sigma}$, we have $f(\boldsymbol{\sigma})=\gloc_f(\boldsymbol{\sigma}, \decomp_{null})=\gloc_f(\boldsymbol{\sigma}, \decomp_{triv})$. To satisfy composition consistency, a rule must satisfy this equality for all non-trivial decompositions too: 
  \begin{definition}[{\citealt[Def. 11]{Laffond96:Composition}}]\label{appdef:oioc}
      An SCF $f$ is \emph{composition-consistent (CC)} if for all preference profiles $\boldsymbol{\sigma}$ and all clone decompositions $\decomp$ w.r.t. $\boldsymbol{\sigma}$, we have $f(\boldsymbol{\sigma}) =\gloc_f(\boldsymbol{\sigma}, \decomp)$.
\end{definition}

Intuitively, an SCF is CC if it chooses the ``best'' candidates from the ``best'' clone sets~\citep{Laffond96:Composition}. Notice that while any other member of the clone set winning after the removal of a winner clone is sufficient by IoC, CC also specifies which exact clones should be winning. We formalize this hierarchy in Proposition \ref{prop:cctoioc} by showing CC implies IoC. Example \ref{appeg:GLOC} already demonstrates that the other direction is untrue, as it proves that $STV$, which is IoC, is not CC. In later sections, we will be analyzing other IoC rules to show whether they are CC.  

While the hierarchical relationship between CC and IoC may seem clear to the reader when presented this way, with few exceptions (more on which below), many papers that mention both axioms do not explicitly identify CC to be a strictly stronger property than IoC. It appears that part of the this unclarity can once again be attributed to SCFs taking a relatively small space in \citet{Laffond96:Composition}, which is mostly dedicated to tournaments. Accordingly, subsequent papers have also studied CC primarily in the context of tournaments~\citep{Cornaz13:Kemeny,Brandt11:Fixed,Brandt18:Extending}, even describing components/CC to be the ``analogue'' of clones/IoC for tournaments~\citep{Elkind11:Cloning,Dellis13:Multiple,Karpov22:Symmetric}. Other works, while identifying a link between IoC and CC, have not been precise about their relationship~\citep{Brandt09:Some,Öztürk20:Consistency,Koray07:Self,Laslier12:Loser,Lederer24:Bivariate,Saitoh22:Characterization,Elkind17:What,Camps12:Continuous,Laslier16:Strategic}, describing them as ``similar'' notions~\citep{Laslier97:Tournament,Heitzig10:Some} or simply ``related''~\citep{Brandt11:Fixed}.

There are papers that come much closer in identifying CC as a stronger axiom than IoC: working in a more general setting where voters' preferences are neither required to be asymmetric (ties are allowed)  nor transitive, \citet{Laslier00:Aggregation} introduces the notion of \textit{cloning consistency}, which he explains is weaker than CC and is the ``same idea'' as \citeauthor{Tideman87:Independence}'s IoC in voting theory. However, there are significant differences between \citeauthor{Laslier00:Aggregation}'s definition and IoC: first, his ``clone set'' definition requires every voter being \textit{indifferent} between any two alternatives in the set (as opposed to having the same relationship to all other candidates), and his cloning consistency dictates that if one clone wins, then so must every other member of the same clone set. Another property for tournament solutions named \emph{weak composition consistency} (this time in fact analogous to IoC) is discussed by \citet{Brandt18:Extending,Kruger18:Permutation} and \citet{Laslier97:Tournament}, although none of them points out the connection to IoC. Perhaps the work that does the most justice to the relationship between CC and IoC is  by \citet{Brandl16:Consistent}, who explicitly state that \citeauthor{Tideman87:Independence}'s IoC (which they refer to as cloning consistency) is weaker than \citeauthor{Laffond96:Composition}'s CC. They work with probabilistic social choice functions (PSCF), which return a set of lotteries over $\cand$ rather than a subset of candidates. As they note, deterministic SCFs can be viewed as PSCFs that return \emph{all} lotteries over a subset of candidates, therefore their observation that CC is stronger than IoC applies to our setting, although the adaptation is not obvious; thus, we formalize this statement in \Cref{prop:cctoioc}. The PSCFs analyzed by \citeauthor{Brandl16:Consistent} are all non-deterministic; hence, to the best of our knowledge, no previous work has studied whether IoC SCFs also satisfy CC, which we do in the main body of the paper.




