\section{Introduction}\label{sec:intro}

On November 6th, 1934, Oregonians took to the polls to elect their 28th governor. Earlier, in a contested Republican primary, Senator Joe E.~Dunne had narrowly defeated Peter Zimmerman, who then decided to run as an independent.
In the subsequent general election, each candidate received the following share of votes~\citep{Needham21:1934}:
\begin{center}
    \begin{tabular}{ |c|c|c|c| } 
    \hline
    Charles Martin & Peter Zimmerman & Joe Dunne \\
    \hline
    116,677 & 95,519 & 86,923 \\ 

    \hline
    \end{tabular}
\end{center}
The Democratic candidate (Martin) won, even though the two Republicans collectively won nearly $60\%$ of the vote. This example motivates the following question: \textit{How can we ensure that similar candidates in an election do not `spoil' the election, preventing each other from winning?}


Naturally, some winner determination rules, or \emph{social choice functions (SCFs)}, are more resilient
to this ``spoilage'' effect than others. 
The field of social choice offers a rich variety of SCFs and formulates various desirable criteria
(axioms) for them. In particular, \citet{Tideman87:Independence} introduces the concept of a {\em clone set}, \emph{i.e.}, a group of candidates (clones) that are ranked consecutively in all voter's rankings, and puts forward an axiom called \emph{independence of clones (IoC)}, which asks 
that if a candidate is an election winner, this should remain the case even if we add\footnote{For example, under veto/anti-plurality, a non-IoC SCF that picks the candidate(s) ranked bottom by the least number of voters, introducing clones can {\em help} the cloned candidate, as the clones split the last places in votes. \label{footnote:veto}} or remove clones of her opponents. In the context of political elections, IoC means that a political party need not be strategic about the number of party representatives participating in an election, as long as it does not care which of its candidates wins. Conversely, if a rule fails IoC, adding/deleting clones may be a viable strategy to change the outcome (Tideman himself recalls winning a grade school election after nominating his opponent's best friend). Moreover, \citet{Elkind11:Cloning} show that the algorithmic problem of purposeful cloning (\emph{i.e.}, to change the outcome of an election) is easy for many common SCFs. Thus, the appeal of SCFs satisfying IoC goes beyond the theoretical.

In settings with more abstract candidates, it is even easier to introduce clones. For example, when candidates consist of drafts of a text---\emph{e.g.}, we are voting over drafts of the guiding principles of our organization---it is straightforward to introduce a near-duplicate of an existing draft.  When candidates are AI systems---\emph{e.g.}, we are ranking LLMs, as done for example on Chatbot Arena~\citep{Chiang24:Chatbot}, to determine the best one---one can introduce a second version of a model, one that is fine-tuned only slightly differently (as has already been pointed out by~\citet{Conitzer24:Position}).  Without IoC, such clones can critically affect the outcome. 

On the other hand, IoC may not be \emph{enough} to dissuade strategic cloning. While IoC dictates that the cloning of a candidate should not change whether \emph{one} of the clones wins, it does not say anything about \emph{which} clone should win, even though one of them can be significantly preferred to the others by the voters. As such, even with an IoC rule, cloning can have a significant impact on the result by changing which candidate among the clones/political party wins.

Moreover, a rule being IoC does not reveal how \emph{obviously} robust it is against strategic nomination. As demonstrated by \citet{Li17:Obviously} in the context of obvious strategy-proofness, the benefits of a property of a mechanism might only be materialized if the agents actually \emph{believe} that the property indeed holds. Even when using an IoC rule, it is not clear that the average voter or candidate can be easily convinced of this property---resulting, for example, in a candidate unnecessarily dropping out of the race, either out of fear of hurting their party, or of being blamed by their voters for doing so.

These drawbacks of IoC might be one potential explanation for why major parties in the United States still hold internal primaries to pick a single nominee for the handful of elections that pick a winner using single transferable vote~\citep{Richie23:Case}, which is in fact IoC. This happens even though consolidating party support behind a single candidate does not improve the chances of the party winning the election, and they could provide the voters with a wider range of choices just by letting \emph{all} of their willing candidates run in the general election.

To this end, we turn to the \emph{stronger} axiom of \emph{composition consistency (CC)}, introduced by \citet{Laffond96:Composition}, which dictates not just which clone sets win, but which clones among those sets win too. CC, as we will argue, also exposes the \emph{obviousness} of a rule's robustness against strategic nomination. When introducing CC, \citeauthor{Laffond96:Composition} were seemingly unaware of the IoC definition by \citet{Tideman87:Independence}, despite using equivalent (but differently-named) concepts. This, among other factors, has led the literature on IoC and CC to progress relatively independently, with few papers identifying them as comparable axioms. By studying these axioms in a unified framework, we hope to help dispel this ambiguity.

\subsection{Our Contributions}
\begin{itemize}[leftmargin=*]
    \item We clarify the relationship between IoC and CC---which has historically been ambiguous (see \Cref{sec:related})---by formally showing that CC is strictly more demanding (\Cref{prop:cctoioc}). 
    \item We provide (to the best of our knowledge) the first ever analysis of whether SCFs that are known to be IoC (\emph{e.g.}, Ranked Pairs, Beatpath/Schulze Method, and Split Cycle) also satisfy the stronger property of CC, thereby also establishing where each rule falls in our hierarchy of barriers to strategic nomination (\Cref{sec:ioc_rules}). While for most of these rules the answer is negative, we identify a variant of Ranked Pairs that satisfies CC. We connect our results to the literature on \emph{tournament solutions (TSs)}, and show that while not every CC TS is a CC SCF, a certain extension of Uncovered Set is.
    \item We introduce an efficient algorithm that modifies \emph{any} (neutral) SCF into a new rule satisfying CC, while preserving various desirable properties, \emph{e.g.}, Condorcet/Smith consistency, among others (\Cref{sec:cc_transform}). Our transformation relies on the observation that clone sets (which can be nested and overlapping) can be represented in PQ-trees~\citep{Elkind10:Clone}, allowing us to recursively zoom into the ``best'' clone set. We show that the resulting rule is poly-time computable if the starting SCF is, and fixed-parameter tractable in terms of the properties of the PQ-tree for all other SCFs (additionally proving the fixed-parameter tractability of all SCFs that are CC to begin with).
    \item We provide the first extension of CC to social preference functions (SPF) and prove that many of our characterization results generalize (\Cref{sec:spf}). Nevertheless, we give a negative result showing that no anonymous SPF can be CC, and discuss ways in which this can be circumvented.
    \item We formalize the connection of IoC/CC to strategic behavior by candidates via the model of \emph{strategic candidacy} \citep{Dutta01:Strategic} (\Cref{sec:oioc}). We show that if the candidates' preferences over each other are dictated by their clone structure, IoC rules ensure running in the election is a dominant strategy, hence achieving a stronger version of \emph{candidate stability}. However, IoC is not enough for \emph{obvious} strategy-proofness, which we show can be achieved by CC rules using our PQ-tree algorithm. 
\end{itemize} 
\subsection{Related Work}\label{sec:related}


\paragraph{Independence of Clones and Composition Consistency.} In this section, we give a conceptual overview of the literature of IoC and CC, and point out potential origins for some inconsistencies. We formalize these claims in \Cref{appsec:bg}. For definitions of all concepts sufficient for our results, see \Cref{sec:bg}.  

In order to identify candidates ``close'' to one another (at least according to voters), \citet{Tideman87:Independence} introduces the notion of \emph{clone sets} and the property of \emph{independence of clones (IoC)} for SCFs that are robust to changes in these sets. Seemingly unaware of \citeauthor{Tideman87:Independence}'s work, \citet{Laffond96:Composition} tackle a similar question by introducing the concept of \emph{components} of candidates and the property of \emph{composition consistency (CC)}. Importantly, \citeauthor{Laffond96:Composition} provide two separate definitions for components, one for a \emph{tournament} (a single asymmetric binary relationship on candidates) and one for a \emph{preference profile} (individual preferences of voters over candidates), and thus two different definitions of CC for \emph{tournament solutions} and \emph{SCFs}, which respectively map tournaments and profiles to winners. While \citeauthor{Laffond96:Composition}'s components for profiles is \emph{equivalent} to \citeauthor{Tideman87:Independence}'s clone sets, later work has described the former as ``more liberal''/``weaker'' \citep{Conitzer24:Position,Holliday24:Simple}, potentially due to focusing on components of tournaments instead (See our \Cref{ex:clones_v_components}).

The relationship between IoC and CC has been similarly unclear, despite the latter being strictly more demanding for SCFs (\Cref{prop:cctoioc}). Potentially due to SCFs taking a relatively small space in the work of \citet{Laffond96:Composition}, subsequent papers have primarily studied CC tournament solutions~\citep{Laslier97:Tournament,Brandt11:Fixed,Brandt18:Extending}, even describing components/CC to be ``analogue'' of clones/IoC for tournaments \citep{Elkind11:Cloning,Dellis13:Multiple,Karpov22:Symmetric}. Other works, while identifying a link between IoC and CC, have not been precise on their relationship~\citep{Brandt09:Some,Öztürk20:Consistency,Koray07:Self,Laslier12:Loser,Lederer24:Bivariate,Saitoh22:Characterization,Elkind17:What,Camps12:Continuous,Laslier16:Strategic}, describing them as ``similar'' notions \citep{Laslier97:Tournament,Heitzig10:Some} or ``related'' \citep{Brandt11:Fixed}.

There are papers that come much closer in identifying CC as a stronger axiom than IoC: working in a more general setting where voters' preferences are neither required to be asymmetric (ties are allowed)  nor transitive, \citet{Laslier00:Aggregation} introduces the notion of \textit{cloning consistency}, which he explains is weaker than CC and is the ``same idea'' as \citeauthor{Tideman87:Independence}'s IoC in voting theory. However, there are significant differences between \citeauthor{Laslier00:Aggregation}'s definition and IoC: first, his ``clone set'' definition requires every voter being \textit{indifferent} between any two alternatives in the set (as opposed to having the same relationship to all other candidates), and his cloning consistency dictates that if one clone wins, then so must every other member of the same clone set. Another property for tournament solutions named \emph{weak composition consistency} (this time in fact analogous to IoC) is discussed by \citet{Brandt18:Extending,Kruger18:Permutation}, and \citet{Laslier97:Tournament}, although none of them point out the connection to IoC. Perhaps the work that does the most justice to the relationship between CC and IoC is  by \citet{Brandl16:Consistent}, who explicitly state that \citeauthor{Tideman87:Independence}'s IoC (which they refer to as cloning consistency) is weaker than \citeauthor{Laffond96:Composition}'s CC. Since they work with the more general model of probabilistic social choice functions (PSCF), their observation that CC is stronger than IoC applies to our setting (although the adaptation is not obvious); we formalize this in \Cref{prop:cctoioc}. The PSCFs analyzed by \citeauthor{Brandl16:Consistent} are all non-deterministic; hence, to the best of our knowledge, no previous work has studied whether IoC SCFs also satisfy CC, which we do in \Cref{sec:ioc_rules}.

\paragraph{Strategic Cloning.}
Cloning and voting rules that are IoC are also of interest from a game theory perspective due to their connection to strategic manipulation in elections \citep{Sornat21:Fine,Elkind10:Clone,Delemazure24:Generalizing,Caragiannis10:Socially}. While most of (computational) social choice research treats the set of candidates as fixed, cloning inevitably goes beyond this assumption. To study manipulative behavior in settings with varying candidates, \citet{Dutta01:Strategic,Dutta02:Successsive} have initiated the study of \emph{strategic candidacy}, where candidates too have preferences over each other and may choose whether to run in the election. They define an SCF as \emph{candidate stable} if no candidate can benefit from not running given that all others are running, and show that this property is failed by every non-dictatorial SCF satisfying unanimity. Subsequent work has analyzed the computational aspects of strategic candidacy in extensions of this model, such as when candidates incur a small cost for running \citep{Obraztsova15:Strategic}, when candidates can decide to rejoin the election after dropping out 
(albeit with possibly less support) 
\citep{Polukarov15:Convergence}, or when both voters and candidates are behaving strategically \citep{Brill15:Strategic}. Many of these papers allude to, but do not formally define, a connection between strategic candidacy and cloning. We do so in \Cref{sec:oioc}, where we use the model of strategic candidacy to analyze the strength of robustness of IoC/CC rules against spoilage by clones.

Recently, \citet{Holliday23:Split} have introduced novel robustness criteria they call \emph{immunity to spoilers} and \emph{immunity to stealers}, which study the impact of adding candidates that are not necessarily clones (but must fulfill some other conditions). These criteria are incomparable in strength to IoC/CC, and while we focus exclusively on the impact of similar candidates (\emph{i.e.}, clones), we believe the methods we develop may be of future interest for studying different types of spoilers.