\section{Analysis of IoC Social Choice Functions}\label{sec:ioc_rules}
In this section, we analyze whether the IoC rules in \Cref{tab:scfs} satisfy CC. The answer turns out to be positive for $\RP$ with a specific tie-breaking rule by \citet{Zavist89:Complete} and $\UCg$, but negative for all other SCFs. We first formalize the CC to IoC relationship (\emph{cf.} \citealt{Brandl16:Consistent}).
\begin{restatable}{proposition}{cctoioc}\label{prop:cctoioc}
    If a given SCF is CC, then it is also IoC.
\end{restatable}

All omitted proofs are in the appendix. We show that the converse of Proposition~\ref{prop:cctoioc} is not true.

\begin{restatable}{theorem}{ccfails}\label{thm:cc_fails}
    $\STV$, $\BP$, $\AS$, and $\SC$ all fail CC.
\end{restatable}
\begin{proof}
Since CC requires $f(\profile)=\gloc_f(\profile, \decomp)$ for \textit{all} profiles $\profile$ and clone decompositions $\decomp$, a single counterexample is sufficient to show that a rule fails CC. The statement for $\STV$ follows from \Cref{eg:ioc,eg:GLOC}. For the $\profile$ and $\decomp$ from these examples, $\AS(\profile)=\STV(\profile)$ and $\gloc_{\AS}(\profile, \decomp)=\gloc_{\STV}(\profile, \decomp)$; hence they also show $\AS$ is not CC. For $\BP$ and $\SC$, we use the profile from Fig.~\ref{fig:eg_prof} (say, $\profile'$), with $\decomp'=\{\{a\},\{b,c\}, \{d\}\}$. We have $\BP(\profile')=\SC(\profile')=\{b,c\}$, whereas $\gloc_{\BP}(\profile',\decomp)=\gloc_{\SC}(\profile',\decomp')=\{b\}$, so both $\BP$ and $\SC$ fail CC;
see \Cref{appsec:extended_ioc} for detailed calculations.
\end{proof}

\subsection{Ranked Pairs}\label{subsec:rp}
Our definition of SCFs deals with ties by returning all candidates that win via \textit{some} tie-breaking method. In particular, for $\RP$ the majority matrix $M$ may contain ties, so we need a \textit{tie-breaking order over unordered pairs} to decide the order of adding edges to the digraph. \citet{Tideman87:Independence} originally defined Ranked Pairs as returning all candidates that win for some tie-breaking order (we refer to this rule as $\RP$); later, \citet{Zavist89:Complete} showed that this rule is \emph{not} IoC. 

\begin{proposition}[\citealt{Zavist89:Complete}]\label{prop:rpfailioc}
        $\RP$ fails IoC.
\end{proposition}

By Proposition~\ref{prop:cctoioc}, this also implies $RP$ fails CC. \citet{Zavist89:Complete} propose breaking ties based on the vote of a fixed voter $i \in \voter$, which makes $\RP$ satisfy IoC.  
Specifically, they use $\sigma_i$ to construct a \emph{tie-breaking ranking} $\Sigma_i$ over unordered pairs in $A$ as follows: (1) order the elements within each pair according to $\sigma_i$; (2) rank the pairs according to $\sigma_i$'s ranking of their first elements; (3) rank pairs with the same first element according to $\sigma_i$'s ranking of the second elements. 
\begin{example}
    For $\cand=\{a,b,c\}$ and $\sigma_i: a\succ b \succ c $ we get
      $\Sigma_i: \{a,a\}\succ\{a,b\}\succ\{a,c\} \succ\{b,b\}\succ \{b,c\}\succ\{c,c\}.$
\end{example}

Using $\Sigma_i$, we can construct a complete \emph{priority order} $\mathcal{L}$ over ordered pairs: pairs are ordered (in non-increasing order) according to $M$, with ties broken by $\Sigma_i$ (in the special case where $M[a,b]=0$, we rank $(a,b) \succ_{\mathcal{L}} (b,a)$ if and only if $a \succ_i b$). Then, \textit{the Ranked Pairs method using voter $i$ as a tie-breaker} (which we will call $\RP_i$) adds edges from $M$ to a digraph according to $\mathcal{L}$, skipping those that create a cycle. \citet{Zavist89:Complete} show that $\RP_i$ is IoC. We now strengthen this result.

\begin{restatable}{theorem}{rpcc}\label{thm:rp}
    $\RP_i$ is CC for any fixed $i \in N$. 
\end{restatable}

\begin{proof}[Proof sketch] The proof relies on an equivalence between the topological orders of the final $RP$ graphs and \emph{stacks} over $\cand$, which are rankings $r$ where $a\succ_r b$ implies there is a path in $M$ from $a$ to $b$ consistent with the ranking $r$ and with each link at least as strong as $M[b,a]$ \citep{Zavist89:Complete}. We extend this equivalence to the specific case of stacks with respect to a priority order $\mathcal{L}$, and show that this definition is satisfied by the $\RP_i$ ranking, its summary using any $\decomp $, and its restriction to any clone set. This allows us to establish an agreement between $\RP_i$ and $\gloc_{\RP_i}$, proving $\RP_i$ is CC. The full proof can be found in \Cref{appsec:rpi}.
\end{proof}

Moreover, $\RP_i$ is poly-time computable, whereas the outputs of $\RP$ are \NP-hard to compute~\citep{Brill12:Price}. However, for any fixed $i$ this rule breaks anonymity, \emph{i.e.}, it fails to treat all voters equally.  \citet{Holliday23:Split} suggest (based on personal communication with Tideman) returning 
$\RP_N(\profile) \equiv \bigcup_{i \in N} \RP_i(\profile)$, \emph{i.e.}, 
declaring an $a \in \cand$ to be a winner if and only if $a \in RP_i(\profile)$ for \emph{some} $i \in N$.  This modification recovers anonymity while preserving IoC and tractability, but we show that it loses CC.
\begin{proposition}\label{prop:rpn}
    $\RP_N$ is IoC, but not CC.
\end{proposition}
\begin{proof}
    IoC holds since $RP_i$ is IoC for all $i \in N$. For failure of CC, consider $\profile$ with $n=2$, $\cand=\{a,b,c\}$, and rankings $a \succ_1 b \succ_1 c$ and  $c \succ_2 b \succ_2 a$. We have $RP_N(\profile)=\{a,c\}$. Using decomposition $\decomp=\{K, \{c\}\}$ with $K=\{a,b\}$, we get $RP_N(\profile^\decomp)=\{K, \{c\}\}$ and $RP_N(\profile|_K)=\{a,b\}$. Hence, $\gloc_{RP_N}(\profile, \decomp)= \{a,b,c\} \neq RP_N(\profile)$.
\end{proof}

Thus, Ranked Pairs without tie-breaking ($\RP$) is neither IoC, CC, nor tractable. Using a voter to break ties ($\RP_i$), we get all three, but lose anonymity. Recovering anonymity by taking a union over all voters ($\RP_N$) keeps IoC and tractability, but loses CC.\footnote{For probabilistic SCFs (PSCFs), a tempting approach is to pick an $i\in N$ uniformly at random and return $\RP_{i}(\profile)$. The counterexample from Prop.~\ref{prop:rpn} also shows that this variant fails the CC definition for PSCFs given by \citet{Brandl16:Consistent}.}  

\usetikzlibrary{backgrounds}

\begin{figure}[t]
    \centering
\begin{tikzpicture}[on background layer]

    \definecolor{set1}{RGB}{255, 204, 204}
    \definecolor{set2}{RGB}{204, 255, 204}
    \definecolor{set3}{RGB}{204, 204, 255}

    \fill[set1] (-4, -1.7) rectangle (4, 1.2);
    \draw[black] (-4, -1.7) rectangle (4, 1.2);
    \node at (0,0.95) {\textbf{\textit{Satisfies neither CC nor IoC}}};
    \node at (0,0.6) {
        \begin{tabular}{l l l}
            $\bullet RP$ &  $\bullet PV$ & $\bullet \UCf^\dagger$
        \end{tabular}
    };

    \fill[set2] (-3.5, -1.6) rectangle (3.5, 0.3);
    \draw[black] (-3.5, -1.6) rectangle (3.5, 0.3);
    \node at (0,0.05) {\textbf{\textit{Satisfies IoC}}};
    \node at (0,-0.35) { 
        \begin{tabular}{@{\hskip 0.2cm} l @{\hskip 0.2cm} l @{\hskip 0.2cm} l @{\hskip 0.2cm} l @{\hskip 0.2cm} l @{\hskip 0.2cm} l @{\hskip 0.2cm} l @{\hskip 0.2cm} l @{\hskip 0.2cm} l}
            $\bullet STV$ & $\bullet AS$ & $\bullet BP$ & $\bullet RP_N$ & $\bullet SC$ & $\bullet Sz^\dagger$ & $\bullet Sm^\dagger$
        \end{tabular}
    };

    \fill[set3] (-2, -1.5) rectangle (2, -0.6); 
    \draw[black] (-2, -1.5) rectangle (2, -0.6);
    \node at (0,-0.85) {\textbf{\textit{Satisfies CC}}}; 
    \node at (0,-1.2) { 
        \begin{tabular}{l l}
             $\bullet RP_i$ & $\bullet \UCg^\dagger$
        \end{tabular}
    };
    
\end{tikzpicture}
\caption{Behavior of SCFs from \Cref{tab:scfs} w.r.t~IoC/CC. $\dagger$ indicates majoritarian SCFs.}\label{fig:diagram}
\end{figure}

\subsection{Aside: Majoritarian SCFs}\label{sec:majoritarian}

While \emph{tournaments} (complete and asymmetric binary relationships over $\cand$) are not the main focus of this paper, it is worth briefly discussing how our results in this section relate to prior results on CC \emph{tournament solutions} (TSs), which map tournaments to sets of winners. As noted in \Cref{sec:related}, \citet{Laffond96:Composition}  introduce two separate definitions of components, in tournaments and in profiles (see \Cref{appdef:component_tour,appdef:component_prof} in our \Cref{appsec:bg}), and thus two separate definitions of CC for SCFs and for TSs. Subsequent work has primarily focused on the latter, showing TSs such as uncovered set, the minimal covering set, and the Banks set are CC~\citep{Laffond96:Composition,Laslier97:Tournament}. 

Of course, if $|\voter|$ is odd, the pairwise defeats in $\profile$ define a tournament, so any TS can be thought of as an SCF that maps $\profile$ to the winners of this induced tournament. However, for a TS to be well-defined as an SCF (without assuming odd $|N|$), it must be \emph{extended} to cases where the pairwise defeat relationship may contain ties (equivalently, to incomplete tournaments). Such induced SCFs are called \emph{majoritarian}. For example, $\UCg$ in \Cref{tab:scfs} is an extension of the TS \emph{uncovered set}. Another extension of the same TS follows from the work of \citet{Fishburn77:Condorcet}, and is defined as follows (recall that we say $a$ \emph{left-covers} $b$ if any $c$ that pairwise defeats $a$ also pairwise defeats $b$):
\begin{align*}
    \UCf(\profile)=\{a\in \cand: \nexists b \in A\text{ such that }b\text{ left covers }a\text{ but }a\text{ does not left-cover }b\}.
\end{align*}
It can be checked that $\UCf(\profile)=\UCg(\profile)$ whenever pairwise defeats have no ties, \emph{i.e.}, they are extensions of the same TS. Crucially, even though uncovered set is CC as a TS, $\UCf$ is not even IoC~\citep{Holliday23:Split}! This demonstrates that \textbf{a TS being CC is not sufficient for its SCF extension to be CC}. As we show next, $\UCg$ (\Cref{tab:scfs}) in fact maintains the CC property.
\begin{restatable}{proposition}{ucg}\label{prop:ucg}
    $\UCg$ is CC.
\end{restatable}

The disparity between $\UCf$ and $\UCg$ motivates future work in investigating whether other TSs known be CC can be extended into SCFs while maintaining CC. Existing negative results for TSs, on the other hand, readily generalize to any of their extensions. This is because \citet{Laffond96:Composition} show that for any tournament and a decomposition $\decomp$ into its (tournament) components, there exists some preference profile (that induces this tournament) for which $\decomp$ is once again a valid decomposition (their Prop. 1); this can be used to show that the CC definitions for TSs and their SCF interpretations coincide under the odd $|N|$ assumption \cite[Prop. 2]{Laffond96:Composition}. Since $\Sm$ and $\Sz$ (\Cref{tab:scfs}) are both SCF extensions of the TS \emph{top cycle}, which is not CC, we get:
\begin{proposition}[{Consequence of \citet[Props. 2, 5]{Laffond96:Composition}}] $\Sm$ and $\Sz$ are not CC.
    
\end{proposition}
\noindent For a summary of our results from \Cref{sec:ioc_rules}, see \Cref{fig:diagram}.
