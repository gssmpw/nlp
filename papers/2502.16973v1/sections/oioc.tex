\section{Obvious Independence of Clones}\label{sec:oioc}
 We now investigate what IoC and CC tell us about \emph{strategic behavior} under an election using an IoC/CC SCF. Unlike \citet{Elkind11:Cloning}, who study strategic cloning in a model where each clone set has a central ``manipulator'' that single-handedly decides how many clones run in the election, we will be looking at the strategies of individual candidates, who can personally decide to run in the election or drop out. Our motivation for this is that, as we have seen in \Cref{sec:cc_transform}, a single candidate can be a member of multiple non-trivial (potentially nested) clone sets of different size, and its preference over these sets (and thus over other candidates in them) may vary. In order to formalize this intuition, we now define a clone-based metric over the candidate set $\cand$. Given $\profile$, for each  $a,b \in \cand$, define $d_{\profile}(a,b)=  |K| -1$, where $K$ is the smallest clone set containing both $a$ and $b$.

\begin{restatable}{proposition}{clonemetric}\label{prop:metric}
    For any $\profile$, $d_{\profile}$ is a metric over $\cand$. 
\end{restatable}
 
Now, given $\profile$ and an SCF $f$, consider a normal-form game (NFG) $\Gamma^f_{\profile}$ where the players are the candidates, and each of them has two actions: run ($R$) and drop out ($D$). For simplicity, we assume $f$ is decisive; \emph{i.e.}, it outputs a single candidate. If exactly $S \subseteq \cand$ play $R$, the utility of any $a \in \cand$ is a (strictly) decreasing function of $d_{\profile}(a, f(\profile|_S))$. This follows from the assumption that clones represent proximity in some space (\emph{e.g.}, for political elections, this could be the ideological landscape): the closer the winner is to a candidate, the happier that candidate is.  If \emph{all} candidates pick $D$, then the election has no winner, which we assume gives everyone the worst utility. 

An action is a \emph{dominant strategy} of player $a$ if it brings (weakly) higher utility than any other action, no matter how $a$'s opponents play. A (pure) strategy profile  $\boldsymbol{s}=(s_a)_{a \in A}$ specifies an action $s_a \in \{R,D\}$ for each player $a \in A$. We say $\boldsymbol{s}$ is a \emph{pure-strategy Nash equilibrium (PNE)} if no player $a\in A$ can strictly increase her utility by unilaterally changing her action~\citep{Nash50:Non}. 

The setting of $\Gamma^f_{\profile}$ is a restriction of the more general \emph{strategic candidacy} model introduced by \citet{Dutta01:Strategic}, where candidates also have a preference over each other, and accordingly choose to run or not. Since $d_{\profile}(a,b)=0$ if and only if $a=b$ by \Cref{prop:metric}, our setting fulfills the condition of \emph{self-supporting} preferences (\emph{i.e.}, all candidates like themselves the best), which is taken as a natural domain restriction by \citeauthor{Dutta01:Strategic}~and much of the subsequent work on strategic candidacy. An SCF is called \emph{candidate stable} if for all profiles, the action profile where \emph{all} candidates are running is a PNE. For example, in our setting with  $\Gamma^f_{\profile}$, $PV$ is not candidate stable.

\begin{example}
    Consider $\Gamma^{PV}_{\profile}$, where $\profile$ is the profile from \Cref{fig:bg_eg} (left). In the strategy profile in which all candidates play $R$ (say $\boldsymbol{s}^R$), $b$ wins. However, if $a_2$ deviates to $D$, then $a_1$ wins. Since $d_{\profile}(a_1,a_2)=1<3=d_{\profile}(b,a_2)$, this deviation increases the utility of $a_2$, proving $\boldsymbol{s}^R$ is not a PNE.
\end{example}

Crucially, \citet{Dutta01:Strategic} show that in the general setting with no restrictions on the preferences of voters or of candidates (except the latter being self-supporting), the \emph{only} SCF that is both unanimous (a candidate is picked if all voters rank her first) and candidate-stable is \emph{dictatorship} (\emph{i.e.}, a single voter decides the outcome). As we show next, this is not the case in our restricted setting.

\begin{restatable}{proposition}{iocds}\label{prop:ioc_ds}
    If $f$ is IoC, then $R$ is a dominant strategy in $\Gamma^f_{\profile}$ for all candidates. 
\end{restatable}

As such, in our setting, IoC rules not only achieve candidate stability, but strengthen it, as all candidates running is a \emph{dominant-strategy Nash equilibrium}. \Cref{prop:ioc_ds} is in line with prior results showing how restricting the rankings of voters (\emph{e.g.}, to profiles with a Condorcet winner~\citep{Dutta01:Strategic}) or the rankings of both the voters and candidates (\emph{e.g.}, to single-peaked preferences~\citep{Samejima07-Strategic}) can circumvent the impossibility result by \citet{Dutta01:Strategic} in the general setting. In our setting, we require that all preferences are consistent with \emph{some} inherent clone structure, but since there are infinitely many such structures, this effectively puts no restriction on the preferences of the voters, but only on those of the candidates. Further, \Cref{prop:ioc_ds} formalizes the interpretation of IoC as ``strategy-proofness for candidates,'' in the sense that any candidate who is willing to run will not drop out due to fear of hurting  like-minded candidates (\emph{e.g.}, her political party).\footnote{This is not the only advantage of an IoC/CC rule; \emph{e.g.}, candidates also cannot make their opponents' clone set lose by nominating more clones in this set. In our model, we focus on the choice of whether to run.}

However, as demonstrated by \citet{Li17:Obviously}, the benefits of a property of a mechanism might only be materialized if the agents actually \emph{believe} that the property does indeed hold. If a candidate needs to go through complicated ``what if'' steps in order to believe that an SCF is indeed IoC, she might still drop out of the race (even though running is her dominant strategy), either out of fear of hurting her party, or of being blamed by the voters for doing so.

In order to characterize the ``obviousness'' of IoC, we turn to \emph{obviously dominant strategies}~\citep{Li17:Obviously}, which inherently deal with extensive-form games (EFG), in which players take actions in turns. Intuitively, an EFG can be represented by a rooted tree; at each node, the player associated with that node takes an action, each leading to a child node. The nodes belonging to each player are partitioned into \emph{information sets}; a player cannot tell any two nodes in the same information set apart. Below, we introduce the definition of obviously dominant strategies informally for games where each player acts once; the full definition (along with that of EFGs) is in \Cref{appsec:efg-ods}.

\begin{definition}[{\citealt[Informal]{Li17:Obviously}}]
     An action $s$ is \emph{obviously dominant} for player $a$ if for any other action $s'$, starting from the point in the game when $a$ must take an action, the best possible outcome from $s'$ is no better than the worst possible outcome from $s$.
\end{definition}

Here, the ``best'' and ``worst'' outcomes are defined over the actions of candidates that act along with or after $a$. For example, interpreting $\Gamma^f_{\profile}$ from above as an EFG where all candidates act simultaneously, running ($R$) may \emph{not} be an obviously dominant strategy, even if $f$ is IoC, as the next example shows.



\begin{figure}[t]
    \tikzset{
        every path/.style={-},
        every node/.style={draw},
    }
    \forestset{
    subgame/.style={regular polygon,
    regular polygon sides=3,anchor=north, inner sep=1pt},
    }
    \centering
        \begin{forest}
            [\scriptsize{$c$},cgs,name=c1,s sep=14pt,l sep=10pt
                [\scriptsize{$b$},bgs,name=b1,el={1}{R}{},s sep=14pt,l sep=10pt
                    [\scriptsize{$a_2$},a2gs,name=a21,el={3}{R}{},s sep=14pt,l sep=10pt
                        [\scriptsize{$a_1$},a1gs,name=a11,el={4}{R}{},s sep=14pt,l sep=10pt
                            [\text{$a_1$},terminal,el={2}{R}{},yshift=-3.3pt]
                            [\text{$a_2$},terminal,el={2}{D}{},yshift=-3.3pt]
                        ]
                        [\scriptsize{$a_1$},a1gs,name=a12,el={4}{D}{},s sep=14pt,l sep=10pt
                            [\text{$a_1$},terminal,el={2}{R}{},yshift=-3.3pt]
                            [\text{$b$},terminal,el={2}{D}{},yshift=-3.3pt]
                        ]
                    ]
                    [\scriptsize{$a_2$},a2gs,name=a22,el={3}{D}{},s sep=14pt,l sep=10pt
                        [\scriptsize{$a_1$},a1gs,name=a13,el={4}{R}{},s sep=14pt,l sep=10pt
                            [\text{$c$},terminal,el={2}{R}{},yshift=-3.3pt]
                            [\text{$c$},terminal,el={2}{D}{},yshift=-3.3pt]
                        ]
                        [\scriptsize{$a_1$},a1gs,name=a14,el={4}{D}{},s sep=14pt,l sep=10pt
                            [\text{$c$},terminal,el={2}{R}{},yshift=-3.3pt]
                            [\text{$c$},terminal,el={2}{D}{},yshift=-3.3pt]
                        ]
                    ]
                ]
                [\scriptsize{$b$},bgs,name=b2,el={1}{D}{},s sep=14pt,l sep=10pt
                    [\scriptsize{$a_2$},a2gs,name=a23,el={3}{R}{},s sep=14pt,l sep=10pt
                        [\scriptsize{$a_1$},a1gs,name=a15,el={4}{R}{},s sep=14pt,l sep=10pt
                            [\text{$a_2$},terminal,el={2}{R}{},yshift=-3.3pt]
                            [\text{$a_2$},terminal,el={2}{D}{},yshift=-3.3pt]
                        ]
                        [\scriptsize{$a_1$},a1gs,name=a16,el={4}{D}{},s sep=14pt,l sep=10pt
                            [\text{$a_1$},terminal,el={2}{R}{},yshift=-3.3pt]
                            [\text{$b$},terminal,el={2}{D}{},yshift=-3.3pt]
                        ]
                    ]
                    [\scriptsize{$a_2$},a2gs,name=a24,el={3}{D}{},s sep=14pt,l sep=10pt
                        [\scriptsize{$a_1$},a1gs,name=a17,el={4}{R}{},s sep=14pt,l sep=10pt
                            [\text{$a_2$},terminal,el={2}{R}{},yshift=-3.3pt]
                            [\text{$a_2$},terminal,el={2}{D}{},yshift=-3.3pt]
                        ]
                        [\scriptsize{$a_1$},a1gs,name=a18,el={4}{D}{},s sep=14pt,l sep=10pt
                            [\text{$a_1$},terminal,el={2}{R}{},yshift=-3.3pt]
                            [\text{$\emptyset$},terminal,el={2}{D}{},yshift=-3.3pt]
                        ]
                    ]
                ]
            ]
            \draw[infosetb] (b1)--(b2);
            \draw[infoseta2] (a21)--(a22)--(a23)--(a24);
            \draw[infoseta1] (a11)--(a12)--(a13)--(a14)--(a15)--(a16)--(a17)--(a18);
        \end{forest}
    \caption{EFG representation of $\Gamma_{\profile}^{\STV}$ for $\profile$ from Fig.~\ref{fig:bg_eg}. Terminals show the winner under that action profile. Information sets are joined by dotted lines. For $a_1$, the worst outcome of running is  $c$ winning, and the best outcome of dropping out is $a_2$ winning, so running is not an obviously dominant strategy for $a_1$.
    \label{fig:nfgstv}}
\end{figure}


\begin{example}

    Consider $\Gamma^{\STV}_{\profile}$, where $\profile$ is from Fig.~\ref{fig:bg_eg}. From the perspective of $a_1$, the \emph{worst} outcome of running ($R$) is if $a_2$ and $c$ also play $R$, but $b$ plays $D$, making $c$ the winner. The \emph{best} outcome of $a_1$ dropping out ($D$) is if everyone else plays $R$, making $a_2$ win. Since $d_{\profile}(a_1,c)=3>1=d_{\profile}(a_1,a_2)$, $R$ is \emph{not} an obviously-dominant strategy for $a_1$. A tree representation of $\Gamma^{\STV}_{\profile}$ is in \Cref{fig:nfgstv}.
\end{example}

What if we had used a composition-consistent SCF $f$ instead? Recall that by (3) of \Cref{thm:cc_transform}, $f$ can be implemented using \Cref{alg:cc-transform}, \emph{i.e.}, by running it on the PQ-tree of the input profile. The key observation is that when running \Cref{alg:cc-transform}, we can postpone asking a candidate if she is running or not until we reach the internal node that is the immediate ancestor of that candidate. More formally, consider then an alternate EFG $\Lambda^f_{\profile}$ where the actions and utilities are the same as  $\Gamma^f_{\profile}$, but the winner is determined by running \Cref{alg:cc-transform} on inputs $f$ and $\profile$, with the following process after each node $\node$ is dequeued from $\queue$:
\begin{itemize}
    \item If $\node$ is a P-node, then all the children of $\node$ that are actual candidates (\emph{i.e.}, leaf nodes) are asked (simultaneously) to pick $R$ or $D$. Given $S'$ are the child nodes that chose $D$, $f$ is run on $\profile^{\text{decomp}(\node,T)}\setminus S'$ to decide which branch to follow. If the winner is a leaf, the game is over. 
    \item If $\node$ is a Q-node, say $W'=f(\profile^\mathcal{K})|_{\{\node_1(\node,T),\node_2(\node,T)\}}$. If $W'=\node_1(\node,T)$ and $\node_1(\node,T)$ is an internal node, then it is enqueued. Otherwise, the (single) candidate corresponding to $\node_1(\node,T)$ is asked to pick $R$, in which case it is the winner, or $D$, in which case the process is repeated with $\node_2(\node,T), \node_3(\node,T),\ldots$ until either an internal node or a candidate that plays $R$ is encountered. If $W'=\node_2(\node,T)$, on the other hand, the identical process is followed, except starting from $\node_\ell(\node,T)$, where $\ell=|\text{decomp}(\node,T)|$, and moving backwards to $\node_{\ell-1}(\node,T),\node_{\ell-2}(\node,T),\ldots$
    
    \item In either case, if all the children of $\node$ are leaf nodes and all play $D$, then the algorithm moves back to the parent node of $\node$, and repeats the computation there (without re-asking all the leaf nodes) with $\node$ also dropped out of the summary.
\end{itemize}

\begin{figure}[t]
    \tikzset{
        every path/.style={-},
        every node/.style={draw},
    }
    \forestset{
    subgame/.style={regular polygon,
    regular polygon sides=3,anchor=north, inner sep=1pt},
    }
    \centering
        \begin{forest}
            [\scriptsize{$c$},cgs,name=c1,s sep=14pt,l sep=10pt
                [\scriptsize{$b$},bgs,name=b1,el={1}{R}{},s sep=14pt,l sep=10pt
                    [\scriptsize{$a_2$},a2gs,name=a21,el={3}{R}{},s sep=14pt,l sep=10pt
                        [\scriptsize{$a_1$},a1gs,name=a11,el={4}{R}{},s sep=14pt,l sep=10pt
                            [\text{$a_2$},terminal,el={2}{R}{},yshift=-3.3pt]
                            [\text{$a_2$},terminal,el={2}{D}{},yshift=-3.3pt]
                        ]
                        [\scriptsize{$a_1$},a1gs,name=a12,el={4}{D}{},s sep=14pt,l sep=10pt
                            [\text{$a_1$},terminal,el={2}{R}{},yshift=-3.3pt]
                            [\text{$b$},terminal,el={2}{D}{},yshift=-3.3pt]
                        ]
                    ]
                        [\text{$c$},terminal,el={3}{D}{},yshift=-3.3pt]
                ]
                [\scriptsize{$b$},bgs,name=b2,el={1}{D}{},s sep=14pt,l sep=10pt
                    [\scriptsize{$a_2$},a2gs,name=a23,el={3}{R}{},s sep=14pt,l sep=10pt
                        [\scriptsize{$a_1$},a1gs,name=a15,el={4}{R}{},s sep=14pt,l sep=10pt
                            [\text{$a_2$},terminal,el={2}{R}{},yshift=-3.3pt]
                            [\text{$a_2$},terminal,el={2}{D}{},yshift=-3.3pt]
                        ]
                        [\scriptsize{$a_1$},a1gs,name=a16,el={4}{D}{},s sep=14pt,l sep=10pt
                            [\text{$a_1$},terminal,el={2}{R}{},yshift=-3.3pt]
                            [\text{$b$},terminal,el={2}{D}{},yshift=-3.3pt]
                        ]
                    ]
                    [\scriptsize{$a_2$},a2gs,name=a24,el={3}{D}{},s sep=14pt,l sep=10pt
                        [\scriptsize{$a_1$},a1gs,name=a17,el={4}{R}{},s sep=14pt,l sep=10pt
                            [\text{$a_2$},terminal,el={2}{R}{},yshift=-3.3pt]
                            [\text{$a_2$},terminal,el={2}{D}{},yshift=-3.3pt]
                        ]
                        [\scriptsize{$a_1$},a1gs,name=a18,el={4}{D}{},s sep=14pt,l sep=10pt
                            [\text{$a_1$},terminal,el={2}{R}{},yshift=-3.3pt]
                            [\text{$\emptyset$},terminal,el={2}{D}{},yshift=-3.3pt]
                        ]
                    ]
                ]
            ]
            \draw[infosetb] (b1)--(b2);
            \draw[infoseta2] (a21)--(a23)--(a24);
            \draw[infoseta1] (a11)--(a12)--(a15)--(a16)--(a17)--(a18);
        \end{forest}
    \caption{$\Lambda^{STV^{CC}}_{\profile}$, for $\profile$ from Fig.~\ref{fig:bg_eg}, the PQ-tree of which is in Fig.~\ref{fig:tree} (right). For $a_1$, best outcome of not running is $a_2$ winning, which is no better than the worst outcome of running, which is also $a_2$ winning. Therefore, running is an obviously dominant strategy for $a_1$. A similar analysis applies for all other candidates.
    \label{fig:efgpvcc}}
\end{figure}
Intuitively, $\Lambda_{\profile}^f$ asks each candidate whether she is running only when this decision becomes relevant. Just like in $\Gamma^f_{\profile}$, each player in $\Lambda^f_{\profile}$ has a single information set, since she is not aware of the actions of the players that are acting before or simultaneously with her; she only knows her parent node is reached. The winner in $\Lambda_{\profile}^f$ is precisely the winner of applying $f^{CC}$ directly to $\profile \setminus S'$, where $S'$ are the players that picked $D$.\footnote{There is a slight caveat here: the leaf nodes that are children of internal nodes that never got visited did not get to play $R$ or $D$ in  $\Lambda^f_{\profile}$. Any choice these candidates could have made does not change the result of $f$ as long as \emph{at least one} candidate of each non-trivial clone set were to pick $R$. This is in line with the assumption made by \citet{Elkind11:Cloning} that at least one clone of each clone set will be in the election. Indeed, this is not a far-fetched assumption: in practice, it is the leadership of a political party that decides to participate in an election, before individual members of the party make up their mind about whether to run.} If $f$ is CC to begin with, this exactly corresponds to $f(\profile \setminus S')$ by \Cref{thm:cc_transform}(3). \Cref{fig:efgpvcc} shows the game tree for $\Lambda_{\profile}^f$ for $\profile$ from \Cref{fig:bg_eg_2} and $f=STV^{CC}$.

Crucially, this implementation of $f^{CC}$ allows us to strengthen Prop.~\ref{prop:ioc_ds}, achieving obviousness.
\begin{restatable}{theorem}{ccods}\label{thm:cc_ods}
    For any neutral $f$, $R$ is an obviously-dominant strategy in $\Lambda^f_{\profile}$ for all candidates.
\end{restatable}
The proof relies on the observation that in  $\Lambda^f_{\profile}$, when a candidate is asked to decide between $R$ and $D$, \Cref{alg:cc-transform} has already reached her parent node, which is the smallest non-trivial clone set containing her. Thus, the best case of $D$ and worst case of $R$ are both one of her second-favorite group of candidates (after herself) winning, achieving obvious strategy-proofness (OSP).

\Cref{thm:cc_ods} has strong practical implications: since any CC rule can be implemented with \Cref{alg:cc-transform}, the decision of a candidate to drop out of an election can be postponed until \emph{after} she learns whether her smallest clone set has won. Hence, using a CC rule, the election result will not change if we replace the candidates' names on the ballots with party names, and hold in-party primaries for the winners afterwards. In contrast, with rules that are just IoC, the results within the party vary based on whether internal primaries are held (\Cref{eg:GLOC}). Without primaries, CC rules can also derive clone sets \emph{a posteriori} from the votes, rather than simply assuming a party to be a clone set.

Importantly, obviousness is also relevant for contexts where candidates (or, for settings with abstract candidates, their deployers) are perfectly capable of reasoning about an SCF and its properties, but they worry about manipulation of the SCF by the entity implementing it (also called \emph{agenda control}~\citep{Lang13:New}). As shown by \citet{Li17:Obviously}, choice rules that are OSP-implementable offer a significant advantage in these settings, as they are exactly those that are supported by \emph{bilateral commitments} (partial commitments by the planner such that, if violated, those violations can be observed by the agents themselves without communicating among each other). In the context of $\Lambda^f_{\profile}$, instead of committing to using a specific SCF, the planner can commit to each candidate he interacts with that, if she decides to run, the winner will be some member of her smallest non-trivial clone set. This (1) is enough to convince the candidate to run and (2) ensures that a violation of this commitment can be observed by the candidate (by looking at the outcome of the election).

In general, the connection we establish between IoC and CC yields new and natural interpretations of the two properties: we can view CC as a way of exposing the \emph{obviousness} of IoC.\footnote{In fact, in prior work, we have introduced a definition for \emph{obvious independence of clones} that is identical to composition consistency, without realizing the connection at the time~\citep{Berker22:Obvious}.}



