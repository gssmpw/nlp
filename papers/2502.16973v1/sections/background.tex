    
\section{Preliminaries}\label{sec:bg}
\paragraph{Profiles and clones.}\label{subsec:clones} We consider a set of \textit{candidates} $\cand$ with $|\cand|=m$ and a set of \textit{voters} $\voter = \{1,\ldots, n\}$. A {\em ranking} over $\cand$ is an asymmetric, transitive, and complete binary relation $\succ$ on $\cand$. Let $\mathcal{L}(\cand)$ denote the set of all rankings over $\cand$; $a \succ_r b$ indicates that $a$ is ranked above $b$ in a ranking $r$. Each voter $i \in \voter$ has a ranking $\sigma_i \in \mathcal{L}(\cand)$; we collect the rankings of all voters in a \textit{preference profile} $\profile \in \mathcal{L}(\cand)^n$. The next definition helps identify sets of similar candidates (according to voters). 

\begin{definition}[{\citealt[\oldS I]{Tideman87:Independence}; \citealt[Def. 4]{Laffond96:Composition}}]\label{def:clones}
    Given a preference profile $\profile$ over candidates $\cand$, a nonempty subset of candidates $\clone \subseteq \cand$ is a \emph{set of clones} with respect to $\profile$ if for each $a, b\in K$ and each $c\in \cand\setminus K$, no voter ranks $c$ between $a$ and $b$.
\end{definition}

All preference profiles admit two types of \textit{trivial} clone sets:\footnote{\citet{Tideman87:Independence} in fact excludes trivial clone sets. We use the definition from the work of \citet{Elkind10:Clone}.} (1) the entire candidate set $\cand$, and (2) for each $a\in \cand$, the singleton $\{a\}$. We call all other clone sets \textit{non-trivial}. For example, for the profile in \Cref{fig:eg_prof}, the only non-trivial clone set is $\{b,c\}$.


\begin{figure}[t]
    \centering
    \begin{tabular}{|c|c|c|c|}
        \hline
        6 voters & 5 voters & 2 voters & 2 voters \\ \hline
        \cgreen $b$ & \cred $d$  & \cyellow $a$ & \cyellow $a$ \\ \hline
       \cblue $c$ & \cblue $c$ & \cred $d$  & \cred $d$  \\ \hline
        \cyellow $a$ &\cgreen $b$ & \cgreen $b$ & \cblue $c$ \\ \hline
        \cred $d$ & \cyellow $a$& \cblue $c$ & \cgreen $b$ \\
        \hline
    \end{tabular}
    \caption{A preference profile. Columns show rankings, with the bottom row ranked last. The first row shows the number of copies of each ranking (\emph{e.g.}, leftmost column indicates 6 voters rank $b\succ c\succ a\succ d$).}\label{fig:eg_prof}
\end{figure} 

\paragraph{Social choice functions and axioms.}\label{subsec:axioms}
A \textit{social choice function (SCF)} is a mapping $f$ that, given a set of candidates $A$ and a profile $\profile$ over $A$,
outputs a nonempty subset of $A$; the candidates in $f(\profile)$ are the {\em winners} in $\profile$ under $f$. An SCF $f$ is {\em decisive} on $\profile$ if $|f(\profile)|=1$. \Cref{tab:scfs} contains the descriptions of the SCFs that we consider.\footnote{While we describe some SCFs by their winner determination \textit{procedures}, the SCFs themselves are the \textit{functions} that  output the respective winners, and these functions may be computed by other---possibly more efficient---algorithms.} 

We list some desirable properties (\emph{axioms}) for SCFs. For example, an SCF is \emph{neutral} (resp., \emph{anonymous}) if its output is robust to relabeling the candidates (resp., voters); for a formal definition, see {\citet[Def. 2.4, 2.5]{Brandt16:Handbook}}. Further, an SCF $f$ satisfies Smith (resp., Condorcet) consistency if $f(\profile) \subseteq Sm(\profile)$ for all $\profile$ (resp., for all $\profile$ with $|Sm(\profile)|=1$), where $Sm$ is defined in Table~\ref{tab:scfs}.

Next, we will present two axioms that both aim to capture the idea of robustness against strategic nomination. In what follows, we write $\profile \setminus \cand'$ to denote the profile obtained by removing the elements of $\cand' \subset \cand$ from each voter's ranking in $\profile$ while preserving the order of all other candidates.

\begin{definition}[\citealt{Zavist89:Complete}]\label{def:ioc}
An SCF $f$ is \emph{independent of clones (IoC)} if for each profile $\profile$ over $A$ and each non-trivial clone set $\clone \subset A$ with respect to $\profile$,

(1) for all $a \in \clone$, 
\[
    \clone \cap f(\profile) \neq \emptyset \Leftrightarrow \left(\clone \setminus \{a\}\right) \cap f(\profile \setminus \{a\}) \neq \emptyset;  
\]
\indent (2) for all $a \in \clone$ and all $b \in A \setminus \clone$,
\[
    b \in f(\profile) \Leftrightarrow b \in f(\profile \setminus \{a\}).
\]
\end{definition}

Intuitively, IoC dictates that deleting one of the clones in a non-trivial clone set $K$ must not alter the winning status of $K$ as a whole, or of any candidate not in $K$. In particular, it implies that one cannot substantively change the election outcome by nominating new copy-cat candidates. 
 \begin{figure}[t]
    \centering
    \begin{tabular}{|c|c|c|c|}
        \hline
    3 voters &  2 voters & 4 voters & 3 voters \\
    \hline
    \cyellow$a_1$ & \cred$a_2$ & \cgreen$b$ & \cblue$c$ \\ \hline
    \cred$a_2$ & \cyellow$a_1$ & \cblue$c$ & \cred$a_2$ \\ \hline
    \cgreen$b$ & \cgreen$b$ & \cred$a_2$ & \cyellow$a_1$ \\ \hline
    \cblue$c$ & \cblue$c$ & \cyellow$a_1$ & \cgreen$b$ \\
    \hline
    \end{tabular} $\xRightarrow[\text{remove }a_2]{}$ \begin{tabular}{|c|c|c|}
        \hline
    5 voters & 4 voters & 3 voters \\  
    \hline
    \cyellow$a_1$ & \cgreen$b$ & \cblue$c$ \\ \hline
      \cgreen$b$ & \cblue$c$ & \cyellow$a_1$ \\ \hline
     \cblue$c$  &  \cyellow$a_1$ & \cgreen$b$ \\ 
    \hline
    \end{tabular} \caption{(Left) Example profile $\profile$. (Right)  $\profile\setminus \{a_2\}$.}\label{fig:bg_eg}
\end{figure}

\begin{example}\label{eg:ioc}
    In the profile $\profile$ in Fig.~\ref{fig:bg_eg} (left), $K = \{a_1,a_2\}$ is a clone set.
    Plurality Voting (PV) outputs $b$ as the unique winner.
    However, $\mathit{PV}(\profile \setminus \{a_2\}) = \{a_1\}$, since with $a_2$ gone, $a_1$ now has 5 voters ranking it first (\Cref{fig:bg_eg}, right). This violates both conditions~(1) and~(2) from Definition~\ref{def:ioc}. 

    In contrast, $\STV$ eliminates
    $a_2$ (whose votes then transfer to $a_1$), then $c$ and finally $b$, 
    so that $a_1$ is elected; moreover, it produces the same result on $\profile\setminus\{a_2\}$.
    This is in line with the fact that $\STV$ is IoC \citep{Tideman87:Independence}. 
\end{example}

To formulate the second axiom that deals with strategic nomination, we must first introduce a few additional concepts.
  \begin{definition}\label{def:clone_grouping}
Given a preference profile $\profile$ over candidates $\cand$, a set of sets $\decomp=\{\clone_1,\clone_2,\ldots,\clone_\ell\}$, where $\clone_i \subseteq A$ for all $i\in [\ell]$, is a \emph{(clone) decomposition} with respect to $\profile$ if
\begin{enumerate}
    \item $\decomp$ is a partition of $\cand$ into pairwise disjoint subsets, and
    \item Each $\clone_i$ is a non-empty clone set with respect to $\profile$.
\end{enumerate}
\end{definition}
Every profile has at least two decompositions: the \textit{null} decomposition $\decomp_{null}=\{A\}$ and the \textit{trivial} decomposition $\decomp_{triv}=\{ \{a\}\}_{a \in A}$. Given a decomposition $\decomp$ with respect to $\profile$, for each $i \in N$ let $\sigma^\decomp_i$ be voter $i$'s ranking over the sets in $\decomp$; this is well-defined, since each clone set forms an interval in $\sigma_i$. The profile $\profile^\decomp = \{\sigma^\decomp_i\}_{i \in N}$ over $\decomp$ is called the \textit{summary} of $\profile$ with respect to the decomposition $\decomp$. For each $K \in \decomp$, we write $\profile|_{K}$ to denote the restriction of $\profile$ to $K$, so that $\profile|_{K} \equiv \profile \setminus (A \setminus K)$. 

\begin{definition}\label{def:gloc} 
    The \emph{composition product} function of an SCF $f$ is a function $\gloc_f$ that takes as input a profile $\profile$ and a clone decomposition $\decomp$ with respect to $\profile$ and outputs $\gloc_f(\profile, \decomp) \equiv \bigcup_{K \in f\left(\profile^\decomp\right)  }f(\profile|_\clone)$.
\end{definition}

Intuitively, $\gloc_f$ first runs the input SCF $f$ on the summary (as specified by $\decomp$), collapsing each clone set into a meta-candidate $\clone_i$. It then ``unpacks'' the clones of each winning clone set, and runs $f$ once again on each.
\begin{example}\label{eg:GLOC}
    For the profile $\profile$ from Figure~\ref{fig:bg_eg} (left), it holds that $\decomp=\{K_a,K_b,K_c\}$ with $K_a=\{a_1,a_2\}$, $K_b=\{b\}$, $K_c=\{c\}$ is a valid clone decomposition with respect to $\profile$. Figure~\ref{fig:bg_eg_2} shows $\profile^\decomp$ and $\profile|_{\clone_a}$. We have $\STV(\profile^\decomp)=\{\clone_a\}$ and $\STV(K_a)=\{a_2\}$, implying $\gloc_{\STV}(\profile, \decomp) =\{a_2\}$.
\end{example}

Together, \Cref{eg:ioc,eg:GLOC} imply that $\STV(\profile) \neq \gloc_{\STV}(\profile, \decomp)$ for this $\profile$ and $\decomp$; \emph{i.e.}, that $\STV$ does not respect this clone decomposition---even though the winners in $\STV$ and $\gloc_\STV$ are from the same clone set. We now state the composition consistency axiom, which precisely requires a rule to respect all possible clone decompositions.

\begin{definition}[{\citealt[Def. 11]{Laffond96:Composition}}]\label{def:oioc}
      A neutral\footnote{\citet{Laffond96:Composition} define composition consistency (CC) only for neutral SCFs; this is without loss of generality, as they treat $\profile^\decomp$ as a profile over candidates $\{1,2,\ldots, |\decomp|\}$, in which case CC automatically implies neutrality by the trivial decomposition. \citet{Brandl16:Consistent} instead use a definition where $\profile^\decomp$ is simply $\profile$ with all but one candidate removed from each $K_i$; nevertheless, they show that in this model too CC implies neutrality (their Lemma 1). We explicitly state this as a prerequisite for simplicity.} SCF $f$ is \emph{composition-consistent (CC)} if for all preference profiles $\profile$ and all clone decompositions $\decomp$ with respect to $\profile$, we have $f(\profile) =\gloc_f(\profile, \decomp)$.
\end{definition}

\begin{figure}[t]
\centering

 \begin{tabular}{|c|c|c|}
        \hline
    5 voters & 4 voters & 3 voters \\
    \hline
    \corange$\clone_a$  & \cgreen$\clone_b$ & \cblue$\clone_c$ \\ \hline
    \cgreen$\clone_b$ & \cblue$\clone_c$ & \corange$\clone_a$ \\ \hline
   \cblue$\clone_c$ & \corange$\clone_a$ & \cgreen$\clone_b$ \\
    \hline
    \end{tabular} \quad \quad
  \begin{tabular}{|c|c|}
        \hline
    3 voters & 9 voters \\
    \hline
  \cyellow{$a_1$} & \cred{$a_2$} \\
  \hline
  \cred{$a_2$} & \cyellow{$a_1$} 
  \\
  \hline
    \end{tabular}


    \captionsetup{width=.8\linewidth}
    \caption{(Left) $\profile^\decomp$, where clone sets from $\profile$ in \Cref{fig:bg_eg} are condensed into candidates $\clone_a$, $\clone_b$, and $\clone_c$. (Right) $\profile|_{\clone_a}$, where $\profile$ is limited to members of $\clone_a$.}
    \label{fig:bg_eg_2}
\end{figure}


CC dictates that an SCF should choose the ``best'' candidates from the ``best'' clone sets. In contrast, IoC is much more permissive when it comes to choosing a candidate from a best clone set. Indeed, in \Cref{prop:cctoioc} we show that CC implies IoC. On the other hand, \Cref{eg:ioc,eg:GLOC} demonstrate that the converse is false: they show that $\STV$, which is IoC, is not CC. Later, we analyze other IoC rules to determine whether they are CC (\Cref{sec:ioc_rules}).  

\paragraph{Social choice functions considered in this paper.}\label{subsec:rules}
Prior work has shown each SCF in \Cref{tab:scfs} (with the exception of $\mathit{PV}$) to be IoC (see \citet{Holliday23:Split} for an overview). For some, winner determination may require tie-breaking (\emph{e.g.}, under $\STV$, candidates may tie for the lowest plurality score). We define the output of such SCFs as the set of candidates that win for \textit{some} tie-breaking rule, also called \emph{parallel-universes tiebreaking}~\citep{Conitzer09:Preference}. Crucially, this variant of $\RP$ is \textit{not} IoC~\citep{Zavist89:Complete}, a nuance we will address in detail in \Cref{subsec:rp}. Lastly, $\Sm, \Sz, \UCg$ are SCF extensions of \emph{tournament solutions} that are known to either fail or satisfy CC. However, whether they satisfy CC as SCFs is a more subtle issue, which we address in \Cref{sec:majoritarian}.

\begin{table*}[h!]
    \begin{threeparttable}
        \centering
        \begin{tabular}{|m{0.1548\linewidth}|m{0.0497\linewidth}|m{0.7095\linewidth}|}
            \hline  \centering \emph{Name of SCF} & \centering  $f$ &  \emph{Description of the SCF's output on input profile $\profile$} \\
            \hline\hline  \cpink \centering Plurality &\cpink \centering $\mathit{PV}$ & \cpink Outputs the candidate(s) ranked first by the most number of voters.\\
            \hline \centering \cellcolor{SeaGreen!20} Single~Transferable Vote &\cellcolor{SeaGreen!20} \centering $\STV$ & \cellcolor{SeaGreen!20} At each round, the candidate ranked top by the fewest voters is eliminated. Eventually a single candidate remains, becoming the winner.  \\\hline \centering  \cpink  \text{Ranked Pairs} \citep{Tideman87:Independence,Zavist89:Complete}& \cpink \centering $\RP$ & \cpink    Given a profile $\profile$ over candidates $\cand =\{a_i\}_{i \in [m]}$, construct the \textit{majority matrix} $M$, whose $ij$ entry is the number of voters who rank $a_i$ ahead of $a_j$ minus those who rank $a_j$ ahead of $a_i$. Construct a digraph over $\cand$ by adding edges for each $M[ij]\geq 0$ in non-increasing order, skipping those that result in a cycle. The winner is the source node.\\
            \hline \centering \cellcolor{SeaGreen!20} Beatpath (Schulze Method) \citep{Schulze10:New} & \cellcolor{SeaGreen!20} \centering  $\BP$  & \cellcolor{SeaGreen!20} Construct $M$ as in $\RP$, and the corresponding weighted digraph over $\cand$ without skipping edges\tnote{$\dagger$}. Let $S[i,j]$ be the width (min. weight edge) of the widest path from $a_i$ to $a_j$, computed, \emph{e.g.}, with the Floyd--Warshall algorithm. Then $a_i$ is a winner iff $S[i, j] \geq S[j,i]$ for all $j \in [m]$.\\
            \hline \cpink  \centering \citet{Schwartz86:Logic} set (GOCHA)&\cpink \centering $\Sz$ &   \cpink Given $\profile$, we say that $B \subseteq A$ is \emph{undominated} if no $a \in A \setminus B$ pairwise defeats (preferred to by a strict majority of voters) any $b \in B$. The winners are the union of minimal (by inclusion) undominated sets.\\
            \hline \centering \cellcolor{SeaGreen!20} \citet{Smith73:Aggregation} set (GETCHA) &\cellcolor{SeaGreen!20} \centering $\Sm$ & \cellcolor{SeaGreen!20} Outputs the smallest set of candidates who all pairwise defeat every candidate outside the set.\\ \hline \centering   \cpink Alternative-Smith~\citep{Tideman06:Collective}&  \cpink \centering $\AS$ &  \cpink (1) Eliminate all candidates not in $\Sm(\profile)$. (2) In the remaining profile, eliminate the candidate ranked top by the fewest voters. Repeat (1)-(2) until a single candidate remains; this is the winner.\\
            \hline \cellcolor{SeaGreen!20} \centering  Split Cycle \citep{Holliday23:Split}& \cellcolor{SeaGreen!20} \centering $\SC$ & \cellcolor{SeaGreen!20}Construct $M$ as in $\RP$ and the corresponding weighted digraph $G$ over $\cand$ without skipping edges. For each \textit{simple cycle} (cycles visiting each vertex at most once) in $G$, label the edge(s) with the smallest weight in that cycle. Discard all labeled edges (at once) to get $G'$. The winners are the candidates with no incoming edge in $G'$. \\
            \hline \centering  \cpink Uncovered Set \citep{Gillies59:Solutions} &  \cpink \centering $\UCg$ & \cpink Given $\profile$ and $a,b \in \cand$, we say that $a$ \emph{left-covers} $b$ if any $c \in \cand$ that pairwise defeats $a$ also pairwise defeats $b$. The winners are all $a \in \cand$ such that there is no $b \in \cand$ that left-covers \emph{and} pairwise defeats $a$.
            \\\hline
        \end{tabular}
        \caption{SCFs considered in this paper. Second column indicates our notation for the SCF as a function.}
        \label{tab:scfs}
        \begin{tablenotes}
            \footnotesize
                \item[$\dagger$]More generally, $\BP$ can be defined with various choices for edge weights \citep{Schulze10:New}. We use the ``margin'' variant; our results easily generalize to others.
        \end{tablenotes}
    \end{threeparttable}
\end{table*}