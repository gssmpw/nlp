\section{Introduction}

Social media platforms are flooded with immense amounts of new content every day. This constant stream of content has led to the reliance on \textit{algorithms} to \textit{curate} the content in users' social media feeds.  \textit{Algorithmic curation} is defined as the process of ``organizing, selecting, and presenting subsets of a corpus of information for consumption''~\cite{rader-gray-folk-theories}. In recent times, the most prominent examples of algorithmic curation are on short-form video platforms like TikTok, Instagram Reels, and YouTube Shorts. However, less thought of are the ``hot'' and ``popular'' feeds that exist on multiple platforms (e.g., GitHub, Reddit, X/Twitter) that also use algorithms to curate what is ``trending.'' 

Despite the prominent usage of algorithmic curation on social media, uncovering \textit{what factors influence curation algorithms} is challenging. This is because curation algorithms are proprietary, preventing those on the outside from knowing how these algorithms work, thus protecting companies' intellectual property. Additionally, algorithmic curation on social media is mostly done for specific users, i.e., what these algorithms recommend changes for each individual user. The prominence of personalized recommendations on social media presents two challenges: (1) understanding how these systems affect users more broadly, and (2) collecting enough data to support studies into these systems. To alleviate these challenges, we present a study that examines the trending feed on Reddit called r/popular.

\begin{figure}
    \centering
    \includegraphics[width=\linewidth]{images/popular-feed.png}
    \caption{A snapshot of the r/popular feed on Reddit.}
    \label{fig:popular-feed}
\end{figure}

\subsection{Reddit's r/popular Feed}

The r/popular feed\footnote{\url{https://www.reddit.com/r/popular}} on Reddit is one of the default feeds that is available to all users, with and without an account. The posts on the r/popular feed come from nearly all consenting communities on the platform---however, there are some subreddits, particularly ``not safe for work'' (NSFW) communities, that are omitted. The r/popular feed is a \textit{ranked} list (see Figure~\ref{fig:popular-feed}) where the rankings are based on a calculated ``hot'' score (likely based on recency, comment rate, and upvote score) as well as some measures for post and discussion quality.\footnote{\url{www.reddit.com/r/changelog/comments/9n3ix9/rpopular_is_changing/}} At a high level, the r/popular feed consolidates the most active posts on Reddit from numerous subreddits and serves as the front page to the platform.

Studying Reddit's r/popular feed comes with several benefits: (1) it is highly visited, (2) it is available to all users, and (3) it orders the posts consistently for all users, making it impactful for broad population. Its prominence also allows us to collect significant amounts of data to uncover what goes into Reddit's ranking decisions. Thus, analyzing the r/popular feed can provide key insights into how algorithmically curated feeds work and how algorithmic ranking decisions can impact user behaviors on platforms like Reddit.

\subsection{Our Contributions}
In this paper, we conduct an algorithmic audit of r/popular with two key objectives. The first is to understand what factors influence algorithmic ranking on r/popular. The second is to quantify how those decisions, specifically the feed's ranking decisions, affect the engagement on posts. Toward these goals, we ask three research questions.

\begin{enumerate}[label=(RQ\arabic*), leftmargin=1.25cm]
    \item What factors affect how long a post stays on the r/popular feed (i.e., the post's tenure on the feed)?
    \item What factors affect the assigned rank/position of the post on the r/popular feed?
    \item How does the post's rank on the r/popular feed affect the engagement on the post?
\end{enumerate}

RQ1 and RQ2 examine the factors that influence Reddit's ranking algorithm on r/popular. Specifically, what factors affect how long a post stays and where it is placed on the feed. RQ3 aims to quantify how algorithmic ranking decisions affect subsequent engagement on posts.

To answer these research questions, we capture a snapshot of the r/popular feed every 2 minutes over an 11-month period. Using over 1.5M consistently collected snapshots, we employ multiple regression analyses to examine the activity and movements (i.e., changes in position on the r/popular feed) of 10K posts from 694 distinct subreddits.

\subsection{Summary of Findings}

Through our analyses, we find that the total number of comments, along with recent commenting and voting activity, were the most predictive factors for a longer stay on the r/popular feed (RQ1) and for upward movement on the feed (RQ2). We also find that undesired comments, i.e., toxic and moderator-removed comments, were also predictive of a longer stay and upward movement, but to a lesser degree. Regarding how ranking on r/popular affects engagement (RQ3),  we find that posts which were higher on the feed received comments at a higher rate, as well as a greater proportion of undesired comments.

By systematically analyzing snapshots of r/popular, we provide insights into how a prominent, algorithmically curated trending feed made its ranking decisions and how those decisions may influence user engagement on the platform. Understanding how these opaque algorithms operate as an external party is inherently difficult. The internal workings of curation algorithms are proprietary and therefore inaccessible to users and researchers. This lack of access limits transparency regarding how user engagement is driven by content ranking and how the content users interact with is influenced by engagement. Our approach can empower users and researchers to examine the algorithmically curated feeds that determine what they interact with on social media.

\section{Related Work}

In this section, we review prior work on trending feeds/popularity, algorithmic curation systems, and algorithmic audits.

\subsection{Popularity \& Trending Feeds}

Popularity on social media has been studied before; however, these studies often focus on how users~\cite{devito-algorithmic-trap}, communities~\cite{increased-attention-github, my-paper}, or topics~\cite{twitter-trending-topics} change after becoming viral. For example, \citet{popularity-shocks-user} found that after users became viral, they increased their posting frequency and changed their posts to be similar to the post that made them viral. In general, the focus of these studies is the impact of popularity on (groups of) users. However, missing from prior work are investigations into why content or users became viral in the first place. \textit{To address this gap, we examine the factors that influence the systems that determine what is viral on social media, specifically the r/popular feed on Reddit.} It is important to understand how these algorithmic curation systems work because they are pervasive on social media and the popularity they induce can be a double-edged sword, particularly for those of marginalized identities~\cite{devito-algorithmic-trap}. Additionally, these systems can also propagate inflammatory content as noted by mainstream media in recent times~\cite{inflammatory-content, radical-ideas, youtube-conspiracies}. 

\subsection{Algorithmic Curation \& Ranking}

According to \citet{rader-gray-folk-theories}, algorithmic curation is the process of ``organizing, selecting, and presenting subsets of a corpus of information for consumptions.'' Within this definition, we are particularly focused on the organizing and presenting aspect because Reddit algorithmically ranks posts on r/popular. However, there exist other definitions~\cite{news-feed-fyi, congress-testimony, cura}. Algorithmic ranking is only one key mechanism in which algorithmic curation systems can exert power~\cite{diakopoulous-black-box}. By ranking content, specifically on social media feeds, these algorithms essentially have the power to determine what is important by prioritizing content or users over one another. To show how impactful algorithmic ranking can be, \citet{joachims-clicks} examined users' clickthrough behavior on Google's result page and found that participants' trust in Google's retrieval/ranking function led them to click on highly ranked links regardless of their quality or relevance to the query. Additionally, \citet{watts-music-markets} found that ranking songs in music markets by total downloads produced more unpredictability and inequality compared to groups who had songs ranked randomly. \textit{We extend this line of work by examining the influence of algorithmic ranking on social media trending feeds.}

\subsection{Algorithmic Audits}

Per \citet{auditing-algorithms-definition}, an \textit{algorithmic audit} is ``a method of repeatedly and systematically querying an algorithm with inputs and observing the corresponding outputs in order to draw inferences about its opaque inner workings.'' In this paper, we query the r/popular feed for 11 months and examine its ranking decisions to infer details about its internals and impact on engagement. Our study complements the large corpus of existing social media algorithm audits performed on platforms like X/Twitter~\cite{lower-higher-quality}, YouTube~\cite{radical-pathways-youtube, resnick-youtube-scrubbing}, Facebook~\cite{facebook-ideological-segregation}, and TikTok~\cite{tiktok-explanations} to name a few. Additionally, prior studies often focus on political bias that may be built into curation algorithms which differs from our focus on the factors that influence, and are influenced by, popularity.
\textit{We extend this line of work and conduct an audit of algorithmic ranking on a relatively underexplored site, Reddit's r/popular feed.}

To conduct these algorithmic audits, researchers commonly utilize sock-puppet accounts~\cite{twitter-sock-puppet-bias, diakopoulos-more-accounts, tiktok-explanations} that ``use code scripts to create simulated users''~\cite{resnick-youtube-scrubbing}. Sock-puppet accounts are necessary because platforms often do user-specific recommendations based on the user's activity. However, that raises the challenge of simulating realistic user behavior which is a limitation to these types of studies. \textit{By studying a trending feed that is available to everyone, we avoid having to use sock-puppet accounts.}