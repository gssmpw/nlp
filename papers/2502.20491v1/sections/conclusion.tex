\section{Conclusion}

Now that algorithmic curation has become integral to online ecosystems, examining it and its effects on user behavior becomes critical. In this paper, we conducted a comprehensive empirical audit of one such system: Reddit's r/popular feed. Through this, we successfully quantified how recent commenting rate and other factors influence algorithmic ranking on r/popular, as well as how rank/position on r/popular correlates with subsequent engagement and undesirable behavior. Our findings are based on millions of snapshots of the top 100 ranks on r/popular consistently collected over 11 months, an approach that can be applied to other platforms. Additionally, we discussed the implications of our findings for stakeholders, including content creators, highlighted moderators' lack of agency in community curation systems, and proposed future research directions on algorithmic curation amid reduced data access. All in all, studying algorithmic curation goes back to user agency: should users have control over how they interact with content and how their content is distributed online? To address this, we must first understand how these systems are embedded in our online environments, thus enabling us to make informed decisions about governing them.