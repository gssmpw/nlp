\section{RQ1: Tenure on r/popular} \label{sec:rq1}

To recall, RQ1 asks what factors are associated with how long a post stays on r/popular (i.e., tenure). To conduct this analysis, we built logistic regression models that estimate whether the post continues to exist on the feed in the next snapshot---approximately 2 minutes later. This section describes the model and its findings in further detail.

\subsection{Model 1: Logistic Regression}

Essentially, the task is to take an r/popular post during a snapshot and predict whether the same post continues to be on r/popular during the \textit{next snapshot}.
Thus, factors that helped a post stay on the feed in the next snapshot also helped elongate its tenure on r/popular. Because we did not observe posts outside the top 100, we built three models for the top 50 (\mff{}), top 25 (\mtw{}), and top 10 (\mtn{}) to have enough observations of posts both inside and outside of these intervals. Additionally, testing multiple rank intervals helped examine the robustness of our results. Each respective model only includes posts that, at some point, reached its respective rank interval (see Table~\ref{tab:top-n} for the exact number of posts). For example, if a post $p$ is in $n$ snapshots before it reaches the top 50 and $m$ snapshots after, then \mff{} uses only the $m$ snapshots after its initial breakthrough.

\subsection{Results}

\begin{table}[t]
    \centering
    \small
    \begin{tabular}{lrrr}
        \textbf{Feature} & \mff{} & \mtw{} & \mtn{}  \\ \midrule
        Link (vs. image)        & 88.93\%*          & 9.77\%*     & -18.32\%*                \\
        Text (vs. image)        & 47.98\%*          & 11.12\%*    & -2.21\%\enspace          \\
        Video (vs. image)       & -0.13\%\enspace   & -6.24\%*    & 12.01\%*                 \\ \midrule
        Age                     & -38.91\%*         & -39.16\%*   & -43.14\%*                \\
        Num. Comments           & 91.52\%*         & 105.37\%*   & 84.11\%*                \\
        Rec. Comments           & 19.08\%*          & 23.71\%*    & 27.76\%*                 \\
        Prop. Undesired         & 2.64\%*           & 5.26\%*     & 16.80\%*                 \\
        Prop. Rec. Undesired    & 3.18\%*           & 2.93\%*     & 3.94\%*                  \\
        Score                   & -47.89\%*         & -52.84\%*   & -57.45\%*                \\
        Rec. Votes              & 161.14\%*         & 125.10\%*   & 74.51\%*                 \\
        Prop. Upvotes           & -0.60\%\enspace   & 4.10\%*     & -4.72\%*                 \\
        Num. Subscribers        & 3.01\%*           & 2.39\%*     & 10.76\%*                 \\ \midrule
        Num. Posts              & 5,697             & 3,499       & 1,875                    \\ 
        Num. Snapshots          & 1,062,717         & 687,386     & 391,167                  \\ \midrule
        $R^2$                   & 0.230             & 0.261       & 0.285
    \end{tabular}
    \caption{
        Results from the three logistic regression models (\mff{}, \mtw{}, \mtn{}) predicting a post's tenure on the r/popular feed (*$p<0.05$, Bonferroni-adjusted). Percentages indicate the expected change in odds that a post will stay on the top 50 (\mff{}), 25 (\mtw{}), and 10 (\mtn{}) in the next snapshot---approximately 2 minutes later---given a unit increase in each respective feature (see Table \ref{tab:descriptive}). For example, a post with 2 times the number of comments as another post is expected to have 91.52\% greater odds of staying in the top 50.
    }
    \label{tab:top-n}
\end{table}


Table~\ref{tab:top-n} presents our findings from the three logistic regression models we built. The percentages in Table~\ref{tab:top-n} correspond to the change in odds of staying in the top 50, 25, or 10 when the respective feature increases multiplicatively by its standardized unit listed in Table~\ref{tab:descriptive}.

\textbf{Impact of Overall Engagement.} We found the strongest factors that increased the odds of staying near the top of the feed were the number of total comments, recent comments, and recent votes. We also found age (i.e., time since post creation) to be a strong factor that decreased the odds of staying near the top. These are consistent across all rank intervals and are natural for an engagement-based popular feed.

We also observed that increasing the score on a post corresponded to a drastic decrease in the odds of staying on the feed. We further discuss how this counterintuitive result may indicate the correlation between a post's score and some unobserved influence in Section~\ref{sec:limitations}.

\textbf{Impact of Undesired Activity.} We found that increasing the proportion of undesired comments and the number of subscribers had a comparatively small but significant effect on increasing the odds that a post stays within the top 50 and top 25, with a stronger effect in the top 10. While these effects do not necessarily indicate that the ranking algorithm is intentionally designed to promote antisocial content to drive more user engagement, these results fail to rule it out. These effects also hint at an interaction between undesired comments and the rank of a post, and that the effects of undesired comments are strongest for posts at the top of the feed.