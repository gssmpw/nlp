Platforms are increasingly relying on algorithms to curate the content within users' social media feeds. However, the growing prominence of proprietary, \textit{algorithmically curated feeds} has concealed \textit{what factors influence the presentation} of content on social media feeds and \textit{how that presentation affects user behavior}. This lack of transparency can be detrimental to users, from reducing users' agency over their content consumption to the propagation of misinformation and toxic content. To uncover details about how these feeds operate and influence user behavior, we conduct an empirical audit of Reddit's algorithmically curated trending feed called \textit{r/popular}. Using 10K r/popular posts collected by taking snapshots of the feed over 11 months, we find that the total number of comments and recent activity (commenting and voting) helped posts remain on r/popular longer and climb the feed. Using over 1.5M snapshots, we examine how differing ranks on r/popular correlated with engagement. More specifically, we find that posts below rank 80 showed a sharp decline in activity compared to posts above, and that posts at the top of r/popular had a higher proportion of undesired comments than those lower down. Our findings highlight that the order in which content is ranked can influence the levels and types of user engagement within algorithmically curated feeds. This relationship between algorithmic rank and engagement highlights the extent to which algorithms employed by social media platforms essentially determine which content is prioritized and which is not. We conclude by discussing how content creators, consumers, and moderators on social media platforms can benefit from empirical audits aimed at improving transparency in algorithmically curated feeds.