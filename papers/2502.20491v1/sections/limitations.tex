\section{Limitations} \label{sec:limitations}

We recognize that our study bears limitations, however, these limitations also suggest interesting future directions.

Our findings are inherently tied to the demographics of Reddit users. Therefore, any attempt to apply our conclusions to other platforms should carefully consider the demographic similarities between Reddit and the platform in question. 

The relatively low $R^2$ our models exhibited indicate that there are other factors outside of the ones we have included that influence the outcomes. One such factor may include details about surrounding posts on the r/popular feed. A hint that our model is incomplete is the observation that increasing the score on a post corresponds to a drastic decrease in the odds of staying on the feed, as found in RQ1 and RQ2. This counterintuitive result may be due to a correlation between a post's score and some unobserved influence. Regardless, future work can build upon these models by including more factors about surrounding posts that are competing for the same finite number of spots on these feeds. Furthermore, future work could also employ more advanced statistical methods like a stochastic transitivity model~\cite{johnson_bayesian_2013} or a causal framework used in prior work~\cite{chandrasekharan2017you, saha2019prevalence, jhaver2024bystanders}. These causal approaches will help examine the causal relationship between the different factors that influence and are influenced by algorithmic ranking.

Lastly, because our findings were based on observational data, future work will have to examine more user-centered effects of algorithmic ranking---similar to \citet{watts-music-markets, joachims-clicks}, but on social media. Specifically, future work can structure controlled experiments that test the impacts of algorithmic rank on individual users and their perceptions of highly ranked content. These experiments can additionally include interviews or surveys to assess affective characteristics like trust, engagement, and perceived fairness.
Together, these studies can offer complementary insights that provide a more comprehensive understanding of algorithmic ranking on social media platforms.