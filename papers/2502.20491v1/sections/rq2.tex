\section{RQ2: Rank on r/popular} \label{sec:rq2}

\begin{figure}[t]
    \centering
    \includegraphics[width=\linewidth]{images/trajectories.pdf}
    \caption{
        The rank trajectories of 10 r/popular posts where hour 0 is the first moment they reached the top 100. Note the stepwise movement as posts tended to jump to different ranks instead of gradually shifting between nearby ranks.
    }
    \label{fig:trajectories}
\end{figure}

In this section, we describe our analyses of the factors associated with changes to a post's rank on r/popular. As seen in Figure~\ref{fig:trajectories}, a post does not gradually change between nearby ranks over time. Instead, the typical rank trajectory of a post consists of two parts: (1) where the post stays at one rank for several minutes, and (2) where the post suddenly jumps to another, distant rank. To characterize this discontinuous behavior, we used two separate regression models to determine which factors affect: (1) \textit{when} a post changes rank, and (2) \textit{to which} rank it changes when it does.

\subsection{Model 2.1: Multinomial Regression}

The first model is a multinomial regression that estimated whether a post will move up the feed, down the feed, or stay in its current position in the next snapshot. Posts tend to stay at the same rank for several minutes at a time (see Figure~\ref{fig:trajectories}), and this model helped us determine which factors accelerated the movement of a post, and in which direction.

\subsection{Model 2.2: Ordinal Regression}

The second model is an ordinal regression that estimated the post's next rank. Only snapshots where a post changed rank were considered so that we could get a more detailed picture of which rank a post changed to when it moved. We interpreted this model as a latent-variable model, where a post's rank is determined by a continuous latent variable ($z$), and the observed rank depends on which cut points $z$ falls between. The model estimated both the cut points and the association between each feature and the expected value of $z$. The flexibility of the cut points within the model makes it well suited to handle the uneven ``distances'' between ranks---for e.g., it might be less likely for a post to move from rank 2 to rank 1 than it is from rank 99 to 98. We implemented this model in a Bayesian framework with strongly informative priors on the cut points and weakly informative priors on the coefficients. We provided more details on the priors in Appendix~\ref{sec:model-2.2-appendix}.

\begin{table}[t]
    \centering
    \small
    \begin{tabularx}{\linewidth}{Xrr}
        \textbf{Feature} & \textbf{Up} & \textbf{Down} \\ \midrule
         Link (vs. image)           & 2.757\%\enspace       & 4.812\%*             \\
         Text (vs. image)           & 0.401\%\enspace       & 7.842\%*             \\
         Video (vs. image)          & 1.147\%\enspace       & -0.727\%\enspace     \\ \midrule
         Age                        & 4.117\%*              & 10.114\%*            \\
         Num. Comments              & -1.997\%*             & -7.200\%*            \\
         Rec. Comments              & 7.333\%*              & -4.127\%*            \\
         Prop. Undesired            & 0.751\%\enspace       & -2.681\%*            \\
         Prop. Rec. Undesired       & 0.285\%\enspace       & -0.801\%\enspace     \\
         Score                      & -11.626\%*            & 11.018\%*            \\
         Rec. Votes                 & 14.796\%*             & 3.886\%*             \\
         Prop. Upvotes              & 1.393\%*              & 0.803\%\enspace      \\
         Num. Subscribers           & 0.097\%\enspace       & -2.937\%*            \\ \midrule
         Num. Observations          & & 1,548,266                                  \\
         Nagelkerke $R^2$           & & 0.043
    \end{tabularx}
    \caption{
        Results from the multinomial regression model predicting when a post moves and in which direction (*$p<0.05$, Bonferroni-adjusted). Percentages represent expected change in odds of moving up or down the feed (compared to no movement) in the next snapshot, given a unit increase in each respective feature (see Table \ref{tab:descriptive}). For example, a post that is 2 times as old  as another post is expected to have 4.117\% greater odds of moving up the feed and 10.114\% lower odds of moving down the feed in next snapshot.
    }
    \label{tab:rank-movements}
\end{table}

\begin{figure}[t]
    \centering
    \includegraphics[width=\linewidth]{images/rank-movements.pdf}
    \caption{
        The probability of a post moving up or down on r/popular in the next snapshot across ranks estimated by a multinomial logistic regression. Note that most of the time a post did not move in the next snapshot.
    }
    \label{fig:rank-movements}
\end{figure}

\subsection{Results}

\begin{table}[t]
    \centering
    \small
    \begin{tabularx}{\linewidth}{Xr}
        \textbf{Feature} & \textbf{$\Delta z$} \\ \midrule
         Link (vs. image)           & 0.001\enspace\enspace     \\
         Text (vs. image)           & 0.033**                   \\
         Video (vs. image)          & -0.015*\enspace\          \\ \midrule
         Age                        & 0.163**                   \\
         Num. Comments              & -0.053**                  \\
         Rec. Comments              & -0.062**                  \\
         Prop. Undesired            & -0.021**                  \\
         Prop. Rec. Undesired       & -0.008**                  \\
         Score                      & 0.132**                   \\
         Rec. Votes                 & -0.081**                  \\
         Prop. Upvotes              & -0.018**                  \\
         Num. Subscribers           & -0.028**                  \\
    \end{tabularx}
    \caption{
        Results from the ordinal regression model estimating how far a post jumps in rank. Values represent expected change in the latent variable $z$ representing rank on r/popular (Figure \ref{fig:cut-points} shows the cut points which map $z$ to ranks) given a unit increase to a feature (see Table \ref{tab:descriptive}). For example, a post that is 2 times older than another post is expected to have a 0.163 increase in $z$ in the next snapshot where the rank changes. Parameters whose \textit{probability of direction}, a Bayesian analog of $p$-values~\cite{makowski2019indices}, are greater than 0.95 are denoted with a single asterisk (*), and those greater than 0.999 are denoted with two (**).
    }
    \label{tab:next-rank}
\end{table}

\begin{figure}
    \centering
    \includegraphics[width=\linewidth]{images/cut-points.pdf}
    \caption{
        The ordinal regression model's estimated cut points for each rank and the distances between adjacent ranks in the latent space used to represent rank. The independent feature associations in Table~\ref{tab:rank-movements} correspond to this latent space.
    }
    \label{fig:cut-points}
\end{figure}

Starting with the multinomial regression model, Table~\ref{tab:rank-movements} shows how a 2 times increase to a feature, excluding the proportion of upvotes, affected the probability of a post moving up or down the feed. Additionally, Figure~\ref{fig:rank-movements} shows the baseline probabilities for each rank movement across the ranks on the feed. Note that the probability of moving up and down do not sum to 100\% because the remaining amount corresponds to the probability that the post will not move in the next snapshot.

\textbf{Impact of Rank on Movement.} We found that posts closer to the top of the feed moved less frequently and less far. Figure~\ref{fig:rank-movements} shows that the probability that a post moves in either direction reduces as its rank gets closer to 1, and Figure~\ref{fig:cut-points} that shows the gaps between cut points close to rank 1 are wider than cut points closer to rank 100. This is consistent with the intuition that there would be less competition toward the top of the feed as it would be rare for posts to achieve that level of ``quality.''

\textbf{Impact of Overall Activity on Rank.} Similar to our results in RQ1, the number of recent comments and recent votes were strong factors that increased the probability that a post moved up in rank, as seen in Table~\ref{tab:rank-movements}. In terms of how far a post moved, Table~\ref{tab:next-rank} indicates that the total number of comments, recent comments, and recent votes had a similar magnitude effect. However, Table~\ref{tab:rank-movements} indicates that the total number of comments tended to ``stabilize'' the rank of a post, decreasing the odds of movement in either direction, but still favoring upward movement. Similarly, the number of subscribers did not seem to increase the probability of upward movement directly, but it still favored upward movement by decreasing the probability of downward movement. These findings are consistent with an engagement-based ranking algorithm, but they highlight the influence of recent activity.

\textbf{Impact of Undesired Activity on Rank.} Table~\ref{tab:rank-movements} indicates that the proportion of undesired comments slightly decreased the likelihood of moving down the feed, and Table~\ref{tab:next-rank} indicates that undesired comments had relatively little impact on where a post jumped in rank to. Thus, while undesired comments had little impact on moving posts up the feed, it reduced the rate at which posts move down.

For reasoning about score's effect on rank, see Section~\ref{sec:limitations}.