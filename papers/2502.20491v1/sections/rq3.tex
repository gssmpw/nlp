\section{RQ3: Engagment on r/popular} \label{sec:rq3}

Our analyses from RQ1 and RQ2 provided empirical insights into the factors associated with ranking on r/popular. Finally, for RQ3, we investigated how \textit{rank} on the r/popular feed influences the engagement on posts, specifically the frequency of comments gained before the next snapshot.

\subsection{Model 3: Negative Binomial Regression}

Our goal was to investigate the impact of features in Section~\ref{sec:features} on commenting rate, and further, how they impacted the rate of undesired comments differently. We measured commenting rate as the number of comments gained between two consecutive snapshots. For example, if a post had 100 comments during a snapshot and at the next snapshot it had 120, then it has gained 20 comments. Since snapshots are taken approximately 2 minutes apart, this would correspond to a commenting rate of 10 comments per minute.

Since we wanted to investigate if there are differences in how features are associated with undesired commenting rate, we first consider the non-undesired commenting rate to be the reference which we compared the undesired commenting rate against. We used a single negative binomial regression model that estimated both non-undesired and undesired commenting rates using a dummy variable, which indicates that we are estimating undesired commenting rates. Thus, the model directly estimated the association between the features and the non-undesired commenting rate, and differences between how features are associated with undesired compared to non-undesired commenting rate are captured in interaction terms with the dummy variable. In other words, these interaction terms allowed us to inspect whether any of the features from Section~\ref{sec:features} or the placement on r/popular has differing effects on undesired comments versus non-undesired comments.

\subsection{Results}

\begin{table}[t]
    \centering
    \small
    \begin{tabularx}{\linewidth}{Xrrr}
        \textbf{Feature} & \textbf{$\neg$Und.} & \multicolumn{1}{r}{\textbf{Und:$\neg$Und.}} & \multicolumn{1}{r}{\textbf{Net Und.}} \\         \midrule
        Link (vs. image)        & -3.440\%*          & +0.342\%             & -3.110\%     \\
        Self (vs. image)        & -2.179\%*          & +0.650\%             & -1.543\%     \\
        Video (vs. image)       & 0.924\%*           & +0.006\%             & 0.930\%      \\ \midrule
        Age                     & -13.140\%*         & -0.467\%\enspace     & -13.545\%    \\
        Num. Comments           & 22.331\%*          & -2.601\%*            & 19.149\%     \\
        Rec. Comments           & 61.740\%*          & +1.056\%*            & 63.449\%     \\
        Prop. Undesired         & -14.467\%*         & +115.633\%*          & 84.437\%     \\
        Prop. Rec. Und.         & -1.500\%\enspace   & +8.713\%*            & 7.082\%      \\
        Score                   & -13.992\%*         & +2.062\%*            & -12.218\%    \\
        Rec. Votes              & 13.530\%*          & -3.422\%*            & 9.645\%      \\
        Prop. Upvotes           & 3.519\%*           & -1.077\%*            &  2.405\%     \\
        Num. Subscribers        & 1.414\%*           & -0.344\%*            & 1.066\%      \\ \midrule
        Pseudo $R^2$            & 0.29 \enspace\enspace &                   &
    \end{tabularx}
    \caption{
        Results from the negative binomial regression model predicting the rate of non-undesired and undesired comments (*$p<0.05$, Bonferroni-adjusted). Percentages in ``$\neg$Und.'' denote expected change in non-undesired comments given a unit increase in the respective feature (see Table \ref{tab:descriptive}). Percentages in ``Und:$\neg$Und.'' denote expected change in the ratio of undesired to non-undesired comments gained. Net expected change to undesired comments is shown in the ``Net Und.'' column. For example, a post that is 2 times older than another is expected to have 13.1\% fewer non-undesired comments and a further 0.47\% fewer undesired comments, for a net of 13.5\% fewer undesired comments.
    }
    \label{tab:engagement}
\end{table}


\begin{figure}[t]
    \centering
    \includegraphics[width=\linewidth]{images/engagement-rank-coefficients.pdf}
    \caption{
        Plots of the intercepts for non-undesired comments (top) and undesired comments (bottom). Error bars indicate 95\% confidence intervals. Assuming a ``standard'' post with features equal to $\mu$ (see Table~\ref{tab:descriptive}), the top plot can be interpreted as the expected number of non-undesired comments gained in 2 minutes at a given rank, and the bottom plot can be interpreted as the expected ratio of undesired comments to non-undesired comments gained at a given rank. For example, a ``standard'' post at rank 1 is expected to gain about 2 non-undesired comments every 2 minutes, and is expected to gain 72.4\% fewer undesired comments than non-undesired comments. The bottom plot also shows a decreasing trend in the ratio of undesired comments with increasing rank, at a rate of about 0.064\% per rank (red dashed line).
    }
    \label{fig:engagement-rank-coefficients}
\end{figure}

Table~\ref{tab:engagement} shows how our feature set from Section~\ref{sec:features}, excluding rank which is visualized in Figure~\ref{fig:engagement-rank-coefficients}, is associated with the rate of non-undesired comments, the reference, the rate of and undesired comments. 

\textbf{Impact of Engagement on Non-Undesired Activity.} From this table, we found that the number of comments, recent commenting activity, and recent voting activity had the strongest associations with future non-undesired activity. Additionally, the impact of recent commenting activity (61.74\%) on non-undesired commenting rate was greater than the one found with the total number of comments (22.33\%). This hints that the active discussions played a more important role to future engagement than the total number of discussions that have already occurred.

\textbf{Impact of Engagement on Undesired Activity.} For undesired activity, we found that the proportion of undesired comments had the strongest association (84.43\%) out of all the ones found in Table~\ref{tab:engagement}. This suggests that having a greater proportion of undesired activity would invite similarly undesired activity in the future.

\textbf{Impact of Rank on Commenting Activity.} Regarding rank's associations with commenting rate, we found that the rate of non-undesired comments dropped fairly consistently from ranks 2 through 80, visualized in the first subplot in Figure~\ref{fig:engagement-rank-coefficients}. Afterward, the rate of non-undesired comments fell precipitously below rank 80.

\textbf{Impact of Rank on Undesired Activity.} From the second subplot in Figure~\ref{fig:engagement-rank-coefficients}, we observed that the ratio of undesired comments fell as the post is placed lower on the feed. Specifically, the ratio of undesired comments to non-undesired comments fell at a rate of 0.064\% per rank ($p < 0.05$). Additionally, ranks 2 and 3 may indicate that there is a sharp rise in the ratio at the peak of the r/popular feed, however, it is difficult to be sure given the confidence intervals. Overall, we found that posts higher on the r/popular feed had a slightly greater tendency to attract undesired comments compared to posts below it. It is also important to note that the direction of causation is unclear as we previously found that undesired comments may have a slight impact on rank as well.