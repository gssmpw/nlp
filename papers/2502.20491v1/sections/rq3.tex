\section{RQ3: Engagment on r/popular} \label{sec:rq3}

Our analyses from RQ1 and RQ2 provided empirical insights into the factors associated with ranking on r/popular. Finally, for RQ3, we investigated how \textit{rank} on the r/popular feed influences the engagement on posts, specifically the frequency of comments gained before the next snapshot.

\subsection{Model 3: Negative Binomial Regression}

Our goal was to investigate the impact of features in Section~\ref{sec:features} on commenting rate, and further, how they impacted the rate of undesired comments differently. We measured commenting rate as the number of comments gained between two consecutive snapshots. For example, if a post had 100 comments during a snapshot and at the next snapshot it had 120, then it has gained 20 comments. Since snapshots are taken approximately 2 minutes apart, this would correspond to a commenting rate of 10 comments per minute.

Since we wanted to investigate if there are differences in how features are associated with undesired commenting rate, we first consider the non-undesired commenting rate to be the reference which we compared the undesired commenting rate against. We used a single negative binomial regression model that estimated both non-undesired and undesired commenting rates using a dummy variable, which indicates that we are estimating undesired commenting rates. Thus, the model directly estimated the association between the features and the non-undesired commenting rate, and differences between how features are associated with undesired compared to non-undesired commenting rate are captured in interaction terms with the dummy variable. In other words, these interaction terms allowed us to inspect whether any of the features from Section~\ref{sec:features} or the placement on r/popular has differing effects on undesired comments versus non-undesired comments.

\subsection{Results}

\begin{figure}[t]
    \centering
    \begin{subfigure}{.38\textwidth}
        \centering
        \includegraphics[width=.7\textwidth]{figs/rq_1/num_comments_cdf_wo.pdf}
        \caption{Number of Comments}
        \label{fig:comments}
    \end{subfigure}
    \quad

    \begin{subfigure}{.38\textwidth}
        \centering
        \includegraphics[width=.7\textwidth]{figs/rq_1/upvote_ratio_cdf_wo.pdf}
        \caption{Upvote Ratio}
        \label{fig:upvote}
    \end{subfigure}
    \caption{Cumulative distributions of per-user mean engagement values, per disclosure type. 
    }
    \label{fig:all_engagement_cdf}
\end{figure}



\begin{figure}[t]
    \centering
    \begin{subfigure}{.5\textwidth}
        \centering
        \includegraphics[width=.85\linewidth]{figs/rq_1/numm_comments.pdf}
        \caption{Effect on Number of Comments}
        \label{fig:engagement_num_comments}
    \end{subfigure}
    \quad
    \begin{subfigure}{.48\textwidth}
        \centering
        \includegraphics[width=.85\linewidth]{figs/rq_1/score_ratio.pdf}
        \caption{Effect on Upvote Ratio}
        \label{fig:engagement_upvotes_ratio}
    \end{subfigure}
    \caption{Regression results for engagement. Each panel is a different regression model. The y-axis and x-axis show confounding factors and corresponding estimate, respectively.%, and error bars show the 95\% confidence interval. 
    }
    \label{fig:all_engagement_regression}
\end{figure}


Next, we explore user engagement with self-disclosing posts, focusing on how interactions in self-disclosure communities (\eg, /r/AMA) may encourage others to disclose, potentially creating privacy-compromising attack vectors. 




%%%%%%%%%%%%%%%%%%%%%%%%%%%%%%%%%%%%%%%%%%%
\subsection{Quantifying Engagement with Self-Disclosure} \label{subsec:engagement_diff}
%%%%%%%%%%%%%%%%%%%%%%%%%%%%%%%%%%%%%%%%%%%

From a privacy perspective, higher engagement reflects potentially increased exposure to users' information. It may also create a feedback loop that influences what users post in the future~\cite{haq_short_2022}.
We therefore start by measuring the difference in engagement levels across self-disclosing posts. 



Figure~\ref{fig:all_engagement_cdf} presents the distribution of two engagement metrics: \one number of comments %\two the difference between upvotes and downvotes, 
and \two the ratio of upvotes to downvotes, across all posts containing self-disclosure. 
A non-parametric Kruskal-Wallis test confirms significant differences across the distribution of each disclosure type (test statistics are in Table~\ref{tab:eng_post_hoc_num_comments} and \ref{tab:eng_post_hoc_upvote_ratio} in Appendix). 
For instance, posts with \sexualOrientation have more comments ($\mu = 16.4$) than posts with \job disclosure ($\mu = 11.6$). 

To systematically analyze these differences, we turn to regression analysis with users and time-fixed effects. For each engagement metric (number of comments and upvote ratio) and self-disclosure type, we design a separate engagement prediction task based on whether the posts contain a particular self-disclosure or not (not including other types of self-disclosure). The posts without self-disclosure from all users remain identical across the models, hence providing a common baseline across models to compare the results with non-disclosing posts and within different types.

Figure~\ref{fig:engagement_num_comments} and~\ref{fig:engagement_upvotes_ratio} show the regression results for number of comments and upvote ratio, respectively. The Y-axis shows the corresponding self-disclosure used as a confounding factor. The X-axis shows the regression estimate with a 95\% confidence interval. The values in the blue color are statistically significant ($p<0.05$), and the grey color shows non-significance.

We see that the presence of self-disclosure does have a significant effect on both the number of comments and the upvote ratio. Interestingly, the effect is not similar across all self-disclosure types, and there is also a disparity in each engagement metric for a given self-disclosure. For instance, the presence of \sexualOrientation disclosure increases the number of comments (2.65x); however, the same has a negative effect on the upvote ratio($\approx-.02x$). 

To further see the difference between heterosexual and potentially more vulnerable non-hetero, we do a keyword filtering of the posts (with sexual orientation disclosure) containing the words (\textit{gay, lesbian, bisexual, and straight}), and use a non-parametric Kruskal-Wallis test ($\chi^2 = 251.2, df = 3, p < 0.001 $), followed by Dunn's test, to see the engagement metric distributions difference across each keyword's post. We observe that posts containing words `gay' receive more comments (almost double) ($\mu = 102$) and lower upvote ($\mu = .80$) than other non-hetero (\eg upvote ratio for `lesbian'  $\mu = 0.83$).
However, it receives fewer comments than the posts containing `straight' keywords ($\mu = 119$).
This contrast shows that, although \sexualOrientation increases engagement in terms of comment count, the engagement quality is less positive, on average. 
We posit that such engagement metrics may influence users' future self-disclosure likelihood, as prior Reddit studies find engagement do affect topic choice \cite{haq_short_2022}. Moreover, a lower upvote ratio with higher comments may signify negative discussions and discontent with the original poster~\cite{risch_top_2020}, potentially leading to negative experiences, especially in disclosures like \sexualOrientation and \ethnicity.




\begin{figure}[t]
    \centering
    \includegraphics[width=0.48\textwidth]{figs/rq_1/regression_analysis_m1.pdf}
    \caption{%Regression estimates for 11 models. 
    Self-disclosure types on the y-axis represent one regression model with two independent variables (\#interactions and Interacted). The x-axis shows the $\beta$ estimates.}
    \label{fig:regression_analysis_m1}
\end{figure}



\subsection{Quantifying Impact of Self-Disclosure Communities} \label{subsec:engagement_effect}


There are various Reddit communities directly related to specific forms of self-disclosure, \eg \texttt{r/aznidentity} is related to discussions on ethnicity, \texttt{AskDocs} has health-related discussions, and \texttt{r/AskMen} has gender-related discussions. 
These pertain to \ethnicity, \health, and \gender, respectively. We hypothesize that engaging with members of such communities may increase one's own likelihood of disclosing, even to other communities. This may create attack surface where malicious actors purposefully post (fake) self-disclosure to encourage others to share. We next explore the potential presence of such behaviors, quantifying the impact of receiving comments that contain self-disclosure.




\pb{Self-Disclosure related Subreddits.} 
For the above analysis, we first obtain a set of subreddits dedicated to self-disclosure. The names of the subreddits indicate their association with a focused topic~\cite{adelani_estimating_2020}. We extract the top 50 largest subreddits, in terms of their number of posts classified as containing each type of self disclosures. As some subreddits are associated with multiple self-disclosure types, we end up with 250 unique subreddits out of 550 extracted subreddits. 
We then manually review the name of each subreddit and annotate whether it is associated with one of our self-disclosure types.
Some examples are shown in Table~\ref{tab:sd_association} in Appendix~\ref{app:sd_association}. 



\pb{Regression Task.} We next ask \one what is the effect of a user receiving an interaction from a users in a self-disclosure related subreddit compared to not having received an interaction? 
and \two If a user receives an interaction, what is the impact on the number of such interactions? We model this as a regression task to predict whether the user will have a self-disclosure in future: 

\[ Y_{st} = \alpha + \beta NI_{t-1} + U + T \]
%
$Y_{st}$ is the number of self-disclosures by a user in a period t (1 week). $\beta$ shows the coefficient for number of interactions($N_{t-1}$) at time $t-1$ with the users from selected subreddit communities (SD-specific or general subreddits). $I = [0,1]$ whether a user received an interaction $I = 1$ or not $I = 0$. 
Note, we consider interaction to be any action initiated by users from a selected community. Thus, interaction occurs when a user from a selected subreddit writes a comment (at any level of the post) on a post by our selected users.
Finally, $U$ represent all users, and $T$ represents time (in weekly brackets) to control any user and time-dependent fixed-effect. In total, we run 11 regression models. Each model is specific to the self-disclosure-specific subreddits labeled with that type of disclosure. Note, we consider any self-disclosure as a positive instance of self-disclosure and do not differentiate within different types. 




\pb{Results.} Following these steps, we run our fixed effect regression model while controlling heteroskedasticity~\cite{gujarati_basic_2009}. Figure~\ref{fig:regression_analysis_m1} plots the regression model results. The figure consists of two panels, one for each of the independent variables.
The lower panel shows the binary impact of the user receiving an interaction or not, whereas the upper panel shows the impact of the number of interactions. The x-axis shows the $\beta$ estimates for variables, and the y-axis refers to the regression model depending on the self-disclosure for which the specific subreddits are used. The blue color shows the results that are statistically significant (with $p-values$ being lower than $0.05$). The gray color shows the corresponding estimates are statistically not significant. The missing values are also statistically not significant and the values are less than zero, so they are not shown due to the figure's x-axis scale.  

Confirming our hypothesis, we observe a positive effect ($\beta$ estimates) for future self-disclosure, \ie users are more likely to share a self-disclosing posts if they receive an interaction from a user who has previously posted in self-disclosure-specific subreddits.
Distinct effect sizes for the \textit{Interacted} variable are observed within each model. 
The most significant effect is observed for \job, with the odds of future self-disclosure being 0.34x, followed by \health at 0.26x, and \relationship at 0.24x. Similarly, the \textit{\#interactions} shows the positive effect for each such interaction. 
Given these confounding factors, we hypothesize two scenarios where Reddit users could be exploited: \one A user may be influenced to self-disclose through repeated interactions with other users who engage in self-disclosure, potentially including bots; or \two The creation of subreddits designed to \textit{maliciously} foster a sense of community to increase the likelihood of users engaging in self-disclosure.



\begin{figure}[t]
    \centering
    \includegraphics[width=\linewidth]{images/engagement-rank-coefficients.pdf}
    \caption{
        Plots of the intercepts for non-undesired comments (top) and undesired comments (bottom). Error bars indicate 95\% confidence intervals. Assuming a ``standard'' post with features equal to $\mu$ (see Table~\ref{tab:descriptive}), the top plot can be interpreted as the expected number of non-undesired comments gained in 2 minutes at a given rank, and the bottom plot can be interpreted as the expected ratio of undesired comments to non-undesired comments gained at a given rank. For example, a ``standard'' post at rank 1 is expected to gain about 2 non-undesired comments every 2 minutes, and is expected to gain 72.4\% fewer undesired comments than non-undesired comments. The bottom plot also shows a decreasing trend in the ratio of undesired comments with increasing rank, at a rate of about 0.064\% per rank (red dashed line).
    }
    \label{fig:engagement-rank-coefficients}
\end{figure}

Table~\ref{tab:engagement} shows how our feature set from Section~\ref{sec:features}, excluding rank which is visualized in Figure~\ref{fig:engagement-rank-coefficients}, is associated with the rate of non-undesired comments, the reference, the rate of and undesired comments. 

\textbf{Impact of Engagement on Non-Undesired Activity.} From this table, we found that the number of comments, recent commenting activity, and recent voting activity had the strongest associations with future non-undesired activity. Additionally, the impact of recent commenting activity (61.74\%) on non-undesired commenting rate was greater than the one found with the total number of comments (22.33\%). This hints that the active discussions played a more important role to future engagement than the total number of discussions that have already occurred.

\textbf{Impact of Engagement on Undesired Activity.} For undesired activity, we found that the proportion of undesired comments had the strongest association (84.43\%) out of all the ones found in Table~\ref{tab:engagement}. This suggests that having a greater proportion of undesired activity would invite similarly undesired activity in the future.

\textbf{Impact of Rank on Commenting Activity.} Regarding rank's associations with commenting rate, we found that the rate of non-undesired comments dropped fairly consistently from ranks 2 through 80, visualized in the first subplot in Figure~\ref{fig:engagement-rank-coefficients}. Afterward, the rate of non-undesired comments fell precipitously below rank 80.

\textbf{Impact of Rank on Undesired Activity.} From the second subplot in Figure~\ref{fig:engagement-rank-coefficients}, we observed that the ratio of undesired comments fell as the post is placed lower on the feed. Specifically, the ratio of undesired comments to non-undesired comments fell at a rate of 0.064\% per rank ($p < 0.05$). Additionally, ranks 2 and 3 may indicate that there is a sharp rise in the ratio at the peak of the r/popular feed, however, it is difficult to be sure given the confidence intervals. Overall, we found that posts higher on the r/popular feed had a slightly greater tendency to attract undesired comments compared to posts below it. It is also important to note that the direction of causation is unclear as we previously found that undesired comments may have a slight impact on rank as well.