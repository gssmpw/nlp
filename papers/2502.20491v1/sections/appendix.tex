\section{Bayesian Priors for Model 2.2} \label{sec:model-2.2-appendix}

\begin{figure}[ht]
    \centering
    \includegraphics[width=\linewidth]{images/next-rank-variance.pdf}
    \caption{
Variance of a post's rank in the next snapshot (y-axis) plotted against a post's rank in the current snapshot (x-axis).}
    \label{fig:next-rank-variance}
\end{figure}

Our intuition is that it is ``easier'' to move between ranks further down the feed (e.g., from 95 to 91) than it is to move between ranks higher up the feed (e.g., from 5 to 1).
Indeed, based on Figure~\ref{fig:next-rank-variance}, we saw that the variance of rank in the next snapshot appears to increase approximately linearly with the rank in the current snapshot, up until rank 60-80. Beyond that, the variance appears to decrease due to the fact that ranks beyond 100 were censored from our dataset.
We used this observation to set strongly informative priors on our cut points. 

First, we assumed that the variance of next rank increases linearly with the current rank, then the standard deviation would increase proportionally to the square root of the current rank. In other words, this means that posts tend to traverse more ranks the further down the feed, and the number of ranks they typically traverse increases proportionally to the square root of the current rank.
We can capture this notion in the priors for our cut points by making the gap between cut points inversely proportional to the square root of rank. Thus we chose the following priors for our cut points $\kappa_k$ for each rank $k$:

$$
    \kappa_{k} - \kappa_{k-1} \sim \text{LogNormal}(\mu=\frac{\alpha}{\sqrt{k}},\sigma=0.1)
$$

$\alpha$ is a scaling factor with a weakly informative prior of $\alpha \sim \text{Exponential}(\lambda = 1)$, and $\kappa_1$ is fixed to $\kappa_1 = \alpha$.  

We also assume that a post's next rank will be close to its current rank. Thus, the intercepts for each current rank $k$ were given weakly informative priors of $\beta_{\text{rank}=k} \sim \text{Normal}(\mu=\kappa_{k}, \sigma=1)$. 

Finally, the remaining coefficients were given weakly informative priors: $\beta \sim \text{Normal}(\mu=0, \sigma=1).$