\begin{table}[t]
    \centering
    \small
    \begin{tabularx}{\linewidth}{Xrr}
        \textbf{Feature} & \textbf{Up} & \textbf{Down} \\ \midrule
         Link (vs. image)           & 2.757\%\enspace       & 4.812\%*             \\
         Text (vs. image)           & 0.401\%\enspace       & 7.842\%*             \\
         Video (vs. image)          & 1.147\%\enspace       & -0.727\%\enspace     \\ \midrule
         Age                        & 4.117\%*              & 10.114\%*            \\
         Num. Comments              & -1.997\%*             & -7.200\%*            \\
         Rec. Comments              & 7.333\%*              & -4.127\%*            \\
         Prop. Undesired            & 0.751\%\enspace       & -2.681\%*            \\
         Prop. Rec. Undesired       & 0.285\%\enspace       & -0.801\%\enspace     \\
         Score                      & -11.626\%*            & 11.018\%*            \\
         Rec. Votes                 & 14.796\%*             & 3.886\%*             \\
         Prop. Upvotes              & 1.393\%*              & 0.803\%\enspace      \\
         Num. Subscribers           & 0.097\%\enspace       & -2.937\%*            \\ \midrule
         Num. Observations          & & 1,548,266                                  \\
         Nagelkerke $R^2$           & & 0.043
    \end{tabularx}
    \caption{
        Results from the multinomial regression model predicting when a post moves and in which direction (*$p<0.05$, Bonferroni-adjusted). Percentages represent expected change in odds of moving up or down the feed (compared to no movement) in the next snapshot, given a unit increase in each respective feature (see Table \ref{tab:descriptive}). For example, a post that is 2 times as old  as another post is expected to have 4.117\% greater odds of moving up the feed and 10.114\% lower odds of moving down the feed in next snapshot.
    }
    \label{tab:rank-movements}
\end{table}