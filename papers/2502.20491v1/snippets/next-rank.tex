\begin{table}[t]
    \centering
    \small
    \begin{tabularx}{\linewidth}{Xr}
        \textbf{Feature} & \textbf{$\Delta z$} \\ \midrule
         Link (vs. image)           & 0.001\enspace\enspace     \\
         Text (vs. image)           & 0.033**                   \\
         Video (vs. image)          & -0.015*\enspace\          \\ \midrule
         Age                        & 0.163**                   \\
         Num. Comments              & -0.053**                  \\
         Rec. Comments              & -0.062**                  \\
         Prop. Undesired            & -0.021**                  \\
         Prop. Rec. Undesired       & -0.008**                  \\
         Score                      & 0.132**                   \\
         Rec. Votes                 & -0.081**                  \\
         Prop. Upvotes              & -0.018**                  \\
         Num. Subscribers           & -0.028**                  \\
    \end{tabularx}
    \caption{
        Results from the ordinal regression model estimating how far a post jumps in rank. Values represent expected change in the latent variable $z$ representing rank on r/popular (Figure \ref{fig:cut-points} shows the cut points which map $z$ to ranks) given a unit increase to a feature (see Table \ref{tab:descriptive}). For example, a post that is 2 times older than another post is expected to have a 0.163 increase in $z$ in the next snapshot where the rank changes. Parameters whose \textit{probability of direction}, a Bayesian analog of $p$-values~\cite{makowski2019indices}, are greater than 0.95 are denoted with a single asterisk (*), and those greater than 0.999 are denoted with two (**).
    }
    \label{tab:next-rank}
\end{table}