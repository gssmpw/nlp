%%%%%%%%%%%%%%%%%%%%%%%%%%%%%%%%%%%%%%%%%%%%%%%%%%%%%%%%%%%%%%%%%%%%%%%%%%%%%%%%
%2345678901234567890123456789012345678901234567890123456789012345678901234567890
%        1         2         3         4         5         6         7         8

\documentclass[letterpaper, 10 pt, conference]{ieeeconf}  % Comment this line out if you need a4paper

%\documentclass[a4paper, 10pt, conference]{ieeeconf}      % Use this line for a4 paper

\IEEEoverridecommandlockouts                              % This command is only needed if 
                                                          % you want to use the \thanks command

\overrideIEEEmargins                                      % Needed to meet printer requirements.

%In case you encounter the following error:
%Error 1010 The PDF file may be corrupt (unable to open PDF file) OR
%Error 1000 An error occurred while parsing a contents stream. Unable to analyze the PDF file.
%This is a known problem with pdfLaTeX conversion filter. The file cannot be opened with acrobat reader
%Please use one of the alternatives below to circumvent this error by uncommenting one or the other
%\pdfobjcompresslevel=0
%\pdfminorversion=4

% See the \addtolength command later in the file to balance the column lengths
% on the last page of the document

% The following packages can be found on http:\\www.ctan.org
%\usepackage{graphics} % for pdf, bitmapped graphics files

\usepackage{epsfig} % for postscript graphics files
%\usepackage{mathptmx} % assumes new font selection scheme installed
%\usepackage{times} % assumes new font selection scheme installed
\usepackage{amsmath} % assumes amsmath package installed
\usepackage{amssymb}  % assumes amsmath package installed
\usepackage{xspace}
\usepackage{xcolor}
\usepackage{color}
\usepackage{cite}
\usepackage{hyperref}
\usepackage{booktabs, multirow}
\usepackage{colortbl}
\usepackage{bm}
\usepackage{tikz}
\usepackage{setspace}
\PassOptionsToPackage{table}{xcolor}
\usepackage[font=footnotesize,labelfont=bf]{caption}
\newcounter{RNum}
\renewcommand{\theRNum}{\arabic{RNum}}
\newcommand{\Remark}{\noindent\textit{\textbf{Remark}~\refstepcounter{RNum}\textbf{\theRNum}: }}

\newcommand{\methodname}{SMART\xspace}

% % The following will change the table caption to lower case
% % following https://tex.stackexchange.com/questions/166814/table-caption-in-uppercase-i-dont-know-why
% \usepackage{etoolbox}
% \makeatletter
% \patchcmd{\@makecaption}
%   {\scshape}
%   {}
%   {}
%   {}
% \makeatletter
% \patchcmd{\@makecaption}
%   {\\}
%   {.\ }
%   {}
%   {}
% \makeatother
% \def\tablename{Table}

\title{\LARGE \bf
% HD Mapping Everywhere: 
\textit{\methodname}: 
Advancing Scalable Map Priors for \\ Driving Topology Reasoning
}


% \author{Junjie Ye$^{1*}$, David Paz$^2$, Hengyuan Zhang$^{3*}$, Yuliang Guo$^2$, Xinyu Huang$^2$, \\Henrik I. Christensen$^3$, Yue Wang$^1$, and Liu Ren$^2$
% \thanks{$^{*}$Work done while interned at Bosch Research North America.}
% \thanks{$^{1}$Junjie Ye and Yue Wang are with the Thomas Lord Department of Computer Science, University of Southern California, CA90007, USA 
%         {\tt\small \{yejunjie, yue.w\}@usc.edu}}%
% \thanks{$^{2}$David Paz, Yuliang Guo, Xinyu Huang, and Liu Ren are with Bosch North America and Bosch Center for AI (BCAI), CA94085, USA
%         {\tt\small \{david.pazruiz, yuliang.guo2, xinyu.huang, liu.ren\}@us.bosch.com}}%
% \thanks{$^{3}$Hengyuan Zhang and Henrik I. Christensen are with  the Contextual Robotics Institute, UC San Diego, CA92122, USA
%         {\tt\small \{hyzhang, hichristensen\}@ucsd.edu}}%
% }

\author{Junjie Ye$^{1*}$, David Paz$^2$, Hengyuan Zhang$^{3*}$, Yuliang Guo$^2$, Xinyu Huang$^2$, \\Henrik I. Christensen$^3$, Yue Wang$^1$, and Liu Ren$^2$
\thanks{$^{*}$Work done while interned at Bosch Research North America.}
\thanks{$^{1}$Thomas Lord Department of Computer Science, University of Southern California 
        {\tt\small \{yejunjie, yue.w\}@usc.edu}}%
\thanks{$^{2}$Bosch North America and Bosch Center for AI (BCAI)
        {\tt\small \{david.pazruiz, yuliang.guo2, xinyu.huang, liu.ren\}@us.bosch.com}}%
\thanks{$^{3}$Contextual Robotics Institute, UC San Diego
        {\tt\small \{hyzhang, hichristensen\}@ucsd.edu}}%
}


\begin{document}



\maketitle
\thispagestyle{empty}
\pagestyle{empty}


%%%%%%%%%%%%%%%%%%%%%%%%%%%%%%%%%%%%%%%%%%%%%%%%%%%%%%%%%%%%%%%%%%%%%%%%%%%%%%%%
\begin{abstract}
Topology reasoning is crucial for autonomous driving as it enables comprehensive understanding of connectivity and relationships between lanes and traffic elements. 
% ensuring accurate decision-making for safe and efficient navigation.
% High-definition (HD) maps are essential for the safety and efficiency of autonomous driving systems, providing critical information about the driving environment. 
% However, traditional pipelines demand significant human effort and resources for annotating and maintaining map semantics.
While recent approaches have shown success in perceiving driving topology using vehicle-mounted sensors, their scalability is hindered by the reliance on training data captured by consistent sensor configurations.
We identify that the key factor in scalable lane perception and topology reasoning is the elimination of this sensor-dependent feature.
To address this, we propose \methodname, a scalable solution that leverages easily available standard-definition (SD) and satellite maps to learn a map prior model, supervised by large-scale geo-referenced high-definition (HD) maps independent of sensor settings. Attributed to scaled training, \methodname alone achieves superior offline lane topology understanding using only SD and satellite inputs. Extensive experiments further demonstrate that \methodname can be seamlessly integrated into any online topology reasoning methods, yielding significant improvements of up to 28\% on the OpenLane-V2 benchmark. Project page: \href{https://jay-ye.github.io/smart}{https://jay-ye.github.io/smart}.

% Recent methods have attempted to perceive lane topology on the fly using vehicle-mounted sensors. Despite the efficacy, these methods are difficult to scale up due to their reliance on training data captured by the same sensor settings. In this work, we propose leveraging easily accessible Standard Definition (SD) and satellite maps to pretrain a scalable map prior model. Attributing to scaled pretraining, the map prior model alone achieves superior offline lane topology understanding taking input as SD and satellite maps. Extensive experiments further demonstrate that the pretrained model can be seamlessly integrated into online mapping methods and enhance online lane topology perceiving by a large margin.
% Despite the efficacy, these methods rely on a large amount of sensory data captured by exactly the same sensor settings to scale up. This work instead proposes to make full use of easily accessible Standard Definition (SD) maps and satellite images to pretrain a map prior model.


\end{abstract}





%%%%%%%%%%%%%%%%%%%%%%%%%%%%%%%%%%%%%%%%%%%%%%%%%%%%%%%%%%%%%%%%%%%%%%%%%%%%%%%%
\section{Introduction}
In recent years, lane perception and topology reasoning of driving scenes has received considerable
attention, providing essential information about the structure and connectivity of road elements for autonomous driving and driver assistance
systems. 
% This task enables vehicles to navigate safely and efficiently. 
While previous map perception methods~\cite{li2021hdmapnet, liu2023vectormapnet, MapTR, maptrv2} primarily focus on identifying road markers, \textit{e.g.}, lane dividers, road boundaries, and pedestrian crossings, driving topology reasoning encompasses the broader understanding of not only the lane geometry but also their connectivity and relationships with traffic elements, such as traffic lights and road signs. Such in-depth topological reasoning is essential for downstream tasks such as trajectory prediction, path planning, and motion control~\cite{sadat2020eccv, Nayakanti2023icra, mao2024agentdriver}.

\begin{figure}[!t]
\centering
  \includegraphics[width=0.9\linewidth]{img/teaser.pdf}
\caption{\textbf{Comparison between baseline and \methodname-OL.}
Existing topology reasoning methods suffer from limited sensor data. \methodname augments online topology reasoning with robust map priors learned from scalable SD and satellite maps, substantially improving lane perception and topology reasoning.
% Online mapping approaches suffer from the limited sensor data. \methodname leverages scalable SD and satellite map pretraining to enhance the lane detection and topology reasoning performance. 
%Comparison of baseline with the proposed \methodname integrated or not.
}
\label{fig:teaser}
\vspace{-13pt}
\end{figure}

In the literature, driving topology is formulated as a graph that captures the location of lane centerlines and traffic elements, along with the connectivity between centerlines and their relationships to each traffic element~\cite{wang2023openlanev2}. Recent advancements~\cite{li2023toponet, wu2024topomlp, ma2024roadpainter} have shown promising progress in this field. 
% For instance, Li \textit{et al.}~\cite{li2023toponet} propose modeling topology relationships using a graph convolutional network (GCN)~\cite{Kipf2017GCN}, while Wu \textit{et al.}~\cite{wu2024topomlp} adopt a simple MLP-based framework for topology reasoning. 
Nevertheless, existing approaches face significant limitations. They are trained on data captured with consistent sensor configurations.
% limiting their adaptability to new sensor settings. 
Scaling these models typically requires large datasets collected from vehicles with uniform sensor setups, which is both costly and time-consuming. Moreover, the low perspective of ground-based vehicles leads to occlusions from other vehicles, buildings, and objects, posing substantial challenges. Some recent works~\cite{wang2023openlanev2, luo2024smerf, zhang2024enhancingonlineroadnetwork} aim to mitigate these occlusions by integrating standard definition (SD) maps with sensor inputs to provide geometric and topological priors. Yet, despite the widespread availability of SD maps, these approaches remain constrained by the limited availability of sensor data for training.

Training models for driving topology reasoning also requires access to high-definition (HD) maps associated with sensor data. Although these models are sensor-dependent, the supervision signals--HD maps--are calibrated to the real world and are geo-referenced. Setting aside the dependency on sensor data, there exists large-scale HD maps independently of sensor sets. For example, as shown in Table~\ref{tab:statistics}, the motion forecasting set of Argoverse 2~\cite{Argoverse2} possesses HD maps from 285 times more scenes than those typically used for online mapping, not to mention HD maps available from many other sources~\cite{nuplan, Sun2020waymo}. The extensive scale of HD maps offers a huge potential for learning generalizable map priors.

On the other hand, SD maps and satellite images are universally accessible from crowd sources~\cite{OpenStreetMap, mapbox_raster_tiles_api} with geo-locations. These \textit{geospatial} maps offer powerful priors for understanding lane structures and are updated at a reasonable frequency. This leads us to an important question: \textit{Can we advance scalable online topology reasoning with easily accessible geospatial maps and large-scale HD maps?}

% The key to this challenge is to disentangle the need for vast amounts of sensory data. 
We address this challenge by proposing a simple yet effective two-stage driving topology reasoning pipeline.
% In light of this, we propose a simple yet effective two-stage lane topology reasoning pipeline. 
As illustrated in Fig.~\ref{fig:teaser}, in the first stage, we introduce \textit{\methodname}, a map prior model targeted at reason lane topology using SD and satellite maps, supervised by large-scale HD maps. In the second stage, \methodname is integrated with \textit{online} topology reasoning models, enhancing them with powerful map priors learned from the first stage. 
This approach shifts the dependency on massive high-quality sensor data with consistent configurations to highly accessible and scalable geospatial maps, enabling scaled learning of adaptable map representations. 
The well-trained \methodname, in turn, provides robust priors that significantly enhance the generalizability of online topology reasoning. 
Remarkably, our map prior model alone achieves state-of-the-art \textit{offline} lane topology\footnote{We refer to generating lane graphs from \textit{offline} SD and satellite maps as \textit{offline} lane topology.} given only SD and satellite inputs. Extensive experiments further demonstrate that integrating \methodname into existing methods boosts performance by a large margin.

% To differentiate from approaches that rely on online sensory data for lane topology generation, we refer to the process of generating lane graphs from pre-existing SD and satellite maps as offline lane topology. We further note that offline lane topology falls short in traffic element detection and reasoning due to aerial perspective.

\begin{table}[!t]\centering
    \begin{spacing}{0.8}
    \caption{\textbf{Statistics of available HD maps in different datasets.} 
    % Each frame has its associated lane graph. 
    The motion forecasting set from Argoverse (AV) 2~\cite{Argoverse2} has over 280 times more scenes compared to the sensor datasets.}\label{tab:statistics}
    \end{spacing}
    \footnotesize
    \resizebox{\linewidth}{!}{%
        \begin{tabular}{cccc}\toprule
        Datasets &Num. of scenes &Num. of HD maps \\\cmidrule{1-3}
        AV 2 sensor - $train$ &700 &22,477 \\
        nuScenes - $train$ &700 &27,968 \\\cmidrule{1-3}
        AV 2 motion forecasting - $val$ &24,988 & 549,736 \\
        AV 2 motion forecasting - $train$ &199,908 &4,397,976\\
        \bottomrule
        \end{tabular}
    }
    \vspace{-10pt}
\end{table}

        % AV 2 motion forecasting - $val$ &25,000 &406,086 \\
        % AV 2 motion forecasting - $train$ &200,000 &3,248,778 \\
        
To summarize, our contributions are three-fold:
\begin{itemize}
    \item A simple yet effective architecture for map prior learning at scale, achieving impressive lane topology reasoning with SD and satellite inputs.
    \item A map prior model that can be seamlessly integrated into any topology reasoning framework, enhancing robustness and generalizability.
    \item Evaluations on the widely-used benchmark~\cite{wang2023openlanev2} underscore the effectiveness of \methodname in driving topology reasoning, achieving state-of-the-art performance. 
    % The source code of the proposed approach, along with the geospatial map fetching pipeline, will be made publicly available to facilitate future research.
    % \item We release an SD and satellite map dataset complementing existing datasets like Argoverse2, providing additional resources for future research.
\end{itemize}

%%% General background
% Autonomous vehicles have great potential to enhance mobility, reduce energy and space consumption [need citation]. Significant progress has been made, which can be observed by the scaling up of Robotaxi services. However, the services are still limited to a handful of places and scaling slowly. One of the challenges is related to the high-definition (HD) maps. These maps provide accurate environmental information such as lanes and traffic signs. This information is critical for downstream tasks such as prediction and planning in the current stack~\cite{Darweesh2021OpenPlanner2, fan2018baidu}. 

%%% Problem of focus
% HD maps require significant human effort to create and maintain. Recent research focuses on creating such maps online~\cite{li2021hdmapnet, MapTR, li2023toponet, wang2023openlanev2}. Early works focus solely on the lane geometry~\cite{li2021hdmapnet, liu2023vectormapnet, MapTR, maptrv2}. Recent works also bring in traffic elements and the topology relationship between lanes and traffic elements~\cite{wang2023openlanev2, li2023toponet}. To enhance model performance on this challenging task, additional prior from more scalable standard definition (SD) maps are also used~\cite{wang2023openlanev2, luo2024smerf, zhang2024enhancingonlineroadnetwork}. SD maps, such as OpenStreetMaps~\cite{haklay2008openstreetmap}, Google maps, provide rough geometry and topology information without lane level details and centimeter localization. 

%%% Our proposal and contribution
% Our work additionally introduced satellite images along with SD maps as prior. Satellite images are also widely accessible and provide more details that complement SD maps. We show that satellite images provide significantly better priors. The method also scales well with the data.

\begin{figure*}[!t]
\centering
  \includegraphics[width=0.95\linewidth]{img/prelane-figures-v4.pdf}
  \vspace{-3pt}
\caption{\textbf{Outline of the proposed approach.} 
In the first stage (bottom row), \methodname is trained at scale using SD and satellite maps for lane graph prediction, supervised by large-scale geo-referenced HD maps. In the second stage (top row), the robust map priors learned by \methodname are seamlessly integrated into any online driving topology reasoning models, significantly enhancing lane perception and topology reasoning. 
% \methodname takes ego-centric SD and satellite maps as input, generates a BEV neural scene prior. During pretraining, the prior is supervised by HD map data. When integrated with online mapping pipelines, the pretrained model is frozen to provide BEV prior. The widely accessible SD and satellite maps enable large-scale pretraining that significantly increases the online mapping performance. 
}
\label{fig:outline}
\vspace{-13pt}
\end{figure*}

\section{Related Works}
\subsection{Online Mapping}

Online mapping seeks to perceive lane structures on the fly, offering instant scene information for downstream tasks, in contrast to offline methods that rely on traffic observations~\cite{zurn2024autograph} or aerial images~\cite{Buchner2023UrbanLaneGraph, Blayney_2024_CVPR_bezier}. Early works primarily focus on estimating the geometry of road elements~\cite{li2021hdmapnet, liu2023vectormapnet, MapTR, maptrv2}. As one of the pioneering works, HDMapNet~\cite{li2021hdmapnet} generates an online semantic map, followed by post-processing to obtain vectorized road elements. Liu \textit{et al.}~\cite{liu2023vectormapnet} model vectorized HD map learning in an end-to-end fashion. Liao \textit{et al.}~\cite{MapTR, maptrv2} propose a permutation invariant loss that boosts map geometry learning. Recent efforts have extended online mapping to include topology reasoning, which involves understanding the connectivity and relationships between lanes and traffic elements~\cite{wang2023openlanev2, li2023toponet, wu2024topomlp, fu2024topologic}. Graph-based approaches like TopoNet~\cite{li2023toponet} use scene graph neural networks to model these relationships, while simpler models like TopoMLP~\cite{wu2024topomlp} still achieve promising improvements in topology reasoning with multilayer perceptrons (MLPs). Other methods~\cite{li2024topo2d, fu2024topologic} explore novel ways to improve the initialization of 3D queries and interpret geometric distances for better topology reasoning. However, a common bottleneck for these methods is the reliance on large amounts of sensor data collected under consistent configurations.

% , while recent advancements extend this to topology reasoning~\cite{wang2023openlanev2, li2023toponet, wu2024topomlp, fu2024topologic}.

% \noindent\textbf{Geometry Only.} Lane geometry estimation involves detecting the position and shape of road elements such as lane dividers, boundaries, and pedestrian crossings. For instance, Li \textit{et al.} propose HDMapNet~\cite{li2021hdmapnet}, which estimates an online semantic map, followed by post-processing to obtain vectorized road elements. Liu et al.~\cite{liu2023vectormapnet} vectorized the pipeline without predicting an intermediate dense semantic map. Liao et al.~\cite{MapTR, maptrv2} propose a permutation invariant loss that boosts map geometry learning. However, these works are limited by estimating the individual road element geometry and does not model the connectivity explicitly, resulting in disconnected lane segments and broken road networks.

% \noindent\textbf{Topology Reasoning.} To facilitate learning the relationship between road elements, a recent dataset OpenLane-V2~\cite{wang2023openlanev2} introduces lane-to-lane and lane-to-traffic-elements topology reasoning. Lane-to-lane topology models the connectivity between lanes, while lane-to-traffic-element topology associates traffic light, and traffic signs to the lanes. Li et al. propose TopoNet~\cite{li2023toponet}, which models the relationship with a scene graph neural network. Wu et al. propose TopoMLP~\cite{wu2024topomlp}, a simple yet strong model that significantly increases the topology reasoning performance. Topo2D~\cite{li2024topo2d} argues that initializing 3d queries from 2D queries with 2D lane priors can generate better results. TopoLogic~\cite{fu2024topologic} leverages an interpretable geometric distance between lane endpoints and lane queries to facilitate the topology reasoning. One of the greatest bottlenecks for topology reasoning is data. These methods often require large amounts of sensor data captured from the same sensor platform for training.

\subsection{Map Priors}
Incorporating prior knowledge has been shown to advance online mapping performance~\cite{xiong2023neuralmapprior, luo2024smerf, Gao2024ICRA, zhang2024enhancingonlineroadnetwork, Jiang2024pmapnet, Yuan2024presight}. Xiong \textit{et al.}~\cite{xiong2023neuralmapprior} assume a multi-traversal setting to enhance map perception with features from previous traversals. More recently, some works~\cite{luo2024smerf, zhang2024enhancingonlineroadnetwork} integrate SD maps with surroundings view images for joint training, demonstrating improved topology modeling performance. Similarly, Gao~\textit{et al.}~\cite{Gao2024ICRA} aid road element detection with satellite imagery. In spite of promising improvements achieved, priors introduced in these works remain consistently coupled to limited sensor data, leading to unsatisfying scalability.
By contrast, we aim to learn a unified map prior representation from massive geospatial maps, featuring generalizability to novel locations and compatibility with any online topology reasoning models.
Additionally, we identify the benefits of combining the road-level topology priors from SD maps and comprehensive bird's eye view (BEV) textures from satellite maps.



% Many works introduce prior knowledge to improve model performance with limited data. Standard definition (SD) maps, such as OpenStreetMaps~\cite{haklay2008openstreetmap} and Google Maps provide the basic road network information without centimeter localization and many details. Luo et al. propose SMERF~\cite{luo2024smerf} that fuses the SD maps with BEV features through cross attention. Zhang et al.~\cite{zhang2024enhancingonlineroadnetwork} demonstrated the benefit of fusing the rasterized and graphical SD maps, showcasing the benefit of handling occlusion and intersection topology. While SD maps provide easily accessible road network prior, they miss the details in lanes. Therefore, we introduce satellite images as additional prior information to enhance the prior quality. Xiong et al.~\cite{xiong2023neuralmapprior} proposed neural map prior, which generates a global map prior that enables longer range and higher quality perception. It is worth noting that they are global map prior which requires sensor data to generate, whereas our model can generate prior for any location with SD and satellite maps.

\section{Methodology}
\subsection{Problem Definition}
Driving topology reasoning aims to perceive the geometric layout of lane centerlines $\mathbf{V}_{\rm l}$ and traffic elements $\mathbf{V}_{\rm t}$, and reason lane-to-lane connectivity $\mathbf{E}_{\rm ll}$ and lane to traffic element relationship $\mathbf{E}_{\rm lt}$. They formulate two graphs, \textit{i.e.}, $(\mathbf{V}{\rm _l}, \mathbf{E}_{\rm ll})$ and $(\mathbf{V}_{\rm t} \cup \mathbf{V}_{\rm l}, \mathbf{E}_{\rm lt})$. Specifically, $\mathbf{V}_{\rm l}$ consists of a set of directed lane instances with each denoted as $\mathbf{v}_{\rm l}=[p_0, ..., p_{n-1}]$, where $p=(x,y,z)\in \mathbb{R}^3$. 

Existing works estimate lane topology directly from perspective images captured by $C$ synchronized surround-view cameras mounted to the ego-vehicle, some with SD maps as additional input. This line of work lacks scalability and generalizability due to its dependence on sensor data. 
% Due to the entanglement of sensor data, this line of work lacks scalability and generalizability. 

The pipeline of the proposed approach is shown in Fig.~\ref{fig:outline}. During inference, we first fetch SD and satellite maps corresponding to the desired GPS location and adopt a well-trained map prior model to extract prior features. These features are then integrated into online topology reasoning models for enhanced centerline detection and relation modeling. To tackle the limited availability of sensory data and make full use of easily accessible SD and satellite maps, we decouple prior and sensor inputs with two-stage training.

\subsection{Offline Map Prior Learning}
In the first stage, we train a map prior model, referred to as \methodname, with the goal of inferring lane graphs $(\mathbf{V}_{\rm l}, \mathbf{E}_{\rm ll})$ from offline SD and satellite maps.

\Remark We focus on lane graph modeling only in the first stage due to the invisibility of traffic elements like traffic lights and signs from the aerial perspective.
\subsubsection{SD map fetching}\label{sec:sd_fetching} We fetch SD maps from OpenStreetMap (OSM)~\cite{OpenStreetMap} as in previous work~\cite{luo2024smerf, zhang2024enhancingonlineroadnetwork}. SD maps supply comprehensive geographic information for locations worldwide. Given a specific GPS location and orientation, we extract a local SD map that contains $m$ road polylines $\mathbf{R}=[\mathbf{r}_1, ..., \mathbf{r}_{m}]$. Each polyline is represented as an ordered set of 2D points and includes $k$ associated attributes $\bm{\alpha} \in \mathbb{R}^{m \times k}$ such as road types and lane counts. The number of polylines and their individual lengths may vary. The extracted map is then rotated and cropped to generate an ego-centric view covering the desired spatial range.

\subsubsection{Satellite map fetching}\label{sec:sate_fetching} We obtain satellite maps from the Mapbox Raster Tiles API~\cite{mapbox_raster_tiles_api}, which segments the Earth's surface into a grid of tiles at various zoom levels. By projecting a given GPS coordinate onto the 2D map tile plane, we identify the appropriate tile indices and query the API to retrieve the relevant tiles. The zoom level is set to 20, corresponding to $\sim$0.11 meters per pixel. Through a process of map tile stitching, rotation, and cropping, we generate the satellite map $\mathbf{S}\in \mathbb{R}^{H\times W}$ that aligns with a specific location, orientation, and spatial range. 

\subsubsection{SD and satellite map encoding}
For SD maps, we begin by evenly sampling $N$ points along each polyline, then pad the number of polylines to $m'$ to maintain a consistent count across maps within each training batch.
% pad the number of polylines to $m'$ by empty polylines to ensure a consistent number of polylines across maps within a training batch, 
This reformulates each SD map as $\mathbf{R'}=[\mathbf{r'}_1, ..., \mathbf{r'}_{m}, \mathbf{0},...]_{m'}$ and $\bm{\alpha'} \in \mathbb{R}^{m' \times k}$, where $\mathbf{r'} \in \mathbb{R}^{N \times 2}$. Following~\cite{luo2024smerf}, we transform $\mathbf{R'}$ from 2D coodinates to corresponding sinusoidal embeddings $\mathbf{R''} \in \mathbb{R}^{m' \times Nd}$ with $d$ dimensions using sinusoidal encodings $\rm E_{sin}$:
\begin{equation}
    \mathbf{R''} = {\rm E_{sin}} (\mathbf{R'}) \quad.
\end{equation}

With the refined road polylines $\mathbf{R''}$ and associated attributes $\bm{\alpha'}$, we generate SD map features $\mathbf{F}_{\rm SD} \in \mathbb{R}^{m' \times C}$ with a linear layer and a Transformer~\cite{vaswani2017attention} encoder:
\begin{equation}
    \mathbf{F}_{\rm SD} = {\rm Enc} ( {\rm Linear} ( {\rm Concat} (\mathbf{R''}, \bm{\alpha'}) ), \mathbf{M} ) \quad ,
\end{equation}
where $\mathbf{M}$ is a binary mask indicating valid polylines.

For satellite maps, we adopt an image backbone $\mathcal{F}$ to extract features as $\mathbf{F}_{\rm Sate}=\mathcal{F}(\mathbf{S})$ and flatten them to the dimension of $\mathbf{F}_{\rm Sate} \in \mathbb{R}^{H_{\rm f}W_{\rm f} \times C}$. 

To encode SD and satellite maps as a unified prior feature, we sequentially cross-attend features extracted from SD and satellite maps to a BEV feature map encoded with position embeddings, denoted as $\mathbf{B} \in \mathbb{R}^{H_{\rm B} W_{\rm B} \times C}$. Therefore, the fused prior features $\mathbf{B}_{\rm prior}$ can be obtained as: 
\begin{equation}
    \begin{split}
    \hat{\mathbf{B}} =& {\rm Softmax}(\mathbf{B} \mathbf{F}_{\rm SD}^{\top})\mathbf{F}_{\rm SD} \\
    \mathbf{B}_{\rm prior} =& {\rm Softmax}(\hat{\mathbf{B}} \mathbf{F}_{\rm Sate}^{\top})\mathbf{F}_{\rm Sate}
    \end{split} \quad .
\end{equation}

\subsubsection{Offline lane graph decoding}

To decode the lane graph from the prior features $\mathbf{B}_{\rm prior}$, we begin by attending them to learnable centerline instance queries $\mathbf{Q}\in \mathbb{R}^{N_{\rm L} \times C}$ with decoder layers in deformable DETR~\cite{zhu2021deformable} that combine self-attention, deformable attention, and feed-forward network. Sequentially, we adopt the simplified GCN in~\cite{li2023toponet} to modulate $\mathbf{Q}$ for enhanced relational modeling. Hence, the enhanced instance queries $\hat{\mathbf{Q}}$ is formulated as:
\begin{equation}
    \hat{\mathbf{Q}} = {\rm GCN} ( {\rm Dec} (\mathbf{Q}, \mathbf{B}_{\rm prior}) ) \quad .
\end{equation}

Next, two sets of three-layer MLPs are adopted to classify and regress $N_{\rm L}$ lanes, each with $N_{\rm P}$ points as follows:
\begin{equation}
    \mathbf{c} ={\rm MLP_{cls}}(\hat{\mathbf{Q}}) \quad\quad \mathbf{P} ={\rm MLP_{reg}}(\hat{\mathbf{Q}}) \quad ,
\end{equation}
where $\mathbf{c}\in \mathbb{R}^{N_{\rm L}}$ are the confidence scores and $\mathbf{P}\in \mathbb{R}^{N_{\rm L} \times (N_{\rm P}\times 3)}$ are regressed lane sets.

On the other hand, to achieve topological reasoning, the instance query $\hat{\mathbf{Q}}$ is fed into 2 MLPs separately, resulting in $\hat{\mathbf{Q}}'_1$ and $\hat{\mathbf{Q}}'_2$, both with a shape of $N_{\rm L} \times \frac{C}{2}$. These two features are then repeated along a new axis and concatenate together as $\Theta_{\rm ll} \in \mathbb{R}^{N_{\rm L} \times N_{\rm L} \times C}$. Subsequently, a binary classifier is operated on $\Theta_{\rm ll}$ to obtain the final topological matrix $\bm \epsilon$.

Consequently, the lane graph $(\mathbf{V}_{\rm l}, \mathbf{E}_{\rm ll})$ is predicted by further filtering out low-confidence centerlines in $\mathbf{P}$ and $\bm \epsilon$.

\subsubsection{Learning objective} Similar to existing works~\cite{li2023toponet, wu2024topomlp, ma2024roadpainter}, the overall loss $\mathcal{L}$ consisting of classification loss $\mathcal{L}_{\rm cls}$, regression loss $\mathcal{L}_{\rm reg}$, and topological loss $\mathcal{L}_{\rm top}$ is defined as:
\begin{equation}
    \mathcal{L} = \mathcal{L}_{\rm cls} + \mathcal{L}_{\rm reg} + \mathcal{L}_{\rm top} \quad ,
\end{equation}
where $\mathcal{L}_{\rm cls}$ and  $\mathcal{L}_{\rm top}$ employ focal loss, $\mathcal{L}_{\rm reg}$ adopts L1 loss.

% Fig.~\ref{fig:sample} shows some sample of retrieved SD and satellite maps, along with corresponding ground-truth lane graph.



% \begin{figure}[!t]
% \centering
%   \includegraphics[width=0.99\linewidth]{img/sd_sate_gt_samples.pdf}
% \caption{Two map samples, with SD map (left) and lane graph (right) on top of satellite map.
% % Samples of SD and satellite maps, along with corresponding lane graph.
% } % change this figure as the gt lines at bottom is not complete
% \label{fig:sample}
% \end{figure}

\subsection{Online Topology Reasoning with \methodname}\label{sec:online_integration}

The well-trained \methodname can be seamlessly integrated into any online topology reasoning models, augmenting them with robust prior features derived from SD and satellite maps. Generally, there are two mainstream pipeline used in existing online approaches: BEV-based and perspective-based methods. The former~\cite{li2023toponet, fu2024topologic, ma2024roadpainter} explicitly projects features extracted from surrounding views into the BEV perspective using BEVFormer~\cite{li2022bevformer}, followed by perception and reasoning heads for topology prediction. In this pipeline, we directly substitute the learnable BEV queries with prior features extracted by \methodname. Conversely, in perspective-based methods~\cite{wu2024topomlp, li2024topo2d}, the lane decoder interacts directly with multi-view visual features for perception and topology reasoning. In this paradigm, we employ a cross-attention layer to align prior features with perspective features. 

To mitigate the risk of overfitting due to the limited sensor data, the weights of \methodname are kept fixed during the training of online topology models, which are trained with their original settings. We term online topology reasoning with \methodname integrated as \textit{\methodname-OL}.

\Remark In practical applications, aside from fetching and inferring prior features online, these features can be precomputed using future geo-locations of the ego-vehicle derived from the navigation route or mission plans.

% Given the sensory observation of ego-vehicle, lane topology aims to perceive the location of surrounding lane centerlines and traffic elements, and reason centerlines' connectivity to each other and relationship to each traffic element. 


% TopoMLP 47.285
% TopoNet 62.9
% ours 45.69



\section{Experiments}

% TODO mention all use resnet50

In this section, we aim to answer the following questions: (1)~How far can we get with \methodname alone? (2)~How much can \methodname boost online topology reasoning?
(3)~How well can \methodname generalize to unseen areas? (4)~Can \methodname benefit from scaled training data? (5)~Can SD and satellite map fusion boost performance? (6)~Can \methodname reduce the reliance on sensor data?

\subsection{Dataset and Metrics}

% The training of \methodname leverages SD and satellite maps, which are readily accessible via crowdsourced platforms like OSM and Google Maps, along with ground-truth lane graph as labels. In contrast to sensor-dependent vehicle observations, lane graphs are calibrated to real-world, enabling annotated lane graphs from any data sources to be used for training, independent of associated sensory data availability. This feature allows \methodname to scale effectively with large datasets. In this work, we source HD maps from the motion forecasting dataset of Argoverse 2~\cite{Argoverse2}, which was previously inapplicable for online mapping due to a lack of corresponding sensory data. Table~\ref{tab:statistics} shows the statistics of available lane graphs in different datasets. As the most widely-used online mapping benchmarks, Argoverse 2 sensory dataset~\cite{Argoverse2} and nuScenes~\cite{caesar_nuscenes_2020} provide only 700 driving scenes, each containing fewer than 30k lane graphs. In contrast, the validation set of the motion forecasting dataset includes 25k scenes with over 406k lane graphs, while the train set contains \textbf{200k} scenes and over \textbf{3.2 million} lane graphs--\textbf{285} times more scenes and over \textbf{120} times the lane graph data compared to the sensor sets. This substantial increase in data supports the learning of significantly more generalizable and scalable map representations. We complement lane graphs with SD and satellite maps fetched using methods described in Secs.~\ref{sec:sd_fetching} and \ref{sec:sd_fetching} for training. 

\subsubsection{Dataset} Without loss of generality, we utilize the Argoverse 2 motion forecasting dataset~\cite{Argoverse2} for training \methodname, which provides geo-referenced HD maps despite lacking associated sensor data. Ground-truth lane graphs are derived from these HD maps by regressing centerlines and connecting lanes based on topology~\cite{wang2023openlanev2}. 
% noting that HD maps from any data sources are applicable independent of associated sensor data availability. 
Table~\ref{tab:statistics} summarizes the number of scenes in different datasets, each comprising consecutive driving frames at 2 Hz, with each frame associated with an HD map segment. While traditional online mapping benchmarks like the Argoverse 2 sensor dataset~\cite{Argoverse2} and nuScenes~\cite{caesar_nuscenes_2020} provide only 700 driving scenes with fewer than 30k HD maps, the validation set of the motion forecasting dataset includes $\sim$25k scenes with over 549k HD maps. The training set contains $\sim$\textbf{200k} scenes and over \textbf{4.3 million} HD maps--\textbf{285} times more scenes and over \textbf{160} times more HD data compared to sensor sets. This substantial increase in data amount and diversity supports the learning of significantly more generalizable and scalable map priors.

\begin{table}[!b]\centering
\vspace{-10pt}
    \begin{spacing}{0.8}
    \caption{\textbf{Comparison of \methodname and online mapping methods on lane graph generation.} Per-frame latency is measured by running individual methods on an Nvidia GeForce RTX 3090 GPU. Benefiting from scaled training, \methodname outperforms \textit{online} mapping methods on both lane perception and topology reasoning with \textit{offline} geospatial maps.
    % With a large-scale SD and satellite map dataset, \methodname pretrained model alone outperforms baseline online mapping methods in direct lane detection and lane topology reasoning.
    }\label{tab:direct_hd_mapping}
    \end{spacing}
    \footnotesize
    \resizebox{\linewidth}{!}{%
    \begin{tabular}{cccccc}\toprule
        Input type &Methods &${\rm DET_l}$ &${\rm TOP_{ll}}$ &Latency (ms) \\\cmidrule{1-5}
        \multirow{2}{*}{Perspective images} 
        % &STSUcan &12.7 &2.9 & \\
        % &VectorMapNet &11.1 &2.7 & \\
        % &MapTR &17.7 &5.9 & \\ 
        &TopoNet &28.6 &10.9 & 172.6 \\ 
        &TopoMLP &28.5 &21.7 & 328.2 \\ 
        % &Topo2D &29.1 &22.3 & \\ 
        % &TopoMLP &28.5 &21.7 & \\\cmidrule{1-5}
        % \multirow{3}{*}{Perspective images + SD maps} &TopoOSMR &30.6 &17.1 & \\
        % &SMERF &33.4 &15.4 & \\ 
        % &TopoLogic* &34.4 &28.9 & \\
        \cmidrule{1-5}
        SD and satellite maps &\methodname (Ours) &\textbf{37.9} &\textbf{31.9} &\textbf{44.3} \\
        \bottomrule
    \end{tabular}
    }
\end{table}


\begin{table*}[!t]\centering
    \begin{spacing}{0.8}
    \caption{\textbf{Overall performance comparison on the OpenLane-V2 Dataset.} Integrating map prior features into online mapping pipelines, \methodname-OL consistently boosts different kinds of online mapping pipelines by a wide margin, yielding state-of-the-art online driving topology reasoning performance.
    % When integrated with online mapping methods, \methodname generated map priors significantly boost online mapping performance. The improvement across different baselines is consistent.
    }\label{tab:prior_online_mapping}
    \end{spacing}
    \footnotesize
    % \resizebox{0.9\linewidth}{!}{%
    \begin{tabular}{ccccccccc}\toprule
        Input type &Venues &Methods &${\rm DET_l}$ &${\rm TOP_{ll}}$ &${\rm DET_t}$ &${\rm TOP_{lt}}$ &\cellcolor{gray!30} OLS \\\cmidrule{1-8}
        \multirow{8}{*}{Perspective images} &ICCV 2021 &STSU~\cite{Can2021STSU} &12.7 &2.9 &43.0 &19.8 &\cellcolor{gray!30}29.3 \\ 
        &ICML 2023 &VectorMapNet~\cite{liu2023vectormapnet} &11.1 &2.7 &41.7 &9.2 &\cellcolor{gray!30}24.9 \\ 
        &ICLR 2023 &MapTR~\cite{MapTR} &17.7 &5.9 &43.5 &15.1 &\cellcolor{gray!30}31.0 \\ 
        &Arxiv 2023 &TopoNet~\cite{li2023toponet} &28.6 &10.9 &48.6 &23.8 &\cellcolor{gray!30}39.8 \\ 
        &Arxiv 2024 &TopoLogic~\cite{fu2024topologic} &29.9 &23.9 &47.2 &25.4 &\cellcolor{gray!30}44.1 \\ 
        &Arxiv 2024 &Topo2D~\cite{li2024topo2d} &29.1 &22.3 &\textbf{50.6} &26.2 &\cellcolor{gray!30}44.5 \\ 
        &ICLR 2024 &TopoMLP~\cite{wu2024topomlp} &28.5 &21.7 &\underline{49.5} &26.9 &\cellcolor{gray!30}44.1 \\ 
        &ECCV 2024 &RoadPainter~\cite{ma2024roadpainter} &30.7 &22.8 &47.7 &27.2 &\cellcolor{gray!30}44.6 \\\cmidrule{1-8}
        \multirow{4}{*}{Perspective images + SD maps}  
        &IROS 2024 &TopoOSMR~\cite{zhang2024enhancingonlineroadnetwork} &30.6 &17.1 &44.6 &26.8 &\cellcolor{gray!30}42.1 \\ &ICRA 2024 &SMERF~\cite{luo2024smerf} &33.4 &15.4 &48.6 &25.4 &\cellcolor{gray!30}42.9 \\
        &Arxiv 2024 &TopoLogic~\cite{fu2024topologic} &34.4 &28.9 &48.3 &28.7 &\cellcolor{gray!30}47.5 \\ 
        &ECCV 2024 &RoadPainter~\cite{ma2024roadpainter} &36.9 &\underline{29.6} &47.1 &29.5 &\cellcolor{gray!30}48.2 \\\cmidrule{1-8}
        \multirow{4}{*}{Perspective images + Map priors} & \multirow{4}{*}{\textbf{Ours}} &\multirow{2}{*}{\textbf{\methodname-OL} (TopoNet)} &\underline{46.1} &27.5 &48.3 &\textbf{33.1} &\cellcolor{gray!30}\underline{51.1} \\ 
        & & & +61.2\% & +152.3\% & -0.6\% & +39.1\% & \cellcolor{gray!30}+28.4\% \\% \cmidrule{3-8}
        & &\multirow{2}{*}{\textbf{\methodname-OL} (TopoMLP)} &\textbf{46.6} &\textbf{37.0} &47.7 &\underline{33.0} &\cellcolor{gray!30}\textbf{53.1} \\
        & & & +63.5\% & +70.5\% & -3.6\% & +22.7\% & \cellcolor{gray!30}+20.4\% \\
        \bottomrule
    \end{tabular}
    % }
    \vspace{-6pt}
\end{table*}

% \begin{table*}[!h]\centering
%     \caption{Generated by Spread-LaTeX}\label{tab: }
%     % \resizebox{\linewidth}{!}{%
%     \begin{tabular}{cccccccc}\toprule
%         Input type &Methods &DET\_l &TOP\_ll &DET\_t &TOP\_lt & \cellcolor{gray!30} OLS \\\cmidrule{1-7}
%         \multirow{6}{*}{Perspective images} &VectorMapNet &11.1 &2.7 &41.7 &9.2 & \cellcolor{gray!30}24.9 \\ 
%         &MapTR &17.7 &5.9 &43.5 &15.1 &\cellcolor{gray!30}31.0 \\ 
%         &TopoNet &28.6 &10.9 &48.6 &23.8 &\cellcolor{gray!30}39.8 \\ 
%         &TopoLogic &29.9 &23.9 &47.2 &25.4 &\cellcolor{gray!30}44.1 \\ 
%         &Topo2D &29.1 &22.3 &\textbf{50.6} &26.2 &\cellcolor{gray!30}44.5 \\ 
%         &TopoMLP &28.5 &21.7 &49.5 &26.9 &\cellcolor{gray!30}44.1 \\\cmidrule{1-7}
%         \multirow{2}{*}{Perspective images + SD maps} &SMERF &33.4 &15.4 &48.6 &25.4 &\cellcolor{gray!30}42.9 \\
%         &TopoLogic* &34.4 &28.9 &48.3 &28.7 &\cellcolor{gray!30}47.5 \\\cmidrule{1-7}
%         \textcolor[rgb]{0.5,0.5,0.5}{SD and satellite maps} & \textcolor[rgb]{0.5,0.5,0.5}{AutoHD (Ours)} & \textcolor[rgb]{0.5,0.5,0.5}{37.9} & \textcolor[rgb]{0.5,0.5,0.5}{31.9} & \textcolor[rgb]{0.5,0.5,0.5}{/} & \textcolor[rgb]{0.5,0.5,0.5}{/} & \textcolor[rgb]{0.5,0.5,0.5}{/} \\\cmidrule{1-7}
%         Perspective images + Map priors &AutoHD-OL (Ours)&\textbf{46.1} &27.5 &48.3 &\textbf{33.1} &\cellcolor{gray!30}\textbf{51.1} \\
%         \bottomrule
%     \end{tabular}
%     % }
% \end{table*}

The evaluation is conducted on the primary set of the OpenLane-V2 dataset~\cite{wang2023openlanev2}, which extends Argoverse 2 sensor set~\cite{Argoverse2} with ground truth for traffic element detection and topology relationship association. The associated 7-view images are only available during training of online mapping.

We extract the corresponding ego-centric SD maps from OSM~\cite{OpenStreetMap} and satellite images from Mapbox Raster Tiles~\cite{mapbox_raster_tiles_api} using the methods described in Sections~\ref{sec:sd_fetching} and~\ref{sec:sate_fetching}. All maps are centered on the ego-vehicle's position and oriented in the direction of travel, covering an area of $100{\rm m} \times 50{\rm m}$ along the longitudinal and lateral directions, respectively. 

% For training SD and satellite map priors, we utilize the Argoverse 2 motion forecasting dataset~\cite{Argoverse2}. Despite being without sensor data, it contains 2,110 km unique roadways with HD map information and geo-referenced locations. We extracted corresponding ego-centric SD maps from OSM and satellite maps from Google Maps. This results in a much larger and more diverse dataset for map prior training. All maps are centered on the vehicle's position and oriented in the direction of travel, covering an area of $100{\rm m} \times 50{\rm m}$ along the longitudinal and lateral directions, respectively. 

%% NOTE decide if we want to mention we transfer original HD map to lane topology

\subsubsection{Metrics} 
% We evaluate \methodname along with state-of-the-art approaches on the widely-used OpenLane-V2~\cite{wang2023openlanev2} dataset subset~A. Derived from the sensor and HD map data from Argoverse 2 sensory set~\cite{Argoverse2}, OpenLane-V2 provides additional 2D traffic element annotations and lane-to-traffic-element relationship. The sensor data includes 7 surround-view cameras. From the HD map data, they extracted the centerline labels as point lists and lane-to-lane connectivity labels as association matrices. For traffic elements, such as traffic lights, traffic signs, road markings, they are labeled as 2D bounding boxes in images, and associated with related lanes as association matrices.
We report the OpenLane-V2 Score (OLS) defined in~\cite{wang2023openlanev2}, which is computed as:
\begin{equation}
    \text{OLS} = \frac{1}{4} \left [ \text{DET}_{\rm l} + \text{DET}_{\rm t} + \sqrt{\text{TOP}_{\rm ll}} + \sqrt{\text{TOP}_{\rm lt}} \right ] \quad ,
\end{equation}
where DET$_{\rm l}$ is the discrete Fréchet distance~\cite{eiter1994frechetdistance} mean average precision (mAP) for lane geometry, DET$_{\rm t}$ is the mAP for traffic elements. TOP$_{\rm ll}$ and TOP$_{\rm lt}$ are the topology scores for lane-to-lane connectivity and lane-to-traffic-element relationship, respectively.



\subsection{Implementation details}
We implement \methodname with PyTorch. The number of sampling points $N$ in SD maps is set to 11. We adopt a 6-layer Transformer encoder as the SD map encoder. For satellite map encoding, we first resize satellite images to $500\times 250$ and then utilize ResNet-50~\cite{He2016resnet} pretrained on ImageNet~\cite{Krizhevsky2012imagenet} as the backbone for features extraction. Multi-scale features from the last three stages of ResNet-50 are employed and cross-attented with a BEV feature map $\mathbf{B}$, which has a size of $H_{\rm B} = 200$, $W_{\rm B} = 100$, and $C=256$. The number of centerline queries $N_{\rm L}$ is set to 200. To keep the training time manageable while maintaining the diversity of HD maps, we sample frames in each scene with a frame rate of 0.5 and filter out still frames. This results in 200k scenes with 810k HD maps, which are used to train \methodname in the first stage, unless otherwise specified. \methodname is trained for 8 epochs with AdamW~\cite{adamw} optimizer on 8 NVIDIA A100 GPUs, which takes $\sim$2 days to complete training. The batch size for each GPU is 12. 

We integrate well-trained \methodname into two state-of-the-art open-sourced baselines for second-stage training, namely TopoNet~\cite{li2023toponet} and TopoMLP~\cite{wu2024topomlp}, which represent the BEV-based and perspective-based models, respectively. We maintain their default setting of using ResNet-50 as the image backbone for multi-view images. All other settings remain unchanged, except for the integration of map prior features from \methodname, as described in Sec.~\ref{sec:online_integration}.

\subsection{How far can we get with \methodname alone?}\label{sec:direct}

\methodname is trained to generate lane graphs solely from geospatial maps in the first stage. Naturally, this raises the question of how accurate the lane graphs predicted from SD and satellite maps are. In Table~\ref{tab:direct_hd_mapping}, we compare \methodname with two state-of-the-art \textit{online} driving topology reasoning methods~\cite{li2023toponet, wu2024topomlp} on the validation set of~\cite{wang2023openlanev2}. Metrics related to traffic elements are excluded due to the impracticality of detecting them from an aerial view. The results show that \methodname achieves superior performance, with ${\rm DET_l}$ of 37.9 compared to 28.6 and 28.5 for TopoNet and TopoMLP. In terms of ${\rm TOP_{ll}}$, \methodname scores 31.9, significantly higher than TopoNet's 10.9 and TopoMLP's 21.7. Additionally, \methodname shows significantly lower per-frame latency of 44.3 ms, which is $3.9\times$ and $7.4\times$ faster than online mapping models.
% outperforming online mapping methods that use multi-view inputs by a large margin. 
Despite differences in input modality and the amount of training data, the remarkable performance of \methodname underscores the abundant prior information contained in geospatial maps and the vast potential for scaling driving topology reasoning from a geospatial map perspective.

% To evaluate the lane perception and reasoning performance of the pretrained \methodname, we compare it with two baseline online mapping methods. Please note that online mapping methods typically take perspective images from live streams as input, and some also incorporate SD maps as prior information. In contrast, we only take SD maps and satellite images as input. We downsize our SD and satellite maps dataset by x, with a size x times larger than the OpenLane-V2 dataset. As shown in Table~\ref{tab:direct_hd_mapping}, with only SD and satellite maps as input, our approach \methodname performs well in HD mapping tasks. While this is not a direct comparison since \methodname is trained with more data, it demonstrates that SD maps and satellite images effectively capture semantics and HD map information.

 % Due to the aerial perspective, offline lane topology is limited to centerline detection and lane connectivity, lacking the ability to detect or reason about traffic elements.

\subsection{How much can \methodname boost online topology reasoning?}

In light of our goal to achieve robust \textit{online} driving topology reasoning, we expect \methodname also to yield state-of-the-art performance in \textit{online} settings. Therefore, we train two kinds of online topology reasoning baselines~\cite{li2023toponet, wu2024topomlp} jointly with map priors extracted by well-trained \methodname, in comparison to baselines along with state-of-the-art approaches. All models are trained for 24 epochs on the training set of~\cite{wang2023openlanev2}. As shown in Table~\ref{tab:prior_online_mapping}, \methodname-OL consistently improves baselines by over 20\% on OLS. Notably, \methodname-OL improves lane detection performances of both baselines by over 60\%, and boosts TopoNet's lane topology ability by 152.3\%. Owing to enhanced lane detection, the performance of lane-to-traffic-element topology is also improved by over 20\%. In comparison to previous works which only introduce SD maps as priors, \methodname-OL remains superior in all metrics. More specifically, TopoNet, when enhanced by \methodname, substantially outperforms both SMERF~\cite{luo2024smerf} and TopoOSMR~\cite{zhang2024enhancingonlineroadnetwork}, which also use TopoNet as their baseline. Some qualitative comparisons are illustrated in Fig.~\ref{fig:vis}, where \methodname-OL generates more complete lane graphs compared to both baselines. These results verify the efficacy of adopting map priors from \methodname to online mapping.

\Remark Since the traffic elements are detected directly from perspective images, as expected, the performance of \methodname-OL on ${\rm DET_t}$ remains similar to the baselines. 

% We would like to validate if SD and satellite maps based prior improve online mapping. We take the network trained in Section~\ref{sec:direct} to extract embeddings as prior for the OpenLane-V2 benchmark. As shown in Table~\ref{tab:prior_online_mapping}, \methodname extracted prior significantly increases the online mapping performance. For TopoNet~\cite{li2023toponet} baseline, there is a 61\% improvement in lane detection mAP and 152\% improvement in lane topology reasoning. For TopoMLP~\cite{wu2024topomlp} baseline, there is also a 64\% improvement in lane detection mAP and 71\% improvement in lane topology reasoning. As shown in Figure~\ref{fig:vis}, \methodname generates more complete lane graphs and consistently improves the two baselines.

% Moreover, \methodname priors are also better or comparable to the approach with only SD map prior. For example, \methodname enhanced TopoMLP surpass state-of-the-art approach RoadPainter~\cite{ma2024roadpainter} in all metrics, with 26\% improvement in lane detection and 25\% in lane topology reasoning. 

\begin{figure}[!t]
\centering
  \includegraphics[width=0.99\linewidth]{img/visualization.pdf}
  \vspace{-18pt}
\caption{\textbf{Qualitative comparison of \methodname-OL to baselines.} The top-left shows the SD map plotted on top of the satellite image. Our method improves baselines consistently, producing more complete lane graphs.
} % change this figure as the gt lines at bottom is not complete
\label{fig:vis}
\vspace{-10pt}
\end{figure}

% \noindent\textbf{Qualitative results.}


\subsection{How well can \methodname generalize to unseen areas?}

\begin{table}[!t]\centering
    \begin{spacing}{0.8}
    \caption{\textbf{Evaluation in unseen areas.} \methodname-OL improves the baseline on all metrics, especially on ${\rm DET_l}$ and $\rm TOP_{ll}$.}\label{tab:prior-geo-disjoint}
    \end{spacing}
    \footnotesize
    % \resizebox{0.8\linewidth}{!}{%
        \begin{tabular}{ccccccc}\toprule
        Methods &$\rm DET_l$ &$\rm TOP_{ll}$ & $\rm DET_t$ &$\rm TOP_{lt}$ & \cellcolor{gray!30}OLS \\\cmidrule{1-6}
        Baseline &16.6 &4.7 &32.9 &13.3 &\cellcolor{gray!30}26.9 \\
        \textbf{\methodname-OL} &\textbf{24.7} &\textbf{11.7} &\textbf{34.7} &\textbf{16.7} &\cellcolor{gray!30}\textbf{33.6} \\\cmidrule{1-6}
        Improvements (\%) &48.8 &148.9 &5.5 &25.6 &\cellcolor{gray!30}24.9 \\
        \bottomrule
        \end{tabular}
    % }
    % \vspace{-7pt}
\end{table}

Previous works have identified that OpenLane-V2~\cite{wang2023openlanev2} contains geographically overlapping areas between training and validation sets~\cite{lilja2024localizationevaluatedataleakage, luo2024smerf}. To evaluate the performance of SMART compared to the baseline in completely unseen areas, we resplit the training and validation sets in~\cite{wang2023openlanev2}, along with the data used for the first-stage training, ensuring geo-disjoint training and evaluation. The performance comparison on the geo-disjoint splits is shown in Table~\ref{tab:prior-geo-disjoint}. Notably, \methodname-OL improves the baseline in terms of lane detection by \textbf{48.8}\% and lane topology reasoning by \textbf{148.9}\%, yielding the OLS of 33.6, significantly improving baseline's generalizability to novel areas.

% Lilja et al.~\cite{lilja2024localizationevaluatedataleakage} pointed out the data leakage issue in using Argoverse 2~\cite{Argoverse2} and NuScenes~\cite{caesar_nuscenes_2020} dataset splits for evaluating online mapping. As these datasets are not designed for mapping, many locations in the evaluation set are seen in the training set. The derived OpenLane-V2~\cite{wang2023openlanev2} also suffers from this problem. Thus, we perform experiments on the geo-disjoint splits of the OpenLane-V2 dataset subset A to remove the data leakage effect. As shown in Table~\ref{tab:prior-geo-disjoint}, we note that \methodname priors improve lane detection by 49\% and lane topology reasoning by 149\%. The results suggest that the enhancement from \methodname prior are consistent and improves the generalization of the baseline models in unseen environments. 



\subsection{Can \methodname benefit from scaled training data?}

% \begin{table}[!h]\centering
%     \caption{Scaling studies for \methodname}\label{tab: }
%     \resizebox{\linewidth}{!}{%
%         \begin{tabular}{cccccc}\toprule
%             Data amount &DET\_l &TOP\_ll &DET\_t &TOP\_lt &\cellcolor{gray!30} OLS \\\cmidrule{1-6}
%             Base &34.8 \textcolor[rgb]{0.5,0.5,0.5}{(33.0)} &20.9 \textcolor[rgb]{0.5,0.5,0.5}{(16.2)} & 47.9 & 27.7 &\cellcolor{gray!30}{45.3} \\
%             x18 &43.8 \textcolor[rgb]{0.5,0.5,0.5}{(34.5)} &24.0 \textcolor[rgb]{0.5,0.5,0.5}{(28.9)} &48.0 &31.6 &\cellcolor{gray!30}49.3 \\
%             x40 &\textbf{46.1} \textcolor[rgb]{0.5,0.5,0.5}{(37.9)} &\textbf{27.5} \textcolor[rgb]{0.5,0.5,0.5}{(31.9)} &\textbf{48.3} &\textbf{33.1} &\cellcolor{gray!30}\textbf{51.1} \\
%             \bottomrule
%         \end{tabular}
%     }
% \end{table}

\begin{table}[!t]\centering
    \begin{spacing}{0.8}
    \caption{\textbf{Scaling studies for data size.} The performances of \methodname in both \textit{offline} and \textit{online} settings improve progressively as the amount of training data grows. 
    % when the data scale increases for both direct offline mapping and as prior for online mapping, showing great potential for scaling up when more data is accessible.
    }\label{tab:scale}
    \end{spacing}
    \footnotesize
    \resizebox{\linewidth}{!}{%
        \begin{tabular}{c|cc|cccccc}\toprule
            \multirow{2}{*}{Data amount} &\multicolumn{2}{c|}{Offline} &\multicolumn{5}{c}{Online} \\\cmidrule{2-8}
            &${\rm DET_l}$ &${\rm TOP_{ll}}$ &${\rm DET_l}$ &${\rm TOP_{ll}}$ &${\rm DET_t}$ &${\rm TOP_{lt}}$ &\cellcolor{gray!30} OLS \\\cmidrule{1-8}
            Base &33.0 &16.2 &34.8 &20.9 &47.9 &27.7 &\cellcolor{gray!30}45.3 \\
            18$\times$ &34.5 &28.9 &43.8 &24.0 &48.0 &31.6 &\cellcolor{gray!30}49.3 \\
            \textbf{40$\times$} &\textbf{37.9} &\textbf{31.9} &\textbf{46.1} &\textbf{27.5} &\textbf{48.3} &\textbf{33.1} &\cellcolor{gray!30}\textbf{51.1} \\
            \bottomrule
        \end{tabular}
    }
\vspace{-8pt}
\end{table}
% 38.7
% 35.0

% 49.1
% 29.3
% 47.6
% 34.1
% 52.3

Scaling law has been found in Transformer-based models, and scaling the data size generally leads to improved performance. In Table~\ref{tab:scale}, we experiment \methodname with different amounts of training data, where \textit{Base} corresponds to the number of HD maps in the training set of sensor dataset~\cite{Argoverse2}. With the increases in data size to 18$\times$ and 40$\times$ larger, the performance of \methodname in offline settings increases progressively. 
We also observe significant improvements in lane topology reasoning as data size increases, growing from 16.2 to 28.9 and 31.9 when trained with 18$\times$ and 40$\times$ larger data size, respectively. Improved performance in offline settings is consistent in online settings when enhancing the online model with the corresponding \methodname trained offline. This scaling property verifies \methodname's great potential for generalizable topology reasoning in complex scenarios.

% As SD maps and satellite maps are easily accessible, we can scale up the \methodname training data to get stronger prior. we create three scales of the dataset, a base dataset with xxx scenarios, x18 and x40 times larger datasets. As shown in Table~\ref{tab:scale}, \methodname performance increases as the dataset size increases for direction HD mapping and online mapping experiments. This demonstrates the scalability of the \methodname model and its potential to improve when more data is available.

\subsection{Can SD and satellite map fusion boost performance?}

We empirically ablate SD or satellite maps to study the importance of each modality. As shown in Table~\ref{tab:ablation_sd_satellite}, the elimination of either modality leads to a degradation in both offline and online settings, with the removal of satellite maps causing a more significant decline in performance. This observation suggests that lane-level textures contained in satellite images are crucial for topology reasoning. The superior performance yielded by fusing both modalities indicates that the structural information from SD maps and the semantics from satellite images complement each other.  

\subsection{Can \methodname reduce the reliance on sensor data?}

\begin{table}[!t]\centering
    \begin{spacing}{0.8}
    \caption{\textbf{Ablation study on SD and satellite maps.} The removal of either modality leads to a degradation in performance, indicating the necessity of fusing both.
    % Satellite maps are shown to be more important than SD maps while the two modalities combined yield the best model.
    }\label{tab:ablation_sd_satellite}
    \end{spacing}
    \footnotesize
    \resizebox{\linewidth}{!}{%
        \begin{tabular}{c|cc|cccccc}\toprule
            \multirow{2}{*}{Prior type} &\multicolumn{2}{c|}{Offline} &\multicolumn{5}{c}{Online} \\\cmidrule{2-8}
            &${\rm DET_l}$ &${\rm TOP_{ll}}$ &${\rm DET_l}$ &${\rm TOP_{ll}}$ &${\rm DET_t}$ &${\rm TOP_{lt}}$ &\cellcolor{gray!30}OLS \\\cmidrule{1-8}
            SD maps &24.7 &12.4 &34.2 &16.5 &\textbf{48.9} &27.8 &\cellcolor{gray!30}44.1 \\
            Satellite maps &35.1 &23.4 &41.2 &22.2 &48.7 &30.7 &\cellcolor{gray!30}48.1 \\\cmidrule{1-8}
            \textbf{SD and satellite maps} &\textbf{37.9} &\textbf{31.9} &\textbf{46.1} &\textbf{27.5} &48.3 &\textbf{33.1} &\cellcolor{gray!30}\textbf{51.1} \\
            \bottomrule
        \end{tabular}
    }
    \vspace{-6pt}
\end{table}

\begin{figure}[!t]
\centering
  \includegraphics[width=0.99\linewidth]{img/data_amount.pdf}
  \vspace{-15pt}
\caption{
\textbf{Impact of varying sensor data availability.}
With only 40\% of sensor data, \methodname-OL achieves performance comparable to using the full sensor data, demonstrating its robustness with reduced sensor data.
% \methodname (red solid line) provides a strong prior such that even when the training data is reduced by 50\%, the model performance is on par with the baseline (gray dash line) trained on full sensor data without \methodname priors.
% Samples of SD and satellite maps, along with corresponding lane graph.
} % change this figure as the gt lines at bottom is not complete
\label{fig:reduce_data}
\vspace{-10pt}
\end{figure}

\methodname trained on massive SD and satellite maps exhibits strong performance of topology reasoning. 
To investigate whether scaling up \methodname can compensate for limited sensor data in online mapping, we compare against the baseline online mapping model enhanced with \methodname trained on geospatial maps corresponding to \textit{sensor} set. We then progressively reduce the amount of sensor data used to train the online mapping model enhanced by \methodname trained on \textit{full} geospatial maps. As shown in Fig.~\ref{fig:reduce_data}, even with just 40\% of sensor data, the model utilizing priors generated from \textit{full} \methodname achieves performance comparable to that of the model using priors from \textit{sensor} \methodname, which adopts the complete sensor dataset for training.
This suggests that scaling up \methodname can significantly enhance online mapping performance, even with reduced sensor data.

% and compare it against the online mapping model using \methodname trained specifically on geospatial maps corresponding to full sensor data. 
% We are curious if this powerful pre-trained model can reduce the reliance on sensor data by online mapping. In this experiment, we integrate the well-trained, frozen \methodname into the online mapping model and train on different amount of sensor data, in comparison to the online mapping model integrated with \methodname trained on geospatial maps corresponding to sensor data trained on full sensor data. 
% As shown in Fig.~\ref{fig:reduce_data}, \methodname outperforms the baseline even when the sensor data is reduced by 50\%. This suggests that by scaling up \methodname, we can significantly enhance online mapping performance, even when sensor data is limited.

% In this experiment, we train the baseline with full sensor data along with corresponding SD and satellite maps. For the \methodname model, we use do pretraining on large-scale SD and satellite maps with HD map labels, then use various portions of the sensor dataset for online mapping training. As shown in Fig.~\ref{fig:reduce_data}, \methodname outperforms the baseline even when the sensor data is reduced by 50\%. This shows that we can leverage \methodname to boost online mapping performance when sensor data is limited.

\section{Conclusion and Future Works}
This paper introduces \methodname, offering a new perspective on scalable and generalizable driving topology reasoning while circumventing the need for extensive sensor data. By leveraging readily available geospatial maps and existing large-scale HD map datasets, \methodname yields impressive offline topology reasoning and supplies robust map prior representations that can be seamlessly integrated into any online driving topology reasoning architectures, achieving state-of-the-art performance.
% \methodname enhances both offline and online topology reasoning, showing superior performance, scalability, and the ability to reduce reliance on costly sensory data collection. 
More broadly, \methodname opens up promising avenues for future research: (1)~scaling up \methodname in both model size and data to develop a comprehensive map foundation model, and (2)~exploring the immense potential of map prior features for other tasks, such as trajectory prediction, motion planning, and end-to-end driving, wherever a robust understanding of lane structures is critical. We strongly believe that this work will considerably advance scalable and generalizable driving topology reasoning in autonomous driving.

% TODO discuss the use of hd maps from other sources as future work

\section*{ACKNOWLEDGMENT}
We'd like to acknowledge our friends and colleagues, including Katie Z Luo, Cheng Zhao, Arun Das, Pranav Ganti, Nikhil Advani, and Sarthak Gupta, for their fruitful discussions and follow-ups.

\bibliographystyle{IEEEtran}
\bibliography{icra25}


\end{document}
