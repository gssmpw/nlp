\section{Related Works}
\subsection{Online Mapping}

Online mapping seeks to perceive lane structures on the fly, offering instant scene information for downstream tasks, in contrast to offline methods that rely on traffic observations~\cite{zurn2024autograph} or aerial images~\cite{Buchner2023UrbanLaneGraph, Blayney_2024_CVPR_bezier}. Early works primarily focus on estimating the geometry of road elements~\cite{li2021hdmapnet, liu2023vectormapnet, MapTR, maptrv2}. As one of the pioneering works, HDMapNet~\cite{li2021hdmapnet} generates an online semantic map, followed by post-processing to obtain vectorized road elements. Liu \textit{et al.}~\cite{liu2023vectormapnet} model vectorized HD map learning in an end-to-end fashion. Liao \textit{et al.}~\cite{MapTR, maptrv2} propose a permutation invariant loss that boosts map geometry learning. Recent efforts have extended online mapping to include topology reasoning, which involves understanding the connectivity and relationships between lanes and traffic elements~\cite{wang2023openlanev2, li2023toponet, wu2024topomlp, fu2024topologic}. Graph-based approaches like TopoNet~\cite{li2023toponet} use scene graph neural networks to model these relationships, while simpler models like TopoMLP~\cite{wu2024topomlp} still achieve promising improvements in topology reasoning with multilayer perceptrons (MLPs). Other methods~\cite{li2024topo2d, fu2024topologic} explore novel ways to improve the initialization of 3D queries and interpret geometric distances for better topology reasoning. However, a common bottleneck for these methods is the reliance on large amounts of sensor data collected under consistent configurations.

% , while recent advancements extend this to topology reasoning~\cite{wang2023openlanev2, li2023toponet, wu2024topomlp, fu2024topologic}.

% \noindent\textbf{Geometry Only.} Lane geometry estimation involves detecting the position and shape of road elements such as lane dividers, boundaries, and pedestrian crossings. For instance, Li \textit{et al.} propose HDMapNet~\cite{li2021hdmapnet}, which estimates an online semantic map, followed by post-processing to obtain vectorized road elements. Liu et al.~\cite{liu2023vectormapnet} vectorized the pipeline without predicting an intermediate dense semantic map. Liao et al.~\cite{MapTR, maptrv2} propose a permutation invariant loss that boosts map geometry learning. However, these works are limited by estimating the individual road element geometry and does not model the connectivity explicitly, resulting in disconnected lane segments and broken road networks.

% \noindent\textbf{Topology Reasoning.} To facilitate learning the relationship between road elements, a recent dataset OpenLane-V2~\cite{wang2023openlanev2} introduces lane-to-lane and lane-to-traffic-elements topology reasoning. Lane-to-lane topology models the connectivity between lanes, while lane-to-traffic-element topology associates traffic light, and traffic signs to the lanes. Li et al. propose TopoNet~\cite{li2023toponet}, which models the relationship with a scene graph neural network. Wu et al. propose TopoMLP~\cite{wu2024topomlp}, a simple yet strong model that significantly increases the topology reasoning performance. Topo2D~\cite{li2024topo2d} argues that initializing 3d queries from 2D queries with 2D lane priors can generate better results. TopoLogic~\cite{fu2024topologic} leverages an interpretable geometric distance between lane endpoints and lane queries to facilitate the topology reasoning. One of the greatest bottlenecks for topology reasoning is data. These methods often require large amounts of sensor data captured from the same sensor platform for training.

\subsection{Map Priors}
Incorporating prior knowledge has been shown to advance online mapping performance~\cite{xiong2023neuralmapprior, luo2024smerf, Gao2024ICRA, zhang2024enhancingonlineroadnetwork, Jiang2024pmapnet, Yuan2024presight}. Xiong \textit{et al.}~\cite{xiong2023neuralmapprior} assume a multi-traversal setting to enhance map perception with features from previous traversals. More recently, some works~\cite{luo2024smerf, zhang2024enhancingonlineroadnetwork} integrate SD maps with surroundings view images for joint training, demonstrating improved topology modeling performance. Similarly, Gao~\textit{et al.}~\cite{Gao2024ICRA} aid road element detection with satellite imagery. In spite of promising improvements achieved, priors introduced in these works remain consistently coupled to limited sensor data, leading to unsatisfying scalability.
By contrast, we aim to learn a unified map prior representation from massive geospatial maps, featuring generalizability to novel locations and compatibility with any online topology reasoning models.
Additionally, we identify the benefits of combining the road-level topology priors from SD maps and comprehensive bird's eye view (BEV) textures from satellite maps.



% Many works introduce prior knowledge to improve model performance with limited data. Standard definition (SD) maps, such as OpenStreetMaps~\cite{haklay2008openstreetmap} and Google Maps provide the basic road network information without centimeter localization and many details. Luo et al. propose SMERF~\cite{luo2024smerf} that fuses the SD maps with BEV features through cross attention. Zhang et al.~\cite{zhang2024enhancingonlineroadnetwork} demonstrated the benefit of fusing the rasterized and graphical SD maps, showcasing the benefit of handling occlusion and intersection topology. While SD maps provide easily accessible road network prior, they miss the details in lanes. Therefore, we introduce satellite images as additional prior information to enhance the prior quality. Xiong et al.~\cite{xiong2023neuralmapprior} proposed neural map prior, which generates a global map prior that enables longer range and higher quality perception. It is worth noting that they are global map prior which requires sensor data to generate, whereas our model can generate prior for any location with SD and satellite maps.