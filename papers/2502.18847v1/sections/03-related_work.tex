\noindent \textbf{Traditional Tabular Machine Learning} 
Historically, the realm of tabular data modeling over the past decade has been largely dominated by conventional machine learning methods~\cite{shwartz2022tabular}. 
Models such as Gradient Boosting~\cite{bentejac2021comparative}, ExtraTrees~\cite{geurts2006extremely}, and Random Forests~\cite{breiman2001random} have been pivotal in learning intricate data patterns and enhancing robustness against overfitting. 
Notable techniques like XGBoost \cite{chen2016xgboost} and LightGBM \cite{ke2017lightgbm} stand out for their efficiency, optimization techniques, and scalability, making them go-to options in various applications. 
Logistic regression \cite{hosmer2013applied} has been particularly applied to binary classification tasks due to its simplicity and interpretability.
Specialized algorithms like CatBoost~\cite{prokhorenkova2018catboost}, designed to handle categorical features seamlessly, have gained prominence. 
This diverse set of models contributes to a versatile toolbox, addressing the intricacies of tabular data modeling with distinct strengths and adaptability \cite{grinsztajn2022tree}. These traditional models provide a solid foundation for tabular data analysis, balancing interpretability, efficiency, and performance essential for real-world applications.
Despite their effectiveness, these models are often limited by their reliance on handcrafted feature engineering and their inability to leverage the representation learning capabilities inherent in deep learning models.\looseness-1

\section{Overview}

\revision{In this section, we first explain the foundational concept of Hausdorff distance-based penetration depth algorithms, which are essential for understanding our method (Sec.~\ref{sec:preliminary}).
We then provide a brief overview of our proposed RT-based penetration depth algorithm (Sec.~\ref{subsec:algo_overview}).}



\section{Preliminaries }
\label{sec:Preliminaries}

% Before we introduce our method, we first overview the important basics of 3D dynamic human modeling with Gaussian splatting. Then, we discuss the diffusion-based 3d generation techniques, and how they can be applied to human modeling.
% \ZY{I stopp here. TBC.}
% \subsection{Dynamic human modeling with Gaussian splatting}
\subsection{3D Gaussian Splatting}
3D Gaussian splatting~\cite{kerbl3Dgaussians} is an explicit scene representation that allows high-quality real-time rendering. The given scene is represented by a set of static 3D Gaussians, which are parameterized as follows: Gaussian center $x\in {\mathbb{R}^3}$, color $c\in {\mathbb{R}^3}$, opacity $\alpha\in {\mathbb{R}}$, spatial rotation in the form of quaternion $q\in {\mathbb{R}^4}$, and scaling factor $s\in {\mathbb{R}^3}$. Given these properties, the rendering process is represented as:
\begin{equation}
  I = Splatting(x, c, s, \alpha, q, r),
  \label{eq:splattingGA}
\end{equation}
where $I$ is the rendered image, $r$ is a set of query rays crossing the scene, and $Splatting(\cdot)$ is a differentiable rendering process. We refer readers to Kerbl et al.'s paper~\cite{kerbl3Dgaussians} for the details of Gaussian splatting. 



% \ZY{I would suggest move this part to the method part.}
% GaissianAvatar is a dynamic human generation model based on Gaussian splitting. Given a sequence of RGB images, this method utilizes fitted SMPLs and sampled points on its surface to obtain a pose-dependent feature map by a pose encoder. The pose-dependent features and a geometry feature are fed in a Gaussian decoder, which is employed to establish a functional mapping from the underlying geometry of the human form to diverse attributes of 3D Gaussians on the canonical surfaces. The parameter prediction process is articulated as follows:
% \begin{equation}
%   (\Delta x,c,s)=G_{\theta}(S+P),
%   \label{eq:gaussiandecoder}
% \end{equation}
%  where $G_{\theta}$ represents the Gaussian decoder, and $(S+P)$ is the multiplication of geometry feature S and pose feature P. Instead of optimizing all attributes of Gaussian, this decoder predicts 3D positional offset $\Delta{x} \in {\mathbb{R}^3}$, color $c\in\mathbb{R}^3$, and 3D scaling factor $ s\in\mathbb{R}^3$. To enhance geometry reconstruction accuracy, the opacity $\alpha$ and 3D rotation $q$ are set to fixed values of $1$ and $(1,0,0,0)$ respectively.
 
%  To render the canonical avatar in observation space, we seamlessly combine the Linear Blend Skinning function with the Gaussian Splatting~\cite{kerbl3Dgaussians} rendering process: 
% \begin{equation}
%   I_{\theta}=Splatting(x_o,Q,d),
%   \label{eq:splatting}
% \end{equation}
% \begin{equation}
%   x_o = T_{lbs}(x_c,p,w),
%   \label{eq:LBS}
% \end{equation}
% where $I_{\theta}$ represents the final rendered image, and the canonical Gaussian position $x_c$ is the sum of the initial position $x$ and the predicted offset $\Delta x$. The LBS function $T_{lbs}$ applies the SMPL skeleton pose $p$ and blending weights $w$ to deform $x_c$ into observation space as $x_o$. $Q$ denotes the remaining attributes of the Gaussians. With the rendering process, they can now reposition these canonical 3D Gaussians into the observation space.



\subsection{Score Distillation Sampling}
Score Distillation Sampling (SDS)~\cite{poole2022dreamfusion} builds a bridge between diffusion models and 3D representations. In SDS, the noised input is denoised in one time-step, and the difference between added noise and predicted noise is considered SDS loss, expressed as:

% \begin{equation}
%   \mathcal{L}_{SDS}(I_{\Phi}) \triangleq E_{t,\epsilon}[w(t)(\epsilon_{\phi}(z_t,y,t)-\epsilon)\frac{\partial I_{\Phi}}{\partial\Phi}],
%   \label{eq:SDSObserv}
% \end{equation}
\begin{equation}
    \mathcal{L}_{\text{SDS}}(I_{\Phi}) \triangleq \mathbb{E}_{t,\epsilon} \left[ w(t) \left( \epsilon_{\phi}(z_t, y, t) - \epsilon \right) \frac{\partial I_{\Phi}}{\partial \Phi} \right],
  \label{eq:SDSObservGA}
\end{equation}
where the input $I_{\Phi}$ represents a rendered image from a 3D representation, such as 3D Gaussians, with optimizable parameters $\Phi$. $\epsilon_{\phi}$ corresponds to the predicted noise of diffusion networks, which is produced by incorporating the noise image $z_t$ as input and conditioning it with a text or image $y$ at timestep $t$. The noise image $z_t$ is derived by introducing noise $\epsilon$ into $I_{\Phi}$ at timestep $t$. The loss is weighted by the diffusion scheduler $w(t)$. 
% \vspace{-3mm}

\subsection{Overview of the RTPD Algorithm}\label{subsec:algo_overview}
Fig.~\ref{fig:Overview} presents an overview of our RTPD algorithm.
It is grounded in the Hausdorff distance-based penetration depth calculation method (Sec.~\ref{sec:preliminary}).
%, similar to that of Tang et al.~\shortcite{SIG09HIST}.
The process consists of two primary phases: penetration surface extraction and Hausdorff distance calculation.
We leverage the RTX platform's capabilities to accelerate both of these steps.

\begin{figure*}[t]
    \centering
    \includegraphics[width=0.8\textwidth]{Image/overview.pdf}
    \caption{The overview of RT-based penetration depth calculation algorithm overview}
    \label{fig:Overview}
\end{figure*}

The penetration surface extraction phase focuses on identifying the overlapped region between two objects.
\revision{The penetration surface is defined as a set of polygons from one object, where at least one of its vertices lies within the other object. 
Note that in our work, we focus on triangles rather than general polygons, as they are processed most efficiently on the RTX platform.}
To facilitate this extraction, we introduce a ray-tracing-based \revision{Point-in-Polyhedron} test (RT-PIP), significantly accelerated through the use of RT cores (Sec.~\ref{sec:RT-PIP}).
This test capitalizes on the ray-surface intersection capabilities of the RTX platform.
%
Initially, a Geometry Acceleration Structure (GAS) is generated for each object, as required by the RTX platform.
The RT-PIP module takes the GAS of one object (e.g., $GAS_{A}$) and the point set of the other object (e.g., $P_{B}$).
It outputs a set of points (e.g., $P_{\partial B}$) representing the penetration region, indicating their location inside the opposing object.
Subsequently, a penetration surface (e.g., $\partial B$) is constructed using this point set (e.g., $P_{\partial B}$) (Sec.~\ref{subsec:surfaceGen}).
%
The generated penetration surfaces (e.g., $\partial A$ and $\partial B$) are then forwarded to the next step. 

The Hausdorff distance calculation phase utilizes the ray-surface intersection test of the RTX platform (Sec.~\ref{sec:RT-Hausdorff}) to compute the Hausdorff distance between two objects.
We introduce a novel Ray-Tracing-based Hausdorff DISTance algorithm, RT-HDIST.
It begins by generating GAS for the two penetration surfaces, $P_{\partial A}$ and $P_{\partial B}$, derived from the preceding step.
RT-HDIST processes the GAS of a penetration surface (e.g., $GAS_{\partial A}$) alongside the point set of the other penetration surface (e.g., $P_{\partial B}$) to compute the penetration depth between them.
The algorithm operates bidirectionally, considering both directions ($\partial A \to \partial B$ and $\partial B \to \partial A$).
The final penetration depth between the two objects, A and B, is determined by selecting the larger value from these two directional computations.

%In the Hausdorff distance calculation step, we compute the Hausdorff distance between given two objects using a ray-surface-intersection test. (Sec.~\ref{sec:RT-Hausdorff}) Initially, we construct the GAS for both $\partial A$ and $\partial B$ to utilize the RT-core effectively. The RT-based Hausdorff distance algorithms then determine the Hausdorff distance by processing the GAS of one object (e.g. $GAS_{\partial A}$) and set of the vertices of the other (e.g. $P_{\partial B}$). Following the Hausdorff distance definition (Eq.~\ref{equation:hausdorff_definition}), we compute the Hausdorff distance to both directions ($\partial A \to \partial B$) and ($\partial B \to \partial A$). As a result, the bigger one is the final Hausdorff distance, and also it is the penetration depth between input object $A$ and $B$.


%the proposed RT-based penetration depth calculation pipeline.
%Our proposed methods adopt Tang's Hausdorff-based penetration depth methods~\cite{SIG09HIST}. The pipeline is divided into the penetration surface extraction step and the Hausdorff distance calculation between the penetration surface steps. However, since Tang's approach is not suitable for the RT platform in detail, we modified and applied it with appropriate methods.

%The penetration surface extraction step is extracting overlapped surfaces on other objects. To utilize the RT core, we use the ray-intersection-based PIP(Point-In-Polygon) algorithms instead of collision detection between two objects which Tang et al.~\cite{SIG09HIST} used. (Sec.~\ref{sec:RT-PIP})
%RT core-based PIP test uses a ray-surface intersection test. For purpose this, we generate the GAS(Geometry Acceleration Structure) for each object. RT core-based PIP test takes the GAS of one object (e.g. $GAS_{A}$) and a set of vertex of another one (e.g. $P_{B}$). Then this computes the penetrated vertex set of another one (e.g. $P_{\partial B}$). To calculate the Hausdorff distance, these vertex sets change to objects constructed by penetrated surface (e.g. $\partial B$). Finally, the two generated overlapped surface objects $\partial A$ and $\partial B$ are used in the Hausdorff distance calculation step.

\noindent \textbf{Transformers for tabular data} 
Following the popularity of Transformer architectures in vision and language, several methods~\cite{hollmann2022tabpfn, arik2021tabnet, zhu2023xtab} have adapted transformers for learning from tabular datasets. 
For instance, FT-Transformer~\cite{gorishniy2021revisiting} showed superior performance in tabular classification and regression tasks by separating numerical and categorical features. 
Additionally, Saint~\cite{somepalli2021saint} introduced row-wise attention, capturing inter-sample interactions, Fastformer~\cite{wu2021fastformer} suggested the use of additive attention which is lightweight with linear complexity, while TransTab~\cite{wang2022transtab} incorporated transfer learning in tabular tasks, all using transformers as backbones. Recent advancements have specifically tailored the transformer architecture to address challenges in data imputation and cross-table learning, incorporating modifications to the attention mechanism and embedding layers~\cite{badaro2023transformers}.\looseness-1

\noindent \textbf{Self-supervised pretraining} 
Furthermore, the emergence of self-supervised pretraining in the tabular domains has paved the way for novel approaches to feature extraction and representation learning, reducing the reliance on labeled data \cite{liu2021self}. Specifically, drawing inspiration from the success of pretraining in vision and language, previous studies have delved into tabular self-supervised learning \cite{yoon2020vime, ucar2021subtab, somepalli2021saint, bahri2021scarf, majmundar2022met, rubachev2022revisiting, wang2022transtab}. 
Authors in~\cite{yoon2020vime, ucar2021subtab} introduced an auto-encoder framework with a pretext task focused on reconstructing missing elements in a table while \cite{bahri2021scarf} utilized contrastive learning~\cite{chen2020simple} as pretraining objective for improving generalizability of trained architectures in tabular tasks. 
Additionally, \cite{rubachev2022revisiting, wang2022transtab} created a target-aware objective by incorporating label columns of tabular tasks in pretraining. 
Although these innovations have largely improved performance over traditional machine learning approaches, these models have been shown to particularly underperform in the presence of heterogeneous feature columns~\cite{tabllm}.\looseness-1

\noindent \textbf{Modality switch for Tabular Deep Learning} Recent research has explored the conversion of tabular data into orthogonal modalities, such as text, image, and graph. 
TabLLM \cite{tabllm} converted tabular data to text for few-shot classification using large language models. Although it can suffer from context loss and inefficiency when handling high-dimensional data, TabLLM successfully captures the semantic information encapsulated within columns in a table.
SuperTML \cite{wang2019supertml} introduced a method to transform tabular data into a super ensemble of image-based data points, enabling the use of convolutional neural networks for tabular tasks. DeepInsight \cite{DeepInsight} proposed projecting tabular data into an image space using t-SNE, enabling the application of image classification models to tabular data. Even though this technique effectively captures underlying feature correlations, the reliance on a single-image representation and t-SNE's specific distance metric limits its ability to capture diverse and multi-faceted relationships inherent in complex tabular datasets. 
Table2Graph~\cite{huang22table2graph} transforms tabular data into a unified weighted graph and IGNNet \cite{ignnet} transforms each record into a fully-connected graph, allowing the application of graph neural networks (GNNs) for tabular data learning. 
Additionally, GCondNet~\cite{margeloiu2023gcondnet} transforms each column into a graph while CARTE~\cite{kim2024carte} mines entities in tables to learn from entity-centric graphs.
Furthermore, models like Graph foundation models \cite{galkin2024towards, zhang2024gnn} and \cite{sun2023gpt} highlight the efficacy of GNNs in capturing relational structures within tabular data.  
HyTrel \cite{chen2023hytrel} enhances tabular data representation by integrating hypergraph structures, which can capture high-order relationships among features, but the complexity of hypergraph construction and the increased computational cost are significant challenges.
Despite their innovative approach, these methods often face scalability issues with large datasets and are sensitive to the graph construction method. Additionally, even though graphs can capture the structural relationships among features in a table, they cannot capture the semantic information of the categorical and text columns, as well as the column headers. This information can provide valuable insight, which is especially valuable when learning from small datasets. \looseness-1

\noindent \textbf{Multi-Modal Learning}
Multi-modal learning integrates data from multiple sources, such as text, image, video, and audio to enhance machine learning models' performance. A pivotal model in this domain is CLIP \cite{radford2021learning}, which aligns text and image representations using contrastive learning, enabling effective zero-shot learning and image-text retrieval. Other significant advancements include~\cite{hegde2023clip3d}, which adapts CLIP to 3D recognition tasks through prompt tuning for language grounding, as well as \cite{Chen_2023}, which introduces cross-modal knowledge distillation, and \cite{ramesh2021zero}, which introduces zero-shot text-to-image generation.\looseness-1

An important lesson from existing literature is that multi-modal models are capable of generalizing to downstream tasks by capturing complementary information from multiple modalities. For instance, they extract complex spatial patterns from images, semantic meaning from text, and structural relationships from graphs.  
We capitalize on this property to design a multi-modal model for tabular machine learning that combines the richness of graph and text modalities into a unified embedding space to improve performance on downstream ML tasks.
To the best of our knowledge, we are the first to introduce multi-modal learning for tabular datasets using a single table as input across several classification based downstream tasks.\looseness-1