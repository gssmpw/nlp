% 
% Annual Cognitive Science Conference
% Sample LaTeX Paper -- Proceedings Format
% 

% Original : Ashwin Ram (ashwin@cc.gatech.edu)       04/01/1994
% Modified : Johanna Moore (jmoore@cs.pitt.edu)      03/17/1995
% Modified : David Noelle (noelle@ucsd.edu)          03/15/1996
% Modified : Pat Langley (langley@cs.stanford.edu)   01/26/1997
% Latex2e corrections by Ramin Charles Nakisa        01/28/1997 
% Modified : Tina Eliassi-Rad (eliassi@cs.wisc.edu)  01/31/1998
% Modified : Trisha Yannuzzi (trisha@ircs.upenn.edu) 12/28/1999 (in process)
% Modified : Mary Ellen Foster (M.E.Foster@ed.ac.uk) 12/11/2000
% Modified : Ken Forbus                              01/23/2004
% Modified : Eli M. Silk (esilk@pitt.edu)            05/24/2005
% Modified : Niels Taatgen (taatgen@cmu.edu)         10/24/2006
% Modified : David Noelle (dnoelle@ucmerced.edu)     11/19/2014
% Modified : Roger Levy (rplevy@mit.edu)     12/31/2018



%% Change "letterpaper" in the following line to "a4paper" if you must.

\documentclass[10pt,letterpaper]{article}

\usepackage{cogsci}
\usepackage{graphicx}
\usepackage{amsmath}

\cogscifinalcopy % Uncomment this line for the final submission 


\usepackage{pslatex}
\usepackage{apacite}
\usepackage{float} % Roger Levy added this and changed figure/table
                   % placement to [H] for conformity to Word template,
                   % though floating tables and figures to top is
                   % still generally recommended!

%\usepackage[none]{hyphenat} % Sometimes it can be useful to turn off
%hyphenation for purposes such as spell checking of the resulting
%PDF.  Uncomment this block to turn off hyphenation.


% \setlength\titlebox{4.5cm}
% You can expand the titlebox if you need extra space
% to show all the authors. Please do not make the titlebox
% smaller than 4.5cm (the original size).
%%If you do, we reserve the right to require you to change it back in
%%the camera-ready version, which could interfere with the timely
%%appearance of your paper in the Proceedings.



\title{Visual and Auditory Aesthetic Preferences Across Cultures}
 
\author{
    {\large \bf Harin Lee\textsuperscript{1,2,*}, Eline Van Geert\textsuperscript{3}, Elif Çelen\textsuperscript{1}, Raja Marjieh\textsuperscript{4}, Pol van Rijn\textsuperscript{1}, Minsu Park\textsuperscript{5}, and Nori Jacoby\textsuperscript{1,6}} \\
  \textsuperscript{1}Max Planck Institute for Empirical Aesthetics, Frankfurt am Main, Germany \\
  \textsuperscript{2}Max Planck Institute for Human Cognitive and Brain Sciences, Leipzig, Germany \\
  \textsuperscript{3}KU Leuven, Leuven, Belgium \\
  \textsuperscript{4}Princeton University, New Jersey, United States \\
  \textsuperscript{5}New York University Abu Dhabi, Abu Dhabi, United Arab Emirates \\
  \textsuperscript{6}Cornell University, New York, United States \\ \\
  *Corresponding author email: \url{harin.lee@ae.mpg.de} \\
}
  

\begin{document}
\maketitle


\begin{abstract}
Research on how humans perceive aesthetics in shapes, colours, and music has predominantly focused on Western populations, limiting our understanding of how cultural environments shape aesthetic preferences. We present a large-scale cross-cultural study examining aesthetic preferences across five distinct modalities extensively explored in the literature: shape, curvature, colour, musical harmony and melody. Our investigation gathers 401,403 preference judgements from 4,835 participants across 10 countries, systematically sampling two-dimensional parameter spaces for each modality. The findings reveal both universal patterns and cultural variations. Preferences for shape and curvature cross-culturally demonstrate a consistent preference for symmetrical forms. While colour preferences are categorically consistent, relational preferences vary across cultures. Musical harmony shows strong agreement in interval relationships despite differing regions of preference within the broad frequency spectrum, while melody shows the highest cross-cultural variation. These results suggest that aesthetic preferences emerge from an interplay between shared perceptual mechanisms and cultural learning.

\textbf{Keywords:} 
% add your choice of indexing terms or keywords; kindly use a
% semicolon; between each term
aesthetics; art and cognition; cross-cultural analysis; vision; music; culture; big data

\end{abstract}



\section{Introduction}
Human aesthetic experiences permeate our daily lives, from the visual arts to music, influencing countless decisions. When choosing the outfit to wear, we consider colour combinations. When listening to music, certain sounds evoke pleasure while others create discomfort. These aesthetic judgements, whilst deeply personal, are embedded within broader cultural frameworks that shape our taste and experiences of the world~\cite{che2018cross}.

The origin of aesthetic preferences has been debated throughout history, from Plato's philosophical discourse on beauty~\cite{pappas_platos_2024} to contemporary cognitive scientists~\cite{palmer_visual_2013}. However, it remains open to what extent aesthetic preferences arise from cultural learning or innate universal principles~\cite{che2018cross,sharman1997anthropology}.

Evidence exists for both perspectives. Research has demonstrated universal preferences for certain features such as symmetric shapes and human faces~\cite{bertamini2020, little_preferences_2007}. Complementary studies have debated whether mathematical ratio rules, like the golden ratio and Fibonacci sequences, create visually balanced and pleasing geometry~\cite{green_all_1995}. Yet, much work refutes these claims, finding no evidence for specific ratio biases~\cite{davis_unity_1991, mcmanus2010}. In music, consonant and dissonant harmonies were long thought to be universally perceived, mathematically explainable through measuring the coherence of sound phases~\cite{helmholtz_sensations_1954}. However, recent cross-cultural insights reveal that certain musical interpretations are largely culturally learnt~\cite{mcdermott_indifference_2016}.

Two major challenges remain in addressing these longstanding debates. First, insights on aesthetic judgements have predominantly focused on Western and English-speaking populations, which does not adequately represent the diversity of humanity~\cite{blasi_over-reliance_2022, che2018cross}. Consequently, the cross-cultural replicability of aesthetic preferences remains unclear, as well as the extent to which one's taste is culturally nurtured. Second, studies rarely compare beyond a single modality (though see~\citeNP{chen_taste_2022, clemente2021, palmer_accounting_2013}). Cross-modal comparisons to measure the relative homogeneity and heterogeneity across cultures can illuminate whether preferences arise innately from human biology or largely through cultural learning.

Here, we address these challenges through a large-scale, cross-cultural investigation. Collecting 401,403 human judgements from 4,835 participants across 10 countries (Figure~\ref{fig:fig1}), we systematically examine aesthetic preferences for shape (aspect ratio of rectangle), curvature (Bézier curve; \citeNP{Farin1993-ac}), colour combinations (differing in hue degrees), musical harmony and melody (pitch intervals). Notably, we focus on two-dimensional representations of these five distinct modalities so that they can be continuously sampled without relying on discrete categories (e.g., canonical aspect ratios in shape and curvature). Thus, this approach allows to capture controlled parametric variation, at the same time, low-dimensionality makes it feasible for large-scale data collection.

\begin{figure*}[t] 
  \centering
    \includegraphics[width=\textwidth]{figures/fig1.png} 
    \caption{
      Schematics of experimental design. 
      (A) In independent experiments for each modality, participants were asked to rate how much they like the seen or heard stimulus on a 7-point scale from 1 being ``not at all'' to 7 ``very much'', translated into their own language. Each presented stimulus was defined at random by sampling two points from the general space (see `Defining Stimulus Space' in Methods).
      (B) Participants were recruited from 10 countries, including all continents which are coloured according to regions defined by WorldBank (\url{www.worldbank.org}).
    } 
  \label{fig:fig1} 
  \end{figure*}
    
\section{Background}
We have deliberately selected modalities that have been extensively explored in empirical aesthetics research for comparisons with our cross-cultural insights. Below, we summarise key findings related to these modalities.

\subsection{Shape}
Early studies have shown that rectangular shapes following specific ratio rule, such as the \textit{golden ratio} (approximately 1:1.618), are often perceived as more aesthetically appealing~\cite{fechner1876}.
However, individual differences in aesthetic preferences for rectangles have emerged across studies~\cite{green_all_1995, mcmanus2010, mcmanus_square_2013}. These variations challenge the assumed universality of the golden ratio~\cite{stieger_time_2015}, highlighting the significant influence of cultural and contextual factors in shape perception.

Some studies indicate the brain processes horizontal proportions with greater ease, as this scanning direction aligns naturally with our landscape-oriented vision system~\cite{mcmanus_square_2013}.

\subsection{Curvature}
Research consistently demonstrates that curved shapes elicit more favourable responses, being perceived as pleasant, calming and beautiful compared to angular forms~\cite{chuquichambi_how_2022}. Moreover, studies indicate that smooth curvature, characterised by gradual transitions along lines or surfaces, induces greater aesthetic appeal than abrupt or angular changes. The concept is notably illustrated by William Hogarth's \textit{line of beauty}~\cite{hogarth_analysis_1753}, an S-shaped curve that embodies ideal curvature. Hogarth posited that moderate curvature strikes the optimal aesthetic balance, avoiding both excessive flatness and extreme undulation.

\subsection{Colour}
Preferences for colour combinations are understood to emerge from a complex interplay of ecological associations, harmony principles, and cultural or personal contexts (\citeNP{vangeert_jacoby_2024}; for meta-review, see~\citeNP{palmer_visual_2013}). Contrasting colours can create compelling visual effects, particularly when warm figures (e.g., red or yellow) appear against cool backgrounds (e.g., blue or green), enhancing perceived saturation~\cite{schloss_aesthetic_2011}. Research has revealed notable cross-cultural differences in colour preferences~\cite{taylor_color_2013}, which may stem from the distinct meaning associations that colours carry within different cultural contexts.

\subsection{Harmony}
Exposure to specific musical systems is thought to shape our mental representations of musical harmony. Recent cross-cultural studies involving small-scale societies suggest that perceptions of pleasantness and harmony are culturally influenced rather than universal~\cite{mcdermott_indifference_2016, McPherson2020-cb}. 

A recent study employed a large-scale approach to systematically explore preferences for harmonies by uniformly covering a broad space of musical chords~\cite{marjieh_timbral_2024}. Their findings revealed distinct patterns in interval preferences that are shaped by timbre.

\subsection{Melody}
Compared to musical harmony, preferences for musical melody are relatively less understood. Experimental work explored the related areas of memory and expectations~\cite{narmour1992analysis,dowling1986music}, while corpus work identified patterns that are common in Western music~\cite{vos1989ascending,rodriguez2013perceptual}. Computational modelling approaches have utilised corpus data to understand probabilistic sequences of musical notes to predict surprise and pleasure~\cite{temperley2008probabilistic}. 

Recent studies using iterated paradigms in online settings show that sung melodies, passed along participant chains, evolve into prototypical sequences, revealing shared internal representations and preferences~\cite{anglada-tort_large-scale_2023}.

\section{Methods}
\subsection{Cross-cultural Recruitment}
We recruited 4,835 participants through the online CINT platform (\url{www.cint.com}), using an approved ethical protocol (Max Planck Ethics Council \#202142). The selection of countries ensured broad geographical and linguistic coverage across all continents, as we theorised these demographical contexts would lead to differences in aesthetic preferences~\cite{majid_establishing_2023}. 
% Table~\ref{table1} presents the demographic distribution by country collected from participant in the end of each session. 
To be eligible for participation, individuals were required to have been born in their respective country and be current residents (Figure~\ref{fig:fig1}B). Participants received compensation at their country's standard wage rate. Sample sizes and specific locations were determined based on a pilot experiment conducted in English and the participant pool availability as reported by CINT.

% \begin{table}[h]
% \centering
% \caption{Participant demographics}
% \begin{tabular}{lccc}
% \hline
% Country & N & Age (SD)\\
% \hline
% France & 503 & 50.7 (13.7) \\
% Germany & 515 & 49.4 (15.4) \\
% India & 514 & 34.5 (11.0) \\
% Japan & 456 & 53.4 (12.1) \\
% Mexico & 298 & 32.0 (10.8) \\
% Nigeria & 512 & 33.6 (10.3) \\
% S.Korea & 495 & 45.0 (13.0) \\
% Poland & 506 & 40.6 (13.6) \\
% Turkey & 523 & 37.0 (10.6) \\
% USA & 513 & 51.1 (15.3) \\
% % \hline
% % Total & 4,835 & 42.7 (12.6) & 35.4 \\
% \hline
% \label{table1}
% \end{tabular}
% \end{table}

\subsection{Defining Stimulus Space}\label{sec:define_space}
We systematically sampled parameters by defining a fixed two-parameter space for each modality. In each trial, two values were randomly selected from the given space to define the stimulus on the spot (Figure~\ref{fig:fig1}A). This way, we could uniformly sample across the entire space without discrete categories and thus pinpoint high and low preference density regions of the full continuous space. In particular, this is advantageous in a cross-cultural context, as it allows us to avoid making assumptions about specific discrete categories that are primarily derived from existing research focused on Western participants~\cite{blasi_over-reliance_2022}.

\subsubsection{Shape space.}
Rectangles were generated with varying width and height, ranging from 20 to 60 pixels, to be presented in a 100$\times$100 pixel window. The range selection represents a careful balance between detecting subtle differences in aspect ratios whilst maintaining a reasonable maximum ratio. We selected values that constrained the maximal aspect ratio between 1:1 and 1:3 (or 3:1).

\subsubsection{Curvature space.}
Smooth curved lines were created using the cubic Bézier curve formula~\cite{Farin1993-ac}:

$B(t) = (1-t)^3P_0 + 3(1-t)^2tP_1 + 3(1-t)t^2P_2 + t^3P_3$

\noindent where the curve always starts and ends at the vertical centerline ($x$=50), and $t$ ranges from 0 to 1. $P_0=(50,0)$ is the start point, $P_1=(50+a,25)$ is the first control point, $P_2 = (50+b,75)$ is the second control point, and $P_3 = (50,100)$ is the endpoint (i.e., 100px in length). The parameters $a$ and $b$ ranged from -50 to 50, controlling the curvature of the line.

\subsubsection{Colour space.}
Pairs of colours were generated using the OKLCH colour space~\cite{oklch}. Whilst there are other colour spaces such as RGB and HSV, we selected OKLCH as it more accurately represents human perception of colour similarities. We set the lightness to 70\% and chroma to 0.15, presenting two colours (25px width, 50px height) side by side with each varying hue angles from 0° to 360°. These parameters were carefully chosen to align with previous research on colour representation~\cite{schloss_aesthetic_2011, vangeert_jacoby_2024}.

\subsubsection{Harmony and melody spaces.}
Pairs of harmonic complex tones (dyads) were played simultaneously for harmony (1s in duration), while sequentially presented for melodies (750ms in duration) by including a 250ms gap, spanning a continuous MIDI range from 60 to 75 (C4 to C5). These tones were synthesised during the experiment using \textit{ToneJS} (\url{tonejs.github.io}), following the same parameter settings as~\citeNP{marjieh_timbral_2024} (10 harmonics, amplitude roll-off=12dB).

\subsection{Experimental Procedure}
Stimuli were presented using $PsyNet$~\cite{harrison2020}, a web-based experimental platform (\url{www.psynet.dev}). Each participant completed 80 trials. In each trial, participants viewed or heard a stimulus and rated ``How pleasant is this [modality type] from a scale of 1 (not at all) to 7 (very much)?'' Responses were collected using a 7-point Likert scale (Figure~\ref{fig:fig1}A). To prevent rapid clicking, participants could only proceed after a 1.5-second delay. Each trial lasted approximately 4 seconds, with the complete experiment taking 8 minutes. 

\subsubsection{Additional screening.}
For colour perception, we accounted for display variations by applying gamma correction to standardised value 1/2.2, proven effective in previous online studies~\cite{epicoco_can_2024}. Participants were instructed to disable night-shift mode and required to pass a colour blindness test consisting of six Ishihara plates~\cite{clark1924}. Participants also verified screen brightness by adjusting until three grey rectangles were visible against a darker grey background.

For harmony and melody perception, participants were asked to adjust the volume to a comfortable level and were instructed to use headphones. An audio screening test required them to identify the odd sound amongst three options.


% RESULTS
\section{Results}

% FIGURE 2
\begin{figure*}[t] 
  \centering
    \includegraphics[width=\textwidth]{figures/fig2.png} 
    \caption{
      Preferred regions in modality spaces.
      (A) Using a fixed bandwidth to smooth the preference ratings, the most preferred regions in each modality space are highlighted in yellow. White diagonal lines indicate where values above and below are equal but in differing two parameter orders.
      (B) Cross-cultural similarity in regions of preference and their variability. Jensen-Shannon distance was used to measure the similarity between country-level matrices and dimensionality reduction was performed using UMAP. Countries positioned closer together share similar regions of preference. Insets show examples from different coordinates of these UMAPs to illustrate variations.
      } 
  \label{fig:fig2} 
  \end{figure*}

% FIGURE 3
\begin{figure*}[t] 
  \centering
    \includegraphics[width=\textwidth]{figures/fig3.png} 
    \caption{
      Relational preference across modalities is assessed as follows: Shape = width-to-height aspect ratios; Curvature = the difference between control points $P1$ and $P2$; Colour = absolute difference in degrees between paired hues; Harmony and melody = pitch intervals between tone pairs in semitones. Each line represents a GAM-fitted curve per country, with colours denoting world regions.
    }
  \label{fig:fig3} 
  \end{figure*}

% FIGURE 4
\begin{figure}[t] 
  \centering
    \includegraphics[width=\columnwidth]{figures/fig4.png} 
    \caption{
      Agreement and disagreement between cultures.
      Between-country correlations in (A) preferred regions in modality spaces, and (B) relational preferences across modalities. Below each of these, we report the reliability using split-half correlations. Error bars indicate 95\% CI.
    }
  \label{fig:fig4} 
  \end{figure}

\subsection{Preferred Regions in Modality Spaces}
We begin by exploring the regions in each modality space that participants found aesthetically pleasing or displeasing. Figure~\ref{fig:fig2}A displays these spaces and preferred regions in yellow, aggregated across all countries. We generated these heatmaps by smoothing the preference ratings across the continuous parameter spaces with a fixed grid. Values above or below the white diagonal lines indicate the two parameters being symmetrical.

Our analysis reveals that preferences in certain modalities depend strongly on the ratios between parameter values. This is evident in the strong preference following the diagonal in shape space, where perfectly squared shapes lie, and in the striped patterns in musical harmony, which reflect structured interval preferences. In contrast, other modalities show more categorical or absolute preferences. For example, colour combinations demonstrate that bluish hues are consistently preferred while dark sandy colours are disliked, regardless of their pairings. For curvature lines, the strongest preference is seen at regions that form \textit{S-shaped} patterns where $P1$ and $P2$ distances mirror each other symmetrically, but also when both control points point equally in the same direction, creating a \textit{bumpy-hill} shape. Notably, melody preferences displays the least structured pattern, suggesting greater cross-cultural or individual variability.

To assess cross-cultural similarities in these preference spaces, we analysed each country separately and measured the similarity between them using the Jensen-Shannon distance, which is suitable for comparing probability distributions. Figure~\ref{fig:fig2}B shows these relationships after applying UMAP for dimension reduction, with various coordinates of the relation space shown as small insets.

The observed pattern reveals clear cross-cultural differences. For instance, Japanese participants uniquely prefer straight lines, a pattern not observed elsewhere---in contrast, straight lines were the least preferred in France. The striped pattern in musical harmony, seen at the global level (Figure~\ref{fig:fig2}A), is particularly prominent among German and US participants (with Germans favouring harmonies in the lower frequency spectrum), whilst it is less visible in countries such as Turkey and Nigeria.

Interestingly, geographical proximity does not necessarily indicate a similar preference. While Korea and Japan consistently cluster together in these preference spaces, European countries and the US show no clear grouping, suggesting that additional cultural or historical factors may be at play.

\subsection{Relational Preference}
The question of whether aesthetic appeals are governed by specific mathematical principles (i.e., ratio rules) has been extensively debated (see meta-review by~\citeNP{palmer_visual_2013}). Building on previous research exploring these structural relationships, as shown in Figure~\ref{fig:fig3}, we analysed relational preferences across five modalities, including the aspect ratios of rectangular shapes, the differences between two control points of cubic Bézier curves, the differences in hue degrees between colour pairs, and pitch intervals for both harmony and melody (translated from frequency to continuous MIDI semitones). We fitted a Generalised Additive Model (GAM) with a fixed smoothing parameter ($k$ knots = 15) to capture the non-linear relationships in marginal distributions.

For rectangular shapes, we see a prominent peak at the 1:1 ratio, representing perfect squares, as was seen with preference along the diagonal in Figure~\ref{fig:fig2}. The preference strength gradually declines as the ratio deviates from this symmetrical form. Notably, we find no evidence of preference peaks at the famously known golden ratio (1.618:1 or its inverse) in any of the countries, aligning with recent observations~\cite{stieger_time_2015}. Interestingly, the data from South Korea shows distinct peaks around the 4:3 and 3:2 aspect ratios, which align with common proportions used in visual media and frame design, suggesting how these familiar aspect ratios could have appeared appealing~\cite{cossar_shape_2009}. 

A particularly intriguing finding present across all cultures is a consistent bias in participants favouring wider (z-score mean = 0.27) rectangles over taller ($m$ = 0.17) ones with equivalent aspect ratios ($t$ = 76.5, $p$ $<$ .001; seen as slight asymmetry with higher ratings above the 1:1 in Figure~\ref{fig:fig3}). This finding aligns with previous works demonstrating this bias, as objects placed wider can indicate stability and thus appear more appealing~\cite{goffaux_horizontal_2010}.

For curved lines, we observe the strongest peaks when the two control points $P1$ and $P2$ sum to zero (i.e., when they are exact opposites). Slight deviations from this symmetry result in reduced preference. Previous research mainly focused on curved lines taken from Hogarth's line of beauty~\cite{hogarth_analysis_1753, Hubner2022-pv}, which examines variations in seven categories of S-shaped curves. Our work extends beyond this finding and demonstrates that curves pointing in the same direction (e.g., difference of 100) are equally, or even more preferred than S-shaped curves. Notably, Japan and Germany uniquely demonstrate a distinct peak when the control points are roughly 75 values apart (e.g., disproportionate S-curves).

For colours, we observe a general declining trend in preference as the hue degrees between paired colours diverge, which aligns with previous observations~\cite{schloss_aesthetic_2011}. Yet, while discretisation of parameters in past works limited the resolution of this decline (though see~\citeNP{vangeert_jacoby_2024} for continuous space exploration), we see an apparent dip at approximately 30° degrees difference. This suggests visual dissonance when colours are similar but slightly mismatched. Several countries exhibit preference peaks at around 45° degrees, corresponding to \textit{semi-analogous} colour combinations, and at around 135° degrees, representing \textit{triadic} relationships. \textit{Complementary} colours (180° degrees apart) are traditionally considered harmonious, but this preference appears to manifest only within US and French responses.

For musical harmonies, we generally replicate the patterns previously observed among the US and Korean participants using the same experimental design~\cite{marjieh_timbral_2024}. Our analysis also reveals strong preferences for \textit{unison} (same tone), \textit{perfect fifth} (7 semitones), and \textit{octaves} (12 semitones). We also find notable dips in \textit{tritone} (6 semitones), which is known to be displeasing. However, some countries deviate from these established patterns. In Mexico, for instance, the tritone is not particularly disliked. Additionally, countries such as India and Nigeria show more uniform preferences (i.e., flatter lines) across intervals.

For melodies, the defined space was identical to harmonies. However, we find strikingly different patterns with substantial cross-cultural variation. Notably, for some countries (e.g., Mexico), preference peaks do not align closely with the standardised pitch intervals (e.g., keys on a piano). Yet, most countries demonstrate tendencies for favouring octave and disfavouring tritone melodies, which is in line with past observation on US and Indian online participants~\cite{anglada-tort_large-scale_2023}.


\subsection{Agreement and Disagreement Between Cultures}
Finally, we empirically quantified the extent of shared agreement on aesthetic preferences between cultures. For all pairwise combinations of countries in each modality, we computed Spearman correlations for the entire preference spaces (Figure~\ref{fig:fig4}A), followed by computing correlations between GAM fittings that describe mathematical relational preferences (Figure~\ref{fig:fig4}B).

High or low between-country correlations can be influenced by the reliability, or noisiness, of each modality. We thus added to the bottom of Figure~\ref{fig:fig4} that reports within-modality reliability by computing split-half correlations within each country with 100 bootstrap simulations, and then aggregating to calculate the mean.

Our analysis shows that preferences for shapes (space rho = 0.84, 95\% CI = [0.83, 0.86]; relational rho = 0.89 [0.88, 0.91]) and curvatures (space rho = 0.77 [0.73, 0.81]; relational rho = 0.80 [0.75, 0.85]) are generally highly agreed across cultures.

For colour, between-country correlations are high when comparing spaces (rho = 0.84 [0.82, 0.86]), which stems from most countries favouring colour combinations with bluish hues (as previously seen in Figure~\ref{fig:fig2}). By contrast, there exists little agreement for relational differences in hue degrees (rho = 0.46 [0.35, 0.57]). This suggests that colour preferences are more absolute and categorical, with the unlikely presence of governing mathematical relations.

An opposite pattern emerges for musical harmony. While there is little agreement between spaces (rho = 0.28 [0.21, 0.34]), we see strong agreement in preferred musical intervals (rho = 0.78 [0.75, 0.80]). This demonstrates that harmony preferences follow certain ratio rules~\cite{helmholtz_sensations_1954}, whereas their spatial preferences can vary (e.g., as seen in Figure~\ref{fig:fig2}B, Germans preferred harmonies lower in the frequency spectrum). By contrast, melody shows consistently the lowest agreement between countries (space rho = 0.25 [0.19, 0.30], relational rho = 0.38 [0.31, 0.46]), which is in line with no apparent structure or clusters visible in Figure~\ref{fig:fig2}.

Notably, modalities with little agreement across cultures generally also have low reliability (i.e., low agreement within-country). This suggests that less universally agreed modalities underlie either (i) noisier cognitive processing for evaluating preferences in those modalities, or (ii) there is higher variability in individual preferences.

% 1) there being no acoustic roughness (beating) as with harmony, (2) preference for melody being highly culture-specific, (3) there being large individual variability as music processing is closely associated with emotions~\cite{juslin_emotions_2022}, or (4) cognitive processing of melodies is a lot noisier when evaluating preference. 


\section{Discussion}
Our large-scale cross-cultural study captures diverse cultural nuances in aesthetic preferences worldwide. We found cultural variation in almost all modalities. Ratio relationships within each modality were almost always important, except for colours, where we observed categorical behaviour that are consistent with previous findings (e.g.,~\citeNP{vangeert_jacoby_2024}). Furthermore, the amount of variability between cultures was substantially different across the modalities: for instance, shape and curvature showed more universal preference, while melodic preferences were highly varied.

These findings together highlight how certain aspects of aesthetic appreciation might be more innate and driven by psychological foundations, while others are significantly influenced by one's cultural upbringing.

% Our findings challenge several observations previously made within Western contexts alone. For instance, preference of harmonic intervals observed across US individuals were uniform across entire frequency spectrum (\citeNP{marjieh_timbral_2024}; replicated in Figure~\ref{fig:fig2}B). However, in other countries, either the strong preference for intervals were reduced (e.g., India) or concentrated in different areas of pitch height (e.g., Germany).

% In parallel, our research reveals certain universal aspects of aesthetic preference. These include the widespread appreciation for symmetrical rectangles and maximally curved lines (with Japan and Korea being notable exceptions), as well as a consistent dislike of dark sandy colours across cultures. 

Conducting the study online enabled broad scalability, but it comes with the cost of some experimental control. Traditional laboratory studies of colour perception utilise precisely calibrated monitors, and music experiments take place in acoustically isolated environments. Online participants are also influenced by global and mainstream media. Future work should replicate the paradigm in laboratory settings with participants from diverse cultures including small-scale societies~\cite{jacoby_universal_2019, Jacoby2024-tw}. Moreover, the specified parameter spaces we used could also have missed other important areas of cultural nuances. Hence, they should be extended to explore different parameter ranges, dimensions (e.g., varying lightness instead of colour hue), and higher dimensions using efficient sampling methods~\cite{harrison2020, Van_Rijn2022-ol}.

To conclude, our comprehensive study to understand global aesthetic preference reveals rich and complex cultural variations. Accordingly, it opens research avenues on the mechanisms underlying this variability, from demographic compositions~\cite{Lee2024-yo}, and emotional associations~\cite{Lee2021-zb}, to the influence of globalisation~\cite{pieterse_2025}. Such insights can have broad implications in cognitive and social sciences, psychology, and empirical aesthetics.


% RULES
% The entire content of a paper (including figures and anything else) can be no longer than six pages in the \textbf{initial submission}. In the \textbf{final submission}, the text of the paper, including an author line, must fit on six pages. An unlimimted number of pages can be used for acknowledgements and references.

% The text of the paper should be formatted in two columns with an
% overall width of 7 inches (17.8 cm) and length of 9.25 inches (23.5
% cm), with 0.25 inches between the columns. Leave two line spaces
% between the last author listed and the text of the paper; the text of
% the paper (starting with the abstract) should begin no less than 2.75 inches below the top of the
% page. The left margin should be 0.75 inches and the top margin should
% be 1 inch.  \textbf{The right and bottom margins will depend on
%   whether you use U.S. letter or A4 paper, so you must be sure to
%   measure the width of the printed text.} Use 10~point Times Roman
% with 12~point vertical spacing, unless otherwise specified.

% The title should be in 14~point bold font, centered. The title should
% be formatted with initial caps (the first letter of content words
% capitalized and the rest lower case). In the initial submission, the
% phrase ``Anonymous CogSci submission'' should appear below the title,
% centered, in 11~point bold font.  In the final submission, each
% author's name should appear on a separate line, 11~point bold, and
% centered, with the author's email address in parentheses. Under each
% author's name list the author's affiliation and postal address in
% ordinary 10~point type.

% Indent the first line of each paragraph by 1/8~inch (except for the
% first paragraph of a new section). Do not add extra vertical space
% between paragraphs.

% Use standard APA citation format. Citations within the text should
% include the author's last name and year. If the authors' names are
% included in the sentence, place only the year in parentheses, as in
% \citeA{NewellSimon1972a}, but otherwise place the entire reference in
% parentheses with the authors and year separated by a comma
% \cite{NewellSimon1972a}. List multiple references alphabetically and
% separate them by semicolons
% \cite{ChalnickBillman1988a,NewellSimon1972a}. Use the
% ``et~al.'' construction only after listing all the authors to a
% publication in an earlier reference and for citations with four or
% more authors.

% \subsection{Footnotes}

% Indicate footnotes with a number\footnote{Sample of the first
% footnote.} in the text. Place the footnotes in 9~point font at the
% bottom of the column on which they appear. Precede the footnote block
% with a horizontal rule.\footnote{Sample of the second footnote.}


% \subsection{Tables}

% Number tables consecutively. Place the table number and title (in
% 10~point) above the table with one line space above the caption and
% one line space below it, as in Table~\ref{sample-table}. You may float
% tables to the top or bottom of a column, and you may set wide tables across
% both columns.

% \begin{table}[H]
% \begin{center} 
% \caption{Sample table title.} 
% \label{sample-table} 
% \vskip 0.12in
% \begin{tabular}{ll} 
% \hline
% Error type    &  Example \\
% \hline
% Take smaller        &   63 - 44 = 21 \\
% Always borrow~~~~   &   96 - 42 = 34 \\
% 0 - N = N           &   70 - 47 = 37 \\
% 0 - N = 0           &   70 - 47 = 30 \\
% \hline
% \end{tabular} 
% \end{center} 
% \end{table}

% \subsection{Figures}

% All artwork must be very dark for purposes of reproduction and should
% not be hand drawn. Number figures sequentially, placing the figure
% number and caption, in 10~point, after the figure with one line space
% above the caption and one line space below it, as in
% Figure~\ref{sample-figure}. If necessary, leave extra white space at
% the bottom of the page to avoid splitting the figure and figure
% caption. You may float figures to the top or bottom of a column, and
% you may set wide figures across both columns.

% \begin{figure}[H]
% \begin{center}
% \fbox{CoGNiTiVe ScIeNcE}
% \end{center}
% \caption{This is a figure.} 
% \label{sample-figure}
% \end{figure}


% \section{Acknowledgments}

% In the \textbf{initial submission}, please \textbf{do not include
%   acknowledgements}, to preserve anonymity.  In the \textbf{final submission},
% place acknowledgments (including funding information) in a section \textbf{at
% the end of the paper}.

% \section{References Instructions}

% Follow the APA Publication Manual for citation format, both within the
% text and in the reference list, with the following exceptions: (a) do
% not cite the page numbers of any book, including chapters in edited
% volumes; (b) use the same format for unpublished references as for
% published ones. Alphabetize references by the surnames of the authors,
% with single author entries preceding multiple author entries. Order
% references by the same authors by the year of publication, with the
% earliest first.

% Use a first level section heading, ``{\bf References}'', as shown
% below. Use a hanging indent style, with the first line of the
% reference flush against the left margin and subsequent lines indented
% by 1/8~inch. Below are example references for a conference paper, book
% chapter, journal article, dissertation, book, technical report, and
% edited volume, respectively.

% \nocite{ChalnickBillman1988a}
% \nocite{Feigenbaum1963a}
% \nocite{Hill1983a}
% \nocite{OhlssonLangley1985a}
% % \nocite{Lewis1978a}
% \nocite{Matlock2001}
% \nocite{NewellSimon1972a}
% \nocite{ShragerLangley1990a}


\bibliographystyle{apacite}

\setlength{\bibleftmargin}{.125in}
\setlength{\bibindent}{-\bibleftmargin}

\bibliography{CogSci_Template}


\end{document}
