%%%%%%%% ICML 2025 EXAMPLE LATEX SUBMISSION FILE %%%%%%%%%%%%%%%%%

\documentclass{article}

% Recommended, but optional, packages for figures and better typesetting:
\usepackage{microtype}
\usepackage{graphicx}
\usepackage{subfigure}
\usepackage{subcaption}
\usepackage{booktabs} % for professional tables
\usepackage{tabularx} %% long table
\usepackage{makecell}
\usepackage{tcolorbox}
\usepackage{paralist}

% hyperref makes hyperlinks in the resulting PDF.
% If your build breaks (sometimes temporarily if a hyperlink spans a page)
% please comment out the following usepackage line and replace
% \usepackage{icml2025} with \usepackage[nohyperref]{icml2025} above.
\usepackage{hyperref}


% Attempt to make hyperref and algorithmic work together better:
\newcommand{\theHalgorithm}{\arabic{algorithm}}

% Use the following line for the initial blind version submitted for review:
% \usepackage{icml2025}

% If accepted, instead use the following line for the camera-ready submission:
\usepackage[accepted]{icml2025}

% For theorems and such
\usepackage{amsmath}
\usepackage{amssymb}
\usepackage{mathtools}
\usepackage{amsthm}

%additional package(s)
\usepackage{graphicx}
\usepackage{booktabs}
\usepackage{multirow}
% \usepackage{subcaption}
\usepackage{float}

\newcommand{\jola}{\textsc{JoLA}}

% if you use cleveref..
\usepackage[capitalize,noabbrev]{cleveref}

%%%%%%%%%%%%%%%%%%%%%%%%%%%%%%%%
% THEOREMS
%%%%%%%%%%%%%%%%%%%%%%%%%%%%%%%%
\theoremstyle{plain}
\newtheorem{theorem}{Theorem}[section]
\newtheorem{proposition}[theorem]{Proposition}
\newtheorem{lemma}[theorem]{Lemma}
\newtheorem{corollary}[theorem]{Corollary}
\theoremstyle{definition}
\newtheorem{definition}[theorem]{Definition}
\newtheorem{assumption}[theorem]{Assumption}
\theoremstyle{remark}
\newtheorem{remark}[theorem]{Remark}

% Todonotes is useful during development; simply uncomment the next line
%    and comment out the line below the next line to turn off comments
%\usepackage[disable,textsize=tiny]{todonotes}
\usepackage[textsize=tiny]{todonotes}


% The \icmltitle you define below is probably too long as a header.
% Therefore, a short form for the running title is supplied here:
\icmltitlerunning{Joint Localization and Activation Editing for Low-Resource Fine-Tuning}

\begin{document}

\twocolumn[
% \icmltitle{Robust Representation Tuning with Dynamic Attention Head Pruning}
% \icmltitle{Robust Representation Tuning in Low Data Scenarios with Gated Dynamic Attention Head Pruning}
\icmltitle{Joint Localization and Activation Editing for Low-Resource Fine-Tuning}

% It is OKAY to include author information, even for blind
% submissions: the style file will automatically remove it for you
% unless you've provided the [accepted] option to the icml2025
% package.

% List of affiliations: The first argument should be a (short)
% identifier you will use later to specify author affiliations
% Academic affiliations should list Department, University, City, Region, Country
% Industry affiliations should list Company, City, Region, Country

% You can specify symbols, otherwise they are numbered in order.
% Ideally, you should not use this facility. Affiliations will be numbered
% in order of appearance and this is the preferred way.
\icmlsetsymbol{equal}{*}

\begin{icmlauthorlist}
\icmlauthor{Wen Lai}{tum,mcml}
\icmlauthor{Alexander Fraser}{tum,mcml}
\icmlauthor{Ivan Titov}{uoe,uva}
\end{icmlauthorlist}

\icmlaffiliation{tum}{Technical University of Munich}
\icmlaffiliation{mcml}{Munich Center for Machine Learning}
\icmlaffiliation{uoe}{University of Edinburgh}
\icmlaffiliation{uva}{University of Amsterdam}

\icmlcorrespondingauthor{Wen Lai}{wen.lai@tum.de}
% \icmlcorrespondingauthor{Ivan Titov}{ititov@inf.ed.ac.uk}
% \icmlcorrespondingauthor{Alexander Fraser}{alexander.fraser@tum.de}

% You may provide any keywords that you
% find helpful for describing your paper; these are used to populate
% the "keywords" metadata in the PDF but will not be shown in the document
\icmlkeywords{Machine Learning, ICML}

\vskip 0.3in
]

% this must go after the closing bracket ] following \twocolumn[ ...

% This command actually creates the footnote in the first column
% listing the affiliations and the copyright notice.
% The command takes one argument, which is text to display at the start of the footnote.
% The \icmlEqualContribution command is standard text for equal contribution.
% Remove it (just {}) if you do not need this facility.

\printAffiliationsAndNotice{}  % leave blank if no need to mention equal contribution
% \printAffiliationsAndNotice{\icmlEqualContribution} % otherwise use the standard text.

\begin{abstract}
Parameter-efficient fine-tuning (PEFT) methods, such as LoRA, are commonly used to adapt LLMs.
However, the effectiveness of standard PEFT methods is limited in low-resource scenarios with only a few hundred examples.
Recent advances in interpretability research have inspired the emergence of activation editing techniques, which modify the activations of specific model components.
These methods, due to their extremely small parameter counts, show promise for small datasets.
However, their performance is highly dependent on identifying the correct modules to edit and often lacks stability across different datasets.
In this paper, we propose Joint Localization and Activation Editing (\jola{}), a method that jointly learns 
(1) which heads in the Transformer to edit
(2) whether the intervention should be additive, multiplicative, or both and 
(3) the intervention parameters themselves - the vectors applied as additive offsets or multiplicative scalings to the head output.
Through evaluations on three benchmarks spanning commonsense reasoning, natural language understanding, and natural language generation, we demonstrate that \jola{} consistently outperforms existing methods.\footnote{The code for the method is released at \url{https://github.com/wenlai-lavine/jola}.}

\end{abstract}

\section{Introduction}
\label{sec:intro}

Foundational models (FMs)~\cite{zhang2024data, zhou2023comprehensive} have shown remarkable progress in the healthcare domain, enabling professional-like assessment of disease diagnosis, treatment decision-making, and monitoring~\cite{zhang2023text, wang2022medclip, lu2023mi-zero}. 
Examples include LLaVA-Med~\cite{li2023llava}, Med-PaLM Multimodal~\cite{tu2024towards}, and Med-Flamingo~\cite{moor2023med}, have demonstrated their capacity on question answering, medical image analysis, and report generation.
These studies follow a predominant top-down model development strategy that requires upstream developers to collect data and train models for downstream tasks. 
Consequently, the developed model capabilities are heavily dependent on the training data, limiting their generalization performance in diverse clinical scenarios. 
For instance, Med-Gemini~\cite{yang2024advancing} reveals promising general capabilities in report generation while it lags behind state-of-the-art (SoTA) models on classification tasks, especially for out-of-domain applications. 
This indicates that while the generalizability of the foundation model is promising, more solutions are expected to meet the various specialized clinical needs.

To address these challenges, multi-center data centralization becomes essential to enhance model capacity and robustness across varied clinical scenarios~\cite{rajpurkar2022ai}. 
Centralizing distributed data can significantly improve model training and inference performance.
However, the process of medical data storage, transfer, and aggregation among centers requires extra efforts to ensure data security and system interoperability~\cite{bradford2020international}.
Moreover, a growing concern for patient privacy makes large-scale multi-center data sharing particularly challenging. 
While efforts like federated learning~\cite{wen2023survey, li2020review} can achieve good model performance on local data, the need for synchronized system coordination presents significant challenges, as clients are unable to update asynchronously. This limitation greatly restricts the practical capability of such approaches.
As a result, without a flexible collaboration, medical community still struggles to fully utilize the isolated data and local computation resources for comprehensive medical AI model development. 
To address this dilemma, open-source platforms encourage public data sharing and knowledge integration~\cite{markiewicz2021openneuro, zenodo}.
However, these platforms focus solely on raw data sharing while seldom providing collaborative model training or cooperation between different institutions.
Recently, collaborative learning has emerged as a viable approach for enhancing multi-model robustness~\cite{boulemtafes2020review}. 
For instance, software-like model development~\cite{raffel2023building} mimics software engineering practices by introducing structured workflows, enabling merging, version control, and continuous model integration.
Under this design, model ability can be strengthened with incremental knowledge updates similar to the version updating in software development. 

Although collaborative learning provides a multi-model collaboration, two key challenges remain in the leakage of raw data during collaboration~\cite{huang2023lorahub} and the synchronization of multiple collaborators~\cite{mcmahan2017communication} in the medical AI community. It is still challenging to integrate decentralized, privacy-sensitive data across institutions, leading to under-utilized insights and fragmented knowledge sharing~\cite{kaissis2020secure, rajpurkar2022ai, abdullah2021ethics}.
 To address these challenges, inspired by the collaborative software development, we propose \textbf{Med}ical \textbf{Fo}undation Models Me\textbf{rg}ing (\textbf{MedForge}), a cooperative workflow enabling continuously community-driven foundation model (FM) development.
MedForge enables a lightweight manner for individual centers to share their knowledge among multiple centers, minimizing the burden of data transmission and integration while enhancing model robustness.
Meanwhile, MedForge facilitates asynchronous and flexible collaboration, allowing individual centers to continuously update and improve medical FMs without the need for real-time synchronization.
Similar to open-source software development, MedForge incrementally updates medical knowledge and follows a sustainable model development scheme. 
This key design emphasizes a bottom-up construction of a multi-task medical FM, allowing downstream users to collaboratively build, refine, and update the upstream model according to their local resources. Our major contributions of MedForge are as below: 
\begin{enumerate}
    \item[$\bullet$] We introduce a collaborative workflow to promote the merging scheme of open-source software development. Our proposed MedForge allows distributed clinical centers to asynchronously contribute to comprehensive medical model construction while reducing transmitting costs among centers and avoiding the leakage of raw data, thus enhancing the utilization of private resources in the healthcare system. 
    \item[$\bullet$] We propose two effective knowledge-merging strategies for the asynchronous branch contribution. The MedForge-Fusion strategy updates the plugin module parameters of the main model during the merging phase, whereas the MedForge-Mixture strategy integrates the output of the plugin module by memorizing each contributor's coefficient. These strategies make MedForge more flexible and versatile. MedForge-Fusion is friendly to implement, while the MedForge-Mixture offers better performance and robustness.
    \item[$\bullet$]  We comprehensively evaluate model merging strategies to accumulate medical knowledge among multiple branch plugin modules. MedForge yields superior performance on medical classification tasks compared to other collaborative baselines across multiple datasets. We demonstrate the robustness of MedForge by shuffling the task order and evaluating various configurations of plugin modules and dataset distillation methods.
\end{enumerate}



\section{Background}
\label{sec:background}
Activation editing in LLMs modifies intermediate activation outputs to steer model behavior.
We categorize existing approaches into three types based on the transformation function applied to activations.
Given an activation output $z_t^{(l,i)} \in \mathbb{R}^{d_l}$ for $i$th component at layer $l$, the general transformation is:
\begin{equation}
    z_t^{(l,i)'} = f(z_t^{(l,i)}),
\end{equation}
where $f(\cdot)$ determines the intervention type:
\begin{compactitem}
\item \textbf{Additive methods} apply a bias vector $a^{i}_{l} \in \mathbb{R}^{d_l}$ : $z_t^{(l,i)'} = z_t^{(l,i)} + a^{(l,i)}$.
\item \textbf{Multiplicative methods} scale activations as $z_t^{(l,i)'} = m^{(l,i)} \odot z_t^{(l,i)}$, where $m^{(l,i)} \in \mathbb{R}^{d_l}$ and $\odot$ is an element-wise product.
\item \textbf{Hybrid methods} combine both transformations: $z_t^{(l,i)'} = m^{(l,i)} \odot z_t^{(l,i)} + a^{(l,i)}$.
\end{compactitem}

Existing methods follow these paradigms but often rely on fixed selections of components for modification, limiting adaptability.
For example, BitFit~\cite{ben-zaken-etal-2022-bitfit} updates bias terms, while RED~\cite{wu-etal-2024-advancing} employs per-dimension scaling vectors and bias vectors. ReFT~\cite{wu2024reft} applies fine-tuned low-rank hidden states with MLP layers, and LoFIT~\cite{yin2024lofit} intervenes in selected attention heads with additive bias vectors but requires manual selection.
\vspace{-5pt}
\section{Method}
\label{sec:method}
\begin{figure*}[t]
\begin{center}
\includegraphics[width=.85\linewidth]{fig_overview_v3.pdf}
\end{center}
\caption{
FastAtlas Overview: In each frame, we compute charts spanning fully or partially visible triangles (a), determine texture space bounding boxes for the visible portions of the view-space projections of each chart, and tightly pack these boxes into atlases (b, here $2K \times 2K$). We simultaneously bijectively parameterize and shade the charts into their atlas boxes, obtaining high quality texture space shading (c), and use this shading to render the shaded frames (d).}
\label{fig:overview}
\label{fig:alg_overview}
\end{figure*}

\section{Overview}
\label{sec:overview}
Our work has two core contributions: a real-time, GPU-based algorithm for tight packing of general parameterized charts into compact atlases; and a real-time TSS method that
utilizes this packing.  

\paragraph*{FastAtlas Packing.}
FastAtlas runs entirely on the GPU as a series of compute shaders. It takes the bounding boxes of parameterized charts as input, and packs them into an atlas (Fig~\ref{fig:overview}b, Sec.~\ref{sec:pack}). As such, the only input it requires are the dimensions of the bounding boxes.
Its outputs are deterministic; identical input charts are packed into identical atlases. This is critical for TSS and similar applications, as it ensures that consecutive frames taken from the same camera view have the same shading. Even minute shading differences across such frames can cause sampling jitter, leading to undesirable flicker \cite{baker2012rock}. 
While prior methods such as \cite{mueller2018shading,hladky2019tessellated,hladky2021snakebinning,Neff2022MSA} cap the dimensions of the charts that can be packed as-is for a given atlas size, and scale down all charts that exceed these dimensions, we scale all charts by the same factor, and do so only when strictly necessary to achieve packing success (Figs~\ref{fig:atlas},~\ref{fig:sas_issues}). 

\paragraph*{TSS using FastAtlas.}
Our end-to-end TSS atlas generation method combines the packing method above with a novel approach for computing seamless per-frame charts. 
We define our charts as the connected components of the visible surfaces in each frame (Fig.~\ref{fig:overview}a), and efficiently compute them using a parallel union-find algorithm (Sec.~\ref{sec:visible}). Since the boundaries of these charts coincide with the contours of the rendered surface, they are {\em invisible} to the viewer. This approach 
eliminates the artifacts caused by shading discontinuities along visible seams (Fig.~\ref{fig:seams}). 

\begin{parWithWrapFigure}
\begin{wrapfigure}{l}{.27\columnwidth}%
\includegraphics[width=\linewidth]{fig_inset_view_plane.pdf}%
\end{wrapfigure}
We bijectively parametrize the {\em visible portions} of our charts by projecting them to view space (inset). This maps a constant number of texels to each pixel in the final rendered output, evenly distributing residual undersampling error across all image pixels. While conceptually straightforward, efficiently parameterizing charts containing partially visible triangles using viewspace projection is non-trivial, as the visible portions may no longer be triangular (e.g. green triangle in the inset); applying naive projection to triangles with vertices behind the camera may produce ill-posed results. Clipping triangles before projection is both computationally expensive and significantly complicates downstream operations. We avoid explicit clipping by observing that all that is required for atlas packing is the dimensions of, potentially conservative, bounding boxes of these projected visible portions. We compute such bounding boxes without explicit chart clipping by adapting a conservative screen coverage estimator \shortcite{Blinn:CalculatingScreenCoverage} (Sec.~\ref{sec:box}). We then pack the computed boxes using FastAtlas. 
\end{parWithWrapFigure}

Finally, we shade the visible portion of each chart into its corresponding atlas bounding box (Fig~\ref{fig:overview}c). 
The resulting texture is then used during rasterization as a standard texture map (Fig. ~\ref{fig:overview}d). 
Our framework is compatible with all existing approaches for texture space shading, including forward shading, raytraced illumination, or deferred shading in texture space \cite{baker:2016}. In the examples shown, we use the standard forward shading based rendering pipeline included in the G3D Innovation Engine \cite{G3D17}, a commercial grade renderer.


Our goal is to increase the robustness of T2I models, particularly with rare or unseen concepts, which they struggle to generate. To do so, we investigate a retrieval-augmented generation approach, through which we dynamically select images that can provide the model with missing visual cues. Importantly, we focus on models that were not trained for RAG, and show that existing image conditioning tools can be leveraged to support RAG post-hoc.
As depicted in \cref{fig:overview}, given a text prompt and a T2I generative model, we start by generating an image with the given prompt. Then, we query a VLM with the image, and ask it to decide if the image matches the prompt. If it does not, we aim to retrieve images representing the concepts that are missing from the image, and provide them as additional context to the model to guide it toward better alignment with the prompt.
In the following sections, we describe our method by answering key questions:
(1) How do we know which images to retrieve? 
(2) How can we retrieve the required images? 
and (3) How can we use the retrieved images for unknown concept generation?
By answering these questions, we achieve our goal of generating new concepts that the model struggles to generate on its own.

\vspace{-3pt}
\subsection{Which images to retrieve?}
The amount of images we can pass to a model is limited, hence we need to decide which images to pass as references to guide the generation of a base model. As T2I models are already capable of generating many concepts successfully, an efficient strategy would be passing only concepts they struggle to generate as references, and not all the concepts in a prompt.
To find the challenging concepts,
we utilize a VLM and apply a step-by-step method, as depicted in the bottom part of \cref{fig:overview}. First, we generate an initial image with a T2I model. Then, we provide the VLM with the initial prompt and image, and ask it if they match. If not, we ask the VLM to identify missing concepts and
focus on content and style, since these are easy to convey through visual cues.
As demonstrated in \cref{tab:ablations}, empirical experiments show that image retrieval from detailed image captions yields better results than retrieval from brief, generic concept descriptions.
Therefore, after identifying the missing concepts, we ask the VLM to suggest detailed image captions for images that describe each of the concepts. 

\vspace{-4pt}
\subsubsection{Error Handling}
\label{subsec:err_hand}

The VLM may sometimes fail to identify the missing concepts in an image, and will respond that it is ``unable to respond''. In these rare cases, we allow up to 3 query repetitions, while increasing the query temperature in each repetition. Increasing the temperature allows for more diverse responses by encouraging the model to sample less probable words.
In most cases, using our suggested step-by-step method yields better results than retrieving images directly from the given prompt (see 
\cref{subsec:ablations}).
However, if the VLM still fails to identify the missing concepts after multiple attempts, we fall back to retrieving images directly from the prompt, as it usually means the VLM does not know what is the meaning of the prompt.

The used prompts can be found in \cref{app:prompts}.
Next, we turn to retrieve images based on the acquired image captions.

\vspace{-3pt}
\subsection{How to retrieve the required images?}

Given $n$ image captions, our goal is to retrieve the images that are most similar to these captions from a dataset. 
To retrieve images matching a given image caption, we compare the caption to all the images in the dataset using a text-image similarity metric and retrieve the top $k$ most similar images.
Text-to-image retrieval is an active research field~\cite{radford2021learning, zhai2023sigmoid, ray2024cola, vendrowinquire}, where no single method is perfect.
Retrieval is especially hard when the dataset does not contain an exact match to the query \cite{biswas2024efficient} or when the task is fine-grained retrieval, that depends on subtle details~\cite{wei2022fine}.
Hence, a common retrieval workflow is to first retrieve image candidates using pre-computed embeddings, and then re-rank the retrieved candidates using a different, often more expensive but accurate, method \cite{vendrowinquire}.
Following this workflow, we experimented with cosine similarity over different embeddings, and with multiple re-ranking methods of reference candidates.
Although re-ranking sometimes yields better results compared to simply using cosine similarity between CLIP~\cite{radford2021learning} embeddings, the difference was not significant in most of our experiments. Therefore, for simplicity, we use cosine similarity between CLIP embeddings as our similarity metric (see \cref{tab:sim_metrics}, \cref{subsec:ablations} for more details about our experiments with different similarity metrics).

\vspace{-3pt}
\subsection{How to use the retrieved images?}
Putting it all together, after retrieving relevant images, all that is left to do is to use them as context so they are beneficial for the model.
We experimented with two types of models; models that are trained to receive images as input in addition to text and have ICL capabilities (e.g., OmniGen~\cite{xiao2024omnigen}), and T2I models augmented with an image encoder in post-training (e.g., SDXL~\cite{podellsdxl} with IP-adapter~\cite{ye2023ip}).
As the first model type has ICL capabilities, we can supply the retrieved images as examples that it can learn from, by adjusting the original prompt.
Although the second model type lacks true ICL capabilities, it offers image-based control functionalities, which we can leverage for applying RAG over it with our method.
Hence, for both model types, we augment the input prompt to contain a reference of the retrieved images as examples.
Formally, given a prompt $p$, $n$ concepts, and $k$ compatible images for each concept, we use the following template to create a new prompt:
``According to these examples of 
$\mathord{<}c_1\mathord{>:<}img_{1,1}\mathord{>}, ... , \mathord{<}img_{1,k}\mathord{>}, ... , \mathord{<}c_n\mathord{>:<}img_{n,1}\mathord{>}, ... , $
$\mathord{<}img_{n,k}\mathord{>}$,
generate $\mathord{<}p\mathord{>}$'', 
where $c_i$ for $i\in{[1,n]}$ is a compatible image caption of the image $\mathord{<}img_{i,j}\mathord{>},  j\in{[1,k]}$. 

This prompt allows models to learn missing concepts from the images, guiding them to generate the required result. 

\textbf{Personalized Generation}: 
For models that support multiple input images, we can apply our method for personalized generation as well, to generate rare concept combinations with personal concepts. In this case, we use one image for personal content, and 1+ other reference images for missing concepts. For example, given an image of a specific cat, we can generate diverse images of it, ranging from a mug featuring the cat to a lego of it or atypical situations like the cat writing code or teaching a classroom of dogs (\cref{fig:personalization}).
\vspace{-2pt}
\begin{figure}[htp]
  \centering
   \includegraphics[width=\linewidth]{Assets/personalization.pdf}
   \caption{\textbf{Personalized generation example.}
   \emph{ImageRAG} can work in parallel with personalization methods and enhance their capabilities. For example, although OmniGen can generate images of a subject based on an image, it struggles to generate some concepts. Using references retrieved by our method, it can generate the required result.
}
   \label{fig:personalization}\vspace{-10pt}
\end{figure}
\begin{table*}[t!]
\centering
% \vspace{5pt}
\begin{small}
\begin{tabular}{l|c|c|c|c|c|c|c}
\toprule
\textbf{Method} & \textbf{Type} & \textbf{ToMi} & \textbf{BigToM} & \textbf{MMToM-QA} & \textbf{MuMA-ToM} & \textbf{Hi-ToM} & \textbf{All} \\
\midrule
SymbolicToM & Specific & \textbf{98.60} & - &  - & - & - & - \\
TimeToM & Specific & 87.80 & - &   - & - & - & - \\
% \textbf{96.00$^*$}
PercepToM & Specific & 82.90 & - & - & - & - & - \\
BIP-ALM & Specific & - & - & 76.70 & 33.90 & - & - \\
LIMP & Specific & - & - & - & 76.60 & - & - \\
\ours w/ Model Spec. & Specific & 88.80 & \textbf{86.75} & \textbf{79.83} & \textbf{84.00} & \textbf{74.00} & \textbf{82.68} \\
\midrule
Llama 3.1 70B & General & 72.00 & 77.83 & 43.83 & 55.78 & 35.00 & 47.41 \\
Gemini 2.0 Flash & General & 66.70 & 82.00 & 48.00 & 55.33 & 52.50 & 60.91\\
Gemini 2.0 Pro & General & 71.90 & 86.33 & 50.84 &  62.22 & 57.50 & 65.76 \\ 
GPT-4o & General & 77.00 & 82.42 & 44.00 & 63.55 & 50.00 & 63.39 \\
SimToM & General & 79.90 & 77.50 & 51.00 & 47.63 & 71.00 & 65.41\\ 
\ours & General & \textbf{88.30} & \textbf{86.92} & \textbf{75.50} & \textbf{81.44} & \textbf{72.50} & \textbf{80.93} \\
\bottomrule
\end{tabular}
\end{small}
\caption{Results of \ours and baselines on all benchmarks. There are two groups of methods: methods that require domain-specific knowledge (e.g., AutoToM w/ Model Spec.) or implementations (e.g., SymbolicToM) and methods that can be generally applied to any domain. ``-'' indicates that the domain-specific method is not applicable to the benchmark. The best results for each method type are highlighted in bold.}
\label{tab:results}
\vspace{-10pt}
\end{table*}



\section{Experiments}
\subsection{Experimental Settings}



We evaluated our method on multiple Theory of Mind benchmarks, including ToMi \citep{le2019revisiting}, BigToM \citep{gandhi2024understanding}, MMToM-QA \cite{jin2024mmtom}, MuMA-ToM \citep{shi2024muma}, and Hi-ToM \cite{he2023hi}. The diversity and complexity of these benchmarks pose significant reasoning challenges. For instance, MMToM-QA and MuMA-ToM incorporate both visual and textual input, while MuMA-ToM and Hi-ToM require higher-order inference. Additionally, MMToM-QA features exceptionally long contexts, and BigToM presents open-ended scenarios.



Besides the full \ours method, we additionally evaluated \ours given manually specified models (AutoToM w/ Model Spec.). 

We compared \ours against state-of-the-art baselines:
    \textbf{LLMs:} Llama 3.1 70B \citep{dubey2024llama}, Gemini 2.0 Flash, Gemini 2.0 Pro \cite{team2023gemini} and GPT-4o \cite{achiam2023gpt};
    
     \textbf{ToM prompting for LLMs:} SymbolicToM \cite{sclar2023minding}, SimToM \cite{wilf2023think}, TimeToM \cite{hou2024timetom}, and PercepToM \citep{jung2024perceptions};
 
  \textbf{Model-based inference:} BIP-ALM \cite{jin2024mmtom} and LIMP \cite{shi2024muma}.


For multimodal benchmarks, MMToM-QA and MuMA-ToM, we adopt the information fusion methods proposed by \citet{jin2024mmtom} and \citet{shi2024muma} to fuse information from visual and text inputs respectively. The fused information is in text form. We ensure that all methods use the same fused information as their input.


We use GPT-4o as the LLM backend for \ours and all ToM prompting and model-based inference baselines to ensure a fair comparison—except for TimeToM, which relies on GPT-4 and is not open-sourced.


\subsection{Results}
The main results are summarized in Table~\ref{tab:results}. Unlike \ours, many recent ToM baselines can only be applied to specific benchmarks. Among general methods, \ours achieves state-of-the-art results across all benchmarks. In particular, it outperforms its LLM backend, GPT-4o, by a large margin. This is because Bayesian inverse planning is more robust for inferring mental states given long contexts with complex environments and agent behavior. It is also more adept at recursive reasoning which is key to higher-order inference. Notably, \ours performs comparably to manually specified models, showing that automatic model discovery without domain knowledge is as effective as human-provided models. We provide additional results and qualitative examples in Appendix~\ref{sec:more_results}.


\subsection{Ablated Study}



\begin{figure}[t!]
  \centering
  \includegraphics[width=0.8\linewidth]{figures/comparison.pdf}
    \vspace{-10pt}
  \caption{Averaged performance and compute of the full \ours method (star) and the ablated methods (circles) on all benchmarks.}
  \label{fig:ablation}
  \vspace{-10pt}
\end{figure}


We evaluated the following variants of \ours for an ablation study: no hypothesis reduction (\textbf{w/o hypo. reduction}); always using POMDP (\textbf{w/ POMDP}); always using the initial model proposal without variable adjustment (\textbf{w/o variable adj.}); only considering the last timestep (\textbf{w/ last timestep}); and considering all timesteps without timestep adjustment (\textbf{w/ all timesteps}).

The results in Figure~\ref{fig:ablation} show that the full \ours method constructs a suitable BToM model, enabling rich ToM inferences while reducing compute. We analyze key model components below:

\textbf{Hypothesis reduction.}
Compared to the full method, \ours w/o hypo. reduction has a similar accuracy but consumes 53\% more tokens on average, demonstrating that hypothesis reduction optimizes efficiency without sacrificing performance.

\textbf{Variable adjustment.}
\ours dynamically identifies relevant variables for ToM inference, generalizing domain-specific BIP approaches to open-ended scenarios. Compared to its variant without variable adjustment, \ours improves performance with minimal additional compute. The variant that always uses POMDP performs well in scenarios aligned with the POMDP assumption (e.g., MMToM-QA) but generalizes poorly elsewhere and incurs much higher computational costs. %, leading to an 8.5% performance deficit.

\textbf{Timestep adjustment.}
By selecting relevant steps for inference, timestep adjustment enhances performance by focusing on essential information. In contrast, the variant using only the last timestep misses crucial details, significantly lowering performance. The variant incorporating all timesteps suffers from higher computational costs and reduced accuracy due to conditioning on unnecessary, potentially distracting information.



Full ablation results are provided in Appendix~\ref{sec:more_results_ablation}.

\section{Results}
\label{sec:res}
This section evaluates the performance of our proposed method, \jola{}, in comparison with various baselines.
Table~\ref{tab:main_res} presents the average performance for all methods across the three tasks, while Figure~\ref{fig:main_res} illustrates the results for individual subtasks.
More detailed numerical results can be found in Appendix~\ref{appendix:full_res}.

%% main res table
\definecolor{myblue}{RGB}{120,145,181}


\begin{table*}[]
\centering
\small
\caption{Performance comparison across different generators and benchmarks. We evaluate different configurations, with critique-revision representing an iterative process where a critic model provides feedback to guide solution improvement. Pass@1 shows the success rate, while $\Delta_\uparrow$ and $\Delta_\downarrow$ indicate the percentage of wrong solutions being correctly revised and correct solutions being revised to wrong solutions, respectively.
Results are averaged over 5 random seeds.}
\label{tab:main}
\vspace{3mm}



\begin{tabular}{lcccccccccc}
\toprule
\multirow{2}{*}{} & \multicolumn{3}{c}{\textbf{CodeContests}} & \multicolumn{3}{c}{\textbf{LiveCodeBench}} & \multicolumn{3}{c}{\textbf{MBPP+}} & \textbf{Average} \\
 & \multicolumn{1}{c}{\textbf{Pass@1}} & \multicolumn{1}{c}{\textbf{$\Delta_\uparrow$}} & \multicolumn{1}{c}{\textbf{$\Delta_\downarrow$}} & \multicolumn{1}{c}{\textbf{Pass@1}} & \multicolumn{1}{c}{\textbf{$\Delta_\uparrow$}} & \multicolumn{1}{c}{\textbf{$\Delta_\downarrow$}} & \multicolumn{1}{c}{\textbf{Pass@1}} & \multicolumn{1}{c}{\textbf{$\Delta_\uparrow$}} & \multicolumn{1}{c}{\textbf{$\Delta_\downarrow$}} & \textbf{Pass@1} \\ \midrule
\rowcolor{gray!10} \multicolumn{11}{c}{\textit{Qwen2.5-Coder as Generator}} \\
Zero-shot & 7.88 & 0.00 & 0.00 & 30.54 & 0.00 & 0.00 & 77.83 & 0.00 & 0.00  & 38.75 \\
\emph{Single-turn Critique-revision} \\
Critique w/ Qwen2.5-Coder & 8.36 & 2.30 & 1.82 & 32.14 & 2.50 & 0.89 & 77.83 & 3.49 & 3.49 & 39.45 \\
Critique w/ GPT-4o & 10.67 & \textbf{4.85} & 2.06 & 32.32 & 2.32 & \textbf{0.54} & 77.46 & \textbf{3.81} & 4.18 & 40.15 \\
\rowcolor{myblue!20} Critique w/ {\ours} & \textbf{11.76} & 4.73 & \textbf{0.85} & \textbf{33.21} & \textbf{3.39} & 0.71 & \textbf{78.84} & 2.43 & \textbf{1.43} & \textbf{41.27} \\
\emph{Multi-turn Critique-revision} \\
Critique$\times 5$ w/ Qwen2.5-Coder & 9.21 & 3.76 & 2.42 & 29.64 & 2.14 & 3.04 & 76.03 & 3.81 & 5.61 & 38.30 \\
Critique$\times 5$ w/ GPT-4o & 12.48 & 7.03 & 2.42 & 32.86 & \textbf{4.82} & 2.50 & 74.60 & \textbf{4.34} & 	\textbf{7.57} &	39.98 \\
\rowcolor{myblue!20} Critique$\times 5$ w/ {\ours} & \textbf{16.24} & \textbf{9.21} & \textbf{0.85} & \textbf{33.39} & 3.75 & \textbf{0.89} & \textbf{78.68} & 3.23 & 2.38 & \textbf{42.77} \\
\midrule
\rowcolor{gray!10} \multicolumn{11}{c}{\textit{GPT-4o as Generator}} \\
Zero-shot & 20.61 & 0.00 & 0.00 & 32.32 & 0.00 & 0.00 & 77.67 & 0.00 & 0.00 & 43.53 \\
\emph{Single-turn Critique-revision} \\
Critique w/ Qwen2.5-Coder & 20.24 & 3.52 & 3.88 & \textbf{35.36} & \textbf{3.93} & 0.89 & 76.67 & 0.85 & 1.85 & 44.09 \\
Critique w/ GPT-4o & 20.97 & 2.30 & \textbf{1.94} & 34.82 & 2.68 & \textbf{0.18} & 77.41 & \textbf{1.01} & 1.27 & 44.40 \\
\rowcolor{myblue!20} Critique w/ {\ours} & \textbf{23.03} & \textbf{4.97} & 2.55 & 33.39 & 2.14 & 1.07 & \textbf{77.83} & 0.53 & \textbf{0.37} & \textbf{44.75} \\
\emph{Multi-turn Critique-revision} \\
Critique$\times 5$ w/ Qwen2.5-Coder & 19.52 & 5.21 & 6.30 & \textbf{35.54} & \textbf{5.36} & 2.14 & 76.08 & 1.53 & 3.12 & 43.71 \\
Critique$\times 5$ w/ GPT-4o & 20.61 & 3.39 & 3.39 & 35.18 & 3.21 & \textbf{0.36} & 76.61 & \textbf{2.06} & 3.12 & 44.13 \\
\rowcolor{myblue!20} Critique$\times 5$ w/ {\ours} & \textbf{25.45} & \textbf{7.88} & \textbf{3.03} & 34.11 & 3.21 & 1.43 &  \textbf{77.94} & 0.79 & \textbf{0.53} & \textbf{45.83} \\ \bottomrule
\end{tabular}







\end{table*}

%%

\noindent\textbf{Performance of Activation-Based Baselines.}
Activation editing baselines exhibit varying levels of success across tasks, but their sensitivity to hyperparameter selection and layer intervention limits their consistency.
For example, BitFit~\cite{ben-zaken-etal-2022-bitfit} demonstrates notable sensitivity to the placement of bias terms within the model.
Adjusting bias terms in dropout layers or attention mechanisms results in performance fluctuations, particularly in low-data scenarios.
Similarly, RED~\cite{wu-etal-2024-advancing} depends on the specific positions where scaling and bias vectors are introduced, leading to inconsistent results.
REPE~\cite{zou2023representation} is highly sensitive to the quality of activation representations across tasks, making it challenging to generalize its performance.
ReFT~\cite{wu2024reft} achieves moderate success by intervening on selected layers but faces challenges in determining the optimal number and choice of layers.
LoFIT~\cite{yin2024lofit}, while effective in leveraging task-relevant attention heads, struggles to maintain consistency across tasks.

\noindent\textbf{Performance of LoRA.}
LoRA achieves noticeable improvements over zero-shot baselines and, somewhat surprisingly, outperforms previous activation editing methods across all tasks when its rank hyperparameter is appropriately tuned. 
In tasks such as natural language generation, LoRA achieves higher BLEU and ROUGE-L scores, highlighting its ability to generate coherent outputs.

\noindent\textbf{Performance of \jola{}}
Our proposed method, \jola{}, consistently outperforms all baselines across the three tasks by a significant margin.
This can be attributed to \jola{}'s dynamic gated attention mechanism, which allows for adaptive activation of attention heads.
Unlike LoFIT~\cite{yin2024lofit}, which requires manual selection of attention heads, \jola{}'s mechanism enables less relevant heads to gradually ``die off" during training, improving robustness and adaptability.
In commonsense reasoning, \jola{} achieves an average improvement of 3.97\% over the best-performing baseline (LoRA) in LLaMA-3, as shown in Table~\ref{tab:main_res}.
For natural language understanding, \jola{} demonstrates consistent performance across diverse domains in the MMLU-Pro benchmark~\cite{wang2024mmlu} across all 14 subtasks as illustrated in Figure~\ref{fig:main_res}.
In natural language generation tasks, \jola{} achieves higher BLEU, ROUGE-L and BERTScore scores compared to activation-based baselines and LoRA.
To further understand the strengths and limitations of our method, we present a qualitative comparison of follow-up questions generated by all models for the same (IQ, IA) pair (\S\ref{sec:analysis:casestudy}), as well as an analysis of the expected information gain from the generated follow-up questions (\S\ref{sec:analysis:infogain}). 

\subsection{Qualitative Analysis}
\label{sec:analysis:casestudy}

In Table~\ref{tab:example_question_long}, we compare follow-up questions generated by the \textit{ORG}, \textit{AUG}, and \textit{FULL} models for a given (IQ, IA) pair. The questions produced by the \textit{ORG} model are often either redundant, such as ``What is a heuristic?'', or tangential, like responding with ``How do you know if you can ELI5?'' to the original responder that mentioned they didn't know if they could explain it to a 5-year-old (ELI5)---diverging from the target concept of heuristics. While the \textit{FULL} model generates a variety of relevant questions, excelling in diversity, it sometimes includes tangential ones or redundant phrasing, such as ``How do heuristics help in segmenting and segmenting information for specific tasks?'', which can affect clarity. In contrast, the \textit{AUG} model offers the best balance of informativeness and diversity, producing focused and insightful questions, such as ``What are some examples of cognitive strategies that rely on heuristics?'' and ``How does the concept of a heuristic relate to the process of problem-solving?''. Additional examples can be found in Appendix~\ref{app:additional_examples}.

\subsection{Quantifying Information Gain}
\label{sec:analysis:infogain}

In Sec.~\ref{sec:results:human} we asked annotators to rate how informative each follow-up question is. Here we propose an alternative, automated method for effectively evaluating informativeness without human annotators---by leveraging the ``comprehensive answers'' generated from GPT-4 (CA; Sec.~\ref{sec:data:augmentation}). We consider CA as a proxy for the set of all and only the relevant information that can be discussed in the context of (IQ, IA) pair. Thus, an informative FQ cannot be answered by IA (otherwise, the answer to FQ would add no new information); but should be answerable from CA (otherwise, it may be irrelevant). Motivated by this, we prompt GPT-4 to evaluate the answerability of each model's generated follow-up questions using each IA and CA. %Follow-up questions are classified as informative if they are only answerable by CA but not by the IA.  

\begin{table}[!t]
    \centering
    \footnotesize
    \setlength{\tabcolsep}{3pt} % Adjust column spacing
    \resizebox{\linewidth}{!}{ 
    \begin{tabular}{lccc}
        \toprule
        \textbf{Model} & \textbf{Human-INF} & \textbf{GPT-INF-All (\%)} & 
        \textbf{GPT-INF-Sel (\%)}\\
        \midrule
        \textit{ORG}  & 0.7755 & 25.17 & 23.29\\
        \textit{AUG}  & \textbf{1.4517} & \textbf{36.19} & \textbf{35.91}\\
        \textit{FULL} & 1.2951 & 34.90 & 32.20 \\
        \bottomrule
    \end{tabular}
    }
    \caption{Comparison of human-annotated informativeness scores and GPT-evaluated informative percentage across models.}
    \label{tab:auto_informative}
\end{table}

% 0.7755 & 0.9563 & \textbf{1.4517*} & 1.1297 & 1.2951 & 0.8223\\

% We compared human-annotated informativeness scores with GPT-4's classification of informative questions across models (
Table~\ref{tab:auto_informative} reassess the findings from the human evaluation that \textit{AUG} produced the most informative follow-up questions (36\%), followed by \textit{FULL} (35\%) and \textit{ORG} (25\%). 

Comparison between the GPT-4 predictions and the human-annotated informativeness scores (\S\ref{sec:results:auto}) validates this automated approach by showing that annotators assigned slightly higher scores to questions classified by GPT-4 as informative (1.29) than to those classified as not informative (1.07). A T-test ($\textit{p-value} = 0.0011$) confirmed statistical significance, however, with a small magnitude (Cohen's $d = 0.215$) \cite{cohen2013statistical}. %, implying that the actual difference in informativeness scores is relatively minor. 

% These findings suggest that GPT-4's evaluation method could serve as a viable alternative to human annotation, offering potential benefits in reducing annotation time and cost. While it may not strongly differentiate between highly informative and weakly informative questions, its validity remains evident. To further improve the accuracy of informativeness assessment refinements could include redefining what constitutes a ``comprehensive answer'' and increasing the sample size to enhance robustness.

% while GPT-4's evaluation method is valid, it does not strongly differentiate between highly informative and weakly informative questions. offering  Given the small effect size, however, further refinements may be needed to improve the accuracy of informativeness assessment, such as redefining what constitutes a ``comprehensive answer'' and increasing the sample size to enhance robustness.


\section{Related Work}\label{sec:related}

In this section, we summarize current LLM hallucination detection (\S\ref{sec:related:detect}) and mitigation methods (\S\ref{sec:related:mitigate}), and categorize them based on internal ways and external ways.

\subsection{LLM Hallucinations Detection}\label{sec:related:detect}

Some previous works aim to detect hallucinated responses during LLM inference.
We divided them into 2 types: external and internal.

\sssec{External}.
Some works detect the hallucinations from the external response.
\cite{niu2024ragtruth, su2024mitigating, hu2024lrp4ragdetecting} utilizes RAG or KG to find the closest response from a ground-truth dataset, and compare it with the response from LLM to detect potential hallucinations.
Considering the high computing and storage cost, some lightweight external methods have also been proposed:
\cite{chen2024inside} propose new EigenScore metrics to evaluate responses' self-consistency.
\cite{zhang2023enhancing, sadat2023delucionqa} check the keywords within the response to check factuality.
\cite{kossen2024semantic} use semantics entropy from multiple queries to detect.

\sssec{Internal}.
Recently, researchers have tried to discover hallucination pattern within LLM inference process.
Most works directly use hidden states\cite{beigi2024internalinspector, chen2024inside, su2024unsupervised, duan2024llms, azaria2023internal} as the classification features, while activation state is also taken into consideration \cite{chen2024context, ji2024llm, he2024llm}.
Some works \cite{chuang2024lookback, yuksekgonul2023attention} also use attention as the classification feature since they have better interpretability.
Some researchers discover that the hallucinated tokens have lower token probability, and use it as a feature \cite{quevedo2024detecting, he2024llm}.

\subsection{LLM Hallucinations Mitigation}\label{sec:related:mitigate}

Previous works have proposed several methods to mitigate hallucinations in LLMs.
We divided previous mitigation methods into 3 types: external, model-based, and internal.

\sssec{External}.
Most works try to use external knowledge or inference paradigms to mitigate.
The most direct methods utilize outside knowledge to enhance LLMs' ability to generate factual responses, which can be categorized into knowledge graph-based \cite{sun2024thinkongraph, luo2024reasoning}, and RAG-based \cite{li2024chainofknowledge, tian2024finetuning}.
LLMs' ability to self-debug their response is also used to mitigate hallucinations \cite{chen2024teaching, gou2024critic}.
Some works also optimize the inference process of LLM, like dividing the task into sub-tasks \cite{pan2023fact}, building a chain for more robust reasoning \cite{luo2024reasoning, li2024chainofknowledge}.
Multi-agent paradigm is also considered in hallucination mitigation \cite{hong2024metagpt}, since the debate across multiple agents can enhance the response from a single agent.

\sssec{Model}.
Some works also aim at modifying models' weights for mitigation.
\cite{liu2024mitigating, li2024chainofknowledge, tian2024finetuning} addresses this issue by introducing a new dataset to fine-tune the model.
Some works aim to use full parameter fine-tuning \cite{tian2024finetuning}, while other works aim to use LoRA for more lightweight fine-tuning \cite{liu2024mitigating, li2024chainofknowledge}.
Besides fine-tuning, model editing is also a popular model-level hallucination mitigation method \cite{meng2023locating}, where developers or admins can only modify a few neural connections within the MLP layer to achieve accurate fact modification.

\sssec{Internal}.
Considering previous external methods and model methods' drawbacks on high cost and potential threat to catastrophic forgetting, some internal methods have been proposed.
Internal methods mainly focus on lightly modifying the decoding process of LLM without changing model weight.
\cite{azaria2023internal, chen2024context} discover potential mode within hidden state, and utilize it to mitigate hallucination.
\cite{chuang2024lookback} finds the significant difference between factual response and hallucinated response within attention scores, and intervenes attention score in the decoding process.
\cite{liu2024litcab} propose to use a new layer to intervene in logit output for better response.
\section{Conclusion}
\label{sec:Conclusion}
This work evaluates proprietary and open-weight models in agentic frameworks for handling ambiguity in software engineering. In code generation, to effectively integrate new information into the solution, an agent must detect ambiguity and ask targeted questions. Our key findings are:
\begin{itemize}[itemsep=0pt, topsep=0pt]
    \item Given an underspecified input, Claude Sonnet 3.5 and Claude Haiku 3.5 with interaction can achieve 80\% of their performance with a well-specified input. In contrast, open-weight models struggle: Deepseek relies on navigational cues to locate relevant files, while Llama 3.1 70B extracts limited information from the user.
    \item LLMs do not interact unless explicitly prompted, and their ambiguity detection is highly sensitive to prompt variations. Only Claude Sonnet 3.5 achieves a higher accuracy of 84\% in distinguishing between well-specified and underspecified input.

    \item Claude Sonnet 3.5, Haiku 3.5, and Deepseek effectively extract new, detailed user information, whereas Llama 3.1 struggles to ask the right questions.
    
\end{itemize}
Despite these advances, a gap remains between resolve rates for underspecified vs. fully specified issues. Open-weight models need better interaction strategies to improve resolution, while proprietary models, particularly Claude Haiku 3.5, require stronger prompting to engage interactively. This work establishes the current state-of-the-art in handling ambiguity through interaction, breaking the resolution process into multiple steps.






% In the unusual situation where you want a paper to appear in the
% references without citing it in the main text, use \nocite
% \nocite{langley00}

\bibliography{acl,custom}
\bibliographystyle{icml2025}


%%%%%%%%%%%%%%%%%%%%%%%%%%%%%%%%%%%%%%%%%%%%%%%%%%%%%%%%%%%%%%%%%%%%%%%%%%%%%%%
%%%%%%%%%%%%%%%%%%%%%%%%%%%%%%%%%%%%%%%%%%%%%%%%%%%%%%%%%%%%%%%%%%%%%%%%%%%%%%%
% APPENDIX
%%%%%%%%%%%%%%%%%%%%%%%%%%%%%%%%%%%%%%%%%%%%%%%%%%%%%%%%%%%%%%%%%%%%%%%%%%%%%%%
%%%%%%%%%%%%%%%%%%%%%%%%%%%%%%%%%%%%%%%%%%%%%%%%%%%%%%%%%%%%%%%%%%%%%%%%%%%%%%%
\newpage
\appendix
\onecolumn
\newpage
\appendix
\onecolumn

\part{}
\section*{\centering \LARGE{Appendix}}
\mtcsettitle{parttoc}{Contents}
\parttoc

\clearpage

\section{Related Work}
\label{sec:relatedwork}
% \paragraph{Tool Usage and Toolchain Management} Research in this area focuses on how intelligent agents design and optimize tool networks to effectively execute complex tasks, particularly by dynamically generating, selecting, and combining tools based on task requirements.This includes methods for automated tool generation and optimization, emphasizing systems that can adaptively choose and adjust tool combinations according to different task needs.

% \paragraph{Multi-Agent Systems and Collaboration} Research in Multi-Agent Systems has explored how multiple intelligent agents can collaboratively solve complex tasks in dynamic environments. One significant contribution is the development of decentralized algorithms that allow agents to autonomously form beneficial collaborations and adapt to changing tasks without the need for a central server (DeLAMA) ~\citep{tang2024decentralizedlifelongadaptivemultiagentcollaborative}. Another key area of study focuses on collaboration among heterogeneous agents, where different agents with varied capabilities work together on complex tasks, such as cleaning large spaces, using hierarchical decision models to allocate sub-tasks effectively~\citep{liu2023heterogeneousembodiedmultiagentcollaboration}. Additionally, collaborative learning approaches like Collaborative Q-learning (CollaQ) enhance agent teamwork by decomposing the Q-function and introducing reward attribution techniques to improve performance in multi-agent environments, such as the StarCraft challenge ~\citep{zhang2020multiagentcollaborationrewardattribution}. Finally, research has also examined how multi-agent collaboration can enhance the performance of large language models (LLMs) in tasks like simulations and software development, highlighting the potential of intelligent agent collaboration to improve task outcomes~\citep{talebirad2023multiagentcollaborationharnessingpower}.

\paragraph{Code Generation and Task Solving with LLMs} Large Language Models (LLMs) have demonstrated remarkable potential in generating code to solve complex tasks. Prior studies highlight their effectiveness in mathematical computation ~\citep{zhou2023solving, wang2023mathcoder, gou2023tora}, tabular reasoning ~\citep{chen2022program, lyu2023faithful, lu2024chameleon}, and visual understanding ~\citep{suris2023vipergpt, choudhury2023zero, gupta2023visual}. Frameworks such as AutoGen ~\citep{wu2023autogen} and CodeActAgent~\citep{wang2024executable} extend this capability to agent-based tasks by interpreting executable code as actions. These models dynamically invoke basic tools based on environmental feedback, significantly expanding their utility. Despite their successes, these approaches often treat program generation processes independently, failing to model shared task features and limiting the reusability of functional modules across tasks.

\paragraph{Reusable Tool Creation and Abstraction} To address the limitations of single-use program generation, recent efforts have focused on creating reusable tools. CREATOR ~\citep{qian2023creator} separates the processes of planning (tool creation) and execution, while LATM ~\citep{cai2023large} and CRAFT ~\citep{yuan2023craft} pre-build tools using training and validation sets for task solving. However, these methods often generate a large number of tools, presenting challenges for their efficient reuse. Furthermore, while abstraction-based approaches like REGAL ~\citep{stengel2024regal} focus on extracting reusable tools from primitive programs, they primarily construct simple tools with limited functional complexity. Similarly, Trove ~\citep{wang2024trove} adopts a training-free approach by dynamically composing high-level tools during testing, but its reliance on self-consistency can lead to hallucinated knowledge, reducing accuracy in complex tasks.

\paragraph{Tool Selection for Complex Task Solving} Currently, research on tool selection and retrieval methods primarily focuses on selecting appropriate tools through retrieval mechanisms and LLM-based approaches. ToolRerank ~\citep{zheng2024toolrerank} uses adaptive truncation and hierarchy-aware reranking to improve retrieval results, while Re-Invoke ~\citep{chen2024reinvoketool} introduces an unsupervised framework with synthetic queries and multi-view ranking, enhancing both single-tool and multi-tool retrieval. COLT ~\citep{Qu_2024COLT} combines semantic matching with graph-based collaborative learning to capture relationships among tools, outperforming larger models in some cases. AvaTaR~\citep{wu2024avataroptimizingllmagents} automates the optimization of LLM prompts for better tool utilization, and DRAFT~\citep{qu2024DAFT} refines tool documentation through iterative feedback and exploration, helping LLMs better understand external tools. Despite progress, existing methods generally overlook cost-effectiveness and scalability in tool selection, and often struggle to efficiently adapt to new tools and task requirements in dynamic environments, leading to performance and efficiency bottlenecks. In contrast, our approach dynamically prioritizes tools by combining their relevance and structural importance, ensuring computational efficiency and scalability, thus enabling more effective solutions for complex tasks.
\section{Experimental Details}
\label{app:apexp}
\subsection{Open-ended Task}
\label{subsec:open}
\paragraph{Benchmark} We employed the benchmark proposed by Voyager~\citep{wang2023voyager}, using Minecraft as the experimental platform. Minecraft provides a sandbox environment where players gather resources and craft tools to achieve various goals. The simulation is built on MineDojo~\citep{fan2022minedojo} and uses Mineflayer~\citep{PrismarineJS2013} JavaScript APIs for motor control. 

\paragraph{Baselines}
We conducted a comprehensive comparison with four baselines. Except for Voyager, these methods were originally designed for NLP tasks without embodiment. Therefore, we had to reinterpret and adapt them for execution within the MineDojo environment, ensuring compatibility with the specific requirements of our experimental setup.
\begin{itemize}
    \item \textbf{ReAct:} ReAct~\citep{yao2022react} uses chain-of-thought prompting [46] by generating both reasoning traces and action
plans with LLMs. We provide it with our environment feedback and the agent states as observations.
    \item \textbf{Reflexion:} Reflexion~\citep{shinn2023reflexion} is built on top of ReAct~\citep{yao2022react}with self-reflection to infer more intuitive future actions.
    \item \textbf{AutoGPT:} AutoGPT~\citep{richardssignificant} is a popular software tool that automates NLP tasks by decomposing a high-level
goal into multiple subgoals and executing them in a ReAct-style loop. We re-implement AutoGPT by using GPT-4O to do task decomposition and provide it with the agent states, environment feedback,
and execution errors as observations for subgoal execution
We provide it with execution errors and our self-verification module.
    \item \textbf{Voyager:} Voyager~\citep{wang2023voyager} is a system that integrates an automated curriculum, a scalable skill library, and an iterative prompting framework based on environmental feedback to explore, store, and accumulate skill library within the Minecraft environment.
\end{itemize}


\paragraph{Metric}
The evaluation metric is based on the number of iterations required to progress through tool upgrades, from wooden to stone, iron, and finally diamond tools. Each execution of code is considered one iteration.

\paragraph{Model}
We leverage GPT-4o for text completion, along with the text-embedding-ada-002 API for text embedding. We set all temperatures to
0 except for the automatic curriculum, which uses temperature = 0.1 to encourage task diversity. 

\paragraph{Setting}
We set the maximum number of iterations to 160. For both \ours\ and Voyager, all agents are controlled by GPT-4o, with the number of tools retrieved per iteration set to 5. To ensure a fairer comparison, we removed the Tool Requirement Stage and bug-free checks in \ours\ , and allowed a maximum of 3 self-checks per iteration.

\paragraph{Item Types and Levels}
In the Minecraft task, there are different types and levels of items. Diamond tools are the highest level, and rare items such as golden apples also exist. High-level tools require some lower-level items to craft. Table \ref{tab:toollist} lists the key items in the Minecraft task.
\begingroup
\begin{table}[H]
\caption{List of item types and levels in the Minecraft task.}
\label{tab:toollist}
\vskip -0.1in
\setlength{\tabcolsep}{10pt} % 调整列间距
\begin{center}
\begin{small}
\begin{sc}
\begin{tabular}{l|c|c}
\toprule
\textnormal{\textbf{Category}} & \textnormal{\textbf{level}} & \textnormal{\textbf{Items}} \\
\midrule         
\midrule
\multirow{4}{*}{\multirow{3}{*}{\normalfont Tools}} 
              & \normalfont Wooden Tools & \normalfont Wooden\_Shovel,Wooden\_Pickaxe,Wooden\_Axe,Wooden\_Hoe,Wooden\_Sword \\
              \cmidrule{2-3}
              & \normalfont Stone Tools &\normalfont stone\_pickaxe, stone\_shovel,Stone\_Axe,Stone\_Hoe,Stone\_Sword   \\
              \cmidrule{2-3}
              & \normalfont Iron Tools &\normalfont iron\_pickaxe, iron\_axe, iron\_sword, iron\_shovel, iron\_hoe    \\
              \cmidrule{2-3}
              & \normalfont Diamond Tools &\normalfont diamond\_pickaxe, diamond\_sword, diamond\_axe, diamond\_shovel    \\
             
\midrule
\multirow{2}{*}{\multirow{1}{*}{\normalfont  Armor}} 
              & \normalfont Iron Armor &\normalfont iron\_chestplate, iron\_helmet, iron\_leggings  \\
              \cmidrule{2-3}
              & \normalfont Diamond Armor &\normalfont diamond\_chestplate, diamond\_helmet, diamond\_leggings, diamond\_boots     \\

\midrule
\multirow{3}{*}{\multirow{2}{*}{\normalfont  Food}} 
              & \normalfont Raw Food &\normalfont chicken, mutton, porkchop, rabbit, raw\_rabbit, spider\_eye, bone  \\
              \cmidrule{2-3}
              & \normalfont Cooked Food &\normalfont cooked\_beef, cooked\_chicken, cooked\_mutton, cooked\_porkchop, cooked rabbit  \\
              \cmidrule{2-3}
              & \normalfont Advanced Food &\normalfont golden apple    \\

\bottomrule
\end{tabular}
\end{sc}
\end{small}
\end{center}
\vskip -0.1in
\end{table}
\endgroup


\subsection{Agent Task}
\label{subsec:agent}
\paragraph{Benchmark}
We conducted experiments on two types of agent tasks, demonstrating {\ours}'s capabilities in both game-related and data science tasks.
\begin{itemize}
     \item \textbf{TextCraft:} We evaluate {\ours} on the TextCraft dataset~\citep{futuyma1988evolution}, which challenges agents to craft Minecraft items in a text-only environment~\citep{cote2019textworld}. Each task instance provides a goal and a sequence of crafting commands, which include distractors. We use depth-2 splits for testing and reserve a subset of depth-1 recipes for development, resulting in a 99/77 train/test split.
    \item \textbf{InfiAgent-DABench:} We also test {\ours} on the InfiAgent-DABench benchmark~\citep{hu2024infiagent}, which evaluates LLM-based agents on end-to-end data analysis tasks. This benchmark consists of 257 questions across 52 CSV files, with each question corresponding to a unique CSV file. Agents are required to generate code to analyze data and produce the specified output format. We randomly selected 20 CSV files and their associated question-answer pairs as training data, resulting in a train/test split of 98/159 instances.
\end{itemize}

\paragraph{Baselines}
We compare \ours\ with three methods described below.
\begin{itemize}
     \item \textbf{ReAct:} In this setting, we employ the executor to interact iteratively with the environment, adopting the think-act-observe prompting style from ReAct~\citep{yao2022react}.
     \item \textbf{Plan-Execution:} In contrast, the Plan-and-Execute approach~\citep{shridhar2023art, yang2023intercode} generates a plan upfront and assigns each sub-task to the executor. To ensure each step is executable without further decomposition, we provide new prompts with more detailed planning instructions.
    \item \textbf{Reflexion:} In the Reflection setting~\citep{shinn2023reflexion}, the agent engages in self-reflection after each step, drawing on environmental feedback and exploration history. 
\end{itemize}

\paragraph{Metric} 
The most practically important aspect of the solutions is correctness. For Textcraft, we verify whether the agent’s inventory contains the goal item. For DABench, we check if the agent’s final answer matches the ground truth.

\paragraph{Model}
During training, we use GPT-4o to construct the tool library with a temperature setting of 0. In the testing phase, we conduct a comprehensive comparison of various open-source and closed-source models. The open-source models include \textit{Qwen2.5-7B-Instruct, Qwen-Coder-7B-Instruct, Qwen2.5-14B-Instruct, Deepseeker-Coder-6.7B-Instruct, and Deepseeker-Coder-33B-Instruct}, while the closed-source models primarily include \textit{gpt-3.5-turbo-1106} and \textit{Claude-3-haiku}. During testing, the temperature is set to 0.3, and each experiment is repeated 3 times, with the average result reported.

\paragraph{Setting} 
For ReAct, Reflexion, and \ours\ , the maximum number of steps is set to 20. For Plan-Execution, the maximum number of steps for each sub-task is set to 8. In \ours\ , the number of tools retrieved during testing is limited to 3.



\subsection{Single-turn Code Task}
\label{subsec:code}
\paragraph{Benchmark}
To further explore {\ours}'s potential, we evaluated it on single-turn code generation tasks spanning mathematical reasoning, date comprehension, and tabular reasoning:
 \begin{itemize}
     \item \textbf{MATH:} We used a subset of the MATH dataset~\citep{hendrycks2021measuring}, focusing on 405 level-4 and level-5 algebra problems (MATH contains 5 levels of difficulty) that require textual understanding and advanced reasoning. We randomly selected 200 examples from the test set of the MATH dataset to construct the tool network, resulting in a train/test split of 200/405.
     \item \textbf{Date:} We use the date understanding task from BigBenchHard~\citep{srivastava2022beyond}, which consists of short word problems requiring date understanding. We follow the data splits provided by REGAL\citep{stengel2024regal}, resulting in a train/test split of 66/180.
     \item \textbf{TabMWP:} We further extend our general experiments on MATH by testing on TabMWP~\citep{grand2023learning}, a tabular reasoning dataset consisting of math word problems about tabular data. Based on the CRAFT~\citep{yuan2023craft} splits, we selected 470 problems from levels 7 and 8 (TabMWP contains 8 levels) from the 1,000 test examples. Additionally, we randomly selected 200 examples from the TabMWP training set, resulting in a train/test split of 200/470.
\end{itemize}

\paragraph{Baselines}
For these tasks, we use Programs of Thoughts (PoT)~\citep{chen2022program} and other existing tool-making methods as baselines for comparison.

\begin{itemize}
    \item \textbf{PoT:} The LLM utilizes a program to reason through the problem step by step~\citep{chen2022program}.
   \item \textbf{LATM:} LATM~\citep{cai2023large} samples 3 examples from the training set and create a tool for the task, which is further verified by 3 samples from the validation set. The created tool is then applied to all test cases.
    \item \textbf{CREATOR:} CREATOR~\citep{qian2023creator} disentangle planning (tool making) from execution, enabling Large Language Models (LLMs) to autonomously create a specific tool for each test case during inference.
     \item \textbf{CRAFT:} CRAFT~\citep{yuan2023craft} constructs task-specific toolsets by generating a tool for each training example. During testing, it utilizes a tool retrieval module and a reasoning process akin to CREATOR, generating a function first and then producing the corresponding invocation code. 
      % \item \textbf{Trove:} Trove~\citep{wang2024trove} introduces a training-free method based on self-consistency, where LMs interact with the toolbox through three modes—IMPORT, SKIP, and CREATE. Each mode is executed K times, and from the 3K responses, the function from the most consistent and optimal response is added to the toolbox.
      \item \textbf{REGAL:} During training, REGAL~\citep{stengel2024regal} refines primitive programs by extracting functions. In the testing phase, it retrieves both tools and refactored programs—comprising original and refactored versions—to generate a program that effectively solves the task. 
\end{itemize}
\paragraph{Metric}
We use correctness as the evaluation metric, measuring whether the execution outcome of the solution program exactly matches the ground-truth answer(s).
\paragraph{Model}
The models for the single-turn code generation task are the same as those used for the Agent Task, as presented in Section \ref{subsec:agent}.
\paragraph{Setting}
To ensure a fair comparison, we make slight adjustments to each method. For all methods, we allow up to 3 times for format checking and correction, as small models may not always follow the required output format. For PoT, we use 6 fixed examples of basic tool usage as few-shot. CREATOR employs the rectifying process, while for CRAFT, we use the same training set as our method and construct the tool library with GPT-4o, retrieving 3 tools during testing. For Regal, we use PoT along with GPT-4o to obtain ground-truth code, select the correct program, and have GPT-4o reconstruct it. To maintain fairness in tool generation quality, we standardize the few-shot examples of basic tools and retrieve 3 tools, along with 3 usage examples from the current tool library, avoiding errors from pruned tools. For our method, we train with GPT-4o, retrieving 3 tools and their corresponding usage examples during testing, while fixing the basic tool few-shot examples to 3, ensuring consistency with PoT’s total few-shot count.
\section{More Results}
\label{app:apresults}
\subsection{Open-ended Task}
\label{subsec:open-results}
\paragraph{More complex tools} 
Our hierarchical graph architecture offers significant advantages in handling complex tasks and large-scale systems. As shown in Figure \ref{fig:toolnet1}, Trial 1 starts with five nodes occupying three layers, and evolves into a five-layer network, with an increasing number of inter-tool calls. As shown in Figure \ref{fig:toolnet2}, Trial 2 starts with four nodes occupying four layers, and evolves into a five-layer network with more inter-tool calls. As shown in Figure \ref{fig:toolnet3}, Trial 3 starts with four nodes occupying three layers, and evolves into a six-layer network structure, with a growing number of inter-tool calls. Our tool graph becomes progressively more complex, flexibly expanding and optimizing its components. These results demonstrate that our method can generate tools that call each other, and combine them into more complex tools. This not only enhances scalability but also facilitates the creation of more sophisticated tools, enabling the solution of increasingly complex problems.


\paragraph{More types of inventory} Our method is able to generate more inventory types than Voyager. As shown in Table \ref{tab:Number}, we can see that {\ours} produces more inventory types in all three trials compared to Voyager.

The inventory collected by {\ours} in each trial is

\begin{itemize}
    \item \textbf{Trial 1:}  \textit{oak\_log, birch\_log, oak\_planks, birch\_planks, crafting\_table, stick, wooden\_pickaxe, dirt, cobblestone, coal, stone\_pickaxe, raw\_copper, furnace, copper\_ingot, andesite, raw\_iron, granite, iron\_ingot, iron\_pickaxe, shield, diorite, raw\_gold, lapis\_lazuli, redstone, diamond, diamond\_pickaxe, bucket, gold\_ingot, iron\_chestplate, arrow, iron\_sword, iron\_helmet, diamond\_sword, diamond\_helmet, lightning\_rod, chest, iron\_axe, iron\_leggings, sandstone, dandelion, spider\_eye, string, iron\_shovel, copper\_block, iron\_door, iron\_hoe, kelp, bow, dried\_kelp, torch, cooked\_beef, gray\_wool, cobbled\_deepslate, tuff, diamond\_leggings, bone, diamond\_chestplate, chicken, white\_banner, cooked\_chicken, egg, feather, oak\_sapling, apple, acacia\_log, golden\_apple, diamond\_axe}

    \item \textbf{Trial 2:}  \textit{oak\_sapling, oak\_log, stick, oak\_planks, crafting\_table, wooden\_pickaxe, dirt, cobblestone, stone\_pickaxe, diorite, raw\_iron, coal, lapis\_lazuli, gravel, furnace, iron\_ingot, raw\_copper, sandstone, granite, iron\_pickaxe, andesite, raw\_gold, gold\_ingot, diamond, diamond\_pickaxe, redstone, cobbled\_deepslate, bucket, iron\_sword, arrow, bow, bone, birch\_log, chest, amethyst\_block, calcite, smooth\_basalt, iron\_chestplate, diamond\_sword, diamond\_helmet, iron\_leggings, diamond\_boots, water\_bucket, string, orange\_tulip, mutton, white\_wool, porkchop, dandelion, cooked\_porkchop, cooked\_mutton}

    \item \textbf{Trial 3:}  \textit{jungle\_log, stick, oak\_sapling, jungle\_planks, crafting\_table, dirt, wooden\_pickaxe, cobblestone, stone\_pickaxe, raw\_iron, raw\_copper, furnace, iron\_ingot, iron\_pickaxe, coal, diorite, lapis\_lazuli, andesite, moss\_block, clay\_ball, redstone, raw\_gold, cobbled\_deepslate, granite, diamond, diamond\_pickaxe, copper\_ingot, gunpowder, bucket, gravel, gold\_ingot, oak\_log, iron\_sword, iron\_chestplate, chest, diamond\_sword, spruce\_sapling, rotten\_flesh, bone, rose\_bush, water\_bucket, string, oak\_planks, grass\_block, diamond\_helmet, iron\_leggings, emerald, snowball, rabbit\_hide, rabbit, spruce\_log, cooked\_rabbit, diamond\_boots}
\end{itemize}


The inventory collected by Voyager in each trial is
\begin{itemize}
    \item \textbf{Trial 1:}  \textit{oak\_log, birch\_log, oak\_sapling, birch\_sapling, oak\_planks, stick, crafting\_table, wooden\_pickaxe, dirt, cobblestone, stone\_pickaxe, raw\_copper, white\_tulip, coal, furnace, copper\_ingot, granite, raw\_iron, iron\_ingot, lightning\_rod, iron\_pickaxe, pink\_tulip, orange\_tulip, sandstone, shears, shield, diorite, cobbled\_deepslate, iron\_block, chest, tuff, lapis\_lazuli, redstone, diamond, raw\_gold, gold\_ingot, diamond\_pickaxe, diamond\_helmet, diamond\_sword, sand, andesite, arrow, bone, iron\_chestplate, beef, leather, oak\_leaves, porkchop, cooked\_beef, leather\_leggings}

    \item \textbf{Trial 2:}  \textit{dirt, oak\_log, oak\_planks, crafting\_table, stick, oak\_sapling, wooden\_pickaxe, cobblestone, coal, stone\_pickaxe, raw\_iron, granite, lapis\_lazuli, raw\_copper, furnace, iron\_ingot, copper\_ingot, iron\_helmet, iron\_pickaxe, diorite, andesite, salmon, ink\_sac, iron\_chestplate, lightning\_rod, cooked\_salmon, stone, stonecutter, rotten\_flesh, gravel, flint, chest, iron\_leggings, copper\_block, cobbled\_deepslate, tuff, diamond, diamond\_pickaxe, raw\_gold, gold\_ingot, redstone, diamond\_sword, egg, diamond\_boots, diamond\_axe}

    \item \textbf{Trial 3:}  \textit{jungle\_log, jungle\_planks, oak\_sapling, oak\_log, crafting\_table, stick, wooden\_pickaxe, dirt, cobblestone, coal, stone\_pickaxe, raw\_copper, furnace, copper\_ingot, magma\_block, lightning\_rod, stone\_axe, jungle\_boat, kelp, sand, sandstone, glass, raw\_iron, granite, lapis\_lazuli, diorite, iron\_ingot, bucket, iron\_pickaxe, chest, andesite, redstone, dried\_kelp, iron\_chestplate, wooden\_sword, shield, iron\_sword}
\end{itemize}

\vskip -0.2in
\begin{table}[H]
\caption{Number of different inventory types produced by each trial}
\label{tab:Number}
% \vskip 0.1in
\setlength{\tabcolsep}{12pt} % 调整列间距
\renewcommand{\arraystretch}{1.0} % 调整行间距
\begin{center}
% \resizebox{\textwidth}{!}{ % 自动调整表格宽度以适应页面
\begin{small}
\begin{sc}
\begin{tabular}{lccc} % 确保列数与标题一致
\toprule
\textnormal{\textbf{Method}} & \textnormal{\textbf{Trial 1}} & \textnormal{\textbf{Trial 2}} & \textnormal{\textbf{Trial 3}}  \\
\midrule
\normalfont Voyager     & 50  & 45  & 37    \\
\normalfont AETG(Ours)  & 67  & 51  & 53    \\
\bottomrule
\end{tabular}
\end{sc}
\end{small}
% }
\end{center}
\vskip -0.1in
\end{table}


\paragraph{Longer exploration path} To better demonstrate the exploration capabilities of the agent, we compared the exploration trajectories and their lengths. As shown in Figure \ref{fig:linermap}, our agent exhibits longer and more persistent exploration capabilities than Voyager. In Table \ref{tab:length}, the trajectory lengths of our agent are consistently much greater than those of Voyager. {\ours}is able to traverse across multiple terrains, with an average distance 2.66 times longer than Voyager. Additionally, {\ours} can explore across different continental plates, while Voyager remains confined to a single plate, highlighting the exceptional exploration capability of {\ours}.

% \vskip -0.2in
\begin{table}[H]
\caption{Exploration trajectory length in each trial, where \textit{Performance Gain} = $\textit{ours}/\textit{voyager}$.}
\label{tab:length}
% \vskip 0.1in
\setlength{\tabcolsep}{12pt} % 调整列间距
% \renewcommand{\arraystretch}{1.0} % 调整行间距
\begin{center}
% \resizebox{\textwidth}{!}{ % 自动调整表格宽度以适应页面
\begin{small}
\begin{sc}
\begin{tabular}{lcccc} % 确保列数与标题一致
\toprule
\textnormal{\textbf{Method}} & \textnormal{\textbf{Trial 1}} & \textnormal{\textbf{Trial 2}} & \textnormal{\textbf{Trial 3}} & \textnormal{\textbf{\textit{Avg}}}\\
\midrule
\normalfont Voyager     & 1925.74  & 4102.99  & 902.13  & 2310.29   \\
\normalfont {\ours}(Ours)  & 5665.75  & 8908.57  & 3895.06 & 6156.46  \\
\midrule
\normalfont \textit{Performance Gain} & 2.94  & 2.17   & 4.32    & 2.66 \\
\bottomrule
\end{tabular}
\end{sc}
\end{small}
% }
\end{center}
\vskip -0.1in
\end{table}


\vskip -0.2in
\begin{figure}[H]
\vskip 0.2in
\begin{center}
\centerline{\includegraphics[width=1\linewidth]{trial-map.png}}
% \vskip -0.2in
\caption{Map coverage: Three bird’s eye views of Minecraft maps. The trajectories are plotted based on the position coordinates where each agent interacts.}
\label{fig:trialmap}
\end{center}
\vskip -0.3in
\end{figure}


\vskip -0.2in
\begin{figure}[H]
\vskip 0.2in
\begin{center}
\centerline{\includegraphics[width=1\linewidth]{liner-map.png}}
% \vskip -0.2in
\caption{Movement trajectory Map: Three bird’s eye views of Minecraft maps. The trajectories are plotted based on the position coordinates where each agent interacts.}
\label{fig:linermap}
\end{center}
\vskip -0.3in
\end{figure}



\paragraph{Efficient Zero-Shot Generalization to Unseen Tasks} Based on the results presented in Table \ref{tab:newtechtree} and Figure \ref{fig:diamon and compass}, we can clearly observe the significant advantages of {\ours} in the open-ended task. Table \ref{tab:newtechtree} shows the number of iterations required for different methods to complete various tasks (Gold Sword, Compass, Diamond Hoe, Lava Bucket), where fewer iterations indicate higher efficiency. Compared to Voyager and {\ours} (w/o toolnet), {\ours} consistently requires significantly fewer iterations across all tasks, demonstrating substantial improvements in efficiency. Notably, in the Gold Sword task, {\ours} (ours) completes the task in just 14.00±1.73 iterations, whereas Voyager requires 46.33±14.57 iterations, showcasing its superior performance.

Figure \ref{fig:diamon and compass} further visualizes the intermediate progress of different methods on the "Craft a Compass" and "Craft a Diamond Hoe" tasks. It is evident that {\ours} learns and masters the necessary skills for crafting items more quickly. As the number of prompting iterations increases, {\ours} reaches the task objectives significantly earlier than the other methods. Additionally, while {\ours}(w/o Tool Graph) performs better than Voyager, it still lags behind {\ours}, indicating that the ToolNet component plays a crucial role in enhancing the model's capability.

Overall, these experimental results demonstrate that {\ours} not only learns new skills and crafting techniques more efficiently but also that its key module, Tool Graph, is essential for overall performance improvement. This further validates the effectiveness of our approach in self-driven exploration and task generalization.


\begingroup
\begin{table}[H]
\caption{The mastery of the tech tree in the Open-ended Task. The number indicates the number of iterations. The fewer the iterations, the more efficient the method. "N/A" indicates that the number of iterations for obtaining the current type of tool is not available.}
\label{tab:newtechtree}
\vskip 0.1in
\setlength{\tabcolsep}{12pt} % 调整列间距
% \renewcommand{\arraystretch}{1.0} % 调整行间距
\begin{center}
% \resizebox{\textwidth}{!}{ % 自动调整表格宽度以适应页面
\begin{small}
\begin{sc}
\begin{tabular}{lccccc} % 确保列数与标题一致
\toprule
\textnormal{\textbf{Method}} & \textnormal{\textbf{Trial}} & \textnormal{\textbf{Gold Sword}} & \textnormal{\textbf{Compass}} & \textnormal{\textbf{Diamond Pickaxe}} & \textnormal{\textbf{Lava Bucket}} \\
\midrule
\multirow{4}{*}{\multirow{2}{*}{\normalfont Voyager}} 
              & \normalfont Trial 1 & 48 & 16 &  24 & N/A         \\
              & \normalfont Trial 2 & 31 & 17 &  25 & 39         \\
              & \normalfont Trial 3 & 60 & 20 & 18  & N/A         \\
              \cmidrule{2-6}
              & \textit{Average} & 46.33$\pm$14.57 & 17.67$\pm$2.08 & 22.33$\pm$3.79 & 39.00$\pm$0.00 \\
\midrule
\multirow{4}{*}{\multirow{2}{*}{\normalfont {\ours}\textit{\small(w/o toolnet)}}} 
               & \normalfont Trial 1 & 26 & 27 & 23  & N/A         \\
              & \normalfont Trial 2 & 18 & 22 & 18  & N/A        \\
              & \normalfont Trial 3 & 56 & 15 & 30  & N/A          \\
              \cmidrule{2-6}
              & \textit{Average} & 33.33$\pm$20.03 & 21.33$\pm$6.03 & 23.67$\pm$6.03 & N/A$\pm$N/A \\
\midrule
\multirow{4}{*}{\multirow{2}{*}{\normalfont {\ours}\textit{\small(ours)}}} 
              & \normalfont Trial 1 & 13 & 28 & 16  & 19       \\
              & \normalfont Trial 2 & 13 & 10 & 14  & 27       \\
              & \normalfont Trial 3 & 16 & 13  & 13  & 18      \\
              \cmidrule{2-6}
              & \textit{Average} & \textbf{14.00$\pm$1.73} & \textbf{17.00$\pm$9.64} & \textbf{14.33$\pm$1.53} & \textbf{21.33$\pm$4.93} \\
             

\bottomrule
\end{tabular}
\end{sc}
\end{small}
% }
\end{center}
\vskip -0.1in
\end{table}
\endgroup



\begin{figure}[H]
\vskip 0.2in
\begin{center}
\centerline{\includegraphics[width=1\linewidth]{compass_and_diamond.png}}
% \vskip -0.2in
\caption{Zero-shot generalization to unseen tasks. Here, we visualize the intermediate progress of each method on the tasks "Craft a Compass" and "Craft a Diamond Hoe."}
\label{fig:diamon and compass}
\end{center}
\vskip -0.3in
\end{figure}



\subsection{Agent Task}
\label{subsec:agent-results}

Figures \ref{fig:toolnet-dabench} and \ref{fig:toolnet-textcraft} present the tool network evolution diagrams of DA-Bench and TextCraft, which visually reflect the call relationships between different tool functions. In these diagrams, each node represents a specific tool function, edges indicate the call dependencies between tools, and the shading of the nodes reflects the frequency of tool calls—darker colors indicate higher call frequency. From Figure \ref{fig:toolnet-dabench}, it can be observed that in DA-Bench, the tool network expands progressively as the task advances, forming multiple core nodes with higher call frequencies. This suggests that certain key tools are frequently called during the task execution, playing a central role. Additionally, the tool call relationships exhibit a hierarchical and well-organized structure, reflecting DA-Bench's efficiency in tool dependency management.

In contrast, Figure \ref{fig:toolnet-textcraft} illustrates the tool network evolution of TextCraft, which also shows a similar expansion trend overall. However, compared to DA-Bench, the tool call frequency in TextCraft is more evenly distributed across multiple nodes, meaning that the system calls a wider variety of tools during task execution, rather than relying on a few core tools. This distribution pattern may suggest that TextCraft adopts a more diverse tool usage strategy in task execution.

A comparative analysis of the two figures reveals that, although both DA-Bench and TextCraft exhibit certain hierarchical and expansive characteristics in their tool call patterns, DA-Bench relies more heavily on a few core tools, whereas TextCraft displays a more dispersed tool call pattern. This contrast not only highlights the differences in tool usage between the two, but also emphasizes the importance and effectiveness of ToolNet.





\subsection{Single-turn Code Task}
\label{subsec:code-results}

As shown in the Figure\ref{fig:toolnet-math} \ref{fig:toolnet-tabmwp}, this illustrates the evolution of the tool graph for the Math and TabMWP tasks. It is evident that the tool graph gradually becomes more complex, creating multiple layers of tools, making the tool graph more intricate. Since the Date task can be solved with fewer tools, there is no evolution of the tool graph. However, the generated tools can still effectively solve the task, while there exists a multi-level calling relationship.


\section{More Ablations}
\label{app:apablation}
\subsection{Open-ended Task}
\label{subsec:open-ablation}

As shown in Figure \ref{fig:ablation}, AETG significantly outperforms methods that lack certain functional modules in discovering new Minecraft items and skills. It can be observed that the performance is worst when "w/o retrieval" is used, indicating that the absence of retrieval has the greatest impact on overall functionality and plays a crucial role, thereby validating the effectiveness of our retrieval method. The performance with "w/o duplication" is slightly better, indicating its importance is weaker than that of "w/o retrieval." The performance of "w/o check" and "w/o pruning" is better, but still far behind AETG, which further demonstrates the importance and effectiveness of each functional component.

\vskip -0.1in
\begin{figure}[H]
% \vskip 0.2in
\begin{center}
\centerline{\includegraphics[width=0.6\linewidth]{toolnumber-ablation.png}}
% \vskip -0.2in
\caption{Ablation study of the iterative prompting mechanism. AETN surpasses all other options, highlighting the essential significance of each functional module in the iterative prompting mechanism.}
\label{fig:ablation}
\end{center}
\vskip -0.3in
\end{figure}


\subsection{Closed-Ended Task}
\label{subsec:closed-ended}
For the Closed-Ended Task, we select Textcraft from the Agent Task and Date from the Single-turn Code Task to evaluate the effectiveness of several components in our method. The results are shown in the Table \ref{tab:closed-toolnumber}.

\begingroup
\begin{table}[H]
\caption{The number of tools in Close-Ended Task.}
\label{tab:closed-toolnumber}
\vskip -0.1in
\setlength{\tabcolsep}{10pt} % 调整列间距
\begin{center}
\begin{small}
\begin{sc}
\begin{tabular}{l|cc}
\toprule
\textnormal{\textbf{Method}} & \textnormal{\textbf{TextCraft}}  & \textnormal{\textbf{Date}} \\
\midrule         

\normalfont W/o Self-Check & 42 & 9 \\
\midrule  
\normalfont W/o Merging & 49 & 11\\
\midrule  
\normalfont W/o pruning & 46 & 9 \\
\midrule  
\normalfont GATE & 44 & 4 \\


\bottomrule
\end{tabular}
\end{sc}
\end{small}
\end{center}
\vskip -0.1in
\end{table}
\endgroup

\section{Tool Making}
\label{app:toolgarph}
\subsection{Basic Tools}
\label{subsec:basic-tools}
As shown in the Table \ref{tab:basictool} , the basic tools generated by each method are displayed.

\begingroup
\begin{table}[H]
\caption{Basic tools in various methods.}
\label{tab:basictool}
\vskip -0.1in
\setlength{\tabcolsep}{10pt} % 调整列间距
\begin{center}
\begin{small}
\begin{sc}
\begin{tabular}{l|p{12cm}}
\toprule
\textnormal{\textbf{Tasks}} & \textnormal{\textbf{Basic Tools}}  \\
\midrule         

\normalfont Other Tasks & \normalfont ToolRequest, NotebookBlock, Terminate, CreateTool, EditTool, Python, Feedback, SendAPI, Feedback, Retrieval \\
\midrule  
\normalfont Minecraft & \normalfont smeltItem, killMob, waitForMobRemoved, givePlacedItemBack, useChest, exploreUntil, craftItem, mineBlock, shoot, placeItem, craftHelper, smeltItem, mineflayer, killMob, useChest, exploreUntil, craftItem, mineBlock, placeItem \\

\bottomrule
\end{tabular}
\end{sc}
\end{small}
\end{center}
\vskip -0.1in
\end{table}
\endgroup


\subsection{Tool construction Lists}
\label{subsec:tool construction}

\paragraph{CREATOR:}
\begin{itemize}[noitemsep, topsep=0pt]
    \item \textbf{MATH:}  \textit{sum of areas, find largest won matches, find K, total distance after bounces, find common ratio sum, count lattice points with distance squared, find c for radius, find circle equation and constants, polynomial degree product, calculate cells, find fiftieth term, find non domain values, inverse function product, find m and n, sum of fractions from roots, find roots of quadratic, main, find coefficients, compute expression, prime factors, find x y, find second largest angle, find y coordinate, find constants, evaluate expression, find b for one solution, find c, find minimum value, find possible s, solve expression, find cone height, solve abc, find minimum expression, \dots, time to hit ground, sum of reciprocals of roots, solve x floor x product, sum of possible x, find constant a, sum of squares of solutions, find cost per extra hour, is triangular number, find smallest b greater than 2011, solve exponential equation, solve club suit equation, find degree of h, f, find vertical asymptotes, domain width, maximize revenue, future value, total savings, find min interest rate, equation, find integers, sum of x coordinates squared, find integer values of a, smallest c for real domain, smallest integer c, find m, required investment, simplify expression, g, distance between midpoints, compute x and power, greatest possible a, find continued fraction value, find a b, solve mnp, compute sum, sum of integers in range,
    }

    \item \textbf{Date:}  \textit{get us thanksgiving date, get date one week from first monday of 2019, calculate anniversary date, calculate yesterday from last day of january, calculate one week ago from first monday, get first monday of 2019, calculate yesterday, calculate yesterday from rescheduled meeting, calculate date a month ago from rescheduled meeting, calculate yesterday from first monday of 2019, get date 10 days before us thanksgiving, calculate one week ago from egg runout, calculate one week ago from end of first quarter, calculate date 24 hours later, calculate date a month ago, calculate date 24 hours after anniversary, calculate one week from today from rescheduled meeting, \dots, get tomorrow from us thanksgiving, calculate yesterday from day before yesterday, calculate yesterday from anniversary, calculate date 10 days ago, calculate one year ago from egg run out date, calculate tomorrow from yesterday, calculate one week from last day of january, calculate one week from anniversary, calculate yesterday from eggs run out, calculate tomorrow from today, calculate tomorrow from day before yesterday, calculate one week ago from today, calculate one week ago, calculate date one month ago from anniversary, calculate one year ago from given date, calculate one week from given date}

    \item \textbf{TabMWP:}  \textit{calculate total cost, smallest points, price difference, cost of river rafts, calculate median, calculate range, calculate total spent, rate of change, cost difference, cost for rides, rate of change vacation days, total participants, calculate mean glasses, find mode of states visited, rate of change straight A students, calculate median basketball hoops, count bins with toys in range, people with at least 3 trips, count teams with fewer than 80 swimmers, calculate median clubs, count exact pushups, children with less than 2 necklaces, people played exactly 3 times, count people with fewer than 80 pullups, range of states visited, find spent amount, \dots, calculate median miles, people with fewer than 3 seashells, calculate median glasses, cost to buy cockatiels, largest broken lights, calculate spent, calculate ice cream cost, range of soccer fields, patrons with at least 2 books, count bushes with 20 roses, total people played golf, range of articles, count shipments with exactly 60 broken plates, total cost for lip balms, rate of change scholarships, count teams with fewer than 50 members, count tests with 34 problems, find mode of soccer fields, rate of change hockey games, find lowest score, count pizzas with exactly 48 pepperoni, count people with at least 30 points, cost of wooden benches, rate of change students, patients with fewer than 2 trips, find mode, total cost for hazelnuts, calculate mean fan letters, readers with at least 4 hats, count classrooms with 41 desks}
\end{itemize}

\paragraph{CRAFT:}
\begin{itemize}[noitemsep, topsep=0pt]
    \item  \textbf{MATH:}  \textit{find pack size, count distinct solutions, calculate points, find tank capacity, solve exponential log equation, total energy equilateral triangle, inverse square law force, find max value, total logs in stack, sum of multiples of 13, calculate exponential growth, gravitational force, find x for piecewise composition, positive difference, specific piecewise func, day exceeds 200 cents, find lattice points, count integer parameters for integer solutions, count zeros in square of power of ten minus one, energy stored, sum of squares of roots, sum odd integers, find d minus e squared, compute complex series sum, total energy configuration, sum of areas, \dots, max item price, solve two variable system, inverse variation power, total distance hopped, is prime, total distance, find constant term of polynomial, total distance moved, find perpendicular slope, calculate inverse proportionality, find value of A, count integer a, find min items for higher score, apply r n times, find min x, day exceeds threshold, calculate area in square yards, solve log equation, total items produced, find variable for distance condition, solve time at speeds, find largest solution, find weight of object, calculate proportional value, calculate material cost, solve for variable, total elements in arithmetic sequence, transformed domain, find day for algae coverage, calculate energy stored, least value of y, solve bowling ball weight, find min froods}

    \item \textbf{Date:} \textit{get today date, calculate one week ago, calculate n days from future date, calculate n days from date in format, calculate date days ago, calculate n months from date, calculate one week from today, calculate date after event, find palindrome day, calculate date a month ago, calculate date after days and months, calculate relative date, calculate n days from reference, calculate one year ago from today, calculate n hours from date, calculate date n days from, get date today, calculate date 10 days ago from deadline, calculate n weeks from date, \dots, calculate n units from date, calculate n years from date, calculate n weeks from first weekday of year, calculate today from tomorrow, find special day, calculate date 10 days ago from future, calculate n days after event, calculate date from days passed, calculate one week from christmas eve, calculate one year ago, calculate date 24 hours later, calculate n weeks from anniversary, calculate tomorrow from uk format date, calculate n days from date, is palindrome, calculate one week from first monday of year, calculate one week ago from anniversary}

    \item \textbf{TabMWP:} \textit{get frequency, calculate volleyballs in lockers, calculate total cost from package prices, calculate total items from group counts, calculate mode, calculate donation difference for person, count bags with 20 to 40 broken cookies, calculate total items from groups and items per group, count commutes of 50 minutes, get received amount, calculate total items for groups, find probability, calculate vacation cost, calculate rate of change, find received amount for transaction, calculate vote difference between two items for group, count customers, find minimum value in stem leaf, calculate metric wrenches, find smallest number, count books with 30 to 50 characters, \dots, count people with 67 pullups, calculate difference in donations for person, calculate total cost from unit price and weight, calculate total items from ratio, calculate total cost from unit weight prices and weight, calculate donation difference between causes, calculate difference, calculate net income, calculate grasshoppers on twigs, count total members in group, calculate expenses on date, find lightest child, calculate difference in amounts, count votes for item from groups, calculate probability from count table, get table cell value, calculate jeans in hampers, count instances with specific value in stem leaf, calculate donation difference for person and causes, calculate total from frequency and additional count, calculate range, calculate total reviews}
\end{itemize}


\paragraph{REGAL:}
\begin{itemize}[noitemsep, topsep=0pt]
    \item \textbf{MATH:}  \textit{solve for largest side, apply function sequence, solve rational equation, calculate expression sum, max sum of products, find b for perpendicular bisector, vertex of quadratic, calculate work days, calculate c for zero coefficient, simplify and rationalize sympy, find a for binomial square, compound interest, calculate inverse variation, expand expression, calculate average speed, calculate rs, sum sequence, solve for p, max consecutive integers, find x intercept, day exceeding threshold, find smallest sum, solve for ac pair, constant function, sum of distances, evaluate expression, sum finite geometric series, factor expression, find common difference, total coins pirates, calculate geometric first term, calculate closest whole number, calculate x minus y squared, solve letter values, find circle center v2, evaluate expression with sqrt, calculate sum of equations, \dots, calculate x3 plus y3, find negative intervals, calculate floor and abs, solve quadratic and find min, calculate y, solve for a, check equations, rationalize and simplify, calculate xyz, calculate distance, solve for x in simplified equation, calculate expression, calculate exponent, sum arithmetic series, complete square form, calculate x2 plus y2
    }

    \item \textbf{Date:}  \textit{subtract weeks from date, add weeks to date, format date, add days to date, subtract months from date, subtract days from date, subtract years from date, calculate date, calculate days between weekdays}

    \item \textbf{TabMWP:}  \textit{count range, find mode, total participants, count bushes with fewer roses, find max frequency, total items, count in range, calculate total items, count below threshold, count teams with minimum size, calculate total, calculate range, calculate fraction, sum frequencies below threshold, sum frequencies, calculate difference, calculate median, total outcomes, count specific height, count numbers in range, difference between groups, access frequency, calculate proportionality constant, count values below threshold, find median, calculate probability, calculate mode, get frequency, convert stem leaf to numbers, find minimum, get total items, count scores above, rate of change, calculate mean}
\end{itemize}



\subsection{The tool graph evolution diagrams of {\ours} for various tasks.}
\label{subsec:tool-graph}
Below are the tool graph evolution diagrams for various tasks. The Date task does not have a tool network evolution diagram, as date reasoning does not heavily rely on tool diversity.


\begin{figure}[H]
\vskip 0.2in
\begin{center}
\centerline{\includegraphics[width=1\linewidth]{toolnet-trial1.png}}
% \vskip -0.2in
\caption{
The tool graph evolution diagram for Minecraft Trial 1. In this diagram, each node represents a tool function, and the edges represent the invocation relationships between tools. The darker the color, the more frequently the tool is invoked. The network consists of a total of 6 layers, with layers 2 to 6 shown here from top to bottom.}
\label{fig:toolnet1}
\end{center}
\vskip -0.3in
\end{figure}

\vskip -0.2in
\begin{figure}[H]
\vskip 0.2in
\begin{center}
\centerline{\includegraphics[width=1\linewidth]{toolnet-trial2.png}}
% \vskip -0.2in
\caption{The tool graph evolution diagram for Minecraft Trial 2. In this diagram, each node represents a tool function, and the edges represent the invocation relationships between tools. The darker the color, the more frequently the tool is invoked. The network consists of a total of 6 layers, with layers 2 to 6 shown here from top to bottom.}
\label{fig:toolnet2}
\end{center}
\vskip -0.3in
\end{figure}

\vskip -0.2in
\begin{figure}[H]
\vskip 0.2in
\begin{center}
\centerline{\includegraphics[width=1\linewidth]{toolnet-trial3.png}}
% \vskip -0.2in
\caption{The tool graph evolution diagram for Minecraft Trial 3. In this diagram, each node represents a tool function, and the edges represent the invocation relationships between tools. The darker the color, the more frequently the tool is invoked. The network consists of a total of 6 layers, with layers 2 to 7 shown here from top to bottom.}
\label{fig:toolnet3}
\end{center}
\vskip -0.3in
\end{figure}


\begin{figure}[H]
\vskip 0.2in
\begin{center}
\centerline{\includegraphics[width=1\linewidth]{toolnet-dabench.png}}
% \vskip -0.2in
\caption{The tool graph evolution diagram of DA-Bench. In this diagram, each node represents a tool function, and the edges represent the invocation relationships between tools. The darker the color, the more frequently the tool is invoked.}
\label{fig:toolnet-dabench}
\end{center}
\vskip -0.3in
\end{figure}

\begin{figure}[H]
\vskip 0.2in
\begin{center}
\centerline{\includegraphics[width=1\linewidth]{toolnet-textcraft.png}}
% \vskip -0.2in
\caption{The tool graph evolution diagram of TextCraft. In this diagram, each node represents a tool function, and the edges represent the invocation relationships between tools. The darker the color, the more frequently the tool is invoked.}
\label{fig:toolnet-textcraft}
\end{center}
\vskip -0.3in
\end{figure}


\begin{figure}[H]
\vskip 0.2in
\begin{center}
\centerline{\includegraphics[width=1\linewidth]{toolnet-math.png}}
% \vskip -0.2in
\caption{The tool graph evolution diagram of MATH. In this diagram, each node represents a tool function, and the edges represent the invocation relationships between tools. The darker the color, the more frequently the tool is invoked.}
\label{fig:toolnet-math}
\end{center}
\vskip -0.3in
\end{figure}

\begin{figure}[H]
\vskip 0.2in
\begin{center}
\centerline{\includegraphics[width=1\linewidth]{toolnet-tabmwp.png}}
% \vskip -0.2in
\caption{The tool graph evolution diagram of TabMWP. In this diagram, each node represents a tool function, and the edges represent the invocation relationships between tools. The darker the color, the more frequently the tool is invoked.}
\label{fig:toolnet-tabmwp}
\end{center}
\vskip -0.3in
\end{figure}
\section{Prompt Template}
\label{app:prompt}
In this section, we provide the prompt templates of different types used throughout our experiment. These prompts were carefully crafted to ensure that the model's output aligns with the specific objectives of each task.

\subsection{Construction Stage}
In open-ended task online training, we made slight modifications to their prompts based on Voyager~\citep{wang2023voyager}. For close-ended tasks, the prompts used during the construction process are as follows:
\begin{tcolorbox}[title=Task Solver's Prompt, breakable, width=\textwidth,top=0mm]
\begin{Verbatim}[breaklines, fontsize=\footnotesize]
# Instruction #
You are the Task Solver in a collaborative team, specializing in reasoning and Python programming. Your role is to analyze tasks, collaborate with the Tool Manager, and solve problems step by step.
Directly solving tasks without tool analysis is not allowed. Request necessary tools before proceeding when needed, based on the task analysis.

# WORKFLOW #
You can decide which step to take based on the environment and current situation, adapting dynamically as the task progresses.
Stage 1. Tool Requests:
    Requesting tool is mandatory. Request generalized and reusable tools to solve the task. Focus on abstract functionality rather than task-specific details to enhance flexibility and adaptability.
Stage 2. Code and Interact: 
    Write notebook blocks incrementally, executing and interacting with the environment step by step. Avoid bundling all steps into a single block; instead, adjust dynamically based on feedback after each interaction.
Stage 3: Validate and Conclude: 
    When confident in the solution, review your work, validate the results, and conclude the task.

# Custom Library #
===api===

# NOTICE #
1. You must fully understand the action space and its parameters before using it.
2. If code execution fails, you should analyze the error and try to resolve it. If you find that the error is caused by the API, please promptly report the error information to the Tool Manager.
3. Regardless of how simple the issue may seem, you should always aim to summarize and refine the tool requirements.


# Tool Request Guidelines #
1. Keep It Simple: Design tools with single and simple functionality to ensure they are easy to implement, understand, and use. Avoid unnecessary complexity.
2. Define Purpose: Clearly outline the tool’s role within broader workflows. Focus on creating reusable tools that solve abstract problems rather than task-specific ones.
3. Specify Input and Output: Define the required input and expected output formats, prioritizing generic structures (e.g., dictionaries or lists) to enhance flexibility and adaptability.
4. Generalize Functionality: Ensure the tool is not tied to a specific task. Abstract its functionality to make it applicable to similar problems in other contexts.


# ACTION SPACE #
You should Only take One action below in one RESPONSE:
## NotebookBlock Action
* Signature: 
NotebookBlock():
```python
executable python script
```
* Description: The NotebookBlock action allows you to create and execute a Jupyter Notebook cell. The action will add a code block to the notebook with the content wrapped inside the paired ``` symbols. If the block already exists, it can be overwritten based on the specified conditions (e.g., execution errors). Once added or replaced, the block will be executed immediately.
* Restrictions: Only one notebook block can be managed or executed per action.
* Example
- Example1: 
NotebookBlock():
```python
# Calculate the area of a circle with a radius of 5
radius = 5
area = 3.1416 * radius ** 2
print(area)
```

## Tool_request Action
* Signature:
{
    "action_name": "tool_request",
    "argument": {
         "request": [
             ...
         ]
    }
}
* Description: The Tool Request Action allows you to send tool requirements to the Tool Manager and request it to create appropriate tools. You need to provide the action in a JSON format, where the argument field contains a request parameter that accepts a list. Each element in the list is a string describing the desired tool.
* Note:
* Examples:
- Example 1:
{
    "action_name": "tool_request",
    "argument": {
        "request": [
            "I need a tool that calculates the average value of a specified column in a dataset. The input should include the column name."
        ]
    }
}
- Example 2:
{
    "action_name": "tool_request",
    "argument": {
        "request": [
            "I need a tool that filters rows in a dataset based on a specified condition. The input should include the column name and the condition to filter by."
        ]
    }
}


## Terminate Action
* Signature: Terminate(result=the result of the task)
* Description: The Terminate action ends the process and provides the task result. The `result` argument contains the outcome or status of task completion.
* Examples:
  - Example1: Terminate(result="A")
  - Example2: Terminate(result="1.23")

# RESPONSE FORMAT #
For each task input, your response should contain:
1. One RESPONSE should only contain One Stage, One Thought and One Action.
2. An current phase of task completion, outlining the steps from planning to review, ensuring progress and adherence to the workflow.  (prefix "Stage: ").
3. An analysis of the task and the current environment, including reasoning to determine the next action based on your role as a SolvingAgent. (prefix "Thought: ").
4. An action from the **ACTION SPACE** (prefix "Action: "). Specify the action and its parameters for this step.

# RESPONSE EXAMPLE #
Observation: ...(the output of last actions, as provided by the environment and the code output, you don't need to generate it)

Stage:...(One Stage from `WORKFLOW`)
Thought: ...
Action: ...(Use an action from the ACTION SPACE no more than once per response.)

# TASK #
===task===
\end{Verbatim}
\end{tcolorbox}

\begin{tcolorbox}[title=Tool Manager's Prompt, breakable, width=\textwidth,top=0mm]
\begin{Verbatim}[breaklines, fontsize=\footnotesize]
# Instruction #
You are a Tool Manager in a collaborative team, specializing in assembling existing APIs to construct hierarchical and reusable abstract tools based on predefined criteria.
You will be provided with a custom library, similar to Python’s built-in modules, containing various functions related to date reasoning. For each task, you will receive:
1. Tool request: The specific goal or functionality the new tool must achieve.
2. Existing tools: A list of available functions from the custom library that you can utilize.
Your task is to analyze the given request and create a reusable tool by effectively leveraging the relevant functions from the existing tools or utilizing basic tools to achieve the desired functionality. 
If an existing tool from the provided library already fully satisfies the requirements, simply return that tool instead of duplicating functionality. Ensure all responses align with reusability and efficiency principles.

# Custom Library #
===api===

# Creation Criteria #
- **Reusability**: The function could be resued for more complex function.
- **Innovation**: Tools should offer innovation, not merely wrap or replicate existing APIs. Simply re-calling an API without significant enhancements does not qualify as innovation.
- **Completeness**: The function should handle potential edge cases to ensure completeness.
- **Leveraging Existing Functions**: The function should effectively utilize existing functions to enhance efficiency and avoid redundancy.
- **Functionality**: Ensure the tool runs successfully and is bug-free, guaranteeing full functionality.

# ACTION SPACE #
You should Only take One action below in one RESPONSE:
## Create tool Action
* Description: The Create Tool action allows you to develop a new tool and temporarily store it in a private repository accessible only to you. Each invocation creates a single tool at a time. You can repeatedly use this action to build smaller components, which can later be assembled into the final tool.
* Signature: 
Create_tool(tool_name=The name of the tool you want to create):
```python
The source code of tool
```
* Example:
Create_tool(tool_name=“calculate_column_statistics”):
```python
def calculate_column_statistics(dataset: pd.DataFrame, column_name: str) -> Dict[str, float]:
    """
    Calculates basic statistics (mean, median, standard deviation) for a specified column in a dataset.
    Parameters:
    - dataset: A pandas DataFrame containing the data.
    - column_name: The name of the column to calculate statistics for.
    Returns:
    - A dictionary containing the mean, median, and standard deviation of the column.
    """
    if column_name not in dataset.columns:
        raise ValueError(f"Column '{column_name}' not found in the dataset.")
    
    column_data = dataset[column_name]
    stats = {
        "mean": column_data.mean(),
        "median": column_data.median(),
        "std_dev": column_data.std()
    }
    return stats
```
## Edit tool Action
* Description: The Edit Tool action allows you to modify an existing tool and temporarily store it in a private repository that only you can access. You must provide the name of the tool to be updated along with the complete, revised code. Please note that only one tool can be edited at a time.
* Signature: 
Edit_tool(tool_name=The name of the tool you want to create):
```python
The edited source code of tool
```
* Examples:
Edit_tool(tool_name="filter_rows_by_condition"):
```python
def filter_rows_by_condition(dataset: pd.DataFrame, column_name: str, condition: str) -> pd.DataFrame:
    """
    Filters rows in a dataset based on a specified condition for a given column.
    Parameters:
    - dataset: A pandas DataFrame containing the data.
    - column_name: The name of the column to apply the condition to.
    - condition: A string representing the condition, e.g., 'value > 10'.
    Returns:
    - A filtered DataFrame containing only the rows that satisfy the condition.
    """
    if column_name not in dataset.columns:
        raise ValueError(f"Column '{column_name}' not found in the dataset.")
    
    try:
        filtered_dataset = dataset.query(f"{column_name} {condition}")
    except Exception as e:
        raise ValueError(f"Invalid condition: {condition}. Error: {e}")
    
    return filtered_dataset
```

# RESPONSE FORMAT #
For each task input, your response should contain:
1. Each response should contain only one "Thought," and one "Action."
2. Determine how to construct your tool to meet tool request and function creation criteria. Check if any functions in the Existing Tool can be invoked to assist in the tool’s development and ensure alignment with the criteria.(prefix "Thought: ").
3. An action dict from the **ACTION SPACE** (prefix "Action: "). Specify the action and its parameters for this step. 

# RESPONSE EXAMPLE  #
1. If you determine that the tool request cannot be solved using existing tools, choose this mode to provide a clear and complete code solution.

Thought: ...
Action: ...

2. If you determine that the tool request is already satisfied by existing tools, choose this mode to directly reference and return the relevant tool without creating additional solutions.
Thought: ...
Tool: {  
    "tool_name": "Name of Existing tools"
}

# NOTICE #
1. You can directly call and use the tool in the custom library in your code or tool without importing it.
2. You can only create or edit one tool per response, so take it one step at a time.

# TASK #
===task===
\end{Verbatim}
\end{tcolorbox}


\begin{tcolorbox}[title=Prompt of Self-Check Step 1, breakable, width=\textwidth,top=0mm]
\begin{Verbatim}[breaklines, fontsize=\footnotesize]
# Instruction #
You are evaluating whether the tools provided by the Tool Manager meet the required standards. 
You follow a defined workflow, take actions from the ACTION SPACE, and apply the evaluation criteria. 

# Evaluation Criteria #
- **Reusability**: The function should be designed for reuse in more complex scenarios. For instance, in the case of the `craft_wooden_sword()` tool, it would be more versatile if it could accept a quantity as an input parameter.
- **Innovation**: Tools should offer innovation, not merely wrap or replicate existing APIs. Simply re-calling an API without significant enhancements does not qualify as innovation. If an existing tool from the provided library already fully satisfies the requirements, simply return that tool instead of duplicating functionality. Ensure all responses align with reusability and efficiency principles.
- **Completeness**: The function should handle potential edge cases to ensure completeness.
- **Leveraging Existing Functions**: Check if any function in "Existing Function" is helpful for completing the task. If such functions exist but are not invoked in the provided code, relevant feedback should be given.

## Tool Abstraction ##
Tool abstraction is essential for enabling tools to adapt to diverse tasks. Key principles include:
- Design generic functions to handle queries of the same type, based on shared reasoning steps, avoiding specific object names or terms.
- Name functions and write docstrings to reflect the core reasoning pattern and data organization, without referencing specific objects.
- Use general variable names and pass all column names as arguments to enhance adaptability.

# ACTION SPACE #
You should Only take One action below in one RESPONSE:
# Feedback Action
* Signature: {
    "action_name": "Feedback",
    "argument": {
        "feedback": ...
        "passed": true/false
    }
}
* Description: The Feedback Action is represented as a JSON string that provides feedback to the Tool Manager or SolvingAgent. The feedback field contains comments or suggestions, while pass indicates whether the tool meets the requirements (true for approval, false for rejection). Feedback should be concise, constructive, and relevant. If pass is true, the feedback can be left empty; otherwise, it must be provided.
* Example:
- Example1:
{
    "action_name": "Feedback",
    "argument": {
        "feedback": "",
        "passed": true
    }
}
- Example2:
{
    "action_name": "Feedback",
    "argument": {
        "feedback": "The tool correctly solves the equation for small numbers, but fails when the coefficients are very large. Consider optimizing the algorithm for handling larger values and improving computational efficiency.",
        "passed": false
    }
}

# RESPONSE FORMAT #
For each task input, your response should contain:
1. One RESPONSE should ONLY contain One Thought and One Action.
2. An comprehensive analysis of the tool code based on the evaluation criteria.(prefix "Thought: ").
3. An action from the **ACTION SPACE** (prefix "Action: "). 

# EXAMPLE RESPONSE #
Observation: ...(output from the last action, provided by the environment and task input, no need for you to generate it)

Thought: 1. Reusability: ...
2. Innovation: ...
3. Completeness: ...
4. Leveraging Existing Functions: ...

Action: ...(Use an action from the ACTION SPACE once per response.)

# Custom Library #
===api===

# TASK #
===task===
\end{Verbatim}
\end{tcolorbox}

\begin{tcolorbox}[title=Prompt of Self-Check Step 2, breakable, width=\textwidth,top=0mm]
\begin{Verbatim}[breaklines, fontsize=\footnotesize]
# Instruction #
You are verifying whether the tools provided by the Tool Manager execute without runtime errors.
You will use a custom library, similar to the built-in library, which provides everything necessary for the tasks. Your task is only to execute the provided tool code and check for runtime errors, not to evaluate the tool’s functionality or correctness.

# Stage and Workflow #
1. **Ensure Bug-Free Tool Operation**:
	- Execute the tool to ensure it runs without any runtime bugs.
	- You don’t need to verify the function’s functionality; simply call it to check for any runtime errors.
	- If the tool is a retrieved API, skip this step and proceed.
2. **Send Feedback**:
	- After executing the code, provide feedback based on the output, indicating whether the operation was successful or not.

# Notice #
1. If any issues with the tool are found, promptly provide clear and critical feedback to the Tool Manager for resolution. 
2. You should not create or edit functions (tools) with the same name as the Existing Functions in the code.
3. You can directly call the APIs from the custom library without needing to import or declare any external libraries.
4. You don’t need to verify the function’s functionality or set up its standard output; simply call it to check for any errors.

# ACTION SPACE #
You should Only take One action below in one RESPONSE:
## Python Action
* Signature: 
Python(file_path=python_file):
```python
executable_python_code
```
* Description: The Python action will create a python file in the field `file_path` with the content wrapped by paired ``` symbols. If the file already exists, it will be overwritten. After creating the file, the python file will be executed. Remember You can only create one python file.
* Examples:
- Example1
Python(file_path="solution.py"):
```python
# Calculate the area of a circle with a radius of 5
radius = 5
area = 3.1416 * radius ** 2
print(f"The area of the circle is {area} square units.")
```
- Example2
Python(file_path="solution.py"):
```python
# Calculate the perimeter of a rectangle with length 8 and width 3
length = 8
width = 3
perimeter = 2 * (length + width)
print(f"The perimeter of the rectangle is {perimeter} units.")
```

# Feedback Action
* Signature: {
    "action_name": "Feedback",
    "argument": {
        "feedback": ...
        "passed": true/false
    }
}
* Description: The Feedback Action is used to provide feedback to the Tool Manager. The feedback field contains detailed comments or suggestions. If the tool encounters an error, you should set passed to false and provide a detailed feedback. If the tool runs without errors, you can set passed to true and leave feedback as an empty string.
* Examples:
- Example 1:
{
    "action_name": "Feedback",
    "argument": {
        "feedback": ""
        "passed": true
    }
}
- Example 2:
{
    "action_name": "Feedback",
    "argument": {
        "feedback": "The tool encountered an error while executing. The variable 'height' is missing in the function call. Please ensure that all required parameters are provided.",
        "passed": false
    }
}

# RESPONSE FORMAT #
For each task input, your response should contain:
1. One RESPONSE should ONLY contain One Thought and One Action.
2. An analysis of the task and current environment, reasoning through the next evaluation step based on your role as CheckingAgent.(prefix "Thought: ").
3. An action from the **ACTION SPACE** (prefix "Action: "). Specify the action and its parameters for this step.

# EXAMPLE RESPONSE #
Observation: ...(output from the last action, provided by the environment and task input, no need for you to generate it)

Thought: ...
Action: ...(Use an action from the ACTION SPACE once per response.)

# Custom Library #
You can use pandas, sklearn, or other Python libraries as part of the custom library.

* Note: You can directly call these tools without importing or redefining them in your code.

Let's think step by step.
# TASK #
===task===
\end{Verbatim}
\end{tcolorbox}

\subsection{Test Stage}
\label{appsub:test_prompt}
During the test stage, the prompts used for different datasets are as follows:
\begin{tcolorbox}[title=Prompt on DABench, breakable, width=\textwidth,top=0mm]
\begin{Verbatim}[breaklines, fontsize=\footnotesize]
# Instruction #
You are a helpful assistant, skilled in data science tasks.
You will be provided with a task description and related files. 
You should complete tasks by writing notebook code to interact with the environment containing the task files.
Additionally, you must strictly adhere to the task constraints. 
Once the task is completed, you need to format the answer as specified in the answer format and invoke the Terminate action to conclude.
You should use actions from the ACTION SPACE, follow the Response Format, and complete the task within 20 steps.

You may also leverage the following helper functions if needed.
===api===


===example===


# Response Format #
Your each response should contain:
1. One RESPONSE should only contain ONLY One Thought and ONLY One Action.
2. Only an analysis of the task and the current environment, including reasoning to determine the next action. (prefix "Thought: ").
3. Only an action from the **ACTION SPACE** (prefix "Action: "). Specify the action and its parameters for this step.

Observation: ...(Provided by the environment, no need for you to generate it.))

Thought: ...
Action: ...

# ACTION SPACE #
## NotebookBlock Action
* Signature: 
NotebookBlock():
```python
executable python script
```
* Description: The NotebookBlock action allows you to create and execute a Jupyter Notebook cell. The action will add a code block to the notebook with the content wrapped inside the paired ``` symbols. If the block already exists, it can be overwritten based on the specified conditions (e.g., execution errors). Once added or replaced, the block will be executed immediately.
* Restrictions: Each response must contain only one notebook block.
* Note: In a single block, you may call multiple tools or single.
* Example:
Action: NotebookBlock():
```python
# Calculate the area of a circle with a radius of 5
radius = 5
area = 3.1416 * radius ** 2
print(area)
```

# Terminate Action
* Signature: Terminate(result="the result of the task")
* Description: The Terminate action marks the completion of a task and presents the final result. It is a formatting guideline, not an executable Python function. The result parameter must contain a clear, specific answer that strictly complies with the task’s output format, with all required values explicitly provided.
Tips:
    - Ensure the result parameter provides a definite and concrete final answer.
    - Do not include unresolved Python expressions, placeholders, or variables (e.g., @value[{x + y}] or @result[{variable_name}] or "@result[{variable_name}]".format(variable_name)).
    - The output must adhere precisely to the task’s formatting specifications, ensuring clarity and consistency.
* Examples:
- Example 1: 
Answer Format: @shapiro_wilk_statistic[test_statistic] @shapiro_wilk_p_value[p_value]
Action: Terminate(result="@shapiro_wilk_statistic[0.56] @shapiro_wilk_p_value[0.0002]")
- Example 2: 
Answer Format: @total_votes_outliers_num[outlier_num]
where "outlier_num" is an integer representing the number of values considered outliers in the 'total_votes' column.
Action: Terminate(result="@total_votes_outliers[10]")
- Example3:
Action: Terminate(result="@normality_test_result[Not Normal] @p_value[0.000]")

## Response Example
Here are four examples of responses.
## Example1
Thought: The dataset has been loaded successfully and it contains the "Close Price" column. Now, we need to calculate the mean of the "Close Price" column using pandas.
Action: NotebookBlock():
```python
# Calculate the mean of the "Close Price" column
mean_close_price = data["Close Price"].mean()
# Round the result to two decimal places
mean_close_price_rounded = round(mean_close_price, 2)
print(mean_close_price_rounded)
```
## Example2
Thought: We need to filter the dataset to only include rows where the “Volume” is greater than 100,000. This will help focus on high-volume trades.
Action: NotebookBlock():
```python
# Filter rows where "Volume" is greater than 100,000
filtered_data = data[data["Volume"] > 100000]
# Display the filtered dataset
print(filtered_data)
```
## Example3
Thought: To analyze the correlation between “Open Price” and “Close Price,” we will calculate the Pearson correlation coefficient using pandas.
Action: NotebookBlock():
```python
# Calculate the correlation between "Open Price" and "Close Price"
correlation = data["Open Price"].corr(data["Close Price"])
# Print the correlation result
print(correlation)
```
## Example4
Thought: To check for missing values in the dataset, we need to check for null values in each column using pandas.
Action: NotebookBlock():
```python
# Check for missing values in each column
missing_values = data.isnull().sum()
# Display the result
print(missing_values)
```

# Begin #
Let's Begin.
## Task 
===task===
\end{Verbatim}
\end{tcolorbox}


\begin{tcolorbox}[title=Prompt on TextCraft, breakable, width=\textwidth,top=0mm]
\begin{Verbatim}[breaklines, fontsize=\footnotesize]
# Instruction #
You are provided with a set of useful crafting recipes to create items in Minecraft.
Crafting commands follow the format: "craft [target object] using [input ingredients]".
You can either "fetch" an object (ingredient) from the inventory or the environment or "craft" the target object using the provided crafting commands.
You are allowed to use only the crafting commands provided; do not invent or use your own crafting commands.
If a crafting command specifies a generic ingredient, such as "planks", you can substitute it with a specific type of that ingredient (e.g., “dark oak planks”).
To complete the crafting tasks, you will write notebook code utilizing tools from the "Custom Library". You should carefully read and understand the tool’s docstrings and code to fully grasp their functionality and usage.
The tools should be invoked by outputting a block of Python code. Additionally, you may incorporate Python constructs such as for-loops, if-statements, and other logic where necessary.
Please always use actions from the ACTION SPACE and follow the Response Format.


# ACTION SPACE #
## NotebookBlock Action
* Signature: 
NotebookBlock():
```python
executable python script
```
* Description: The NotebookBlock action creates and executes a Jupyter Notebook cell. It adds a code block wrapped in ``` symbols, overwriting existing blocks if specified (e.g., after execution errors). The block is executed immediately after being added or replaced.
* Note: In a single block, you may call multiple tools.

## Terminate Action
* Signature: Terminate(result=the result of the task)
* Description: The Terminate action ends the process and provides the task result. The `result` argument contains the outcome or status of task completion. Only the CheckingAgent has the authority to decide whether a task is finished.
* Examples:
  - Example1: Action: Terminate(result="3")
  - Example2: Action: Terminate(result="Successfully craft 2 oak planks")
  - Example3: Action: Terminate(result="Successfully craft 1 milk")


# Response Format #
For each task input, your response should contain:
1. One RESPONSE should only contain ONLY One Thought and ONLY One Action.
2. An analysis of the task and the current environment, including reasoning to determine the next action. (prefix "Thought: ").
3. An action from the **ACTION SPACE** (prefix "Action: "). Specify the action and its parameters for this step.

## Response Example
### Example1
Thought: To craft a crimson fence, I need to use the crafting command: “craft 3 crimson fence using 2 stick, 4 crimson planks.” This means I need two ingredients: 2 sticks and 4 crimson planks. To obtain 4 crimson planks, I can use the command “craft 4 crimson planks using 1 crimson stems,” so I need to fetch 1 crimson stem first. For the sticks, I can use the command “craft 4 stick using 2 planks.” Since the crafting of sticks requires planks, I can directly use the crimson planks I craft earlier for this purpose.
Steps: 
1. Fetch 1 crimson stems.
2. Craft 4 crimson planks using 1 crimson stems.
3. Craft 4 sticks using 2 crimson planks.
4. Craft 3 crimson fences using 2 sticks and 4 crimson planks.
Action: NotebookBlock()
```python
get_object('1 crimson stems')
```
### Example2
Thought: The inventory is currently empty, which means I need to retrieve the necessary spruce logs from the environment. Since we need spruce planks to craft the spruce trapdoor, and each spruce trapdoor requires 6 spruce planks, I should start by fetching an adequate number of spruce logs. Since each spruce log can craft 4 spruce planks, we need at least 2 spruce logs to ensure we can craft 8 spruce planks (more than enough for 2 spruce trapdoors if needed in the future).
Action: NotebookBlock()
```python
# Step 2: Fetch 2 spruce logs from the environment
get_object('2 spruce logs')
```
### Example3
Thought: We have successfully retrieved 2 spruce logs from the environment. 
Action: Terminate(result="Successfully crafted 8 spruce planks")


# Custom Library #
### Tool `check_inventory`
Source Code:
```python
def check_inventory() -> str:
    """
    Retrieves the current inventory state from the environment.
    The function sends an 'inventory' command to the environment
    and processes the observation to return a string representation
    of the inventory, listing items and their quantities.
    Returns:
        str: A string describing the inventory in the format:
             "Inventory: [item_name] (quantity)"
    """
    obs, _ = step('inventory')
    return obs
```
Usage Example:
```python
check_inventory() 
# If the environment has no items, Output: Inventory: You are not carrying anything.
# If the environment contains 2 oak planks, Output: Inventory: [oak planks] (2)
```
### Tool `get_object`
Source Code:
```python
def get_object(target: str) -> None:
    """
    Retrieves an item from the environment.

    The function prints the response message from the environment, 
    indicating whether the retrieval was successful or not.

    Args:
        target (str): The name of the item to be retrieved.

    Returns:
        None
    """
    obs, _ = step("get " + target)
    print(obs)
```
Usage Example:
Craft Command:
craft 2 yellow dye using 1 sunflower
craft 8 yellow carpet using 8 white carpet, 1 yellow dye
```python
get_object("1 sunflower") # Ouput: Got 1 sunflower
get_object("2 sunflower") # Ouput: Got 2 sunflower
# Note: You cannot retrieve yellow dye directly from the environment; it must first be crafted using sunflowers.
get_object("1 yellow dye") # Output: Could not find yellow dye
```
### Tool `craft_object`
Source Code:
```python
def craft_object(target: str, ingredients: List[str]) -> None:
    """
    Crafts a specified item using the given ingredients.

    This function's `target` and `ingredients` parameters correspond to the craft command: 
    "Craft 'target' using [ingredients]".
    
    **Note:** The `ingredients` must exactly match the command format. For example, if the command requires 
    '1 oak logs', providing '1 oak log' instead will not be recognized.

    Prints the environment's response to indicate whether the crafting operation was successful or not.

    Args:
        target (str): The item to craft along with its quantity (e.g., '4 oak planks').
        ingredients (List[str]): A list of required ingredients with their respective quantities 
                                (e.g., ['1 oak logs']).

    Returns:
        None

    """
    obs, _ = step("craft " + target + " using " + ", ".join(ingredients))
    print(obs)
```
Usage Example:
- Example1
Query: Craft 1 black dye using 1 wither rose
Craft Command:
craft 1 black dye using 1 wither rose
craft 1 black dye using 1 ink sac

```python
get_object('1 wither rose')
craft_object('1 black dye', ['1 wither rose'])
# Ouput:
# Got 1 wither rose
# Crafted 1 minecraft:black dye
```
- Example2
Craft Commands: 4 oak planks using 1 oak logs
```python
get_object("1 oak logs") # Got 1 oak logs
check_inventory() # Inventory: ["oak logs"] (1)
craft_object('4 oak planks', ['1 oak logs'])
# Crafted 4 minecraft:oak planks
```

===api===


===example===

# Example #
Here is a demo of how to fetch and craft objects.
Crafting commands:
craft 3 dark oak sign using 6 dark oak planks, 1 stick
craft 4 dark oak planks using 1 dark oak logs
craft 1 stick using 1 planks
craft 4 stick using 2 bamboo
craft 4 oak planks using 1 oak logs
craft 1 dark oak fence using 2 stick, 4 dark oak planks
craft 1 warped stairs using 6 warped planks
craft 3 oak sign using 6 oak planks, 1 stick

craft 2 oak planks

Thought: To craft 2 oak planks, I need to first check if I can fetch them directly from the environment or my inventory. If they are not available, I will craft them using the appropriate crafting recipe. Based on the crafting commands, I know that I can craft 4 oak planks using 1 oak logs. My plan is as follows:
1. Check if 2 oak planks are already in my inventory or can be fetched from the environment.
2. If not, verify if I have the required ingredient (1 oak logs).
3. Gather the missing ingredient, if necessary.
4. Craft 4 oak planks and ensure 2 are available for the task.
Action: NotebookBlock():
```python
check_inventory()
```

Observation: Inventory: [stick] (1) 

Thought: I could not find 2 oak planks or the required ingredient (1 oak logs) in my inventory. I need to fetch 1 oak logs first.
Action: NotebookBlock():
```
get_object("1 oak logs")
```

Observation: Got 1 oak logs

Thought: I now have the required ingredient to craft oak planks. I will use the crafting command: craft 4 oak planks using 1 oak logs.
Action: NotebookBlock():
```
craft_object("4 oak planks", ["1 oak logs"])
check_inventory()
```

Observation: Crafted 4 minecraft:oak_planks
Inventory: [stick] (1) [oak_planks] (4)

Thought: I now have 2 oak planks in my inventory. Task Completed!
Action: Terminate(result=“Successfully craft 2 oak planks”)


# Begin #
Let's Begin. Please follow the Instruction and focus on solve the task.
# Current items in the inventory #
These are the items currently in your inventory. You cannot use items that are not in your inventory to craft new items.
===inventory===

# Task #
===task===
\end{Verbatim}
\end{tcolorbox}

\begin{tcolorbox}[title=Prompt on MATH, breakable, width=\textwidth,top=0mm]
\begin{Verbatim}[breaklines, fontsize=\footnotesize]
Your task is to solve math competition problems by writing Python programs.

You may also leverage the following helper functions, but must avoid fabricating and calling undefined function names.
```python
===api===
```

Examples: 

Examples: 
Query: Point $P$ lies on the line $x= -3$ and is 10 units from the point $(5,2)$. Find the product of all possible $y$-coordinates that satisfy the given conditions.
Program: 
```python
from sympy import symbols, Eq, solve
# Define symbolic variable for y-coordinate of point P
y = symbols('y')
# Step 1: Given conditions
x1 = -3  # Point P lies on the vertical line x = -3
x2, y2 = 5, 2  # Coordinates of the given point (5, 2)
d = 10  # Distance between point P and (5,2)
# Step 2: Apply the distance formula
# Distance formula: sqrt((x2 - x1)^2 + (y - y2)^2) = d
# Squaring both sides to eliminate the square root:
# (x2 - x1)^2 + (y - y2)^2 = d^2
distance_equation = Eq((x2 - x1)**2 + (y - y2)**2, d**2)
# Step 3: Solve for possible values of y
y_solutions = solve(distance_equation, y)
# Step 4: Compute the product of all possible y-values
product = y_solutions[0] * y_solutions[1]
# Step 5: Output the final result
print("Final Answer:", product)
```

Query: If $3p+4q=8$ and $4p+3q=13$, what is $q$ equal to?
Program:
```python
from sympy import symbols, Eq, solve
# Define symbolic variables for the unknowns p and q
p, q = symbols('p q')
# Step 1: Define the given system of equations
eq1 = Eq(3 * p + 4 * q, 8)  # Equation 1: 3p + 4q = 8
eq2 = Eq(4 * p + 3 * q, 13)  # Equation 2: 4p + 3q = 13
# Step 2: Solve the system of equations for p and q
solution = solve((eq1, eq2), (p, q))
# Step 3: Extract and output the value of q
print("Final Answer:", solution[q])
```

Query: Simplify $\frac{3^4+3^2}{3^3-3}$. Express your answer as a common fraction.
Program:
```python
from sympy import symbols, simplify
# Define the variable
x = symbols('x')
# Define the expression
numerator = 3**4 + 3**2
denominator = 3**3 - 3
fraction = numerator / denominator
# Simplify the fraction
simplified_fraction = simplify(fraction)
# Print the result
print("Final Answer:", simplified_fraction)
```

===example===

## Begin !
Please generate ONLY the code wrapped in ```python...``` to solve the query below.

Query: ===task===
Program:
\end{Verbatim}
\end{tcolorbox}



\begin{tcolorbox}[title=Prompt on Date, breakable, width=\textwidth,top=0mm]
\begin{Verbatim}[breaklines, fontsize=\footnotesize]
Your task is to solve simple word problems by creating Python programs.

You may also leverage the following helper functions, but must avoid fabricating and calling undefined function names, such as `calculate_date_by_years`.
```python
===api===
```

Examples:

Query: In the US, Thanksgiving is on the fourth Thursday of November. Today is the US Thanksgiving of 2001. What is the date one week from today in MM/DD/YYYY?
Program:
```python
# import relevant packages
from datetime import date, time, datetime
from dateutil.relativedelta import relativedelta
import calendar
# 1. What is the date of the first Thursday of November? (independent, support: [])
date_1st_thu = date(2001,11,1)
while date_1st_thu.weekday() != calendar.THURSDAY:
    date_1st_thu += relativedelta(days=1)
# 2. How many days are there in a week? (independent, support: ["External knowledge: There are 7 days in a week."])
n_days_of_a_week = 7
# 3. What is the date today? (depends on 1 and 2, support: ["Today is the US Thanksgiving of 2001", "Thanksgiving is on the fourth Thursday of November"])
days_from_1st_to_4th_thu = (4-1) * n_days_of_a_week
date_today = date_1st_thu + relativedelta(days=days_from_1st_to_4th_thu)
# 4. What is the date one week from today? (depends on 3, support: [])
date_1week_from_today = date_today + relativedelta(weeks=1)
# 5. Final Answer: What is the date one week from today in MM/DD/YYYY? (depends on 4, support: [])
answer = date_1week_from_today.strftime("%m/%d/%Y")
# print the answer
print(answer)
```

Query: Yesterday was 12/31/1929. Today could not be 12/32/1929 because December has only 31 days. What is the date tomorrow in MM/DD/YYYY?
Program:
```python
# import relevant packages
from datetime import date, time, datetime
from dateutil.relativedelta import relativedelta
# 1. What is the date yesterday? (independent, support: ["Yesterday was 12/31/1929"])
date_yesterday = date(1929,12,31)
# 2. What is the date today? (depends on 1, support: ["Today could not be 12/32/1929 because December has only 31 days"])
date_today = date_yesterday + relativedelta(days=1)
# 3. What is the date tomorrow? (depends on 2, support: [])
date_tomorrow = date_today + relativedelta(days=1)
# 4. Final Answer: What is the date tomorrow in MM/DD/YYYY? (depends on 3, support: [])
answer = date_tomorrow.strftime("%m/%d/%Y")
# print the answer
print(answer)
```

Query: The day before yesterday was 11/23/1933. What is the date one week from today in MM/DD/YYYY?
Program:
```python
# import relevant packages
from datetime import date, time, datetime
from dateutil.relativedelta import relativedelta
# 1. What is the date the day before yesterday? (independent, support: ["The day before yesterday was 11/23/1933"])
date_day_before_yesterday = date(1933,11,23)
# 2. What is the date today? (depends on 1, support: [])
date_today = date_day_before_yesterday + relativedelta(days=2)
# 3. What is the date one week from today? (depends on 2, support: [])
date_1week_from_today = date_today + relativedelta(weeks=1)
# 4. Final Answer: What is the date one week from today in MM/DD/YYYY? (depends on 3, support: [])
answer = date_1week_from_today.strftime("%m/%d/%Y")
# print the answer
print(answer)
```

===example===

## Begin !
Please generate ONLY the code wrapped in ```python...``` to solve the query below.

Query: ===task===
Program:
\end{Verbatim}
\end{tcolorbox}



\begin{tcolorbox}[title=Prompt on TabMWP, breakable, width=\textwidth,top=0mm]
\begin{Verbatim}[breaklines, fontsize=\footnotesize]
Your task is to solve table-reasoning problems by writing Python programs.
You are given a table. The first row is the name for each column. Each column is seperated by "|" and each row is seperated by "\n".
Pay attention to the format of the table, and what the question asks.

You may also leverage the following helper functions, but must avoid fabricating and calling undefined function names.
```python
===api===
```


Examples: 
### Table
Name: None
Unit: $
Content:
Date | Description | Received | Expenses | Available Funds
 | Balance: end of July | | | $260.85
8/15 | tote bag | | $6.50 | $254.35
8/16 | farmers market | | $23.40 | $230.95
8/22 | paycheck | $58.65 | | $289.60
### Question
This is Akira's complete financial record for August. How much money did Akira receive on August 22?
### Solution code
```python
records = {
    "7/31": {"Description": "Balance: end of July", "Received": "", "Expenses": "", "Available Funds": 260.85},
    "8/15": {"Description": "tote bag", "Received": "", "Expenses": 6.5, "Available Funds": ""},
    "8/16": {"Description": "farmers market", "Received": "", "Expenses": 23.4, "Available Funds": ""},
    "8/22": {"Description": "paycheck", "Received": 58.65, "Expenses": "", "Available Funds": ""}
}
# Access the amount received on August 22
received_aug_22 = records["8/22"]["Received"]
print("Final Answer: ", received_aug_22)
```

### Table
Name: Orange candies per bag
Unit: bags
Content:
Stem | Leaf 
2 | 2, 3, 9
3 | 
4 | 
5 | 0, 6, 7, 9
6 | 0
7 | 1, 3, 9
8 | 5
### Question
A candy dispenser put various numbers of orange candies into bags. How many bags had at least 32 orange candies?
### Solution code
```python
data = {
    2: [2, 3, 9],
    3: [],
    4: [],
    5: [0, 6, 7, 9],
    6: [0],
    7: [1, 3, 9],
    8: [5]
}
# Initialize the count to zero
count = 0
# Iterate over the keys in the dictionary
for key in data:
    # Combine tenth digit and unit digit
    if key * 10 + data[key] >= 32:
        # Increment the count
        count += 1

# Output the result
print("Final Answer: ", count)
```

### Table
Name: Monthly Savings  
Unit: $  
Content:  
Date  | Description       | Received | Expenses | Available Funds |
       | Balance: end of May |   |   | $500.00 |
6/10  | groceries        |   | $45.75 | $454.25 |
6/15  | gas refill       |   | $30.20 | $424.05 |
6/25  | salary           | $1200.00 |   | $1624.05 |
### Question
How much money did Akira receive on June 25?
### Solution code
```python
import pandas as pd
records = {
    "5/31": {"Description": "Balance: end of May", "Received": "", "Expenses": "", "Available Funds": 500.00},
    "6/10": {"Description": "groceries", "Received": "", "Expenses": 45.75, "Available Funds": ""},
    "6/15": {"Description": "gas refill", "Received": "", "Expenses": 30.2, "Available Funds": ""},
    "6/25": {"Description": "salary", "Received": 1200.00, "Expenses": "", "Available Funds": ""}
}
# Access the amount received on June 25
received_june_25 = records["6/25"]["Received"]
print("Final Answer: ", received_june_25)
```

===example===

## Begin!
Please solve the task below and enclose your code within a single code block using ```python```  format.

===task===
### Solution code
\end{Verbatim}
\end{tcolorbox}









\section{Examples}
\label{app:example}
\subsection{Generated Tools}

\textbf{The tools generated for the Open-ended Tasks are as follows:}
\begin{tcolorbox}[title=CraftDiamondHelmet, width=\textwidth,top=0mm,  breakable]
\begin{Verbatim}[breaklines=true, breakanywhere=true, fontsize=\footnotesize]
async function craftDiamondHelmet(bot) {
  const mcData = require('minecraft-data')(bot.version);
  const Vec3 = require('vec3').Vec3;

  // Check inventory for the number of diamonds
  const diamondCount = bot.inventory.count(mcData.itemsByName["diamond"].id);
  const requiredDiamonds = 5;

  // If not enough diamonds, mine diamond ores
  if (diamondCount < requiredDiamonds) {
    const diamondsToMine = requiredDiamonds - diamondCount;
    bot.chat(`Need ${diamondsToMine} more diamond(s). Mining...`);
    await mineBlock(bot, "diamond_ore", diamondsToMine);
  }

  // Ensure a crafting table is placed nearby
  let craftingTable = bot.findBlock({
    matching: mcData.blocksByName.crafting_table.id,
    maxDistance: 32
  });
  if (!craftingTable) {
    bot.chat("Placing crafting table...");
    await placeItem(bot, "crafting_table", bot.entity.position.offset(1, 0, 0));
    craftingTable = bot.findBlock({
      matching: mcData.blocksByName.crafting_table.id,
      maxDistance: 32
    });
  }

  // Craft the diamond helmet
  bot.chat("Crafting diamond helmet...");
  await craftItem(bot, "diamond_helmet", 1);
  bot.chat("Diamond helmet crafted successfully.");
}
\end{Verbatim}
\end{tcolorbox}


\begin{tcolorbox}[title=CraftItemWithMaterials, width=\textwidth,top=0mm,  breakable]
\begin{Verbatim}[breaklines=true, breakanywhere=true, fontsize=\footnotesize]
async function craftItemWithMaterials(bot, itemName, requiredMaterials) {
  const mcData = require('minecraft-data')(bot.version);
  const Vec3 = require('vec3').Vec3;

  // Check inventory for required materials
  for (const material of requiredMaterials) {
    let itemCount = bot.inventory.count(mcData.itemsByName[material.name].id);
    if (itemCount < material.count) {
      const requiredCount = material.count - itemCount;
      bot.chat(`Need ${requiredCount} more ${material.name}(s).`);
      if (material.name === "diamond") {
        let diamondOre = await bot.findBlock({
          matching: mcData.blocksByName["diamond_ore"].id,
          maxDistance: 32
        });
        if (!diamondOre) {
          bot.chat("No diamond ore found nearby. Exploring...");
          diamondOre = await exploreUntil(bot, new Vec3(1, 0, 1), 60, () => {
            return bot.findBlock({
              matching: mcData.blocksByName["diamond_ore"].id,
              maxDistance: 32
            });
          });
        }
        if (diamondOre) {
          await mineBlock(bot, "diamond_ore", requiredCount);
        } else {
          bot.chat("Failed to find diamond ore after exploring.");
          return;
        }
      } else if (material.name === "stick") {
        const woodenPlanksCount = bot.inventory.count(mcData.itemsByName["oak_planks"].id) + bot.inventory.count(mcData.itemsByName["birch_planks"].id);
        if (woodenPlanksCount < 2) {
          const requiredLogs = Math.ceil((2 - woodenPlanksCount) / 4);
          bot.chat(`Need more wooden planks. Gathering ${requiredLogs} logs...`);
          await obtainWoodLogs(bot, requiredLogs);
          await craftItem(bot, "oak_planks", requiredLogs);
        }
        bot.chat("Crafting sticks...");
        await craftItem(bot, "stick", 1);
      }
    }
  }

  // Ensure a crafting table is placed nearby
  let craftingTable = bot.findBlock({
    matching: mcData.blocksByName.crafting_table.id,
    maxDistance: 32
  });
  if (!craftingTable) {
    bot.chat("Placing crafting table...");
    await placeItem(bot, "crafting_table", bot.entity.position.offset(1, 0, 0));
    craftingTable = bot.findBlock({
      matching: mcData.blocksByName.crafting_table.id,
      maxDistance: 32
    });
  }

  // Craft the item
  bot.chat(`Crafting ${itemName}...`);
  await craftItem(bot, itemName, 1);
  bot.chat(`${itemName} crafted successfully.`);
}

async function craftDiamondAxe(bot) {
  const requiredMaterials = [{
    name: "diamond",
    count: 3
  }, {
    name: "stick",
    count: 2
  }];
  await craftItemWithMaterials(bot, "diamond_axe", requiredMaterials);
}
\end{Verbatim}
\end{tcolorbox}


\textbf{The tools generated for the Agent Tasks are as follows:}
Here, we can clearly see the call relationships between functions, thus forming more complex tools.
\begin{tcolorbox}[title=Tools for DA-Bench, width=\textwidth,top=0mm,  breakable]
\begin{Verbatim}[breaklines=true, breakanywhere=true, fontsize=\footnotesize]
def filter_rows_by_non_null(df: pd.DataFrame, column_name: str) -> pd.DataFrame:
    """
    Filters rows in a dataset based on non-null values in a specified column.
    
    Parameters:
    - df (pd.DataFrame): The input DataFrame.
    - column_name (str): The name of the column to filter by non-null values.
    
    Returns:
    - pd.DataFrame: A DataFrame with rows containing non-null values in the specified column.
    
    Raises:
    - ValueError: If the specified column is not found in the DataFrame.
    """
    # Check if the column exists in the DataFrame
    if column_name not in df.columns:
        raise ValueError(f"Column '{column_name}' not found in the DataFrame.")
    
    # Filter rows based on non-null values in the specified column
    filtered_df = df.dropna(subset=[column_name])
    
    return filtered_df

def convert_column_to_numeric(df: pd.DataFrame, column_name: str) -> pd.DataFrame:
    """
    Converts a specified column in a DataFrame to numeric values, handling non-numeric values appropriately.
    
    Parameters:
    - df (pd.DataFrame): The input DataFrame.
    - column_name (str): The name of the column to convert to numeric values.
    
    Returns:
    - pd.DataFrame: The DataFrame with the specified column converted to numeric values.
    
    Raises:
    - ValueError: If the specified column is not found in the DataFrame.
    """
    # Check if the column exists in the DataFrame
    if column_name not in df.columns:
        raise ValueError(f"Column '{column_name}' not found in the DataFrame.")
    
    # Convert the specified column to numeric values, setting non-numeric values to NaN
    df[column_name] = pd.to_numeric(df[column_name], errors='coerce')
    
    # Filter out rows with non-numeric values in the specified column using the existing tool
    df = filter_rows_by_non_null(df, column_name)
    
    return df

def create_sum_feature(df: pd.DataFrame, new_column_name: str, columns_to_sum: list) -> pd.DataFrame:
    """
    Creates a new feature by summing specified columns in a DataFrame.
    
    Parameters:
    - df (pd.DataFrame): The input DataFrame.
    - new_column_name (str): The name of the new column to be created.
    - columns_to_sum (list): A list of column names to sum.
    
    Returns:
    - pd.DataFrame: The DataFrame with the new feature added.
    
    Raises:
    - ValueError: If any of the specified columns are not found in the DataFrame.
    """
    # Check if all specified columns exist in the DataFrame
    for column in columns_to_sum:
        if column not in df.columns:
            raise ValueError(f"Column '{column}' not found in the DataFrame.")
    
    # Convert specified columns to numeric values
    for column in columns_to_sum:
        df = convert_column_to_numeric(df, column)
    
    # Create the new feature by summing the specified columns
    df[new_column_name] = df[columns_to_sum].sum(axis=1)
    
    return df
\end{Verbatim}
\end{tcolorbox}


\begin{tcolorbox}[title=Tools for TextCraft, width=\textwidth,top=0mm, breakable]
\begin{Verbatim}[breaklines=true, breakanywhere=true, fontsize=\footnotesize]
def gather_materials_for_dye(required_materials: dict) -> bool:
    """
    Gathers the required materials for crafting any dye.
    
    Parameters:
    - required_materials (dict): A dictionary where keys are material names and values are the required quantities.
    
    The tool checks the inventory for these materials and gathers them if they are missing.
    
    Returns:
    - bool: True if all materials were successfully gathered, False otherwise.
    """
    # Gather the required materials
    if not gather_materials(required_materials):
        return False
    
    # Check if we have white dye, if not gather bone meal or lily of the valley to craft it
    inventory = check_inventory()
    if "white dye" in required_materials and "white dye" not in inventory:
        if not gather_materials({"bone meal": 1}) and not gather_materials({"lily of the valley": 1}):
            return False
        # Craft white dye using bone meal or lily of the valley
        if "bone meal" in inventory:
            craft_object("1 white dye", ["1 bone meal"])
        elif "lily of the valley" in inventory:
            craft_object("1 white dye", ["1 lily of the valley"])
    
    # Recheck the inventory to ensure all materials are gathered
    missing_items = check_missing_items([f"{qty} {item}" for item, qty in required_materials.items()])
    if missing_items:
        print(f"Missing items: {missing_items}")
        return False
    
    # Successfully gathered all materials
    return True

def craft_orange_dye(quantity: int) -> bool:
    """
    Crafts the specified quantity of orange dye.
    
    Parameters:
    - quantity (int): The number of orange dye to craft.
    
    Returns:
    - bool: True if the orange dye was successfully crafted, False otherwise.
    """
    # Define the required materials for crafting orange dye
    required_materials = {"orange tulip": quantity, "red dye": quantity, "yellow dye": quantity}
    
    # Gather the required materials using the existing gather_materials_for_dye function
    if not gather_materials_for_dye(required_materials):
        return False
    
    # Check the inventory for available materials
    inventory = check_inventory()
    
    # Craft orange dye using orange tulip if available
    if "orange tulip" in inventory:
        craft_object(f"{quantity} orange dye", [f"{quantity} orange tulip"])
        print(f"Crafted {quantity} orange dye using {quantity} orange tulip")
        return True
    
    # Craft orange dye using red dye and yellow dye if available
    if "red dye" in inventory and "yellow dye" in inventory:
        craft_object(f"{quantity} orange dye", [f"{quantity} red dye", f"{quantity} yellow dye"])
        print(f"Crafted {quantity} orange dye using {quantity} red dye and {quantity} yellow dye")
        return True
    
    # If neither method was successful, return False
    print("Failed to craft orange dye.")
    return False
\end{Verbatim}
\end{tcolorbox}


\textbf{The tools generated for the Single-turn Code Task are as follows:}
\begin{tcolorbox}[title=Tools for MATH, width=\textwidth,top=0mm, breakable]
\begin{Verbatim}[breaklines=true, breakanywhere=true, fontsize=\footnotesize]
def find_integer_satisfying_condition(condition):
    """
    Find the smallest positive integer that satisfies the given condition.

    Parameters:
        condition (function): A lambda function representing the condition to be checked.

    Returns:
        int: The smallest positive integer that satisfies the condition.
    """
    x = 1
    while True:
        if condition(x):
            return x
        x += 1

def calculate_min_correct_answers(total_problems, passing_percentage):
    """
    Calculate the minimum number of correct answers required to pass a test based on the total number of problems and the passing percentage.

    Parameters:
        total_problems (int): The total number of problems on the test.
        passing_percentage (float): The passing percentage required to pass the test.

    Returns:
        int: The minimum number of correct answers required to pass the test.
    """
    if total_problems <= 0:
        return "Total number of problems must be greater than zero."
    if not (0 <= passing_percentage <= 100):
        return "Passing percentage must be between 0 and 100."

    required_correct_answers = (passing_percentage / 100) * total_problems

    # Use find_integer_satisfying_condition to find the minimum integer satisfying the condition
    min_correct_answers = find_integer_satisfying_condition(lambda x: x >= required_correct_answers)
    
    return min_correct_answers
\end{Verbatim}
\end{tcolorbox}

\begin{tcolorbox}[title=Tools for Date, width=\textwidth,top=0mm, breakable]
\begin{Verbatim}[breaklines=true, breakanywhere=true, fontsize=\footnotesize]
def calculate_date_by_days(start_date_str: str, days_to_add: int, date_format="%m/%d/%Y") -> str:
    """
    Calculates the date a specified number of days before or after a given date.

    Parameters:
    - start_date_str (str): The starting date as a string in the format MM/DD/YYYY.
    - days_to_add (int): The number of days to add (positive) or subtract (negative) from the start date.
    - date_format (str): The format of the input and output date string. Default is 'MM/DD/YYYY'.

    Returns:
    - str: The resulting date in the format MM/DD/YYYY.
    
    Raises:
    - ValueError: If the input date string does not match the specified format.
    - OverflowError: If the resulting date is out of the valid range for dates.
    """
    from datetime import datetime, timedelta

    try:
        # Parse the input date string into a date object using the provided format
        start_date = datetime.strptime(start_date_str, date_format).date()

        # Calculate the new date by adding the specified number of days
        new_date = start_date + timedelta(days=days_to_add)

        # Format the new date back into the desired string format
        result_date_str = new_date.strftime(date_format)

        return result_date_str
    except ValueError as e:
        raise ValueError("Incorrect date format. Please ensure the date string matches the provided format.") from e
    except OverflowError as e:
        raise OverflowError("The resulting date is out of the valid range for dates.") from e

def calculate_date_by_days_uk_format(start_date_str: str, days_to_add: int) -> str:
    """
    Calculates the date a specified number of days before or after a given date in UK format (DD/MM/YYYY).

    Parameters:
    - start_date_str (str): The starting date as a string in the format DD/MM/YYYY.
    - days_to_add (int): The number of days to add (positive) or subtract (negative) from the start date.

    Returns:
    - str: The resulting date in the format MM/DD/YYYY.
    
    Raises:
    - ValueError: If the input date string does not match the specified format.
    """
    from datetime import datetime

    try:
        # Convert the input date from DD/MM/YYYY to MM/DD/YYYY
        start_date = datetime.strptime(start_date_str, "%d/%m/%Y")
        
        # Use the existing tool to calculate the new date
        result_date_str = calculate_date_by_days(start_date.strftime("%m/%d/%Y"), days_to_add, "%m/%d/%Y")
        
        return result_date_str
    except ValueError as e:
        raise ValueError("Incorrect date format. Please ensure the date string matches the provided format.") from e
\end{Verbatim}
\end{tcolorbox}


\begin{tcolorbox}[title=Tools for TabMWP, width=\textwidth,top=0mm, breakable]
\begin{Verbatim}[breaklines=true, breakanywhere=true, fontsize=\footnotesize]
import pandas as pd

def stem_and_leaf_to_dataframe(stem_leaf_dict: dict) -> pd.DataFrame:
    """
    Converts a stem-and-leaf plot into a DataFrame.

    Parameters:
    - stem_leaf_dict (dict): A dictionary where keys are the stems and values are lists of leaves.

    Returns:
    - pd.DataFrame: A DataFrame with a single column containing the combined values of stems and leaves.
    """
    # Initialize an empty list to store the combined values
    combined_values = []

    # Iterate through the dictionary to combine stems and leaves
    for stem, leaves in stem_leaf_dict.items():
        for leaf in leaves:
            combined_value = int(f"{stem}{leaf}")
            combined_values.append(combined_value)

    # Create a DataFrame from the combined values
    df = pd.DataFrame(combined_values, columns=["Values"])
    
    return df

import pandas as pd

def count_value_occurrences(stem_leaf_dict: dict, value) -> int:
    """
    Counts the occurrences of a specific value in a DataFrame column created from a stem-and-leaf plot.

    Parameters:
    - stem_leaf_dict (dict): A dictionary where keys are the stems and values are lists of leaves.
    - value: The value to count in the DataFrame.

    Returns:
    - int: The count of the specified value in the DataFrame.
    """
    # Convert the stem-and-leaf plot to a DataFrame using the existing tool
    df = stem_and_leaf_to_dataframe(stem_leaf_dict)
    
    # Count the occurrences of the specified value in the DataFrame
    count = df["Values"].value_counts().get(value, 0)
    
    return count
\end{Verbatim}
\end{tcolorbox}

%%%%%%%%%%%%%%%%%%%%%%%%%%%%%%%%%%%%%%%%%%%%%%%%%%%%%%%%%%%%%%%%%%%%%%%%%%%%%%%
%%%%%%%%%%%%%%%%%%%%%%%%%%%%%%%%%%%%%%%%%%%%%%%%%%%%%%%%%%%%%%%%%%%%%%%%%%%%%%%


\end{document}
