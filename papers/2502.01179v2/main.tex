%%%%%%%% ICML 2025 EXAMPLE LATEX SUBMISSION FILE %%%%%%%%%%%%%%%%%

\documentclass{article}

% Recommended, but optional, packages for figures and better typesetting:
\usepackage{microtype}
\usepackage{graphicx}
\usepackage{subfigure}
\usepackage{subcaption}
\usepackage{booktabs} % for professional tables
\usepackage{tabularx} %% long table
\usepackage{makecell}
\usepackage{tcolorbox}
\usepackage{paralist}

% hyperref makes hyperlinks in the resulting PDF.
% If your build breaks (sometimes temporarily if a hyperlink spans a page)
% please comment out the following usepackage line and replace
% \usepackage{icml2025} with \usepackage[nohyperref]{icml2025} above.
\usepackage{hyperref}


% Attempt to make hyperref and algorithmic work together better:
\newcommand{\theHalgorithm}{\arabic{algorithm}}

% Use the following line for the initial blind version submitted for review:
% \usepackage{icml2025}

% If accepted, instead use the following line for the camera-ready submission:
\usepackage[accepted]{icml2025}

% For theorems and such
\usepackage{amsmath}
\usepackage{amssymb}
\usepackage{mathtools}
\usepackage{amsthm}

%additional package(s)
\usepackage{graphicx}
\usepackage{booktabs}
\usepackage{multirow}
% \usepackage{subcaption}
\usepackage{float}

\newcommand{\jola}{\textsc{JoLA}}

% if you use cleveref..
\usepackage[capitalize,noabbrev]{cleveref}

%%%%%%%%%%%%%%%%%%%%%%%%%%%%%%%%
% THEOREMS
%%%%%%%%%%%%%%%%%%%%%%%%%%%%%%%%
\theoremstyle{plain}
\newtheorem{theorem}{Theorem}[section]
\newtheorem{proposition}[theorem]{Proposition}
\newtheorem{lemma}[theorem]{Lemma}
\newtheorem{corollary}[theorem]{Corollary}
\theoremstyle{definition}
\newtheorem{definition}[theorem]{Definition}
\newtheorem{assumption}[theorem]{Assumption}
\theoremstyle{remark}
\newtheorem{remark}[theorem]{Remark}

% Todonotes is useful during development; simply uncomment the next line
%    and comment out the line below the next line to turn off comments
%\usepackage[disable,textsize=tiny]{todonotes}
\usepackage[textsize=tiny]{todonotes}


% The \icmltitle you define below is probably too long as a header.
% Therefore, a short form for the running title is supplied here:
\icmltitlerunning{Joint Localization and Activation Editing for Low-Resource Fine-Tuning}

\begin{document}

\twocolumn[
% \icmltitle{Robust Representation Tuning with Dynamic Attention Head Pruning}
% \icmltitle{Robust Representation Tuning in Low Data Scenarios with Gated Dynamic Attention Head Pruning}
\icmltitle{Joint Localization and Activation Editing for Low-Resource Fine-Tuning}

% It is OKAY to include author information, even for blind
% submissions: the style file will automatically remove it for you
% unless you've provided the [accepted] option to the icml2025
% package.

% List of affiliations: The first argument should be a (short)
% identifier you will use later to specify author affiliations
% Academic affiliations should list Department, University, City, Region, Country
% Industry affiliations should list Company, City, Region, Country

% You can specify symbols, otherwise they are numbered in order.
% Ideally, you should not use this facility. Affiliations will be numbered
% in order of appearance and this is the preferred way.
\icmlsetsymbol{equal}{*}

\begin{icmlauthorlist}
\icmlauthor{Wen Lai}{tum,mcml}
\icmlauthor{Alexander Fraser}{tum,mcml}
\icmlauthor{Ivan Titov}{uoe,uva}
\end{icmlauthorlist}

\icmlaffiliation{tum}{Technical University of Munich}
\icmlaffiliation{mcml}{Munich Center for Machine Learning}
\icmlaffiliation{uoe}{University of Edinburgh}
\icmlaffiliation{uva}{University of Amsterdam}

\icmlcorrespondingauthor{Wen Lai}{wen.lai@tum.de}
% \icmlcorrespondingauthor{Ivan Titov}{ititov@inf.ed.ac.uk}
% \icmlcorrespondingauthor{Alexander Fraser}{alexander.fraser@tum.de}

% You may provide any keywords that you
% find helpful for describing your paper; these are used to populate
% the "keywords" metadata in the PDF but will not be shown in the document
\icmlkeywords{Machine Learning, ICML}

\vskip 0.3in
]

% this must go after the closing bracket ] following \twocolumn[ ...

% This command actually creates the footnote in the first column
% listing the affiliations and the copyright notice.
% The command takes one argument, which is text to display at the start of the footnote.
% The \icmlEqualContribution command is standard text for equal contribution.
% Remove it (just {}) if you do not need this facility.

\printAffiliationsAndNotice{}  % leave blank if no need to mention equal contribution
% \printAffiliationsAndNotice{\icmlEqualContribution} % otherwise use the standard text.

\begin{abstract}
Parameter-efficient fine-tuning (PEFT) methods, such as LoRA, are commonly used to adapt LLMs.
However, the effectiveness of standard PEFT methods is limited in low-resource scenarios with only a few hundred examples.
Recent advances in interpretability research have inspired the emergence of activation editing techniques, which modify the activations of specific model components.
These methods, due to their extremely small parameter counts, show promise for small datasets.
However, their performance is highly dependent on identifying the correct modules to edit and often lacks stability across different datasets.
In this paper, we propose Joint Localization and Activation Editing (\jola{}), a method that jointly learns 
(1) which heads in the Transformer to edit
(2) whether the intervention should be additive, multiplicative, or both and 
(3) the intervention parameters themselves - the vectors applied as additive offsets or multiplicative scalings to the head output.
Through evaluations on three benchmarks spanning commonsense reasoning, natural language understanding, and natural language generation, we demonstrate that \jola{} consistently outperforms existing methods.\footnote{The code for the method is released at \url{https://github.com/wenlai-lavine/jola}.}

\end{abstract}

%!TEX root = gcn.tex
\section{Introduction}
Graphs, representing structural data and topology, are widely used across various domains, such as social networks and merchandising transactions.
Graph convolutional networks (GCN)~\cite{iclr/KipfW17} have significantly enhanced model training on these interconnected nodes.
However, these graphs often contain sensitive information that should not be leaked to untrusted parties.
For example, companies may analyze sensitive demographic and behavioral data about users for applications ranging from targeted advertising to personalized medicine.
Given the data-centric nature and analytical power of GCN training, addressing these privacy concerns is imperative.

Secure multi-party computation (MPC)~\cite{crypto/ChaumDG87,crypto/ChenC06,eurocrypt/CiampiRSW22} is a critical tool for privacy-preserving machine learning, enabling mutually distrustful parties to collaboratively train models with privacy protection over inputs and (intermediate) computations.
While research advances (\eg,~\cite{ccs/RatheeRKCGRS20,uss/NgC21,sp21/TanKTW,uss/WatsonWP22,icml/Keller022,ccs/ABY318,folkerts2023redsec}) support secure training on convolutional neural networks (CNNs) efficiently, private GCN training with MPC over graphs remains challenging.

Graph convolutional layers in GCNs involve multiplications with a (normalized) adjacency matrix containing $\numedge$ non-zero values in a $\numnode \times \numnode$ matrix for a graph with $\numnode$ nodes and $\numedge$ edges.
The graphs are typically sparse but large.
One could use the standard Beaver-triple-based protocol to securely perform these sparse matrix multiplications by treating graph convolution as ordinary dense matrix multiplication.
However, this approach incurs $O(\numnode^2)$ communication and memory costs due to computations on irrelevant nodes.
%
Integrating existing cryptographic advances, the initial effort of SecGNN~\cite{tsc/WangZJ23,nips/RanXLWQW23} requires heavy communication or computational overhead.
Recently, CoGNN~\cite{ccs/ZouLSLXX24} optimizes the overhead in terms of  horizontal data partitioning, proposing a semi-honest secure framework.
Research for secure GCN over vertical data  remains nascent.

Current MPC studies, for GCN or not, have primarily targeted settings where participants own different data samples, \ie, horizontally partitioned data~\cite{ccs/ZouLSLXX24}.
MPC specialized for scenarios where parties hold different types of features~\cite{tkde/LiuKZPHYOZY24,icml/CastigliaZ0KBP23,nips/Wang0ZLWL23} is rare.
This paper studies $2$-party secure GCN training for these vertical partition cases, where one party holds private graph topology (\eg, edges) while the other owns private node features.
For instance, LinkedIn holds private social relationships between users, while banks own users' private bank statements.
Such real-world graph structures underpin the relevance of our focus.
To our knowledge, no prior work tackles secure GCN training in this context, which is crucial for cross-silo collaboration.


To realize secure GCN over vertically split data, we tailor MPC protocols for sparse graph convolution, which fundamentally involves sparse (adjacency) matrix multiplication.
Recent studies have begun exploring MPC protocols for sparse matrix multiplication (SMM).
ROOM~\cite{ccs/SchoppmannG0P19}, a seminal work on SMM, requires foreknowledge of sparsity types: whether the input matrices are row-sparse or column-sparse.
Unfortunately, GCN typically trains on graphs with arbitrary sparsity, where nodes have varying degrees and no specific sparsity constraints.
Moreover, the adjacency matrix in GCN often contains a self-loop operation represented by adding the identity matrix, which is neither row- nor column-sparse.
Araki~\etal~\cite{ccs/Araki0OPRT21} avoid this limitation in their scalable, secure graph analysis work, yet it does not cover vertical partition.

% and related primitives
To bridge this gap, we propose a secure sparse matrix multiplication protocol, \osmm, achieving \emph{accurate, efficient, and secure GCN training over vertical data} for the first time.

\subsection{New Techniques for Sparse Matrices}
The cost of evaluating a GCN layer is dominated by SMM in the form of $\adjmat\feamat$, where $\adjmat$ is a sparse adjacency matrix of a (directed) graph $\graph$ and $\feamat$ is a dense matrix of node features.
For unrelated nodes, which often constitute a substantial portion, the element-wise products $0\cdot x$ are always zero.
Our efficient MPC design 
avoids unnecessary secure computation over unrelated nodes by focusing on computing non-zero results while concealing the sparse topology.
We achieve this~by:
1) decomposing the sparse matrix $\adjmat$ into a product of matrices (\S\ref{sec::sgc}), including permutation and binary diagonal matrices, that can \emph{faithfully} represent the original graph topology;
2) devising specialized protocols (\S\ref{sec::smm_protocol}) for efficiently multiplying the structured matrices while hiding sparsity topology.


 
\subsubsection{Sparse Matrix Decomposition}
We decompose adjacency matrix $\adjmat$ of $\graph$ into two bipartite graphs: one represented by sparse matrix $\adjout$, linking the out-degree nodes to edges, the other 
by sparse matrix $\adjin$,
linking edges to in-degree nodes.

%\ie, we decompose $\adjmat$ into $\adjout \adjin$, where $\adjout$ and $\adjin$ are sparse matrices representing these connections.
%linking out-degree nodes to edges and edges to in-degree nodes of $\graph$, respectively.

We then permute the columns of $\adjout$ and the rows of $\adjin$ so that the permuted matrices $\adjout'$ and $\adjin'$ have non-zero positions with \emph{monotonically non-decreasing} row and column indices.
A permutation $\sigma$ is used to preserve the edge topology, leading to an initial decomposition of $\adjmat = \adjout'\sigma \adjin'$.
This is further refined into a sequence of \emph{linear transformations}, 
which can be efficiently computed by our MPC protocols for 
\emph{oblivious permutation}
%($\Pi_{\ssp}$) 
and \emph{oblivious selection-multiplication}.
% ($\Pi_\SM$)
\iffalse
Our approach leverages bipartite graph representation and the monotonicity of non-zero positions to decompose a general sparse matrix into linear transformations, enhancing the efficiency of our MPC protocols.
\fi
Our decomposition approach is not limited to GCNs but also general~SMM 
by 
%simply 
treating them 
as adjacency matrices.
%of a graph.
%Since any sparse matrix can be viewed 

%allowing the same technique to be applied.

 
\subsubsection{New Protocols for Linear Transformations}
\emph{Oblivious permutation} (OP) is a two-party protocol taking a private permutation $\sigma$ and a private vector $\xvec$ from the two parties, respectively, and generating a secret share $\l\sigma \xvec\r$ between them.
Our OP protocol employs correlated randomnesses generated in an input-independent offline phase to mask $\sigma$ and $\xvec$ for secure computations on intermediate results, requiring only $1$ round in the online phase (\cf, $\ge 2$ in previous works~\cite{ccs/AsharovHIKNPTT22, ccs/Araki0OPRT21}).

Another crucial two-party protocol in our work is \emph{oblivious selection-multiplication} (OSM).
It takes a private bit~$s$ from a party and secret share $\l x\r$ of an arithmetic number~$x$ owned by the two parties as input and generates secret share $\l sx\r$.
%between them.
%Like our OP protocol, o
Our $1$-round OSM protocol also uses pre-computed randomnesses to mask $s$ and $x$.
%for secure computations.
Compared to the Beaver-triple-based~\cite{crypto/Beaver91a} and oblivious-transfer (OT)-based approaches~\cite{pkc/Tzeng02}, our protocol saves ${\sim}50\%$ of online communication while having the same offline communication and round complexities.

By decomposing the sparse matrix into linear transformations and applying our specialized protocols, our \osmm protocol
%($\prosmm$) 
reduces the complexity of evaluating $\numnode \times \numnode$ sparse matrices with $\numedge$ non-zero values from $O(\numnode^2)$ to $O(\numedge)$.

%(\S\ref{sec::secgcn})
\subsection{\cgnn: Secure GCN made Efficient}
Supported by our new sparsity techniques, we build \cgnn, 
a two-party computation (2PC) framework for GCN inference and training over vertical
%ly split
data.
Our contributions include:

1) We are the first to explore sparsity over vertically split, secret-shared data in MPC, enabling decompositions of sparse matrices with arbitrary sparsity and isolating computations that can be performed in plaintext without sacrificing privacy.

2) We propose two efficient $2$PC primitives for OP and OSM, both optimally single-round.
Combined with our sparse matrix decomposition approach, our \osmm protocol ($\prosmm$) achieves constant-round communication costs of $O(\numedge)$, reducing memory requirements and avoiding out-of-memory errors for large matrices.
In practice, it saves $99\%+$ communication
%(Table~\ref{table:comm_smm}) 
and reduces ${\sim}72\%$ memory usage over large $(5000\times5000)$ matrices compared with using Beaver triples.
%(Table~\ref{table:mem_smm_sparse}) ${\sim}16\%$-

3) We build an end-to-end secure GCN framework for inference and training over vertically split data, maintaining accuracy on par with plaintext computations.
We will open-source our evaluation code for research and deployment.

To evaluate the performance of $\cgnn$, we conducted extensive experiments over three standard graph datasets (Cora~\cite{aim/SenNBGGE08}, Citeseer~\cite{dl/GilesBL98}, and Pubmed~\cite{ijcnlp/DernoncourtL17}),
reporting communication, memory usage, accuracy, and running time under varying network conditions, along with an ablation study with or without \osmm.
Below, we highlight our key achievements.

\textit{Communication (\S\ref{sec::comm_compare_gcn}).}
$\cgnn$ saves communication by $50$-$80\%$.
(\cf,~CoGNN~\cite{ccs/KotiKPG24}, OblivGNN~\cite{uss/XuL0AYY24}).

\textit{Memory usage (\S\ref{sec::smmmemory}).}
\cgnn alleviates out-of-memory problems of using %the standard 
Beaver-triples~\cite{crypto/Beaver91a} for large datasets.

\textit{Accuracy (\S\ref{sec::acc_compare_gcn}).}
$\cgnn$ achieves inference and training accuracy comparable to plaintext counterparts.
%training accuracy $\{76\%$, $65.1\%$, $75.2\%\}$ comparable to $\{75.7\%$, $65.4\%$, $74.5\%\}$ in plaintext.

{\textit{Computational efficiency (\S\ref{sec::time_net}).}} 
%If the network is worse in bandwidth and better in latency, $\cgnn$ shows more benefits.
$\cgnn$ is faster by $6$-$45\%$ in inference and $28$-$95\%$ in training across various networks and excels in narrow-bandwidth and low-latency~ones.

{\textit{Impact of \osmm (\S\ref{sec:ablation}).}}
Our \osmm protocol shows a $10$-$42\times$ speed-up for $5000\times 5000$ matrices and saves $10$-2$1\%$ memory for ``small'' datasets and up to $90\%$+ for larger ones.

\section{Background}
\label{sec:background}
Activation editing in LLMs modifies intermediate activation outputs to steer model behavior.
We categorize existing approaches into three types based on the transformation function applied to activations.
Given an activation output $z_t^{(l,i)} \in \mathbb{R}^{d_l}$ for $i$th component at layer $l$, the general transformation is:
\begin{equation}
    z_t^{(l,i)'} = f(z_t^{(l,i)}),
\end{equation}
where $f(\cdot)$ determines the intervention type:
\begin{compactitem}
\item \textbf{Additive methods} apply a bias vector $a^{i}_{l} \in \mathbb{R}^{d_l}$ : $z_t^{(l,i)'} = z_t^{(l,i)} + a^{(l,i)}$.
\item \textbf{Multiplicative methods} scale activations as $z_t^{(l,i)'} = m^{(l,i)} \odot z_t^{(l,i)}$, where $m^{(l,i)} \in \mathbb{R}^{d_l}$ and $\odot$ is an element-wise product.
\item \textbf{Hybrid methods} combine both transformations: $z_t^{(l,i)'} = m^{(l,i)} \odot z_t^{(l,i)} + a^{(l,i)}$.
\end{compactitem}

Existing methods follow these paradigms but often rely on fixed selections of components for modification, limiting adaptability.
For example, BitFit~\cite{ben-zaken-etal-2022-bitfit} updates bias terms, while RED~\cite{wu-etal-2024-advancing} employs per-dimension scaling vectors and bias vectors. ReFT~\cite{wu2024reft} applies fine-tuned low-rank hidden states with MLP layers, and LoFIT~\cite{yin2024lofit} intervenes in selected attention heads with additive bias vectors but requires manual selection.
\vspace{-5pt}
\section{Method}
\label{sec:method}
\section{Overview}

\revision{In this section, we first explain the foundational concept of Hausdorff distance-based penetration depth algorithms, which are essential for understanding our method (Sec.~\ref{sec:preliminary}).
We then provide a brief overview of our proposed RT-based penetration depth algorithm (Sec.~\ref{subsec:algo_overview}).}



\section{Preliminaries }
\label{sec:Preliminaries}

% Before we introduce our method, we first overview the important basics of 3D dynamic human modeling with Gaussian splatting. Then, we discuss the diffusion-based 3d generation techniques, and how they can be applied to human modeling.
% \ZY{I stopp here. TBC.}
% \subsection{Dynamic human modeling with Gaussian splatting}
\subsection{3D Gaussian Splatting}
3D Gaussian splatting~\cite{kerbl3Dgaussians} is an explicit scene representation that allows high-quality real-time rendering. The given scene is represented by a set of static 3D Gaussians, which are parameterized as follows: Gaussian center $x\in {\mathbb{R}^3}$, color $c\in {\mathbb{R}^3}$, opacity $\alpha\in {\mathbb{R}}$, spatial rotation in the form of quaternion $q\in {\mathbb{R}^4}$, and scaling factor $s\in {\mathbb{R}^3}$. Given these properties, the rendering process is represented as:
\begin{equation}
  I = Splatting(x, c, s, \alpha, q, r),
  \label{eq:splattingGA}
\end{equation}
where $I$ is the rendered image, $r$ is a set of query rays crossing the scene, and $Splatting(\cdot)$ is a differentiable rendering process. We refer readers to Kerbl et al.'s paper~\cite{kerbl3Dgaussians} for the details of Gaussian splatting. 



% \ZY{I would suggest move this part to the method part.}
% GaissianAvatar is a dynamic human generation model based on Gaussian splitting. Given a sequence of RGB images, this method utilizes fitted SMPLs and sampled points on its surface to obtain a pose-dependent feature map by a pose encoder. The pose-dependent features and a geometry feature are fed in a Gaussian decoder, which is employed to establish a functional mapping from the underlying geometry of the human form to diverse attributes of 3D Gaussians on the canonical surfaces. The parameter prediction process is articulated as follows:
% \begin{equation}
%   (\Delta x,c,s)=G_{\theta}(S+P),
%   \label{eq:gaussiandecoder}
% \end{equation}
%  where $G_{\theta}$ represents the Gaussian decoder, and $(S+P)$ is the multiplication of geometry feature S and pose feature P. Instead of optimizing all attributes of Gaussian, this decoder predicts 3D positional offset $\Delta{x} \in {\mathbb{R}^3}$, color $c\in\mathbb{R}^3$, and 3D scaling factor $ s\in\mathbb{R}^3$. To enhance geometry reconstruction accuracy, the opacity $\alpha$ and 3D rotation $q$ are set to fixed values of $1$ and $(1,0,0,0)$ respectively.
 
%  To render the canonical avatar in observation space, we seamlessly combine the Linear Blend Skinning function with the Gaussian Splatting~\cite{kerbl3Dgaussians} rendering process: 
% \begin{equation}
%   I_{\theta}=Splatting(x_o,Q,d),
%   \label{eq:splatting}
% \end{equation}
% \begin{equation}
%   x_o = T_{lbs}(x_c,p,w),
%   \label{eq:LBS}
% \end{equation}
% where $I_{\theta}$ represents the final rendered image, and the canonical Gaussian position $x_c$ is the sum of the initial position $x$ and the predicted offset $\Delta x$. The LBS function $T_{lbs}$ applies the SMPL skeleton pose $p$ and blending weights $w$ to deform $x_c$ into observation space as $x_o$. $Q$ denotes the remaining attributes of the Gaussians. With the rendering process, they can now reposition these canonical 3D Gaussians into the observation space.



\subsection{Score Distillation Sampling}
Score Distillation Sampling (SDS)~\cite{poole2022dreamfusion} builds a bridge between diffusion models and 3D representations. In SDS, the noised input is denoised in one time-step, and the difference between added noise and predicted noise is considered SDS loss, expressed as:

% \begin{equation}
%   \mathcal{L}_{SDS}(I_{\Phi}) \triangleq E_{t,\epsilon}[w(t)(\epsilon_{\phi}(z_t,y,t)-\epsilon)\frac{\partial I_{\Phi}}{\partial\Phi}],
%   \label{eq:SDSObserv}
% \end{equation}
\begin{equation}
    \mathcal{L}_{\text{SDS}}(I_{\Phi}) \triangleq \mathbb{E}_{t,\epsilon} \left[ w(t) \left( \epsilon_{\phi}(z_t, y, t) - \epsilon \right) \frac{\partial I_{\Phi}}{\partial \Phi} \right],
  \label{eq:SDSObservGA}
\end{equation}
where the input $I_{\Phi}$ represents a rendered image from a 3D representation, such as 3D Gaussians, with optimizable parameters $\Phi$. $\epsilon_{\phi}$ corresponds to the predicted noise of diffusion networks, which is produced by incorporating the noise image $z_t$ as input and conditioning it with a text or image $y$ at timestep $t$. The noise image $z_t$ is derived by introducing noise $\epsilon$ into $I_{\Phi}$ at timestep $t$. The loss is weighted by the diffusion scheduler $w(t)$. 
% \vspace{-3mm}

\subsection{Overview of the RTPD Algorithm}\label{subsec:algo_overview}
Fig.~\ref{fig:Overview} presents an overview of our RTPD algorithm.
It is grounded in the Hausdorff distance-based penetration depth calculation method (Sec.~\ref{sec:preliminary}).
%, similar to that of Tang et al.~\shortcite{SIG09HIST}.
The process consists of two primary phases: penetration surface extraction and Hausdorff distance calculation.
We leverage the RTX platform's capabilities to accelerate both of these steps.

\begin{figure*}[t]
    \centering
    \includegraphics[width=0.8\textwidth]{Image/overview.pdf}
    \caption{The overview of RT-based penetration depth calculation algorithm overview}
    \label{fig:Overview}
\end{figure*}

The penetration surface extraction phase focuses on identifying the overlapped region between two objects.
\revision{The penetration surface is defined as a set of polygons from one object, where at least one of its vertices lies within the other object. 
Note that in our work, we focus on triangles rather than general polygons, as they are processed most efficiently on the RTX platform.}
To facilitate this extraction, we introduce a ray-tracing-based \revision{Point-in-Polyhedron} test (RT-PIP), significantly accelerated through the use of RT cores (Sec.~\ref{sec:RT-PIP}).
This test capitalizes on the ray-surface intersection capabilities of the RTX platform.
%
Initially, a Geometry Acceleration Structure (GAS) is generated for each object, as required by the RTX platform.
The RT-PIP module takes the GAS of one object (e.g., $GAS_{A}$) and the point set of the other object (e.g., $P_{B}$).
It outputs a set of points (e.g., $P_{\partial B}$) representing the penetration region, indicating their location inside the opposing object.
Subsequently, a penetration surface (e.g., $\partial B$) is constructed using this point set (e.g., $P_{\partial B}$) (Sec.~\ref{subsec:surfaceGen}).
%
The generated penetration surfaces (e.g., $\partial A$ and $\partial B$) are then forwarded to the next step. 

The Hausdorff distance calculation phase utilizes the ray-surface intersection test of the RTX platform (Sec.~\ref{sec:RT-Hausdorff}) to compute the Hausdorff distance between two objects.
We introduce a novel Ray-Tracing-based Hausdorff DISTance algorithm, RT-HDIST.
It begins by generating GAS for the two penetration surfaces, $P_{\partial A}$ and $P_{\partial B}$, derived from the preceding step.
RT-HDIST processes the GAS of a penetration surface (e.g., $GAS_{\partial A}$) alongside the point set of the other penetration surface (e.g., $P_{\partial B}$) to compute the penetration depth between them.
The algorithm operates bidirectionally, considering both directions ($\partial A \to \partial B$ and $\partial B \to \partial A$).
The final penetration depth between the two objects, A and B, is determined by selecting the larger value from these two directional computations.

%In the Hausdorff distance calculation step, we compute the Hausdorff distance between given two objects using a ray-surface-intersection test. (Sec.~\ref{sec:RT-Hausdorff}) Initially, we construct the GAS for both $\partial A$ and $\partial B$ to utilize the RT-core effectively. The RT-based Hausdorff distance algorithms then determine the Hausdorff distance by processing the GAS of one object (e.g. $GAS_{\partial A}$) and set of the vertices of the other (e.g. $P_{\partial B}$). Following the Hausdorff distance definition (Eq.~\ref{equation:hausdorff_definition}), we compute the Hausdorff distance to both directions ($\partial A \to \partial B$) and ($\partial B \to \partial A$). As a result, the bigger one is the final Hausdorff distance, and also it is the penetration depth between input object $A$ and $B$.


%the proposed RT-based penetration depth calculation pipeline.
%Our proposed methods adopt Tang's Hausdorff-based penetration depth methods~\cite{SIG09HIST}. The pipeline is divided into the penetration surface extraction step and the Hausdorff distance calculation between the penetration surface steps. However, since Tang's approach is not suitable for the RT platform in detail, we modified and applied it with appropriate methods.

%The penetration surface extraction step is extracting overlapped surfaces on other objects. To utilize the RT core, we use the ray-intersection-based PIP(Point-In-Polygon) algorithms instead of collision detection between two objects which Tang et al.~\cite{SIG09HIST} used. (Sec.~\ref{sec:RT-PIP})
%RT core-based PIP test uses a ray-surface intersection test. For purpose this, we generate the GAS(Geometry Acceleration Structure) for each object. RT core-based PIP test takes the GAS of one object (e.g. $GAS_{A}$) and a set of vertex of another one (e.g. $P_{B}$). Then this computes the penetrated vertex set of another one (e.g. $P_{\partial B}$). To calculate the Hausdorff distance, these vertex sets change to objects constructed by penetrated surface (e.g. $\partial B$). Finally, the two generated overlapped surface objects $\partial A$ and $\partial B$ are used in the Hausdorff distance calculation step.

Our goal is to increase the robustness of T2I models, particularly with rare or unseen concepts, which they struggle to generate. To do so, we investigate a retrieval-augmented generation approach, through which we dynamically select images that can provide the model with missing visual cues. Importantly, we focus on models that were not trained for RAG, and show that existing image conditioning tools can be leveraged to support RAG post-hoc.
As depicted in \cref{fig:overview}, given a text prompt and a T2I generative model, we start by generating an image with the given prompt. Then, we query a VLM with the image, and ask it to decide if the image matches the prompt. If it does not, we aim to retrieve images representing the concepts that are missing from the image, and provide them as additional context to the model to guide it toward better alignment with the prompt.
In the following sections, we describe our method by answering key questions:
(1) How do we know which images to retrieve? 
(2) How can we retrieve the required images? 
and (3) How can we use the retrieved images for unknown concept generation?
By answering these questions, we achieve our goal of generating new concepts that the model struggles to generate on its own.

\vspace{-3pt}
\subsection{Which images to retrieve?}
The amount of images we can pass to a model is limited, hence we need to decide which images to pass as references to guide the generation of a base model. As T2I models are already capable of generating many concepts successfully, an efficient strategy would be passing only concepts they struggle to generate as references, and not all the concepts in a prompt.
To find the challenging concepts,
we utilize a VLM and apply a step-by-step method, as depicted in the bottom part of \cref{fig:overview}. First, we generate an initial image with a T2I model. Then, we provide the VLM with the initial prompt and image, and ask it if they match. If not, we ask the VLM to identify missing concepts and
focus on content and style, since these are easy to convey through visual cues.
As demonstrated in \cref{tab:ablations}, empirical experiments show that image retrieval from detailed image captions yields better results than retrieval from brief, generic concept descriptions.
Therefore, after identifying the missing concepts, we ask the VLM to suggest detailed image captions for images that describe each of the concepts. 

\vspace{-4pt}
\subsubsection{Error Handling}
\label{subsec:err_hand}

The VLM may sometimes fail to identify the missing concepts in an image, and will respond that it is ``unable to respond''. In these rare cases, we allow up to 3 query repetitions, while increasing the query temperature in each repetition. Increasing the temperature allows for more diverse responses by encouraging the model to sample less probable words.
In most cases, using our suggested step-by-step method yields better results than retrieving images directly from the given prompt (see 
\cref{subsec:ablations}).
However, if the VLM still fails to identify the missing concepts after multiple attempts, we fall back to retrieving images directly from the prompt, as it usually means the VLM does not know what is the meaning of the prompt.

The used prompts can be found in \cref{app:prompts}.
Next, we turn to retrieve images based on the acquired image captions.

\vspace{-3pt}
\subsection{How to retrieve the required images?}

Given $n$ image captions, our goal is to retrieve the images that are most similar to these captions from a dataset. 
To retrieve images matching a given image caption, we compare the caption to all the images in the dataset using a text-image similarity metric and retrieve the top $k$ most similar images.
Text-to-image retrieval is an active research field~\cite{radford2021learning, zhai2023sigmoid, ray2024cola, vendrowinquire}, where no single method is perfect.
Retrieval is especially hard when the dataset does not contain an exact match to the query \cite{biswas2024efficient} or when the task is fine-grained retrieval, that depends on subtle details~\cite{wei2022fine}.
Hence, a common retrieval workflow is to first retrieve image candidates using pre-computed embeddings, and then re-rank the retrieved candidates using a different, often more expensive but accurate, method \cite{vendrowinquire}.
Following this workflow, we experimented with cosine similarity over different embeddings, and with multiple re-ranking methods of reference candidates.
Although re-ranking sometimes yields better results compared to simply using cosine similarity between CLIP~\cite{radford2021learning} embeddings, the difference was not significant in most of our experiments. Therefore, for simplicity, we use cosine similarity between CLIP embeddings as our similarity metric (see \cref{tab:sim_metrics}, \cref{subsec:ablations} for more details about our experiments with different similarity metrics).

\vspace{-3pt}
\subsection{How to use the retrieved images?}
Putting it all together, after retrieving relevant images, all that is left to do is to use them as context so they are beneficial for the model.
We experimented with two types of models; models that are trained to receive images as input in addition to text and have ICL capabilities (e.g., OmniGen~\cite{xiao2024omnigen}), and T2I models augmented with an image encoder in post-training (e.g., SDXL~\cite{podellsdxl} with IP-adapter~\cite{ye2023ip}).
As the first model type has ICL capabilities, we can supply the retrieved images as examples that it can learn from, by adjusting the original prompt.
Although the second model type lacks true ICL capabilities, it offers image-based control functionalities, which we can leverage for applying RAG over it with our method.
Hence, for both model types, we augment the input prompt to contain a reference of the retrieved images as examples.
Formally, given a prompt $p$, $n$ concepts, and $k$ compatible images for each concept, we use the following template to create a new prompt:
``According to these examples of 
$\mathord{<}c_1\mathord{>:<}img_{1,1}\mathord{>}, ... , \mathord{<}img_{1,k}\mathord{>}, ... , \mathord{<}c_n\mathord{>:<}img_{n,1}\mathord{>}, ... , $
$\mathord{<}img_{n,k}\mathord{>}$,
generate $\mathord{<}p\mathord{>}$'', 
where $c_i$ for $i\in{[1,n]}$ is a compatible image caption of the image $\mathord{<}img_{i,j}\mathord{>},  j\in{[1,k]}$. 

This prompt allows models to learn missing concepts from the images, guiding them to generate the required result. 

\textbf{Personalized Generation}: 
For models that support multiple input images, we can apply our method for personalized generation as well, to generate rare concept combinations with personal concepts. In this case, we use one image for personal content, and 1+ other reference images for missing concepts. For example, given an image of a specific cat, we can generate diverse images of it, ranging from a mug featuring the cat to a lego of it or atypical situations like the cat writing code or teaching a classroom of dogs (\cref{fig:personalization}).
\vspace{-2pt}
\begin{figure}[htp]
  \centering
   \includegraphics[width=\linewidth]{Assets/personalization.pdf}
   \caption{\textbf{Personalized generation example.}
   \emph{ImageRAG} can work in parallel with personalization methods and enhance their capabilities. For example, although OmniGen can generate images of a subject based on an image, it struggles to generate some concepts. Using references retrieved by our method, it can generate the required result.
}
   \label{fig:personalization}\vspace{-10pt}
\end{figure}
\section{Experiments}

\subsection{Experimental Setup}

\textbf{Datasets.} We use three categories from the Amazon Reviews dataset~\cite{mcauley2015image} for our experiments: ``Sports and Outdoors'' (\textbf{Sports}), ``Beauty'' (\textbf{Beauty}), and ``CDs and Vinyl'' (\textbf{CDs}). Each user’s historical reviews are considered ``actions'' and are sorted chronologically as action sequences, with earlier reviews appearing first. To evaluate the models, we adopt the widely used leave-last-out protocol~\cite{kang2018sasrec,zhao2022revisiting,rajput2023tiger}, where the last item and second-to-last item in each action sequence are used for testing and validation, respectively. More details about the datasets can be found in~\Cref{app:datasets}.

\textbf{Compared methods.} We compare the performance of ActionPiece with the following methods: (1)~ID-based sequential recommendation methods, including BERT4Rec~\cite{sun2019bert4rec}, and SASRec~\cite{kang2018sasrec}; (2)~feature-enhanced sequential recommendation methods, such as FDSA~\cite{zhang2019fdsa}, S$^3$-Rec~\cite{zhou2020s3}, and VQ-Rec~\cite{hou2023vqrec}; and (3)~generative recommendation methods, including P5-CID~\cite{hua2023p5cid}, TIGER~\cite{rajput2023tiger}, LMIndexer~\cite{jin2024lmindexer}, HSTU~\cite{zhai2024hstu}, and SPM-SID~\cite{singh2024spmsid}, each representing a different action tokenization method (\Cref{tab:act_tokenization}). A detailed description of these baselines is provided in~\Cref{appendix:baselines}.

\textbf{Evaluation settings.} Following~\citet{rajput2023tiger}, we use Recall@$K$ and NDCG@$K$ as metrics to evaluate the methods, where $K \in \{5, 10\}$. Model checkpoints with the best performance on the validation set are used for evaluation on the test set. We run the experiments with five random seeds and report the average metrics.

\textbf{Implementation details.} Please refer to~\Cref{appendix:implementation} for detailed implementation and hyperparameter settings.


\subsection{Overall Performance}

We compare ActionPiece with sequential recommendation and generative recommendation baselines, which use various action tokenization methods, across three public datasets. The results are shown in~\Cref{tab:performance}. 

For the compared methods, we observe that those using item features generally outperform item ID-only methods. This indicates that incorporating features enhances recommendation performance. Among the methods leveraging item features (``Feature + ID'' and ``Generative''), generative recommendation models achieve better performance. These results further confirm that injecting semantics into item indexing and optimizing at a sub-item level enables generative models to better use semantic information and improve recommendation performance. Among all the baselines, SPM-SID achieves the best results. By incorporating the SentencePiece model~\cite{kudo2018sentencepiece}, SPM-SID replaces popular semantic ID patterns within each item with new tokens, benefiting from a larger vocabulary.

\begin{table}[t!]
    \small
    \centering
	\caption{Ablation analysis of ActionPiece. The recommendation performance is measured using NDCG@$10$. The best performance is denoted in \textbf{bold} fonts.}
	\label{tab:ablation}
	\vskip 0.1in
% 	\setlength{\tabcolsep}{1mm}{
% \resizebox{2.1\columnwidth}{!}{
    \begin{tabular}{lccc}
	\toprule
	\multicolumn{1}{c}{\textbf{Variants}} & \textbf{Sports} & \textbf{Beauty} & \textbf{CDs} \\
	\midrule
	\midrule
    \multicolumn{4}{@{}c}{\textit{TIGER with larger vocabularies}} \\
    \midrule
    (1.1) TIGER\ -\ 1k ($4 \times 2^8$) & 0.0225 & 0.0384 & 0.0411 \\
    (1.2) TIGER-49k ($6 \times 2^{13}$) & 0.0162 & 0.0317 & 0.0338 \\
    (1.3) TIGER-66k ($4 \times 2^{14}$) & 0.0194 & N/A$^\dag$ & 0.0319 \\
    \midrule
    \multicolumn{4}{@{}c}{\textit{Vocabulary construction}} \\
    \midrule
    (2.1) \emph{w/o} tokenization & 0.0215 & 0.0389 & 0.0346 \\
    (2.2) \emph{w/o} context-aware & 0.0258 & 0.0416 & 0.0429 \\
    (2.3) \emph{w/o} weighted counting & 0.0257 & 0.0412 & 0.0435 \\
    \midrule
    \multicolumn{4}{@{}c}{\textit{Set permutation regularization}} \\
    \midrule
    (3.1) only for inference & 0.0192 & 0.0316 & 0.0329 \\
    (3.2) only for training & 0.0244 & 0.0387 & 0.0422 \\
    \midrule
    ActionPiece (40k) & \textbf{0.0264} & \textbf{0.0424} & \textbf{0.0451} \\
    \bottomrule
	\end{tabular}
	\vspace{0.05cm}
	\begin{flushleft}
        $^\dag$ not applicable as $2^{14}$ is larger than \#items in Beauty.
    \end{flushleft}
% 	}}
    \vskip -0.2in
\end{table}

\begin{figure*}[t!]
    \begin{center}
    \includegraphics[width=\linewidth]{fig/vocab_size.pdf}
    \vskip -0.1in
    \caption{Analysis of recommendation performance (NDCG@10, $\uparrow$) and average tokenized sequence length (NSL, $\downarrow$) \wrt vocabulary size across three datasets.
    % NSL refers to the normalized sequence length, calculated relative to the number of initial tokens.
    ``N/A’’ indicates that ActionPiece is not applied, \ie action sequences are represented solely by initial tokens.}
    \label{fig:vocab_size}
    \end{center}
    \vskip -0.2in
\end{figure*}

Our proposed ActionPiece consistently outperforms all baselines across three datasets, achieving a significant improvement in NDCG@$10$. It surpasses the best-performing baseline method by $6.00\%$ to $12.82\%$. Unlike existing methods, ActionPiece is the first context-aware action sequence tokenizer, \ie the same action can be tokenized into different tokens depending on its surrounding context. This allows ActionPiece to capture important sequence-level feature patterns that enhance recommendation performance.

% \begin{figure}[t]
% % \vskip 0.2in
% \begin{center}
% \centerline{\includegraphics[width=0.85\columnwidth]{fig/ndcg_vs_vocab_size.pdf}}
% \end{center}
% % \vskip -0.3in
% \vspace{-0.3in}
% \caption{Comparison of performance and vocabulary size (\#token for TIGER, SPM-SID, and ActionPiece; \#item for SASRec; and \#item+\#attribute for S$^3$-Rec) on ``Sports'' dataset.
% % By adjusting the vocabulary size, ActionPiece is shown to balance memory efficiency and recommendation performance.
% }
% \label{fig:intro}
% % \vskip -0.2in
% \vspace{-0.1in}
% \end{figure}


\subsection{Ablation Study}\label{sec:ablation}

We conduct ablation analyses in~\Cref{tab:ablation} to study how each proposed technique contributes to ActionPiece.\\
\hspace*{3mm} (1)~We increase the vocabulary size of TIGER, to determine whether the performance gain of ActionPiece is solely due to scaling up the number of tokens in the vocabulary. By increasing the number of semantic ID digits per item~($4 \rightarrow 6$) and the number of candidate semantic IDs per digit~($2^8 \rightarrow 2^{13}\ \text{or}\ 2^{14}$), we create two variants with vocabularies larger than ActionPiece. However, these TIGER variants perform worse than ActionPiece, and even the original TIGER with only $1024$ tokens. The experimental results suggest that scaling up the vocabulary size for generative recommendation models is challenging, consistent with the observations from~\citet{zhang2024moc}.\\
\hspace*{3mm} (2)~To evaluate the effectiveness of the proposed vocabulary construction techniques, we introduce the following variants: \emph{(2.1)~w/o tokenization}, which skips vocabulary construction, using item features directly as tokens; \emph{(2.2)~w/o context-aware}, which only considers co-occurrences and merges tokens within each action during vocabulary construction and segmentation; and \emph{(2.3)~w/o weighted counting}, which treats all token pairs equally rather than using the weights defined in~\Cref{eq:p_one_set,eq:p_two_sets}. The results indicate that removing any of these techniques reduces performance, demonstrating the importance of these methods for building a context-aware tokenizer.\\
\hspace*{3mm} (3)~To evaluate the effectiveness of SPR, we revert to naive segmentation, as described in~\Cref{subsubsec:segmentation}, during model training and inference, respectively. The results show that replacing SPR with naive segmentation in either training or inference degrades performance.

\begin{figure}[t!]
    \begin{center}
    \includegraphics[width=0.95\columnwidth]{fig/token_util.pdf}
    \vskip -0.1in
    \caption{Analysis of token utilization rate (\%) during model training \wrt segmentation strategy.
    % ``SPR'' denotes set permutation regularization.
    }
    \label{fig:token_util}
    \end{center}
    % \vskip -0.3in
    \vskip -0.3in
\end{figure}

\subsection{Further Analysis}

% In this section, we analyze the impact of key hyperparameters in vocabulary construction and segmentation.

\subsubsection{Performance and Efficiency \wrt Vocabulary Size}

Vocabulary size is a key hyperparameter for language tokenizers~\cite{meta2024llama3,dagan2024getting}. In this study, we investigate how adjusting vocabulary size affects the generative recommendation models. We use the normalized sequence length (NSL)~\cite{dagan2024getting} to measure the length of tokenized sequences, where a smaller NSL indicates fewer tokens per tokenized sequence. We experiment with vocabulary sizes in \{N/A, 5k, 10k, 20k, 30k, 40k\}, where ``N/A'' represents the direct use of item features as tokens. As shown in~\Cref{fig:vocab_size}, increasing the vocabulary size improves recommendation performance and reduces the tokenized sequence length. Conversely, reducing the vocabulary size lowers the number of model parameters, improving memory efficiency. This analysis demonstrates that adjusting vocabulary size enables a trade-off between model performance, sequence length, and memory efficiency.

\subsubsection{Token Utilization Rate \wrt Segmentation Strategy}\label{sec:token_utilization}

As described in~\Cref{subsubsec:training}, applying SPR augments the training corpus by producing multiple token sequences that share the same semantics. In~\Cref{tab:ablation}, we observe that incorporating SPR significantly improves recommendation performance. One possible reason is that SPR increases token utilization rates. To validate this assumption, we segment the action sequences in each training epoch using two strategies: naive segmentation and SPR. As shown in~\Cref{fig:token_util}, naive segmentation uses only $56.89\%$ of tokens for model training, limiting the model's ability to generalize to unseen action sequences. In contrast, SPR achieves a token utilization rate of $87.01\%$ after the first training epoch, with further increases as training progresses. These results demonstrate that the proposed SPR segmentation strategy improves the utilization of ActionPiece tokens, enabling better generalization and enhanced performance.


\subsubsection{Performance \wrt Inference-Time Ensembles}\label{sec:inference_time_ensemble}

As described in~\Cref{subsubsec:inference}, ActionPiece supports inference-time ensembling by using SPR segmentation. We vary the number of ensembled segments, $q$, in \{N/A, 1, 3, 5, 7\}, where ``N/A'' indicates using naive segmentation during model inference. As shown in~\Cref{fig:ensemble}, ensembling more tokenized sequences improves ActionPiece's recommendation performance. However, the performance gains slow down as $q$ increases to $5$ and $7$. Since a higher $q$ also increases the computational cost of inference, this creates a trade-off between performance and computational budget in practice.

\begin{figure}[t!]
    \begin{center}
    \includegraphics[width=\columnwidth]{fig/ensemble.pdf}
    \vskip -0.15in
    \caption{Analysis of performance (NDCG@10, $\uparrow$) \wrt the number of ensembled segments $q$ during model inference.}
    \label{fig:ensemble}
    \end{center}
    \vskip -0.25in
\end{figure}

\subsection{Case Study}\label{subsec:case}

To understand how GR models benefit from the unordered feature setting and context-aware action sequence tokenization, we present an illustrative example in~\Cref{fig:case}.

Each item in the action sequence is represented as a feature set, with each item consisting of five features. The features within an item do not require a specific order. The first step of tokenization leverages the unordered nature of the feature set and applies set permutation regularization~(\Cref{subsubsec:segmentation}). This process arranges each feature set into a specific permutation and iteratively groups features based on the constructed vocabulary~(\Cref{subsubsec:vocab_construct}). This results in different segments that convey the same semantics. Each segment is represented as a sequence of sets, where each set corresponds to a token in the vocabulary.

By examining the segments and their corresponding token sequences, we identify four types of tokens, as annotated in~\Cref{fig:case}: (1) a subset of features from a single item (token {\setlength{\fboxsep}{0pt}\colorbox{myblue}{14844}} corresponds to features {\setlength{\fboxsep}{0pt}\colorbox{myblue}{747}} and {\setlength{\fboxsep}{0pt}\colorbox{myblue}{923}} of the T-shirt); (2) a set containing a single feature (feature {\setlength{\fboxsep}{0pt}\colorbox{mygreen}{76}} of the socks); (3) all features of a single item (token {\setlength{\fboxsep}{0pt}\colorbox{myyellow}{7995}} corresponds to all features of the shorts); and (4) features from multiple items (\eg token {\setlength{\fboxsep}{0pt}\colorbox{myblue}{83}\colorbox{mygreen}{16}} includes feature {\setlength{\fboxsep}{0pt}\colorbox{myblue}{923}} from the T-shirt and feature {\setlength{\fboxsep}{0pt}\colorbox{mygreen}{679}} from the socks, while token {\setlength{\fboxsep}{0pt}\colorbox{mygreen}{19}\colorbox{myyellow}{895}} includes feature {\setlength{\fboxsep}{0pt}\colorbox{mygreen}{1100}} from the socks as well as features {\setlength{\fboxsep}{0pt}\colorbox{myyellow}{560}} and {\setlength{\fboxsep}{0pt}\colorbox{myyellow}{943}} from the shorts). Notably, the fourth type of token demonstrates that the features of one action can be segmented and grouped with features from adjacent actions. This results in different tokens for the same action depending on the surrounding context, showcasing the context-aware tokenization process of ActionPiece.


\section{Simulation}
\label{sec:result}
In this section, we present our simulation results,
which show that \acp\ is very effective in reducing the user request TTS.
Moreover, the fidelity of end-to-end EPs is improved with the help of entanglement purification.

\begin{table}[ht]
    \vspace{-0.1in}
    \caption{Simulation parameter}
    \vspace{-0.1in}
    \centering
    \begin{tabular}{|c|c||c|c|}
        \hline
        End node processing delay & 100 \unit{\us} & Forward delay & 20 \unit{\us}  \\
        \hline
         \# of memory per node & 10  & Fiber attenuation & 0.2 dB/km  \\
        \hline
         Max \# of memory for \acp & 5 & Link distance & 10 km \\
        \hline 
        Quantum memory efficiency & 0.6 &  Speed of light & $2\text{e}8$ m/s \\
        \hline
        Photon detector efficiency & 0.95  & Coherence time & 2 s \\
        \hline
        Initial EP fidelity & 0.95 &Pauli errors & [$\frac{1}{3},\frac{1}{3},\frac{1}{3}$] \\
        \hline
        Swapping success rate & 1 & Gate fidelity & 0.99\\
        \hline
        Photonic BSM success rate & 0.5 & Measure fidelity & 0.99 \\
        \hline
        Request arrival rate & 10 Hz &  $\delta$ in Algo.~\ref{algo:adaptive} & 0.05  \\
        \hline
    \end{tabular}
    \vspace{-0.1in}
    \label{tab:sim-parameter}
\end{table}


\subsection{Simulation Setting}

In this subsection, we discuss the network topology, request pattern, classical communication latency, and the entanglement generation strategies under study.
Table~\ref{tab:sim-parameter} summarizes the key parameters for our simulations. 

\begin{wrapfigure}{r}{1.45in}
    \vspace{-0.2in}
    \hspace{-0.2in}
    \centering
    \includegraphics[width=1.05\linewidth]{figures/topology.png}
    \vspace{-0.1in}
    \caption{Network topologies.}
    \vspace{-0.1in}
    \label{fig:topology}
\end{wrapfigure}

\subsubsection{Network Topology} 
We consider three network topologies: a small-scale two-node network, a medium-scale 20-node bottleneck network, and a large-scale 200-node autonomous system (AS) network as shown in Fig.~\ref{fig:topology}(a)$\sim$(c).
We assume that all links have a uniform distance of 10~km, each node has 10 quantum memories, and the \acp\ is allowed to use a maximum of five quantum memories at each node.





% \begin{figure}
%     \centering
%     \includegraphics[width=0.3\linewidth]{figures/topology.png}
%     \caption{Will use the wrapped figure}
%     \label{fig:topology}
% \end{figure}

\subsubsection{Request Pattern}
Requests are sampled from a \emph{traffic matrix} whose elements sum to one.
The element $[i,j]$ represents the probability of generating a request in which the initiator is Node-$i$ and the responder is Node-$j$.
The request arrival rate is 10~Hz throughout all experiments (i.e., a new request arrives every 0.1 s).
We assume each request asks for one end-to-end EP with fidelity $> 0.5$, and the difference between the \texttt{start time} and \texttt{end time} is slightly $< 0.1$ s (i.e., ideally a request is served before a new request arrives).
For the two-node topology in Fig.~\ref{fig:topology}(a), there is only one entanglement path with zero hops.
For the bottleneck topology in Fig.~\ref{fig:topology}(b), all the request initiators are on the left side of the bottleneck link, while the responders are on the right side.
As a result, the entanglement paths of the end-to-end EP have two hops (intermediate nodes).
For the AS topology in Fig.~\ref{fig:topology}(c), we create a traffic matrix such that all the entanglement paths have exactly four hops.



\subsubsection{Classical Communication Latency} 
Classical communication is needed for sending measurement results during entanglement generation, swapping, and purification, as well as for protocol coordination tasks.
% However, classical communication is often naively simplified in the literature, such as assuming all pair of nodes have direct fiber connection.
It is considered a major bottleneck for quantum networks~\cite{rozped-thesis-19}.
In this paper, the classical communication latency between node $u$ and $v$ is
\begin{equation}
    l_{(u,v)} = \frac{d_{(u,v)}}{c} + \mathrm{hop}_{(u,v)} \times D_\mathrm{forward} + D_\mathrm{end~process},
    \label{eqn:latency}
\end{equation}
where $d_{(u,v)}$ is the length of the path from $u$ to $v$,
$c$ is the speed of light in optical fiber,
$\mathrm{hop}_{(u,v)}$ is the number of hops along the path,
$D_\mathrm{forward}$ is the delay for forwarding the packet at intermediate nodes, and
$D_\mathrm{end~process}$ is the delay for processing the packets at the two end nodes.
The values of $D_\mathrm{end~process}$ and $D_\mathrm{forward}$ are in the first row of Table~\ref{tab:sim-parameter}.
We assume the packet size of classical communication in quantum networks is \red{under a hundred bytes, considering the payload being a few bytes to account for a few measurement results}. 
For simplicity, we neglect transmission and queueing delays.

\subsubsection{Methods Compared}
We compare the \acp\ with an ``On Demand Only'' (\odo) strategy, where the quantum network starts generating link EPs only \emph{after} a request arrives
\red{and does not include entanglement purification.}
We also compare the \acp\ with a ``Uniform Continuous Entanglement Generation Protocol'' (\ucp), which has the same continuousness (Sec.~\ref{subsec:continuousness}) of the \acp\ but it lacks the adaptive control (Sec.~\ref{subsec:adaptive-control}), i.e. a node continuously generates EPs with neighbor nodes where the neighbors are selected by a uniform probability distribution.
Moreover, the \acp\ has two variations without entanglement purification, where multiple EPs could have been generated on the same neighbor link.
These two variations apply two different strategies: (1) select the freshest EP and (2) select a random EP.
% \acp\ has purification enabled.
All these strategies will be compared by two metrics: request TTS (see Eqn.~\ref{eqn:tts}) and end-to-end EP fidelity.
% \red{\odo\ does not include purification. Purification could increase the TTS at least two times for \odo, but has minimal impact for \ucp\ and \acp.}

\begin{figure}
    \centering
    \vspace{-0.1in}
    \includegraphics[width=0.96\linewidth]{figures/2node.png}
    \vspace{-0.1in}
    \caption{Simulation results for the two-node topology network.}
    \vspace{-0.1in}
    \label{fig:2node}
\end{figure}


\subsection{Simulation Results}

\subsubsection{Two-Node Network Setting}
The \acp\ reduces the TTS by $94\%$ at most compared to the \odo.
As shown in Fig.~\ref{fig:2node}(a), the average TTS for the \odo\ is 5.95~ms, while the average TTS for the \acp\ with and without entanglement purification is 0.39~ms and 0.3~ms, respectively.
0.3~ms is the delay of a classical communication round trip time between two neighbors, see Fig.~\ref{fig:eg} and Eqn.~\ref{eqn:latency}.
The overhead of 0.09 ms is mainly because of failures in entanglement purification.
Occasionally, EPs generated by the \acp\ do not serve any request, because they are all destroyed during a failed entanglement purification attempt.
Despite a small overhead in TTS, the fidelity in Fig.~\ref{fig:2node}(b) shows an improvement of 0.01$\sim$0.02. The \acp's fidelity is 0.955, while the fidelities of the two variations of the \acp\ without entanglement purification are 0.943 and 0.937.
The \odo\ has a constant fidelity 0.949.
%If compare \acp\ with \odo, we are delighted to observe that 
This simulation demonstrates that the \acp\ can not only drastically reduce the TTS, but also improve the fidelity of end-to-end EPs.

\subsubsection{20-Node Network Setting}
The \acp\ reduces the TTS by $70\%$ compared to the \odo\ and $60\%$ compared to the \ucp.
In Fig.~\ref{fig:20node}(a), the TTS for the \acp\ starts at 9~ms, then it gradually decreases to 4~ms before a change in the traffic matrix occurs. 
After the traffic matrix changes, the TTS suddenly increases to 10~ms --which is worse than the beginning-- then it gradually goes down to 4~ms again.
This trend shows the effectiveness of the adaptive control (i.e., it allows a node to dynamically adjust to changing traffic patterns).
In comparison, the TTS for the \odo\ remains oscillating around 13.3~ms, while for the \ucp\ it fluctuates at 10.3~ms.
For fidelity in Fig.~\ref{fig:20node}(b), the \acp's fidelity starts at 0.84 and it gradually increases before the traffic matrix changes, which leads to a sudden decrease in fidelity.
After this sudden decrease, the fidelity gradually increases again and reaches 0.86, which is nearly 0.02 higher than the \odo\ and the \ucp's fidelity.


\subsubsection{200-Node Network Setting}
The trend depicted in Fig.\ref{fig:200as} is very similar to the one observed in Fig.~\ref{fig:20node}.
\red{The \acp\ (6.8~ms) reduces the TTS to $57\%$ and $50\%$ compared to the \odo\ (15.7~ms) and the \ucp\ (13.6~ms), respectively.}
Furthermore, we observe that the fidelity is increased by nearly 0.05 (see Fig.~\ref{fig:200as}(b)).
This improvement is significantly larger than the two-node and the 20-node scenarios.
This is because the fidelity improvement by the specific entanglement purification we use is the highest when the input fidelity is between 0.7 and 0.8 and in Fig.~\ref{fig:200as}(b), the fidelity of EPs happens to fall in this range.

% Another observation is that although the 200-node scenario has a larger network size and more hops in entanglement paths compared to the 20-node scenario, the TTS is lower. This is because the bottleneck in the 20-node scenario is the critical resource and the amount of memory \acp\ at each node is limited to five.
% This highlights the important role of simulations for capacity planning of future quantum networks.

\begin{figure}
    \centering
    \includegraphics[width=0.96\linewidth]{figures/20bottleneck.png}
    \vspace{-0.1in}
    \caption{Simulation results for the 20-node bottleneck topology network.}
    \vspace{-0.1in}
    \label{fig:20node}
\end{figure}
% bottleneck is causing the delay to be higher compared with the autonomous systems, although the node number is a lot smaller
% also the variation is higher

\begin{figure}
    \centering
    \includegraphics[width=0.96\linewidth]{figures/200as.png}
    \vspace{-0.1in}
    \caption{\red{Simulation results for the 200-Node AS topology network.}}
    \vspace{-0.1in}
    \label{fig:200as}
\end{figure}


\section{Analysis}
\label{sect:analysis}

\begin{figure}[t]
    \centering
    \includegraphics[width=0.86\linewidth]{figure/6-analysis/prefilling_attn_kernel.pdf}
    \caption{Prefilling Stage Attention Kernel Evaluation.} 
        
    \label{fig:ana:prefilling_attention}
\end{figure}

\begin{figure}[t]
    \centering
    \includegraphics[width=\linewidth]{figure/6-analysis/hierarchy_NIAH.pdf}
    \caption{\textbf{Hierarchical paging} enables \system to preserve the long-context retrieval capabilities of the original model without increasing the key-value (KV) token budget. We use Llama-3-8B for the ablation.}

    
    \label{fig:ana:our_larger_page}
\end{figure}

\begin{figure}[t]
    \centering
    \includegraphics[width=\linewidth]{figure/6-analysis/selector_overhead.pdf}
    \caption{\textbf{Effect of Reusable Page Selection}. The overhead of the dynamic page selector is significant, as its complexity increases linearly with input sequence length. Our \textit{Reusable Page Selection} effectively mitigates this issue. The latency breakdown is evaluated on Llama-3-8B.} 
    \label{fig:ana:selector_overhead}
\end{figure}


In this section, we present in-depth analysis on our design choices in the \system system from both the accuracy and the efficiency perspective. We also scrutinize the sources of performance gains in \sect{sect:results}.

\subsection{Prefilling Stage Sparse Attention Kernel}

We benchmark the performance of our block sparse attention kernel for the prefilling stage in Figure~\ref{fig:ana:prefilling_attention}. Compared with the implementation by MInference~\cite{jiang2024minference}, our kernel consistently achieves 1.3$\times$ speedup at the same sparsity level. Oracle stands for the theoretical upper-bound speedup ratio: $\text{Latency}_{\text{oracle}} = \text{Latency}_{\text{dense}} * (1-\text{sparsity})$.


\subsection{Effectiveness of Hierarchical Paging}



We use the Needle-in-a-Haystack ~\cite{LLMTest_NeedleInAHaystack} test to demonstrate that the hierarchical paging design effectively maintains the model's long-context capability on larger page blocks without increasing the token budget. In contrast to the performance drop observed with increased page granularity in Figure~\ref{fig:ana:naive-larger-page}, \system leverages a hierarchical page structure to decouple the pruning algorithm’s page granularity from the physical memory layout of the KV cache. This approach enables our sparse attention mechanism to remain both accurate and hardware-efficient. Figure~\ref{fig:ana:our_larger_page} highlights this improvement: with a page size of 64 and the same token budget, \system achieves accuracy comparable to the baseline algorithm~\cite{tang2024quest}, which prunes history tokens at a granularity of 16.

\subsection{Mitigating Page Selection Overhead}



\begin{table}[t]
\centering
\caption{The reusable page selector in \system preserves the model's long-context accuracy while significantly reducing selection overhead by \textbf{4$\times$} with a reuse interval of 4. We evaluate Llama-3-8B on RULER~\cite{nvidia_ruler} at a sequence length of 64K. LServe-$N$ denotes that the token budget for dynamic sparsity is $N$.}

\footnotesize
\scalebox{0.95}{
\begin{tabular}{ccccccc}


\toprule
Reuse Interval & Dense    & 1    & 2    & 4    & 8    & 16   \\ 
\midrule
LServe-4096 & 86.8 & 86.2 & 85.6 & 85.6 & 84.8 & 83.2 \\ 
\midrule			
LServe-8192 & 86.8 & 86.1 & 85.8 & 85.5 & 85.6 & 84.8\\ 
\bottomrule
\end{tabular}
}
\label{tab:ana:reusable_accuracy}
\vspace{10pt}
\end{table}


\paragraph{Reusable Page Selection.} During decoding, although the attention kernel maintains constant complexity due to a capped number of historical KV tokens, the complexity of the page selector still scales linearly with sequence length. As illustrated in Figure~\ref{fig:ana:selector_overhead}, for a sequence length of 128K and a 4K token budget for sparse attention, the page selector (0.24 ms) is already twice as slow as the sparse attention kernel (0.12 ms). With our reusable page selector, however, \system significantly reduces page selection overhead by a factor of $C$, where $C$ is the reuse interval. We further show that \system is resilient to different reuse interval choices. Table~\ref{tab:ana:reusable_accuracy} demonstrates no significant performance degradation until the reuse interval exceeds 8, so we set it to 4 by default in \system.

\paragraph{Context Pooling Overhead.} To enable page selection during decoding, we must calculate representative features using min-max pooling in the prefilling stage. It is important to note that a single pooling kernel executes under \textbf{1 ms}, while the entire prefilling stage completes in approximately 17 seconds with 128K context length. Consequently, the context pooling overhead is negligible.

\subsection{Sparse Attention Kernel for Decoding Stage}

\begin{figure}[t]
    \centering
    \includegraphics[width=\linewidth]{figure/6-analysis/decoding_attn_kernel.pdf}
    \caption{\textbf{Efficiency gains from static and dynamic sparsity in \system}. These sparsity patterns contribute to a compound speedup effect, with static sparsity being more effective at shorter contexts, and dynamic sparsity offering greater benefits at longer contexts. We report the latency of a single attention layer in Llama-2-7B.}
    \label{fig:ana:decoding_attn_kernel}
\end{figure}



We analyze the effectiveness of different sparsity patterns in decoding attention. In Figure~\ref{fig:ana:decoding_attn_kernel}, we apply \textit{static} sparsity by converting 50\% of attention heads to streaming heads, achieving a \textbf{1.3-1.7$\times$} speedup across various input sequence lengths. Additionally, we introduce dynamic sparsity with a fixed KV budget of 4096 tokens, which bounds the computational complexity of decoding attention to a \textbf{constant}, delivering a \textbf{30$\times$} speedup over the dense baseline for an input length of 256K.  Although sparsity selection introduces minor overhead for shorter sequences, this is mitigated by reusable page selection. Additionally, we also perform the end-to-end ablation study in Section \ref{sect:End-to-End Ablation}.



\subsection{End-to-End Speedup Breakdown}
\label{sect:End-to-End Ablation}


\begin{figure}[htb]
\small
\begin{tcolorbox}[left=3pt,right=3pt,top=3pt,bottom=3pt,title=\textbf{Conversation History:}]
[human]: Craft an intriguing opening paragraph for a fictional short story. The story should involve a character who wakes up one morning to find that they can time travel.

...(Human-Bot Dialogue Turns)... \textcolor{blue}{(Topic: Time-Travel Fiction)}

[human]: Please describe the concept of machine learning. Could you elaborate on the differences between supervised, unsupervised, and reinforcement learning? Provide real-world examples of each.

...(Human-Bot Dialogue Turns)... \textcolor{blue}{(Topic: Machine learning Concepts and Types)}


[human]: Discuss antitrust laws and their impact on market competition. Compare the antitrust laws in US and China along with some case studies

...(Human-Bot Dialogue Turns)... \textcolor{blue}{(Topic: Antitrust Laws and Market Competition)}

[human]: The vertices of a triangle are at points (0, 0), (-1, 1), and (3, 3). What is the area of the triangle?

...(Human-Bot Dialogue Turns)... \textcolor{blue}{(Topic: Area and Circumcircle of Triangle)}

...(Human-Bot Dialogue Interaction on Other Topics)... 

\end{tcolorbox}
\begin{tcolorbox}[left=3pt,right=3pt,top=3pt,bottom=3pt,title=\textbf{Question:}]
What title reflects the thematic essence of the Doctor's experience with time travel as described in the story?
\end{tcolorbox}
\begin{tcolorbox}[left=3pt,right=3pt,top=3pt,bottom=3pt,title=\textbf{Ground-Truth Answer:}]
Awakening to Eternity: The Doctor's New Dawn
\end{tcolorbox}
\begin{tcolorbox}[left=3pt,right=3pt,top=3pt,bottom=3pt,title=\textbf{Retrieved Memory (Turn-Level):}]
[human]: Describe the process and write out the balanced chemical equation for the reaction that occurs when solid calcium carbonate reacts with hydrochloric acid to form aqueous calcium chloride, carbon dioxide, and water. What type of reaction is this, and what observations might indicate that the reaction is taking place?

[bot]: When solid calcium carbonate ($\text{CaCO}_{3}$) reacts with hydrochloric acid (HCl), the products formed are aqueous calcium chloride ($\text{CaCl}_{2}$), carbon dioxide ($\text{CO}_{2}$), and water ($\text{H}_{2}\text{O}$). The balanced chemical equation for this reaction is: ... \textcolor{blue}{(Topic: Calcium Carbonate and Hydrochloric Acid Reaction)}

\textcolor{brown}{
[human]: Write a title for your story.}

\textcolor{brown}{
[bot]: "Awakening to Eternity: The Doctor's New Dawn".
} \textcolor{blue}{(Topic: Time-Travel Fiction)}

[human]: Analyze the following customer reviews from different sources for three different smartphones - the latest iPhone, Samsung Galaxy, and Google Pixel - and provide an overall rating for each phone on a scale of 1 to 10. Consider the following complex and contradictory reviews: - TechRadar's review of the latest iPhone: The new iPhone is a stunning triumph of engineering that sets a new bar for smartphone performance and camera quality. However, the incremental design and high price mean it lacks the 'wow' factor of previous iPhones. Still, its power and intelligence are unrivaled. - CNET's review of the latest Samsung Galaxy: ... Return the answer as a JSON object with the overall ratings for each phone out of 10, to one decimal place.

[bot]: ```json { "iPhone": 8.5, "Samsung Galaxy": 8.0, "Google Pixel": 7.5 } ```

\textcolor{blue}{(Topic: Smartphone Review Ratings Analysis)}
\end{tcolorbox}
\begin{tcolorbox}[left=3pt,right=3pt,top=3pt,bottom=3pt,title=\textbf{Answer (Turn-Level Retrieval):}]
As an AI language model, I do not have access to the story you mentioned earlier. Please provide me with the story or the necessary details so that I can assist you better.

\textbf{\textcolor{red}{(The next part of this figure~\ref{fig: case_study_segment_vs_turn} will be continued on the next page.)}}
\end{tcolorbox}
\end{figure}

\begin{figure}[htb]
% \ContinuedFloat
\small
\begin{tcolorbox}[left=3pt,right=3pt,top=3pt,bottom=3pt,title=\textbf{Retrieved Memory (Ours):}]
[human]: Craft an intriguing opening paragraph for a fictional short story. The story should involve a character who wakes up one morning to find that they can time travel.

[bot]: In the hushed pre-dawn hours, when the world still slumbers under the cloak of night, Alex found themselves caught in the fragile boundary between sleep and wakefulness. ...

[human]: Summarize the story with three bullet points using only nouns and adjectives, without verbs.

[bot]: - Alex, pre-dawn awakening, time-travel discovery - Disorientation, vertigo, reality's malleability - Sunlight, new day, astonishing truth

[human]: Is there any famous fictional character that can time travel? If so, can you rewrite your story with him or her?

[bot]: - \textcolor{brown}{Doctor Who, a renowned time-traveling character from British television, is a perfect fit for this scenario.} In the silent embrace of early morning, where shadows merge with the fading night, the Doctor found themselves in the thin veil separating dreams from consciousness. It was here, amidst the soft whispers of time, that an unusual sensation enveloped them—a sense of time's fabric slipping through their fingers like grains of sand. As dawn's light broke through the darkness, painting the room in hues of gold and amber, the Doctor's eyes opened to what appeared to be an ordinary day. \textcolor{brown}{However, upon standing, a wave of dizziness overwhelmed them, the surroundings blurred, and in that moment of confusion, the Doctor realized a profound truth: they had not only awakened to a new dawn but had also stumbled upon the incredible ability to navigate the currents of time at will.}

\textcolor{brown}{[human]: Write a title for your story.}

\textcolor{brown}{[bot]: "Awakening to Eternity: The Doctor's New Dawn"} \textcolor{blue}{(Topic: Time-Travel Fiction)}
\end{tcolorbox}
\begin{tcolorbox}[left=3pt,right=3pt,top=3pt,bottom=3pt,title=\textbf{Answer (Ours):}]
The title "Awakening to Eternity: The Doctor's New Dawn" reflects the thematic essence of the Doctor's experience with time travel as described in the story. It captures the idea of a new beginning, a sense of eternity, and the awakening of a new power that will change the course of the Doctor's life forever.
\end{tcolorbox}
\caption{Case study comparing our \sysname\ with the \textit{turn-level} memory baseline. When constructing the memory bank at the \textit{turn-level}, the agent fails to provide a satisfactory response due to fragmentation: the retrieved dialogue turns can not capture a complete human-agent interaction related to the query, missing critical information highlighted in \textcolor{brown}{brown}. The dialogue topic is emphasized in \textcolor{blue}{blue}. Some context is omitted (replaced by ...) due to length constraints.}
\label{fig: case_study_segment_vs_turn}
\end{figure}

\newpage

\begin{figure}[htbp]
\small
\begin{tcolorbox}[left=3pt,right=3pt,top=3pt,bottom=3pt,title=\textbf{Conversation History:}]
[human]: Explain what's base rate fallacy and list five specific examples of ...

... (Human-Bot Dialogue Turns) ... \textcolor{blue}{(Topic: Base Rate Fallacy In Politics)}


[human]: How many integers are in the solution of the inequality $|x + 5| < 10$?

... (Human-Bot Dialogue Turns) ... \textcolor{blue}{(Topic: Number of Integers in Inequalities)}

[human]: How have the Alps and Rhine River influenced settlement and agriculture in Western Europe?

... (Human-Bot Dialogue Turns) ... \textcolor{blue}{(Topic: Alps and Rhine River Impact)}

[human]: In a survey conducted at a local high school, preferences for a new school color were measured: 58\% of students liked the color blue, 45\% preferred green, and 22\% liked both colors. If we randomly pick a student from the school, what's the probability that they would like neither blue nor green?

... (Human-Bot Dialogue Turns) ... \textcolor{blue}{(Topic: Probability of Liking Colors in Survey)}

[human]: Imagine you are participating in a race with a group of people. If you have just overtaken the second person, what's your current position? Where is the person you just overtook?

... (Human-Bot Dialogue Turns) ... \textcolor{blue}{(Topic: Overtaking Positions in Race)}
\end{tcolorbox}
\begin{tcolorbox}[left=3pt,right=3pt,top=3pt,bottom=3pt,title=\textbf{Question:}]
[human]: In the inequality question, how many positive integers are in the solution of the second inequality?
\end{tcolorbox}
\begin{tcolorbox}[left=3pt,right=3pt,top=3pt,bottom=3pt,title=\textbf{Ground-Truth Answer:}]
In the second inequality, $(|x + 10| < 5)$, the solution range is $(-15 < x < -5)$. Within this range, there are no positive integers, as all values of $(x)$ that satisfy the inequality are negative.
\end{tcolorbox}
\begin{tcolorbox}[left=3pt,right=3pt,top=3pt,bottom=3pt,title=\textbf{Retrieved Memory (Session-Level):}]
[human]: Explain what's base rate fallacy and list five specific examples of ...

... (Human-Bot Dialogue Turns) ... \textcolor{blue}{(Topic: Base Rate Fallacy In Politics)}

[human]: How many integers are in the solution of the inequality $|x + 5| < 10$?

[bot]: To solve the inequality $|x + 5| < 10$, ... 

\textcolor{brown}{
[human]: What about $|x + 10| < 5$?
}

\textcolor{brown}{
[bot]: To solve the inequality $|x + 10| < 5$, we ...} \textcolor{gray}{(``overwhelmed'' by other irrelevant, redundant content)}

... (Human-Bot Dialogue Turns) ...  \textcolor{blue}{(Topic: Number of Integers in Inequalities)}

% \textcolor{brown}{
% [human]: What about $|x + 10| < 5$
% }

% \textcolor{brown}{
% [bot]: To solve the inequality $|x + 10| < 5$, ...
% } \textcolor{blue}{(Topic: Number of Integers in Inequalities)}

[human]: How have the Alps and Rhine River influenced settlement and agriculture in Western Europe?

... (Human-Bot Dialogue Turns) ... \textcolor{blue}{(Topic: Alps and Rhine River Impact)}

[human]: In a survey conducted at a local high school, ...

... (Human-Bot Dialogue Turns) ... \textcolor{blue}{(Topic: Probability of Liking Colors in Survey)}
\end{tcolorbox}
\begin{tcolorbox}[left=3pt,right=3pt,top=3pt,bottom=3pt,title=\textbf{Answer (Session-Level Retrieval):}]
I apologize for the confusion earlier. To answer your question, the second inequality was not provided in our previous conversation. If you could provide the inequality, I would be happy to help you find the solution and the number of positive integers in the solution set.

\textbf{\textcolor{red}{(The next part of this figure~\ref{fig: case_study_segment_vs_session} will be continued on the next page.)}}
\end{tcolorbox}
\end{figure}
\begin{figure}[htbp]
% \ContinuedFloat
\small
\begin{tcolorbox}[left=3pt,right=3pt,top=3pt,bottom=3pt,title=\textbf{Retrieved Memory (Ours):}]
[human]: How many integers are in the solution of the inequality $|x + 5| < 10$?

[bot]: To solve the inequality $|x + 5| < 10$, we break it into two cases based on the definition of absolute value: 
1. $x + 5 < 10$ 
2. $-(x + 5) < 10$, which simplifies to $x + 5 > -10$ 
Solving each case: 
1. $x < 5$ 
2. $x > -15$ 
Combining these, we get $-15 < x < 5$. This inequality means that $x$ can be any number greater than $-15$ and less than $5$. To find how many integers satisfy this condition, we list them: $-14, -13, -12, -11, -10, -9, -8, -7, -6, -5, -4, -3, -2, -1, 0, 1, 2, 3, 4$. There are 19 integers that satisfy the inequality $|x + 5| < 10$.

\textcolor{brown}{
[human]: What about $|x + 10| < 5$?
}

\textcolor{brown}{
[bot]: To solve the inequality $|x + 10| < 5$, we again consider two cases: 
1. $x + 10 < 5$ 
2. $-(x + 10) < 5$, which simplifies to $x + 10 > -5$ 
Solving each case: 
1. $x < -5$ 
2. $x > -15$ 
Combining these, we get $-15 < x < -5$. This inequality means that $x$ can be any number greater than $-15$ and less than $-5$. To find how many integers satisfy this condition, we list them: $-14, -13, -12, -11, -10, -9, -8, -7, -6$. There are 9 integers that satisfy the inequality $|x + 10| < 5$.
}

[human]: What about $|x + 10| < 10$?

[bot]: To solve the inequality $|x + 10| < 10$, we consider two cases: 
1. $x + 10 < 10$ 
2. $-(x + 10) < 10$, which simplifies to $x + 10 > -10$ 
Solving each case: 
1. $x < 0$ 
2. $x > -20$ 
Combining these, we get $-20 < x < 0$. This inequality means that $x$ can be any number greater than $-20$ and less than $0$. To find how many integers satisfy this condition, we list them: $-19, -18, -17, -16, -15, -14, -13, -12, -11, -10, -9, -8, -7, -6, -5, -4, -3, -2, -1$. There are 19 integers that satisfy the inequality $|x + 10| < 10$. \textcolor{blue}{(Topic: Number of Integers in Inequalities)}
\end{tcolorbox}
\begin{tcolorbox}[left=3pt,right=3pt,top=3pt,bottom=3pt,title=\textbf{Answer (Ours):}]
To find the number of positive integers that satisfy the inequality $(|x + 10| < 5)$, we need to look for positive integers in the solution set $(-15 < x < -5)$. Since there are no positive integers in this interval, the answer is 0.

\end{tcolorbox}
\caption{Case study comparing our \sysname\ with the \textit{session-level} memory baseline. When constructing the memory bank at the \textit{session-level}, the agent fails to provide a satisfactory response due to the inclusion of excessive irrelevant information within a session. This distracts the generator from focusing on the truly useful information highlighted in \textcolor{brown}{brown}. The dialogue topic is emphasized in \textcolor{blue}{blue}. Some context is omitted (replaced by ...) due to length constraints.}
\label{fig: case_study_segment_vs_session}
\end{figure}



\begin{figure}[htb]
\small
\begin{tcolorbox}[left=3pt,right=3pt,top=3pt,bottom=3pt,title=\textbf{Conversation History:}]
[human]: Photosynthesis is a vital process for life on Earth. Could you outline the two main stages of photosynthesis, including where they take place within the chloroplast, and the primary inputs and outputs for each stage? ... (Human-Bot Dialogue Turns)... \textcolor{blue}{(Topic: Photosynthetic Energy Production)}

[human]: Please assume the role of an English translator, tasked with correcting and enhancing spelling and language. Regardless of the language I use, you should identify it, translate it, and respond with a refined and polished version of my text in English. 

... (Human-Bot Dialogue Turns)...  \textcolor{blue}{(Topic: Language Translation and Enhancement)}

[human]: Suggest five award-winning documentary films with brief background descriptions for aspiring filmmakers to study.

\textcolor{brown}{[bot]: ...
5. \"An Inconvenient Truth\" (2006) - Directed by Davis Guggenheim and featuring former United States Vice President Al Gore, this documentary aims to educate the public about global warming. It won two Academy Awards, including Best Documentary Feature. The film is notable for its straightforward yet impactful presentation of scientific data, making complex information accessible and engaging, a valuable lesson for filmmakers looking to tackle environmental or scientific subjects.}

... (Human-Bot Dialogue Turns)... 
\textcolor{blue}{(Topic: Documentary Films Recommendation)}

[human]: Given the following records of stock prices, extract the highest and lowest closing prices for each month in the year 2022. Return the results as a CSV string, with one line allocated for each month. Date,Open,High,Low,Close,Volume ... ... (Human-Bot Dialogue Turns)...  \textcolor{blue}{(Topic: Stock Prices Analysis)}

[human]: The city of Vega intends to build a bridge that will span the Vegona River, covering a distance of 1.8 kilometers. The proposed location falls within a seismically active area that has experienced several high-magnitude earthquakes. Given these circumstances, what would be the best approach to constructing the bridge? ... (Human-Bot Dialogue Turns)...  \textcolor{blue}{(Topic: Seismic Bridge Construction)}
\end{tcolorbox}
\begin{tcolorbox}[left=3pt,right=3pt,top=3pt,bottom=3pt,title=\textbf{Question:}]
Which documentary won two Academy Awards and what lesson does it offer to filmmakers?
\end{tcolorbox}
\begin{tcolorbox}[left=3pt,right=3pt,top=3pt,bottom=3pt,title=\textbf{Ground-Truth Answer:}]
``An Inconvenient Truth" (2006) won two Academy Awards. It offers the lesson of making complex information accessible and engaging, a valuable lesson for filmmakers looking to tackle environmental or scientific subjects.
\end{tcolorbox}
\begin{tcolorbox}[left=3pt,right=3pt,top=3pt,bottom=3pt,title=\textbf{Memory generated by RecurSum~\citep{wang2023recursively}:}]
The dialogue provided detailed information on various topics, including the stages of photosynthesis, energy production estimates through photosynthesis for different organisms, translations and enhancements of phrases in different languages, recommendations for documentary films, a pitch for a film about overcoming adversity, analysis of stock prices, and considerations for constructing a bridge in a seismically active area. Here's a summary of the key points: ...

\textbf{Documentary Film Recommendations}: Suggested five award-winning documentary films for aspiring filmmakers to study, with brief background descriptions for each. The response included:

-  "March of the Penguins" (2005) - Emperor penguins' journey in Antarctica; noted for its cinematography and portrayal of the penguins' life cycle.

-  "Citizenfour" (2014) - Edward Snowden and NSA spying scandal; known for its real-time exploration of Snowden's revelations.

-  "20 Feet from Stardom" (2013) - Stories of backup singers behind musical legends; celebrates unsung heroes in the music industry.

-  "Man on Wire" (2008) - Philippe Petit's 1974 high-wire walk between the Twin Towers; combines personal footage, interviews, and re-enactments.

-  \textcolor{brown}{"An Inconvenient Truth" (2006) - Al Gore's documentary on global warming; known for its presentation of scientific data.}
\end{tcolorbox}
\begin{tcolorbox}[left=3pt,right=3pt,top=3pt,bottom=3pt,title=\textbf{Answer (RSum):}]
I'm sorry, but based on the updated context you provided, I don't see any previous discussion related to the question you asked. However, to answer your question, the documentary that won two Academy Awards is "Free Solo," which won for Best Documentary Feature and Best Achievement in Film Editing in 2019. 
% The lesson it offers to filmmakers is the importance of taking risks and pushing boundaries in order to create something truly unique and impactful.
....
The film follows climber Alex Honnold as he attempts to free solo climb El Capitan in Yosemite National Park, and the filmmakers had to take risks themselves in order to capture the footage and tell the story.

\textbf{\textcolor{red}{(The next part of this figure~\ref{fig: case_study_segment_vs_rsum} will be continued on the next page.)}}
\end{tcolorbox}
\end{figure}
\begin{figure}[htbp]
% \ContinuedFloat
\small
\begin{tcolorbox}[left=3pt,right=3pt,top=3pt,bottom=3pt,title=\textbf{Retrieved Memory (Ours):}]
[human]: Suggest five award-winning documentary films with brief background descriptions for aspiring filmmakers to study.

[bot]: 1. **"March of the Penguins" (2005)** - Directed by Luc Jacquet, this French feature-length nature documentary delves into the annual journey of Emperor penguins across the Antarctic. It won the Academy Award for Best Documentary Feature. The film is celebrated for its stunning cinematography and its detailed, heartfelt portrayal of the penguins' life cycle, making it a must-study for filmmakers interested in wildlife documentaries. 2. **"Citizenfour" (2014)** ... 5. \textcolor{brown}{**"An Inconvenient Truth" (2006)** - Directed by Davis Guggenheim and featuring former United States Vice President Al Gore, this documentary aims to educate the public about global warming. It won two Academy Awards, including Best Documentary Feature.} The film is notable for its straightforward yet impactful presentation of scientific data, making complex information accessible and engaging, \textcolor{brown}{a valuable lesson for filmmakers looking to tackle environmental or scientific subjects.}

[human]: With the spirit in the first film, craft a succinct and persuasive pitch for a film about overcoming adversity.

[bot]: Drawing inspiration from the resilience and majestic journey of Emperor penguins in "March of the Penguins," our documentary, "Unyielding Spirits: The Human Odyssey," will explore the profound stories of individuals around the globe who have faced and overcome staggering adversities. Through breathtaking cinematography and intimate storytelling, we ...
\textcolor{blue}{(Topic: Documentary Films Recommendation)}
\end{tcolorbox}
\begin{tcolorbox}[left=3pt,right=3pt,top=3pt,bottom=3pt,title=\textbf{Answer (Ours):}]
``An Inconvenient Truth" won two Academy Awards, including Best Documentary Feature. The film offers a valuable lesson to filmmakers looking to tackle environmental or scientific subjects, as it presents complex information in a straightforward yet impactful way, making it accessible and engaging to a wider audience.


\end{tcolorbox}
\caption{Case study comparing our \sysname\ with RecurSum~\citep{wang2023recursively}. When using RecurSum, the agent fails to provide a satisfactory answer due to information loss during the summarization process. Summarization omits essential details needed to answer the user's new request, as highlighted in \textcolor{brown}{brown}. The dialogue topic is emphasized in \textcolor{blue}{blue}. Some context is omitted (replaced by ...) due to length constraints.}
\label{fig: case_study_segment_vs_rsum}
\end{figure}




\begin{figure}[htb]
\small
\begin{tcolorbox}[left=3pt,right=3pt,top=3pt,bottom=3pt,title=\textbf{Conversation History:}]
[human]: Photosynthesis is a vital process for life on Earth. Could you outline the two main stages of photosynthesis, including where they take place within the chloroplast, and the primary inputs and outputs for each stage? ... (Human-Bot Dialogue Turns)... \textcolor{blue}{(Topic: Photosynthetic Energy Production)}

[human]: Please assume the role of an English translator, tasked with correcting and enhancing spelling and language. Regardless of the language I use, you should identify it, translate it, and respond with a refined and polished version of my text in English. 

... (Human-Bot Dialogue Turns)...  \textcolor{blue}{(Topic: Language Translation and Enhancement)}

[human]: Suggest five award-winning documentary films with brief background descriptions for aspiring filmmakers to study.

\textcolor{brown}{[bot]: ...
5. \"An Inconvenient Truth\" (2006) - Directed by Davis Guggenheim and featuring former United States Vice President Al Gore, this documentary aims to educate the public about global warming. It won two Academy Awards, including Best Documentary Feature. The film is notable for its straightforward yet impactful presentation of scientific data, making complex information accessible and engaging, a valuable lesson for filmmakers looking to tackle environmental or scientific subjects.}

... (Human-Bot Dialogue Turns)... 
\textcolor{blue}{(Topic: Documentary Films Recommendation)}

[human]: Given the following records of stock prices, extract the highest and lowest closing prices for each month in the year 2022. Return the results as a CSV string, with one line allocated for each month. Date,Open,High,Low,Close,Volume ... ... (Human-Bot Dialogue Turns)...  \textcolor{blue}{(Topic: Stock Prices Analysis)}

[human]: The city of Vega intends to build a bridge that will span the Vegona River, covering a distance of 1.8 kilometers. The proposed location falls within a seismically active area that has experienced several high-magnitude earthquakes. Given these circumstances, what would be the best approach to constructing the bridge? ... (Human-Bot Dialogue Turns)...  \textcolor{blue}{(Topic: Seismic Bridge Construction)}
\end{tcolorbox}
\begin{tcolorbox}[left=3pt,right=3pt,top=3pt,bottom=3pt,title=\textbf{Question:}]
Which documentary won two Academy Awards and what lesson does it offer to filmmakers?
\end{tcolorbox}
\begin{tcolorbox}[left=3pt,right=3pt,top=3pt,bottom=3pt,title=\textbf{Ground-Truth Answer:}]
"An Inconvenient Truth" (2006) won two Academy Awards. It offers the lesson of making complex information accessible and engaging, a valuable lesson for filmmakers looking to tackle environmental or scientific subjects.
\end{tcolorbox}
\begin{tcolorbox}[left=3pt,right=3pt,top=3pt,bottom=3pt,title=\textbf{Memory generated by ConditionMem~\citep{yuan2023evolving}:}]
- \textcolor{brown}{The user is seeking recommendations for award-winning documentary films, along with brief background descriptions, for educational purposes.} This request likely stems from a desire to learn from successful examples in the documentary filmmaking field, possibly to improve their own filmmaking skills or to gain inspiration for their projects. 

- \textcolor{brown}{User is interested in award-winning documentary films for educational purposes, aiming to study their backgrounds and filmmaking techniques.}

- The user's request for crafting a succinct and persuasive pitch for a film about overcoming adversity is inspired by the spirit of "March of the Penguins," a documentary that beautifully captures the Emperor penguins' annual journey across the Antarctic, highlighting themes of endurance and survival against the odds. The user seeks to emulate the emotional depth and cinematic quality that made "March of the Penguins" compelling, aiming to create a pitch that resonates with similar themes of resilience and triumph over challenges. 

- User is interested in creating content that focuses on themes of overcoming adversity, inspired by the emotional depth and cinematic excellence of documentaries like "March of the Penguins." They value storytelling that captures the essence of endurance and survival against the odds.
% : The user's request to write a budget plan for the video production directly follows the task of creating a video script aimed at training new employees of a car wash business in Japan, focusing on business etiquette norms.
% The user is likely seeking to understand the financial requirements to bring the previously crafted script to life, indicating a move from the planning and content development phase to the practical aspects of production. 

... (Other Irrelevant Memory Records) ...
\end{tcolorbox}
\begin{tcolorbox}[left=3pt,right=3pt,top=3pt,bottom=3pt,title=\textbf{Answer (Conditional Memory):}]
The documentary that won two Academy Awards is "Free Solo," which follows climber Alex Honnold as he attempts to free solo climb El Capitan in Yosemite National Park. The film offers a lesson to filmmakers on the importance of capturing authentic and compelling human stories, as well as the technical challenges of filming in extreme environments.

\textbf{\textcolor{red}{(The next part of this figure~\ref{fig: case_study_segment_vs_condmem} will be continued on the next page.)}}
\end{tcolorbox}
\end{figure}
\begin{figure}[htbp]
% \ContinuedFloat
\small
\begin{tcolorbox}[left=3pt,right=3pt,top=3pt,bottom=3pt,title=\textbf{Retrieved Memory (Ours):}]
[human]: Suggest five award-winning documentary films with brief background descriptions for aspiring filmmakers to study.

[bot]: 1. **"March of the Penguins" (2005)** - Directed by Luc Jacquet, this French feature-length nature documentary delves into the annual journey of Emperor penguins across the Antarctic. It won the Academy Award for Best Documentary Feature. The film is celebrated for its stunning cinematography and its detailed, heartfelt portrayal of the penguins' life cycle, making it a must-study for filmmakers interested in wildlife documentaries. 2. **"Citizenfour" (2014)** ... 5. \textcolor{brown}{**"An Inconvenient Truth" (2006)** - Directed by Davis Guggenheim and featuring former United States Vice President Al Gore, this documentary aims to educate the public about global warming. It won two Academy Awards, including Best Documentary Feature.} The film is notable for its straightforward yet impactful presentation of scientific data, making complex information accessible and engaging, a valuable lesson for filmmakers looking to tackle environmental or scientific subjects.

[human]: With the spirit in the first film, craft a succinct and persuasive pitch for a film about overcoming adversity.

[bot]: Drawing inspiration from the resilience and majestic journey of Emperor penguins in "March of the Penguins," our documentary, "Unyielding Spirits: The Human Odyssey," will explore the profound stories of individuals around the globe who have faced and overcome staggering adversities. Through breathtaking cinematography and intimate storytelling, we ...
\textcolor{blue}{(Topic: Documentary Films Recommendation)}
\end{tcolorbox}
\begin{tcolorbox}[left=3pt,right=3pt,top=3pt,bottom=3pt,title=\textbf{Answer (Ours):}]
``An Inconvenient Truth" won two Academy Awards, including Best Documentary Feature. The film offers a valuable lesson to filmmakers looking to tackle environmental or scientific subjects, as it presents complex information in a straightforward yet impactful way, making it accessible and engaging to a wider audience.
\end{tcolorbox}
\caption{Case study comparing our \sysname\ with ConditionMem~\citep{yuan2023evolving}. When using ConditionMem, the agent fails to provide a satisfactory answer due to (1) information loss during the summarization process and (2) the incorrect discarding of turns that are actually useful, as highlighted in \textcolor{brown}{brown}. The dialogue topic is emphasized in \textcolor{blue}{blue}. Some context is omitted (replaced by ...) due to length constraints.}
\label{fig: case_study_segment_vs_condmem}
\end{figure}


In Figure~\ref{fig:ana:case_study}, we highlight the sources of performance improvement in \system. By leveraging static sparsity, \system achieves end-to-end speedups of up to \textbf{1.7$\times$} over the dense baseline. Additionally, dynamic sparsity, aided by a reusable page selector, significantly reduces generation latency, yielding a \textbf{7.7$\times$} speedup for sequence lengths of 256K. Lastly, \system configures sparse patterns through offline profiling, effectively avoiding slowdowns from dynamic sparsity at shorter context lengths.


























\section{Related work}
\label{sec:literature}

\subsection{Code review automation}
Several approaches were proposed to assist code review. Gupta et al. \cite{gupta2018intelligent} introduced an LSTM-based model trained on positive and negative (code, review) pairs, selecting candidate reviews based on code similarity and predicting relevance scores. Siow et al. \cite{siow2020core} enhanced this with multi-level embeddings, leveraging word-level and character-level representations to better capture the semantics of code and reviews.

With the advent of large language models, the focus has shifted toward generative models to fully automate code review tasks. Tufano et al. \cite{tufan2021towards} developed a transformer-based model to suggest code edits before and after code review, later enhancing it by pre-training T5 on Java and technical English~\cite{tufano2022using}. Li et al. \cite{li2022automating} pre-trained CodeT5 on a multilingual dataset and fine-tuned it for code review tasks like quality estimation, review generation, and code refinement. Sghaier et al. \cite{ben2024improving} further advanced this area by applying cross-task knowledge distillation to address successive code review tasks jointly, enhancing performance and promoting tasks' interdependence.

Current research efforts have significantly advanced the automation of code review, introducing different and innovative approaches that enhance code review tasks. However, the performance of these automated approaches remains limited in terms of correctness, as indicated by low BLEU scores, suggesting that further refinement is needed to achieve higher accuracy and reliability as expected in practical software development contexts.


\subsection{LLMs and static analysis combination}

Recent research has increasingly focused on enhancing LLM-based solutions for software engineering using several techniques. One is by integrating them with static analysis tools, addressing the challenge of reducing inaccurate or incomplete results. 

In automated program repair, RepairAgent \cite{bouzenia2024repairagent} employs static analysis to gather contextual data that guides LLM-driven code correction, while PyTy \cite{chow2024pyty} relies on type-checking mechanisms to validate the accuracy of LLM-generated candidates in resolving static type inconsistencies. For software testing, approaches like TECO \cite{nie2023learning} apply static analysis to derive semantic features for training transformers in test completion, while ChatTester \cite{yuan2023no} and TestPilot \cite{schafer2023adaptive} utilize similar techniques to prepare contextual information that supports iterative LLM-based test code repair processes. For bug detection, LLMs were combined with static analysis to reduce false positives. SkipAnalyzer \cite{mohajer2023skipanalyzer} and GPTScan~\cite{sun2024gptscan} use static analysis to validate LLM predictions, while D2A~\cite{zheng2021d2a} and ReposVul \cite{wang2024reposvul} refine bug labeling and re-rank predictions.
For code completion, STALL+ \cite{liu2024stall+} integrates static analyzers with LLMs through a multi-phase approach involving prompting, decoding, and post-processing.

These studies illustrate the effectiveness of combining LLMs with static analysis across tasks like program repair, bug detection, testing, and code completion. However, this integration has yet to be explored for code review.


% ==================================================================

% Research is increasingly focusing on improving LLM-based solutions for software engineering tasks by integrating them with static analysis tools and techniques. This combination addresses one of the key challenges in using LLMs by reducing the risk of generating inaccurate or incomplete results. For instance, in automated program repair, several approaches leverage static analysis to validate or improve LLM-generated patches. RepairAgent \cite{bouzenia2024repairagent} uses static analysis to extract relevant information that guides LLM-based repair suggestions, while PyTy \cite{chow2024pyty} employs a type checker to ensure that each candidate fix generated by the LLM resolves type-related issues. Repolit \cite{wei2023copiloting} integrates a code completion engine to refine patch generation dynamically.

% For software testing, researchers have developed methods that combine traditional test generation with LLMs to improve test coverage and accuracy. TECO \cite{nie2023learning} employs a transformer model trained for test completion, which relies on semantic features obtained through static analysis. Additionally, ChatTester \cite{yuan2023no} uses static analysis to provide contextual information for iterative test repair guided by LLMs. 

% In the area of bug detection, combining LLMs with static analysis has shown promise in reducing false positives. For example, SkipAnalyzer \cite{mohajer2023skipanalyzer} filters warnings produced by both the LLM and the Infer static analysis tool, and GPTScan \cite{sun2024gptscan} applies static analysis to validate key variables and statements recognized by the LLM, increasing confidence in detecting vulnerabilities. Additionally, D2A \cite{zheng2021d2a} leverages LLMs alongside differential analysis of bug-fixing commits to improve labeling and reduce false positives in static analysis outputs. Similarly, ReposVul \cite{wang2024reposvul} employs a combination of static analysis tools and LLMs to evaluate the relevance of code changes to vulnerability fixes. During model inference, static analysis is employed to adjust the model's predicted token probabilities. After model inference, static analysis is used to filter and re-rank these candidates.

% STALL+ \cite{liu2024stall+} integrates static analyzers with LLMs for code completion through a multi-phase approach involving prompting, decoding, and post-processing. Before inference, static analysis is used to extract useful code contexts from other files in the repository. 

% These studies demonstrate the effectiveness of combining LLMs with static analysis across various software engineering tasks, including program repair, bug detection, test generation, and code completion.

% \imene{can we say that here?} While these approaches underscore the potential benefits of integrating LLMs and static analysis, specific work on using this combination for code review has not yet been explored. 





\section{Conclusion}
We reveal a tradeoff in robust watermarks: Improved redundancy of watermark information enhances robustness, but increased redundancy raises the risk of watermark leakage. We propose DAPAO attack, a framework that requires only one image for watermark extraction, effectively achieving both watermark removal and spoofing attacks against cutting-edge robust watermarking methods. Our attack reaches an average success rate of 87\% in detection evasion (about 60\% higher than existing evasion attacks) and an average success rate of 85\% in forgery (approximately 51\% higher than current forgery studies). 



% In the unusual situation where you want a paper to appear in the
% references without citing it in the main text, use \nocite
% \nocite{langley00}

\bibliography{acl,custom}
\bibliographystyle{icml2025}


%%%%%%%%%%%%%%%%%%%%%%%%%%%%%%%%%%%%%%%%%%%%%%%%%%%%%%%%%%%%%%%%%%%%%%%%%%%%%%%
%%%%%%%%%%%%%%%%%%%%%%%%%%%%%%%%%%%%%%%%%%%%%%%%%%%%%%%%%%%%%%%%%%%%%%%%%%%%%%%
% APPENDIX
%%%%%%%%%%%%%%%%%%%%%%%%%%%%%%%%%%%%%%%%%%%%%%%%%%%%%%%%%%%%%%%%%%%%%%%%%%%%%%%
%%%%%%%%%%%%%%%%%%%%%%%%%%%%%%%%%%%%%%%%%%%%%%%%%%%%%%%%%%%%%%%%%%%%%%%%%%%%%%%
\newpage
\appendix
\onecolumn
\newpage
\appendix
\section{Appendix}

\subsection{Conversational agent prompts for generating stable diffusion prompts in art-making phase}

\textbf{Role:} You will be able to capture the essence of the sessions and drawings in the recordings based on the art therapy session recordings I have given you and summarize them into a short sentence that will be used to guide the PROMPT for the Stable Diffusion model.

\vspace{0.5em} % 添加一些垂直间距

\textbf{Example input:}

\begin{itemize}[leftmargin=*]
    \item \textbf{USER:} [user-drawn] I drew the ocean. [canvas content] There is nothing on the canvas right now.
    \item \textbf{ASSISTANT:} What kind of ocean is this?
    \item \textbf{USER:} [user-drawn] I drew grass. [canvas content] Now there is an ocean on the canvas.
    \item \textbf{ASSISTANT:} What kind of grass is this?
    \item \textbf{USER:} [user-drawn] I drew the sky. [canvas content] Now there is grass and ocean on the canvas.
    \item \textbf{ASSISTANT:} What kind of sky is this?
    \item \textbf{USER:} [user-drawn] I drew mountains. [canvas content] Now there is sky, grass, and ocean on the canvas.
    \item \textbf{ASSISTANT:} What kind of mountain is this?
    \item \textbf{USER:} [user-drawn] I drew clouds. [canvas content] Now there is sky, mountain, grass, and ocean on the canvas.
    \item \textbf{ASSISTANT:} What kind of cloud is this?
    \item \textbf{USER:} [user dialogue] Colorful clouds, emerald green mountains and grass, choppy ocean
\end{itemize}

\vspace{0.5em} % 添加一些垂直间距

\textbf{Task:}

\begin{enumerate}[label=\textbf{Step \arabic*:}]
    \item \textbf{[Step 0]} Read the given transcript of the art therapy session, focusing on the content of \texttt{user: [user drawing]} and \texttt{user: [user dialog]}: Go to \textbf{[Step 1]}.
    \item \textbf{[Step 1]} Based on the input, find the last entry of user's input with \texttt{[canvas content]}, find the keywords of the screen elements that the canvas now contains (in the example input above, it is: sky, grass, sea), separate the keywords of each element with a comma, and add them to the generated result. Examples: [keyword1], [keyword2], [keyword3], \dots, [keyword n].
    \item \textbf{[Step 2]} Find whether there are more specific descriptions of the keywords of the painting elements in \texttt{[Step 1]} in \texttt{[User Dialog]} according to the input. If not, this step ends into \textbf{[Step 3]}; if there are, combine these descriptions and the keywords corresponding to the descriptions into a new descriptive phrase, and replace the previous keywords with the new phrases. Examples: [description of keyword 1] [keyword 1], [keyword 2 description of keyword 2], [description of keyword 3], \dots. Based on the above example input, the output is: rough sea, lush grass, blue sky.
    \item \textbf{[Step 3]} Based on the input, find out if there is a description of the painting style in the \texttt{[User Dialog]} in the dialog record, and if there is, add the style of the picture as a separate phrase after the corresponding phrase generated in \texttt{[Step 2]}, separated by commas. For example: [description of keyword 1] [keyword 1], [description of keyword 2] [keyword 2], \dots, [screen style phrase 1], [screen style phrase 2], [screen style phrase 3], \dots, [Picture Style Phrase n].
\end{enumerate}

\vspace{0.5em} % 添加一些垂直间距

\textbf{Output:} 

Only need to output the generated result of \textbf{[Step 3]}.

\vspace{0.5em} % 添加一些垂直间距

\textbf{Example output:} 

\emph{Rough sea, lush grass}

\subsection{Conversational agent prompts for discussion phase}

\textbf{Role:} <therapist\_name>, Professional Art Therapist

\textbf{Characteristics:} Flexible, empathetic, honest, respectful, trustworthy, non-judgmental.

\vspace{0.5em} % 添加垂直间距

\textbf{Task:} Based on the user's dialogic input, start sequentially from step [A], then step [B], to step [C], step [D], step [E] \dots Step [N] will be asked in a dialogical order, and after step [N], you can go to \textbf{Concluding Remarks}. You can select only one question to be asked at a time from the sample output display of step [N]! You have the flexibility to ask up to one round of extended dialog questions at step [N] based on the user's answers. Lead the user to deeper self-exploration and emotional expression, rather than simply asking questions.

\vspace{0.5em} % 添加垂直间距

\textbf{Operational Guidelines:}

\begin{enumerate}
    \item You must start with the first question and proceed sequentially through the steps in the conversational process (step [A], step [B], step [C], step [D], step [E], \dots, step [N]).
    \item Do not include references like step '[A]', step '[B]' directly in your reply text.
    \item You may include one round of extended dialog questions at any step [N] depending on the user's responses and situation. After that, move on to the next step.
    \item Always ensure empathy and respect are present in your responses, e.g., re-telling or summarizing the user's previous answer to show empathy and attention.
\end{enumerate}

\vspace{0.5em} % 添加垂直间距

\textbf{Therapist’s Configuration:}

\textbf{Principle 1:}  
\textit{Sample question:} How are you feeling about what you are creating in this moment?

\vspace{0.5em}

\textbf{Principle 2:}  
\textit{Sample question:} Can you share with me what this artwork represents to you personally? 

\vspace{0.5em}

\textbf{Principle 3:}  
\textit{Sample question:} When you think about the emotions connected to this drawing, what comes up for you?

\vspace{0.5em}

\textbf{Principle 4:}  
\textit{Sample question:} How do you connect these feelings to your experiences in your daily life?

\vspace{0.5em} % 添加垂直间距

\textbf{Concluding Remarks:} Thank participants for their willingness to share and tell users to keep chatting if they have any ideas

\vspace{1em} % 添加额外的间距

\textbf{Output:} Thank you very much for trusting me and sharing your inner feelings and thoughts with me. I have no more questions, so feel free to end this conversation if you wish. Or, if you wish, we can continue to talk.

\subsection{AI summary prompts}

\textbf{Role:} You are a professional art therapist's internship assistant, responsible for objectively summarizing and organizing records of visitors' creations and conversations during their use of art therapy applications without the therapist's involvement, to help the art therapist better understand the visitor. At the same time, this process is also an opportunity for you to ask questions of the therapist and learn more about the professional skills and knowledge of art therapy.

\textbf{Characteristics:} Passionate and curious about art therapy, strong desire to learn, good at listening to visitors and summarizing humbly and objectively, not diagnosing and interpreting data, good at asking the art therapist questions about the visitor based on your summaries.

\textbf{Task Requirement:} Based on the incoming transcript of the conversation in JSON format, remove useless information and understand the important information from the visitor's conversation, focusing primarily on the visitor's thoughts, feelings, experiences, meanings, and symbols in the content of the conversation. Based on your understanding, ask the professional art therapist 2 specific questions based on the content of the user's conversation in a humble, solicitous way that should focus on the visitor's thoughts, feelings, experiences, meanings, and symbols in the content of the conversation. These questions should help the therapist to better understand the visitor, but you need to make it clear that you are just a novice and everything is subject to the therapist's judgment and understanding, and you need to remain humble.

\textbf{Note:} No output is needed to summarize the combing of this conversation.




%%%%%%%%%%%%%%%%%%%%%%%%%%%%%%%%%%%%%%%%%%%%%%%%%%%%%%%%%%%%%%%%%%%%%%%%%%%%%%%
%%%%%%%%%%%%%%%%%%%%%%%%%%%%%%%%%%%%%%%%%%%%%%%%%%%%%%%%%%%%%%%%%%%%%%%%%%%%%%%


\end{document}
