\section{Comparison of the number of parameters
%Compare 
with Full Parameter Fine-Tuning, PEFT and Activation Editing}
\label{appendix:compare}
To better demonstrate the advantages of our approach over full parameter fine-tuning, LoRA~\cite{hu2021lora}, and existing activation editing methods, we compare them across five key aspects: the percentage of modified parameters, the intervention modules, dynamic localization of intervention modules, data efficiency, and robustness across various tasks.
Using the fine-tuning of LLaMA-3~\cite{dubey2024llama} as an example, as shown in Table~\ref{tab:intro}, compared to full parameter fine-tuning and LoRA, activation editing requires editing only a small fraction of parameters, significantly fewer than the parameters from LoRA.
Activation editing is also a data-efficient approach, often effective even with just 100 examples.
However, we find that existing activation editing methods lack robustness across different datasets, primarily due to their reliance on manually configuring intervention module parameters, which are highly sensitive to task-specific variations.
In contrast, our proposed method, \jola{}, achieves comparable parameter efficiency to the best activation editing methods (e.g., LoFIT \cite{yin2024lofit}), while also delivering robust performance across tasks in low-resource settings.
This is due to dynamically adjusting the intervention module configurations rather than relying on manual parameter tuning.

%% Table: comparison
% \vspace{-1em}
\section{Backup: compare with previous works}

\paragraph{Comparison with Theorem 1 of \cite{srikant2024rates}.} While the framework of our proof of Theorem \ref{thm:Srikant-generalize} is mainly inspired by the proof of Theorem 1 of \cite{srikant2024rates}, there are some noteworthy differences. Most importantly, we observe that in the equation beginning from the bottom of Page 7 and continuing to the start of Page 8, the right-most side contains a term
\begin{align}\label{eq:Srikant-error}
-\frac{1}{n-k+1} \mathsf{Tr}\left(\bm{\Sigma}_{\infty}^{-\frac{1}{2}}(\bm{\Sigma}_k - \bm{\Sigma}_{\infty})\bm{\Sigma}_{\infty}^{-\frac{1}{2}}\mathbb{E}[\nabla^2 f(\tilde{\bm{Z}}_k)]\right);
\end{align}
the author argued that ``by taking an expectation to remove conditioning, and defining $\bm{A}_k$ to be $\mathbb{E}[\nabla^2 f(\tilde{\bm{Z}}_k)]$'', this term can be transformed to the term
\begin{align}\label{eq:Srikant-wrong}
-\frac{1}{n-k+1} \mathsf{Tr}\left(\bm{A}_k \left(\bm{\Sigma}_{\infty}^{-\frac{1}{2}} \mathbb{E}[\bm{\Sigma}_k]\bm{\Sigma}_{\infty}^{-\frac{1}{2}}-\bm{I}\right)\right)
\end{align}
in the expression of Theorem 1. However, we note that the function $f(\cdot)$, as defined on Page 6 as the solution to the Stein's equation with respect to $\tilde{h}(\cdot)$, is \emph{dependent on} $\mathcal{F}_{k-1}$; in fact, $f$ corresponds to the function $f_k$ in our proof. Consequently, the terms $\bm{A}_k = \mathbb{E}[\nabla^2 f(\tilde{\bm{Z}}_k)]$ (which is actually a conditional expectation with respect to $\mathcal{F}_{k-1}$), and $\bm{\Sigma}_k$ (which corresponds to $\bm{V}_k$ in our proof), are confounded by $\mathcal{F}_{k-1}$ and hence \emph{not independent}. Therefore, taking expectation, with respect to $\mathcal{F}_0$, on \eqref{eq:Srikant-error} should yield
\begin{align}\label{eq:Srikant-right}
-\frac{1}{n-k+1} \mathbb{E}\left\{\mathsf{Tr}\left(\bm{A}_k \left(\bm{\Sigma}_{\infty}^{-\frac{1}{2}} \bm{\Sigma}_k\bm{\Sigma}_{\infty}^{-\frac{1}{2}}-\bm{I}\right)\right)\right\}
\end{align}
Notice that the expectation is taken over the trace as a whole, instead of only $\bm{\Sigma}_k$. However, also due to the confounding bewteen $\bm{A}_k$ and $\bm{\Sigma}_k$, there is no guarantee that the sum of \eqref{eq:Srikant-right} is bounded as shown in the proof of Theorem 2 in \cite{srikant2024rates} on page 10. In other words, the framework of the proof needs a substantial correction to obtain a meaningful Berry-Esseen bound. 

Our solution in the proof of Theorem \ref{thm:Srikant-generalize} is to replace the matrix $\bm{Q}=\sqrt{n-k+1}\bm{\Sigma}_{\infty}$, as defined on Page 6 of \cite{srikant2024rates}, with the matrix $\bm{P}_k$, following the precedent of \cite{JMLR2019CLT}. This essentially eliminates the term \eqref{eq:Srikant-right}, but would require $\bm{P}_k$ to be measurable with respect to $\mathcal{F}_{k-1}$. For this purpose, we impose the assumption that $\bm{P}_1 = n\bm{\Sigma}_n$ almost surely, also following the precedent of \cite{JMLR2019CLT}. The relaxation of this assumption would be addressed in Theorem \ref{thm:Berry-Esseen-mtg}. 

Another important improvement we made in Theroem \ref{thm:Srikant-generalize} is to tighten the upper bound through a closer scrutiny of the smoothness of the solution to the Stein's equation, as is indicated in Proposition \ref{prop:Stein-smooth}. This paves the way for Corollary \ref{cor:Wu}, the proof of which we present in the next subsection. 

%% Table: Dataset
\section{Dataset}
\label{sec:dataset}

\subsection{Data Collection}

To analyze political discussions on Discord, we followed the methodology in \cite{singh2024Cross-Platform}, collecting messages from politically-oriented public servers in compliance with Discord's platform policies.

Using Discord's Discovery feature, we employed a web scraper to extract server invitation links, names, and descriptions, focusing on public servers accessible without participation. Invitation links were used to access data via the Discord API. To ensure relevance, we filtered servers using keywords related to the 2024 U.S. elections (e.g., Trump, Kamala, MAGA), as outlined in \cite{balasubramanian2024publicdatasettrackingsocial}. This resulted in 302 server links, further narrowed to 81 English-speaking, politics-focused servers based on their names and descriptions.

Public messages were retrieved from these servers using the Discord API, collecting metadata such as \textit{content}, \textit{user ID}, \textit{username}, \textit{timestamp}, \textit{bot flag}, \textit{mentions}, and \textit{interactions}. Through this process, we gathered \textbf{33,373,229 messages} from \textbf{82,109 users} across \textbf{81 servers}, including \textbf{1,912,750 messages} from \textbf{633 bots}. Data collection occurred between November 13th and 15th, covering messages sent from January 1st to November 12th, just after the 2024 U.S. election.

\subsection{Characterizing the Political Spectrum}
\label{sec:timeline}

A key aspect of our research is distinguishing between Republican- and Democratic-aligned Discord servers. To categorize their political alignment, we relied on server names and self-descriptions, which often include rules, community guidelines, and references to key ideologies or figures. Each server's name and description were manually reviewed based on predefined, objective criteria, focusing on explicit political themes or mentions of prominent figures. This process allowed us to classify servers into three categories, ensuring a systematic and unbiased alignment determination.

\begin{itemize}
    \item \textbf{Republican-aligned}: Servers referencing Republican and right-wing and ideologies, movements, or figures (e.g., MAGA, Conservative, Traditional, Trump).  
    \item \textbf{Democratic-aligned}: Servers mentioning Democratic and left-wing ideologies, movements, or figures (e.g., Progressive, Liberal, Socialist, Biden, Kamala).  
    \item \textbf{Unaligned}: Servers with no defined spectrum and ideologies or opened to general political debate from all orientations.
\end{itemize}

To ensure the reliability and consistency of our classification, three independent reviewers assessed the classification following the specified set of criteria. The inter-rater agreement of their classifications was evaluated using Fleiss' Kappa \cite{fleiss1971measuring}, with a resulting Kappa value of \( 0.8191 \), indicating an almost perfect agreement among the reviewers. Disagreements were resolved by adopting the majority classification, as there were no instances where a server received different classifications from all three reviewers. This process guaranteed the consistency and accuracy of the final categorization.

Through this process, we identified \textbf{7 Republican-aligned servers}, \textbf{9 Democratic-aligned servers}, and \textbf{65 unaligned servers}.

Table \ref{tab:statistics} shows the statistics of the collected data. Notably, while Democratic- and Republican-aligned servers had a comparable number of user messages, users in the latter servers were significantly more active, posting more than double the number of messages per user compared to their Democratic counterparts. 
This suggests that, in our sample, Democratic-aligned servers attract more users, but these users were less engaged in text-based discussions. Additionally, around 10\% of the messages across all server categories were posted by bots. 

\subsection{Temporal Data} 

Throughout this paper, we refer to the election candidates using the names adopted by their respective campaigns: \textit{Kamala}, \textit{Biden}, and \textit{Trump}. To examine how the content of text messages evolves based on the political alignment of servers, we divided the 2024 election year into three periods: \textbf{Biden vs Trump} (January 1 to July 21), \textbf{Kamala vs Trump} (July 21 to September 20), and the \textbf{Voting Period} (after September 20). These periods reflect key phases of the election: the early campaign dominated by Biden and Trump, the shift in dynamics with Kamala Harris replacing Joe Biden as the Democratic candidate, and the final voting stage focused on electoral outcomes and their implications. This segmentation enables an analysis of how discourse responds to pivotal electoral moments.

Figure \ref{fig:line-plot} illustrates the distribution of messages over time, highlighting trends in total messages volume and mentions of each candidate. Prior to Biden's withdrawal on July 21, mentions of Biden and Trump were relatively balanced. However, following Kamala's entry into the race, mentions of Trump surged significantly, a trend further amplified by an assassination attempt on him, solidifying his dominance in the discourse. The only instance where Trump’s mentions were exceeded occurred during the first debate, as concerns about Biden’s age and cognitive abilities temporarily shifted the focus. In the final stages of the election, mentions of all three candidates rose, with Trump’s mentions peaking as he emerged as the victor.

\section{Datasets}
\label{appendix:dataset}
We conduct experiments across three tasks: commonsense reasoning~\cite{hu-etal-2023-llm}, natural language understanding~\cite{wang2024mmlu}, and natural language generation~\cite{gehrmann-etal-2022-gemv2}.
Table~\ref{tab:dataset} provides a brief overview of the sub-datasets or sub-tasks within the three benchmarks evaluated.
The commonsense reasoning task is framed as a multiple-choice problem, where the correct answer is selected from 2 to 4 possible options.
The natural language understanding task also follows a multiple-choice format, but with ten options.
The natural language generation task, on the other hand, is an end-to-end text generation task, where unstructured data (such as commonsense concepts or data) is converted into coherent text.
In the training phase, we simulate a low-resource scenario by using 200 examples.
Section~\ref{subsec:data_model_size} further explores experiments with varying numbers of samples.
To ensure consistency across experiments, we used the same random seed (seed$=42$) for data sampling, ensuring identical training samples in all runs.

%% Table: prompt

\definecolor{titlecolor}{rgb}{0.9, 0.5, 0.1}
\definecolor{anscolor}{rgb}{0.2, 0.5, 0.8}
\definecolor{labelcolor}{HTML}{48a07e}
\begin{table*}[h]
	\centering
	
 % \vspace{-0.2cm}
	
	\begin{center}
		\begin{tikzpicture}[
				chatbox_inner/.style={rectangle, rounded corners, opacity=0, text opacity=1, font=\sffamily\scriptsize, text width=5in, text height=9pt, inner xsep=6pt, inner ysep=6pt},
				chatbox_prompt_inner/.style={chatbox_inner, align=flush left, xshift=0pt, text height=11pt},
				chatbox_user_inner/.style={chatbox_inner, align=flush left, xshift=0pt},
				chatbox_gpt_inner/.style={chatbox_inner, align=flush left, xshift=0pt},
				chatbox/.style={chatbox_inner, draw=black!25, fill=gray!7, opacity=1, text opacity=0},
				chatbox_prompt/.style={chatbox, align=flush left, fill=gray!1.5, draw=black!30, text height=10pt},
				chatbox_user/.style={chatbox, align=flush left},
				chatbox_gpt/.style={chatbox, align=flush left},
				chatbox2/.style={chatbox_gpt, fill=green!25},
				chatbox3/.style={chatbox_gpt, fill=red!20, draw=black!20},
				chatbox4/.style={chatbox_gpt, fill=yellow!30},
				labelbox/.style={rectangle, rounded corners, draw=black!50, font=\sffamily\scriptsize\bfseries, fill=gray!5, inner sep=3pt},
			]
											
			\node[chatbox_user] (q1) {
				\textbf{System prompt}
				\newline
				\newline
				You are a helpful and precise assistant for segmenting and labeling sentences. We would like to request your help on curating a dataset for entity-level hallucination detection.
				\newline \newline
                We will give you a machine generated biography and a list of checked facts about the biography. Each fact consists of a sentence and a label (True/False). Please do the following process. First, breaking down the biography into words. Second, by referring to the provided list of facts, merging some broken down words in the previous step to form meaningful entities. For example, ``strategic thinking'' should be one entity instead of two. Third, according to the labels in the list of facts, labeling each entity as True or False. Specifically, for facts that share a similar sentence structure (\eg, \textit{``He was born on Mach 9, 1941.''} (\texttt{True}) and \textit{``He was born in Ramos Mejia.''} (\texttt{False})), please first assign labels to entities that differ across atomic facts. For example, first labeling ``Mach 9, 1941'' (\texttt{True}) and ``Ramos Mejia'' (\texttt{False}) in the above case. For those entities that are the same across atomic facts (\eg, ``was born'') or are neutral (\eg, ``he,'' ``in,'' and ``on''), please label them as \texttt{True}. For the cases that there is no atomic fact that shares the same sentence structure, please identify the most informative entities in the sentence and label them with the same label as the atomic fact while treating the rest of the entities as \texttt{True}. In the end, output the entities and labels in the following format:
                \begin{itemize}[nosep]
                    \item Entity 1 (Label 1)
                    \item Entity 2 (Label 2)
                    \item ...
                    \item Entity N (Label N)
                \end{itemize}
                % \newline \newline
                Here are two examples:
                \newline\newline
                \textbf{[Example 1]}
                \newline
                [The start of the biography]
                \newline
                \textcolor{titlecolor}{Marianne McAndrew is an American actress and singer, born on November 21, 1942, in Cleveland, Ohio. She began her acting career in the late 1960s, appearing in various television shows and films.}
                \newline
                [The end of the biography]
                \newline \newline
                [The start of the list of checked facts]
                \newline
                \textcolor{anscolor}{[Marianne McAndrew is an American. (False); Marianne McAndrew is an actress. (True); Marianne McAndrew is a singer. (False); Marianne McAndrew was born on November 21, 1942. (False); Marianne McAndrew was born in Cleveland, Ohio. (False); She began her acting career in the late 1960s. (True); She has appeared in various television shows. (True); She has appeared in various films. (True)]}
                \newline
                [The end of the list of checked facts]
                \newline \newline
                [The start of the ideal output]
                \newline
                \textcolor{labelcolor}{[Marianne McAndrew (True); is (True); an (True); American (False); actress (True); and (True); singer (False); , (True); born (True); on (True); November 21, 1942 (False); , (True); in (True); Cleveland, Ohio (False); . (True); She (True); began (True); her (True); acting career (True); in (True); the late 1960s (True); , (True); appearing (True); in (True); various (True); television shows (True); and (True); films (True); . (True)]}
                \newline
                [The end of the ideal output]
				\newline \newline
                \textbf{[Example 2]}
                \newline
                [The start of the biography]
                \newline
                \textcolor{titlecolor}{Doug Sheehan is an American actor who was born on April 27, 1949, in Santa Monica, California. He is best known for his roles in soap operas, including his portrayal of Joe Kelly on ``General Hospital'' and Ben Gibson on ``Knots Landing.''}
                \newline
                [The end of the biography]
                \newline \newline
                [The start of the list of checked facts]
                \newline
                \textcolor{anscolor}{[Doug Sheehan is an American. (True); Doug Sheehan is an actor. (True); Doug Sheehan was born on April 27, 1949. (True); Doug Sheehan was born in Santa Monica, California. (False); He is best known for his roles in soap operas. (True); He portrayed Joe Kelly. (True); Joe Kelly was in General Hospital. (True); General Hospital is a soap opera. (True); He portrayed Ben Gibson. (True); Ben Gibson was in Knots Landing. (True); Knots Landing is a soap opera. (True)]}
                \newline
                [The end of the list of checked facts]
                \newline \newline
                [The start of the ideal output]
                \newline
                \textcolor{labelcolor}{[Doug Sheehan (True); is (True); an (True); American (True); actor (True); who (True); was born (True); on (True); April 27, 1949 (True); in (True); Santa Monica, California (False); . (True); He (True); is (True); best known (True); for (True); his roles in soap operas (True); , (True); including (True); in (True); his portrayal (True); of (True); Joe Kelly (True); on (True); ``General Hospital'' (True); and (True); Ben Gibson (True); on (True); ``Knots Landing.'' (True)]}
                \newline
                [The end of the ideal output]
				\newline \newline
				\textbf{User prompt}
				\newline
				\newline
				[The start of the biography]
				\newline
				\textcolor{magenta}{\texttt{\{BIOGRAPHY\}}}
				\newline
				[The ebd of the biography]
				\newline \newline
				[The start of the list of checked facts]
				\newline
				\textcolor{magenta}{\texttt{\{LIST OF CHECKED FACTS\}}}
				\newline
				[The end of the list of checked facts]
			};
			\node[chatbox_user_inner] (q1_text) at (q1) {
				\textbf{System prompt}
				\newline
				\newline
				You are a helpful and precise assistant for segmenting and labeling sentences. We would like to request your help on curating a dataset for entity-level hallucination detection.
				\newline \newline
                We will give you a machine generated biography and a list of checked facts about the biography. Each fact consists of a sentence and a label (True/False). Please do the following process. First, breaking down the biography into words. Second, by referring to the provided list of facts, merging some broken down words in the previous step to form meaningful entities. For example, ``strategic thinking'' should be one entity instead of two. Third, according to the labels in the list of facts, labeling each entity as True or False. Specifically, for facts that share a similar sentence structure (\eg, \textit{``He was born on Mach 9, 1941.''} (\texttt{True}) and \textit{``He was born in Ramos Mejia.''} (\texttt{False})), please first assign labels to entities that differ across atomic facts. For example, first labeling ``Mach 9, 1941'' (\texttt{True}) and ``Ramos Mejia'' (\texttt{False}) in the above case. For those entities that are the same across atomic facts (\eg, ``was born'') or are neutral (\eg, ``he,'' ``in,'' and ``on''), please label them as \texttt{True}. For the cases that there is no atomic fact that shares the same sentence structure, please identify the most informative entities in the sentence and label them with the same label as the atomic fact while treating the rest of the entities as \texttt{True}. In the end, output the entities and labels in the following format:
                \begin{itemize}[nosep]
                    \item Entity 1 (Label 1)
                    \item Entity 2 (Label 2)
                    \item ...
                    \item Entity N (Label N)
                \end{itemize}
                % \newline \newline
                Here are two examples:
                \newline\newline
                \textbf{[Example 1]}
                \newline
                [The start of the biography]
                \newline
                \textcolor{titlecolor}{Marianne McAndrew is an American actress and singer, born on November 21, 1942, in Cleveland, Ohio. She began her acting career in the late 1960s, appearing in various television shows and films.}
                \newline
                [The end of the biography]
                \newline \newline
                [The start of the list of checked facts]
                \newline
                \textcolor{anscolor}{[Marianne McAndrew is an American. (False); Marianne McAndrew is an actress. (True); Marianne McAndrew is a singer. (False); Marianne McAndrew was born on November 21, 1942. (False); Marianne McAndrew was born in Cleveland, Ohio. (False); She began her acting career in the late 1960s. (True); She has appeared in various television shows. (True); She has appeared in various films. (True)]}
                \newline
                [The end of the list of checked facts]
                \newline \newline
                [The start of the ideal output]
                \newline
                \textcolor{labelcolor}{[Marianne McAndrew (True); is (True); an (True); American (False); actress (True); and (True); singer (False); , (True); born (True); on (True); November 21, 1942 (False); , (True); in (True); Cleveland, Ohio (False); . (True); She (True); began (True); her (True); acting career (True); in (True); the late 1960s (True); , (True); appearing (True); in (True); various (True); television shows (True); and (True); films (True); . (True)]}
                \newline
                [The end of the ideal output]
				\newline \newline
                \textbf{[Example 2]}
                \newline
                [The start of the biography]
                \newline
                \textcolor{titlecolor}{Doug Sheehan is an American actor who was born on April 27, 1949, in Santa Monica, California. He is best known for his roles in soap operas, including his portrayal of Joe Kelly on ``General Hospital'' and Ben Gibson on ``Knots Landing.''}
                \newline
                [The end of the biography]
                \newline \newline
                [The start of the list of checked facts]
                \newline
                \textcolor{anscolor}{[Doug Sheehan is an American. (True); Doug Sheehan is an actor. (True); Doug Sheehan was born on April 27, 1949. (True); Doug Sheehan was born in Santa Monica, California. (False); He is best known for his roles in soap operas. (True); He portrayed Joe Kelly. (True); Joe Kelly was in General Hospital. (True); General Hospital is a soap opera. (True); He portrayed Ben Gibson. (True); Ben Gibson was in Knots Landing. (True); Knots Landing is a soap opera. (True)]}
                \newline
                [The end of the list of checked facts]
                \newline \newline
                [The start of the ideal output]
                \newline
                \textcolor{labelcolor}{[Doug Sheehan (True); is (True); an (True); American (True); actor (True); who (True); was born (True); on (True); April 27, 1949 (True); in (True); Santa Monica, California (False); . (True); He (True); is (True); best known (True); for (True); his roles in soap operas (True); , (True); including (True); in (True); his portrayal (True); of (True); Joe Kelly (True); on (True); ``General Hospital'' (True); and (True); Ben Gibson (True); on (True); ``Knots Landing.'' (True)]}
                \newline
                [The end of the ideal output]
				\newline \newline
				\textbf{User prompt}
				\newline
				\newline
				[The start of the biography]
				\newline
				\textcolor{magenta}{\texttt{\{BIOGRAPHY\}}}
				\newline
				[The ebd of the biography]
				\newline \newline
				[The start of the list of checked facts]
				\newline
				\textcolor{magenta}{\texttt{\{LIST OF CHECKED FACTS\}}}
				\newline
				[The end of the list of checked facts]
			};
		\end{tikzpicture}
        \caption{GPT-4o prompt for labeling hallucinated entities.}\label{tb:gpt-4-prompt}
	\end{center}
\vspace{-0cm}
\end{table*}
%% parameter search
\begin{table}[!thp]
\caption{
\label{tab:hyperparam}
Hyperparameter configurations for the baseline methods evaluated in our experiments. These settings are used across multiple tasks to assess model performance in low-resource settings, as discussed in Sections~\ref{sec:intro} and Section~\ref{sec:exp}.
}
\resizebox{\textwidth}{!}{
\begin{tabular}{l|l|p{8cm}}
\toprule
\textbf{Baseline} & \textbf{Hyperparameter} & \textbf{Values} \\
\midrule
\multirow{2}{*}{BitFit~\cite{ben-zaken-etal-2022-bitfit}} & Bias Moudule & bias of Q,K and V from attention/bias of LayerNorm from attention outputs/bias of LayerNorm from hidden outputs \\ \cmidrule{2-3}
 & Learning Rate & 1e-4/ 5e-4 \\
\midrule
\multirow{2}{*}{RED~\cite{wu-etal-2024-advancing}} & Rank & 8 / 16 \\
\cmidrule{2-3} 
 & Learning Rate & 5e-5/ 2e-4 / 6e-2 \\
\midrule
REPE~\cite{zou2023representation} & method & Representation Reading / Representation Control \\
\midrule
\multirow{2}{*}{ReFT~\cite{wu2024reft}} & Prefix + suffix posotion & p7 + s7 / p11 + s11 \\
\cmidrule{2-3} 
 & Layers & all / 4,6,10,12,14,18,20,22/3,9,18,24 \\
\midrule
\multirow{2}{*}{LoFIT~\cite{yin2024lofit}} & number of attention heads & 32/64/128 \\
\cmidrule{2-3} 
 & Learning Rate & 5e-4 / 5e-3 \\
\bottomrule
\end{tabular}
}
\end{table}


\section{Prompt Setting}
\label{appendix:prompt}
Recent studies~\cite{he2024does,lai-etal-2024-llms} have highlighted the substantial impact of prompt design on model performance.
In our experiments, we adopt the same prompt configurations as~\citet{hu-etal-2023-llm} for the commonsense reasoning benchmark, and used the prompts from the original paper for the MMLU-Pro benchmark~\cite{wang2024mmlu}.
For the GEM benchmark~\cite{gehrmann-etal-2022-gemv2}, where the original paper did not provide the prompt settings, we utilized commonly used prompts curated from PromptSource\footnote{\url{https://github.com/bigscience-workshop/promptsource}}.
To ensure reproducibility of our results, we present the prompts employed in our experiments in Table~\ref{tab:prompt}.

\section{Experiment Configurations}
\label{appendix:exp_config}
\subsection{Training Setup}
\label{appendix:traiing_config}
We conduct all experiments using the HuggingFace Transformers\footnote{\url{https://github.com/huggingface/transformers}} library and fine-tuned the models with the TRL toolkit\footnote{\url{https://github.com/huggingface/trl}}.
The AdamW optimizer~\cite{loshchilov2017decoupled} was used for fine-tuning, with $\epsilon = 1e-6$ and one epoch of warm-up.
In addition, we employ an exponentially decaying~\cite{li2019exponential} learning rate schedule, defined by the following formula:
\begin{equation}
    \text{lr}(t) = \text{lr}_0 \cdot \lambda^{t} \cdot e^{-\text{decay} \cdot t}
\end{equation}
where $\text{lr}_0$ is the initial learning rate $lr_{0}$set to $5 \times 10^{-4}$ , $\lambda$ is 0.1, and the decay rate is 0.01.
For the gating units, we used a temperature of 0.33 in the Gumbel Softmax~\cite{jang2017categorical}.
Fine-tuning was performed in full precision for the 7B, 8B, 1B, and 3B models, while for the 70B model, we applied 4-bit quantization to enable single-precision fine-tuning.

\subsection{Computational Resources}
\label{appendix:comp_resource}
All experiments for the 1B, 3B, 8B, and 13B models were conducted on a single NVIDIA A100 80GB GPU server.
The 70B model, described in Section~\ref{subsec:data_model_size}, was evaluated on an NVIDIA H100 94GB GPU server.
As an example, with the 8B LLaMA-3 model, \jola{} converged within 2 GPU hours on most tasks in the low-resource setting, using only 200 training samples.

\subsection{Hyperparameter Search for Baselines}
\label{appendix:hyper_search}
As discussed in Section~\ref{sec:intro} and Section~\ref{sec:exp}, the performance of baseline methods in low-resource settings is highly sensitive to hyperparameters across different tasks.
However, it is impractical to conduct hyperparameter searches for each task individually, given that we evaluate 26 tasks in total, and performing a separate search for each would be time-consuming.
To mitigate this, we use five different sets of hyperparameters during the baseline experiments.
The average results of these five configurations are reported in Table~\ref{tab:main_res} and Figure~\ref{fig:main_res}.
The specific hyperparameters used for each method are detailed in Table~\ref{tab:hyperparam}.

\section{Full Results Across all Tasks}
\label{appendix:full_res}
Due to page limitations, we present the average performance across the 26 tasks in Table~\ref{tab:main_res} and Figure~\ref{fig:main_res}.
In this section, we provide detailed performance metrics for each individual task.
Specifically, Table~\ref{tab:llama_reason_full} reports the accuracy of LLaMA-3 on the commonsense reasoning task, while Table~\ref{tab:qwen_reason_full} presents the accuracy of Qwen-2.5 on the same task.
Table~\ref{tab:llama_mmlu_full} shows the accuracy of LLaMA-3 on the natural language understanding task, and Table~\ref{tab:qwen_mmlu_full} shows the corresponding accuracy for Qwen-2.5.
Finally, Table~\ref{tab:llama_gem_full} presents the BLEU, ROUGE-L, and BERTScore for LLaMA-3 on the natural language generation task, with Table~\ref{tab:qwen_gem_full} displaying the corresponding metrics for Qwen-2.5.

%% full result
\begin{table*}[h]
  \centering
  % \vspace{-1mm}
  \scalebox{0.9}{%
    \begin{tabular}{l||c|c|c||c|c|c}
        % \toprule
        \hline
        \multirow{2}{*}{method} & \multicolumn{3}{c||}{$b=3$} &\multicolumn{3}{c}{$b=5$} \bigstrut \\
        % \cmidrule{2-7}
        \cline{2-7}        
         & \WL & \CL & \EL  & \WL & \CL & \EL  \bigstrut\\
        % \midrule
        \hline
        \hline
        \textsc{Random} & 0.0016 (0.0214) & -0.0133 (0.1105) & 0.0010 (0.0182)  &
        -0.0045 (0.0216) & 0.0010 (0.1188) & 0.0081 (0.0158) \bigstrut[t]\\
        OdC & 0.0022 (0.0225) & 0.0063 (0.1077) & 0.0123 (0.0207)  &
        0.0117 (0.0238) & 0.0352 (0.1534) & 0.0119 (0.0407)  \\
        BC & -0.0006 (0.0226) & 0.0006 (0.0860) & -0.0011 (0.0208)  &
        0.0019 (0.0220) & -0.0057 (0.0992) & -0.0028 (0.0206) \\
        PR &-0.0017 (0.0175) & -0.0036 (0.1204) & -0.0002 (0.0187)  &
        -0.0017 (0.0201) & -0.0010 (0.1077) & -0.0124 (0.0182) \bigstrut[b] \\
        % \midrule
        \hline
        \BPM & \textbf{O.O.T} & 0.0159 (0.1221) & 0.0387 (0.0471)  &
        \textbf{O.O.T} & -0.0021 (0.1074) & 0.0477 (0.0527) \bigstrut[t] \\
        \KED &-0.0006 (0.0174) & 0.0109 (0.1162) & -0.0015 (0.0180)  &
        0.0029 (0.0178) & 0.0097 (0.1317) & -0.0029 (0.0556) \\
        \MDS &0.0098 (0.0254) & -0.0004 (0.1085) & 0.0298 (0.0366)  &
        0.0173 (0.0229) & 0.0202 (0.1341) & 0.0368 (0.0421) \bigstrut[b] \\
        % \midrule
        \hline
        %\MBPM-10 & . & . & . & .\\
        \MBPM-100 & 0.0105 (0.0622) & 0.0108 (0.1358) & 0.0093 (0.0610)  &
        0.0098 (0.0630) & -0.0076 (0.1215) & 0.0116 (0.0594) \bigstrut[t] \\
        \MBPM-1000 & 0.1342 (0.2195) & -0.0013 (0.1308) & 0.1396 (0.2096)  &
        0.1589 (0.2245) & -0.0052 (0.1187) & 0.1767 (0.2150) \\
        \MBPM-10000 & 0.2811 (0.2140) & 0.0136 (0.1350) & 0.3836 (0.2014)  &
        0.3160 (0.2170) & 0.0183 (0.1335) & 0.4273 (0.2127) \\
        \MBPM-100000 & \textbf{O.O.T} & 0.2046 (0.2338) & 0.4468 (0.1766)  &
        \textbf{O.O.T} & \textbf{O.O.T} & 0.5284 (0.1790) \bigstrut[b] \\
        % \midrule
        \hline
        \Greedy-10 & 0.0059 (0.0481) & -0.0182 (0.0976) & 0.0244 (0.0995)  &
        0.0062 (0.0459) & -0.0052 (0.1040) & 0.0236 (0.1011) \bigstrut[t] \\
        \Greedy-100 & \textbf{O.O.T} & -0.0052 (0.1144) & 0.1560 (0.2363)  &
        \textbf{O.O.T} & 0.0052 (0.1374) & 0.1635 (0.2356) \bigstrut[b] \\
        % \midrule
        \hline
        \RIS-0.6 & 0.2280 (0.1720) & 0.3813 (0.1941) & 0.2937 (0.1968) & \textbf{O.O.T} & 0.5045 (0.2064) &   0.3641 (0.2102) \bigstrut[t] \\
        \RIS-0.4 & \textbf{O.O.T} &  0.4428 (0.1960) & 0.3202 (0.2078) & \textbf{O.O.T} & 0.5227 (0.2136) & 0.4156 (0.2163)\\
        \RIS-0.2 & \textbf{O.O.T} & 0.4456 (0.1893) & 0.3688 (0.2100) & \textbf{O.O.T} &  0.5591 (0.2217) & 0.4513 (0.2212) \bigstrut[b]  \\
        \hline
        %\naive (2GCN) & 0.3316 (0.1942) & 0.4789 (0.1912) & 0.4435 (0.1950)  &
        % 0.3977 (0.1819) & 0.6096 (0.1853) & 0.5224 (0.1959) \bigstrut[t] \\
        \naive & 0.3609 (0.1796) & 0.5240 (0.1876) & 0.4675 (0.1855)  &
        0.4311 (0.1726) & 0.6547 (0.1877) & 0.5613 (0.1828) \bigstrut[t] \\
        %\hline
        %\adv (2GCN) & 0.3180 (0.1817) & 0.4905 (0.2247) & 0.4246 (0.2119)  &
        %0.3897 (0.1779) & 0.6261 (0.2151) & 0.5126 (0.2022) \bigstrut[t] \\
        \adv  & 0.2713 (0.1978) & 0.5172 (0.1930) & 0.4415 (0.1871)  &
        0.3914 (0.1796) & 0.6614 (0.1957) & 0.5332 (0.1845)  \\
        %\hline        
        %\advp (2GCN) & 0.3061 (0.1906) & 0.4985 (0.2176) & 0.4195 (0.2065) &
        %0.3886 (0.1855) & 0.6348 (0.2169) & 0.5048 (0.2049) \bigstrut[t] \\
        \advp  & 0.3236 (0.1856) & 0.5125 (0.1956) & 0.3512 (0.2002)  &
        0.3876 (0.1759) & 0.6583 (0.1948) & 0.4718 (0.2066)  \\
        % \midrule[\heavyrulewidth]
        \hline        
        \multicolumn{7}{c}{}\\
        \hline        
        \multirow{2}{*}{method} & \multicolumn{3}{c||}{$b=7$} &\multicolumn{3}{c}{$b=10$} \bigstrut\\
        % \cmidrule{2-7}
        \cline{2-7}
         & \WL & \CL & \EL  & \WL & \CL & \EL  \bigstrut \\
         % \midrule
         \hline
         \hline
        \textsc{Random}  &  -0.0011 (0.0187) & -0.0187 (0.1244) & 0.0014 (0.0210)  & 
        0.0002 (0.0201) & -0.0072 (0.1150) & 0.0005 (0.0182) \bigstrut[t] \\
        OdC  &  0.0275 (0.0330) & 0.0106 (0.1120) & 0.0315 (0.0439)  & 
        0.0327 (0.0338) & 0.0263 (0.1294) & 0.0448 (0.0494) \\
        BC  &  0.0027 (0.0194) & 0.0196 (0.1118) & 0.0090 (0.0557)  & 
        0.0020 (0.0234) & 0.0054 (0.1013) & 0.0158 (0.0534) \\
        PR  &  -0.0035 (0.0225) & 0.0187 (0.1071) & -0.0020 (0.0203) & 
        0.0006 (0.0239) & 0.0257 (0.1377) & 0.0002 (0.0218) \bigstrut[b] \\
        % \midrule
        \hline
        \BPM  &  \textbf{O.O.T} & 0.0016 (0.0980) & 0.0882 (0.1145)  & 
        \textbf{O.O.T} & -0.0085 (0.1030) & 0.1047 (0.1432) \bigstrut[t] \\
        \KED  &  -0.0010 (0.0158) & 0.0017 (0.1316) & 0.0079 (0.0569)  & 
         -0.0012 (0.0197) & -0.0100 (0.1319) & 0.0113 (0.0579)\\
        \MDS  &  0.0160 (0.0244) & 0.0015 (0.1112) & 0.0372 (0.0442)  & 
        0.0163 (0.0258) & 0.0230 (0.1354) & 0.0396 (0.0424) \bigstrut[b] \\
        % \midrule
        \hline
        \MBPM-100  &  0.0461 (0.1659) & 0.0091 (0.1202) & 0.0258 (0.1008)  & 
        0.0507 (0.1711) & -0.0080 (0.0778) & 0.0252 (0.1015)\bigstrut[t] \\
        \MBPM-1000  &  0.1587 (0.2318) & 0.0009 (0.1315) & 0.1778 (0.2219)  & 
        0.1641 (2315) & 0.0128 (0.1172) & 0.2014 (0.2220) \\
        \MBPM-10000  & 0.3365 (0.2215) & 0.0153 (0.1437) & 0.4673 (0.2089)  & 
        0.3545 (2243) & 0.0289 (0.1307) & 0.4991 (0.2072) \\
        \MBPM-100000  &  \textbf{O.O.T} & \textbf{O.O.T} & \textbf{O.O.T}  & 
        \textbf{O.O.T} & \textbf{O.O.T} & \textbf{O.O.T} \bigstrut[b] \\
        % \midrule
        \hline
        \Greedy-10  &  0.0076 (0.0493) & -0.0152 (0.1241) & 0.0224 (0.1024)  & 
        0.0136 (0.0654) & -0.0300 (0.1453) & 0.0321 (0.1111) \bigstrut[t] \\
        \Greedy-100  &  \textbf{O.O.T} & 0.0030 (0.1299) & 0.1809 (0.2340)  & 
        \textbf{O.O.T} & -0.0074 (0.1414) & 0.1977 (0.2358) \bigstrut[b]\\
        \hline
        \RIS-0.6 & \textbf{O.O.T} & 0.5732 (0.2292) & 0.4172 (0.2134) & \textbf{O.O.T} & 0.7590 (0.2020) & 0.6390 (0.1443) \bigstrut[t]\\
        \RIS-0.4 & \textbf{O.O.T} & 0.6093 (0.2117) & 0.4688 (0.2131) & \textbf{O.O.T} & 0.8113 (0.1836) & 0.6517 (0.1534) \\
        \RIS-0.2 & \textbf{O.O.T} & \textbf{O.O.T} & 0.5116 (0.2195) & \textbf{O.O.T} & \textbf{O.O.T} & \textbf{O.O.T} \bigstrut[b] 
        \\
        % \midrule
        \hline
        %\naive (2GCN)  &  0.4479 (0.1778) & 0.7029 (0.2012) & 0.5885 (0.1877)  & 
        %0.5000 (0.1698) & 0.7882 (0.2085) & 0.6519 (0.1778) \bigstrut[t]\\
        \naive  &  \textbf{O.O.T} & 0.7455 (0.1871) & 0.6171 (0.1757) & 
        \textbf{O.O.T} & 0.8124 (0.1870) & 0.6780 (0.1625) \bigstrut[t] \\
      % \adv (2GCN) &  0.4353 (0.1739) & 0.7291 (0.2218) & 0.5731 (0.1970) & 
        % 0.4833 (0.1729) & 0.8151 (0.2153) & 0.6339 (0.1892) \bigstrut[t] \\
        \adv &  0.4301 (0.1754) & 0.7423 (0.2034) & 0.5881 (0.1794)  &
        0.4758 (0.1663) & 0.8352 (0.1957) & 0.6439 (0.1725)  \\
      %  \advp (2GCN)&  0.4376 (0.1719) & 0.7268 (0.2191) & 0.5663 (0.1881)  & 
       % 0.4819 (0.1738) & 0.8145 (0.2157) & 0.6271 (0.1826) \bigstrut[t]\\
        \advp  &  0.4289 (0.1714) & 0.7417 (0.1914) & 0.5286 (0.2078)  & 
        0.4772 (0.1655) & 0.8346 (0.1919) & 0.6023 (0.1980) \\ 
    % \bottomrule
    \hline
    \end{tabular}
    }
    \caption{The effectiveness (the reduced ratio of influence) of each method with the standard deviations, with budget $b \in \Set{3, 5, 7, 10}$.
  O.O.T denotes out-of-time, i.e., the method does not terminate within one hour on a single seed set in the corresponding setting.}
    \label{tab:app_perf}
\end{table*}

\section{Case Study}
\label{appendix:case_study}
To provide an intuitive evaluation of the advantages of our method, we select one representative case from each of the tasks: commonsense reasoning, natural language understanding, and natural language generation.
The results generated by the baseline and our approach are presented below.

%% case study
\section{Case Study}\label{sec:appendix_casestudy}

We present cases across multiple coding datasets, comparing compressed and original code examples. For instance, as demonstrated in Figures 8, 9, and 10, \ourtool prioritizes discarding \textbf{Invocation} tokens first, followed by \textbf{Symbol} tokens.
\begin{figure}[!h]
\begin{tcolorbox}
\begin{lstlisting}[language=Java,frame=single,framerule=0pt]
### FOCAL_METHOD 
getProduction(java.lang.String) { 
 return productionsByName.get(name); }  
### UNIT_TEST  
testJustifications() { 
 runTest("testJustifications", 2); org.jsoar.kernel.Production j = agent.getProductions() .getProduction("justification-1"); "<AssertPlaceHolder>"; 
}    
\end{lstlisting}
\end{tcolorbox}
\caption{Original Code Examples of Assertion Generation (63 tokens)}
\label{fig:code-example}
\end{figure}

\begin{figure}[!h]
\begin{tcolorbox}
\begin{lstlisting}[language=Java,frame=single,framerule=0pt]
### FOCAL_METHOD 
getProduction(java.lang.String) { 
 return productionsByName; }  
### UNIT_TEST  
testJustifications() { 
 ; 
 org.jsoar.kernel.Production j = agent.getProductions() .getProduction("justification-1"); "<AssertPlaceHolder>"; 
}    
\end{lstlisting}
\end{tcolorbox}
\caption{Compressed Code Examples of Assertion Generation (55 tokens, $\tau_{code}$: 0.1)}
\label{fig:code-example}
\end{figure}

\begin{figure}[!h]
\begin{tcolorbox}
\begin{lstlisting}[language=Java,frame=single,framerule=0pt]
### FOCAL_METHOD 
getProduction(java.lang.String)  
 return productionsByName;     
### UNIT_TEST  
testJustifications()  
 ; 
 org.jsoar.kernel.Production j = agent;
  "<AssertPlaceHolder>"; 
\end{lstlisting}
\end{tcolorbox}
\caption{Compressed Code Examples of Assertion Generation (39 tokens, $\tau_{code}$: 0.4)}
\label{fig:code-example}
\end{figure}


\begin{figure}[!h]
\begin{tcolorbox}
\begin{lstlisting}[language=Java,frame=single,framerule=0pt]
### BUGGY_CODE 
public static TYPE_1 init(java.lang.String name, java.util.Date date) {
   TYPE_1 VAR_1 = new TYPE_1();
   VAR_1.METHOD_1(name);
   java.util.Calendar VAR_2 = java.util.Calendar.getInstance();
   VAR_2.METHOD_2(date);
   VAR_1.METHOD_3(VAR_2);
   return VAR_1;
}
### FIXED_CODE   
public static TYPE_1 init(java.lang.String name, java.util.Date date) {
   TYPE_1 VAR_1 = new TYPE_1();
   VAR_1.METHOD_1(name);
   java.util.Calendar VAR_2 = null;
   if (date != null) {
       VAR_2 = java.util.Calendar.getInstance();
       VAR_2.METHOD_2(date);
   } 
   VAR_1.METHOD_3(VAR_2);
   return VAR_1;
}
\end{lstlisting}
\end{tcolorbox}
\caption{Original Code Examples of Bugs2Fix (195 tokens)}
\label{fig:code-example}
\end{figure}

\begin{figure}[!h]
\begin{tcolorbox}
\begin{lstlisting}[language=Java,frame=single,framerule=0pt]
### BUGGY_CODE 
public static TYPE_1 init(java.lang.String name, java.util.Date date) {
    = new TYPE_1();
   ;
   java.util.Calendar = java.util.Calendar;
   .METHOD_2(date);
   .METHOD_3(VAR_2);
   return ;
}
### FIXED_CODE   
public static TYPE_1 init(java.lang.String name, java.util.Date date) {
    = new TYPE_1();
   ;
   java.util.Calendar = null;
   if (date != null) {
        = java.util.Calendar;
       .METHOD_2(date);
   } 
   .METHOD_3(VAR_2);
   return ;
}
\end{lstlisting}
\end{tcolorbox}
\caption{Compressed Code Examples of Bugs2Fix (136 tokens, $\tau_{code}$: 0.3)}
\label{fig:code-example}
\end{figure}

\begin{figure}[!h]
\begin{tcolorbox}
\begin{lstlisting}[language=Java,frame=single,framerule=0pt]
### METHOD_HEADER 
protected final void fastPathEmit ( U value , boolean delayError , Disposable dispose )
### WHOLE_METHOD  
protected final void fastPathEmit(U value, boolean delayError, Disposable dispose) {
   final Observer<? super V> s = actual;
   final SimplePlainQueue<U> q = queue;
   if (wip.get() == 0 && wip.compareAndSet(0, 1)) {
       accept(s, value);
       if (leave(-1) == 0) {
           return;
       }
   } else {
       q.offer(value);
       if (!enter()) {
           return;
       }
   }
   QueueDrainHelper.drainLoop(q, s, delayError, dispose, this);
}
\end{lstlisting}
\end{tcolorbox}
\caption{Original Code Examples  of \taskthree (157 tokens, $\tau_{code}$: 0.3)}
\label{fig:code-example}
\end{figure}


\begin{figure}[!h]
\begin{tcolorbox}
Original Code Examples (121 tokens, $\tau_{code}$: 0.3)
\begin{lstlisting}[language=Java,frame=single,framerule=0pt]
### METHOD_HEADER 
protected final void fastPathEmit ( U value , boolean delayError , Disposable dispose )
### WHOLE_METHOD  
   final Observer<? super V> = 
   final SimplePlainQueue<U> = 
   if (wip.get() == 0 && wip.compareAndSet(0, 1)) 
       ;
       if (leave(-1) == 0) 
           return;    
    else 
       .offer(value);
       if (!enter()) 
           return;
   .drainLoop(q, s, delayError, dispose, this);
\end{lstlisting}
\end{tcolorbox}
\caption{Compressed Code Examples  of \taskthree (121 tokens, $\tau_{code}$: 0.3)}
\end{figure}








