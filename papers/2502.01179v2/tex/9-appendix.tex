\section{Comparison of the number of parameters
%Compare 
with Full Parameter Fine-Tuning, PEFT and Activation Editing}
\label{appendix:compare}
To better demonstrate the advantages of our approach over full parameter fine-tuning, LoRA~\cite{hu2021lora}, and existing activation editing methods, we compare them across five key aspects: the percentage of modified parameters, the intervention modules, dynamic localization of intervention modules, data efficiency, and robustness across various tasks.
Using the fine-tuning of LLaMA-3~\cite{dubey2024llama} as an example, as shown in Table~\ref{tab:intro}, compared to full parameter fine-tuning and LoRA, activation editing requires editing only a small fraction of parameters, significantly fewer than the parameters from LoRA.
Activation editing is also a data-efficient approach, often effective even with just 100 examples.
However, we find that existing activation editing methods lack robustness across different datasets, primarily due to their reliance on manually configuring intervention module parameters, which are highly sensitive to task-specific variations.
In contrast, our proposed method, \jola{}, achieves comparable parameter efficiency to the best activation editing methods (e.g., LoFIT \cite{yin2024lofit}), while also delivering robust performance across tasks in low-resource settings.
This is due to dynamically adjusting the intervention module configurations rather than relying on manual parameter tuning.

%% Table: comparison
% \vspace{-1em}
\begin{table}
\centering
% \footnotesize
\resizebox{\linewidth}{!}{
\setlength{\tabcolsep}{5pt}
\begin{tabular}[t]{l|ccc}
\toprule
 \makecell[c]{\textbf{Method}} & \makecell[c]{\textbf{Self}\\\textbf{Reflection}} & \makecell[c]{\textbf{Memory}} & \makecell[c]{\textbf{Length}\\\textbf{Generalization}} \\
\midrule
Revision~\cite{DBLP:journals/corr/abs-2408-03314} & \redcross & \greencheck & \redcross \\
Self-Refine~\cite{DBLP:conf/nips/MadaanTGHGW0DPY23} & \greencheck & \greencheck & \redcross \\
Best-of-N~\cite{DBLP:journals/corr/abs-2407-21787} & \redcross & \redcross & \greencheck \\
Beam Search~\cite{ow1988filtered} & \redcross & \redcross & \greencheck \\
Guided Beam Search~\cite{DBLP:conf/nips/XieKZZKHX23} & \greencheck & \redcross & \greencheck \\
\midrule
\textbf{FTTT (ours)} & \greencheck & \greencheck & \greencheck \\
\bottomrule
\end{tabular}
}
% \vspace{-5pt}
\caption{Comparing the advantages and drawbacks of FTTT and related works.}
\label{tab:compare}
% \vspace{-0.5cm}
\end{table}
%% Table: Dataset
\section{Dataset Generation}
\label{sec:dataset}
\revise{
To train the proposed GNN, we constructed a dataset of building structures and a subset of these structures were subjected to fire simulations using FEA. The dataset generation process is illustrated in \figref{fig:dataset_generation_procedure}. Initially, a total of 33,000 building structures with geometrical details, material properties, and gravity loads were created. Due to randomness in generating these structures, a filter is applied to remove unreasonable data after gravity load simulation, which included 15,377 structures. A trade-off between computational feasibility and model performance is made among the remaining 17,623 structures. As further labeling structures with MIDR requires resource-intensive fire simulations via OpenSeesRT, a large proportion of 16,050 structures is selected as unlabeled dataset. On the other hand, each of the other 1,573 structures was further subjected to 30 different fire simulations, forming the labeled dataset containing $1,573\times 30 = 47,190$ fire cases.} This section details the step-by-step process for generating the dataset, including geometry creation, material property assignment, and simulations due to gravity loads and fire scenarios. 
% To train the proposed neural network, we constructed a dataset comprising building structure data and a subset of fire scenario data. The dataset generation process is illustrated in \figref{fig:dataset_generation_procedure}. 
% A total of 33,000 building structures with geometric details, material properties, and gravity loads were initially created. Out of these, 3,000 structures were selected as labeled data, and the remaining 30,000 were designated as unlabeled data. Further, about half of them filtered out due to instability under gravity loads only. 
\begin{figure*}[h!]
    \centering
    \includegraphics[width=0.8\linewidth]{figures/dataset_filter_procedure.pdf}
    \caption{Workflow for dataset generation (geometry, material property, gravity loads, and fire scenarios).}
    \label{fig:dataset_generation_procedure}
\end{figure*}

\subsection{Geometry Generation}
\label{subsec:geometry_generation}
The geometry of the building structures forms the foundation of the dataset. Regular 
\revise{3D structures} resembling multi-story parking structures or shopping malls were generated, with parameters such as building floor dimensions and story heights selected randomly. Each building structure is composed of multiple rooms, which serve as the basic unit in this study. A room herein is a cuboid space defined by specific length, width, and height. Within a structure, rooms of the same dimensions are uniformly arranged along the length, width, and height, corresponding to the $x$-, $y$-, and $z$-axes, respectively. Structures vary in room size and number of rooms along each axis. Specifically, the room length, width, and height are independently sampled from a uniform distribution within the interval $[2, 5]$ meters along the three directions of the structure. Similarly, the room number along each axis is uniformly sampled independently as an integer within the interval $[2, 7]$, i.e., the maximum number of stories of the buildings simulated in this study is 7.

To introduce variability and simulate real-world scenarios, approximately $8\%$ of structural elements (beams or columns) are randomly removed after initial geometry creation. 
\revise{Such removal is not fire-induced damage, but reflects functional diversity often observed in real buildings, such as open spaces designed for activities in shopping malls, e.g., ice skating rinks. Examples of the generated geometries are illustrated in \figref{fig:example_generated_geometry}, showcasing the diversity and realism of the dataset. This element removal does not affect the definition of room's geometry in the structure and nor does it affect the number of considered fire scenarios.} 

\revise{A range of coefficient of variation values ($3.3\%$ to $17.5\%$) was derived from prior studies that investigated the statistics of geometrical and material properties of structural components of buildings (e.g., \cite{mirza1979variations, lee2004probabilistic}). These studies provide empirical data on the natural variability in parameters such as Young's modulus, yield strength, and dimensions of structural elements due to manufacturing tolerances and material inconsistencies. By selecting $8\%$ for the removal of structural elements in our database, we aimed to maintain a level of variability that is representative of real-world uncertainties while ensuring computational feasibility. This choice ensures that the database captures realistic deviations without introducing extreme cases that may not be commonly encountered in practice.}

\begin{figure*}[h!]
    \centering
    \includegraphics[width=\linewidth]{figures/example_generated_geometry.pdf}
    \caption{Examples of generated structural geometry of different sizes (all dimensions in meters).}
    \label{fig:example_generated_geometry} 
\end{figure*}

{\blockRevise

In this study, we opted for a deterministic square, dimension of $0.1$ m, solid cross-sectional steel elements due to their simplicity in modeling and analysis. Square sections exhibit uniform geometrical properties in all directions, simplifying the computation of structural responses and avoiding complications associated with more complex shapes, such as wide-flange sections, facilitating the computational efficiency and scalability to generate a large dataset. This choice also helps to mitigate issues related to stress concentrations and facilitates a more straightforward representation of structural behavior under thermal loads. 

\textit{Remark:} The selected cross-section provides a comparable flexural rigidity to a $W 130 \times 130 \times 28.1$ wide-flange section (metric units), albeit with significantly higher axial rigidity. This cross-section is acceptable for gravity-load-designed frames under service loading conditions where the models assume fully rigid, moment-resisting beam-column connections for the evaluation of the IDR under thermal loading. This assumption is reasonable in this computational study where the primary interest is to understand the global deformation response of frames under fire conditions. The selection of uniform square cross-sections for both beams and columns, rather than adherence to standard capacity design principles, was made here primarily for computational efficiency and to reduce design parameters in the database generation process. This choice allows for simplified and scalable approach to analyze the fire-induced response of generic steel frames without the need for large section variations, where this study mainly focuses on the fire vulnerability assessment using ML-based predictions. However, if additional loading conditions, e.g., seismic or wind loads, were to be considered, larger sections, strong-column/weak-beam principle, and ductile detailing would be required in the generated buildings for realistic structural behavior under combined loading conditions. Future studies may also consider investigating the influence of variable cross-sectional dimensions and semi-rigid connections on the structural performance under fire conditions. 
} % blockRevise

\subsection{Material Properties}
Steel is chosen as the material for the structures. To reflect real-world variations, we randomly assign one of five slightly different steel material types to each structural element. \revise{
The ranges of material properties are provided in \tabref{tab:material_property_ranges} and the properties are sampled from uniform distributions of the corresponding ranges. These variations simulate differences arising from manufacturing batches or regional material properties. That these properties are at ambient temperature and change when the temperature rises due to a fire. The selection of materials with varying properties is aimed at increasing the diversity of the data. Our goal is to represent as wide a range of data as possible with a limited amount of building structure data, thereby enhancing the generalization ability of the GNN. Our assumed material property ranges are expected to be wider than the real-world conditions based on findings in \cite{mirza1979variations, lee2004probabilistic}. Therefore, we are essentially tackling a more challenging and general task. If we can solve this problem, we are confident that our method will perform equally well or even better in real-world scenarios.
}
\begin{table}[h!]
    \centering
    \caption{Material properties ranges for considered steel structures.}
    \begin{tabular}{lc}
        \toprule
        Property & Range \\
        \midrule
        Young's modulus & [168, 252] GPa \\
        Yield strength & [220, 330] MPa \\
        Strain-hardening ratio & [0.8, 1.2] \% \\
        \bottomrule
    \end{tabular}
    \label{tab:material_property_ranges}
\end{table}

\subsection{Gravity Loads}
Gravity loads are applied to columns and beams based on their \revise{influence (tributary) areas as typically conducted in structural analysis. The considered ``service'' load conditions include the column self-weight and the additional loads directly supported on the beams from their self-weight and weights of the reinforced concrete slabs, people as live load, and building content. An edge beam typically carries approximately half the gravity load supported by a parallel interior beam}. The ranges of gravity loads are listed in \tabref{tab:gravity_load_ranges}. \revise{The loads are sampled from uniform distributions of the corresponding ranges.} Structures that failed to meet an MIDR threshold of $1\%$ under gravity loads were deemed unacceptable designs and filtered out, as such configurations of randomly chosen geometry, material, and gravity load combinations were considered unrealistic from a regulatory and practicality points of view.
\begin{table}[h!]
    \centering
    \caption{Gravity load ranges for considered beams and columns.}
    \begin{tabular}{lc}
        \toprule
        Element & Range (kN/m)  \\
        \midrule
        Column & [0.5, 1.0]  \\
        Edge beam & [1.5, 4.5]  \\
        Interior beam & [3.0, 7.5]  \\
        \bottomrule
    \end{tabular}
    \label{tab:gravity_load_ranges}
\end{table} 

\subsection{Rule-based Thermal Load Generation}
\label{subsec:thermal_load_generation}
To evaluate a building's structural response during a fire event, we employed a simplified rule-based approach for thermal load generation. 
% Previous studies \cite{nan_structuralfire_2023} have demonstrated that steel structures rapidly equilibrate with surrounding gases temperatures due to efficient heat exchange. Consequently, gas temperatures can be directly used as inputs for FEA tools, e.g., OpenSees, simplifying the process of modeling thermal loads. 
% Accurately simulating temperature fields in fire scenarios poses significant challenges. Advanced thermodynamic simulations, such as those performed using Fire Dynamics Simulator (FDS) \cite{mcgrattan_fire_2000}, provide precise temperature predictions. However, these methods are hindered by high computational costs, prolonging execution times, and limited scalability, making them impractical for generating large datasets. Additionally, real-world fire loads often display substantial spatial variability across different rooms \cite{dundar_fire_2023}, resulting in scenario-specific temperature fields with limited generalizability. For example, studies on bridge fires \cite{he_study_2024} have demonstrated that environmental factors, such as wind speeds, can significantly influence temperature distributions. Furthermore, even within identical scenarios, variations in fire modeling methodologies can produce distinctly different temperature fields \cite{zhang_temperature_2020, du_new_2012}. These challenges emphasize the need for efficient and adaptable methods to generate fire temperature data.
% To address these issues, we adopted a rule-based approach to model temperature variations. 
According to \cite{spearpoint_fire_2008}, a typical fire development follows a predictable pattern. During the {\em{growth stage}}, the temperature rises slowly and approximately linearly after ignition. This is followed by the {\em{flashover stage}}, where temperatures increase rapidly to peak values. After reaching the peak, the temperature either stabilizes or continues to rise slowly until the {\em{decay stage}} begins. Inspired by this fire development pattern, we describe the temperature evolution in time, $t$, prior to the decay stage in two distinct stages:
\begin{enumerate}
    \item {\bf{Initial linear increase stage}}: For $t \in [0, t_1)$, temperature increases gradually and linearly as the fire spreads through the building. This stage represents the time before the fire directly affects a structural element.  
    \item {\bf{ISO 834 fire curve stage}}: For $t \in [t_1, t_{\thre}]$, temperature rises rapidly following the ISO 834 curve \cite{ISO834}, modeling the direct impact of the fire on the structural element. 
\end{enumerate}
The slope of the linear temperature increase, $c$, and the transition time, $t_1$, are influenced by the spatial relationship between the fire source and the structural element. For the second stage of temperature evolution, we utilize the ISO 834 curve, a widely accepted standard for fire resistance testing. This standardized fire curve describes the temperature rise over time, enabling rapid and consistent thermal fields across various scenarios. The duration of fire simulation in this study is set to $t_{\thre}=60$ minutes. This value represents the upper limit for the temperature evolution of each structural element, providing a consistent basis for analyzing the structural response to fire.

Let $(x, y, z)$ represents the midpoint of a structural element and $(x_{\subfire}, y_{\subfire}, z_{\subfire})$ the fire source point. \revise{Integer parameters $h$ and $h_{\subfire}$ correspond to the respective floor levels of the element and the fire source}. The temperature evolution for each element is expressed as follows:
\begin{enumerate}
    \item Linear increase stage ($0 < t < t_1$):
    \begin{equation}
    T(t) = c \cdot t,
    \end{equation}
    where $c$, the rate of temperature increase ($^\circ\mathrm{C}/\mathrm{min}$), depends on the height difference between the element, $h$, and the fire source, $h_{\subfire}$:
    \begin{equation}
        c = 
        \begin{cases} 
        5\left/\left(h - h_{\subfire} + 1\right)\right., & h \geq h_{\subfire}, \\
        2\left/\left(h_{\subfire} - h\right)\right., & h < h_{\subfire}.
        \end{cases}
    \end{equation}
     \item ISO 834 stage ($t \geq t_1$):
\begin{equation}
    T(t) = c \cdot t_1 + 345 \log_{10} \left(8 \left(t - t_1\right) + 1\right).
\end{equation}
\end{enumerate}

The transition (arrival) time $t_1$, marking the end of the linear stage, depends on the spatial distance between the fire source and the element. We define the following two Euclidean distances $L_p$ in the $xy$ plane and $L_s$ in the $xyz$ space:
\begin{eqnarray}
L_p & \triangleq & \sqrt{(x - x_{\subfire})^2 + (y - y_{\subfire})^2}, \\
\label{eq:Lp}
L_s & \triangleq & \sqrt{(x - x_{\subfire})^2 + (y - y_{\subfire})^2 + (z - z_{\subfire})^2}.
\label{eq:Ls}
\end{eqnarray}
Accordingly, the transition time, $t_1$, is expressed as follows:
\begin{equation}
    t_1 = 
    \begin{cases}
    \beta_{1} \cdot \left(1 - \exp\left\{- L_s\left/\alpha_{1}\right.\right\}\right), & h > h_{\subfire}, \\
    \beta_{2} \cdot \left(1 - \exp\left\{- L_p\left/\alpha_{2}\right.\right\}\right), & h = h_{\subfire}, \\
    \beta_{3} \cdot \left(1 - \exp\left\{- L_s\left/\alpha_{3}\right.\right\}\right), & h < h_{\subfire} .
    \end{cases}
    \label{eq:t1}
\end{equation}
The parameters $\beta_i$ and $\alpha_i$ for determining $t_1$ are summarized in Table~\ref{tab:fire_spread_parameters}. In this study, we take $r_{\mathrm{up}}=0.95$ and $r_{\mathrm{down}}=0.97$.
\begin{table}[ht]
    \centering
    \caption{Fire spread parameters for $t_1$ calculations.}
    \begin{tabular}{lcc}
        \toprule
        Case  & $\beta_i$ & $\alpha_i$  \\
        \midrule
        $i=1$, Upward spread & $16 \left.\left(1-r_{\mathrm{up}}^{\left|h-h_{\subfire}\right|}\right)\right/\left(1-r_{\mathrm{up}}\right)$ & $10$  \\
        $i=2$, Horizontal spread & $18$ & $18$  \\
        $i=3$, Downward spread & $30 \left.\left(1-r_{\mathrm{down}}^{\left|h-h_{\subfire}\right|}\right)\right/\left(1-r_{\mathrm{down}}\right)$ & $5$  \\
        \bottomrule
    \end{tabular}
    \label{tab:fire_spread_parameters}
\end{table}

\figref{fig:t1_curve} illustrates the $t_1$ curves for various fire scenarios: (1) fire originating on the lower floor, $h-h_{\subfire}=1$ with rapid upward spread, (2) fire on the same floor, $h=h_{\subfire}$ with the fastest spread, and (3) fire on the upper floor, $h_{\subfire}-h=1$ with slow downward spread. The exponential decay in $t_1$ reflects the accelerating fire propagation speed as the distance increases. \figref{fig:t1_curve} also indicates that the employed simplified model is consistent with the Markov chain-based dynamic model given by \cite{cheng_dynamic_2011}, where the rooms at the same floor of the fire point start flashover slightly before the corresponding upper floors. Additionally, $\beta_{1}$ and $\beta_{3}$ are the summation of a geometric sequence, where story level $h$ is the index. The common ratios $r_{\mathrm{up}}<1$ in $\beta_{1}$ and $r_{\mathrm{down}}<1$ in $\beta_{3}$ indicate that the fire speeds up to spread through the next story, which is consistent with the real-world fire spread mechanism given in \cite{hokugo_mechanism_2000}. The temperature profile within the range $t \in [0, t_{\thre}]$ is subsequently used as the thermal load in OpenSeesRT simulations to compute displacements at each structural node at time $t_{\thre}$.
\begin{figure}[h!]
    \centering
    \includegraphics[width=0.8\linewidth]{figures/m204_t1_curve.pdf}
    \caption{Three examples for the $t_1$ curve.}
    \label{fig:t1_curve}
\end{figure}

\revise{
\textit{Remark:} The effects of structural elements, such as concrete floor slabs and partitions, are not explicitly modeled in our approach. Instead, their influence is implicitly captured through the careful selection of the parameters $ \alpha, \beta, r_\mathrm{up} $, and $ r_\mathrm{down} $. This parameterization provides a unified framework for generating temperature fields. Indeed, fire propagation is governed by a multitude of factors and remains an open research question. For instance, if the fire resistance of a floor slab is enhanced by fire protective coating, the corresponding model can account for this by decreasing $\alpha_1$ \& $\alpha_3$, increasing $\beta_1$ \& $\beta_3$, and adopting larger values for $r_\mathrm{up}$ \& $r_\mathrm{down}$, which collectively slow down the vertical spread of fire. Conversely, scenarios involving higher amounts of combustible materials would warrant the opposite adjustments. This flexible and integrated approach avoids the need to design separate models for different fire propagation scenarios while still capturing the essential effects.
}

\revise{
In conclusion, our rule-based approach is a computationally efficient method for approximating fire temperature fields, enabling large-scale dataset generation to train predictive models. By combining ISO 834 fire curves with spatial considerations and embedding structural effects through parameter calibration, the method achieves a balanced trade-off between accuracy and scalability, making it a practical solution for thermal load modeling in fire scenarios. After generating the temperature of each beam or column according to the middle point, the temperature is applied as uniform thermal load to the elements of the structure in question using OpenSeesRT. 
}

% In conclusion, this rule-based approach is a computationally efficient method to approximate fire temperature fields, enabling large-scale dataset generation to train predictive models. By combining ISO 834 fire curves with spatial considerations, the method balances accuracy and scalability, making it a practical solution for thermal load modeling in fire scenarios.

% \subsection{Interstory Drift Ratio}
\subsection{OpenSeesRT Simulation}
\label{subsec:opensees_simulation}

The thermal and mechanical responses of 3D frame structures under combined fire and gravity loads are simulated using OpenSeesRT \cite{perez2024openseesrt}. \revise{In the simulation, the IDR of each node at $t_{\thre}$ is computed using the computed nodal displacements. Each structural model features six degrees of freedom per node (3 translational  and 3 rotational), with linear geometrical transformations (\texttt{geomTransf: Linear}) defining how the element local coordinate systems are mapped to the global coordinate system and assuming small displacements and rotations. Although OpenSeesRT allows a variety of options for modeling finite deformations, in the present simulations and mainly for simplicity, we did not consider large deformations. All bottom nodes (nodes on the ground) are fully constrained in all six degrees of freedom, while degrees of freedom os all other nodes are free.} Material behavior is temperature-dependent and modeled with \texttt{Steel01Thermal}, while fiber-based sections (\texttt{FiberThermal}) capture nonlinear interactions between thermal and mechanical responses at the cross-section level. \revise{Structural elements are represented as displacement-based Euler-Bernoulli beam-columns (\texttt{dispBeamColumnThermal}). This element  formulation accounts for thermal strains (temperature gradients) in the section, which is discretized into fibers. Numerical integration is used along the length of each element using three integration (Gauss) points, one at each end and the third in the middle of the element.}

{\revise{Thermal expansion of steel members plays a crucial role in IDR development. In reality, reinforced concrete floor slabs heat at a different rate than steel members due to their higher thermal mass and lower thermal conductivity. This differential heating can lead to restrained thermal expansion, introducing axial compression in beams and affecting the overall structural response. In this study, explicit {\em{composite action}} between steel members and concrete slabs is not modeled. Instead, our approach focuses on isolating the response of the steel structural frame, which is often the critical load-bearing component in fire scenarios. This assumption aligns with prior studies \cite{Possidente_2024} demonstrating that steel structures reach thermal equilibrium with surrounding gases quickly, allowing the use of uniform thermal loading in fire analysis. Future work could enhance this framework by incorporating slab-beam interaction effects, through a refined FEA for an extended dataset where constraints imposed by floor slabs are explicitly considered.}

The analysis begins with the application of gravity loads, followed by incremental thermal loads simulating the fire exposure. A static nonlinear solver using  \texttt{ExpressNewton} algorithm ensures convergence, while the \texttt{NormDispIncr} test maintains accuracy. An incremental \texttt{LoadControl} scheme with small step sizes is employed to guarantee numerical stability, using 10\% for gravity loads and 1\% for thermal loads. 

\revise{
In the thermal load analysis, uniform thermal load is applied to each beam or column, i.e., the temperature of each element is set to be that at the middle point, according to \secref{subsec:thermal_load_generation}. The \texttt{Steel01Thermal} material allows the properties (e.g., Young's modulus and yield strength) to be adjusted at increasing temperatures according to \cite{EN1993} using its Table 3.1: Reduction factors for the stress-strain relationship of carbon steel at elevated temperatures. For example, if the Young’s modulus at ambient temperature is $E_0$, then as the temperature ($T$) increases, the modulus changes as $E(T) = \eta (T) \times E_0$. \cite{EN1993} directly provides the values of $\eta(T) \in \left[0,1\right] $ at every $100 ^\circ\mathrm{C}$ interval and recommends using linear interpolation to obtain $\eta(T)$ for intermediate values of $T$.
} OpenSeesRT documentation \cite{OpenSeesThermalExamples} provides several examples of thermal analyses.

This modeling framework accommodates variations in material properties, cross-sectional geometries, and temperature profiles, providing robust simulations of structural behavior under fire conditions. The primary settings and configurations for the OpenSeesRT simulations are summarized in \tabref{tab:ops_detail}.
\begin{table}[h!]
    \centering
        \caption{Key settings of OpenSeesRT simulations.}
    \begin{tabular}{l|>{\raggedright\arraybackslash}p{0.6\linewidth}} %
    \toprule
    Modeling Aspect     & Details \\
    \midrule
    Geometry            & 3D models; 6 degrees of freedom per node \\
    Transformation      & geomTransf: Linear \\ 
    Material            & Steel01Thermal \\
    Section             & FiberThermal; Cross-section: $0.1$ m $\times$ $0.1$ m \\ 
    Element type        & {dispBeamColumnThermal} \\ 
    Loading             & Gravity loads: {beamUniform}; Thermal loads: {beamThermal} \\
    Integration scheme  & Incremental {LoadControl}; Step size: $10\%$ (gravity analysis), $1\%$ (thermal analysis) \\
    Nonlinear solver    & {ExpressNewton} algorithm; {UmfPack} solver; Convergence test: {NormDispIncr} tolerance: $10^{-8}$; Maximum \# iterations per step: $1000$. \\ 
    \bottomrule
    \end{tabular}
    \label{tab:ops_detail}
\end{table}

For each structure in the labeled dataset, 30 fire points are selected using a dual-granularity approach, \revise{i.e., two-stage sampling strategy,} to ensure they are well-distributed. Specifically, rooms are sequentially selected, with one fire point randomly chosen within each selected room. If a building is large and contains more than 30 rooms, we randomly select 30 rooms without replacement, i.e., ensuring that no more than one fire point is located in the same room. Conversely, if the building is small and has fewer than 30 rooms, all rooms are initially selected, with one fire point randomly assigned to each room. Additionally, rooms are then selected with replacement until a total of 30 fire points are assigned. \revise{The room-level sampling prioritizes selecting distinct rooms to avoid spatial clustering of fire points, while the point-level sampling ensures intra-room variability. This approach aligns with stratified sampling principles commonly used for efficient spatial representation, where multi-stage sampling strategies optimize coverage and variability, e.g., \cite{arunachalam_generalized_2023}, and enables a more comprehensive characterizing of how the structures respond under fire conditions.}
% This selection method prevents fire points from clustering too closely while maintaining an element of randomness. By distributing fire points in this manner, the 30 fire scenarios are effectively utilized, enabling a more comprehensive characterizing of how the structures respond under fire conditions.

\subsection{Summary of the Dataset Generation}
As discussed in this section and related to  \figref{fig:dataset_generation_procedure}, three key steps were considered in the development of the dataset: 
\begin{enumerate}
    \item {\bf{Filtering process}}: Structures with MIDR exceeding $1\%$ under gravity loads were excluded,  resulting in $1,573$ labeled structures retained for fire simulation and $16,050$ unlabeled structures for training the MFSP predictor.
    \item {\bf{Fire simulations}}: For each retained labeled structure, 30 fire scenarios were simulated using OpenSeesRT, yielding $47,190$ fire cases.
    \item {\bf{Data distribution check}}: MIDR distributions for labeled and unlabeled data under gravity loads were highly similar, because both datasets were generated using the same method. Under fire conditions, the MIDR distribution shifted, reflecting significant structural deformation with values reaching a maximum of about 6\%, an average of 1.70\%, and a standard deviation of 1.12\%. This step ensured a diverse and comprehensive dataset for the proposed predictive framework.
\end{enumerate}
The statistical distribution histograms for MIDR (after applying the $1\%$ filtering threshold \revise{for gravity load responses}) under different loading conditions are plotted in \figref{fig:histogram_mdr}. Figures \ref{fig:histogram_mdr}(a) and \ref{fig:histogram_mdr}(b) show the MIDR distributions of the labeled and unlabeled data, respectively, under gravity loads only. \figref{fig:histogram_mdr}(c) shows the MIDR distribution of the labeled data under the combined effects of gravity and fire loads. Fire load causes the structures to significantly deform, leading to a noticeably \revise{right-skewed} MIDR distribution.

\begin{figure*}[h!]
    \centering
    \includegraphics[width=\linewidth]{figures/histogram_mdr.pdf}
    \caption{Histograms of MIDR for labeled and unlabeled structures with gravity loads and fire cases.}
    \label{fig:histogram_mdr}
\end{figure*}

\revise{
This dataset provides the basis for training and testing the performance of the GNN-based framework. Although we employed a simplified rule-based thermal load generation method compared with conventional CFD-based simulations, the temperature field, the changes of the material properties, and the response of the structures, are all still highly nonlinear and complex. Therefore, it is still a challenging task for the NN to predict the MIDRs based on this dataset.
}

\section{Datasets}
\label{appendix:dataset}
We conduct experiments across three tasks: commonsense reasoning~\cite{hu-etal-2023-llm}, natural language understanding~\cite{wang2024mmlu}, and natural language generation~\cite{gehrmann-etal-2022-gemv2}.
Table~\ref{tab:dataset} provides a brief overview of the sub-datasets or sub-tasks within the three benchmarks evaluated.
The commonsense reasoning task is framed as a multiple-choice problem, where the correct answer is selected from 2 to 4 possible options.
The natural language understanding task also follows a multiple-choice format, but with ten options.
The natural language generation task, on the other hand, is an end-to-end text generation task, where unstructured data (such as commonsense concepts or data) is converted into coherent text.
In the training phase, we simulate a low-resource scenario by using 200 examples.
Section~\ref{subsec:data_model_size} further explores experiments with varying numbers of samples.
To ensure consistency across experiments, we used the same random seed (seed$=42$) for data sampling, ensuring identical training samples in all runs.

%% Table: prompt
\begin{tcolorbox}[title={The Prompt used for Translation}]
You are a highly skilled translator tasked with translating various types of content from English into \{\{ language \}\}. Follow these instructions carefully to complete the translation task.

You will receive a user-bot conversation in XML format. Please follow a three-step translation process:

\begin{enumerate}
  \item \textbf{Initial Translation:} Translate the input content into \{\{ language \}\}, preserving the original intent and keeping the original paragraph and text format unchanged. Do not delete or omit any content, and ensure that all original Markdown elements (e.g., images, code blocks) are preserved.
  \item \textbf{Reflection and Feedback:} Carefully review both the source text and your translation. Provide constructive criticism and specific suggestions to improve the translation in terms of:
    \begin{enumerate}[label=(\roman*)]
      \item \textbf{Accuracy:} Correct errors of addition, mistranslation, omission, or untranslated text.
      \item \textbf{Fluency:} Apply \{\{ language \}\} grammar, spelling, and punctuation rules while avoiding unnecessary repetitions.
      \item \textbf{Style:} Ensure that the translation reflects the style of the source text and considers any relevant cultural context.
    \end{enumerate}
  \item \textbf{Refinement:} Based on your reflections, refine and polish your translation.
  \item \textbf{Fallback:} If you are not confident in translating the conversation, please return ``\texttt{<stop></stop>}''.
\end{enumerate}

\bigskip
\textbf{Output:}

For each step of the translation process, output your results within the appropriate XML tags as follows:
\begin{verbatim}
<step1_initial_translation>
[Insert your initial translation here]
</step1_initial_translation>

<step2_reflection>
[Insert your reflection on the translation, including a list 
of specific, helpful, and constructive suggestions for 
improvement. Each suggestion should address a specific 
part of the translation.]
</step2_reflection>

<step3_refined_translation>
[Insert your refined and polished translation here]
</step3_refined_translation>
\end{verbatim}

Ensure that your final translation in step 3 accurately reflects the original meaning while sounding natural in \{\{ language \}\}.

Here is the original conversation:
\label{box:trans_prompt}
\end{tcolorbox}

%% parameter search
\begin{table}[!thp]
\caption{
\label{tab:hyperparam}
Hyperparameter configurations for the baseline methods evaluated in our experiments. These settings are used across multiple tasks to assess model performance in low-resource settings, as discussed in Sections~\ref{sec:intro} and Section~\ref{sec:exp}.
}
\resizebox{\textwidth}{!}{
\begin{tabular}{l|l|p{8cm}}
\toprule
\textbf{Baseline} & \textbf{Hyperparameter} & \textbf{Values} \\
\midrule
\multirow{2}{*}{BitFit~\cite{ben-zaken-etal-2022-bitfit}} & Bias Moudule & bias of Q,K and V from attention/bias of LayerNorm from attention outputs/bias of LayerNorm from hidden outputs \\ \cmidrule{2-3}
 & Learning Rate & 1e-4/ 5e-4 \\
\midrule
\multirow{2}{*}{RED~\cite{wu-etal-2024-advancing}} & Rank & 8 / 16 \\
\cmidrule{2-3} 
 & Learning Rate & 5e-5/ 2e-4 / 6e-2 \\
\midrule
REPE~\cite{zou2023representation} & method & Representation Reading / Representation Control \\
\midrule
\multirow{2}{*}{ReFT~\cite{wu2024reft}} & Prefix + suffix posotion & p7 + s7 / p11 + s11 \\
\cmidrule{2-3} 
 & Layers & all / 4,6,10,12,14,18,20,22/3,9,18,24 \\
\midrule
\multirow{2}{*}{LoFIT~\cite{yin2024lofit}} & number of attention heads & 32/64/128 \\
\cmidrule{2-3} 
 & Learning Rate & 5e-4 / 5e-3 \\
\bottomrule
\end{tabular}
}
\end{table}


\section{Prompt Setting}
\label{appendix:prompt}
Recent studies~\cite{he2024does,lai-etal-2024-llms} have highlighted the substantial impact of prompt design on model performance.
In our experiments, we adopt the same prompt configurations as~\citet{hu-etal-2023-llm} for the commonsense reasoning benchmark, and used the prompts from the original paper for the MMLU-Pro benchmark~\cite{wang2024mmlu}.
For the GEM benchmark~\cite{gehrmann-etal-2022-gemv2}, where the original paper did not provide the prompt settings, we utilized commonly used prompts curated from PromptSource\footnote{\url{https://github.com/bigscience-workshop/promptsource}}.
To ensure reproducibility of our results, we present the prompts employed in our experiments in Table~\ref{tab:prompt}.

\section{Experiment Configurations}
\label{appendix:exp_config}
\subsection{Training Setup}
\label{appendix:traiing_config}
We conduct all experiments using the HuggingFace Transformers\footnote{\url{https://github.com/huggingface/transformers}} library and fine-tuned the models with the TRL toolkit\footnote{\url{https://github.com/huggingface/trl}}.
The AdamW optimizer~\cite{loshchilov2017decoupled} was used for fine-tuning, with $\epsilon = 1e-6$ and one epoch of warm-up.
In addition, we employ an exponentially decaying~\cite{li2019exponential} learning rate schedule, defined by the following formula:
\begin{equation}
    \text{lr}(t) = \text{lr}_0 \cdot \lambda^{t} \cdot e^{-\text{decay} \cdot t}
\end{equation}
where $\text{lr}_0$ is the initial learning rate $lr_{0}$set to $5 \times 10^{-4}$ , $\lambda$ is 0.1, and the decay rate is 0.01.
For the gating units, we used a temperature of 0.33 in the Gumbel Softmax~\cite{jang2017categorical}.
Fine-tuning was performed in full precision for the 7B, 8B, 1B, and 3B models, while for the 70B model, we applied 4-bit quantization to enable single-precision fine-tuning.

\subsection{Computational Resources}
\label{appendix:comp_resource}
All experiments for the 1B, 3B, 8B, and 13B models were conducted on a single NVIDIA A100 80GB GPU server.
The 70B model, described in Section~\ref{subsec:data_model_size}, was evaluated on an NVIDIA H100 94GB GPU server.
As an example, with the 8B LLaMA-3 model, \jola{} converged within 2 GPU hours on most tasks in the low-resource setting, using only 200 training samples.

\subsection{Hyperparameter Search for Baselines}
\label{appendix:hyper_search}
As discussed in Section~\ref{sec:intro} and Section~\ref{sec:exp}, the performance of baseline methods in low-resource settings is highly sensitive to hyperparameters across different tasks.
However, it is impractical to conduct hyperparameter searches for each task individually, given that we evaluate 26 tasks in total, and performing a separate search for each would be time-consuming.
To mitigate this, we use five different sets of hyperparameters during the baseline experiments.
The average results of these five configurations are reported in Table~\ref{tab:main_res} and Figure~\ref{fig:main_res}.
The specific hyperparameters used for each method are detailed in Table~\ref{tab:hyperparam}.

\section{Full Results Across all Tasks}
\label{appendix:full_res}
Due to page limitations, we present the average performance across the 26 tasks in Table~\ref{tab:main_res} and Figure~\ref{fig:main_res}.
In this section, we provide detailed performance metrics for each individual task.
Specifically, Table~\ref{tab:llama_reason_full} reports the accuracy of LLaMA-3 on the commonsense reasoning task, while Table~\ref{tab:qwen_reason_full} presents the accuracy of Qwen-2.5 on the same task.
Table~\ref{tab:llama_mmlu_full} shows the accuracy of LLaMA-3 on the natural language understanding task, and Table~\ref{tab:qwen_mmlu_full} shows the corresponding accuracy for Qwen-2.5.
Finally, Table~\ref{tab:llama_gem_full} presents the BLEU, ROUGE-L, and BERTScore for LLaMA-3 on the natural language generation task, with Table~\ref{tab:qwen_gem_full} displaying the corresponding metrics for Qwen-2.5.

%% full result
\begin{table*}[h]
  \centering
  % \vspace{-1mm}
  \scalebox{0.9}{%
    \begin{tabular}{l||c|c|c||c|c|c}
        % \toprule
        \hline
        \multirow{2}{*}{method} & \multicolumn{3}{c||}{$b=3$} &\multicolumn{3}{c}{$b=5$} \bigstrut \\
        % \cmidrule{2-7}
        \cline{2-7}        
         & \WL & \CL & \EL  & \WL & \CL & \EL  \bigstrut\\
        % \midrule
        \hline
        \hline
        \textsc{Random} & 0.0016 (0.0214) & -0.0133 (0.1105) & 0.0010 (0.0182)  &
        -0.0045 (0.0216) & 0.0010 (0.1188) & 0.0081 (0.0158) \bigstrut[t]\\
        OdC & 0.0022 (0.0225) & 0.0063 (0.1077) & 0.0123 (0.0207)  &
        0.0117 (0.0238) & 0.0352 (0.1534) & 0.0119 (0.0407)  \\
        BC & -0.0006 (0.0226) & 0.0006 (0.0860) & -0.0011 (0.0208)  &
        0.0019 (0.0220) & -0.0057 (0.0992) & -0.0028 (0.0206) \\
        PR &-0.0017 (0.0175) & -0.0036 (0.1204) & -0.0002 (0.0187)  &
        -0.0017 (0.0201) & -0.0010 (0.1077) & -0.0124 (0.0182) \bigstrut[b] \\
        % \midrule
        \hline
        \BPM & \textbf{O.O.T} & 0.0159 (0.1221) & 0.0387 (0.0471)  &
        \textbf{O.O.T} & -0.0021 (0.1074) & 0.0477 (0.0527) \bigstrut[t] \\
        \KED &-0.0006 (0.0174) & 0.0109 (0.1162) & -0.0015 (0.0180)  &
        0.0029 (0.0178) & 0.0097 (0.1317) & -0.0029 (0.0556) \\
        \MDS &0.0098 (0.0254) & -0.0004 (0.1085) & 0.0298 (0.0366)  &
        0.0173 (0.0229) & 0.0202 (0.1341) & 0.0368 (0.0421) \bigstrut[b] \\
        % \midrule
        \hline
        %\MBPM-10 & . & . & . & .\\
        \MBPM-100 & 0.0105 (0.0622) & 0.0108 (0.1358) & 0.0093 (0.0610)  &
        0.0098 (0.0630) & -0.0076 (0.1215) & 0.0116 (0.0594) \bigstrut[t] \\
        \MBPM-1000 & 0.1342 (0.2195) & -0.0013 (0.1308) & 0.1396 (0.2096)  &
        0.1589 (0.2245) & -0.0052 (0.1187) & 0.1767 (0.2150) \\
        \MBPM-10000 & 0.2811 (0.2140) & 0.0136 (0.1350) & 0.3836 (0.2014)  &
        0.3160 (0.2170) & 0.0183 (0.1335) & 0.4273 (0.2127) \\
        \MBPM-100000 & \textbf{O.O.T} & 0.2046 (0.2338) & 0.4468 (0.1766)  &
        \textbf{O.O.T} & \textbf{O.O.T} & 0.5284 (0.1790) \bigstrut[b] \\
        % \midrule
        \hline
        \Greedy-10 & 0.0059 (0.0481) & -0.0182 (0.0976) & 0.0244 (0.0995)  &
        0.0062 (0.0459) & -0.0052 (0.1040) & 0.0236 (0.1011) \bigstrut[t] \\
        \Greedy-100 & \textbf{O.O.T} & -0.0052 (0.1144) & 0.1560 (0.2363)  &
        \textbf{O.O.T} & 0.0052 (0.1374) & 0.1635 (0.2356) \bigstrut[b] \\
        % \midrule
        \hline
        \RIS-0.6 & 0.2280 (0.1720) & 0.3813 (0.1941) & 0.2937 (0.1968) & \textbf{O.O.T} & 0.5045 (0.2064) &   0.3641 (0.2102) \bigstrut[t] \\
        \RIS-0.4 & \textbf{O.O.T} &  0.4428 (0.1960) & 0.3202 (0.2078) & \textbf{O.O.T} & 0.5227 (0.2136) & 0.4156 (0.2163)\\
        \RIS-0.2 & \textbf{O.O.T} & 0.4456 (0.1893) & 0.3688 (0.2100) & \textbf{O.O.T} &  0.5591 (0.2217) & 0.4513 (0.2212) \bigstrut[b]  \\
        \hline
        %\naive (2GCN) & 0.3316 (0.1942) & 0.4789 (0.1912) & 0.4435 (0.1950)  &
        % 0.3977 (0.1819) & 0.6096 (0.1853) & 0.5224 (0.1959) \bigstrut[t] \\
        \naive & 0.3609 (0.1796) & 0.5240 (0.1876) & 0.4675 (0.1855)  &
        0.4311 (0.1726) & 0.6547 (0.1877) & 0.5613 (0.1828) \bigstrut[t] \\
        %\hline
        %\adv (2GCN) & 0.3180 (0.1817) & 0.4905 (0.2247) & 0.4246 (0.2119)  &
        %0.3897 (0.1779) & 0.6261 (0.2151) & 0.5126 (0.2022) \bigstrut[t] \\
        \adv  & 0.2713 (0.1978) & 0.5172 (0.1930) & 0.4415 (0.1871)  &
        0.3914 (0.1796) & 0.6614 (0.1957) & 0.5332 (0.1845)  \\
        %\hline        
        %\advp (2GCN) & 0.3061 (0.1906) & 0.4985 (0.2176) & 0.4195 (0.2065) &
        %0.3886 (0.1855) & 0.6348 (0.2169) & 0.5048 (0.2049) \bigstrut[t] \\
        \advp  & 0.3236 (0.1856) & 0.5125 (0.1956) & 0.3512 (0.2002)  &
        0.3876 (0.1759) & 0.6583 (0.1948) & 0.4718 (0.2066)  \\
        % \midrule[\heavyrulewidth]
        \hline        
        \multicolumn{7}{c}{}\\
        \hline        
        \multirow{2}{*}{method} & \multicolumn{3}{c||}{$b=7$} &\multicolumn{3}{c}{$b=10$} \bigstrut\\
        % \cmidrule{2-7}
        \cline{2-7}
         & \WL & \CL & \EL  & \WL & \CL & \EL  \bigstrut \\
         % \midrule
         \hline
         \hline
        \textsc{Random}  &  -0.0011 (0.0187) & -0.0187 (0.1244) & 0.0014 (0.0210)  & 
        0.0002 (0.0201) & -0.0072 (0.1150) & 0.0005 (0.0182) \bigstrut[t] \\
        OdC  &  0.0275 (0.0330) & 0.0106 (0.1120) & 0.0315 (0.0439)  & 
        0.0327 (0.0338) & 0.0263 (0.1294) & 0.0448 (0.0494) \\
        BC  &  0.0027 (0.0194) & 0.0196 (0.1118) & 0.0090 (0.0557)  & 
        0.0020 (0.0234) & 0.0054 (0.1013) & 0.0158 (0.0534) \\
        PR  &  -0.0035 (0.0225) & 0.0187 (0.1071) & -0.0020 (0.0203) & 
        0.0006 (0.0239) & 0.0257 (0.1377) & 0.0002 (0.0218) \bigstrut[b] \\
        % \midrule
        \hline
        \BPM  &  \textbf{O.O.T} & 0.0016 (0.0980) & 0.0882 (0.1145)  & 
        \textbf{O.O.T} & -0.0085 (0.1030) & 0.1047 (0.1432) \bigstrut[t] \\
        \KED  &  -0.0010 (0.0158) & 0.0017 (0.1316) & 0.0079 (0.0569)  & 
         -0.0012 (0.0197) & -0.0100 (0.1319) & 0.0113 (0.0579)\\
        \MDS  &  0.0160 (0.0244) & 0.0015 (0.1112) & 0.0372 (0.0442)  & 
        0.0163 (0.0258) & 0.0230 (0.1354) & 0.0396 (0.0424) \bigstrut[b] \\
        % \midrule
        \hline
        \MBPM-100  &  0.0461 (0.1659) & 0.0091 (0.1202) & 0.0258 (0.1008)  & 
        0.0507 (0.1711) & -0.0080 (0.0778) & 0.0252 (0.1015)\bigstrut[t] \\
        \MBPM-1000  &  0.1587 (0.2318) & 0.0009 (0.1315) & 0.1778 (0.2219)  & 
        0.1641 (2315) & 0.0128 (0.1172) & 0.2014 (0.2220) \\
        \MBPM-10000  & 0.3365 (0.2215) & 0.0153 (0.1437) & 0.4673 (0.2089)  & 
        0.3545 (2243) & 0.0289 (0.1307) & 0.4991 (0.2072) \\
        \MBPM-100000  &  \textbf{O.O.T} & \textbf{O.O.T} & \textbf{O.O.T}  & 
        \textbf{O.O.T} & \textbf{O.O.T} & \textbf{O.O.T} \bigstrut[b] \\
        % \midrule
        \hline
        \Greedy-10  &  0.0076 (0.0493) & -0.0152 (0.1241) & 0.0224 (0.1024)  & 
        0.0136 (0.0654) & -0.0300 (0.1453) & 0.0321 (0.1111) \bigstrut[t] \\
        \Greedy-100  &  \textbf{O.O.T} & 0.0030 (0.1299) & 0.1809 (0.2340)  & 
        \textbf{O.O.T} & -0.0074 (0.1414) & 0.1977 (0.2358) \bigstrut[b]\\
        \hline
        \RIS-0.6 & \textbf{O.O.T} & 0.5732 (0.2292) & 0.4172 (0.2134) & \textbf{O.O.T} & 0.7590 (0.2020) & 0.6390 (0.1443) \bigstrut[t]\\
        \RIS-0.4 & \textbf{O.O.T} & 0.6093 (0.2117) & 0.4688 (0.2131) & \textbf{O.O.T} & 0.8113 (0.1836) & 0.6517 (0.1534) \\
        \RIS-0.2 & \textbf{O.O.T} & \textbf{O.O.T} & 0.5116 (0.2195) & \textbf{O.O.T} & \textbf{O.O.T} & \textbf{O.O.T} \bigstrut[b] 
        \\
        % \midrule
        \hline
        %\naive (2GCN)  &  0.4479 (0.1778) & 0.7029 (0.2012) & 0.5885 (0.1877)  & 
        %0.5000 (0.1698) & 0.7882 (0.2085) & 0.6519 (0.1778) \bigstrut[t]\\
        \naive  &  \textbf{O.O.T} & 0.7455 (0.1871) & 0.6171 (0.1757) & 
        \textbf{O.O.T} & 0.8124 (0.1870) & 0.6780 (0.1625) \bigstrut[t] \\
      % \adv (2GCN) &  0.4353 (0.1739) & 0.7291 (0.2218) & 0.5731 (0.1970) & 
        % 0.4833 (0.1729) & 0.8151 (0.2153) & 0.6339 (0.1892) \bigstrut[t] \\
        \adv &  0.4301 (0.1754) & 0.7423 (0.2034) & 0.5881 (0.1794)  &
        0.4758 (0.1663) & 0.8352 (0.1957) & 0.6439 (0.1725)  \\
      %  \advp (2GCN)&  0.4376 (0.1719) & 0.7268 (0.2191) & 0.5663 (0.1881)  & 
       % 0.4819 (0.1738) & 0.8145 (0.2157) & 0.6271 (0.1826) \bigstrut[t]\\
        \advp  &  0.4289 (0.1714) & 0.7417 (0.1914) & 0.5286 (0.2078)  & 
        0.4772 (0.1655) & 0.8346 (0.1919) & 0.6023 (0.1980) \\ 
    % \bottomrule
    \hline
    \end{tabular}
    }
    \caption{The effectiveness (the reduced ratio of influence) of each method with the standard deviations, with budget $b \in \Set{3, 5, 7, 10}$.
  O.O.T denotes out-of-time, i.e., the method does not terminate within one hour on a single seed set in the corresponding setting.}
    \label{tab:app_perf}
\end{table*}

\section{Case Study}
\label{appendix:case_study}
To provide an intuitive evaluation of the advantages of our method, we select one representative case from each of the tasks: commonsense reasoning, natural language understanding, and natural language generation.
The results generated by the baseline and our approach are presented below.

%% case study
\begin{table*}[t]
\centering
\caption{A case study examining \model's response to a query about ``\emph{graders}'' in OpenAI's 12-day edu    cational series demonstrates its video retrieval capabilities and knowledge synthesis.}
\label{tab:case study}
\small
\begin{tabularx}{\textwidth}{X X}

\toprule
\textbf{Query:} Explain the purpose and functionality of 'graders' in the context of reinforcement fine-tuning.\\

\midrule
\textbf{Input Videos:} \href{https://www.youtube.com/playlist?list=PLOXw6I10VTv9lin5AzsHAHCTrC7BdVdEM}{12 Days of OpenAI} (A total of 12 videos) \\

\midrule

\textbf{\model\ Response:} 

\texttt{\#\#\# Understanding Gradation in Reinforcement Fine-Tuning}

\texttt{**Purpose of Graders**}

\texttt{Graders are integral components in the process of reinforcement fine-tuning, mainly utilized to evaluate and improve the performance of machine learning models...} (Omitted for brevity)

\texttt{**Functionality of Graders**}

\texttt{\colorbox{lightgreen}{The grading system operates by comparing a model's outputs against those of}}

\texttt{\colorbox{lightgreen}{verified correct responses. The mechanics of this assessment involve assigning}}

\texttt{\colorbox{lightgreen}{scores to the model's outputs based on their correctness. These scores range}}

\texttt{\colorbox{lightgreen}{from zero to one, where:}}


\texttt{\colorbox{lightblue}{- A score of **0** indicates that the model's answer is completely incorrect.}}

\texttt{\colorbox{lightblue}{- A score of **1** signals that the answer is fully correct.}}

\texttt{- Scores can also be fractional, allowing for partial credit. For instance, \colorbox{lightpink}{a score of **0.7** may denote that the model's suggestion was }}
\texttt{\colorbox{lightpink}{correct but not the highest in a ranked list of possible answers.}} 

\\

\texttt{The process includes:}

\texttt{1. **Input Comparison**: The grader takes both the model’s output and the correct answer, analyzing them to derive a score.}

\texttt{2. **Graded Outputs**: This interaction allows for a dynamic reflection of the model's performance, where the scores directly influence how the model is fine-tuned in subsequent training iterations.}

\texttt{3. **Feedback Loop**: The scores obtained from graders provide feedback that can be used to adjust learning parameters and improve the model’s future responses. This highlights the feedback mechanism that is crucial for reinforcement learning.}

\\

(The remaining content are omitted for brevity.)

\\

\midrule
\textbf{Retrieved Sources by \model:}

\model\ accurately retrieved four continuous video segments from the OpenAI's 12-day show, ranging from 10:00 to 12:00 in Day 2 of the video "\href{https://www.youtube.com/watch?v=yCIYS9fx56U}{Reinforcement Fine-Tuning}." Here, we highlight key moments relevant to the detailed content in the answer. From left to right, these are retrieved moments at timestamps \colorbox{lightgreen}{10:35}, \colorbox{lightblue}{10:39}, and \colorbox{lightpink}{11:10}, which provide informative insights that help \model\ give a comprehensive answer to the query.

\\

\begin{tabular}{ccc}
    {\includegraphics[width=0.3\textwidth]{figs/openai-1.png}} &
    {\includegraphics[width=0.3\textwidth]{figs/openai-2.png}} &
    {\includegraphics[width=0.3\textwidth]{figs/openai-3.png}} \\
\end{tabular}

\\

\bottomrule

\end{tabularx}
\vspace{-0.2in}
\end{table*}





