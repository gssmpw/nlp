
\documentclass[11pt]{article}  
\usepackage{amsmath}
\usepackage{amsfonts}
\usepackage{amssymb}
\usepackage{amsthm}
% \usepackage{theorem}
\usepackage{euscript}
\usepackage{ifthen}
\usepackage{xifthen}
\usepackage{xcolor}
\usepackage{fullpage}
\usepackage[english]{babel}
\usepackage{mathtools}
\usepackage{hyperref}
\usepackage{authblk}
\hypersetup{colorlinks = true,linkcolor=black,citecolor=black}
%\mathtoolsset{showonlyrefs}
\renewcommand{\labelenumi}{\theenumi}
\renewcommand{\theenumi}{(\roman{enumi})}
% inversion map
\newcommand{\inv}{\mathrm {inv}}

\usepackage{algorithm} 
\usepackage{algorithmicx}
\usepackage[noend]{algpseudocode}

%\usepackage{showkeys}
%
% \usepackage{pstricks}
% \usepackage{vntex}
%\usepackage{refcheck}
%
% %

\usepackage{array}

\usepackage{imakeidx}
\makeindex

% -------------------------------------------------------------------------
\topmargin -0.0cm
\oddsidemargin -0.1cm
\textwidth  16.4cm 
\headheight 0.0cm
\textheight 21.3cm
\parindent  6mm
\parskip    9pt
\tolerance  1000



\usepackage{hyperref}
\hypersetup
{
	colorlinks,
	citecolor=black,
	filecolor=black,
	linkcolor=black,
	urlcolor=blue
}
\hypersetup{linktocpage}
% -------------------------------------------------------------------------
\newtheorem{theorem}{Theorem}[section]
\newtheorem{lemma}{Lemma}[section]
\newtheorem{corollary}{Corollary}[section]
\numberwithin{equation}{section}
\newtheorem{assumption}[theorem]{Assumption}
% -------------------------------------------------------------------------
 
\newtheorem{prop}[theorem]{Proposition}
\theoremstyle{definition}
\newtheorem{definition}[theorem]{Definition} 
\theoremstyle{remark}
\newtheorem{remark}[theorem]{Remark}



%%%%%%%%%%%%%%%%%%%%%%%%%%%%%%%%%%%%
\newcommand{\brac}[1]{\left(#1\right)}
\newcommand{\brap}[1]{\left[#1\right]}
\newcommand{\brab}[1]{\left\{#1\right\}}
\newcommand{\set}[2]{\{#1\,:\,#2\}}
%%%%%%%%%%%%%%%%%%%%%%%%%%%%%%%%%%%%
\newcommand{\clr}[1]{{\color{red}{ #1 }}}
\newcommand{\clb}[1]{{\color{blue}{#1}}}
\newcommand{\bss}{\boldsymbol{S}}
\newcommand{\norm}[2]{\left\|{#1}\right\|_{#2}}
\newcommand{\snorm}[2]{\left|{#1}\right|_{#2}}
\newcommand{\abs}[1]{\left|#1\right|}

%%%%%%%%%%%%%%%%%%%%%%%%%%%%%%%%%%%%%%%%
\newcommand{\ba}{{\boldsymbol{a}}}
\newcommand{\bb}{{\boldsymbol{b}}}
\newcommand{\bc}{{\boldsymbol{c}}}
\newcommand{\bd}{{\boldsymbol{d}}}
\newcommand{\be}{{\boldsymbol{e}}}
\newcommand{\bj}{{\boldsymbol{j}}}
\newcommand{\bk}{{\boldsymbol{k}}}
\newcommand{\bell}{{\boldsymbol{\ell}}}

\newcommand{\bh}{{\boldsymbol{h}}}
\newcommand{\bl}{{\boldsymbol{l}}}
\newcommand{\bm}{{\boldsymbol{m}}}
\newcommand{\bn}{{\boldsymbol{n}}}
\newcommand{\bp}{{\boldsymbol{p}}}
\newcommand{\bq}{{\boldsymbol{q}}}
\newcommand{\br}{{\boldsymbol{r}}}
\newcommand{\bs}{{\boldsymbol{s}}}
\newcommand{\bt}{{\boldsymbol{t}}}
\newcommand{\bw}{{\boldsymbol{w}}}
\newcommand{\bx}{{\boldsymbol{x}}}
\newcommand{\bX}{{\boldsymbol{X}}}
\newcommand{\by}{{\boldsymbol{y}}}
\newcommand{\bY}{{\boldsymbol{Y}}}
\newcommand{\bz}{{\boldsymbol{z}}}
\newcommand{\bM}{{\boldsymbol{M}}}
\newcommand{\bW}{{\boldsymbol{W}}}
\newcommand{\bN}{{\boldsymbol{N}}}

\newcommand{\bLambda}{{\boldsymbol{\Lambda}}}
\newcommand{\brho}{{\boldsymbol{\rho}}}
\newcommand{\bvarrho}{{\boldsymbol{\varrho}}}
\newcommand{\bsigma}{{\boldsymbol{\sigma}}}
\newcommand{\bphi}{{\boldsymbol{\phi}}}
\newcommand{\bPhi}{{\boldsymbol{\Phi}}}
\newcommand{\balpha}{{\boldsymbol{\alpha}}}
\newcommand{\bnu}{{\boldsymbol{\nu}}}
\newcommand{\bbeta}{{\boldsymbol{\beta}}}
\newcommand{\rd}{{\rm d}} 
\newcommand{\rev}{{\rm ev}} 
\newcommand{\support}{{\rm support}}
\newcommand{\Int}{{\rm Int}} 
% --------------------------------------------------------------------------
\def\II{\mathbb I}
\def\EE{\mathbb E}
\def\ZZd{{\mathbb Z}^d}
%\def\TTd{{\mathbb T}^d}
\def\IId{{\mathbb I}^d}
\def\ZZ{{\mathbb Z}}
\def\ZZdp{{\mathbb Z}^d_+}
\def\RR{{\mathbb R}}
\def\RRd{{\mathbb R}^d}
\def\RRdp{{\mathbb R}^d_+}
\def\NN{{\mathbb N}}
\def\NNd{{\NN}^d}

\def\II{{\mathbb I}}
\def\CC{{\mathbb C}}
\def\CC{{\mathbb C}}
\def\DD{{\mathbb D}}
\def\NN{{\mathbb N}}
\def\RR{{\mathbb R}}
\def\GG{{\mathbb G}}
\def\Vv{{\mathbb P}}
\def\UU{{\mathbb U}}
\def\Vv{{\mathcal P}}
\def\Kk{{\mathcal K}}
\def\FF{{\mathcal F}}
\def\Bb{{\mathcal B}}
\def\TT{{\mathbb T}}
\def\IId{{\mathbb I}^d}
\def\CCd{{\mathbb C}^d}
\def\DDd{{\mathbb D}^d}
\def\NNd{{\mathbb N}^d}
\def\RRd{{\mathbb R}^d}
\def\RRm{{\mathbb R}^m}
\def\TTd{{\mathbb T}^d}
\def\TTm{{\mathbb T}^m}
\def\GGd{{\mathbb G}^d}
\def\ZZd{{\mathbb Z}^d}
\def\ZZm{{\mathbb Z}^m}
\def\IIm{{\mathbb I}^m}
\def\ZZmp{{\mathbb Z}^m_+}
\def\IIi{{\mathbb I}^\infty}
\def\RRi{{\mathbb D}^\infty}
\def\CCi{{\mathbb C}^\infty}
\def\NNi{{\mathbb N}^\infty}
\def\NNo{{\mathbb N}_0}
\def\NNom{{\mathbb N}_0^m}
\def\NMO{{\mathbb N}_0^M}
\def\RRi{{\mathbb R}^\infty}
\def\TTi{{\mathbb T}^\infty}
\def\GGi{{\mathbb G}^\infty}
\def\RRdp{{\mathbb R}^\infty_+}
\def\RRM{{\mathbb R}^M}
\def\RRm{{\mathbb R}^m}
\def\RRip{{\mathbb R}^{\infty}_+}
\def\ZZi{{\mathbb Z}^{\infty}}
\def\ZZi{{\mathbb Z}^{\infty}}
\def\UUi{{\mathbb U}^{\infty}}
\def\ZZis{{\mathbb Z}^{\infty}_*}
\def\ZZip{{\mathbb Z}^\infty_+}
%\def\NNom{{\mathbb Z}^\infty_{+*}}
\def\NNis{{\mathbb N}^\infty_*}
%-----------------------------------
\def\Aa{{\mathcal A}}
\def\Bb{{\mathcal B}}
\def\Cc{{\mathcal C}}
\def\Dd{{\mathcal D}}
\def\Ee{{\mathcal E}}
\def\Ff{{\mathcal F}}
\def\Gg{{\mathcal G}}
\def\Hh{{\mathcal H}}
\def\Ii{{\mathcal I}}
\def\Jj{{\mathcal J}}
\def\Kk{{\mathcal K}}
\def\Ll{{\mathcal L}}
\def\NN{{\mathcal N}}
\def\Oo{{\mathcal O}}
\def\Pp{{\mathcal P}}
\def\Qq{{\mathcal Q}}
\def\Ss{{\mathcal S}}
\def\Tt{{\mathcal T}}
\def\Uu{{\mathcal U}}
\def\Vv{{\mathcal V}}
\def\Ww{{\mathcal W}}
%-------------------------------
\def\II{{\mathbb I}}
\def\CC{{\mathbb C}}
\def\ZZ{{\mathbb Z}}
\def\NN{{\mathbb N}}
\def\RRR{{\mathbb R}}
\def\GG{{\mathbb G}}
\def\FF{{\mathbb F}}
%----------------------------------
\def\TT{{\mathbb T}}
\def\IId{{\mathbb I}^d}
\def\CCd{{\mathbb C}^d}
\def\NNd{{\mathbb N}^d}
\def\RRRd{{\mathbb R}^d}
\def\TTd{{\mathbb T}^d}
\def\GGd{{\mathbb G}^d}
\def\RRRdp{{\mathbb R}^d_+}
\def\RRRi{{\mathbb R}^\infty}
\def\ZZdp{{\mathbb Z}^d_+}
\def\RRRdpe{{\mathbb R}^d_+(e)}
\def\ZZdpe{{\mathbb Z}^d_+(e)}
%--------------------------------------------------------------------------
\def\supp{\operatorname{supp}}
\def\dv{\operatorname{div}}
\def\rev{{\operatorname{ev}}}
% --------------------------------------------------------------------------
\def\LiD{L_\infty(D)}
\def\LiV{L_\infty(\IIi,V,d\mu)}
\def\AAid{{\mathbb A}^\infty_\delta}
\def\AAidB{{\mathbb A}^\infty_{\delta,B}}
\def\UUir{{\mathbb U}^\infty_\rho}
\def\Wi{W^1_\infty(D)}
\def\div{\operatorname{div}}
%%%%%%%%%%%%%%%%%%%%%%%%%%%%%%%%%%%%%%%%

%\newcommand{\norm}[2]{\left\|{#1}\right\|_{#2}}
%\newcommand{\abs}[1]{\left|#1\right|}

%%%%%%%%%%%%%%%%%%%%%%%%%%%%%%%%%%%%%%%%
% --------------------------------------------------------------------------
\def\proof{\noindent{\it Proof}. \ignorespaces}
\def\endproof{\vbox{\hrule height0.6pt\hbox{\vrule height1.3ex% 
width0.6pt\hskip0.8ex\vrule width0.6pt}\hrule height0.6pt}}
% --------------------------------------------------------------------------
% 


\title{\sffamily  Simultaneous spatial-parametric collocation approximation for parametric PDEs with log-normal random inputs} 


	
\author[*]{Dinh D\~ung}
\affil[*]{Information Technology Institute, Vietnam National University, Hanoi
	\protect\\
	144 Xuan Thuy, Cau Giay, Hanoi, Vietnam
	\protect\\
	Email: dinhzung@gmail.com}


\date{\today}
 \tolerance 2500
% --------------------------------------------------------------------------
\begin{document}
\maketitle

\begin{abstract}
	We proved convergence rates of fully discrete multi-level simultaneous   linear collocation approximation of solutions to parametric elliptic PDEs on bounded polygonal domain with log-normal random inputs, based on  a finite number of their values at points in the spatial-parametric domain. These convergence rates  significantly improve the best-known convergence rates of fully discrete collocation approximation and with some logarithm factors coincide with the convergence rates of best $n$-term approximation.
	These results are obtained as consequences of general results on multi-level linear sampling recovery by extended least squares algorithms in abstract Bochner spaces.
	
	\medskip
	\noindent
	{\bf Keywords and Phrases}: High dimensional approximation; Sampling recovery; Bochner spaces; Linear collocation approximation; Least squares approximation; Parametric PDEs with random inputs; Infinite dimensional holomorphic function; Convergence rate.
	
	\medskip
	\noindent
	{\bf Mathematics Subject Classifications (2020)}: 65C30, 65N15, 65N35, 41A25, 41A65. 	
\end{abstract}


\section{Introduction and main results}
\label{Introduction and main results}

Let $D \subset \RRR^2$ be a bounded  polygonal domain (recall that a  polygonal domain in $\RR^2$ is a polygon (which may have
precludes cusps and slits)  with a finite number of straight sides).  Consider the divergence-form diffusion elliptic equation 
\begin{equation} \label{ellip}
	- \dv (a\nabla u)
	\ = \
	f \quad \text{in} \quad D,
	\quad u|_{\partial D} \ = \ 0, 
\end{equation}
for  a given fixed right-hand side $f$ and a 
spatially variable scalar diffusion coefficient $a$.
Denote by $V:= H^1_0(D)$ the energy space and $V' = H^{-1}(D)$ the dual space of $V$. Assume that  $f \in H^{-1}(D)$ and  $a \in L_\infty(D)$ (in what follows this preliminary assumption always holds without mention). If $a$ satisfies the ellipticity assumption
\begin{equation} \nonumber
	0<a_{\min} \leq a \leq a_{\max}<\infty,
\end{equation}
by the well-known Lax-Milgram lemma, there exists a unique weak 
solution $u \in V$  to the equation~\eqref{ellip}  satisfying the variational equation
\begin{equation} \nonumber
	\int_{D} a\nabla u \cdot \nabla v \, \rd \bx
	\ = \
	\langle f , v \rangle,  \quad \forall v \in V.
\end{equation}
For the equation~\eqref{ellip}, 
we consider  the diffusion coefficients having a parametric form $a=a(\by)$, where $\by=(y_j)_{j \in \NN}$
is a sequence of real-valued parameters ranging in the set 
$\RRRi$.
Denote by $u(\by)$ the weak solution to the 
parametric  diffusion divergence-form elliptic equation 
\begin{equation} \label{parametricPDE}
	- {\rm div} (a(\by)\nabla u(\by))
	\ = \
	f \quad \text{in} \quad D,
	\quad u(\by)|_{\partial D} \ = \ 0. 
\end{equation}	
In the present paper, we consider the case when the diffusion coefficient $a$  is of the lognormal form
\begin{equation} \label{lognormal}
	a(\by)=\exp(b(\by)), \quad {\text{with }}\ b(\by)=\sum_{j = 1}^\infty y_j\psi_j,
\end{equation}
and $y_j$ are i.i.d. standard Gaussian random 
variables,
where  $\psi_j \in L_\infty(D)$.



For $r\in\NN_0$ and $\varkappa\in\RR$,  the Kondrat'ev spaces
 $\Kk^{r}_{\varkappa}(D)$ and $\Ww^{r}_\infty(D)$ are defined as  the weighted normed spaces of functions on $D$ equipped with norms
\begin{equation*}
	\|u\|_{ \Kk^r_\varkappa(D)}
	:= \
	\sum_{|\balpha|\leq r}\|\tau_D^{|\balpha|-\varkappa}D^\balpha u\|_{L^2(D)}
	\qquad\text{and}\qquad
	\|u\|_{ \Ww^r_\infty(D)}
	:= \
	\sum_{|\balpha|\leq r}\|\tau_D^{|\balpha|}D^\balpha u\|_{L^\infty(D)},
\end{equation*}
respectively. Here, 
$\tau_{D}:D\to [0,1]$ denotes a fixed smooth function that coincides
with the distance to the nearest corner, in a neighborhood of each
corner, $D^\balpha$ denotes the
weak partial derivative  of order $\balpha\in \NN_0^2$,  and $|\balpha|:= \alpha_1 +\alpha_2$.
The function spaces $\Kk^r_\varkappa(D)$ and
$\Ww^r_\infty(D)$ endowed with these norms are Banach spaces,
and $\Kk^r_\varkappa(D)$ are separable Hilbert spaces.  
An embedding of these spaces is
$ \Kk^1_1(D) \hookrightarrow H^1_0(D)$ \cite[Lemma 1.5]{BNZ2005}, i.e., there exits a positive constant $C$ such that
\begin{equation} \label{EmbeddingInequality}
	\|v\|_{H^1_0(D)}
	\ \le \ 	
	C \|v\|_{\Kk^1_1(D)}, \ \  v \in \Kk^1_1(D).
\end{equation}



%For $s\in \NN_0$ and $\varkappa\in \RR$ we define \index{space!Kondrat'ev $\sim$}
%\begin{equation*} \Kk^s_\varkappa(D): = \big\{ u: D \to
%	\CC: \ r_D^{|\balpha|-\varkappa}D^\balpha u\in L^2(D),
%	|\balpha|\leq s \big\}
%\end{equation*}
%%
%and
%%
%\begin{equation*}
%	\Ww^s_\infty(D):
%	=
%	\big\{u: D\to \CC: \ r_D^{|\balpha|}D^\balpha u\in L^\infty(D),\
%	|\balpha|\leq s \big\}.
%\end{equation*}
%

By \cite[Theorem 3.29]{DNSZ2023}, for $r\ge 2$,
$f\in \Kk^{r-2}_{\varkappa-1}(D)$ and $a\in \Ww^{r-1}_\infty(D)$, the
weak solution to \eqref{ellip} belongs to
$\Kk_{\varkappa+1}^{r}(D)$ provided that with
\begin{equation*}
	\rho(a) := \underset{\bx\in D}{\operatorname{ess\,inf}}\, a(\bx)>0,
	\ \ \ \text{and} \ \ \
	|\varkappa|<\frac{\rho(a)}{\nu \norm{a}{L_\infty(D)}},
\end{equation*}
where $\nu$ is a constant depending on $D$ and $r$.  



In computational uncertainty quantification, the problem of efficient   approximation for  parametric and stochastic PDEs has been of great interest and achieved  significant progress in recent years. Depending on a particular setting,  as usual, this problem  leads to an approximation problem in a Bochner space $L_2(U,X;\mu)$ with an appropriate separable Hilbert space $X$, an infinite-dimensional domain $U$ and a probability measure $\mu$ on $U$, where parametric solutions $u(\by)$, $\by \in U$, to parametric and stochastic PDEs, are treated as elements of  $L_2(U,X;\mu)$ and $U$ the parametric domain.   There is a  vast number of works on this topic to not mention all of them.  We  point out just some works \cite{ABDM2024,ADM2024,BNT2007,BCDC17,BCDM17,BCM17,BD2024,BTNT12,CCS15,CoDe15a,CCS13,CCMNT2015,CDS10,CDS11,Dung19,Dung21,DNSZ2023, EST18,HoSc14,MNST2014,NTW2008,NTW2008a,ZDS19,ZS20} which are directly related to the  problem setting in our paper.  In semi-discrete (intrusive and non-intrusive) approximations, \cite{ADM2024,BNT2007,BCDM17,BCM17,BTNT12,CCS15,CoDe15a,CCS13,CCMNT2015,Dung21,DD-Erratum23,DNSZ2023,EST18,MNST2014,NTW2008,NTW2008a,ZS20} discretizations are processed only with respect to the parametric variables, but not spatial variables, the approximants still belong to the infinite dimensional spatial space and hence  are not useful for practical applications. 
They  should themselves be approximated by spatial discretizations by means of finite elements, wavelets  or spectral Galerkin methods, etc.. The problem of fully discrete (and multilevel)  approximation of parametric PDEs with  random inputs and of relevant infinite-dimensional holomorphic functions has been investigated  in the works
\cite{BCDC17,CDS10,CDS11,Dung19,Dung21,DD-Erratum23,DNSZ2023,HoSc14,ZDS19} based on some spatial approximation properties and parametric summabilities of the GPC expansion coefficients of solutions.  Some bounds for convergence rates of spectral and collocation fully discrete approximation have been proven in these papers.  Observe that by the problem setting a convergence rate of any fully discrete approximation  is not better than the convergence rate given by the spatial approximation properties. Moreover, it was proven that they coincide  in the case of sufficiently small spatial regularity. In the case of higher spatial regularity, the known bounds of convergence rates of fully discrete approximation are less than this regularity and depend essentially on parametric summability properties  of the generalized polynomial chaos (GPC) expansion coefficients of solutions (see the above cited papers for detail). 

A natural question arising is whether the known convergence rates are optimal and whether one can improve them. The aim of the present paper is to improve these known bounds in the case of higher spatial regularity of the solution of the parametric equation \eqref{parametricPDE} with log-normal inputs \eqref{lognormal}.  
More precisely,
we are interested in the problem of  fully discrete  simultaneous spatial-parametric linear  collocation approximation of solutions $u$ to parametric PDEs \eqref{parametricPDE} on bounded polygonal domain $D$ with log-normal random inputs  \eqref{lognormal} and its convergence rate, based on a finite number of its particular spatial-parametric values $u(\bx_1,\by_1),..., u(\bx_n,\by_n)$. The approximation error is measured by the norm of   the Bochner space $L_2(\RRi,V;\gamma)= V \otimes L_2(\RRi;\gamma)$  associated with the energy space $V$ and the infinite standard Gaussian measure $\gamma$ on $\RRi$ (see Subsection~\ref{Extended least squares sampling algorithms in Bochner spaces} for definition of Bochner space).	The present paper can be considered as  a continuation of the works \cite{BD2024, Dung21,DNSZ2023}.  In \cite{Dung21,DNSZ2023}, we have investigated the problem of fully discrete multi-level collocation approximation  of stochastic and parametric elliptic PDEs with log-normal inputs by using sparse-grid GPC Lagrange interpolation at the zeros of Hermite  polynomials.  In the recent paper \cite{BD2024}, we have investigated semi-discrete collocation approximation of these equations by employing extended least squares sampling algorithms in Bochner spaces.  In the present paper, we develop and combine the different techniques used in  these papers for solving the formulated problem.


%This problem is a particular case of the general problem of  linear sampling recovery in normed spaces as follows.
%While the above results allow us to establish algebraic convergence rates
%of the properly truncated series in terms of the number $n$
%of retained coefficients, they do not yet yield practically realizable approximations.
%Indeed, the coefficients $u_\nu\in \{t_\nu,v_\nu,w_\nu\}$ belong to the infinite
%dimensional space $V$ and should themselves be approximated by means
%of spatial discretizations. Such discretizations are typically performed 
%by means of finite elements or wavelets,
%and the number $n_\nu$ of allocated degrees of freedom may vary with
%the retained index $\nu$.
%These results by themselves do not imply practical applications, because they do not cover the approximation of the expansion coefficients which are functions of the spatial variable. 
%The key condition which emerged as governing the
%convergence rates of numerical integration and interpolation methods for a parametric solution $u(\by)$ in $L_2(U,X;\mu)$ is
%a sparsity of the coefficients of its GPC expansion.
%The sparsity is quantified by 
%$\ell_p$-summability or weighted $\ell_2$-summability of
%these coefficients which are appropriate for non-linear $n$-term approximation or linear approximation, respectively. 
%
%The problem of adaptive nonlinear collocation approximation was investigated in \cite{CCS15,CoDe15a,ABDM2024,CCMNT2015,CCS13},  and of  non-adaptive  linear collocation approximation in 
%\cite{BNT2007,Dung19,Dung21,DD-Erratum23, EST18,MNST2014,NTW2008,NTW2008a,DNSZ2023,ZDS19,ZS20}. 
%The last problem naturally leads to the problem of linear sampling recovery in a Bochner space $L_2(U,X;\mu)$. Let us formulate a setting of this problem which will cover the linear collocation approximation problem for a wide class of parametric PDEs with random inputs as well as of infinite dimensional holomorphic functions.

%Let  $E$ be  a normed space of functions on $\Omega$. Given sample points $\bx_1,\ldots,\bx_n \in \Omega$, we consider the approximate recovery  of a continuous function $f$ on $\Omega$  from their values $f(\bx_1),\ldots, f(\bx_n)$ by a linear sampling algorithm $S_n$
%on $\Omega$ of the form
%\begin{equation} \label{S_k}
%	S_n(f): = \sum_{i=1}^n  f(\bx_i) h_i, 
%\end{equation}
%where  $h_1,\ldots,h_n$ are given  functions on $\Omega$.  For convenience, we assume that some of the sample points $\bx_i$ may coincide. The approximation error is measured by the norm 
%$\|f - S_n(f)\|_E$.  
%Denote by $\Ss_n$  the family of all linear sampling algorithms $S_k$ of the form \eqref{S_k} with $k \le n$.
%Let $F \subset E$ be a set of  functions on $\Omega$.   To study the optimality  of linear  sampling algorithms  from $\Ss_n$ for  $F$ and their convergence rates we use  the  (linear) sampling $n$-width
%\begin{equation} \label{rho_n}
%	\varrho_n(F, E) :=\inf_{S_n \in \Ss_n} \ \sup_{f\in F} 
%	\|f - S_n (f)\|_E.%{L_{q,w}(\RRd)}.
%\end{equation}

We give a short description of main contribution of the present paper with some comments.

Let	
$r\in\NN$, $r \ge  2$, $f\in \Kk_{\varkappa-1}^{r-2}(D)$ and
$\psi_k\in W^{1}_\infty(D)\cap\Ww_{\infty}^{r-1}(D)$, $k \in \NN$. 
For $i=1,2$, let the sequences $\bb_i:=(b_{i,j})_{j\in\NN}$ be defined by
\begin{equation*}
	b_{1,j}:=\norm{\psi_j}{L^\infty},\ \ \text{and} \ \
	b_{2,j}:=\max\big\{\norm{\psi_j}{W^{1}_\infty(D)},\norm{\psi_j}{\Ww_\infty^{r-1}(D)}\big\},
\end{equation*}
 and 
satisfy the condition 
$\bb_i\in\ell^{p_i}(\NN)$, respectively, with $0< p_1\le  p_2 < 1$ and $p_1 < 2/3$. 
	Let the numbers $\alpha$ and  $\beta$ be defined by
		\begin{equation*} 
		\alpha:= \frac{r-1}{2}, \quad	\beta := \brac{\frac 1 {p_1} - 1} \frac{\alpha}{\alpha + \delta}, \quad 
		\delta := \frac 1 {p_1} - \frac 1 {p_2}.
	\end{equation*}	
%Let $L_2(\RRi,V;\gamma)= H^1_0(D) \otimes L_2(\RRi;\gamma)$ be the Bochner space associated with the energy space $V$ and the infinite standard Gaussian measure $\gamma$ on $\RRi$ (see Subsection \ref{Extended least squares sampling algorithms in Bochner spaces} for definition of Bochner space).
	Then for all $n \ge 2$ there exist  points $(\bx_1,\by_1),...,(\bx_n,\by_n) \in D \times \RRi$ and functions 
	$\varphi_1,...,\varphi_n \in V$ and  $h_1,...,h_n \in L_2(\RRi,\RR;\gamma)$  such that for the linear sampling algorithm $S_n$ on the spatial-parametric domain $D\times \RRi$ defined  by
	\begin{equation*}
		S_n(v)(\bx,\by): = \sum_{i=1}^n  v(\bx_i,\by_i) \varphi_i (\bx) h_i(\by),  
		\ \ \bx \in D, \ \ \by \in \RRi,
	\end{equation*}	
	and for the parametric solution $u$ to the equation \eqref{parametricPDE} with log-normal random inputs \eqref{lognormal},
	it holds true the error bounds
	 \begin{equation} 	\label{main-result}
		\begin{split}
			\big\|u-S_n u\big\|_{L_2(\RRi,V;\gamma)} 
			\ \le \
			\begin{cases}
				C n^{-\alpha}\quad &{\rm if }  \ \alpha \le 1/p_2 - 3/2, \\
				C_\varepsilon	n^{-\alpha}(\log n )^{\alpha + 1 +\varepsilon} 
				\quad &{\rm if }  \ 1/p_2 - 3/2 <  \alpha < 1/p_2 - 1,\\
			C_\varepsilon	n^{-\alpha}(\log n )^{\alpha + 2 +\varepsilon}\quad &{\rm if }  \ \alpha = 1/p_2 -1,  \\
			C n^{-\beta}(\log n )^{\beta + \alpha/(\alpha - \beta)}\ \ \quad  &{\rm if }  \ \alpha > 1/p_2 - 1, 
			\end{cases}
		\end{split}
	\end{equation}
	for  an arbitrarily   small number  $ \varepsilon > 0$,	
	where  the constants $C$ and $C_\varepsilon$ are independent of $u$ and $n$.
	
	The convergence rate $n^{-\alpha}$ in \eqref{main-result} in the case of small spatial regularity $\alpha \le 1/p_2 - 3/2$ is as expected.  
	The convergence rates in \eqref{main-result} in the cases of higher spatial regularity $\alpha > 1/p_2 - 3/2$  significantly improve the best-known convergence rates of fully discrete collocation approximation of $u$ which is $n^{-\beta'}$ where $\beta':= \beta - \frac{1}{2}\frac{\alpha}{\alpha + \delta}$ (cf. \cite{Dung21,DD-Erratum23,DNSZ2023}).
	Moreover, in the particular intermediate case $1/p_2 - 3/2< \alpha \le 1/p_2 -1$, the convergence rate $n^{-\alpha}$ still holds with a logarithm factor. Notice also that  with some logarithm factors, the convergence rates in \eqref{main-result} coincide with the convergence rates $n^{-\min(\alpha,\beta)}$ of  fully discrete best $n$-term approximation of the parametric solution $u$ in the space  $L_2(\RRi,V;\gamma)$ 
	(cf. \cite{BCDC17,CDS10,CDS11,Dung19,Dung21,DD-Erratum23,DNSZ2023,HoSc14,ZDS19}).
	
Notice that $S_n$ is a fully discrete multi-level collocation  approximation method.	At each level  in $S_n$, the spatial component  is based on finite element Lagrange interpolation associated with triangulations of the spatial domain $D$, while  the parametric component is based on Hermite-Lagrange GPC interpolation in the case $\alpha \le 1/p_2 - 3/2$, and on  extended least squares sampling algorithms in the cases $\alpha > 1/p_2 - 3/2$. 

 Following the approach in \cite{BD2024, Dung21,DNSZ2023}, the convergence rate results in \eqref{main-result} and constructions of the linear sampling algorithms  $S_n$ realizing them, are obtained as consequences of general results on multi-level linear sampling recovery in abstract Bochner spaces. In the case of small spatial regularity $\alpha \le 1/p_2 - 3/2$, we use a modification of methods of  sparse-grid Hermite-Lagrange GPC interpolation in Bochner spaces in \cite{Dung21}. In the cases of higher spatial regularity $\alpha > 1/p_2 - 3/2$, we used
extended least squares sampling algorithms in Bochner spaces and  some recent  results on linear sampling recovery of functions in reproducing kernel Hilbert spaces \cite{BSU23, DKU2023,KUV21,KU21a}. We would like to emphasize that the general theory of sampling recovery in abstract Bochner spaces presented  in this paper as well as  \cite{BD2024, Dung21,DNSZ2023} is applicable to uncertainty quantification for a wide class of parametric and stochastic PDEs.

This  paper is organized as follows. 

In Section \ref{Sampling recovery in Bochner spaces}, we investigate   multi-level linear sampling recovery  of functions in an abstract Bochner space $L_2(U,X;\mu)$ for a Hilbert space $X$ and a probability measure space $(U,\Sigma,\mu)$, based on some weighted $\ell_2$-summability of their coefficients of an orthonormal expansion, and some approximation properties in the space $X$. The received results are then applied to $L_2(\RRi,X;\gamma)$, the Bochner space associated with a Hilbert space $X$ and the infinite standard Gaussian measure $\gamma$ on $\RRi$. 
In Section \ref{Applications to  holomorphic functions}, we apply the results on  multi-level linear sampling recovery in the space $L_2(\RRi,X;\gamma)$ in the previous section to $(\bb,\xi,\delta,X)$-holomorphic functions on $\RRi$.
In Section \ref{Multi-level sparse-grid interpolation algorithms},
we construct fully discrete multi-level sparse-grid  sampling algorithms
for  $(\bb,\xi,\delta,X)$-holomorphic functions, based on GPC Lagrange-Hermite interpolation, and prove convergence rates of the approximation by them. 
In Section \ref{Applications to  parametric PDEs}, 
 we prove that the parametric solution $u$ to
the parametric elliptic PDEs \eqref{ellip} on bounded polygonal domain with log-normal inputs \eqref{lognormal} is a $(\bb_j,\xi,\delta,V)$-holomorphic function. This allows prove convergence rates of 
a fully discrete multi-level collocation  algorithm by applying the results in Sections \ref{Applications to  holomorphic functions} and \ref{Multi-level sparse-grid interpolation algorithms}.
In Section \ref{Extensions},
we  discuss various  least squares sampling algorithms  for functions in the reproducing kernel Hilbert space, and inequalities between sampling $n$-widths and Kolmogorov $n$-widths of the unit ball of this space, and how to apply these inequalities to obtain corresponding convergence rates of  multi-level linear sampling recovery in abstract Bochner spaces and  of fully discrete multi-level collocation  approximation of  the parametric solution $u$ to
the parametric elliptic PDEs \eqref{ellip} on bounded polygonal domain with log-normal inputs \eqref{lognormal}.



\medskip
\noindent
{\bf Notation} \  As usual, $\NN$ denotes the natural numbers, $\ZZ$  the integers, $\RR$ the real numbers, $\CC$ the complex numbers,  and 
$ \NN_0:= \{s \in \ZZ: s \ge 0 \}$.
We denote  $\RR^\infty$ the
set of all sequences $\by = (y_j)_{j\in \NN}$ with $y_j\in \RR$.
 Denote by $\FF$  the set of all sequences of non-negative integers $\bs=(s_j)_{j \in \NN}$ such that their support $\supp (\bs):= \{j \in \NN: s_j >0\}$ is a finite set.  
If  $\ba= (a_j)_{j \in \Jj}$  is a set of positive numbers with  any index set $\Jj$, then we use the notation 
$\ba^{-1}:= (a_j^{-1})_{j \in \Jj}$.
We use letters $C$  and $K$ to denote general 
positive constants which may take different values, and $C_{a,b,...,d}$  and $K_{a,b,...,d}$ constants depending on $a,b,...,d$.
 For the quantities $A_n(f,\bk)$ and $B_n(f,\bk)$ depending on 
 $n \in \NN$, $f \in W$, $\bk \in \ZZd$,  
 we write  $A_n(f,\bk) \ll B_n(f,\bk)$, $f \in W$, $\bk \in \ZZd$ ($n \in \NN$ is specially dropped),  
 if there exists some constant $C >0$ such that 
 $A_n(f,\bk) \le CB_n(f,\bk)$ for all $n \in \NN$,  $f \in W$, $\bk \in \ZZd$ (the notation $A_n(f,\bk) \gg B_n(f,\bk)$ has the obvious opposite meaning), and  
 $A_n(f,\bk) \asymp B_n(f,\bk)$ if $A_n(f,\bk) \ll B_n(f,\bk)$
 and $B_n(f,\bk) \ll A_n(f,\bk)$.  Denote by $|G|$ the cardinality of the set $G$.
 
 
%  \section{Linear approximations and $n$-widths  in Bochner spaces}
% \label{Linear approximation  in Bochner spaces}

\section{Sampling recovery in Bochner spaces}
\label{Sampling recovery in Bochner spaces}	
 
In this section, we investigate    multi-level linear sampling recovery  of functions in  abstract Bochner spaces $L_2(U,X;\mu)$ for a Hilbert space $X$ and a probability measure space $(U,\Sigma,\mu)$, based on some weighted $\ell_2$-summability of their coefficients of an orthonormal expansion, and some approximation properties in the space $X$. We develop  the methods used in \cite{BD2024, Dung21,DNSZ2023} to construct   fully discrete (multi-level) sampling algorithms by using the approximation properties in the space $X$ combined with extended least squares sampling algorithms at each level, and prove convergence rates of approximation by them.  The received results are then applied to $L_2(\RRi,X;\gamma)$, the Bochner space associated with a Hilbert space $X$ and the infinite standard Gaussian measure $\gamma$ on $\RRi$.
%The optimality is studied in terms of sampling widths.
 

 \subsection{Function spaces}
 \label{Function spaces}
 Let $(U,\Aa,\mu)$ be a probability measure space with $\Aa$ being countably generated and let $X$  be a complex separable Hilbert space.   Denote by $L_2(U,X;\mu)$  the Bochner space  of  strongly $\mu$-measurable mappings $v$ from $U$ to $X$, equipped with the norm
 \begin{equation} \label{BochnerSpaceNorm}
 	\|v\|_{L_2(U,X;\mu)}
 	:= \
 	\left(\int_{U} \|v(\by)\|_X^2 \, \rd \mu(\by) \right)^{1/2}.
 \end{equation}
 
 Notice that  because $\Aa$ is countably generated, $L_2(U,\CC;\mu)$ is separable by \cite[Proposition~3.4.5]{Cohn13}.
 Hence $L_2(U,X;\mu)$ is a separable complex Hilbert space and, moreover,  
 $$
 L_2(U,X;\mu) = L_2(U,\CC;\mu) \otimes X.
 $$
 
 
 Let $\brac{\varphi_s}_{s \in \NN}$ be a fixed orthonormal basis of  $L_2(U,\CC;\mu)$.  Then a function $v \in L_2(U,X;\mu)$  can be represented  by the expansion
 \begin{equation} \label{series}
 	v(\by)=\sum_{s \in \NN} v_s \,\varphi_s(\by), \quad v_s \in X,
 \end{equation}
 with the series convergence in $L_2(U,X;\mu)$,	
 where
 \begin{equation*}
 	v_s:=\int_U v(\by)\,\overline{\varphi_s(\by)}\, \rd\mu (\by), \quad s \in \NN.
 	\label{hermite}
 \end{equation*}
 Moreover, for every $v \in L_2(U,X;\mu)$  represented by the 
 series \eqref{series},  Parseval's identity holds
 \begin{equation} \nonumber
 	\|v\|_{L_2(U,X; \mu)}^2
 	\ = \ \sum_{s\in \NN} \|v_s\|_X^2.
 \end{equation}
 
 
 %For a function $v$ defined on $U$, we make use of the convention $v \in L_2(U,X;\mu)$ if $v$ is a representative of  an element of $L_2(U,X;\mu)$. 
 Let $\bsigma=\brac{\sigma_{s}}_{s \in \NN}$ be  a non-decreasing sequence of positive numbers strictly larger than $1$ such that 
 $\bsigma^{-1}:=\brac{\sigma_{s}^{-1}}_{s \in \NN} \in \ell_2(\NN)$. 
 For given $U$ and $\mu$, denote by $H_{X,\bsigma}$ the linear subspace of all functions 
 $v \in L_2(U,X;\mu)$ such that the norm
 \begin{equation} \nonumber
 	\|v\|_{H_{X,\bsigma}}
 	:= \
 	\brac{\sum_{s \in \NN} \brac{\sigma_{s} \|v_s\|_X}^2}^{1/2} < \infty.		
 \end{equation}
 In particular, the space $H_{\CC,\bsigma}$ is the linear subspace in $L_2(U,\CC;\mu)$ equipped with its own inner product
 \begin{equation} \nonumber
 	\langle f,g \rangle_{H_{\CC,\bsigma}}
 	:= \
 	\sum_{s \in \NN} \sigma_{s}^2 
 	\langle f,\varphi_s \rangle_{L_2(U,\CC;\mu)}
 	\overline{\langle g,\varphi_s \rangle_{L_2(U,\CC;\mu)}}.
 \end{equation}
 The space $H_{\CC,\bsigma}$ is a reproducing kernel Hilbert space with the reproducing kernel
 \begin{equation} \nonumber
 	K(\cdot,\by)
 	:= \
 	\sum_{s \in \NN} \sigma_{s}^{-2} \varphi_s(\cdot)\overline{\varphi_s(\by)}
 \end{equation}
 having the eigenfunctions $\brac{\varphi_s}_{s \in \NN}$ and the eigenvalues 
 $\brac{\sigma_{1,s}^{-1}}_{s \in \NN}$. Moreover, $K(\bx,\by)$ satisfies the finite trace assumption
 \begin{equation*}\label{trace assumption}
 	\int_U K(\by,\by) \rd \mu(\by) \ < \ \infty.
 \end{equation*}
 %
 %The aims of the present paper is to investigate the approximate recovery  of functions in the space  $H_{X,\bsigma}$ with $\bsigma^{-1} \in \ell_q(\NN)$ for some $0<q<2$ from a finite number of their sample values. We would like to establish convergence rates of the sampling recovery by extensions of several least squares methods which are different with respect to their constructiveness. Obtained results will be applied to linear collocation approximation for parametric PDEs with log-normal or affine inputs as well as for infinite dimensional holomorphic functions.
 
 Let $(V_k)_{k \in \NN_0}$ be a given sequence of subspaces $V_k \subset X$ of dimension $2^k$, and $(P_k)_{k \in \NN_0}$  a given sequence of uniformly bounded linear projectors  $P_k$ from $X$ onto $V_k$. 
We extend the operators $P_k$ (which with an abuse is denoted again by $P_k$) to $L_2(U,X;\mu)$ by the formula
 $$
 (P_k v)(\by):= P_k (v(\by)).
 $$
For $i=1,2$, let   $0< q_i \le 2$ and $\bsigma_i:=(\sigma_{i;s})_{s \in \NN}$  be given sequences of numbers strictly larger than $1$, such that $\bsigma_i^{-1}:=\brac{\sigma_{i,s}^{-1}}_{s \in \NN} \in \ell_{q_i}(\NN)$. 
For  a given number $\alpha> 0$, denote by $ H_X^{\alpha}$ the linear subspace in $L_2(U,X;\mu)$ of all $v$ such that the norm
 \begin{equation} \label{H_{X}^{alpha}}
 	\|v\|_{H_{X}^{\alpha}}
 	:= \
 	\brac{\|v\|_{H_{X,\bsigma_1}}^2 + \|v\|_{H_{X,\bsigma_2}^{\alpha}}^2}^{1/2} < \infty, 
 \end{equation}
 where the semi-norm
 $\norm{v}{H_{X,\bsigma_2}^{\alpha}}$ is defined by 
 \begin{equation} \label{H_{X}^{alpha,tau}}
 	\|v\|_{H_{X,\bsigma_2}^{\alpha}}
 	:= \
 	\sup_{k \in \NN_0}	2^{\alpha k} \norm{v - P_k v} {H_{X,\bsigma_2}}.
 \end{equation}
 
 Similarly, for given numbers $\alpha> 0$ and $\tau > 0$, denote by $ H_X^{\alpha,\tau}$ the linear subspace in $L_2(U,X;\mu)$ of all $v$ such that the norm
 \begin{equation} \label{H_{X}^{alpha,tau}}
 	\|v\|_{H_{X}^{\alpha,\tau}}
 	:= \
 	\brac{\|v\|_{H_{X,\bsigma_1}}^2 + \|v\|_{H_{X,\bsigma_2}^{\alpha,\tau}}^2}^{1/2} < \infty, 
 \end{equation}
 where the semi-norm
 $\norm{v}{H_{X,\bsigma_2}^{\alpha,\tau}}$ is defined by 
 \begin{equation} \label{H_{X}^{alpha,tau}semi-norm}
 	\|v\|_{H_{X,\bsigma_2}^{\alpha,\tau}}
 	:= \
 	\brac{\sum_{k \in \NN_0}	
 		\brac{2^{\alpha k}k^{-\tau}  \norm{v - P_k v} {H_{X,\bsigma_2}}}^2}^{1/2} < \infty.
 \end{equation}
 
Later on we will see that under certain assumptions the parametric solution $u$ to the parametric elliptic PDE \eqref{parametricPDE} with log-normal inputs \eqref{lognormal} belongs to a certain space   $H_{X}^{\alpha}$.  However, such a space has not Hilbertian structure and hence  is not appropriate for constructing least squares sampling algorithms. Due to the continuous embedding 
$H_{X}^{\alpha} \hookrightarrow H_X^{\alpha,\tau}$ for any $\tau > 1$, the parametric solution $u$ can be treated as an element of  the  Hilbert space $H_X^{\alpha,\tau}$ for which we are able to construct least squares sampling algorithms.
 
% 
% Given numbers $\alpha> 0$ and $\tau > 1$, denote by $H_{X,\bsigma_2}^{\alpha,\tau}$ the linear subspace in $L_2(U,X;\mu)$ of all $v$ such that the norm
% \begin{equation} \label{H_{X}^{alpha,tau}}
% 	\|v\|_{H_{X,\bsigma_2}^{\alpha,\tau}}
% 	:= \
% 	\brac{\sum_{k \in \NN_0}	
% 		\brac{2^{\alpha k}k^{-\tau}  \norm{v - P_k v} {H_{X,\bsigma_2}}}^2}^{1/2} < \infty.
% \end{equation}
%
% We denote $H_X:= H_X^{\alpha,\tau}: = H_{X,\bsigma_1} \cap H_{X,\bsigma_2}^{\alpha,\tau}$ the linear subspace in $L_2(U,X;\mu)$ of all $v$ such that the norm
% \begin{equation} \label{H_{X}}
% 	\|v\|_{H_{X}^{\alpha,\tau}}
% 	:= \
% 	\brac{\|v\|_{H_{X,\bsigma_1}}^2 + \|v\|_{H_{X,\bsigma_2}^{\alpha,\tau}}^2}^{1/2} < \infty, 
% \end{equation}
% 
% We define the space $H_{X}^{\alpha}: = H_{X,\bsigma_1} \cap H_{X,\bsigma_2}^{\alpha}$  in a similar way of  by replacing in \eqref{H_{X}^{alpha,tau}} the norm of $H_{X,\bsigma_2}^{\alpha,\tau}$ with the norm 
% $H_{X,\bsigma_2}^{\alpha}$ given by
% \begin{equation} \nonumber
% 	\|v\|_{H_{X,\bsigma_2}^{\alpha}}
% 	:= \
% 	\sup_{k \in \NN_0}	2^{\alpha k} \norm{v - P_k v} {H_{X,\bsigma_2}}
% 	< \infty. 
% \end{equation}
We will need the following lemma.

 \begin{lemma}\label{lemma:norms-equality}
 	We have 
 	\begin{equation*} %\label{norms-equality1}
 		\|v- P_k v\|_{H_{X,\bsigma_2}}
 		\ = \
 		\brac{	\sum_{s\in\NN} 
 			\brac{\sigma_{2;s} \|v_s - P_k v_s\|_{X}}^2}^{1/2}, \ \ 	v \in H_{X,\bsigma_2}.
 	\end{equation*}
 \end{lemma}
 
 \begin{proof} By the definition,
 	 \begin{equation} \nonumber
 		\|v- P_k v\|_{H_{X,\bsigma_2}}
 		:= \
 		\brac{\sum_{s \in \NN} \brac{\sigma_{s} \|(v- P_k v)_s\|_X}^2}^{1/2} < \infty.		
 	\end{equation}
 Due to the equality 	$(v - P_k v)_s= v_s - (P_k v)_s$, to prove the lemma it is sufficient to show $(P_k v)_s= P_k v_s$.
 	Since $\bsigma_2^{-1}\in \ell_2(\NN)$, then the series \eqref{series} converges unconditionally  in  $L_2(U,X;\mu)$  to $v$ for $v \in H_{X,\bsigma_2}$.
Indeed, by the H\"older inequality we have for $v \in H_{X,\bsigma_2}$, 
 	\begin{equation} \nonumber
 		\begin{split}
 			\sum_{s\in \NN} \|v_s  \varphi_s\|_{L_2(U,X;\mu)}
 			\ &= \
 			\sum_{s\in\NN} \|v_s\|_{X} \|\varphi_s\|_{L_2(U, \CC;\mu)}
 			\ = \
 			\sum_{s\in\NN} \|v_s\|_{X}
 			\\
 			&\leq 
 			\bigg( \sum_{s\in \NN} (\sigma_s \|v_s\|_{X})^2\bigg)^{1/2} 
 			\bigg(\sum_{s\in \NN} \sigma_s^{-2}\bigg)^{1/2} \ < \ \infty.		
 		\end{split}
 	\end{equation}
 	This yields that the series \eqref{series} absolutely and hence unconditionally converges in $L_2(U, \CC;\mu)$ to $v$, since in a Banach space the absolute convergence implies the unconditional convergence. From this unconditional convergence  and the uniform boundedness of the linear projectors $(P_k)_{k \in \NN_0}$ we derive
	 \begin{equation} \nonumber
	P_k v
	\ = \
	\sum_{s \in \NN} P_k v_s \varphi_s,
\end{equation}
and therefore, 	$(P_k v)_s= P_k v_s$.
 	\hfill
 \end{proof}
 
 
 % 
% For a  finite subset $G$ in $ \NN_0 \times \NN$, 
% denote by $\Vv(G)$ the subspace in $L_2(U,X;\mu)$ of  all functions $v$
% of the form
% \begin{equation} \nonumber
% 	v
% 	\ = \
% 	\sum_{(k,s) \in G} v_k \, H_s, \quad v_k \in V_{k}.
% \end{equation}
% %Let Assumption A hold for the Hilbert spaces $X$  and  $v \in L_2(U,X;\mu)$. 
% We define the linear operator $\Ss_G: \, L_2(U,X;\mu)\to \Vv(G)$ by
% \[
% \Ss_G v
% := \
% \sum_{(k,s) \in G} (\delta_k v)_s\, H_s.
% \]
% 
% 
% \begin{theorem}\label{thm[L_2-approx]}
% 	Define for $\xi>0$
% 	\begin{equation} \label{G(xi)G}
% 		G(\xi)
% 		:= \ 
% 		\begin{cases}
% 			\big\{(k,s) \in \NN_0 \times\NN: \, \sigma_{2;s} \leq \xi^{1/q_2} 2^{-k/q_2} \big\} \quad &
% 			{\rm if }  \ \alpha \le 1/q_2;\\			
% 			\big\{(k,s) \in \NN_0 \times\NN: \, 
% 			\sigma_{2;s} \leq \xi^{1/q_1}2^{-\alpha k}, \ \sigma_{1;s}\le \xi^{1/q_1} \big\} \quad  & 
% 			{\rm if }  \ \alpha > 1/q_2.
% 		\end{cases}
% 	\end{equation}
% Let
% 	\begin{equation} 	\label{[beta]2}
% 		\beta := \frac 1 {q_1}\frac{\alpha}{\alpha + \delta}, \quad 
% 		\delta := \frac 1 {q_1} - \frac 1 {q_2}.
% 	\end{equation}
% 	Then for all $n \ge 2$ there exists a number $\xi_n$  such that   $\dim(\Vv(G(\xi_n)) \le n$, and
% 	\begin{equation} 	\label{|v-Ss_{G(xi_n)}v|<}
% 			\sup_{v\in B_{X}^\alpha}		\|v-\Ss_{G(\xi_n)}v\|_{L_2(U,X;\mu)} 
% 			\ \le \
% 			C
% 			\begin{cases}
% 				n^{-\alpha}\quad & \text{if }  \ \alpha \le 1/q_2,\\
% 				n^{-\beta}\quad  & \text{ if }  \ \alpha > 1/q_2. 
% 			\end{cases}
% 	\end{equation}
% 	Moreover,   if $\alpha \le 1/q_2$,
%	\begin{equation} 	\label{d_n1}
%		d_n(B_{X}^\alpha,L_2(U,X;\mu)) 
%	\ \asymp \
%			n^{-\alpha},		
%	\end{equation} 
%	and   if $\alpha \le 1/q_2$ and the sequence $(V_k)_{k \in \NN_0}$ is nested, for any small $\varepsilon > 0$,
%	\begin{equation} 	\label{d_n2}
%	C_\varepsilon	 n^{-\beta}(\log n)^{-\beta (1 + 1/(\alpha q_1)) - \varepsilon}
%\ \le \		
%\sup_{\|\bsigma_i^{-1}\|_{\ell_{q_i}} \le 1, \ i =1,2} d_n(B_{X}^\alpha,L_2(U,X;\mu))	
%\	\le	\ 
%C n^{-\beta}.
%\end{equation} 
%\end{theorem}
% 
% \begin{proof} The proof of the upper bounds \eqref{|v-Ss_{G(xi_n)}v|<} is   a modification of   the proof of \cite[Theorem 2.3]{Dung21}, we omit it.  From \eqref{|v-Ss_{G(xi_n)}v|<} we derive the upper bounds in  \eqref{d_n1} and \eqref{d_n2}. Let us prove the lower bounds in \eqref{d_n1} and \eqref{d_n2}. 
% 	
% 	We will need the following equality proven by Tikhomirov  \cite[Theorem 1]{Tikh1960}. 
% 		If $E$ is a Banach space and $B$ the ball of radius $\rho >0$ in a linear $n+1$-dimensional subspace of $E$, then 
% 		\begin{equation} \label{TikhomirovEq}
% 		d_n(B,E)=\rho.
% \end{equation}
% 	
% \noindent
% 		{\bf The case $\alpha > 1/q_2$. } 
% 	For $i=1,2$ and any fixed $\varepsilon > 1/q_1$, we take  $\bsigma_{i,s}=(\sigma_{i,s})_{s \in \NN}$ defined by
% \begin{equation} \label{bsigma_i}
% 	\sigma_{i,s}= c_i s^{1/q_i} (\log (s+1))^{\varepsilon}.
% 	 \end{equation}
% 	 Since $\bsigma_{i,s}^{-1} \in \ell_{q_i}(\NN)$,  $i=1,2$, we can take $c_i >0$ satisfying the condition 
% 	 $\norm{\bsigma_{i,s}^{-1}}{\ell_{q_i}(\NN)} = 1$.  For $\xi > 1$, consider the subspace in $L_2(U,X;\mu)$
% $$
%B(\xi):= \brab{u = \sum_{(k,s) \in G(\xi)} (P_k v_s) \varphi_s: \ \norm{v}{L_2(U,X;\mu)} \le 1},
% $$
% where $G(\xi)$ is the set defined as in \eqref{G(xi)G} for the case $\alpha > 1/q_2$. Let us estimate the norm 
% $\norm{u}{H_X^\alpha}$ for $u \in B(\xi)$ represented as inside the brackets. From this unconditional convergence  and the uniform boundedness of the linear projectors $(P_k)_{k \in \NN_0}$ we derive 
% \begin{equation} \label{|u|}
%	\begin{split}
%		 \|u\|_{H_{X,\bsigma_1}}^2 
% 	&= \ 
% \sum_{\sigma_{1;s}\le  \xi^{1/q_1} } (\sigma_{1;s}\|P_k v_s\|_{X})^2 
% 	\ \le \ 
%C \sum_{\sigma_{1;s}\le  \xi^{1/q_1} } (\sigma_{1;s}\|v_s\|_{X})^2
%\\
%& 
% 	\ \le \ 
% C \xi^{1/q_1}\sum_{s \in \NN} (\|v_s\|_{X})^2 \ \le \ C \xi^{1/q_1}.
%	\end{split}
%	 \end{equation}
%	We have for arbitrary $k' \in \NN_0$, 
%	\begin{equation} \nonumber
%		\begin{split}
%			2^{\alpha 2k'} \norm{u - P_{k'} u} {H_{X,\bsigma_2}}^2		
%			&:= \
%		2^{\alpha 2k'} \brac{\sum_{s\in\NN} 
%			\sigma_{2;s} \|u_s - P_{k'} u_s\|_{X}}^2
%			\\&
%			\ = \ 
%				2^{2\alpha k'}	\sum_{\sigma_{1;s}\le \xi^{1/q_1}} 	
%				\brac{\sigma_{12;s} \norm{
%			\sum_{2^{\alpha k}\sigma_{2;s} \leq \xi^{1/q_1}}
%		P_k v_s - P_{k'}(P_k v_s)}{X}}^2 
%		\\&
%	\ = \ 
%	2^{2\alpha k'}	\sum_{\sigma_{1;s}\le \xi^{1/q_1}} 	\brac{\sigma_{2;s} 
%		\norm{
%			\sum_{2^{\alpha k}\sigma_{2;s} \leq \xi^{1/q_1}, \ k > k'}
%			P_k v_s - P_{k'}(P_k v_s)}{X}}^2 
%			\\&
%		\ \le \ C^2
%		2^{2\alpha k'}	\sum_{\sigma_{1;s}\le \xi^{1/q_1}} 	\brac{\sigma_{2;s} 			
%				\sum_{2^{\alpha k}\sigma_{2;s} \leq \xi^{1/q_1}, \ k > k'}
%				\norm{ v_s}{X}}^2 
%				\\&
%			\ \le \ C^2
%			2^{2\alpha k'}	\sum_{\sigma_{1;s}\le \xi^{1/q_1}} 	\brac{\sigma_{2;s} 			
%				\sum_{2^{\alpha k}\sigma_{2;s} \leq \xi^{1/q_1}, \ k > k'} 
%				2^{-\alpha k}\sigma_{2;s}^{-1} \xi^{1/q_1}
%				\norm{ v_s}{X}}^2 
%					\\&
%				\ \le \ C^2 \xi^{2/q_1}
%					\sum_{\sigma_{1;s}\le \xi^{1/q_1}} 		\norm{ v_s}{X}^2\brac{			
%					\sum_{k > k'} 
%					2^{-\alpha (k - k')}}^2 
%			\\&
%				\ \le \ 
%			C^2 \sum_{\sigma_{1;s}\le  \xi^{1/q_1} } \|v_s\|_{X}^2 
%			\ \le \
%			C^2 \xi^{2/q_1}
%			\norm{v}{L_2(U,X;\mu)}^2 \le C^2 \xi^{2/q_1}.
%		\end{split}
%	\end{equation}
%(In the third step we  used the the equality $P_k v_s - P_{k'}(P_k v_s) = 0$  for $k \le k'$ as  by the assumption the sequence $(V_k)_{k \in \NN_0}$ is nested.) This proves	that
%  \begin{equation} \nonumber
% 		\|u\|_{H_{X,\bsigma_2}^{\alpha}} \ \le \
% 		C\xi^{1/q_1},
% \end{equation}
% which together with \eqref{|u|} implies
%   \begin{equation} \label{|u|<}
% 	\|u\|_{H_{X}^{\alpha}} \ \le \
% 	C\xi^{1/q_1}, \ \ u \in B(\xi).
% \end{equation}
% We estimate now from bellow the dimension of the  set $B(\xi)$ for $\xi > 1$. 
%With $\bsigma_i$, $i=1,2$, as in \eqref{bsigma_i},  by the definition we have for any $\varepsilon > 1/q_1$,
% \begin{equation} \nonumber
% 	\begin{split}
% 	\dim B(\xi) = |B(\xi)|
% 		\ &= \
% \sum_{\sigma_{1;s}\le \xi^{1/q_1}}				
% \sum_{2^{\alpha k}\sigma_{2;s} \leq \xi^{1/q_1}} 2^k			
% 		\ =  \  
% 	\sum_{\sigma_{1;s}\le \xi^{1/q_1}}				
% 	\sum_{2^{k}\leq \brac{\xi^{1/q_1}\sigma_{2;s}^{-1}}^{1/\alpha}} 2^k
% 		\\&
% 		\ \ge \ C
% 		\sum_{\sigma_{1;s}\le \xi^{1/q_1}} \brac{\xi^{1/q_1}\sigma_{2;s}^{-1}}^{1/\alpha}
% 			\ \ge \ C \xi^{1/(\alpha q_1)}
% 		\sum_{\sigma_{1;s}\le \xi^{1/q_1}} \sigma_{2;s}^{-1/\alpha}
% 		\\&
% 	\ \ge \ C \xi^{1/(\alpha q_1)}
% 	\sum_{c_1 s^{1/q_1} (\log (s+1))^{\varepsilon}\le \xi^{1/q_1}} \brac{c_2 s^{1/q_2} (\log (s+1))^{\varepsilon}}^{-1/\alpha}
% 	\\&
% \ \ge \ C \xi^{1/(\alpha q_1)}
% \sum_{s(\log (s+1))^{q_1\varepsilon}\le 	c_1^{-q_1}\xi}
%  s^{-1/(q_2\alpha)} (\log (s+1))^{-\varepsilon/\alpha}	
%  \end{split}
% \end{equation} 
% Since $1/(q_2\alpha)  < 1$, we can estimate from below  the sum in the right-side hand as 
% $$
% \sum_{s(\log (s+1))^{q_1\varepsilon}\le 	c_1^{-q_1}\xi}
%s^{-1/(q_2\alpha)} (\log (s+1))^{-\varepsilon/\alpha}
%\ \ge \
%	C_\varepsilon  \xi^{1- 1/(q_2\alpha)}(\log \xi)^{-\varepsilon/\alpha - q_1\varepsilon}.	 
% $$
%hence we obtain that 
%\begin{equation} \nonumber
%	\dim B(\xi) \ge K_\varepsilon  \xi^{1 + \delta/\alpha}(\log \xi)^{-\varepsilon(1/\alpha + q_1)},
%\end{equation}
%for some positive constant independent of $\xi$. For any $n \in \NN$, letting $\xi_n$ be a number satisfying the inequalities 
%\begin{equation} \label{xi_n}
%\frac{1}{2}	K_\varepsilon  \xi_n^{1 + \delta/\alpha}(\log \xi_n)^{-\varepsilon(1/\alpha + q_1)}
%	\ < \
%	n + 1
%	\ \le \ K _\varepsilon \xi_n^{1 + \delta/\alpha}(\log \xi_n)^{-\varepsilon(1/\alpha + q_1)},
%\end{equation}
%we derive that  $ \dim \Vv(G(\xi_n)) \ge  n + 1$.
%On the other hand, by $\eqref{|u|<}$ we have that 
%$$
%C \xi_n^{-1/q_1} B(\xi_n) \subset B_X^\alpha
%$$
%  for some positive number $C$ independent of $n$.
%Hence, by \eqref{TikhomirovEq} and \eqref{|u|<} we deduce that 
%\begin{align*}
%	d_n(B_X^\alpha, L_2(U,X;\mu))
%	\ &\ge \
%	d_n(C \xi_n^{-1/q_1} B(\xi_n) , L_2(U,X;\mu))
%	\\&
%	\ge \ 
%	C_\varepsilon \xi_n^{-1/q_1}
%\	\ge \
%	C_\varepsilon	n^{- \beta} (\log n)^{- \varepsilon\beta (1/\alpha + q_1)}
%		\\&
%	\ = \	
%	C_\varepsilon n^{- \beta}(\log n)^{-\beta (1 + 1/(\alpha q_1)) - \beta(1/q_1 - \varepsilon)}.
%\end{align*}
%This proves the lower bound in \eqref{d_n2}.
%
%		{\bf The case $\alpha \le 1/q_2$. } The lower bound of \eqref{d_n1} can be proven in a similar but much simpler way. Let us briefly process the proof.
%To this end,   for $k \in \NN_0$ consider the subspace in $L_2(U,X;\mu)$
%$$
%B'(k):= \brab{u =  (P_k v_1) \varphi_1: \ \norm{v}{L_2(U,X;\mu)} \le 1},
%$$
%We have $\dim B'(k) = \dim V_k = 2^k$ and by the definitions,
%\begin{equation} \label{|u|<2}
%	\|u\|_{H_{X}^{\alpha}} \ \le \
%	C 2^{\alpha k}, \ \ u \in B'(k).
%\end{equation}
% For any $n \in \NN$, letting $k_n$ be a number satisfying the inequalities 
%\begin{equation} \label{k_n}
%2^{k_n -1}	\ < \
%	n + 1
%	\ \le \ 2^{k_n},
%\end{equation}
%we derive that  $ \dim \Vv(G(\xi_n)) \ge  n + 1$. On the other hand, by $\eqref{|u|<}$ we have that 
%$$
%C 2^{-\alpha k_n}B'(k_n) \subset B_X^\alpha
%$$
%for some positive number $C$ independent of $n$.
%Hence, by \eqref{TikhomirovEq}, \eqref{|u|<2} and \eqref{k_n}  we deduce that 
%\begin{align*}
%	d_n(B_X^\alpha, L_2(U,X;\mu))
%	\ &\ge \
%	d_n(C 2^{-\alpha k_n} B'(k_n) , L_2(U,X;\mu))
%	\\&
%	\ge \ 
%	C 2^{-\alpha k_n}
%	\asymp 
%	n^{- \alpha}.
%\end{align*}
%This proves the lower bound in \eqref{d_n1}.
% 	\hfill
% \end{proof}
%  
 

%In this section, we show that the problem of linear sampling recovery of functions in the  space $H_{X,\bsigma}$ for a general separable Hilbert space $X$ can be reduced to the particular case of  the reproducing kernel Hilbert space	$H_{\CC,\bsigma}$. This allows, in particular,  to extend linear least squares sampling algorithms in $H_{\CC,\bsigma}$ to  $H_{X,\bsigma}$ with preserving the accuracy of approximation. Hence, we are able to derive convergence rates of various extended  linear least squares sampling algorithms
%for functions in $B_{X,\bsigma}^q$ based on some recent
% results on inequality between sampling widths and Kolmogorov widths of the unit ball $B_{\CC,\bsigma}$ which are fulfilled by the relevant linear least squares sampling algorithms.
	
	\subsection{Extended least squares sampling algorithms in Bochner spaces}
	\label{Extended least squares sampling algorithms in Bochner spaces}
	
			
Let $A^X$ be a  linear operator in $L_2(U,X;\mu)$ defined for $v\in L_2(U,X;\mu)$ by 	
\begin{equation} \label{A^X}
	v \mapsto \sum_{k \in \NN}  
	\brac{\sum_{s \in \NN} a_{k,s}	v_s}\varphi_k,
\end{equation}
where $(a_{k,s})_{(k,s) \in \NN^2}$	 is an infinite dimensional matrix. Then $A^X$ uniquely defines the linear operator $A^\CC$ in $L_2(U,\CC;\mu)$  given for $f\in L_2(U,\CC;\mu)$ by 	
\begin{equation} \label{A^CC}
	f \mapsto \sum_{k \in \NN}  
	\brac{\sum_{s \in \NN} a_{k,s}	f_k}\varphi_k.
\end{equation}
Conversely, a linear operator $A^\CC$ in $L_2(U,\CC;\mu)$  of the form \eqref{A^CC} uniquely defines the $A^X$  linear operator of the form \eqref{A^X} in $L_2(U,X;\mu)$, i.e., this is an one-onto-one correspondence between the linear operators in $L_2(U,X;\mu)$ and $L_2(U,\CC;\mu)$. Similarly, there is  an one-onto-one correspondence between the sets of linear subspaces in $L_2(U,X;\mu)$ and $L_2(U,\CC;\mu)$. Based on these correspondences, we can see that there are  one-onto-one correspondences between the spaces 
$H_\CC^{\alpha}$,  $H_\CC^{\alpha,\tau}$, $H_\CC$ and  $H_X^{\alpha}$, $H_X^{\alpha,\tau}$,  $H_X$, respectively.
%\eqref{H \eqref{H_{X}^{alpha,tau}}, \eqref{H_{X}}.
%	 In what follows, with an abuse, we always denote by $H_\CC^{\alpha,\tau}$ and $H_\CC$ the spaces which corresponding to  the spaces $H_X^{\alpha,\tau}$ and $H_X$.

Let  $E$ be  a normed space of functions on $\Omega$. Given sample points $\bx_1,\ldots,\bx_n \in \Omega$, we consider the approximate recovery  of a  function $f$ on $\Omega$  from their values $f(\bx_1),\ldots, f(\bx_n)$ by a linear sampling algorithm $S_n$
on $\Omega$ of the form
\begin{equation} \label{S_k}
	S_n(f): = \sum_{i=1}^n  f(\bx_i) h_i, 
\end{equation}
where  $h_1,\ldots,h_n$ are given  functions on $\Omega$.  For convenience, we assume that some of the sample points $\bx_i$ may coincide. The approximation error is measured by the norm 
$\|f - S_n(f)\|_E$.   Denote by $\Ss_n$  the family of all linear sampling algorithms $S_k$ of the form \eqref{S_k} with $k \le n$.
Let $F \subset E$ be a set of  functions on $\Omega$.   To study the optimality  of linear  sampling algorithms  from $\Ss_n$ for  $F$ and their convergence rates we use  the  (linear) sampling $n$-width
\begin{equation*} %\label{rho_n}
	\varrho_n(F, E) :=\inf_{S_n \in \Ss_n} \ \sup_{f\in F} 
	\|f - S_n (f)\|_E.
\end{equation*}

Denote by $B_{X,\bsigma}$, $B_X^{\alpha}$ and $B_X^{\alpha,\tau}$ the unit balls in the spaces $H_{X,\bsigma}$, $H_X^{\alpha}$ and $H_X^{\alpha,\tau}$, respectively.

The following result has been proven in \cite{BD2024}.
\begin{lemma}\label{lemma:sampling-equality}
	Given arbitrary sample points $\by_1,\ldots,\by_n \in U$ and functions $h_1,\ldots,h_n \in L_2(\RRi,\CC;\mu)$,	for the sampling algorithm $S_n^X$  in $L_2(U,X;\mu)$ defined by \eqref{S_k}, we have
	\begin{equation*} \label{sampling-equality}
		\sup_{v \in B_{X,\bsigma}} \norm{v- S_n^X v}{L_2(U,X;\mu)}
		= \sup_{f \in B_{\CC,\bsigma}} \norm{f - S_n^\CC f}{L_2(U,\CC;\mu)}.
	\end{equation*}		
\end{lemma}

In the next step we extend this result  to the space $H_X^{\alpha,\tau}$. 
For $k \in \NN_0$ and $v \in H_X^{\alpha,\tau}$,  we define
\begin{equation} \label{delta_k}
	\delta_k v
	:= \
	P_{k} v  - P_{k-1} v, \ k \in \NN, \quad \delta_0 v = P_0 v.
\end{equation}
We then can represent every $v \in H_X^{\alpha,\tau}$  by the series
\begin{equation*} %\label{v=}
	v
	\ = \
	\sum_{k \in \NN_0}\delta_k v
\end{equation*}
converging in $L_2(U,X;\mu)$ absolutely and hence, unconditionally, and satisfying 
\begin{equation*} %\label{2^{alpha k} norm{delta_k v}}
	\sup_{k \in \NN_0}	2^{\alpha k} \norm{\delta_k v} {H_{X,\bsigma_2}^{\alpha,\tau}}
	\ \le \
	2\|v\|_{H_{X,\bsigma_2}^{\alpha,\tau}}
	\ < \infty. 
\end{equation*}

For $m \in \NN_0$, we introduce the sets
\begin{equation*} \label{Lambda_sigma(xi)}
	\delta_mB_X^{\alpha,\tau}
	:= \ 
	\ \big\{\delta_m v:\,  v \in B_X^{\alpha,\tau}\big\}, \ \ \
	\delta_mB_\CC^{\alpha,\tau}
	:= \ 
	\ \big\{\delta_m f:\,  f \in B_\CC^{\alpha,\tau}\big\}.
\end{equation*}	

\begin{lemma} \label{lemma:OperatorNormEquality}
	Assume that there is a constant $K$ such that for every $v \in H_X^{\alpha,\tau}$ and every $m \in \NN_0$, $\norm{P_m v}{H_X^{\alpha,\tau}} \le K\norm{v}{H_X^{\alpha,\tau}}$.  
Then for every $m \in \NN_0$ there holds the equality
	\begin{equation} \label{OperatorNormEquality}
	\sup_{0 \not=\delta_m v\in \delta_mB_X^{\alpha,\tau}}
	\frac{\big\|A^X\delta_m v\big\|_{L_2(U,X;\mu)}}{\norm{\delta_m v}{H_X^{\alpha,\tau}}}
		\ = \
		\sup_{0 \not=\delta_m f\in \delta_mB_\CC^{\alpha,\tau}}	
	\frac{\big\|A^X\delta_m f\big\|_{L_2(U,\CC;\mu)}}{\norm{\delta_m f}{H_\CC^{\alpha,\tau}}}.	
	\end{equation}
\end{lemma}

\begin{proof} From the  assumptions we can see that for every $v \in H_X^{\alpha,\tau}$ and every $m \in \NN_0$, $\norm{\delta_m v}{H_X^{\alpha,\tau}} \le 2 K\norm{v}{H_X^{\alpha,\tau}}$. 
	Denote by $N_X$ and $N_\CC$ the left-hand side and right-hand side of \eqref{OperatorNormEquality}, respectively. 
	For $\delta_m f \in \delta_m B_\CC^{\alpha,\tau}$ represented as
	\begin{equation} \nonumber
		\delta_m f
		= \
		\sum_{s \in \NN} 	(\delta_m f)_s \varphi_s,
	\end{equation}
	we have 
	\begin{equation} \nonumber
		\|A^{\CC} \delta_m f\|_{L_2(U,\CC;\mu)}^2
		\le
		N_\CC^2 	
		\brac{\|\delta_m f\|_{H_{\CC,\bsigma_1}}^2 + \|\delta_m f\|_{H_{\CC,\bsigma_2}^{\alpha,\tau}}^2}.
	\end{equation}
	The last inequality is equivalent to inequality
	\begin{equation*} %\label{ineq1}
		\sum_{i \in \NN}  
		\left|\sum_{s \in \NN} a_{i,s} (\delta_m f)_s\right|^2
		\le
	N_\CC^2 	
		\brac{\sum_{s\in\NN}
			\brac{ \sigma_{1;s} |(\delta_m f)_s|}^2  
			+	\sum_{k \in \NN_0}	\sum_{s\in\NN} 
			\brac{2^{\alpha k}k^{-\tau} \sigma_{2;s} |(\delta_m f - P_k (\delta_m f))_s|}^2}. 
	\end{equation*}
	%for every $k \in \NN_0$ and 
	%for every sequence $\brac{2^{\alpha k} k^{-\tau}\sigma_{2;s} |(f - P_k f)_s|}_{s,k \in \NN}\in\ell_2\otimes \ell_2 $ such that
	%	\begin{equation} \nonumber
		%		\sum_{k \in \NN_0}	\sum_{s\in\NN} 
		%		\brac{2^{\alpha k} k^{-\tau}\sigma_{2;s} |(f - P_k f)_s|}^2 < \infty.
		%	\end{equation}
	For $\delta_m v \in \delta_mH_X^{\alpha,\tau}$, we have 
	\begin{equation} \nonumber
		\delta_m v
		= \
		\sum_{s\in \NN} 
		(\delta_m v)_s \varphi_s
\end{equation}
with			
	\begin{equation} \nonumber	
		\sum_{s\in\NN}
		\brac{ \sigma_{1;s} \|(\delta_m v)_s\|_X}^2  
		+	\sum_{k \in \NN_0}	\sum_{s\in\NN} 
		\brac{2^{\alpha k}k^{-\tau} \sigma_{2;s} \|(\delta_m v - P_k (\delta_m v))_s\|_X}^2 
		< \infty,
	\end{equation}	
	and 
	\begin{equation} \label{A^Xdelta_m v=}
		\|A^X\delta_m v\|_{L_2(U,X;\mu)}^2
		= \
		\sum_{i \in \NN} 	\left\|\sum_{s \in \NN} a_{i,s} (\delta_m v)_s \right\|_X^2.
	\end{equation}
	Let $\brac{\psi_j}_{j \in \NN}$ be an orthonormal basis of  $X$.  Then $\brac{\varphi_k \psi_j}_{k,j \in \NN}$ is an orthonormal basis of  $L_2(U,X;\mu)$. We have
	\begin{equation} \nonumber
		\delta_m v
		= \
		\sum_{j \in \NN} 	\sum_{s \in \NN} 
		(\delta_m v)_{s}^j\psi_j \varphi_s 
		= 
		\sum_{j \in \NN} 	
		(\delta_m v^j)\psi_j.
	\end{equation}	
%	where $\delta_m v^j \in \delta_mB_\CC^{\alpha,\tau}$.
	Then,
	\begin{equation} \nonumber
		A^X\delta_m v
		= \
		\sum_{s \in \NN} \sum_{i \in \NN} \sum_{j \in \NN} a_{i,s}(\delta_m v)^j_s \psi_j  \varphi_m.
	\end{equation}
	By applying  \eqref{A^Xdelta_m v=}  to $\delta_m f = \delta_m v^j$, we obtain for every $k \in \NN_0$,
	\begin{equation} \nonumber
		\begin{aligned}
			&	\|A^X\delta_m v\|_{L_2(U,X;\mu)}^2
			\\ &	= \
			\sum_{j \in \NN}\sum_{i \in \NN}
			\left|\sum_{s \in \NN}  a_{i,s} 	(\delta_m v)_{s}^j  \right|^2
			= \sum_{j \in \NN}\|A^\CC (\delta_m v)^j\|_{L_2(U,\CC;\mu)}^2
			\\&
			\le \
			N_X^2
			\brac{\sum_{j \in \NN} 
				\sum_{s\in\NN}
				\brac{ \sigma_{1;s} |((\delta_m v)^j)_s|}^2  
				+	\sum_{j \in \NN} \sum_{k \in \NN_0}	\sum_{s\in\NN} 
				\brac{2^{\alpha k}k^{-\tau} \sigma_{2;s} |((\delta_m v)^j - P_k ((\delta_m v)^j))_s|}^2}
			\\&
			=:  	N_X^2
			(B_1^2 + B_2^2).
		\end{aligned}	
	\end{equation}	
	For the term $B_2^2$, we have
	\begin{equation} \nonumber
		\begin{aligned}
			B_2^2
			&= \		
			\sum_{k \in \NN_0}\sum_{s \in \NN}\brac{2^{\alpha k}  k^{-\tau}\sigma_{2;s}}^2 
			\sum_{j\in\NN} |((\delta_m v)^j - P_k ((\delta_m v))^j)_s|^2	
			\\&
			=
			\sum_{k \in \NN_0}	\sum_{s\in\NN} 
			\brac{2^{\alpha k} k^{-\tau}\sigma_{2;s} |((\delta_m v)^j - P_k ((\delta_m v))^j)_s|}^2
			=
			\|\delta_m v\|_{H_{X,\bsigma_2}^{\alpha,\tau}}^2.
		\end{aligned}	
	\end{equation}	
	By the same way we can show
	\begin{equation} \nonumber
		B_1
		\ = \		
		\|\delta_m v\|_{H_{X,\bsigma_1}}.
	\end{equation}	
	Summing up 	we prove  the inequality	
	\begin{equation} \nonumber
	N_X
		\le \  N_{\CC}.
	\end{equation}
	In order to prove the inverse inequality, let 
	$\brac{f^n}_{n \in \NN} \subset H_\CC^{\alpha,\tau}$ be  a sequence  such that 	 
	$\|(\delta_m f)^n\|_{H_\CC^{\alpha,\tau}} = 1$, and 
	$$
	\lim_{n \to \infty} \big\|A^{\CC}(\delta_m f)^n\big\|_{L_2(U,\CC;\mu)} = N_{\CC}.
	$$
	Define $(\delta_m v)^n:= (\delta_m f)^n\psi_1$. Then  $\|(\delta_m v)^n\|_{H_X^{\alpha,\tau}}= 1$ and
	\begin{equation} \nonumber
		\begin{aligned}
			\big\|A^X(\delta_m v)^n\big\|_{L_2(U,X;\mu)}^2
			&	= \
			\sum_{k \in \NN} 	
			\left\|\sum_{s \in \NN} a_{k,s}	\langle (\delta_m f)^n,\varphi_s \rangle_{L_2(U, \CC; \mu)} \psi_1 \right\|_X^2
			\\&
			= \
			\sum_{k \in \NN} 	
			\left|\sum_{s \in \NN} a_{k,s}	\langle (\delta_m f)^n,\varphi_s \rangle_{L_2(U, \CC; \mu)} \right|^2
			\\&
			= \
			\big\|A^{\CC}(\delta_m f)^n\big\|_{L_2(U,\CC;\mu)}^2	 \ \to 	\	N_\CC \ \ \text{as} \ \ n \to \infty.
		\end{aligned}	
	\end{equation}	
	This proves the inequality	
	\begin{equation} \nonumber
		N_X
		\ge \ 
	N_\CC.
	\end{equation}
	\hfill
\end{proof}

From this lemma we can extend the  result of Lemma \ref{lemma:sampling-equality} to the sets $\delta_mB_X^{\alpha,\tau}$ and $\delta_mB_\CC^{\alpha,\tau}$.

\begin{corollary}\label{corollary:OperatorNormEquality}
	Assume that there is a constant $K$ such that for every $v \in H_X^{\alpha,\tau}$ and every $m \in \NN_0$, $\norm{P_m v}{H_X^{\alpha,\tau}} \le K\norm{v}{H_X^{\alpha,\tau}}$. 
	Given arbitrary sample points $\by_1,\ldots,\by_k \in U$ and functions $h_1,\ldots,h_k \in L_2(U,\CC;\mu)$,	for the sampling algorithm $S_n^X$  in $L_2(U,X;\mu)$ defined by \eqref{S_k}, we have for every $m \in \NN_0$,
	\begin{equation} \nonumber
\sup_{0 \not=\delta_m v\in \delta_mB_X^{\alpha,\tau}}			
		\frac{\norm{\delta_m v - S_n^X (\delta_m v)}{L_2(U,X;\mu)}}{\norm{\delta_m v}{H_X^{\alpha,\tau}}}
		\ = \
			\sup_{0 \not=\delta_m f\in \delta_mB_\CC^{\alpha,\tau}}	
		\frac{\norm{\delta_m f - S_n^\CC (\delta_m f)}{L_2(U,\CC;\mu)}}{\norm{\delta_m f}{H_\CC^{\alpha,\tau}}}.	
	\end{equation}
\end{corollary}

\begin{proof}
	This corollary is  Lemma \ref{lemma:OperatorNormEquality} for $A^X = I^X - S_n^X$, where
	$I^X$ denotes the identity operator in $L_2(U,X;\mu)$. 
	\hfill
\end{proof}

Let $n \in \NN$ and $E$ be a normed space and $F$ a central symmetric compact set in $E$. 
Then the Kolmogorov $n$-width of $F$ is defined by
\begin{equation*}
	d_n(F,E):= \ \inf_{L_{n}}\sup_{f\in F}\inf_{g\in L_n}\|f-g\|_E,
\end{equation*}
where the left-most infimum is taken over all subspaces $L_{n}$ of dimension at most $n$ in $E$. The Kolmogorov $n$-width $d_n(F,E)$ is a characterization of the best approximation of elements in $F$ by elements in linear subspaces of dimension at most $n$.

We recall a concept  of  weighted least squares sampling algorithm in the space $L_2(U,\CC;\mu)$ (cf. also \cite{BD2024}). Recall that  $\brac{\varphi_s}_{s \in \NN}$ is the fixed orthonormal basis of  $L_2(U,\CC;\mu)$ considered in Subsection  \ref{Extended least squares sampling algorithms in Bochner spaces}.
For $c,n,m\in\NN$ with $cn\ge m$, let $\by_1, \dots, \by_{cn}\in U$ be points, $\omega_1, \dots, \omega_{cn}\ge 0$ be weights, and $\Phi_m = \operatorname{span}\{\varphi_s\}_{s=1}^{m}$ the subspace spanned by the functions $\varphi_s$,   $s=1,...,m$.
The weighted least squares sampling algorithm $S_{cn}^{\CC} f = S_{cn}^{\CC}(\by_1, \dots, \by_{cn}, \omega_1, \dots, \omega_{cn}, \Phi_m) f$ of a function $f\colon U\to\CC$ is given by
\begin{equation} \label{least-squares-sampling1}
	S_{cn}^{\CC} f
	= \operatorname{arg\,min}_{g\in \Phi_m} \sum_{i=1}^{cn} \omega_i |f(\by_i) - g(\by_i)|^2 .
\end{equation}
This weighted least squares sampling algorithm can be computed using the Moore-Penrose inverse, which gives the solution of smallest error for over-determined systems where no exact solution can be expected.
In particular, for $\boldsymbol L = [\varphi_s(\by_i)]_{i=1,\dots,cn; s=1,\dots,m}$ and $\boldsymbol W = \operatorname{diag}(\omega_1, \dots, \omega_n)$ we have
\begin{equation} \label{least-squares-sampling2}
	S_{cn}^\CC f
	= \sum_{s=1}^{m} \hat g_{s} \varphi_s
	\quad\text{with}\quad
	(\hat g_1, \dots, \hat g_{m})^\top
	= (\boldsymbol L^\ast\boldsymbol W\boldsymbol L)^{-1}\boldsymbol L^\ast\boldsymbol W (f(\by_1), \dots, f(\by_{cn}))^{\top}.
\end{equation} 
Notice that $S_{cn}^\CC$ is a linear sampling algorithm of the form
\begin{equation}\label{least-squares3}
	S_{cn}^\CC f
	= \sum_{i=1}^{cn} f(\by_i) h_i(\by)
\end{equation}
for some $h_1,..., h_{cn}$ in $L_2(U,\CC;\mu)$.


The extended least squares algorithm to the Bochner space $L_2(U,X;\mu)$ can be defined 
by replacing $f\in L_2(U,\CC;\mu)$ with $v\in L_2(U,X;\mu)$:
\begin{equation}  \label{least-squares-sampling-extension}
	S_{cn}^X v
	= \sum_{i=1}^{cn} v(\by_i) h_i(\by).
\end{equation}

For $m\in\NN$ let the probability  measure $\nu = \nu(m)$ be defined by
\begin{equation}\label{nu}
	\mathrm d\nu(\by)
	:= \varrho(\by)\mathrm d\mu(\by)
	:= \frac{1}{2}\left( \frac{1}{m}\sum_{s=1}^{m}|\varphi_s(\by)|^2
	+ \frac{\sum_{s=m+1}^{\infty}|\sigma_{s}^{-1}\varphi_s(\by)|^2}{\sum_{s=m+1}^{\infty}\sigma_{s}^{-2}} \right)\mathrm d\mu(\by) .
\end{equation}
		Let $m := n$ and $\lceil 20 n\log n\rceil$ sample points be drawn i.i.d.\ with respect to $\nu$.
Let further $\by_1, \dots, \by_{c_pn} \in U$ be the subset of the sample points fulfilling \cite[Theorem~3]{DKU2023} with $c_p n \asymp m$ and $\omega_i := \frac{c_pn}{\lceil 20  n\log n\rceil}(\varrho(\by_i))^{-1}$.
For every  $n \in \NN$,  let
\begin{equation} \label{S_nCC}
	S_{c_p n}^\CC f : = \sum_{i=1}^{c_p n}  f(\by_i) h_i,
\end{equation}	
be the least squares sampling algorithm constructed as in \eqref{least-squares-sampling1}--\eqref{least-squares3} for these sample points and weights, where  
$h_1,...,h_{c_p n} \in L_2(U,\CC;\mu)$.

The following result \cite[Theorem 3]{DKU2023} gives a error bound of the approximation by 
$S_{c_p n}^\CC$ via Kolmogorov $n$-widths.

\begin{lemma}\label{lemma:sampling inequality}
	Let $0 < p <2$ and $d_s := d_s(F,L_2(U,\CC;\mu))$. Assume that $F \subset L_2(U,\CC;\mu)$ is such that there is a metric on $F$ satisfying the condition that $F$ is continuously embedded into $L_2(U,\CC;\mu)$, and the function evaluation $f \mapsto f(\by)$ is continuous on $F$ for every $\by \in F$. 
	Then there is a constant $c_p  \in \NN$ such that  the least squares sampling algorithms 
	$S_{c_p n}^\CC$ defined as in \eqref{S_nCC} satisfy the inequalities		
			\begin{equation} \nonumber
				\varrho_n(F, L_2(U,\CC;\mu)) 
				\ \le \
			\sup_{f\in F} \norm{v - S_{c_p n}^\CC f}{L_2(U,\CC;\mu)}
			\ \le \brac{\frac{1}{n} \sum_{s \ge n} d_s^p}^{1/p}.
		\end{equation}		
\end{lemma}

 The sampling algorithms $S_{c_p n}^\CC$  in Lemma \ref{lemma:sampling inequality} is a weighted least squares algorithm using samples from a set of $c_p n$ points that is subsampled from a set of $c_p n \log n$ i.i.d. random points with respect to the probability measure $\nu$. Depending on the function class $F$, the algorithm using the full set of random points may be constructive but the subsampling is  not constructive. We will explain and discuss it and other least squares sampling algorithms  in detail in Section \ref{Extensions}.

\begin{lemma}\label{lemma:sampling inequalityX}
Let $0 < p <2$ and $d_j := d_j(B_{\CC,\bsigma},L_2(U,\CC;\mu))$. 
Let  the least squares linear sampling algorithm $S_{c_p n}^\CC$ be as in \eqref{S_nCC}.
Then the extended least squares linear sampling algorithm
\begin{equation*} %\label{S_nX}
	S_{c_p n}^X  v : = \sum_{i=1}^{c_p n}  v(\by_i) h_i
\end{equation*}
defined by the formula \eqref{least-squares-sampling-extension}, satisfies the inequality		
\begin{equation} \nonumber
	\varrho_n(B_{X,\bsigma},L_2(U,X;\mu))
	\ \le \
	\sup_{v\in B_{X,\bsigma}} \norm{v - S_{c_p n}^X v}{L_2(U,X;\mu)}
	\ \le \brac{\frac{1}{n} \sum_{j \ge n} d_j^p}^{1/p}.
\end{equation}		
\end{lemma}


\begin{proof}
	We first consider
	the particular case when $X = \CC$. In this case $B_{\CC,\sigma}$ is a subset of the unit ball  of the reproducing kernel Hilbert space $H_{\CC,\bsigma}$ equipped with a metric as the norm of  this space, which is continuously embedded into $L_2(U,X;\mu)$  and for which  the function evaluation $f \mapsto f(\by)$ is continuous on $B_{\CC,\sigma}$ for every $\by \in B_{\CC,\sigma}$. This means that the assumption of Lemma~\ref{lemma:sampling inequality}  is satisfied for $F= B_{\CC,\sigma}$. The lemma has been proven for the particular case when $X = \CC$.
	Hence, by 
	using  Lemma \ref{lemma:sampling-equality} we prove the lemma in the general case.
	\hfill
\end{proof}

In a similar way from Lemma \ref{lemma:sampling inequality} and Corollary \ref{corollary:OperatorNormEquality} we derive 

\begin{lemma}\label{lemma:sampling inequality-delta_m}
	Let $0 < p <2$, $m \in \NN_0$,  and $d_j := d_j(\delta_mB_\CC^{\alpha,\tau},L_2(U,\CC;\mu))$. 	 Assume that there is a constant $K$ such that for every $v \in H_X^{\alpha,\tau}$ and every $m \in \NN_0$, $\norm{P_m v}{H_X^{\alpha,\tau}} \le K\norm{f}{H_X^{\alpha,\tau}}$. 
	Then there is a constant $c_p \in \NN$ such that we have the following.  For every $m \in \NN_0$ and every  $n \in \NN$, there are  
	$\by_1,...,\by_{c_p n}$	and $h_1,...,h_{c_p n} \in L_2(U,\CC;\mu)$ such that the linear least squares sampling algorithm 
	\begin{equation*} %\label{S_n^X}
		S_{c_p n}^X  v : = \sum_{i=1}^{c_p n}  v(\by_i) h_i
	\end{equation*}
	satisfies the inequalities		
	\begin{equation} \nonumber
			\varrho_n(\delta_mB_X^{\alpha,\tau}, L_2(U,X;\mu)) 
		\ \le \
		\sup_{\delta_m v \in \delta_mB_X^{\alpha,\tau}} \norm{\delta_m v - S_{c_p n}^X (\delta_m v)}{L_2(U,X;\mu)}
		\ \le \brac{\frac{1}{n} \sum_{j \ge n} d_j^p}^{1/p}.
	\end{equation}		
\end{lemma}


%\subsection{Convergence rates}

Let $\Lambda \subset \NN$ be a finite set. For
$v \in L_2(U,X;\mu)$,
we define the truncation of the  GPC expansion of $v$
\begin{equation*} %\label{truncation}
	S_{\Lambda}v := \sum_{s\in\Lambda} (\delta_m v)_s \varphi_s.
\end{equation*}

The following lemma has been proven in \cite[Lemma 2.2]{BD2024}

\begin{lemma} \label{lemma:d_n><}
	Let $0 < q\le 2$ and $\bsigma = (\sigma_{s})_{s \in \NN}$  a given sequence of numbers strictly larger than $1$, $\bsigma^{-1} \in \ell_q(\NN)$. For $\xi>0$ and $M > 0$, we introduce the set	
	\begin{equation*} \label{Lambda_sigma(xi)}
		\Lambda(\xi)
		:= \ 
		\ \big\{s \in \NN: \, \sigma_s \le \xi^{1/q} \big\}.
	\end{equation*}
	We have
	\begin{equation} \label {sup_{B_{X,bsigma}^q}}
		\sup_{v \in B_{X,\bsigma}}	\|v- S_{\Lambda(\xi)}v\|_{L_2(U,X;\mu)}		
			\ \le \ \xi^{-1/q}.
	\end{equation} 
Moreover, if in addition $\norm{\bsigma^{-1}}{\ell_q(\NN)} \le 1$, then for all $n \ge 2$ there exists a number $\xi_n$  such that 
$
\dim(\Vv(\Lambda(\xi_n))  \le  n,
$
 and	
		\begin{equation} \label{d_n><}
		d_n(B_{\CC,\bsigma},L_2(U,\CC;\mu))
				\ \le \ 
		\sup_{f \in B_{\CC,\bsigma}}	
		\|f- S_{\Lambda(\xi_n)}f\|_{L_2(U,\CC;\mu)}		
		\ \le 2^{1/q}  n^{-1/q} \ \ \forall n \in \NN.
	\end{equation}
\end{lemma}	


\begin{lemma} \label{lemma:d_n(delta_kB_X)}
	Let $0 < q_1 \le q_2 <2$ and $\alpha > 1/q_2$. 
	Let the numbers $\beta$ and $\delta$ be defined as 
	\begin{equation} 	\label{delta, beta}
		\beta := \frac 1 {q_1}\frac{\alpha}{\alpha + \delta}, \quad 
		\delta := \frac 1 {q_1} - \frac 1 {q_2}.
	\end{equation}	
	 Assume that there is a constant $K$ such that for every $v \in H_X^{\alpha,\tau}$ and every $m \in \NN_0$, $\norm{P_m v}{H_X^{\alpha,\tau}} \le K\norm{v}{H_X^{\alpha,\tau}}$.  For $\xi>0$ and $m \in \NN_0$, we introduce the set
	 \begin{equation*} %\label{}
	 	\Lambda_m(\xi)
	 	:= \ 	 	
	 		\big\{\sigma_{2;s} \leq \xi^{1/q_1}2^{-\alpha m}m^\tau, \ \sigma_{1;s}\le \xi^{1/q_1} \big\}. 
	 \end{equation*}
	We have
	\begin{equation} \label{sup_{B_{X}}}
		\|\delta_m v - S_{\Lambda(\xi)}(\delta_m v)\|_{L_2(U,X;\mu)}		
		\ \le \ 2K\xi^{-2/q_1}\|\delta_m  v\|_{H_X^{\alpha,\tau}}, \ \ \ v \in H_X^{\alpha,\tau}.
	\end{equation} 
	Moreover, for all $n \ge 2$ there exists a number $\xi_n$  such that 
	$\dim(\Vv(\Lambda_m(\xi_n))  \le  n$,
	and	
	\begin{equation} \label{d_n(delta_mB)}
		\begin{split}
		d_n(\delta_m B_X^{\alpha,\tau},L_2(U,X;\mu))
		\ &\le \ 
	\sup_{\delta_m v \in \delta_m B_X^{\alpha,\tau}}
		\|\delta_m v -  S_{\Lambda_m(\xi_n)}(\delta_m v)\|_{L_2(U,X;\mu)}
		\\		
		\ &\le \  C\, (2^m n)^{-\beta} (m + \log n)^{\beta \tau /\alpha} \ \ \forall n \in \NN.
			\end{split}
	\end{equation}
\end{lemma}	

\begin{proof}	 From the  assumptions we can see that for every $v \in H_X^{\alpha,\tau}$ and every $m  \in \NN_0$, $\norm{\delta_m  v}{H_X^{\alpha,\tau}} \le 2 K\norm{v}{H_X^{\alpha,\tau}}$.  For a function $v\in B_\CC^{\alpha,\tau}$,
	 putting
	\[
	(\delta_m  v)_\xi
	:=
	\sum_{\sigma_{1;s}^{q_1} \le \xi} (\delta_m  v)_s \varphi_s,
	\]
	we get
	\begin{equation} \nonumber
		\|(\delta_m  v)- S_{\Lambda_m (\xi)} (\delta_m  v)\|_{L_2(U,X;\mu)}
		\ \le \
		\|(\delta_m  v)- (\delta_m  v)_\xi\|_{L_2(U,X;\mu)} + \|(\delta_m  v)_\xi - S_{\Lambda_m (\xi)} (\delta_m  v)\|_{L_2(U,X;\mu)}.
	\end{equation}
	For the norm $\|(\delta_m  v)- (\delta_m  v)_\xi\|_{L_2(U,X;\mu)}$ we have that
	\begin{equation*} %\label{ineq:|v-v_xi|}
		\begin{split}
			\|(\delta_m  v)- (\delta_m  v)_\xi\|_{L_2(U,X;\mu)} ^2
			\ &= \
			\sum_{\sigma_{1;s}> \xi^{1/q_1} } \|(\delta_m  v)_s\|_{X}^2 
			\ = \
			\sum_{\sigma_ {1;s} > \xi^{1/q_1} } (\sigma_{1;s}\|(\delta_m  v)_s\|_{X})^2 \sigma_{1;s}^{-2}
			\\[1.5ex]
			\ &\le \
			\xi^{-2/q_1}
			\sum_{\sigma_{1;s}> \xi^{1/q_1} } (\sigma_{1;s}\|(\delta_m  v)_s\|_{X})^2 
			\\[1.5ex]
			\ &\le \
			 (2K)^2\xi^{-2/q_1}\|\delta_m  v\|_{H_X^{\alpha,\tau}}^2.
		\end{split}
	\end{equation*}
	For the norm $\|(\delta_m  v)_\xi - S_{\Lambda_m (\xi)} (\delta_m  v)\|_{L_2(U,X;\mu)}$,  we obtain 
	\begin{equation} \nonumber
		\begin{split}
			\|(\delta_m  v)_\xi - S_{\Lambda_m (\xi)} (\delta_m  v)\|_{L_2(U,X;\mu)}^2
			\ &= \
			\sum_{\sigma_{1;s} \le \xi^{1/q_1}, \ \sigma_{2;s} > \xi^{1/q_1}2^{-\alpha m }m ^\tau} \norm{ (\delta_m  v)_s}{X}^2
					\\&
			\ \le \
			\sum_{\sigma_{1;s} \le \xi^{1/q_1}, \ \sigma_{2;s} > \xi^{1/q_1}2^{-\alpha m }m ^\tau} 
			\brac{ 2^{\alpha  m } m ^{-\tau}\xi^{-1/q_1}\sigma_{2;s} \norm{ (\delta_m  v)_s}{X}}^2
			\\[1.5ex]
			\ &\le \
			\xi^{-2/q_1} \sum_{s \in \NN} 
			\brac{ 2^{\alpha  m } m ^{-\tau}\sigma_{2;s} \norm{ (\delta_m  v)_s}{X}}^2
		\\[1.5ex]
		\ &\le \
		(2K)^2\xi^{-2/q_1}\|\delta_m  v\|_{H_X^{\alpha,\tau}}^2.
		\end{split}
	\end{equation}
	These estimates yield \eqref{sup_{B_{X}}}.
	
	For the dimension of the space $\Vv(\Lambda_m (\xi))$, 	with $q:= q_2 \alpha > 1$ and $1/q' + 1/q = 1$, 
 we have that
	\begin{equation} \nonumber
		\begin{split}
			\dim \Vv(\Lambda_m (\xi))
			\ & \ = \ |\Lambda_m (\xi)| \ = \ \sum_{s \in \Lambda_m (\xi)} 1
			\ = \ 
			2^{-m }\sum_{\sigma_{1;s}\le \xi^{1/q_1}, 
				\ \sigma_{2;s} > \xi^{1/q_1}2^{-\alpha m }m ^\tau} 2^m  
			\\ \  &\le \
		2^{-m }\sum_{s \in\NN: \  \sigma_{1;s}\le \xi^{1/q_1}}
			\ \sum_{(k,s)\in \NN_0 \times \NN: \ \sigma_{2;s} > \xi^{1/q_1}2^{-\alpha k}k^\tau} 2^{k} 
			\\ \  &\le \
			2^{-m } \sum_{\sigma_{1;\bs}\le \xi^{1/q_1}}   \xi^{1/(q_1\alpha)}
			\brac{\log\xi}^{\tau/\alpha} \sigma_{2;\bs}^{-1/\alpha} 
			\\ \ &\le \
			2^{-m } \xi^{1/(q_1 \alpha)}\brac{\log\xi}^{\tau/\alpha}
			\left(\sum_{\sigma_{1;\bs}\le \xi^{1/q_1}}\sigma_{2;\bs}^{-q_2} \right)^{1/q}
			\left(\sum_{\sigma_{1;\bs}\le \xi^{1/q_1}} 1 \right)^{1/q'}
			\\ \  &\le 
			2^{-m } \xi^{1/(q_1 \alpha)}\brac{\log\xi}^{\tau/\alpha}\left(\sum_{\bs \in \FF} \sigma_{2;\bs}^{-q_2} \right)^{1/q}
			\left(\sum_{\bs \in \FF} \xi \sigma_{1;\bs}^{-q_1}  \right)^{1/q'}
				\\ \  &\le 
			 2^{-m }\xi^{1 + \delta/\alpha}\brac{\log\xi}^{\tau/\alpha}.
		\end{split}
	\end{equation}
	
	For any $n \in \NN$, letting $\xi_n$ be a number satisfying the inequalities 
	\begin{equation} \label{[xi_n]3}
		 \xi^{1 + \delta/\alpha}\brac{\log\xi}^{\tau/\alpha}  2^{-m }
		\ \le \
		n
		\ < \ 
		2\xi^{1 + \delta/\alpha}\brac{\log\xi}^{\tau/\alpha}  2^{-m },
	\end{equation}
	we derive that  $ \dim \Vv(\Lambda_m (\xi_n)) \le  n$.
	On the other hand, by \eqref{[xi_n]3},
	\begin{equation} \nonumber
		\begin{split}
			\xi_n^{-1/q_1} 
			\ &\le \
			C \, \brac{2^m  n (\log n + m )^{-\tau/\alpha}}^{-\frac{1}{q_1} \frac{\alpha}{\alpha + \delta}} 
			\\&
			\ \le \  C\, (2^m  n)^{-\beta} (m  + \log n)^{\beta \tau /\alpha}.
		\end{split}
	\end{equation}	
	This together with \eqref{sup_{B_{X}}} proves \eqref{d_n(delta_mB)}.	
		\hfill
\end{proof}

From Corollary \ref{corollary:OperatorNormEquality} and Lemma \ref{lemma:d_n(delta_kB_X)} we obtain 

\begin{corollary} \label{corollary:d_n(delta_kB_CC)}
	Under the assumption and notation of Lemma \ref{lemma:d_n(delta_kB_X)},
	we have
	\begin{equation*} %\label {sup_{B_{CC}}}
		\|\delta_m f - S_{\Lambda(\xi)}(\delta_m f)\|_{L_2(U,\CC;\mu)}		
		\ \le \ 2K\xi^{-2/q_1}\|\delta_m  f\|_{H_\CC^{\alpha,\tau}}, \ \   \forall f \in H_\CC^{\alpha,\tau}.
	\end{equation*} 
	Moreover, for all $n \ge 2$ there exists a number $\xi_n$  such that 
	$\dim(\Vv(\Lambda_m(\xi_n))  \le  n$,
	and	
	\begin{equation*} 
		\begin{split}
			d_n(\delta_m B_\CC^{\alpha,\tau},L_2(U,\CC;\mu))
			\ &\le \ 
			\sup_{\delta_m f \in \delta_m B_\CC^{\alpha,\tau}}
			\|\delta_m f -  S_{\Lambda_m(\xi_n)}(\delta_m f)\|_{L_2(U,\CC;\mu)}
			\\		
			\ &\le \  C\, (2^m n)^{-\beta} (m + \log n)^{\beta \tau /\alpha} \ \ \forall n \in \NN.
		\end{split}
	\end{equation*}
\end{corollary}	
%%%%%%%%%%%%%%%

\subsection{Multi-level least squares sampling algorithms}

Let $0 < p < 2$ and $(\xi_{n})_{n \in \NN}$ be a sequence of increasing positive numbers whose values will be selected later, such that $\xi_{n} \to \infty$ as $n \to \infty$. For $n \in \NN$,  we consider the multi-level least squares  operator $\Ss_n^X$  defined by
\begin{equation}  \label{Ss_{n}^X}
	\Ss_{n}^X v
	\ = \ 
	\sum_{k=0}^{k_n}  S_{c_p n_k}^X (\delta_k v),
\end{equation}
where  
\begin{equation} \label{n_k,k_n}
n_k := \lfloor  c_p^{-1} n 2^{-k}\rfloor, \ \ \ k_n:= \lfloor \log \xi_n \rfloor,
\end{equation}
and
\begin{equation} \label{S_n_k}
	S_{c_p n_k}^X  v : = \sum_{i=1}^{c_p n_k}  v(\by_{k,i}) h_{k,i}
\end{equation}
are the least squares sampling algorithms 
%as in Lemma~\ref{lemma:sampling inequality} for  $\by_{k,1},...,\by_{k,n_k} \in U$ and  $h_{k,1},...,h_{k,n_k} \in L_2(U,\CC;\mu)$,
and $c_p > 0$  the constant as in Lemma~\ref{lemma:sampling inequality}. 

Since $\delta_k v$ belongs to the subspace $V_k + V_{k-1}$ of dimension at most $2^k + 2^{k-1}$ and the number of sampling points in the sampling algorithms $S_{c_p n_k}^X$ is $n_k$. The computational complexity 
$\operatorname{Comp} \brac{\Ss_{n}^X }$ of the operator  $\Ss_{n}^X$ can be defined as 
\begin{equation}  \nonumber
\operatorname{Comp} \brac{\Ss_{n}^X}
	 := \ 
	\sum_{k=0}^{k_n} \brac{2^k + 2^{k-1}}n_k.
\end{equation}

\begin{theorem}\label{thm:sampling}
	Let $0 < q_1 \le q_2 <2$ and $p$ be a fixed number such that $1/q_2 < p <2$. Let the numbers $\beta$ and $\delta$ be defined as in \eqref{delta, beta}.		
Let $\tau$ be any number such that  $\tau > 1$
	if $\alpha \le 1/q_2$, and $\tau =\alpha/(\alpha - \beta)$ if $\alpha > 1/q_2$. Let $(\xi_{n})_{n \in \NN}$ be a sequence of increasing positive numbers $\xi_n$ satisfying the condition
	\begin{equation} 	\label{xi_n}
			\begin{cases}
\xi_n \ \le \ n\ < \ 2\xi_n	\quad &{\rm if }  \ \alpha \le 1/q_2,\\
\xi_n^{q_1(\alpha + \delta)} \ \le \ n \ < \ 2\, \xi_n^{q_1(\alpha + \delta)} \quad  & {\rm if }  \ \alpha > 1/q_2. 
			\end{cases}
	\end{equation}
	 Assume that there is a constant $K:=K_\tau$ such that for every $v \in H_X^{\alpha,\tau}$ and every $m \in \NN_0$, 
	 $\norm{P_m v}{H_X^{\alpha,\tau}} \le K\norm{v}{H_X^{\alpha,\tau}}$. 
	Then for all $n \ge 2$ there exist  $\by_{k,1},...,\by_{k,n_k} \in U$ and $h_{k,1},...,h_{k,n_k} \in L_2(U,\CC;\mu)$, $k=0,...,k_n$,  such that for the operator $\Ss_{n}^X$ defined as in  \eqref{Ss_{n}^X}--\eqref{S_n_k},
	\begin{equation*} %\label{comp1}
			\operatorname{Comp} \brac{\Ss_{n}^X }	
			\ \le  \
			C n \log n.
	\end{equation*} 
	\begin{equation*} 	\label{L_2-rate}
		\begin{split}
			\sup_{v\in B_X^{\alpha,\tau}}		\|v-\Ss^X_n v\|_{L_2(U,X;\mu)} 
		\ \le \	C_\tau
			\begin{cases}
				n^{-\alpha}(\log n )^\tau\quad &{\rm if }  \ \alpha < 1/q_2,\\
				n^{-\alpha}(\log n )^{1 + \tau}\quad &{\rm if }  \ \alpha = 1/q_2,\\
				n^{-\beta}(\log n )^\tau \quad  & {\rm if }  \ \alpha > 1/q_2.
			\end{cases}
		\end{split}
	\end{equation*}
\end{theorem}

\begin{proof}
 Let $v \in B_X^{\alpha,\tau}$	be given. By the assumptions we have 
	\begin{equation} \label{norm{delta_k v}}
\norm{\delta_k v}{H_X^{\alpha,\tau}} \le 2K \ \ \forall k \in \NN_0.
\end{equation}
By the triangle inequality,
	\begin{equation*} %\label{ineq1}
		\|v- \Ss_n^X v\|_{L_2(U,X;\mu)}
		\ \le \ 
		\|v- P_{k_n} v\|_{L_2(U,X;\mu)}
		\ + \
		\sum_{k=0}^{k_n}	\|\delta_k v -  S_{n_k}^X (\delta_k v)\|_{L_2(U,X;\mu)}.
	\end{equation*}

	
		By  Parseval's identity  and the equality $k_n:= \lfloor \log \xi_n \rfloor$ we deduce that
	\begin{equation} \nonumber
		\begin{split}
			\|v- P_{k_n} v\|_{L_2(U,X;\mu)}^2
			\ &= \
			\sum_{s \in \NN}\|(v- P_{k_n} v)_s\|_{X}^2
			\\
			\ &\le \
			2^{-2\alpha k_n} \brac{k_n}^{2\tau} 2^{2\alpha k_n}\brac{k_n}^{-2\tau}
			\sum_{s \in \NN}\brac{\sigma_{2;s} \|(v- P_{k_n} v)_s\|_{X}}^2
			\\ 
			\ &\le \
			2^{-2\alpha \lfloor \log \xi_n \rfloor}\brac{k_n}^{2\tau}\norm{v}{H_{X,\bsigma_2}^{\alpha}} 
			\ \le \		2^{-2\alpha \lfloor \log \xi_n \rfloor}\brac{\log \xi_n}^{2\tau}.
		\end{split}
	\end{equation}	
	This means that 
	\begin{equation} \label{|v- P_{k_{xi_n}} v|}
			\|v- P_{k_n} v\|_{L_2(U,X;\mu)}
			\ \le \		2^{-\alpha \lfloor \log \xi_n \rfloor}\brac{\log \xi_n}^{\tau}.
	\end{equation}	

Assume  $\alpha \le 1/q_2$. By \eqref{norm{delta_k v}},	
\begin{equation} \label{norm{delta_k v}2}
	\norm{\delta_k v} {H_{X,\bsigma_2}}
	\ \le \
	2K 2^{-\alpha k} k^\tau, \ \ k \in \NN_0.
\end{equation} 
	By  Lemma \ref{lemma:sampling inequalityX} for
$\bsigma = \bsigma_2$  and  $q=q_2$,
there exist  $\by_{k,1},...,\by_{k,n_k} \in U$, $h_{k,1},...,h_{k,n_k} \in L_2(U,\CC;\mu)$ such that
\begin{equation*} %\label{ineq2}
	\begin{split}			
		\|\delta_k v -  S_{n_k}^X (\delta_k v)\|_{L_2(U,X;\mu)}
		\ &\le \ 
		\norm{\delta_k v}{H_{X,\bsigma_2}}	\brac{\frac{1}{n_k} \sum_{j \ge n_k} d_j^p}^{1/p},
	\end{split}
\end{equation*}
where $d_j := d_j(B_{\CC,\bsigma_2},L_2(U,\CC;\mu))$.
Hence, by  Lemma \ref{lemma:d_n><} for 
$\bsigma = \bsigma_2$  and  $q=q_2$,  \eqref{norm{delta_k v}2} and the inequality $p/q_2 > 1$,
we derive for $k \le k_n$,
\begin{equation} \label{ineq3}
	\begin{split}			
		\|\delta_k v -  S_{n_k}^X (\delta_k v)\|_{L_2(U,X;\mu)}
		\ &\le \ 
		C\, 2^{-\alpha k} k^\tau
		\brac{\frac{1}{n_k} \sum_{j \ge n_k} j^{-p/q_2}}^{1/p}
		\ \le \ 
		C\, 2^{\brac{1/q_2-\alpha}k}  k^\tau n^{-1/q_2}. 
	\end{split}
\end{equation}

For any $n \in \NN$, choose $\xi_n$ as a number satisfying the inequalities 
\begin{equation*} %\label{[xi_n]1}
	\xi_n 
	\ \le \
	n
	\ < \ 2\xi_n.
\end{equation*}	
By  the equality $k_n:= \lfloor \log \xi_n \rfloor$ and the equivalence $\xi_n \asymp n$, from 
\eqref{|v- P_{k_{xi_n}} v|} and \eqref{ineq3} we deduce that
\begin{equation} \nonumber
	\|v- P_{k_n} v\|_{L_2(U,X;\mu)}
	\ \le \ 
2^{-\alpha \lfloor \log \xi_n \rfloor} (\log \xi_n )^\tau
	\ \le \	2^{\alpha} n^{- \alpha}  (\log n )^\tau,
\end{equation}	
and 
 for  $\alpha < 1/q_2$,
\begin{equation} \nonumber
	\sum_{k=0}^{k_n} 	\|\delta_k v -  S_{c_p n}^X (\delta_k v)\|_{L_2(U,X;\mu)}
	\ \le \ C\, n^{- 1/q_2}
	\sum_{k=0}^{k_n}   2^{\brac{1/q_2-\alpha}k}k ^\tau
	\ \le \ C \,n^{- \alpha}(\log n )^\tau,
\end{equation}
and for  $\alpha = 1/q_2$,
\begin{equation} \nonumber
	\sum_{k=0}^{k_n} 	\|\delta_k v -  S_{c_p n}^X (\delta_k v)\|_{L_2(U,X;\mu)}
	\ \le \ C\, n^{- 1/q_2}
	\sum_{k=0}^{k_n}  k ^\tau
	\ \le \ C \,n^{- \alpha}(\log n )^{1 + \tau}.
\end{equation}
Summing up, we arrive at the inequalities
\begin{equation} \nonumber
	\begin{split}
		\|v - \Ss_n^X v\|_{L_2(U,X;\mu)}
		\ &\le \
		C \, n^{- \alpha}(\log n )^{\tau}, \ \ \text{if} \ \ \alpha < 1/q_2,
	\end{split}
\end{equation}	
and 
\begin{equation} \nonumber
	\begin{split}
		\|v - \Ss_n^X v\|_{L_2(U,X;\mu)}
		\ &\le \
		C \, n^{- \alpha}(\log n )^{1 + \tau}, \ \ \text{if} \ \  \alpha = 1/q_2.
	\end{split}
\end{equation}	

For the computational complexity $\operatorname{Comp} \brac{\Ss_{n}^X v}$  we have that
\begin{equation*} %\label{comp1}
	\begin{split}
		\operatorname{Comp} \brac{\Ss_{n}^X v}	
		\ \le 
		2\sum_{k=0}^{k_n}  2^kn_k
		\ & \le  \ 
		2 \sum_{k=0}^{k_n}  2^k (\xi_n 2^{-k})
		\ =  \
2 \xi_n	\sum_{k=0}^{k_n}  1
		\ \le  \
	 C n \log n.
	\end{split}
\end{equation*} 



 	 We now consider the case $ \alpha > 1/q_2 $.  In this case we have $\alpha > \beta$. For any $n \in \NN$, choose $\xi_n$ as a number satisfying the inequalities 
 	 \begin{equation*} %\label{[xi_n]2}
 	 	\xi_n^{q_1(\alpha + \delta)} \ \le \ n \ < \ 2\, \xi_n^{q_1(\alpha + \delta)}.
 	 \end{equation*}	 
 By  the  equivalence 
 $\xi_n \asymp n^{\frac{1}{q_1}\frac{1}{\alpha + \delta}}$, from \eqref{|v- P_{k_{xi_n}} v|}  we deduce that
 \begin{equation} \nonumber
 	\|v- P_{k_n} v\|_{L_2(U,X;\mu)}
 	\ \le \
 	2^{-\alpha \lfloor \log \xi_n \rfloor}\brac{\log \xi_n}^{\tau}
 	\ \le \	
 C\,  n^{- \beta}\brac{\log  n}^{\tau}.
 \end{equation}	
 	By  Lemma \ref{lemma:d_n(delta_kB_X)},
 there exist  $\by_{k,1},...,\by_{k,n_k} \in U$, $h_{k,1},...,h_{k,n_k} \in L_2(U,\CC;\mu)$ such that
 \begin{equation*} %\label{ineq2}
 	\begin{split}			
 		\|\delta_k v -  S_{n_k}^X (\delta_k v)\|_{L_2(U,X;\mu)}
 		\ &\le \ 
 		\norm{\delta_k v}{H_X^{\alpha,\tau}}	\brac{\frac{1}{n_k} \sum_{j \ge n_k} d_j^p}^{1/p},
 	\end{split}
 \end{equation*}
 where $d_j := d_j(\delta_k B_\CC^{\alpha,\tau},L_2(U,\CC;\mu))$.
 Hence, by Corollary \ref{corollary:d_n(delta_kB_CC)} and the inequality $p \beta >1$,
 we derive for $k \le k_n$,
 \begin{equation*} %\label{ineq2}
 	\begin{split}			
 		\|\delta_k v -  S_{n_k}^X (\delta_k v)\|_{L_2(U,X;\mu)}
 		\ &\le \ 
 		C\, 
 	(2^k n_k)^{-\beta} (k + \log n_k)^{\beta \tau /\alpha}
 			\ \le \
 	C\, n^{-\beta}(\log n)^{\beta \tau /\alpha}.
 	\end{split}
 \end{equation*}
 Hence,   we have
 \begin{equation} \nonumber
 	\sum_{k=0}^{k_n} 	\|\delta_k v -  S_{c_p n}^X (\delta_k v)\|_{L_2(U,X;\mu)}
 	\ \le \
 	\sum_{k=0}^{k_n} C \,  n^{- \beta}(\log n)^{\beta \tau /\alpha}
 	\ \le \ C \,n^{- \beta} (\log n)^{1 + \beta \tau /\alpha},
 \end{equation}
   Since $\tau = \frac{\alpha}{\alpha - \beta}$, we have $1 + \beta \tau /\alpha = \tau$.
 Hence,   we have
 \begin{equation} \nonumber
 	\sum_{k=0}^{k_n} 	\|\delta_k v -  S_{c_p n}^X (\delta_k v)\|_{L_2(U,X;\mu)}
 	\ \le \
 	\ C \,n^{- \beta} (\log n)^{\tau},
 \end{equation}
 Summing up, we arrive at the inequality
 \begin{equation} \nonumber
 	\begin{split}
 		\|v - \Ss_n^X v\|_{L_2(U,X;\mu)}
 		\ &\le \
 		C \, n^{- \beta} (\log n)^{\tau}, \ \ \text{if} \ \ \alpha  >  1/q_2.  
 	\end{split}
 \end{equation}	
For the computational complexity $\operatorname{Comp} \brac{\Ss_{n}^X v}$  we have that
\begin{equation*} %\label{comp1}
	\begin{split}
		\operatorname{Comp} \brac{\Ss_{n}^X v}	
		\ \le 
		2\sum_{k=0}^{k_n}  2^kn_k
		\ & \le  \ 
		2 \sum_{k=0}^{k_n} 2^k \brac{ n 2^{-k}}
		\ =  \
		2 n	\sum_{k=0}^{k_n}  1
		\ \le  \
		C n \log n.
	\end{split}
\end{equation*}  	 
	\hfill
\end{proof}

By  the continuous embedding  $ H_X^{\alpha} \hookrightarrow  H_X^{\alpha, 1+\varepsilon}$ for any $\varepsilon > 0$, Theorem \ref{thm:sampling} can be reformulated more conveniently  for the space $ H_X^{\alpha}$ as follows.

  \begin{theorem}\label{thm:sampling2}
  	Let $0 < q_1 \le q_2 <2$ and $\varepsilon$ be any positive number. Let the number $\beta$ be defined as in \eqref{delta, beta}.
 Let $(\xi_{n})_{n \in \NN}$ be a sequence of increasing positive numbers $\xi_n$ satisfying the condition \eqref{xi_n}. 
  	Assume that there is a constant $K_\varepsilon$ such that for every $v \in H_X^{\alpha,1+\varepsilon}$ and every $m \in \NN_0$, 
  	$\norm{P_m v}{H_X^{\alpha, 1+\varepsilon}} \le K_\varepsilon \norm{f}{H_X^{\alpha,1+\varepsilon}}$. 
  	Then for all $n \ge 2$ there exist  $\by_{k,1},...,\by_{k,n_k} \in U$ and $h_{k,1},...,h_{k,n_k} \in L_2(U,\CC;\mu)$, $k=0,...,k_{\lceil n/\log n \rceil}$,  such that for the operator $\bar{\Ss}^X_n:= \Ss_{\lceil n/\log n \rceil}^X$ defined as in \eqref{Ss_{n}^X}--\eqref{S_n_k}, 
  	\begin{equation*} \label{comp1}
  		\operatorname{Comp} \brac{\bar{\Ss}^X_n}	
  		\ \le  \
  		 n,
  	\end{equation*} 
  	 and
  	\begin{equation*} 	\label{L_2-rate}
  		\begin{split}
  			\sup_{v\in B_X^{\alpha}}		\|v-\bar{\Ss}^X_n v\|_{L_2(U,X;\mu)} 
%  			\ \le \
%  			\sup_{v\in B_X^{\alpha, 1+\varepsilon}}		\|v-\bar{\Ss}^X_n v\|_{L_2(U,X;\mu)} 
  			\ \le \
  			\begin{cases}
  			C_\varepsilon	n^{-\alpha}(\log n )^{\alpha + 1 +\varepsilon}\quad &{\rm if }  \ \alpha < 1/q_2,\\
  			C_\varepsilon	n^{-\alpha}(\log n )^{\alpha + 2 +\varepsilon}\quad &{\rm if }  \ \alpha = 1/q_2, \\
  			C	n^{-\beta}(\log n )^{\beta + \alpha/(\alpha - \beta)}\ \ \quad  &{\rm if }  \ \alpha > 1/q_2.
  			\end{cases}
  		\end{split}
  	\end{equation*}
  \end{theorem}
   
Next, we apply  Theorem \ref{thm:sampling2}
 to Bochner spaces with infinite tensor-product standard Gaussian  measure relevant to the applications  to  holomorphic functions and solutions to parametric PDEs with random inputs  in Sections 
 \ref{Applications to  holomorphic functions} and \ref{Applications to  parametric PDEs}, respectively.

 We recall a concept of
 standard Gaussian measure $\gamma(\by)$ on $\RRi$ as 
 the infinite tensor product of copies of the one-dimensional standard Gaussian
 measure $\gamma(y_i)$:
 \begin{equation} \nonumber
 	\gamma(\by) 
 	:= \ 
 	\bigotimes_{j \in \NN} \gamma(y_j) , \quad \by = (y_j)_{j \in \NN} \in \RRi.
 \end{equation}
 (The sigma algebra for $\gamma(\by)$ is generated by the set of cylinders $A:= \prod_{j \in \NN} A_j$, where $A_j \subset \RRR$ are univariate $\gamma$-measurable sets and only a finite number of $A_i$ are different from $\RRR$. For such a set $A$, we have $\gamma(A) = \prod_{j \in \NN} \gamma(A_j)$).
 
 Let $X$ be a separable Hilbert space. Then a function $v \in L_2(\RRi,X;\gamma)$  can be represented  by the Hermite GPC expansion
 \begin{equation} \label{GPCexpansion}
 	v=\sum_{\bs\in\FF} v_\bs \,H_\bs, \quad v_\bs \in X,
 \end{equation}
 with
 \begin{equation*}
 	H_\bs(\by)=\bigotimes_{j \in \NN} H_{s_j}(y_j),\quad 
 	v_\bs:=\int_{\RRi} v(\by)\,H_\bs(\by)\, \rd\gamma (\by), \quad 
 	\bs \in \FF.
 \end{equation*}
 Here $(H_s)_{s \in \NN_0}$  are the univariate orthonormal Hermite polynomials, $\FF$ is the set of all sequences of non-negative integers $\bs=(s_j)_{j \in \NN}$ such that their support 
 $\supp (\bs):= \{j \in \NN: s_j >0\}$ is a finite set.
 Notice that the complex-valued  $(H_\bs)_{\bs \in \FF}$ are an orthonormal basis of $L_2(\RRi,\CC;\gamma)$. 
 Moreover, for every $v \in L_2(\RRi,X;\gamma)$  represented by the 
 series \eqref{GPCexpansion},  Parseval's identity holds
 \begin{equation} \nonumber
 	\|v\|_{L_2(\RRi,X;\gamma)}^2
 	\ = \ \sum_{\bs\in\FF} \|v_\bs\|_X^2.
 \end{equation}
 
 
 Let $\bsigma=\brac{\sigma_{\bs}}_{\bs \in \FF}$ be a set of positive numbers strictly larger than $1$ such that 
 $\bsigma^{-1}:=\brac{\sigma_{\bs}^{-1}}_{\bs \in \FF} \in \ell_2(\FF)$. 
 Denote by $H_{X,\bsigma}$ the linear subspace in $L_2(\RRi,X;\gamma)$ of all $v$ such that the norm
 \begin{equation} \nonumber
 	\|v\|_{H_{X,\bsigma}}
 	:= \
 	\brac{\sum_{\bs \in \FF} \brac{\sigma_{\bs} \|v_\bs\|_X}^2}^{1/2} < \infty.		
 \end{equation}
 In particular, the space $H_{\CC,\bsigma}$ is the linear subspace in $L_2(\RRi,\CC;\gamma)$ equipped with its own inner product
 \begin{equation} \nonumber
 	\langle f,g \rangle_{H_{\CC,\bsigma}}
 	:= \
 	\sum_{\bs \in \FF} \sigma_{\bs}^2 
 	\langle f,H_\bs \rangle_{L_2(\RRi,\CC;\gamma)}
 	\overline{\langle g,H_\bs \rangle_{L_2(\RRi,\CC;\gamma)}}.
 \end{equation}
 The space $H_{\CC,\bsigma}$ is a reproducing kernel Hilbert space with the reproducing kernel
 \begin{equation} \nonumber
 	K(\cdot,\by)
 	:= \
 	\sum_{\bs \in \FF} \sigma_{\bs}^{-2} H_\bs (\cdot)\overline{H_\bs (\by)}
 \end{equation}
 with the eigenfunctions $\brac{H_\bs}_{\bs \in \FF}$ and the eigenvalues 
 $\brac{\sigma_{\bs}^{-1}}_{\bs \in \FF}$. Moreover, $K(\bx,\by)$ satisfies the finite trace assumption
 \begin{equation*}
 	\int_{\RRi} K(\bx,\bx) \rd \gamma(\bx) \ < \ \infty.
 \end{equation*}
 
 Let $(V_k)_{k \in \NN_0}$ be a given sequence of subspaces $V_k \subset X$ of dimension $2^k$, and $(P_k)_{k \in \NN_0}$  a given sequence of uniformly bounded linear projectors  $P_k$ from $X$ onto $V_k$. 
For $i=1,2$,  let $0< q_i \le 2$ and   $\bsigma_i:=(\sigma_{i;\bs})_{\bs \in \FF}$ be given sets of numbers strictly larger than $1$, such that $\bsigma_i^{-1}:=\brac{\sigma_{i,\bs}^{-1}}_{\bs \in \FF} \in \ell_{q_i}(\FF)$. 
 For   given numbers $\alpha> 0$ and  $\tau > 0$,  the linear subspaces  $ H_X^{\alpha}$ and $ H_X^{\alpha,\tau}$ in $L_2(\RRi,X;\gamma)$  are defined in the same manner as \eqref{H_{X}^{alpha}} and \eqref{H_{X}^{alpha,tau}}, respectively.
 
 Theorem \ref{thm:sampling2} for the space $L_2(\RRi,X;\gamma)$ is read as 
 \begin{theorem}\label{thm:samplingDD}
 	Let the assumptions of Theorem \ref{thm:sampling2} hold for the space $L_2(\RRi,X;\gamma)$.
 	Then for all $n \ge 2$ there exist  $\by_{k,1},...,\by_{k,n_k} \in \RRi$ and $h_{k,1},...,h_{k,n_k} \in L_2(\RRi,\CC;\gamma)$, $k=0,...,k_{\lceil n/\log n \rceil}$,  such that for the operator $\bar{\Ss}^X_n:= \Ss_{\lceil n/\log n \rceil}^X$ defined as in \eqref{Ss_{n}^X}--\eqref{S_n_k}, 
 	\begin{equation*} \label{comp1}
 		\operatorname{Comp} \brac{\bar{\Ss}^X_n}	
 		\ \le  \
 		n,
 	\end{equation*} 
 	and
 	\begin{equation*} 	\label{L_2-rate}
 		\begin{split}
 			\sup_{v\in B_X^{\alpha}}		\|v-\bar{\Ss}^X_n v\|_{L_2(\RRi,X;\gamma)} 
 			\ \le \
 			\begin{cases}
 				C_\varepsilon	n^{-\alpha}(\log n )^{\alpha + 1 +\varepsilon}\quad &{\rm if }  \ \alpha < 1/q_2,\\
 				C_\varepsilon	n^{-\alpha}(\log n )^{\alpha + 2 +\varepsilon}\quad &{\rm if }  \ \alpha = 1/q_2, \\
 				C	n^{-\beta}(\log n )^{\beta + \alpha/(\alpha - \beta)}\ \ \quad  &{\rm if }  \ \alpha > 1/q_2.
 			\end{cases}
 		\end{split}
 	\end{equation*}
 	\end{theorem}
 
 
 	
 	
 	\section{Applications to  holomorphic functions} 
\label{Applications to  holomorphic functions}

In this section, we apply the results on  multi-level linear sampling recovery in the space $L_2(\RRi,X;\gamma)$ to holomorphic functions.

We recall  a concept of  $(\bb,\xi,\delta,X)$-holomorphic
functions  which has been introduced in \cite[Definition 4.1]{DNSZ2023}. 
For $m\in\NN$ and a positive sequence $\bvarrho=(\varrho_j)_{j=1}^m$,  we put
\begin{equation*}
	\label{eq:Sjrho}
	\Ss(\bvarrho) := \set{\bz\in \CC^m}{|\mathfrak{Im}z_j| < \varrho_j~\forall j}\qquad\text{and}\qquad
	\Bb(\bvarrho) := \set{\bz\in\CC^m}{|z_j|<\varrho_j~\forall j}.
\end{equation*}

%
Let $X$ be a complex separable Hilbert space,
$\bb=(b_j)_{j\in\NN}$ a positive sequence, and $\xi>0$, $\delta > 0$.
For $m\in\NN$ we say that a positive sequence $\bvarrho=(\varrho_j)_{j=1}^m$ is
\emph{$(\bb,\xi)$-admissible} if
\begin{equation*}\label{eq:adm}
	\sum_{j=1}^m b_j\varrho_j\leq \xi\,.
\end{equation*}
A function $v\in L_2(\RRi,X;\gamma)$ is called
$(\bb,\xi,\delta,X)$-holomorphic if
%
\begin{enumerate}
	\item[{\rm (i)}]\label{item:hol} for every $m\in\NN$ there exists
	$v_m:\RR^m\to X$, which, for every $(\bb,\xi)$-admissible
	$\bvarrho$, admits a holomorphic extension
	(denoted again by $v_m$) from $\Ss(\bvarrho)\to X$; furthermore,
	for all $m<m'$
	\begin{equation*}\label{eq:un=um}
		v_m(y_1,\dots,y_m)=v_{m'}(y_1,\dots,y_m,0,\dots,0)\qquad\forall (y_j)_{j=1}^m\in\RR^m,
	\end{equation*}
	
	\item[{\rm (ii)}]\label{item:varphi} for every $m\in\NN$ there exists
	$\varphi_m:\RR^m\to\RR_+$ such that
	$\norm{\varphi_m}{L^2(\RR^m;\gamma)}\le\delta$ and
	%
	\begin{equation*} \label{ineq[phi]}
		\sup_{\text{$\brho$ is $(\bb,\xi)$-adm.}}~\sup_{\bz\in
			\Bb(\bvarrho)}\norm{v_m(\by+\bz)}{X}\le
		\varphi_m(\by)\qquad\forall\by\in\RR^m,
	\end{equation*}
	\item[{\rm (iii)}]\label{item:vN} with $\tilde v_m:\RRRi\to X$ defined by
	$\tilde v_m(\by) :=v_m(y_1,\dots,y_m)$ for $\by\in \RRRi$ it holds
	\begin{equation*}
		\lim_{m\to\infty}\norm{v-\tilde v_m}{L_2(\RRi,X;\gamma)}=0.
	\end{equation*}
\end{enumerate}

The following key result on  weighted $\ell_2$-summability of $(\bb,\xi,\delta,X)$-holomorphic functions has been proven in \cite[Corollary 4.9]{DNSZ2023}.

\begin{lemma} \label{lemma:weighted summability, holomorphic} Let $v$ be
	$(\bb,\xi,\delta,X)$-holomorphic for some $\bb\in \ell_p(\NN)$ with $0< p <1$.  Let $\eta\in\NN_0$ and let the sequence $\brho=(\rho_j)_{j \in \NN}$ be defined by
	$$
	\rho_j:=b_j^{p-1}\frac{\xi}{4\sqrt{\eta!}} \norm{\bb}{\ell_p(\NN)}.
	$$
	Assume that $\bb$ is a non-increasing sequence and that $b_j^{p-1}\frac{\xi}{4\sqrt{\eta!}} \norm{\bb}{\ell_p(\NN)}>1$ for all $j\in \NN$.
	Then we have 
		\begin{equation*} \label{ell_2-summability}
			\left(\sum_{\bs\in\FF} (\sigma_\bs \|v_\bs\|_{X})^2\right)^{1/2} \ \le M \ <\infty, \ \ \text{with} \ \
			\norm{\bsigma^{-1}}{\ell_q(\NN)} \le N < \infty,
		\end{equation*}	
		where $q := p/(1-p)$,	
	with $|\bs'|_{\infty}:= \sup_{j \in \NN} s'_j$ 	the set $\bsigma:=(\sigma_\bs)_{\bs \in \FF}$ is defined by
\begin{equation} \label{sigma_s}
	\sigma_{\bs}^2:=\sum_{|\bs'|_{\infty}\leq \eta}{\bs\choose \bs'} \prod_{j \in \NN}\rho_j^{2s_j'},
\end{equation}	
		$M= \delta C_{\bb,\xi,\eta}$ with some positive  constant $C_{\bb,\xi,\eta}$, and
		$N=  K_{\bb,\xi,\eta}$.
\end{lemma}

 This theorem allows us to apply weighted $\ell_2$-summability for collocation approximation of solutions  $u(\by)$  as $(\bb,\xi,\delta,X)$-holomorphic functions on various function spaces $X$, to a wide range of parametric and stochastic PDEs with log-normal inputs, specially,  Kondrat'ev spaces $X= K^r_\varkappa(D)$ for  the parametric elliptic PDEs \eqref{ellip} on bounded polygonal domain with log-normal inputs \eqref{lognormal} a detailed analysis of which will be presented in Section \ref{Applications to  parametric PDEs}.  For more information see \cite{DNSZ2023}.

For  a function $v$ defined on $\RRi$ taking values in a separable Hilbert space $X$, we say that $v$ satisfies Assumption~\ref{ass:ml} if

\begin{assumption}\label{ass:ml}
	$\alpha > 0$,  $\eta \in \NN$, $0<p_1\le p_2< 1$, $0<p_1 < 2/3$, $\bb_i\in\ell_{p_1}(\NN)$,
	$i=1,2$, are given sequences, $\xi>0$, $\delta>0$ such that
\begin{equation} \label{rho_{i,j}}
	\rho_{i,j}:=	b_{i,j}^{p_i-1}\frac{\xi}{4\sqrt{\eta!}} \norm{\bb_i}{\ell_{p_j}(\NN)}>1,  \ j\in \NN.
\end{equation}	
	  $\Vv=(V_k)_{k \in \NN_0}$  is a given sequence of subspaces $V_k \subset L_2(\RRi,X;\gamma)$ of dimension $2^k$, and $\Pp =(P_k)_{k \in \NN_0}$  a given sequence of linear uniformly bounded projectors  from $L_2(\RRi,X;\gamma)$ onto $V_k$ such that
	\begin{enumerate}
		\item $v\in L_2(\RRi,X;\gamma)$ and is $(\bb_1,\xi,\delta,X)$-holomorphic,
		%
		\item\label{item:u-ujbound} $(v- P_k v)\in L_2(\RRi,X;\gamma)$ and is
		$(\bb_1,\xi,\delta,X)$-holomorphic for every $k\in\NN_0$,
		\item\label{item:u-ujclose} $(v-P_k v)$ is
		$(\bb_2,\xi,\delta 2^{-{\alpha k}},X)$-holomorphic for every
		$k\in\NN_0$,
		\item\label{propertiesP_m}
		For every $m \in \NN_0$ and every linear bounded operator $Q$ in $L_2(\RRi,X;\gamma)$ with $\|Q\| \le K$, the function 
		$w:=Q v$ possesses the properties  (i)--(iii) with the parameter $\delta$ replaced by $K\delta$, the same parameters 
		$\alpha, \eta, p_1, p_2, \xi$ and sequences $\bb_1, \bb_2, \Vv, \Pp$.
	\end{enumerate}
\end{assumption}

For a space $H_X^{\alpha}$ with $\norm{\bsigma_i^{-1}}{\ell_q(\NN)} \le N$, $i=1,2$, denote
\begin{equation*} 
	B_X^{\alpha}(M,N):= \brab{v \in H_X^{\alpha,\tau}: \, \norm{v}{H_X^{\alpha,\tau}} \le M}.
\end{equation*}

\begin{lemma} \label{lemma:B_X, holomorphic} Let $H_X^{\alpha}$ and $H_X^{\alpha,\tau}$ with $\tau > 1$ be defined so that  for  $i=1,2$, $q_i := p_i/(1-p_i)$ and
	$\bsigma_i:=(\sigma_{i,\bs})_{\bs \in \FF}$ be given by \eqref{sigma_s} for $\brho_i$ as in \eqref{rho_{i,j}}, $\norm{\bsigma_i^{-1}}{\ell_{q_i}(\FF)} \le N$, respectively.  Then every function $v$ satisfying Assumption~\ref{ass:ml} belongs to 
	$	B_X^{\alpha}(M,N)$ for  $M= \delta C_{\bb,\xi,\eta}$ and $N=  K_{\bb,\xi,\eta,\tau}$ with some positive  constants $C_{\bb,\xi,\eta}$, and $K_{\bb,\xi,\eta}$. Moreover,
	$$
	\norm{P_m v}{H_X^{\alpha,\tau}} \le K_{\bb,\xi,\eta,\tau} M\norm{v}{H_X^{\alpha,\tau}}, \ \ m \in \NN_0.
	$$
\end{lemma}

\begin{proof}	
Assume that $v$ satisfies Assumption~\ref{ass:ml}. 	
 Then 
we have by Lemma \ref{lemma:weighted summability, holomorphic},
\begin{equation*} \label{ell_2-summability1}
	\left(\sum_{\bs\in\FF} (\sigma_{1,\bs} \|v_\bs\|_{X})^2\right)^{1/2} \ \le M, \ \
\text{with} \ \
\norm{\bsigma_1^{-1}}{\ell_q(\FF)} \le N < \infty,
\end{equation*}	
and  for every $k \in \NN_0$,
\begin{equation*} \label{ell_2-summability2}
	\left(\sum_{\bs\in\FF} (\sigma_{2,\bs} \|(v - P_k v)_\bs\|_{X})^2\right)^{1/2} \ \le M 2^{-\alpha k}\ <\infty, \ \ 
	\text{with} \ \
	\norm{\bsigma_2^{-1}}{\ell_q(\FF)} \le N < \infty,
\end{equation*}	
and in addition, for a every $m \in \NN_0$, 
\begin{equation*} \label{ell_2-summability1}
	\left(\sum_{\bs\in\FF} (\sigma_{1,\bs} \|(P_m)_\bs\|_{X})^2\right)^{1/2} \ \le \ K_{\bb,\xi,\eta}M, \ \
	\left(\sum_{\bs\in\FF} (\sigma_{2,\bs} \|(P_m - P_k (P_m))_\bs\|_{X})^2\right)^{1/2} 
	\ \le \ K_{\bb,\xi,\eta}M 2^{-\alpha k}.
\end{equation*}			
Hence,  $v \in B_X^{\alpha}(M,N)$.  Moreover,
$\norm{P_m v}{H_X^{\alpha,\tau}} \le K_{\bb,\xi,\eta,\tau}M\norm{v}{H_X^{\alpha,\tau}}$,  
$m \in \NN_0$, since  $\tau > 1$. 
	\hfill
\end{proof}


By applying Theorem \ref{thm:samplingDD}, from Lemma \ref{lemma:B_X, holomorphic}  we obtain

\begin{theorem}\label{thm:sampling-lognormal-holomorphicK} 	Let $v$ satisfy Assumption \ref{ass:ml}. 
	Let $q_i := p_i/(1-p_i)$ for $i=1,2$, and $\varepsilon$ be any positive number.
Let the number $\beta$ be defined as in \eqref{delta, beta}. Let $(\xi_{n})_{n \in \NN}$ be a sequence of increasing positive numbers $\xi_n$ satisfying the condition \eqref{xi_n}.
	Then for all $n \ge 2$ there exist  $\by_{k,1},...,\by_{k,n_k} \in \RRi$ and $h_{k,1},...,h_{k,n_k} \in L_2(\RRi,\CC;\gamma)$, $k=0,...,k_{\lceil n/\log n \rceil}$,  such that for the operator $\bar{\Ss}^X_n:= \Ss_{\lceil n/\log n \rceil}^X$ defined as in \eqref{Ss_{n}^X}--\eqref{S_n_k}, 
	\begin{equation*} \label{comp1}
		\operatorname{Comp} \brac{\bar{\Ss}^X_n}	
		\ \le  \
		n,
	\end{equation*} 
	and 
	\begin{equation*} 	\label{L_2-rate}
		\begin{split}
			\|v-\bar{\Ss}^X_n v\|_{L_2(\RRi,X;\gamma)} 
			\ \le \
			\begin{cases}
				C_\varepsilon	n^{-\alpha}(\log n )^{\alpha + 1 +\varepsilon}\quad &{\rm if }  \ \alpha < 1/q_2,\\
				C_\varepsilon	n^{-\alpha}(\log n )^{\alpha + 2 +\varepsilon}\quad &{\rm if }  \ \alpha = 1/q_2, \\
				C	n^{-\beta}(\log n )^{\beta + \alpha/(\alpha - \beta)}\ \ \quad  &{\rm if }  \ \alpha > 1/q_2.
			\end{cases}
		\end{split}
	\end{equation*}
	\end{theorem}	
	
\section{Multi-level sparse-grid interpolation algorithms}
\label{Multi-level sparse-grid interpolation algorithms}

In this section, we construct fully discrete multi-level sparse-grid  sampling algorithms
for the $(\bb,\xi,\delta,X)$-holomorphic functions satisfying Assumption \ref{ass:ml} except the item 4, based on GPC Lagrange-Hermite interpolation, and prove convergence rates of the approximation by them. It terns out that in the case $\alpha \le  1/q_2 - 1/2$,
these sampling algorithms give a convergence  rate better than the extended least squares sampling 
algorithm in Theorem \ref{thm:sampling-lognormal-holomorphicK}. The construction and techniques used in this section are a modification of those in \cite{Dung21,DD-Erratum23}.


For $m \in \NN_0$, let $Y_m = (y_{m;k})_{k \in \pi_m}$ be the increasing sequence of  the $m+1$ roots of the Hermite polynomial $H_{m+1}$, ordered as
$$
y_{m,-j} < \cdots < y_{m,-1} < y_{m,0} = 0 < y_{m,1} < \cdots < y_{m,j} \quad {\rm if} \  m = 2j,
$$
$$
y_{m,-j} < \cdots < y_{m,-1} < y_{m,1} < \cdots < y_{m,j} \quad {\rm if} \  m = 2j - 1,
$$
where 
$$
\pi_m:= 
\begin{cases}
	\{-j,-j+1,..., -1, 0, 1, ...,j-1,j \} \ & \ \text{if} \  m = 2j; \\
	\{-j,-j+1,...,-1, 1,...,j-1,j \} \ & \ \text{if} \  m = 2j-1.
\end{cases}
$$
(in particular, $Y_0 = (y_{0;0})$ with $y_{0;0} = 0$).

For  a function $v$ on $\RR$, taking values in a Hilbert space $X$ and $m \in \NN_0$, we define the   Lagrange interpolation operator $I_m$  by
\begin{equation*} %\label{I_(v)}
	I_m(v):= \ \sum_{k\in \pi_m} v(y_{m;k}) L_{m;k}, \quad 
	L_{m;k}(y) := \prod_{j \in \pi_m \ j\not=k}\frac{y - y_{m;j}}{y_{n;k} - y_{m;j}},
\end{equation*}		
(in particular, $I_0(v) = v(y_{0,0})L_{0,0}(y)= v(0)$ and $L_{0,0}(y)=1$). Notice that $I_m(v)$ is a function on $\RR$ taking values in $X$ and  interpolating $v$ at $y_{m;k}$, i.e., $I_m(v)(y_{m;k}) = v(y_{m;k})$.   

%Moreover, for a function $v: \, \RR \to \RR$, the function $I_m(v)$ is the Lagrange polynomial having degree 
%$\le m$, and that $I_m(\varphi) = \varphi$ for every polynomial $\varphi$ of degree $\le m$.
%
%Let
%\begin{equation} \label{lambda_m}
%	\lambda_m:= \ \sup_{\|v\|_{L_\infty^{\sqrt{g}}(\RR)} \le 1} \|I_m(v)\|_{L_\infty^{\sqrt{g}}(\RR)} 
%\end{equation}	
%be the Lebesgue constant. 
%It was proven in \cite{Mat94a,Mat94b,Sza97} that
%\begin{equation} \nonumber
%	\lambda_m
%	\ \le \
%	C(m+1)^{1/6}, \quad m \in \NN,
%\end{equation}
%for some positive constant $C$ independent of $m$ (with the obvious inequality $\lambda_0(Y_0) \le 1$). Hence, for every $\varepsilon > 0$, there exists a positive constant $C_\varepsilon \ge 1$ independent of $m$ such that
%\begin{equation} \label{ineq[lambda_m]2}
%	\lambda_m
%	\ \le \
%	(1 + C_\varepsilon m)^{1/6 + \varepsilon}, \quad \forall m \in \NN_0.
%\end{equation}

We define the univariate operator $\Delta_m$ for $m \in \NN_0$ by
\begin{equation} \nonumber
	\Delta_m
	:= \
	I_m - I_{m-1},
\end{equation} 
with the convention $I_{-1} = 0$. 

%\begin{lemma} \label{lemma[Delta_{bs}]}
%	For every $\varepsilon > 0$, there exists a positive constant $C_\varepsilon$ independent of $m$ such that for every function  $v$ on $\RR$,
%	\begin{equation} \label{ineq[Delta]1}
%		\|\Delta_m(v)\|_{L_\infty^{\sqrt{g}}(\RR)}	
%		\ \le \
%		(1 + C_\varepsilon m)^{1/6 + \varepsilon} \|v\|_{L_\infty^{\sqrt{g}}(\RR)}	, \quad \forall m \in \NN_0,
%	\end{equation}
%	whenever the norm in the right-hand side is finite. 
%\end{lemma}
%
%\begin{proof} 
%	From the assumptions we have that
%	\begin{equation} \nonumber
%		\|\Delta_m(v)\|_{L_\infty^{\sqrt{g}}(\RR)}	
%		\ \le \
%		2C(1 + m)^{1/6} \|v\|_{L_\infty^{\sqrt{g}}(\RR)}	, \quad \forall m \in \NN_0,
%	\end{equation}
%	which implies \eqref{ineq[Delta]1}.  
%	\hfill
%\end{proof}

%We will use a sparse-grid Lagrange GPC interpolation as an intermediate approximation in the deep ReLU neural network approximation of functions $v \in L_2(\RRi,X;\gamma)$. In order to have a correct definition of interpolation operator we have to impose some necessary restrictions on $v$. Let $\Ee$ be a $\gamma$-measurable subset in $U$ such that $\gamma(\Ee) =1$ and $\Ee$ contains  all $\by \in U$ with $|\by|_0 < \infty$ in the case $U=\RRi$, where $|\by|_0$ denotes the number of nonzero components $y_j$ of $\by$. For a given $\Ee$ and Hilbert space $X$, we define $L_2^\Ee(U,X,\gamma)$ as the subspace in $L_2(\RRi,X;\gamma)$  of all elements $v$ such that the point value $v(\by)$ (of a representative of $v$) is well-defined for all $\by \in \Ee$. In what follows, $\Ee$ is fixed.

For  a function $v$ on $\RRi$, taking values in a Hilbert space $X$, we introduce the tensor product operator $\Delta_\bs$, $\bs \in \FF$, by
\begin{equation*} %\label{Delta_bs(v)}
	\Delta_\bs(v)
	:= \
	\bigotimes_{j \in \NN} \Delta_{s_j}(v),
\end{equation*}
where the univariate operator
$\Delta_{s_j}$ is successively applied to the univariate function $\bigotimes_{i<j} \Delta_{s_i}(v)$ by considering it as a 
function of  variable $y_j$ with the other variables held fixed.
%From the definition of $L_2^\Ee(U,X,\gamma)$ one can see that the operators  $\Delta_\bs$ are well-defined for all $\bs \in \FF$. 
We define for $\bs \in \FF$,
\begin{equation} \nonumber
	I_\bs(v)
	:= \
	\bigotimes_{j \in \NN} I_{s_j}(v), \quad
	L_{\bs;\bk}
	:= \
	\bigotimes_{j \in \NN} L_{s_j;k_j}, \quad
	\pi_\bs
	:= \
	\prod_{j \in \NN} \pi_{s_j},
\end{equation}
(the operator  $I_\bs$ is defined in the same manner as $\Delta_\bs$).

For $\bs \in \FF$ and $\bk \in \pi_\bs$, let $E_\bs$ be the subset in $\FF$ of all $\be$ such that $e_j$ is either $1$ or $0$ if $s_j > 0$, and $e_j$ is $0$ if $s_j = 0$, and let $\by_{\bs;\bk}:= (y_{s_j;k_j})_{j \in \NN} \in \RRi$. Put $|\bs|_1 := \sum_{j \in \NN} s_j$ for  $\bs \in \FF$. It is easy to check that the interpolation operator $\Delta_\bs$ can be represented in the form
\begin{equation} \label{Delta_bs=}
	\Delta_\bs(v)				
	\ = \
	\sum_{\be \in E_\bs} (-1)^{|\be|_1} I_{\bs - \be} (v)
	\ = \
	\sum_{\be \in E_\bs} (-1)^{|\be|_1} \sum_{\bk \in \pi_{\bs - \be}} v(\by_{\bs - \be;\bk}) L_{\bs - \be;\bk}.
\end{equation}

For a given finite set $\Lambda \subset \FF$, we introduce the GPC interpolation operator $I_\Lambda$  by
\begin{equation*} %\label{I_Lambda}
	I_\Lambda
	:= \
	\sum_{\bs \in \Lambda} \Delta_\bs.
\end{equation*}
From \eqref{Delta_bs=} we obtain the sparse-grid interpolation sampling algorithm
\begin{equation} \label{I_Lambda=}
	I_\Lambda(v)				
	\ = \
	\sum_{\bs \in \Lambda} \sum_{\be \in E_\bs} (-1)^{|\be|_1} \sum_{\bk \in \pi_{\bs - \be}} v(\by_{\bs - \be;\bk}) L_{\bs - \be;\bk}.
\end{equation}

%A set $\Lambda \subset \FF$ is called downward closed if the inclusion $\bs \in \Lambda$ yields the inclusion $\bs' \in \Lambda$ for every $\bs' \in \FF$ such that $\bs' \le \bs$. 
%
%For $\theta, \lambda \ge 0$, we define the set $\bp(\theta, \lambda):= \brac{p_\bs(\theta, \lambda) }_{\bs \in \FF}$ by 
%\begin{equation} \label{[p_s]}
%	p_\bs(\theta, \lambda) := \prod_{j \in \NN} (1 + \lambda s_j)^\theta, \quad \bs \in \FF,
%\end{equation}
%with abbreviations $p_\bs(\theta):= p_\bs(\theta, 1)$ and $\bp(\theta):= \bp(\theta, 1)$. 


Next, we introduce  the multi-level sparse-grid interpolation  sampling algorithm  $\Ii_\xi$ for  $\xi >1$  by
\begin{equation}  \label{Ii_{xi}v}
	\Ii_{\xi} v 
	\ = \ 
	\sum_{k=0}^{\lfloor \log_2 \xi \rfloor} I_{\Lambda_k(\xi)}(\delta_k v).
\end{equation}
where 
\begin{equation} \nonumber
	\Lambda_k(\xi)
	:= \ 
	\begin{cases}
		\big\{\bs \in \FF: \,\sigma_{2;\bs} \leq \brac{2^{-k}\xi}^{1/q_2 }\big\} \quad &{\rm if }  \ 
		\alpha \le 1/q_2 - 1/2;\\
		\big\{\bs \in \FF: \, \sigma_{1;\bs} \le \xi^{1/q_1}, \  
		\sigma_{2;\bs} \leq 2^{- (\alpha +1/2) k}\xi^\vartheta \big\} \quad  & {\rm if }  \ 
		\alpha > 1/q_2 - 1/2,
	\end{cases}
\end{equation}
with $\vartheta:= 1/q_1  + (1/q_1 - 1/q_2)/(2\alpha)$.

Since $\delta_k v$ belongs to the subspace $V_k + V_{k-1}$ of dimension at most $2^k + 2^{k-1}$ and the number of sampling points in the sampling algorithms $ I_{\Lambda_k(\xi)}$ is 
$$
n_k:= 	\sum_{\bs \in \Lambda} \sum_{\be \in E_\bs}
|\pi_{\bs - \be}|,
$$ 
the computational complexity 
$\operatorname{Comp} \brac{\Ii_{\xi}}$ of the operator  $\Ii_{\xi}$ can be defined as 
\begin{equation}  \nonumber
	\operatorname{Comp} \brac{\Ii_{\xi}}
	:= \ 
	\sum_{k=0}^{\lfloor \log_2 \xi \rfloor}  \brac{2^k + 2^{k-1}}n_k,
\end{equation}

\begin{theorem}\label{thm:fully-discrete-lognormal-holomorphic} 
	Let $v$ satisfy Assumption \ref{ass:ml} except the item 4.  	Let $q_i := p_i/(1-p_i)$ for $i=1,2$. 	Let	
	\begin{equation*} 	%\label{beta}
		\beta := \brac{\frac 1 {q_1} - \frac{1}{2}}\frac{\alpha}{\alpha + \delta}, \quad 
		\delta := \frac 1 {q_1} - \frac 1 {q_2}.
	\end{equation*}		
	Then for each $n \in \NN$ there exists a number $\xi_n$  such that  
	\begin{equation*} \label{comp1}
		\operatorname{Comp} \brac{\Ii_{\xi_n}}	
		\ \le  \
		n, 
	\end{equation*} 
	and
	\begin{equation*} 	\label{L_2-rate}
		\begin{split}
			\|v- \Ii_{\xi_n} v\|_{L_2(\RRi,X;\gamma)} 
			\ \le \	C
			\begin{cases}
				n^{-\alpha}\quad &{\rm if }  \ \alpha \le  1/q_2 - 1/2,\\
				n^{-\beta}  \quad  & {\rm if }  \ \alpha > 1/q_2 - 1/2,
			\end{cases}
		\end{split}
	\end{equation*}
where the constant $C$ is independent of $v$ and $n$.
\end{theorem}	
\begin{proof}
	The operator $\Ii_{\xi}$ can be rewritten in the form
	\begin{equation} \label{Ii_xi= Ii_{G(xi)}}
		\Ii_\xi  \ = \ \Ii_{G(\xi)}  
		:= \
		\sum_{(k,\bs) \in G(\xi)} \Delta^{{\rm I}}_\bs \delta_k,
	\end{equation}
	where
	\begin{equation*} %\label{G_(xi)}
	G(\xi)
	:= \ 
	\begin{cases}
		\big\{(k,\bs) \in \NN_0 \times\FF: \, 2^k \sigma_{2;\bs}^{q_2} \leq \xi\big\} 
		\quad &{\rm if }  \ \alpha \le 1/q_2 - 1/2;\\
		\big\{(k,\bs) \in \NN_0 \times\FF: \, \sigma_{1;\bs}^{q_1} \le \xi, \  
		2^{(\alpha+1/2)  k} \sigma_{2;\bs}\leq \xi^\vartheta \big\} 
		\quad  & {\rm if }  \ \alpha > 1/q_2 - 1/2.
	\end{cases}
\end{equation*}	
On the other hand, it holds the following statement which can be proven in a way similar to the proof of \cite[Theorem 3.8]{Dung21} (see also \cite{DD-Erratum23} for some proof corrections) with some slight modification.  For each $n \in \NN$ there exists a number $\xi_n$  such that  
\begin{equation*} \label{comp1}
	\operatorname{Comp} \brac{\Ii_{G(\xi_n)}}	
	\ \le  \
	n, 
\end{equation*} 
and
\begin{equation*} 	\label{L_2-rate}
	\begin{split}
		\|v- \Ii_{G(\xi_n)} v\|_{L_2(\RRi,X;\gamma)} 
		\ \le \	C
		\begin{cases}
			n^{-\alpha}\quad &{\rm if }  \ \alpha \le  1/q_2 - 1/2,\\
			n^{-\beta}  \quad  & {\rm if }  \ \alpha > 1/q_2 - 1/2,
		\end{cases}
	\end{split}
\end{equation*}
where the constant $C$ is independent of $v$ and $n$. From this statement and \eqref{Ii_xi= Ii_{G(xi)}} we prove the theorem.
	\hfill
\end{proof}	



\section{Applications to  parametric elliptic PDEs} 
\label{Applications to  parametric PDEs} 
In this section, we prove that the parametric solution $u$ to
the parametric elliptic PDEs \eqref{ellip} on a bounded polygonal domain with log-normal inputs \eqref{lognormal} is a $(\bb_j,\xi,\delta,V)$-holomorphic function on $\RRi$ satisfying  Assumption \ref{ass:ml} with $\bb_j$ as in \eqref{eq:b1b2ml}, $j=1,2$. This allows to establish convergence rates of 
a fully discrete multi-level collocation  algorithm $S_n$.	The spatial components in $S_n$  are based on finite element Lagrange interpolations associated with triangulations of the spatial domain $D$. While   the parametric components  are based on Hermite-Lagrange GPC interpolation as in Theorem \ref{thm:sampling-HermiteInterpolation} in the case of small spatial regularity $\alpha \le 1/p_2 - 3/2$, and on  extended least squares sampling algorithms as in Theorem \ref{thm:sampling-lognormal-holomorphicK} in the cases of higher spatial regularity $\alpha > 1/p_2 - 3/2$.

It is convenient to consider the extended equation \eqref{ellip} with complex-valued data $a$ and $f$. We rewrite the solution to the equation \eqref{ellip} as the  mapping $a \mapsto \Uu(a)$ from $\Ww^{r-1}_\infty(D)$ to $\Kk_{\varkappa+1}^{r}(D)$ satisfying the equation
\begin{equation}\label{eq:ellipticK}
	- \div(a \nabla \Uu(a))=f\quad\text{in }D,\qquad
	\Uu(a) =0\quad\text{on }\partial D.
\end{equation}

In what follows, we assume that
\begin{equation*}
	\rho(a) := \underset{\bx\in D}{\operatorname{ess\,inf}}\, \Re (a(\bx)) > 0,
	\ \ \ \text{and} \ \ \
	|\varkappa|<\frac{\rho(a)}{\nu \norm{a}{L_\infty(D)}},
\end{equation*}
where $\nu$ is a constant depending on $D$ and $r$.  
The following result has been proven in \cite[Theorem 7.8]{DNSZ2023}
\begin{lemma}\label{lemma:bacuta2}
	Let $D\subset\RR^2$ be a bounded polygonal domain and
	$r\in\NN$, $r\ge 2$.  Then there exist $\varkappa>0$ and $C_r>0$
	depending on $D$ and $r$ such that for all
	$a\in W^{1}_\infty(D)\cap\Ww_{\infty}^{r-1}(D)$ and all
	$f\in \Kk_{\varkappa-1}^{r-2}(D)$ the weak solution
	$\Uu\in H_0^1(D)$ of \eqref{eq:ellipticK} belongs to the space $\Kk_{\varkappa+1}^{r}(D)$ and  satisfies with
	$N_r:=\frac{r(r-1)}{2}$
	\begin{equation*}%\label{eq:Uuapriori}
		\norm{\Uu}{\Kk_{\varkappa+1}^{r}(D)}
		%  
		\le C_r \frac{1}{\rho(a)}\left(\frac{\norm{a}{\Ww^{r-1}_{\infty}(D)}+\norm{a}{W^{1}_\infty(D)}}{\rho(a)}\right)^{N_r}\norm{f}{\Kk_{\varkappa-1}^{r-2}(D)}.
	\end{equation*}
\end{lemma}

We recall a concept of Lagrange finite element. Let $r \in \NN$ be given and $\Tt$  a triangulation of  $D$ with triangles $T$. Let $V(\Tt)$  be the finite element space which consists  of those continuous functions on $D$ that restrict to polynomials of order  at most $r$ on each triangle $T \in \Tt$.  For any function $v \in C(D)$, define $I v :=I(\Tt) v$ as the interpolating function associated to $v$, the interpolation nodes $\bx_1,...,\bx_{m(\Tt)} \in D$ being obtained by taking points with baricentric coordinates $r^{-1}\ZZ$. The operator $I$ is called Lagrange interpolation operator associated with $\Tt$. Thus, for any continuous function $v \in C(\bar{D)}$, the operator $I$ is uniquely determined by the conditions that 
$I v(\bx_i)= v(\bx_i)$ for any interpolation node $\bx_i$, $i=1,...,m(\Tt)$, and $I v \in V(\Tt)$. From the definitions we can see that
\begin{equation} \label{I v}
	(I v)(\bx): = \sum_{i=1}^{m(\Tt)}  v(\bx_i) \phi_{i}(\bx),
\end{equation}
where $\phi_{i}$, $i=1,...,m(\Tt)$, are the nodal basic of $V(\Tt)$. If $\Sigma:= \brab{\sigma_1,...,\sigma_{m(\Tt)}}$ are the linear forms such that $\sigma_i(v):= v(\bx_i)$, $i=1,...,m(\Tt)$, for $v \in V(\Tt)$,  then the triple $(D,V(\Tt),\Sigma)$ is a Lagrange finite element.
%\begin{equation}\label{dimV}
%	\dim V(\Tt) \ = \ m(\Tt).
%\end{equation}	

We consider a special class of the so-called shape-regular triangulation  $\Tt$ which satisfies the 
condition $\frac{R_{T}}{r_{T}}\le c$ for every $T \in \Tt$, 
where $c$ is a constant  and $r_{T}$ 
($R_{T}$) is the radius of the largest (smallest) ball contained in (containing) $T$. The following result is well-known (see, e.g., \cite{Brenner,Ciarlet}).

\begin{lemma}\label{lemma:InterpolationApprox}
	Let $r \in \NN$ and $I:= I(\Tt)$, $m:= m(\Tt)$.
Assume that  the triangulation  $\Tt$ is shape-regular.  Then there exists a constant $C = C_{r,a,c}$  such that  for any $v \in H^{r}(D)$
	\begin{equation*} %\label{norm{v- I v}{H^1}}
		\norm{v- I v}{H^1(D)}
		\ \le \
	C m^{- \frac{r-1}{2}}	{\norm{v}{H^{r}(D)}}.	
	\end{equation*}
	%	and 
	%		\begin{equation}\label{}
		%	\sup_{k\in \NN_0}	\ \sup_{0\neq u\in \Kk_{\varkappa+1}^{s}}
		%		\frac{\norm{P_k u}{V}}{\norm{u}{\Kk_{\varkappa+1}^{s}}}
		%		\ \le \ K.
		%	\end{equation}
\end{lemma}





For the reader's convenience, we recall the definition of  the class $\Cc$  of   triangulations $\Tt$ of $D$ introduced  in \cite[Definition~4.1]{BNZ2005}. 
For simplicity of presentation we assume that $D$ is a triangle all of whose angles are acute. The general case can be considered in a similar way by first dividing $D$ in triangles with acute angles. We also assume that  $D$ is an open set. Let $l$ be the length of shortest edge of $D$. Let $D_\delta$ be the union of three isosceles triangles that have equal sides of length $\delta$. The complement $D \setminus D_\delta$ is a hexagon when $\delta < l$.
Fix $r \in \NN$ and let $\varkappa \in (0,1]$, $h >0$, $\epsilon, b \in (0,1)$, and $a \in (0,\pi/2)$ be parameters. We define $\Cc:= \Cc(r,h,\varkappa,\epsilon, a,b)$ to be the set of triangulations $\Tt$ defined as follows. Choose $m$ such that 
$$
\epsilon^{\varkappa m} 
\ \le \ M(l,\epsilon l /8,a) h^r.
$$
We decompose $D$ as the union of  $\Omega_0:= D \setminus D_{l/4}$, 
$\Omega_1:= D_{l/4} \setminus D_{\epsilon l/4}$,..., 
$\Omega_m:= D_{\epsilon^{m-1} l/4}\setminus D_{\epsilon^m l/4}$, and  
$\tilde{\Omega}_{m+1}:= D_{\epsilon^m l/4}$.
For each $j = 0,...,m$, we triangulate $\Omega_j$ with triangles  with all angles $\ge a$, and edges of length at most 
$$
h_{m,j}:= h \epsilon^{(1 - \varkappa/r)j}
$$
and at least $b h_{m,j}$. Then $\Tt$ is the union of the triangles appearing in the triangulations $\Omega_j$, $j \le m$, and of three triangles forming $\tilde{\Omega}_{m+1}$. 

Notice that the finite element space $V(\Tt)$ with $\Tt \in \Cc$, $r$ is the degree of the polynomials used in the approximation, $h$ is the largest admissible length of the sides of the triangles in the partition, $\epsilon$ controls the decay of the triangles as they approach a vertex, $a$ is the minimum admissible angle of a triangle in the partition , $0< \varkappa < \pi/a_1$, where $a_1$ is the largest angle of the polygon, and $b$ controls the ratio of the sizes of close triangles. The constants $r,h,\varkappa,\epsilon,a,b$ must satisfy certain conditions for the class $\Cc:= \Cc(r,h,\varkappa,\epsilon,a,b)$ to be non-empty. 
The following result \cite[Theorem 0.3]{BNZ2005} is therefore relevant.
For any polygon $D$, there exist $0 < \varkappa  \le 1$, $a  > 0$, $1 > b > 0$ and a sequence $h_k = h_0 2^{-k}$ such that, 
if $\epsilon = 2^{-r/\varkappa}$, the class $\Cc_k:= \Cc(r,h_k,\varkappa,\epsilon, a,b)$ is not empty, where $h_0 > 0$ is a certain constant.

If $\Tt_k \in \Cc_k$, there are positive constants $C$ and $C'$ such that 
\begin{equation*} %\label{h_n}
	C n \le h_n^{-2} \le C' n, \ \ \dim V_k = m_k \le 2^k.
\end{equation*}	
where we denote $V_k:= V(\Tt_k)$ and $m_k:= m(\Tt_k)$. 

Let $\Tt_k \in \Cc_k$,  $k \in \NN_0$.   Let   $P_k :=I(\Tt_k)$ be  the Lagrange interpolation operator  defined as in~\eqref{I v}:
\begin{equation} \label{P_k}
	(P_k v)(\bx): = \sum_{i_k=1}^{m_k}  v(\bx_{k,i_k}) \phi_{k,i_k}(\bx),
\end{equation}
 where  $\bx_{k,1},...,\bx_{k,m_k} \in D$  are  interpolation nodes and $\phi_{k,1},...,\phi_{k,m_k} \in C(D)$  are the nodal basis of $V_k$ whose restriction to every $T \in \Tt_k$ is a polynomial of degree at most $r$.
%let $(X_k)_{k\in\NN}$, $X_k:=V_{2^{k+k_0}}$,  be  the associated finite element spaces  $(P_k)_{k\in\NN}$,  $P_k:=I_{2^{k+k_0}}$,  the associated  linear interpolation operators. 
From \cite[Theorem~4.4]{BNZ2005} one can derive the following

\begin{lemma}\label{lemma:FEM}
Let $r\in\NN$. Then there exists a constant $C>0$  such that 
 we have for every $k \in \NN_0$,
	\begin{equation*}%\label{eq:conv}
			\sup_{0\neq v\in \Kk_{\varkappa+1}^{r}(D)}
		\frac{\norm{v- P_k v}{\Kk^1_1(D)}}{\norm{v}{\Kk_{\varkappa+1}^{r}(D)}}
		\ \le \ C 2^{- \frac{r-1}{2} k}.
	\end{equation*}
\end{lemma}


Throughout the rest of this section,  we consider the equation \eqref{eq:ellipticK} with 
 the parametric diffusion coefficient \eqref{lognormal} satisfying  the condition
$\psi_j\in W^{1}_\infty(D)\cap\Ww_\infty^{r-1}(D)$, $j\in\NN$, and $f\in \Kk_{\varkappa-1}^{r-2}(D)$. Denote
\begin{equation}\label{eq:b1b2ml}
b_{1,j}:=\norm{\psi_j}{L^\infty},\quad
b_{2,j}:=\max\big\{\norm{\psi_j}{W^{1}_\infty(D)},\norm{\psi_j}{\Ww_\infty^{r-1}(D)}\big\}
\end{equation}
and $\bb_1:=(b_{1,j})_{j\in\NN}$, $\bb_2:=(b_{2,j})_{j\in\NN}$.

%
%Let $D=[0,1]$ and $\psi_j(x)=\sin(jx)j^{-r}$ for some $r>2$. Then
%$\bb_1\in\ell^{p_1}(\NN)$ for every $p_1>\frac{1}{r}$ and
%$\bb_2\in\ell^{p_2}(\NN)$ for every $p_2>\frac{1}{r-(s-1)}$.
%
%  

\begin{lemma}\label{lemma:holomorphyKspace}
	Let $D\subset\RR^2$ be a bounded polygonal domain and
	$r\in\NN$, $r \ge  2$.  Let
	$\alpha = \frac{r-1}{2}$. Let 
	$\psi_k\in W^{1}_\infty(D)\cap\Ww_{\infty}^{r-1}(D)$, $k \in \NN$, and 
	$f\in \Kk_{\varkappa-1}^{r-2}(D)$. Let $0< p_1,  p_2 < 1$.
	For $i=1,2$, let the sequences $\bb_i:=(b_{i,j})_{j\in\NN}$ be defined as in \eqref{eq:b1b2ml} with
	$\bb_i\in\ell^{p_i}(\NN)$. 
Then there exist $\xi>0$ and $\delta>0$ such that for the parametric solution to \eqref{eq:ellipticK} with inputs \eqref{lognormal}
\begin{equation*}%\label{eq:uby}
	u(\by):=\Uu\Bigg(\exp\Bigg(\sum_{j\in\NN}y_j\psi_j\Bigg)\Bigg)
\end{equation*}
and for every $k\in\NN$,
\begin{enumerate}
	\item[{\rm (1)}]
	$u$ is
	$(\bb_1,\xi,\delta,V)$-holomorphic,
	\item[{\rm (2)}]
	$P_k u$ is
	$(\bb_1,\xi,\delta,V)$-holomorphic,
	\item[{\rm (3)}]
	 $u-P_k u$ is
	$(\bb_1,\xi,\delta,V)$-holomorphic,
	\item[{\rm (4)}]
	 $u-P_k u$ is
	$(\bb_2,\xi,\delta 2^{-\alpha k},V)$-holomorphic,
		\item[{\rm (5)}]
	For every $m \in \NN_0$,  the function 
	$w:=P_m u$ possesses the properties  {\rm (1)--(4)} with the parameter $\delta$ replaced by $C\delta$, the same parameters 
	$\alpha, \eta, p_1, p_2, \xi$ and sequences $\bb_1, \bb_2$.
\end{enumerate}
\end{lemma}
 \begin{proof}	The proof of this lemma is a modification of the proof of 
 	\cite[Proposition 7.12]{DNSZ2023}. We  apply \cite[Theorem 4.11]{DNSZ2023} on holomorphy of composite functions with
 	$E=L^\infty(D)$ and $X=V$ to prove the claims  {\rm (1)--(5)}. 
 	
 		\noindent		
	{\bf Step 1.} 	We verify the claim (1). By \cite[Proposition 7.12]{DNSZ2023}  there exist $\xi_1>0$ and $\delta_1>0$ such that $u(\by)$
	is $(\bb_1,\xi_1,\delta_1,V)$-holomorphic on the open set
		$$
		O_1=\set{a\in L^\infty(D;\CC)}{\rho(a)>0} \subset L^\infty(D; \CC)
		$$  
for some constants $\xi_1>0$ and $\delta_1>0$ depending on $O_1$. 
		
			\noindent		
	{\bf Step 2.} 
	Concerning the claim (2),
	by assumption, $b_{1,j}=\norm{\psi_j}{L^\infty}$ satisfies
	$\bb_1=(b_{1,j})_{j\in\NN}\in\ell^{p_1}(\NN )\subseteq
	\ell^1(\NN)$, which corresponds to the assumption (iv) of
	Theorem \cite[Theorem 4.11]{DNSZ2023}. It remains to verify  the
	assumptions (i), (ii) and
	(iii) of  \cite[Theorem 4.11]{DNSZ2023} for $P_k u$, $k \in \NN_0$.
	
\noindent
		(i) \  
		$P_k\Uu:O_1\to V$ is holomorphic since  the map
		$a\mapsto P_k\Uu(a)$ is a composition of holomorphic
		functions. 
		
\noindent		
		(ii)\  
		For all $a\in O_1$, we have  by Lemma \ref{lemma:InterpolationApprox} and 
		the well-known estimate 
		$\norm{\Uu(a)}{V} \le \frac{\norm{f}{V'}}{\rho(a)}$,
		\begin{equation} \label{norm2}
			\begin{split}		
			\norm{P_k\Uu(a)}{V}
			\ &\le \
			\norm{\Uu(a)}{V} + \norm{\Uu(a)- P_k\Uu(a)}{V}
			\\
			\ &\le \
			C\norm{\Uu(a)}{V}
		\	\le \ C \frac{\norm{f}{V'}}{\rho(a)}
			= C \frac{\norm{f}{H^{-1}(D)}}{\rho(a)}.				
			\end{split}
		\end{equation}
		
		\noindent		
		(iii)\  
	 For all $a$, $b\in O_1$ we have  by Lemma \ref{lemma:InterpolationApprox} and \cite[(4.21)]{DNSZ2023},
		\begin{equation} \label{norm}
			\norm{P_k\Uu(a)-P_k\Uu(b)}{V}
			\le 
			C \norm{\Uu(a)-\Uu(b)}{V}\le \norm{f}{H^{-1}(D)}
			\frac{1}{\min\{\rho(a),\rho(b)\}^2}
			\norm{a-b}{L^\infty}.
		\end{equation}
	According to \cite[Theorem 4.11]{DNSZ2023} 
	the map
	$$\Uu \mapsto P_k\Uu\in  L^2(U,V;\gamma)$$ is
	$(\bb_1,\xi_2,\delta_2,V)$-holomorphic, for some fixed
	constants $\xi_2>0$ and $\delta_2>0$ depending on $O_1$ but	independent of $k$.
	
		\noindent		
	{\bf Step 3.}	 
	The claim (3) follows directly from Steps 1 and 2 that the
	difference $u - P_k u$ is $(\bb_1,\xi_3,\delta_3,V)$-holomorphic for some constants $\xi_3>0$ and $\delta_3>0$ depending on $O_1$ but	independent of $k$.
	
		\noindent		
	{\bf Step 4.} To show the claim (4), we set
	$$O_2=\set{a\in W^{1}_\infty(D)\cap\Ww^{s-1}_\infty(D)}{\rho(a)>0},$$ and verify again the
	assumptions (i), (ii) and
	(iii) of  \cite[Theorem 4.11]{DNSZ2023} with
	$E = W^{1}_\infty(D)\cap\Ww^{s-1}_\infty(D)$. First, observe
	that with
	$$b_{2,j}:=\max\big\{\norm{\psi_j}{\Ww^{s-1}_\infty(D)},\norm{\psi_j}{W^{1}_\infty}(D)\big\},$$
	by assumption
	$$\bb_2=(b_{2,j})_{j\in\NN}\in\ell^{p_2}(\NN)\hookrightarrow \ell^1(\NN)$$
	which corresponds to  the assumption (iv) of
	Theorem \cite[Theorem 4.11]{DNSZ2023}.
	
	\noindent
	(i) \  
	 For every $k\in\NN$, the mapping $\Uu- P_k\Uu:O_2\to V$ is holomorphic: Since $O_2$ can
		be considered a subset of $O_1$ (and $O_2$ is equipped with a
		stronger topology than $O_1$), Fr\'echet differentiability
		follows by Fr\'echet differentiability of
		$$\Uu- P_k\Uu: O_1\to V,$$ which holds by Step 3.
		
		\noindent
		(ii) \   For every $a\in O_2$, by 	the embedding inequality \eqref{EmbeddingInequality} and Lemmata  \ref{lemma:FEM} and \ref{lemma:bacuta2},
		\begin{equation*} 
			\norm{(\Uu- P_k\Uu)(a)}{V}
			\le  \delta_k
			\frac{(\norm{a}{W^{1}_\infty(D)}+\norm{a}{\Ww^{r-1}_{\infty}(D)})^{N_r+1}}{\rho(a)^{N_r+2}},
		\end{equation*}	
		where $\delta_k:= K 2^{-\alpha k}  \norm{f}{\Kk_{\varkappa-1}^{r-2}}$.
		
		\noindent
		(iii) \   For every $a$, $b\in O_2\subseteq O_1$, by \eqref{norm} and \cite[(4.21)]{DNSZ2023}, 
		\begin{equation} \label{norm1}
			\begin{aligned}
			\norm{(\Uu- P_k\Uu)(a)-(\Uu- P_k\Uu)(b)}{V} &\le
			\norm{\Uu(a)-\Uu(b)}{V}
			+\norm{P_k\Uu(a)- P_k\Uu(b)}{V}\nonumber\\
			&\le
			C \norm{f}{H^{-1}(D)}
			\frac{2}{\min\{\rho(a),\rho(b)\}^2}
			\norm{a-b}{L_\infty(D)}. 
		\end{aligned}
	\end{equation}	
		We conclude with \cite[Theorem 4.11]{DNSZ2023} 
		that there exist $\xi_4$ and $C_4$
		depending on $O_2$, $D$ but independent
		of $k$ such that $u-P_k u$ is
		$(\bb_2,\xi_4, C_4\delta_k,V)$-holomorphic.

	
	Summing up, the claims (1)--(4) hold with
	\begin{equation*}
		\xi:=\min\{\xi_1,\xi_2,\xi_3,\xi_4,\}\qquad\text{and}\qquad
		\delta:=\max\Big\{\tilde C_1, C_2, C_3, C_4 K\norm{f}{\Kk_{\varkappa-1}^{r-2}}\Big\}.
	\end{equation*}
	
		\noindent		
		{\bf Step 5.} The claim (5) can be proven with the same arguments as ones in Steps 1--4 by using  the inequality  \eqref{norm2}:
		\begin{align*}
			\norm{(P_m\Uu)(a)}{V}
			\ \le \
			C \norm{\Uu(a)}{V}
		\end{align*}
		for every $m \in \NN_0$ and every $a \in O_2$,
	\hfill
\end{proof}

Let $\Tt_k \in \Cc_k$. Recall that we denote $V_k:= V(\Tt_k)$ and $m_k:= m(\Tt_k)$ and  
  $P_k :=I(\Tt_k)$   the interpolation operator defined as in \eqref{P_k}.
%\begin{equation} \label{P_kTt_k}
%	(P_k v)(\bx): = \sum_{i_k=1}^{m_k}  v(\bx_{k,i_k}) \phi_{k,i_k}(\bx), \ \ v \in C(D),
%\end{equation}
%associated with $\Tt_k$, where  $\bx_{k,1},...,\bx_{k,2^k} \in D$  are  interpolation points and functions $\phi_{k,1},...,\phi_{k,2^k} \in C^{r-1}(D)$ whose restriction to every $T \in \Tt_k$ is a polynomial of degree at most $r$, 
Notice that there are positive constants $C$ and $C'$ such that 
\begin{equation*}
	C n \le h_n^{-2} \le C' n, \ \ \dim V_k = m_k \le 2^k.
\end{equation*}	
 Let the operators $\delta_k$, $k \in \NN_0$, be defined in \eqref{delta_k} for  the sequence $(P_k)_{k\in\NN}$. Then we have 
 \begin{equation*} 
 	(\delta_k v)(\bx): = \sum_{i_k=1}^{m_k}  v(\bx_{k,i_k}) \phi_{k,i_k}(\bx) 
 	- \sum_{i_{k-1}=1}^{m_{k-1}}  v(\bx_{k-1,i_{k-1}}) \phi_{k-1,i_{k-1}}(\bx), \ \ v \in C(D),
 \end{equation*}
which can be rewritten as 
\begin{equation} \label{delta_k}
	(\delta_k v)(\bx): = \sum_{i_k=1}^{\bar{m}_k}  v(\bar{\bx}_{k,i_k}) \bar{\phi}_{k,i_k}(\bx), \ \ v \in C(D),
\end{equation}
where $\bar{m}_k:= m_k + m_{k-1}$; 
$\bar{\bx}_{k,i_k} := \bx_{k,i_k}$, 
$\bar{\phi}_{k,i_k}:= \phi_{k,i_k}$  for $i_k= 1,..., m_k$, and 
$\bar{\bx}_{k,i_k} := \bx_{k-1,i_{k-1}}$,
 $\bar{\phi}_{k,i_k}:= - \phi_{k-1,i_{k-1}}$ for $i_k= m_k + 1,..., \bar{m}_k$. 
 
 Recall that for $n_k \in \NN_0$,  the operator
 \begin{equation*} 
 	S_{c_p n_k}^V  v : = \sum_{j_k=1}^{c_p n_k}  v\brac{\by_{k,j_k}} h_{k,j_k}
 \end{equation*}
 has been defined in \eqref{S_n_k} with $X=V$ for  $\by_{k,1},...,\by_{k,n_k} \in U$ and  $h_{k,1},...,h_{k,n_k} \in L_2(\RRi,\CC;\gamma)$,
 and $c_p > 0$ is the constant as in Lemma \ref{lemma:sampling inequality}.  
Then  for all $n \ge 2$, with $\delta_k$ as in \eqref{delta_k} the operator $\bar{\Ss}^V_n:= \Ss_{\lceil n/\log n \rceil}^V$ defined as in \eqref{Ss_{n}^X} and \eqref{S_n_k} can be rewritten as
\begin{equation}  \label{barSs_{n}^V}
	\bar{\Ss}^V_n v
	\ = \ 
	\sum_{k=0}^{k_{\lceil n/\log n \rceil}} 
	\sum_{i_k=1}^{\bar{m}_k} \sum_{j_k=1}^{c_p n_k} v\brac{\bar{\bx}_{k,i_k},\by_{k,j_k}} \Phi_{k,i_k,j_k}(\bx,\by),
\end{equation}
where $\Phi_{k,i_k,j_k}(\bx,\by):= \bar{\phi}_{k,i_k}(\bx) h_{k,j_k}(\by)$ and $k_{\lceil n/\log n \rceil}$ is defined by \eqref{n_k,k_n} for  a sequence  $(\xi_{n})_{n \in \NN}$ of increasing positive numbers $\xi_n$ satisfying the condition \eqref{xi_n}.
The operator 
$\bar{\Ss}^V_n$ is a linear sampling algorithm in the space 
$L_2(\RRi,V; \gamma)$ defined for functions $v(\bx,\by)$ on  the spatial-parametric domain $D \times \RRi$ and based on the  sampling points 
$$
\operatorname{SamplePts}\brac{\bar{\Ss}^V_n}
\ =\
\brab{\big(\bx_{k,i_k}, \by_{k,j_k}\big): \ i_k=1,...,\bar{m}_k,\ j_k = 1,...,c_p n_k, \ k = 0,..., k_{\lceil n/\log n \rceil}}.
$$

\begin{assumption} \label{assumptionK}
	$D\subset\RR^2$ is a bounded polygonal domain and
	$r\in\NN$, $r \ge  2$; $f\in \Kk_{\varkappa-1}^{r-2}(D)$ and
	$\psi_k\in W^{1}_\infty(D)\cap\Ww_{\infty}^{r-1}(D)$, $k \in \NN$. 
	For $i=1,2$, the sequences $\bb_i:=(b_{i,j})_{j\in\NN}$ defined as in \eqref{eq:b1b2ml} satisfy the condition 
	$\bb_i\in\ell^{p_i}(\NN)$ with $0< p_1\le  p_2 < 1$ and $p_1 < 2/3$. 
\end{assumption}

\begin{theorem}\label{thm:sampling-Kspace} 
	Let Assumption \ref{assumptionK} hold.
	Let the numbers $\alpha$ and  $\beta$ be defined by
	\begin{equation} 	\label{alpha,delta, beta}
	\alpha:= \frac{r-1}{2}, \quad	\beta := \brac{\frac 1 {p_1} - 1} \frac{\alpha}{\alpha + \delta}, \quad 
		\delta := \frac 1 {p_1} - \frac 1 {p_2}.
	\end{equation}	
		Then for all $n \ge 2$ there exist  $\by_{k,1},...,\by_{k,n_k} \in \RRi$ and 
	$h_{k,1},...,h_{k,n_k} \in L_2(\RRi,\CC;\gamma)$, $k=0,...,k_{\lceil n/\log n \rceil}$,  such that for the linear sampling algorithm defined by \eqref{barSs_{n}^V}, we have that
	\begin{equation*} \label{comp1}
	\big|\operatorname{SamplePts}\brac{\bar{\Ss}^V_n}\big|
		\ \le  \
		n, 
	\end{equation*} 
	and for the parametric solution $u$ to the equation \eqref{parametricPDE} with log-normal random inputs,
	\begin{equation*} 	\label{L_2-rate}
	\begin{split}
		\big\|u-\bar{\Ss}^V_n u\big\|_{L_2(\RRi,V;\gamma)} 
		\ \le \
		\begin{cases}
			C_\varepsilon	n^{-\alpha}(\log n )^{\alpha + 1 +\varepsilon}\quad &{\rm if }  \ \alpha < 1/q_2,\\
			C_\varepsilon	n^{-\alpha}(\log n )^{\alpha + 2 +\varepsilon}\quad &{\rm if }  \ \alpha = 1/q_2, \\
			C	n^{-\beta}(\log n )^{\beta + \alpha/(\alpha - \beta)}\ \ \quad  &{\rm if }  \ \alpha > 1/q_2.
		\end{cases}
	\end{split}
\end{equation*}
for  an arbitrarily   small number  $ \varepsilon > 0$,
%
%	\begin{equation*} 	\label{L_2-rate}
%		\begin{split}
%			\big\|u-\bar{\Ss}^V_n u\big\|_{L_2(\RRi,V;\gamma)} 
%			\ \le \	C_\varepsilon
%			\begin{cases}
%				n^{-\alpha}(\log n )^{\alpha + 1 +\varepsilon}\quad &{\rm if }  \ \alpha < 1/q_2,\\
%				n^{-\alpha}(\log n )^{\alpha + 2 +\varepsilon}\quad &{\rm if }  \ \alpha = 1/q_2,
%			\end{cases}
%		\end{split}
%	\end{equation*}
%	for  an arbitrarily   small number  
%	$ \varepsilon > 0$,	and
%	\begin{equation*} 	\label{L_2-rate}
%	\big\|u-\bar{\Ss}^V_n u\big\|_{L_2(\RRi,V;\gamma)} 
%		\ \le \	C
%		n^{-\beta}(\log n )^{\beta + \alpha/(\alpha - \beta)}\ \ \quad  {\rm if }  \ \alpha > 1/q_2.
%	\end{equation*}	
	where  the constants $C$ and $C_\varepsilon$ is independent of $u$ and $n$.
	\end{theorem}	
	
	\begin{proof}	Under the hypothesis of this theorem, all the assumptions of Lemma \ref{lemma:holomorphyKspace} are satisfied. 	This yields that  Assumption \ref{ass:ml} holds for $u\in L_2(\RRi,V;\gamma)$. Hence, Theorem \ref{thm:sampling-lognormal-holomorphicK} is true for $u$. To complete the proof it is sufficient to notice that
		$$
		\big|\operatorname{SamplePts}\brac{\bar{\Ss}^V_n}\big|
		\ =\
	\operatorname{Comp} \brac{\bar{\Ss}^V_n}	
	\ \le  \
	n.
		$$
	\hfill
	\end{proof}
	
	 Let the operators $\delta_k$, $k \in \NN_0$, be defined in \eqref{delta_k} for  the sequence $(P_k)_{k\in\NN}$ given by \eqref{P_k}. Then 
 by the formulas \eqref{I_Lambda=} and \eqref{delta_k} we  can  represent the operator $\Ii_{\xi}$ defined in \eqref{Ii_{xi}v}, as
 \begin{equation} \label{I_xi1}
 	\Ii_{\xi} v(\bx,\by) 
 	\ = \ 
 	\sum_{k=0}^{\lfloor \log_2 \xi \rfloor} \sum_{i_k=1}^{\bar{m}_k}  
 	\sum_{(\bs_k,\be_k,\bj_k) \in \Gamma_k(\xi)}  
 	(-1)^{|\be_k|_1} v(\bar{\bx}_{k,i_k},\by_{\bs_k - \be_k;\bj_k})
 	\bar{\phi}_{k,i_k}(\bx)L_{\bs_k - \be_k;\bj_k}(\by),
 \end{equation}
 where	
 \begin{equation*} 
 	\Gamma_k(\xi)				
 	:= \
 	\{(\bs_k,\be_k,\bj_k) \in \FF \times \FF \times \FF: 
 	\, \bs_k \in \Lambda_k(\xi), \ \be_k \in E_{\bs_k}, \ \bj_k \in \pi_{\bs_k - \be_k} \}.
 \end{equation*}
 We  rewrite  $\Ii_{\xi}$ in \eqref{I_xi1} in the form of sampling algorithm in $L_2(\RRi,V;\gamma)$
 \begin{equation*} 
 	\Ii_{\xi} v(\bx,\by) 
 	\ = \ 
 	\sum_{k=0}^{\lfloor \log_2 \xi \rfloor} \sum_{i_k=1}^{\bar{m}_k}  
 	\sum_{(\bs_k,\be_k,\bj_k) \in \Gamma_k(\xi)}  
 	v(\bar{\bx}_{k,i_k},\by_{\bs_k - \be_k;\bj_k})
 	\Phi_{i_k,\bs_k,\be_k,\bj_k}(\bx, \by)	,
 \end{equation*}
 where 
 $$
 \Phi_{i_k,\bs_k,\be_k,\bj_k}(\bx, \by)
 := 
 (-1)^{|\be_k|_1}\bar{\phi}_{k,i_k}(\bx)L_{\bs_k - \be_k;\bj_k}(\by).
 $$
 
 In a similar way to the proof of Theorem \ref{thm:sampling-Kspace},  from 
 Theorem \ref{thm:fully-discrete-lognormal-holomorphic} and Lemma \ref{lemma:holomorphyKspace} we derive

\begin{theorem}\label{thm:sampling-HermiteInterpolation} 
	Let Assumption \ref{assumptionK} hold.
	Let the numbers $\alpha$ and  $\beta$ be defined by
	\begin{equation} 	\label{alpha,delta, beta}
		\alpha:= \frac{r-1}{2}, \quad	\beta := \brac{\frac 1 {p_1} - \frac{3}{2}} \frac{\alpha}{\alpha + \delta}, \quad 
		\delta := \frac 1 {p_1} - \frac 1 {p_2}.
	\end{equation}	
	Then for each $n \in \NN$ there exists a number $\xi_n$  such that  
	\begin{equation*} \label{comp1}
		\big|\operatorname{SamplePts}\Ii_{\xi_n}\big|
		\ \le  \
		n, 
	\end{equation*} 
	and for the parametric solution $u$ to the equation \eqref{parametricPDE} with log-normal random inputs \eqref{lognormal}, 
	\begin{equation*} 	\label{L_2-rate}
		\begin{split}
			\big\|u - \Ii_{\xi_n} u\big\|_{L_2(\RRi,V;\gamma)} 
			\ \le \	C
			\begin{cases}
				n^{-\alpha} \quad &{\rm if }  \ \alpha \le 1/p_2 - 3/2,\\
				n^{-\beta} \quad  & {\rm if }  \ \alpha > 1/p_2 - 3/2,
			\end{cases}
		\end{split}
	\end{equation*}
	where  the constant $C$ is independent of $u$ and $n$.
\end{theorem}	

By combining Theorems \ref{thm:sampling-Kspace} and \ref{thm:sampling-HermiteInterpolation} we obtain the following final results.

\begin{theorem}\label{thm:sampling-final} 
	Let Assumption \ref{assumptionK} hold.
	Let the numbers $\alpha$ and  $\beta$ be defined by \eqref{alpha,delta, beta}.
	Then for all $n \ge 2$ there exist  points $(\bx_1,\by_1),...,(\bx_n,\by_n) \in D \times \RRi$ and functions
	$\varphi_1,...,\varphi_n \in V$ and $h_1,...,h_n \in L_2(\RRi,\RR;\gamma)$  such that for the linear sampling algorithm $S_n$ on the spatial-parametric domain $D\times \RRi$ defined 
	%for functions $v$ on   $D \times \RRi$ 
	by
	\begin{equation*}
		S_n(v)(\bx,\by): = \sum_{i=1}^n  v(\bx_i,\by_i) \varphi_i (\bx) h_i(\by),  
		\ \ \bx \in D, \ \ \by \in \RRi,
	\end{equation*}	
	and for the parametric solution $u$ to the equation \eqref{parametricPDE} with log-normal random inputs 	 \eqref{lognormal}, 
	it holds the error bounds
	 \begin{equation*} 	\label{L_2-rate}
		\begin{split}
			\big\|u-S_n u\big\|_{L_2(\RRi,V;\gamma)} 
			\ \le \	
				\begin{cases}
			C	n^{-\alpha}\quad &{\rm if }  \ \alpha \le 1/p_2 - 3/2,\\
			C_\varepsilon	n^{-\alpha}(\log n )^{\alpha + 1 +\varepsilon} 
				\quad &{\rm if }  \ 1/p_2 - 3/2 <  \alpha < 1/p_2 - 1,\\
			C_\varepsilon	n^{-\alpha}(\log n )^{\alpha + 2 +\varepsilon}\quad &{\rm if }  \ \alpha = 1/p_2 -1,\\
				C		n^{-\beta}(\log n )^{\beta + \alpha/(\alpha - \beta)} \quad  &{\rm if }  \ \alpha > 1/p_2 - 1,
			\end{cases}
		\end{split}
	\end{equation*}
	for  an arbitrarily   small number  
	$ \varepsilon > 0$,	
%	and
%	\begin{equation*} 	\label{L_2-rate}
%			\big\|u-S_n u\big\|_{L_2(\RRi,V;\gamma)} 
%		\ \le \	C
%		n^{-\beta}(\log n )^{\beta + \alpha/(\alpha - \beta)}\ \ \quad  {\rm if }  \ \alpha > 1/p_2 - 1.
%	\end{equation*}	
	where  the constants $C$ and $C_\varepsilon$ are independent of $u$ and $n$.
\end{theorem}	
%%%%%%%%%%%%%%%%%%%%%%

\section{Extensions of least squares sampling algorithms}
\label{Extensions}
In this section, we  discuss various  least squares sampling algorithms  for functions in the reproducing kernel Hilbert space $H_{\CC,\bsigma}$, and inequalities between sampling $n$-widths and Kolmogorov $n$-widths of the unit ball $B_{\CC,\bsigma}$  of this space. We explain then how to apply these inequalities to obtain corresponding convergence rates of  multi-level linear sampling recovery in abstract Bochner spaces and  of fully discrete multi-level collocation  approximation of  the parametric solution $u$ to
the parametric elliptic PDEs \eqref{ellip} on a bounded polygonal domain with log-normal inputs \eqref{lognormal}.

Recall that in Subsection \ref{Extended least squares sampling algorithms in Bochner spaces}, a notion of weighted least squares sampling algorithm $S_{cn}^\CC$ in $L_2(U,\CC;\mu)$ is  introduced as in \eqref{least-squares-sampling1}--\eqref{least-squares3}. Extensions of $S_{cn}^\CC$ to $L_2(U,X;\mu)$ is defined as in \eqref{least-squares-sampling-extension}. By the help of extended  weighted least squares sampling algorithms  with the special choice of sample points $\by_1, \dots, \by_{cn}$ and weights $\omega_1, \dots, \omega_{cn}$, and the bounds of the approximation error as in Lemma \ref{lemma:sampling inequality}, we constructed efficient multi-level least squares sampling algorithms in the Bochner space  $L_2(U,X;\mu)$ for functions in $H^\alpha_{X}$, and proved the convergence rates by them as in Theorem~\ref{thm:sampling2}. 
The choice of sample points $\by_1, \dots, \by_{cn}$, weights $\omega_1, \dots, \omega_{cn}$, and approximation space $\Phi_m$ is crucial for the error of the least squares sampling algorithm.
We recall three choices presented in \cite{BD2024}, with a trade-off between constructiveness and tightness of the error bound. The third choice has been considered in Lemma \ref{lemma:sampling inequality}.

\begin{assumption}\label{assum:samplepoints}
	Let $n\in\NN$, $n\ge 90$, $c_1\ge 1$, $c_2 > 1+\frac{1}{n}$, and $c_3
	\ge 3284$. Let the probability measure $\nu$ be defined as in \eqref{nu}.
	\begin{itemize}
				\item[\rm (1)]
		Let $m := \lfloor n/(20\log n)\rfloor$.
		Let further $\by_1  ,\dots \by_{c_1n}\in U$ be points drawn i.i.d.\ with respect to $\nu$ and $\omega_i := (\varrho(\by_i))^{-1}$.
		\item[\rm (2)]
		Let $m := n$ and $\lceil 20 n\log n\rceil$ points be drawn i.i.d.\ with respect to $\nu$ and subsampled using \cite[Algorithm~3]{BSU23} to $c_2n \asymp m$ points.
		Denote the resulting points by $\by_1, \dots, \by_{c_2n} \in U$ and $\omega_i = \frac{c_2n}{\lceil 20 n\log n\rceil}(\varrho(\by_i))^{-1}$.
		\item[\rm (3)]
		Let $m := n$ and $\lceil 20 n\log n\rceil$ points be drawn i.i.d.\ with respect to $\nu$.
		Let further $\by_1, \dots, \by_{c_3n} \in U$ be the subset of points fulfilling \cite[Theorem~1]{DKU2023} with $c_3n \asymp m$ and $\omega_i := \frac{c_3n}{\lceil 20  n\log n\rceil}(\varrho(\by_i))^{-1}$.
	\end{itemize}
\end{assumption}
%We make use of the abbreviation $d_n := d_n(B_{\CC,\bsigma},L_2(U,\CC;\mu))$.
%In our setting, we know  that $d_n = \sigma_{n+1}^{-1}$.


\begin{lemma}\label{lemma:leastsquaresbounds}
	Let $0 < p < 2$. Let  $d_n := d_n(B_{\CC,\bsigma},L_2(U,\CC;\mu))$. For $c,n,m\in\NN$ with $cn\ge m$, let $S_{cn}^\CC$ be the least squares sampling algorithm  defined as in \eqref{least-squares-sampling1}--\eqref{least-squares3}.
	There are constants $c_1, c_2, c_3 \in \NN$ depending on $p$ such that for all  $n \ge 2$ we have the following.
	\begin{enumerate}
		\item[\rm (1)]
		The points from Assumption \ref{assum:samplepoints}{\rm (1)} fulfill with high probability
		\begin{equation} \nonumber
				\varrho_n(B_{\CC,\bsigma}, L_2(\RRi,\CC;\mu)) 
			\ \le \
			\sup_{v\in B_{\CC,\bsigma}} \norm{v - S_{c_1n}^X v}{L_2(U,\CC;\mu)}
				\ \le \brac{\frac{\log n}{n} \sum_{j \ge n/\log n} d_j^p}^{1/p}
		\end{equation}
		\item[\rm (2)]
		The points from Assumption~\ref{assum:samplepoints}{\rm (2)} fulfill with high probability
		\begin{equation}\nonumber
				\varrho_n(B_{\CC,\bsigma}, L_2(\RRi,\CC;\mu)) 
			\ \le \
			\sup_{v\in B_{\CC,\bsigma}} \norm{v - S_{c_2n}^X v}{L_2(U,\CC;\mu)}
				\ \le \brac{\frac{\log n}{n} \sum_{j \ge n} d_j^p}^{1/p}.
		\end{equation}
		\item[\rm (3)]
		The points from Assumption~\ref{assum:samplepoints}{\rm (3)} fulfill with high probability
		\begin{equation}\nonumber
				\varrho_n(B_{\CC,\bsigma}, L_2(\RRi,\CC;\mu)) 
			\ \le \
			\sup_{v\in B_{\CC,\bsigma}} \norm{v - S_{c_2n}^X v}{L_2(U,\CC;\mu)}
				\ \le \brac{\frac{1}{n} \sum_{j \ge n} d_j^p}^{1/p}.
		\end{equation}
	\end{enumerate}
\end{lemma}

This lemma has been formulated in \cite{BD2024}.
As mentioned there, the claims (1) and (3) have  been proven in 
\cite[Theorem~8]{KU21} and
\cite[Theorem~1]{DKU2023}, respectively.	
The claim (2) can be proven a similar way based on the result
\cite[Theorem~6.7]{BSU23}.


As commented in \cite{BD2024}, regarding the constructiveness of the linear least squares sampling algorithms in Lemma~\ref{lemma:leastsquaresbounds},  the bound  Lemma~\ref{lemma:leastsquaresbounds}(1) is the most coarse bound, but the points construction requires only a random draw, which is computationally inexpensive.
The sharper bound in Lemma~\ref{lemma:leastsquaresbounds}(2) uses an additional constructive subsampling step.
This was implemented and numerically tested in \cite{BSU23} for up to 1000 basis functions.
For larger problem sizes the current algorithm is to slow as its runtime is cubic in the number of basis functions.
The sharpest  bound in Lemma~\ref{lemma:leastsquaresbounds}(3) is a pure existence result.
So, up to now, the only way to obtain this point set is to brute-force every combination, which is computational infeasible.

As mentioned above, the linear least squares sampling algorithms in Lemma \ref{lemma:sampling inequality}(3) are based on the sample points in Assumption~\ref{assum:samplepoints}{\rm (3)} which give the best convergence rate among the sample points in Assumption~\ref{assum:samplepoints}{\rm (1)}--{\rm (3)}, but are least constructive. Hence,   the parametric components in the linear sampling algorithms in Theorems 
%\ref{thm:fully-discrete-lognormal-holomorphic}, 
%\ref{thm:leastsquaresbounds}, 
\ref{thm:sampling}, 
\ref{thm:sampling2}, 
\ref{thm:sampling-lognormal-holomorphicK} and
%\ref{thm:samplingDD}, 
\ref{thm:sampling-Kspace}
%\ref{thm:sampling-final}, 
%\ref{thm:sampling-HermiteInterpolation} 
are  based on such points. The linear sampling algorithms in Lemma~\ref{lemma:sampling inequality}(1)--(2)  based on the sample points in Assumption~\ref{assum:samplepoints}(1)--(2), respectively, are pure least squares algorithms or least squares algorithms with constructive subsampling, and therefore, constructive. But they give slightly worse  error bounds.

Again, let $(\xi_{n})_{n \in \NN}$ be a sequence of increasing positive numbers whose values will be selected later, such that $\xi_{n} \to \infty$ as $n \to \infty$. For $n \in \NN$ and $j =1,2,3$, we consider the multilevel least squares  operator $\Ss_{(j),n}^X$ defined by 
\begin{equation*}  
	\Ss_{(j),n}^X v
	\ = \ 
	\sum_{k=0}^{k_n}  S_{c_j n_{j,k}}^X (\delta_k v),
\end{equation*}
where  
%$n_k := \lfloor  c_p^{-1} n 2^{-k}\rfloor$, $k_n:= \lfloor \log \xi_n \rfloor$,
\begin{equation*} 
	n_{j,k} := \lfloor  c_j^{-1} n 2^{-k}\rfloor, \ \ \ k_n:= \lfloor \log \xi_n \rfloor,
\end{equation*}
and
\begin{equation*} 
	S_{c_j n_{j,k}}^X  v : = \sum_{i=1}^{c_j n_k}  v\brac{\by_{k,i}^{(j)}} h_{k,i}^{(j)}
\end{equation*}
the extended least squares approximations with sample points
$\by_{k,1}^{(j)},...,\by_{k,n_k}^{(j)} \in U$  as in Assumption~\ref{assum:samplepoints}(j), $h_{k,1}^{(j)},...,h_{k,n_k}^{(j)} \in L_2(\RRi,\CC;\mu)$,
and $c_j > 0$ are the constants as in Lemma~\ref{lemma:leastsquaresbounds}, respectively. We define the operators:
\begin{equation}  \label{Ss_{(j),n}^X}
	\bar{\Ss}_{(j),n}^X 
	:= \ 
	\Ss_{(j),\lfloor n / \log n\rfloor}^X.
\end{equation}
Theorem \ref{thm:sampling2} gives the error bounds of approximation of  $v\in B_X^{\alpha,\tau}$ by the operators $\bar{\Ss}_{(3),n}^X = \bar{\Ss}_{n}^X$ based on the parametric sample points in Assumption~\ref{assum:samplepoints}(3) and proven with the help of Lemma~\ref{lemma:leastsquaresbounds}(3). Below we formulate its counterparts based on the parametric sample points in Assumption~\ref{assum:samplepoints}(1)--(2) and proven with the help of Lemma~\ref{lemma:leastsquaresbounds}(1)--(2), respectively, which can be proven in a similar way with slight modifications.

\begin{theorem}\label{thm:sampling-modification(1)} 
	Let the assumptions of Theorem \ref{thm:sampling2} hold.
	Then for $j =1,2$ and all $n \ge 2$, there exist  $\by_{k,1},...,\by_{k,n_k} \in U$ and $h_{k,1},...,h_{k,n_k} \in L_2(\RRi,\CC;\mu)$, $k=0,...,k_{\lceil n/\log n \rceil}$,  such that for the operator $\bar{\Ss}^X_{(j),n}$ defined as in \eqref{Ss_{(j),n}^X}, 
	\begin{equation*} \label{comp1}
		\operatorname{Comp} \brac{\bar{\Ss}^X_{(1),n}}	
		\ \le  \
		n,
	\end{equation*} 
	and
	\begin{equation*} 	\label{L_2-rate}
		\begin{split}
			\sup_{v\in B_X^{\alpha}}		\|v-\bar{\Ss}^X_{(1),n} v\|_{L_2(U,X;\mu)} 
			\ \le \	
			\begin{cases}
			C_\varepsilon	n^{-\alpha}(\log n )^{\alpha + 1 + 1/q_2 +\varepsilon}\quad &{\rm if }  \ \alpha < 1/q_2,\\
			C_\varepsilon	n^{-\alpha}(\log n )^{\alpha + 2  + 1/q_2+\varepsilon}\quad &{\rm if }  \ \alpha = 1/q_2, \\
			C	n^{-\beta}(\log n )^{2\beta + \alpha/(\alpha - \beta)}\ \ \quad  &{\rm if }  \ \alpha > 1/q_2,
			\end{cases}
		\end{split}
	\end{equation*}
	for  an arbitrarily   small number 	$ \varepsilon > 0$.
%	,	and
%	\begin{equation*} 	\label{L_2-rate}
%		\sup_{v\in B_X^{\alpha,\tau}}		\|v-\bar{\Ss}^X_{(1),n} v\|_{L_2(U,X;\mu)} 
%		\ \le \	C
%		n^{-\beta}(\log n )^{2\beta + \alpha/(\alpha - \beta)}\ \ \quad  {\rm if }  \ \alpha > 1/q_2.
%	\end{equation*}
\end{theorem}

\begin{theorem}\label{thm:sampling-modification(2)} 
	Let the assumptions of Theorem \ref{thm:sampling2} hold.
	Then for $j =1,2$ and all $n \ge 2$, there exist  $\by_{k,1},...,\by_{k,n_k} \in U$ and $h_{k,1},...,h_{k,n_k} \in L_2(\RRi,\CC;\mu)$, $k=0,...,k_{\lceil n/\log n \rceil}$,  such that for the operator $\bar{\Ss}^X_{(2),n}$ defined as in \eqref{Ss_{(j),n}^X}, 
	\begin{equation*} \label{comp1}
		\operatorname{Comp} \brac{\bar{\Ss}^X_{(2),n}}	
		\ \le  \
		n,
	\end{equation*} 
	and
	\begin{equation*} 	\label{L_2-rate}
		\begin{split}
			\sup_{v\in B_X^{\alpha}}		\|v-\bar{\Ss}^X_{(2),n} v\|_{L_2(U,X;\mu)} 
			\ \le \	C_\varepsilon
			\begin{cases}
				n^{-\alpha}(\log n )^{\alpha + 3/2 +\varepsilon}\quad &{\rm if }  \ \alpha < 1/q_2,\\
				n^{-\alpha}(\log n )^{\alpha + 5/2+\varepsilon}\quad &{\rm if }  \ \alpha = 1/q_2, \\
				n^{-\beta}(\log n )^{3\beta/2 + \alpha/(\alpha - \beta)+\varepsilon} \  \quad  &{\rm if }  \ \alpha > 1/q_2,
			\end{cases}
		\end{split}
	\end{equation*}
	for an arbitrarily   small number  $ \varepsilon > 0$.
\end{theorem}

From Theorems \ref{thm:sampling-modification(1)} and \ref{thm:sampling-modification(2)} one can derive respective results similar to Theorem \ref{thm:sampling-Kspace} and the others in Sections \ref{Applications to  holomorphic functions} and \ref{Applications to  parametric PDEs}, based on the least squares sampling algorithms defined as in \eqref{least-squares-sampling1}--\eqref{least-squares3} with the choice of sample points and weights as in Assumption \ref{assum:samplepoints}(1) or (2).
 	
 \medskip
 \noindent
 {\bf Acknowledgments:}  
 This work is funded by the Vietnam National Foundation for Science and Technology Development (NAFOSTED) in the frame of the NAFOSTED--SNSF Joint Research Project under  Grant 
 IZVSZ2$_{ - }$229568.  
 A part of this work was done when  the author was working at the Vietnam Institute for Advanced Study in Mathematics (VIASM). He would like to thank  the VIASM  for providing a fruitful research environment and working condition.
  
 %%%%%%%%%%%%%%%%%%%
\bibliographystyle{abbrv}
\bibliography{Sampling&Coll-SPDE.bib}
\end{document}



