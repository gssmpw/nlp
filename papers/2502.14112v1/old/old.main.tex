\documentclass[12pt]{article}
\usepackage{newtxtext,newtxmath}
\usepackage{titling}
\usepackage[utf8]{inputenc}
\usepackage{amsmath}
\usepackage{appendix}
\usepackage{graphicx}
\usepackage{float}
\usepackage{graphicx}
\usepackage{caption}
\usepackage{subcaption}
\usepackage{color}
\usepackage{url}
\usepackage[square]{natbib}
\usepackage[margin=1in]{geometry}
%\usepackage[title]{appendix}
\usepackage[textsize=scriptsize]{todonotes}
\usepackage{xcolor}
\definecolor{blue}{RGB}{0, 93, 170}			%Go Big Blue!
%\newcommand{\nick}[1]{\todo[color=blue!40]{Nick says: #1}}
\usepackage{amssymb}

\newtheorem{prop}{Proposition}

\def\Comments{1} % change to 0 to hide comments
\newcommand{\kibitz}[2]{\ifnum\Comments=1{\color{#1}{#2}}\fi}
\newcommand{\rmr}[1]{\kibitz{red}{[RESHEF:#1]}}
\newcommand{\hdl}[1]{\kibitz{blue}{[HODAYA:#1]}}
\newcommand{\knt}[1]{\kibitz{green}{[KINNERET:#1]}}


\def\cite{\citealp}
\def\shortcite{\citeyearpar}

\title{The Effect of Protecting Initial Discoveries on Exploration Behaviors }
\author{Hodaya Lampert\thanks{Israel Institute of Technology, Email: hodaya.lampert@gmail.com} ~ Reshef Meir \thanks{Faculty of Industrial Engineering and Management, Technion, Israel Institute of Technology, Email: reshefm@technion.ac.il } ~ Kinneret Teodorescu \thanks{Faculty of Industrial Engineering and Management, Technion, Israel Institute of Technology, Email: kinnerett@technion.ac.il }}
\date{\today}
\linespread{1.7}

\usepackage{graphicx}
\graphicspath{ {./images/} }
\renewcommand{\baselinestretch}{2}
\begin{document}

\maketitle
\newpage
\section*{Declarations}
\subsection*{Funding}
This project was funded by the Israeli Ministry of Science and Technology, grant no. 3-15284.
\subsection*{Conflicts of interest/Competing interests}
Not applicable
\subsection*{Availability of data and material}
Data can be found in https://github.com/hodayal/The-Effect-of-Protecting-Initial-Discoveries-on-Exploration
\subsection*{Code availability}
Code can be found in
https://github.com/hodayal/The-Effect-of-Protecting-Initial-Discoveries-on-Exploration
\newpage
\begin{abstract}
     Exploring new ideas is a fundamental aspect of research and development (R\&D), which often occurs in competitive environments. Most ideas are subsequent, i.e. one idea today leads to more ideas tomorrow. According to one approach, the best way to encourage innovative activity is by granting protection on discoveries to the first innovator. Correspondingly, only the one who made the first discovery can use the new knowledge and benefit from subsequent discoveries, which in turn should increase the initial motivation to explore. An alternative approach to promote exploration favors the sharing of knowledge from discoveries among researchers allowing explorers to use each other’s discoveries to develop further knowledge. With no protection, all explorers have access to all existing discoveries and new research directions are explored faster as more teams are involved, leading to earlier subsequent discoveries. This paper contrasts these approaches by developing a game that simulates a simplified interaction among R\&D investors. We present a game theoretic analysis of the game which clarifies the expected advantages and disadvantages of the two approaches under full information. We then compare the theoretical predictions with the observed behavior of actual players in the lab who operate under partial information conditions in a ``with protection” and ``no-protection” worlds. We show that although the no protection approach leads to more exploratory efforts, the protection approach yields more discoveries overall, due to a more efficient exploration.
     
We further discuss how underweighting of rare events contributes to the observed effect of
discovery protection, as well as alternative ways to achieve efficient exploration without the
detrimental disadvantages of discovery protection.

     
\end{abstract}

\section{Introduction}

Stimulating innovation that leads to advances in technology have always been a core challenge for policy designers. To this end, some proponents advocate free competition while others argue for the benefits of rights protection. Although competition among agents usually has a positive influence on their incentives \citep{nickell1996competition,blundell1999market,younge2018competitive}, competition in the context of innovation might represent an exceptional case. This exception can be attributed to the cumulative nature of discoveries \citep{scotchmer1991standing}: when a new radical discovery is made it paves the path to a whole field of research possibilities. This new knowledge can be used to easily, and cheaply, make many more subsequent, though incremental, discoveries. Thus, if the new knowledge is publicly shared then other inventors can use it to generate incremental discoveries. However, only the first inventor bears the cost of the whole discovery process. Therefore, discoveries and inventions could, arguably, be classified as a public good, and as such might receive insufficient contributions (i.e. exploratory efforts) in a competitive environment \citep{grossman1990trade}.

One common way to overcome this problem is to grant original inventors exclusive rights to explore related incremental discoveries. This type of protection aims at encouraging radical innovation efforts by blocking others from competing on subsequent developments, which in turn increases the potential rewards for original inventors \citep{kaufer2012economics}. For example, to motivate searches for natural resources such as oil or gas, governments often provide firms with exclusive franchise to search in a specific area. Franchises promise firms a monopoly right to search in a given area and to benefit from any discovery it yields. Similarly, technology firms protect their breakthrough discoveries with patents. Patents make it difficult  for other firms to use the protected knowledge and through this action, give the patentees a significant advantage in competition for subsequent products. In academic research, a policy that allows researchers to keep their data private, increases the reward from collecting the data, giving the researcher an advantage over other researchers from the same field (who cannot access and explore the data set).

Beyond increasing the incentives to search for radical innovations, another potential advantage of protecting initial discoveries is relative specialization. When a research team or a technology firm specializes in investigating one initial discovery, they can learn from their own previous experience which research directions work best and which will fail with high probability. Moreover, providing the original inventors exclusive rights for subsequent searches reduces the chances that the same discovery will be made by several teams in parallel, which in turn increases the overall efficiency of the search process \citep{baron2013cooperates,denicolo2000two}. Accordingly, blocking others from searching for subsequent discoveries may lead to more efficient exploration processes.  

Yet, blocking others from using existing knowledge reduces competition for subsequent discoveries which in fact might slow down the discovery process \citep{llanes2009anticommons,boldrin2005economics,bessen2009sequential,galasso2014patents}. For example, conferences and journals have recently started to condition publication of papers on making the data public \citep{stieglitz2020researchers,zhu2020open}, allowing other researchers to explore the data and extract novel insights as well as find errors in the original studies. This approach already underlies existing open-source platforms, where developers share their source code in a public domain and use other developers’ code in their own programs \citep{lerner2006dynamics}. However, notice that open policies, might suffer from the disadvantages protection is assumed to solve, i.e. insufficient investment in radical exploration and inefficient exploration processes. In the current work, we aim to shed light on the assumed advantages and disadvantages of protecting initial discoveries. Specifically, we examine how the fundamental economic factors underlying competition with and without a protection policy affect exploration behaviors and performance measurements such as the amount of discoveries made, the speed of making discoveries and the efficiency of the exploration process. 

As mentioned above competitive search for discoveries occurs in many real-life environments such as searching for natural resources, technological advances and scientific insights. Each context has its own distinct characteristics, but all share the fundamental economic features that are involved in exploration processes. A key feature is search cost, which is heterogeneous among competitors and varies over time due to dynamic environmental factors like weather, energy prices and employee availability. Additionally, in many searching processes, successes are public information, while failures are private. Firms and researchers tend to publish their achievements to increase their reputation, their value in the stock market or their profits.\footnote{In some cases firms prefer to keep their successes as trade secrets. This option can be available when granting a patent is too expansive, or legally impossible. However, in many cases the possibility of reverse engineering the final product reduces the effectiveness of this choice.} However, failures such as unsuccessful attempts to find gas or oil, failed experiments or disappointing development endeavors often remain private information that is kept far from the competitors’ eyes. 

Importantly, another shared attribute of competitive innovation environments is the distribution of rewards: initial exploration in unfamiliar areas is less likely to succeed but offers higher rewards for radical discoveries, while subsequent, incremental, discoveries tend to yield lower rewards with higher frequency. The interplay between the magnitude and the frequency of rewards and its effect on behavior has been extensively explored in the “Decisions from Experience” (DfE) literature. One of the most robust findings in this literature is that in repeated choice settings, people tend to underweight rare events \citep{baron2013cooperates,hertwig2004decisions,teodorescu2021enforcement}. Specifically, in exploration tasks, participants were found to under-explore in a ``rare treasure environment", where exploration is disappointing most of the time but on rare occasions can yield very high reward (discovery) \citep{teodorescu2014decision,teodorescu2014learned}. 

In the context of innovation, since initial discoveries can be thought of as rare treasures, invested exploration efforts may be below optimum. Moreover, since protecting initial discoveries reduces the probability for others to make subsequent discoveries, it decreases the average probability to make a subsequent discovery \citep{bessen2009sequential}. Thus, underweighting of rare events implies that increasing the magnitude of a rare reward (via protection) will have a smaller than expected effect on exploratory efforts to find initial discoveries. Underweighting of rare events also implies that when exploration is frequently rewarding (i.e. in searches for incremental discoveries) disappointing exploration efforts are more rare and thus people might over-search for subsequent discoveries (searching even when it is not optimal to do so).
\\
\newpage
\textbf{Previous experimental studies}

Only a few experimental studies tackled the effect of discovery protection on innovative behavior. \citet{torrance2009patents} used an interactive R\&D simulation, finding that protection reduced both the quantity and quality of innovations, and decreased welfare compared to a no-protection condition. Similarly, \citet{bruggemann2016intellectual} using a Scrabble like creativity task, found that protection reduced innovations' quantity and quality and also reduced welfare. However, \citet{buchanan2014experiment},  using a color generation studio task and \citet{dimmig2012quasi}, using a two-player duopoly game, found no significant or only minor effects of discovery protection. \citet{ullberg2012dynamic,ullberg2017coordination} further highlighted that low patent validity impairs coordination in a licenses market. Importantly, the limited number of experimental studies in competitive environment have employed highly complex tasks, which may increase external validity, but make causal relationships difficult to analyze. For example,  \citet{torrance2009patents} complex simulation does not clarify whether the adverse effect of protection stemmed from patenting cost, probability of making a discovery, licensing availability/fees, or other factors, nor whether participants' behaviors were rational response or influenced by behavioral biases. Additionally, probably due to the complexity, most of the above experiments lasted more than an hour yet included a relatively small amount of trials (10-25 per session). Since participants receive feedback only at the end of each trial, the limited number of trials makes it difficult to address learning and long-term effects.\footnote{\citet{torrance2009patents} are an exception, not employing a distinct-trials setting but rather using a fixed time limit of 25 min.}
\\
\textbf{The current framework}

In the current study we investigate exploration with and without protection over many trials and with immediate feedback. The core objective of the current work is to shed light on the fundamental causal effect of initial discovery protection on exploration, learning and performance within a competitive sequential environment. To this end, we developed a simplified game in which players compete to find hidden treasures on a spatial map. The competition is sequential, such that exploration decisions are based on existing knowledge that was discovered in previous periods. In this framework, treasures represent successful innovation efforts, i.e. making a new discovery is simulated by finding a treasure. The game is played under two conditions, ``Protection” and ``No Protection. Under the ``Protection" condition, the information gained from a treasure discovery can be used exclusively by the finder,\footnote{Hence, the protection here means that the finder can exclusively enjoy incremental improvement of initial discoveries.} and in the ``No Protection" condition, players can use the location of any treasure to guide the search for subsequent treasures. In addition, in both conditions the players’ failed exploratory efforts are private information, while their successes are public information. Within this simplified framework, we focus on investigating the effect of protection on exploration for initial and subsequent discoveries as well as on exploration efficiency. 

Unlike some of the previous experimental studies, here we do not focus on the innovation process itself (which involves creativity and entrepreneurship abilities, as in \citet{bruggemann2016intellectual}) but rather on the more basic economic variables such as search costs, the probability to make a discovery, the magnitude of reward obtained following discoveries etc. In addition, a simple game shortens the overall game time, and allows for dynamically tracking individual behavior over time. Most importantly, the current setting also allows derivation of proxies to the optimal strategies with and without protection and the comparison of these proxies with actual behavior. Optimal strategy analysis assumes players act rationally and base decisions on full information regarding their payoffs structure. However, given the uncertain nature of innovative activity and evidence for bounded rationality, deviations from optimality might occur, as will be discussed below. Importantly, the current, simple, setting enables identification of systematic behavioral deviations from optimality under full information assumptions, which could be crucial in deriving efficient and ecologically valid policy implementations. 

The rest of the paper proceeds as follows: Section \ref{sec:golddigger} presents the game in detail. Section \ref{sec:theory} contains a theoretic analysis of the optimal strategies in the Protection condition, and the subgame perfect equilibrium of the game in the No Protection condition. In addition, Section \ref{sec:simulations} presents computer simulations of money maximizer players and concludes with describing how underweighting of rare events can lead to deviations from the theoretical predictions. Section \ref{sec:experiment} presents the lab experiment in which the theoretical and behavioral predictions were tested, and section \ref{sec:conclusion} summarizes the main results and discusses theoretical and practical implications.


\section{The Competitive Treasure Hunt Game}\label{sec:golddigger}

``The Competitive Treasure Hunt" game is played in groups of 4 players. In this game, players are faced with a hive of white hexagons and need to find treasures. 5\% of the hexagons are hidden treasures that simulate discoveries in the real world. 

 Every three treasures are arranged in clusters  which form a tight triangle. We define the three linked treasures as a ``gold mine." Therefore, discovering one treasure increases the probability of finding the second treasure in the mine from (roughly) 0.05 to at least 0.33, the finding of which, in turn, increases the probability to find the third one to at least 0.5.\footnote{When the player has a previous failed exploration experience in adjacent hexagons, the probabilities to find the second and the third treasures are higher.} 
 
 Searching for treasures is costly to simulate R\&D costs. The first treasure to be found in each mine simulates a breakthrough discovery and the other two treasures simulate sequential discoveries. While finding an initial innovation is rarer, it provides knowledge that increases the probability of sequential innovations, or in our game, subsequent treasure discoveries. 
 
 To resemble the fact that in most real-life situations the initial discovery is worth more than subsequent, incremental improvements (but see discussion for alternative scenarios), in this study we set the first treasure discovered in a mine to have a value of 320 points while the second and the third are worth 80 points each. The game is played 4 times with 50 rounds each. The objective of the game is to maximize the expected payoff in each round. 
 
 The costs of exploration for each round are uniformly distributed over $\{5,10,15,20,25,30,35\}$, and are sampled independently for every player in each round. Each player is informed of his current cost of exploration at the beginning of each round.\footnote{The variation in search costs is intended to create heterogeneity between the players that also exists in the real world, where sometimes certain players have more skill (or knowledge, or resources) that allows them lower cost compared to others.} The players choose simultaneously whether to explore or to skip the round. Players who decide to skip the round obtain 0, and players who decide to explore, get to search one of the hexagons in the hive. They must pay the costs of exploration, and their total payoff in the round depends on whether they find a treasure or not, under which condition they play, and the decisions of the other players.\footnote{The reason we chose this payoffs and cost structures is because we designed the optimal search cost threshold strategies to be roughly in the middle of the cost range, to reduce ceiling or floor effects. The calculation of the optimal strategies can be found in the chapter of the theoretical analysis.} 

After clicking on a hexagon, if a player does not find a treasure, the hexagon he choose is colored in black on his board, but not on the other players' boards. If a player finds a treasure, the hexagon is colored in yellow on his board, and in red on the other players' boards (thus treasures are public, but failed exploration efforts are not).  

After Once a hexagon is colored in any color, the player cannot choose this hexagon in future rounds of that game. The mines are not adjacent to each other. Also, the treasure map was built so that all the mines contained exactly 3 treasures.\footnote{Regarding the edges, the proportion between treasures and empty hexagons approximately remains, so that the probabilities to find a treasure were not affected by the mine's location.}   

The game is played under two conditions: ``Protection" and ``No Protection". Under the ``Protection" condition, whenever a player finds the first treasure in a new mine, he also obtains the exclusive right to explore the adjacent hexagons (note that this area covers the entire gold mine). The protected area is marked on the board for all players, and   the marking is removed once the entire mine was discovered (see Fig. 1). Hence, no other player can profit from the information revealed after finding the first treasure in a new mine, since collecting the payoff from the two other treasures is not possible. In addition, when two or more players find the same first treasure simultaneously, they all pay the costs of exploration, and the computer randomly chooses one of them to receive the gains of the payoff from the treasure and the protection of the mine, while the others obtain 0 and cannot profit from subsequent discoveries in that mine. 

When a protected treasure is discovered, the protection allows the player to profit exclusively from all hexes adjacent to the treasure. A protection boundary is created that signals to the player with the protection and to the other players that there is an active protection. The protection boundary continues to be marked until all the treasures in the mine have been discovered. 

Under the ``No Protection" condition, when a player finds a treasure, this does not restrict the future search of other players. However, if two or more players choose the same hexagon simultaneously, the payoffs that each player obtains follow this rule: if two players find the same treasure, each of them obtains 0.2 from the original reward of this treasure (which amount to 64 if this is the first treasure in the mine, and 16 if this is the second or the third). If three players find the same treasure, each of them obtains 0.05 from the original reward and if four players find it, each of them obtains 0.\footnote{This rule was designed to account for the fact that real life competition decreases the total producers surplus.} After choosing a hexagon, it is colored as in the case of the Protection condition.

See Figures~\ref{fig:screenshotpatent} and \ref{fig:screenshotnopatent} for screenshot examples. E.g. in Fig.\ref{fig:screenshotpatent}  we can see some failed searches, one mine that was fully discovered by the current player, and two mines that are partially discovered: one protected by the current player (with a single discovered treasure); and one protected by another player (with two treasures discovered out of three).  In Fig. 2 we can see two fully discovered mines, where the current player managed to obtain some of the profit.\footnote{Examples of screenshots of typical end games in both conditions are presented on Appendix~\ref{apdx:screenshots}}

\begin{figure}[H]
\includegraphics[width=0.6\textwidth]{"patent".png}
\centering
\caption{A screenshot of the game, the Protection condition. Black hexagons represent failed searches, red hexagons are treasures that were found by other players and yellow hexagons are hexagons that were found by the player himself.}
\label{fig:screenshotpatent}
\end{figure}
 \begin{figure}[H]
\includegraphics[width=0.6\textwidth]{"no patent".png}
\centering
\caption{A screenshot of the game, the No Protection condition. Notice that in this condition, there are no protected areas thus each mine can be discovered by more than one player.}
\label{fig:screenshotnopatent}
\end{figure}

In addition, we refer to a control ``Singleton"  condition (see Figure \ref{fig:screenshotsingleton}  below).\footnote{A similar approach was taken by~\citet{levy2018understanding} in a different setting where the researchers study the effect of competition on the players' behavior in simple contests.}

\begin{figure}[H]
\includegraphics[width=0.6\textwidth]{"singleton map".png}
\centering
\caption{A screenshot of the game, the Singleton condition.}
\label{fig:screenshotsingleton}
\end{figure}

under which, each player plays as a singleton player, completely unaffected by other players  (his payoff and his board is independent of the other players' choices).  
The players can only observe their own treasures (colored yellow) and failed exploratory efforts (colored black).


\section{Theoretical Analysis}\label{sec:theory}
In this section we present an analytic solution from a game theoretical point of view. The theoretical analysis assumes Fully Informed Bayesian Players (FIBP) play the treasure hunt game and know the a-priori payoff structure (including the probability of finding a treasure in a new mine). It is important to note that in the real world, full information is rarely available to the operating agents (e.g., innovators) and thus in the experimental investigation presented in section \ref{sec:experiment}, the game was played without providing real participants a-priori full information about the incentive structure (see Section \ref{sec:experiment} for the information provided to real players in the experiment). Accordingly, we refer to the analytic solution as a benchmark for optimality under simplifying assumptions of full information. A final simulation is then used to estimate the game results of this benchmark solution.

The first decision each FIBP takes at the beginning of each round is whether to explore a new hexagon or not given the current state of the board and the costs of exploration.

A deterministic strategy is essentially a threshold in each of these board state cases. Hence, our main objective is to investigate the optimal threshold cost, for each stage of the game, for which the FIBP should explore if and only if the exploration cost in a specific round is lower than the threshold\footnote{Notice that a higher threshold value implies more exploration.}.


In the Protection condition, we can treat each specific exploration stage in the current mine, and compute an appropriate threshold. In the No Protection condition, we roughly divide it into two main situations: (1) search for a first treasure and (2) search for subsequent treasures. We are interested in specifying how this threshold changes under the different experimental conditions. 

In both cases we analyzed optimal (or equilibrium) strategies for FIBP assuming an infinite board and infinite time. The meaning of an infinite board is that without prior knowledge, there is a 5 percent probability of finding treasure in each hexagon. The meaning of infinite time is that, from the player's point of view, the optimal strategy cannot be calculated by calculating backwards from the last step in the game.


\subsection{Protection}

In this section we analyze the optimal exploration strategy under the Protection condition. After being granted Protection, a player can exclusively explore around his first treasure. Therefore, our approach is to examine what is the best threshold for each exploration stage. Notice that the optimal exploration strategy in each stage is affected by the exploration strategy that will be taken in the future during the search prosses at the same mine. For this reason, we calculate the optimal behaviour backwards- from finding the third treasure to searching for the first one.   To keep the text more fluent, we present here only the bottom line results, and present the calculation in Appendix~\ref{apdx:patent}.

\begin{prop}
Over an infinite time span, the optimal exploration threshold under the Protection condition in each stage of the exploration process is $20.42$. 
\end{prop}
In fact, we show that in this stochastic search game the cost threshold is fixed in all rounds. To show the robustness of this result, we prove that it also holds in a somewhat simpler game, a 2-step deterministic game with general reward and uniform cost. The proof can be found in Appendix \ref{apdx:patent}.

Intuitively, the threshold does not depend on the current state of the board (i.e. whether the player searches for first or subsequent treasure) since the alternative cost of waiting, which is the loss of one round in the game, is constant in all exploration stages. Therefore, it makes sense that the threshold value will stay the same within each stage. 


\subsection{No Protection}
Under the No Protection condition, we consider two scenarios. The first is when the players explore for a first treasure and the second is when one of the players find a treasure and all the players can use the information about the treasure's location, in order to find subsequent treasures in the same mine. 

We start the analysis with the second case, and then calculate backwards the cost threshold in the first case. The challenge in the second case is to compute the cost thresholds and the expected payoffs in equilibrium when these depend on the other players' choices.

The decision to explore around a first treasure yields two types of revenues. The first is the expected payoff from the treasure itself. This revenue decreases with the number of players that explore around this mine, as explained above.

The second type of gain from exploration is the information gain. Whether the player finds a treasure or not, any search reveals information that helps narrow down the search in the next round. For example, after a player finds a first treasure in an unexplored area, both successful and failed subsequent searches increase the probability of finding a treasure in the next round from 0.33 to 0.5.\footnote{One failed subsequent search left 4 options for the next 2 treasures (notice they both have to be linked) so if the next search is one of the 3 hexagons that are not linked to the failed search, the payer has probability of 0.5 to find a treasure.}

The value of this information depends on its distribution among the players, and on their future choices. In addition, the players' choices (the cost thresholds they pick and the hexagons they choose) depend on their expected future payoffs. 

Analyzing all the different scenarios is complicated, and the accurate number is likely not as important as the boundary values.
Therefore, we focus on computing lower and upper bounds to the cost threshold. We describe the result in the main text, and leave the detailed calculation for Appendix~\ref{apdx:nopatent}. 
\begin{prop}
Over an infinite time span, 
\begin{enumerate}
    \item the equilibrium cost threshold in the case of searching for a subsequent treasure is between 21.28 and 25.1,
\\ and
    \item the equilibrium cost threshold in the case of searching for a first treasure is between 16 and 16.5.
\end{enumerate} 
\end{prop}

To summarize, optimal/equilibrium searching cost thresholds by condition are:
\begin{center}
 \begin{tabular}{c|| c c } 
 \hline
  & First treasures & Subsequent treasures  \\ [0.5ex] 
 \hline\hline
 Protection & 20.42 & 20.42 \\ 
 \hline
 No Protection & 16-16.5 & 21.28-25.1  \\
 \hline

\end{tabular}
\end{center}


\subsection{Singleton}
Under the Singleton condition, each player plays in his own hive, and his payoff is independent of the other players' choices. The analysis is the same as in the Protection condition, since, from a theoretical point of view, searching for a first treasure is independent of the other players choices, and searching for subsequent treasure as a singleton player, is the same as the exclusivity search in the Protection condition.

\section{Simulation Results}\label{sec:simulations} 
We programmed artificial Fully Informed Baysian Players (FIBP) in both conditions, and let them play ``The Competitive Treasure Hunt" game, in order to supply provide a theoretical prediction regarding the players' performance in the game. A player is defined by a pair of thresholds: cost thresholds for exploration for first and for subsequent treasures. We focus on symmetric strategies, i.e. within each simulation, all players use the same combination of strategies. The number of treasures found and their payoffs were the outputs. We repeated the game 10,000 times for each possible combination of thresholds. 

The results are presented in Appendix~\ref{apdx:simulation}, and includes the number of treasures each player found, the number of treasures each group of players found and the players' payoff. 

The simulations show that under a rational behavior assumption FIBP find more treasures in the Protection condition, both at the group and at the individual levels. In addition, the simulations show that the number of treasures found by more than one player, which indicates the inefficiency of exploration, increases as the thresholds increase in the No Protection condition, since higher thresholds result in more search activity over a limited area. The results of this analysis provide the theoretical prediction that under profit maximizing assumption, exploration will be more efficient (less effort leads to more discoveries) under the Protection conditionizing assumption, exploration will be more efficient (less effort leads to more discoveries) under the Protection condition.
 

\subsection{Theoretical Predictions}
Following the theoretical analysis and simulations, we present three theoretical predictions under profit maximization and full information assumptions:


\textit{\textbf{Theoretical Prediction 1:}} Protection increases exploration activity for first treasures.

\textit{\textbf{Theoretical Prediction 2:}} Protection decreases exploration activity for subsequent treasures.

\textit{\textbf{Theoretical Prediction 3:}} Protection increases the efficiency of exploration.

The first and the second theoretical predictions stem directly from the theoretical analysis, by comparing the thresholds of rational FIBP. Intuitively, the first prediction stems from the fact that protection of initial discoveries increases the expected payoffs for the initial discoverers, and therefore the searching cost threshold is higher under the protection condition. The second prediction stems from the fact that searching for subsequent treasures is less profitable under no protection, since the total payoff is shared among all the players who find it, whereas the full cost is paid by each. The third theoretical prediction is derived from the simulation results that show differences between the performance in the individual and group levels, evident in duplicated treasures.  




\subsection{Behavioral effects}
It is important to note that, the theoretical predictions were based on the assumption of FIBP who know the a-priori probability of finding a treasure and follow the optimal exploration threshold from start. In reality (and in our lab experiment, section 5), the a-priori probability is unknown to the competing players in advance, who can only learn it through experience. 

Under such partial information, rational players may take time to converge to a consistent exploration threshold, but assuming effective learning this threshold should eventually approximate the optimal one. However, if participants value the likelihood of profit over its magnitude (in line with under-weighting of rare events findings in Decisions-from-Experience \citet{barron2003small,hertwig2004decisions,camilleri2011and,erev2014maximization,plonsky2015reliance,teodorescu2021frequency,teodorescu2014learned,teodorescu2014decision}) this could alter our hypotheses. First, in the case of initial discoveries, where the probability of finding a treasure is low, we expect the increase in exploration activity due to protection to be \textbf{weaker} in the actual data, and in contrast to theoretical prediction (1). Second, in the case of subsequent discoveries, where the probability of finding a treasure is high (0.3-1), we expect exploration rates will be higher than the theoretically optimal ones in both conditions. Since this effect is expected to be similar under both the Protection and the No Protection conditions, it is hard to determine how the behavioral effect would change theoretical Prediction 2. As for theoretical Prediction 3, over-exploration for subsequent treasures does not change the efficiency of exploration in the Protection condition, since players act as singletons, and the problem of overlapping exploration does not exist. Nevertheless, as the exploration rate for subsequent treasures increases, exploration becomes less efficient in the No Protection condition since duplicated exploration efforts become more likely (See Figure~\ref{fig:duplicated} in Appendix~\ref{apdx:simulation}). Therefore, we expect to find a stronger positive effect of protection on the efficiency of exploration, compared to the theoretical analysis.  


\section{The Lab Experiment}\label{sec:experiment}
\subsection{Method}
\subsubsection{Participants}
One-hundred and fifty four (81 Female) 
Technion and Ben Gurion University 
students, with an average age of 25, participated in the study in exchange for monetary compensation. The study included participants aged 18 and older who signed a consent form to participate in the experiment. The forms were signed by hand on a page in front of the research team and kept in the laboratory. The study was carried out between January 21, 2019 and April 28, 2019.
They were divided into groups of 4, for a total of 15 groups in each experimental condition, plus 34 participants who
played in the singleton condition\footnote{One student was mistakenly invited to the lab twice, and therefore her second session was removed.}. 

A performance based payment was added to (if positive) or subtracted from (if negative) a show-up fee of 30 NIS. 
\footnote{Participants obtained a total of 29 NIS (that equals about \$8.3) on average, in a game lasting around 35 minutes. Note that the mean payoff is lower than the show-up fee, which means that on average, the performance based payment was negative. This is a first indication that participants did not behave optimally (they could guarantee the show-up fee by skipping all rounds).}

\subsubsection{Experimental Design}
Participants played a lab adaptation of the ``The Competitive Treasure Hunt" game that was described in Section~\ref{sec:golddigger}. The lab adaptation of the game included the following differences from the theoretical analysis:

\begin{enumerate}
  \item The hive included 2100 hexagons (70X30), rather than infinite number of hexagons. This modification implies that after each round, information is revealed and (1) the probability of finding a first treasure decreases after a treasure is found (due to fewer remaining treasures) as well as (2) the probability of finding a first treasure increases after a failed search only for the player making the move, since players can observe all successful searches, but only their own failures. Consequently, the overall probability of finding a first treasure tends to decrease over time. However, since around each gold mine there are several known empty hexagons, the overall probability decreases only slightly.\footnote{We computed the probability of finding a first treasure in the last round of each game, and obtained on average the probabilities 0.0445, 0.0458 and 0.0492 in the Protection, No Protection and Singleton conditions, respectively. As previously described, the probability decreases on average over time, so these numbers estimate the minimum probability in each game. We can thus see that these probabilities are relatively close to 0.05.}
  \item Each game included 50 rounds, rather than infinite number of rounds. There is strong evidence that although in traditional game theoretic analysis any finite horizon may completely change the structure of equilibria, human players only take this into account (end-game effect) very close to the actual termination. E.g. in \citet{normann2012impact} end-game effect was explicitly measured only in the last 3 rounds out of 22 of the Repeated Prisoners' Dilemma. They also compared behavior under known, unknown, and random termination rules and find that differences in behavior only start $\sim$10 rounds before termination. 
  Moreover,  RPD is a deterministic game. Adding randomness to the game (as in our case) substantially reduces endgame effect, since it negates the value of looking ahead in general. E.g. while medium-level Chess programs typically consider $\sim$15 steps ahead, the Backgammon programs only need to look 2 steps ahead in order to beat the best human players \citep{tesauro1994td}.
  As we explain below, to compare participants’ behavior to the theoretical, infinite time horizon benchmark, we excluded the last 12 rounds in each game from the analysis.
  \item Players were not informed in advance of the probability to find a treasure. As noted, with sufficient experience the learnt probability to find a treasure should converge to the actual one and lead rational players to a stable optimal threshold. The experiment included 4 games of 50 rounds each, which should allow for sufficient learning.\footnote{Indeed, analysis of potential changes throughout the games revealed quick learning and no significant differences between the first and the last game participants played.}
\end{enumerate}

The two conditions (Protection, and No Protection) were manipulated between groups. As described in Section~\ref{sec:golddigger}, the main difference between the conditions is who can potentially gain further from a first treasure find. Under the Protection condition, only the player who finds the first treasure can gain from finding the second and the third ones, and under the No Protection condition, all the players can gain from it. 

In addition, under the Protection and No Protection conditions each player can observe the other players' successes but cannot observe their failures. 
The exploration costs were randomized between and within participants (i.e., across rounds). 

For each game, the computer chose randomly one of ten different possible ``maps" of treasures. 

\subsubsection{Procedure}
In each experiment's session, students invited to the lab were randomly assigned into groups of four. Each group was randomly assigned into the ``Protection" or the ``No Protection" condition. 
All the remaining participants were assigned to the ``Singleton" condition. 
Each participant played four games with 50 rounds per game.\footnote{18 participants from the Singleton condition played 5 rather than 4 games. For those participants, we analyzed only the first four games played.}

The players were not informed about the other players' identities, but they did know their group's size. 

The participants received three pages of instructions, which included pictures of different states of the game with explanation about the shape of mines and the meaning of different colors and marked areas (see Appendix~\ref{apdx:instructions}). Participants were informed about the structure of mines (i.e. a tight triangle), but not about their frequency and their location in the hive. In the Protection condition, the instructions explained that when a player finds the first treasure, he obtains an exclusive right to explore surrounding (adjacent) hexagons and benefit from the subsequent treasures. In the No Protection condition the instructions explained that the more players who find the same treasure, the lower the reward it yields. To make sure that the players understood the instructions well, before starting the first game they had to answer a short quiz with questions concerning the instructions of the game. The game started only after all the examinees answered all the questions correctly.
We did not mention any economic or domain specific terms such as ``protection", ``innovation" etc in any of the condition’s instructions.

Note that in all conditions, the players receive the same instructions (except for the introduction of protection rules in the protection condition), and the treasures location as well as the payoff procedure were the same. 


 The game progressed as follows. First, the computer displayed the hive, containing 2100 hexagons. The players received a message stating the exploration cost for this round, and asking each player if he wants to skip the round or to explore under the current exploration cost. After making their choice, players were asked to wait for the other players to make their choices.\footnote{This message appeared also in the Singleton condition to match this condition to the other conditions.} 

If a player decided to skip, he gained 0, and the round was over for him. If a player decided to explore, he could choose one of the hexagons in the hive that was not yet colored. At the end of each round, the players received a message with their payoff from the round, calculated as the reward obtained minus the current exploration cost.   

\subsubsection{Data Analysis}

Our first objective is to examine the decision to explore, and how it is affected by the different situations in the game. We are also interested in the players' performance, reflected by the number of treasures found and their payoffs.

In order to measure the correlation between participants and groups, we analyzed the data using a linear mixed effects model (LMM). We allow random intercepts for the players' ID and for the group of players.



\subsection{Experimental Results}

\subsubsection{Observed Exploration Rates}\label{sec:searchbhvr}
As in the theoretical analysis, we distinguish between two situations in the game: the first state is when the player can only explore for a first treasure in a new mine. This situation arises when there are no open mines (i.e., partially exposed mines) that the player can exploit for subsequent searches. These rounds are categorized as ``initial search" rounds, for which the probability of finding a new treasure is 0.05. The second state in contrast, is defined by the existence of open mines, and the player’s ability to search for a subsequent treasure around a discovery made in previous rounds. These rounds are categorized as “subsequent search rounds”, for which the probability of finding a treasure is equal to or higher than 0.33.\footnote{Notice that in the No Protection condition, all players are always simultaneously in the same state because the ability to search for subsequent treasures is open to all players regardless of who found the first initial treasure. In contrast, in the protection condition the states are defined for each player independently, because one cannot exploit protected open mines of other players.}

Each state of the game is analyzed separately.


\paragraph{Initial search:} The first question addressed is how protecting first treasures affects exploration activity for initial search. Table~\ref{table:Searching}, column 1 presents the results of the exploration rates for first treasures. It shows that the coefficient of $Protection$ variable is insignificant, suggesting that when players explore for the first treasure in each mine, there is no significant difference between their behavior under the Protection and the No Protection conditions.

Figure~\ref{fig:first} presents the exploration rate for first treasures, showing there is slightly more exploration activity under the Protection condition. However, as previously shown, this effect is not significant.



According to theoretical Prediction 1, protection will increase exploration activity for first treasures. However, our data analysis does not support this prediction, as presented in Result 1.  

\textit{\textbf{Result 1:}} Protection does not significantly affect the exploration rate for first treasures.

\paragraph{Discussion:} We showed in section~\ref{sec:theory} that the expected reward from a mine in the Protection condition is higher than in the No Protection condition.\footnote{Therefore the cost threshold of first treasures is lower in the No Protection condition.} This result is also obtained from actual data analysis (see Appendix~\ref{apdx:minepayoff}). Hence, the interpretation of Result 1 is that the expected reward from a treasure has less influence on the decision to explore than the theoretical analysis would predict.

To check the robustness of this result, we ran the Singleton condition (which is similar to the Protection condition from a theoretical point of view) 
using two levels of reward for a first treasure: 320 (which is the same as the reward for first treasures in the Protection and No Protection conditions) and 260. The results show no significant effect of the reward level on the exploration rates, supporting the interpretation of Result 1 whereby players are under-sensitive to the reward level (see supporting evidence in Appendix~\ref{apdx:levelsingleton}). 


Result 1 is in line with studies in the DfE literature (e.g., \citet{teodorescu2014learned,teodorescu2014decision} which found that the frequency of a reward is more important than its magnitude in repeated settings where the environment is learned from experience. This finding can explain why the theoretically predicted positive effect of protecting first treasures on exploration is not significant. One may argue that the probability to find a radical innovation in the real world is lower than 5\% (which was the probability to find a first treasure in our game). However, according to the underweighting of rare events phenomenon, a lower probability will only increase underweighting and strengthen Result 1 \citep{teodorescu2021frequency}.

To further investigate the deviation of observed exploration rates from the theoretical predictions, we also compared the observed insignificant effect of protection to the predicted effect of protection, as implied by the theoretical benchmark. The difference between optimal thresholds in searching for first discoveries with and without protection is about 4 (16.5 vs. 20.42), thus according to the theoretical benchmark, the predicted difference in exploration rates is 13\%.\footnote{Probabilities can be easily derived from thresholds due to the uniform distribution of search costs.} The observed exploration rates for first discoveries with and without protection was 0.60 and 0.53, respectively, implying an observed difference of 7\%, which is significantly lower than 13\% (t(df)=- 2.56, p<0.01),\footnote{This result was obtained using each game as one observation. When we average all four games per individual, the observed difference in exploration rates from the predicted 0.13\% difference, becomes only marginally significant smaller than the predicted 0.13\% difference (T(df)=- 1.46, p= 0.07).} in line with the underweighting of rare events phenomenon.


%Notice that risk aversion and/or diminishing sensitivity to higher gains can also lead to behavior in line with under-weighting of rare events in the current environment (i.e. rare treasures). However, it is important to note that underweighting of rare events has been also observed in other environments, which cannot be explained by risk aversion (e.g. in rare disasters environments, people suboptimally prefer the risky option, see for example \citet{erev2008loss,newell2016rare}). Accordingly, other explanations, mainly ones that refer to the learning mechanism, are more commonly used to explain why people behave as if they underweight rare events (e.g. \citet{rakow2008biased,erev2012surprise,plonsky2015reliance}). 


\paragraph{Sequential search:}
We next analyze the observed exploration rates for subsequent treasures. Qualitatively, there is slightly more exploration activity for subsequent treasures under the No Protection condition than under the Protection condition.\footnote{Exploration rates of 0.72 and 0.76 in the Protection and the No Protection conditions, respectively.}
However, observations in the case of exploration for subsequent treasures are highly correlated within participants. The decision to explore in each round depends on the successful and unsuccessful previous attempts to find a subsequent treasure in that mine and in the distribution of information among the players. Therefore, a na\"ive comparison of the conditions is inadequate.

It is appropriate to examine the effect of protecting first treasures on exploration for subsequent treasures by comparing the Protection and No Protection conditions, only for the cases where a player found a first treasure in the previous round. In this case, the observations are not correlated within participants, since only one decision in each mine's discovery process is taken into account. This comparison also allows us to ignore the asymmetric definitions for ``subsequent" treasure between the Protection and No Protection conditions.\footnote{A ``subsequent" treasure is defined under the Protection condition as a case where the player himself found a treasure in the previous rounds, and now he can explore for the second or the third treasure in the mine, and in the No Protection condition as a case where \textbf{any one} of the players has found a treasure in previous rounds, and exploration for a second or a third treasure is now available for all players.} 
In other words, we investigate the decision to explore only in observations where the player found a first treasure in a previous round, in both the Protection and No Protection condition. 

Table~\ref{table:Searching} column 2 presents the results. It shows that protection on first treasures significantly decreases the overall tendency to explore for a subsequent treasure by 10 percent ($p<0.05$). According to theoretical Prediction 2, protection decreases the exploration rate for subsequent treasures and data analysis corroborate theoretical Prediction 2, as presented in Result 2.

\textit{\textbf{Result 2}}: Protection significantly reduces exploration efforts for subsequent treasures.



Figure~\ref{fig:ceiling} shows that there is a ceiling effect for exploration under low exploration costs. Therefore, we estimated this model only for costs that are higher than 20, as presented in Table~\ref{table:Searching} column 3. Limiting the analysis only to high costs leads to a stronger effect.

\paragraph{Discussion:}
Our findings show that protecting subsequent searches around initial discoveries has a negative effect on the tendency to explore for subsequent discoveries. This is in line with theoretical predictions that suggest higher cost threshold in the Protection condition for exploring subsequent discoveries.
In the game, sequential treasures are organized in a specific structure and all players know the possible ways to find them. In this case less search activity means slower discovery process. In reality, however, subsequent discoveries are not a certain result. Therefore, sequential discoveries may be completely missed altogether due to the reduction in search efforts.

\begin{figure}[H]
  \centering
  \begin{minipage}[t]{0.45\textwidth}
  \textbf{Exploration rate for first treasures}\par\medskip
    \includegraphics[width=\textwidth]{search_first.png}
    \caption{Exploration rates for first treasures in the Protection and the No Protection conditions.}
    \label{fig:first}
  \end{minipage}
  \hfill
  \begin{minipage}[t]{0.45\textwidth}
    \textbf{Exploration rate for subsequent treasures}\par\medskip
    \includegraphics[width=\textwidth]{search_sequential.png}
    \caption{Exploration rates for subsequent treasures in the Protection and the No Protection conditions. Only the first search after the player found a first treasure was accounted.}
    \label{fig:ceiling}
  \end{minipage}
\end{figure}

\begin{table}[H]
\begin{center}
\begin{tabular}{l c c c }
\hline
 & search & search & search \\
 & for first treasure & for self-subsequent treasure & for self-subsequent treasure\\
 & & & for high-costs only\\
\hline
(Intercept)                 & $1.11^{***}$  & $1.22^{***}$  & $1.45^{***}$  \\
                            & $(0.04)$      & $(0.05)$      & $(0.11)$      \\
Cost                           & $-0.03^{***}$ & $-0.02^{***}$ & $-0.02^{***}$ \\
                            & $(0.00)$      & $(0.00)$      & $(0.00)$      \\
Protection             & $0.06$        & $-0.10^{*}$   & $-0.17^{*}$   \\
                            & $(0.05)$      & $(0.05)$      & $(0.07)$      \\
\hline
AIC                         & 14734.27      & 273.42        & 305.32        \\
BIC                         & 14781.58      & 299.04        & 327.84        \\
Log Likelihood              & -7361.13      & -130.71       & -146.66       \\
Num. obs.                   & 19659         & 529           & 315           \\
Num. groups: ID             & 119           & 111           & 104           \\
Num. groups: GroupIndex     & 30            & 30            & 30            \\
Var: ID (Intercept)         & 0.06          & 0.02          & 0.06          \\
Var: GroupIndex (Intercept) & 0.00          & 0.01          & 0.01          \\
Var: Residual               & 0.12          & 0.08          & 0.10          \\
\hline
\multicolumn{4}{l}{\scriptsize{$^{***}p<0.001$, $^{**}p<0.01$, $^*p<0.05$}}
\end{tabular}
\caption{Search rate for first (column 1) and subsequent (column 2,3) treasures, column 3 presents search rate in high cost only.}
\label{table:Searching}
\end{center}
\end{table}



\subsubsection{Actual Exploration Costs Thresholds }
Another important question is if players behave strategically, i.e. do they use a consistent cost threshold in different situations of the game, and is this threshold the same as the one suggested by the theoretical analysis. As before, we classify the game into two main groups: exploration for first treasures and exploration for subsequent treasures. In the Singleton condition, we consider only cases where the payoff structure from treasures is the same as the Protection condition.

We computed decision thresholds for first and subsequent searches for every player using the methodology described in Appendix~\ref{apdx:actualthreshold}.
Figure~\ref{fig:threshold} presents the distribution of thresholds over the players in each condition and each state of the game. Dots represent optimal/equilibrium thresholds computed from Section~\ref{sec:theory} and a range of values in the No Protection condition.
The data has a right censored normal distribution, reflected in inflated right tail (at the top of the figure), especially in the subsequent treasures cases, probably due to specifying the cost range and limiting it with an upper bound. Therefore we used the Wilcoxon test for non-parametric statistical hypothesis tests. Cases with significant differences between observed and optimal/equilibrium thresholds are marked by a star. The 12 last rounds from every game were removed to bring the real game setup closer to the theoretical assumption of an infinite time horizon. That is, the maximum observed number of rounds needed to reveal a mine was 12 rounds, hence, a player could approximately adopt a strategy assuming an infinite time horizon up until round 38. We also removed from the data the few rounds where players could explore for subsequent treasures, but decided to explore for first treasures. Thresholds were computed over all four games, since there were no significant differences in the players' behavior between games.

\begin{figure}[H]
    \centering
    \includegraphics[width=0.7\textwidth]{Thresholds.png}
    \caption{Threshold distribution by condition and by the state of the game. Dots represent optimal/equilibrium thresholds. Stars ($*$) represent significant differences between median value of observed threshold and optimal/equilibrium threshold, determined by the Wilcoxon test. Horizontal lines represent data deviation in quartiles.}
    \label{fig:threshold}
\end{figure}




\paragraph{Results:}
The basic result from the empirical analysis is that most of the players hold a relatively consistent cost threshold (evidence is provided in Appendix~\ref{apdx:actualthreshold}).   

In addition, comparing the optimal thresholds implied by the theoretical benchmark analysis and the observed thresholds provides several insights. First, in all conditions players applied higher thresholds for subsequent treasures (where the probability to find a treasure is relatively high) than the optimal ones suggested by the theoretical benchmark analysis. Moreover, in the Protection and the Singleton conditions, they used significantly higher thresholds for subsequent treasures than for first treasures, even though the optimal benchmark thresholds are the same. 

\paragraph{Discussion:}
One potential contributor to the deviations of the observed behavior from the theoretical benchmark  is a behavioral phenomenon according to which during learning people are more sensitive to the frequency of obtaining a reward from exploration than to its average magnitude \citep{teodorescu2014learned,teodorescu2014decision}. In both initial and  subsequent treasures the players' payoffs on average were the same, but in subsequent treasures the probability of making a discovery was much higher.  

Figure~\ref{fig:threshold} also shows that under the Protection condition players tend to use higher threshold values than in the No Protection condition when they explore for first treasures, and lower thresholds when they explore for subsequent treasures. Although not significant, this effect is in line with the results presented in section~\ref{sec:searchbhvr}. 


\subsubsection{Efficiency}
In this section we investigate the relationship between exploration efforts and performance, and examine the efficiency of the exploration process.

First, we examine the cost per treasure found as a measure to search efficiency. The results are presented in figure~\ref{fig:costPerTr}

\begin{figure}[H]
    \centering
    \includegraphics[width=0.3\textwidth]{"cost per treasure".png}
    \caption{The cost per treasure found}
    \label{fig:costPerTr}
\end{figure}

One source of inefficiency in the No Protection condition is that players cannot observe each other's failures when they explore for sequential discoveries. Accordingly, the same empty hexagon next to a first discovery is likely to be explored by multiple players because they did not observe previous failures. This inefficiency is prevented under the Protection condition, since only one player explores around each mine. so he can learn from his failures and avoids from revisiting an ``empty" hexagon. 

Another source of inefficiency in the No Protection condition is that players cannot gain the full return from their failed exploration. Each failed exploration yields new information about the location of subsequent treasures. Under the Protection condition players can use this information with probability 1 to find these treasures, but under the No Protection condition other players may find those treasures before them, so the player who paid the exploration cost will not necessarily enjoy its full return.


The interpretation of figure \ref{fig:costPerTr} corroborates theoretical Prediction 3 that predicts more efficient exploration under the Protection condition, and is reflected in Result 3.

\textit{\textbf{Result 3:}} Protection increases the efficiency of exploration.


Figure~\ref{fig:NroundFailures} presents the number of failures as a function of the number of rounds. On one hand, the lack of competition in the Protection condition led to fewer failed exploratory attempts and therefore to more efficient exploration. On the other hand, treasures were revealed slower in the Protection condition. Presumably, competition in the No Protection condition forced players to invest more in exploration, since it allowed more players to participate in the exploration process.\footnote{For ease of viewing, in both figures we insert uncorrelated white noise, normally distributed with mean 0 to the observed data, so that points do not overlap.}


\begin{figure}[H]
\includegraphics[width=0.6\textwidth]{"search_efficiency".png}
\centering
\caption{The number of failures, as a function of the number of rounds.}

\label{fig:NroundFailures}
\end{figure}

\paragraph{Discussion:} Granting protection on a first treasure plays the role of marking the territory for the first finder, and signals other players to explore elsewhere. By doing so, it increases the coordination among the players. This allows the first finder to invest their exploration efforts more carefully, and to search more efficiently by maximizing the information gain from their successes as well as from their failures.    

\citet{bessen2009sequential} discussed complementary research and its effect on the patent protection efficiency. They defined complementary research as a case where
\begin{quote}
    ``each potential innovator takes a different research line and thereby enhances the overall probability that a particular goal is reached within a given
time"
\end{quote}
 They claimed that in the case of sequential and complementary research, the inventor and the society would be better off with no patent protection since
 \begin{quote}
    ``it helps the imitator develop further inventions and because the imitator may have valuable ideas not available to the original discoverer"
\end{quote}
In our design, this is reflected in sharing searching opportunities with players who have lower current costs, or more information regarding the subsequent treasures' location.  
As in \citet{bessen2009sequential}, when two players explore simultaneously, the probability of finding a treasure is higher than the same probability when only one player explores, but it is less than double. We argue that the result of \citet{bessen2009sequential} still holds in our framework. 
We observed more exploration activity under the No Protection condition and the players benefit from each other's treasures. However, since the players were not able to benefit from each other's failures (exploration is complementary, but not perfectly complementary) the advantage of the no protection policy comes with the price of inefficient exploration which results in less treasures being found. 

 As can be seen in Figure~\ref{fig:threshold}, players explored for subsequent treasures more than they would under the optimal strategy. This result is in line with the underweighting of rare events phenomenon. However, this behavioral phenomenon affects exploration activity of the Protection and No Protection conditions in the same direction, so it is not possible to delineate how it affects the way protection changes exploration behavior for subsequent treasures. 

Notice however, that underweighting of rare events affects the way protecting first discoveries influences the efficiency of exploration. In the No Protection condition, the fact that more players decide to explore for subsequent treasures increases duplicated exploration, and exacerbates
the problem of players that cannot fully benefit from their failures. Thus, in the absence of protection, underweighting of rare events indirectly leads to increased exploration inefficiency (See figure~\ref{fig:duplicated} in Appendix~\ref{apdx:simulation}).

Nevertheless, over-exploration has no influence on the efficiency of exploration in the Protection condition, since the players play as singletons when they explore for subsequent treasures, so there is no reason for duplicated exploratory efforts or sub-exploitation of failures.
Therefore, underweighting of rare events makes the positive effect of protection on exploration efficiency even stronger.

Obviously, changing the payoff structure would affect Result 3. For instance, if there were more than two subsequent discoveries for each initial discovery, the inefficiency search problem in the case of no protection condition would get worse, but also the slow finding problem in the case of protection condition. With higher payoffs from subsequent treasures, relative to first treasure, more players will choose to search them in the NP condition, which in turn will increase inefficiency. On the other hand, in the  P condition, subsequent treasures will be discovered sooner, since the searching-cost threshold will decrease.

\subsubsection{The Effect of Observing Others' Success}\label{sec:forgone}

In this section we estimate the effect of observing the other players' success on future exploration activity. From a theoretical point of view, in the Protection condition this behavior should not be influenced by successful exploration activity of others.\footnote{Since the hive is large enough, the information gain from revealing other players' first treasure has a minor influence on the probability of finding other first treasures in the Protection condition.} However, data analysis shows that this is not the case in actual behavior. 
 

The Singleton condition is similar to the Protection condition except for one feature: In the singleton game players cannot see the other players' actions.
Therefore, comparing the exploration rates under the Singleton and the Protection conditions can reveal consequences of the effect of observing the other player's treasures on observed exploration rates.  

We estimate this for cases where the players explore for a first treasure. We find that in the Protection condition players explore 9 percentage points more than under the Singleton condition. However, this effect is not significant.

In addition, we estimate the effect of other players' previous round treasure on the exploration rate in the current round in the Protection condition. We find that players explore 2 percentage points more for first treasures after another player finds a treasure in the previous round ($p < 0.01$). 

Further details can be found in Appendix~\ref{apdx:forgone}. 

\paragraph{Discussion:} These findings provide support for the argument in favor of revealing information about new innovations, even when the information cannot be used.

These results are in line with \citep{plonsky2020influence} who found that exposing participants to a positive forgone payoff, increases risky exploration. In our context, observing the other players’ treasures serves as positive forgone payoffs for untaken exploratory directions, which could in turn contributed to the increased exploration rates. 



\section{Conclusions and General Discussion}\label{sec:conclusion}
In this study we developed a new paradigm to investigate the question of the effectiveness of protecting discoveries as a tool to encourage innovation. This paradigm distinguishes between first treasures, that represent initial discoveries, and subsequent treasures, that represent subsequent discoveries. First treasures are found with low probability, and require no previous information. Subsequent treasures are found with higher probability and rely on information obtained from first treasures. 

Our findings show that the benefit of protecting subsequent searches around initial discoveries stems from increasing exploration efficiency, rather than encouraging exploration intensity. While the theoretical benchmark analysis imply that protection should increase the search for first treasures, the observed exploration rates for first treasures did not differ significantly between the experimental conditions. This result is in line with behavioral studies that found that when probabilities are learnt through on going feedback (like in the current experiment and in most real life scenarios), people tend to underweight rare events \citep{ teodorescu2014learned, teodorescu2014decision}. Since finding a first treasure is a rare event, players were under-sensitive to the reward it yields, thus increasing this reward through protecting first treasures did not enhance exploration activity as theoretically predicted.

Furthermore, discovery protection decreased exploration for subsequent treasures. This result is in line with previous studies regarding patent protection (e.g. \cite{bessen2009sequential,boldrin2008against,galasso2014patents}) and attests to the negative aspects of discovery protection i.e. the inhibition of cumulative innovative activity. 

We found that the main advantage of such protection is by improving coordination among players and thereby increasing exploration efficiency. Introducing protection forces a wider distribution of exploration efforts, and reduces duplicated searches. It also allows more efficient exploitation of knowledge about failed exploratory efforts. Hence, other knowledge management mechanisms that maintain these benefits of discovery protection, but without inhibiting competition for innovations, should be considered.

One suggestion can be to encourage communication between explorers about unsuccessful searches. In the case of protecting discoveries, this kind of information can be obtained by employing a market for failed R\&D efforts. Currently, firms' research failures are treated as trade secrets and withholding this information from other researchers leads to an inefficient allocation of exploration efforts. Since information about failure is valuable, allowing a trading mechanism may cause Pareto improvements in the R\&D market.

Another possible alternative to the protection system is to implement an insurance mechanism in the innovation market. Insurance companies could compensate innovators for failed exploratory efforts, and charge a share of their profit from successful exploration. Insurance companies will have an incentive to reveal information about failed exploratory efforts to minimize paying compensation costs to other inventors exploring the same direction. This would improve coordination among innovators, without the social cost of legal monopoly.

In the case of innovating behavior within organizations, researchers can improve their coordination by forming a dedicated forum where they discuss their failures and learn from them. Researchers often tend to cover their failures to avoid bad reputation. Therefore, managers should motivate their employees to share their failures  and draw conclusions for future trials. 

A different policy implication we derive from our findings is that the disclosure of discoveries plays an important role in encouraging innovation. When an inventor observes a successful discovery made by a rival inventor, it encourages him to explore more intensively. Protecting discoveries by granting the researcher who made them an exclusive right to search for subsequent discoveries may increase search efficiency (as our results demonstrate) but protecting discoveries in the sense of keeping their very existence a secret (such as trade secrets) can lower the motivation of others to explore .

Last, it is important to note that investigating the effect of protection through a simplified game setting, bears some limitations. While this approach facilitates the collection of more tractable data, it may reduce ecological validity by not accounting for factors like creativity and inspiration. The simplified setup also restricts the analysis to two boundary regimes—protection and no protection—despite most real-world scenarios lying on a spectrum between these extremes. Additionally, our setting excludes cases where improvements are more valuable than initial discoveries, as with mRNA vaccines, where the adaptation to Covid was more lucrative than the original discovery. We also assumed uniform cost distribution among players and treasures, which may not reflect reality. Furthermore, we did not include licensing in the "Protection" condition, which could allow innovators with lower search costs to buy rights and improve efficiency\citep{phelps2018need}. However, licensing faces challenges like transaction costs and market failures \citep{galasso2014patents}.
Future research could explore the optimal length or area of protection, player heterogeneity, and licensing to improve ecological validity.   



\bibliographystyle{apalike}
\bibliography{patent.bib}
\appendix
\subsection{Lloyd-Max Algorithm}
\label{subsec:Lloyd-Max}
For a given quantization bitwidth $B$ and an operand $\bm{X}$, the Lloyd-Max algorithm finds $2^B$ quantization levels $\{\hat{x}_i\}_{i=1}^{2^B}$ such that quantizing $\bm{X}$ by rounding each scalar in $\bm{X}$ to the nearest quantization level minimizes the quantization MSE. 

The algorithm starts with an initial guess of quantization levels and then iteratively computes quantization thresholds $\{\tau_i\}_{i=1}^{2^B-1}$ and updates quantization levels $\{\hat{x}_i\}_{i=1}^{2^B}$. Specifically, at iteration $n$, thresholds are set to the midpoints of the previous iteration's levels:
\begin{align*}
    \tau_i^{(n)}=\frac{\hat{x}_i^{(n-1)}+\hat{x}_{i+1}^{(n-1)}}2 \text{ for } i=1\ldots 2^B-1
\end{align*}
Subsequently, the quantization levels are re-computed as conditional means of the data regions defined by the new thresholds:
\begin{align*}
    \hat{x}_i^{(n)}=\mathbb{E}\left[ \bm{X} \big| \bm{X}\in [\tau_{i-1}^{(n)},\tau_i^{(n)}] \right] \text{ for } i=1\ldots 2^B
\end{align*}
where to satisfy boundary conditions we have $\tau_0=-\infty$ and $\tau_{2^B}=\infty$. The algorithm iterates the above steps until convergence.

Figure \ref{fig:lm_quant} compares the quantization levels of a $7$-bit floating point (E3M3) quantizer (left) to a $7$-bit Lloyd-Max quantizer (right) when quantizing a layer of weights from the GPT3-126M model at a per-tensor granularity. As shown, the Lloyd-Max quantizer achieves substantially lower quantization MSE. Further, Table \ref{tab:FP7_vs_LM7} shows the superior perplexity achieved by Lloyd-Max quantizers for bitwidths of $7$, $6$ and $5$. The difference between the quantizers is clear at 5 bits, where per-tensor FP quantization incurs a drastic and unacceptable increase in perplexity, while Lloyd-Max quantization incurs a much smaller increase. Nevertheless, we note that even the optimal Lloyd-Max quantizer incurs a notable ($\sim 1.5$) increase in perplexity due to the coarse granularity of quantization. 

\begin{figure}[h]
  \centering
  \includegraphics[width=0.7\linewidth]{sections/figures/LM7_FP7.pdf}
  \caption{\small Quantization levels and the corresponding quantization MSE of Floating Point (left) vs Lloyd-Max (right) Quantizers for a layer of weights in the GPT3-126M model.}
  \label{fig:lm_quant}
\end{figure}

\begin{table}[h]\scriptsize
\begin{center}
\caption{\label{tab:FP7_vs_LM7} \small Comparing perplexity (lower is better) achieved by floating point quantizers and Lloyd-Max quantizers on a GPT3-126M model for the Wikitext-103 dataset.}
\begin{tabular}{c|cc|c}
\hline
 \multirow{2}{*}{\textbf{Bitwidth}} & \multicolumn{2}{|c|}{\textbf{Floating-Point Quantizer}} & \textbf{Lloyd-Max Quantizer} \\
 & Best Format & Wikitext-103 Perplexity & Wikitext-103 Perplexity \\
\hline
7 & E3M3 & 18.32 & 18.27 \\
6 & E3M2 & 19.07 & 18.51 \\
5 & E4M0 & 43.89 & 19.71 \\
\hline
\end{tabular}
\end{center}
\end{table}

\subsection{Proof of Local Optimality of LO-BCQ}
\label{subsec:lobcq_opt_proof}
For a given block $\bm{b}_j$, the quantization MSE during LO-BCQ can be empirically evaluated as $\frac{1}{L_b}\lVert \bm{b}_j- \bm{\hat{b}}_j\rVert^2_2$ where $\bm{\hat{b}}_j$ is computed from equation (\ref{eq:clustered_quantization_definition}) as $C_{f(\bm{b}_j)}(\bm{b}_j)$. Further, for a given block cluster $\mathcal{B}_i$, we compute the quantization MSE as $\frac{1}{|\mathcal{B}_{i}|}\sum_{\bm{b} \in \mathcal{B}_{i}} \frac{1}{L_b}\lVert \bm{b}- C_i^{(n)}(\bm{b})\rVert^2_2$. Therefore, at the end of iteration $n$, we evaluate the overall quantization MSE $J^{(n)}$ for a given operand $\bm{X}$ composed of $N_c$ block clusters as:
\begin{align*}
    \label{eq:mse_iter_n}
    J^{(n)} = \frac{1}{N_c} \sum_{i=1}^{N_c} \frac{1}{|\mathcal{B}_{i}^{(n)}|}\sum_{\bm{v} \in \mathcal{B}_{i}^{(n)}} \frac{1}{L_b}\lVert \bm{b}- B_i^{(n)}(\bm{b})\rVert^2_2
\end{align*}

At the end of iteration $n$, the codebooks are updated from $\mathcal{C}^{(n-1)}$ to $\mathcal{C}^{(n)}$. However, the mapping of a given vector $\bm{b}_j$ to quantizers $\mathcal{C}^{(n)}$ remains as  $f^{(n)}(\bm{b}_j)$. At the next iteration, during the vector clustering step, $f^{(n+1)}(\bm{b}_j)$ finds new mapping of $\bm{b}_j$ to updated codebooks $\mathcal{C}^{(n)}$ such that the quantization MSE over the candidate codebooks is minimized. Therefore, we obtain the following result for $\bm{b}_j$:
\begin{align*}
\frac{1}{L_b}\lVert \bm{b}_j - C_{f^{(n+1)}(\bm{b}_j)}^{(n)}(\bm{b}_j)\rVert^2_2 \le \frac{1}{L_b}\lVert \bm{b}_j - C_{f^{(n)}(\bm{b}_j)}^{(n)}(\bm{b}_j)\rVert^2_2
\end{align*}

That is, quantizing $\bm{b}_j$ at the end of the block clustering step of iteration $n+1$ results in lower quantization MSE compared to quantizing at the end of iteration $n$. Since this is true for all $\bm{b} \in \bm{X}$, we assert the following:
\begin{equation}
\begin{split}
\label{eq:mse_ineq_1}
    \tilde{J}^{(n+1)} &= \frac{1}{N_c} \sum_{i=1}^{N_c} \frac{1}{|\mathcal{B}_{i}^{(n+1)}|}\sum_{\bm{b} \in \mathcal{B}_{i}^{(n+1)}} \frac{1}{L_b}\lVert \bm{b} - C_i^{(n)}(b)\rVert^2_2 \le J^{(n)}
\end{split}
\end{equation}
where $\tilde{J}^{(n+1)}$ is the the quantization MSE after the vector clustering step at iteration $n+1$.

Next, during the codebook update step (\ref{eq:quantizers_update}) at iteration $n+1$, the per-cluster codebooks $\mathcal{C}^{(n)}$ are updated to $\mathcal{C}^{(n+1)}$ by invoking the Lloyd-Max algorithm \citep{Lloyd}. We know that for any given value distribution, the Lloyd-Max algorithm minimizes the quantization MSE. Therefore, for a given vector cluster $\mathcal{B}_i$ we obtain the following result:

\begin{equation}
    \frac{1}{|\mathcal{B}_{i}^{(n+1)}|}\sum_{\bm{b} \in \mathcal{B}_{i}^{(n+1)}} \frac{1}{L_b}\lVert \bm{b}- C_i^{(n+1)}(\bm{b})\rVert^2_2 \le \frac{1}{|\mathcal{B}_{i}^{(n+1)}|}\sum_{\bm{b} \in \mathcal{B}_{i}^{(n+1)}} \frac{1}{L_b}\lVert \bm{b}- C_i^{(n)}(\bm{b})\rVert^2_2
\end{equation}

The above equation states that quantizing the given block cluster $\mathcal{B}_i$ after updating the associated codebook from $C_i^{(n)}$ to $C_i^{(n+1)}$ results in lower quantization MSE. Since this is true for all the block clusters, we derive the following result: 
\begin{equation}
\begin{split}
\label{eq:mse_ineq_2}
     J^{(n+1)} &= \frac{1}{N_c} \sum_{i=1}^{N_c} \frac{1}{|\mathcal{B}_{i}^{(n+1)}|}\sum_{\bm{b} \in \mathcal{B}_{i}^{(n+1)}} \frac{1}{L_b}\lVert \bm{b}- C_i^{(n+1)}(\bm{b})\rVert^2_2  \le \tilde{J}^{(n+1)}   
\end{split}
\end{equation}

Following (\ref{eq:mse_ineq_1}) and (\ref{eq:mse_ineq_2}), we find that the quantization MSE is non-increasing for each iteration, that is, $J^{(1)} \ge J^{(2)} \ge J^{(3)} \ge \ldots \ge J^{(M)}$ where $M$ is the maximum number of iterations. 
%Therefore, we can say that if the algorithm converges, then it must be that it has converged to a local minimum. 
\hfill $\blacksquare$


\begin{figure}
    \begin{center}
    \includegraphics[width=0.5\textwidth]{sections//figures/mse_vs_iter.pdf}
    \end{center}
    \caption{\small NMSE vs iterations during LO-BCQ compared to other block quantization proposals}
    \label{fig:nmse_vs_iter}
\end{figure}

Figure \ref{fig:nmse_vs_iter} shows the empirical convergence of LO-BCQ across several block lengths and number of codebooks. Also, the MSE achieved by LO-BCQ is compared to baselines such as MXFP and VSQ. As shown, LO-BCQ converges to a lower MSE than the baselines. Further, we achieve better convergence for larger number of codebooks ($N_c$) and for a smaller block length ($L_b$), both of which increase the bitwidth of BCQ (see Eq \ref{eq:bitwidth_bcq}).


\subsection{Additional Accuracy Results}
%Table \ref{tab:lobcq_config} lists the various LOBCQ configurations and their corresponding bitwidths.
\begin{table}
\setlength{\tabcolsep}{4.75pt}
\begin{center}
\caption{\label{tab:lobcq_config} Various LO-BCQ configurations and their bitwidths.}
\begin{tabular}{|c||c|c|c|c||c|c||c|} 
\hline
 & \multicolumn{4}{|c||}{$L_b=8$} & \multicolumn{2}{|c||}{$L_b=4$} & $L_b=2$ \\
 \hline
 \backslashbox{$L_A$\kern-1em}{\kern-1em$N_c$} & 2 & 4 & 8 & 16 & 2 & 4 & 2 \\
 \hline
 64 & 4.25 & 4.375 & 4.5 & 4.625 & 4.375 & 4.625 & 4.625\\
 \hline
 32 & 4.375 & 4.5 & 4.625& 4.75 & 4.5 & 4.75 & 4.75 \\
 \hline
 16 & 4.625 & 4.75& 4.875 & 5 & 4.75 & 5 & 5 \\
 \hline
\end{tabular}
\end{center}
\end{table}

%\subsection{Perplexity achieved by various LO-BCQ configurations on Wikitext-103 dataset}

\begin{table} \centering
\begin{tabular}{|c||c|c|c|c||c|c||c|} 
\hline
 $L_b \rightarrow$& \multicolumn{4}{c||}{8} & \multicolumn{2}{c||}{4} & 2\\
 \hline
 \backslashbox{$L_A$\kern-1em}{\kern-1em$N_c$} & 2 & 4 & 8 & 16 & 2 & 4 & 2  \\
 %$N_c \rightarrow$ & 2 & 4 & 8 & 16 & 2 & 4 & 2 \\
 \hline
 \hline
 \multicolumn{8}{c}{GPT3-1.3B (FP32 PPL = 9.98)} \\ 
 \hline
 \hline
 64 & 10.40 & 10.23 & 10.17 & 10.15 &  10.28 & 10.18 & 10.19 \\
 \hline
 32 & 10.25 & 10.20 & 10.15 & 10.12 &  10.23 & 10.17 & 10.17 \\
 \hline
 16 & 10.22 & 10.16 & 10.10 & 10.09 &  10.21 & 10.14 & 10.16 \\
 \hline
  \hline
 \multicolumn{8}{c}{GPT3-8B (FP32 PPL = 7.38)} \\ 
 \hline
 \hline
 64 & 7.61 & 7.52 & 7.48 &  7.47 &  7.55 &  7.49 & 7.50 \\
 \hline
 32 & 7.52 & 7.50 & 7.46 &  7.45 &  7.52 &  7.48 & 7.48  \\
 \hline
 16 & 7.51 & 7.48 & 7.44 &  7.44 &  7.51 &  7.49 & 7.47  \\
 \hline
\end{tabular}
\caption{\label{tab:ppl_gpt3_abalation} Wikitext-103 perplexity across GPT3-1.3B and 8B models.}
\end{table}

\begin{table} \centering
\begin{tabular}{|c||c|c|c|c||} 
\hline
 $L_b \rightarrow$& \multicolumn{4}{c||}{8}\\
 \hline
 \backslashbox{$L_A$\kern-1em}{\kern-1em$N_c$} & 2 & 4 & 8 & 16 \\
 %$N_c \rightarrow$ & 2 & 4 & 8 & 16 & 2 & 4 & 2 \\
 \hline
 \hline
 \multicolumn{5}{|c|}{Llama2-7B (FP32 PPL = 5.06)} \\ 
 \hline
 \hline
 64 & 5.31 & 5.26 & 5.19 & 5.18  \\
 \hline
 32 & 5.23 & 5.25 & 5.18 & 5.15  \\
 \hline
 16 & 5.23 & 5.19 & 5.16 & 5.14  \\
 \hline
 \multicolumn{5}{|c|}{Nemotron4-15B (FP32 PPL = 5.87)} \\ 
 \hline
 \hline
 64  & 6.3 & 6.20 & 6.13 & 6.08  \\
 \hline
 32  & 6.24 & 6.12 & 6.07 & 6.03  \\
 \hline
 16  & 6.12 & 6.14 & 6.04 & 6.02  \\
 \hline
 \multicolumn{5}{|c|}{Nemotron4-340B (FP32 PPL = 3.48)} \\ 
 \hline
 \hline
 64 & 3.67 & 3.62 & 3.60 & 3.59 \\
 \hline
 32 & 3.63 & 3.61 & 3.59 & 3.56 \\
 \hline
 16 & 3.61 & 3.58 & 3.57 & 3.55 \\
 \hline
\end{tabular}
\caption{\label{tab:ppl_llama7B_nemo15B} Wikitext-103 perplexity compared to FP32 baseline in Llama2-7B and Nemotron4-15B, 340B models}
\end{table}

%\subsection{Perplexity achieved by various LO-BCQ configurations on MMLU dataset}


\begin{table} \centering
\begin{tabular}{|c||c|c|c|c||c|c|c|c|} 
\hline
 $L_b \rightarrow$& \multicolumn{4}{c||}{8} & \multicolumn{4}{c||}{8}\\
 \hline
 \backslashbox{$L_A$\kern-1em}{\kern-1em$N_c$} & 2 & 4 & 8 & 16 & 2 & 4 & 8 & 16  \\
 %$N_c \rightarrow$ & 2 & 4 & 8 & 16 & 2 & 4 & 2 \\
 \hline
 \hline
 \multicolumn{5}{|c|}{Llama2-7B (FP32 Accuracy = 45.8\%)} & \multicolumn{4}{|c|}{Llama2-70B (FP32 Accuracy = 69.12\%)} \\ 
 \hline
 \hline
 64 & 43.9 & 43.4 & 43.9 & 44.9 & 68.07 & 68.27 & 68.17 & 68.75 \\
 \hline
 32 & 44.5 & 43.8 & 44.9 & 44.5 & 68.37 & 68.51 & 68.35 & 68.27  \\
 \hline
 16 & 43.9 & 42.7 & 44.9 & 45 & 68.12 & 68.77 & 68.31 & 68.59  \\
 \hline
 \hline
 \multicolumn{5}{|c|}{GPT3-22B (FP32 Accuracy = 38.75\%)} & \multicolumn{4}{|c|}{Nemotron4-15B (FP32 Accuracy = 64.3\%)} \\ 
 \hline
 \hline
 64 & 36.71 & 38.85 & 38.13 & 38.92 & 63.17 & 62.36 & 63.72 & 64.09 \\
 \hline
 32 & 37.95 & 38.69 & 39.45 & 38.34 & 64.05 & 62.30 & 63.8 & 64.33  \\
 \hline
 16 & 38.88 & 38.80 & 38.31 & 38.92 & 63.22 & 63.51 & 63.93 & 64.43  \\
 \hline
\end{tabular}
\caption{\label{tab:mmlu_abalation} Accuracy on MMLU dataset across GPT3-22B, Llama2-7B, 70B and Nemotron4-15B models.}
\end{table}


%\subsection{Perplexity achieved by various LO-BCQ configurations on LM evaluation harness}

\begin{table} \centering
\begin{tabular}{|c||c|c|c|c||c|c|c|c|} 
\hline
 $L_b \rightarrow$& \multicolumn{4}{c||}{8} & \multicolumn{4}{c||}{8}\\
 \hline
 \backslashbox{$L_A$\kern-1em}{\kern-1em$N_c$} & 2 & 4 & 8 & 16 & 2 & 4 & 8 & 16  \\
 %$N_c \rightarrow$ & 2 & 4 & 8 & 16 & 2 & 4 & 2 \\
 \hline
 \hline
 \multicolumn{5}{|c|}{Race (FP32 Accuracy = 37.51\%)} & \multicolumn{4}{|c|}{Boolq (FP32 Accuracy = 64.62\%)} \\ 
 \hline
 \hline
 64 & 36.94 & 37.13 & 36.27 & 37.13 & 63.73 & 62.26 & 63.49 & 63.36 \\
 \hline
 32 & 37.03 & 36.36 & 36.08 & 37.03 & 62.54 & 63.51 & 63.49 & 63.55  \\
 \hline
 16 & 37.03 & 37.03 & 36.46 & 37.03 & 61.1 & 63.79 & 63.58 & 63.33  \\
 \hline
 \hline
 \multicolumn{5}{|c|}{Winogrande (FP32 Accuracy = 58.01\%)} & \multicolumn{4}{|c|}{Piqa (FP32 Accuracy = 74.21\%)} \\ 
 \hline
 \hline
 64 & 58.17 & 57.22 & 57.85 & 58.33 & 73.01 & 73.07 & 73.07 & 72.80 \\
 \hline
 32 & 59.12 & 58.09 & 57.85 & 58.41 & 73.01 & 73.94 & 72.74 & 73.18  \\
 \hline
 16 & 57.93 & 58.88 & 57.93 & 58.56 & 73.94 & 72.80 & 73.01 & 73.94  \\
 \hline
\end{tabular}
\caption{\label{tab:mmlu_abalation} Accuracy on LM evaluation harness tasks on GPT3-1.3B model.}
\end{table}

\begin{table} \centering
\begin{tabular}{|c||c|c|c|c||c|c|c|c|} 
\hline
 $L_b \rightarrow$& \multicolumn{4}{c||}{8} & \multicolumn{4}{c||}{8}\\
 \hline
 \backslashbox{$L_A$\kern-1em}{\kern-1em$N_c$} & 2 & 4 & 8 & 16 & 2 & 4 & 8 & 16  \\
 %$N_c \rightarrow$ & 2 & 4 & 8 & 16 & 2 & 4 & 2 \\
 \hline
 \hline
 \multicolumn{5}{|c|}{Race (FP32 Accuracy = 41.34\%)} & \multicolumn{4}{|c|}{Boolq (FP32 Accuracy = 68.32\%)} \\ 
 \hline
 \hline
 64 & 40.48 & 40.10 & 39.43 & 39.90 & 69.20 & 68.41 & 69.45 & 68.56 \\
 \hline
 32 & 39.52 & 39.52 & 40.77 & 39.62 & 68.32 & 67.43 & 68.17 & 69.30  \\
 \hline
 16 & 39.81 & 39.71 & 39.90 & 40.38 & 68.10 & 66.33 & 69.51 & 69.42  \\
 \hline
 \hline
 \multicolumn{5}{|c|}{Winogrande (FP32 Accuracy = 67.88\%)} & \multicolumn{4}{|c|}{Piqa (FP32 Accuracy = 78.78\%)} \\ 
 \hline
 \hline
 64 & 66.85 & 66.61 & 67.72 & 67.88 & 77.31 & 77.42 & 77.75 & 77.64 \\
 \hline
 32 & 67.25 & 67.72 & 67.72 & 67.00 & 77.31 & 77.04 & 77.80 & 77.37  \\
 \hline
 16 & 68.11 & 68.90 & 67.88 & 67.48 & 77.37 & 78.13 & 78.13 & 77.69  \\
 \hline
\end{tabular}
\caption{\label{tab:mmlu_abalation} Accuracy on LM evaluation harness tasks on GPT3-8B model.}
\end{table}

\begin{table} \centering
\begin{tabular}{|c||c|c|c|c||c|c|c|c|} 
\hline
 $L_b \rightarrow$& \multicolumn{4}{c||}{8} & \multicolumn{4}{c||}{8}\\
 \hline
 \backslashbox{$L_A$\kern-1em}{\kern-1em$N_c$} & 2 & 4 & 8 & 16 & 2 & 4 & 8 & 16  \\
 %$N_c \rightarrow$ & 2 & 4 & 8 & 16 & 2 & 4 & 2 \\
 \hline
 \hline
 \multicolumn{5}{|c|}{Race (FP32 Accuracy = 40.67\%)} & \multicolumn{4}{|c|}{Boolq (FP32 Accuracy = 76.54\%)} \\ 
 \hline
 \hline
 64 & 40.48 & 40.10 & 39.43 & 39.90 & 75.41 & 75.11 & 77.09 & 75.66 \\
 \hline
 32 & 39.52 & 39.52 & 40.77 & 39.62 & 76.02 & 76.02 & 75.96 & 75.35  \\
 \hline
 16 & 39.81 & 39.71 & 39.90 & 40.38 & 75.05 & 73.82 & 75.72 & 76.09  \\
 \hline
 \hline
 \multicolumn{5}{|c|}{Winogrande (FP32 Accuracy = 70.64\%)} & \multicolumn{4}{|c|}{Piqa (FP32 Accuracy = 79.16\%)} \\ 
 \hline
 \hline
 64 & 69.14 & 70.17 & 70.17 & 70.56 & 78.24 & 79.00 & 78.62 & 78.73 \\
 \hline
 32 & 70.96 & 69.69 & 71.27 & 69.30 & 78.56 & 79.49 & 79.16 & 78.89  \\
 \hline
 16 & 71.03 & 69.53 & 69.69 & 70.40 & 78.13 & 79.16 & 79.00 & 79.00  \\
 \hline
\end{tabular}
\caption{\label{tab:mmlu_abalation} Accuracy on LM evaluation harness tasks on GPT3-22B model.}
\end{table}

\begin{table} \centering
\begin{tabular}{|c||c|c|c|c||c|c|c|c|} 
\hline
 $L_b \rightarrow$& \multicolumn{4}{c||}{8} & \multicolumn{4}{c||}{8}\\
 \hline
 \backslashbox{$L_A$\kern-1em}{\kern-1em$N_c$} & 2 & 4 & 8 & 16 & 2 & 4 & 8 & 16  \\
 %$N_c \rightarrow$ & 2 & 4 & 8 & 16 & 2 & 4 & 2 \\
 \hline
 \hline
 \multicolumn{5}{|c|}{Race (FP32 Accuracy = 44.4\%)} & \multicolumn{4}{|c|}{Boolq (FP32 Accuracy = 79.29\%)} \\ 
 \hline
 \hline
 64 & 42.49 & 42.51 & 42.58 & 43.45 & 77.58 & 77.37 & 77.43 & 78.1 \\
 \hline
 32 & 43.35 & 42.49 & 43.64 & 43.73 & 77.86 & 75.32 & 77.28 & 77.86  \\
 \hline
 16 & 44.21 & 44.21 & 43.64 & 42.97 & 78.65 & 77 & 76.94 & 77.98  \\
 \hline
 \hline
 \multicolumn{5}{|c|}{Winogrande (FP32 Accuracy = 69.38\%)} & \multicolumn{4}{|c|}{Piqa (FP32 Accuracy = 78.07\%)} \\ 
 \hline
 \hline
 64 & 68.9 & 68.43 & 69.77 & 68.19 & 77.09 & 76.82 & 77.09 & 77.86 \\
 \hline
 32 & 69.38 & 68.51 & 68.82 & 68.90 & 78.07 & 76.71 & 78.07 & 77.86  \\
 \hline
 16 & 69.53 & 67.09 & 69.38 & 68.90 & 77.37 & 77.8 & 77.91 & 77.69  \\
 \hline
\end{tabular}
\caption{\label{tab:mmlu_abalation} Accuracy on LM evaluation harness tasks on Llama2-7B model.}
\end{table}

\begin{table} \centering
\begin{tabular}{|c||c|c|c|c||c|c|c|c|} 
\hline
 $L_b \rightarrow$& \multicolumn{4}{c||}{8} & \multicolumn{4}{c||}{8}\\
 \hline
 \backslashbox{$L_A$\kern-1em}{\kern-1em$N_c$} & 2 & 4 & 8 & 16 & 2 & 4 & 8 & 16  \\
 %$N_c \rightarrow$ & 2 & 4 & 8 & 16 & 2 & 4 & 2 \\
 \hline
 \hline
 \multicolumn{5}{|c|}{Race (FP32 Accuracy = 48.8\%)} & \multicolumn{4}{|c|}{Boolq (FP32 Accuracy = 85.23\%)} \\ 
 \hline
 \hline
 64 & 49.00 & 49.00 & 49.28 & 48.71 & 82.82 & 84.28 & 84.03 & 84.25 \\
 \hline
 32 & 49.57 & 48.52 & 48.33 & 49.28 & 83.85 & 84.46 & 84.31 & 84.93  \\
 \hline
 16 & 49.85 & 49.09 & 49.28 & 48.99 & 85.11 & 84.46 & 84.61 & 83.94  \\
 \hline
 \hline
 \multicolumn{5}{|c|}{Winogrande (FP32 Accuracy = 79.95\%)} & \multicolumn{4}{|c|}{Piqa (FP32 Accuracy = 81.56\%)} \\ 
 \hline
 \hline
 64 & 78.77 & 78.45 & 78.37 & 79.16 & 81.45 & 80.69 & 81.45 & 81.5 \\
 \hline
 32 & 78.45 & 79.01 & 78.69 & 80.66 & 81.56 & 80.58 & 81.18 & 81.34  \\
 \hline
 16 & 79.95 & 79.56 & 79.79 & 79.72 & 81.28 & 81.66 & 81.28 & 80.96  \\
 \hline
\end{tabular}
\caption{\label{tab:mmlu_abalation} Accuracy on LM evaluation harness tasks on Llama2-70B model.}
\end{table}

%\section{MSE Studies}
%\textcolor{red}{TODO}


\subsection{Number Formats and Quantization Method}
\label{subsec:numFormats_quantMethod}
\subsubsection{Integer Format}
An $n$-bit signed integer (INT) is typically represented with a 2s-complement format \citep{yao2022zeroquant,xiao2023smoothquant,dai2021vsq}, where the most significant bit denotes the sign.

\subsubsection{Floating Point Format}
An $n$-bit signed floating point (FP) number $x$ comprises of a 1-bit sign ($x_{\mathrm{sign}}$), $B_m$-bit mantissa ($x_{\mathrm{mant}}$) and $B_e$-bit exponent ($x_{\mathrm{exp}}$) such that $B_m+B_e=n-1$. The associated constant exponent bias ($E_{\mathrm{bias}}$) is computed as $(2^{{B_e}-1}-1)$. We denote this format as $E_{B_e}M_{B_m}$.  

\subsubsection{Quantization Scheme}
\label{subsec:quant_method}
A quantization scheme dictates how a given unquantized tensor is converted to its quantized representation. We consider FP formats for the purpose of illustration. Given an unquantized tensor $\bm{X}$ and an FP format $E_{B_e}M_{B_m}$, we first, we compute the quantization scale factor $s_X$ that maps the maximum absolute value of $\bm{X}$ to the maximum quantization level of the $E_{B_e}M_{B_m}$ format as follows:
\begin{align}
\label{eq:sf}
    s_X = \frac{\mathrm{max}(|\bm{X}|)}{\mathrm{max}(E_{B_e}M_{B_m})}
\end{align}
In the above equation, $|\cdot|$ denotes the absolute value function.

Next, we scale $\bm{X}$ by $s_X$ and quantize it to $\hat{\bm{X}}$ by rounding it to the nearest quantization level of $E_{B_e}M_{B_m}$ as:

\begin{align}
\label{eq:tensor_quant}
    \hat{\bm{X}} = \text{round-to-nearest}\left(\frac{\bm{X}}{s_X}, E_{B_e}M_{B_m}\right)
\end{align}

We perform dynamic max-scaled quantization \citep{wu2020integer}, where the scale factor $s$ for activations is dynamically computed during runtime.

\subsection{Vector Scaled Quantization}
\begin{wrapfigure}{r}{0.35\linewidth}
  \centering
  \includegraphics[width=\linewidth]{sections/figures/vsquant.jpg}
  \caption{\small Vectorwise decomposition for per-vector scaled quantization (VSQ \citep{dai2021vsq}).}
  \label{fig:vsquant}
\end{wrapfigure}
During VSQ \citep{dai2021vsq}, the operand tensors are decomposed into 1D vectors in a hardware friendly manner as shown in Figure \ref{fig:vsquant}. Since the decomposed tensors are used as operands in matrix multiplications during inference, it is beneficial to perform this decomposition along the reduction dimension of the multiplication. The vectorwise quantization is performed similar to tensorwise quantization described in Equations \ref{eq:sf} and \ref{eq:tensor_quant}, where a scale factor $s_v$ is required for each vector $\bm{v}$ that maps the maximum absolute value of that vector to the maximum quantization level. While smaller vector lengths can lead to larger accuracy gains, the associated memory and computational overheads due to the per-vector scale factors increases. To alleviate these overheads, VSQ \citep{dai2021vsq} proposed a second level quantization of the per-vector scale factors to unsigned integers, while MX \citep{rouhani2023shared} quantizes them to integer powers of 2 (denoted as $2^{INT}$).

\subsubsection{MX Format}
The MX format proposed in \citep{rouhani2023microscaling} introduces the concept of sub-block shifting. For every two scalar elements of $b$-bits each, there is a shared exponent bit. The value of this exponent bit is determined through an empirical analysis that targets minimizing quantization MSE. We note that the FP format $E_{1}M_{b}$ is strictly better than MX from an accuracy perspective since it allocates a dedicated exponent bit to each scalar as opposed to sharing it across two scalars. Therefore, we conservatively bound the accuracy of a $b+2$-bit signed MX format with that of a $E_{1}M_{b}$ format in our comparisons. For instance, we use E1M2 format as a proxy for MX4.

\begin{figure}
    \centering
    \includegraphics[width=1\linewidth]{sections//figures/BlockFormats.pdf}
    \caption{\small Comparing LO-BCQ to MX format.}
    \label{fig:block_formats}
\end{figure}

Figure \ref{fig:block_formats} compares our $4$-bit LO-BCQ block format to MX \citep{rouhani2023microscaling}. As shown, both LO-BCQ and MX decompose a given operand tensor into block arrays and each block array into blocks. Similar to MX, we find that per-block quantization ($L_b < L_A$) leads to better accuracy due to increased flexibility. While MX achieves this through per-block $1$-bit micro-scales, we associate a dedicated codebook to each block through a per-block codebook selector. Further, MX quantizes the per-block array scale-factor to E8M0 format without per-tensor scaling. In contrast during LO-BCQ, we find that per-tensor scaling combined with quantization of per-block array scale-factor to E4M3 format results in superior inference accuracy across models. 



\end{document}