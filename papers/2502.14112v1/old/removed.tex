\subsection{Learning Processes}

Participants play ``The Gold Digger" game four times. This section investigates their learning processes throughout the experiment.

Figure~\ref{fig:payoff} shows the average payoff per round. In the Patent and Singleton conditions, players obtained a positive payoff, and in the No Patent condition they obtained a negative one. 

\begin{figure}[H]
\includegraphics[width=0.6\textwidth]{"payoff".png}
\centering
\caption{Average payoff per round.}
\label{fig:payoff}
\end{figure}

The question that arises is whether the players learned over time that exploration under the No Patent condition is not profitable, and accordingly reduced their exploratory efforts in subsequent games. We therefore added the game's order as a independent variable, ranging from 1 to 4, to predict exploration under the No Patent condition.

Figure~\ref{fig:learn} presents the learning effect in the No Patent condition and the linear regression results. Error bars represent a confidence interval of 0.95. 
\begin{figure}[H]
\includegraphics[width=0.6\textwidth]{"learn".png}
\centering
\caption{Learning process in the No Patent condition.\rmr{not clear if this is initial or subsequent exploration.}}
\label{fig:learn}
\end{figure}

To conclude, results show that players do learn over time to adjust their exploration rate to the average outcome, and explore less over time in the No Patent condition. \rmr{this brings more obvious questions that we do not answer - after they learn, is payoff positive? is it optimal?  perhaps this can be better shown be seeing how the mean/median threshold changes over games (because you can compare to the optimal threshold. Perhaps replacing this figure with the bottom-left part of the figure in Appendix J}


More analyses of the players performance throughout the game can be found in Appendix~\ref{apdx:learning}.

\section{}\label{apdx:learning}
\begin{table}[H]
\begin{center}
\begin{tabular}{l c }
\hline
 & search \\
\hline
(Intercept)                 & $1.15^{***}$  \\
                            & $(0.04)$      \\
$Cost$                           & $-0.03^{***}$ \\
                            & $(0.00)$      \\
$Game$                           & $-0.02^{***}$ \\
                            & $(0.00)$      \\
\hline
AIC                         & 7192.98       \\
BIC                         & 7236.02       \\
Log Likelihood              & -3590.49      \\
Num. obs.                   & 9644          \\
Num. groups: ID             & 60            \\
Num. groups: GroupIndex     & 15            \\
Var: ID (Intercept)         & 0.07          \\
Var: GroupIndex (Intercept) & 0.00          \\
Var: Residual               & 0.12          \\
\hline
\multicolumn{2}{l}{\scriptsize{$^{***}p<0.001$, $^{**}p<0.01$, $^*p<0.05$}}
\end{tabular}
\caption{The effect of learning on the search rate for first discovery at the No Patent condition. We can see that players do learn to explore for first discoveries about 2 percentage points ($p<0.001$) less in each game under the No Patent condition.}
\label{table:learningNoPatent}
\end{center}
\end{table}
\begin{table}[H]
\begin{center}
\begin{tabular}{l c }
\hline
 & search \\
\hline
(Intercept)                 & $1.19^{***}$  \\
                            & $(0.03)$      \\
$Cost$                           & $-0.03^{***}$ \\
                            & $(0.00)$      \\
$Game$                           & $0.00$        \\
                            & $(0.00)$      \\
\hline
AIC                         & 8888.03       \\
BIC                         & 8932.29       \\
Log Likelihood              & -4438.02      \\
Num. obs.                   & 11800         \\
Num. groups: ID             & 59            \\
Num. groups: GroupIndex     & 15            \\
Var: ID (Intercept)         & 0.05          \\
Var: GroupIndex (Intercept) & 0.00          \\
Var: Residual               & 0.12          \\
\hline
\multicolumn{2}{l}{\scriptsize{$^{***}p<0.001$, $^{**}p<0.01$, $^*p<0.05$}}
\end{tabular}
\caption{Learning effect on exploration in the Patent condition. We can see that there is no change in the exploration behaviour throughout the game. }
\label{table:coefficients}
\end{center}
\end{table}

\begin{figure}[H]
\includegraphics[width=0.9\textwidth]{"learning".png}
\caption{Learning effects by conditions and by the by the states of the game. We can see that the main effect is in the no patent case, when the players explore for first discoveries.}
\label{fig:learning}
\end{figure}

Figure~\ref{fig:threshold} presents a histogram of thresholds in the different states and conditions of the game.\footnote{9 observations of subjects that had the most inconsistent threshold (less than 0.7 consistent rate) was removed in all figures and tables in this section, see more details in appendix~\ref{apdx:actualthreshold}} 

The most striking observation is that there is a clear bimodal distribution: some subjects consistently always search, while the others are distributed around their mean decision threshold. 

Tables~\ref{table:threshold1} and~\ref{table:threshold2} present a summary of this bimodal distribution by computing the fraction of workers that always search and the average value of all other subjects.The optimal threshold is also represented (range of possible values in the No Patent condition) showing the values that were computed in section~\ref{sec:theory}.

\begin{figure}[!ht]
  \centering
  \begin{minipage}[t]{0.45\textwidth}
    \includegraphics[width=\textwidth]{threshHistFirst.png}
  \end{minipage}
  \hfill
  \begin{minipage}[t]{0.45\textwidth}
    \includegraphics[width=\textwidth]{threshHistSub.png}
  \end{minipage}
  \caption{Histogram of exploration costs thresholds by condition and by the state of the game. The lines represent the optimal thresholds (range of values in the No Patent condition, Green line for Patent and Singleton optimal thresholds) computed in section~\ref{sec:theory}. }
  \label{fig:threshold}
\end{figure}

\begin{table}[H]
      \centering
        \begin{tabular}{rlrrrrr}
  \hline
 Condition & First treasure & Subsequent treasure \\ 
  \hline
 Singleton & 0.06 & 0.30 \\ 
 Patent & 0.21 & 0.36  \\ 
 No Patent & 0.18 & 0.40\\
   \hline
   
        \end{tabular}
        \caption{Fraction of players that always search.}
   \label{table:threshold1}
\end{table}

\begin{table}[H]
      \centering
        \begin{tabular}{rlrrrrr}
  \hline
  Condition & Average threshold - & 
  Average threshold - \\& first treasure & subsequent treasure \\
  \hline
 Singleton & 18.6 & 23.5 \\ 
 Patent & 19.7  & 23.2  \\ 
No Patent & 17.7  & 23.8 \\
   \hline
   \end{tabular}
   \caption{Average thresholds of players who have mixed choices whether to search or not.}
   \label{table:threshold2}
\end{table}

Appendix~\ref{apdx:singleton} presents more results regarding the Singleton condition.

\section{}\label{apdx:singleton}
This section presents some results regarding the Singleton condition.  

Table~\ref{table:3conditionsDiscoverylevel} presents the results in the mine level. Regressions show that all the differences between the Patent and the Singleton conditions are significant, except the difference in the number of failures in the searching process. 


\begin{table}[H]
\centering

\begin{tabular}{rlrrrrr}
  \hline
 & Condition & Nplayers & Nround & IsParallel & Failures & Nsearch \\ 
  \hline
1 & No Patent & 3.16 & 2.43 & 1.06 & 3.20 & 6.35 \\ 
  2 & Patent & 1.30 & 6.43 & 1.58 & 1.75 & 3.96 \\ 
  3 & Singleton & 0.98 & 5.41 & 1.08 & 1.46 & 3.35 \\ 
   \hline
\end{tabular}
\caption{Comparison between the three conditions in the mine levels}
\label{table:3conditionsDiscoverylevel}
\end{table}
Table~\ref{table:3conditionsroundlevel} presents the results in the player-round data level. Column 1 shows the exploration rate for first treasures, column 2 presents the exploration rate for subsequent treasure, column 3 presents the number of treasures found and column 4 presents the average payoff in each condition. Regressions show that except the difference in the number of treasures, which is clearly significant, all the other differences between the Patent and the Singleton conditions are not significant.  
\begin{table}[H]
\centering
\begin{tabular}{rlrrrr}
  \hline
 & Condition & First treasure & Subsequent treasure & Number of treasures & Payoff \\ 
  \hline
1 & No Patent & 0.53 & 0.76 & 11.40 & -0.86 \\ 
  2 & Patent & 0.59 & 0.72 & 14.42 & 2.41 \\ 
  3 & Singleton & 0.49 & 0.73 & 2.96 & 2.19 \\ 
   \hline
\end{tabular}
\caption{Comparison between the three conditions in the round-player data level}
\label{table:3conditionsroundlevel}
\end{table}