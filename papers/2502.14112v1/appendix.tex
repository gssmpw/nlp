\newpage
\setcounter{page}{1}
%\maketitle
\begin{center}
    Online Appendices
\end{center}
\
\section{Details of the Experiment} \label{apx:experiment}
One-hundred and fifty four (81 Female) 
Technion and Ben Gurion University 
students, with an average age of 25, participated in the study in exchange for monetary compensation. The study included participants aged 18 and older who signed a consent form to participate in the experiment. The forms were signed by hand on a page in front of the research team and kept in the laboratory. The study was carried out between January 21, 2019 and April 28, 2019.
We planned to have at least 15 groups of 4 participants in each of the Protection and No Protection conditions, and stopped data collection once this goal was reached. Eventually we collected data from 60 participants in the Protection condition,\footnote{One student was mistakenly invited to the lab twice, and therefore her second session was removed.} 60 participants in the No Protection condition,
and 34 participants in the Singleton condition. 

A performance based payment was added to (if positive) or subtracted from (if negative) a show-up fee of 30 NIS. 
\footnote{Participants obtained a total of 29 NIS (that equals about \$8.3) on average, in a game lasting around 35 minutes. Note that the mean payoff is lower than the show-up fee, which means that on average, the performance based payment was negative. This is a first indication that participants did not behave optimally (they could guarantee the show-up fee by skipping all rounds).}


\subsubsection{Experimental Design}
Participants played a lab adaptation of the ``The Competitive Treasure Hunt" game that was described in Section~\ref{sec:golddigger}. 
\begin{enumerate}
  \item The hive included 2100 hexagons (70X30), rather than infinite number of hexagons. This modification implies that after each round, information is revealed and the probabilities change in the following ways: (1) the probability of finding a first treasure decreases after a treasure is found, since overall fewer treasures are left; (2) the probability of finding a first treasure increases after a failed search only for the player making the move, since players can observe all successful searches, but only their own failures. Consequently, the overall probability of finding a first treasure tends to decrease over time. However, since around each gold mine there are several known empty hexagons, the overall probability decreases only slightly.\footnote{We computed the probability of finding a first treasure in the last round of each game, and obtained on average the probabilities 0.0445, 0.0458 and 0.0492 in the Protection, No Protection and Singleton conditions, respectively. As previously described, the probability decreases on average over time, so these numbers estimate the minimum probability in each game. We can thus see that these probabilities are relatively close to 0.05.}
  \item Each game included 50 rounds, rather than infinite number of rounds. There is strong evidence that although in traditional game theoretic analysis any finite horizon may completely change the structure of equilibria, human players only take this into account (end-game effect) very close to the actual termination. E.g. in \citet{normann2012impact} end-game effect was explicitly measured only in the last 3 rounds out of 22 of the Repeated Prisoners' Dilemma. They also compared behavior under known, unknown, and random termination rules and find that differences in behavior only start $\sim$10 rounds before termination. 
  Moreover,  RPD is a deterministic game. Adding randomness to the game (as in our case) substantially reduces endgame effect, since it negates the value of looking ahead in general. E.g. while medium-level Chess programs typically consider $\sim$15 steps ahead, the Backgammon programs only need to look 2 steps ahead in order to beat the best human players \citep{tesauro1994td}.
  As we explain below, to compare participants' behavior to the theoretical, infinite time horizon benchmark, we excluded the last 12 rounds in each game from the analysis.
  \item Players were not informed in advance of the probability to find a treasure. As noted, with sufficient experience the learnt probability to find a treasure should converge to the actual one and lead rational players to a stable optimal threshold. The experiment included 4 games of 50 rounds each, which should allow for sufficient learning.\footnote{Indeed, analysis of potential changes throughout the games revealed quick learning and no significant differences between the first and the last game participants played.}
  \item If two or more players choose the same hexagon simultaneously, the payoffs that each player obtains follow this rule: if two players find the same treasure, each of them obtains 0.2 from the original reward of this treasure (which amount to 64 if this is the first treasure in the mine, and 16 if this is the second or the third). If three players find the same treasure, each of them obtains 0.05 from the original reward and if four players find it, each of them obtains 0.\footnote{This rule was designed to account for the fact that real life competition decreases the total producers surplus.} 
\end{enumerate}






\subsubsection{Procedure}
In each experiment's session, students invited to the lab were randomly assigned into groups of four. Each group was randomly assigned into the ``Protection" or the ``No Protection" condition. 
All the remaining participants were assigned to the ``Singleton" condition. 
Each participant played four games with 50 rounds per game.\footnote{18 participants from the Singleton condition played 5 rather than 4 games. For those participants, we analyzed only the first four games played.}

The players were not informed about the other players' identities, but they did know their group's size. 

The participants received three pages of instructions, which included pictures of different states of the game with explanation about the shape of mines and the meaning of different colors and marked areas (see Appendix~\ref{apdx:instructions}). Participants were informed about the structure of mines (i.e. a tight triangle), but not about their frequency and their location in the hive. In the Protection condition, the instructions explained that when a player finds the first treasure, he obtains an exclusive right to explore surrounding (adjacent) hexagons and benefit from the subsequent treasures. In the No Protection condition the instructions explained that the more players who find the same treasure, the lower the reward it yields. To make sure that the players understood the instructions well, before starting the first game they had to answer a short quiz with questions concerning the instructions of the game. The game started only after all the examinees answered all the questions correctly.
We did not mention any economic or domain specific terms such as ``protection", ``innovation" etc. in any of the condition's instructions.

Note that in all conditions, the players receive the same instructions (except for the introduction of protection rules in the protection condition), and the treasures location as well as the payoff procedure were the same. 


 The game progressed as follows. First, the computer displayed the hive, containing 2100 hexagons. The players received a message stating the exploration cost for this round, and asking each player if he wants to skip the round or to explore under the current exploration cost. After making their choice, players were asked to wait for the other players to make their choices.\footnote{This message appeared also in the Singleton condition to match this condition to the other conditions.} 

If a player decided to skip, he gained 0, and the round was over for him. If a player decided to explore, he could choose one of the hexagons in the hive that was not yet colored. At the end of each round, the players received a message with their payoff from the round, calculated as the reward obtained minus the current exploration cost.   


\subsection{Screenshots of the Competitive Treasure Hunt Game}\label{apdx:screenshots}
\begin{figure}[H]
    \centering
    \includegraphics[width = \textwidth]{EndGamePatent.png}
    \caption{A screenshot of a typical end game under the protection condition.}
    \label{}
\end{figure}
\begin{figure}[H]
    \centering
    \includegraphics[width = \textwidth]{EndGameNoPatent.png}
    \caption{A screenshot of a typical end game under the no protection condition. }
    \label{}
\end{figure}


\section{Simulations}\label{apdx:simulation}
This section, presents the simulation results. The simulation obtained 7 different values for the cost threshold for first treasures and 7 different values for the cost threshold for subsequent treasures, overall 49 combinations of both. We ran the simulation 10000 times in any combination, and took mean values of the number of treasures and payoffs. 

Figures \ref{fig:individuallevel} and~\ref{fig:grouplevel} show that under a rational behaviour assumption (which is marked in a red circle),\footnote{In the No Protection condition, we took the cost within the possible range.} players find more treasures in the Protection condition, both at the group and at the individual levels. Figure \ref{fig:payoffsimulation} shows that in the Protection condition, optimal strategy leads to a payoff maximization, and in the No Protection condition, players could increase their payoffs by collectively deciding to explore less for subsequent treasures, below the equilibrium strategy. This result shows how lack of coordination among the players in the No Protection case, causes a reduction in payoffs.

In the simulation, we also measured the efficiency of exploration by the number of duplicated treasures, which is the amount of treasures that were found by more than one player. Figure~\ref{fig:duplicated} presents the number of duplicated treasures in the No Protection condition, and shows that this number increases as the thresholds increase. Structurally, there are no duplicated treasures in the Protection condition simulation.
\begin{figure}[H]
    \centering
    \includegraphics[width = \textwidth]{"individual_level".png}
    \caption{The number of treasures each player found, as a function of the chosen cost threshold in searching for first and subsequent treasures. In the No Protection condition we consider only symmetric strategies, where all players choose the same strategy. We can see that choosing the optimal strategies yields almost the same number of treasures in both conditions. }
    \label{fig:individuallevel}
\end{figure}
\begin{figure}[H]
    \centering
    \includegraphics[width = \textwidth]{grouplevel.png}
    \caption{The number of treasures players found at the group level. In the No Protection condition we consider only symmetric strategies, where all players choose the same strategy. We can see that players found more treasures under the Protection condition.}
    \label{fig:grouplevel}
\end{figure}
\begin{figure}[H]
    \centering
    \includegraphics[width = \textwidth]{payoffsimulation.png}
    \caption{The payoffs as a function of cost threshold. In the No Protection condition we consider only symmetric strategies.}
    \label{fig:payoffsimulation}
\end{figure}
\begin{figure}[H]
    \centering
    \includegraphics[width = 0.5\textwidth]{duplicated.png}
    \caption{The number of treasures that were found by more than one player under the No Protection condition. }
    \label{fig:duplicated}
\end{figure}
\section{The Game Instructions}\label{apdx:instructions}
\begin{figure}[H]
\includegraphics[width=0.9\textwidth]{"Instructions1".png}
\caption{Instructions that are common to all conditions}
\label{fig:Instruction1}
\end{figure}
\begin{figure}[H]
\includegraphics[width=0.9\textwidth]{"Instructions2".JPG}

\label{fig:Instruction3}
\end{figure}
\begin{figure}[H]

\includegraphics[width=0.9\textwidth]{"Instructions3".JPG}
\caption{Instructions to the Protection condition}
\label{fig:Instruction4}
\end{figure}

\begin{figure}[H]
\includegraphics[width=0.9\textwidth]{"Instructions4".JPG}

\label{fig:Instruction5}
\end{figure}
\begin{figure}[H]

\includegraphics[width=0.9\textwidth]{"Instructions6".JPG}
\caption{Instructions to the No Protection condition}
\label{fig:Instruction6}
\end{figure}
\begin{figure}[H]
\includegraphics[width=0.9\textwidth]{"Instructions4".JPG}

\label{fig:Instruction7}
\end{figure}
\begin{figure}[H]

\includegraphics[width=0.9\textwidth]{"Instructions7".JPG}
\caption{Instructions to the Singleton condition}
\label{fig:Instruction8}
\end{figure}


\section{}\label{apdx:minepayoff}
\begin{figure}[H]
    \centering
    \includegraphics[width=0.6\textwidth]{"payoff_from_mine".png}
    \caption{Average payoff from the second and
 the third treasure of any mine.}
    \label{fig:my_label}
\end{figure}

\section{}\label{apdx:levelsingleton}
\begin{figure}[H]
\includegraphics[width=0.6\textwidth]{"singleton".png}
\caption{Search rate for the first treasure in the Singleton condition when the reward from the first treasure equals 320 and 260, respectively. }
\label{fig:singleton}
\end{figure}

\section{Statistical analysis}\label{apx:stat}


In order to measure the correlation between participants and groups, we analyzed the data using a linear mixed effects model (LMM). We allow random intercepts for the players' ID and for the group of players.

\begin{table}[H]
\begin{center}
\begin{scriptsize}
\begin{tabular}{l c c c }
\hline
 & search & search & search \\
 & for first treasure & for self-subsequent treasure & for self-subsequent treasure\\
 & & & for high-costs only\\
\hline
(Intercept)                 & $1.11^{***}$  & $1.22^{***}$  & $1.45^{***}$  \\
                            & $(0.04)$      & $(0.05)$      & $(0.11)$      \\
Cost                           & $-0.03^{***}$ & $-0.02^{***}$ & $-0.02^{***}$ \\
                            & $(0.00)$      & $(0.00)$      & $(0.00)$      \\
Protection             & $0.06$        & $-0.10^{*}$   & $-0.17^{*}$   \\
                            & $(0.05)$      & $(0.05)$      & $(0.07)$      \\
\hline
AIC                         & 14734.27      & 273.42        & 305.32        \\
BIC                         & 14781.58      & 299.04        & 327.84        \\
Log Likelihood              & -7361.13      & -130.71       & -146.66       \\
Num. obs.                   & 19659         & 529           & 315           \\
Num. groups: ID             & 119           & 111           & 104           \\
Num. groups: GroupIndex     & 30            & 30            & 30            \\
Var: ID (Intercept)         & 0.06          & 0.02          & 0.06          \\
Var: GroupIndex (Intercept) & 0.00          & 0.01          & 0.01          \\
Var: Residual               & 0.12          & 0.08          & 0.10          \\
\hline
\multicolumn{4}{l}{\scriptsize{$^{***}p<0.001$, $^{**}p<0.01$, $^*p<0.05$}}
\end{tabular}
\end{scriptsize}
\caption{Search rate for first (column 1) and subsequent (column 2,3) treasures, column 3 presents search rate in high cost only.}
\label{table:Searching}
\end{center}
\end{table}


\paragraph{Initial search:} The first question addressed is how protecting first treasures affects exploration activity for initial search. Table~\ref{table:Searching}, column 1 presents the results of the exploration rates for first treasures. It shows that the coefficient of $Protection$ variable is insignificant, suggesting that when players explore for the first treasure in each mine, there is no significant difference between their behavior under the Protection and the No Protection conditions.

 Column~2 presents the results. It shows that protection on first treasures significantly decreases the overall tendency to explore for a subsequent treasure by 10 percent ($p<0.05$).

 Column~3 shows the same analysis for costs larger than 15. 

 %%%%%%%%%%%%%%%%%%%%%%%%%%%
\section{Computing Thresholds}\label{apdx:actualthreshold}
In this section the methodology of computing the actual thresholds is presented.
We denote the states of the game by $F$ for exploration for a first treasure and $S$ for exploration for subsequent treasure. 
Notice that the definition for $S$ imposes asymmetric treatment between the Protection and the No Protection conditions. In the Protection condition, subsequent treasures are available only when the player finds the first treasure in the mine by himself. 
However, in the No Protection condition, subsequent treasures are available whenever any player finds the first treasure.\footnote{We removed from the analysis all observations in which we identified a search for a first discovery when it is possible to search for a subsequent discovery. As for the non-search classification when subsequent search is possible, it is in fact a non-search of both a first discovery and a subsequent discovery. It can therefore be classified as a non-search for a subsequent discovery, and so we did. Since there is no reason to believe that the threshold for a first search when a subsequent discovery is available, will be different from the threshold for a first search when it is not, removing the observations of these exceptional searches should not change the results.}
Thus, each round in the game is classified into these two states. For each player $i$, in each state of the game $\omega \in \{F , S\}$, we calculate the threshold cost value $T_{i,\omega}$ by the following process. 

First, for each possible cost $c_j \in \{5,10,15,20,25,30,35\}$ and for each round, $r$, and for each player $i$, we let $c_r^i$ be the realization of the cost for player $i$ in round $r$. We define the specification function: 
\begin{equation*}
S_i(c_j,c_r^i) = \begin{cases}
1 &\text{if $c_r^i \geq c_j$ and $search = 0$ or if $c_r^i < c_j$ and $search = 1$}\\
0 &\text{otherwise}
\end{cases}
\end{equation*} 
when ``search" is a variable that is set to be 1 when the player chooses to search, and to 0 when he decides to skip. Then for each $i,c_j,\omega$ we define the specification quality, which is essentially a 1-dimensional classifier with the 0-1 loss function, by: 
$$SQ(i,c_j,\omega) = \frac{\sum_{r\in \omega}S_{i}(c_j,c_r^i)}{|\omega_i|}$$ 
where $|\omega_i|$ is the number of rounds that player $i$ is in the state of the game $\omega$. Finally we define the threshold of each player in each state of the game as $T_{i,\omega}=argmax_{c_j}\{SQ(i,c_j,\omega)\}$ and the threshold quality by $TQ_{i,\omega} = max_{c_j}\{SQ(i,c_j,\omega)\}$ 

In other words, we consider each possible cost as a potential threshold. If this cost is the ``real" threshold, the player should explore whenever the cost realization is lower, and skip whenever it is higher. Then we took the potential threshold that yields the least number of mis-specifications of the ``real" threshold. Finally we define the threshold quality as the fraction of the number of rounds that the player's action was consistent with his threshold. 

Finally, we would like to find evidence for a consistent behaviour, that is expressed by a high quality of thresholds. Figure~\ref{fig:threshold_quality} represents the histogram of the thresholds quality. We can see that players tend to be highly consistent, where more than 84\% of the players had a threshold quality of 0.8 or more.

\begin{figure}[H]
\includegraphics[width=0.6\textwidth]{"threshold_quality".png}
\centering
\caption{Histogram of the threshold's quality}
\label{fig:threshold_quality}
\end{figure}


Fig.~\ref{fig:threshold} shows the distribution of thresholds we found in each condition and context.
\begin{figure}[H]
    \centering
    \includegraphics[width=0.7\textwidth]{Thresholds.png}
    \caption{Threshold distribution by condition and by the state of the game. Dots represent optimal/equilibrium thresholds. Stars ($*$) represent significant differences between median value of observed threshold and optimal/equilibrium threshold, determined by the Wilcoxon test. Horizontal lines represent data deviation in quartiles.}
    \label{fig:threshold}
\end{figure}





\section{Protection vs. Singleton}\label{apdx:forgone}
We estimate the effect of the Protection condition on the exploration activity, compared to the Singleton condition, for cases where the players explore for a first treasure. 
Table~\ref{table:otherDiscoveries} column 1 presents the results of this estimation. We can see that under the Protection condition players explore 9 percentage points more than under the Singleton condition, however this effect is not significant. Although the coefficient of the condition is not precisely estimated, this result supports the hypothesis that revealing information about others' treasures encourages innovative activity of other inventors. 

Another way to show this hypothesis is to measure the effect of observing the other players' treasures within the Protection condition. We consider all observations in the Protection condition where players explore for a first treasure, and create a dummy variable, ``$other\_treasure$". 
This variable indicates whether there is another player that found a treasure in the previous round. We estimate the effect of this variable on the exploration activity in the current round. The result can be found in Table~\ref{table:otherDiscoveries} column 2. 
We can see that the players tend to explore 2 percentage points more for a first treasure after another player found treasure ($p<0.01$).
\begin{table}[H]
\begin{center}
\begin{tabular}{l c c }
\hline
 & search & search \\
\hline
(Intercept)                 & $1.12^{***}$  & $1.19^{***}$  \\
                            & $(0.04)$      & $(0.03)$      \\
$cost$                           & $-0.03^{***}$ & $-0.03^{***}$ \\
                            & $(0.00)$      & $(0.00)$      \\
$Protection$              & $0.09$        &               \\
                            & $(0.05)$      &               \\
$other\_treasure$            &               & $0.02^{**}$   \\
                            &               & $(0.01)$      \\
\hline
AIC                         & 13187.09      & 7528.65       \\
BIC                         & 13225.82      & 7571.92       \\
Log Likelihood              & -6588.54      & -3758.32      \\
Num. obs.                   & 17087         & 10015         \\
Num. groups: ID             & 94            & 59            \\
Var: ID (Intercept)         & 0.05          & 0.05          \\
Var: Residual               & 0.12          & 0.12          \\
Num. groups: GroupIndex     &               & 15            \\
Var: GroupIndex (Intercept) &               & 0.01          \\
\hline
\multicolumn{3}{l}{\scriptsize{$^{***}p<0.001$, $^{**}p<0.01$, $^*p<0.05$}}
\end{tabular}
\caption{The effect of observing the other players' treasures. Column 1 compares the Protection and the Singleton conditions and Column 2 estimates the effect of other players' successful search in the previous round, within the Protection condition.}
\label{table:otherDiscoveries}
\end{center}
\end{table}


