\section{Introduction}
\label{sec:Introduction}

Barrier functions (BFs) are Lyapunov-like functions \cite{tee2009barrier} whose use can be traced back to optimization problems \cite{boyd2004convex}. They have been utilized to prove set invariance \cite{aubin2011viability} \cite{prajna2007framework} and to derive multi-objective control \cite{panagou2013multi} \cite{wang2016multi}. More recently, a less restrictive form of a barrier function, which is allowed to grow when far away from the boundary of the set, was proposed in \cite{ames2016control}. Another approach that allows
a barrier function to reach the boundary of
the set was proposed in \cite{glotfelter2017nonsmooth}. 

In \cite{boyd2004convex}, Control Barrier Functions (CBFs) are extensions of BFs used to render a set forward invariant for an affine control system. It has been shown that stabilizing a control-affine system to admissible states while optimizing a quadratic cost subject to state and control constraints can be reduced to a sequence of Quadratic Programs (QPs) \cite{ames2016control} by unifying CBFs and Control Lyapunov Functions (CLFs) \cite{ames2012control}. 
Exponential CBFs \cite{nguyen2016exponential} are introduced in order to adapt CBFs to high relative degree systems. A more general form of exponential CBFs, called High Order CBFs (HOCBFs), has been proposed in \cite{xiao2021high}. The CBF method has been widely used to enforce safety in many applications, including adaptive cruise control \cite{ames2016control}, bipedal robot walking, \cite{hsu2015control} and robot swarming \cite{borrmann2015control}. However, the aforementioned CBF-based QP might be infeasible in the presence of tight or time-varying control bounds due to the conflict between CBF constraints and control bounds.

There are several approaches that aim to enhance the feasibility of the CBF method while guaranteeing safety. One can formulate CBFs as constraints under a Nonlinear Model Predictive Control (NMPC) framework, which allows the controller to predict future state information up to a horizon larger than one. This leads to a less aggressive control strategy \cite{zeng2021enhancing}. However, the corresponding
optimization is overall nonlinear and non-convex, 
which could be computationally expensive for
nonlinear systems. A convex MPC with linearized, discrete-time CBFs under an iterative approach was proposed in \cite{liu2023iterative} to address the above challenges, but this comes at the price of losing safety guarantees. The works in \cite{gurriet2018online,singletary2019online,gurriet2020scalable,chen2021backup} recently developed approaches
in which a known backup set or backup policy is defined that can be used to extend the safe set to a larger viable set
to enhance the feasible space of the system in a finite time horizon
under input constraints. This backup approach has further been generalized to infinite time horizons \cite{squires2018constructive} \cite{breeden2021high}. One limitation of these approaches is that they require prior knowledge of finding appropriate backup sets, policy or nominal control law, which are difficult to be predefined. Another limitation is that they only focus on feasibility, which may introduce over-aggressive or over-conservative control strategies. Sufficient conditions have been proposed in \cite{xiao2022sufficient} to guarantee the feasibility of the CBF-based QP, but they may be hard to find for general constrained control problems. All these approaches only consider time-invariant control limitations.

In order to account for time-varying control bounds, adaptive CBFs (aCBFs) \cite{xiao2021adaptive} have been proposed by
introducing penalty functions in HOCBFs constraints, which provide flexible and adaptive control strategies over time. However, this approach requires extensive parameter tuning. 
% Unlike the AVBCBFs {\color{blue} acronym not defined. also we do not need B for "based", it should be AVCBF} proposed in this paper, but additional constraint design and complicated parameter tuning process However, choosing appropriate penalty variables for class $\kappa$ functions may not be straightforward and does need additional constraint design and complicated parameter tuning process.
To address this issue,
we propose a novel type of aCBFs to safety-critical control problems. Specifically, the contributions of this paper are as follows:

\begin{itemize}
\item We propose Auxiliary-Variable Adaptive CBFs (AVCBFs) that can improve the feasibility of the CBF method under tight and time-varying control bounds.

\item We show that the proposed AVCBFs are analogous to existing CBF methods such that excessive additional constraints are not required. Most importantly, the AVCBFs preserve the adaptive property of aCBFs \cite{xiao2021adaptive}, while generating non-overshooting control policies near the boundaries of safe sets.


\item We demonstrate the effectiveness of the proposed AVCBFs on an adaptive cruise control problem with tight and time-varying control bounds, and compare it with existing CBF methods. The results show that the proposed approach can generate smoother and more adaptive control compared to existing methods, without requiring design of excessive additional constraints and complicated parameter-tuning procedures.
\end{itemize}
