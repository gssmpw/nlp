\section{Preliminaries}
\label{sec:preliminaries}

Consider an affine control system expressed as 
\begin{equation}
\label{eq:affine-control-system}
\dot{\boldsymbol{x}}=f(\boldsymbol{x})+g(\boldsymbol{x})\boldsymbol{u},
\end{equation}
 where $\boldsymbol{x}\in \mathbb{R}^{n}, f:\mathbb{R}^{n}\to\mathbb{R}^{n}$ and $g:\mathbb{R}^{n}\to\mathbb{R}^{n\times q}$ are locally Lipschitz, and $\boldsymbol{u}\in \mathcal U\subset \mathbb{R}^{q}$ denotes the control constraint set, which is defined as 
 
\begin{equation}
\label{eq:control-constraint}
\mathcal U \coloneqq \{\boldsymbol{u}\in \mathbb{R}^{q}:\boldsymbol{u}_{min}\le \boldsymbol{u} \le \boldsymbol{u}_{max} \}, \end{equation}
 with $\boldsymbol{u}_{min},\boldsymbol{u}_{max}\in \mathbb{R}^{q}$ (the vector inequalities are interpreted componentwise).
 
\begin{definition}[Class $\kappa$ function~\cite{Khalil:1173048}]
\label{def:class-k-f}
A continuous function $\alpha:[0,a)\to[0,+\infty],a>0$ is called a class $\kappa$ function if it is strictly increasing and $\alpha(0)=0.$
\end{definition}

\begin{definition}
\label{def:forward-inv}
A set $\mathcal C\subset \mathbb{R}^{n}$ is forward invariant for system \eqref{eq:affine-control-system} if its solutions for some $\boldsymbol{u} \in \mathcal U$ starting from any $\boldsymbol{x}(0) \in \mathcal C$ satisfy $\boldsymbol{x}(t) \in \mathcal C, \forall t \ge 0.$
\end{definition}

\begin{definition}
\label{def:relative-degree}
The relative degree of a differentiable function $b:\mathbb{R}^{n}\to\mathbb{R}$ is the minimum number of times we need to differentiate it along dynamics \eqref{eq:affine-control-system} until every component of $\boldsymbol{u}$ explicitly shows. 
\end{definition}

In this paper, safety is defined as the forward invariance of set $\mathcal C$. The relative degree of function $b$ is also referred to as the relative degree of constraint $b(\boldsymbol{x}) \ge 0$. For a constraint $b(\boldsymbol{x})\ge0$ with relative degree $m$, \ $b:\mathbb{R}^{n}\to\mathbb{R}$ and $\psi_{0}(\boldsymbol{x})\coloneqq b(\boldsymbol{x}),$ we can define a sequence of functions as $\psi_{i}:\mathbb{R}^{n}\to\mathbb{R},\ i\in \{1,...,m\}:$

\begin{equation}
\label{eq:sequence-f1}
\psi_{i}(\boldsymbol{x})\coloneqq\dot{\psi}_{i-1}(\boldsymbol{x})+\alpha_{i}(\psi_{i-1}(\boldsymbol{x})),\ i\in \{1,...,m\}, 
\end{equation}
where $\alpha_{i}(\cdot ),\ i\in \{1,...,m\}$ denotes a $(m-i)^{th}$ order differentiable class $\kappa$ function. A sequence of sets $\mathcal C_{i}$ are defined based on \eqref{eq:sequence-f1} as
\begin{equation}
\label{eq:sequence-set1}
\mathcal C_{i}\coloneqq \{\boldsymbol{x}\in\mathbb{R}^{n}:\psi_{i}(\boldsymbol{x})\ge 0\}, \ i\in \{0,...,m-1\}. 
\end{equation}

\begin{definition}[HOCBF~\cite{xiao2021high}]
\label{def:HOCBF}
Let $\psi_{i}(\boldsymbol{x}),\ i\in \{1,...,m\}$ be defined by \eqref{eq:sequence-f1} and $\mathcal C_{i},\ i\in \{0,...,m-1\}$ be defined by \eqref{eq:sequence-set1}. A function $b:\mathbb{R}^{n}\to\mathbb{R}$ is a High Order Control Barrier Function (HOCBF) with relative degree $m$ for system \eqref{eq:affine-control-system} if there exist $(m-i)^{th}$ order differentiable class $\kappa$ functions $\alpha_{i},\ i\in \{1,...,m\}$ such that
\begin{equation}
\label{eq:highest-HOCBF}
\begin{split}
\sup_{\boldsymbol{u}\in \mathcal U}[L_{f}^{m}b(\boldsymbol{x})+L_{g}L_{f}^{m-1}b(\boldsymbol{x})\boldsymbol{u}+O(b(\boldsymbol{x}))
+\\
\alpha_{m}(\psi_{m-1}(\boldsymbol{x}))]\ge 0,
\end{split}
\end{equation}
$\forall \boldsymbol{x}\in \mathcal C_{0}\cap,...,\cap \mathcal C_{m-1},$ where $O(\cdot)=\sum_{i=1}^{m-1}L_{f}^{i}(\alpha_{m-1}\circ\psi_{m-i-1})(\boldsymbol{x})$; $L_{f}^{m}$ denotes the $m$-th Lie derivative along $f$ and $L_{g}$ denotes the matrix of Lie derivatives along the columns of $g$. $\psi_{i}(\boldsymbol{x})\ge0$ is referred to as the $i^{th}$ order HOCBF constraint. We assume that $L_{g}L_{f}^{m-1}b(\boldsymbol{x})\boldsymbol{u}\ne0$ on the boundary of set $\mathcal C_{0}\cap,...,\cap \mathcal C_{m-1}.$ 
\end{definition}

\begin{theorem}[Safety Guarantee~\cite{xiao2021high}]
\label{thm:safety-guarantee}
Given a HOCBF $b(\boldsymbol{x})$ from Def. \ref{def:HOCBF} with corresponding sets $\mathcal{C}_{0}, \dots,\mathcal {C}_{m-1}$ defined by \eqref{eq:sequence-set1}, if $\boldsymbol{x}(0) \in \mathcal {C}_{0}\cap \dots \cap \mathcal {C}_{m-1},$ then any Lipschitz controller $\boldsymbol{u}$ that satisfies the constraint in \eqref{eq:highest-HOCBF}, $\forall t\ge 0$ renders $\mathcal {C}_{0}\cap \dots \cap \mathcal {C}_{m-1}$ forward invariant for system \eqref{eq:affine-control-system}, $i.e., \boldsymbol{x} \in \mathcal {C}_{0}\cap \dots \cap \mathcal {C}_{m-1}, \forall t\ge 0.$
\end{theorem}

\begin{definition}[CLF~\cite{ames2012control}]
\label{def:control-l-f}
A continuously differentiable function $V:\mathbb{R}^{n}\to\mathbb{R}$ is an exponentially stabilizing Control Lyapunov Function (CLF) for system \eqref{eq:affine-control-system} if there exist constants $c_{1}>0, c_{2}>0,c_{3}>0$ such that for $\forall \boldsymbol{x} \in \mathbb{R}^{n}, c_{1}\left \|  \boldsymbol{x} \right \| ^{2} \le V(\boldsymbol{x}) \le c_{2}\left \|  \boldsymbol{x} \right \| ^{2},$
\begin{equation}
\label{eq:clf}
\inf_{\boldsymbol{u}\in \mathcal U}[L_{f}V(\boldsymbol{x})+L_{g}V(\boldsymbol{x})\boldsymbol{u}+c_{3}V(\boldsymbol{x})]\le 0.
\end{equation}
\end{definition}

The existing works \cite{nguyen2016exponential},\cite{xiao2021high} combine HOCBFs \eqref{eq:highest-HOCBF} for systems with high relative degree with quadratic costs to form safety-critical optimization problems.
CLFs \eqref{eq:clf} can also be incorporated in optimization problems (see \cite{xiao2022sufficient},\cite{xiao2021adaptive}) if exponential convergence of some states is desired. In these works, time is discretized into time intervals,
and an optimization problem with constraints
given by HOCBFs and CLFs is solved in each
time interval. Since the state value is fixed
at the beginning of the interval, these
constraints are linear in control, therefore each optimization problem is a QP. The optimal control obtained by solving each QP is applied at the beginning of the interval and held constant for the
whole interval. During each interval, the state is updated using dynamics \eqref{eq:affine-control-system}. This method, which is referred to as CBF-CLF-QP, works conditioned on the fact that solving the QP at every time interval is feasible. However, this is not guaranteed, in particular
under tight or time-varying control bounds. 
 The authors of \cite{xiao2021adaptive} proposed a new type of HOCBF called PACBF, which introduced a time-varying penalty variable $p_{i}(t)$ in front of the class $\kappa$ function in the $i^{th}$ order HOCBF constraint ($i\in \{1,\dots,m\}$), trying to maximize the feasibility of solving CBF-CLF-QPs. However, the formulation of PACBFs requires the design of many additional constraints. Defining such constraints may not be straightforward, and may result in complicated parameter-tuning processes. To address this issue, we develop a new type of adaptive CBFs, called Auxiliary-Variable Adaptive Control Barrier Functions (AVCBFs), which is described in the next section.


% The process to construct PACBFs in \cite{xiao2021adaptive} can be summarized in next paragraphs.

% The first step is to define a time-varying penalty function $\boldsymbol{p}(t)\coloneqq(p_{1}(t),\dots,p_{m}(t)),$ then we need to define auxiliary dynamics for corresponding $p_{i}(t)$ as  Given a relative degree $m$ PACBF $b:\mathbb{R}^{n}\to\mathbb{R}$ and a time-varying penalty function  we can define a sequence of functions 
% \cite{xiao2021adaptive}

% \begin{equation}
% \label{eq:PACBF-sequence}
% \begin{split}
% &\psi_{1}(\boldsymbol{x},\boldsymbol{p}(t))\coloneqq\dot{\psi}_{0}(\boldsymbol{x})+p_{1}(t)\alpha_{1}(\psi_{0}(\boldsymbol{x})),\\ 
% &\psi_{i}(\boldsymbol{x},\boldsymbol{p}(t))\coloneqq\dot{\psi}_{i-1}(\boldsymbol{x},\boldsymbol{p}(t))+p_{i}(t)\alpha_{i}(\psi_{i-1}(\boldsymbol{x},\boldsymbol{p}(t))),
% \end{split}
% \end{equation}
% where $\psi_{0}(\boldsymbol{x})\coloneqq b(\boldsymbol{x}),  i\in \{2,...,m\}.$ Note that we need to define HOCBFs for each $p_{i}(t)$ to require $p_{i}(t)\ge0$ and also define auxiliary variables as virtual states and control inputs as Eq. (14) in \cite{xiao2021adaptive} to extend each $p_{i}(t)$ to $m^{th}$ order PACBF constraint. We further define a sequence of sets $\mathcal{C}_{i}$ associated with \eqref{eq:PACBF-sequence} in the form 
% \begin{equation}
% \label{eq:PACBF-set}
% \begin{split}
% &\mathcal C_{0}\coloneqq \{\boldsymbol{x}\in\mathbb{R}^{n}:\psi_{0}(\boldsymbol{x})\ge 0\}, \\
% &\mathcal C_{i}\coloneqq \{(\boldsymbol{x},\boldsymbol{p}(t)) \in \mathbb{R}^{n} \times \mathbb{R}^{m}:\psi_{i}(\boldsymbol{x},\boldsymbol{p}(t))\ge 0\}. 
% \end{split}
% \end{equation}
% where $i\in \{1,...,m-1\}.$ By ensuring solutions of Eq. (17) in \cite{xiao2021adaptive}, we hope to generate $\boldsymbol{x} \in \mathcal{C}_{0}, (\boldsymbol{x},\boldsymbol{p})\in \mathcal C_{1}\cap,...,\cap \mathcal C_{m-1}, \forall t \ge 0.$

% Note that in \eqref{eq:PACBF-sequence}, the multiplication of penalty function and class $\kappa$ function $p_{i}(t)\alpha_{i}(\psi_{i-1}(\boldsymbol{x},\boldsymbol{p}(t)))$ does not satisfy the property of strictly increasing mentioned in Def. \ref{def:class-k-f}, i.e., the multiplication may increase when $\psi_{i-1}(\boldsymbol{x},\boldsymbol{p}(t))$ decreases, as $p_{i}(t)$ may increase much faster, therefore ensuring $\psi_{i}(\boldsymbol{x},\boldsymbol{p}(t))\ge 0$ in \eqref{eq:PACBF-sequence} can not guarantee $\psi_{i-1}(\boldsymbol{x},\boldsymbol{p}(t))\ge 0$, which is different from the proof of Thm. \ref{thm:safety-guarantee} (see \cite{xiao2021high}). To address this issue, the authors defined CLFs $V_{i}(\boldsymbol{x})\coloneqq(p_{i}(t)-p_{i}^{\ast})^{2}$ to stabilize each $p_{i}(t)$ to a small enough value $p_{i}^{\ast}$ to make PACBF effective \cite{xiao2021high}. The major drawback of this method comes from the complicated parameters-tuning process. On the one hand, the number of hyperparameters $p_{i}^{\ast}$ is large if the relative degree of PACBF is high. On the other hand, the relative degree of CLF in \eqref{eq:clf} is $1,$ but the relative degree of $V_{i}(\boldsymbol{x})$ may be above $1,$ therefore we have to define a desired state feedback form like Eq. (24) in \cite{xiao2021high} to extend $V_{i}(\boldsymbol{x})$ to the desired high relative degree. This process will introduce many other hyperparameters to tune which affects the performance of CLF-CBF-QP. These issues call for the development of a new type of adaptive CBFs which could be easily used and named Auxiliary variable-based Control Barrier Functions (AVBCBFs) in next section.
