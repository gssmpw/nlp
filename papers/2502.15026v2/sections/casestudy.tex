\section{Implementation and Results}
\label{sec:Implementation}
In this section, we show how our proposed AVCBFs will provide the adaptivity and outperform the PACBFs in solving the optimal ACC problem under conservative control constraints. We consider the Prob. \ref{prob:ACC-prob} with time-varying control bounds \eqref{eq:constraint-u} (due to smoothness of vehicle tires and road surfaces), and implement AVCBFs or PACBFs as safety constraints for solving Prob. \ref{prob:ACC-prob} in MATLAB. We utilized ode45 to integrate the dynamic system for every $0.1s$ time-interval and quadprog to solve QP. Both methods show varying degrees of adaptivity to complexity of road conditions.

The parameters are $v_{p}=13.89m/s, v_{d}=24m/s, M=1650kg, g=9.81m/s^{2},z(0)=100m, l_{p}=10m, f_{0}=0.1N, f_{1}=5Ns/m, f_{2}=0.25Ns^{2}/m, c_{a}(t)=0.4.$
\subsection{Implementation with AVCBFs}
Define $b(\boldsymbol{x}(t))=z(t)-l_{p},$ the relative degree of $b(\boldsymbol{x}(t))$ with respect to dynamics \eqref{eq:ACC-dynamics} is 2. For simplicity, based on Rem. \ref{rem: sufficient-con}, we just introduce one auxiliary variable as $\boldsymbol{a}(t)=a_{1}(t).$ 
Motivated by Sec. \ref{sec:AVCBFs}, we define the auxiliary dynamics as
\begin{small}
\begin{equation}
\label{eq:Auxiliary-dynamics1}
\underbrace{\begin{bmatrix}
\dot{a_{1}}(t) \\
\dot{\pi}_{1,2}(t) 
\end{bmatrix}}_{\dot{\boldsymbol{\pi}}_{1}(t)}  
=\underbrace{\begin{bmatrix}
 \pi_{1,2}(t) \\
 0
\end{bmatrix}}_{F_{1}(\boldsymbol{{\pi}}_{1}(t))} 
+ \underbrace{\begin{bmatrix}
  0 \\
  1 
\end{bmatrix}}_{G_{1}(\boldsymbol{{\pi}}_{1}(t))}\nu_{1}(t).
\end{equation}
\end{small}
The HOCBFs for $a_{1}(t)$ are defined as 
\begin{equation}
\label{eq:SHOCBF-sequence-ACC}
\begin{split}
&\varphi_{1,0}(\boldsymbol{{\pi}}_{1}(t))\coloneqq a_{1}(t),\\
&\varphi_{1,1}(\boldsymbol{{\pi}}_{1}(t))\coloneqq \dot{\varphi}_{1,0}(\boldsymbol{{\pi}}_{1}(t))+l_{1}\varphi_{1,0}(\boldsymbol{{\pi}}_{1}(t)),\\
&\varphi_{1,2}(\boldsymbol{{\pi}}_{1}(t))\coloneqq \dot{\varphi}_{1,1}(\boldsymbol{{\pi}}_{1}(t))+l_{2}\varphi_{1,1}(\boldsymbol{{\pi}}_{1}(t)),
\end{split}
\end{equation}
where $\alpha_{1,1}(\cdot),\alpha_{1,2}(\cdot)$ are defined as linear functions. The AVCBFs are then defined as
\begin{equation}
\label{eq:AVBCBF-sequence-ACC}
\begin{split}
&\psi_{0}(\boldsymbol{x},\boldsymbol{{\pi}}_{1}(t))\coloneqq a_{1}(t)b(\boldsymbol{x}),\\
&\psi_{1}(\boldsymbol{x},\boldsymbol{{\pi}}_{1}(t))\coloneqq \dot{\psi}_{0}(\boldsymbol{x},\boldsymbol{{\pi}}_{1}(t))+k_{1}\psi_{0}(\boldsymbol{x},\boldsymbol{{\pi}}_{1}(t)),\\
&\psi_{2}(\boldsymbol{x},\boldsymbol{{\pi}}_{1}(t))\coloneqq \dot{\psi}_{1}(\boldsymbol{x},\boldsymbol{{\pi}}_{1}(t))+k_{2}\psi_{1}(\boldsymbol{x},\boldsymbol{{\pi}}_{1}(t)),
\end{split}
\end{equation}
where $\alpha_{1}(\cdot),\alpha_{2}(\cdot)$ are set as linear functions. By formulating constraints from HOCBFs \eqref{eq:SHOCBF-sequence-ACC}, AVCBFs \eqref{eq:AVBCBF-sequence-ACC}, CLF \eqref{eq:ACC-clf} and acceleration \eqref{eq:constraint-u}, we can define cost function 
 for QP as
 \begin{small}
\begin{equation}
\label{eq:AVBCBF-cost}
\begin{split}
\min_{u(t),\nu_{1}(t),\delta(t)} \int_{0}^{T}[(\frac{u(t)-F_{r}(v(t))}{M})^{2}\\+W_{1}(\nu_{1}(t)-a_{1,w})^{2}+Q\delta(t)^{2}]dt.
\end{split}
\end{equation}
\end{small}
Other parameters are set as $v(0)=6m/s, a_{1}(0)=1, \pi_{1,2}(0)=1, c_{3}=2, W_{1}=Q=1000,\epsilon=10^{-10}.$
\begin{figure}[ht]
    \centering
    \includegraphics[scale=0.58]{figures/Avbcbf-differentroads.png}
    \caption{Control input $u(t)$ varies as $b(\boldsymbol{x(t)})$ goes to 0 under different lower control bounds. The arrows denote the changing trend for $b(\boldsymbol{x(t)})$ and $c_{d}(t)$ over time. $b(\boldsymbol{x(0)})=90$ and $b(\boldsymbol{x(t)})\ge 0$ implies safety. Hyperparameters are set as $k_{1}=k_{2}=l_{1}=l_{2}=0.1, a_{1,w}=1,T=50s.$ }
    \label{fig:AVBCBFs-braking}
\end{figure} 
We first test the adaptivity to deceleration by changing the lower control bound $-c_{d}(t)Mg.$ In each test in Fig. \ref{fig:AVBCBFs-braking}, we set deceleration coefficient $c_{d}(t)$ to different constant or linearly decreasing variable due to different road conditions. For each case, the ego vehicle first accelerates to the same velocity ($b(\boldsymbol{x(t)})$ increases to the same value), then starts to decelerate at the same time. Due to different braking capability, the ego vehicle reaches different maximum deceleration (denoted by arrows) and finally keeps a constant velocity the same as $v_{p}$. The safe distance $l_{p}$ is maintained for all $t\in[0,50s].$ Note that under poor road conditions (e.g., the road is very slippery or the smoothness of road varies), shown by red or magenta curves, QPs are still feasible by using AVCBFs method, which shows good adaptivity to control constraints. 
\begin{figure}[ht]
    \centering
    \includegraphics[scale=0.58]{figures/Avbcbf-hyperparameters.png}
    \caption{Control input $u(t)$ varies as $b(\boldsymbol{x(t)})$ goes to 0 under different lower control bounds. The arrows denote the changing trend for $b(\boldsymbol{x(t)})$ and $c_{d}(t)$ over 50 seconds. $b(\boldsymbol{x(0)})=90$ and $b(\boldsymbol{x(t)})\ge 0$ implies safety. Different sets of hyperparameters for class $\kappa$ functions are tested.}
    \label{fig:AVBCBFs-hyperparameters}
\end{figure} 

Next, we test the adaptivity to conservativeness of control strategy by changing the hyperparameters $k_{1},k_{2},l_{1},l_{2}$ inside the class $\kappa$ functions in \eqref{eq:SHOCBF-sequence-ACC},\eqref{eq:AVBCBF-sequence-ACC}. We also change $c_{d}(t)$ for different hyperparameters. For each case in Fig. \ref{fig:AVBCBFs-hyperparameters}, the ego car first accelerates to the same velocity ($b(\boldsymbol{x(t)})$ increases to the same value), then starts to decelerate at different time (the orange and red curves represent breaking earliest, while the green and blue curves represent breaking later and the magenta curve represents breaking latest). Due to different braking capability, the ego vehicle reaches different maximum deceleration (denoted by arrows) and finally keeps a constant velocity the same as $v_{p}$. The safe distance $l_{p}$ is maintained for all $t\in[0,50s].$ Note that from comparing magenta curve with orange curve, larger hyperparameters allow the ego vehicle to brake later and to reach larger deceleration, but if the $c_{d}(t)$ is very small (like green curve), larger hyperparameters always cause infeasibility of QPs (the ego vehicle does not have enough long distance to brake down safely, therefore infeasible at 25.3s). We can make ego vehicle brake faster (as blue curve shows) by adjusting $a_{1,w}$ in cost function \eqref{eq:AVBCBF-cost}, which shows good adaptivity of AVCBFs to managing control strategy.

\subsection{Implementation with PACBFs}

Similar to AVCBFs, we define $b(\boldsymbol{x}(t))=z(t)-l_{p}$ for PACBFs. We use the same penalty function and auxiliary dynamics in \cite{xiao2021adaptive} as $\boldsymbol{p}(t)=(p_{1}(t), p_{2}(t))$ and
\begin{equation}
\label{eq:AVBCBFs-PACBFs-1}
\begin{split}
\dot{p_{1}}(t)=\nu_{1}(t),\\
p_{2}(t)=\nu_{2}(t).
\end{split}
\end{equation}
To make $p_{1}(t)$ converge to a small enough value, we define CLF constraint as
\begin{equation}
\label{eq:HOCBFs-CLFs}
\begin{split}
2(p_{1}(t)-p_{1}^{\ast })\nu_{1}(t)+\rho (p_{1}(t)-p_{1}^{\ast})^{2}\le \delta_{p}(t),
\end{split}
\end{equation}
where $\rho=10, p_{1}^{\ast}=0.103, \delta_{p}(t)$ is relaxed variable. We define HOCBF constraints for $p_{1}(t)$ as 
\begin{equation}
\label{eq:Auxiliary-PACBFs}
\begin{split}
-\nu_{1}(t)+(3-p_{1}(t))\ge0,\\
\nu_{1}(t)+p_{1}(t)\ge0,
\end{split}
\end{equation}
which confines $p_{1}(t)$ into $[0,3].$ The PACBFs are then defined as 
\begin{equation}
\label{eq:AVBCBFs-PACBFs-2}
\begin{split}
&\psi_{0}(\boldsymbol{x})\coloneqq b(\boldsymbol{x}),\\
&\psi_{1}(\boldsymbol{x},\boldsymbol{p}(t))\coloneqq \dot{\psi}_{0}(\boldsymbol{x})+p_{1}(t)\psi_{0}(\boldsymbol{x})^{2},\\
&\psi_{2}(\boldsymbol{x},\boldsymbol{p}(t),\boldsymbol{\nu}(t))\coloneqq \dot{\psi}_{1}(\boldsymbol{x},\boldsymbol{p}(t))+\nu_{2}(t)\psi_{1}(\boldsymbol{x},\boldsymbol{p}(t))
\end{split}
\end{equation}
to guarantee the safety. By formulating constraints from HOCBFs \eqref{eq:Auxiliary-PACBFs}, CLFs \eqref{eq:HOCBFs-CLFs}\eqref{eq:ACC-clf}, PACBFs \eqref{eq:AVBCBFs-PACBFs-2} and acceleration \eqref{eq:constraint-u}, we can define cost function 
 for QP as
 \begin{small}
\begin{equation}
\label{eq:PACBF-cost}
\begin{split}
\min_{u(t),\nu_{1}(t),\nu_{2}(t),\delta(t),\delta_{p}(t)} \int_{0}^{T}[(\frac{u(t)-F_{r}(v(t))}{M})^{2}+W_{1}\nu_{1}(t)\\+W_{2}(\nu_{2}(t)-1)^{2}+Q\delta(t)^{2}+Q_{p}\delta_{p}(t)^{2}]dt. 
\end{split}
\end{equation}
\end{small}
Other parameters are set as $p_{1}(0)=0.103, p_{2}(0)=1, c_{3}=10, W_{1}=W_{2}=2e^{12},Q=Q_{p}=1.$
\subsection{Comparison between AVCBFs and PACBFs}
\begin{figure}[ht]
    \centering
    \includegraphics[scale=0.58]{figures/AVBCBF-PACBF.png}
    \caption{Control input $u(t)$ varies as $b(\boldsymbol{x(t)})$ goes to 0 under different lower control bounds. The arrows denote the changing trend for $b(\boldsymbol{x(t)})$ and $c_{d}(t)$ over 50 seconds. $b(\boldsymbol{x(0)})=90$ and $b(\boldsymbol{x(t)})\ge 0$ implies safety. Solid curves denote AVCBFs and dashed curves denote PACBFs. }
    \label{fig:AVBCBFs-PACBFs-1}
\end{figure} 
% \vspace{-0.5cm}

We compare our proposed AVCBFs with the state of the art (PACBFs) for a more urgent braking case by making initial velocity large as $v(0)=20m/s.$ In Fig. \ref{fig:AVBCBFs-PACBFs-1}, we change the lower control bound $-c_{d}(t)Mg$ to different value to compare both methods' adaptivity. Hyperparameter of CLF \eqref{eq:ACC-clf} for maganta and cyan curves is set as $c_{3}=70$ and for orange, green curves, we set $c_{3}=100$. Other hyperparameters for AVCBFs are set as $k_{1}=k_{2}=l_{1}=l_{2}=0.1, W_{1}=2e^{5}, Q=7e^{5}.$ We choose the cases denoted by cyan, orange and red curves where $c_{d}(t)=0.23$ in Fig. \ref{fig:AVBCBFs-PACBFs-1} and further analyze them in Fig. \ref{fig:AVBCBFs-PACBFs-2}. Based on Figs. \ref{fig:AVBCBFs-PACBFs-1} and \ref{fig:AVBCBFs-PACBFs-2}, by both methods, the ego vehicle first accelerates to desired velocity around $24m/s,$ then decelerates until $v=v_{p}.$ Even both methods work well for the cases shown in two figures, AVCBFs show more adaptivity to the road condition by simply adjusting $a_{1,w}$ thus the ego vehicle can brake earlier with even more strict limitation of deceleration (i.e., the smaller $c_{d}(t)$ due to more slippery road condition). For PACBFs, it is difficult to adjust hyperparameters to make ego vehicle adaptive to poor road conditions as $p_{1}(t)$ is always required to be a small enough by additional CLF \eqref{eq:HOCBFs-CLFs}. We also note that AVCBFs can generate a smoother optimal controller in Fig. \ref{fig:AVBCBFs-PACBFs-1}, as there is no noticeable overshoot (for blue and red curves by PACBFs, the controller shows overshooting strategy near the end, which is not smooth).
\begin{figure}[ht]
    \centering
    \includegraphics[scale=0.58]{figures/AVBCBF-PACBF-dynamics.png}
    \caption{Control input $u(t)$, velocity $v(t)$, time-varying $p_{1}(t)$ and distance between two vehicles $b(\boldsymbol{x(t)})$ over 30 seconds for AVCBFs and PACBFs. $b(\boldsymbol{x(t)})\ge 0$ implies safety. Solid curves denote AVCBFs and dashed curve denotes PACBFs.}
    \label{fig:AVBCBFs-PACBFs-2}
\end{figure} 

