\section{Related works}
\noindent\textbf{Fine-tuning diffusion models. } Fine-tuning diffusion models aims to adapt pre-trained models to boost the reward on downstream tasks. Methods in this domain include directly backpropagating the reward \citep{clarkdirectly}, RL-based fine-tuning \citep{fan2024reinforcement, black2023training}, direct latent optimization \citep{tang2024tuning, wallace2023end}, guidance-based approach \citep{guo2024gradient, chung2022diffusion} and optimal control \citep{uehara2024fine}. Although entropy regularization is often incorporated into the reward to prevent over-optimization, no existing work has explored designing an efficient bilevel method to tune its strength. 

\noindent\textbf{Noise scheduling in diffusion models. } Noise schedule is crucial to balance the computational efficiency with data fidelity during image generation. Early works, such as DDPM \citep{ho2020denoising}, employed simple linear schedules for noise variance, while \citet{nichol2021improved} introduced cosine schedules to enhance performance. Recent studies \citep{lin2024common,chen2023importance} have highlighted limitations in traditional noise schedules and proposed new parameterization to improve the image quality. However, none of them considered using bilevel optimization to automatically learn the noise schedule. 

\noindent\textbf{Bilevel hyperparameter optimization. } Bilevel optimization has been explored as an efficient hyperparameter optimization framework, including hypernetwork search \citep{mackayself,liu2018darts}, hyper-representation \citep{franceschi2018bilevel}, regularization learning \citep{shaban2019truncated} and data reweighting \citep{shaban2019truncated,franceschi2017forward}. Recently, it has been explored in federated learning \citep{tarzanagh2022fednest} and LLM fine-tuning \citep{shen2024seal,zakarias2024bissl}. None of the existing works have explored hyperparameter optimization in diffusion models, and the methods proposed so far are inapplicable due to the infinite-dimensional probability space and the high computational cost of sampling. 


\begin{figure}[tbp]
    % \vspace{-0.3cm}
    \centering
    \setlength{\tabcolsep}{1pt} % Reduce column spacing
    \resizebox{0.5\textwidth}{!}{ % Scale the table to fit the page width
        \begin{tabular}{ccc}
            \includegraphics[width=0.12\textwidth]{figures/visualization_mnist/second_exp/bilevel2.png} &
            \includegraphics[width=0.12\textwidth]{figures/visualization_mnist/second_exp/default.png} &
            \includegraphics[width=0.12\textwidth]{figures/visualization_mnist/second_exp/bayesian.png} \\
            \scriptsize{(a) Bilevel} & \scriptsize{(b) DDIM (default)} & \scriptsize{(c) Bayesian} \\
        \end{tabular}
    }
    \vspace{-0.5cm}
    \caption{Visualization of the final generated images by different methods using cosine parameterization.}
    \label{fig:results_visualization}
    \vspace{-.7cm}
\end{figure} 


\vspace{-0.2cm}