\section{Related works}
\noindent\textbf{Fine-tuning diffusion models. } Fine-tuning diffusion models aims to adapt pre-trained models to boost the reward on downstream tasks. Methods in this domain include directly backpropagating the reward ____, RL-based fine-tuning ____, direct latent optimization ____, guidance-based approach ____ and optimal control ____. Although entropy regularization is often incorporated into the reward to prevent over-optimization, no existing work has explored designing an efficient bilevel method to tune its strength. 

\noindent\textbf{Noise scheduling in diffusion models. } Noise schedule is crucial to balance the computational efficiency with data fidelity during image generation. Early works, such as DDPM ____, employed simple linear schedules for noise variance, while ____ introduced cosine schedules to enhance performance. Recent studies ____ have highlighted limitations in traditional noise schedules and proposed new parameterization to improve the image quality. However, none of them considered using bilevel optimization to automatically learn the noise schedule. 

\noindent\textbf{Bilevel hyperparameter optimization. } Bilevel optimization has been explored as an efficient hyperparameter optimization framework, including hypernetwork search ____, hyper-representation ____, regularization learning ____ and data reweighting ____. Recently, it has been explored in federated learning ____ and LLM fine-tuning ____. None of the existing works have explored hyperparameter optimization in diffusion models, and the methods proposed so far are inapplicable due to the infinite-dimensional probability space and the high computational cost of sampling. 


\begin{figure}[tbp]
    % \vspace{-0.3cm}
    \centering
    \setlength{\tabcolsep}{1pt} % Reduce column spacing
    \resizebox{0.5\textwidth}{!}{ % Scale the table to fit the page width
        \begin{tabular}{ccc}
            \includegraphics[width=0.12\textwidth]{figures/visualization_mnist/second_exp/bilevel2.png} &
            \includegraphics[width=0.12\textwidth]{figures/visualization_mnist/second_exp/default.png} &
            \includegraphics[width=0.12\textwidth]{figures/visualization_mnist/second_exp/bayesian.png} \\
            \scriptsize{(a) Bilevel} & \scriptsize{(b) DDIM (default)} & \scriptsize{(c) Bayesian} \\
        \end{tabular}
    }
    \vspace{-0.5cm}
    \caption{Visualization of the final generated images by different methods using cosine parameterization.}
    \label{fig:results_visualization}
    \vspace{-.7cm}
\end{figure} 


\vspace{-0.2cm}