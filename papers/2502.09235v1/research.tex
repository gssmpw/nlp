\section{Research}\label{sec:research}

% \subsection{Goals of the Research}
My research focuses on the foundations of hybrid ASP with the goal of both
understanding better its fundamental properties and
exploiting this knowledge to guide and improve solver implementations.
%
Further, as this research is motivated by problems in real-world applications,
another goal is to better understand the essence of these problems and how they can be solved using hybrid ASP.
%
More precisely, the objective is to find mathematical or logical formalizations of these problems which subsequently serve as basis for succint but general ASP representations.
%
These two goals are reciprocally beneficial as a deeper insight into real-world problems will make clearer the necessary research directions on the foundational level.
%
In the context of applications, my current focus lies on problems in the realm of (industrial product) configuration.

\subsection{Contributions and Future Work}
% \subsection{Current Status of the Research}
Regarding the theoretical aspects of my research I am currently working on the theoretical foundations of solver \fclingo\
with the goal of improving the current implementation.
%
Here, one of the open issues is that current results in \HTC\ only allow for the use of GZ aggregate semantics (see Sec.~\ref{sec:semantics}) in \fclingo.
%
However, we would like to be able to use Ferraris aggregate semantics which guarantee definedness as known from \clingo.
%
Our current approach here consists of finding a suitable translation between the two semantics.

Another open issue is the formalization of solver \clingodl\ by means of logic \HTLB.
%
The concept of assigning a minimal, founded value to integer variables of \HTLB\ seems like a natural match with
the difference constraints in \clingodl\ which are defined as inequalities, thus, generally have multiple valid solutions
but only one or a few minimal ones.
%
% Currently, there exists a characterization for the former in terms of so-called abstract theories in \cite{cafascwa23a}
% but it is left to prove a correspondence between the notion of answer sets in the former and equilibrium models in \HTLB.
%
% More distant future work possibly includes the further study of different aggregates semantics.
% %
% We already mentioned two approaches for which a characterization in terms of \HTC\ exists.
% %
% Another one for which this has not been done is \cite{sonpon07b}.

% \subsection{Preliminary Results}
On the practical side of my research, preliminary results have been found in application of (plain and hybrid) ASP to configuration problems.
%
In \cite{ruscst23a} we developed a principled approach to configuration that included a mathematical formalization
of configuration problems with an ASP-based solution.
%
We defined a configuration problem in terms of an abstract model and a concrete instantiation.
While the model serves as a blueprint for all possible configurations, the instantation represents a solution.
%
This work was accompanied by a corresponding fact format and two ASP encoding, one for \clingo\ and one for \fclingo,
which were subsequently made public \footnote{\url{https://github.com/potassco/configuration-encoding}}.

A similar but slightly different work has been done in \cite{baheosreruscwa24a} where we developed the \coomsuite\ \footnote{\url{https://github.com/potassco/coom-suite}},
a workbench for experimentation with industrial-scale product configuration problems.
%
The \coomsuite\ is built around product configuration language \coom \cite{coomlang}
\footnote{\Coom\ is a domain-specific language developed by denkbares GmbH and used in numerous industrial applications}
and provides a \coom\ grammar for parsing, a specialized ASP translator for conversion into facts, two encodings (one for \clingo\ and one for \fclingo)
as well as various benchmark sets.
%
The intention is to ease the development of powerful methods able to perform in industrial settings.

Future work here includes the further study of suitable representations for hybrid solver \fclingo.
%
The current \fclingo\ encodings do not necessarily use all features the solver has to offer, eg.\ undefinedness of numeric variables,
but rather leaves this to non-hybrid ASP.
%
The reason for this is that these encodings have been constructed with a plain ASP encoding as base,
only modifying the necessary parts.
%
An approach we want to pursue here is to find a logical formalization of configuration problems in terms of \HTC\
and use this as basic for new encodings which make more natural use of \fclingo's features.
%
We expect that this will not only improve the knowledge representation but also the performance of the solver.
