\section{Background}\label{sec:background}

\subsection{Hybrid Solvers}\label{sec:systems}
Nowadays, ASP solver \clingo\ supports so-called \emph{theory atoms} which allow for foreign inference methods \cite{jakaosscscwa17a,karoscwa21a}
following the approach of SAT \emph{modulo theories} (SMT; \cite{SATHandbook}).
%
This has greatly facilitated the development of ASP-based special-purpose systems which make use of dedicated inference methods for
certain subclasses of constraints such as difference logic and linear programming.
%
The general idea is that some external theory serves as an oracle by certifying some of a program's stable models and has been characterized for \clingo\ in \cite{cafascwa23a}.
%
We proceed by giving a quick introduction of some of the hybrid solvers that are part of the \potassco\ suite \footnote{\url{https://potassco.org/}}.

The system \clingcon\ is a solver for \emph{Constraint Answer Set Programming} (CASP) and extends the input language of \clingo\ with linear equations,
represented as theory atoms of the form
\begin{align}\label{clingcon:linear:constraint}
    \code{\&sum\{\mathit{k_1*x_1};\dots;\mathit{k_n*x_n}\}} \prec k_0
\end{align}
where $x_i$ is an integer variable and $k_i\in\mathbb{Z}$ an integer constant for $0\leq i\leq n$;
and $\prec$ is a comparison symbol such as \code{<=}, \code{=}, \code{!=}, \code{<}, \code{>}, \code{>=}.
In \clingo, theory predicates are preceded by `\code{\&}'.

System \clingodl\ has a more restricted syntax which allows for \emph{difference constraints} over integers.
This is a subset of the syntax in (\ref{clingcon:linear:constraint}) where theory atoms have the fixed form
$\code{\&sum\{1 * \mathit{x}; (-1)*\mathit{y}\} <= \mathit{k}}$
but are rewritten instead as:
%
\begin{align}\label{clingodl:difference:constraint}
    \diffc{x}{y}{k}
\end{align}
%
with $x$ and $y$ integer variables and $k \in\mathbb{Z}$.

A third system is \clingolp\ \footnote{\url{https://github.com/potassco/clingoLP}} which extends \clingo\ to solve linear constraints
as dealt with in Linear Programming (LP).
The syntax is identical to (\ref{clingcon:linear:constraint}) but the domain now ranges over the real numbers.
%
Notably, \cite{cafascwa23a} contains a formal characterization of all three just mentioned systems.

Lastly, a recent addition is system \fclingo\ \footnote{\url{https://github.com/potassco/fclingo}} which makes use of \clingcon\
to solve ASP modulo conditional linear constraints with founded variables.
%
While in \clingcon\ all integer variables need to have a value assigned,
\fclingo\ adds a notion of undefinedness and foundedness as known from ASP,
ie.\ there needs to be a justification in the logic program if a variable receives a value in an answer set.
%
Further, the conditional aspect of the linear constraints can be seen as a generalization of the concept of aggregates commonly used in ASP.
%
The syntax of \fclingo\ accomodates so-called \emph{assignments} which guarantee that a variable only gets assigned a value if
all other variables in its definition are itself defined, ie.\ justified at some other part in the logic program.
For instance, the expression
\begin{align*}\label{fclingo:assignment}
    \code{\&in\{\mathit{y}..\mathit{y}\} =: \textit{x}}
\end{align*}
only assigns the value of $y$ to $x$ if $y$ has been defined by some other rule.
%
Omitting the assignment would permit $y$ and $x$ to take arbitrary values if not defined elsewhere, thereby circumventing the principle of foundedness.

Further systems not developed by \potassco\ include ASP solver \dlvhex\ \cite{redl16a} which supports a similar concept of theory atoms as \clingo.
%
Other CASP systems include \dingo\ \cite{janise09a}, \mingo\ \cite{lijani12a} and \ezsmt\ \cite{liesus16a}.
Different from the aforementioned systems, all three rely on translations to non-ASP solvers.


\subsection{The Logic of Here-and-There and Hybrid Extensions}\label{sec:semantics}
The logics \HT\ and \emph{Equilibrium Logic} nowawadays serve as a logical foundation for (plain) ASP,
having brought upon fundamental results such as the notion of \emph{strong equivalence} \cite{lipeva01a}.
%
The idea of \HT\ is that of two worlds $h$ and $t$, generally called \emph{here} and \emph{there}.
\footnote{This is based on Kripke semantics for intuitionistic logic, see \cite{dalen86a}}
%
More precisely, an \HT-interpretation is a pair \handt\ of sets of atoms such that $H \subseteq T$.
%
This gives rise to a three-valued logic where atoms can either be \textit{true}, \textit{false} or \textit{undefined}.
%
A formula $\varphi$ is \emph{satisfied} or \emph{holds} in a model \handt\, in symbols $\handt \models \varphi$, if it is true in the model, ie.\ satisfied at the $h$\nobreakdash-world.
%
A model \handt\ of a theory $\Gamma$ is called an \emph{equilibrium model} if
(i) it is total, ie.\ $H = T$, and
(ii) for any $H'$ such that $H' \subset T$, $\handt \not\models \Gamma$.
%
The term equilibrium model was coined in \cite{pearce96a} and there is complete agreement between equilibrium models and the stable models of logic programs as defined in \cite{gellif88b}.


In an attempt to provide a solid, logical foundation for hybrid systems such as the ones introduced in Section~\ref{sec:systems},
a number of extensions of \HT\ for incorporating constraints have been introduced.

The Logic of \emph{Here-and-There with constraints} (\HTC; \cite{cakaossc16a}) allows for capturing constraint theories in the non-monotonic setting
and has subsequently been extended with aggregate functions over constraint values and variables \cite{cafascwa20a,cafascwa20b}.
%
In \cite{cafascwa20b} specifications for aggregate functions in terms of \HTC\ are given based on two different semantic principles.
%
While the semantics given in \cite{ferraris11a} ensures that aggregate terms are always defined,
\cite{gelzha14a} prohibits so-called \emph{vicious cycles}. % which essentially means that no object can be derived by a definition depending on that object itself.
% While the former has from the beginning been defined in terms of propositional logic,
% a propositional characterization for the latter has been given later in \cite{cafascsc17b}.
%
We also refer to the former as \emph{Ferraris} and to the latter as \emph{Gelfond-Zhang} (GZ) aggregate semantics.

The Logic of \emph{Here-and-there with lower bound founded variables} (\HTLB; \cite{cafascsc19a}) generalizes the concept of \emph{foundedness} to integer variables.
The idea is that variables get assigned the smallest integer value that can be justified.
This can be seen as a generalization of plain \HT\ if one regards Boolean truth values as ordered by letting \textit{true} be greater than \textit{false}.

Both of these extensions can be seen as \emph{black-box} approaches in the sense that the constraints are incorporated
as special entities whose syntax and satisfaction relations are generally left open.
Thus, the intricacies of the hybrid part are mostly unknown from the logic program perspective.
%
Another \HT\ extension with a \emph{white-box} approach of constraints is \ASPAC\ \cite{eitkie20a}
which generalizes logical connectives as a particular case of more general operations on weighted formulas over semirings.
In this setting, operators like logical conjunction $\wedge$ become just one more possible operation that can be combined with others,
such as addition or multiplication (depending on the underlying semiring).
%
This results in a very expressive and powerful formalism but at the price of a more complex semantics and the requirement of a semiring structure.

Further white-box approaches are based on the incorporation of intensional or non-Herbrand functions in ASP.
For instance, \cite{cabalar11a} added partial intensional functions to a quantified First-Order version of \HT\ \cite{peaval04b} and
later extended this to sets and aggregates \cite{cafafape18a}.
% \comment{Elaborate}

% \comment{Aggregates? Son-Pontelli and Future work with HT?}
\subsection{Configuration}
A wide range of approaches exist for representing and solving configuration problems
across various paradigms~\cite{junker06a,hofestrybawo14a}.
%
In recent years, ASP has emerged as a promising alternative,
as evidenced by several applications~%
\cite{%
    gekasc11c,%
    fefaateruraz17a,%
    gescer19a,%
    hebasasc22a%
}.
%
Moreover, \cite{faryscsh15a} developed an object-oriented approach to configuration by directly defining concepts in ASP.
%
In the context of interactive configuration,
\cite{fahakrscscta20a}~conducted a comparative evaluation of various systems,
including the ASP solver \clingo\ as well as SAT and CP systems,
for their suitability in this context, finding \clingo\ to be as capable as any other system.
