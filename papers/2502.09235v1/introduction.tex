\section{Introduction}\label{sec:introduction}
Answer Set Programming (ASP; \cite{lifschitz19a}) is being increasingly applied to solve problems from the real-world.
%
However, in many cases these problems have a heterogenous nature which requires features beyond the current capacities of solvers like \clingo\ \cite{gekakasc17a}.
%
Consider for example the field of configuration \cite{kbc14}; one of the early successful applications of ASP \cite{gekasc11c,gescer19a}.
%
A configuration problem usually consists of (at least) a partonomy where parts are parameterized by attributes whose values in turn are restricted by constraints.
% an (abstract) model and concrete instantiations \cite{ruscst23a}.
%
While in simple cases these attributes are discrete, many industrial applications require attributes
that range over large numeric domains (eg.\ precisions in the milimeter range might be needed).
Further, calculations over these attributes can be of linear nature (eg.\ calculating the total weight by summing up the weight of all parts) as well as
non-linear (eg.\ area or volume of an object, inclination of a conveyor belt, etc).
%
Standards ASP solvers like \clingo\ quickly reach their limits when dealing with numeric ranges and calculations
as they need to explicitly ground all possible values.
Apart from this, representing constraints in ASP that go beyond simple arithmetic expressions or aggregations generally requires considerable effort.

Over the last years, hybrid solvers such as
\clingcon\ \cite{bakaossc16a}\footnote{\url{https://potassco.org/clingcon}} and
\clingodl\ \cite{jakaosscscwa17a}\footnote{\url{https://github.com/potassco/clingo-dl}}
which make use of dedicated inference methods for certain kinds of constraints over finite integer domains
have already been successfully applied to many problems such as train scheduling \cite{abjoossctowa21a} and warehouse delivery \cite{rascwachliso23a}.
%
However, what is still missing, is a solid, semantic underpinning of these systems. % there remain many open questions regarding their semantic foundations.

This issue has first been addressed by introducing the Logic of \emph{Here-and-There with constraints} (\HTC; \cite{cakaossc16a}) as
an extension of the Logic of \emph{Here-and-There} (\HT; \cite{heyting30a}) and its non-monotone extension \emph{Equilibrium Logic} \cite{pearce06a}.
%
Nowadays, \HT\ serves as a logical foundation for ASP and has facilitated a broader understanding of this paradigm.
The idea is that \HTC\ (and other extensions; see Section~\ref{sec:background}) play an analogous role for hybrid ASP.

There remain many open questions about these logics regarding their fundamental characteristics
as well as their practical use in solvers, ie.\ how they can guide the implementation.
%
Having a formal understanding of these hybrid logics is also needed
to better understand the inherent structure of the (real-world) problems they are applied to, eg.\ configuration, and to improve their representations in ASP.


%
% \begin{itemize}
%       \item Many real-world applications have a het cannot be solved by pure ASP but require hybrid reasoning.
%       \item Consider for example the field of (product) configuration, one of the earliest successful applications of ASP \comment{cite}.
%       \item A configuration problem usually consists of an (abstract) model and concrete instantiations \cite{ruscst23a}.
%             The model usually consists of (at least) a partonomy where parts are parameterized by attributes whose values in turn are restricted by constraints.
%       \item In simple cases these attributes are discrete, however, many industrial applications require attributes with large ranges over the integers, rational or even real numbers (eg.\ precisions in the milimeter range might be needed)
%             Further, calculations over these attributes might require linear calculations (eg.\ sum or count aggregates) as well as
%             non-linear calculations (eg.\ area or volume of an object, inclination of a conveyor belt, etc.).
%       \item Recent versions of ASP solver \clingo\ allow for so-called \emph{theory atoms} which allow for foreign inference methods.
%       \item However, \clingo\ quickly reaches its limits when dealing with numeric ranges and calculations as it needs to explicitly ground the numeric ranges and possible values of, say, arithmetic calculations.
%       \item In recent years hybrid solvers as \clingcon, \clingodl, and \fclingo\ have started to tackle these problems by introducing dedicated inference methods for certain kinds of problems \comment{Introduce these systems}.
%       \item However, they are lacking a semantic (theoretic) foundation.
%       \item There are various approaches which have started to tackle this
%             \begin{itemize}
%                   \item based on denotations~\cite{cakaossc16a,cafascsc19a,cafascwa20a,cafascwa20b}
%                         \comment{Add: Towards a semantics for Hybrid ASP systems?}
%                   \item
%             \end{itemize}
%       \item Regarding the applications, formal understandings of these systems are needed also to understand the inherent structure of these problems and to improve hybrid encodings.
%       \item Current fclingo encodings do not use features like undefinedness, etc\dots . This part is left to clingo.
%       \item A formalization of eg.\ configuration problems in terms of a hybrid language such as \HTC\ could improve the encodings.
% \end{itemize}
