\documentclass[submission,copyright,creativecommons]{eptcs}
\providecommand{\event}{ICLP/LPNMR-DC 2024}

\usepackage{underscore}
\usepackage[utf8]{inputenc}
\usepackage{amsmath}
\usepackage{amssymb}
\usepackage{microtype}
\usepackage{url}\urlstyle{tt}

\section{OUR APPROACH: DQuaG}

\begin{figure*}[tb]
\centering
\includegraphics[width=0.75\textwidth]{Figures/framework_adqv.png}
\vspace{-0.6\baselineskip}
\caption{Data Quality Validation Framework Using GNN and VAE. Top: Training on clean data for Approach Establishment. Bottom: Validating unseen datasets by reconstruction error comparison.}
\vspace*{-0.3cm}
%\soror{highlight in the caption the process at the top and the one at the bottom of teh figure }
\label{fig:framework}
\end{figure*}

%In this section, we detail our novel approach to automated data quality validation using graph representation learning and a Variational Autoencoder (VAE) framework. Assuming we start with a clean dataset, our method addresses the limitations of traditional data quality verification techniques through a series of steps designed to capture intrinsic relationships within tabular data and assess data quality with minimal expert intervention.


In this section, we present DQuaG (Data Quality Graph), a novel approach for data quality validation. 
Figure~\ref{fig:framework} illustrates the framework of our approach, which includes both the training process using a clean dataset to train the GNN and VAE, and the data quality validation process using these trained models. 
%\qw{For the training phase, }
%\qw{For the validation phase, }

Note that DQuaG requires a clean dataset to train its models, which is a common assumption when embracing VAE in relevant problems.
The clean dataset serves as the foundational benchmark for our model, providing a reference state of high data quality against which data errors are identified. 
%This dataset is not only used to train the GNN and VAE models, but also to define a normative baseline for what constitutes acceptable data quality.

% \subsection{Training GNN and VAE on Clean Data for Data Quality Validation}
\subsection{Training GNN and VAE on Clean Data}
\subsubsection{\textbf{Feature Graph Construction}}

The initial step in our approach involves constructing a feature graph from clean tabular data to capture intrinsic relationships and dependencies between data features.
First, we address the challenge of diverse data types: categorical variables are transformed using label encoding, and timestamp data is broken into components (i.e., day, month, year). 
This uniform input format is critical for graph-based processing.

We then use ChatGPT-4~\cite{openai2024gpt4} to automate the feature graph construction. 
Given a clean dataset, we extract the feature names \( F \) and their descriptions \( D \) from the data source. We then randomly sample 100 data points from the dataset, denoted as \( S \). These feature names, descriptions, and sample data points are provided to the ChatGPT-4, structured as follows: \(\text{Input} = \{ F, D, S \}\), then ChatGPT-4 generates a JSON file capturing feature relationships.
The output format is \(\text{Feature\_Relationships} = \{ (f_i, f_j) \mid f_i, f_j \in F \}\), indicating that there is a relationship between features \( f_i \) and \( f_j \).

Using these relationships, we construct the knowledge-based feature graph \( G = (V, E) \), where \( V \) represents features and \( E \) represents edges indicating relationships between features.

% \subsubsection{\textbf{Feature Graph Construction}}

% The initial step in our methodology involves constructing a feature graph from clean tabular data, which is essential for capturing the intrinsic relationships and dependencies between different data features.

% To facilitate this process, our approach first addresses the challenge of handling diverse data types. 
% In the preprocessing stage, categorical variables are transformed using label encoding, which assigns each unique category a unique integer based on alphabetical ordering. 
% For timestamp data, we extract significant components such as day, month, and year. 
% This uniform input format is critical for the subsequent graph-based processing. 

% Following the preprocessing, we utilize a large language model, ChatGPT-4 \cite{openai2024gpt4}, to automate the construction of the feature graph. This integration allows for a more nuanced capture of feature relationships and dependencies, reducing reliance on expert knowledge and manual effort.

% Given a clean dataset, we extract the feature names \( F = \{f_1, f_2, \ldots, \\f_n\} \) and their descriptions \( D = \{d_1, d_2, \ldots, d_n\} \) from the data source. We then randomly sample 100 data points from the dataset, denoted as \( S = \{s_1, s_2, \ldots, s_{100}\} \). These feature names, descriptions, and sample data points are provided to the LLM, structured as follows: \(\text{Input} = \{ F, D, S \}\).

% The LLM analyzes the provided input and generates a structured JSON file capturing the relationships between features. The output format is \(\text{Feature\_Relationships} = \{ (f_i, f_j) \mid f_i, f_j \in F \}\), indicating that there is a relationship between features \( f_i \) and \( f_j \).

% Using the relationships provided by the LLM, we construct the feature graph \( G = (V, E) \) where \( V = F \) (nodes representing features) and \( E = \{(f_i, f_j) \mid (f_i, f_j) \in \text{Feature\_Relationships} \} \) (edges representing relationships). 

%----------------------------
%This graph-based representation allows us to model complex interdependencies within the data that are often overlooked by traditional methods, enhancing our ability to perform thorough data quality assessments.



% \subsubsection{\textbf{Training the Graph Neural Network (GNN) and Representing the Clean Dataset}}
\subsubsection{\textbf{Training GNN and preparing training data for VAE}}

Once the feature graph is constructed, we train a Graph Convolutional Network (GCN)~\cite{zhang2019graph} to generate feature embeddings. 
The GCN processes both the feature graph \( G = (V, E) \) and the original tabular data. Assume each instance in the original tabular data is an \( n \)-dimensional tuple \( \mathbf{x} \in \mathbb{R}^n \).

The GCN leverages the feature graph to learn the intrinsic relationships between features and produces embeddings that reflect the underlying structure of the clean data. Specifically, for each instance \( \mathbf{x} \), the GCN generates an \( n \)-dimensional embedding \( \mathbf{z} \in \mathbb{R}^n \). 
%This embedding \( \mathbf{z} \) captures the information from each feature's value and incorporates the relationships between features as learned from the feature graph.
% We train the GCN using the Adam optimizer, which is well-suited for handling graph-based data. The loss function used during training is the mean-squared error (MSE) between the predicted and true values for a set of labeled data. 
Formally, let \( \mathbf{Z} \) be the matrix of embeddings, where each row \( \mathbf{z}_i \) corresponds to an instance \( \mathbf{x}_i \). 
The GCN updates the embeddings by aggregating information from neighboring nodes in the feature graph, ensuring that the final embedding \( \mathbf{z}_i \) incorporates both the feature values and the relationships between features.

%These embeddings \( \mathbf{Z} \) serve as a compact and informative representation of the data's quality attributes, providing a robust basis for subsequent data quality assessment.


% \subsubsection{\textbf{Training the VAE for Encoding and Decoding}}
\subsubsection{\textbf{Training the VAE}}
%\qw{the input of VAE is composed by both the original data and the GNN output, i.e., move the above embedding part here}
The feature embeddings \( \mathbf{Z} \) generated by the GNN are then used to train a Variational Autoencoder (VAE). 
The VAE consists of an encoder and a decoder. The encoder maps the embeddings \( \mathbf{z} \) into a latent space, and the decoder reconstructs the embeddings back to their original feature space. This training is performed using the embeddings from the clean data, allowing the VAE to learn a probabilistic model of the normal data distribution.

\subsubsection{\textbf{Collecting the statistics of reconstruction errors}}
During training, we record the reconstruction error for each instance. The reconstruction error is essentially the loss for each instance. This results in a list of reconstruction errors, \(\mathcal{E}\). Given that even cleaned datasets may contain undetected errors, we do not set the maximum reconstruction error as the threshold for identifying problematic instances. Instead, we set the threshold at the 95th percentile of \(\mathcal{E}\), denoted as \( e_{clean} \). Instances with reconstruction errors above \( e_{clean} \) are flagged as potentially problematic.

%\qw{adjsut this paragraph}
%This process ensures that the VAE effectively learns the characteristics of the clean data while providing a robust method for detecting deviations from the norm in new datasets.


\subsection{Data Quality Validation Process}

\noindent{\textbf{Detecting Data Quality Issues by Reconstruction Errors}.}
With the GNN and VAE trained on the embeddings of the clean data, we proceed to assess the quality of new, unseen datasets. These unseen datasets must keep the same schema as the original clean dataset. The process involves several steps.
First, we generate embeddings for the new dataset using the trained GNN. Let \( \mathbf{Z}_{\text{new}} \) be the embeddings of the new dataset instances. These embeddings are then input into the trained VAE to obtain a list of reconstruction errors, denoted as \(\mathcal{E}_{\text{new}}\).
Next, we compare each reconstruction error in \(\mathcal{E}_{\text{new}}\) with the threshold \( e_{clean} \) from the clean dataset. 
We calculate the proportion of instances in the new dataset with reconstruction errors exceeding \( e_{clean} \), denoted as \( R_{error} \). 
Since the threshold was set at the 95th percentile for the clean dataset, we expect around 5\% of clean data instances to exceed this value. 

To account for data variability, if \( R_{error} \) exceeds \( 5\% \times n \), we classify the new dataset as problematic. This means if more than \( 5n\% \) of instances in the new dataset have errors greater than \( e_{clean} \), we will report the dataset has data quality issues. The parameter \( n \) can be adjusted based on observed reconstruction errors after deployment.
In our experiments, we set \( n = 1.2 \), which exhibited good performance.
Finally, we report the indices of all instances in the new dataset with reconstruction errors above \( e_{clean} \), clearly identifying problematic samples.

% Next, we compare each reconstruction error in \(\mathcal{E}_{\text{new}}\) with the previously determined threshold \( e_{cleaned} \). We calculate the proportion of instances in the new dataset that have reconstruction errors exceeding \( e_{cleaned} \), denoted as \( R_{error} \). Since the threshold was set at the 95th percentile for the clean dataset, it is expected that approximately 5\% of the clean dataset instances would have reconstruction errors above this threshold.
% To account for data variability, if \( R_{error} \) exceeds \( 5\% \times n = 5n\% \), we classify the new dataset as problematic. This means that if more than 5n\% of the instances in the new dataset have reconstruction errors greater than \( e_{cleaned} \), the dataset is flagged as having data quality issues.

% Finally, we report the indices of all instances in the new dataset that have reconstruction errors above \( e_{cleaned} \), providing a clear indication of which specific samples are problematic.

%This process allows us to effectively detect data quality issues in new datasets, capturing both explicit errors and subtle inconsistencies that traditional approaches may miss.


\noindent{\textbf{Detecting Feature Errors}.}
Each instance's reconstruction error \( e \) is a list corresponding to each feature's loss. To identify specific problematic features, we detect outliers with significantly higher reconstruction errors.
For an instance \( \mathbf{x}_i \), let \( \mathbf{e}_i = [e_{i1}, e_{i2}, \ldots, e_{in}] \) be the reconstruction errors for the \( n \) features. We calculate the mean \( \mu_i \) and standard deviation \( \sigma_i \) of the errors. Features with errors greater than \( \mu_i + 5\sigma_i \) are flagged as problematic.

%By reporting these outlier features, we can pinpoint which specific parts of an instance contribute most to data quality issues. 
This drill-down process helps identify exact feature-level problems within instances, facilitating targeted data cleaning.


%Our approach offers several key advantages over traditional data quality verification methods. By leveraging GNNs and VAEs, it automatically identifies data quality issues without predefined constraints and detects hidden relationships within the data. This reduces the need for continuous expert input, making the process more efficient and scalable. Additionally, it can pinpoint problematic samples and specific features, facilitating targeted data cleaning and correction.





\section{Definitions and basic properties}\label{sec:logics}

For $k\in \N$, we denote $[k]\coloneqq \{1,\ldots,k\}$.
All graphs are finite, simple (i.e. without loops or parallel edges), and undirected, unless explicitly stated. They are also possibly {\em{vertex-colored}}: the vertex set is equipped with a number of distinguished subsets -- colors.
We treat such graphs as relational structures consisting of the vertex set, the binary symmetric edge relation (denoted $E$), and a number of unary predicates signifying colors. Note that a vertex may belong to several different colors simultaneously, or to no color at~all.

For a set $A$ of vertices of $G$, we denote by $G[A]$ the subgraph of $G$ induced by $A$, i.e. the graph with vertex set $A$, edge set consisting of all edges of $G$ with both endpoints in $A$, and colors inherited naturally.
For a vertex $v$ of $G$ and a set of vertices $A \subseteq V(G)$, we denote by $E(v, S)$ the set of neighbors of $v$ in $S$, i.e. $E(v, S) = \setof{u \in S}{uv \in E(G)}$. The {\em{neighborhood}} of a vertex is $N(v)\coloneqq E(v,V(G))$.

We assume reader's familiarity with the first-order logic \fo and monadic second-order logic \mso. For convenience, in all formulas of all the considered logics, including \fo and its extensions, we allow free set variables. If $X$ is such a set variable and $x$ is a vertex variable, then we allow membership tests of the form~$x\in X$.
By $\tup x$ we denote a tuple of vertex variables and by $\wtup X$ we denote a tuple of set variables.
For a graph $G$ and a tuple of vertices $\tup x$, we denote by $G^{\tup x}$ the set of evaluations of $\tup x$ in $G$, i.e. functions from the variables in $\tup x$ to the vertices of $G$. Similarly, for a tuple of set variables $\wtup X$, we denote by $G^{\wtup X}$ the set of evaluations of $\wtup X$ in $G$.
For brevity we might identify tuples of vertices with respective evaluations.
We say that a logic $\Ll$ is an {\em{extension}} of \fo if $\Ll$ contains \fo as its fragment.

For a formula $\varphi(x, \wtup Y, \tup z)$ of a logic $\Ll$, a graph $G$, and evaluations $\wtup B$ of $\wtup Y$ and $\tup c$ of $\tup z$, we say that $\varphi(x, \wtup B, \tup c)$ {\em{defines}} a set $A \subseteq V(G)$ in $G$ if
\[
    A = \setof{v \in V(G)}{G \models \varphi(v, \wtup B, \tup c)}.
\]
We also denote by $\varphi(G, \wtup B, \tup c)$ the set defined by $\varphi(x, \wtup B, \tup c)$ in $G$.

\paragraph*{Low rank \mso.} We have already defined low rank \mso in \cref{sec:intro}. Let us make here a few simple remarks about the choices made in the definition.

For readers not familiar with the notions of rankwidth and of cutrank, measuring the complexity of a binary matrix by its rank over $\F_2$ may not be the most intuitive choice. Let us explain that the  selection of rank over $\F_2$ is immaterial, as any similar choice would lead to a logic with the same expressive~power.

For a $\{0,1\}$-matrix $M$ and a field $\F$, by $\rkk_\F(M)$ we denote the rank of $M$ over~$\F$. Further, let the {\em{diversity}} of $M$, denoted $\dv(M)$, be the number of different rows of $M$ plus the number of different columns of~$M$. We have the following simple algebraic~fact.

\begin{lemma}\label{lem:equiv-measures}
 Let $M$ be a matrix with entries in $\{0,1\}$ and $\F$ be a finite field. Then
 \[\rkk_\F(M)\leq \rkk_\Q(M)\leq \dv(M)/2\leq |\F|^{\rkk_\F(M)}.\]
\end{lemma}
\begin{proof}
 For the first inequality, every set of columns of $M$ that is dependent over $\Q$ is also dependent over~$\F$. For the second inequality, the rank of a matrix over any field is always bounded by the number different rows, as well as by the number of different columns. For the last inequality, if the rank of $M$ over $\F$ is $k$, then the columns of $M$ are contained in the span of a base of size $k$. This span has at most $|\F|^k$ different vectors, hence $M$ has at most $|\F|^k$ different columns. A symmetric argument shows that $M$ has at most $|\F|^k$ different rows as well.
\end{proof}

\cref{lem:equiv-measures} implies that for a $\{0,1\}$-matrix $M$, whether we measure the diversity of $M$, or its rank over~$\F_2$, or its rank over any other finite field $\F$, or its rank over $\Q$, all these measurements yield values that are bounded by functions of each other. Hence, if $M$ has one of those measures bounded, then all the other measures are bounded as well. Let us also observe that testing the value of any of the considered measures can be defined in \fo.

\begin{lemma}\label{lem:inter-def}
 For every $k\in \N$ and a finite field $\F$, there is an \fo formula $\varphi_{k,\F}(X)$ that for a graph $G$ and $X\subseteq V(G)$, tests whether $\rk_\F(\Adj_G[X,\wh X])\leq k$, where $\wh X=V(G)\setminus X$. Similarly, there are formulas $\varphi_{k,\Q}(X)$ and $\varphi_{k,\dv}(X)$ that test whether $\rk_\Q(\Adj_G[X,\wh X])\leq k$ and $\dv(\Adj_G[X,\wh X])\leq k$,~respectively.
\end{lemma}
\begin{proof}
 We first construct the formula $\varphi_\F(X)$. Call two vertices $u,u'\in X$ {\em{twins}} if $u$ and $u'$ have the same neighborhood in $\wh X$; equivalently, $u$ and $u'$ define equal rows in $M\coloneqq \Adj_G[X,\wh X]$. Similarly, $v,v'\in \wh X$ are twins if they have the same neighborhood in $X$, or equivalently they define equal columns of $M$. Let $C$ be an inclusionwise maximal subset of $X$ consisting of pairwise non-twins, and similarly let $D\subseteq \wh X$ be an inclusionwise maximal subset of $\wh X$ consisting of pairwise non-twins. Note that $|C|+|D|=\dv(M)\leq 2 \cdot |\F|^{\rkk_\F(M)}$ by \cref{lem:equiv-measures}. Therefore, formula $\varphi_\F(X)$ can be constructed by (i) existentially quantifying $C$ and $D$ as sets of total size at most $|\F|^k$, (ii) checking that $C$ and $D$ are inclusionwise maximal subsets of $X$ and of $\wh X$, respectively, consisting of pairwise non-twins, and (iii) verifying that the rank over $\F$ of the minor of $M$ induced by the rows of $C$ and the columns of $D$ is at most $k$, by making a disjunction over all possible adjacency relations between the vertices of $C$ and of $D$. Formula $\varphi_{k,\Q}(X)$ can be defined in the same way (here we have $|C|+|D|\leq 2\cdot 2^k$ by \cref{lem:equiv-measures} for $\F=\F_2$), while in formula $\varphi_{k,\dv}(X)$ we only need to make sure that $|C|+|D|\leq k$.
\end{proof}

From \cref{lem:equiv-measures,lem:inter-def} we conclude that in the definition of low rank \mso, regardless whether in set quantification we require providing an explicit bound on the rank of the bipartite adjacency matrix over~$\F_2$, or over any other fixed finite field $\F$, or over $\Q$, or even on the diversity of the adjacency matrix, all these logics will have the same expressive power. This is because if we have two measures $\mu_1,\mu_2$ among the above, to quantify over $X$ with $\mu_2(\Adj_G[X,\wh X])\leq k$, it suffices to quantify over $X$ with $\mu_1(\Adj_G[X,\wh X])\leq f(k)$, where $f\colon \N\to \N$ is such that $\mu_1(M)\leq f(\mu_2(M))$ for every $\{0,1\}$-matrix $M$, and verify that indeed $\mu_2(\Adj_G[X,\wh X])\leq k$ using a formula provided by \cref{lem:inter-def}. Therefore, following the literature on rankwidth we make the arbitrary choice of defining low rank \mso using the cutrank function that relies on ranks over $\F_2$. In the remainder of this paper, we denote $\rkk\coloneqq \rkk_{\F_2}$ for brevity. Also, the cutrank of a set will be called just {\em{rank}}.

\paragraph*{Separator logic, flip-connectivity logic, and flip-reachability logic.}
Separator logic has also been introduced in \cref{sec:intro}. Recall that it is defined as the extension of \fo on graphs by predicates $\conn_k$ for $k\in \N$, each of arity $k+2$, with the following semantics: if $G$ is a graph and $s,t,a_1,\ldots,a_k$ are vertices of $G$, then $\conn_k(s,t,a_1,\ldots,a_k)$ holds in $G$ if and only if there is a path with endpoints $s$ and $t$ that does not pass through any of the vertices $a_1,\ldots,a_k$.

We now define flip-connectivity logic and flip-reachability logic. Flip-reachability logic works naturally on directed graphs ({\em{digraphs}}) and flip-connectivity logic will be a special case of the definition, hence we need to introduce some terminology on digraphs.

A~directed graph (digraph) is a~pair $G=(V, E)$ consisting of a~set $V=V(G)$ of vertices and a~set $E=E(G)$ of \emph{arcs}.
An~arc from $u$ to $v$ is denoted $\vec{uv}$ and has tail $u$ and head $v$.
We specify that digraphs do not contain self-loops, so $u \neq v$ for each arc $\vec{uv}$, and neither do they contain multiple copies of an~arc.
However, we allow parallel arcs connecting two vertices in the opposite directions.

The \emph{atomic type} of a~$k$-tuple $\tup{v} = (v_1, \ldots, v_k)$ of vertices of an~undirected graph $G$, denoted $\atp(\tup{v}) = \atp(v_1, \ldots, v_k)$, is the set of all atomic formulas of the form $x_i = x_j$, $E(x_i, x_j)$, and $U(x_i)$ for some $i,j\in \{1,\ldots,k\}$ and a unary predicate $U$ in the language of $G$, satisfied by $\tup{v}$ in $G$.
Naturally, two $p$-tuples $\tup{u}$, $\tup{v}$ of vertices satisfy the same quantifier-free formulas of first-order logic with no parameters if and only if $\atp(\tup{u}) = \atp(\tup{v})$.
We define \emph{edge type} of a~$k$-tuple $\tup v$ similarly to the atomic type, but we do not consider the unary predicates.

Let $\atp^k$ denote the set of all possible atomic types of $k$-tuples of vertices of an~undirected graph. Note that $\atp^k$ is finite and of size bounded by a function of $k$ and the number of unary predicates interpreted in $G$. Atomic types could be also naturally defined for directed graphs, but we will use them only in the undirected context.

Next, we define flips of (undirected) graphs with respect to a~tuple of parameters.
Let $k \in \N$, $A \subseteq \atp^{k+1} \times \atp^{k+1}$, $G$ be an~undirected graph and $\tup{a} = (a_1, \ldots, a_k)$ be a~$k$-tuple of vertices of $G$ -- the \emph{parameters} of the flip.
Then the \emph{$A$-flip of $G$ with parameters $\tup{a}$}, denoted $G \oplus_{\tup{a}} A$, is the directed graph with the same vertex set as~$G$, where for distinct $u, v \in V(G)$, we have
\[
    \vec{uv} \in E(G \oplus_{\tup{a}} A)\qquad\textrm{if and only if}\qquad[uv \in E(G)]\ \textrm{xor}\ [(\atp(u, \tup{a}), \atp(v, \tup{a})) \in A].
\]
Note that if the relation $A$ is symmetric, then the resulting graph is always undirected, in the sense that every arc $\vec{uv}$ is accompanied with the opposite arc $\vec{vu}$.
In this case, we will say that the $A$-flip is \emph{symmetric} and consider its result to be an undirected graph.

This definition now allows us to formally introduce the flip-reachability logic and its weaker flip-connectivity counterpart.

\begin{definition}[Flip-reachability logic]
    For $k \in \N$ and $A \subseteq \atp^{k+1} \times \atp^{k+1}$, we introduce the flip-reachability predicate
    \[
        \flipreach_{k, A}(s, t, a_1, \ldots, a_k)
    \]
    as the relation on vertices in a~graph that holds precisely when there exists a~directed path from $s$ to $t$ in $G \oplus_{\tup{a}} A$.
    Flip-reachability logic is first-order logic over graphs in which the universe is formed by the vertices of a~graph, and the available relations are the binary edge relation, the unary predicates in the language, as well as the flip-reachability predicates.
\end{definition}

\begin{definition}[Flip-connectivity logic]
    For $k \in \N$ and symmetric $A \subseteq \atp^{k+1} \times \atp^{k+1}$, we introduce the flip-connectivity predicate
    \[
        \flipconn_{k, A}(s, t, a_1, \ldots, a_k)
    \]
    as the relation on vertices in a~graph that holds precisely when $s$ and $t$ are in the same connected component of $G \oplus_{\tup{a}} A$.
    Flip-connectivity logic is first-order logic over graphs in which the universe is formed by the vertices of a~graph, and the available relations are the binary edge relation, the unary predicates in the language, as well as the flip-connectivity predicates.
\end{definition}

\paragraph*{Easy comparisons.} Let us first establish the straightforward relations between the considered logics in terms of expressive power. We first note the following.

\begin{lemma}\label{lem:freach-in-lrmso}
    Every flip-reachability predicate can be expressed in low rank \mso.
\end{lemma}
\begin{proof}
    Fix $k\in \N$, $A \subseteq \atp^{k+1} \times \atp^{k+1}$, a graph $G$, and vertices $s,t,a_1,\ldots,a_k$; denote $\tup a=(a_1,\ldots,a_k)$.
    Observe that the predicate $\flipreach_{k, A}(s, t, a_1, \ldots, a_k)$ is false if and only if there is a set $X$ of vertices of $G$ such that $s\in X$, $t\notin X$, and there are no arcs from $X$ to $\wh X$ in $G \oplus_{\tup{a}} A$, where we denote $\wh X=V(G)\setminus X$.
    Note that these conditions, for a given set $X$, can be encoded by a first-order formula taking $X,s,t,a_1,\ldots,a_k$ as free variables.
    So it remains to show that quantification over such sets $X$ can be done using a low-rank quantifier, that is, that $X$ has rank bounded by a function of~$k$.

    For this, we argue that the set $X$ of vertices of $G$ that are reachable from $s$ in $G \oplus_{\tup{a}} A$ has rank at most $|\atp^{k+1}|$.
    To see this, consider two vertices $u,u'\in X$ such that $\atp(u, \tup{a}) = \atp(u', \tup{a})$ (in $G$), as well as a vertex $v\in \wh X$. Note that $v$ is adjacent in $G$ to either both $u$ and $u'$ or to none of them, for otherwise either $\vec{uv}$ or $\vec{u'v}$ would appear as an arc in $G \oplus_{\tup{a}} A$, which would contradict the assumption that $v$ is not reachable from $s$.
    Hence, whether there is an edge in $G$ between a vertex $u$ in $X$ and a vertex $v$ in $\wh X$ depends only on the atomic type $\atp(u,\tup{a})$. It follows that the adjacency matrix $\Adj_G[X, \wh X]$ has at most $|\atp^{k+1}|$ distinct rows, hence its rank is at most $|\atp^{k+1}|$.
\end{proof}

We conclude the following.

\begin{proposition}\label{lem:easy-comparison}
 The following holds:
 \begin{itemize}[nosep]
  \item For every formula of separator logic there is an equivalent formula of flip-connectivity logic.
  \item For every formula of flip-connectivity logic there is an equivalent formula of flip-reachability logic.
  \item For every formula of flip-reachability logic there is an equivalent formula of low rank \mso.
 \end{itemize}
\end{proposition}
\begin{proof}
 For the first point, it suffices to observe that there is a~symmetric $A\subseteq \atp^{k+1} \times \atp^{k+1}$ such that for every graph $G$ and $\tup a\in V(G)^k$, the flip $G\oplus_{\tup a} A$ is equal to $G$ with all edges incident to the vertices of $\tup a$ removed; then the predicate $\conn_k(s,t,a_1,\ldots,a_k)$ is equivalent to $\flipconn_{k,A}(s,t,a_1,\ldots,a_k)$. The second point is obvious: flip-connectivity predicates are special cases of flip-reachability predicates obtained by restricting attention to symmetric relations $A$. The third point follows immediately from \cref{lem:freach-in-lrmso}.
\end{proof}

Finally, we note that flip-connectivity logic has a strictly larger expressive power than separator logic. A distinguishing property is {\em{co-connectivity}}: connectivity of the complement of the graph.

\begin{proposition}
 There is a sentence of flip-connectivity logic that verifies whether a graph is co-connected. However, there is no such sentence in separator logic.
\end{proposition}
\begin{proof}
 For the sentence of flip-connectivity logic verifying co-connectivity, we can take \[\forall s\,\forall t\, \flipconn_{0,A}(s,t),\qquad\textrm{where }A=\atp^1\times \atp^1.\]
 We are left with proving that the property cannot be expressed in separator logic.

 For the sake of contradiction, suppose there is a sentence $\varphi$ of separator logic that holds exactly in graphs whose complements are connected. Since $\varphi$ is finite, there is some number $k\in \N$ such that all connectivity predicates present in $\varphi$ are of arity at most $k+2$. Now, for every integer $n>k+2$, consider the following two graphs: $G_n$ is the complement of a cycle on $n$ vertices, and $H_n$ is the complement of the disjoint union of two cycles on $n$ vertices. Note that $G_n$ is co-connected while $H_n$ is not, hence $\varphi$ holds in $G_n$ and does not hold in $H_n$. Further, it can be easily verified that both $G_n$ and $H_n$ are $k$-connected, that is, no pair of vertices can be disconnected by a $k$-tuple of other vertices. Hence, every predicate $\conn_\ell(s,t,a_1,\ldots,a_\ell)$, for $\ell\leq k$, is equivalent on $\{G_n,H_n\colon n>k+2\}$ to the first-order formula
 \[\bigwedge_{i\in [\ell]} (s\neq a_i \wedge t\neq a_i).\]
 By replacing all the connectivity predicates with the formulas above, we turn $\varphi$ into a first-order sentence $\varphi'$ such that for all $n>k+2$, $\varphi'$ holds in $G_n$ and does not hold in $H_n$. However, a standard argument based on Ehrenfeucht--Fra\"isse games for first-order logic shows that such a sentence $\varphi'$ distinguishing $G_n$ from $H_n$ does not exist.
\end{proof}

\input{./macros/ht}
%%%%%%%%%%%%%%%%%%%%%%%%%%%%%%%%%%%%%%%%%%%%%%%%%%%%%%%
%%%%%%%%%%%%%%%    theorems %%%%%%%%%%%%%%%%%%%%%%%%%%%
%%%%%%%%%%%%%%%%%%%%%%%%%%%%%%%%%%%%%%%%%%%%%%%%%%%%%%%
% \usepackage{mdframed}
\usepackage{kantlipsum}

%%%%%%%%%%%%%%%%%%%%%%%%%%%%%%%%%%%%%%%%%%%%%%%%%%%%%%%
%%%%%%%%%%%%%%%    theorems %%%%%%%%%%%%%%%%%%%%%%%%%%%
%%%%%%%%%%%%%%%%%%%%%%%%%%%%%%%%%%%%%%%%%%%%%%%%%%%%%%%
\theoremstyle{plain}
\newtheorem{theorem}{Theorem}[section]
\newtheorem{proposition}[theorem]{Proposition}
\newtheorem{lemma}[theorem]{Lemma}
\newtheorem{example}[theorem]{Example}
\newtheorem{corollary}[theorem]{Corollary}
\theoremstyle{definition}
\newtheorem{definition}[theorem]{Definition}
\newtheorem{assumption}[theorem]{Assumption}
\theoremstyle{remark}
\newtheorem{remark}[theorem]{Remark}


% \titleformat{\subsection}[runin]% runin puts it in the same paragraph
%        {\normalfont\bfseries}% formatting commands to apply to the whole heading
%        {\thesubsection}% the label and number
%        {0.5em}% space between label/number and subsection title
%        {}% formatting commands applied just to subsection title
%        [.]% punctuation or other commands following subsection title


%%%%%%%%%%%%%%%%%%%%%%%%%%%%%%%%%%%%%%%%%%%%%%%%%%%%%%%
%%%%%%%%%%%%%%%  mathematical notations%%%%%%%%%%%%%%%%
% \usepackage[english]{babel}
% \usepackage[utf8]{inputenc}
% \usepackage[T1]{fontenc}

%% Figures, tables and lists
\usepackage[dvipsnames]{xcolor}
\usepackage{paralist}
\usepackage{graphicx}
\usepackage{subcaption}
\usepackage{longtable} 
\usepackage{multirow}
\usepackage{listings}
\usepackage{makecell}
\usepackage{array}
\usepackage{float}
\usepackage{dsfont}
\usepackage{rotating}
\usepackage{booktabs}
\usepackage{enumerate}
\usepackage{tikz}
\usepackage{pgf}
\usepackage{enumitem}
\usepackage{lipsum} % for generating filler text
\usepackage{titlesec}

%% Math
% \usepackage{amssymb, amsthm,bbm}
\usepackage{mathtools}
\usepackage{mathrsfs}
%% References and author info 
\mathtoolsset{showonlyrefs}
\usepackage{natbib}
\usepackage{authblk}
\usepackage{todonotes}
\usepackage{xr-hyper}


%%%%%%%%%%%%%%%%%%%%%%%%%%%%%%%%%%%%%%%%%%%%%%%%%%%%%%%
\newcommand{\R}{\mathbb R}
\newcommand{\EE}{\mathbb{E}}

\DeclareMathOperator{\Tr}{Tr}
\DeclareMathOperator*{\argmin}{argmin}
\DeclareMathOperator*{\argmax}{argmax}

\newcommand{\bs}[1]{\ensuremath{\boldsymbol{#1}}}
\newcommand{\mc}{\mathcal}
\newcommand{\opt}{^\star}


\newcommand{\diff}{\textnormal{d}}


\def \iid {\stackrel{\textnormal{i.i.d.}}{\sim}}
\def \iidtext {\textnormal{i.i.d.}}





%%%%%%%%%%%%%%%%%%%%%%%%%%%%%%%%%%%%%%%%%%%%%%%%%%%%%%%
%%%%%%%%%%%%%%%%%%%%% colors     %%%%%%%%%%%%%%%%%%%%%%
%%%%%%%%%%%%%%%%%%%%%%%%%%%%%%%%%%%%%%%%%%%%%%%%%%%%%%%
\definecolor{myblue}{rgb}{.8, .8, 1}
\definecolor{mathblue}{rgb}{0.2472, 0.24, 0.6} % mathematica's Color[1, 1--3]
\definecolor{mathred}{rgb}{0.6, 0.24, 0.442893}
\definecolor{mathyellow}{rgb}{0.6, 0.547014, 0.24}


% May add more in future.







\title{Hybrid Answer Set Programming: Foundations and Applications}

\author{%
  Nicolas R\"uhling \\
  \institute{University of Potsdam, Germany \\ \textit{Institute of Computer Science} \\ \textit{An der Bahn 2, 14476 Potsdam}}
  \email{nruehling@uni-potsdam.de}
}

\def\titlerunning{Hybrid ASP: Foundations and Applications}
\def\authorrunning{N. R\"uhling}

\hypersetup{
  bookmarksnumbered,
  pdftitle    = {\titlerunning},
  pdfauthor   = {\authorrunning},
  pdfsubject  = {ICLP/LPNMR-DC 2024},               % Consider adding a more appropriate subject or description
  pdfkeywords = {Answer Set Programming, Logic of Here-and-there, Configuration} % Uncomment and enter keywords specific to your paper
}

\begin{document}
\maketitle

\documentclass[../main.tex]{subfiles}
\graphicspath{{../images/}}
\makeatletter
\def\input@path{{../images/}}
\makeatother
\begin{document}
\section{Introduction}
\begin{figure}
\centering
\begin{tikzpicture}
\node[inner sep=0pt] (ws) at (0, 0) {
\includegraphics[height=.4\textwidth, trim={10cm 0 10cm 0},clip]{world_space.png}};
\node[inner sep=0pt] (cs) at (6,0) {\includegraphics[height=.4\textwidth, trim={10cm 1cm 10cm 4cm},clip]{conf_space.png}};
\end{tikzpicture}
\vspace{-5pt}
\label{fig:pbrm_intro}
\caption{\textbf{Left}: Shows world space obstacles as grey spheres. Robots start and goal configuration is colored red and green, respectively. Configurations along the computed path are colored transparent blue. \textbf{Right:} Mapped world space scenario to configuration space. Obstacle region is the grey mesh. Red spheres are collision-free regions computed by the neural SCDF. The optimized shortest path in the convex corridor is the blue curve.}
\vspace{-25pt}
\end{figure}
Motion planning is the problem of finding a collision-free trajectory that connects a given start and goal configuration. The planning takes place in the configuration space of the robot. For single body robots, like mobile robots or drones, the configuration space and the world space are usually the same. This simplifies the planning, since explicit obstacle representations are available which enables geometrical tools like separating hyperplanes, smallest distance to obstacles etc., to be used when designing motion planning algorithms. For multi-body robots like manipulators, the situation is completely different. The world space obstacles are usually mapped to non-convex regions, and to make the problem even harder, the mapping is usually not known. Forming explicit representations of the obstacle region in the configuration space is usually too expensive or intractable. Despite all of this, sampling based planners are used with great success, which mainly is due to their use of implicit representations of the obstacle region. The basic idea is to construct a graph in the configuration space that covers and connects the collision-free region. From this graph, a path can be extracted that connects a given start and goal configuration. The approach is computationally expensive, since the graph is constructed with the smallest geometrical building block available, points, which represents a collision-check. Furthermore, the extracted paths from the graph are non-smooth and jagged due to the stochastic nature of the approach. This adds an additional post-processing step to the process, where the paths are shortcutted and smoothened, before the path can be used for tracking. Clearly a lot of time is invested to form this graph and produce smooth paths. Thus, if the obstacles start to move, then all of this work is done in no use, since all points that make up this graph need to be re-verified, which is simply too time consuming to be done in real time.
\\\\
In this work, we want to address the existing drawbacks of the sampling based planners. Our main contribution is an improved motion planner where each vertex in the graph covers a collision-free region in the form of a sphere instead of a point and where the edges are formed with neighboring intersecting spheres. This representation has the advantage of instead of returning piecewise linear paths, returning a sequence of overlapping spheres, i.e. a convex corridor, that connects a given start and goal configuration, illustrated in Figure \ref{fig:pbrm_intro}. This convex corridor allows us to use convex optimization to produce smooth trajectories, instead of computationally expensive post-processing methods. The representation further allows us to estimate the coverage of the collision-free space, which gives us awareness and feedback in the offline roadmap construction phase. Finally, our representation is simple to adapt to moving obstacles, simply requery for the new radii and recheck for intersections. 
\\\\
The spherical collision-free regions are formed using a signed distance function (SDF), which is a function that returns the smallest distance from an arbitrary point to the boundary of an obstacle. As the name implies, the distance is signed, thus if the point is inside the obstacle it is negative otherwise positive. If the distance is positive, a sphere with radius equal to the distance is guaranteed to cover a collision-free region. Using an SDF in motion planning is not new, but what is novel about our approach is that we express the distance in the configuration space instead of the world space and by doing so allows us to form these convex collision-free regions. We refer to the resulting SDF as a signed configuration distance function (SCDF). Computing an SCDF analytically is non-trivial, our approach is therefore to parameterize the SCDF with a deep neural network and learn the mapping by supervised learning. Our resulting neural SCDF can compute distances for different parameter values of obstacle shapes and we also show how multiple distances can be combined, thus making our approach flexible.
\section{Related work}
Motion planning algorithms can roughly be divided into three families, grid-based, sampling based and optimization based methods. Grid-based methods (GBM) discretize the planning space from which a graph is then compiled. A standard search method is A$^\star$ \citep{a_star}, which is classified as an \textit{informed} search method, since it employs a heuristic function to speed up the search. A$^\star$ guarantees to return an optimal path at the level of discretization used. GBMs usually discretize the planning space by a regular lattice and this limits the GBMs to problems with low dimensionality due to the curse of dimensionality. Thus, GBMs are usually limited to single-body robots where the degrees of freedom (DOF) are low. To overcome the inherent scaling problem with the GBMs, stochastic methods are usually used for multi-body robots. These methods are termed as sampling-based methods (SBM) and core members within this family are the rapidly-exploring random trees (RRT) \citep{rrt} and the probabilistic roadmap (PRM) \citep{prm}. RRT grows a tree from the start configuration and explores the collision-free region in a rapid way until it is able to connect to the goal region. RRT is usually improved by bi-directional planning \citep{rrt_connect}, i.e. an additional tree is grown from the goal configuration and the trees are tested for connection after any tree has been expanded. RRT is a single-query method, thus it searches for a path from scratch each time it is queried. Contrary to this, PRM is a multi-query method, which solves for multiple queries without starting from scratch. PRM does this by creating a roadmap (graph) that covers the collision-free space as an offline step. The graph is then used to solve for multiple queries. PRMs are used in cases where the environment does not change since the extra offline step is too computationally costly and needs to be re-done if the environment is changed. In our work, we address this inherent issue by using a different roadmap representation. Our vertices in the graph cover a collision-free region in the form of spheres and we form the edges by checking for intersecting spheres. If something in the environment changes, we recompute the spheres radii and recheck the intersections, without relying on collision detection. We use a trained neural network to compute the sphere radius, therefore querying for the radius can be done fast, hence our representation enables the PRM for dynamic environments.
\\\\
In the recent decades, optimization based methods (OBM) \citep{chomp, schulman, itomp, stomp} have been introduced as an alternative to SBM for multi-body robots. Like the SBM, the OBMs scale well to higher dimensional problems and produce smoother motion. It is common to use a SDF in the optimization since it is a smooth function, thus enabling gradient-based methods. However, the standard way of expressing the SDF is in world space. The distance therefore needs to be mapped to the configuration space by the forward kinematics. This mapping makes the optimization problem a non-linear program (NLP), which is computationally expensive to solve. Recently, a different approach has been proposed. In \cite{mp_gcs} motion planning is formulated as a convex optimization problem by using the graph of convex sets framework \citep{gcs}. The underlying idea is to decompose the collision-free space into intersecting convex sets from which a convex optimization problem is formulated. In cases where an explicit representation of the obstacles in the configuration space exists, like for single-body robots, creating collision-free convex regions can be done fast \citep{iris}. For multi-body robots, this is non-trivial. Existing work does this successfully \citep{iris_nlp, iris_c} by an optimization based approach, but the methods are still too time consuming to be used in the presence of moving obstacles. Our approach is instead to use deep learning to learn an SDF expressed in the configuration space. With this, we can query for shortest distances to the collision boundary, which allows us to expand spherical regions which are collision-free. Our approach is fast and therefore enables our suggested roadmap planner to be used in dynamic environments.
\\\\
Recent research has focused on learning collision detection \citep{fk_kernel_distance, diffco, graphdistnet} by predicting the signed distance between the robot links and the surrounding obstacles in the world space. The learned SDF is used in trajectory optimization but since the distance is expressed in the world space, the problem becomes an NLP and therefore takes a long time to solve. We take a novel approach and suggest to instead express the signed distance in the configuration space. This allows us to improve the PRM at the same time as it enables convex optimization for trajectory optimization, which runs faster and is more reliable than NLP solvers. In \cite{cspf} a learned signed distance function in the configuration space is proposed similar to our approach. However, their approach is restricted to point cloud representations, while we propose to represent the obstacles as parameterized geometric shapes, e.g. spheres. Furthermore, we also show how to use our learned SCDF to improve an existing roadmap planner.
\section{Problem formulation}
A robot is located in the world space, $\W \subset \R^3 $. The unique location of the robot is given by its configuration $\q \in \C$, where $\C$ is the configuration space. The set of points covered by the robots bodies at a certain configuration is expressed as $\B(\q) \subset \W$. The robot is surrounded by $\NrObst$ obstacles $\O = \bigcup_{i=1}^{\NrObst} \O_i$, where  $\O_i \subset \W$. The representation of the obstacle in the configuration space is the set $\C\O_i = \{\q \in \C \: |\: \B(\q) \cap \O_i \neq \emptyset \}$. The obstacle space is formed as $\Co = \bigcup_{i=1}^{\NrObst} \C \O_i$. The complement is referred to as the free space, $\Cf = \C \setminus \Co$. The path planning problem is a tuple, ($\Cf$, $\qStart$, $\qGoal$), where we want to connect a query pair, consisting of a start, $\qStart$, and goal configuration, $\qGoal$, with a geometric path, $\q(s): [0, 1] \mapsto \Cf$, such that $\q(0)=\qStart$ and $\q(1)=\qGoal$, or report correctly when such a path does not exist.
\end{document}

\section{Basic Background: Supervised Learning and the PAC Model}
\label{sec:background}

At this point almost everyone has heard of machine learning (ML). Anyone likely to stumble upon this article will have also heard of its most influential special case, supervised learning, and those theoretically inclined will also be familiar with the PAC model. Nonetheless, I will set the stage by  recapping the basics.

\subsection{Basics of Supervised Learning}%Let's set the stage in any case

\emph{Supervised Learning} is the task of ``coming up'' with a function $f: \X \to \Y$ to ``explain'' or ``fit'' a sequence of input/output examples   $(x_1,y_1), \ldots, (x_n,y_n)$, with $x_i \in \X$ and $y_i \in \Y$.  Here $\X$ is a \emph{data domain} consisting of \emph{datapoints} $x \in \X$, $\Y$ is a \emph{label set} consisting of \emph{labels} $y \in \Y$, and the sequence $(x_1,y_1),\ldots,(x_n,y_n)$ is the \emph{training data} consisting of \emph{labeled examples (a.k.a. samples)}~$(x_i,y_i)$.  I~will refer to the chosen function $f$ as a \emph{predictor}, and to $n$ as the \emph{sample size}. A \emph{learning algorithm} takes as input training data, and outputs (some representation of) a predictor $f \in \Y^\X$.\footnote{Note that this describes the usual \emph{batch}, a.k.a.~\emph{offline}, setting of supervised learning. I do not discuss other paradigms such as online or active learning in this article.} 



Success in supervised learning is defined as \emph{generalization} to  future examples: For a typical \emph{test example}  $(x_{\tst},y_{\tst})$, the predicted label $y'_{\tst}=f(x_{\tst})$ should ``equal'' $y_{\tst}$, perhaps approximately. We usually assume the test example is drawn from the same  ``source'' as the training data  --- commonly, i.i.d.~from the same distribution. The quality of the prediction is quantified by $\ell(y'_{\tst},y_{\tst})$, where $\ell:~\Y~\times~\Y \to \RR_{\geq 0}$ is a \emph{loss function} chosen as part of the problem definition. Common loss functions include the 0-1 loss $\ell_{0-1}(y',y) = [y' \neq y]$ for \emph{classification} problems,\footnote{The notation $[P]$ denotes $1$ when predicate $P$ is true, and denotes $0$ when $P$ is false.} as well as the absolute loss $|y'-y|$ or squared loss $(y'-y)^2$ for \emph{regression problems} featuring $\Y  \sse \RR$.

Nontrivial generalization properties are typically only possible if one assumes something about the data.\footnote{The need for such an assumption is formalized by the  \emph{no free lunch theorems} of supervised learning \cite{wolpert_connection_1992,wolpert_lack_1996,schaffer_conservation_1994}.} The Bayesian approach to  machine learning, common in many applications, assumes some parametric form for the distribution generating the data, and postulates a prior on the parameters. This is not the approach I will take in this article. Instead, I will focus on the frequentist --- and some would say ``worst-case'' or ``adversarial'' ---  approach that is common in the computational learning theory community, embodied by the PAC model. Here we assume that the (training and test) data can be explained, perhaps approximately, by a function in some ``simple enough to learn'' class of functions $\H \sse \Y^\X$, often called the \emph{hypotheses}. Equivalently, we  seek a predictor which explains the unseen data roughly  as well as the best hypothesis $h^* \in \H$, whether or not we assume that $h^*$ itself provides a perfect explanation.



 \paragraph{Common Algorithmic Templates.} Perhaps the best known general-purpose supervised learning algorithm is \emph{empirical risk minimization (ERM)}, which chooses as its predictor a hypothesis $f \in \H$ minimizing $\frac{1}{n} \sum_{i=1}^n \ell(f(x_i),y_i)$ --- a quantity called the \emph{training error}, \emph{empirical error}, or \emph{empirical risk} of $f$. %\footnote{When multiple hypotheses minimize the empirical risk, we assume ERM breaks ties arbitrarily.}
A common template for generalizing ERM involves adding a \emph{regularization term} $\psi(f)$ to the  objective function, typically chosen to measure some notion of ``hypothesis complexity.'' An algorithm instantiating this template is known as a \emph{structural risk minimizer (SRM)}, and chooses as its predictor the hypothesis $f \in \H$ minimizing the \emph{structural risk} $\frac{1}{n} \sum_{i=1}^n \ell(f(x_i),y_i) + \psi(f)$. Other well-known algorithms, such as gradient descent and its variations,  can frequently be interpreted as approximate implementations of ERM or SRM.


\paragraph{Proper vs Improper Learning.} A learning algorithm is said to be \emph{proper} if its predictor $f$ is always chosen from the hypothesis class, i.e., $f \in \H$, otherwise it is said to be \emph{improper}. ERM  is an example of a proper learning algorithm, as are SRM algorithms of the form described above.  In the \emph{proper regime} of learning, algorithms are required to be proper. This article will be concerned with the more flexible \emph{improper regime} (a.k.a \emph{representation-independent learning}), where no such constraint is placed on the learner. In other words, all we care about is predictive power at test time, rather than any insights derived from the functional form or representation of the predictor~itself.


\subsection{The PAC Model}
A standard mathematical setup for evaluation of supervised learning algorithms, at least in the theoretical computer science community, is Valiant's \emph{Probably Approximately Correct (PAC) model} of learning (see e.g.~\cite{kearns_introduction_1994,mohri_foundations_2018}). Here, we assume there is an unknown distribution $\D$ on $\X \times \Y$ from which training and test data are  drawn.  Specifically, the labeled datapoints of the training set  $(x_1,y_1), \ldots, (x_n,y_n)$, as well as the test data  $(x_\tst,y_\tst)$, are i.i.d.~from $\D$. Often it is assumed that $\D$ lies in some class of distributions of interest. The \emph{true expected loss}, or simply \emph{loss}, of a predictor $f: \X \to \Y$ is the expected loss it incurs on draws from $\D$, written $L_\D(f) = \Ex_{(x,y) \sim \D} \ell(f(x),y)$.


There are two main ``settings'' in PAC learning. The  \emph{realizable setting} only requires that the data be perfectly explained by some hypothesis in $\H$. More generally, the \emph{agnostic setting} makes no assumption relating the data to the hypotheses, but shifts the goalposts as necessary to allow nontrivial guarantees: the expected loss at test time is evaluated only ``relative'' to that of the best hypothesis $h^* \in \H$. There are other settings which make more nuanced assumptions, such as $\D$ being of a particular parametric form or its support living in some (unknown) lower-dimensional space, etc. I will mostly discuss the realizable and agnostic settings in this article, those being the simplest and most studied from a theoretical perspective. %TODO:We will briefly discuss other settings in Section ??

The PAC model demands high probability guarantees of learners, in the worst case over distributions of interest. Consider first the realizable setting, where $\D$ is such that $\min_{h \in \H} L_{\D}(h) = 0$. A PAC learner has \emph{error} $\epsilon=\epsilon(n)$ and \emph{confidence} $\delta=\delta(n)$ if, when training data consists of $n$ i.i.d~samples from a realizable distribution $\D$, it produces a predictor $f$  satisfying $L_\D(f) \leq \epsilon$ with probability at least $1-\delta$. In the agnostic setting, where $\D$ can be arbitrary, we require $L_\D(f) - \min_{h \in \H} L_\D(h) \leq \epsilon$ with probability $1-\delta$.

In both the realizable and agnostic settings, we look for PAC learners with small $\epsilon$ and $\delta$ as a function of the sample size $n$. An equivalent perspective looks at the sample complexity $m(\epsilon,\delta)$, which is the minimum sample size which guarantees error  at most $\epsilon$ with probability at least $1-\delta$. We say a problem is \emph{PAC learnable} if its PAC sample complexity is finite whenever $\epsilon,\delta > 0$.

For most PAC learning problems, learnability and sample complexity are characterized in terms of a  ``dimension'' of the hypothesis class. Most prominently this is the \emph{VC dimension} for binary classification, the \emph{fat shattering dimension} for agnostic regression, and the \emph{DS dimension} for multiclass classification (see \cite{anthony_neural_1999,daniely_optimal_2014,brukhim_characterization_2022}). Treatment of these is beyond the scope of this article. The unfamiliar reader need not worry, however,  as dimensions will feature only tangentially in our~discussion.




%\paragraph{Learning settings: Realizable, Agnostic, etc.} In learning theory, evaluating a supervised learning algorithm requires specifying a data model and an objective. We will leave the details of the data model flexible for now, to allow for both the PAC model and the adversarial transductive model. Nonetheless we will describe two variations, which we call ``settings'', which cut across different models. The  \emph{realizable setting}  requires only that the data be perfectly explained by some hypothesis $h \in \H$ --- i.e., there exists a hypothesis which is guaranteed to suffer a loss of $0$ on training and test data. The performance of the learning algorithm is its expected loss at test time for some ``worst case'' realizable instance. More generally, the \emph{agnostic setting} makes no assumption relating the data to the hypotheses, but shifts the goalposts as necessary to allow nontrivial guarantees: the expected loss at test time is evaluated only ``relative'' to that of the best hypothesis $h^* \in \H$, again for some ``worst case'' instance. There are other settings which make more nuanced assumptions about the data, such as it is drawn from a distribution of a particular parametric form, or that it lives in some (unknown) lower-dimensional space, etc. We will mostly discuss the realizable and agnostic settings, those being the simplest and most studied from a theoretical perspective.




%%% Local Variables:
%%% mode: latex
%%% TeX-master: "learning_matching"
%%% End:


%#############################################################################################    
\section{The research Life Cycle}
\label{section:research}
%#############################################################################################

Experimentally-driven research should be grounded on a solid methodology that is understood and implemented by other disciplines. This is somehow the ambition of the European EOSC initiative. As a consequence, SLICES does not target only the deployment of the instrument/facility but as importantly, addresses the full research life-cycle, including open data, data management and reproducibility.

Researchers and research stakeholders nowadays require that research data is made available for other researchers to examine, experiment and develop further. Additionally, preserving the data in conjunction with how conclusions from the data were drawn, accelerates the discovery process, enable easier reproducibility of the results and thus supports evidence. It is then necessary to develop policies and procedures for regulating the management and publication of research data in order to make them interoperable and widely available.

In Europe, it is recommended to conform with the European Open Science~\cite{euos} and Open Access policy \cite{euoa}, Open Research Data Pilot \cite{eu_data_pilot} and FAIR \cite{fair} principles in producing and managing research data. This requires defining appropriate metadata (including compatible experiment description) on the data produced by or integrated into the infrastructure with the objective to ensure eventually data accessibility, trustworthiness, reusability and interoperability with data produced by similar infrastructures/experiments for enabling complex experiments and multi-domain research. 
Alignments with the relevant recommendations such as the ones published by EOSC FAIRsFAIR \cite{fairsfair} project, GO FAIR initiative \cite{gofair} and RDA for FAIR data management \cite{rda2020fair}, and general European Open Access to research publications and Open Research Data Pilot policies, are of utmost importance.

The FAIR (Findable, Accessible, Interoperable, and Reusable) \cite{fair-principles} Data Principles were developed to be used as guidelines for data producers and publishers, with regards to data management and stewardship. One important aspect that differentiates FAIR from any other related initiatives is that they move beyond the traditional data and they place specific emphasis on automatic computation, thus considering both human-driven and machine-driven data activities. Since their publication, FAIR principles became widely accepted and used.
To this end, SLICES fully endorses and adopts the FAIR principles, acting as a catalyst to enable and foster the data-driven science and scientific data-sharing in this area.


Understanding the data collected and processed within SLICES becomes essential to understand data usage from the target user groups. This should allow to develop an appropriate information model that represents the data collected from the SLICES testbeds, experimental equipment and applications. We consider that the datasets generated by the usage of the SLICES hardware and software infrastructure can be roughly organized into five main categories:

\begin{itemize}
    \item[-] \textbf{Observational Data:}  collected using methods such as surveys (e.g. online questionnaires) or recording of measurements (e.g. through sensors). The data include mostly data related to signal or performance measurements, and network or service log data that allow for experiment evaluation and reproducibility. 
    \item[-] \textbf{Experimental Data: } where researchers introduce an intervention and study the effects of certain variables, trying to determine their impact.
    \item[-] \textbf{Simulation Data: } is generated by using computer models that simulate the operation of a real-world process or system. These may use observational data.
    \item[-] \textbf{Derived Data: } involves the analysis (e.g. cleaning, transformation, summarization, predictive modeling) of existing data, often coming from different datasets (e.g. the results of two experiments), to create a new dataset for a specific purpose. 
    \item[-] \textbf{Metadata:  } concerns data that provides descriptors about all categories of data mentioned above. This information is essential in making the discovery of data easier and ensuring their interoperability.
\end{itemize}

SLICES, as an open platform, promotes interoperability, thus non-proprietary, unencrypted, uncompressed, and commonly used by the research community formats should be adopted. In addition, SLICES end users should have the ability to decide on a suitable license and attach it to their data. 

Our preliminary estimations for SLICES include up to 5,000 users and their data, accounting for up to 50GB per user on the individual nodes and up to 1TB on the cloud. This provides us with a preliminary estimation of 0.25PB-1PB of data storage for all datacenters residing on SLICES nodes, and 5PB for the cloud-based datacenter.

As a consequence, SLICES will setup a data management framework to support the efficient and effective operation of the SLICES infrastructure. To accomplish this, the data management framework sets its own design goals, which are summarized below.
%and presented in \Cref{fig:Data-managment-framework}.

\begin{itemize}
    \item[-] \textbf{Data Governance: } A systemic and effective Data Governance structure to support the data management operations through a hierarchical structure with appropriate roles (e.g. Data Manager, Data Protection Officer and Metadata administrator), implement all related policies and processes, and adopt standards and leading practices.
    \item[-] \textbf{Data Architecture: } An agile Data Architecture that can perform efficiently to fulfill the SLICES infrastructure requirements, scales gracefully to accommodate for increased workloads, is flexible to integrate new processes and technologies, and is open to interact with other systems and infrastructures. 
    \item[-] \textbf{Data Quality:} Appropriate data transformation mechanisms to ensure Data Quality across multiple dimensions (e.g. accuracy, completeness, integrity), in order to improve data utility (e.g. further processing, analysis).
    \item[-] \textbf{Metadata:} Appropriate metadata management mechanisms to facilitate collaboration between users by providing the means to share their data and also support FAIR data.  
    \item[-] \textbf{Interoperability:  } Facilitate seamless interaction with other systems and infrastructures.
    \item[-] \textbf{Analytics:} Deployment of statistical, machine learning and artificial intelligence techniques to draw valuable insights from data and appropriate visualisation techniques to interpret them.
    \item[-] \textbf{Data Security:  } Mechanisms to protect data from unauthorized access and protect its integrity.
    \item[-] \textbf{Privacy: } Strict controls to manage the sharing of data, both internally and externally.
    
\end{itemize} 


%FIGURE: Data Managment Framework
%\begin{figure}[h]
%   \centering
%   \includegraphics{././figures/png/Data-managment-framework.png}
%   \caption{Data Management Framework}
%	\label{fig:Data-managment-framework}
%\end{figure}





\bibliographystyle{eptcs}
\begin{thebibliography}{99}

\bibitem{chaplot2020neural} Chaplot, Devendra Singh, et al. "Neural topological slam for visual navigation." Proceedings of the IEEE/CVF conference on computer vision and pattern recognition. 2020.

\bibitem{maksymets2021thda} Maksymets, Oleksandr, et al. "Thda: Treasure hunt data augmentation for semantic navigation." Proceedings of the IEEE/CVF International Conference on Computer Vision. 2021.

\bibitem{mezghan2022memory} Mezghan, Lina, et al. "Memory-augmented reinforcement learning for image-goal navigation." 2022 IEEE/RSJ International Conference on Intelligent Robots and Systems (IROS). IEEE, 2022.

\bibitem{al2022zero} Al-Halah, Ziad, Santhosh Kumar Ramakrishnan, and Kristen Grauman. "Zero experience required: Plug \& play modular transfer learning for semantic visual navigation." Proceedings of the IEEE/CVF Conference on Computer Vision and Pattern Recognition. 2022.

\bibitem{ye2021auxiliary} Ye, Joel, et al. "Auxiliary tasks and exploration enable objectgoal navigation." Proceedings of the IEEE/CVF international conference on computer vision. 2021.

\bibitem{chaplot2020object} Chaplot, Devendra Singh, et al. "Object goal navigation using goal-oriented semantic exploration." Advances in Neural Information Processing Systems 33 (2020)

\bibitem{ramakrishnan2022poni} Ramakrishnan, Santhosh Kumar, et al. "Poni: Potential functions for objectgoal navigation with interaction-free learning." Proceedings of the IEEE/CVF Conference on Computer Vision and Pattern Recognition. 2022.

\bibitem{ramrakhya2022habitat} Ramrakhya, Ram, et al. "Habitat-web: Learning embodied object-search strategies from human demonstrations at scale." Proceedings of the IEEE/CVF Conference on Computer Vision and Pattern Recognition. 2022.

\bibitem{mousavian2019visual} Mousavian, Arsalan, et al. "Visual representations for semantic target driven navigation." 2019 International Conference on Robotics and Automation (ICRA). IEEE, 2019.

\bibitem{dhariwal2021diffusion} Dhariwal, Prafulla, and Alexander Nichol. "Diffusion models beat gans on image synthesis." Advances in neural information processing systems 34 (2021)

\bibitem{ho2022classifier} Ho, Jonathan, and Tim Salimans. "Classifier-free diffusion guidance." arXiv preprint arXiv:2207.12598 (2022).

\bibitem{nichol2021glide} Nichol, Alex, et al. "Glide: Towards photorealistic image generation and editing with text-guided diffusion models." arXiv preprint arXiv:2112.10741 (2021)

\bibitem{brooks2023instructpix2pix} Brooks, Tim, Aleksander Holynski, and Alexei A. Efros. "Instructpix2pix: Learning to follow image editing instructions." Proceedings of the IEEE/CVF Conference on Computer Vision and Pattern Recognition. 2023.

\bibitem{fu2023guiding} Fu, Tsu-Jui, et al. "Guiding instruction-based image editing via multimodal large language models." arXiv preprint arXiv:2309.17102 (2023).

\bibitem{geng2024instructdiffusion} Geng, Zigang, et al. "Instructdiffusion: A generalist modeling interface for vision tasks." Proceedings of the IEEE/CVF Conference on Computer Vision and Pattern Recognition. 2024.

\bibitem{zhou2024minedreamer} Zhou, Enshen, et al. "Minedreamer: Learning to follow instructions via chain-of-imagination for simulated-world control." arXiv preprint arXiv:2403.12037 (2024).

\bibitem{zhou2023esc} Zhou, Kaiwen, et al. "Esc: Exploration with soft commonsense constraints for zero-shot object navigation." International Conference on Machine Learning. PMLR, 2023.

\bibitem{yu2023l3mvn} Yu, Bangguo, Hamidreza Kasaei, and Ming Cao. "L3mvn: Leveraging large language models for visual target navigation." 2023 IEEE/RSJ International Conference on Intelligent Robots and Systems (IROS). IEEE, 2023.

\bibitem{gadre2023cows} Gadre, Samir Yitzhak, et al. "Cows on pasture: Baselines and benchmarks for language-driven zero-shot object navigation." Proceedings of the IEEE/CVF Conference on Computer Vision and Pattern Recognition. 2023.

\bibitem{shah2023navigation} Shah, Dhruv, et al. "Navigation with large language models: Semantic guesswork as a heuristic for planning." Conference on Robot Learning. PMLR, 2023.

\bibitem{cai2024bridging} Cai, Wenzhe, et al. "Bridging zero-shot object navigation and foundation models through pixel-guided navigation skill." 2024 IEEE International Conference on Robotics and Automation (ICRA). IEEE, 2024.

\bibitem{yu2023co} Yu, Bangguo, Hamidreza Kasaei, and Ming Cao. "Co-NavGPT: Multi-robot cooperative visual semantic navigation using large language models." arXiv preprint arXiv:2310.07937 (2023).

\bibitem{wu2024voronav} Wu, Pengying, et al. "Voronav: Voronoi-based zero-shot object navigation with large language model." arXiv preprint arXiv:2401.02695 (2024).

\bibitem{qin2023mp5} Qin, Yiran, et al. "Mp5: A multi-modal open-ended embodied system in minecraft via active perception." arXiv preprint arXiv:2312.07472 (2023).

\bibitem{du2024learning} Du, Yilun, et al. "Learning universal policies via text-guided video generation." Advances in Neural Information Processing Systems 36 (2024).

\bibitem{ajay2024compositional} Ajay, Anurag, et al. "Compositional foundation models for hierarchical planning." Advances in Neural Information Processing Systems 36 (2024).

\bibitem{liang2024skilldiffuser} Liang, Zhixuan, et al. "Skilldiffuser: Interpretable hierarchical planning via skill abstractions in diffusion-based task execution." Proceedings of the IEEE/CVF Conference on Computer Vision and Pattern Recognition. 2024.

\bibitem{heusel2017gans} Heusel, Martin, et al. "Gans trained by a two time-scale update rule converge to a local nash equilibrium." Advances in neural information processing systems 30 (2017).

\bibitem{zhang2018unreasonable} Zhang, Richard, et al. "The unreasonable effectiveness of deep features as a perceptual metric." Proceedings of the IEEE conference on computer vision and pattern recognition. 2018.

\bibitem{brown2020language} Brown, Tom B. "Language models are few-shot learners." arXiv preprint arXiv:2005.14165 (2020).

\bibitem{podell2023sdxl} Podell, Dustin, et al. "Sdxl: Improving latent diffusion models for high-resolution image synthesis." arXiv preprint arXiv:2307.01952 (2023).

\bibitem{brohan2022rt} Brohan, Anthony, et al. "Rt-1: Robotics transformer for real-world control at scale." arXiv preprint arXiv:2212.06817 (2022).

\bibitem{brohan2023rt} Brohan, Anthony, et al. "Rt-2: Vision-language-action models transfer web knowledge to robotic control." arXiv preprint arXiv:2307.15818 (2023).

\bibitem{li2024manipllm} Li, Xiaoqi, et al. "Manipllm: Embodied multimodal large language model for object-centric robotic manipulation." Proceedings of the IEEE/CVF Conference on Computer Vision and Pattern Recognition. 2024.

\bibitem{shah2023vint} Shah, Dhruv, et al. "ViNT: A foundation model for visual navigation." arXiv preprint arXiv:2306.14846 (2023).

\bibitem{liu2024visual} Liu, Haotian, et al. "Visual instruction tuning." Advances in neural information processing systems 36 (2024).

\bibitem{hu2021lora} Hu, Edward J., et al. "Lora: Low-rank adaptation of large language models." arXiv preprint arXiv:2106.09685 (2021).

\bibitem{qin2023supfusion} Qin, Yiran, et al. "SupFusion: Supervised LiDAR-camera fusion for 3D object detection." Proceedings of the IEEE/CVF International Conference on Computer Vision. 2023.

\bibitem{qin2024worldsimbench} Qin, Yiran, et al. "Worldsimbench: Towards video generation models as world simulators." arXiv preprint arXiv:2410.18072 (2024).

\bibitem{yu2025gamefactory} Yu, Jiwen, et al. "GameFactory: Creating New Games with Generative Interactive Videos." arXiv preprint arXiv:2501.08325 (2025).

\bibitem{zhou2024code} Zhou, Enshen, et al. "Code-as-Monitor: Constraint-aware Visual Programming for Reactive and Proactive Robotic Failure Detection." arXiv preprint arXiv:2412.04455 (2024).

\bibitem{zhang2024ad} Zhang, Zaibin, et al. "AD-H: Autonomous Driving with Hierarchical Agents." arXiv preprint arXiv:2406.03474 (2024).

\bibitem{wang2024toward} Wang, Chaoqun, et al. "Toward Accurate Camera-based 3D Object Detection via Cascade Depth Estimation and Calibration." arXiv preprint arXiv:2402.04883 (2024).

\bibitem{huang2024story3d} Huang, Yuzhou, et al. "Story3d-agent: Exploring 3d storytelling visualization with large language models." arXiv preprint arXiv:2408.11801 (2024).

\bibitem{savinov2018semi} Savinov, Nikolay, Alexey Dosovitskiy, and Vladlen Koltun. "Semi-parametric topological memory for navigation." arXiv preprint arXiv:1803.00653 (2018).

\bibitem{majumdar2022zson} Majumdar, Arjun, et al. "Zson: Zero-shot object-goal navigation using multimodal goal embeddings." Advances in Neural Information Processing Systems 35 (2022): 32340-32352.

\bibitem{yadav2023offline} Yadav, Karmesh, et al. "Offline visual representation learning for embodied navigation." Workshop on Reincarnating Reinforcement Learning at ICLR 2023. 2023.

\bibitem{yadav2023ovrl} Yadav, Karmesh, et al. "Ovrl-v2: A simple state-of-art baseline for imagenav and objectnav." arXiv preprint arXiv:2303.07798 (2023).

\bibitem{sun2024fgprompt} Sun, Xinyu, et al. "FGPrompt: fine-grained goal prompting for image-goal navigation." Advances in Neural Information Processing Systems 36 (2024).

\bibitem{zhu2017target} Zhu, Yuke, et al. "Target-driven visual navigation in indoor scenes using deep reinforcement learning." 2017 IEEE international conference on robotics and automation (ICRA). IEEE, 2017.

\bibitem{koh2024generating} Koh, Jing Yu, Daniel Fried, and Russ R. Salakhutdinov. "Generating images with multimodal language models." Advances in Neural Information Processing Systems 36 (2024).

\bibitem{krantz2022instance} Krantz, Jacob, et al. "Instance-specific image goal navigation: Training embodied agents to find object instances." arXiv preprint arXiv:2211.15876 (2022).

\bibitem{schulman2017proximal} Schulman, John, et al. "Proximal policy optimization algorithms." arXiv preprint arXiv:1707.06347 (2017).

\bibitem{anderson2018evaluation} Anderson, Peter, et al. "On evaluation of embodied navigation agents." arXiv preprint arXiv:1807.06757 (2018).

\bibitem{lin2024navcot} Lin, Bingqian, et al. "NavCoT: Boosting LLM-Based Vision-and-Language Navigation via Learning Disentangled Reasoning." arXiv preprint arXiv:2403.07376 (2024).

\bibitem{NavGPT} Zhou, Gengze, Yicong Hong, and Qi Wu. "Navgpt: Explicit reasoning in vision-and-language navigation with large language models." Proceedings of the AAAI Conference on Artificial Intelligence.

\bibitem{hahn2021no} Hahn, Meera, et al. "No rl, no simulation: Learning to navigate without navigating." Advances in Neural Information Processing Systems 34 (2021): 26661-26673.

\bibitem{li2025t2isafety} Li, Lijun, et al. "T2ISafety: Benchmark for Assessing Fairness, Toxicity, and Privacy in Image Generation." arXiv preprint arXiv:2501.12612 (2025).

\bibitem{an2024agfsync} An, Jingkun, et al. "AGFSync: Leveraging AI-Generated Feedback for Preference Optimization in Text-to-Image Generation." arXiv preprint arXiv:2403.13352 (2024).


\end{thebibliography}

\end{document}
