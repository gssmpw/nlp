\begin{abstract}
Link prediction for directed graphs is a crucial task with diverse real-world applications. Recent advances in embedding methods and Graph Neural Networks (GNNs) have shown promising improvements. However, these methods often lack a thorough analysis of embedding expressiveness and suffer from ineffective benchmarks for a fair evaluation. In this paper, we propose a unified framework to assess the expressiveness of existing methods, highlighting the impact of dual embeddings and decoder design on performance. To address limitations in current experimental setups, we introduce \textbf{DirLinkBench}, a robust new benchmark with comprehensive coverage and standardized evaluation. 
% Empirical results show that current methods struggle to achieve strong performance on the new benchmark, while DiGAE outperforms others overall. We further revisit DiGAE from a theoretical perspective, demonstrating that the graph convolution operation in DiGAE corresponds to the GCN convolution applied to the undirected bipartite graph. 
%\yanping{Empirical results demonstrate that current methods struggle to achieve strong performance on the new benchmark. The key factor in achieving high performance is constructing a graph convolution that aligns with the GCN convolution applied to the undirected bipartite graph. }
The results show that current methods struggle to achieve strong performance on the new benchmark, while DiGAE outperforms others overall. We further revisit DiGAE theoretically, showing its graph convolution aligns with GCN on an undirected bipartite graph.
Inspired by these insights, we propose a novel spectral directed graph auto-encoder \textbf{SDGAE} that achieves SOTA results on DirLinkBench. Finally, we analyze key factors influencing directed link prediction and highlight open challenges. The code is available at \href{https://github.com/ivam-he/DirLinkBench-SDGAE}{here}.



%Directed graphs are a key type of unstructured data, widely present in various real-world applications. Graph Embedding methods and Graph Neural Networks (GNNs) designed for directed graphs have seen significant development in recent years. Node classification and link prediction are the common tasks used to evaluate these methods. In this case, link prediction is more important, as it directly reflects the directionality of edges. However, current research lacks effective benchmarks for link prediction in directed graphs, and differences in datasets, data splits, and experimental setups have hindered the fair evaluation and progress of new methods. 
%In this paper, we rethink the link prediction task for directed graphs. We first analyze the key limitations of existing benchmarks, including issues with task setup, dataset cleaning, data splitting, and the choice of evaluation metrics. Based on these insights, we propose a new, more robust benchmark that ensures fair task setup, incorporates a diverse range of datasets, and standardizes data splits. 

%Building on this new benchmark, we conduct a detailed theoretical analysis of the link prediction task for directed graphs. Our findings reveal that most existing directed GNNs struggle with simple graph structures, such as ring graphs, and we demonstrate that embeddings with more than two real dimensions are necessary to accurately represent the structure of directed graphs. Additionally, we investigate whether GNNs can effectively model the in-degree and out-degree distributions of directed graphs. Finally, we validate our analysis on the proposed benchmark, and the results offer valuable insights into current research directions for directed graph representation, guiding future work in the field. %The code is available at \url{https://github.com/ivam-he/Digraph_LP_Benchmark}.
\end{abstract}