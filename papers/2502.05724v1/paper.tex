%%%%%%%% ICML 2025 EXAMPLE LATEX SUBMISSION FILE %%%%%%%%%%%%%%%%%

\documentclass{article}

% Recommended, but optional, packages for figures and better typesetting:
\usepackage{microtype}
\usepackage{graphicx}
\usepackage{subfigure}
\usepackage{booktabs} % for professional tables
\usepackage{multirow}

% hyperref makes hyperlinks in the resulting PDF.
% If your build breaks (sometimes temporarily if a hyperlink spans a page)
% please comment out the following usepackage line and replace
% \usepackage{icml2025} with \usepackage[nohyperref]{icml2025} above.
\usepackage{hyperref}
\usepackage[table]{xcolor} 
\usepackage{colortbl}  

% Attempt to make hyperref and algorithmic work together better:
\newcommand{\theHalgorithm}{\arabic{algorithm}}


\definecolor{myred}{RGB}{255, 0, 0}    % 
\definecolor{myblue}{RGB}{0, 0, 249}   % 
\definecolor{mygreen}{RGB}{0, 150, 0}  % 

\newcommand{\hig}[2]{%
  \ifnum#1=1
    \textcolor{myred}{\textbf{#2}}%
  \else\ifnum#1=2
    \textcolor{myblue}{\textbf{#2}}%
  \else\ifnum#1=3
    \textcolor{mygreen}{\textbf{#2}}%
  \else
    #2%
  \fi\fi\fi
}

% Use the following line for the initial blind version submitted for review:
%\usepackage{icml2025}

% If accepted, instead use the following line for the camera-ready submission:
\usepackage[accepted]{icml2025}

% For theorems and such
\usepackage{amsmath}
\usepackage{amssymb}
\usepackage{mathtools}
\usepackage{amsthm}
\usepackage{enumitem}
\usepackage{makecell}
%\usepackage{caption}
\usepackage{wrapfig}
\usepackage{hyperref}
\usepackage{siunitx}
%%%%% NEW MATH DEFINITIONS %%%%%

\usepackage{amsmath,amsfonts,bm}
\usepackage{derivative}
% Mark sections of captions for referring to divisions of figures
\newcommand{\figleft}{{\em (Left)}}
\newcommand{\figcenter}{{\em (Center)}}
\newcommand{\figright}{{\em (Right)}}
\newcommand{\figtop}{{\em (Top)}}
\newcommand{\figbottom}{{\em (Bottom)}}
\newcommand{\captiona}{{\em (a)}}
\newcommand{\captionb}{{\em (b)}}
\newcommand{\captionc}{{\em (c)}}
\newcommand{\captiond}{{\em (d)}}

% Highlight a newly defined term
\newcommand{\newterm}[1]{{\bf #1}}

% Derivative d 
\newcommand{\deriv}{{\mathrm{d}}}

% Figure reference, lower-case.
\def\figref#1{figure~\ref{#1}}
% Figure reference, capital. For start of sentence
\def\Figref#1{Figure~\ref{#1}}
\def\twofigref#1#2{figures \ref{#1} and \ref{#2}}
\def\quadfigref#1#2#3#4{figures \ref{#1}, \ref{#2}, \ref{#3} and \ref{#4}}
% Section reference, lower-case.
\def\secref#1{section~\ref{#1}}
% Section reference, capital.
\def\Secref#1{Section~\ref{#1}}
% Reference to two sections.
\def\twosecrefs#1#2{sections \ref{#1} and \ref{#2}}
% Reference to three sections.
\def\secrefs#1#2#3{sections \ref{#1}, \ref{#2} and \ref{#3}}
% Reference to an equation, lower-case.
\def\eqref#1{equation~\ref{#1}}
% Reference to an equation, upper case
\def\Eqref#1{Equation~\ref{#1}}
% A raw reference to an equation---avoid using if possible
\def\plaineqref#1{\ref{#1}}
% Reference to a chapter, lower-case.
\def\chapref#1{chapter~\ref{#1}}
% Reference to an equation, upper case.
\def\Chapref#1{Chapter~\ref{#1}}
% Reference to a range of chapters
\def\rangechapref#1#2{chapters\ref{#1}--\ref{#2}}
% Reference to an algorithm, lower-case.
\def\algref#1{algorithm~\ref{#1}}
% Reference to an algorithm, upper case.
\def\Algref#1{Algorithm~\ref{#1}}
\def\twoalgref#1#2{algorithms \ref{#1} and \ref{#2}}
\def\Twoalgref#1#2{Algorithms \ref{#1} and \ref{#2}}
% Reference to a part, lower case
\def\partref#1{part~\ref{#1}}
% Reference to a part, upper case
\def\Partref#1{Part~\ref{#1}}
\def\twopartref#1#2{parts \ref{#1} and \ref{#2}}

\def\ceil#1{\lceil #1 \rceil}
\def\floor#1{\lfloor #1 \rfloor}
\def\1{\bm{1}}
\newcommand{\train}{\mathcal{D}}
\newcommand{\valid}{\mathcal{D_{\mathrm{valid}}}}
\newcommand{\test}{\mathcal{D_{\mathrm{test}}}}

\def\eps{{\epsilon}}


% Random variables
\def\reta{{\textnormal{$\eta$}}}
\def\ra{{\textnormal{a}}}
\def\rb{{\textnormal{b}}}
\def\rc{{\textnormal{c}}}
\def\rd{{\textnormal{d}}}
\def\re{{\textnormal{e}}}
\def\rf{{\textnormal{f}}}
\def\rg{{\textnormal{g}}}
\def\rh{{\textnormal{h}}}
\def\ri{{\textnormal{i}}}
\def\rj{{\textnormal{j}}}
\def\rk{{\textnormal{k}}}
\def\rl{{\textnormal{l}}}
% rm is already a command, just don't name any random variables m
\def\rn{{\textnormal{n}}}
\def\ro{{\textnormal{o}}}
\def\rp{{\textnormal{p}}}
\def\rq{{\textnormal{q}}}
\def\rr{{\textnormal{r}}}
\def\rs{{\textnormal{s}}}
\def\rt{{\textnormal{t}}}
\def\ru{{\textnormal{u}}}
\def\rv{{\textnormal{v}}}
\def\rw{{\textnormal{w}}}
\def\rx{{\textnormal{x}}}
\def\ry{{\textnormal{y}}}
\def\rz{{\textnormal{z}}}

% Random vectors
\def\rvepsilon{{\mathbf{\epsilon}}}
\def\rvphi{{\mathbf{\phi}}}
\def\rvtheta{{\mathbf{\theta}}}
\def\rva{{\mathbf{a}}}
\def\rvb{{\mathbf{b}}}
\def\rvc{{\mathbf{c}}}
\def\rvd{{\mathbf{d}}}
\def\rve{{\mathbf{e}}}
\def\rvf{{\mathbf{f}}}
\def\rvg{{\mathbf{g}}}
\def\rvh{{\mathbf{h}}}
\def\rvu{{\mathbf{i}}}
\def\rvj{{\mathbf{j}}}
\def\rvk{{\mathbf{k}}}
\def\rvl{{\mathbf{l}}}
\def\rvm{{\mathbf{m}}}
\def\rvn{{\mathbf{n}}}
\def\rvo{{\mathbf{o}}}
\def\rvp{{\mathbf{p}}}
\def\rvq{{\mathbf{q}}}
\def\rvr{{\mathbf{r}}}
\def\rvs{{\mathbf{s}}}
\def\rvt{{\mathbf{t}}}
\def\rvu{{\mathbf{u}}}
\def\rvv{{\mathbf{v}}}
\def\rvw{{\mathbf{w}}}
\def\rvx{{\mathbf{x}}}
\def\rvy{{\mathbf{y}}}
\def\rvz{{\mathbf{z}}}

% Elements of random vectors
\def\erva{{\textnormal{a}}}
\def\ervb{{\textnormal{b}}}
\def\ervc{{\textnormal{c}}}
\def\ervd{{\textnormal{d}}}
\def\erve{{\textnormal{e}}}
\def\ervf{{\textnormal{f}}}
\def\ervg{{\textnormal{g}}}
\def\ervh{{\textnormal{h}}}
\def\ervi{{\textnormal{i}}}
\def\ervj{{\textnormal{j}}}
\def\ervk{{\textnormal{k}}}
\def\ervl{{\textnormal{l}}}
\def\ervm{{\textnormal{m}}}
\def\ervn{{\textnormal{n}}}
\def\ervo{{\textnormal{o}}}
\def\ervp{{\textnormal{p}}}
\def\ervq{{\textnormal{q}}}
\def\ervr{{\textnormal{r}}}
\def\ervs{{\textnormal{s}}}
\def\ervt{{\textnormal{t}}}
\def\ervu{{\textnormal{u}}}
\def\ervv{{\textnormal{v}}}
\def\ervw{{\textnormal{w}}}
\def\ervx{{\textnormal{x}}}
\def\ervy{{\textnormal{y}}}
\def\ervz{{\textnormal{z}}}

% Random matrices
\def\rmA{{\mathbf{A}}}
\def\rmB{{\mathbf{B}}}
\def\rmC{{\mathbf{C}}}
\def\rmD{{\mathbf{D}}}
\def\rmE{{\mathbf{E}}}
\def\rmF{{\mathbf{F}}}
\def\rmG{{\mathbf{G}}}
\def\rmH{{\mathbf{H}}}
\def\rmI{{\mathbf{I}}}
\def\rmJ{{\mathbf{J}}}
\def\rmK{{\mathbf{K}}}
\def\rmL{{\mathbf{L}}}
\def\rmM{{\mathbf{M}}}
\def\rmN{{\mathbf{N}}}
\def\rmO{{\mathbf{O}}}
\def\rmP{{\mathbf{P}}}
\def\rmQ{{\mathbf{Q}}}
\def\rmR{{\mathbf{R}}}
\def\rmS{{\mathbf{S}}}
\def\rmT{{\mathbf{T}}}
\def\rmU{{\mathbf{U}}}
\def\rmV{{\mathbf{V}}}
\def\rmW{{\mathbf{W}}}
\def\rmX{{\mathbf{X}}}
\def\rmY{{\mathbf{Y}}}
\def\rmZ{{\mathbf{Z}}}

% Elements of random matrices
\def\ermA{{\textnormal{A}}}
\def\ermB{{\textnormal{B}}}
\def\ermC{{\textnormal{C}}}
\def\ermD{{\textnormal{D}}}
\def\ermE{{\textnormal{E}}}
\def\ermF{{\textnormal{F}}}
\def\ermG{{\textnormal{G}}}
\def\ermH{{\textnormal{H}}}
\def\ermI{{\textnormal{I}}}
\def\ermJ{{\textnormal{J}}}
\def\ermK{{\textnormal{K}}}
\def\ermL{{\textnormal{L}}}
\def\ermM{{\textnormal{M}}}
\def\ermN{{\textnormal{N}}}
\def\ermO{{\textnormal{O}}}
\def\ermP{{\textnormal{P}}}
\def\ermQ{{\textnormal{Q}}}
\def\ermR{{\textnormal{R}}}
\def\ermS{{\textnormal{S}}}
\def\ermT{{\textnormal{T}}}
\def\ermU{{\textnormal{U}}}
\def\ermV{{\textnormal{V}}}
\def\ermW{{\textnormal{W}}}
\def\ermX{{\textnormal{X}}}
\def\ermY{{\textnormal{Y}}}
\def\ermZ{{\textnormal{Z}}}

% Vectors
\def\vzero{{\bm{0}}}
\def\vone{{\bm{1}}}
\def\vmu{{\bm{\mu}}}
\def\vtheta{{\bm{\theta}}}
\def\vphi{{\bm{\phi}}}
\def\va{{\bm{a}}}
\def\vb{{\bm{b}}}
\def\vc{{\bm{c}}}
\def\vd{{\bm{d}}}
\def\ve{{\bm{e}}}
\def\vf{{\bm{f}}}
\def\vg{{\bm{g}}}
\def\vh{{\bm{h}}}
\def\vi{{\bm{i}}}
\def\vj{{\bm{j}}}
\def\vk{{\bm{k}}}
\def\vl{{\bm{l}}}
\def\vm{{\bm{m}}}
\def\vn{{\bm{n}}}
\def\vo{{\bm{o}}}
\def\vp{{\bm{p}}}
\def\vq{{\bm{q}}}
\def\vr{{\bm{r}}}
\def\vs{{\bm{s}}}
\def\vt{{\bm{t}}}
\def\vu{{\bm{u}}}
\def\vv{{\bm{v}}}
\def\vw{{\bm{w}}}
\def\vx{{\bm{x}}}
\def\vy{{\bm{y}}}
\def\vz{{\bm{z}}}

% Elements of vectors
\def\evalpha{{\alpha}}
\def\evbeta{{\beta}}
\def\evepsilon{{\epsilon}}
\def\evlambda{{\lambda}}
\def\evomega{{\omega}}
\def\evmu{{\mu}}
\def\evpsi{{\psi}}
\def\evsigma{{\sigma}}
\def\evtheta{{\theta}}
\def\eva{{a}}
\def\evb{{b}}
\def\evc{{c}}
\def\evd{{d}}
\def\eve{{e}}
\def\evf{{f}}
\def\evg{{g}}
\def\evh{{h}}
\def\evi{{i}}
\def\evj{{j}}
\def\evk{{k}}
\def\evl{{l}}
\def\evm{{m}}
\def\evn{{n}}
\def\evo{{o}}
\def\evp{{p}}
\def\evq{{q}}
\def\evr{{r}}
\def\evs{{s}}
\def\evt{{t}}
\def\evu{{u}}
\def\evv{{v}}
\def\evw{{w}}
\def\evx{{x}}
\def\evy{{y}}
\def\evz{{z}}

% Matrix
\def\mA{{\bm{A}}}
\def\mB{{\bm{B}}}
\def\mC{{\bm{C}}}
\def\mD{{\bm{D}}}
\def\mE{{\bm{E}}}
\def\mF{{\bm{F}}}
\def\mG{{\bm{G}}}
\def\mH{{\bm{H}}}
\def\mI{{\bm{I}}}
\def\mJ{{\bm{J}}}
\def\mK{{\bm{K}}}
\def\mL{{\bm{L}}}
\def\mM{{\bm{M}}}
\def\mN{{\bm{N}}}
\def\mO{{\bm{O}}}
\def\mP{{\bm{P}}}
\def\mQ{{\bm{Q}}}
\def\mR{{\bm{R}}}
\def\mS{{\bm{S}}}
\def\mT{{\bm{T}}}
\def\mU{{\bm{U}}}
\def\mV{{\bm{V}}}
\def\mW{{\bm{W}}}
\def\mX{{\bm{X}}}
\def\mY{{\bm{Y}}}
\def\mZ{{\bm{Z}}}
\def\mBeta{{\bm{\beta}}}
\def\mPhi{{\bm{\Phi}}}
\def\mLambda{{\bm{\Lambda}}}
\def\mSigma{{\bm{\Sigma}}}

% Tensor
\DeclareMathAlphabet{\mathsfit}{\encodingdefault}{\sfdefault}{m}{sl}
\SetMathAlphabet{\mathsfit}{bold}{\encodingdefault}{\sfdefault}{bx}{n}
\newcommand{\tens}[1]{\bm{\mathsfit{#1}}}
\def\tA{{\tens{A}}}
\def\tB{{\tens{B}}}
\def\tC{{\tens{C}}}
\def\tD{{\tens{D}}}
\def\tE{{\tens{E}}}
\def\tF{{\tens{F}}}
\def\tG{{\tens{G}}}
\def\tH{{\tens{H}}}
\def\tI{{\tens{I}}}
\def\tJ{{\tens{J}}}
\def\tK{{\tens{K}}}
\def\tL{{\tens{L}}}
\def\tM{{\tens{M}}}
\def\tN{{\tens{N}}}
\def\tO{{\tens{O}}}
\def\tP{{\tens{P}}}
\def\tQ{{\tens{Q}}}
\def\tR{{\tens{R}}}
\def\tS{{\tens{S}}}
\def\tT{{\tens{T}}}
\def\tU{{\tens{U}}}
\def\tV{{\tens{V}}}
\def\tW{{\tens{W}}}
\def\tX{{\tens{X}}}
\def\tY{{\tens{Y}}}
\def\tZ{{\tens{Z}}}


% Graph
\def\gA{{\mathcal{A}}}
\def\gB{{\mathcal{B}}}
\def\gC{{\mathcal{C}}}
\def\gD{{\mathcal{D}}}
\def\gE{{\mathcal{E}}}
\def\gF{{\mathcal{F}}}
\def\gG{{\mathcal{G}}}
\def\gH{{\mathcal{H}}}
\def\gI{{\mathcal{I}}}
\def\gJ{{\mathcal{J}}}
\def\gK{{\mathcal{K}}}
\def\gL{{\mathcal{L}}}
\def\gM{{\mathcal{M}}}
\def\gN{{\mathcal{N}}}
\def\gO{{\mathcal{O}}}
\def\gP{{\mathcal{P}}}
\def\gQ{{\mathcal{Q}}}
\def\gR{{\mathcal{R}}}
\def\gS{{\mathcal{S}}}
\def\gT{{\mathcal{T}}}
\def\gU{{\mathcal{U}}}
\def\gV{{\mathcal{V}}}
\def\gW{{\mathcal{W}}}
\def\gX{{\mathcal{X}}}
\def\gY{{\mathcal{Y}}}
\def\gZ{{\mathcal{Z}}}

% Sets
\def\sA{{\mathbb{A}}}
\def\sB{{\mathbb{B}}}
\def\sC{{\mathbb{C}}}
\def\sD{{\mathbb{D}}}
% Don't use a set called E, because this would be the same as our symbol
% for expectation.
\def\sF{{\mathbb{F}}}
\def\sG{{\mathbb{G}}}
\def\sH{{\mathbb{H}}}
\def\sI{{\mathbb{I}}}
\def\sJ{{\mathbb{J}}}
\def\sK{{\mathbb{K}}}
\def\sL{{\mathbb{L}}}
\def\sM{{\mathbb{M}}}
\def\sN{{\mathbb{N}}}
\def\sO{{\mathbb{O}}}
\def\sP{{\mathbb{P}}}
\def\sQ{{\mathbb{Q}}}
\def\sR{{\mathbb{R}}}
\def\sS{{\mathbb{S}}}
\def\sT{{\mathbb{T}}}
\def\sU{{\mathbb{U}}}
\def\sV{{\mathbb{V}}}
\def\sW{{\mathbb{W}}}
\def\sX{{\mathbb{X}}}
\def\sY{{\mathbb{Y}}}
\def\sZ{{\mathbb{Z}}}

% Entries of a matrix
\def\emLambda{{\Lambda}}
\def\emA{{A}}
\def\emB{{B}}
\def\emC{{C}}
\def\emD{{D}}
\def\emE{{E}}
\def\emF{{F}}
\def\emG{{G}}
\def\emH{{H}}
\def\emI{{I}}
\def\emJ{{J}}
\def\emK{{K}}
\def\emL{{L}}
\def\emM{{M}}
\def\emN{{N}}
\def\emO{{O}}
\def\emP{{P}}
\def\emQ{{Q}}
\def\emR{{R}}
\def\emS{{S}}
\def\emT{{T}}
\def\emU{{U}}
\def\emV{{V}}
\def\emW{{W}}
\def\emX{{X}}
\def\emY{{Y}}
\def\emZ{{Z}}
\def\emSigma{{\Sigma}}

% entries of a tensor
% Same font as tensor, without \bm wrapper
\newcommand{\etens}[1]{\mathsfit{#1}}
\def\etLambda{{\etens{\Lambda}}}
\def\etA{{\etens{A}}}
\def\etB{{\etens{B}}}
\def\etC{{\etens{C}}}
\def\etD{{\etens{D}}}
\def\etE{{\etens{E}}}
\def\etF{{\etens{F}}}
\def\etG{{\etens{G}}}
\def\etH{{\etens{H}}}
\def\etI{{\etens{I}}}
\def\etJ{{\etens{J}}}
\def\etK{{\etens{K}}}
\def\etL{{\etens{L}}}
\def\etM{{\etens{M}}}
\def\etN{{\etens{N}}}
\def\etO{{\etens{O}}}
\def\etP{{\etens{P}}}
\def\etQ{{\etens{Q}}}
\def\etR{{\etens{R}}}
\def\etS{{\etens{S}}}
\def\etT{{\etens{T}}}
\def\etU{{\etens{U}}}
\def\etV{{\etens{V}}}
\def\etW{{\etens{W}}}
\def\etX{{\etens{X}}}
\def\etY{{\etens{Y}}}
\def\etZ{{\etens{Z}}}

% The true underlying data generating distribution
\newcommand{\pdata}{p_{\rm{data}}}
\newcommand{\ptarget}{p_{\rm{target}}}
\newcommand{\pprior}{p_{\rm{prior}}}
\newcommand{\pbase}{p_{\rm{base}}}
\newcommand{\pref}{p_{\rm{ref}}}

% The empirical distribution defined by the training set
\newcommand{\ptrain}{\hat{p}_{\rm{data}}}
\newcommand{\Ptrain}{\hat{P}_{\rm{data}}}
% The model distribution
\newcommand{\pmodel}{p_{\rm{model}}}
\newcommand{\Pmodel}{P_{\rm{model}}}
\newcommand{\ptildemodel}{\tilde{p}_{\rm{model}}}
% Stochastic autoencoder distributions
\newcommand{\pencode}{p_{\rm{encoder}}}
\newcommand{\pdecode}{p_{\rm{decoder}}}
\newcommand{\precons}{p_{\rm{reconstruct}}}

\newcommand{\laplace}{\mathrm{Laplace}} % Laplace distribution

\newcommand{\E}{\mathbb{E}}
\newcommand{\Ls}{\mathcal{L}}
\newcommand{\R}{\mathbb{R}}
\newcommand{\emp}{\tilde{p}}
\newcommand{\lr}{\alpha}
\newcommand{\reg}{\lambda}
\newcommand{\rect}{\mathrm{rectifier}}
\newcommand{\softmax}{\mathrm{softmax}}
\newcommand{\sigmoid}{\sigma}
\newcommand{\softplus}{\zeta}
\newcommand{\KL}{D_{\mathrm{KL}}}
\newcommand{\Var}{\mathrm{Var}}
\newcommand{\standarderror}{\mathrm{SE}}
\newcommand{\Cov}{\mathrm{Cov}}
% Wolfram Mathworld says $L^2$ is for function spaces and $\ell^2$ is for vectors
% But then they seem to use $L^2$ for vectors throughout the site, and so does
% wikipedia.
\newcommand{\normlzero}{L^0}
\newcommand{\normlone}{L^1}
\newcommand{\normltwo}{L^2}
\newcommand{\normlp}{L^p}
\newcommand{\normmax}{L^\infty}

\newcommand{\parents}{Pa} % See usage in notation.tex. Chosen to match Daphne's book.

\DeclareMathOperator*{\argmax}{arg\,max}
\DeclareMathOperator*{\argmin}{arg\,min}

\DeclareMathOperator{\sign}{sign}
\DeclareMathOperator{\Tr}{Tr}
\let\ab\allowbreak


\usepackage{titletoc}

% if you use cleveref..
\usepackage[capitalize,noabbrev]{cleveref}

%%%%%%%%%%%%%%%%%%%%%%%%%%%%%%%%
% THEOREMS
%%%%%%%%%%%%%%%%%%%%%%%%%%%%%%%%
\theoremstyle{plain}
\newtheorem{theorem}{Theorem}[section]
\newtheorem{proposition}[theorem]{Proposition}
\newtheorem{lemma}[theorem]{Lemma}
\newtheorem{corollary}[theorem]{Corollary}
\theoremstyle{definition}
\newtheorem{definition}[theorem]{Definition}
\newtheorem{assumption}[theorem]{Assumption}
\theoremstyle{remark}
\newtheorem{remark}[theorem]{Remark}
\newcommand{\yanping}[1]{\textcolor{red}{#1}}

% Todonotes is useful during development; simply uncomment the next line
%    and comment out the line below the next line to turn off comments
%\usepackage[disable,textsize=tiny]{todonotes}
\usepackage[textsize=tiny]{todonotes}


% The \icmltitle you define below is probably too long as a header.
% Therefore, a short form for the running title is supplied here:
\icmltitlerunning{Rethinking Link Prediction for Directed Graphs}

\begin{document}
\setlength{\textfloatsep}{10pt}

\twocolumn[
\icmltitle{Rethinking Link Prediction for Directed Graphs}
%\title{\textbf{Rethinking Link Prediction for Directed Graphs}}

%\date{} 
%\noindent\rule{\textwidth}{4pt}
%\vspace{-4.0em}
%\maketitle
%\noindent\rule{\textwidth}{1pt}
%\vspace{-2.5em}



% It is OKAY to include author information, even for blind
% submissions: the style file will automatically remove it for you
% unless you've provided the [accepted] option to the icml2025
% package.

% List of affiliations: The first argument should be a (short)
% identifier you will use later to specify author affiliations
% Academic affiliations should list Department, University, City, Region, Country
% Industry affiliations should list Company, City, Region, Country

% You can specify symbols, otherwise they are numbered in order.
% Ideally, you should not use this facility. Affiliations will be numbered
% in order of appearance and this is the preferred way.
\icmlsetsymbol{equal}{*}

\begin{icmlauthorlist}
\icmlauthor{Mingguo He}{ruc}
\icmlauthor{Yuhe Guo}{ruc}
\icmlauthor{Yanping Zheng}{ruc}
\icmlauthor{Zhewei Wei}{ruc}
\icmlauthor{Stephan Günnemann}{tum}
\icmlauthor{Xiaokui Xiao}{nus}
%\icmlauthor{Firstname7 Lastname7}{comp}
%\icmlauthor{}{sch}
%\icmlauthor{Firstname8 Lastname8}{sch}
%\icmlauthor{Firstname8 Lastname8}{yyy,comp}
%\icmlauthor{}{sch}
%\icmlauthor{}{sch}
\end{icmlauthorlist}

\icmlaffiliation{ruc}{Gaoling School of Artifical Intelligence, Renmin University of China}
\icmlaffiliation{tum}{Technical University of Munich}
\icmlaffiliation{nus}{National University of Singapore}

\icmlcorrespondingauthor{Mingguo He}{mingguo@ruc.edu.cn}
%\icmlcorrespondingauthor{Firstname2 Lastname2}{first2.last2@www.uk}

% You may provide any keywords that you
% find helpful for describing your paper; these are used to populate
% the "keywords" metadata in the PDF but will not be shown in the document
\icmlkeywords{Machine Learning, ICML}

\vskip 0.3in
]

% this must go after the closing bracket ] following \twocolumn[ ...

% This command actually creates the footnote in the first column
% listing the affiliations and the copyright notice.
% The command takes one argument, which is text to display at the start of the footnote.
% The \icmlEqualContribution command is standard text for equal contribution.
% Remove it (just {}) if you do not need this facility.

\printAffiliationsAndNotice{}  % leave blank if no need to mention equal contribution
%\printAffiliationsAndNotice{\icmlEqualContribution} % otherwise use the standard text.

\begin{abstract}  
Test time scaling is currently one of the most active research areas that shows promise after training time scaling has reached its limits.
Deep-thinking (DT) models are a class of recurrent models that can perform easy-to-hard generalization by assigning more compute to harder test samples.
However, due to their inability to determine the complexity of a test sample, DT models have to use a large amount of computation for both easy and hard test samples.
Excessive test time computation is wasteful and can cause the ``overthinking'' problem where more test time computation leads to worse results.
In this paper, we introduce a test time training method for determining the optimal amount of computation needed for each sample during test time.
We also propose Conv-LiGRU, a novel recurrent architecture for efficient and robust visual reasoning. 
Extensive experiments demonstrate that Conv-LiGRU is more stable than DT, effectively mitigates the ``overthinking'' phenomenon, and achieves superior accuracy.
\end{abstract}  
\section{Introduction}
\label{sec:introduction}
The business processes of organizations are experiencing ever-increasing complexity due to the large amount of data, high number of users, and high-tech devices involved \cite{martin2021pmopportunitieschallenges, beerepoot2023biggestbpmproblems}. This complexity may cause business processes to deviate from normal control flow due to unforeseen and disruptive anomalies \cite{adams2023proceddsriftdetection}. These control-flow anomalies manifest as unknown, skipped, and wrongly-ordered activities in the traces of event logs monitored from the execution of business processes \cite{ko2023adsystematicreview}. For the sake of clarity, let us consider an illustrative example of such anomalies. Figure \ref{FP_ANOMALIES} shows a so-called event log footprint, which captures the control flow relations of four activities of a hypothetical event log. In particular, this footprint captures the control-flow relations between activities \texttt{a}, \texttt{b}, \texttt{c} and \texttt{d}. These are the causal ($\rightarrow$) relation, concurrent ($\parallel$) relation, and other ($\#$) relations such as exclusivity or non-local dependency \cite{aalst2022pmhandbook}. In addition, on the right are six traces, of which five exhibit skipped, wrongly-ordered and unknown control-flow anomalies. For example, $\langle$\texttt{a b d}$\rangle$ has a skipped activity, which is \texttt{c}. Because of this skipped activity, the control-flow relation \texttt{b}$\,\#\,$\texttt{d} is violated, since \texttt{d} directly follows \texttt{b} in the anomalous trace.
\begin{figure}[!t]
\centering
\includegraphics[width=0.9\columnwidth]{images/FP_ANOMALIES.png}
\caption{An example event log footprint with six traces, of which five exhibit control-flow anomalies.}
\label{FP_ANOMALIES}
\end{figure}

\subsection{Control-flow anomaly detection}
Control-flow anomaly detection techniques aim to characterize the normal control flow from event logs and verify whether these deviations occur in new event logs \cite{ko2023adsystematicreview}. To develop control-flow anomaly detection techniques, \revision{process mining} has seen widespread adoption owing to process discovery and \revision{conformance checking}. On the one hand, process discovery is a set of algorithms that encode control-flow relations as a set of model elements and constraints according to a given modeling formalism \cite{aalst2022pmhandbook}; hereafter, we refer to the Petri net, a widespread modeling formalism. On the other hand, \revision{conformance checking} is an explainable set of algorithms that allows linking any deviations with the reference Petri net and providing the fitness measure, namely a measure of how much the Petri net fits the new event log \cite{aalst2022pmhandbook}. Many control-flow anomaly detection techniques based on \revision{conformance checking} (hereafter, \revision{conformance checking}-based techniques) use the fitness measure to determine whether an event log is anomalous \cite{bezerra2009pmad, bezerra2013adlogspais, myers2018icsadpm, pecchia2020applicationfailuresanalysispm}. 

The scientific literature also includes many \revision{conformance checking}-independent techniques for control-flow anomaly detection that combine specific types of trace encodings with machine/deep learning \cite{ko2023adsystematicreview, tavares2023pmtraceencoding}. Whereas these techniques are very effective, their explainability is challenging due to both the type of trace encoding employed and the machine/deep learning model used \cite{rawal2022trustworthyaiadvances,li2023explainablead}. Hence, in the following, we focus on the shortcomings of \revision{conformance checking}-based techniques to investigate whether it is possible to support the development of competitive control-flow anomaly detection techniques while maintaining the explainable nature of \revision{conformance checking}.
\begin{figure}[!t]
\centering
\includegraphics[width=\columnwidth]{images/HIGH_LEVEL_VIEW.png}
\caption{A high-level view of the proposed framework for combining \revision{process mining}-based feature extraction with dimensionality reduction for control-flow anomaly detection.}
\label{HIGH_LEVEL_VIEW}
\end{figure}

\subsection{Shortcomings of \revision{conformance checking}-based techniques}
Unfortunately, the detection effectiveness of \revision{conformance checking}-based techniques is affected by noisy data and low-quality Petri nets, which may be due to human errors in the modeling process or representational bias of process discovery algorithms \cite{bezerra2013adlogspais, pecchia2020applicationfailuresanalysispm, aalst2016pm}. Specifically, on the one hand, noisy data may introduce infrequent and deceptive control-flow relations that may result in inconsistent fitness measures, whereas, on the other hand, checking event logs against a low-quality Petri net could lead to an unreliable distribution of fitness measures. Nonetheless, such Petri nets can still be used as references to obtain insightful information for \revision{process mining}-based feature extraction, supporting the development of competitive and explainable \revision{conformance checking}-based techniques for control-flow anomaly detection despite the problems above. For example, a few works outline that token-based \revision{conformance checking} can be used for \revision{process mining}-based feature extraction to build tabular data and develop effective \revision{conformance checking}-based techniques for control-flow anomaly detection \cite{singh2022lapmsh, debenedictis2023dtadiiot}. However, to the best of our knowledge, the scientific literature lacks a structured proposal for \revision{process mining}-based feature extraction using the state-of-the-art \revision{conformance checking} variant, namely alignment-based \revision{conformance checking}.

\subsection{Contributions}
We propose a novel \revision{process mining}-based feature extraction approach with alignment-based \revision{conformance checking}. This variant aligns the deviating control flow with a reference Petri net; the resulting alignment can be inspected to extract additional statistics such as the number of times a given activity caused mismatches \cite{aalst2022pmhandbook}. We integrate this approach into a flexible and explainable framework for developing techniques for control-flow anomaly detection. The framework combines \revision{process mining}-based feature extraction and dimensionality reduction to handle high-dimensional feature sets, achieve detection effectiveness, and support explainability. Notably, in addition to our proposed \revision{process mining}-based feature extraction approach, the framework allows employing other approaches, enabling a fair comparison of multiple \revision{conformance checking}-based and \revision{conformance checking}-independent techniques for control-flow anomaly detection. Figure \ref{HIGH_LEVEL_VIEW} shows a high-level view of the framework. Business processes are monitored, and event logs obtained from the database of information systems. Subsequently, \revision{process mining}-based feature extraction is applied to these event logs and tabular data input to dimensionality reduction to identify control-flow anomalies. We apply several \revision{conformance checking}-based and \revision{conformance checking}-independent framework techniques to publicly available datasets, simulated data of a case study from railways, and real-world data of a case study from healthcare. We show that the framework techniques implementing our approach outperform the baseline \revision{conformance checking}-based techniques while maintaining the explainable nature of \revision{conformance checking}.

In summary, the contributions of this paper are as follows.
\begin{itemize}
    \item{
        A novel \revision{process mining}-based feature extraction approach to support the development of competitive and explainable \revision{conformance checking}-based techniques for control-flow anomaly detection.
    }
    \item{
        A flexible and explainable framework for developing techniques for control-flow anomaly detection using \revision{process mining}-based feature extraction and dimensionality reduction.
    }
    \item{
        Application to synthetic and real-world datasets of several \revision{conformance checking}-based and \revision{conformance checking}-independent framework techniques, evaluating their detection effectiveness and explainability.
    }
\end{itemize}

The rest of the paper is organized as follows.
\begin{itemize}
    \item Section \ref{sec:related_work} reviews the existing techniques for control-flow anomaly detection, categorizing them into \revision{conformance checking}-based and \revision{conformance checking}-independent techniques.
    \item Section \ref{sec:abccfe} provides the preliminaries of \revision{process mining} to establish the notation used throughout the paper, and delves into the details of the proposed \revision{process mining}-based feature extraction approach with alignment-based \revision{conformance checking}.
    \item Section \ref{sec:framework} describes the framework for developing \revision{conformance checking}-based and \revision{conformance checking}-independent techniques for control-flow anomaly detection that combine \revision{process mining}-based feature extraction and dimensionality reduction.
    \item Section \ref{sec:evaluation} presents the experiments conducted with multiple framework and baseline techniques using data from publicly available datasets and case studies.
    \item Section \ref{sec:conclusions} draws the conclusions and presents future work.
\end{itemize}






%\newpage
%\section{Preliminaries}

%\textbf{Notation}. We consider a directed unweighted graph $\gG=(V,E)$ with node set $V$ and edge set $E$. Let $n=|V|$ and $m=|E|$ denote the number of nodes and edges in $\gG$. We use $\mA$ to denote the adjacency matrix of $G$, with $\mA_{ij} = 1$ if there is a directed edge from node $i$ to $j$, and $0$ otherwise. We use $\mD_{\rm out} = {\rm diag}(\mA\vone) $ and $\mD_{\rm int} = {\rm diag}(\mA^{\top}\vone)$ to denote the out- and in-degree matrix of $\mA$, respectively, where $\vone$ is the all-in-one vector. Let $\mX \in \mathbb{R}^{n \times d}$ denote the feature matrix of the node, where each node is described by a vector of dimensions $d$.




%Let $\hat{\mathbf{A}}=\mathbf{A}+\mathbf{I}$ denote the adjacency matrix with self-loops, and we use $\hat{\mathbf{D}}_{\rm in}$ and $\hat{\mathbf{D}}_{\rm out}$ to denote the in- and out-degree matrix of $\hat{\mathbf{A}}$. We use $\mathbf{x}\in \mathbb{R}^n$ to denote the graph attributes, where $\mathbf{x}(i)$ denotes the attribute at node $i$. Note that in the general case of GNNs where the input feature is a matrix $\mathbf{X}\in \mathbb{R}^{n\times f}$, we can treat each column of $\mathbf{X}$ as a graph attribute.

%\textbf{Vanilla Graph Convolutions}~\cite{chebnet, gcn}. Originally defined for undirected graphs, vanilla graph convolutions utilize the adjacency matrix, denoted as $\mathbf{A}_{\rm u}$, of the undirected graph transformed from $G$. The normalized Laplacian matrix is represented as $\mathbf{L}_{\rm u} = \mathbf{I} - \mathbf{D}_{\rm u}^{-1/2} \mathbf{A}_{\rm u} \mathbf{D}_{\rm u}^{-1/2}$, where $\mathbf{D}_{\rm u}$ is the degree matrix of $\mathbf{A}_{\rm u}$. It is established that $\mathbf{L}_{\rm u}$ is a positive semidefinite matrix. The eigendecomposition of $\mathbf{L}_{\rm u}$ is expressed as $\mathbf{L}_{\rm u} = \mathbf{U} \mathbf{\Lambda} \mathbf{U}^T$. Consequently, vanilla graph convolutions are further approximable by the $K$-th order polynomial of Laplacians
%\begin{equation}\label{eq:van_GC}
%    \mathbf{y}=\mathbf{U}h \left(\mathbf{\Lambda}\right)\mathbf{U}^T\mathbf{x} = h(\mathbf{L})\mathbf{x} \approx \sum\limits_{k=0}^K w_k \mathbf{L}_{\rm u}^k\mathbf{x},
%\end{equation}
%where $h(\lambda)$ is the spectral filter, which is a function of eigenvalues of the Laplacian matrix $\mathbf{L}_{\rm u}$. The $w_k$ denotes the polynomial filter weights. Many studies have concentrated on learning and approximating Equation~\eqref{eq:van_GC}~\cite{chebnet,appnp,gprgnn,bernnet,chebnetii,jacobiconv,optbasis}. For instance, GPRGNN~\cite{gprgnn} focuses on directly learning the coefficients $w_k$, while JacobiConv~\cite{jacobiconv} and OptiBasisGNN~\cite{optbasis} investigate the potential of various polynomial bases in approximating and learning the filter $h(\Lambda)$. Although these methods have attained a degree of success, their direct application to directed graphs is problematic. This difficulty arises from the fact that the adjacency matrix $\mathbf{A}$ of directed graphs is not self-adjoint, which complicates the process of eigendecomposition and the subsequent definition of corresponding graph convolutions.




\begin{figure*}[ht]
    \centering
   \vspace{-2mm}
   \hspace{-4.8mm}
   \subfigure[Direction Prediction (DP)]{
   \includegraphics[width=58mm]{figure/magnet_mlp_dp.pdf}
   \label{fig:dp}
   }
   \hspace{-4.8mm}
   %\vspace{-2mm}
   \subfigure[Existence Prediction (EP)]{
   \includegraphics[width=58mm]{figure/magnet_mlp_ep.pdf}
   \label{fig:ep}
   }
   %\vspace{-2mm}
   \hspace{-4.8mm}
   \subfigure[Class imbalance issue]{
   \includegraphics[width=58mm]{figure/cora_bar_chart.pdf}
   \label{fig:imbalance}
   }
   \vspace{-3mm}
   \hspace{-4.8mm}
   \caption{(a) and (b) are the results of MagNet~\cite{magnet} as reported in the original paper, alongside the reproduced MagNet and MLP results. (c) is the number of samples and the accuracy for each class of DUPLEX~\cite{duplex} on the Cora dataset in the 4C task.}
   \vspace{-4mm}
   \label{fig:magnet_mlp}
\end{figure*}





%\section{Issues with Existing Setting}

\section{Rethinking Directed Link Prediction}
In this section, we will first revisit the link prediction task for directed graphs and introduce a unified framework to assess the expressiveness of existing methods. Next, we examine the current experimental setups for directed link prediction and highlight several associated issues.

\textbf{Notation.} We consider a directed, unweighted graph $\gG = (V, E)$, with node set $V$ and edge set $E$. Let $n = |V|$ and $m = |E|$ represent the number of nodes and edges in $\gG$, respectively. We use $\mA$ to denote the adjacency matrix of $\gG$, where $\mA_{uv} = 1$ if there exists a directed edge from node $u$ to node $v$, and $\mA_{uv} = 0$ otherwise.
The Hermitian adjacency matrix of $\gG$ is denoted by $\mH$ and defined as $\mH = \mA_{\rm s} \odot \exp\left(i\frac{\pi}{2}\mathbf{\Theta}\right)$. Here, $\mA_{\rm s} = \mA \cup \mA^{\top}$ is the adjacency matrix of the undirected graph derived from $\gG$, and $\mathbf{\Theta} = \mA - \mA^{\top}$ is a skew-symmetric matrix.
We denote the out-degree and in-degree matrices of $\mA$ by $\mD_{\rm out} = {\rm diag}(\mA\vone)$ and $\mD_{\rm int} = {\rm diag}(\mA^{\top}\vone)$, respectively, where $\vone$ is the all-one vector. Let $\mX \in \mathbb{R}^{n \times d^{\prime}}$ denote the node feature matrix, where each node has a $d^{\prime}$-dimensional feature vector. Key notations of this paper are summarized in Appendix~\ref{app_notation}.



%In this section, we will replicate the link prediction tasks on directed graphs in the existing setup and discuss the associated issues. In addition, we will analyze the deep-seated reasons for these issues.

\subsection{Unified Framework for Directed Link Prediction}
The link prediction task on directed graphs is to predict potential directed links (edges) in an observed graph \( \gG^{\prime} \), with the given structure of \( \gG^{\prime} \) and node feature \( \mX \). Formally,
\begin{definition}\label{de_dlp}
\textbf{Directed link prediction problem}. Given an observed graph $\gG^{\prime}=(V, E^{\prime})$ and node feature $\mX$, the goal of directed link prediction is to predict the likelihood of a directed edge $(u,v) \in E^{\star}$ existing, where $E^{\star} \subseteq (V \times V) \setminus E^{\prime}$. The probability of edge $(u,v)$ existing is given by 
\begin{equation}
    p(u,v) = f(u \to v \mid \gG^{\prime}, \mX).
\end{equation}
\end{definition}
The $f(\cdot)$ denotes a prediction model, such as embedding methods or GNNs. 
%The threshold $\epsilon \in [0,1]$ determines whether an edge exists.
%\begin{equation}
%    E^{\star} = \left\{(u,v) \in (V \times V) \setminus E^{\prime} \mid p(u,v) \geq \epsilon \right\},
%\end{equation}
Unlike link prediction on undirected graphs~\cite{zhang2018link}, for directed graphs, it is necessary to account for directionality. Specifically, $p(u,v)$  and $p(v,u)$ are not equal; they represent the probability of a directed edge existing from node $u$ to node $v$, and from node $v$ to node $u$, respectively. To evaluate the expressiveness of existing methods for directed link prediction, we propose a unified framework: 
\begin{align}
    ({\bm \theta}_u, {\bm \phi}_u) &= \mathrm{Enc}(\gG^{\prime}, \mX, u), \quad \forall u \in V, \label{eq_encoder} \\
    p(u, v) &= \mathrm{Dec}({\bm \theta}_u, {\bm \phi}_u, {\bm \theta}_v, {\bm \phi}_v), \quad \forall (u, v) \in E^{\star}. \label{eq_decoder}
\end{align}
Here, $\mathrm{Enc}(\cdot)$ represents an encoder function, which includes various methods described in the Sec.~\ref{intro}. %These methods can be categorized into four types: source-target methods, gravity-inspired methods, single real-valued methods, and complex-valued methods. 
And ${\bm \theta}_u \in \mathbb{R}^{d_{\theta}}$, ${\bm \phi}_u \in \mathbb{R}^{d_{\phi}}$ are real-valued dual embeddings of dimensions ${d_{\theta}}$ and ${d_{\phi}}$, respectively. $\mathrm{Dec}(\cdot)$ is a decoder function tailored to the specific encoder method. This framework unifies existing methods for directed link prediction, as summarized in Table~\ref{table_fram}. More details are provided in Appendix~\ref{app_uni_fram}. Based on this framework, we have the following theorem.
\begin{theorem}\label{th_dual}
For the framework defined by Equations (\ref{eq_encoder}) and (\ref{eq_decoder}), if ${d_{\theta}}, {d_{\phi}} > 0$ and sufficiently large, there exist embeddings ${\bm \theta}_u, {\bm \phi}_u$ and a decoder $\mathrm{Dec}(\cdot)$ that can correctly compute the probability $p(u,v)$ of any directed edge $(u,v)$ in an arbitrary graph. Conversely, if ${d_{\theta}} = 0$ or ${d_{\phi}} = 0$, no such embeddings or decoders can compute the correct probability of any edges in an arbitrary graph.
\end{theorem}
The intuition behind the proof of this theorem is that dual embeddings can effectively reconstruct the structure of a graph. For example, in the source-target embeddings, we have $\mA_{uv} = \vs_u^{\top}\vt_v$, and in the complex-valued method, $\mH_{uv} = \vz_u \overline{\vz}_v$. The full proof can be found in Appendix~\ref{app_proof_th_dual}. Theorem~\ref{th_dual} demonstrates that dual embeddings are critical for preserving the asymmetry of directed graphs, and that a suitable decoder function is equally important. %To further highlight the significance of the decoder function,
%For decoder design, we present the following corollary:
\begin{corollary}\label{coro_de}
With dual embeddings $\bm{\theta}_u$ and $\bm{\phi}_u$, if there is no suitable decoder $\mathrm{Dec(\cdot)}$, the probability $p(u,v)$ of any edge $(u,v)$ cannot be computed correctly. In contrast, even with one single embedding ($\bm{\theta}_u$ or $\bm{\phi}_u = \varnothing$), a suitable decoder can improve the ability to compute edge probabilities.
\end{corollary}
The proof is given in Appendix~\ref{app_proof_cora_de}. This corollary shows that dual embeddings require suitable decoders for theoretical expressiveness, while single embeddings, though fundamentally limited, can still benefit from specific decoders in practice. For example, complex-valued methods using $\mathrm{MLP}\bigl({\bm\theta}_u \|{\bm\theta}_v \|\bm{\phi}_u \|\bm{\phi}_v\bigr)$~\cite{magnet} and single real-valued methods using $\mathrm{MLP}\bigl(\vh_u \| \vh_v\bigr)$~\cite{dpyg} have equal expressiveness if embedding dimensions are large.

%Existing methods for directed graph link prediction can be broadly categorized into \textbf{embedding-based methods} and \textbf{graph neural networks (GNNs)}. Embedding-based methods aim to preserve the asymmetry of directed graphs by generating two separate embeddings for each node: a source embedding $\vs_u$ and a target embedding $\vt_u$~\cite{eltra,odin}, which are also known as content/context representations~\cite{strap}. For an edge $(u, v)$, these methods use the source embedding $\vs_u$ 
%of node $u$ and the target embedding $\vt_v$ of node $v$ 
%to compute the probability $p(u,v) = \mathrm{Dec}(\vs_u, \vt_v)$. Here, $\vs_u, \vt_v \in \mathbb{R}^d$ are $d$-dimensional real-valued embeddings, and 
%$\mathrm{Dec}(\cdot)$ represents the decoder function, commonly chosen as the inner product or logistic regression.

%GNN-based methods, on the other hand, can be further divided into four distinct classes according to the types of embedding they generate. (1) The first class is similar to embedding-based methods, employing specialized propagation mechanisms to learn separate source and target embeddings for each node. (2) The second class, inspired by Newton’s law of universal gravitation, learns a real-valued embedding
%$\vh_u \in \mathbb{R}^d$ and a mass parameter $m_u \in \mathbb{R}^{+}$ for each node $u$, then computes $p(u,v) = \mathrm{Dec}(\vh_u, \vh_v, m_v)$. (3) The third class adopts a more traditional approach, learning a single embedding $\vh_u \in \mathbb{R}^d$ for each node and computing the probabilities as $p(u,v) = \mathrm{Dec}(\vh_u, \vh_v)$. (4) The last class is based on Hermitian adjacency matrices, learning complex-valued embeddings $\vz_u \in \mathbb{C}^d$ for each node and using $p(u,v) = \mathrm{Dec}(\vz_u, \vz_v)$ to obtain the edge probabilities.
%We have the following theorem.










%A directed graph \( G = (V, E) \), where \( V \) is the set of nodes, and \( E \subseteq V \times V \) is the set of directed edges. Node feature matrix \( \mathbf{X} \in \mathbb{R}^{|V| \times d} \), where each node has a feature vector of dimension \( d \).


   
%\textbf{Directionality Determination}:
%The directionality of edges between two nodes \( u \) and \( v \) is determined by evaluating the predictions \( f(u, v) \) and \( f(v, u) \):

%If \( f(u, v) \to 1 \) and \( f(v, u) \to 0 \), it indicates that there is a directed edge from \( u \) to \( v \).
%If \( f(v, u) \to 1 \) and \( f(u, v) \to 0 \), it indicates that there is a directed edge from \( v \) to \( u \).
%If \( f(u, v) \to 1 \) and \( f(v, u) \to 1 \), this suggests the presence of an undirected edge between \( u \) and \( v \) (i.e., a bidirectional edge).
%If \( f(u, v) \to 0 \) nor \( f(v, u) \to 0 \), then no edge exists between \( u \) and \( v \) in either direction.

%This approach ensures that the model not only predicts the existence of edges but also accurately captures the directionality or undirected nature of the relationship between nodes.


%\begin{theorem}
%Let $\gG = (V, E)$ be a directed graph with node set $V$ and edge set $E$, its adjacency matrix and Hermitian adjacency matrix denote as $\mA \in \mathbb{R}^{n \times n}$ and $\mH \in \mathbb{C}^{n \times n}$, respectively,  and let $d$ be a positive integer.

%\begin{enumerate}[label=(\alph*)]
%    \item If each node $v \in V$ is assigned an embedding $\vx_r(v) \in \mathbb{R}^d$, then there does not exist a function $f: \mathbb{R}^d \times \mathbb{R}^d \rightarrow \mathbb{R}$ such that the adjacency relationships of $G$ can be fully captured by $\mA_{ij} = f(\vx_r(v_i), \vx_r(v_j))$. Specifically, certain structures, such as the triangle with directed edges $(1 \rightarrow 2),\ (3 \rightarrow 1),\ (3 \rightarrow 2)$, cannot be effectively represented.

 %    \item If each node $v \in V$ is assigned an embedding $\vx_c(v) \in \mathbb{C}^d$, then the graph structure can be effectively represented. The Hermitian adjacency matrix $\mH$ can be decomposed as $\mH_{ij} = \vx_c(v_i)^* \overline{\vx}_c(v_j)$, from which the adjacency relationships of $G$ can be recovered.

%     \item If each node $v \in V$ is assigned two embeddings $\vx_s(v),\ \vx_t(v) \in \mathbb{R}^d$, then the graph structure can be effectively represented. There exists a function $g: \mathbb{R}^d \times \mathbb{R}^d \rightarrow \mathbb{R}$ such that the adjacency matrix $\mA$ of $G$ can be reconstructed via $\mA_{ij} = g(\vx_s(v_i), \vx_t(v_j))$.
% \end{enumerate}
% \end{theorem}

\begin{table*}[th]
\centering
\caption{Link prediction results on Cora dataset under the DUPLEX~\cite{duplex} setup: results without superscripts are from the DUPLEX paper, $^\dagger$ indicates reproduction with test set edges in training, and $^\ddagger$ indicates reproduction without test set edges in training.}
%\resizebox{\textwidth}{!}{
\begin{tabular}{@{}lllllll@{}}
\toprule

%\multicolumn{1}{l}{\multirow{2}{*}{Dataset}} & \multirow{2}{*}{Method} & \multicolumn{2}{c}{EP} & \multicolumn{2}{c}{DP} & 3C  & 4C  \\ \cmidrule(l){3-8} 
Method &EP(ACC) &EP(AUC) &DP(ACC) &DP(AUC) &3C(ACC) &4C(ACC) \\ \midrule
%\multicolumn{1}{l}{} & & ACC  & AUC  & ACC   & AUC  & ACC & ACC \\ \midrule
 MagNet &81.4$\pm$0.3 &89.4$\pm$0.1 &88.9$\pm$0.4 &95.4$\pm$0.2 &66.8$\pm$0.3 &63.0$\pm$0.3 \\
 DUPLEX &93.2$\pm$0.1 &95.9$\pm$0.1 &95.9$\pm$0.1 &97.9$\pm$0.2 &92.2$\pm$0.1 &88.4$\pm$0.4  \\ \midrule

 DUPLEX$^\dagger$ &93.49$\pm$0.21 &95.61$\pm$0.20 &95.25$\pm$0.16 &96.34$\pm$0.23  &92.41$\pm$0.21 &89.76$\pm$0.25 \\

 MLP$^\dagger$ &88.53$\pm$0.22	&\textbf{95.46$\pm$0.18}	&\textbf{95.76$\pm$0.21}	&\textbf{99.25$\pm$0.06}	&79.97$\pm$0.48	&78.49$\pm$0.26  \\ \midrule

 DUPLEX$^\ddagger$ &87.43$\pm$0.20 &91.16$\pm$0.24 &88.43$\pm$0.16 &91.74$\pm$0.38 &84.53$\pm$0.34 &81.36$\pm$0.46 \\
 MLP$^\ddagger$ &84.00$\pm$0.29	&\textbf{91.52$\pm$0.25}	&\textbf{90.83$\pm$0.16}	&\textbf{96.48$\pm$0.28}	&72.93$\pm$0.21	&71.51$\pm$0.20  \\
  \bottomrule

\end{tabular}
\label{tb:duplex}
\end{table*}




\subsection{Issues with Existing Experimental Setup}\label{issues}
The existing directed link prediction experimental setups can be broadly categorized into two types. The first is the \textbf{multiple subtask setup}, which includes existence prediction (EP), direction prediction (DP), three-type prediction (3C), and four-type prediction (4C). This approach treats directed link prediction as a multi-class classification problem that requires the prediction of positive, inverse, bidirectional, and nonexistent edges. More details are provided in Appendix~\ref{app_issues_subtask}. This setup is widely adopted by existing methods~\cite{magnet,dpyg,fiorini2023sigmanet,lin2023magnetic,duplex,lightdic}.

The other category is the \textbf{non-standardized setting} defined in various papers~\cite{strap,odin,dhypr,digae,coba}. These settings involve different datasets, inconsistent splitting methods, and varying evaluation metrics. We discuss the four issues with existing setups below. 


\textbf{Issue 1: The Multi-layer Perceptron (MLP) is a neglected but powerful baseline.} 
Most existing setups do not report MLP performance. We evaluate MLP across three popular multiple subtask setups: MagNet~\cite{magnet}, DUPLEX~\cite{duplex}, and PyGSD~\cite{dpyg} covering their different datasets and baselines. Figures~\ref{fig:dp} and~\ref{fig:ep} present the results of our reproduced MagNet experiments alongside the MLP results, showing that MLP performs comparably to MagNet on DP and EP tasks. Table~\ref{tb:duplex} shows the replicated DUPLEX experiments and MLP results on the Cora dataset, demonstrating that MLP achieves competitive performance. Interestingly, while DUPLEX represents the latest advancement in directed graph learning, MLP outperforms it in certain tasks. %Additional results for PyGSD and other datasets are provided in Appendix~\ref{app_issues_results}. 
These findings highlight the absence of basic baselines in prior work and suggest that current benchmarks lack sufficient challenge.

%In most existing setups, researchers have not reported the performance of MLP. We evaluate MLP in three recent and widely used multiple subtask setups: MagNet~\cite{magnet}, DUPLEX~\cite{duplex}, and PyGSD~\cite{dpyg}. Although these setups belong to the multiple subtask framework, they use different datasets and split settings, so we evaluate MLP across all of them. In Figures~\ref{fig:dp} and~\ref{fig:ep} , we reproduce the MagNet experiments and present the MLP results. The findings show that MLP performs comparably with MagNet in five datasets for both DP and EP tasks. In Table~\ref{tb:duplex}, we replicate DUPLEX experiments and report the MLP results on Cora, demonstrating that MLP delivers competitive performance. Notably, although DUPLEX is the latest advancement in directed graph learning research, MLP outperforms it on some tasks. Additional results for PyGSD and other datasets are provided in Appendix~\ref{app_issues_results}. These results show that MLP can perform strongly under the existing multiple subtask setups. On the one hand, this highlights the absence of basic baselines in prior research; on the other hand, it suggests that the current benchmark settings lack challenge.

%In most existing settings, researchers have not reported the results of the MLP. We evaluate MLP in three recent and common multiple subtask settings, including MagNet~\cite{magnet}, DUPLEX~\cite{duplex} and PyGSD~\cite{dpyg}. Although they are all use the multiple subtask setting, they have different dataset and splitting, so we evaluate in all of them. In Figure~\ref{fig:magnet_mlp}, we reproduce MagNet experiments and present the MLP results. The results indicate that MLP performs on par with MagNet in five datasets for both directed and existence link prediction tasks. In Table~\ref{tb:duplex}, we replicate the DUPLEX's experiments and report MLP results on Cora dataset, which also show MLP delivers competitive performance and consistently outperforms MagNet. Note that DUPLEX is the latest advancement research for directed graphs, but MLP outperforms it on EP and DP tasks. Due to the space constraints, we report more results of PyGSD and different datasets in the Appendix.

%These results naturally raise the question of why MLP can achieve such strong link prediction performance, even outperforming some GNN models specifically designed for directed graphs, such as MagNet and DUPLEX. 


%We believe two aspects are worth considering: (1) the current benchmark experiments are poorly designed and fail to fairly evaluate the ability of different methods to perform link prediction on directed graphs, and (2) MLP possesses an inherent capability to perform link prediction on directed graphs, making it essential to conduct a detailed analysis of their underlying mechanisms.

%We evaluated MLP in three recent and common benchmark settings and found that it performs comparable to state-of-the-art methods within these frameworks. The pioneering and widely used setting is derived from MagNet~\cite{magnet}, and numerous studies adopt this setup in their link prediction experiments~\cite{dpyg, duplex, lightdic}. 
%Two other important benchmark settings are PyGSD~\cite{dpyg} and DUPLEX~\cite{duplex}, both of which improve the MagNet setting. PyGSD is a software package designed for Signed and Directed Graphs, built on PyTorch Geometric (PyG). 
%We compare the baselines reported in PyGSD with MLP and present the results in Tables 2 and 3. Notably, in the Direction Link Prediction (DP) and Existence Link Prediction (EP) tasks across six datasets, MLP demonstrates competitive performance and achieves SOTA results on both tasks for the Texas and Wisconsin datasets. 
%DUPLEX represents the latest advancement in GNN research for directed graphs. Its paper provides a detailed comparison of the performance of existing methods for link prediction on directed graphs. We replicated their experiments and reported the MLP results under the same conditions, as outlined in Table 6 of the Appendix. The findings show that MLP delivers competitive performance and consistently outperforms MagNet.


%\begin{figure}[t]
%    \centering
%   \vspace{-2mm}
%   %\hspace{-3mm}  
%   \includegraphics[width=75mm]{figure/cora_bar_chart.pdf}
%   \vspace{-3mm}
%   \caption{Performance of DUPLEX's link prediction on Cora for the 4C task. Blue bars represent the number of samples in each class, while orange bars show the corresponding accuracy.}
%   \vspace{-2mm}
%    \label{fig:duplex_cls}
% \end{figure}




\textbf{Issue 2: Many benchmarks suffer from label leakage.}
%Link prediction is defined as the task of predicting which pairs of nodes may have unobserved edges between them, based on the structural information of the observed nodes and edges in a graph. In the link prediction task, edge information in the test set should not be accessible during model training; otherwise, it constitutes label leakage. The benchmarks established in MagNet and PyGSD exhibit label leakage, which occurs during the negative sampling process in training. Specifically, the model samples a large number of negative edges alongside positive edges during training, and this negative sampling inadvertently observed edges from the test set. For example, consider a social network where the user \( u \) follows the user \( v \). If the trained model is tasked with predicting this relationship but has already known the \( u \)-to-\( v \) relationship during negative sampling, the relationship will not be used as a negative sample. This type of label leakage in negative sampling has a limited effect when the number of nodes \( n \) in the graph is large, as there are more negative samples available. However, it significantly affects graphs with smaller \( n \), such as Texas or Cornell, where there are fewer negative samples available for selection.
As defined in Definition~\ref{de_dlp}, directed link prediction aims to predict potential edges from observed graphs, with the key principle that test edges must remain hidden during training to avoid label leakage. However, current setups often violate this principle. For example, 
1) MagNet, PyGSD, and DUPLEX expose test edges during negative edge sampling in the training process, indirectly revealing the test edges' presence to the model.
%resulting in test edges never being treated as negative samples and indirectly revealing their presence to the model.
2) LighDiC~\cite{lightdic} uses eigenvectors of the Laplacian matrix of the entire graph as input features, embedding test edge information in the training input. 
3) DUPLEX propagates information across the entire graph during training, making the test edges directly visible to the model. %These issues highlight serious label leakage concerns in existing setups. 
To investigate, we experiment with DUPLEX using its original code. As shown in Table~\ref{tb:duplex}, DUPLEX$^\dagger$ (original settings with label leakage) clearly outperforms DUPLEX$^\ddagger$ (propagation restricted to training edges) due to label leakage. A similar result is observed with MLP: MLP$^\dagger$ (using in/out degrees from test edges) significantly outperforms MLP$^\ddagger$ (using only training-edge degrees). These findings underscore that even the leakage of degree information can significantly impact performance.

%Since the setup of DUPLEX involves significant label leakage, we conduct experiments using the code provided, and the results on Cora are presented in Table~\ref{tb:duplex}. Here, DUPLEX$^\dagger$ represents the results reproduced with the original settings, while DUPLEX$^\ddagger$ is the results using only the training edges for propagation. DUPLEX$^\dagger$ performs better, probably due to label leakage. Interestingly, a similar result is observed with MLP: MLP$^\dagger$ uses in/out degrees derived from test edges as input features, while MLP$^\ddagger$ relies only on degrees from the training edges. The results show a significant performance boost for MLP$^\dagger$. These findings underscore that even leakage of degree information can greatly impact performance.
%and 2) degree information is crucial for link prediction tasks.

%The experimental setup of DUPLEX reveals significant label leakage, not only in negative sampling but also in the graph propagation process during training, where edges from the test set are used to propagate. This constitutes a critical label leakage, as graph propagation is heavily based on structural information, and accessing test edges during training is highly unreasonable. We conducted experiments using the code provided by DUPLEX and present the results in Table 6. In these experiments, DUPLEX$^\dagger$ represents the results reproduced based on the original settings, while DUPLEX$^\ddagger$ reflects the results obtained when only the training graph is used for propagation. The comparison reveals that DUPLEX$^\dagger$ achieves better results, primarily due to label leakage. 

%Interestingly, a similar phenomenon is observed with MLP. During training, MLP$^\dagger$ can also access edge information from the test set. However, since MLP does not perform graph propagation, we use the in/out degrees of the nodes as its input feature. Specifically, the input feature for MLP$^\dagger$ includes the in/out degrees derived from the test edges, while for MLP$^\ddagger$, it is limited to those of the training graph. The results demonstrate a significant performance boost for MLP$^\dagger$ over MLP$^\ddagger$, with MLP$^\dagger$ even outperforming MagNet. This observation highlights two key points: first, even slight leakage of degree information can substantially influence performance, and second, degree information plays a pivotal role in link prediction tasks.


%These issues underscore significant concerns about label leakage in current link prediction setups.



%\begin{figure}[t]
%    \centering
%   \vspace{-1mm}
%   %\hspace{-3mm}
%   \subfigure[Cora]{
%   \includegraphics[width=65mm]{figure/cora_bar_chart.pdf}
%   }
%   %\hspace{3mm}
%   %\vspace{-2mm}
%   \subfigure[Citeseer]{
%   \includegraphics[width=65mm]{figure/citeseer_bar_chart.pdf}}
%   \vspace{-1mm}
%   \caption{The link prediction results of MagNet as reported in the original paper, along with those of the reproduced MagNet and MLP.}
%   \vspace{-1mm}
%   \label{fig:duplex_cls}
% \end{figure}

%\textbf{Observation 3: The evaluation metric is too simple}. Most existing methods rely on accuracy as the evaluation metric, which is relatively simplistic compared to the metrics used for link prediction in undirected graphs.




\textbf{Issue 3: Multiple subtask setups result in class imbalances and limited evaluation metrics.}
The multiple subtask setups treat directed link prediction as a multi-class classification problem, causing significant class imbalances that hinder model training. For example, the 4C task in DUPLEX classifies edges into reverse, positive, bidirectional, and nonexistent. However, bidirectional edges are rare in real-world directed graphs, and reverse edges are often arbitrarily assigned, lacking meaningful physical significance. Figure~\ref{fig:imbalance} highlights the class imbalance and the challenge of predicting bidirectional edges on the Cora dataset. Additionally, these setups rely heavily on accuracy as an evaluation metric, which provides a limited and potentially misleading
assessment. Given the nature of link prediction, ranking metrics such as Hits@K and Mean Reciprocal Rank (MRR) are more suitable—a perspective well established in link prediction for undirected graphs~\cite{li2023evaluating}.

%Ranking metrics like Hits@K and Mean Reciprocal Rank (MRR), commonly used for undirected graphs~\cite{li2023evaluating}, are more suitable for link prediction.

%As described in Section 2, many existing methods approach link prediction on directed graphs as a multiclass classification problem.
%The multiple subtask setups treat directed link prediction as a multi-classification problem, leading to significant class imbalances and complicating model training. For example, the 4C task in DUPLEX aims to classify edges into four categories: reverse, positive, bidirectional, and nonexistent. However, in real-world directed graphs, bidirectional edges are typically rare, resulting in a low proportion. Additionally, reverse edges are often arbitrarily assigned in homogeneous graphs, lacking meaningful physical significance. %To address imbalances, the counts of reverse and non-existent edges are often artificially adjusted to equalize the number of edges in each class.
%Figure~\ref{fig:imbalance} illustrates the number of samples and the accuracy for each class of DUPLEX on the Cora dataset in the 4C task, showcasing the pronounced class imbalance and the difficulty in predicting bidirectional edges. Furthermore, this setting predominantly uses accuracy as an evaluation metric, which provides a limited and potentially misleading assessment. Given the nature of link prediction, ranking metrics such as Hits@K and Mean Reciprocal Rank (MRR) are more suitable—a perspective well-established in link prediction for undirected graphs~\cite{li2023evaluating}.

%The intuition behind link prediction is to estimate the likelihood that an edge will form between two nodes in the future. Therefore, 

%DUPLEX~\cite{duplex} defines the directed link prediction task as a four-class problem, with labels 0, 1, 2, and 3 representing reverse, positive, bidirectional, and nonexistent edges, respectively. 

%However, this approach results in significant class imbalance, complicating model training. 



\textbf{Issue 4: Lack of standardization in dataset splits and feature inputs.}
Current settings face inconsistent dataset splits. In multiple subtask setups, edges are typically split into 80\% for training, 5\% for validation, and 15\% for testing. However, class proportions are further manually adjusted for balance, which leads to varying training and testing ratios across different datasets. Non-standardized setups are more confusing, e.g., ELTRA~\cite{eltra} uses 90\% for training, STRAP~\cite{strap} uses 50\%, and DiGAE~\cite{digae} uses 85\%, making cross-study results difficult to evaluate.
%Current settings suffer from inconsistent dataset splits. In multiple subtask settings, the edges are typically split, with 80\% for training, 5\% for validation, and 15\% for testing. However, class proportions are further manually adjusted for balance, leading to varying training and testing ratios across datasets. Non-standardized settings are more confusing; for example, ELTRA~\cite{eltra} uses 90\% edges for training, STRAP~\cite{strap} uses 50\%, and DiGAE~\cite{digae} uses 85\%, making it difficult to compare the results reported between studies.
%Existing approaches lack standardized feature inputs. Embedding methods often ignore node features, while GNNs require them. For example, MagNet~\cite{magnet} and PyGSD~\cite{dpyg} use in/out degrees, DUPLEX~\cite{duplex} uses random normal distributions, LightDiC~\cite{lightdic} uses original node features or Laplacian eigenvectors, and DHYPR~\cite{dhypr} uses identity matrices. This inconsistency hinders reproducibility. As shown in Figure~\ref{fig:magnet_mlp} and Table~\ref{tb:duplex}, our reproduced results differ significantly from the reported ones, with some being better and others worse.
Feature input standards are also lacking. Embedding methods often omit node features, while GNNs require them. MagNet and PyGSD use in/out degrees, DUPLEX uses random normal distributions, LightDiC uses original features or Laplacian eigenvectors, and DHYPR~\cite{dhypr} uses identity matrices. This inconsistency undermines reproducibility. As shown in Figure~\ref{fig:magnet_mlp} and Table~\ref{tb:duplex}, reproduced results often deviate significantly, with some better and others worse.

%Due to the space constraints, more details and experimental results of this part are provided in Appendix~\ref{app_issues_results}. These issues highlight the need for a unified benchmark for directed graph link prediction, allowing fair evaluation and providing a foundation for future research.
More details and experimental results are provided in Appendix~\ref{app_issues_results}. These issues emphasize the need for a new benchmark for directed link prediction, allowing fair evaluation and providing a foundation for future research.











%\begin{table}[ht]
%\centering
%\caption{Performance comparison of different models on Three Classes Prediction (3C) task.}
%\resizebox{\textwidth}{!}{
%\begin{tabular}{ccccccc}
%\toprule
%Method & Cora-ML & Citeseer & Telegram & Cornell & Texas & Wisconsin \\ \midrule
%MLP  & 70.56 $\pm$ 0.78  & 64.97 $\pm$ 0.72  & 77.33 $\pm$ 0.49  & 63.05 $\pm$ 2.92  & 69.31 $\pm$ 3.91  & 67.33 $\pm$ 2.06 \\ \midrule 
%DGCN      & 69.05 $\pm$ 0.64  & 64.81 $\pm$ 0.59  & 78.65 $\pm$ 0.75  & 59.64 $\pm$ 4.83  & 65.29 $\pm$ 3.81  & 67.68 $\pm$ 3.29  \\ 
%DiGCN     & 67.88 $\pm$ 0.61  & 64.99 $\pm$ 0.90  & 78.40 $\pm$ 0.44  & 59.37 $\pm$ 2.90  & 64.53 $\pm$ 1.48  & 65.10 $\pm$ 3.57  \\
%DiGCNIB   & 70.71 $\pm$ 0.74  & 68.47 $\pm$ 0.69  & 80.01 $\pm$ 0.68  & 61.74 $\pm$ 3.32  & 67.47 $\pm$ 2.93  & 70.00 $\pm$ 2.36  \\ 
%MagNet    & 71.26 $\pm$ 0.64  & 65.07 $\pm$ 0.80  & 82.66 $\pm$ 0.38  & 60.43 $\pm$ 3.18  & 67.12 $\pm$ 3.15  & 67.89 $\pm$ 2.95 \\  \bottomrule
%\end{tabular}}
%\end{table}



\begin{table*}[htbp]
    \centering
    \caption{Results of various methods under on the \textbf{Hits@100} metric (mean ± standard error\%). Results ranked \hig{1}{first}, \hig{2}{second}, and \hig{3}{third} are highlighted. TO indicates methods that did not finish running within 24 hours, and OOM indicates methods that exceeded memory limits.}
    \label{tab:performance_comparison}
    \resizebox{\textwidth}{!}{
    \begin{tabular}{
        lcccccccr 
    }
        \toprule
        {Method} & {Cora-ML} & {CiteSeer} & {Photo} & {Computers} & {WikiCS} & {Slashdot} & {Epinions} & {Avg.\ Rank $\downarrow$} \\
        \midrule
        STRAP & 79.09$\pm$1.57 & 69.32$\pm$1.29 & \hig{1}{69.16$\pm$1.44} & \hig{2}{51.87$\pm$2.07} &\hig{1}{76.27$\pm$0.92} & 31.43$\pm$1.21 & \hig{1}{58.99$\pm$0.82} &\hig{3}{5.14}\\
        
        ODIN & 54.85$\pm$2.53 & 63.95$\pm$2.98 & 14.13$\pm$1.92 & 12.98$\pm$1.47 & 9.83$\pm$0.47 & 34.17$\pm$1.19 & 36.91$\pm$0.47 &12.71 \\
        
        ELTRA &\hig{2}{87.45$\pm$1.48} & 84.97$\pm$1.90 & 20.63$\pm$1.93 & 14.74$\pm$1.55  & 9.88$\pm$0.70 & 33.44$\pm$1.00  & 41.63$\pm$2.53 &8.57 \\ \midrule
        
        MLP        & 60.61$\pm$6.64 & 70.27$\pm$3.40 & 20.91$\pm$4.18 & 17.57$\pm$0.85 & 12.99$\pm$0.68 & 32.97$\pm$0.51 & 44.59$\pm$1.62 &10.57  \\ %\midrule
        
        GCN        & 70.15$\pm$3.01 & 80.36$\pm$3.07 & \hig{3}{58.77$\pm$2.96} & \hig{3}{43.77$\pm$1.75} & 38.37$\pm$1.51 & 33.16$\pm$1.22 & 46.10$\pm$1.37 &5.71 \\
        
        GAT        & 79.72$\pm$3.07 & \hig{3}{85.88$\pm$4.98} & 58.06$\pm$4.03 & 40.74$\pm$3.22 & 40.47$\pm$4.10  & 30.16$\pm$3.11 & 43.65$\pm$4.88 &5.86 \\
        
        APPNP      & 86.02$\pm$2.88 & 83.57$\pm$4.90 & 47.51$\pm$2.51 & 32.24$\pm$1.40 & 20.23$\pm$1.72 & 31.87$\pm$1.43 & 41.99$\pm$1.23 &8.00 \\ \midrule
        
        DGCN       & 63.32$\pm$2.59 & 68.97$\pm$3.39 & 51.61$\pm$6.33 & 39.92$\pm$1.94 & 25.91$\pm$4.10 & TO               & TO               &10.29  \\
        
        DiGCN      & 63.21$\pm$5.72 & 70.95$\pm$4.67 & 40.17$\pm$2.38 & 27.51$\pm$1.67 & 25.31$\pm$1.84 & TO               & TO               &11.29 \\
        
        DiGCNIB    & 80.57$\pm$3.21 & 85.32$\pm$3.70 & 48.26$\pm$3.98 & 32.44$\pm$1.85 & 28.28$\pm$2.44 & TO               & TO               &8.43  \\
        
        DirGNN     & 76.13$\pm$2.85 & 76.83$\pm$4.24 & 49.15$\pm$3.62 & 35.65$\pm$1.30 & \hig{3}{50.48$\pm$0.85} &\hig{3}{41.74$\pm$1.15} &50.10$\pm$2.06 &6.00 \\ \midrule
        
        MagNet     & 56.54$\pm$2.95 & 65.32$\pm$3.26 & 13.89$\pm$0.32 & 12.85$\pm$0.59 & 10.81$\pm$0.46  & 31.98$\pm$1.06 & 28.01$\pm$1.72 &13.14  \\
        
        DUPLEX &69.00$\pm$2.52	&73.39$\pm$3.42	&17.94$\pm$0.66	&17.90$\pm$0.71	&8.52$\pm$0.60	&18.42$\pm$2.59	&16.50$\pm$4.34  &12.14\\ \midrule
        
        DHYPR & \hig{3}{86.81$\pm$1.60} & \hig{2}{92.32$\pm$3.72} & 20.93$\pm$2.41 &TO &TO &OOM/TO &OOM/TO &10.57\\
        
        DiGAE & 82.06$\pm$2.51 & 83.64$\pm$3.21 & 55.05$\pm$2.36 & 41.55$\pm$1.62 & 29.21$\pm$1.36 &\hig{2}{41.95$\pm$0.93} & \hig{3}{55.14$\pm$1.96} & \hig{2}{4.43}\\ \midrule
        
        SDGAE & \hig{1}{90.37$\pm$1.33} &\hig{1}{93.69$\pm$3.68} &\hig{2}{68.84$\pm$2.35} &\hig{1}{53.79$\pm$1.56} &\hig{2}{54.67$\pm$2.50} &\hig{1}{42.42$\pm$1.15} &\hig{2}{55.91$\pm$1.77} &\hig{1}{1.43}\\
        \bottomrule
        
    \end{tabular}}
\end{table*}

\section{New Benchmark: DirLinkBench}
%In this section, we introduce a new unified benchmark for the directed link prediction task, abbreviated as \textbf{DirLinkBench}. This benchmark comprises seven graphs of varying sizes sourced from various fields. We carefully select and report the baseline results on this benchmark.
%\subsection{Dataset}
In this section, we introduce a novel robust benchmark for directed link prediction, \textbf{DirLinkBench}, offering three principal advantages: 1) \underline{Comprehensive coverage}, incorporating seven real-world datasets spanning diverse domains and 15 baseline models; 2) \underline{Standardized evaluation}, establishing a unified framework for dataset splitting, feature initialization, and task settings to ensure fairness and reliability; and 3) \underline{Modular extensibility}, built on PyG~\cite{pyg} to enable easy integration of new datasets, model architectures, and configurable modules (e.g., feature initialization, decoder types, and negative sampling). We then detail the implementation of DirLinkBench.

\textbf{Dataset}. We select seven publicly available directed graphs from diverse domains. These datasets include two citation networks, Cora-ML~\cite{mccallum2000cora_ml,bojchevski2018cora_ml} and CiteSeer~\cite{sen2008citeseer}; two co-purchasing networks, Photo and Computers~\cite{shchur2018computerandphoto}; a weblink network, WikiCS~\cite{mernyei2020wiki}; and two social networks, Slashdot~\cite{ordozgoiti2020slash} and Epinions~\cite{massa2005epinion}. These directed graphs differ in both size and average degree. %ranging from CiteSeer with the smallest average degree (3.5) to WikiCS with the largest (51.3). a
Except for Slashdot and Epinions, each dataset includes original node features. We provide the statistical details and additional descriptions in Appendix~\ref{app_dataset_baseline}.
%Table~\ref{dataset_info} provides statistical details of these datasets, and additional descriptions are included in the Appendix. 
To establish standard evaluation conditions for link prediction methods, we preprocess these datasets by eliminating duplicate edges and self-loop links. Following~\cite{appnp,shchur2018computerandphoto}, we also removed isolated nodes and used the largest
weakly connected component. %As a result, all directed graphs in DirLinkBench are weakly connected.



%\subsection{Setup}
\textbf{Task setup}. %Unlike previous settings, such as those in~\cite{magnet,dpyg}, which define link prediction for directed graphs as multiple subtasks, 
%Unlike the multiple subtask setup, which defines directed link prediction as a multi-class classification problem.
We simplify directed link prediction to a binary classification task: the model determines whether a directed edge exists from $u$ to $v$. 
%For any directed edge $(u, v)$, the goal is to predict whether an edge exists from $u$ to $v$. 
Specifically, given a preprocessed directed graph $\gG$, we randomly split 15\% of edges for testing, 5\% for validation, and use the remaining 80\% for training, and ensure that the training graph $\gG^{\prime}$ remains weakly connected~\cite{dpyg}. For testing and validation, we sample an equal number of negative edges under the full graph $\gG$ visible, while for training, only the training graph $\gG^{\prime}$ is visible. To ensure fairness, we generate 10 random splits using fixed seeds, and all models share the same splits. The model learns from the training graphs and feature inputs to compute $p(u, v)$ for test edges.
%For a link prediction model, it learns from the training set and feature inputs, then computes $p(u, v)$ for the edges in the test set. 
Feature inputs are provided in three forms: original node features, in/out degrees from the training graph $\gG^{\prime}$~\cite{magnet}, or a random normal distribution matrix~\cite{duplex}.

%\subsection{Baseline}
\textbf{Baseline}. We carefully select 15 state-of-the-art baselines, including three embedding methods: STRAP~\cite{strap}, ODIN~\cite{odin}, ELTRA~\cite{eltra}; a basic method MLP; three classic undirected GNNs: GCN~\cite{gcn}, GAT~\cite{gat}, APPNP~\cite{appnp}; four single real-valued methods: DGCN~\cite{dgcn-tong}, DiGCN~\cite{digcn}, DiGCNIB~\cite{digcn}, DirGNN~\cite{dirgnn}; two complex-valued methods: MagNet~\cite{magnet}, DUPLEX~\cite{duplex}; a gravity-inspired method: DHYPR~\cite{dhypr}; and a source-target GNN: DiGAE~\cite{digae}. We exclude some recent approaches (e.g., CoBA~\cite{coba}, BLADE~\cite{blade}, NDDGNN~\cite{nddgnn}) due to unavailable code.
%Note that some recent methods, such as CoBA~\cite{coba}, BLADE~\cite{blade}, and NDDGNN~\cite{nddgnn}, are not included due to unavailable code. 

\textbf{Baseline Setting}. 
For the baseline implementations, we rely on the authors’ original released code or popular libraries like PyG and the PyTorch Geometric Signed Directed library~\cite{dpyg}. For methods without released link-prediction code (e.g., GCN, DGCN), we provide various decoders and loss functions, while for methods with available link-prediction code (e.g., MagNet, DiGAE), we strictly follow the reported settings. We tune hyperparameters using grid search, adhering to the configurations specified in each paper. Further details of each method are in Appendix~\ref{app_dataset_baseline}.
%For the implementation of baselines, we use the original codes released by the authors, and implementations from some popular libraries, such as PyG and the PyTorch Geometric Signed Directed library~\cite{dpyg}. %For other baselines, we utilize the original code released by the authors. 
%For methods without released code for the link prediction task, such as GCN, DGCN, etc., we designed different decoders and loss functions. For methods with available link prediction task codes, such as MagNet, DiGAE, etc., we strictly follow the reported settings. Hyperparameters were carefully selected using grid search, adhering to the configurations specified in the respective papers. More setting details are provided in the Appendix.

%\subsection{Results}
%\subsection{Metric}

\textbf{Results.} 
We report results on seven metrics—Hits@20, Hits@50, Hits@100, MRR, AUC, AP, and ACC (detailed metric descriptions are provided in Appendix~\ref{app_bench_metric}), with complete results for each dataset in Appendix~\ref{app_complete_res}. Table~\ref{tab:performance_comparison} highlights the Hits@100 results. These results first show that embedding methods retain a strong advantage even without feature inputs. Second, early single real-valued undirected and directed GNNs also perform competitively, while newer directed GNNs (e.g., MagNet, DUPLEX, DHYPR) exhibit weaker performance or scalability issues. This is because complex-based methods have unsuitable loss functions and decoders, while DHYPR’s preprocessing requires $O(Kn^3)$ time and $O(Kn^2)$ space complexity. We provide further analysis in Sec.~\ref{analysis}. Notably, DiGAE, a simple directed graph auto-encoder, emerges as the best performer overall, yet it underperforms on certain datasets (e.g., Cora-ML, CiteSeer), leading to a worse average ranking. This observation motivates us to revisit DiGAE’s design and propose new methods to enhance directed link prediction.

%We report results in seven metrics: Hits@20, Hits@50, Hits@100, MRR, AUC, AP, and ACC (see Appendix~\ref{app_bench_metric} for these metrics' instructions).  See Appendix~\ref{app_complete_res} for complete results for each metric on each dataset.Table~\ref{tab:performance_comparison} shows the results for the Hits@100 metric. According to these results, we first observe that embedding methods retain a significant advantage without using feature inputs. Second, the early single real-valued undirected GNNs and directed GNNs also have competitive performance. In contrast, the latest directed GNNs show poor performance or scalability, such as DUPLEX and DHYPR. After our analysis, we found that the poor performance of complex-based methods (e.g., MagNet, DUPLEX) is mainly due to their unsuitable loss function and decoder, while the unscalability of DHYPR is due to its preprocessing stage with $O(Kn^3)$ time complexity and $O(Kn^2)$ space complexity. We will further analyze these results in Section 5.  Notably, the best-performing baseline is DiGAE, a simple graph auto-encoder for directed graphs. Although it is the best overall performer, it is worse on certain datasets, such as Cora-ML and CiteSeer, resulting in a poor average ranking. This motivates us to revisit the model design of DiGAE to propose new methods for improving the performance of directed link prediction.


%We report results in seven metrics: Hits@20, Hits@50, Hits@100, MRR, AUC, AP, and ACC (see the Appendix for details). Table~\ref{tab:performance_comparison} highlights the results for the Hits@100 metric. Notably, we observe that embedding methods retain a significant advantage, even without using node features, while traditional undirected graph GNNs also perform well. Surprisingly, the latest GNNs designed specifically for directed graphs show no clear advantage, prompting a re-evaluation of their current design.

%In the following sections, we will first analyze the best performing GNN method, DiGAE~\cite{digae}, from the spectral domain and propose a new method, the Spectral Directed Graph Autoencoder. We will then examine the trends of existing results in detail, provide key observations, and pose several open questions.





%\textbf{Dataset}:
%We select seven directed graphs from various domains to perform the link prediction task. These include the citation networks CoraML and Citeseer, the co-purchasing networks Computer and Photo, the Weblink network WikiCS, and the social networks Slashdot and Epinions. We show the statistical details of these datasets in Table~\ref{dataset_info}.






\section{New Method: SDGAE}
In this section, we first revisit the mode structure of DiGAE and analyze its encoder graph convolution. We then propose a novel Spectral Directed Graph Auto-Encoder (SDGAE).

\subsection{Understand the Graph Convolution of DiGAE}
DiGAE~\cite{digae} is a graph auto-encoder designed for directed graphs. Its encoder graph convolutional layer is denoted as %is inspired by GCN~\cite{gcn} and denoted as
\begin{align}
\mS^{(\ell+1)} &= \sigma\left(\hat{\mD}_{\rm out}^{-\beta}\hat{\mA}\hat{\mD}_{\rm in}^{-\alpha}\mT^{(\ell)}\mW_{T}^{(\ell)}\right), \label{digae_s}\\
\mT^{(\ell+1)} &= \sigma\left(\hat{\mD}_{\rm in}^{-\alpha}\hat{\mA}^{\top}\hat{\mD}_{\rm out}^{-\beta}\mS^{(\ell)}\mW_{S}^{(\ell)}\right).\label{digae_t}
\end{align}
Here, $\hat{\mA} = \mA + \mI$ denotes the adjacency matrix with added self-loops, and $\hat{\mD}_{\rm out}$ and $\hat{\mD}_{\rm in}$ represent the corresponding out-degree and in-degree matrices, respectively. 
$\mS^{(\ell)}$ and $\mT^{(\ell)}$ denote the source and target embeddings at the $\ell$-th layer, initialized as $\mS^{(0)} = \mT^{(0)} = \mX$.
The hyperparameters $\alpha$ and $\beta$ are degree-based normalization factors, $\sigma$ is the activation function (e.g., $\mathrm{ReLU}$), and $\mW_{T}^{(\ell)}, \mW_{S}^{(\ell)}$ represents the learnable weight matrices.

%The motivation for this graph convolution stems from GCN~\cite{gcn} and the 1-WL~\cite{wl} algorithm connection to directed graphs. However, its underlying principles in the spectral domain remain unexplained, and the heuristic hyperparameters $\alpha$ and $\beta$ introduce significant challenges in tuning.

The design of this graph convolution is inspired by the connection between GCN~\cite{gcn} and the 1-WL~\cite{wl} algorithm for directed graphs. However, the underlying principles of this convolution remain unexplored,  leaving its meaning unclear. Additionally, the heuristic hyperparameters $\alpha$ and $\beta$ introduce significant challenges for parameter optimization.

\begin{figure}[t]
    \centering
   \includegraphics[width=75mm]{figure/sdgconv_demo.pdf}
   \vspace{-4mm}
    \caption{The bipartite representations of two toy directed graphs.}
    \label{fig:bg}
    \vspace{-1mm}
\end{figure}


\begin{figure*}[t]
    \centering
   \vspace{-3mm}
   \hspace{-3mm}
   \subfigure{
   \includegraphics[width=37mm]{fig_degree/degree_WikiCS.pdf}
   }
   \hspace{-7mm}
   \subfigure{
   \includegraphics[width=37mm]{fig_degree/degree_WikiCS-STRAP.pdf}
   }
   \hspace{-7mm}
   \subfigure{
   \includegraphics[width=37mm]{fig_degree/degree_WikiCS-DirGNN.pdf}
   }
   \hspace{-7mm}
   \subfigure{
   \includegraphics[width=37mm]{fig_degree/degree_WikiCS-MagNet.pdf}
   }
   \hspace{-7mm}
   \subfigure{
   \includegraphics[width=37mm]{fig_degree/degree_WikiCS-SDGAE.pdf}
   }
   
   \vspace{-3mm}
   \caption{Degree distribution of WikiCS graph and its reconstruction graphs.}
   \vspace{-3mm}
   \label{fig:degree_wiki}
\end{figure*}
%Given a directed graph $\gG$, its bipartite representation is denoted as $\mathcal{S}(\hat{\mA})$~\cite{book_digraph}. Specifically,

We revisit the graph convolution of DiGAE and observe it corresponds to the GCN convolution~\cite{gcn} applied to an undirected bipartite graph. Specifically, given a directed graph $\gG$, its bipartite representation~\cite{book_digraph} is expressed as:
\begin{equation}
    \mathcal{S}(\hat{\mA}):=
    \left[ 
        \begin{array}{cc}
            \mathbf{0} & \hat{\mA} \\
            \hat{\mA}^{\top} & \mathbf{0}
        \end{array}
    \right] \in \mathbb{R}^{2n \times 2n}.
\end{equation}
Here, $\mathcal{S}(\hat{\mA})$ is a block matrix consisting of $\hat{\mA}$ and $\hat{\mA}^{\top}$. It is evident that $\mathcal{S}(\hat{\mA})$ represents the adjacency matrix of an undirected bipartite graph with $2n$ nodes. Figure~\ref{fig:bg} provides two toy examples of directed graphs and their corresponding undirected bipartite graph representations. Notably, the self-loop in $\hat{\mA}$ serves a fundamentally different purpose than in GCN~\cite{gcn}. In this context, the self-loop ensures the connectivity of the bipartite graph. Without the self-loop, as illustrated in graphs (b) and (e) of Figure~\ref{fig:bg}, the graph structure suffers from significant connectivity issues. Based on this, we present the following lemma.

%It can be observed that $\mathcal{S}(\hat{\mA})$ represents the adjacency matrix of a bipartite graph with $2n$ nodes. 
%Figure~\ref{fig:bg} illustrates examples of two toy directed graphs turned into such undirected bipartite graphs. Notably, the self-loop in $\hat{\mA}$ serves a purpose fundamentally different from the one in GCN. Here, the self-loop ensures the connectivity of the bipartite graph. Without the self-loop, as seen in graphs (b) and (e) in Fig.~\ref{fig:bg}, the graph structure suffers from severe connectivity issues. Then, we present the following lemma to understand DiGAE.


\begin{lemma}\label{lemma_digae_conv}
When omitting degree-based normalization in Equations (\ref{digae_s}) and (\ref{digae_t}), the graph convolution of DiGAE is
\begin{equation*}
\left[\mS^{(\ell+1)},\mT^{(\ell+1)}\right]^{\top} =\sigma\left(\mathcal{S}(\hat{\mA})\left[\mS^{(\ell)}\mW_{S}^{(\ell)},\mT^{(\ell)}\mW_{T}^{(\ell)}\right]^{\top} \right).
\end{equation*}
\end{lemma}
The proof and its extension with degree-based normalization are provided in Appendix~\ref{app_proof_lemma_digae_conv}. Lemma~\ref{lemma_digae_conv} and extension in its proof reveal that the graph convolution of DiGAE essentially corresponds to the GCN convolution applied to the undirected bipartite graph $\mathcal{S}(\hat{\mA})$.

%, similar to the convolution mechanism in GCN~\cite{gcn}.



\subsection{Spectral Directed Graph Auto-Encoder (SDGAE)}
%Motivated by the understanding of DiGAE, we propose a new Spectral Directed Graph Auto-Encoder (SDGAE), which uses the polynomial weights of spectral-based GNNs~\cite{gprgnn,bernnet}. The convolutional layer of SDGAE is denoted as 

Building on our understanding of DiGAE, we identify two main drawbacks. 1) DiGAE struggles to use deep networks for capturing richer structural information due to excessive learnable weight matrices (as highlighted in a recent study on deep GCNs~\cite{pengover}). We validate this issue experimentally in the next section. 2) It is difficult to optimize parameters due to heuristic hyperparameters. To address these issues, we propose SDGAE, which uses polynomial weights inspired by spectral-based GNNs~\cite{gprgnn,bernnet} and incorporates symmetric normalization of the adjacency matrix. The convolutional layer of SDGAE is defined as: $\left[\mS^{(\ell+1)},\mT^{(\ell+1)}\right]^{\top} =$
\begin{equation}\label{eq:bg_A_hat}
    \left[ 
        \begin{array}{c}
            w^{(\ell)}_S \\
            w^{(\ell)}_T
        \end{array}
    \right] 
    \odot
    \left[ 
        \begin{array}{cc}
            \mathbf{0} & \Tilde{\mA} \\
            \Tilde{\mA}^{\top} & \mathbf{0}
        \end{array}
    \right]
    \left[ 
        \begin{array}{c}
            \mS^{(\ell)} \\
            \mT^{(\ell)}
        \end{array}
    \right] 
    +
    \left[ 
        \begin{array}{c}
            \mS^{(\ell)} \\
            \mT^{(\ell)}
        \end{array}
    \right],
\end{equation}
where $w^{(\ell)}_S,w^{(\ell)}_T \in \mathbb{R}$ are learnable scalar weights, initialized to one. 
$\Tilde{\mA}=\hat{\mD}_{\rm out}^{-1/2}\hat{\mA}\hat{\mD}_{\rm in}^{-1/2}$ denotes the normalized adjacency matrix, and lemma~\ref{le:norm} demonstrates that this normalization corresponds to the symmetric normalization of $\mathcal{S}(\hat{\mA})$. Each layer of this graph convolution performs weighted propagation over the normalized undirected bipartite graph $\mathcal{S}(\Tilde{\mA})$, with the addition of a residual connection.
%note that the below lemma~\ref{le:norm} understands that this regularization is essentially regularizing the adjacency matrix $\mathcal{S}(\hat{\mA})$ of the undirected graph. 
\begin{lemma}\label{le:norm}
    The symmetrically normalized block adjacency matrix $ \mD_{\mathcal{S}}^{-1/2} \mathcal{S}(\hat{\mA}) \mD_{\mathcal{S}}^{-1/2} = \mathcal{S}(\Tilde{\mA})$, where $\mD_{\mathcal{S}} = \mathrm{diag}(\hat{\mD}_{\rm out}, \hat{\mD}_{\rm in})$ is the diagonal degree matrix of \(\mathcal{S}(\hat{\mA})\).
\end{lemma}
The proof of Lemma~\ref{le:norm} is provided in Appendix~\ref{app_proof_le_norm}. For the initialization of $\mS^{(0)}$ and $\mT^{(0)}$ in the implementation, we follow spectral-based GNNs~\cite{gprgnn} and use two MLPs: $\mS^{(0)} = \mathrm{MLP}_S(\mX)$ and $\mT^{(0)} = \mathrm{MLP}_T(\mX)$. For the decoder, we apply the inner product $\sigma(\vs_u | \vt_v)$, as well as MLP scoring functions $\mathrm{MLP}(\vs_u \odot \vt_v)$ and $\mathrm{MLP}(\vs_u | \vt_v)$.

Table~\ref{tab:performance_comparison} presents SDGAE's results on DirLinkBench for the Hits@100 metric, where it achieves state-of-the-art performance on four of the seven datasets, while delivering competitive results on the remaining three. Furthermore, SDGAE significantly outperforms DiGAE. These results demonstrate the effectiveness of SDGAE for the directed link prediction task. Additional details and experimental settings for SDGAE are provided in Appendix~\ref{app_sdgae}.

%The proof of lemma~\ref{le:norm} is in Appendix~\ref{app_proof_le_norm}. In the implementation of SDGAE, for the initialization $\mS^{(0)}$ and $\mT^{(0)}$, we use two MLPs following spectral-based GNNs~\cite{gprgnn}, i.e., $\mS^{(0)}=\mathrm{MLP}_S(\mX)$ and $ \mT^{(0)}=\mathrm{MLP}_T(\mX)$. For the decoder function, we use
%inner product $\sigma(\vs_u \| \vt_v)$ and MLP score $\mathrm{MLP}(\vs_u \odot \vt_v)$, $\mathrm{MLP}(\vs_u \| \vt_v)$. In Table~\ref{tab:performance_comparison} we show the results of SDGAE on DirLinkBench; SDGAE achieves SOTA on top of four of the seven datasets, the other three achieve competitive results and SDGAE significantly outperforms of DiGAE. These results illustrates the effectiveness of SDGAE on directed connectivity prediction. We show more details and experimental setting of SDGAE in the appendix.


%This choice of decoder function differs from DiGAE, 



%and we will analyze the impact of various decoders in the next section.


%For the initialization $\mS^{(0)}$ and $\mT^{(0)}$, we use two MLPs following~\cite{appnp}, that is
%\begin{equation}
%    \mS^{(0)}=\mathrm{MLP}_S(\mX), \quad \mT^{(0)}=\mathrm{MLP}_T(\mX).
%\end{equation}
%The intuition behind SDGAE is to define a spectral domain graph convolution on the bipartite graph $\mathcal{S}(\hat{\mA})$ to generate source and target embeddings. This design allows us to use deeper propagation layers and simplifies adjacency matrix regularization. For the decoder function, we use $\sigma(\mathrm{MLP}(\vs_u \odot \vt_v))$ and $\sigma(\mathrm{MLP}(\vs_u \| \vt_v))$, where $\sigma$ represents the activation function (i.e., $\mathrm{Sigmoid}$). This choice of decoder function differs from DiGAE, and we will analyze the impact of various decoders in the next section. 

%Notably, most previous works have prioritized the design of the encoder while paying less attention to the decoder. We list more details of SDGAE in the Appendix.



\section{Analysis}
\label{sec:analysis}
In the following sections, we will analyze European type approval regulation\footnote{Strictly speaking, the German enabling act (AFGBV) does not regulate type-approval, but how test \& operating permits are issued for SAE-Level-4 systems. Type-approval regulation for SAE-Level-3 systems follows UN Regulation No. 157 (UN-ECE-ALKS) \parencite{un157}.} regarding the underlying notions of ``safety'' and ``risk''.
We will classify these notions according to their absolute or relative character, underlying risk sources, or underlying concepts of harm.

\subsection{Classification of Safety Notions}
\label{sec:safety-notions}
We will refer to \emph{absolute} notions of safety as conceptualizations that assume the complete absence of any kind of risk.
Opposed to this, \emph{relative} notions of safety are based on a conceptualization that specifically includes risk acceptance criteria, e.g., in terms of ``tolerable'' risk or ``sufficient'' safety.

For classifying notions of safety by their underlying risk (or rather ``hazard'') sources, and different concepts of harm, \Cref{fig:hazard-sources} provides an overview of our reasoning, which is closely in line with the argumentation provided by Waymo in \parencite{favaro2023}.
We prefer ``hazard sources'' over ``risk sources'', as a risk must always be related to a \emph{cause} or \emph{source of harm} (i.e., a hazard \parencite[p.~1, def. 3.2]{iso51}).
Without a concrete (scenario) context that the system is operating in, a hazard is \emph{latent}: E.g., when operating in public traffic, there is a fundamental possibility that a \emph{collision with a pedestrian} leads to (physical) harm for that pedestrian. 
However, only if an automated vehicle shows (potentially) hazardous behavior (e.g., not decelerating properly) \emph{and} is located near a pedestrian (context), the hazard is instantiated and leads to a hazardous event.
\begin{figure*}
    \includeimg[width=.9\textwidth]{hazard-sources0.pdf}
    \caption{Graphical summary of a taxonomy of risk related to automated vehicles, extended based on ISO 21448 (\parencite{iso21448}) and \parencite{favaro2023}. Top: Causal chain from hazard sources to actual harm; bottom: summary of the individual elements' contributions to a resulting risk. Graphic translated from \parencite{nolte2024} \label{fig:hazard-sources}}
\end{figure*}
If the hazardous event cannot be mitigated or controlled, we see a loss event in which the pedestrian's health is harmed.
Note that this hypothetical chain of events is summarized in the definition of risk:
The probability of occurrence of harm is determined by a) the frequency with which hazard sources manifest, b) the time for which the system operates in a context that exposes the possibility of harm, and c) by the probability with which a hazardous event can be controlled.
A risk can then be determined as a function of the probability of harm and the severity of the harm potentially inflicted on the pedestrian.

In the following, we will apply this general model to introduce different types of hazard sources and also different types of harm.
\cref{fig:hazard-sources} shows two distinct hazard sources, i.e., functional insufficiencies and E/E-failures that can lead to hazardous behavior.
ISO~21488 \parencite{iso21448} defines functional insufficiencies as insufficiencies that stem from an incomplete or faulty system specification (specification insufficiencies).
In addition, the standard considers insufficiencies that stem from insufficient technical capability to operate inside the targeted Operational Design Domain (performance insufficiencies).
Functional insufficiencies are related to the ``Safety of the Intended Functionality (SOTIF)'' (according to ISO~21448), ``Behavioral Safety'' (according to Waymo \parencite{waymo2018}), or ``Operational Safety'' (according to UN Regulation No. 157 \parencite{un157}).
E/E-Failures are related to classic functional safety and are covered exhaustively by ISO~26262 \parencite{iso2018}.
Additional hazard sources can, e.g., be related to malicious security attacks (ISO~21434), or even to mechanical failures that should be covered (in the US) in the Federal Motor Vehicle Safety Standards (FMVSS).

For the classification of notions of safety by the related harm, in \parencite{salem2024, nolte2024}, we take a different approach compared to \parencite{koopman2024}:
We extend the concept of harm to the violation of stakeholder \emph{values}, where values are considered to be a ``standard of varying importance among other such standards that, when combined, form a value pattern that reduces complexity for stakeholders [\ldots] [and] determines situational actions [\ldots].'' \parencite{albert2008}
In this sense, values are profound, personal determinants for individual or collective behavior.
The notion of values being organized in a weighted value pattern shows that values can be ranked according to importance.
For automated vehicles, \emph{physical wellbeing} and \emph{mobility} can, e.g., be considered values which need to be balanced to achieve societal acceptance, in line with the discussion of required tradeoffs in \cref{sec:terminology}.
For the analysis of the following regulatory frameworks, we will evaluate if the given safety or risk notions allow tradeoffs regarding underlying stakeholder values. 

\subsection{UN Regulation No. 157 \& European Implementing Regulation (EU) 2022/1426}
\label{sec:enabling-act}
UN Regulation No. 157 \parencite{un157} and the European Implementing Regulation 2022/1426 \parencite{eu1426} provide type approval regulation for automated vehicles equipped with SAE-Level-3 (UN Reg. 157) and Level 4 (EU 2022/1426) systems on an international (UN Reg. 157) and European (EU 2022/1426) level.

Generally, EU type approval considers UN ECE regulations mandatory for its member states ((EU) 2018/858, \parencite{eu858}), while the EU largely forgoes implementing EU-specific type approval rules, it maintains the right to alter or to amend UN ECE regulation \parencite{eu858}.

In this respect, the terminology and conceptualizations in the EU Implementing Act closely follow those in UN Reg. No. 157.
The EU Implementing Act gives a clear reference to UN Reg. No. 157 \parencite[][Preamble,  Paragraph 1]{eu1426}.
Hence, the documents can be assessed in parallel.
Differences will be pointed out as necessary.

Both acts are written in rather technical language, including the formulation of technical requirements (e.g., regarding deceleration values or speeds in certain scenarios).
While providing exhaustive definitions and terminology, neither of both documents provide an actual definition of risk or safety.
The definition of ``unreasonable'' risk in both documents does not define risk, but only what is considered \emph{unreasonable}. It states that the ``overall level of risk for [the driver, (only in UN Reg. 157)] vehicle occupants and other road users which is increased compared to a competently and carefully driven manual vehicle.''
The pertaining notions of safety and risk can hence only be derived from the context in which they are used.

\subsubsection{Absolute vs. Relative Notions of Safety}
In line with the technical detail provided in the acts, both clearly imply a \emph{relative} notion of safety and refer to the absence of \emph{unreasonable} risk throughout, which is typical for technical safety definitions.

Both acts require sufficient proof and documentation that the to-be-approved automated driving systems are ``free of unreasonable safety risks to vehicle occupants and other road users'' for type approval.\footnote{As it targets SAE-Level-3 systems, UN Reg. 157 also refers to the driver, where applicable.}
In this respect, both acts demand that the manufacturers perform verification and validation activities for performance requirements that include ``[\ldots] the conclusion that the system is designed in such a way that it is free from unreasonable risks [\ldots]''.
Additionally, \emph{risk minimization} is a recurring theme when it comes to the definition of Minimum Risk Maneuvers (MRM).

Finally, supporting the relative notions of safety and risk, UN Reg. 157 introduces the concept of ``reasonable foreseeable and preventable'' \parencite[Article 1, Clause 5.1.1.]{un157} collisions, which implies that a residual risk will remain with the introduction of automated vehicles.
\parencite[][Appendix 3, Clause 3.1.]{un157} explicitly states that only \emph{some} scenarios that are unpreventable for a competent human driver can actually be prevented by an automated driving system.
While this concept is not applied throughout the EU Implementing Act, both documents explicitly refer to \emph{residual} risks that are related to the operation of automated driving systems (\parencite[][Annex I, Clause 1]{un157}, \parencite[][Annex II, Clause 7.1.1.]{eu1426}).

\subsubsection{Hazard Sources}
Hazard sources that are explicitly differentiated in UN Reg. 157 and (EU) 2022/1426 are E/E-failures that are in scope of functional safety (ISO~26262) and functional insufficiencies that are in scope of behavioral (or ``operational'') safety (ISO~21448).
Both documents consistently differentiate both sources when formulating requirements.

While the acts share a common definition of ``operational'' safety (\parencite[][Article 2, def. 30.]{eu1426}, \parencite[][Annex 4, def. 2.15.]{un157}), the definitions for functional safety differ.
\parencite{un157} defines functional safety as the ``absence of unreasonable risk under the occurrence of hazards caused by a malfunctioning behaviour of electric/electronic systems [\ldots]'', \parencite{eu1426} drops the specification of ``electric/electronic systems'' from the definition.
When taken at face value, this definition would mean that functional safety included all possible hazard sources, regardless of their origin, which is a deviation from the otherwise precise usage of safety-related terminology.

\subsubsection{Harm Types}
As the acts lack explicit definitions of safety and risk, there is no consistent and explicit notion of different harm types that could be differentiated.

\parencite{un157} gives little hints regarding different considered harm types.
``The absence of unreasonable risk'' in terms of human driving performance could hence be related to any chosen performance metric that allows a comparison with a competent careful human driver including, e.g., accident statistics, statistics about rule violations, or changes in traffic flow.

In \parencite{eu1426}, ``safety'' is, implicitly, attributed to the absence of unreasonable risk to life and limb of humans.
This is supported by the performance requirements that are formulated:
\parencite[][Annex II, Clause 1.1.2. (d)]{eu1426} demands that an automated driving system can adapt the vehicle behavior in a way that it minimizes risk and prioritizes the protection of human life.

Both acts demand the adherence to traffic rules (\parencite[][Annex 2, Clause 1.3.]{eu1426}, \parencite[][Clause 5.1.2.]{un157}).
\parencite[][Annex II, Clause 1.1.2. (c)]{eu1426} also demands that an automated driving system shall adapt its behavior to surrounding traffic conditions, such as the current traffic flow.
With the relative notion of risk in both acts, the unspecific clear statement that there may be unpreventable accidents \parencite{un157}, and a demand of prioritization of human life in \parencite{eu1426}, both acts could be interpreted to allow developers to make tradeoffs as discussed in \cref{sec:terminology}.


\subsubsection{Conclusion}
To summarize, the UN Reg. 157 and the (EU) 2022/1426 both clearly support the technical notion of safety as the absence of unreasonable risk.
The notion is used consistently throughout both documents, providing a sufficiently clear terminology for the developers of automated vehicles.
Uncertainty remains when it comes to considered harm types: Both acts do not explicitly allow for broader notions of safety, in the sense of \parencite{koopman2024} or \parencite{salem2024}.
Finally, a minor weak spot can be seen in the definition of risk acceptance criteria: Both acts take the human driving performance as a baseline.
While (EU) 2022/1426 specifies that these criteria are specific to the systems' Operational Design Domain \parencite[][Annex II, Clause 7.1.1.]{eu1426}, the reference to the concrete Operational Design Domain is missing in UN Reg. 157.
Without a clearly defined notion of safety, however, it remains unclear, how aspects beyond net accident statistics (which are given as an example in \parencite[][Annex II, Clause 7.1.1.]{eu1426}), can be addressed practically, as demanded by \parencite{koopman2024}.

\subsection{German Regulation (StVG \& AFGBV)}
\label{sec:afgbv}
The German L3 (Automated Driving Act) and L4 (Act on Autonomous Driving) Acts from 2017 and 2021,\footnote{Formally, these are amendments to the German Road Traffic Act (StVG): 06/21/2017, BGBl. I p. 1648, 07/12/2021 BGBl. I p. 3108.} respectively, provide enabling regulation for the operation of SAE-Level-3 and 4 vehicles on German roads.
The German Implementing Regulation (\parencite{afgbv}, AFGBV) defines how this enabling regulation is to be implemented for granting testing permits for SAE-Level-3 and -4 and driving permits for SAE-Level-3 and -4 automated driving systems.\footnote{Note that these permits do not grant EU-wide type approval, but serve as a special solution for German roads only. At the same time, the AFGBV has the same scope as (EU) 2022/1426.}
With all three acts, Germany was the first country to regulate the approval of automated vehicles for a domestic market.
All acts are subject to (repeated) evaluation until the year 2030 regarding their impact on the development of automated driving technology.
An assessment of the German AFGBV and comparisons to (EU) 2022/1426 have been given in \cite{steininger2022} in German.

Just as for UN Reg. 157 and (EU) 2022/1426, neither the StVG nor the AFGBV provide a clear definition of ``safety'' or ``risk'' -- even though the "safety" of the road traffic is one major goal of the StVG and StVO.
Again, different implicit notions of both concepts can only be interpreted from the context of existing wording.
An additional complication that is related to the German language is that ``safety'' and ``security'' can both be addressed as ``Sicherheit'', adding another potential source of unclarity.
Literal Quotations in this section are our translations from the German act.

\subsubsection{Absolute vs. Relative Notions of Safety}
For assessing absolute vs. relative notions of safety in German regulation, it should be mentioned that the main goal of the German StVO is to ensure the ``safety and ease of traffic flow'' -- an already diametral goal that requires human drivers to make tradeoffs.\footnote{For human drivers, this also creates legal uncertainty which can sometimes only be settled in a-posteriori court cases.}
While UN and EU regulation clearly shows a relative notion of safety\footnote{And even the StVG contains sections that use wording such as ``best possible safety for vehicle occupants'' (§1d (4) StVG) and acknowledges that there are unavoidable hazards to human life (§1e (2) No. 2c)).}, the German AFGBV contains ambiguous statements in this respect:
Several paragraphs contain a demand for a hazard free operation of automated vehicles.
§4 (1) No. 4 AFGBV, e.g., states that ``the operation of vehicles with autonomous driving functions must neither negatively impact road traffic safety or traffic flow, nor endanger the life and limb of persons.''
Additionally, §6 (1) AFGBV states that the permits for testing and operation have to be revoked, if it becomes apparent that a ``negative impact on road traffic safety or traffic flow, or hazards to the life and limb of persons cannot be ruled out''.
The same wording is used for the approval of operational design domains regulated in §10 (1) No. 1.
A particularly misleading statement is made regarding the requirements for technical supervision instances which are regulated in §14 (3) AFGBV which states that an automated vehicle has to be  ``immediately removed from the public traffic space if a risk minimal state leads to hazards to road traffic safety or traffic flow''.
Considering the argumentation in \cref{sec:terminology}, that residual risks related to the operation of automated driving systems are inevitable, these are strong statements which, if taken at face value, technically prohibit the operation of automated vehicles.
It suggests an \emph{absolute} notion of safety that requires the complete absence of risk.  
The last statement above is particularly contradictory in itself, considering that a risk \emph{minimal} state always implies a residual risk.

In addition to these absolute safety notions, there are passages which suggest a relative notion of safety:
The approval for Operational Design Domains is coupled to the proof that the operation of an automated vehicle ``neither negatively impacts road traffic safety or traffic flow, nor significantly endangers the life and limb of persons beyond the general risk of an impact that is typical of local road traffic'' (§9 (2) No. 3 AFGBV).
The addition of a relative risk measure ``beyond the general risk of an impact'' provides a relaxation (cf. also \cite{steininger2022}, who criticizes the aforementioned absolute safety notion) that also yields an implicit acceptance criterion (\emph{statistically as good as} human drivers) similar to the requirements stated in UN Reg. 157 and (EU) 2022/1426.

Additional hints for a relative notion of safety can be found in Annex 1, Part 1, No. 1.1 and Annex 1, Part 2, No. 10.
Part 1, No 1.1 specifies collision-avoidance requirements and acknowledges that not all collisions can be avoided.\footnote{The same is true for Part 2, No. 10, Clause 10.2.5.}
Part 2, No. 10 specifies requirements for test cases.
It demands that test cases are suitable to provide evidence that the ``safety of a vehicle with an autonomous driving function is increased compared to the safety of human-driven vehicles''.
This does not only acknowledge residual risks, but also yields an acceptance criterion (\emph{better} than human drivers) that is different from the implied acceptance criterion given in §9 (2) No. 3 AFGBV.

\subsubsection{Hazard Sources}
Regarding hazard sources, Annex 1 and 3 AFGBV explicitly refer to ISO~26262 and ISO~21448 (or rather its predecessor ISO/PAS~21448:2019).
However, regarding the discussion of actual hazard sources, the context in which both standards are mentioned is partially unclear:
Annex 1, Clause 1.3 discusses requirements for path and speed planning.
Clause 1.3 d) demands that in intersections, a Time to Collision (TTC) greater than 3 seconds must be guaranteed.
If manufacturers deviate from this, it is demanded that ``state-of-the-art, systematic safety evaluations'' are performed.
Fulfillment of the state of the art is assumed if ``the guidelines of ISO~26262:2018-12 Road Vehicles -- Functional Safety are fulfilled''.
Technically, ISO~26262 is not suitable to define the state of the art in this context, as the requirements discussed fall in the scope of operational (or behavioral) safety (ISO~21448).
A hazard source ``violated minimal time to collision'' is clearly a functional insufficiency, not an E/E-failure.

Similar unclarity presents itself in Annex 3, Clause 1 AFGBV: 
Clause 1 specifies the contents of the ``functional specification''.
The ``specification of the functionality'' is an artifact which is demanded in ISO~21448:2022 (Clause 5.3) \parencite{iso21448}.
However, Annex 3, Clause 1 AFGBV states that the ``functional specification'' is considered to comply to the state of the art, if the ``functional specification'' adheres to ISO~26262-3:2018 (Concept Phase).
Again, this assumes SOTIF-related contents as part of ISO~26262, which introduces the ``Item Definition'' as an artifact, which is significantly different from the ``specification of the functionality'' which is demanded by ISO~21448.
Finally, Annex 3, Clause 3 AFGBV demands a ``documentation of the safety concept'' which ``allows a functional safety assessment''.
A safety concept that is related to operational / behavioral safety is not demanded.
Technically, the unclarity with respect to the addressed harm types lead to the fact that the requirements provided by the AFGBV do not comply with the state of the art in the field, providing questionable regulation.

\subsubsection{Harm Types}
Just like UN Reg. 157 and (EU) 2022/1426, the German StVG and AFGBV do not explicitly differentiate concrete harm types for their notions of safety.
However, the AFGBV mentions three main concerns for the operation of automated vehicles which are \emph{traffic flow} (e.g., §4 (1) No. 4 AFGBV), compliance to \emph{traffic law} (e.g., §1e (2) No. 2 StVG), and the \emph{life and limb of humans} (e.g., §4 (1) No. 4 AFGBV).

Again, there is some ambiguity in the chosen wording:
The conflict between traffic flow and safety has already been argued in \cref{sec:terminology}.
The wording given in §4 (1) No. 4 and §6 (1) AFGBV  demand to ensure (absolute) safety \emph{and} traffic flow at the same time, which is impossible (cf. \cref{sec:terminology}) from an engineering perspective.
§1e (2) No. 2 StVG defines that ``vehicles with an autonomous driving function must [\ldots] be capable to comply to [\ldots] traffic rules in a self-contained manner''.
Taken at face value, this wording implies that an automated driving system could lose its testing or operating permit as soon as it violates a traffic rule.
A way out could be provided by §1 of the German Traffic Act (StVO) which demands careful and considerate behavior of all traffic participants and by that allows judgement calls for human drivers.
However, if §1 is applicable in certain situations is often settled in court cases. 
For developers, the application of §1 StVO during system design hence remains a legal risk.

While there are rather absolute statements as mentioned above, sections of the AFGBV and StVG can be interpreted to allow tradeoffs:
§1e (2) No. 2 b) demands that a system,  ``in case of an inevitable, alternative harm to legal objectives, considers the significance of the legal objectives, where the protection of human life has highest priority''.
This exact wording \emph{could} provide some slack for the absolute demands in other parts of the acts, enabling tradeoffs between (tolerable) risk and mobility as discussed in \cref{sec:terminology}.
However, it remains unclear if this interpretation is legally possible.

\subsubsection{Conclusion}
Compared to UN Reg. 157 and (EU) 2022/1426, the German StVG and AFGBV introduce openly inconsistent notions of safety and risk which are partially directly contradictory:
The wording partially implies absolute and relative notions of safety and risk at the same time.
The implied validation targets (``better'' or ``as good as'' human drivers) are equally contradictory. 
The partially implied absolute notions of safety, when taken at face value, prohibit engineers from making the tradeoffs required to develop a system that is safe and provides customer benefit at the same time. 
In consequence, the wording in the acts is prone to introducing legal uncertainty.
This uncertainty creates additional clarification need and effort for manufacturers and engineers who design and develop SAE-Level-3 and -4 automated driving systems. The use of undefined legal terms not only makes it more difficult for engineers to comply with the law, but also complicates the interpretation of the law and leads to legal uncertainty.

\subsection{UK Automated Vehicles Act 2024 (2024 c. 10)}
The UK has issued a national enabling act for regulating the approval of automated vehicles on the roads in the UK.
To the best of our knowledge, concrete implementing regulation has not been issued yet.
Regarding terminology, the act begins with a dedicated terminology section to clarify the terms used in the act \parencite[Part 1, Chapter 1, Section 1]{ukav2024}.
In that regard, the act defines a vehicle to drive ```autonomously' if --- (a)
it is being controlled not by an individual but by equipment of the vehicle, and (b) neither the vehicle nor its surroundings are being monitored by an individual with a view to immediate intervention in the driving of the vehicle.''
The act hence covers SAE-Level-3 to SAE-Level-5 automated driving systems.

\subsubsection{Absolute vs. Relative Notions of Safety}
While not providing an explicit definition of safety and risk, the UK Automated Vehicles Act (``UK AV Act'') \parencite{ukav2024} explicitly refers to a relative notion of safety.
Part~1, Chapter~1, Section~1, Clause (7)~(a) defines that an automated vehicle travels ```safely' if it travels to an acceptably safe standard''.
This clarifies that absolute safety is not achievable and that acceptance criteria to prove the acceptability of residual risk are required, even though a concrete safety definition is not given.
The act explicitly tasks the UK Secretary of State\footnote{Which means, that concrete implementation regulation needs to be enacted.} to install safety principles to determine the ``acceptably safe standard'' in Part~1, Chapter~1, Section~1, Clause (7)~(a).
In this respect, the act also provides one general validation target as it demands that the safety principles must ensure that ``authorized automated vehicles will achieve a level of safety equivalent to, or higher than, that of careful and competent human drivers''.
Hence, the top-level validation risk acceptance criterion assumed for UK regulation is ``\emph{at least as good} as human drivers''.

\subsubsection{Hazard Sources}
The UK AV Act contains no statements that could be directly related to different hazard sources.
Note that, in contrast to the rest of the analyzed documents, the UK AV Act is enabling rather than implementing regulation.
It is hence comparable to the German StVG, which does not refer to concrete hazard sources as well.

\subsubsection{Types of Harm}
Even though providing a clear relative safety notion, the missing definition of risk also implies a lack of explicitly differentiable types of harm.
Implicitly, three different types of harm can be derived from the wording in the act.
This includes the harm to life and limb of humans\footnote{Part~1, Chapter~3, Section~25 defines ``aggravated offence where death or serious injury occurs'' \parencite{ukav2024}.}, the violation of traffic rules\footnote{Part~1, Chapter~1, Clause~(7)~(b) defines that an automated vehicle travels ```legally' if it travels with an acceptably low risk of committing a traffic infraction''}, and the cause of inconvenience to the public \parencite[Part~1, Chapter~1, Section~58, Clause (2)~(d)]{ukav2024}.

The act connects all the aforementioned types of harm to ``risk'' or ``acceptable safety''.
While the act generally defines criminal offenses for providing ``false or misleading information about safety'', it also acknowledges possible defenses if it can be proven that ``reasonable precautions'' were taken and that ``due diligence'' was exercised to ``avoid the commission of the offence''.
This statement could enable tradeoffs within the scope of ``reasonable risk'' to the life and limb of humans, the violation of traffic rules, or to the cause of inconvenience to the public, as we argued in \cref{sec:terminology}.

\subsubsection{Conclusion}
From the set of reviewed documents, the current UK AV Act is the one with the most obvious relative notions of safety and risk and the one that seems to provide a legal framework for permitting tradeoffs.
In our review, we did not spot major inconsistency beyond a missing definitions of safety and risk\footnote{Note that with the Office for Product Safety and Standards (OPSS), there is a British government agency that maintains an exhaustive and widely focussed ``Risk Lexicon'' that provides suitable risk definitions. For us, it remains unclear, to what extent this terminology is assumed general knowledge in British legislation.}.
The general, relative notion of safety and the related alleged ability for designers to argue well-founded development tradeoffs within the legal framework could prove beneficial for the actual implementation of automated driving systems.
While the act thus appears as a solid foundation for the market introduction of automated vehicles, without accompanying implementing regulation, it is too early to draw definite conclusions.
\section{Conclusion}
In this work, we propose a simple yet effective approach, called SMILE, for graph few-shot learning with fewer tasks. Specifically, we introduce a novel dual-level mixup strategy, including within-task and across-task mixup, for enriching the diversity of nodes within each task and the diversity of tasks. Also, we incorporate the degree-based prior information to learn expressive node embeddings. Theoretically, we prove that SMILE effectively enhances the model's generalization performance. Empirically, we conduct extensive experiments on multiple benchmarks and the results suggest that SMILE significantly outperforms other baselines, including both in-domain and cross-domain few-shot settings.

 


\newpage
\clearpage
%\section*{Impact Statements:} This work advances machine learning methods for directed graph analysis, with potential applications in practice. There are many potential societal consequences of our work, none of which we feel must be specifically highlighted here.
\bibliography{ref}
\bibliographystyle{icml2025}


\subsection{Lloyd-Max Algorithm}
\label{subsec:Lloyd-Max}
For a given quantization bitwidth $B$ and an operand $\bm{X}$, the Lloyd-Max algorithm finds $2^B$ quantization levels $\{\hat{x}_i\}_{i=1}^{2^B}$ such that quantizing $\bm{X}$ by rounding each scalar in $\bm{X}$ to the nearest quantization level minimizes the quantization MSE. 

The algorithm starts with an initial guess of quantization levels and then iteratively computes quantization thresholds $\{\tau_i\}_{i=1}^{2^B-1}$ and updates quantization levels $\{\hat{x}_i\}_{i=1}^{2^B}$. Specifically, at iteration $n$, thresholds are set to the midpoints of the previous iteration's levels:
\begin{align*}
    \tau_i^{(n)}=\frac{\hat{x}_i^{(n-1)}+\hat{x}_{i+1}^{(n-1)}}2 \text{ for } i=1\ldots 2^B-1
\end{align*}
Subsequently, the quantization levels are re-computed as conditional means of the data regions defined by the new thresholds:
\begin{align*}
    \hat{x}_i^{(n)}=\mathbb{E}\left[ \bm{X} \big| \bm{X}\in [\tau_{i-1}^{(n)},\tau_i^{(n)}] \right] \text{ for } i=1\ldots 2^B
\end{align*}
where to satisfy boundary conditions we have $\tau_0=-\infty$ and $\tau_{2^B}=\infty$. The algorithm iterates the above steps until convergence.

Figure \ref{fig:lm_quant} compares the quantization levels of a $7$-bit floating point (E3M3) quantizer (left) to a $7$-bit Lloyd-Max quantizer (right) when quantizing a layer of weights from the GPT3-126M model at a per-tensor granularity. As shown, the Lloyd-Max quantizer achieves substantially lower quantization MSE. Further, Table \ref{tab:FP7_vs_LM7} shows the superior perplexity achieved by Lloyd-Max quantizers for bitwidths of $7$, $6$ and $5$. The difference between the quantizers is clear at 5 bits, where per-tensor FP quantization incurs a drastic and unacceptable increase in perplexity, while Lloyd-Max quantization incurs a much smaller increase. Nevertheless, we note that even the optimal Lloyd-Max quantizer incurs a notable ($\sim 1.5$) increase in perplexity due to the coarse granularity of quantization. 

\begin{figure}[h]
  \centering
  \includegraphics[width=0.7\linewidth]{sections/figures/LM7_FP7.pdf}
  \caption{\small Quantization levels and the corresponding quantization MSE of Floating Point (left) vs Lloyd-Max (right) Quantizers for a layer of weights in the GPT3-126M model.}
  \label{fig:lm_quant}
\end{figure}

\begin{table}[h]\scriptsize
\begin{center}
\caption{\label{tab:FP7_vs_LM7} \small Comparing perplexity (lower is better) achieved by floating point quantizers and Lloyd-Max quantizers on a GPT3-126M model for the Wikitext-103 dataset.}
\begin{tabular}{c|cc|c}
\hline
 \multirow{2}{*}{\textbf{Bitwidth}} & \multicolumn{2}{|c|}{\textbf{Floating-Point Quantizer}} & \textbf{Lloyd-Max Quantizer} \\
 & Best Format & Wikitext-103 Perplexity & Wikitext-103 Perplexity \\
\hline
7 & E3M3 & 18.32 & 18.27 \\
6 & E3M2 & 19.07 & 18.51 \\
5 & E4M0 & 43.89 & 19.71 \\
\hline
\end{tabular}
\end{center}
\end{table}

\subsection{Proof of Local Optimality of LO-BCQ}
\label{subsec:lobcq_opt_proof}
For a given block $\bm{b}_j$, the quantization MSE during LO-BCQ can be empirically evaluated as $\frac{1}{L_b}\lVert \bm{b}_j- \bm{\hat{b}}_j\rVert^2_2$ where $\bm{\hat{b}}_j$ is computed from equation (\ref{eq:clustered_quantization_definition}) as $C_{f(\bm{b}_j)}(\bm{b}_j)$. Further, for a given block cluster $\mathcal{B}_i$, we compute the quantization MSE as $\frac{1}{|\mathcal{B}_{i}|}\sum_{\bm{b} \in \mathcal{B}_{i}} \frac{1}{L_b}\lVert \bm{b}- C_i^{(n)}(\bm{b})\rVert^2_2$. Therefore, at the end of iteration $n$, we evaluate the overall quantization MSE $J^{(n)}$ for a given operand $\bm{X}$ composed of $N_c$ block clusters as:
\begin{align*}
    \label{eq:mse_iter_n}
    J^{(n)} = \frac{1}{N_c} \sum_{i=1}^{N_c} \frac{1}{|\mathcal{B}_{i}^{(n)}|}\sum_{\bm{v} \in \mathcal{B}_{i}^{(n)}} \frac{1}{L_b}\lVert \bm{b}- B_i^{(n)}(\bm{b})\rVert^2_2
\end{align*}

At the end of iteration $n$, the codebooks are updated from $\mathcal{C}^{(n-1)}$ to $\mathcal{C}^{(n)}$. However, the mapping of a given vector $\bm{b}_j$ to quantizers $\mathcal{C}^{(n)}$ remains as  $f^{(n)}(\bm{b}_j)$. At the next iteration, during the vector clustering step, $f^{(n+1)}(\bm{b}_j)$ finds new mapping of $\bm{b}_j$ to updated codebooks $\mathcal{C}^{(n)}$ such that the quantization MSE over the candidate codebooks is minimized. Therefore, we obtain the following result for $\bm{b}_j$:
\begin{align*}
\frac{1}{L_b}\lVert \bm{b}_j - C_{f^{(n+1)}(\bm{b}_j)}^{(n)}(\bm{b}_j)\rVert^2_2 \le \frac{1}{L_b}\lVert \bm{b}_j - C_{f^{(n)}(\bm{b}_j)}^{(n)}(\bm{b}_j)\rVert^2_2
\end{align*}

That is, quantizing $\bm{b}_j$ at the end of the block clustering step of iteration $n+1$ results in lower quantization MSE compared to quantizing at the end of iteration $n$. Since this is true for all $\bm{b} \in \bm{X}$, we assert the following:
\begin{equation}
\begin{split}
\label{eq:mse_ineq_1}
    \tilde{J}^{(n+1)} &= \frac{1}{N_c} \sum_{i=1}^{N_c} \frac{1}{|\mathcal{B}_{i}^{(n+1)}|}\sum_{\bm{b} \in \mathcal{B}_{i}^{(n+1)}} \frac{1}{L_b}\lVert \bm{b} - C_i^{(n)}(b)\rVert^2_2 \le J^{(n)}
\end{split}
\end{equation}
where $\tilde{J}^{(n+1)}$ is the the quantization MSE after the vector clustering step at iteration $n+1$.

Next, during the codebook update step (\ref{eq:quantizers_update}) at iteration $n+1$, the per-cluster codebooks $\mathcal{C}^{(n)}$ are updated to $\mathcal{C}^{(n+1)}$ by invoking the Lloyd-Max algorithm \citep{Lloyd}. We know that for any given value distribution, the Lloyd-Max algorithm minimizes the quantization MSE. Therefore, for a given vector cluster $\mathcal{B}_i$ we obtain the following result:

\begin{equation}
    \frac{1}{|\mathcal{B}_{i}^{(n+1)}|}\sum_{\bm{b} \in \mathcal{B}_{i}^{(n+1)}} \frac{1}{L_b}\lVert \bm{b}- C_i^{(n+1)}(\bm{b})\rVert^2_2 \le \frac{1}{|\mathcal{B}_{i}^{(n+1)}|}\sum_{\bm{b} \in \mathcal{B}_{i}^{(n+1)}} \frac{1}{L_b}\lVert \bm{b}- C_i^{(n)}(\bm{b})\rVert^2_2
\end{equation}

The above equation states that quantizing the given block cluster $\mathcal{B}_i$ after updating the associated codebook from $C_i^{(n)}$ to $C_i^{(n+1)}$ results in lower quantization MSE. Since this is true for all the block clusters, we derive the following result: 
\begin{equation}
\begin{split}
\label{eq:mse_ineq_2}
     J^{(n+1)} &= \frac{1}{N_c} \sum_{i=1}^{N_c} \frac{1}{|\mathcal{B}_{i}^{(n+1)}|}\sum_{\bm{b} \in \mathcal{B}_{i}^{(n+1)}} \frac{1}{L_b}\lVert \bm{b}- C_i^{(n+1)}(\bm{b})\rVert^2_2  \le \tilde{J}^{(n+1)}   
\end{split}
\end{equation}

Following (\ref{eq:mse_ineq_1}) and (\ref{eq:mse_ineq_2}), we find that the quantization MSE is non-increasing for each iteration, that is, $J^{(1)} \ge J^{(2)} \ge J^{(3)} \ge \ldots \ge J^{(M)}$ where $M$ is the maximum number of iterations. 
%Therefore, we can say that if the algorithm converges, then it must be that it has converged to a local minimum. 
\hfill $\blacksquare$


\begin{figure}
    \begin{center}
    \includegraphics[width=0.5\textwidth]{sections//figures/mse_vs_iter.pdf}
    \end{center}
    \caption{\small NMSE vs iterations during LO-BCQ compared to other block quantization proposals}
    \label{fig:nmse_vs_iter}
\end{figure}

Figure \ref{fig:nmse_vs_iter} shows the empirical convergence of LO-BCQ across several block lengths and number of codebooks. Also, the MSE achieved by LO-BCQ is compared to baselines such as MXFP and VSQ. As shown, LO-BCQ converges to a lower MSE than the baselines. Further, we achieve better convergence for larger number of codebooks ($N_c$) and for a smaller block length ($L_b$), both of which increase the bitwidth of BCQ (see Eq \ref{eq:bitwidth_bcq}).


\subsection{Additional Accuracy Results}
%Table \ref{tab:lobcq_config} lists the various LOBCQ configurations and their corresponding bitwidths.
\begin{table}
\setlength{\tabcolsep}{4.75pt}
\begin{center}
\caption{\label{tab:lobcq_config} Various LO-BCQ configurations and their bitwidths.}
\begin{tabular}{|c||c|c|c|c||c|c||c|} 
\hline
 & \multicolumn{4}{|c||}{$L_b=8$} & \multicolumn{2}{|c||}{$L_b=4$} & $L_b=2$ \\
 \hline
 \backslashbox{$L_A$\kern-1em}{\kern-1em$N_c$} & 2 & 4 & 8 & 16 & 2 & 4 & 2 \\
 \hline
 64 & 4.25 & 4.375 & 4.5 & 4.625 & 4.375 & 4.625 & 4.625\\
 \hline
 32 & 4.375 & 4.5 & 4.625& 4.75 & 4.5 & 4.75 & 4.75 \\
 \hline
 16 & 4.625 & 4.75& 4.875 & 5 & 4.75 & 5 & 5 \\
 \hline
\end{tabular}
\end{center}
\end{table}

%\subsection{Perplexity achieved by various LO-BCQ configurations on Wikitext-103 dataset}

\begin{table} \centering
\begin{tabular}{|c||c|c|c|c||c|c||c|} 
\hline
 $L_b \rightarrow$& \multicolumn{4}{c||}{8} & \multicolumn{2}{c||}{4} & 2\\
 \hline
 \backslashbox{$L_A$\kern-1em}{\kern-1em$N_c$} & 2 & 4 & 8 & 16 & 2 & 4 & 2  \\
 %$N_c \rightarrow$ & 2 & 4 & 8 & 16 & 2 & 4 & 2 \\
 \hline
 \hline
 \multicolumn{8}{c}{GPT3-1.3B (FP32 PPL = 9.98)} \\ 
 \hline
 \hline
 64 & 10.40 & 10.23 & 10.17 & 10.15 &  10.28 & 10.18 & 10.19 \\
 \hline
 32 & 10.25 & 10.20 & 10.15 & 10.12 &  10.23 & 10.17 & 10.17 \\
 \hline
 16 & 10.22 & 10.16 & 10.10 & 10.09 &  10.21 & 10.14 & 10.16 \\
 \hline
  \hline
 \multicolumn{8}{c}{GPT3-8B (FP32 PPL = 7.38)} \\ 
 \hline
 \hline
 64 & 7.61 & 7.52 & 7.48 &  7.47 &  7.55 &  7.49 & 7.50 \\
 \hline
 32 & 7.52 & 7.50 & 7.46 &  7.45 &  7.52 &  7.48 & 7.48  \\
 \hline
 16 & 7.51 & 7.48 & 7.44 &  7.44 &  7.51 &  7.49 & 7.47  \\
 \hline
\end{tabular}
\caption{\label{tab:ppl_gpt3_abalation} Wikitext-103 perplexity across GPT3-1.3B and 8B models.}
\end{table}

\begin{table} \centering
\begin{tabular}{|c||c|c|c|c||} 
\hline
 $L_b \rightarrow$& \multicolumn{4}{c||}{8}\\
 \hline
 \backslashbox{$L_A$\kern-1em}{\kern-1em$N_c$} & 2 & 4 & 8 & 16 \\
 %$N_c \rightarrow$ & 2 & 4 & 8 & 16 & 2 & 4 & 2 \\
 \hline
 \hline
 \multicolumn{5}{|c|}{Llama2-7B (FP32 PPL = 5.06)} \\ 
 \hline
 \hline
 64 & 5.31 & 5.26 & 5.19 & 5.18  \\
 \hline
 32 & 5.23 & 5.25 & 5.18 & 5.15  \\
 \hline
 16 & 5.23 & 5.19 & 5.16 & 5.14  \\
 \hline
 \multicolumn{5}{|c|}{Nemotron4-15B (FP32 PPL = 5.87)} \\ 
 \hline
 \hline
 64  & 6.3 & 6.20 & 6.13 & 6.08  \\
 \hline
 32  & 6.24 & 6.12 & 6.07 & 6.03  \\
 \hline
 16  & 6.12 & 6.14 & 6.04 & 6.02  \\
 \hline
 \multicolumn{5}{|c|}{Nemotron4-340B (FP32 PPL = 3.48)} \\ 
 \hline
 \hline
 64 & 3.67 & 3.62 & 3.60 & 3.59 \\
 \hline
 32 & 3.63 & 3.61 & 3.59 & 3.56 \\
 \hline
 16 & 3.61 & 3.58 & 3.57 & 3.55 \\
 \hline
\end{tabular}
\caption{\label{tab:ppl_llama7B_nemo15B} Wikitext-103 perplexity compared to FP32 baseline in Llama2-7B and Nemotron4-15B, 340B models}
\end{table}

%\subsection{Perplexity achieved by various LO-BCQ configurations on MMLU dataset}


\begin{table} \centering
\begin{tabular}{|c||c|c|c|c||c|c|c|c|} 
\hline
 $L_b \rightarrow$& \multicolumn{4}{c||}{8} & \multicolumn{4}{c||}{8}\\
 \hline
 \backslashbox{$L_A$\kern-1em}{\kern-1em$N_c$} & 2 & 4 & 8 & 16 & 2 & 4 & 8 & 16  \\
 %$N_c \rightarrow$ & 2 & 4 & 8 & 16 & 2 & 4 & 2 \\
 \hline
 \hline
 \multicolumn{5}{|c|}{Llama2-7B (FP32 Accuracy = 45.8\%)} & \multicolumn{4}{|c|}{Llama2-70B (FP32 Accuracy = 69.12\%)} \\ 
 \hline
 \hline
 64 & 43.9 & 43.4 & 43.9 & 44.9 & 68.07 & 68.27 & 68.17 & 68.75 \\
 \hline
 32 & 44.5 & 43.8 & 44.9 & 44.5 & 68.37 & 68.51 & 68.35 & 68.27  \\
 \hline
 16 & 43.9 & 42.7 & 44.9 & 45 & 68.12 & 68.77 & 68.31 & 68.59  \\
 \hline
 \hline
 \multicolumn{5}{|c|}{GPT3-22B (FP32 Accuracy = 38.75\%)} & \multicolumn{4}{|c|}{Nemotron4-15B (FP32 Accuracy = 64.3\%)} \\ 
 \hline
 \hline
 64 & 36.71 & 38.85 & 38.13 & 38.92 & 63.17 & 62.36 & 63.72 & 64.09 \\
 \hline
 32 & 37.95 & 38.69 & 39.45 & 38.34 & 64.05 & 62.30 & 63.8 & 64.33  \\
 \hline
 16 & 38.88 & 38.80 & 38.31 & 38.92 & 63.22 & 63.51 & 63.93 & 64.43  \\
 \hline
\end{tabular}
\caption{\label{tab:mmlu_abalation} Accuracy on MMLU dataset across GPT3-22B, Llama2-7B, 70B and Nemotron4-15B models.}
\end{table}


%\subsection{Perplexity achieved by various LO-BCQ configurations on LM evaluation harness}

\begin{table} \centering
\begin{tabular}{|c||c|c|c|c||c|c|c|c|} 
\hline
 $L_b \rightarrow$& \multicolumn{4}{c||}{8} & \multicolumn{4}{c||}{8}\\
 \hline
 \backslashbox{$L_A$\kern-1em}{\kern-1em$N_c$} & 2 & 4 & 8 & 16 & 2 & 4 & 8 & 16  \\
 %$N_c \rightarrow$ & 2 & 4 & 8 & 16 & 2 & 4 & 2 \\
 \hline
 \hline
 \multicolumn{5}{|c|}{Race (FP32 Accuracy = 37.51\%)} & \multicolumn{4}{|c|}{Boolq (FP32 Accuracy = 64.62\%)} \\ 
 \hline
 \hline
 64 & 36.94 & 37.13 & 36.27 & 37.13 & 63.73 & 62.26 & 63.49 & 63.36 \\
 \hline
 32 & 37.03 & 36.36 & 36.08 & 37.03 & 62.54 & 63.51 & 63.49 & 63.55  \\
 \hline
 16 & 37.03 & 37.03 & 36.46 & 37.03 & 61.1 & 63.79 & 63.58 & 63.33  \\
 \hline
 \hline
 \multicolumn{5}{|c|}{Winogrande (FP32 Accuracy = 58.01\%)} & \multicolumn{4}{|c|}{Piqa (FP32 Accuracy = 74.21\%)} \\ 
 \hline
 \hline
 64 & 58.17 & 57.22 & 57.85 & 58.33 & 73.01 & 73.07 & 73.07 & 72.80 \\
 \hline
 32 & 59.12 & 58.09 & 57.85 & 58.41 & 73.01 & 73.94 & 72.74 & 73.18  \\
 \hline
 16 & 57.93 & 58.88 & 57.93 & 58.56 & 73.94 & 72.80 & 73.01 & 73.94  \\
 \hline
\end{tabular}
\caption{\label{tab:mmlu_abalation} Accuracy on LM evaluation harness tasks on GPT3-1.3B model.}
\end{table}

\begin{table} \centering
\begin{tabular}{|c||c|c|c|c||c|c|c|c|} 
\hline
 $L_b \rightarrow$& \multicolumn{4}{c||}{8} & \multicolumn{4}{c||}{8}\\
 \hline
 \backslashbox{$L_A$\kern-1em}{\kern-1em$N_c$} & 2 & 4 & 8 & 16 & 2 & 4 & 8 & 16  \\
 %$N_c \rightarrow$ & 2 & 4 & 8 & 16 & 2 & 4 & 2 \\
 \hline
 \hline
 \multicolumn{5}{|c|}{Race (FP32 Accuracy = 41.34\%)} & \multicolumn{4}{|c|}{Boolq (FP32 Accuracy = 68.32\%)} \\ 
 \hline
 \hline
 64 & 40.48 & 40.10 & 39.43 & 39.90 & 69.20 & 68.41 & 69.45 & 68.56 \\
 \hline
 32 & 39.52 & 39.52 & 40.77 & 39.62 & 68.32 & 67.43 & 68.17 & 69.30  \\
 \hline
 16 & 39.81 & 39.71 & 39.90 & 40.38 & 68.10 & 66.33 & 69.51 & 69.42  \\
 \hline
 \hline
 \multicolumn{5}{|c|}{Winogrande (FP32 Accuracy = 67.88\%)} & \multicolumn{4}{|c|}{Piqa (FP32 Accuracy = 78.78\%)} \\ 
 \hline
 \hline
 64 & 66.85 & 66.61 & 67.72 & 67.88 & 77.31 & 77.42 & 77.75 & 77.64 \\
 \hline
 32 & 67.25 & 67.72 & 67.72 & 67.00 & 77.31 & 77.04 & 77.80 & 77.37  \\
 \hline
 16 & 68.11 & 68.90 & 67.88 & 67.48 & 77.37 & 78.13 & 78.13 & 77.69  \\
 \hline
\end{tabular}
\caption{\label{tab:mmlu_abalation} Accuracy on LM evaluation harness tasks on GPT3-8B model.}
\end{table}

\begin{table} \centering
\begin{tabular}{|c||c|c|c|c||c|c|c|c|} 
\hline
 $L_b \rightarrow$& \multicolumn{4}{c||}{8} & \multicolumn{4}{c||}{8}\\
 \hline
 \backslashbox{$L_A$\kern-1em}{\kern-1em$N_c$} & 2 & 4 & 8 & 16 & 2 & 4 & 8 & 16  \\
 %$N_c \rightarrow$ & 2 & 4 & 8 & 16 & 2 & 4 & 2 \\
 \hline
 \hline
 \multicolumn{5}{|c|}{Race (FP32 Accuracy = 40.67\%)} & \multicolumn{4}{|c|}{Boolq (FP32 Accuracy = 76.54\%)} \\ 
 \hline
 \hline
 64 & 40.48 & 40.10 & 39.43 & 39.90 & 75.41 & 75.11 & 77.09 & 75.66 \\
 \hline
 32 & 39.52 & 39.52 & 40.77 & 39.62 & 76.02 & 76.02 & 75.96 & 75.35  \\
 \hline
 16 & 39.81 & 39.71 & 39.90 & 40.38 & 75.05 & 73.82 & 75.72 & 76.09  \\
 \hline
 \hline
 \multicolumn{5}{|c|}{Winogrande (FP32 Accuracy = 70.64\%)} & \multicolumn{4}{|c|}{Piqa (FP32 Accuracy = 79.16\%)} \\ 
 \hline
 \hline
 64 & 69.14 & 70.17 & 70.17 & 70.56 & 78.24 & 79.00 & 78.62 & 78.73 \\
 \hline
 32 & 70.96 & 69.69 & 71.27 & 69.30 & 78.56 & 79.49 & 79.16 & 78.89  \\
 \hline
 16 & 71.03 & 69.53 & 69.69 & 70.40 & 78.13 & 79.16 & 79.00 & 79.00  \\
 \hline
\end{tabular}
\caption{\label{tab:mmlu_abalation} Accuracy on LM evaluation harness tasks on GPT3-22B model.}
\end{table}

\begin{table} \centering
\begin{tabular}{|c||c|c|c|c||c|c|c|c|} 
\hline
 $L_b \rightarrow$& \multicolumn{4}{c||}{8} & \multicolumn{4}{c||}{8}\\
 \hline
 \backslashbox{$L_A$\kern-1em}{\kern-1em$N_c$} & 2 & 4 & 8 & 16 & 2 & 4 & 8 & 16  \\
 %$N_c \rightarrow$ & 2 & 4 & 8 & 16 & 2 & 4 & 2 \\
 \hline
 \hline
 \multicolumn{5}{|c|}{Race (FP32 Accuracy = 44.4\%)} & \multicolumn{4}{|c|}{Boolq (FP32 Accuracy = 79.29\%)} \\ 
 \hline
 \hline
 64 & 42.49 & 42.51 & 42.58 & 43.45 & 77.58 & 77.37 & 77.43 & 78.1 \\
 \hline
 32 & 43.35 & 42.49 & 43.64 & 43.73 & 77.86 & 75.32 & 77.28 & 77.86  \\
 \hline
 16 & 44.21 & 44.21 & 43.64 & 42.97 & 78.65 & 77 & 76.94 & 77.98  \\
 \hline
 \hline
 \multicolumn{5}{|c|}{Winogrande (FP32 Accuracy = 69.38\%)} & \multicolumn{4}{|c|}{Piqa (FP32 Accuracy = 78.07\%)} \\ 
 \hline
 \hline
 64 & 68.9 & 68.43 & 69.77 & 68.19 & 77.09 & 76.82 & 77.09 & 77.86 \\
 \hline
 32 & 69.38 & 68.51 & 68.82 & 68.90 & 78.07 & 76.71 & 78.07 & 77.86  \\
 \hline
 16 & 69.53 & 67.09 & 69.38 & 68.90 & 77.37 & 77.8 & 77.91 & 77.69  \\
 \hline
\end{tabular}
\caption{\label{tab:mmlu_abalation} Accuracy on LM evaluation harness tasks on Llama2-7B model.}
\end{table}

\begin{table} \centering
\begin{tabular}{|c||c|c|c|c||c|c|c|c|} 
\hline
 $L_b \rightarrow$& \multicolumn{4}{c||}{8} & \multicolumn{4}{c||}{8}\\
 \hline
 \backslashbox{$L_A$\kern-1em}{\kern-1em$N_c$} & 2 & 4 & 8 & 16 & 2 & 4 & 8 & 16  \\
 %$N_c \rightarrow$ & 2 & 4 & 8 & 16 & 2 & 4 & 2 \\
 \hline
 \hline
 \multicolumn{5}{|c|}{Race (FP32 Accuracy = 48.8\%)} & \multicolumn{4}{|c|}{Boolq (FP32 Accuracy = 85.23\%)} \\ 
 \hline
 \hline
 64 & 49.00 & 49.00 & 49.28 & 48.71 & 82.82 & 84.28 & 84.03 & 84.25 \\
 \hline
 32 & 49.57 & 48.52 & 48.33 & 49.28 & 83.85 & 84.46 & 84.31 & 84.93  \\
 \hline
 16 & 49.85 & 49.09 & 49.28 & 48.99 & 85.11 & 84.46 & 84.61 & 83.94  \\
 \hline
 \hline
 \multicolumn{5}{|c|}{Winogrande (FP32 Accuracy = 79.95\%)} & \multicolumn{4}{|c|}{Piqa (FP32 Accuracy = 81.56\%)} \\ 
 \hline
 \hline
 64 & 78.77 & 78.45 & 78.37 & 79.16 & 81.45 & 80.69 & 81.45 & 81.5 \\
 \hline
 32 & 78.45 & 79.01 & 78.69 & 80.66 & 81.56 & 80.58 & 81.18 & 81.34  \\
 \hline
 16 & 79.95 & 79.56 & 79.79 & 79.72 & 81.28 & 81.66 & 81.28 & 80.96  \\
 \hline
\end{tabular}
\caption{\label{tab:mmlu_abalation} Accuracy on LM evaluation harness tasks on Llama2-70B model.}
\end{table}

%\section{MSE Studies}
%\textcolor{red}{TODO}


\subsection{Number Formats and Quantization Method}
\label{subsec:numFormats_quantMethod}
\subsubsection{Integer Format}
An $n$-bit signed integer (INT) is typically represented with a 2s-complement format \citep{yao2022zeroquant,xiao2023smoothquant,dai2021vsq}, where the most significant bit denotes the sign.

\subsubsection{Floating Point Format}
An $n$-bit signed floating point (FP) number $x$ comprises of a 1-bit sign ($x_{\mathrm{sign}}$), $B_m$-bit mantissa ($x_{\mathrm{mant}}$) and $B_e$-bit exponent ($x_{\mathrm{exp}}$) such that $B_m+B_e=n-1$. The associated constant exponent bias ($E_{\mathrm{bias}}$) is computed as $(2^{{B_e}-1}-1)$. We denote this format as $E_{B_e}M_{B_m}$.  

\subsubsection{Quantization Scheme}
\label{subsec:quant_method}
A quantization scheme dictates how a given unquantized tensor is converted to its quantized representation. We consider FP formats for the purpose of illustration. Given an unquantized tensor $\bm{X}$ and an FP format $E_{B_e}M_{B_m}$, we first, we compute the quantization scale factor $s_X$ that maps the maximum absolute value of $\bm{X}$ to the maximum quantization level of the $E_{B_e}M_{B_m}$ format as follows:
\begin{align}
\label{eq:sf}
    s_X = \frac{\mathrm{max}(|\bm{X}|)}{\mathrm{max}(E_{B_e}M_{B_m})}
\end{align}
In the above equation, $|\cdot|$ denotes the absolute value function.

Next, we scale $\bm{X}$ by $s_X$ and quantize it to $\hat{\bm{X}}$ by rounding it to the nearest quantization level of $E_{B_e}M_{B_m}$ as:

\begin{align}
\label{eq:tensor_quant}
    \hat{\bm{X}} = \text{round-to-nearest}\left(\frac{\bm{X}}{s_X}, E_{B_e}M_{B_m}\right)
\end{align}

We perform dynamic max-scaled quantization \citep{wu2020integer}, where the scale factor $s$ for activations is dynamically computed during runtime.

\subsection{Vector Scaled Quantization}
\begin{wrapfigure}{r}{0.35\linewidth}
  \centering
  \includegraphics[width=\linewidth]{sections/figures/vsquant.jpg}
  \caption{\small Vectorwise decomposition for per-vector scaled quantization (VSQ \citep{dai2021vsq}).}
  \label{fig:vsquant}
\end{wrapfigure}
During VSQ \citep{dai2021vsq}, the operand tensors are decomposed into 1D vectors in a hardware friendly manner as shown in Figure \ref{fig:vsquant}. Since the decomposed tensors are used as operands in matrix multiplications during inference, it is beneficial to perform this decomposition along the reduction dimension of the multiplication. The vectorwise quantization is performed similar to tensorwise quantization described in Equations \ref{eq:sf} and \ref{eq:tensor_quant}, where a scale factor $s_v$ is required for each vector $\bm{v}$ that maps the maximum absolute value of that vector to the maximum quantization level. While smaller vector lengths can lead to larger accuracy gains, the associated memory and computational overheads due to the per-vector scale factors increases. To alleviate these overheads, VSQ \citep{dai2021vsq} proposed a second level quantization of the per-vector scale factors to unsigned integers, while MX \citep{rouhani2023shared} quantizes them to integer powers of 2 (denoted as $2^{INT}$).

\subsubsection{MX Format}
The MX format proposed in \citep{rouhani2023microscaling} introduces the concept of sub-block shifting. For every two scalar elements of $b$-bits each, there is a shared exponent bit. The value of this exponent bit is determined through an empirical analysis that targets minimizing quantization MSE. We note that the FP format $E_{1}M_{b}$ is strictly better than MX from an accuracy perspective since it allocates a dedicated exponent bit to each scalar as opposed to sharing it across two scalars. Therefore, we conservatively bound the accuracy of a $b+2$-bit signed MX format with that of a $E_{1}M_{b}$ format in our comparisons. For instance, we use E1M2 format as a proxy for MX4.

\begin{figure}
    \centering
    \includegraphics[width=1\linewidth]{sections//figures/BlockFormats.pdf}
    \caption{\small Comparing LO-BCQ to MX format.}
    \label{fig:block_formats}
\end{figure}

Figure \ref{fig:block_formats} compares our $4$-bit LO-BCQ block format to MX \citep{rouhani2023microscaling}. As shown, both LO-BCQ and MX decompose a given operand tensor into block arrays and each block array into blocks. Similar to MX, we find that per-block quantization ($L_b < L_A$) leads to better accuracy due to increased flexibility. While MX achieves this through per-block $1$-bit micro-scales, we associate a dedicated codebook to each block through a per-block codebook selector. Further, MX quantizes the per-block array scale-factor to E8M0 format without per-tensor scaling. In contrast during LO-BCQ, we find that per-tensor scaling combined with quantization of per-block array scale-factor to E4M3 format results in superior inference accuracy across models. 


\end{document}


% This document was modified from the file originally made available by
% Pat Langley and Andrea Danyluk for ICML-2K. This version was created
% by Iain Murray in 2018, and modified by Alexandre Bouchard in
% 2019 and 2021 and by Csaba Szepesvari, Gang Niu and Sivan Sabato in 2022.
% Modified again in 2023 and 2024 by Sivan Sabato and Jonathan Scarlett.
% Previous contributors include Dan Roy, Lise Getoor and Tobias
% Scheffer, which was slightly modified from the 2010 version by
% Thorsten Joachims & Johannes Fuernkranz, slightly modified from the
% 2009 version by Kiri Wagstaff and Sam Roweis's 2008 version, which is
% slightly modified from Prasad Tadepalli's 2007 version which is a
% lightly changed version of the previous year's version by Andrew
% Moore, which was in turn edited from those of Kristian Kersting and
% Codrina Lauth. Alex Smola contributed to the algorithmic style files.
