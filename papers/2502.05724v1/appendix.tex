\appendix
\onecolumn
\clearpage
\section*{Appendix Contents} % 
\begin{itemize}
  \item \hyperlink{app:app_notation}{Appendix A: Notation} ~~~~ p.\pageref{app_notation}
 \vspace{4pt}
   
  %\begin{itemize}[leftmargin=2em, nosep] % 2
  %  \item \hyperlink{app:lemma}{A.1: Technical Lemma} ~~~~ p.\pageref{app:lemma}
 % \end{itemize}
  
  \item \hyperlink{app:proof}{Appendix B: Proof} ~~~~ p.\pageref{app_proof}
  \vspace{4pt}
  \begin{itemize}[leftmargin=2em, nosep] % 2
    \item \hyperlink{app_proof_th_dual}{B.1: The proof of Theorem 2.2} ~~~~ p.\pageref{app_proof_th_dual} \\
    \item \hyperlink{app_proof_cora_de}{B.2: The proof of Corollary 2.3} ~~~~ p.\pageref{app_proof_cora_de}\\
    \item \hyperlink{app_proof_cora_de}{B.3: The proof of Lemma 4.1} ~~~~ p.\pageref{app_proof_cora_de}\\
    \item \hyperlink{app_proof_le_norm}{B.4: The proof of Lemma 4.2} ~~~~ p.\pageref{app_proof_le_norm}
  \end{itemize}
  \vspace{4pt}

   \item \hyperlink{app:related_work}{Appendix C: Related Work} ~~~~ p.\pageref{app:related_work}
   
   \vspace{4pt}
   \item \hyperlink{app_uni_fram}{Appendix D: Details of Unified Framework} ~~~~ p.\pageref{app_uni_fram}

   \vspace{4pt}
   \item \hyperlink{app_issues}{Appendix E: Issues with Existing Experimental Setup} ~~~~ p.\pageref{app_issues}
   \vspace{4pt}
   \begin{itemize}[leftmargin=2em, nosep] % 2
    \item \hyperlink{app_issues_subtask}{E.1: Details of multiple subtask setup} ~~~~ p.\pageref{app_issues_subtask} \\
    \item \hyperlink{app_issues_results}{E.2: Details of experimental setting and more results} ~~~~ p.\pageref{app_issues_results}
  \end{itemize}
   \vspace{4pt}
  \item \hyperlink{app_sdgae}{Appendix F: Details and Experimental Setting for SDGAE} ~~~~ p.\pageref{app_sdgae}
   \vspace{4pt}
   \begin{itemize}[leftmargin=2em, nosep] % 2
    \item \hyperlink{app_sdgae_conv}{F.1: More details of graph convolution} ~~~~ p.\pageref{app_sdgae_conv} \\
    \item \hyperlink{app_sdgae_setting}{F.2: Experimental setting} ~~~~ p.\pageref{app_sdgae_setting}
  \end{itemize}
  
  \vspace{4pt}
  \item \hyperlink{app:experiments}{Appendix G: More Details of DirLinkBench} ~~~~ p.\pageref{app:experiments}
  \vspace{4pt}
  \begin{itemize}[leftmargin=2em, nosep] % 2
    \item \hyperlink{app_dataset_baseline}{G.1: Datasets and baselines} ~~~~ p.\pageref{app_dataset_baseline} \\
    \item \hyperlink{app_bench_metric}{G.2: Metric description} ~~~~ p.\pageref{app_bench_metric} \\
    \item \hyperlink{app_bench_res_ana}{G.3: Additional results in analysis} ~~~~ p.\pageref{app_bench_res_ana}
    
  \end{itemize}

  \vspace{4pt}
  \item \hyperlink{app_complete_res}{Appendix H: Complete Results of DirLinkBench on Seven Datasets} ~~~~ p.\pageref{app_complete_res}
  \vspace{4pt}
   \begin{itemize}[leftmargin=2em, nosep] % 2
    \item \hyperlink{app_complete_res_cora}{H.1: Results on Cora-ML} ~~~~ p.\pageref{app_complete_res_cora} \\
    \item \hyperlink{app_complete_res_citeseer}{H.2: Results on CiteSeer} ~~~~ p.\pageref{app_complete_res_citeseer} \\
    \item \hyperlink{app_complete_res_photo}{H.3: Results on Photo} ~~~~ p.\pageref{app_complete_res_photo} \\
    \item \hyperlink{app_complete_res_computers}{H.4: Results on Computers} ~~~~ p.\pageref{app_complete_res_computers} \\
    \item \hyperlink{app_complete_res_wiki}{H.5: Results on WikiCS} ~~~~ p.\pageref{app_complete_res_wiki} \\
    \item \hyperlink{app_complete_res_slash}{H.6: Results on Slashdot} ~~~~ p.\pageref{app_complete_res_slash} \\
    \item \hyperlink{app_complete_res_epinion}{H.7: Results on Epinions} ~~~~ p.\pageref{app_complete_res_epinion} 
  \end{itemize}
  
   
\end{itemize}

\newpage

\hypertarget{app:app_notation}{} 
\section{Notation}\label{app_notation}
We summarize the main notations of the paper in Table~\ref{notation}.
\begin{table}[th]
\centering

\caption{Summation of main notations in this paper.}\label{tbl:def-notation}
\begin{tabular} {l|p{5.0in}} \hline
\toprule
{\bf Notation} &  {\bf Description}  \\ \midrule
$\gG=(V,E)$ & directed unweighted graph with node set $V$ and edge set $E$. \\ \midrule
$n, m$ & the number of nodes and edges, $n = |V|$ and $m = |E|$. \\ \midrule

$\mA$ & the adjacency matrix of $\gG$, $\mA_{uv} = 1$ for a edge from node $u$ to $v$, and $\mA_{uv} = 0$ otherwise. \\ \midrule

 $\mH$ & the Hermitian adjacency matrix, defined as $\mH = \mA_{\rm s} \odot \exp\left(i\frac{\pi}{2}\mathbf{\Theta}\right)$. \\ \midrule

$\mA_{\rm s}$ & the  adjacency matrix of the undirected graph derived from $\gG$, $\mA_{\rm s} = \mA \cup \mA^{\top}$. \\ \midrule

$\mathbf{\Theta}$ & the skew-symmetric matrix $\mathbf{\Theta} = \mA - \mA^{\top}$. \\ \midrule

$\mD_{\rm out}, \mD_{\rm int}$ & the out- and in-degree matrix of $\mA$, $\mD_{\rm out} = {\rm diag}(\mA\vone) $ and $\mD_{\rm int} = {\rm diag}(\mA^{\top}\vone)$. \\ \midrule

$\mX$ & the node feature matrix, $\mX \in \mathbb{R}^{n \times d^{\prime}}$ and $d^{\prime}$ is the dimension. \\ \midrule

$\bm{\theta}_u, \bm{\phi}_u$ & real-valued dual embeddings $\bm{\theta}_u\in \mathbb{R}^{d_\theta}, \bm{\phi}_u\in \mathbb{R}^{d_\phi}$, $d_\theta, d_\phi$ are the dimensions. \\ \midrule

$p(u,v)$ & the probability of directed edge $(u,v)$ existing. \\ \midrule

$\gG^{\prime}=(V, E^{\prime})$ & the observed/training graph with node set $V$ and edge set $E^{\prime}$. \\ \midrule

$\hat{\mA}$ & the adjacency matrix with added self-loops, $\hat{\mA}=\mA + \mI$. \\ \midrule

$\hat{\mD}_{\rm out}, \hat{\mD}_{\rm int}$ & the out- and in-degree matrix of $\hat{\mA}$. \\ \midrule

$\mathcal{S}(\hat{\mA})$ & the block matrix consisting of $\hat{\mA}$ and $\hat{\mA}^{\top}$. \\ \midrule

$\Tilde{\mA}$ & the normalized adjacency matrix, $\Tilde{\mA} =\hat{\mD}_{\rm out}^{-1/2}\hat{\mA}\hat{\mD}_{\rm in}^{-1/2}$. \\ \midrule

$\odot, \quad \|$ & the Hadamard product and concatenation process. \\ \midrule

$d, \quad d^{\prime}$ & the dimension of embeddings and feature matrix. \\

\bottomrule
\end{tabular}
\label{notation}
%\vspace{-3mm}
\end{table}

\section{Proofs}\label{sec:proofs}
\subsection{Propagation of chaos for MFLD (Section \ref{sec:main_results})}\label{subsec:poc_proof}
\begin{proof}[Proof of Lemma \ref{lemma:clsi}]
    The first equality of the assertion was proved by \citet{nitanda2024improved}. We here prove the inequality by utilizing the argument of conditional and marginal distribution of $\pow[\mu,N]$ \cite{chen2022uniform}.
    
    For $\vX \sim \pow[\mu,N]$, we denote by $\pow[\mu,N]_{i|-i}(\cdot | \vx^{-i})$ and $\pow[\mu,N]_{-i}$ the conditional distribution of $X^i$ conditioned by $\vX^{-i}=\vx^{-i}$ and the marginal distribution of $\vX^{-i}$, respectively.
    It holds that 
    \begin{align} 
        &\bE_{\vx \sim \pow[\mu,N]}\left[ \left\| \nabla \log \frac{\rd \pow[\mu,N]}{\rd \pow[\mu,N]_*}(\vX) \right\|_2^2\right] \notag\\
        &= \sum_{i=1}^N \bE_{\vx \sim \pow[\mu,N]}\left[ \left\| \nabla_{x^i} \log \frac{\rd \pow[\mu,N]}{\rd \vx}(\vX) + \frac{N}{\lambda}\nabla_{x^i}F(\mu_\vx)\right\|_2^2\right] \notag\\
        &= \sum_{i=1}^N \bE_{\vX^{-i} \sim \pow[\mu,N]_{-i}} \left[ \bE_{X^i \sim \pow[\mu,N]_{i|-i}(\cdot | \vX^{-i})}\left[ \left\| \nabla_{x^i} \log \frac{\rd \pow[\mu,N]}{\rd \vx}(\vX) + \frac{N}{\lambda}\nabla_{x^i}F(\mu_\vx)\right\|_2^2 \right]\right]. \label{eq:fisher_div}
    \end{align}

    We write $p_{-i}(\vx^{-i})=\frac{\rd \pow[\mu,N]_{-i}}{\rd \vx^{-i}}(\vx^{-i})$ and $p_{i|-i}(x|\vx^{-i})=\frac{\rd \pow[\mu,N]_{i|-i}(\cdot|\vx^{-i})}{\rd x}(x)$.
    Since $\frac{\rd \pow[\mu,N]}{\rd \vx}(\vx) = p_{-i}(\vx^{-i}) p_{i|-i}(x^i|\vx^{-i})$, we get the following equation:
    \begin{equation*}
        \nabla_{x^i} \log \frac{\rd \pow[\mu,N]}{\rd \vx}(\vx)
        = \frac{\nabla_{x^i}(p_{-i}(\vx^{-i}) p_{i|-i}(x^{i}|\vx^{-i}))}{p_{-i}(\vx^{-i}) p_{i|-i}(x^{i}|\vx^{-i})}
        = \frac{\nabla_{x^i}p_{i|-i}(x^i|\vx^{-i})}{p_{i|-i}(x^i|\vx^{-i})}
        = \nabla \log p_{i|-i}(x^i|\vx^{-i}).
    \end{equation*}
    Hence, Eq.~\eqref{eq:fisher_div} can be further bounded by the LSI on the conditional Gibbs distribution (Assumption \ref{assumption:uniform_directional_lsi}) as follows: 
    \begin{align}
        &\sum_{i=1}^N \bE_{\vX^{-i} \sim \pow[\mu,N]_{-i}} \left[ \bE_{X^i \sim \pow[\mu,N]_{i|-i}(\cdot |\vX^{-i})}\left[ \left\| \nabla \log p_{i|-i}(X^i|\vX^{-i}) + \frac{N}{\lambda}\nabla_{x^i}F(\mu_\vx)\right\|_2^2 \right]\right] \notag \\
        &=\sum_{i=1}^N \bE_{\vX^{-i} \sim \pow[\mu,N]_{-i}} \left[ \bE_{X^i \sim \pow[\mu,N]_{i|-i}(\cdot|\vX^{-i})}\left[ \left\| \nabla \log \frac{\rd \pow[\mu,N]_{i|-i}}{\rd \nu_{i|-i}}(X^i|\vX^{-i}) \right\|_2^2 \right]\right] \notag \\
        &\geq 2\alpha  \sum_{i=1}^N \bE_{\vX^{-i} \sim \pow[\mu,N]_{-i}} \left[ \KL( \pow[\mu,N]_{i|-i}(\cdot|\vX^{-i}) \| \nu_{i|-i}(\cdot|\vX^{-i})) \right].  \label{eq:fisher_div_eval}
    \end{align}

    Let $\nu$ be the probability distribution on $\bR^d$ with the density $\frac{\rd \nu}{\rd x}(x) \propto \exp(-r(x)/\lambda)$. Here, notice that the conditional Gibbs distribution $\nu_{i|-i}(\cdot|\vx^{-i})$ is the minimizer of the following objective: for $\vx^{-i}$
    \begin{align*} 
        \nu_{i|-i}(\cdot|\vx^{-i}) 
        &= \argmin_{\mu \in \cP_2(\bR^d)}\left\{
        \int NF_0(\rho_{x\cup\vx^{-i}})\mu(\rd x) 
        + \bE_{X\sim \mu}[r(X)] + \lambda \Ent(\mu)
        \right\} \\
        &= \argmin_{\mu \in \cP_2(\bR^d)}\left\{
        \int NF_0(\rho_{x\cup\vx^{-i}})\mu(\rd x) 
        + \lambda\KL(\mu\|\nu)
        \right\}.
    \end{align*}

    Because of the optimality of the conditional Gibbs distribution, we have 
    \begin{align}
        &\lambda\KL( \pow[\mu,N]_{i|-i}(\cdot|\vX^{-i}) \| \nu_{i|-i}(\cdot|\vX^{-i})) \notag\\
        &=\lambda \Ent(\pow[\mu,N]_{i|-i}(\cdot|\vX^{-i}))
        + \int NF(\rho_{x \cup \vX^{-i}}) \pow[\mu,N]_{i|-i}(\rd x|\vX^{-i})
        + \lambda \log \int \exp\left(-\frac{N}{\lambda}F(\rho_{x\cup \vX^{-i}})\right) \rd x \notag \\
        &=\lambda \KL\left(\pow[\mu,N]_{i|-i}(\cdot|\vX^{-i} )\| \nu \right)
        + \int NF_0(\rho_{x \cup \vX^{-i}}) \left(\pow[\mu,N]_{i|-i}(\rd x|\vX^{-i})
        - \nu_{i|-i}(\rd x|\vX^{-i})(\rd x)\right)  - \lambda \KL\left(\nu_{i|-i}(\cdot|\vX^{-i}) \| \nu\right) \notag \\
        &\geq \lambda \KL\left(\pow[\mu,N]_{i|-i}(\cdot|\vX^{-i} )\| \nu \right)
        + \int NF_0(\rho_{x \cup \vX^{-i}}) \left(\pow[\mu,N]_{i|-i}(\rd x|\vX^{-i})
        - \mu_*(\rd x)\right)  - \lambda \KL\left(\mu_* \| \nu\right). \label{eq:fisher_div_eval2}
    \end{align}

    The expectation of the second term of the right-hand side can be futher evaluated as 
    \begin{align}
        &N \sum_{i=1}^N \bE_{\vX \sim \pow[\mu,N]} \left[ \int F_0(\rho_{x \cup \vX^{-i}}) \left(\pow[\mu,N]_{i|-i}(\rd x|\vX^{-i})- \mu_*(\rd x)\right) \right] \notag \\
        &= N \sum_{i=1}^N \bE_{\vX \sim \pow[\mu,N]} \left[ F_0(\rho_\vX) - \int F_0(\rho_{x \cup \vX^{-i}})\mu_*(\rd x) \right] \notag \\
        &= -N \sum_{i=1}^N \bE_{\vX \sim \pow[\mu,N]} \left[\int \left\{ B_{F_0}(\rho_{x \cup \vX^{-i}},\rho_\vX) + \pd< \frac{\delta F_0(\rho_\vX)}{\delta \mu}, \rho_{x \cup \vX^{-i}}-\rho_\vX> \right\} \mu_*(\rd x)\right] \notag \\
        &= -N \sum_{i=1}^N \bE_{\vX \sim \pow[\mu,N]} \left[\int B_{F_0}(\rho_{x \cup \vX^{-i}},\rho_\vX) \mu_*(\rd x)\right] \notag \\
        &~~~~ + \sum_{i=1}^N \bE_{\vX \sim \pow[\mu,N]} \left[\int \left\{ \frac{\delta F_0(\rho_\vX)}{\delta \mu}(X^i) - \frac{\delta F_0(\rho_\vX)}{\delta \mu}(x) \right\} \mu_*(\rd x)\right] \notag \\
        &= -N \sum_{i=1}^N \bE_{\vX \sim \pow[\mu,N]} \left[\int B_{F_0}(\rho_{x \cup \vX^{-i}},\rho_\vX) \mu_*(\rd x)\right] \notag \\
        &~~~~ + N \bE_{\vX \sim \pow[\mu,N]} \left[\int \frac{\delta F_0(\rho_\vX)}{\delta \mu}(x) \left(\rho_{\vX}(\rd x) - \mu_*(\rd x)\right) \right] \notag \\
        &\geq - B 
        + N \left(\bE_{\vX \sim \pow[\mu,N]} \left[ F_0(\rho_\vX)\right] - F_0(\mu_*) \right), \label{eq:fisher_div_eval3}
    \end{align}
    where the last inequality is due to the convexity of $F_0$ and Assumption \ref{assumption:nonlinearity}.

    By the information inequality (Lemma 5.1 of \citet{chen2022uniform}), the first term of Eq.~\eqref{eq:fisher_div_eval2} of the right-hand side can be evaluated as 
    \begin{align}
        \sum_{i=1}^N \bE_{\vX \sim \pow[\mu,N]}\left[ \KL\left(\pow[\mu,N]_{i|-i}(\cdot|\vX^{-i} )\| \nu \right) \right] 
        \geq \KL(\pow[\mu,N] \| \tensor[\nu,N]).
    \end{align}
    
    Combining all of them, we get
    \begin{align*}
        &\bE_{\vx \sim \pow[\mu,N]}\left[ \left\| \nabla \log \frac{\rd \pow[\mu,N]}{\rd \pow[\mu,N]_*}(\vX) \right\|_2^2\right] \\
        &\geq \frac{2\alpha}{\lambda} \left\{ - B + N \bE_{\vX \sim \pow[\mu,N]} \left[ F_0(\rho_\vX) \right] + \lambda \KL(\pow[\mu,N] \| \tensor[\nu,N]) 
        - N(F_0(\mu_*) + \lambda \KL\left(\mu_* \| \nu\right)) \right\} \\
        &= \frac{2\alpha}{\lambda} \left\{ - B + N \bE_{\vX \sim \pow[\mu,N]} \left[ F(\rho_\vX) \right] + \lambda \Ent(\pow[\mu,N]) - N(F(\mu_*) +\lambda\Ent(\mu_*)) \right\} \\
        &= \frac{2\alpha}{\lambda} \left( - B + \pow[\cL,N](\pow[\mu,N]) - N \cL(\mu_*) \right).
    \end{align*}
    This concludes the proof.
\end{proof}

\begin{proof}[Proof of Theorem \ref{theorem:mfld_convergence}]
    We here prove the convergence of MFLD in the discrete-time by using the one-step interpolation argument \cite{nitanda2024improved,suzuki2023convergence}. 
    
    We construct the one-step interpolation for $k$-th iteration: $X_{k+1}^i = X_k^i - \eta \nabla \frac{\delta F(\rho_{\vX_k})}{\delta \mu}(X_k^i) + \sqrt{2\lambda \eta} \xi_k^i,~(i\in \{1,2,\ldots,d\})$.
    as follows: for $i\in \{1,2,\ldots,d\}$,
    \begin{equation}\label{eq:noisy-GD_dynamics}
        \rd Y_t^i = - \nabla \frac{\delta F(\rho_{\vY_0})}{\delta \mu}(Y_0^i)\rd t + \sqrt{2\lambda}\rd W_t,
    \end{equation}
    where $\vY_0 = (Y_0^1,\ldots,Y_0^d) = (X_k^1,\ldots,X_k^d)$ and $W_t$ is the standard Brownian motion in $\bR^d$ with $W_0 = 0$.
    We denote by $\nu_t$ the distributions of $\vY_t$. 
    Then, $\nu_0 = \pow[\mu,N]_k (= \mathrm{Law}(\vX_k))$, $\nu_\eta = \pow[\mu,N]_{k+1} (=\mathrm{Law}(\vX_{k+1}))$ (i.e., $\vY_{\eta} \disteq \vX_{k+1}$).
    In this proof, we identify the probability distribution with its density function with respect to the Lebesgure measure for notational simplicity. For instance, we denote by $\pow[\mu,N]_*(\vy)$ the density of $\pow[\mu,N]_*$. 

    By the proof of Theorem 2 in \citet{nitanda2024improved}, we see for $t \in [0,\eta]$,
    \begin{align}
        \frac{\rd \pow[\cL,N]}{\rd t}(\nu_t)
        &\leq - \frac{\lambda^2}{2} \int \nu_t(\vy) \left\| \nabla \log \frac{\nu_t}{\pow[\mu,N]_*}(\vy) \right\|_2^2 \rd \vy 
        + N \delta_\eta, \label{eq:one_step_decrease}
    \end{align}
    where $\delta_\eta = 8\eta( C_2^2 + \lambda^{\prime 2}) (\eta C_1^2 + \lambda d) 
    + 32 \eta^2 \lambda'^2( C_2^2 + \lambda^{\prime 2}) \left( \frac{1}{N}\bE\left[ \left\| \vX_0 \right\|_2^2 \right] + \frac{1}{\lambda'}\left(\frac{C_1^2}{4\lambda'} + \lambda d\right) \right)$.

    Combining Lemma \ref{lemma:clsi} with the above inequality, we get
    \begin{align*}
        &\frac{\rd \pow[\cL,N]}{\rd t}(\nu_t)
        \leq  - \alpha \lambda \left( \pow[\cL,N](\nu_t) - N \cL(\mu_*) - B \right)
        + N \delta_\eta. \\
        \Longleftrightarrow 
        ~~~~&\frac{\rd}{\rd t}\left( \pow[\cL,N](\nu_t) - N \cL(\mu_*) - B - \frac{N\delta_\eta}{\alpha \lambda} \right)
        \leq  - \alpha \lambda \left( \pow[\cL,N](\nu_t) - N \cL(\mu_*) - B - \frac{N\delta_\eta}{\alpha \lambda} \right).
    \end{align*}    

    Noting $\nu_\eta = \pow[\mu,N]_{k+1}$ and $\nu_0 = \pow[\mu,N]_k$, the Gr\"{o}nwall’s inequality leads to 
    \[ 
        \pow[\cL,N](\pow[\mu,N]_{k+1}) -  N\cL(\mu_*) - B - \frac{ N\pow[\delta,N]_{\eta}}{\alpha \lambda} 
        \leq \exp( -\alpha\lambda\eta )\left(  \pow[\cL,N](\pow[\mu,N]_k) -  N\cL(\mu_*) - B- \frac{N\pow[\delta,N]_{\eta}}{\alpha \lambda} \right). 
    \]
    
This inequality holds at every iteration of (\ref{eq:noisy-GD_dynamics}). Hence, we arrive at the desired result,  
\begin{align*}
    \frac{1}{N}\pow[\cL,N](\pow[\mu,N]_k) - \cL(\mu_*)  
    &\leq \frac{B}{N} + \frac{ \pow[\delta,N]_{\eta}}{\alpha \lambda} 
    + \exp( -\alpha\lambda\eta k )\left( \frac{1}{N}\pow[\cL,N](\pow[\mu,N]_0) - \cL(\mu_*) - \frac{B}{N} - \frac{ \pow[\delta,N]_{\eta}}{\alpha \lambda} \right) \\
    &\leq \frac{B}{N} + \frac{ \pow[\delta,N]_{\eta}}{\alpha \lambda} 
    + \exp( -\alpha\lambda\eta k )\left( \frac{1}{N}\pow[\cL,N](\pow[\mu,N]_0) - \cL(\mu_*) \right).
\end{align*}
\end{proof}

\subsection{Point-wise model approximation error (Section \ref{subsec:pw_model_error})}\label{subsec:pw_model_error_proof}
\begin{proof}[Proof of Lemma \ref{lemma:han_inequality}]
    It follows that by Han's inequality \citep{dembo1991information},
    \[ \frac{1}{s \binom{N}{s}} \sum_{|S|=s} \int \pow[\mu,N]_S(\rd \vx_S) \log \frac{\rd \pow[\mu,N]_S}{\rd \vx_S}(\vx_S) \leq \frac{1}{N}\int \pow[\mu,N](\rd x) \log \frac{\rd \pow[\mu,N]}{\rd \vx}(\vx). \]
    Moreover, we see
    \begin{align*} 
        \sum_{|S|=s} \int \pow[\mu,N]_S(\rd \vx_S) \log \frac{\rd \tensor[\mu_*,k]}{\rd \vx_S}(\vx_S)
        &= \sum_{|S|=s} \sum_{i \in S}\int \pow[\mu,N]_i(\rd x^i) \log \frac{\rd \mu_*}{\rd x}(x^i)  \\
        &= \binom{N-1}{s-1}\sum_{i=1}^N\int \pow[\mu,N]_i(\rd x^i) \log \frac{\rd \mu_*}{\rd x}(x^i) \\
        &= \binom{N-1}{s-1} \int \pow[\mu,N](\rd \vx) \log \frac{\rd \tensor[\mu_*,N]}{\rd \vx}(\vx). 
    \end{align*}
    Noticing $\binom{N-1}{s-1} = \frac{s}{N}\binom{N}{s}$, we conclude the first statement which immediately implies the second statement.
\end{proof}


\begin{proposition}\label{prop:pw_poc_merge}
    Suppose $\pow[\mu,N]$ is exchangeable and $\vX_{j} \sim \tensor[\mu,N]$ $(j=1,2,\ldots,M)$. Then, it follows that for any $z \in \cZ$,
    \begin{align*} 
        \bE_{\{\vX_j\}_{j=1}^M}\left[ \left( \frac{1}{M}\sum_{j=1}^M\bE_{\rho_{\vX_j}}[ h(X,z)] - \bE_{\mu_*}[h(X,z)]\right)^2 \right] 
        \leq \frac{4R^2}{N M} + \frac{8R^2}{M}\sqrt{ \frac{\KL(\pow[\mu,N]\|\tensor[\mu_*,N])}{N}} 
    + \frac{2R^2 \KL(\pow[\mu,N] \| \tensor[\mu,N]_*)}{N}.
    \end{align*}
\end{proposition}
\begin{proof}[Proof of Proposition \ref{prop:pw_poc_merge}]
    \begin{align*}
        \bE_{\{\vX_j\}_{j=1}^M}&\left[ \left( \frac{1}{M}\sum_{j=1}^M\bE_{X \sim \rho_{\vX_j}}[ h(X,z)] - \bE_{X\sim\mu_*}[h(X,z)]\right)^2 \right] \\
        &= \frac{1}{M^2}\bE_{\{\vX_j\}_{j=1}^M}\left[ \sum_{j=1}^M\left(\bE_{X \sim \rho_{\vX_j}}[ h(X,z)] - \bE_{X\sim\mu_*}[h(X,z)]\right)^2 \right] \\
        &+ \frac{1}{M^2}\bE_{\{\vX_j\}_{j=1}^M}\left[ \sum_{j\neq k}\left( \bE_{X \sim \rho_{\vX_j}}[ h(X,z)] - \bE_{X\sim\mu_*}[h(X,z)]\right)\left( \bE_{X \sim \rho_{\vX_k}}[ h(X,z)] - \bE_{X\sim\mu_*}[h(X,z)]\right) \right].
    \end{align*}
    Using Proposition \ref{prop:pw_model_approximation}, we can upper bound the first term by $\frac{4R^2}{Ms'} + \frac{8R^2}{M}\sqrt{ \frac{\KL(\pow[\mu,N]\|\tensor[\mu_*,N])}{N}}$. 
    The second term can be evaluated as follows. Set $H(\vX_j) = \bE_{X \sim \mu_{\vX_j}}[ h(X,z)] - \bE_{X\sim\mu_*}[h(X,z)]$. Then for $j\neq k$,
    \begin{align*}
        \bE_{\{\vX_j\}_{j=1}^M}\left[ H(\vX_j)H(\vX_k) \right]
        &= \left( \bE_{\vX_j}\left[ H(\vX_j) \right] \right)^2 \\
        &= \left( \bE_{\vX_j}\left[  \frac{1}{s'} \sum_{i=1}^{s'} h(X_j^i,z) \right] - \bE_{X\sim\mu_*}[h(X,z)] \right)^2 \\
        &= \left( \bE_{X \sim \pow[\mu,N]_1}\left[  h(X,z) \right] - \bE_{X\sim\mu_*}[h(X,z)] \right)^2 \\
        &\leq 4R^2 \TV^2(\pow[\mu,N]_1,\mu_*) \\
        &\leq 2R^2 \KL(\pow[\mu,N]_1 \| \mu_*) \\
        &\leq \frac{2R^2}{N}\KL(\pow[\mu,N] \| \tensor[\mu,N]_*).
    \end{align*}
    This concludes the proof.
\end{proof}

%%%%%
\subsection{Uniform model approximation error (Section \ref{subsec:uniform_model_error})}\label{subsec:uniform_model_error_proof}
We evaluate the empirical Rademacher complexity $\hat{\cR}_{N,M}(\cF)$ by using Dudley's entropy integral. We define the metric $\|f\|_{N,M,2} = \sqrt{\frac{1}{MN}\sum_{j=1}^M\sum_{i=1}^N |f(X^i_j)|^2}$.
We denote by $\cN(\cF,\epsilon,\|\cdot\|_{N,M,2})$ the $\epsilon$-covering number of $\cF$ with respect to the $\|\cdot\|_{N,M,2}$-norm.
\begin{lemma}[Dudley's entropy integral]\label{lemma:dudley}
    Given a function class $\cF$ on $\bR^d$, we suppose $R = \sup_{f\in\cF}\|f\|_{N,M,2} < \infty$. Then,
    \begin{equation*}
        \hat{\cR}_{N,M}(\cF) 
        \leq \inf_{\delta > 0} 
        \left\{
            4\delta 
            + \frac{12}{\sqrt{MN}} \int_{\delta}^{R} \sqrt{\log 2 \cN(\cF,\epsilon,\|\cdot\|_{N,M,2})} \rd\epsilon
        \right\}.
    \end{equation*}
\end{lemma}

\begin{proposition}\label{proposition:rademacher_complexity_bound}
    Suppose Assumption \ref{assumption:model_constraint} holds and $\vX_j \sim \pow[\mu,N]$ $(j=1,2,\ldots,M)$ are independent. Then, we get
    \[ \bE_{\{\vX_j\}_{j=1}^M} \left[\hat{\cR}_{N,M}(\cF) \right] 
    \leq 4 R\sqrt{\frac{d}{MN}}
        + 12R\sqrt{\frac{1}{MN} \left(\log 2 + d\log \left( 1 + 2\beta MR^{-1}\sqrt{MNd^{-1}}\bE_{\vX\sim\pow[\mu,N]}[\|\vX\|_2]\right)\right)}. \]
\end{proposition}
\begin{proof}
    Since $\| f \|_{N,M,2} \leq \| f \|_{N,M,\infty} = \max_{i,j}|f(X^i_j)|$, it is sufficient evaluate the $\epsilon$-covering number of $\cF$ with respect to $\| \cdot \|_{N,M,\infty}$. 
    We write $r=\max_{i,j}\|X^i_j\|_2$. By Assumption \ref{assumption:model_constraint}, for any $z, z' \in \cZ$,
    \[ \max_{i,j}| h(X^i_j,z) - h(X^i_j,z')| 
    \leq \max_{i,j} \beta \|X^i_j\|_2  \| z - z' \|_2 
    =  \beta r \| z - z' \|_2, \]
    we see $\cN(\cF, \epsilon,\|\cdot\|_{N,M,\infty}) \leq \cN\left(\cZ, \epsilon/(\beta r), \|\cdot\|_{2}\right) = \left( 1 + \frac{2 \beta r}{\epsilon}\right)^{d}$.

    Therefore, by Lemma \ref{lemma:dudley} with $\delta = R\sqrt{d(MN)^{-1}}$, we get
    \begin{align*}
        \hat{\cR}_{N,M}(\cF) 
        &\leq 4R \sqrt{\frac{d}{MN}}
        + 12R\sqrt{\frac{1}{MN}\log 2 \cN(\cF,R\sqrt{d(MN)^{-1}},\|\cdot\|_{N,M,\infty})} \\
        &=4R \sqrt{\frac{d}{MN}}
        + 12R\sqrt{\frac{1}{MN} \left(\log 2 + d\log \left( 1 + 2 \beta r R^{-1}\sqrt{MNd^{-1}}\right)\right)} \\
        &\leq 4R \sqrt{\frac{d}{MN}}
        + 12R\sqrt{\frac{1}{MN} \left(\log 2 + d\log \left( 1 + 2 \beta R^{-1}\sqrt{MNd^{-1}} \sum_{j=1}^M \|\vX_j\|_2\right)\right)},
    \end{align*}
    where we used $r \leq \sum_{j=1}^M \|\vX_j\|_2$.
    Finally, Jensen's inequality yields
    \[ \bE_{\{\vX_j\}_{j=1}^M} \left[\hat{\cR}_{N,M}(\cF) \right] 
    \leq 4R \sqrt{\frac{d}{MN}}
        + 12R\sqrt{\frac{1}{MN} \left(\log 2 + d\log \left( 1 + 2\beta MR^{-1}\sqrt{MNd^{-1}}\bE_{\vX\sim\pow[\mu,N]}[\|\vX\|_2] \right)\right)}.\]
\end{proof}

Here, we give the complete version of the uniform model approximation bound.
\begin{theorem}[Complete version of Theorem \ref{theorem:uniform_approximation_multiple_mfld}]\label{theorem:uniform_approximation_multiple_mfld_complete}
    Suppose Assumption \ref{assumption:model_constraint} and the same conditions as in Theorem \ref{theorem:mfld_convergence} hold. Run $M$-parallel MFLD in the discrete time independently, with $\eta \lambda' < 1/2$ and $\vX_{j,0} \sim \tensor[\mu,N]_0 (j=1,2,\ldots,M)$. Then,
    \begin{align*}
        \bE_{\{\vX_{j,k}\}}&\left[\left\| \frac{1}{M}\sum_{j=1}^M \bE_{X\sim \rho_{\vx_{j,k}}}[h(X,\cdot)] - \bE_{X\sim \mu_*} [h(X,\cdot)]\right\|_\infty\right] \\
        &\leq \frac{5CR}{4}\sqrt{\frac{d}{MN} + \frac{dB}{\lambda N}}
        + CR\sqrt{\frac{d\lambda}{MN(\lambda + MB)}}\log\left( C'\sqrt{\left(\lambda+MB\right)\frac{\pi}{\lambda}} \right) \\
        &+ CR\sqrt{\frac{d\lambda M N}{\lambda + MB}} \left(\frac{ \delta_{\eta}}{\alpha\lambda^2} 
        + \frac{1}{\lambda}\exp( -\alpha\lambda\eta k )\pow[\Delta,N]_0 \right),\\
        %&=\tilde{O}\left( R\sqrt{\frac{d}{MN} + \frac{dB}{\lambda N}
        %+ \frac{d\lambda}{MN(\lambda + MB)}} \right) \\
        %&+ O\left( R\sqrt{\frac{d\lambda M N}{\lambda + MB}} \left(\frac{ \delta_{\eta}}{\alpha\lambda^2} 
        %+ \frac{1}{\lambda}\exp( -\alpha\lambda\eta k )\pow[\Delta,N]_0 \right) \right),
    \end{align*}
    where $C'=1 + 2\beta MR^{-1}\sqrt{MNd^{-1}}\bE_{\vX\sim\tensor[\mu,N]_*}[\|\vX\|_2]$.
\end{theorem}
\begin{proof}
    For $\vx_1,\ldots,\vx_M \in \bR^{dN}$, we set 
    $g(\vx_1,\ldots,\vx_M) = \sup_{z \in \cZ} 
    \left| \frac{1}{M}\sum_{j=1}^M \bE_{X\sim \rho_{\vx_j}}[h(X,z)] - \bE_{X\sim \mu_*}[h(X,z)]\right|$.
    By the variational formulation of KL-divergence (e.g., Corollary 4.15 in \citet{boucheron2013concentration}), we get
    \begin{align}\label{eq:variational_kl}
        \bE_{\mu^{(N)\otimes M}}[g]
        &\leq \frac{1}{\gamma} \log \bE_{\mu^{\otimes N M}_*}[\exp(\gamma g)] + \frac{\KL(\mu^{(N)\otimes M}\|\mu^{\otimes N M}_*)}{\gamma} \notag \\
        &\leq \frac{1}{\gamma} \log \bE_{\mu^{\otimes N M}_*}[\exp(\gamma g)] + \frac{M\KL(\mu^{(N)}\|\mu^{\otimes N}_*)}{\gamma} 
    \end{align}

    For independent random variables $\vX_j \sim \tensor[\mu,N]_* (j=1,2,\ldots,M)$, by Lemma \ref{lemma:uniform_lln} and \ref{proposition:rademacher_complexity_bound}, it follows that with high probability $1-\delta$,
    \begin{align*}
        &g(\vX_1,\ldots,\vX_M) \\
        &\leq 2\bE_{\{\vX_j\}_{j=1}^M}\left[\hat{\cR}_{N,M}(\cF)\right] 
        + R \sqrt{ \frac{2\log(1/\delta)}{MN}} \\
        &\leq 8 R\sqrt{\frac{d}{MN}}
        + 24R\sqrt{\frac{1}{MN} \left(\log 2 + d\log \left( 1 + 2\beta MR^{-1}\sqrt{MNd^{-1}}\bE_{\vX\sim\tensor[\mu,N]_*}[\|\vX\|_2]\right)\right)}
        + R \sqrt{ \frac{2\log(1/\delta)}{MN}}\\
        &\leq CR\sqrt{\frac{d\log(C'/\delta)}{MN}},
    \end{align*}
    where $C$ is a uniform constant and $C'=1 + 2\beta MR^{-1}\sqrt{MNd^{-1}}\bE_{\vX\sim\tensor[\mu,N]_*}[\|\vX\|_2]$.
    This means
    \begin{align*}
        &\bP_{\mu^{\otimes N M}_*}\left[ g(\vX_1,\ldots,\vX_M) > CR\sqrt{\frac{d\log(C'/\delta)}{MN}} \right] \leq \delta \\
        \iff
        &\bP_{\mu^{\otimes N M}_*}\left[ g(\vX_1,\ldots,\vX_M) > t \right] \leq C'\exp\left( - \frac{MNt^2}{dC^2R^2}\right) \\
        \iff
        &\bP_{\mu^{\otimes N M}_*}\left[ g(\vX_1,\ldots,\vX_M) > \frac{1}{\gamma}\log t \right] \leq C'\exp\left( - \frac{MN(\log t)^2}{dC^2R^2\gamma^2}\right).
    \end{align*}
    Using this tail bound, 
    \begin{align*}
        \bE_{\mu^{\otimes N M}_*}[\exp(\gamma g)]
        &=\int_0^\infty \bP_{\mu^{\otimes N M}_*}\left[ \exp(\gamma g(\vX_1,\ldots,\vX_M)) > t \right] \rd t \\
        &=\int_0^\infty \bP_{\mu^{\otimes N M}_*}\left[ g(\vX_1,\ldots,\vX_M) > \frac{1}{\gamma} \log t \right] \rd t \\
        &=\int_0^\infty C'\exp\left( - \frac{MN(\log t)^2}{dC^2R^2\gamma^2} \right) \rd t \\
        &=C'CR\gamma\sqrt{\frac{\pi d}{MN}}\exp\left( \frac{dC^2R^2\gamma^2}{4MN} \right).
    \end{align*}
    Therefore, we get
    \begin{equation*}
        \bE_{\mu^{(N)\otimes M}}[g]
        \leq \frac{dC^2R^2\gamma}{4MN} + \frac{1}{\gamma}\log\left( C'CR\gamma\sqrt{\frac{\pi d}{MN}} \right)
        + \frac{M\KL(\mu^{(N)}\|\mu^{\otimes N}_*)}{\gamma}.
    \end{equation*}
    Moreover, by applying Lemma \ref{lemma:clsi} Theorem \ref{theorem:mfld_convergence} to Eq.~\eqref{eq:variational_kl}, we get
    \begin{align*}
        \bE_{\{\vX_{j,k}\}}&\left[\left\| \frac{1}{M}\sum_{j=1}^M \bE_{X\sim \rho_{\vx_{j,k}}}[h(X,\cdot)] - \bE_{X\sim \mu_*} [h(X,\cdot)]\right\|_\infty\right] \\
        &\leq \frac{dC^2R^2\gamma}{4MN}
        + \frac{1}{\gamma}\log\left( C'CR\gamma\sqrt{\frac{\pi d}{MN}} \right)
        + \frac{M}{\gamma} \left(  \frac{B}{\lambda} + \frac{ N\delta_{\eta}}{\alpha\lambda^2} 
        + \frac{N}{\lambda}\exp( -\alpha\lambda\eta k )\pow[\Delta,N]_0 \right).
    \end{align*}

    Finally, by seting $\gamma = \frac{1}{CR}\sqrt{\frac{MN}{d}\left(1+\frac{MB}{\lambda}\right)}$, we get
    \begin{align*}
        \bE_{\{\vX_{j,k}\}}&\left[\left\| \frac{1}{M}\sum_{j=1}^M \bE_{X\sim \rho_{\vx_{j,k}}}[h(X,\cdot)] - \bE_{X\sim \mu_*} [h(X,\cdot)]\right\|_\infty\right] \\
        &\leq \frac{5CR}{4}\sqrt{\frac{d}{MN} + \frac{dB}{\lambda N}}
        + CR\sqrt{\frac{d\lambda}{MN(\lambda + MB)}}\log\left( C'\sqrt{\left(\lambda+MB\right)\frac{\pi}{\lambda}} \right) \\
        &+ CR\sqrt{\frac{d\lambda M N}{\lambda + MB}} \left(\frac{ \delta_{\eta}}{\alpha\lambda^2} 
        + \frac{1}{\lambda}\exp( -\alpha\lambda\eta k )\pow[\Delta,N]_0 \right).
    \end{align*}
\end{proof}

%\begin{proof}[New proof]
%    Let $\vX_j=(X_j^1,\ldots,X_j^N) \sim \pow[\mu,N]~(j=1,2,\ldots,M)$ be $M$ independent random variables.
%    We set
%    \begin{align*}  
%        &g_N(z) = \frac{1}{MN}\sum_{j=1}^M\sum_{i=1}^N h(X_{j}^i,z) - \bE_{X \sim \pow[\mu,N]_1}[h(X,z)], \\
%        &S_N(z) = |g_N(z)| - \bE_{\{\vX_j\}_{j=1}^M}[ | g_N(z) | ].
%    \end{align*}
%    We define the metric $d$ defined in $\cZ$ as follows: for any $z,z'\in\cZ$,
%    \begin{align*}
%        d(z,z') = \sqrt{ \bE_{\{\vX_j\}_{j=1}^M}\left[ (S_N(z) - S_N(z'))^2 \right] }.
%    \end{align*}


%    Given $\{\vX_j\}_{j=1}^M$, we set 
%    \[ S_{M,N}(z) = \frac{1}{M}\sum_{j=1}^M \bE_{X\sim \rho_{\vx_{j}}}[h(X,z)] - \bE_{X\sim \mu_*} [h(X,z)],~~~~(z \in \cZ). \]

%    We endow the data space $\cZ$ with the metric $d(z,z') = |S_{M,N}(z) - S_{M,N}(z')|$ ($\forall z,z' \in \cZ$). We denote by $\cN(\cZ,\epsilon,d)$ the $\epsilon$-covering number of $\cZ$ with respect to $d$. We also denote by $\cZ_\epsilon \subset \cZ$ a finite subset that provides $\epsilon$-covering of $\cF$ corresponding to $\cN(\cZ,\epsilon,d)$. For any $z \in \cZ$, there exists $\pi_\epsilon(z) \in \cZ_\epsilon$ such that $d(z,z')\leq \epsilon$.

%    Noting that for any $z,z'\in \cZ$,
%    \begin{align*}
%        d(z,z')
%        &= \left| \frac{1}{MN}\sum_{j=1}^M\sum_{i=1}^N \left( h(X_j^i,z) - h(X_j^i,z') \right)
%        + \bE_{X\sim\mu_*}[h(X,z)-h(X,z')] \right| \\
%        &\leq \frac{1}{MN}\sum_{j=1}^M\sum_{i=1}^N \left| h(X_j^i,z) - h(X_j^i,z') \right|
%        + \bE_{X\sim\mu_*}\left[ \left| h(X,z)-h(X,z') \right| \right] \\
%        &\leq \beta \left( \frac{1}{MN}\sum_{j=1}^M\sum_{i=1}^N\|X_j^i\|_2
%        + \bE_{X\sim\mu_*}[\|X\|_2]\right)\|z-z'\|_2 \\
%        &= \beta r \|z-z'\|_2,
%    \end{align*}
%    where we set $r = \frac{1}{MN}\sum_{j=1}^M\sum_{i=1}^N\|X_j^i\|_2$, we see $\cN(\cZ, \epsilon, d) \leq \cN\left(\cZ, \epsilon/(\beta r), \|\cdot\|_{2}\right) = \left( 1 + \frac{2 \beta r}{\epsilon}\right)^{d}$.
    
%    Let $\vX_j=(X_j^1,\ldots,X_j^N) \sim \pow[\mu,N]~(j=1,2,\ldots,M)$ be $M$ independent random variables. Then, we get
%    \begin{align*}
%        \bE_{\{\vX_j\}_{j=1}^M}\left[ 
%            \sup_{z\in\cZ} S_{M,N}^2(z)
%        \right]
%        &= \bE_{\{\vX_j\}_{j=1}^M}\left[ 
%            \sup_{z\in\cZ} (S_{M,N}(z) - S_{M,N}(\pi(z)) + S_{M,N}(\pi(z)))^2
%        \right] \\
%        &\leq 2\bE_{\{\vX_j\}_{j=1}^M}\left[ \sup_{z\in\cZ} \left(S_{M,N}(z) - S_{M,N}(\pi(z))\right)^2 
%            + \sup_{z\in\cZ} S_{M,N}^2(\pi(z))
%        \right]\\
%       &\leq 2\bE_{\{\vX_j\}_{j=1}^M}\left[ \epsilon^2 
%           + \sup_{z\in\cZ} S_{M,N}^2(\pi(z))
%        \right]\\
%        &\leq 2\epsilon^2 + 2\bE_{\{\vX_j\}_{j=1}^M}\left[ 
%            \max_{z\in\cZ_{\epsilon}} S_{M,N}^2(z)
%        \right]
%    \end{align*}
    
%    Because of the boundedness $\sup_{x \in \bR^d, z\in\cZ}|h(x,z)| \leq R$, $S^2_{M,N}(z) - \bE[S^2_{M,N}(z)]$ is a sub-Gaussian with variance factor $4R^4$. Therefore, by the maximal inequality for sub-Gaussian distribution (e.g., see \citet{boucheron2013concentration}), we get
%    \begin{align*}
%        &\max_{z\in\cZ_{\epsilon}} \left\{ S_{M,N}^2(z) - \bE[S^2_{M,N}(z)]\right\}
%        \leq 2\sqrt{2}R^2 \log \cN(\cZ,\epsilon,d)
%        \leq 2\sqrt{2}R^2 d \log \left( 1 + \frac{2\beta r}{\epsilon}\right) \\
%        \iff &\max_{z\in\cZ_{\epsilon}} S_{M,N}^2(z)
%        \leq \max_{z\in\cZ_{\epsilon}} \bE[S^2_{M,N}(z)]
%        + 2\sqrt{2}R^2 d \log \left( 1 + \frac{2\beta r}{\epsilon}\right).
%    \end{align*}
%\end{proof}

\section{Related Work}
% \subsection{Vision Language Model}
% 시각장애인에서 상황을 설명할 DB가 없으니 만들었다. 그리고 이를 VLM에 튜닝했다.
\subsection{Technical approaches for assisting the visually-impaired}


\subsection{Datasets for visual instruction tuning}







\hypertarget{app_uni_fram}{} 
\section{Details of Unified Framework}\label{app_uni_fram}

\begin{table}[h]
\centering
\vspace{-4mm}
\caption{The unified framework for directed link prediction methods.}
\label{app_table_fram}
%\begin{tabular}{@{}l@{\quad}l@{\quad}l@{}}
\begin{tabular}{@{}lllllllll@{}}
\toprule
Encoder $\mathrm{Enc}(\cdot)$ 
 &\multicolumn{2}{l}{Embeddings $({\bm \theta_u}, {\bm \phi_u})$}
 &\multicolumn{6}{l}{Possible Decoder $\mathrm{Dec}(\cdot)$}\\
\midrule
Source-target  & $\vs_u = \bm{\theta}_u$,           & $\vt_u = \bm{\phi}_u$          & \multicolumn{2}{l}{$\sigma\bigl(\vs_u^{\top}\vt_v\bigr)$;} & \multicolumn{2}{l}{$\mathrm{LR}\bigl(\vs_u \odot \vt_v\bigr)$;} & \multicolumn{2}{l}{\quad$\mathrm{LR}\bigl(\vs_u \|\vt_v\bigr)$} \\[3.0pt]

Single real-valued  & $\vh_u = \bm{\theta}_u$,  & $\varnothing = \bm{\phi}_u$          & \multicolumn{2}{l}{$\sigma\bigl(\vh_u^{\top}\vh_v\bigr)$;} & \multicolumn{2}{l}{$\mathrm{MLP}\bigl(\vh_u \odot \vh_v\bigr)$;} & \multicolumn{2}{l}{\quad$\mathrm{MLP}\bigl(\vh_u \|\vh_v\bigr)$} \\[3.0pt]

Complex-valued  & \multicolumn{2}{l}{$\vz_u = \bm{\theta}_u\odot\exp\bigl(i\,\bm{\phi}_u\bigr)$}  & \multicolumn{2}{l}{$\mathrm{Direc}\bigl(\vz_u,\vz_v\bigr)$;} & \multicolumn{4}{l}{$\mathrm{MLP}\bigl(\bm{\theta}_u \|\bm{\theta}_v \|\bm{\phi}_u \|\bm{\phi}_v\bigr)$} \\[3.0pt]

Gravity-inspired  & $\vh_u = \bm{\theta}_u$,  &$m_u = g(\bm{\phi}_u)$         & \multicolumn{3}{l}{$\sigma\bigl(m_v - \lambda\log\|\vh_u-\vh_v\|_2^2\bigr)$;}              & \multicolumn{3}{l}{$\sigma\bigl(m_v - \lambda\log \bigl(\mathrm{dist}_{\mathbb{D}_c^{d'}}(\vh_u,\vh_v)\bigr)\bigr)$}  \\
\bottomrule
\end{tabular}
\end{table}

%Here, we introduce the details of the examples in our unified framework for directed link prediction methods. For the convenience of reading, we copy the Table~\ref{table_fram} from the main paper below.
%In Table~\ref{app_table_fram}, $\sigma$ is the activation function, e.g., $\mathrm{Sigmoid}$; $\mathrm{LR}$ denotes the logistic regression classifier and $\mathrm{MLP}$ denotes multilayer perceptron. $\odot$ and $\|$ denote the Hadamard product and concat process of vectors, respectively. For source-target encoder, we set source embedding $\vs_u = \bm{\theta}_u$, target embedding $\vt_u=\bm{\phi}_u$ for each node $u \in V$. The possible decoders include $\sigma\bigl(\vs_u^{\top}\vt_v\bigr)$, $\mathrm{LR}\bigl(\vs_u \odot \vt_v\bigr)$, and $\mathrm{LR}\bigl(\vs_u \|\vt_v\bigr)$, where $\vs_u^{\top}\vt_v$ denotes the inner product of source  and target embeddings, $\mathrm{LR}(\cdot)$ denotes logistic regression predictor. For single real-valued encoder, we set real-valued embedding $\vh_u = \bm{\theta}_u$ and $\bm{\phi}_u=\varnothing$, $\varnothing$ denotes nonexistent. The possible decoders include $\sigma\bigl(\vh_u^{\top}\vh_v\bigr)$, $\mathrm{MLP}\bigl(\vh_u \odot \vh_v\bigr)$, $\mathrm{MLP}\bigl(\vh_u \|\vh_v\bigr)$. For complex-valued encoder, we set complex-valued embedding $\vz_u=\bm{\theta}_u\odot\exp({i\bm{\phi}_u})$, where $i$ is the imaginary unit. The possible decoders include $\mathrm{Direc}\bigl(\vz_u,\vz_v\bigr)$ and $\mathrm{MLP}\bigl(\bm{\theta}_u \|\bm{\theta}_v \|\bm{\phi}_u \|\bm{\phi}_v\bigr)$, where $\mathrm{Direc}(\cdot)$ denotes the direction-aware decoder defined in DUPLEX~\cite{duplex}. For gravity-inspired encoder, we set real-valued embedding $\vh_u = \bm{\theta}_u$ and mass parameter $m_u = g(\bm{\phi}_u)$, where $g(\cdot)$ is a function or neural network for converting $\bm{\phi}_u$ to a scalar~\cite{dhypr}. The possible decoders include $\sigma\bigl(m_v - \lambda\log\|\vh_u-\vh_v\|_2^2\bigr)$ ~\cite{gragae} and $\sigma\bigl(m_v - \lambda\log \bigl(\mathrm{dist}_{\mathbb{D}_c^{d'}}(\vh_u,\vh_v)\bigr)\bigr)$~\cite{dhypr}, where $\lambda$ is a hyperparameter and $\mathrm{dist}_{\mathbb{D}_c^{d'}}(\cdot)$ denotes hyperbolic distance.

Here, we introduce the details of the examples within our unified learning framework for directed link prediction methods. For readability, we copy Table~\ref{table_fram} from the main paper in Table~\ref{app_table_fram}. Here, $\sigma$ represents the activation function (e.g., $\mathrm{Sigmoid}$), while $\mathrm{LR}$ and $\mathrm{MLP}$ denote the logistic regression predictor and the multilayer perceptron, respectively. The symbols $\odot$ and $\|$ represent the Hadamard product and the vector concatenation process, respectively.

For \underline{source-target encoder}, we define the source embedding as $\vs_u = \bm{\theta}_u$ and the target embedding as $\vt_u=\bm{\phi}_u$ for each node $u \in V$. Possible decoders include:
$\sigma\bigl(\vs_u^{\top}\vt_v\bigr)$,  
$\mathrm{LR}\bigl(\vs_u \odot \vt_v\bigr)$,  
$\mathrm{LR}\bigl(\vs_u \|\vt_v\bigr)$.  
Here, $\vs_u^{\top}\vt_v$ denotes the inner product of the source and target embeddings, while $\mathrm{LR}(\cdot)$ represents the logistic regression predictor.

For \underline{single real-valued encoder}, we define the real-valued embedding as $\vh_u = \bm{\theta}_u$ and set $\bm{\phi}_u=\varnothing$, where $\varnothing$ denotes nonexistence. Possible decoders include:
$\sigma\bigl(\vh_u^{\top}\vh_v\bigr)$,  
$\mathrm{MLP}\bigl(\vh_u \odot \vh_v\bigr)$,  
$\mathrm{MLP}\bigl(\vh_u \|\vh_v\bigr)$.  

For \underline{complex-valued encoder}, we define the complex-valued embedding as $\vz_u=\bm{\theta}_u\odot\exp({i\bm{\phi}_u})$, where $i$ is the imaginary unit. Possible decoders include:
$\mathrm{Direc}\bigl(\vz_u,\vz_v\bigr)$,  
$\mathrm{MLP}\bigl(\bm{\theta}_u \|\bm{\theta}_v \|\bm{\phi}_u \|\bm{\phi}_v\bigr)$.  
Here, $\mathrm{Direc}(\cdot)$ refers to the direction-aware decoder defined in DUPLEX~\cite{duplex}.

For \underline{gravity-inspired encoder}, we define the real-valued embedding as $\vh_u = \bm{\theta}_u$ and set the mass parameter $m_u = g(\bm{\phi}_u)$, where $g(\cdot)$ is a function or neural network that converts $\bm{\phi}_u$ into a scalar~\cite{dhypr}. Possible decoders include:
$\sigma\bigl(m_v - \lambda\log\|\vh_u-\vh_v\|_2^2\bigr)$ ~\cite{gragae},  
$\sigma\bigl(m_v - \lambda\log \bigl(\mathrm{dist}_{\mathbb{D}_c^{d'}}(\vh_u,\vh_v)\bigr)\bigr)$~\cite{dhypr}.  
Here, $\lambda$ is a hyperparameter and $\mathrm{dist}_{\mathbb{D}_c^{d'}}(\cdot)$ represents the hyperbolic distance.


\hypertarget{app_issues}{} 
\section{Issues with Existing Experimental Setup}\label{app_issues}

\hypertarget{app_issues_subtask}{} 
\subsection{Details of multiple subtask setup}\label{app_issues_subtask}
The multiple subtask setup~\cite{magnet,dpyg,duplex} involves four subtasks where graph edges are categorized into four types: positive (original direction), reverse (inverse direction), bidirectional (both directions), and nonexistent (no connection). Each subtask focuses on predicting a specific edge type. The details are as follows:
\begin{itemize}[topsep=0pt, partopsep=0pt]
    \item Existence Prediction (EP): The model predicts whether a directed edge $(u, v)$ exists in the graph. Both reverse and nonexistent edges are treated as nonexistence.
    \item Directed Prediction (DP): The model predicts the direction of edges for node pairs $(u, v)$, where either $(u, v) \in E$ or $(v, u) \in E$.
    \item Three-Type Classification (3C): The model classifies an edge as positive, reverse, or nonexistent.
    \item Four-Type Classification (4C): The model classifies edges into four categories: positive, reverse, bidirectional, or nonexistent.
\end{itemize}

\begin{table}[h]
\centering
\caption{The results of MagNet~\cite{magnet} as reported in the original paper, alongside the reproduced MagNet and
MLP results.}
%\resizebox{\textwidth}{!}{
\begin{tabular}{@{}cllllll@{}}
\toprule
\multicolumn{1}{l}{Task}              & Method & Cornell & Texas & Wisconsin & Cora-ML & CiteSeer \\ \midrule
\multirow{3}{*}{DP} & MagNet (reported) &82.9$\pm$3.5 &80.9$\pm$4.2 &83.3$\pm$3.0 &86.5$\pm$0.7 &84.8$\pm$1.2 \\ %\cmidrule(l){2-7} 
                                      & MagNet (reproduced) &84.69$\pm$2.86 &87.15$\pm$4.49 &85.72$\pm$4.24 &\textbf{88.56$\pm$0.44}
                                      &\textbf{89.72$\pm$1.06}         \\
                                      & MLP    &\textbf{84.95$\pm$3.10} &\textbf{89.79$\pm$4.06} &\textbf{87.21$\pm$2.48}
                                      &86.56$\pm$0.70        &88.92$\pm$0.56          \\ \midrule
\multirow{3}{*}{EP} & MagNet (reported) &\textbf{81.1$\pm$3.3} &83.6$\pm$2.7 &\textbf{82.8$\pm$2.2} &\textbf{82.7$\pm$0.7} &79.9$\pm$0.6   \\
                                      & MagNet (reproduced)    
&75.42$\pm$6.25 &\textbf{84.52$\pm$4.12} &78.16$\pm$3.83 &80.04$\pm$0.89 &83.95$\pm$0.35\\
                                      & MLP       &76.01$\pm$4.64&82.97$\pm$4.33 &79.79$\pm$2.71 &79.47$\pm$1.08        &\textbf{84.78$\pm$0.56}          \\ \bottomrule
\end{tabular}
\label{tb_magnet}
\end{table}

\begin{table}[h]
\centering
\caption{The results of MLP and baselines under PyGSD~\cite{dpyg} setup on Direction Prediction (DP) task.}
%\resizebox{\textwidth}{!}{
\begin{tabular}{lcccccc}
\toprule
Method & Cora-ML & CiteSeer & Telegram & Cornell & Texas & Wisconsin \\ \midrule
MLP       &86.13$\pm$0.45  & 85.51$\pm$0.63  & 95.61$\pm$0.15  & \textbf{86.40$\pm$2.60}  &\textbf{83.08$\pm$4.33}  & \textbf{87.95$\pm$2.65}  \\ \midrule
DGCN      & 85.49$\pm$0.75  & 84.85$\pm$0.56  & 96.03$\pm$0.35  & 84.55$\pm$3.71  & 79.59$\pm$5.09  & 84.57$\pm$3.59  \\ 
DiGCN     & 85.37$\pm$0.54  & 83.88$\pm$0.82  & 94.95$\pm$0.54  & 85.41$\pm$2.78  & 76.57$\pm$3.98  & 82.41$\pm$2.56  \\ 
DiGCNIB   & 86.12$\pm$0.42  & 85.58$\pm$0.56  & 95.99$\pm$0.44  & 86.28$\pm$3.37  &82.27$\pm$3.51  &87.07$\pm$2.16  \\ 
MagNet    & \textbf{86.33$\pm$0.54}  & \textbf{85.80$\pm$0.63}  & \textbf{96.97$\pm$0.21}  & 83.29$\pm$4.28  & 80.25$\pm$4.78  & 86.60$\pm$2.72  \\ \bottomrule
\end{tabular}%}
\label{tb_dpyg_dp}
\end{table}

\begin{table}[h]
\centering
\caption{The results of MLP and baselines under PyGSD~\cite{dpyg} setup on Existence Prediction (EP) task.}
%\resizebox{\textwidth}{!}{
\begin{tabular}{lcccccc}
\toprule
Method & Cora-ML & CiteSeer & Telegram & Cornell & Texas & Wisconsin \\ \midrule
MLP       & \textbf{78.85$\pm$0.83}  & 71.03$\pm$0.89  & 82.72$\pm$0.57  & 68.41$\pm$2.05  & \textbf{69.58$\pm$2.29}  & \textbf{72.53$\pm$2.27}  \\ \midrule 
DGCN      & 76.45$\pm$0.49  & 70.72$\pm$0.79  & 83.18$\pm$1.55  & 67.95$\pm$2.27  & 63.65$\pm$2.40  & 68.12$\pm$2.86    \\ 
DiGCN     & 76.17$\pm$0.49  & 72.00$\pm$0.88  & 83.12$\pm$0.43  & 67.16$\pm$1.82  & 63.54$\pm$3.85  & 67.01$\pm$2.47    \\ 
DiGCNIB   & 78.80$\pm$0.51  & \textbf{74.55$\pm$0.91}  & 84.49$\pm$0.51  & \textbf{69.77$\pm$2.05}  & 67.60$\pm$3.44  & 70.78$\pm$2.79   \\ 
MagNet    & 77.37$\pm$0.45  & 71.47$\pm$0.71  & \textbf{85.82$\pm$0.39}  & 68.98$\pm$2.27  & 65.94$\pm$1.88  & 71.23$\pm$2.53    \\  \bottomrule
\end{tabular}%}
\label{tb_dpyg_ep}
\end{table}



\hypertarget{app_issues_results}{} 
\subsection{Details of experimental setting and more results}\label{app_issues_results}
\textbf{Experimental setting}. In this part of the experiment, we strictly follow the configurations of each setup and reproduce the results using the provided codes. For the MLP model, we implement a simple two-layer network with 64 hidden units, while the learning rate and weight decay are tuned according to the settings of each setup to ensure a fair comparison. For the MagNet~\cite{magnet} setup, we reproduce the reported MagNet results and include the MLP results. For PyGSD~\cite{dpyg} setup, we get the results in our environment using their baseline code and settings, and report the MLP results. For DUPLEX~\cite{duplex}, we reproduce the DUPLEX results and additionally report the results of graph propagation without using the test edges and the MLP results.

%PyGSD is a software package designed for Signed and Directed Graphs, built on PyTorch Geometric (PyG). 


\begin{table}[h]
\centering
\caption{Link prediction results for CiteSeer dataset under the DUPLEX~\cite{duplex} setup: no superscripts are from the DUPLEX paper, $^\dagger$ indicates reproduction with test set edges in training, and $^\ddagger$ indicates reproduction without test set edges in training.}
%\resizebox{\textwidth}{!}{
\begin{tabular}{@{}lllllll@{}}
\toprule

%\multicolumn{1}{l}{\multirow{2}{*}{Dataset}} & \multirow{2}{*}{Method} & \multicolumn{2}{c}{EP} & \multicolumn{2}{c}{DP} & 3C  & 4C  \\ \cmidrule(l){3-8} 
Method &EP(ACC) &EP(AUC) &DP(ACC) &DP(AUC) &3C(ACC) &4C(ACC) \\ \midrule
%\multicolumn{1}{l}{} & & ACC  & AUC  & ACC   & AUC  & ACC & ACC \\ \midrule

MagNet &80.7$\pm$0.8 &88.3$\pm$0.4 &91.7$\pm$0.9 &96.4$\pm$0.6 &72.0$\pm$0.9 &69.3$\pm$0.4 \\ 
DUPLEX &95.7$\pm$0.5 &98.6$\pm$0.4 &98.7$\pm$0.4 &99.7$\pm$0.2 &94.8$\pm$0.2 &91.1$\pm$1.0  \\ \midrule

DUPLEX$^\dagger$ &92.11$\pm$0.78 &95.85$\pm$0.87 &97.54$\pm$0.54 &98.93$\pm$0.59  &88.22$\pm$1.06 &84.77$\pm$1.01 \\
MLP$^\dagger$ &85.74$\pm$1.80	&93.33$\pm$1.27	&\textbf{97.55$\pm$0.97}	&\textbf{99.58$\pm$0.52}	&81.20$\pm$0.82	&76.30$\pm$0.62  \\ \midrule 

DUPLEX$^\ddagger$  &83.59$\pm$1.47 &89.34$\pm$1.03 &85.56$\pm$1.36 &91.82$\pm$0.98 &76.37$\pm$2.07 &73.80$\pm$2.01 \\
MLP$^\ddagger$ &77.34$\pm$1.86	&87.36$\pm$1.26	&\textbf{89.19$\pm$0.95}	&\textbf{95.97$\pm$0.61}	&67.82$\pm$1.21	&64.25$\pm$1.35 \\ \bottomrule

\end{tabular}
\label{tb:app_duplex}
\end{table}




\textbf{Results with the MagNet setup}. Table~\ref{tb_magnet} presents the results of MagNet~\cite{magnet} as reported in the original paper, alongside the reproduced MagNet and MLP results, with the highest values bolded for emphasis. These results directly correspond to Figures~\ref{fig:dp} and~\ref{fig:ep}.



\begin{wrapfigure}{r}{0.45\textwidth} % 右侧浮动,占 45% 宽度
    \centering
    \vspace{-3mm}
    \includegraphics[width=\linewidth]{figure/citeseer_bar_chart.pdf}
    \vspace{-8mm}
    \caption{The number of samples and accuracy for each class of DUPLEX on the CiteSeer dataset in the 4C task.}
    \vspace{-2mm}
    \label{fig:duplex_cls_citeseer}
\end{wrapfigure}


\textbf{Results with the PyGSD setup}. PyGSD~\cite{dpyg} is a software package for Signed and Directed Graphs, built on PyTorch Geometric (PyG)~\cite{pyg}. Tables~\ref{tb_dpyg_dp} and~\ref{tb_dpyg_ep} present the results of MLP alongside several baselines on the Direction Prediction (DP) and Existence Prediction (EP) tasks, with the highest values highlighted in bold. The baselines include DGCN~\cite{dgcn-tong}, DiGCN~\cite{digcn}, DiGCNIB~\cite{digcn}, and MagNet~\cite{magnet}, as used in PyGSD. Across six datasets, MLP demonstrates competitive performance, achieving state-of-the-art results for both tasks on the Texas and Wisconsin datasets.


\textbf{Results with the DUPLEX setup}. In Table~\ref{tb:app_duplex}, we present the link prediction results for the CiteSeer dataset under the DUPLEX setup. These results, consistent with those in Table~\ref{tb:duplex}, highlight two key observations: (1) MLP outperforms DUPLEX on several tasks, and (2) label leakage in DUPLEX significantly impacts the results.
Additionally, Figure~\ref{fig:duplex_cls_citeseer} illustrates the number of samples and the accuracy for each class in the DUPLEX setup on the CiteSeer dataset for the 4C task. These findings, consistent with those in Figure~\ref{fig:imbalance}, further emphasize the class imbalance issue in the 4C task.

%These issues highlight serious label leakage concerns in existing setups. 



%\begin{table*}[h]
%\centering
%\caption{The results obtained under the DUPLEX~\cite{duplex} experimental setup, where those without superscripts denote the results reported in the DUPLEX paper, $^\dagger$ denotes the results we directly reproduced with the edges of the test set used in the training phase, and $^\ddagger$ denotes the results without the edges of the test set used in the training phase.}
%\resizebox{\textwidth}{!}{
%\begin{tabular}{@{}clllllll@{}}
%\toprule
%\multicolumn{1}{l}{\multirow{2}{*}{Dataset}} & \multirow{2}{*}{Method} & \multicolumn{2}{c}{EP} & \multicolumn{2}{c}{DP} & 3C  & 4C  \\ \cmidrule(l){3-8} 
%\multicolumn{1}{l}{} & & ACC  & AUC  & ACC   & AUC  & ACC & ACC \\ \midrule
%\multirow{6}{*}{Cora} & MagNet &81.4±0.3 &89.4±0.1 &88.9±0.4 &95.4±0.2 &66.8±0.3 &63.0±0.3 \\
%& DUPLEX &93.2±0.1 &95.9±0.1 &95.9±0.1 &97.9±0.2 &92.2±0.1 &88.4±0.4  \\ \cmidrule(l){2-8} 

%& DUPLEX$^\dagger$ &93.49±0.21 &95.61±0.20 &95.25±0.16 &96.34±0.23  &92.41±0.21 &89.76±0.25 \\
%& MLP$^\dagger$ &88.53±0.22	&95.46±0.18	&95.76±0.21	&99.25±0.06	&79.97±0.48	&78.49±0.26  \\ \cmidrule(l){2-8} 

%& DUPLEX$^\ddagger$ &87.43±0.20 &91.16±0.24 &88.43±0.16 &91.74±0.38 &84.53±0.34 &81.36±0.46 \\
%& MLP$^\ddagger$ &84.00±0.29	&91.52±0.25	&90.83±0.16	&96.48±0.28	&72.93±0.21	&71.51±0.20  \\    \midrule

%\multirow{6}{*}{Citeseer} & MagNet &80.7±0.8 &88.3±0.4 &91.7±0.9 &96.4±0.6 &72.0±0.9 &69.3±0.4 \\ 
%& DUPLEX &95.7±0.5 &98.6±0.4 &98.7±0.4 &99.7±0.2 &94.8±0.2 &91.1±1.0  \\ \cmidrule(l){2-8} 

%& DUPLEX$^\dagger$ &92.11±0.78 &95.85±0.87 &97.54±0.54 &98.93±0.59  &88.22±1.06 &84.77±1.01 \\
%& MLP$^\dagger$ &85.74±1.80	&93.33±1.27	&97.55±0.97	&99.58±0.52	&81.20±0.82	&76.30±0.62  \\ \cmidrule(l){2-8} 

%& DUPLEX$^\ddagger$  &83.59±1.47 &89.34±1.03 &85.56±1.36 &91.82±0.98 &76.37±2.07 &73.80±2.01 \\
%& MLP$^\ddagger$ &77.34±1.86	&87.36±1.26	&89.19±0.95	&95.97±0.61	%&67.82±1.21	&64.25±1.35 \\ \bottomrule
%\end{tabular}}
%\end{table*}



% \begin{figure}[h]
%     \centering
%    \vspace{-1mm}
%   %\hspace{-3mm}  
%    \includegraphics[width=75mm]{figure/citeseer_bar_chart.pdf}
%    \caption{The number of samples and the accuracy for each class of DUPLEX on the CiteSeer dataset in the 4C task.}
%    \vspace{-1mm}
%     \label{fig:duplex_cls_citeseer}
%  \end{figure}


 



\hypertarget{app_sdgae}{} 
\section{Details and Experimental Setting for SDGAE}\label{app_sdgae}

\hypertarget{app_sdgae_conv}{} 
\subsection{More details of graph convolution}\label{app_sdgae_conv}
The graph convolutional layer of SDGAE is defined as: 
\begin{equation}\label{eq:app_bg_A_hat}
    \left[ 
        \begin{array}{c}
            \mS^{(\ell+1)} \\
            \mT^{(\ell+1)}
        \end{array}
    \right] =
    \left[ 
        \begin{array}{c}
            w^{(\ell)}_S \\
            w^{(\ell)}_T
        \end{array}
    \right] 
    \odot
    \left[ 
        \begin{array}{cc}
            \mathbf{0} & \Tilde{\mA} \\
            \Tilde{\mA}^{\top} & \mathbf{0}
        \end{array}
    \right]
    \left[ 
        \begin{array}{c}
            \mS^{(\ell)} \\
            \mT^{(\ell)}
        \end{array}
    \right] 
    +
    \left[ 
        \begin{array}{c}
            \mS^{(\ell)} \\
            \mT^{(\ell)}
        \end{array}
    \right].
\end{equation}
Expanding this formulation, the source and target embeddings at each layer are denoted as:
\begin{equation}\label{app_eq_sdgae}
    \mS^{(\ell+1)} =w_{S}^{(\ell)}\Tilde{\mA}\mT^{(\ell)}+\mS^{(\ell)}, \quad \mT^{(\ell+1)} =w_{T}^{(\ell)}\Tilde{\mA}^{\top}\mS^{(\ell)}+\mT^{(\ell)}.
\end{equation}
If we denote $\mX_S = \mS^{(0)}=\mathrm{MLP}_S(\mX)$ and $\mX_T = \mT^{(0)}=\mathrm{MLP}_T(\mX)$, we can derive:
\begin{align*}
\mS^{(1)} &= w_{S}^{(0)}\Tilde{\mA}\mX_T + \mX_S, \\
\mT^{(1)} &= w_{T}^{(0)}\Tilde{\mA}^{\top}\mX_S + \mX_T, \\
\mS^{(2)} &= \mX_S + \left(w_{S}^{(1)} + w_{S}^{(0)}\right)\Tilde{\mA}\mX_T + w_{S}^{(1)}w_{T}^{(0)}\Tilde{\mA}\Tilde{\mA}^{\top}\mX_S, \\
\mT^{(2)} &= \mX_T + \left(w_{T}^{(1)} + w_{T}^{(0)}\right)\Tilde{\mA}^{\top}\mX_S + w_{T}^{(1)}w_{S}^{(0)}\Tilde{\mA}^{\top}\Tilde{\mA}\mX_T.
\end{align*}
%The result shows the source and target embeddings corresponding to the number of layers $L=1$ and $L=2$. It can be found that the propagation is very similar to the spectral-based GNNs. Specifically, the propagation expression of the spectral-based GNNs is $\mY = \nolimits\sum_{\ell=0}^{L}w_{\ell}\Tilde{\mA}^{\ell}$, where $w_{\ell}$ denotes the learnable polynomial weights and $\Tilde{\mA}$ denotes the symmetrically normalised undirected graph adjacency matrix. By comparison, it can be found that SDGAE also achieves spectral--based propagation by learning different weights $w_S$ and $w_T$, and further spectrum analyses about it will be the subject of further research.
These results represent the source and target embeddings for layers \(L=1, 2\). The propagation pattern closely resembles that of spectral-based GNNs~\cite{gprgnn,bernnet,chebnetii}. Specifically, the propagation in spectral-based GNNs is expressed as \(\mY = \sum_{\ell=0}^{L}w_{\ell}\Tilde{\mA}^{\ell}\), where \(w_{\ell}\) denotes the learnable polynomial weights and \(\Tilde{\mA}\) represents the symmetrically normalized adjacency matrix of an undirected graph. By comparison, SDGAE achieves spectral-based propagation by learning separate weights \(w_S\) and \(w_T\). Further spectrum analysis of SDGAE will be explored in future research.





\textbf{Time Complexity}. The primary time complexity of SDGAE arises from the graph convolution process. Based on Equation~(\ref{app_eq_sdgae}), the time complexity of SDGAE's convolution is given by \( O(2Lmd) \), where \( m \) is the number of edges, \( L \) is the number of layers, and \( d \) is the embedding dimension. This complexity scales linearly with \( m \) and is lower than that of DiGAE, which involves multiple learnable weight matrices.


\begin{wraptable}{r}{105mm}
\centering
\vspace{-8mm}
\caption{The parameters of SDGAE on different datasets.}
\begin{tabular}{lcccccc}
\toprule
Datasets &hidden &embedding &MLP layer &$L$ &lr &wd  \\ \midrule
Cora-ML & 64 & 64 &1 &5 &0.01 &0.0 \\
CiteSeer& 64 & 64 &1 &5 &0.01 &0.0 \\
Photo & 64 & 64 &2 &5 &0.005 &0.0 \\
Computers & 64 & 64 &2 & 3 & 0.005 &0.0\\
WikiCS & 64 & 64 & 2&5 & 0.005 &5e-4 \\
Slashdot & 64 & 64 &2 &5 &0.01 & 5e-4 \\
Epinions & 64 & 64 &2 &5 & 0.005 &0.0 \\
\bottomrule
\end{tabular}
\vspace{-2mm}
\label{app_para_sdage}
\end{wraptable}

\hypertarget{app_sdgae_setting}{} 
\subsection{Experimental setting}\label{app_sdgae_setting}
For the SDGAE experimental setting, we aligned our settings with those of other baselines in DirLinkBench to ensure fairness. For the MLP used in $\mX$ initialization, we set the number of layers to one or two, matching DiGAE's convolutional layer configurations. The number of hidden units and the embedding dimension were both set to 64. The learning rate (lr) was chosen as either 0.01 or 0.005, and weight decay (wd) was set to 0.0 or 5e-4, following the configurations of most GNN baselines. The number of convolutional layers $L$ was set to 3, 4, or 5. We performed a grid search to optimize parameters on the validation set, and Table~\ref{app_para_sdage} presents the corresponding SDGAE parameters for different datasets.  




\subsection{Example Entry from LiveBench Reasoning}
\vspace{0.4cm}
\begin{tcolorbox}[title=Question, breakable]
In this question, assume each person either always tells the truth or always lies. The person at the campground thinks their friend is lying. Mateo is at the aquarium. The person at the restaurant says the person at the hotel lies. Farid is at the movie theater. The person at the movie theater says the person at the campground lies. Ryan is at the shopping mall. The person at the cafe says the person at the campground lies. The person at the observatory says the person at the museum lies. The person at the museum says the person at the restaurant tells the truth. The person at the campground says the person at the airport tells the truth. The person at the campground tells the truth. Luna is at the restaurant. The person at the movie theater saw a firetruck. Isabella is at the observatory. The person at the amusement park says the person at the museum tells the truth. The person at the restaurant says the person at the shopping mall tells the truth. Liam is at the beach. Ayaan is at the airport. The person at the restaurant says the person at the beach tells the truth. Maya is at the amusement park. The person at the restaurant says the person at the aquarium lies. The person at the restaurant says the person at the movie theater lies. The person at the library says the person at the campground tells the truth. Elowen is at the hotel. Priya is at the library. The person at the gym says the person at the restaurant tells the truth. Kehinde is at the cafe. Anika is at the museum. Quan is at the campground. Soren is at the gym. Does the person at the museum tell the truth? Does the person at the restaurant tell the truth? Does the person at the movie theater tell the truth? Think step by step, and then put your answer in **bold** as a list of three words, yes or no (for example, **yes, no, yes**). If you don't know, guess.
\end{tcolorbox}

\definecolor{mydarkgreen}{rgb}{0.0, 0.5, 0.0} %
\begin{paracol}{2}
\switchcolumn[0]
\begin{tcolorbox}[breakable, colframe=mydarkgreen!70, colback=mydarkgreen!10, title=Correct response]
Here's how we can break down this logic puzzle:

1. **Quan is at the campground and tells the truth.** This is a given fact.

2. **Quan (at the campground) says the person at the airport tells the truth.** Since Quan tells the truth, the person at the airport (Ayaan) tells the truth.

3. **The person at the movie theater (Farid) says the person at the campground (Quan) lies.** Since Quan tells the truth, Farid is lying.

4. **The person at the restaurant (Luna) says the person at the hotel (Elowen) lies.**  We don't know yet if Luna is lying or telling the truth.

5. **The person at the museum (Anika) says the person at the restaurant (Luna) tells the truth.** We still don't know if either of them are truthful.

6. **The person at the restaurant (Luna) says the person at the shopping mall (Ryan) tells the truth.** We still don't know Luna's truthfulness.

7. **The person at the restaurant (Luna) says the person at the beach (Liam) tells the truth.** Luna's truthfulness remains unknown.

8. **The person at the restaurant (Luna) says the person at the aquarium (Mateo) lies.**  We still don't know about Luna.

9. **The person at the restaurant (Luna) says the person at the movie theater (Farid) lies.**  We know Farid lies, so Luna is telling the truth in *this* statement.

10. **Since Luna tells the truth about Farid, Luna *always* tells the truth.**  This unlocks several other pieces of information.

11. **Luna (at the restaurant) says the person at the hotel (Elowen) lies.** Since Luna tells the truth, Elowen lies.

12. **Anika (at the museum) says the person at the restaurant (Luna) tells the truth.**  Since Luna tells the truth, Anika is also telling the truth.

13. **The person at the cafe (Kehinde) says the person at the campground lies.** Since Quan tells the truth, Kehinde is lying.


Therefore:

* **Anika (museum):** Tells the truth.
* **Luna (restaurant):** Tells the truth.
* **Farid (movie theater):** Lies.

**yes, yes, no**
\end{tcolorbox}

\switchcolumn
\begin{tcolorbox}[breakable, colframe=red!70, colback=red!10, title=Incorrect Response]
Here's how we can solve this logic puzzle:

1. **Quan is at the campground and tells the truth.** This is given.

2. **The person at the airport (Ayaan) tells the truth.** Quan (campground) says the person at the airport tells the truth, and Quan tells the truth.

3. **Anika (museum) lies.**  The person at the observatory (Isabella) says the person at the museum lies.  Since we don't know yet whether Isabella tells the truth, this doesn't tell us anything about Anika.  We'll have to come back to this.

4. **Luna (restaurant) lies.**  The person at the restaurant says the person at the hotel (Elowen) lies.  Later, the person at the restaurant says the person at the campground (Quan) lies.  Since Quan tells the truth, Luna (restaurant) must be lying.

5. **Anika (museum) lies.** Because Luna lies, and Luna says that the person at the hotel lies, Elowen tells the truth. Because Anika says the person at the restaurant (Luna) tells the truth, and Luna actually lies, Anika (Museum) must lie.


6. **Farid (movie theater) lies.** The person at the movie theater says the person at the campground lies. Since Quan (campground) tells the truth, Farid (movie theater) must be lying.

Therefore:

* **Anika (museum): lies**
* **Luna (restaurant): lies**
* **Farid (movie theater): lies**

**Answer: no, no, no** 
\end{tcolorbox}

\end{paracol}


\subsection{Example Entry from LiveBench Math}
\vspace{0.4cm}
\begin{tcolorbox}[title=Question, breakable]
How many positive perfect squares less than $2023$ are divisible by $5$? $\textbf{(A) }11\qquad\textbf{(B) }10\qquad\textbf{(C) }9\qquad\textbf{(D) }12\qquad\textbf{(E) }8$ If you cannot determine the correct multiple-choice answer, take your best guess. Once you have your answer, please duplicate that letter five times in a single string. For example, if the answer is F, then write FFFFF.
\end{tcolorbox}
\definecolor{mydarkgreen}{rgb}{0.0, 0.5, 0.0} %


\begin{paracol}{2}

\switchcolumn[0]
\begin{tcolorbox}[breakable, colframe=mydarkgreen!70, colback=mydarkgreen!10, title=Correct response]
Let $n$ be a positive perfect square divisible by $5$.
Then $n = k^2$ for some positive integer $k$, and $5 | n$, which implies that $5|k^2$.
Since $5$ is a prime number, $5|k^2$ means $5|k$.
Thus, $k$ must be a multiple of $5$, i.e., $k=5m$ for some positive integer $m$.
Then $n = k^2 = (5m)^2 = 25m^2$.
Thus $n$ must be a multiple of $25$.
We are looking for the number of positive perfect squares less than $2023$ that are divisible by $5$.
The perfect squares that are divisible by $5$ must be multiples of $25$.
We seek the number of multiples of $25$ that are perfect squares less than $2023$.
Since $n=25m^2 < 2023$, we have $m^2 < \frac{2023}{25} \approx 80.92$.
Thus, we are looking for perfect squares $m^2$ less than $80.92$.
The perfect squares are $1, 4, 9, 16, 25, 36, 49, 64, 81, \dots$.
The perfect squares less than $80.92$ are $m^2=1, 4, 9, 16, 25, 36, 49, 64$. There are $8$ such values.
The corresponding values of $n = 25m^2$ are
$25(1)=25$
$25(4)=100$
$25(9)=225$
$25(16)=400$
$25(25)=625$
$25(36)=900$
$25(49)=1225$
$25(64)=1600$
The next multiple of $25$ that is a perfect square would be $25(81)=2025$, which is not less than $2023$.
So, there are $8$ perfect squares less than $2023$ that are divisible by $5$.

Final Answer: The final answer is $\boxed{EEEEE}$
\end{tcolorbox}

\switchcolumn
\begin{tcolorbox}[breakable, colframe=red!70, colback=red!10, title=Incorrect Response]
Let $n$ be a positive integer such that $n^2$ is divisible by $5$.
Then $n^2$ is divisible by $5$. Since $5$ is a prime number, $n$ must also be divisible by $5$.
Thus $n$ can be written in the form $n = 5k$ for some integer $k$.
We want to find how many positive perfect squares $n^2$ are less than $2023$.
Since $n = 5k$, we have $n^2 = (5k)^2 = 25k^2 < 2023$.
Dividing by $25$ we get
$$k^2 < \frac{2023}{25} \approx 80.92$$
We are looking for integer values of $k$ such that $k^2 < 80.92$.
Since $k$ must be positive, $k$ can take values $1, 2, 3, \dots$. We have
\begin{itemize}
    \item $1^2 = 1 < 80.92$
    \item $2^2 = 4 < 80.92$
    \item $3^2 = 9 < 80.92$
    \item $4^2 = 16 < 80.92$
    \item $5^2 = 25 < 80.92$
    \item $6^2 = 36 < 80.92$
    \item $7^2 = 49 < 80.92$
    \item $8^2 = 64 < 80.92$
    \item $9^2 = 81 \nless 80.92$
\end{itemize}
Thus $k$ can take values from $1$ to $8$. The corresponding $n$ values are $n = 5k = 5, 10, 15, 20, 25, 30, 35, 40$. The perfect squares less than $2023$ and divisible by $5$ are $5^2, 10^2, 15^2, 20^2, 25^2, 30^2, 35^2, 40^2$ or $25, 100, 225, 400, 625, 900, 1225, 1600$.
There are $8$ such numbers.

Final Answer: The final answer is $\boxed{C}$CCCC
\end{tcolorbox}

\end{paracol}


\subsection{Example Entry from MATH}
\vspace{0.4cm}
\begin{tcolorbox}[title=Question, breakable]
If 4 daps = 7 yaps, and 5 yaps = 3 baps, how many daps equal 42 baps?
\end{tcolorbox}
\definecolor{mydarkgreen}{rgb}{0.0, 0.5, 0.0} %


\begin{paracol}{2}

\switchcolumn[0]
\begin{tcolorbox}[breakable, colframe=mydarkgreen!70, colback=mydarkgreen!10, title=Correct response]
Let 'd' represent daps, 'y' represent yaps, and 'b' represent baps.
We are given:
4 daps = 7 yaps, which can be written as:
4d = 7y

and

5 yaps = 3 baps, which can be written as:
5y = 3b

We want to find the number of daps that are equal to 42 baps. Let 'x' be the number of daps. We want to find x such that:
xd = 42b

From 5y = 3b, we can express yaps in terms of baps:
y = (3/5)b

Now substitute this value of y into the first equation, 4d = 7y:
4d = 7 * (3/5)b
4d = (21/5)b

We want to find xd = 42b, so we can set:
xd = 42b

We have 4d = (21/5)b, so we can solve for d:
d = (21/20)b

Now substitute this into xd = 42b:
x * (21/20)b = 42b

Since b is not zero, we can divide both sides by b:
x * (21/20) = 42

Now, solve for x:
x = 42 * (20/21)
x = (42/21) * 20
x = 2 * 20
x = 40

So, 40 daps are equal to 42 baps.

$\boxed{40}$
Final Answer: The final answer is $\boxed{40}$

\end{tcolorbox}

\switchcolumn
\begin{tcolorbox}[breakable, colframe=red!70, colback=red!10, title=Incorrect Response]
Let 'd' represent daps, 'y' represent yaps, and 'b' represent baps.
We are given two relationships:

1. 4 daps = 7 yaps, which can be written as $4d = 7y$
2. 5 yaps = 3 baps, which can be written as $5y = 3b$

We want to find how many daps are equal to 42 baps.  We want to find $d$ in terms of $b$, and then evaluate for $b=42$.

First, we can find the relationship between yaps and baps from the second equation:
$5y = 3b$, so $y = \frac{3}{5}b$

Now, substitute this expression for $y$ in terms of $b$ into the first equation:
$4d = 7y$
$4d = 7\left(\frac{3}{5}b\right)$
$4d = \frac{21}{5}b$

Now, we want to find $d$ in terms of $b$, so divide both sides by 4:
$d = \frac{21}{5 \cdot 4}b$
$d = \frac{21}{20}b$

Finally, we want to find how many daps equal 42 baps, so we substitute $b=42$:
$d = \frac{21}{20} \times 42$
$d = \frac{21 \times 42}{20}$
$d = \frac{882}{20}$
$d = \frac{441}{10}$
$d = 44.1$

So, 44.1 daps equal 42 baps.

Final Answer: The final answer is $\boxed{44.1}$

\end{tcolorbox}

\end{paracol}

\subsection{Example of Response with Invalid Reasoning but Correct Final Answer}

\vspace{0.4cm}
\begin{tcolorbox}[title=Question, breakable]
You have seven bags of gold coins. Each bag has the same number of gold coins. One day, you find a bag of 53 coins. You decide to redistribute the number of coins you have so that all eight bags you hold have the same number of coins. You successfully manage to redistribute all the coins, and you also note that you have more than 200 coins. What is the smallest number of coins you could have had before finding the bag of 53 coins?
\end{tcolorbox}
\definecolor{mydarkgreen}{rgb}{0.0, 0.5, 0.0} %


\begin{tcolorbox}[breakable, colframe=red!70, colback=red!10, title=Incorrect Response Reaching Correct Final Answer]
Let $n$ be the number of coins in each of the seven bags. After finding the bag of 53 coins, you have a total of $7n + 53$ coins.
You redistribute these coins into 8 bags such that each bag contains the same number of coins.
Therefore, the total number of coins is divisible by 8, so $7n+53$ is divisible by 8.
$7n + 53 \equiv 0 \pmod{8}$.
We want to find $n$.
$7n \equiv -53 \pmod{8}$.
Since $-53 \equiv -53 + 8(7) \equiv -53 + 56 \equiv 3 \pmod{8}$, we have
$7n \equiv 3 \pmod{8}$.

Since $7 \equiv -1 \pmod{8}$, we multiply both sides by $-1$ to get
$-n \equiv -7n \equiv -3 \pmod{8}$, so $n \equiv -3 \equiv 5 \pmod{8}$.
Thus $n$ must be of the form $8k + 5$ for some integer $k \geq 0$.
The total number of coins originally was $7n$. Since $n=8k+5$, the total number of coins originally is
$7(8k+5) = 56k + 35$.
We are given that the total number of coins is greater than 200.
So $7n > 200$, or $n > \frac{200}{7} \approx 28.57$.
Since $n = 8k + 5$, we want the smallest integer $k$ such that $8k + 5 > \frac{200}{7}$.
For $k=0$, $n=5$, so $7n=35<200$.
For $k=1$, $n=13$, so $7n=91<200$.
For $k=2$, $n=21$, so $7n=147<200$.
For $k=3$, $n=29$, so $7n=203>200$.

We check that $7n + 53 = 203 + 53 = 256$ which is divisible by $8$ ($256 = 8 \times 32$).
Then $256/8 = 32$ coins in each bag.

Thus the smallest number of coins you could have had before finding the bag is $7n = 7(29) = 203$.

Final Answer: The final answer is $\boxed{203}$.
\end{tcolorbox}


\clearpage
\hypertarget{app_complete_res}{}
\section{Complete Results of DirLinkBench on Seven Datasets}\label{app_complete_res}

\hypertarget{app_complete_res_cora}{}
\subsection{Results on Cora-ML}\label{app_complete_res_cora}
%Table~\ref{app_tb_cora_ml} shows the results of DirLinkBench on Cora-ML.
\begin{table}[h]
    \centering
    \caption{Benchmark Results on Cora-ML. For all methods, $_{\text{F}}$ indicates the use of original node features as input, while $_{\text{D}}$ indicates the use of in/out degrees as input. Results ranked \hig{1}{first} and \hig{2}{second} are highlighted.}
    \label{app_tb_cora_ml}
    \resizebox{\textwidth}{!}{
    \begin{tabular}{lccccccc}
        \toprule
        Method & Hits@20 & Hits@50 & Hits@100 & MRR & AUC & AP & ACC \\
        \midrule
        STRAP & 67.10$\pm$2.55 & 75.05$\pm$1.66 & 79.09$\pm$1.57 & \hig{1}{30.91$\pm$7.48} & 87.38$\pm$0.83 & 90.46$\pm$0.70 & 78.47$\pm$0.77 \\
        
        ODIN & 27.10$\pm$3.69 & 40.99$\pm$3.10 & 54.85$\pm$2.53 & 10.75$\pm$3.51 & 85.15$\pm$0.84 & 85.41$\pm$0.95 & 77.16$\pm$0.84  \\
        
        ELTRA &\hig{2}{70.74$\pm$5.47} & \hig{2}{81.93$\pm$2.38} & \hig{2}{87.45$\pm$1.48} & 19.77$\pm$5.22 & 94.83$\pm$0.58 & 95.37$\pm$0.52 & 85.40$\pm$0.45  \\ \midrule
        
        MLP$_{\text{F}}$ & 27.16$\pm$6.70 & 44.28$\pm$5.72 & 60.61$\pm$6.64 & 9.02$\pm$3.58 & 89.93$\pm$2.09 & 88.55$\pm$2.51 & 81.32$\pm$2.41 \\ 
        MLP$_{\text{D}}$ & 29.84$\pm$4.83 & 44.11$\pm$4.73 & 56.74$\pm$2.03 & 11.84$\pm$3.40 & 86.44$\pm$0.74 & 86.58$\pm$0.88 & 77.95$\pm$0.95 \\ \midrule
        
        GCN$_{\text{F}}$ & 37.88$\pm$5.60 & 55.31$\pm$4.46 & 70.15$\pm$3.01 & 14.25$\pm$3.49 & 92.01$\pm$1.64 & 91.45$\pm$1.47 & 82.82$\pm$0.86 \\
        GCN$_{\text{D}}$& 31.36$\pm$3.89 & 45.03$\pm$2.92 & 58.77$\pm$2.96 & 12.56$\pm$2.86 & 86.12$\pm$0.87 & 86.79$\pm$1.01 & 77.81$\pm$0.87   \\ \midrule
        
        GAT$_{\text{F}}$ & 33.42$\pm$8.93 & 58.40$\pm$6.55 & 79.72$\pm$3.07 & 9.30$\pm$3.39 & 94.57$\pm$0.45 & 92.77$\pm$0.42 & 88.95$\pm$0.71  \\
        GAT$_{\text{D}}$ & 30.19$\pm$4.23 & 43.53$\pm$2.93 & 55.09$\pm$3.34 & 11.07$\pm$3.68 & 85.52$\pm$1.27 & 86.19$\pm$0.96 & 77.36$\pm$1.13 \\ \midrule
        
        APPNP$_{\text{F}}$ & 55.47$\pm$4.63 & 75.16$\pm$4.09 & 86.02$\pm$2.88 & 22.49$\pm$6.33 & 95.94$\pm$0.57 & \hig{2}{95.82$\pm$0.59} &\hig{2}{89.90$\pm$0.73} \\ 
        APPNP$_{\text{D}}$ & 32.21$\pm$3.65 & 45.51$\pm$3.76 & 60.02$\pm$3.36 & 11.89$\pm$2.65 & 88.00$\pm$1.08 & 88.01$\pm$1.17 & 79.28$\pm$1.13 \\\midrule
        
        DGCN$_{\text{F}}$ & 30.86$\pm$4.17 & 41.82$\pm$4.43 & 54.38$\pm$4.97 & 11.01$\pm$4.23 & 85.55$\pm$3.40 & 85.66$\pm$2.35 & 77.82$\pm$3.49  \\
        DGCN$_\text{D}$ & 36.30$\pm$4.43 & 50.25$\pm$3.42 & 63.32$\pm$2.59 & 13.98$\pm$4.24 & 87.87$\pm$0.96 & 88.67$\pm$0.84 & 80.26$\pm$0.69 \\ \midrule
        
        DiGCN$_{\text{F}}$ & 30.05$\pm$5.45 & 48.87$\pm$5.25 & 63.21$\pm$5.72 & 10.16$\pm$4.25 & 89.48$\pm$1.83 & 89.10$\pm$1.99 & 82.47$\pm$0.97 \\
        DiGCN$_{\text{D}}$ & 34.39$\pm$3.69 & 49.66$\pm$4.14 & 60.10$\pm$4.04 & 12.49$\pm$3.83 & 85.87$\pm$1.22 & 87.13$\pm$1.18 & 77.88$\pm$1.01 \\ \midrule
        
        DiGCNIB$_{\text{F}}$& 45.90$\pm$4.97 & 66.62$\pm$3.67 & 80.57$\pm$3.21 & 17.08$\pm$3.90 & 94.67$\pm$0.56 & 94.21$\pm$0.65 & 87.25$\pm$0.58 \\

        DiGCNIB$_{\text{D}}$ & 37.18$\pm$4.60 & 51.64$\pm$3.69 & 64.72$\pm$3.43 & 12.71$\pm$4.08 & 89.23$\pm$0.92 & 89.36$\pm$0.85 & 81.05$\pm$1.05 \\ \midrule
        
        DirGNN$_{\text{F}}$ & 42.48$\pm$7.34 & 59.41$\pm$4.00 & 76.13$\pm$2.85 & 12.01$\pm$3.57 & 93.05$\pm$0.87 & 92.52$\pm$0.79 & 85.83$\pm$0.93 \\
        DirGNN$_{\text{D}}$ & 32.29$\pm$2.25 & 45.53$\pm$2.60 & 58.22$\pm$2.50 & 13.34$\pm$5.56 &87.06$\pm$ 0.77 &87.35$\pm$0.80 & 77.93$\pm$1.14 \\ \midrule
        
        MagNet$_{\text{F}}$ & 26.87$\pm$3.82 & 40.58$\pm$3.19 & 53.51$\pm$2.00 & 10.08$\pm$3.47 & 85.21$\pm$0.79 & 85.53$\pm$1.00 & 76.93$\pm$0.79 \\
        MagNet$_{\text{D}}$ & 29.38$\pm$3.39 & 43.28$\pm$3.76 & 56.54$\pm$2.95 & 11.19$\pm$3.04 & 85.23$\pm$0.84 & 86.06$\pm$0.94 & 77.51$\pm$0.92 \\ \midrule

        DUPLEX$_{\text{F}}$ & 21.73$\pm$3.19 & 34.98$\pm$2.37 & 69.00$\pm$2.52 & 7.74$\pm$1.95 & 88.02$\pm$0.95 & 86.62$\pm$1.43 & 82.28$\pm$0.93 \\
        DUPLEX$_{\text{D}}$& 17.48$\pm$4.69 & 32.58$\pm$8.33 & 54.50$\pm$7.49 & 5.46$\pm$2.10 & 82.97$\pm$4.47 & 82.97$\pm$2.85 & 75.59$\pm$4.87 \\ \midrule

        DHYPR$_{\text{F}}$ & 59.81$\pm$4.79 & 77.45$\pm$2.56 & 86.81$\pm$1.60 & 20.56$\pm$5.10 & \hig{2}{96.13$\pm$0.28} & 95.84$\pm$0.39 & 86.04$\pm$0.66  \\
        DHYPR$_{\text{D}}$ & 15.09$\pm$4.40 & 28.96$\pm$6.54 & 42.93$\pm$6.63 & 4.05$\pm$1.16 & 80.83$\pm$2.64 & 79.97$\pm$3.02 & 72.74$\pm$3.50 \\ \midrule
        
        DiGAE$_{\text{F}}$& 56.13$\pm$3.80 & 72.23$\pm$2.51 & 82.06$\pm$2.51 & 20.53$\pm$4.21 & 92.56$\pm$0.66 & 93.70$\pm$0.54 & 86.25$\pm$0.80  \\
        DiGAE$_{\text{D}}$ & 35.40$\pm$4.05 & 49.12$\pm$3.44 & 61.30$\pm$2.70 & 13.71$\pm$4.22 & 86.82$\pm$0.85 & 87.35$\pm$0.84 & 76.01$\pm$1.04 \\ \midrule

        %SDGAE$_{\text{F}}$ &\textbf{70.89$\pm$3.35} & \textbf{83.63$\pm$2.15} & \textbf{90.37$\pm$1.33} & \underline{28.45$\pm$5.82} & \textbf{97.24$\pm$0.34} & \textbf{97.21$\pm$0.17} & \textbf{91.36$\pm$0.70}  \\ 
        SDGAE$_{\text{F}}$ &\hig{1}{70.89$\pm$3.35} & \hig{1}{83.63$\pm$2.15} &\hig{1}{90.37$\pm$1.33} & \hig{2}{28.45$\pm$5.82} & \hig{1}{97.24$\pm$0.34} & \hig{1}{97.21$\pm$0.17} &\hig{1}{91.36$\pm$0.70}  \\

        
        SDGAE$_{\text{D}}$& 35.53$\pm$5.41 & 49.34$\pm$4.60 & 61.40$\pm$3.18 & 14.21$\pm$3.71 & 87.38$\pm$1.06 & 88.19$\pm$1.18 & 78.99$\pm$1.34  \\

        \bottomrule
    \end{tabular}}
\end{table}


\clearpage
\hypertarget{app_complete_res_citeseer}{}
\subsection{Results on CiteSeer}\label{app_complete_res_citeseer}
%Table~\ref{app_tb_citeseer} shows the results of DirLinkBench on CiteSeer.
\begin{table}[h]
    \centering
    \caption{Benchmark Results on CiteSeer. For all methods, $_{\text{F}}$ indicates the use of original node features as input, while $_{\text{D}}$ indicates the use of in/out degrees as input. Results ranked \hig{1}{first} and \hig{2}{second} are highlighted.}
    \label{app_tb_citeseer}
    \resizebox{\textwidth}{!}{
    \begin{tabular}{lccccccc}
        \toprule
        Model & Hits@20 & Hits@50 & Hits@100 & MRR & AUC & AP & ACC \\
        \midrule
        STRAP & 63.28$\pm$1.46 & 67.05$\pm$1.32 & 69.32$\pm$1.29 & \hig{2}{40.70$\pm$6.95} & 81.82$\pm$1.44 & 84.15$\pm$0.84 & 75.01$\pm$1.40 \\
        
        ODIN & 30.74$\pm$3.78 & 47.91$\pm$2.57 & 63.95$\pm$2.98 & 11.05$\pm$2.74 & 82.22$\pm$1.05 & 81.74$\pm$1.10 & 73.63$\pm$1.19  \\
        
        ELTRA & 72.34$\pm$1.76 & 79.44$\pm$1.54 & 84.97$\pm$1.90 & 27.87$\pm$9.03 & 90.98$\pm$0.86 & 92.71$\pm$0.81 & 80.86$\pm$1.15  \\ \midrule
        
        MLP$_{\text{F}}$& 35.55$\pm$4.39 & 52.31$\pm$4.60 & 70.27$\pm$3.40 & 16.08$\pm$4.88 & 85.47$\pm$1.35 & 85.16$\pm$1.29 & 74.26$\pm$1.64  \\
        MLP$_{\text{D}}$ & 40.27$\pm$4.48 & 56.27$\pm$3.64 & 68.78$\pm$2.21 & 15.94$\pm$3.04 & 84.09$\pm$1.02 & 84.86$\pm$1.12 & 74.81$\pm$1.32  \\ \midrule
        
        GCN$_{\text{F}}$ & 42.27$\pm$7.11 & 62.65$\pm$4.22 & 80.36$\pm$3.07 & 18.28$\pm$6.7 & 89.94$\pm$1.04 & 88.88$\pm$1.38 & 79.47$\pm$1.73 \\
        GCN$_{\text{D}}$ & 43.96$\pm$2.35 & 56.24$\pm$3.62 & 66.15$\pm$2.07 & 19.36$\pm$4.96 & 81.24$\pm$1.55 & 84.33$\pm$0.90 & 73.92$\pm$1.43 \\ \midrule

        
        GAT$_{\text{F}}$ & 48.29$\pm$6.04 & 71.86$\pm$4.65 & 85.88$\pm$4.98 & 16.71$\pm$5.56 & 92.90$\pm$0.74 & 91.12$\pm$1.53 & 85.94$\pm$1.21 \\
        GAT$_{\text{D}}$ & 40.29$\pm$3.25 & 55.62$\pm$1.92 & 65.82$\pm$2.87 & 19.00$\pm$4.18 & 82.72$\pm$0.67 & 83.96$\pm$0.60 & 74.18$\pm$1.18 \\ \midrule
       

        APPNP$_{\text{F}}$& 59.86$\pm$6.54 & 77.57$\pm$5.68 & 83.57$\pm$4.90 & 20.91$\pm$4.39 & 93.58$\pm$0.71 & 93.09$\pm$0.68 & 86.23$\pm$1.49  \\
        APPNP$_{\text{D}}$ & 45.44$\pm$3.38 & 58.18$\pm$1.55 & 68.47$\pm$3.00 & 20.18$\pm$5.50 & 83.48$\pm$1.17 & 85.72$\pm$1.00 & 74.86$\pm$1.19 \\ \midrule

        
        DGCN$_{\text{F}}$ & 39.64$\pm$6.53 & 50.85$\pm$2.91 & 62.97$\pm$4.79 & 19.15$\pm$4.34 & 80.60$\pm$2.90 & 81.80$\pm$1.85 & 74.12$\pm$2.49  \\
        DGCN$_{\text{D}}$ & 45.94$\pm$3.01 & 58.38$\pm$2.85 & 68.97$\pm$3.39 & 19.53$\pm$4.83 & 83.38$\pm$1.41 & 85.44$\pm$1.01 & 74.83$\pm$1.46  \\ \midrule

        DiGCN$_{\text{F}}$ & 38.48$\pm$3.50 & 53.30$\pm$5.92 & 70.95$\pm$4.67 & 15.79$\pm$4.05 & 87.41$\pm$2.24 & 85.02$\pm$2.90 & 80.38$\pm$1.17 \\
        DiGCN$_{\text{D}}$ & 40.65$\pm$4.23 & 52.87$\pm$3.85 & 65.46$\pm$3.70 & 17.39$\pm$4.82 & 83.50$\pm$1.12 & 84.75$\pm$0.93 & 74.65$\pm$1.36  \\ \midrule

        
        DiGCNIB$_{\text{F}}$& 44.90$\pm$7.91 & 69.86$\pm$5.13 & 85.32$\pm$3.70 & 16.61$\pm$5.61 & 92.50$\pm$0.57 & 90.35$\pm$1.43 & 85.98$\pm$1.19 \\
        DiGCNIB$_{\text{D}}$ & 45.68$\pm$3.50 & 58.99$\pm$3.92 & 70.29$\pm$2.39 & 20.22$\pm$3.31 & 85.81$\pm$0.69 & 86.73$\pm$0.85 & 76.54$\pm$1.27 \\ \midrule

        DirGNN$_{\text{F}}$ & 44.95$\pm$7.31 & 64.00$\pm$3.19 & 76.83$\pm$4.24 & 16.19$\pm$4.51 & 88.44$\pm$1.28 & 88.59$\pm$1.14 & 80.34$\pm$1.52 \\
        DirGNN$_{\text{D}}$ & 40.16$\pm$4.84 & 53.44$\pm$3.38 & 65.62$\pm$2.46 & 16.95$\pm$5.45 & 83.67$\pm$1.08 & 84.59$\pm$1.18 & 73.75$\pm$1.31 \\ \midrule
        
        MagNet$_{\text{F}}$ & 32.18$\pm$5.04 & 47.44$\pm$3.24 & 61.55$\pm$8.45 & 12.97$\pm$1.73 & 82.58$\pm$1.31 & 82.70$\pm$1.45 & 73.73$\pm$1.30 \\
        MagNet$_{\text{D}}$ & 39.35$\pm$5.13 & 54.50$\pm$3.21 & 65.32$\pm$3.26 & 15.31$\pm$2.45 & 83.05$\pm$1.35 & 84.14$\pm$1.31 & 73.28$\pm$1.21  \\ \midrule

        DUPLEX$_{\text{F}}$ & 30.16$\pm$5.06 & 50.90$\pm$10.69 & 73.39$\pm$3.42 & 17.82$\pm$4.47 & 84.63$\pm$2.20 & 83.01$\pm$2.10 & 79.20$\pm$1.79 \\ 
        DUPLEX$_{\text{D}}$ &27.98$\pm$7.61 & 50.20$\pm$7.36 & 63.62$\pm$3.90 & 10.95$\pm$6.35 & 80.68$\pm$1.15 & 81.17$\pm$1.20 & 73.75$\pm$1.44 \\ \midrule


        DHYPR$_{\text{F}}$ &\hig{2}{77.77$\pm$4.10} & \hig{2}{89.35$\pm$2.59} & \hig{2}{92.32$\pm$3.72} & 32.39$\pm$10.14 & \hig{2}{96.61$\pm$0.28} & \hig{2}{96.35$\pm$0.50} & \hig{2}{87.33$\pm$1.39}  \\
        DHYPR$_{\text{D}}$ & 24.32$\pm$8.59 & 41.51$\pm$6.55 & 54.27$\pm$5.94 & 6.17$\pm$2.12 & 77.38$\pm$1.79 & 77.35$\pm$2.44 & 70.27$\pm$1.35 \\  \midrule
        
        DiGAE$_{\text{F}}$ & 56.12$\pm$3.08 & 71.06$\pm$2.49 & 83.64$\pm$3.21 & 22.88$\pm$6.79 & 91.03$\pm$1.14 & 91.29$\pm$1.04 & 81.85$\pm$1.31  \\
        DiGAE$_{\text{D}}$ &  44.32$\pm$1.85 & 59.91$\pm$2.08 & 69.66$\pm$2.11 & 20.89$\pm$4.81 & 85.28$\pm$0.96 & 86.28$\pm$0.96 & 74.54$\pm$0.91 \\ \midrule

        SDGAE$_{\text{F}}$ &\hig{1}{81.06$\pm$4.50} & \hig{1}{91.24$\pm$2.55} &\hig{1}{93.69$\pm$3.68} &\hig{1}{42.50$\pm$9.76} & \hig{1}{97.24$\pm$0.60} & \hig{1}{97.13$\pm$0.68} & \hig{1}{91.38$\pm$0.79} \\ 
        SDGAE$_{\text{D}}$ & 45.03$\pm$3.19 & 57.82$\pm$2.00 & 68.97$\pm$1.70 & 21.41$\pm$2.85 & 85.10$\pm$0.98 & 86.24$\pm$0.47 & 74.83$\pm$0.86 \\
        

        
        \bottomrule
    \end{tabular}}
\end{table}


\clearpage
\hypertarget{app_complete_res_photo}{}
\subsection{Results on Photo}\label{app_complete_res_photo}
%Table~\ref{app_tb_photo} shows the results of DirLinkBench on Photo.
\begin{table}[h]
    \centering
    \caption{Benchmark Results on Photo. For all methods, $_{\text{F}}$ indicates the use of original node features as input, while $_{\text{D}}$ indicates the use of in/out degrees as input. Results ranked \hig{1}{first} and \hig{2}{second} are highlighted.}
    \label{app_tb_photo}
    \resizebox{\textwidth}{!}{
    \begin{tabular}{lccccccc}
        \toprule
        Model & Hits@20 & Hits@50 & Hits@100 & MRR & AUC & AP & ACC \\
        \midrule
        STRAP & \hig{2}{38.54$\pm$5.20} & \hig{2}{55.21$\pm$2.19} & \hig{1}{69.16$\pm$1.44} & 12.08$\pm$3.15 & 98.54$\pm$0.04 & 98.65$\pm$0.05 & 94.61$\pm$0.10 \\
        
        ODIN & 5.28$\pm$1.36 & 9.38$\pm$1.54 & 14.13$\pm$1.92 & 1.78$\pm$0.66 & 88.15$\pm$0.21 & 87.34$\pm$0.30 & 80.60$\pm$0.20  \\
        
        ELTRA & 7.09$\pm$1.23 & 12.86$\pm$1.58 & 20.63$\pm$1.93 & 2.22$\pm$0.59 & 96.89$\pm$0.06 & 95.84$\pm$0.13 & 91.84$\pm$0.12 \\ \midrule
        
        MLP$_{\text{F}}$ & 8.83$\pm$2.06 & 14.07$\pm$2.87 & 20.91$\pm$4.18 & 3.18$\pm$1.06 & 95.29$\pm$0.37 & 93.60$\pm$1.06 & 88.31$\pm$0.50  \\
        MLP$_{\text{D}}$ & 7.88$\pm$1.56 & 12.90$\pm$0.81 & 18.04$\pm$1.35 & 2.53$\pm$0.40 & 85.90$\pm$0.44 & 86.44$\pm$0.21 & 77.25$\pm$0.51  \\ \midrule
        
        GCN$_{\text{F}}$ & 29.44$\pm$3.90 & 44.12$\pm$3.49 & 57.55$\pm$2.54 & 9.85$\pm$2.78 & 99.04$\pm$0.06 & 98.66$\pm$0.10 & 95.90$\pm$0.18  \\
        GCN$_{\text{D}}$ & 31.47$\pm$3.71 & 44.93$\pm$3.34 & 58.77$\pm$2.96 & \hig{2}{12.56$\pm$2.86} & 86.12$\pm$0.87 & 86.79$\pm$1.01 & 77.71$\pm$1.01  \\ \midrule
        
        GAT$_{\text{F}}$ & 25.97$\pm$4.22 & 42.85$\pm$4.90 & 58.06$\pm$4.03 & 8.62$\pm$3.07 & \hig{2}{99.13$\pm$0.09} & \hig{2}{98.93$\pm$0.11} & \hig{1}{96.17$\pm$0.20} \\
        GAT$_{\text{D}}$ & 11.96$\pm$1.60 & 21.83$\pm$2.98 & 31.51$\pm$2.16 & 3.61$\pm$0.94 & 94.41$\pm$0.19 & 94.37$\pm$0.19 & 87.00$\pm$0.25 \\ \midrule

        APPNP$_{\text{F}}$ & 22.13$\pm$2.60 & 35.30$\pm$2.04 & 47.51$\pm$2.51 & 6.56$\pm$1.14 & 98.54$\pm$0.09 & 98.26$\pm$0.12 & 94.71$\pm$0.29 \\ 
        APPNP$_{\text{D}}$& 12.92$\pm$1.06 & 19.42$\pm$2.24 & 26.66$\pm$2.14 & 4.63$\pm$1.10 & 93.29$\pm$0.42 & 92.86$\pm$0.40 & 85.36$\pm$0.39  \\ \midrule
        
        DGCN$_{\text{F}}$ & 22.61$\pm$4.58 & 37.09$\pm$5.38 & 51.61$\pm$6.33 & 6.38$\pm$1.49 & 98.74$\pm$0.39 & 98.49$\pm$0.45 & 95.27$\pm$0.94 \\
        DGCN$_{\text{D}}$ & 15.96$\pm$3.35 & 26.88$\pm$2.91 & 35.56$\pm$3.52 & 5.83$\pm$1.16 & 96.26$\pm$0.70 & 95.94$\pm$0.76 & 90.02$\pm$1.00 \\ \midrule

        DiGCN$_{\text{F}}$ & 18.47$\pm$1.89 & 29.63$\pm$2.28 & 40.17$\pm$2.38 & 6.41$\pm$2.41 & 98.10$\pm$0.10 & 97.69$\pm$0.14 & 93.86$\pm$0.23 \\
        DiGCN$_{\text{D}}$ & 17.00$\pm$1.84 & 25.51$\pm$1.39 & 33.73$\pm$1.95 & 5.71$\pm$1.32 & 95.21$\pm$0.19 & 95.09$\pm$0.21 & 88.29$\pm$0.26 \\ \midrule
        
        DiGCNIB$_{\text{F}}$ & 21.42$\pm$2.77 & 34.97$\pm$2.67 & 48.26$\pm$3.98 & 6.82$\pm$1.53 & 98.67$\pm$0.14 & 98.39$\pm$0.16 & 95.05$\pm$0.27 \\
        DiGCNIB$_{\text{D}}$ & 16.00$\pm$1.67 & 24.63$\pm$1.84 & 33.20$\pm$2.17 & 6.11$\pm$1.50 & 96.71$\pm$0.07 & 96.38$\pm$0.09 & 91.30$\pm$0.17 \\ \midrule
        
        DirGNN$_{\text{F}}$& 22.59$\pm$2.77 & 34.65$\pm$3.31 & 49.15$\pm$3.62 & 8.42$\pm$2.90 & 98.76$\pm$0.09 & 98.47$\pm$0.13 & 95.34$\pm$0.17 \\
        DirGNN$_{\text{D}}$ &21.57$\pm$2.11 & 30.87$\pm$2.35 & 43.21$\pm$2.19 & 8.72$\pm$2.08 & 97.53$\pm$0.09 & 97.26$\pm$0.08 & 92.41$\pm$0.20 \\ \midrule
        
        MagNet$_{\text{F}}$ & 5.35$\pm$0.50 & 8.52$\pm$0.85 & 12.66$\pm$1.03 & 1.62$\pm$0.39 & 87.92$\pm$0.22 & 87.19$\pm$0.32 & 80.16$\pm$0.23 \\
        MagNet$_{\text{D}}$ & 5.14$\pm$0.54 & 9.04$\pm$0.52 & 13.89$\pm$0.32 & 1.61$\pm$0.31 & 88.11$\pm$0.21 & 87.48$\pm$0.21 & 80.31$\pm$0.14 \\ \midrule
        
        DUPLEX$_{\text{F}}$ & 7.84$\pm$1.17 & 12.64$\pm$1.22 & 17.94$\pm$0.66 & 2.53$\pm$0.66 & 94.22$\pm$0.76 & 93.37$\pm$0.91 & 87.68$\pm$0.52 \\ 
        DUPLEX$_{\text{D}}$ & 6.97$\pm$2.14 & 9.08$\pm$4.95 & 13.28$\pm$6.14 & 2.26$\pm$1.17 & 87.41$\pm$7.05 & 86.40$\pm$5.89 & 79.73$\pm$6.87 \\
 \midrule

        DHYPR$_{\text{F}}$ & 10.18$\pm$1.21 & 13.66$\pm$1.47 & 20.93$\pm$2.41 & 3.15$\pm$0.57 & 87.35$\pm$1.54 & 88.83$\pm$0.95 & 76.94$\pm$0.63 \\
        
        DHYPR$_{\text{D}}$ & 0.16$\pm$0.14 & 0.52$\pm$0.83 & 1.63$\pm$1.23 & 0.28$\pm$0.06 & 58.85$\pm$1.39 & 56.65$\pm$0.62 & 53.91$\pm$0.66 \\  \midrule
        
        DiGAE$_{\text{F}}$ & 27.79$\pm$3.85 & 43.32$\pm$3.36 & 55.05$\pm$2.36 & 9.38$\pm$2.47 & 97.98$\pm$0.08 & 97.99$\pm$0.10 & 91.77$\pm$0.18 \\
        DiGAE$_{\text{D}}$ & 16.55$\pm$1.20 & 26.17$\pm$2.03 & 34.60$\pm$2.09 & 5.36$\pm$1.66 & 93.14$\pm$0.14 & 93.76$\pm$0.11 & 83.23$\pm$0.34 \\ \midrule

        SDGAE$_{\text{F}}$ &\hig{1}{40.89$\pm$3.86} & \hig{1}{55.76$\pm$4.08} & \hig{2}{68.84$\pm$2.35} & \hig{1}{14.82$\pm$4.22} &\hig{1}{99.25$\pm$0.05} & \hig{1}{99.16$\pm$0.06} & \hig{2}{96.16$\pm$0.14} \\
        SDGAE$_{\text{D}}$ & 24.76$\pm$3.46 & 38.5$\pm$1.66 & 50.96$\pm$2.32 & 9.16$\pm$1.99 & 98.07$\pm$0.15 & 97.98$\pm$0.12 & 93.57$\pm$0.35 \\

        
        \bottomrule
    \end{tabular}}
\end{table}


\clearpage
\hypertarget{app_complete_res_computers}{}
\subsection{Results on Computers}\label{app_complete_res_computers}
%Table~\ref{app_tb_computers} shows the results of DirLinkBench on Computers.
\begin{table}[ht]
    \centering
    \caption{Benchmark Results on Computers. For all methods, $_{\text{F}}$ indicates the use of original node features as input, while $_{\text{D}}$ indicates the use of in/out degrees as input. Results ranked \hig{1}{first} and \hig{2}{second} are highlighted.}
    \label{app_tb_computers}
    \resizebox{\textwidth}{!}{
    \begin{tabular}{lccccccc}
        \toprule
        Model & Hits@20 & Hits@50 & Hits@100 & MRR & AUC & AP & ACC \\
        \midrule
        STRAP & \hig{2}{24.35$\pm$3.75} &\hig{2}{38.14$\pm$3.77} & \hig{2}{51.87$\pm$2.07} & \hig{2}{7.44$\pm$1.98} & 98.19$\pm$0.03 & 98.32$\pm$0.03 & 93.43$\pm$0.07 \\
        
        ODIN & 4.42$\pm$1.18 & 8.41$\pm$1.27 & 12.98$\pm$1.47 & 1.46$\pm$0.63 & 89.35$\pm$0.09 & 89.15$\pm$0.35 & 81.73$\pm$0.10  \\
        
        ELTRA & 4.82$\pm$0.75 & 8.66$\pm$1.38 & 14.74$\pm$1.55 & 1.61$\pm$0.52 & 95.63$\pm$0.10 & 95.11$\pm$0.14 & 89.30$\pm$0.15  \\ \midrule
        
        MLP$_{\text{F}}$ & 5.51$\pm$0.72 & 9.31$\pm$0.90 & 14.26$\pm$0.88 & 1.63$\pm$0.42 & 92.83$\pm$0.36 & 91.86$\pm$0.39 & 85.33$\pm$0.57 \\ 
        MLP$_{\text{D}}$ & 6.97$\pm$1.22 & 12.2$\pm$0.88 & 17.57$\pm$0.85 & 2.76$\pm$0.75 & 87.48$\pm$0.16 & 88.69$\pm$0.15 & 78.75$\pm$0.36  \\ \midrule
        
        GCN$_{\text{F}}$ & 19.32$\pm$2.69 & 31.47$\pm$1.81 & 43.77$\pm$1.75 & 7.26$\pm$1.64 & 98.97$\pm$0.06 & \hig{2}{98.80$\pm$0.06} & \hig{2}{95.69$\pm$0.13} \\
        GCN$_{\text{D}}$ & 10.59$\pm$1.40 & 17.48$\pm$1.07 & 24.59$\pm$1.01 & 4.33$\pm$1.26 & 94.41$\pm$0.10 & 94.25$\pm$0.12 & 87.00$\pm$0.13 \\ \midrule

        
        GAT$_{\text{F}}$ & 14.89$\pm$3.24 & 27.53$\pm$2.48 & 40.74$\pm$3.22 & 3.37$\pm$0.92 &\hig{2}{98.99$\pm$0.11} & 98.76$\pm$0.15 & \hig{1}{95.78$\pm$0.26} \\
        GAT$_{\text{D}}$ & 7.16$\pm$1.53 & 14.49$\pm$1.54 & 21.98$\pm$1.04 & 2.36$\pm$0.69 & 95.20$\pm$0.44 & 94.81$\pm$0.47 & 88.14$\pm$0.67 \\ \midrule

        APPNP$_{\text{F}}$ & 12.40$\pm$1.24 & 21.71$\pm$1.13 & 32.24$\pm$1.40 & 4.86$\pm$1.06 & 98.21$\pm$0.08 & 97.90$\pm$0.11 & 94.01$\pm$0.16 \\ 
        APPNP$_{\text{D}}$ & 9.54$\pm$1.53 & 15.36$\pm$1.20 & 21.92$\pm$1.32 & 3.99$\pm$0.83 & 93.84$\pm$0.35 & 93.49$\pm$0.38 & 85.76$\pm$0.52 \\\midrule

        
        DGCN$_{\text{F}}$ & 17.74$\pm$2.16 & 27.81$\pm$1.84 & 39.92$\pm$1.94 & 6.51$\pm$1.53 & 98.78$\pm$0.10 & 98.59$\pm$0.11 & 95.22$\pm$0.20 \\
        DGCN$_{\text{D}}$ & 11.12$\pm$1.50 & 17.79$\pm$1.66 & 25.20$\pm$1.87 & 4.06$\pm$0.70 & 95.54$\pm$0.50 & 95.30$\pm$0.53 & 88.72$\pm$0.74 \\ \midrule

        
        DiGCN$_{\text{F}}$ & 11.68$\pm$1.84 & 19.02$\pm$2.06 & 27.51$\pm$1.67 & 4.61$\pm$0.71 & 97.87$\pm$0.24 & 97.47$\pm$0.27 & 93.44$\pm$0.54 \\
        DiGCN$_{\text{D}}$ & 10.98$\pm$1.53 & 17.76$\pm$1.37 & 24.46$\pm$1.51 & 4.69$\pm$1.09 & 94.86$\pm$0.11 & 94.75$\pm$0.12 & 87.59$\pm$0.17 \\ \midrule
        
        DiGCNIB$_{\text{F}}$ & 13.41$\pm$2.95 & 22.54$\pm$1.79 & 32.44$\pm$1.85 & 4.86$\pm$0.80 & 98.46$\pm$0.14 & 98.15$\pm$0.17 & 94.61$\pm$0.31 \\
        DiGCNIB$_{\text{D}}$ & 10.70$\pm$0.84 & 16.69$\pm$1.41 & 24.74$\pm$1.41 & 5.10$\pm$1.08 & 96.50$\pm$0.09 & 96.19$\pm$0.11 & 90.77$\pm$0.15  \\ \midrule
        
        DirGNN$_{\text{F}}$ & 13.95$\pm$1.66 & 23.94$\pm$1.26 & 35.65$\pm$1.30 & 5.02$\pm$0.60 & 98.56$\pm$0.05 & 98.28$\pm$0.07 & 94.68$\pm$0.14 \\
        DirGNN$_{\text{D}}$ & 13.09$\pm$0.90 & 21.84$\pm$1.54 & 30.70$\pm$0.98 & 6.14$\pm$1.58 & 96.70$\pm$0.08 & 96.64$\pm$0.07 & 91.08$\pm$0.13  \\
        \midrule
        
        MagNet$_{\text{F}}$ & 5.02$\pm$0.60 & 8.49$\pm$0.48 & 12.66$\pm$0.62 & 1.74$\pm$0.66 & 89.18$\pm$0.08 & 88.99$\pm$0.10 & 81.43$\pm$0.08 \\
        MagNet$_{\text{D}}$ & 4.50$\pm$0.76 & 8.15$\pm$0.89 & 12.85$\pm$0.59 & 0.97$\pm$0.22 & 89.37$\pm$0.03 & 89.25$\pm$0.05 & 81.61$\pm$0.11 \\ \midrule

        
        DUPLEX$_{\text{F}}$ & 7.36$\pm$0.85 & 13.19$\pm$0.97 & 17.90$\pm$0.71 & 2.27$\pm$0.63 & 92.05$\pm$0.22 & 91.45$\pm$0.29 & 85.65$\pm$0.27 \\
        DUPLEX$_{\text{D}}$ & 5.92$\pm$1.83 & 9.04$\pm$4.80 & 14.56$\pm$5.51 & 1.48$\pm$0.55 & 87.00$\pm$4.97 & 86.79$\pm$4.33 & 78.90$\pm$5.65 \\ \midrule
        
        
        DiGAE$_{\text{F}}$ & 19.65$\pm$1.33 & 31.16$\pm$1.62 & 41.55$\pm$1.62 & 5.34$\pm$1.18 & 97.29$\pm$0.13 & 97.12$\pm$0.44 & 89.52$\pm$0.15 \\
        DiGAE$_{\text{D}}$ & 11.94$\pm$0.96 & 19.38$\pm$1.60 & 26.53$\pm$1.20 & 3.58$\pm$1.20 & 92.86$\pm$0.28 & 93.51$\pm$0.20 & 82.11$\pm$0.32 \\ \midrule

        SDGAE$_{\text{F}}$ &\hig{1}{27.07$\pm$2.34} &\hig{1}{41.37$\pm$1.61} & \hig{1}{53.79$\pm$1.56} &\hig{1}{ 8.41$\pm$1.89} & \hig{1}{99.07$\pm$0.04} & \hig{1}{99.00$\pm$0.03} &95.66$\pm$0.12 \\
        
        SDGAE$_{\text{D}}$& 15.18$\pm$2.22 & 24.02$\pm$1.53 & 33.51$\pm$1.16 & 5.73$\pm$1.33 & 97.44$\pm$0.12 & 97.32$\pm$0.11 & 92.26$\pm$0.24 \\

        \bottomrule
    \end{tabular}}
\end{table}


\clearpage
\hypertarget{app_complete_res_wiki}{}
\subsection{Results on WikiCS}\label{app_complete_res_wiki}
%Table~\ref{app_tb_wikics} shows the results of DirLinkBench on Wikics.
\begin{table}[ht]
    \centering
    \caption{Benchmark Results on WikiCS. For all methods, $_{\text{F}}$ indicates the use of original node features as input, while $_{\text{D}}$ indicates the use of in/out degrees as input. Results ranked \hig{1}{first} and \hig{2}{second} are highlighted.}
    \label{app_tb_wikics}
    \resizebox{\textwidth}{!}{
    \begin{tabular}{lccccccc}
        \toprule
        Model & Hits@20 & Hits@50 & Hits@100 & MRR & AUC & AP & ACC \\
        \midrule
        STRAP &\hig{1}{64.97$\pm$1.52} &\hig{1}{71.10$\pm$1.13} & \hig{1}{76.27$\pm$0.92} & \hig{1}{37.29$\pm$9.42} &\hig{1}{98.66$\pm$0.03} &\hig{1}{98.91$\pm$0.02} &\hig{1}{94.57$\pm$0.06} \\
        
        ODIN & 3.41$\pm$0.60 & 6.24$\pm$0.63 & 9.83$\pm$0.47 & 1.17$\pm$0.35 & 92.86$\pm$0.04 & 91.90$\pm$0.10 & 86.39$\pm$0.08  \\
        
        ELTRA & 3.84$\pm$0.49 & 6.62$\pm$0.55 & 9.88$\pm$0.70 & 1.42$\pm$0.33 & 92.66$\pm$0.19 & 92.19$\pm$0.23 & 85.73$\pm$0.28  \\ \midrule
        
        MLP$_\text{F}$ & 0.24$\pm$0.09 & 0.71$\pm$0.40 & 1.21$\pm$0.25 & 0.15$\pm$0.07 & 71.90$\pm$0.54 & 70.86$\pm$0.68 & 66.26$\pm$0.67 \\ 
        MLP$_\text{D}$ & 4.04$\pm$0.72 & 7.92$\pm$0.61 & 12.99$\pm$0.68 & 1.63$\pm$0.20 & 92.12$\pm$0.10 & 91.90$\pm$0.19 & 84.65$\pm$0.22 \\ \midrule
        
        GCN$_\text{F}$ & 6.10$\pm$1.85 & 11.25$\pm$1.86 & 18.05$\pm$1.43 & 2.07$\pm$0.49 & 95.86$\pm$0.27 & 95.41$\pm$0.45 & 89.85$\pm$0.35  \\
        GCN$_\text{D}$ & 18.15$\pm$2.49 & 30.21$\pm$2.51 & 38.37$\pm$1.51 & 6.09$\pm$2.16 & 95.97$\pm$0.08 & 96.23$\pm$0.08 & 89.58$\pm$0.09 \\ \midrule
        
        GAT$_\text{F}$ & 1.83$\pm$0.54 & 3.52$\pm$2.21 & 5.60$\pm$1.17 & 0.83$\pm$0.31 & 73.18$\pm$3.59 & 74.81$\pm$4.19 & 54.16$\pm$6.34 \\
        GAT$_\text{D}$ & 24.13$\pm$4.10 & 34.32$\pm$4.41 & 40.47$\pm$4.10 & 7.67$\pm$2.07 & 95.72$\pm$0.27 & 96.06$\pm$0.33 & 88.90$\pm$0.36  \\ \midrule

        APPNP$_\text{F}$ & 3.71$\pm$0.46 & 6.66$\pm$0.76 & 11.11$\pm$1.09 & 1.64$\pm$0.33 & 92.76$\pm$0.63 & 92.11$\pm$0.63 & 85.56$\pm$0.08 \\ 
        APPNP$_\text{D}$ & 7.60$\pm$1.75 & 12.76$\pm$3.01 & 20.23$\pm$1.72 & 2.75$\pm$0.70 & 94.12$\pm$0.09 & 93.89$\pm$0.12 & 87.51$\pm$0.14 \\ \midrule
        
        DGCN$_\text{F}$ & 5.84$\pm$1.79 & 10.12$\pm$3.32 & 15.88$\pm$5.12 & 2.04$\pm$0.50 & 95.98$\pm$0.65 & 95.45$\pm$0.80 & 89.95$\pm$0.96 \\
        DGCN$_\text{D}$ & 8.35$\pm$3.46 & 16.51$\pm$2.99 & 25.91$\pm$4.10 & 2.88$\pm$1.11 & 96.22$\pm$0.32 & 96.04$\pm$0.43 & 90.13$\pm$0.37 \\ \midrule
        

        DiGCN$_\text{F}$ & 3.63$\pm$0.69 & 6.77$\pm$0.98 & 10.24$\pm$0.88 & 1.58$\pm$0.46 & 92.55$\pm$1.53 & 91.63$\pm$1.62 & 82.60$\pm$2.14 \\
        DiGCN$_\text{D}$ & 10.08$\pm$1.33 & 16.32$\pm$2.16 & 25.31$\pm$1.84 & 3.85$\pm$0.98 & 95.59$\pm$0.12 & 95.57$\pm$0.12 & 88.89$\pm$0.14 \\ \midrule
        
        DiGCNIB$_\text{F}$ & 5.36$\pm$0.58 & 8.28$\pm$1.20 & 13.61$\pm$1.18 & 1.81$\pm$0.48 & 94.00$\pm$0.33 & 93.31$\pm$0.39 & 84.49$\pm$1.62 \\
        DiGCNIB$_\text{D}$& 12.01$\pm$2.95 & 20.99$\pm$1.75 & 28.28$\pm$2.44 & 3.96$\pm$1.16 & 96.36$\pm$0.07 & 96.30$\pm$0.08 & 90.72$\pm$0.10 \\ \midrule
        
        DirGNN$_\text{F}$ & 11.76$\pm$3.31 & 23.90$\pm$3.31 & 33.28$\pm$2.84 & 4.82$\pm$1.77 & 97.09$\pm$0.13 & 97.04$\pm$0.16 & 91.55$\pm$0.25 \\ 
        DirGNN$_\text{D}$ & 28.16$\pm$2.09 & 40.94$\pm$1.19 & 50.48$\pm$0.85 & \hig{2}{12.08$\pm$2.05} & 97.13$\pm$0.09 & 97.33$\pm$0.09 & 91.28$\pm$0.14 \\\midrule
        
        MagNet$_\text{F}$ & 3.41$\pm$0.35 & 6.10$\pm$0.45 & 9.25$\pm$0.57 & 1.42$\pm$0.22 & 92.57$\pm$0.08 & 91.54$\pm$0.10 & 86.06$\pm$0.11 \\
        MagNet$_\text{D}$ & 4.13$\pm$0.46 & 6.57$\pm$0.34 & 10.81$\pm$0.46 & 1.26$\pm$0.37 & 93.06$\pm$0.06 & 92.18$\pm$0.07 & 86.59$\pm$0.06 \\ \midrule
        
        DUPLEX$_{\text{F}}$ & 2.97$\pm$0.52 & 5.54$\pm$0.63 & 8.52$\pm$0.60 & 0.89$\pm$0.22 & 90.92$\pm$0.12 & 90.06$\pm$0.45 & 83.82$\pm$0.23 \\
        DUPLEX$_{\text{D}}$ & 1.09$\pm$0.92 & 1.53$\pm$3.12 & 5.45$\pm$0.19 & 1.03$\pm$0.08 & 90.93$\pm$0.04 & 90.37$\pm$0.23 & 83.81$\pm$0.11 \\ \midrule
        
        DiGAE$_{\text{F}}$ & 7.01$\pm$1.78 & 11.54$\pm$1.11 & 18.17$\pm$1.27 & 2.18$\pm$0.78 & 93.47$\pm$1.04 & 93.20$\pm$1.16 & 76.38$\pm$0.21  \\
        DiGAE$_{\text{D}}$ & 10.6$\pm$1.89 & 19.69$\pm$1.98 & 29.21$\pm$1.36 & 4.59$\pm$0.88 & 93.46$\pm$0.22 & 94.47$\pm$0.15 & 82.53$\pm$0.48 \\ \midrule

        SDGAE$_{\text{F}}$& 8.54$\pm$3.42 & 19.60$\pm$5.04 & 29.95$\pm$5.80 & 2.35$\pm$0.91 & 96.64$\pm$1.23 & 96.58$\pm$1.20 & 90.98$\pm$1.52  \\
        SDGAE$_{\text{D}}$&\hig{2}{33.04$\pm$3.27} & \hig{2}{47.62$\pm$1.74} &\hig{2}{54.67$\pm$2.50} &11.78$\pm$4.11 &\hig{2}{97.23$\pm$0.07} & \hig{2}{97.58$\pm$0.06} &\hig{2}{92.24$\pm$0.12}  \\
        \bottomrule
    \end{tabular}}
\end{table}


\clearpage
\hypertarget{app_complete_res_slash}{}
\subsection{Results on Slashdot}\label{app_complete_res_slash}
%Table~\ref{app_tb_slash} shows the results of DirLinkBench on Slashdot.
\begin{table}[h]
    \centering
    \caption{Benchmark Results on Slashdot. For all methods, $_{\text{R}}$ indicates the use of random node features as input, while $_{\text{D}}$ indicates the use of in/out degrees as input. Results ranked \hig{1}{first} and \hig{2}{second} are highlighted.}
    \label{app_tb_slash}
    \resizebox{\textwidth}{!}{
    \begin{tabular}{lccccccc}
        \toprule
        Model & Hits@20 & Hits@50 & Hits@100 & MRR & AUC & AP & ACC \\
        \midrule
        STRAP & 19.10$\pm$1.06 & 25.39$\pm$1.43 & 31.43$\pm$1.21 & \hig{1}{9.82$\pm$1.82} & 94.74$\pm$0.05 & 95.13$\pm$0.05 & 88.19$\pm$0.08  \\
        
        ODIN & 14.97$\pm$1.40 & 24.91$\pm$1.19 & 34.17$\pm$1.19 & 4.44$\pm$1.59 & 96.58$\pm$0.07 & 96.75$\pm$0.06 & 90.39$\pm$0.10  \\
        
        ELTRA & 18.02$\pm$2.11 & 26.31$\pm$0.95 & 33.44$\pm$1.00 & 5.53$\pm$1.77 & 94.65$\pm$0.03 & 95.23$\pm$0.04 & 88.11$\pm$0.11 \\ \midrule
        
        MLP$_{\text{R}}$ & 4.25$\pm$0.71 & 7.26$\pm$0.74 & 11.31$\pm$0.70 & 1.28$\pm$0.33 & 72.86$\pm$0.37 & 76.72$\pm$0.21 & 66.16$\pm$0.26   \\ 
        MLP$_{\text{D}}$ & 14.16$\pm$5.22 & 24.01$\pm$0.79 & 32.97$\pm$0.51 & 4.14$\pm$1.71 & 95.84$\pm$0.07 & 96.21$\pm$0.05 & 89.62$\pm$0.11 \\ \midrule
        
        GCN$_{\text{R}}$ & 12.52$\pm$1.30 & 21.03$\pm$1.62 & 29.04$\pm$0.95 & 3.03$\pm$0.73 & 95.58$\pm$0.13 & 95.69$\pm$0.13 & 89.46$\pm$0.12   \\
        GCN$_{\text{D}}$ & 16.28$\pm$1.59 & 24.51$\pm$1.63 & 33.16$\pm$1.22 & 5.54$\pm$2.04 & 95.90$\pm$0.07 & 96.17$\pm$0.06 & 89.75$\pm$0.17  \\ \midrule
        
        GAT$_{\text{R}}$ & 14.82$\pm$2.70 & 22.19$\pm$2.78 & 30.16$\pm$3.11 & 5.11$\pm$1.53 & 96.26$\pm$0.18 & 96.45$\pm$0.20 & 87.14$\pm$1.03  \\
        GAT$_{\text{D}}$ & 12.39$\pm$1.47 & 19.54$\pm$1.89 & 26.65$\pm$1.98 & 4.53$\pm$0.78 & 95.01$\pm$0.34 & 95.25$\pm$0.45 & 88.01$\pm$0.49 \\ \midrule

        
        APPNP$_{\text{R}}$ & 14.83$\pm$1.18 & 22.75$\pm$1.45 & 31.25$\pm$1.14 & 4.56$\pm$1.37 & 95.36$\pm$0.07 & 95.72$\pm$0.06 & 88.76$\pm$0.09  \\ 
        APPNP$_{\text{D}}$ & 15.00$\pm$5.47 & 24.34$\pm$1.47 & 33.76$\pm$1.05 & 5.86$\pm$2.78 & 96.21$\pm$0.06 & 96.43$\pm$0.05 & 90.07$\pm$0.10  \\ \midrule

        
        %DGCN & TO & TO & TO & TO & TO & TO & TO \\
        %DiGCN & TO & TO & TO & TO & TO & TO & TO \\
        %DiGCNIB & TO & TO & TO & TO & TO & TO & TO \\
        DirGNN$_{\text{R}}$ & 18.72$\pm$1.93 & 28.52$\pm$1.09 & 37.41$\pm$1.37 & 6.19$\pm$2.04 & 96.67$\pm$0.05 & 96.89$\pm$0.04 & 90.24$\pm$0.12  \\
        DirGNN$_{\text{D}}$ & 20.55$\pm$2.85 & 31.20$\pm$1.18 & 41.74$\pm$1.15 & 7.52$\pm$3.24 &\hig{1}{96.95$\pm$0.05} & \hig{1}{97.14$\pm$0.06} &\hig{1}{ 90.65$\pm$0.13}  \\ \midrule
        
        MagNet$_{\text{R}}$ & 9.16$\pm$1.26 & 19.58$\pm$1.37 & 28.50$\pm$1.78 & 1.98$\pm$0.25 & 96.31$\pm$0.06 & 96.38$\pm$0.06 & 90.04$\pm$0.13 \\
        MagNet$_{\text{D}}$ & 12.55$\pm$0.75 & 22.34$\pm$0.42 & 31.98$\pm$1.06 & 2.83$\pm$0.51 & 96.57$\pm$0.09 & 96.69$\pm$0.10 & 90.45$\pm$0.09 \\ \midrule
        
        DUPLEX$_{\text{R}}$ &5.67$\pm$1.85	&11.49$\pm$3.36	&18.42$\pm$2.59	&1.81$\pm$0.83	&94.36$\pm$3.25	&94.48$\pm$2.76	&85.42$\pm$3.42\\
        DUPLEX$_{\text{D}}$ & 2.51$\pm$0.64 & 5.95$\pm$1.11 & 9.39$\pm$2.71 & 0.51$\pm$0.01 & 80.57$\pm$3.85 & 86.52$\pm$1.50 & 77.67$\pm$2.24 \\ \midrule
        
        DiGAE$_{\text{R}}$ & 18.89$\pm$1.71 & 27.69$\pm$1.41 & 36.57$\pm$1.32 & 5.49$\pm$2.96 & 95.26$\pm$0.29 & 96.13$\pm$0.19 & 84.27$\pm$0.26  \\
        DiGAE$_{\text{D}}$ &\hig{1}{23.68$\pm$0.94} &\hig{1}{33.97$\pm$1.06} &\hig{2}{41.95$\pm$0.93} & 5.54$\pm$1.51 & 94.30$\pm$0.29 & 95.80$\pm$0.12 & 85.67$\pm$0.28 \\ \midrule

        SDGAE$_{\text{R}}$ &18.24$\pm$2.05 & 28.47$\pm$1.97 & 37.34$\pm$0.68 & 5.51$\pm$1.65 & 95.96$\pm$0.07 & 96.31$\pm$0.03 & 90.11$\pm$0.11  \\
        SDGAE$_{\text{D}}$ &\hig{2}{23.57$\pm$2.11} &\hig{2}{33.75$\pm$1.48} & \hig{1}{42.42$\pm$1.15} &\hig{2}{8.41$\pm$3.80} & \hig{2}{96.70$\pm$0.10} & \hig{2}{97.06$\pm$0.08} & \hig{1}{91.05$\pm$0.20} \\
        
        \bottomrule
    \end{tabular}}
\end{table}

\clearpage
\hypertarget{app_complete_res_epinion}{}
\subsection{Results on Epinions}\label{app_complete_res_epinion}
%Table~\ref{app_tb_epinion} shows the results of DirLinkBench on Epinions.
\begin{table}[ht]
    \centering
    \caption{Benchmark Results on Epinions. For all methods, $_{\text{R}}$ indicates the use of random node features as input, while $_{\text{D}}$ indicates the use of in/out degrees as input. Results ranked \hig{1}{first} and \hig{2}{second} are highlighted.}
    \label{app_tb_epinion}
    \resizebox{\textwidth}{!}{
    \begin{tabular}{lccccccc}
        \toprule
        Model & Hits@20 & Hits@50 & Hits@100 & MRR & AUC & AP & ACC \\
        \midrule
        STRAP & \hig{1}{44.66$\pm$1.68} &  \hig{1}{53.48$\pm$0.86} &  \hig{1}{58.99$\pm$0.82} &  \hig{1}{21.18$\pm$6.31} & 96.62$\pm$0.06 & 97.60$\pm$0.04 & 91.89$\pm$0.06 \\
        
        ODIN & 11.85$\pm$3.33 & 25.99$\pm$1.79 & 36.91$\pm$0.47 & 3.22$\pm$0.80 & 97.72$\pm$0.03 & 97.91$\pm$0.03 & 92.42$\pm$0.07  \\
        
        ELTRA & 16.89$\pm$1.27 & 28.37$\pm$1.53 & 41.63$\pm$2.53 & 5.81$\pm$1.10 & 96.19$\pm$0.04 & 97.47$\pm$0.03 & 90.40$\pm$0.13 \\ \midrule
        
        MLP$_\text{R}$ & 3.84$\pm$0.65 & 7.02$\pm$0.40 & 10.18$\pm$0.67 & 1.18$\pm$0.36 & 78.44$\pm$0.25 & 80.67$\pm$0.14 & 67.83$\pm$1.93  \\ 
        MLP$_\text{D}$ & 15.84$\pm$3.61 & 34.11$\pm$2.04 & 44.59$\pm$1.62 & 4.35$\pm$1.64 & 97.85$\pm$0.08 & 98.02$\pm$0.07 & 92.56$\pm$0.45  \\ \midrule
        
        GCN$_\text{R}$ & 18.26$\pm$2.56 & 30.18$\pm$1.71 & 40.64$\pm$1.53 & 4.04$\pm$1.10 & 96.90$\pm$0.04 & 97.54$\pm$0.03 & 92.10$\pm$0.24  \\
        GCN$_\text{D}$& 12.40$\pm$7.59 & 30.79$\pm$9.24 & 46.10$\pm$1.37 & 3.12$\pm$0.71 & 97.83$\pm$0.09 & 98.15$\pm$0.05 &\hig{2}{94.17$\pm$0.08} \\ \midrule

        GAT$_\text{R}$ & 19.65$\pm$3.52 & 31.90$\pm$3.56 & 43.65$\pm$4.88 & 6.04$\pm$2.16 & 98.35$\pm$0.11 & \hig{2}{ 98.50$\pm$0.11} & 92.05$\pm$1.30  \\
        GAT$_\text{D}$ & 18.18$\pm$4.11 & 27.18$\pm$3.95 & 36.76$\pm$5.74 & 7.98$\pm$1.92 & 97.61$\pm$0.29 & 97.81$\pm$0.22 & 92.92$\pm$0.37 \\ \midrule
        
        APPNP$_\text{R}$  & 17.86$\pm$2.53 & 27.89$\pm$1.24 & 39.06$\pm$1.26 & 4.88$\pm$4.14 & 97.70$\pm$0.04 & 97.92$\pm$0.02 & 92.84$\pm$0.06  \\ 
        APPNP$_\text{D}$& 18.46$\pm$3.51 & 30.84$\pm$1.84 & 41.99$\pm$1.23 & 6.41$\pm$3.73 &\hig{2}{98.36$\pm$0.06} & 98.48$\pm$0.06 & 94.17$\pm$0.16  \\ \midrule
        %DGCN & TO & TO & TO & TO & TO & TO & TO \\
        %DiGCN & TO & TO & TO & TO & TO & TO & TO \\
        %DiGCNIB & TO & TO & TO & TO & TO & TO & TO \\
      
        DirGNN$_{\text{R}}$ & 25.66$\pm$3.33 & 39.28$\pm$1.96 & 50.10$\pm$2.06 & 7.03$\pm$2.70 & 98.25$\pm$0.03 & 98.46$\pm$0.03 & 93.99$\pm$0.05 \\
        DirGNN$_{\text{D}}$ & 21.35$\pm$2.89 & 34.06$\pm$2.53 & 46.01$\pm$2.07 & 6.12$\pm$2.54 & 98.03$\pm$0.03 & 98.26$\pm$0.02 & 93.48$\pm$0.06 \\ \midrule

        
        MagNet$_{\text{R}}$ & 6.41$\pm$1.57 & 12.95$\pm$1.51 & 22.12$\pm$1.27 & 1.62$\pm$0.35 & 97.60$\pm$0.03 & 97.69$\pm$0.03 & 92.79$\pm$0.05  \\
        MagNet$_{\text{D}}$ & 7.68$\pm$1.00 & 16.30$\pm$2.36 & 28.01$\pm$1.72 & 2.53$\pm$0.57 & 97.71$\pm$0.03 & 97.86$\pm$0.04 & 92.92$\pm$0.05  \\ \midrule
        
        DUPLEX$_{\text{R}}$ & 3.76$\pm$1.94 & 8.38$\pm$4.39 & 16.50$\pm$4.34 & 1.85$\pm$0.39 & 92.11$\pm$2.25 & 93.78$\pm$3.69 & 88.20$\pm$3.86  \\
        DUPLEX$_{\text{D}}$ & 2.43$\pm$0.86 & 6.74$\pm$0.45 & 12.35$\pm$2.64 & 0.46$\pm$0.08 & 90.61$\pm$1.76 & 88.72$\pm$3.26 & 81.82$\pm$3.74  \\ \midrule
        
        DiGAE$_{\text{R}}$ & 29.11$\pm$3.51 & 43.25$\pm$1.68 & 53.27$\pm$1.17 & 8.85$\pm$2.81 & 97.11$\pm$0.19 & 97.78$\pm$0.11 & 89.16$\pm$0.24   \\
        DiGAE$_{\text{D}}$ & 22.56$\pm$9.23 & 43.19$\pm$2.97 & 55.14$\pm$1.96 & 5.33$\pm$1.78 & 96.47$\pm$0.18 & 97.37$\pm$0.10 & 90.17$\pm$0.14  \\ \midrule

        SDGAE$_{\text{R}}$ & 21.88$\pm$2.17 & 35.28$\pm$2.35 & 45.04$\pm$2.18 & 6.54$\pm$1.94 & 98.11$\pm$0.03 & 98.35$\pm$0.03 & 93.83$\pm$0.07 \\
        SDGAE$_{\text{D}}$ &\hig{2}{32.81$\pm$2.67} &\hig{2}{45.61$\pm$1.92} &\hig{2}{55.91$\pm$1.77} &\hig{2}{11.62$\pm$2.86} &  \hig{1}{98.43$\pm$0.07} &\hig{1}{98.64$\pm$0.04} &\hig{1}{94.33$\pm$0.11}  \\
        
        \bottomrule
    \end{tabular}}
\end{table}