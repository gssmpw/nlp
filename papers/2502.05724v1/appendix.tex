\appendix
\onecolumn
\clearpage
\section*{Appendix Contents} % 
\begin{itemize}
  \item \hyperlink{app:app_notation}{Appendix A: Notation} ~~~~ p.\pageref{app_notation}
 \vspace{4pt}
   
  %\begin{itemize}[leftmargin=2em, nosep] % 2
  %  \item \hyperlink{app:lemma}{A.1: Technical Lemma} ~~~~ p.\pageref{app:lemma}
 % \end{itemize}
  
  \item \hyperlink{app:proof}{Appendix B: Proof} ~~~~ p.\pageref{app_proof}
  \vspace{4pt}
  \begin{itemize}[leftmargin=2em, nosep] % 2
    \item \hyperlink{app_proof_th_dual}{B.1: The proof of Theorem 2.2} ~~~~ p.\pageref{app_proof_th_dual} \\
    \item \hyperlink{app_proof_cora_de}{B.2: The proof of Corollary 2.3} ~~~~ p.\pageref{app_proof_cora_de}\\
    \item \hyperlink{app_proof_cora_de}{B.3: The proof of Lemma 4.1} ~~~~ p.\pageref{app_proof_cora_de}\\
    \item \hyperlink{app_proof_le_norm}{B.4: The proof of Lemma 4.2} ~~~~ p.\pageref{app_proof_le_norm}
  \end{itemize}
  \vspace{4pt}

   \item \hyperlink{app:related_work}{Appendix C: Related Work} ~~~~ p.\pageref{app:related_work}
   
   \vspace{4pt}
   \item \hyperlink{app_uni_fram}{Appendix D: Details of Unified Framework} ~~~~ p.\pageref{app_uni_fram}

   \vspace{4pt}
   \item \hyperlink{app_issues}{Appendix E: Issues with Existing Experimental Setup} ~~~~ p.\pageref{app_issues}
   \vspace{4pt}
   \begin{itemize}[leftmargin=2em, nosep] % 2
    \item \hyperlink{app_issues_subtask}{E.1: Details of multiple subtask setup} ~~~~ p.\pageref{app_issues_subtask} \\
    \item \hyperlink{app_issues_results}{E.2: Details of experimental setting and more results} ~~~~ p.\pageref{app_issues_results}
  \end{itemize}
   \vspace{4pt}
  \item \hyperlink{app_sdgae}{Appendix F: Details and Experimental Setting for SDGAE} ~~~~ p.\pageref{app_sdgae}
   \vspace{4pt}
   \begin{itemize}[leftmargin=2em, nosep] % 2
    \item \hyperlink{app_sdgae_conv}{F.1: More details of graph convolution} ~~~~ p.\pageref{app_sdgae_conv} \\
    \item \hyperlink{app_sdgae_setting}{F.2: Experimental setting} ~~~~ p.\pageref{app_sdgae_setting}
  \end{itemize}
  
  \vspace{4pt}
  \item \hyperlink{app:experiments}{Appendix G: More Details of DirLinkBench} ~~~~ p.\pageref{app:experiments}
  \vspace{4pt}
  \begin{itemize}[leftmargin=2em, nosep] % 2
    \item \hyperlink{app_dataset_baseline}{G.1: Datasets and baselines} ~~~~ p.\pageref{app_dataset_baseline} \\
    \item \hyperlink{app_bench_metric}{G.2: Metric description} ~~~~ p.\pageref{app_bench_metric} \\
    \item \hyperlink{app_bench_res_ana}{G.3: Additional results in analysis} ~~~~ p.\pageref{app_bench_res_ana}
    
  \end{itemize}

  \vspace{4pt}
  \item \hyperlink{app_complete_res}{Appendix H: Complete Results of DirLinkBench on Seven Datasets} ~~~~ p.\pageref{app_complete_res}
  \vspace{4pt}
   \begin{itemize}[leftmargin=2em, nosep] % 2
    \item \hyperlink{app_complete_res_cora}{H.1: Results on Cora-ML} ~~~~ p.\pageref{app_complete_res_cora} \\
    \item \hyperlink{app_complete_res_citeseer}{H.2: Results on CiteSeer} ~~~~ p.\pageref{app_complete_res_citeseer} \\
    \item \hyperlink{app_complete_res_photo}{H.3: Results on Photo} ~~~~ p.\pageref{app_complete_res_photo} \\
    \item \hyperlink{app_complete_res_computers}{H.4: Results on Computers} ~~~~ p.\pageref{app_complete_res_computers} \\
    \item \hyperlink{app_complete_res_wiki}{H.5: Results on WikiCS} ~~~~ p.\pageref{app_complete_res_wiki} \\
    \item \hyperlink{app_complete_res_slash}{H.6: Results on Slashdot} ~~~~ p.\pageref{app_complete_res_slash} \\
    \item \hyperlink{app_complete_res_epinion}{H.7: Results on Epinions} ~~~~ p.\pageref{app_complete_res_epinion} 
  \end{itemize}
  
   
\end{itemize}

\newpage

\hypertarget{app:app_notation}{} 
\section{Notation}\label{app_notation}
We summarize the main notations of the paper in Table~\ref{notation}.
\begin{table}[th]
\centering

\caption{Summation of main notations in this paper.}\label{tbl:def-notation}
\begin{tabular} {l|p{5.0in}} \hline
\toprule
{\bf Notation} &  {\bf Description}  \\ \midrule
$\gG=(V,E)$ & directed unweighted graph with node set $V$ and edge set $E$. \\ \midrule
$n, m$ & the number of nodes and edges, $n = |V|$ and $m = |E|$. \\ \midrule

$\mA$ & the adjacency matrix of $\gG$, $\mA_{uv} = 1$ for a edge from node $u$ to $v$, and $\mA_{uv} = 0$ otherwise. \\ \midrule

 $\mH$ & the Hermitian adjacency matrix, defined as $\mH = \mA_{\rm s} \odot \exp\left(i\frac{\pi}{2}\mathbf{\Theta}\right)$. \\ \midrule

$\mA_{\rm s}$ & the  adjacency matrix of the undirected graph derived from $\gG$, $\mA_{\rm s} = \mA \cup \mA^{\top}$. \\ \midrule

$\mathbf{\Theta}$ & the skew-symmetric matrix $\mathbf{\Theta} = \mA - \mA^{\top}$. \\ \midrule

$\mD_{\rm out}, \mD_{\rm int}$ & the out- and in-degree matrix of $\mA$, $\mD_{\rm out} = {\rm diag}(\mA\vone) $ and $\mD_{\rm int} = {\rm diag}(\mA^{\top}\vone)$. \\ \midrule

$\mX$ & the node feature matrix, $\mX \in \mathbb{R}^{n \times d^{\prime}}$ and $d^{\prime}$ is the dimension. \\ \midrule

$\bm{\theta}_u, \bm{\phi}_u$ & real-valued dual embeddings $\bm{\theta}_u\in \mathbb{R}^{d_\theta}, \bm{\phi}_u\in \mathbb{R}^{d_\phi}$, $d_\theta, d_\phi$ are the dimensions. \\ \midrule

$p(u,v)$ & the probability of directed edge $(u,v)$ existing. \\ \midrule

$\gG^{\prime}=(V, E^{\prime})$ & the observed/training graph with node set $V$ and edge set $E^{\prime}$. \\ \midrule

$\hat{\mA}$ & the adjacency matrix with added self-loops, $\hat{\mA}=\mA + \mI$. \\ \midrule

$\hat{\mD}_{\rm out}, \hat{\mD}_{\rm int}$ & the out- and in-degree matrix of $\hat{\mA}$. \\ \midrule

$\mathcal{S}(\hat{\mA})$ & the block matrix consisting of $\hat{\mA}$ and $\hat{\mA}^{\top}$. \\ \midrule

$\Tilde{\mA}$ & the normalized adjacency matrix, $\Tilde{\mA} =\hat{\mD}_{\rm out}^{-1/2}\hat{\mA}\hat{\mD}_{\rm in}^{-1/2}$. \\ \midrule

$\odot, \quad \|$ & the Hadamard product and concatenation process. \\ \midrule

$d, \quad d^{\prime}$ & the dimension of embeddings and feature matrix. \\

\bottomrule
\end{tabular}
\label{notation}
%\vspace{-3mm}
\end{table}

\section{Theoretical results}\label{sec:appendix_proof}




\subsection{Detailed notations}
\label{sec:apx_detailed_notation}

\begin{definition}[Embedding Layer]\label{defi:embedding_layer}
Given a finite vocabulary $\mathcal{V}$, embedding dimension $d\in\mathbb{N}^+$, token embedding parameter $\theta_{\tokenembedding}\in\mathbb{R}^{d\times \abs{\mathcal{V}}}$ and position embedding parameter $\theta_{\posencoding}\in \mathbb{R}^{d\times n_{\max}}$, we define the \emph{embedding layer} as a sequence-to-sequence map, denoted by $\embed_{\theta_\tokenembedding,\theta_\posencoding}:\mathcal{V}^n\to (\mathbb{R}^d)^n$ for any $1\le n\le n_{\max}$, where
\begin{equation}
    \embed_{\theta_\tokenembedding,\theta_\posencoding}(v_1,\ldots,v_n) = \left(\theta_\tokenembedding(v_1)+ \theta_\posencoding(1), \ldots, \theta_\tokenembedding(v_n)+ \theta_\posencoding(n)  \right). 
\end{equation}
\end{definition}

\paragraph{Multi-Head Self-Attention Mechanism:} Given attention parameters $\theta_\attn = \{W_Q, W_K, W_V, W_O\}$, where each $W_Q^m, W_K^m, W_V^m, W_O^m \in \mathbb{R}^{d_{\attn} \times d }$, we define the \emph{Self-Attention} layer with a causal mask for a decoder-only transformer in \Cref{alg:defi_attn}. We also define a \emph{Multi-Head Attention} layer as a collection of self-attention layer with non-shared parameters $\theta_\mha = \{\theta_{\attn}^{(h)}\}_{h=1}^H$, and its output is the sum of the outputs from each head. That is, $\mha_{\theta_\mha} = \sum_{h=1}^H \attn_{\theta_\attn^{(h)}}$.\footnote{Though in this paper we focus on attention with casual mask, our definition of looped transformer generalizes to the cases with other attention masks.}

\begin{algorithm}
    \caption{Causal Self-Attention, $\attn$}\label{alg:defi_attn}
    \begin{algorithmic}[1]
    \Require  Parameter $\theta_\attn = (W_Q,W_K,W_V,W_O)$, Input embedding $ x_1,\ldots, x_n)\in \left(\mathbb{R}^{d}\right)^n$.
    \Ensure Output embedding $x'= (x'_1,\ldots, x'_n) \triangleq \attn_{\theta_\attn}(x_1,\ldots, x_n)$.
    \State $q_i \triangleq W_Q  x_i, k_i \triangleq W_K x_i, v_i \triangleq W_V x_i, \forall i\in[n]$
    \State $s_i \triangleq \softmax(\inner{q_i}{k_1},\ldots,\inner{q_i}{k_i}) \| (0,\ldots, 0) $. 
    \State $h'_i \triangleq W_O^\top \sum_{j=1}^n (s_i)_j v_j$.
    \end{algorithmic}
    \end{algorithm}
\paragraph{Feed-Forward Network:} Given the parameters of the fully-connected feedforward network layer $\theta_\ff = (W_1, b_1, W_2, b_2) \in \mathbb{R}^{x_{\ff} \times d} \times \mathbb{R}^{d_{\ff}} \times \mathbb{R}^{d \times d_{\ff}} \times \mathbb{R}^{d_{\ff}}$, we define the feedforward layer $\ff_{\theta_\ff}: \mathbb{R}^{d} \to \mathbb{R}^{d}$ as $\ff_{\theta_{\ff}}(h) \triangleq W_2, \relu(W_1 h + b_1) + b_2$.

\iffalse
\begin{definition}[Transformer Block]\label{defi:transformer_block}
Given number of layers $L\in\mathbb{N}^+$ and parameter $\theta_\tfblock = (\theta^{(l)}_\mha,\theta^{(l)}_\ff )_{l=0}^{L-1}$, we define the $L$-layer transformer block $\tfblock_{\theta_\tfblock}:(\mathbb{R}^d)^n\to (\mathbb{R}^d)^n$ for any $n\in\mathbb{N}^+$ as \begin{equation}
    \tfblock_{\theta_\tfblock} \triangleq (\id+ \ff_{\theta^{(L-1)}_\ff})\circ (\id+ \mha_{\theta^{(L-1)}_\mha})\circ \cdots (\id+ \ff_{\theta^{(0)}_\ff})\circ (\id+ \mha_{\theta^{(0)}_\mha})
\end{equation}
\end{definition}
\fi

\begin{definition}[Output Layer]\label{defi:output_layer}
    Given parameter $\theta_\transoutput \in\mathbb{R}^{d\times |\mathcal{V}|}$, we denote the output layer as $\transoutput_{\theta_\transoutput}:(\mathbb{R}^d)^n\to \Delta^{|\mathcal{V}|-1}$, where 
    \begin{align}
        \transoutput_{\theta_\transoutput}(x_1,\ldots,x_n) \triangleq \softmax(x_n^\top \theta_{\transoutput})
    \end{align}
\end{definition}

Finally, we define the entire transformer model $p_{\theta}: \cup_{n\le n_{\max}} \mathcal{V}^n\to \Delta^{|\mathcal{V}|-1}$ as 
\begin{align}
p_{\theta}\triangleq \transoutput_{\theta_\transoutput}\circ \tfblock_{\theta_\tfblock}\circ \embed_{\theta_\tokenembedding,\theta_\posencoding}
\end{align}
for any $\theta = (\theta_\tfblock,\theta_\tokenembedding,\theta_\posencoding,\theta_\transoutput)$.For convenience, we also write $\left[p_{\theta}(v_1,\ldots,v_n)\right](v)$ as $p_{\theta}(v\mid v_1,\ldots,v_n)$.
In particular, we use $\transformer_{\theta}(v_1,\ldots,v_n) \triangleq \argmax_{v\in\mathcal{V}} p_{\theta}(v|v_1,\ldots,v_n)$ to denote the deterministic version of the transformer model. 

\iffalse
\begin{definition}[Looped Transformer]\label{defi:looped_transformer}
Given number of loops $T\in\mathbb{N}^+$,  parameters $\theta = (\theta_{\tfblock},\theta_\tokenembedding,\theta_\posencoding,\theta_\transoutput)$, where $\theta_{\transformer} = (\theta^{(l)}_\mha,\theta^{(l)}_\ff )_{l=0}^{L-1}$, we define the \emph{looped transformer} as  $p_{\theta,T}\triangleq \transoutput_{\theta_\transoutput}\circ \left(\tfblock_{\theta_\tfblock}\right)^T\circ \embed_{\theta_\tokenembedding,\theta_\posencoding}$ and the corresponding deterministic version as $\transformer_{\theta,T}(v_1,\ldots,v_n) \triangleq \argmax_{v\in\mathcal{V}} p_{\theta,T}(v|v_1,\ldots,v_n)$.
\end{definition}
\fi



\paragraph{Finite-precision Modeling:} In this paper we assume the transformer is of finite precision. More specifically, we follow the setting in \citet{li2024chain} and use the shorthand $\Floating_{s}\triangleq \{c\cdot k\cdot 2^{-s}\mid c\in \{-1,1\}, 0\le k\le 2^{2s}-1, k\in\mathbb{N}\}$ to denote fixed-point numbers of constant precision $s$ and rounding operation $\rds{\cdot}:\mathbb{R}\to \Floating_s$ to denote the correcting rounding, namely the mapping from $\mathbb{R}$ to the closest representable number in $\Floating_s$. (We break the tie by picking the number with smaller absolute value). We assume that (1). all the parameters of the transformer are in $\Floating_{s}$ and (2). correct rounding is performed after every binary operation in the forward pass of the transformer. We will refer the readers to \citet{li2024chain} for detailed discussion on such finite-precision modeling and only list important notations and lemmas that will be used in this paper below. 

We use $1_s$ to denote all-one vectors of length $s$. Similarly we define $\inner{\cdot}{\cdot}_s$, $\times_s$, and $\softmax_s$. We recall that for any $s\in \mathbb{N}^+$ and integer $0\le x\le 2^s-1$, we use $\bin_s(x)\in\{0,1\}^s$ to denote the usual binary encoding of integer $x$ using $s$ binary bits in the sense that $x = \sum_{i=1}^s 2^i (\bin_s(x))_i$ and $\sbin_s(x)\in\{-1,1\}^s$ to denote the signed binary encoding, which is $2\bin_s(x)-(1,\ldots,1)$. Finally we define $B_s = \max \Floating_s = 2^s-2^{-s}$.



%
%\begin{proof}[Proof of \Cref{lem:cot_1_is_squal_to_O1}]
%	This is a special case of \Cref{thm:cot_dpolyn_slogn_main}.
%\end{proof}

\begin{lemma}\label{lem:exp_rounding}[Lemma E.1, \citep{li2024chain}]
	For any $s\in\mathbb{N}^+$, it holds that $\rds{\exp(-B_s)} = 0$.
\end{lemma}

\begin{lemma}\label{lem:exp_rounding_up}[Lemma E.2, \citep{li2024chain}]
	For any $s\in\mathbb{N}^+$, it holds that $\rds{\exp(B_s)} = B_s$.
\end{lemma}

\begin{lemma}\label{lem:FB_simulating_boolean_gates}[Lemma E.5, \citep{li2024chain}]
Unlimited-fanin $\AND,\OR$ (resp. $\MAJORITY) :\{0,1\}^n\to \{0,1\}$ can be simulated by some 2-layer feedforward ReLU network with constant (resp. $\log n$) bits of precision constant hidden dimension and additional $n$ constant inputs of value 1. 

Mathematically, let $\ff[s(n)]$ be the set of functions $C:\{0,1\}^n\to\{0,1\}$ which can be a two-layer feedforward ReLU network with at most $s(n)$ bits of precision and constant hidden dimension $\ff_{\theta}:\{0,1\}^{2n}\to \{0,1\}, \ff_{\theta}(x') = W_2\times_s\relu(\rds{W_1\times_s x'+b_1})$, where $\theta = (W_2,W_1,b_1)$, such that for any $x\in\{0,1\}^n$, 
\begin{align}
\ff_\theta(x_1,1,x_2,1,\ldots,x_n,1) = C(x).	
\end{align}
We have unlimited-fanin $\AND,\OR\in \ff[1]$ and $\MAJORITY\in \ff[\log n]$.  
\end{lemma}

Given two vectors $x,y$ of the same length $s$, we use $\interleave{x}{y}$ to denote their interleaving, that is, $(\interleave{x}{y})_{2i-1} = x_i, (\interleave{x}{y})_{2i} = y_i$ for all $i\in [e]$. 
\begin{lemma}\label{lem:attention_rounding}[Lemma E.3, \citep{li2024chain}]
	For any $s\in\mathbb{N}^+$, let $q_i = \interleave{\sbin_s(i)}{1_s}$ and $k_i = B_s\cdot (\interleave{\sbin_s(i)}{(-1_s)})$ for all $i\in [2^s-1]$, it holds that $\rds{\exp(\inner{q_i}{k_j}_s)}=\indct{i=j}$ for all $i,j\in [2^s-1]$.
\end{lemma}

\subsection{Proofs}


\subsubsection{Looped models can simulate non-looped models}
\label{sec:apx_simulate_nonlooped}


\begin{proof}[Proof of \Cref{thm:main}]

We start by introduce some more notations. We will proof the theorem for any fixed sequence of $\vv = (v_1,\ldots, v_n)$. We use $\vx^{(l)}=(x_{i}^{(l)})_{i=1}^n$ to denote the intermediate embedding of $p_{\theta}$ in the $l$th layer. More specifically, we define 

\begin{align}
    \vx^{l} = (\id+ \ff_{\theta^{(l-1)}_\ff})\circ (\id+ \mha_{\theta^{(l-1)}_\mha})\circ \cdots (\id+ \ff_{\theta^{(0)}_\ff})\circ (\id+ \mha_{\theta^{(0)}_\mha})\circ \embed_{\theta_{\embed}}(\vv).
\end{align}

We also use $\vx^{(l+0.5)}=(x_{i}^{(l+0.5)})_{i=1}^n$ to denote the intermediate embedding of $p_{\theta}$ in the $l$th layer after the attention layer.
\begin{align}
    \vx^{l+0.5} =  (\id+ \mha_{\theta^{(l-1)}_\mha})(\vx^{l}).
\end{align}

Similarly, for the constructed looped transformer $p_{\theta,T}$, we use $\vv'=(\#,\vv_1,\ldots,\vv_n)$ to denote its input. For simplicity, we use the convention that $\#$ is at position $0$. The proof still works if $\#$ starts at position $1$ because we can just transfer the later tokens by $1$ position. We define $\vx'^{(l)}=(x'^{(l)}_0,x'^{(l)}_1,\ldots,x'^{(l)}_n)$ as the intermediate embedding of $p_{\theta}$ in the $l$th layer and $\vx'^{(l+0.5)}=(x'^{(l+0.5)}_0,x'^{(l+0.5)}_1,\ldots,x'^{(l+0.5)}_n)$ as the intermediate embedding of $p_{\theta}$ in the $l$th layer.

Below we first state the key properties that our construction will satisfy, which imply the correctness of the theorem and then we state our construction of $p_{\theta',T}$ and show the properties are actually satisfied:
\begin{itemize}
    \item $x'^{(l)}_{i} = (x^{(l)}_i, \bm{1}_{R} - e_{r(l)},l, \indct{i=0})$.
    \item $x'^{(l+0.5)}_{i} = (x^{(l+0.5)}_i, \bm{1}_{R} - e_{r(l)},l, \indct{i=0})$.
    \item $x^{(l)}_0 = x^{(l+0.5)}_0 = \bm{0}$.\footnote{Here we abuse the notation for simplicity of presentation. $x^{(l)}_0 = x^{(l+0.5)}_0$ are not defined in the original non-looped transformer. The key point here is that they are 0 vectors throughout the forward pass.}
\end{itemize}

% \begin{itemize}
%     \item $\theta'_{\ff} $ is essentially concatenation of $\theta^{(i)}_\ff$ for $i\in[r]$ in the hidden dimension of $\ff$. The goal is to make $\ff_{\theta'_{\ff}} = \sum_{i=1}^r \ff_{\theta^{(i)}_\ff}\indct{r(l)=i}$ at $l$th layer. 
%     \item  $\theta'_{\mha} $ is essentially concatenation of $\theta^{(i)}_\mha$ for $i\in[r]$ in the hidden dimension of $\mha$. The goal is to make $\mha_{\theta'_{\mha}} = \sum_{i=1}^r \mha_{\theta^{(i)}_\ff}\indct{r(l)=i}$ at $l$th layer.
%     \item Use extra dimension to count the depth $l$ and maintain $\bm{1} - e_{r(l)}$ using $R+1$ dimension on top of the original embedding size $d$ and use it to do switch. Such piecewise linear function can be approximated by a neural network with $O(L)$ hidden dimension.
%     \item use $\bm{1} - e_{r(l)}$ together with ReLU activation to do the select in MLP. Use $\indct - e_{r(l)}$ in attention to force all attention heads to attend to the dummy token $\#$ in the beginning of the sentence.
% \end{itemize}

To get the last coordinate $l$, which is a depth counter, we just need to add $1$ more hidden dimension in MLP. 

Next we show we can use two-layer with $L+2$ MLP to get the mapping from $\ell\mapsto \bm{1}_{R} - e_{r(l)}$. Let $(\theta^{(i)}_\ff,\theta^{(i)}_\mha)_{i=1}^R$ be the parameters of the $R$ distinct layers in $\theta$. We assume in the $l$th layer, $r(l)$'s parameters are used. This is because $e_{r(l)} = \sum_{i=1}^L e_{r(i)}0.5*( [l-i+1 ]_+-2[l-i]_+ + [l-i-1]_+)$.

Now we explain how to deactivate the undesired MLP neurons. In other words, our construction of $\theta'_{\ff} $ is essentially concatenation of $\theta^{(i)}_\ff$ for $i\in[r]$ in the hidden dimension of $\ff$, with the additional control that  $\ff_{\theta'_{\ff}}((x^{(l)}_i, \bm{1}_{R} - e_{r(l)},l),\indct{i=0}) = \sum_{i=1}^R \ff_{\theta^{(i)}_\ff}(x^{(l)}_i)\indct{r(l)=i,i\neq 0}$ at $l$th layer. This control can be done by subtracting $\bm{1} - e_{r(l)} + \indct{i=0}$ by a constant which is larger than the maximum pre-activation in the hidden layer.

Finally we explain how to deactivate the undesired attention. We will only use attention to update the first part of the embedding, which is $x^{(l+0.5)}_i$. A crucial step here is that we set the token embedding of $\#$ as $0$
We construct keys and queries as follows: 
\begin{enumerate}
    \item ${W'}_Q^{(r')}(x'^{(l)}_{i}) = (W_Q^{(r')}x^{(l)}_{i}, 1- \indct{r'=r(l)} $ for $r' \in [R]$ and $i=0,\ldots,n$
    \item ${W'}_K^{(r')}(x'^{(l)}_{i}) = (W_K^{(r')}x^{(l)}_{i}, -B\indct{i=0}) $ for $r' \in [R]$ and $i=0,\ldots,n$, where $B$ is some constant larger than the maximum previous inner product in attention, $\max_{l\in[L],i,j} \inner{(W_K x^{(l)}_{i}}{(W_Qx^{(l)}_{i}}$. 
    \item ${W'}_O^{(r')}{W'}_V^{(r')}(x'^{(l)}_{i}) = (W_O^{(r')}W_V^{(r')}x^{(l)}_{i}, \bm{0}, 0,0)$.
\end{enumerate}
This construction works because only the `desired' attention head $r=r(l)$ will be activated and behave as in the non-looped case, because otherwise all position in that attention head will be completely attended to position $0$ and returns a zero vector. (We can choose $B$ to be large enough and distribution calculated by the attention score is delta distribution) at position $0$, which yields a zero vector as its value. This completes the proof.
\end{proof}





\subsubsection{Group composition.}
\label{sec:apx_group_composition}






\begin{algorithm}[t]
\caption{Group Composition}\label{alg:group_composition}
\begin{algorithmic}[1]
\Require  Group elements $g_0,g_1,\ldots,g_n\in G$, where $g_0=e$.
\Ensure $g_0\circ g_1\circ \ldots g_n$.
\State $g^{(0)}_i = g_i$, $\forall0\le i\le n$.
\For{$l=1\to \lceil \log_2 n\rceil$}
\State $a^{(l)}_i = g^{(l-1)}_{[2i-n-1]_+},b^{(l)}_i = g^{(l-1)}_{[2i-n]_+}$  $\forall0\le i\le n$.
\State $g^{(l)}_i = a^{(l)}_i\circ b^{(l)}_i$, $\forall0\le i\le n$.
\EndFor
\State \Return $g^{(\lceil\log _2 n\rceil)}_{n}$.
\end{algorithmic}
\end{algorithm}



The landmark result in automata theory, Krohn-Rhodes Decompotision Theorem~\citep{krohn1965algebraic}, shows that all semi-automaton with solvable transformation group (which includes composition problem of solvable groups) can be simulated by a cascade of permutation-reset automata, which can be simulated by $\TC^0$ circuits. \citep{liu2022transformers} further showed that such automaton with solvable transformation group can be continuously simulated by constant-depth transformers. However, it is also shown~\citep{barrington1986bounded} that the composition problem of unsolvable groups are $\NC^1$-complete, for example, the composition of permutation group over $5$ elements, $S_5$. Under the common hardness assumption that $\NC^1\neq \TC^0)$,   constant depth transformer cannot solve composition of $S_5$ using a single forward pass~\citep{merrill2023parallelism,liu2022transformers,li2024chain}. But with CoT, very shallow transformers (depth equal to one or two) can simulate the composition problem of any group\citep{li2024chain,merrill2023expresssive}.



\begin{proof}[Proof of \Cref{thm:group_composition_log_depth}]
We will set the token embedding of $\#$ the same as that of $e$, which is the identity of $G$. In the following proof, we will just treat $\#$ as $e$. We will construct the transformer simulating group composition following the following algorithm~\Cref{alg:group_composition}, which gives the high-level idea of the construction. The correctness of \Cref{alg:group_composition} follows from the associativity of group composition. More concretely, we can verify by induction that $g^{l}_0\circ g^{l}_1\circ\ldots g^{l}_n$ is the same for all $l=0,\ldots, \lceil \log_2 n\rceil$ and in the final round, i.e., when $l=\lceil \log_2 n\rceil$, $g^{(l)}_i=e$  for all $i<n$. 



Below we show how to construct a transformer of the given sizes to simulate the above \Cref{alg:group_composition}.
We will embed each $g\in G$ as a different vector  $\overline{g} \in \{-1,1\}^{\lceil\log_2 |G|\rceil}$ and each position $0\le i\le n$ as its binary representation in $\overline{i} \in\{-1,1\}^{\lceil\log_2 n+1\rceil}$, which is a shorthand for $\sbin_s(i)$ with $s=\lceil\log_2 n+1\rceil$.
We concatenate them to get $\{x^{(0)}_i\}_{i=0}^n$, that is, $x^{(0)}_i = (\overline{g_i},\overline{i}, \overline{[2i-n-1]_+}, \overline{[2i-n-1]_+},0^{\lceil \log_2 |G|\rceil},0^{\lceil \log_2 |G|\rceil})$.
For convenience, we will drop the 0's in the end (also in the other proofs of the paper) and write it as $x^{(0)}_i = (\overline{g_i},\overline{i} , \overline{[2i-n-1]_+}, \overline{[2i-n-1]_+})$.
Below we show we can construct 1-layer transformer block with parameter $(\theta_\mha,\theta_\ff)$ satisfying that
\begin{enumerate}
    \item $\left[ \mha_{\theta_\mha}\left((\overline{g_i},\overline{i},\overline{[2i-n-1]_+}, \overline{[2i-n-1]_+})_{i=0}^n\right)\right]_k = (0^{\lceil\log_2 |G|\rceil +3\lceil\log_2 n+1\rceil}, \overline{g_{[2k-n-1]_+}}, \overline{g_{[2k-n]_+}})$ \\for all $g_0=e,g_i\in G \forall i\in[n]$, $k=0,\ldots,n$;
    \item $\ff_{\theta_\ff}(\overline{g},\overline{i}, \overline{j}, \overline{k},\overline{g'}, \overline{g''}) = (\overline{g'\circ g''}-\overline{g}, 0^{3\lceil\log_2 n+1\rceil},-\overline{g'}, -\overline{g''})$, for all $i,j,k=0,\ldots,n$, $g,g',g''\in G$. 
\end{enumerate}

The first claim is because we can use two attention heads to retrieve $ \overline{g_{[2k-n-1]_+}}$ and $ \overline{g_{[2k-n]_+}}$ respectively, where both of them use $\overline{k}$ as the key and use $-\overline{[2k-n-1]_+}$ and $-\overline{[2k-n]_+}$ as queries respectively. This is possible because all the required information are already in $x_i$. We further make attention temperature low enough so the probability returned by attention is a one-hot distribution at the position whose key is equal to the negative query after rounding.

Now we turn to the second claim about MLP. We will use $|G|^2$ neurons with ReLU activation and bias to simulate the product of $g'$ and $g''$. We can index each neuron 
by $(h,h')$ for $h,h'\in G$ and set its incoming weight $[W_1]_{(h,h'),:} = (\overline{h},\overline{h'})$ and set bias $(b_1)_{(h,h')} = - 2\lceil\log_2 |G|\rceil+1$, which ensures that the activation of neuron $(h,h')$ will only be $1$ when $g'=h,g''=h'$ and be $0$ otherwise. Then setting the outgoing weight of neuron $(h,h')$ as $\overline{h\circ h'}$ and the bias in the second layer to be $0$ finishes the construction for simulating the group composition. Finally we use the remaining $6\lceil\log_2 |G|\rceil$ to simulate negative identity mapping $x\to-x$ for the remaining $3\lceil\log_2 |G|\rceil$ embedding dimension. This completes the proof.
\end{proof}






% \iffalse


\subsection{Connection to chain-of-thought reasoning}
\label{sec:apx_cot_connection}


In this section, we establish a connection betwee looped models and CoT reasoning. We first define the recursion for CoT reasoning as follows:
$$
\transformer^{i}_\theta(v_1,\ldots,v_n)\triangleq \transformer^{i-1}_\theta(v_1,\ldots,v_n, \transformer_\theta(v_1,\ldots,v_n)),$$ for $i, n \in \mathbb{N}^+$ satisfying $i+n\le n_{\max}-1$ along with the base case of $\transformer^{1}_\theta(v_1,\ldots,v_n)\triangleq\transformer_\theta(v_1,\ldots,v_n)$. For all $0\le i\le n_{\max} - n-1$, the output with $i$ steps of CoT is
$v_{n+i+1}  = \transformer^{i+1}_\theta(v_1,\ldots,v_n) = \transformer_\theta(v_1,\ldots,v_n,v_{n+1},\ldots,v_{n+i})$.
% \ns{Connection to CoT reasoning} \sjr{this part is very unclear}

\iffalse
\begin{theorem}[Informal]\label{thm:cot_informal}
    For any $L$-layer non-looped transformer $\transformer_\theta$, there exists a looped transformer with parameter $\theta'$ and $L+\mathcal{O}(1)$ layers, such that for any input $\vx$ and integer $m$,  the output of non-looped transformer after $m$ steps of CoT is the same as that of the looped transformer on input $x$ concatenated by $m$ dummy tokens with $m$ loops.
\end{theorem}
\fi




We first give the formal statement below.
% of \Cref{thm:cot_informal}, which is \Cref{thm:cot_formal} below. 
\begin{theorem}[Looped transformer simulates CoT]\label{thm:cot_formal}
    For any $L$-layer non-looped transformer $\transformer_\theta$, there exists a looped transformer $\transformer_{\theta'}$ with $L+\mathcal{O}(1)$ layers, constantly many more dimensions in embedding, MLP and attention layers and constantly many more attention heads, such that for any input $\vv= (v_i)_{i=1}^n$ and integer $m$, the output of non-looped transformer after $m$ steps of CoT, $\transformer^m_\theta(\vv)$, is the same as that of the looped transformer on input $x$ concatenated by $m$ dummy tokens with $m$ loops, $\transformer_{\theta',m}(\vv,\#^m)$.
\end{theorem}


Below are some helping lemmas towards showing \Cref{thm:cot_formal} is at least as powerful as CoT.
\begin{lemma}[Simulating $\argmax$ using MLP]\label{lem:simulating_hard_argmax}
    For every $d\in \mathbb{N}$ and precision $s\in\mathbb{N}^+$, there exists a 3-layer network with $\relu$ activation and $d^2$ width $f$ with $s$ precision, such that for any $x\in\Floating_s^d$, if there is $k\in [d]$, such that $x_k >\max_{j\neq k,j\in [d]}x_j$, $f(x) = e_k$. 
\end{lemma}


\begin{proof}[Proof of \Cref{lem:simulating_hard_argmax}]
    Define $g_i = 2^s\cdot \relu(2^{-s} - \sum_{j\neq i}\relu(x_j-x_i))$ for each $i\in [n]$. We claim that if there is $k\in [d]$, such that $x_k -\max_{j\neq k,j\in [d]}x_j\ge 2^{-s}$, $g_i = 1$ iff $i=k$ for all $i\in [d]$. First $g_k =2^s\cdot \relu(2^{-s}) =1 $. Next for $i\neq k$, it clearly holds that  $\sum_{j\neq i}\relu(x_j-x_i)\ge 2^{-s}$ and thus $g_i\le 0$. 
    This construction can clearly be implemented by a 3-layer $\relu$ network with $s$ precision.
\end{proof}

\begin{lemma}[Simulating Decoding and Embedding using MLP]\label{lem:simulating_decoding_embedding}
    Given any $s$-precision $\theta_{\tokenembedding}\in\mathbb{R}^{d\times \Sigma}$ and $\theta_{\transoutput}$, there is a 5-layer network $f:\mathbb{R}^d\to \mathbb{R}^d$ with $\relu$ activation and $\max(|\Sigma|^2)$ width with $s$-precision, such that for all $s$-precision $x\in\mathbb{R}^d$ which admits unique $\argmax$ for $v\triangleq \argmax_{o\in\Sigma} (x^\top\theta_{\transoutput})(o)$, it holds that
    \begin{align*}
        f(x) =  \theta_{\tokenembedding}(v).
    \end{align*}
\end{lemma}
\begin{proof}[Proof of \Cref{lem:simulating_decoding_embedding}]
    This is a simple application of \Cref{lem:simulating_hard_argmax}.
\end{proof}

\begin{lemma}[Control Gate]\label{lem:control_gate}
    A 2-layer $\relu$ network with precision $s$ can implement $F:\Floating_s\times \Floating_s\times \{0,1\}, F(x,y,M) = Mx + (1-M)y $. 
\end{lemma}
\begin{proof}[Proof of \Cref{lem:control_gate}]
    Note that $F(x,y,M) = \relu(x-2^s\cdot (1-M)) - \relu(-x-2^s\cdot (1-M)) + \relu(y-2^s\cdot M) - \relu(-y-2^s\cdot M)$. The proof is completed.
\end{proof}

\begin{definition}[Transformer Block with Mask]\label{defi:embedding_layer_w_mask}
Given number of layers $L\in\mathbb{N}^+$, parameter $\theta_\tfblock = (\theta^{(l)}_\mha,\theta^{(l)}_\ff )_{l=0}^{L-1}$, and mask function $M:\mathbb{N}\to\{0,1\}$, we define the $L$-layer \emph{transformer block with mask}$\tfblock_{\theta_\tfblock,M}:(\mathbb{R}^d)^n\to (\mathbb{R}^d)^n$ for any $n\in\mathbb{N}^+$ as 
\begin{equation}
    [\tfblock_{\theta_\tfblock,M}(\vx)]_i \triangleq (1-M(i))x_i + M(i)[\tfblock_{\theta_\tfblock}(\vx)]_i
\end{equation}
\end{definition}

\begin{definition}[Looped Transformer with Mask]\label{defi:looped_transformer_w_mask}
Given number of loops $T\in\mathbb{N}^+$,  parameters $\theta = (\theta_{\tfblock},\theta_\tokenembedding,\theta_\posencoding,\theta_\transoutput)$, and mask functions $\{M^t\}_{t=1}^T$, where $\theta_{\transformer} = (\theta^{(l)}_\mha,\theta^{(l)}_\ff )_{l=0}^{L-1}$, we define the \emph{looped transformer with mask} as  $p_{\theta,T,M}\triangleq \transoutput_{\theta_\transoutput}\circ \tfblock_{\theta_\tfblock,M^T}\circ \cdots \tfblock_{\theta_\tfblock,M^1}\circ \embed_{\theta_\tokenembedding,\theta_\posencoding}$ and the corresponding deterministic version as $\transformer_{\theta,T,M}(v_1,\ldots,v_n) \triangleq \argmax_{v\in\mathcal{V}} p_{\theta,T,M}(v|v_1,\ldots,v_n)$.
\end{definition}

\begin{definition}[Shifting Layer]\label{defi:shifting_layer}
    We define the \emph{shifting layer} $\shift:(\mathbb{R}^d)^n\to (\mathbb{R}^d)^n$ as the following for any $d,n\in\mathbb{N}^+$ and $x_1,\ldots, x_n\in\mathbb{R}^{d}$:
    \begin{align}
        \shift(x_1,x_2,x_3, \ldots,x_n) = (x_1,x_1,x_2,x_3,\ldots,x_{n-1}).
    \end{align}
\end{definition}

\begin{lemma}\label{lem:shifting_layer}
    For input sequence length up to some integer $n$, $\shift$ could be implemented by a attention layer by concatenating each embedding $x_i$ with $(\sbin_s(i), \sbin_s(f(i)))$, where $n= \lceil \log_2 n +1\rceil$.
\end{lemma}

\begin{proof}[Proof of \Cref{lem:shifting_layer}]
    It is equivalent to show we can construct an attention heads which computes $x_{f(i)}$ at each position $i$. 

    To do so, we just need to invoke \Cref{lem:simulating_hard_argmax} and use that to set key and query, so position $i$ attends to position $f(i)$. We set value at position $i$ to be $x_i$. This completes the proof. 
\end{proof}


\begin{lemma}\label{lem:mlp_addition}
    For any positive integer $s>0$, there is a constant-depth MLP $F$ with $O(s)$ hidden neurons per layer and parameters in $\Floating_s$, such that for any input $\bin(x)\in\{-1,+1\}^s$, where $0\le x\le 2^s-1$, it holds that  
    \begin{align*}
        F(x) = \bin(x+1).
    \end{align*}
\end{lemma}

\begin{proof}[Proof of \Cref{lem:mlp_addition}]
    By \Cref{lem:FB_simulating_boolean_gates}, it suffices to show that we can simulate $\bin(x)\mapsto \bin(x)+1$ using $O(s)$ wide, constant-depth boolean circuits with $\AND,\OR,\NOT$ gates with unlimited fan-ins. This is immediate by noting that 
    \begin{align}
        [\bin(x+1)]_i = [\bin(x+1)]_i \oplus \bigwedge_{j=1}^{i-1}[\bin(x+1)]_j
    \end{align}
\end{proof}

\begin{lemma}\label{lem:mlp_comparison}
    For any positive integer $s>0$, there is a constant-depth MLP $F$ with $O(s)$ hidden neurons per layer and parameters in $\Floating_s$, such that for any input $(\bin(x),\bin(y))\in\{-1,+1\}^s$, where $0\le x,y\le 2^s-1$, it holds that  
    \begin{align*}
        F(x,y) = \indct{x > y}.
    \end{align*}
\end{lemma}

\begin{proof}[Proof of \Cref{lem:mlp_comparison}]
    By \Cref{lem:FB_simulating_boolean_gates}, it suffices to show that we can simulate $\bin(x),\bin(y)\mapsto \indct{x> y}$ using $O(s)$ wide, constant-depth boolean circuits with $\AND,\OR,\NOT$ gates with unlimited fan-ins. This is immediate by noting that 
    \begin{align}
\indct{x > y} = \bigvee\limits_{i\in[s]} \left( ([\bin(x)]_i = 1) \wedge ([\bin(y)]_i = 0) \wedge \bigwedge\limits_{1\le j< i} \left([\bin(x)]_j = [\bin(y)]_j\right) \right).
    \end{align}
\end{proof}
\begin{proof}[Proof of \Cref{thm:cot_formal}]

We consider mask $M^t(i) = \indct{i-t\ge n}$, which we call it CoT masking. Let $s = \lfloor \log (n+m) +1 \rfloor$, we use $\overline i$ to denote $\sbin_s(i)$ for $1\le i\le n+m$ for convenience. The embedding size of our newly constructed transformer is larger than the target transformer to be simulated by an additive constant of $3s$. We denote the new embedding at position $i$ after $t$th loop by $(x_i^{(t)}, p_i^{(t)})$, where $x_i^{(t)}$ will be the original embedding of the transformer to be simulated, and $p_i^{(t)}$ is of dimension $3s$ and only depends on $i$ and $t$. In particular, we can show that we can set $p_i^{(t)} \triangleq (\overline i,\overline{[i-2]_+ +1}, \overline{n+t})$ to save information about position and the number of loops --- $p_i^{(0)}$ is from the positional embedding and the update is due to \Cref{lem:mlp_addition}. The size of hidden dimension of MLP and attention (key, query, value) will also be increased by $O(s)$.

The proof contains two steps:
\begin{enumerate}
    \item To show that there is a transformer with CoT masks simulating the target transformer with $m$ steps of CoT and $L$ layers by looping its own $L+O(1)$ layer block $m$ times.
    \item To show that the above looped transformer with CoT masks can be simulated by a standard looped transformer without mask and with constantly many more layers.
\end{enumerate}

For the first claim, starting from the same parameter of the transformers with CoT $\theta$, we build a new looped model with parameter $\theta'$ with constantly many more layers in each transformer block and at most constantly many more heads per attention layer. First, we can add constantly many more layers to use MLP to simulate the decoding-encoding process using \Cref{lem:simulating_decoding_embedding}.
Next, we can add one more transformer layer in each block and use the attention layer to simulate the shifting layer by~\Cref{lem:shifting_layer}, since we have the additional position embedding $p_i^(t)$.
In particular, the embedding we get at position $n+t$ after $t$ loops, $x_{n+t}^{(t)}$,  now simulates the token embedding of $n+t$ of the CoT transformer.
By the way we define CoT mask $M$, for every $t\ge -n+1$, the embedding $\hat{x}_{n+t}^{(t')}$ will keep the same for all $t'\ge \max(t,0)$. 
In $t$th loop, the only embedding update that matters happens at $n+t$th position, because no updates happen at earlier positions, and updates at later positions $n+t'$ for some $t'>t$ will be overwritten eventually in the future loops $t'$, by some value which is independent of their value at the current loop $t$. In the end, we know the embedding $x_i^{(T)}$ in \Cref{defi:looped_transformer_w_mask} is exactly equal to that in CoT transformer, and so does the final output.

For the second claim, because CoT mask can be computed by a $O(\log(n+m))$ wide, constant-depth MLP~(\Cref{lem:mlp_comparison}), together with \Cref{lem:control_gate}, we know it suffices to increase the number of layers per transformer block and embedding size and hidden dimensions by constant to simulate the transformer with mask by a transformer without mask. 
\end{proof}


% \fi



% \ns{Connecting to our $p$-hop experiments}
\begin{theorem}[Restatement of Theorem 4.2, \citep{sanford2024transformers}]\label{thm:khop_log_depth}
    $p$-hop problem (\Cref{defi:khop}) can be solved by $\lfloor\log_2 p\rfloor+2$-layer non-looped transformer with $\log n$ bits of precision, at most $3$ different layers, $d=d_{\ff}=d_{\attn}=O(1)$ embedding size, hidden dimension for MLP and attention, $1$ attention head.
\end{theorem}




\section{RELATED WORK}
\label{sec:relatedwork}
In this section, we describe the previous works related to our proposal, which are divided into two parts. In Section~\ref{sec:relatedwork_exoplanet}, we present a review of approaches based on machine learning techniques for the detection of planetary transit signals. Section~\ref{sec:relatedwork_attention} provides an account of the approaches based on attention mechanisms applied in Astronomy.\par

\subsection{Exoplanet detection}
\label{sec:relatedwork_exoplanet}
Machine learning methods have achieved great performance for the automatic selection of exoplanet transit signals. One of the earliest applications of machine learning is a model named Autovetter \citep{MCcauliff}, which is a random forest (RF) model based on characteristics derived from Kepler pipeline statistics to classify exoplanet and false positive signals. Then, other studies emerged that also used supervised learning. \cite{mislis2016sidra} also used a RF, but unlike the work by \citet{MCcauliff}, they used simulated light curves and a box least square \citep[BLS;][]{kovacs2002box}-based periodogram to search for transiting exoplanets. \citet{thompson2015machine} proposed a k-nearest neighbors model for Kepler data to determine if a given signal has similarity to known transits. Unsupervised learning techniques were also applied, such as self-organizing maps (SOM), proposed \citet{armstrong2016transit}; which implements an architecture to segment similar light curves. In the same way, \citet{armstrong2018automatic} developed a combination of supervised and unsupervised learning, including RF and SOM models. In general, these approaches require a previous phase of feature engineering for each light curve. \par

%DL is a modern data-driven technology that automatically extracts characteristics, and that has been successful in classification problems from a variety of application domains. The architecture relies on several layers of NNs of simple interconnected units and uses layers to build increasingly complex and useful features by means of linear and non-linear transformation. This family of models is capable of generating increasingly high-level representations \citep{lecun2015deep}.

The application of DL for exoplanetary signal detection has evolved rapidly in recent years and has become very popular in planetary science.  \citet{pearson2018} and \citet{zucker2018shallow} developed CNN-based algorithms that learn from synthetic data to search for exoplanets. Perhaps one of the most successful applications of the DL models in transit detection was that of \citet{Shallue_2018}; who, in collaboration with Google, proposed a CNN named AstroNet that recognizes exoplanet signals in real data from Kepler. AstroNet uses the training set of labelled TCEs from the Autovetter planet candidate catalog of Q1–Q17 data release 24 (DR24) of the Kepler mission \citep{catanzarite2015autovetter}. AstroNet analyses the data in two views: a ``global view'', and ``local view'' \citep{Shallue_2018}. \par


% The global view shows the characteristics of the light curve over an orbital period, and a local view shows the moment at occurring the transit in detail

%different = space-based

Based on AstroNet, researchers have modified the original AstroNet model to rank candidates from different surveys, specifically for Kepler and TESS missions. \citet{ansdell2018scientific} developed a CNN trained on Kepler data, and included for the first time the information on the centroids, showing that the model improves performance considerably. Then, \citet{osborn2020rapid} and \citet{yu2019identifying} also included the centroids information, but in addition, \citet{osborn2020rapid} included information of the stellar and transit parameters. Finally, \citet{rao2021nigraha} proposed a pipeline that includes a new ``half-phase'' view of the transit signal. This half-phase view represents a transit view with a different time and phase. The purpose of this view is to recover any possible secondary eclipse (the object hiding behind the disk of the primary star).


%last pipeline applies a procedure after the prediction of the model to obtain new candidates, this process is carried out through a series of steps that include the evaluation with Discovery and Validation of Exoplanets (DAVE) \citet{kostov2019discovery} that was adapted for the TESS telescope.\par
%



\subsection{Attention mechanisms in astronomy}
\label{sec:relatedwork_attention}
Despite the remarkable success of attention mechanisms in sequential data, few papers have exploited their advantages in astronomy. In particular, there are no models based on attention mechanisms for detecting planets. Below we present a summary of the main applications of this modeling approach to astronomy, based on two points of view; performance and interpretability of the model.\par
%Attention mechanisms have not yet been explored in all sub-areas of astronomy. However, recent works show a successful application of the mechanism.
%performance

The application of attention mechanisms has shown improvements in the performance of some regression and classification tasks compared to previous approaches. One of the first implementations of the attention mechanism was to find gravitational lenses proposed by \citet{thuruthipilly2021finding}. They designed 21 self-attention-based encoder models, where each model was trained separately with 18,000 simulated images, demonstrating that the model based on the Transformer has a better performance and uses fewer trainable parameters compared to CNN. A novel application was proposed by \citet{lin2021galaxy} for the morphological classification of galaxies, who used an architecture derived from the Transformer, named Vision Transformer (VIT) \citep{dosovitskiy2020image}. \citet{lin2021galaxy} demonstrated competitive results compared to CNNs. Another application with successful results was proposed by \citet{zerveas2021transformer}; which first proposed a transformer-based framework for learning unsupervised representations of multivariate time series. Their methodology takes advantage of unlabeled data to train an encoder and extract dense vector representations of time series. Subsequently, they evaluate the model for regression and classification tasks, demonstrating better performance than other state-of-the-art supervised methods, even with data sets with limited samples.

%interpretation
Regarding the interpretability of the model, a recent contribution that analyses the attention maps was presented by \citet{bowles20212}, which explored the use of group-equivariant self-attention for radio astronomy classification. Compared to other approaches, this model analysed the attention maps of the predictions and showed that the mechanism extracts the brightest spots and jets of the radio source more clearly. This indicates that attention maps for prediction interpretation could help experts see patterns that the human eye often misses. \par

In the field of variable stars, \citet{allam2021paying} employed the mechanism for classifying multivariate time series in variable stars. And additionally, \citet{allam2021paying} showed that the activation weights are accommodated according to the variation in brightness of the star, achieving a more interpretable model. And finally, related to the TESS telescope, \citet{morvan2022don} proposed a model that removes the noise from the light curves through the distribution of attention weights. \citet{morvan2022don} showed that the use of the attention mechanism is excellent for removing noise and outliers in time series datasets compared with other approaches. In addition, the use of attention maps allowed them to show the representations learned from the model. \par

Recent attention mechanism approaches in astronomy demonstrate comparable results with earlier approaches, such as CNNs. At the same time, they offer interpretability of their results, which allows a post-prediction analysis. \par








\hypertarget{app_uni_fram}{} 
\section{Details of Unified Framework}\label{app_uni_fram}

\begin{table}[h]
\centering
\vspace{-4mm}
\caption{The unified framework for directed link prediction methods.}
\label{app_table_fram}
%\begin{tabular}{@{}l@{\quad}l@{\quad}l@{}}
\begin{tabular}{@{}lllllllll@{}}
\toprule
Encoder $\mathrm{Enc}(\cdot)$ 
 &\multicolumn{2}{l}{Embeddings $({\bm \theta_u}, {\bm \phi_u})$}
 &\multicolumn{6}{l}{Possible Decoder $\mathrm{Dec}(\cdot)$}\\
\midrule
Source-target  & $\vs_u = \bm{\theta}_u$,           & $\vt_u = \bm{\phi}_u$          & \multicolumn{2}{l}{$\sigma\bigl(\vs_u^{\top}\vt_v\bigr)$;} & \multicolumn{2}{l}{$\mathrm{LR}\bigl(\vs_u \odot \vt_v\bigr)$;} & \multicolumn{2}{l}{\quad$\mathrm{LR}\bigl(\vs_u \|\vt_v\bigr)$} \\[3.0pt]

Single real-valued  & $\vh_u = \bm{\theta}_u$,  & $\varnothing = \bm{\phi}_u$          & \multicolumn{2}{l}{$\sigma\bigl(\vh_u^{\top}\vh_v\bigr)$;} & \multicolumn{2}{l}{$\mathrm{MLP}\bigl(\vh_u \odot \vh_v\bigr)$;} & \multicolumn{2}{l}{\quad$\mathrm{MLP}\bigl(\vh_u \|\vh_v\bigr)$} \\[3.0pt]

Complex-valued  & \multicolumn{2}{l}{$\vz_u = \bm{\theta}_u\odot\exp\bigl(i\,\bm{\phi}_u\bigr)$}  & \multicolumn{2}{l}{$\mathrm{Direc}\bigl(\vz_u,\vz_v\bigr)$;} & \multicolumn{4}{l}{$\mathrm{MLP}\bigl(\bm{\theta}_u \|\bm{\theta}_v \|\bm{\phi}_u \|\bm{\phi}_v\bigr)$} \\[3.0pt]

Gravity-inspired  & $\vh_u = \bm{\theta}_u$,  &$m_u = g(\bm{\phi}_u)$         & \multicolumn{3}{l}{$\sigma\bigl(m_v - \lambda\log\|\vh_u-\vh_v\|_2^2\bigr)$;}              & \multicolumn{3}{l}{$\sigma\bigl(m_v - \lambda\log \bigl(\mathrm{dist}_{\mathbb{D}_c^{d'}}(\vh_u,\vh_v)\bigr)\bigr)$}  \\
\bottomrule
\end{tabular}
\end{table}

%Here, we introduce the details of the examples in our unified framework for directed link prediction methods. For the convenience of reading, we copy the Table~\ref{table_fram} from the main paper below.
%In Table~\ref{app_table_fram}, $\sigma$ is the activation function, e.g., $\mathrm{Sigmoid}$; $\mathrm{LR}$ denotes the logistic regression classifier and $\mathrm{MLP}$ denotes multilayer perceptron. $\odot$ and $\|$ denote the Hadamard product and concat process of vectors, respectively. For source-target encoder, we set source embedding $\vs_u = \bm{\theta}_u$, target embedding $\vt_u=\bm{\phi}_u$ for each node $u \in V$. The possible decoders include $\sigma\bigl(\vs_u^{\top}\vt_v\bigr)$, $\mathrm{LR}\bigl(\vs_u \odot \vt_v\bigr)$, and $\mathrm{LR}\bigl(\vs_u \|\vt_v\bigr)$, where $\vs_u^{\top}\vt_v$ denotes the inner product of source  and target embeddings, $\mathrm{LR}(\cdot)$ denotes logistic regression predictor. For single real-valued encoder, we set real-valued embedding $\vh_u = \bm{\theta}_u$ and $\bm{\phi}_u=\varnothing$, $\varnothing$ denotes nonexistent. The possible decoders include $\sigma\bigl(\vh_u^{\top}\vh_v\bigr)$, $\mathrm{MLP}\bigl(\vh_u \odot \vh_v\bigr)$, $\mathrm{MLP}\bigl(\vh_u \|\vh_v\bigr)$. For complex-valued encoder, we set complex-valued embedding $\vz_u=\bm{\theta}_u\odot\exp({i\bm{\phi}_u})$, where $i$ is the imaginary unit. The possible decoders include $\mathrm{Direc}\bigl(\vz_u,\vz_v\bigr)$ and $\mathrm{MLP}\bigl(\bm{\theta}_u \|\bm{\theta}_v \|\bm{\phi}_u \|\bm{\phi}_v\bigr)$, where $\mathrm{Direc}(\cdot)$ denotes the direction-aware decoder defined in DUPLEX~\cite{duplex}. For gravity-inspired encoder, we set real-valued embedding $\vh_u = \bm{\theta}_u$ and mass parameter $m_u = g(\bm{\phi}_u)$, where $g(\cdot)$ is a function or neural network for converting $\bm{\phi}_u$ to a scalar~\cite{dhypr}. The possible decoders include $\sigma\bigl(m_v - \lambda\log\|\vh_u-\vh_v\|_2^2\bigr)$ ~\cite{gragae} and $\sigma\bigl(m_v - \lambda\log \bigl(\mathrm{dist}_{\mathbb{D}_c^{d'}}(\vh_u,\vh_v)\bigr)\bigr)$~\cite{dhypr}, where $\lambda$ is a hyperparameter and $\mathrm{dist}_{\mathbb{D}_c^{d'}}(\cdot)$ denotes hyperbolic distance.

Here, we introduce the details of the examples within our unified learning framework for directed link prediction methods. For readability, we copy Table~\ref{table_fram} from the main paper in Table~\ref{app_table_fram}. Here, $\sigma$ represents the activation function (e.g., $\mathrm{Sigmoid}$), while $\mathrm{LR}$ and $\mathrm{MLP}$ denote the logistic regression predictor and the multilayer perceptron, respectively. The symbols $\odot$ and $\|$ represent the Hadamard product and the vector concatenation process, respectively.

For \underline{source-target encoder}, we define the source embedding as $\vs_u = \bm{\theta}_u$ and the target embedding as $\vt_u=\bm{\phi}_u$ for each node $u \in V$. Possible decoders include:
$\sigma\bigl(\vs_u^{\top}\vt_v\bigr)$,  
$\mathrm{LR}\bigl(\vs_u \odot \vt_v\bigr)$,  
$\mathrm{LR}\bigl(\vs_u \|\vt_v\bigr)$.  
Here, $\vs_u^{\top}\vt_v$ denotes the inner product of the source and target embeddings, while $\mathrm{LR}(\cdot)$ represents the logistic regression predictor.

For \underline{single real-valued encoder}, we define the real-valued embedding as $\vh_u = \bm{\theta}_u$ and set $\bm{\phi}_u=\varnothing$, where $\varnothing$ denotes nonexistence. Possible decoders include:
$\sigma\bigl(\vh_u^{\top}\vh_v\bigr)$,  
$\mathrm{MLP}\bigl(\vh_u \odot \vh_v\bigr)$,  
$\mathrm{MLP}\bigl(\vh_u \|\vh_v\bigr)$.  

For \underline{complex-valued encoder}, we define the complex-valued embedding as $\vz_u=\bm{\theta}_u\odot\exp({i\bm{\phi}_u})$, where $i$ is the imaginary unit. Possible decoders include:
$\mathrm{Direc}\bigl(\vz_u,\vz_v\bigr)$,  
$\mathrm{MLP}\bigl(\bm{\theta}_u \|\bm{\theta}_v \|\bm{\phi}_u \|\bm{\phi}_v\bigr)$.  
Here, $\mathrm{Direc}(\cdot)$ refers to the direction-aware decoder defined in DUPLEX~\cite{duplex}.

For \underline{gravity-inspired encoder}, we define the real-valued embedding as $\vh_u = \bm{\theta}_u$ and set the mass parameter $m_u = g(\bm{\phi}_u)$, where $g(\cdot)$ is a function or neural network that converts $\bm{\phi}_u$ into a scalar~\cite{dhypr}. Possible decoders include:
$\sigma\bigl(m_v - \lambda\log\|\vh_u-\vh_v\|_2^2\bigr)$ ~\cite{gragae},  
$\sigma\bigl(m_v - \lambda\log \bigl(\mathrm{dist}_{\mathbb{D}_c^{d'}}(\vh_u,\vh_v)\bigr)\bigr)$~\cite{dhypr}.  
Here, $\lambda$ is a hyperparameter and $\mathrm{dist}_{\mathbb{D}_c^{d'}}(\cdot)$ represents the hyperbolic distance.


\hypertarget{app_issues}{} 
\section{Issues with Existing Experimental Setup}\label{app_issues}

\hypertarget{app_issues_subtask}{} 
\subsection{Details of multiple subtask setup}\label{app_issues_subtask}
The multiple subtask setup~\cite{magnet,dpyg,duplex} involves four subtasks where graph edges are categorized into four types: positive (original direction), reverse (inverse direction), bidirectional (both directions), and nonexistent (no connection). Each subtask focuses on predicting a specific edge type. The details are as follows:
\begin{itemize}[topsep=0pt, partopsep=0pt]
    \item Existence Prediction (EP): The model predicts whether a directed edge $(u, v)$ exists in the graph. Both reverse and nonexistent edges are treated as nonexistence.
    \item Directed Prediction (DP): The model predicts the direction of edges for node pairs $(u, v)$, where either $(u, v) \in E$ or $(v, u) \in E$.
    \item Three-Type Classification (3C): The model classifies an edge as positive, reverse, or nonexistent.
    \item Four-Type Classification (4C): The model classifies edges into four categories: positive, reverse, bidirectional, or nonexistent.
\end{itemize}

\begin{table}[h]
\centering
\caption{The results of MagNet~\cite{magnet} as reported in the original paper, alongside the reproduced MagNet and
MLP results.}
%\resizebox{\textwidth}{!}{
\begin{tabular}{@{}cllllll@{}}
\toprule
\multicolumn{1}{l}{Task}              & Method & Cornell & Texas & Wisconsin & Cora-ML & CiteSeer \\ \midrule
\multirow{3}{*}{DP} & MagNet (reported) &82.9$\pm$3.5 &80.9$\pm$4.2 &83.3$\pm$3.0 &86.5$\pm$0.7 &84.8$\pm$1.2 \\ %\cmidrule(l){2-7} 
                                      & MagNet (reproduced) &84.69$\pm$2.86 &87.15$\pm$4.49 &85.72$\pm$4.24 &\textbf{88.56$\pm$0.44}
                                      &\textbf{89.72$\pm$1.06}         \\
                                      & MLP    &\textbf{84.95$\pm$3.10} &\textbf{89.79$\pm$4.06} &\textbf{87.21$\pm$2.48}
                                      &86.56$\pm$0.70        &88.92$\pm$0.56          \\ \midrule
\multirow{3}{*}{EP} & MagNet (reported) &\textbf{81.1$\pm$3.3} &83.6$\pm$2.7 &\textbf{82.8$\pm$2.2} &\textbf{82.7$\pm$0.7} &79.9$\pm$0.6   \\
                                      & MagNet (reproduced)    
&75.42$\pm$6.25 &\textbf{84.52$\pm$4.12} &78.16$\pm$3.83 &80.04$\pm$0.89 &83.95$\pm$0.35\\
                                      & MLP       &76.01$\pm$4.64&82.97$\pm$4.33 &79.79$\pm$2.71 &79.47$\pm$1.08        &\textbf{84.78$\pm$0.56}          \\ \bottomrule
\end{tabular}
\label{tb_magnet}
\end{table}

\begin{table}[h]
\centering
\caption{The results of MLP and baselines under PyGSD~\cite{dpyg} setup on Direction Prediction (DP) task.}
%\resizebox{\textwidth}{!}{
\begin{tabular}{lcccccc}
\toprule
Method & Cora-ML & CiteSeer & Telegram & Cornell & Texas & Wisconsin \\ \midrule
MLP       &86.13$\pm$0.45  & 85.51$\pm$0.63  & 95.61$\pm$0.15  & \textbf{86.40$\pm$2.60}  &\textbf{83.08$\pm$4.33}  & \textbf{87.95$\pm$2.65}  \\ \midrule
DGCN      & 85.49$\pm$0.75  & 84.85$\pm$0.56  & 96.03$\pm$0.35  & 84.55$\pm$3.71  & 79.59$\pm$5.09  & 84.57$\pm$3.59  \\ 
DiGCN     & 85.37$\pm$0.54  & 83.88$\pm$0.82  & 94.95$\pm$0.54  & 85.41$\pm$2.78  & 76.57$\pm$3.98  & 82.41$\pm$2.56  \\ 
DiGCNIB   & 86.12$\pm$0.42  & 85.58$\pm$0.56  & 95.99$\pm$0.44  & 86.28$\pm$3.37  &82.27$\pm$3.51  &87.07$\pm$2.16  \\ 
MagNet    & \textbf{86.33$\pm$0.54}  & \textbf{85.80$\pm$0.63}  & \textbf{96.97$\pm$0.21}  & 83.29$\pm$4.28  & 80.25$\pm$4.78  & 86.60$\pm$2.72  \\ \bottomrule
\end{tabular}%}
\label{tb_dpyg_dp}
\end{table}

\begin{table}[h]
\centering
\caption{The results of MLP and baselines under PyGSD~\cite{dpyg} setup on Existence Prediction (EP) task.}
%\resizebox{\textwidth}{!}{
\begin{tabular}{lcccccc}
\toprule
Method & Cora-ML & CiteSeer & Telegram & Cornell & Texas & Wisconsin \\ \midrule
MLP       & \textbf{78.85$\pm$0.83}  & 71.03$\pm$0.89  & 82.72$\pm$0.57  & 68.41$\pm$2.05  & \textbf{69.58$\pm$2.29}  & \textbf{72.53$\pm$2.27}  \\ \midrule 
DGCN      & 76.45$\pm$0.49  & 70.72$\pm$0.79  & 83.18$\pm$1.55  & 67.95$\pm$2.27  & 63.65$\pm$2.40  & 68.12$\pm$2.86    \\ 
DiGCN     & 76.17$\pm$0.49  & 72.00$\pm$0.88  & 83.12$\pm$0.43  & 67.16$\pm$1.82  & 63.54$\pm$3.85  & 67.01$\pm$2.47    \\ 
DiGCNIB   & 78.80$\pm$0.51  & \textbf{74.55$\pm$0.91}  & 84.49$\pm$0.51  & \textbf{69.77$\pm$2.05}  & 67.60$\pm$3.44  & 70.78$\pm$2.79   \\ 
MagNet    & 77.37$\pm$0.45  & 71.47$\pm$0.71  & \textbf{85.82$\pm$0.39}  & 68.98$\pm$2.27  & 65.94$\pm$1.88  & 71.23$\pm$2.53    \\  \bottomrule
\end{tabular}%}
\label{tb_dpyg_ep}
\end{table}



\hypertarget{app_issues_results}{} 
\subsection{Details of experimental setting and more results}\label{app_issues_results}
\textbf{Experimental setting}. In this part of the experiment, we strictly follow the configurations of each setup and reproduce the results using the provided codes. For the MLP model, we implement a simple two-layer network with 64 hidden units, while the learning rate and weight decay are tuned according to the settings of each setup to ensure a fair comparison. For the MagNet~\cite{magnet} setup, we reproduce the reported MagNet results and include the MLP results. For PyGSD~\cite{dpyg} setup, we get the results in our environment using their baseline code and settings, and report the MLP results. For DUPLEX~\cite{duplex}, we reproduce the DUPLEX results and additionally report the results of graph propagation without using the test edges and the MLP results.

%PyGSD is a software package designed for Signed and Directed Graphs, built on PyTorch Geometric (PyG). 


\begin{table}[h]
\centering
\caption{Link prediction results for CiteSeer dataset under the DUPLEX~\cite{duplex} setup: no superscripts are from the DUPLEX paper, $^\dagger$ indicates reproduction with test set edges in training, and $^\ddagger$ indicates reproduction without test set edges in training.}
%\resizebox{\textwidth}{!}{
\begin{tabular}{@{}lllllll@{}}
\toprule

%\multicolumn{1}{l}{\multirow{2}{*}{Dataset}} & \multirow{2}{*}{Method} & \multicolumn{2}{c}{EP} & \multicolumn{2}{c}{DP} & 3C  & 4C  \\ \cmidrule(l){3-8} 
Method &EP(ACC) &EP(AUC) &DP(ACC) &DP(AUC) &3C(ACC) &4C(ACC) \\ \midrule
%\multicolumn{1}{l}{} & & ACC  & AUC  & ACC   & AUC  & ACC & ACC \\ \midrule

MagNet &80.7$\pm$0.8 &88.3$\pm$0.4 &91.7$\pm$0.9 &96.4$\pm$0.6 &72.0$\pm$0.9 &69.3$\pm$0.4 \\ 
DUPLEX &95.7$\pm$0.5 &98.6$\pm$0.4 &98.7$\pm$0.4 &99.7$\pm$0.2 &94.8$\pm$0.2 &91.1$\pm$1.0  \\ \midrule

DUPLEX$^\dagger$ &92.11$\pm$0.78 &95.85$\pm$0.87 &97.54$\pm$0.54 &98.93$\pm$0.59  &88.22$\pm$1.06 &84.77$\pm$1.01 \\
MLP$^\dagger$ &85.74$\pm$1.80	&93.33$\pm$1.27	&\textbf{97.55$\pm$0.97}	&\textbf{99.58$\pm$0.52}	&81.20$\pm$0.82	&76.30$\pm$0.62  \\ \midrule 

DUPLEX$^\ddagger$  &83.59$\pm$1.47 &89.34$\pm$1.03 &85.56$\pm$1.36 &91.82$\pm$0.98 &76.37$\pm$2.07 &73.80$\pm$2.01 \\
MLP$^\ddagger$ &77.34$\pm$1.86	&87.36$\pm$1.26	&\textbf{89.19$\pm$0.95}	&\textbf{95.97$\pm$0.61}	&67.82$\pm$1.21	&64.25$\pm$1.35 \\ \bottomrule

\end{tabular}
\label{tb:app_duplex}
\end{table}




\textbf{Results with the MagNet setup}. Table~\ref{tb_magnet} presents the results of MagNet~\cite{magnet} as reported in the original paper, alongside the reproduced MagNet and MLP results, with the highest values bolded for emphasis. These results directly correspond to Figures~\ref{fig:dp} and~\ref{fig:ep}.



\begin{wrapfigure}{r}{0.45\textwidth} % 右侧浮动,占 45% 宽度
    \centering
    \vspace{-3mm}
    \includegraphics[width=\linewidth]{figure/citeseer_bar_chart.pdf}
    \vspace{-8mm}
    \caption{The number of samples and accuracy for each class of DUPLEX on the CiteSeer dataset in the 4C task.}
    \vspace{-2mm}
    \label{fig:duplex_cls_citeseer}
\end{wrapfigure}


\textbf{Results with the PyGSD setup}. PyGSD~\cite{dpyg} is a software package for Signed and Directed Graphs, built on PyTorch Geometric (PyG)~\cite{pyg}. Tables~\ref{tb_dpyg_dp} and~\ref{tb_dpyg_ep} present the results of MLP alongside several baselines on the Direction Prediction (DP) and Existence Prediction (EP) tasks, with the highest values highlighted in bold. The baselines include DGCN~\cite{dgcn-tong}, DiGCN~\cite{digcn}, DiGCNIB~\cite{digcn}, and MagNet~\cite{magnet}, as used in PyGSD. Across six datasets, MLP demonstrates competitive performance, achieving state-of-the-art results for both tasks on the Texas and Wisconsin datasets.


\textbf{Results with the DUPLEX setup}. In Table~\ref{tb:app_duplex}, we present the link prediction results for the CiteSeer dataset under the DUPLEX setup. These results, consistent with those in Table~\ref{tb:duplex}, highlight two key observations: (1) MLP outperforms DUPLEX on several tasks, and (2) label leakage in DUPLEX significantly impacts the results.
Additionally, Figure~\ref{fig:duplex_cls_citeseer} illustrates the number of samples and the accuracy for each class in the DUPLEX setup on the CiteSeer dataset for the 4C task. These findings, consistent with those in Figure~\ref{fig:imbalance}, further emphasize the class imbalance issue in the 4C task.

%These issues highlight serious label leakage concerns in existing setups. 



%\begin{table*}[h]
%\centering
%\caption{The results obtained under the DUPLEX~\cite{duplex} experimental setup, where those without superscripts denote the results reported in the DUPLEX paper, $^\dagger$ denotes the results we directly reproduced with the edges of the test set used in the training phase, and $^\ddagger$ denotes the results without the edges of the test set used in the training phase.}
%\resizebox{\textwidth}{!}{
%\begin{tabular}{@{}clllllll@{}}
%\toprule
%\multicolumn{1}{l}{\multirow{2}{*}{Dataset}} & \multirow{2}{*}{Method} & \multicolumn{2}{c}{EP} & \multicolumn{2}{c}{DP} & 3C  & 4C  \\ \cmidrule(l){3-8} 
%\multicolumn{1}{l}{} & & ACC  & AUC  & ACC   & AUC  & ACC & ACC \\ \midrule
%\multirow{6}{*}{Cora} & MagNet &81.4±0.3 &89.4±0.1 &88.9±0.4 &95.4±0.2 &66.8±0.3 &63.0±0.3 \\
%& DUPLEX &93.2±0.1 &95.9±0.1 &95.9±0.1 &97.9±0.2 &92.2±0.1 &88.4±0.4  \\ \cmidrule(l){2-8} 

%& DUPLEX$^\dagger$ &93.49±0.21 &95.61±0.20 &95.25±0.16 &96.34±0.23  &92.41±0.21 &89.76±0.25 \\
%& MLP$^\dagger$ &88.53±0.22	&95.46±0.18	&95.76±0.21	&99.25±0.06	&79.97±0.48	&78.49±0.26  \\ \cmidrule(l){2-8} 

%& DUPLEX$^\ddagger$ &87.43±0.20 &91.16±0.24 &88.43±0.16 &91.74±0.38 &84.53±0.34 &81.36±0.46 \\
%& MLP$^\ddagger$ &84.00±0.29	&91.52±0.25	&90.83±0.16	&96.48±0.28	&72.93±0.21	&71.51±0.20  \\    \midrule

%\multirow{6}{*}{Citeseer} & MagNet &80.7±0.8 &88.3±0.4 &91.7±0.9 &96.4±0.6 &72.0±0.9 &69.3±0.4 \\ 
%& DUPLEX &95.7±0.5 &98.6±0.4 &98.7±0.4 &99.7±0.2 &94.8±0.2 &91.1±1.0  \\ \cmidrule(l){2-8} 

%& DUPLEX$^\dagger$ &92.11±0.78 &95.85±0.87 &97.54±0.54 &98.93±0.59  &88.22±1.06 &84.77±1.01 \\
%& MLP$^\dagger$ &85.74±1.80	&93.33±1.27	&97.55±0.97	&99.58±0.52	&81.20±0.82	&76.30±0.62  \\ \cmidrule(l){2-8} 

%& DUPLEX$^\ddagger$  &83.59±1.47 &89.34±1.03 &85.56±1.36 &91.82±0.98 &76.37±2.07 &73.80±2.01 \\
%& MLP$^\ddagger$ &77.34±1.86	&87.36±1.26	&89.19±0.95	&95.97±0.61	%&67.82±1.21	&64.25±1.35 \\ \bottomrule
%\end{tabular}}
%\end{table*}



% \begin{figure}[h]
%     \centering
%    \vspace{-1mm}
%   %\hspace{-3mm}  
%    \includegraphics[width=75mm]{figure/citeseer_bar_chart.pdf}
%    \caption{The number of samples and the accuracy for each class of DUPLEX on the CiteSeer dataset in the 4C task.}
%    \vspace{-1mm}
%     \label{fig:duplex_cls_citeseer}
%  \end{figure}


 



\hypertarget{app_sdgae}{} 
\section{Details and Experimental Setting for SDGAE}\label{app_sdgae}

\hypertarget{app_sdgae_conv}{} 
\subsection{More details of graph convolution}\label{app_sdgae_conv}
The graph convolutional layer of SDGAE is defined as: 
\begin{equation}\label{eq:app_bg_A_hat}
    \left[ 
        \begin{array}{c}
            \mS^{(\ell+1)} \\
            \mT^{(\ell+1)}
        \end{array}
    \right] =
    \left[ 
        \begin{array}{c}
            w^{(\ell)}_S \\
            w^{(\ell)}_T
        \end{array}
    \right] 
    \odot
    \left[ 
        \begin{array}{cc}
            \mathbf{0} & \Tilde{\mA} \\
            \Tilde{\mA}^{\top} & \mathbf{0}
        \end{array}
    \right]
    \left[ 
        \begin{array}{c}
            \mS^{(\ell)} \\
            \mT^{(\ell)}
        \end{array}
    \right] 
    +
    \left[ 
        \begin{array}{c}
            \mS^{(\ell)} \\
            \mT^{(\ell)}
        \end{array}
    \right].
\end{equation}
Expanding this formulation, the source and target embeddings at each layer are denoted as:
\begin{equation}\label{app_eq_sdgae}
    \mS^{(\ell+1)} =w_{S}^{(\ell)}\Tilde{\mA}\mT^{(\ell)}+\mS^{(\ell)}, \quad \mT^{(\ell+1)} =w_{T}^{(\ell)}\Tilde{\mA}^{\top}\mS^{(\ell)}+\mT^{(\ell)}.
\end{equation}
If we denote $\mX_S = \mS^{(0)}=\mathrm{MLP}_S(\mX)$ and $\mX_T = \mT^{(0)}=\mathrm{MLP}_T(\mX)$, we can derive:
\begin{align*}
\mS^{(1)} &= w_{S}^{(0)}\Tilde{\mA}\mX_T + \mX_S, \\
\mT^{(1)} &= w_{T}^{(0)}\Tilde{\mA}^{\top}\mX_S + \mX_T, \\
\mS^{(2)} &= \mX_S + \left(w_{S}^{(1)} + w_{S}^{(0)}\right)\Tilde{\mA}\mX_T + w_{S}^{(1)}w_{T}^{(0)}\Tilde{\mA}\Tilde{\mA}^{\top}\mX_S, \\
\mT^{(2)} &= \mX_T + \left(w_{T}^{(1)} + w_{T}^{(0)}\right)\Tilde{\mA}^{\top}\mX_S + w_{T}^{(1)}w_{S}^{(0)}\Tilde{\mA}^{\top}\Tilde{\mA}\mX_T.
\end{align*}
%The result shows the source and target embeddings corresponding to the number of layers $L=1$ and $L=2$. It can be found that the propagation is very similar to the spectral-based GNNs. Specifically, the propagation expression of the spectral-based GNNs is $\mY = \nolimits\sum_{\ell=0}^{L}w_{\ell}\Tilde{\mA}^{\ell}$, where $w_{\ell}$ denotes the learnable polynomial weights and $\Tilde{\mA}$ denotes the symmetrically normalised undirected graph adjacency matrix. By comparison, it can be found that SDGAE also achieves spectral--based propagation by learning different weights $w_S$ and $w_T$, and further spectrum analyses about it will be the subject of further research.
These results represent the source and target embeddings for layers \(L=1, 2\). The propagation pattern closely resembles that of spectral-based GNNs~\cite{gprgnn,bernnet,chebnetii}. Specifically, the propagation in spectral-based GNNs is expressed as \(\mY = \sum_{\ell=0}^{L}w_{\ell}\Tilde{\mA}^{\ell}\), where \(w_{\ell}\) denotes the learnable polynomial weights and \(\Tilde{\mA}\) represents the symmetrically normalized adjacency matrix of an undirected graph. By comparison, SDGAE achieves spectral-based propagation by learning separate weights \(w_S\) and \(w_T\). Further spectrum analysis of SDGAE will be explored in future research.





\textbf{Time Complexity}. The primary time complexity of SDGAE arises from the graph convolution process. Based on Equation~(\ref{app_eq_sdgae}), the time complexity of SDGAE's convolution is given by \( O(2Lmd) \), where \( m \) is the number of edges, \( L \) is the number of layers, and \( d \) is the embedding dimension. This complexity scales linearly with \( m \) and is lower than that of DiGAE, which involves multiple learnable weight matrices.


\begin{wraptable}{r}{105mm}
\centering
\vspace{-8mm}
\caption{The parameters of SDGAE on different datasets.}
\begin{tabular}{lcccccc}
\toprule
Datasets &hidden &embedding &MLP layer &$L$ &lr &wd  \\ \midrule
Cora-ML & 64 & 64 &1 &5 &0.01 &0.0 \\
CiteSeer& 64 & 64 &1 &5 &0.01 &0.0 \\
Photo & 64 & 64 &2 &5 &0.005 &0.0 \\
Computers & 64 & 64 &2 & 3 & 0.005 &0.0\\
WikiCS & 64 & 64 & 2&5 & 0.005 &5e-4 \\
Slashdot & 64 & 64 &2 &5 &0.01 & 5e-4 \\
Epinions & 64 & 64 &2 &5 & 0.005 &0.0 \\
\bottomrule
\end{tabular}
\vspace{-2mm}
\label{app_para_sdage}
\end{wraptable}

\hypertarget{app_sdgae_setting}{} 
\subsection{Experimental setting}\label{app_sdgae_setting}
For the SDGAE experimental setting, we aligned our settings with those of other baselines in DirLinkBench to ensure fairness. For the MLP used in $\mX$ initialization, we set the number of layers to one or two, matching DiGAE's convolutional layer configurations. The number of hidden units and the embedding dimension were both set to 64. The learning rate (lr) was chosen as either 0.01 or 0.005, and weight decay (wd) was set to 0.0 or 5e-4, following the configurations of most GNN baselines. The number of convolutional layers $L$ was set to 3, 4, or 5. We performed a grid search to optimize parameters on the validation set, and Table~\ref{app_para_sdage} presents the corresponding SDGAE parameters for different datasets.  





\hypertarget{app:experiments}{} 
\section{More Details of DirLinkBench} \label{app:experiments}

\hypertarget{app_dataset_baseline}{} 
\subsection{Datasets and baselines}\label{app_dataset_baseline}
\textbf{Datasets Details.}
Table~\ref{dataset_info} summarizes the statistical characteristics of seven directed graphs. Avg. Degree indicates average node connectivity and \%Directed Edges reflects inherent directionality. Detailed descriptions:
\begin{itemize}
    \item Cora-ML~\cite{mccallum2000cora_ml,bojchevski2018cora_ml} and CiteSeer~\cite{sen2008citeseer} are two citation networks. 
    Nodes represent academic papers, and edges represent directed citation relationships.
    \item Photo and Computers~\cite{shchur2018computerandphoto} are two Amazon co-purchasing networks. Nodes denote products, and directed edges denote the sequential purchase relationships.
    \item WikiCS~\cite{mernyei2020wiki} is a weblink network where nodes represent computer science articles from Wikipedia and directed edges correspond to hyperlinks between articles.
    \item Slashdot~\cite{ordozgoiti2020slash} and Epinions~\cite{massa2005epinion} are two social networks, where nodes represent users. Edges in Slashdot indicate directed social interactions between users, and edges in Epinions represent unidirectional trust relationships between users.
\end{itemize}


\begin{table*}[h]
\centering
\caption{Statistics of DirLinkBench datasets.}
\resizebox{\textwidth}{!}{
\begin{tabular}{lrrrrcr}
\toprule
Datasets & \#Nodes & \#Edges &Avg. Degree & \#Features & \%Directed Edges &Description \\ \midrule
Cora-ML & 2,810 & 8,229 & 5.9 &2,879 & 93.97 & citation network \\

CiteSeer & 2,110 & 3,705 & 3.5 & 3,703 & 98.00 & citation network \\
Photo & 7,487 & 143,590 & 38.4 & 745 & 65.81 & co-purchasing network \\
Computers & 13,381 & 287,076 & 42.9 & 767 & 71.23 & co-purchasing network \\
WikiCS & 11,311 & 290,447 & 51.3 & 300 & 48.43 & weblink network \\
Slashdot & 74,444 & 424,557 & 11.4 &- & 80.17 & social network \\
Epinions & 100,751 & 708,715 & 14.1 &- & 65.04 & social network \\
\bottomrule
\end{tabular}}
\label{dataset_info}
\end{table*}


\textbf{Baseline Implementations}. For MLP, GCN, GAT, and APPNP, we use the PyTorch Geometric (PyG) library~\cite{pyg} implementations. For DCN, DiGCN, and DiGCNIB, we rely on the PyTorch Geometric Signed Directed (PyGSD) library~\cite{dpyg} implementations. For other baselines, we use the original code released by the authors. Here are the links to each repository.
\begin{itemize}[topsep=0pt, partopsep=0pt]
    \item Pytorch Geometric library:
    \href{https://github.com/pyg-team/pytorch_geometric/tree/master/benchmark}{https://github.com/pyg-team/pytorch\_geometric/tree/master/benchmark}
    \item PyTorch Geometric Signed Directed library: \href{https://github.com/SherylHYX/pytorch_geometric_signed_directed}{https://github.com/SherylHYX/pytorch\_geometric\_signed\_directed}
    \item STRAP: \href{https://github.com/yinyuan1227/STRAP-git}{https://github.com/yinyuan1227/STRAP-git}
    \item ODIN: \href{https://github.com/hsyoo32/odin}{https://github.com/hsyoo32/odin} 
    \item ELTRA: \href{https://github.com/mrhhyu/ELTRA}{https://github.com/mrhhyu/ELTRA} 
    %\item \textbf{GravityGAE:}  \href{https://github.com/deezer/gravity_graph_autoencoders}{https://github.com/deezer/gravity\_graph\_autoencoders}
    \item DiGAE:
    \href{https://github.com/gidiko/DiGAE}{https://github.com/gidiko/DiGAE}
    \item DHYPR:
    \href{https://github.com/hongluzhou/dhypr}{https://github.com/hongluzhou/dhypr}
    \item DirGNN:
    \href{https://github.com/emalgorithm/directed-graph-neural-network}{https://github.com/emalgorithm/directed-graph-neural-network}
    \item MagNet:
    \href{https://github.com/matthew-hirn/magnet}{https://github.com/matthew-hirn/magnet}
    %\item \textbf{LightDiC:}
    %\href{https://github.com/xkLi-Allen/LightDiC}{https://github.com/xkLi-Allen/LightDiC}
    \item DUPLEX:
    \href{https://github.com/alipay/DUPLEX}{https://github.com/alipay/DUPLEX}
\end{itemize}
\textbf{Baseline setting}. For STRAP, ODIN, and ELTRA, we use four decoders for selection:  
\(\sigma\bigl(\vs_u^{\top}\vt_v\bigr)\), \(\mathrm{LR}\bigl(\vs_u \odot \vt_v\bigr)\), \(\mathrm{LR}\bigl(\vs_u \|\vt_v\bigr)\), and \(\mathrm{LR}\bigl(\vs_u \| \vs_v \| \vt_u \| \vt_v\bigr)\)~\cite{odin}.  
For embedding generation, we follow the parameter settings specified in their respective papers and refer to the code for further details.
For MLP, GCN, GAT, APPNP, DGCN, DiGCN, DiGCNIB, and DirGNN, we use four combinations of loss functions and decoders for selection: CE loss with \(\mathrm{MLP}\bigl(\vh_u\|\vh_v \bigr)\), BCE loss with \(\mathrm{MLP}\bigl( \vh_u\|\vh_v\bigr)\), BCE loss with \(\mathrm{MLP}\bigl( \vh_u \odot \vh_v\bigr)\), and BCE loss with \(\sigma\bigl( \vh_u^{\top}\vh_v\bigr)\). For MagNet, DUPLEX, DHYPR, and DiGAE, we follow the reported loss function and decoder settings from their respective papers. For all GNNs, we set the number of epochs to 2000 with an early stopping criterion of 200 and run each experiment 10 times to report the mean and standard deviation differences.  
The hyperparameter settings for the baselines are detailed below, where 'hidden' represents the number of hidden units, 'embedding' refers to the embedding dimension, 'undirected' indicates whether an undirected training graph is used, 'lr' stands for the learning rate, and 'wd' denotes weight decay.
\begin{itemize}[topsep=0pt, partopsep=0pt]
    \item MLP: hidden: 64, embedding: 64, layer: 2, lr: \{0.01, 0.005\}, wd:  \{0.0, 5e-4\}.
    \item GCN: hidden: 64, embedding: 64, layer: 2, undirected: \{True, False\}, lr: \{0.01, 0.005\}, wd:  \{0.0, 5e-4\}.
    \item GAT: hidden: 8, heads: 8, embedding: 64, layer: 2,  undirected: \{True, False\}, lr: \{0.01, 0.005\}, wd:  \{0.0, 5e-4\}
    \item APPNP: hidden: 64, embedding: 64, $K$: 10, $\alpha$: \{0.1,0.2\}, undirected: \{True, False\}, lr: \{0.01, 0.005\}, wd:  \{0.0, 5e-4\}.
    \item DGCN: hidden: 64, embedding: 64, lr: \{0.01, 0.005\}, wd:  \{0.0, 5e-4\}.
    \item DiGCN and DiGCNIB: hidden: 64, embedding: 64, $\alpha$: \{0.1,0.2\}, layer: 2, lr: \{0.01, 0.005\}, wd:  \{0.0, 5e-4\}.
    \item DirGNN:  hidden: 64, embedding: 64, layer: 2, $\alpha$: \{0.0, 0.5, 1.0\},  jk: \{'cat','max'\}, normalize: \{True, False\}, lr: \{0.01, 0.005\}, wd:  \{0.0, 5e-4\}.
    \item MagNet: hidden: 64, embedding: 64, layer: 2, $K$: \{1, 2\}, $q$: \{0.05, 0.1, 0.15, 0.2, 0.25\}, lr: \{0.01, 0.005\}, wd:  \{0.0, 5e-4\}.
    \item DUPLEX: hidden: 64, embedding: 64, layer: 3, head: 1, loss weight:\{0.1, 0.3\}, loss decay: \{0.0, 1e-2, 1e-4\}, lr: \{0.01, 0.005\}, wd:  \{0.0, 5e-4\}.
    \item DHYPR: hidden: 64, embedding: 32, proximity: \{1, 2\}, $\lambda$: \{0.01, 0.05, 1, 5\}, lr: \{0.01, 0.001\}, wd:  \{0.0, 0.001\}.
    \item DiGAE: hidden: 64, embedding: 64, single layer: \{True, False\}, $(\alpha,\beta): \{0.0, 0.2, 0.4, 0.6, 0.8\}^2$, lr: \{0.01, 0.005\}, wd:  \{0.0, 5e-4\}.
\end{itemize}


\hypertarget{app_bench_metric}{}
\subsection{Metric description}\label{app_bench_metric}
\textbf{Mean Reciprocal Rank (MRR)} evaluates the capability of models to rank the first correct entity in link prediction tasks. It assigns higher weights to top-ranked predictions by computing the average reciprocal rank of the first correct answer across queries: $\text{MRR} = \frac{1}{|Q|} \sum_{i=1}^{|Q|} \frac{1}{\text{rank}_i}$, where \(|Q|\) is the total number of queries and \(\text{rank}_i\) denotes the position of the first correct answer for the \(i\)-th query. MRR emphasizes early-ranking performance, making it sensitive to improvements in top predictions.


\textbf{Hits@K} measures the proportion of relevant items that appear in the top-$\text{K}$ positions of the ranked list of items. For \(N\) queries, Hits@K$=\frac{1}{N}\sum\nolimits_{i=1}^{N}\mathbf{1}(\text{rank}_i\leq \text{K})$,
where rank$_i$ is the rank of the $i$-th sample and the indicator function $\mathbf{1}$ is 1 if rank$_i \leq \text{K}$, and 0 otherwise. Following the OGB benchmark~\cite{ogb}, link prediction implementations compare each positive sample's score against a set of negative sample scores. A "hit" occurs if the positive sample's score surpasses at least K-1 negative scores, with final results averaged across all queries.


\textbf{Area Under the Curve (AUC)} measures the likelihood that a positive sample is ranked higher than a random negative sample. \(\text{AUC} = \frac{\sum_{i=1}^{M}\sum_{j=1}^{N}\mathbf{1}(s_i^{\text{pos}} > s_j^{\text{neg}})}{M \times N}\), where \(M\) and \(N\) are positive/negative sample counts, \(s_i^{\text{pos}}\) and \(s_j^{\text{neg}}\) their prediction scores. Values approaching 1 indicate perfect separation of positive and negative edges.

\textbf{Average Precision (AP)} is defined as the area under the Precision-Recall (PR) curve. Formally, \(\text{AP} = \sum_{i=1}^{N} (R_i - R_{i-1}) \times P_i,\) where \(P_i\) is the precision at the \(i\)-th threshold, \(R_i\) is the recall at the \(i\)-th threshold, and \(N\) is the number of thresholds considered. 

\textbf{Accuracy (ACC)} measures the proportion of correctly predicted samples among all predictions. Formally, \(\text{ACC} = \frac{TP + TN}{TP + TN + FP + FN},\) where \(TP\), \(TN\), \(FP\), and \(FN\) represent true positives, true negatives, false positives, and false negatives, respectively.

\begin{figure*}[h]
    \centering
   \vspace{-2mm}
   \hspace{-3mm}
   \subfigure[Photo]{
   \label{fig:app_feature_photo}
   \includegraphics[width=57mm]{figure/feature_photo.pdf}
   }
   \hspace{-5mm}
   \subfigure[WikiCS]{
   \label{fig:app_feature_wiki}
   \includegraphics[width=57mm]{figure/feature_wiki.pdf}}
   \hspace{-4mm}
   \subfigure[Slashdot]{
   \label{fig:app_feature_slsh}
   \includegraphics[width=57mm]{figure/feature_slash.pdf}}
   \hspace{-4mm}
   \vspace{-2mm}
   \caption{Comparison of various methods' performance using original features, in/out degrees, or random features as inputs on 3 datasets.}
    \vspace{-2mm}
   \label{fig:app_feature}
\end{figure*}

\hypertarget{app_bench_res_ana}{}
\subsection{Additional results in analysis}
\label{app_bench_res_ana}

\subsubsection{Feture inputs}\label{app_feature_input}
In Figures~\ref{fig:app_feature_photo} and~\ref{fig:app_feature_wiki}, we compare the performance of various methods using original features and in/out degrees as inputs on Photo and WikiCS. Figure~\ref{fig:app_feature_slsh} presents a similar comparison using random features and in/out degrees as inputs on Slashdot. The results highlight the significant impact of feature inputs on GNN performance and also demonstrate that in-/out-degree information plays a crucial role in link prediction.

 \begin{figure}[ht]
    \centering
   \vspace{-1mm}
%   %\hspace{-3mm} 
    \subfigure[Cora-ML]{
   \includegraphics[width=72mm]{figure/loss_embedding_cora.pdf}
   }
   \subfigure[Slashdot]{
   \includegraphics[width=72mm]{figure/loss_embedding_slashdot.pdf}
   }
    \vspace{-2mm}
   \caption{Comparison of different decoders on embedding methods}
   \vspace{-1mm}
    \label{fig:app_loss_embedding}
 \end{figure}

 \begin{figure}[ht]
    \centering
   \vspace{-1mm}
%   %\hspace{-3mm} 
    \subfigure[Cora-ML]{
   \includegraphics[width=72mm]{figure/loss_gnn_cora.pdf}
   }
   \subfigure[Photo]{
   \includegraphics[width=72mm]{figure/loss_gnn_photo.pdf}
   }
   \vspace{-2mm}
   \caption{Comparison of various loss functions with different decoders on GNNs.}
    \label{fig:app_loss_gnn}
 \end{figure}

\subsubsection{Loss function and decoder}\label{app_loss_function}
We present a comparison of different decoders on embedding methods in Figure~\ref{fig:app_loss_embedding} and a comparison of various loss functions with different decoders on GNNs in Figure~\ref{fig:app_loss_gnn}. These results highlight the significant impact of the loss function and decoder on the performance of directed link prediction methods. For embedding methods, even when using the same embeddings, different decoders can lead to substantial variations in performance. Notably, on the Slashdot dataset, ELTRA and ODIN are particularly sensitive to decoder settings. Furthermore, the results from GNNs on Cora-ML and Photo demonstrate that BCE loss provides a clear advantage. These findings emphasize the importance of carefully selecting loss functions and decoder configurations in link prediction tasks.


\begin{figure}[h]
\centering
   \vspace{-2mm}
   \hspace{-2mm}
   \subfigure{
   \includegraphics[width=37mm]{fig_degree/degree_WikiCS.pdf}
   }
   \hspace{-7mm}
   \subfigure{
   \includegraphics[width=37mm]{fig_degree/degree_WikiCS-DirGNN_F.pdf}
   }
   \hspace{-7mm}
   \subfigure{
   \includegraphics[width=37mm]{fig_degree/degree_WikiCS-MagNet_F.pdf}
   }
   \hspace{-7mm}
   \subfigure{
   \includegraphics[width=37mm]{fig_degree/degree_WikiCS-DiGAE_F.pdf}
   }
   \hspace{-7mm}
   \subfigure{
   \includegraphics[width=37mm]{fig_degree/degree_WikiCS-SDGAE_F.pdf}
   }
   \vspace{-5mm}
   \centering
   \caption{Degree distribution of WikiCS graph and its reconstruction graph generated by GNNs using the original feature as inputs.}
   %\vspace{-2mm}
   \label{fig:app_degree_feature}
\end{figure}

\begin{figure}[t]
    \centering
   \hspace{-3mm}
   \subfigure{
   \includegraphics[width=37mm]{fig_degree/degree_WikiCS-ELTRA.pdf}
   }
   \hspace{-7mm}
   \subfigure{
   \includegraphics[width=37mm]{fig_degree/degree_WikiCS-DirGNN_D.pdf}
   }
   \hspace{-7mm}
   \subfigure{
   \includegraphics[width=37mm]{fig_degree/degree_WikiCS-MagNet_D.pdf}
   }
   \hspace{-7mm}
   \subfigure{
   \includegraphics[width=37mm]{fig_degree/degree_WikiCS-DiGAE_D.pdf}
   }
   \hspace{-7mm}
   \subfigure{
   \includegraphics[width=37mm]{fig_degree/degree_WikiCS-SDGAE_D.pdf}
   }
   \vspace{-3mm}
   \caption{Degree distribution of WikiCS's reconstruction graph generated by ELTRA and GNNs using the in/out degrees as inputs.}
    \label{fig:app_degree_degree}
\end{figure}


\subsubsection{Degree distribution}\label{app_degree}
We present the degree distribution of the WikiCS graph alongside its reconstructed versions generated by ELTRA and GNNs in Figures~\ref{fig:app_degree_feature} and~\ref{fig:app_degree_degree}. The GNNs use either the original features or the in/out degrees as inputs. First, we observe that ELTRA does not effectively preserve the degree distribution of WikiCS. This may be because ELTRA is primarily designed for directed graphs with many directed edges~\cite{eltra}, whereas WikiCS contains many undirected edges, which likely contributes to its poor performance on this dataset. Second, when GNNs use in/out degrees as inputs, they better preserve the degree distribution, which explains their improved performance on WikiCS for the link prediction task. This finding further underscores the importance of degree distribution in link prediction. Finally, comparing DiGAE and SDGAE, we find that SDGAE preserves the degree distribution more effectively, demonstrating the advantages of our proposed SDGAE.

\begin{table}[h]
\centering
\caption{The impact of normalization of adjacency matrix on DiGAE and SDGAE.}
\begin{tabular}{lcccc}
\toprule
Methods &Cora-ML	&CiteSeer	&Photo	&Computers  \\ \midrule
DiGAE ($\hat{\mD}_{\rm out}^{-\beta}\hat{\mA}\hat{\mD}_{\rm in}^{-\alpha}$) &82.06$\pm$2.51 	&83.64$\pm$3.21	&55.05$\pm$2.36	&41.55$\pm$1.62  \\
DiGAE ($\hat{\mD}_{\rm out}^{-1/2}\hat{\mA}\hat{\mD}_{\rm in}^{-1/2}$)&70.89$\pm$3.59	&71.60$\pm$6.21	&38.75$\pm$5.20	&32.73$\pm$5.28   \\
SDGAE ($\hat{\mD}_{\rm out}^{-1/2}\hat{\mA}\hat{\mD}_{\rm in}^{-1/2}$) &90.37$\pm$1.33	&93.69$\pm$3.68 	&68.84$\pm$2.35 	&53.79$\pm$1.56 \\
\bottomrule
\end{tabular}
\label{app_norm}
\end{table}


\subsubsection{SDGAE versus DiGAE}\label{app_sdgae_digae}
In Table~\ref{app_norm}, we compare the impact of different normalizations of the adjacency matrix on DiGAE and SDGAE. Here, $\hat{\mD}_{\rm out}^{-\beta}\hat{\mA}\hat{\mD}_{\rm in}^{-\alpha}$ represents the original settings in the DiGAE paper~\cite{digae}. We searched for the optimal parameters within the given range, $(\alpha,\beta) \in \{0.0, 0.2, 0.4, 0.6, 0.8\}^2$, noting that this does not include $\hat{\mD}_{\rm out}^{-1/2}\hat{\mA}\hat{\mD}_{\rm in}^{-1/2}$. As shown in Table~\ref{app_norm}, even when DiGAE uses $\hat{\mD}_{\rm out}^{-1/2}\hat{\mA}\hat{\mD}_{\rm in}^{-1/2}$, its performance did not improve. This indicates that SDGAE’s performance gains are not solely due to changes in the normalization of the adjacency matrix.

\begin{table}[h]
    \centering
    \caption{Comparison of negative sampling strategy on Cora-ML and CiteSeer datasets.}
    \resizebox{\textwidth}{!}{
    \begin{tabular}{llccccccc}
        \toprule
        Dataset & Sample & MLP & GCN & DiGCNIB & DirGNN & MagNet & DiGAE & SDGAE \\
        \midrule
        \multirow{2}{*}{Cora-ML} &each run& 60.61$\pm$6.64  & 70.15$\pm$3.01  & 80.57$\pm$3.21  & 76.13$\pm$2.85  & 56.54$\pm$2.95  & 82.06$\pm$2.51  & 90.37$\pm$1.33  \\
        & each epoch & 34.15$\pm$3.54  & 59.86$\pm$9.89  & 56.74$\pm$4.08  & 49.89$\pm$3.59  & 54.79$\pm$2.98  & 79.76$\pm$3.28  & 89.71$\pm$2.36  \\
        \midrule
        \multirow{2}{*}{CiteSeer} & each run  & 70.27$\pm$3.40  & 80.36$\pm$3.07  & 85.32$\pm$3.70  & 76.83$\pm$4.24  & 65.32$\pm$3.26  & 83.64$\pm$3.21  & 93.69$\pm$3.68  \\
        &  each epoch & 66.92$\pm$6.50  & 69.48$\pm$6.60  & 61.69$\pm$6.87  & 53.8$\pm$12.41  & 70.56$\pm$2.06  & 87.32$\pm$3.79  & 92.12$\pm$3.96  \\
        \bottomrule
    \end{tabular}}
    
    \label{app_sample}
\end{table}

\subsubsection{Metric and negative sampling strategy}\label{app_metrci_sampling}
We report the DirLinkBench results under the Accuracy metric in Table~\ref{tab:accuracy_performance} and compute the average rank and average score for each baseline. From these results, we observe that simple undirected graph GNNs can achieve competitive performance, with only a small gap between different methods. Notably, some methods that perform well under the Hits@100 metric (e.g., STRAP, DiGAE) perform relatively poorly in Accuracy. Additionally, in Appendix~\ref{app_complete_res}, we provide the complete DirLinkBench results for each metric across the seven datasets, including AUC, AP, and others. Analyzing these results, we find that AUC and AP exhibit minimal differentiation across baselines, making it difficult to accurately assess link prediction effectiveness. This aligns with ongoing discussions regarding the limitations of these metrics for evaluating link prediction tasks on undirected graphs~\cite{yang2015evaluating,li2023evaluating}. Therefore, we argue that ranking-based metrics, such as Hits@K and MRR, are more suitable for link prediction tasks.

We compare the impact of different negative sampling strategies on model performance during training in Table~\ref{app_sample}, presenting results for seven GNNs on Cora-ML and CiteSeer. Here, ``each run" refers to the default setting in DirLinkBench, where a random negative sample is generated for each run and shared across all models. In contrast, ``each epoch" represents a strategy where different models randomly sample negative edges in each training epoch. The positive sample splitting and test set remain consistent across comparisons. These results indicate that modifying the negative sampling strategy during training significantly affects model performance, particularly for single real-valued GNNs, where test performance drops substantially. These findings suggest that negative sampling strategies during training deserve further research. For instance, heuristic-based methods have been proposed as alternatives to random sampling in undirected graphs~\cite{li2023evaluating}, which could be explored in future work.


\begin{table*}[t]
    \centering
    \caption{Results of various methods under the \textbf{Accuracy} metric (mean ± standard error\%). Results ranked \hig{1}{first}, \hig{2}{second}, and \hig{3}{third} are highlighted. TO indicates methods that did not finish running within 24 hours, and OOM indicates methods that exceeded memory limits.}
    \label{tab:accuracy_performance}
    \resizebox{\textwidth}{!}{
    \begin{tabular}{
        lcccccccrr 
    }
        \toprule
        {Method} & {Cora-ML} & {CiteSeer} & {Photo} & {Computers} & {WikiCS} & {Slashdot} & {Epinions} & {\makecell{Avg.\\ Rank $\downarrow$}} & {\makecell{Avg.\\ Score $\uparrow$}} \\
        \midrule
        STRAP & 78.47$\pm$0.77 & 75.01$\pm$1.40 & 94.61$\pm$0.10 & 93.43$\pm$0.07 & \hig{1}{94.57$\pm$0.06} & 88.19$\pm$0.08 & 91.89$\pm$0.06 & 8.71 & 88.02 \\
        ODIN & 77.16$\pm$0.84 & 73.63$\pm$1.19 & 80.60$\pm$0.20 & 81.73$\pm$0.10 & 86.39$\pm$0.08 & 90.39$\pm$0.10 & 92.42$\pm$0.07 & 11.86 & 83.19 \\
        ELTRA & 85.40$\pm$0.45 & 80.86$\pm$1.15 & 91.84$\pm$0.12 & 89.30$\pm$0.15 & 85.73$\pm$0.28 & 88.11$\pm$0.11 & 90.40$\pm$0.13 & 9.57 & 87.38 \\ \midrule
        
        MLP & 81.32$\pm$2.41 & 74.81$\pm$1.32 & 88.31$\pm$0.50 & 85.33$\pm$0.57 & 84.65$\pm$0.22 & 89.62$\pm$0.11 & 92.56$\pm$0.45 & 11.14 & 85.23 \\ \midrule
        
        GCN & 82.82$\pm$0.86 & 79.47$\pm$1.73 & \hig{3}{95.90$\pm$0.18} & \hig{2}{95.69$\pm$0.13} & 89.85$\pm$0.35 & 89.75$\pm$0.17 & \hig{3}{94.17$\pm$0.08} & 5.50 & 89.66 \\
        GAT & \hig{3}{88.95$\pm$0.71} & 85.94$\pm$1.21 & \hig{1}{96.17$\pm$0.20} & \hig{1}{95.78$\pm$0.26} & 88.90$\pm$0.36 & 88.01$\pm$0.49 & 92.92$\pm$0.37 & \hig{2}{4.64} & \hig{2}{90.95} \\
        APPNP & \hig{2}{89.90$\pm$0.73} & \hig{3}{86.23$\pm$1.49} & 94.71$\pm$0.29 & 94.01$\pm$0.16 & 87.51$\pm$0.14 & 90.07$\pm$0.10 & \hig{2}{94.17$\pm$0.16} & 5.07 & \hig{3}{90.94} \\ \midrule
        
        DGCN & 80.26$\pm$0.69 & 74.83$\pm$1.46 & 95.27$\pm$0.94 & 95.22$\pm$0.20 & 90.13$\pm$0.37 & TO & TO & 9.43 & 87.14 \\
        DiGCN & 82.47$\pm$0.97 & 80.38$\pm$1.17 & 93.86$\pm$0.23 & 93.44$\pm$0.54 & 88.89$\pm$0.14 & TO & TO & 9.86 & 87.81 \\
        DiGCNIB & 87.25$\pm$0.58 & 85.98$\pm$1.19 & 95.05$\pm$0.27 & 94.61$\pm$0.31 & 90.72$\pm$0.10 & TO & TO & 7.14 & 90.72 \\
        DirGNN & 85.83$\pm$0.93 & 80.34$\pm$1.52 & 95.34$\pm$0.17 & 94.68$\pm$0.14 & \hig{3}{91.55$\pm$0.25} & \hig{2}{90.65$\pm$0.13} & 93.99$\pm$0.05 & \hig{3}{4.86} & 90.34 \\ \midrule
        
        MagNet & 77.51$\pm$0.92 & 73.73$\pm$1.30 & 80.16$\pm$0.23 & 81.43$\pm$0.08 & 86.59$\pm$0.06 & \hig{3}{90.45$\pm$0.09} & 92.92$\pm$0.05 & 11.21 & 83.26 \\
        DUPLEX & 82.28$\pm$0.93 & 79.20$\pm$1.79 & 87.68$\pm$0.52 & 85.65$\pm$0.27 & 83.82$\pm$0.23 & 85.42$\pm$3.42 & 88.20$\pm$3.86 & 12.14 & 84.61 \\ \midrule
        
        DHYPR & 86.04$\pm$0.66 & \hig{2}{87.33$\pm$1.39} & 76.94$\pm$0.63 & TO & TO & OOM/TO & OOM/TO & 11.71 & 83.44 \\
        DiGAE & 86.25$\pm$0.80 & 81.85$\pm$1.31 & 91.77$\pm$0.18 & 89.52$\pm$0.15 & 82.53$\pm$0.48 & 85.67$\pm$0.28 & 90.17$\pm$0.14 & 9.86 & 86.82 \\ \midrule
        
        SDGAE & \hig{1}{91.36$\pm$0.70} & \hig{1}{91.38$\pm$0.79} & \hig{2}{96.16$\pm$0.14} & \hig{3}{95.66$\pm$0.12} & \hig{2}{92.24$\pm$0.12} & \hig{1}{91.05$\pm$0.20} & \hig{1}{94.33$\pm$0.11} & \hig{1}{1.57} & \hig{1}{93.17} \\
        \bottomrule
    \end{tabular}}
\end{table*}



%\subsection{Link prediction on directed graph problem}


%\textbf{Loss Function} Binary Cross-Entropy Loss (BCE-Loss) and Cross-Entropy Loss (CE-Loss).






%\begin{table*}[th]
%\centering
%\caption{The results on the DUPLEX's setting.}
%\resizebox{\textwidth}{!}{
%\begin{tabular}{@{}llllllll@{}}
%\toprule
%\multicolumn{1}{l}{Dataset}              & Method & EP(ACC) & EP(AUC) & DP(ACC)  & DP(AUC) & 3C(ACC) & 4C(ACC) \\ \midrule
%\multirow{3}{*}{CiteSeer} & DUPLEX (original) &95.7±0.5 &98.6±0.4 &98.7±0.4 &99.7±0.2 &94.8±0.2 &91.1±1.0 \\ %\cmidrule(l){2-7} 
%& DUPLEX (reproduced) &92.11±0.78 &95.85±0.87 &97.54±0.54 &98.93±0.59  &88.22±1.06 &84.77±1.01           \\
%                                      & MLP             \\ \midrule
%\multirow{3}{*}{Cora} & DUPLEX (original) &93.2±0.1 &95.9±0.1 &95.9±0.1 &97.9±0.2 &92.2±0.1 &88.4±0.4    \\
%& DUPLEX (reproduced) &93.49±0.21 &95.61±0.20 &95.25±0.16 &96.34±0.23  &92.41±0.21 &89.76±0.25     \\
%                                      & MLP             \\ \bottomrule
%\end{tabular}}
%\end{table*}

%\begin{figure}[ht]
%    \centering
%   \vspace{-1mm}
%   %\hspace{-3mm}  
%   \includegraphics[width=80mm]{figure/feature_cora.pdf}
%   \caption{Cora-ml.}
%   \vspace{-1mm}
    %\label{fig:duplex_cls_citeseer}
% \end{figure}



















\clearpage
\hypertarget{app_complete_res}{}
\section{Complete Results of DirLinkBench on Seven Datasets}\label{app_complete_res}

\hypertarget{app_complete_res_cora}{}
\subsection{Results on Cora-ML}\label{app_complete_res_cora}
%Table~\ref{app_tb_cora_ml} shows the results of DirLinkBench on Cora-ML.
\begin{table}[h]
    \centering
    \caption{Benchmark Results on Cora-ML. For all methods, $_{\text{F}}$ indicates the use of original node features as input, while $_{\text{D}}$ indicates the use of in/out degrees as input. Results ranked \hig{1}{first} and \hig{2}{second} are highlighted.}
    \label{app_tb_cora_ml}
    \resizebox{\textwidth}{!}{
    \begin{tabular}{lccccccc}
        \toprule
        Method & Hits@20 & Hits@50 & Hits@100 & MRR & AUC & AP & ACC \\
        \midrule
        STRAP & 67.10$\pm$2.55 & 75.05$\pm$1.66 & 79.09$\pm$1.57 & \hig{1}{30.91$\pm$7.48} & 87.38$\pm$0.83 & 90.46$\pm$0.70 & 78.47$\pm$0.77 \\
        
        ODIN & 27.10$\pm$3.69 & 40.99$\pm$3.10 & 54.85$\pm$2.53 & 10.75$\pm$3.51 & 85.15$\pm$0.84 & 85.41$\pm$0.95 & 77.16$\pm$0.84  \\
        
        ELTRA &\hig{2}{70.74$\pm$5.47} & \hig{2}{81.93$\pm$2.38} & \hig{2}{87.45$\pm$1.48} & 19.77$\pm$5.22 & 94.83$\pm$0.58 & 95.37$\pm$0.52 & 85.40$\pm$0.45  \\ \midrule
        
        MLP$_{\text{F}}$ & 27.16$\pm$6.70 & 44.28$\pm$5.72 & 60.61$\pm$6.64 & 9.02$\pm$3.58 & 89.93$\pm$2.09 & 88.55$\pm$2.51 & 81.32$\pm$2.41 \\ 
        MLP$_{\text{D}}$ & 29.84$\pm$4.83 & 44.11$\pm$4.73 & 56.74$\pm$2.03 & 11.84$\pm$3.40 & 86.44$\pm$0.74 & 86.58$\pm$0.88 & 77.95$\pm$0.95 \\ \midrule
        
        GCN$_{\text{F}}$ & 37.88$\pm$5.60 & 55.31$\pm$4.46 & 70.15$\pm$3.01 & 14.25$\pm$3.49 & 92.01$\pm$1.64 & 91.45$\pm$1.47 & 82.82$\pm$0.86 \\
        GCN$_{\text{D}}$& 31.36$\pm$3.89 & 45.03$\pm$2.92 & 58.77$\pm$2.96 & 12.56$\pm$2.86 & 86.12$\pm$0.87 & 86.79$\pm$1.01 & 77.81$\pm$0.87   \\ \midrule
        
        GAT$_{\text{F}}$ & 33.42$\pm$8.93 & 58.40$\pm$6.55 & 79.72$\pm$3.07 & 9.30$\pm$3.39 & 94.57$\pm$0.45 & 92.77$\pm$0.42 & 88.95$\pm$0.71  \\
        GAT$_{\text{D}}$ & 30.19$\pm$4.23 & 43.53$\pm$2.93 & 55.09$\pm$3.34 & 11.07$\pm$3.68 & 85.52$\pm$1.27 & 86.19$\pm$0.96 & 77.36$\pm$1.13 \\ \midrule
        
        APPNP$_{\text{F}}$ & 55.47$\pm$4.63 & 75.16$\pm$4.09 & 86.02$\pm$2.88 & 22.49$\pm$6.33 & 95.94$\pm$0.57 & \hig{2}{95.82$\pm$0.59} &\hig{2}{89.90$\pm$0.73} \\ 
        APPNP$_{\text{D}}$ & 32.21$\pm$3.65 & 45.51$\pm$3.76 & 60.02$\pm$3.36 & 11.89$\pm$2.65 & 88.00$\pm$1.08 & 88.01$\pm$1.17 & 79.28$\pm$1.13 \\\midrule
        
        DGCN$_{\text{F}}$ & 30.86$\pm$4.17 & 41.82$\pm$4.43 & 54.38$\pm$4.97 & 11.01$\pm$4.23 & 85.55$\pm$3.40 & 85.66$\pm$2.35 & 77.82$\pm$3.49  \\
        DGCN$_\text{D}$ & 36.30$\pm$4.43 & 50.25$\pm$3.42 & 63.32$\pm$2.59 & 13.98$\pm$4.24 & 87.87$\pm$0.96 & 88.67$\pm$0.84 & 80.26$\pm$0.69 \\ \midrule
        
        DiGCN$_{\text{F}}$ & 30.05$\pm$5.45 & 48.87$\pm$5.25 & 63.21$\pm$5.72 & 10.16$\pm$4.25 & 89.48$\pm$1.83 & 89.10$\pm$1.99 & 82.47$\pm$0.97 \\
        DiGCN$_{\text{D}}$ & 34.39$\pm$3.69 & 49.66$\pm$4.14 & 60.10$\pm$4.04 & 12.49$\pm$3.83 & 85.87$\pm$1.22 & 87.13$\pm$1.18 & 77.88$\pm$1.01 \\ \midrule
        
        DiGCNIB$_{\text{F}}$& 45.90$\pm$4.97 & 66.62$\pm$3.67 & 80.57$\pm$3.21 & 17.08$\pm$3.90 & 94.67$\pm$0.56 & 94.21$\pm$0.65 & 87.25$\pm$0.58 \\

        DiGCNIB$_{\text{D}}$ & 37.18$\pm$4.60 & 51.64$\pm$3.69 & 64.72$\pm$3.43 & 12.71$\pm$4.08 & 89.23$\pm$0.92 & 89.36$\pm$0.85 & 81.05$\pm$1.05 \\ \midrule
        
        DirGNN$_{\text{F}}$ & 42.48$\pm$7.34 & 59.41$\pm$4.00 & 76.13$\pm$2.85 & 12.01$\pm$3.57 & 93.05$\pm$0.87 & 92.52$\pm$0.79 & 85.83$\pm$0.93 \\
        DirGNN$_{\text{D}}$ & 32.29$\pm$2.25 & 45.53$\pm$2.60 & 58.22$\pm$2.50 & 13.34$\pm$5.56 &87.06$\pm$ 0.77 &87.35$\pm$0.80 & 77.93$\pm$1.14 \\ \midrule
        
        MagNet$_{\text{F}}$ & 26.87$\pm$3.82 & 40.58$\pm$3.19 & 53.51$\pm$2.00 & 10.08$\pm$3.47 & 85.21$\pm$0.79 & 85.53$\pm$1.00 & 76.93$\pm$0.79 \\
        MagNet$_{\text{D}}$ & 29.38$\pm$3.39 & 43.28$\pm$3.76 & 56.54$\pm$2.95 & 11.19$\pm$3.04 & 85.23$\pm$0.84 & 86.06$\pm$0.94 & 77.51$\pm$0.92 \\ \midrule

        DUPLEX$_{\text{F}}$ & 21.73$\pm$3.19 & 34.98$\pm$2.37 & 69.00$\pm$2.52 & 7.74$\pm$1.95 & 88.02$\pm$0.95 & 86.62$\pm$1.43 & 82.28$\pm$0.93 \\
        DUPLEX$_{\text{D}}$& 17.48$\pm$4.69 & 32.58$\pm$8.33 & 54.50$\pm$7.49 & 5.46$\pm$2.10 & 82.97$\pm$4.47 & 82.97$\pm$2.85 & 75.59$\pm$4.87 \\ \midrule

        DHYPR$_{\text{F}}$ & 59.81$\pm$4.79 & 77.45$\pm$2.56 & 86.81$\pm$1.60 & 20.56$\pm$5.10 & \hig{2}{96.13$\pm$0.28} & 95.84$\pm$0.39 & 86.04$\pm$0.66  \\
        DHYPR$_{\text{D}}$ & 15.09$\pm$4.40 & 28.96$\pm$6.54 & 42.93$\pm$6.63 & 4.05$\pm$1.16 & 80.83$\pm$2.64 & 79.97$\pm$3.02 & 72.74$\pm$3.50 \\ \midrule
        
        DiGAE$_{\text{F}}$& 56.13$\pm$3.80 & 72.23$\pm$2.51 & 82.06$\pm$2.51 & 20.53$\pm$4.21 & 92.56$\pm$0.66 & 93.70$\pm$0.54 & 86.25$\pm$0.80  \\
        DiGAE$_{\text{D}}$ & 35.40$\pm$4.05 & 49.12$\pm$3.44 & 61.30$\pm$2.70 & 13.71$\pm$4.22 & 86.82$\pm$0.85 & 87.35$\pm$0.84 & 76.01$\pm$1.04 \\ \midrule

        %SDGAE$_{\text{F}}$ &\textbf{70.89$\pm$3.35} & \textbf{83.63$\pm$2.15} & \textbf{90.37$\pm$1.33} & \underline{28.45$\pm$5.82} & \textbf{97.24$\pm$0.34} & \textbf{97.21$\pm$0.17} & \textbf{91.36$\pm$0.70}  \\ 
        SDGAE$_{\text{F}}$ &\hig{1}{70.89$\pm$3.35} & \hig{1}{83.63$\pm$2.15} &\hig{1}{90.37$\pm$1.33} & \hig{2}{28.45$\pm$5.82} & \hig{1}{97.24$\pm$0.34} & \hig{1}{97.21$\pm$0.17} &\hig{1}{91.36$\pm$0.70}  \\

        
        SDGAE$_{\text{D}}$& 35.53$\pm$5.41 & 49.34$\pm$4.60 & 61.40$\pm$3.18 & 14.21$\pm$3.71 & 87.38$\pm$1.06 & 88.19$\pm$1.18 & 78.99$\pm$1.34  \\

        \bottomrule
    \end{tabular}}
\end{table}


\clearpage
\hypertarget{app_complete_res_citeseer}{}
\subsection{Results on CiteSeer}\label{app_complete_res_citeseer}
%Table~\ref{app_tb_citeseer} shows the results of DirLinkBench on CiteSeer.
\begin{table}[h]
    \centering
    \caption{Benchmark Results on CiteSeer. For all methods, $_{\text{F}}$ indicates the use of original node features as input, while $_{\text{D}}$ indicates the use of in/out degrees as input. Results ranked \hig{1}{first} and \hig{2}{second} are highlighted.}
    \label{app_tb_citeseer}
    \resizebox{\textwidth}{!}{
    \begin{tabular}{lccccccc}
        \toprule
        Model & Hits@20 & Hits@50 & Hits@100 & MRR & AUC & AP & ACC \\
        \midrule
        STRAP & 63.28$\pm$1.46 & 67.05$\pm$1.32 & 69.32$\pm$1.29 & \hig{2}{40.70$\pm$6.95} & 81.82$\pm$1.44 & 84.15$\pm$0.84 & 75.01$\pm$1.40 \\
        
        ODIN & 30.74$\pm$3.78 & 47.91$\pm$2.57 & 63.95$\pm$2.98 & 11.05$\pm$2.74 & 82.22$\pm$1.05 & 81.74$\pm$1.10 & 73.63$\pm$1.19  \\
        
        ELTRA & 72.34$\pm$1.76 & 79.44$\pm$1.54 & 84.97$\pm$1.90 & 27.87$\pm$9.03 & 90.98$\pm$0.86 & 92.71$\pm$0.81 & 80.86$\pm$1.15  \\ \midrule
        
        MLP$_{\text{F}}$& 35.55$\pm$4.39 & 52.31$\pm$4.60 & 70.27$\pm$3.40 & 16.08$\pm$4.88 & 85.47$\pm$1.35 & 85.16$\pm$1.29 & 74.26$\pm$1.64  \\
        MLP$_{\text{D}}$ & 40.27$\pm$4.48 & 56.27$\pm$3.64 & 68.78$\pm$2.21 & 15.94$\pm$3.04 & 84.09$\pm$1.02 & 84.86$\pm$1.12 & 74.81$\pm$1.32  \\ \midrule
        
        GCN$_{\text{F}}$ & 42.27$\pm$7.11 & 62.65$\pm$4.22 & 80.36$\pm$3.07 & 18.28$\pm$6.7 & 89.94$\pm$1.04 & 88.88$\pm$1.38 & 79.47$\pm$1.73 \\
        GCN$_{\text{D}}$ & 43.96$\pm$2.35 & 56.24$\pm$3.62 & 66.15$\pm$2.07 & 19.36$\pm$4.96 & 81.24$\pm$1.55 & 84.33$\pm$0.90 & 73.92$\pm$1.43 \\ \midrule

        
        GAT$_{\text{F}}$ & 48.29$\pm$6.04 & 71.86$\pm$4.65 & 85.88$\pm$4.98 & 16.71$\pm$5.56 & 92.90$\pm$0.74 & 91.12$\pm$1.53 & 85.94$\pm$1.21 \\
        GAT$_{\text{D}}$ & 40.29$\pm$3.25 & 55.62$\pm$1.92 & 65.82$\pm$2.87 & 19.00$\pm$4.18 & 82.72$\pm$0.67 & 83.96$\pm$0.60 & 74.18$\pm$1.18 \\ \midrule
       

        APPNP$_{\text{F}}$& 59.86$\pm$6.54 & 77.57$\pm$5.68 & 83.57$\pm$4.90 & 20.91$\pm$4.39 & 93.58$\pm$0.71 & 93.09$\pm$0.68 & 86.23$\pm$1.49  \\
        APPNP$_{\text{D}}$ & 45.44$\pm$3.38 & 58.18$\pm$1.55 & 68.47$\pm$3.00 & 20.18$\pm$5.50 & 83.48$\pm$1.17 & 85.72$\pm$1.00 & 74.86$\pm$1.19 \\ \midrule

        
        DGCN$_{\text{F}}$ & 39.64$\pm$6.53 & 50.85$\pm$2.91 & 62.97$\pm$4.79 & 19.15$\pm$4.34 & 80.60$\pm$2.90 & 81.80$\pm$1.85 & 74.12$\pm$2.49  \\
        DGCN$_{\text{D}}$ & 45.94$\pm$3.01 & 58.38$\pm$2.85 & 68.97$\pm$3.39 & 19.53$\pm$4.83 & 83.38$\pm$1.41 & 85.44$\pm$1.01 & 74.83$\pm$1.46  \\ \midrule

        DiGCN$_{\text{F}}$ & 38.48$\pm$3.50 & 53.30$\pm$5.92 & 70.95$\pm$4.67 & 15.79$\pm$4.05 & 87.41$\pm$2.24 & 85.02$\pm$2.90 & 80.38$\pm$1.17 \\
        DiGCN$_{\text{D}}$ & 40.65$\pm$4.23 & 52.87$\pm$3.85 & 65.46$\pm$3.70 & 17.39$\pm$4.82 & 83.50$\pm$1.12 & 84.75$\pm$0.93 & 74.65$\pm$1.36  \\ \midrule

        
        DiGCNIB$_{\text{F}}$& 44.90$\pm$7.91 & 69.86$\pm$5.13 & 85.32$\pm$3.70 & 16.61$\pm$5.61 & 92.50$\pm$0.57 & 90.35$\pm$1.43 & 85.98$\pm$1.19 \\
        DiGCNIB$_{\text{D}}$ & 45.68$\pm$3.50 & 58.99$\pm$3.92 & 70.29$\pm$2.39 & 20.22$\pm$3.31 & 85.81$\pm$0.69 & 86.73$\pm$0.85 & 76.54$\pm$1.27 \\ \midrule

        DirGNN$_{\text{F}}$ & 44.95$\pm$7.31 & 64.00$\pm$3.19 & 76.83$\pm$4.24 & 16.19$\pm$4.51 & 88.44$\pm$1.28 & 88.59$\pm$1.14 & 80.34$\pm$1.52 \\
        DirGNN$_{\text{D}}$ & 40.16$\pm$4.84 & 53.44$\pm$3.38 & 65.62$\pm$2.46 & 16.95$\pm$5.45 & 83.67$\pm$1.08 & 84.59$\pm$1.18 & 73.75$\pm$1.31 \\ \midrule
        
        MagNet$_{\text{F}}$ & 32.18$\pm$5.04 & 47.44$\pm$3.24 & 61.55$\pm$8.45 & 12.97$\pm$1.73 & 82.58$\pm$1.31 & 82.70$\pm$1.45 & 73.73$\pm$1.30 \\
        MagNet$_{\text{D}}$ & 39.35$\pm$5.13 & 54.50$\pm$3.21 & 65.32$\pm$3.26 & 15.31$\pm$2.45 & 83.05$\pm$1.35 & 84.14$\pm$1.31 & 73.28$\pm$1.21  \\ \midrule

        DUPLEX$_{\text{F}}$ & 30.16$\pm$5.06 & 50.90$\pm$10.69 & 73.39$\pm$3.42 & 17.82$\pm$4.47 & 84.63$\pm$2.20 & 83.01$\pm$2.10 & 79.20$\pm$1.79 \\ 
        DUPLEX$_{\text{D}}$ &27.98$\pm$7.61 & 50.20$\pm$7.36 & 63.62$\pm$3.90 & 10.95$\pm$6.35 & 80.68$\pm$1.15 & 81.17$\pm$1.20 & 73.75$\pm$1.44 \\ \midrule


        DHYPR$_{\text{F}}$ &\hig{2}{77.77$\pm$4.10} & \hig{2}{89.35$\pm$2.59} & \hig{2}{92.32$\pm$3.72} & 32.39$\pm$10.14 & \hig{2}{96.61$\pm$0.28} & \hig{2}{96.35$\pm$0.50} & \hig{2}{87.33$\pm$1.39}  \\
        DHYPR$_{\text{D}}$ & 24.32$\pm$8.59 & 41.51$\pm$6.55 & 54.27$\pm$5.94 & 6.17$\pm$2.12 & 77.38$\pm$1.79 & 77.35$\pm$2.44 & 70.27$\pm$1.35 \\  \midrule
        
        DiGAE$_{\text{F}}$ & 56.12$\pm$3.08 & 71.06$\pm$2.49 & 83.64$\pm$3.21 & 22.88$\pm$6.79 & 91.03$\pm$1.14 & 91.29$\pm$1.04 & 81.85$\pm$1.31  \\
        DiGAE$_{\text{D}}$ &  44.32$\pm$1.85 & 59.91$\pm$2.08 & 69.66$\pm$2.11 & 20.89$\pm$4.81 & 85.28$\pm$0.96 & 86.28$\pm$0.96 & 74.54$\pm$0.91 \\ \midrule

        SDGAE$_{\text{F}}$ &\hig{1}{81.06$\pm$4.50} & \hig{1}{91.24$\pm$2.55} &\hig{1}{93.69$\pm$3.68} &\hig{1}{42.50$\pm$9.76} & \hig{1}{97.24$\pm$0.60} & \hig{1}{97.13$\pm$0.68} & \hig{1}{91.38$\pm$0.79} \\ 
        SDGAE$_{\text{D}}$ & 45.03$\pm$3.19 & 57.82$\pm$2.00 & 68.97$\pm$1.70 & 21.41$\pm$2.85 & 85.10$\pm$0.98 & 86.24$\pm$0.47 & 74.83$\pm$0.86 \\
        

        
        \bottomrule
    \end{tabular}}
\end{table}


\clearpage
\hypertarget{app_complete_res_photo}{}
\subsection{Results on Photo}\label{app_complete_res_photo}
%Table~\ref{app_tb_photo} shows the results of DirLinkBench on Photo.
\begin{table}[h]
    \centering
    \caption{Benchmark Results on Photo. For all methods, $_{\text{F}}$ indicates the use of original node features as input, while $_{\text{D}}$ indicates the use of in/out degrees as input. Results ranked \hig{1}{first} and \hig{2}{second} are highlighted.}
    \label{app_tb_photo}
    \resizebox{\textwidth}{!}{
    \begin{tabular}{lccccccc}
        \toprule
        Model & Hits@20 & Hits@50 & Hits@100 & MRR & AUC & AP & ACC \\
        \midrule
        STRAP & \hig{2}{38.54$\pm$5.20} & \hig{2}{55.21$\pm$2.19} & \hig{1}{69.16$\pm$1.44} & 12.08$\pm$3.15 & 98.54$\pm$0.04 & 98.65$\pm$0.05 & 94.61$\pm$0.10 \\
        
        ODIN & 5.28$\pm$1.36 & 9.38$\pm$1.54 & 14.13$\pm$1.92 & 1.78$\pm$0.66 & 88.15$\pm$0.21 & 87.34$\pm$0.30 & 80.60$\pm$0.20  \\
        
        ELTRA & 7.09$\pm$1.23 & 12.86$\pm$1.58 & 20.63$\pm$1.93 & 2.22$\pm$0.59 & 96.89$\pm$0.06 & 95.84$\pm$0.13 & 91.84$\pm$0.12 \\ \midrule
        
        MLP$_{\text{F}}$ & 8.83$\pm$2.06 & 14.07$\pm$2.87 & 20.91$\pm$4.18 & 3.18$\pm$1.06 & 95.29$\pm$0.37 & 93.60$\pm$1.06 & 88.31$\pm$0.50  \\
        MLP$_{\text{D}}$ & 7.88$\pm$1.56 & 12.90$\pm$0.81 & 18.04$\pm$1.35 & 2.53$\pm$0.40 & 85.90$\pm$0.44 & 86.44$\pm$0.21 & 77.25$\pm$0.51  \\ \midrule
        
        GCN$_{\text{F}}$ & 29.44$\pm$3.90 & 44.12$\pm$3.49 & 57.55$\pm$2.54 & 9.85$\pm$2.78 & 99.04$\pm$0.06 & 98.66$\pm$0.10 & 95.90$\pm$0.18  \\
        GCN$_{\text{D}}$ & 31.47$\pm$3.71 & 44.93$\pm$3.34 & 58.77$\pm$2.96 & \hig{2}{12.56$\pm$2.86} & 86.12$\pm$0.87 & 86.79$\pm$1.01 & 77.71$\pm$1.01  \\ \midrule
        
        GAT$_{\text{F}}$ & 25.97$\pm$4.22 & 42.85$\pm$4.90 & 58.06$\pm$4.03 & 8.62$\pm$3.07 & \hig{2}{99.13$\pm$0.09} & \hig{2}{98.93$\pm$0.11} & \hig{1}{96.17$\pm$0.20} \\
        GAT$_{\text{D}}$ & 11.96$\pm$1.60 & 21.83$\pm$2.98 & 31.51$\pm$2.16 & 3.61$\pm$0.94 & 94.41$\pm$0.19 & 94.37$\pm$0.19 & 87.00$\pm$0.25 \\ \midrule

        APPNP$_{\text{F}}$ & 22.13$\pm$2.60 & 35.30$\pm$2.04 & 47.51$\pm$2.51 & 6.56$\pm$1.14 & 98.54$\pm$0.09 & 98.26$\pm$0.12 & 94.71$\pm$0.29 \\ 
        APPNP$_{\text{D}}$& 12.92$\pm$1.06 & 19.42$\pm$2.24 & 26.66$\pm$2.14 & 4.63$\pm$1.10 & 93.29$\pm$0.42 & 92.86$\pm$0.40 & 85.36$\pm$0.39  \\ \midrule
        
        DGCN$_{\text{F}}$ & 22.61$\pm$4.58 & 37.09$\pm$5.38 & 51.61$\pm$6.33 & 6.38$\pm$1.49 & 98.74$\pm$0.39 & 98.49$\pm$0.45 & 95.27$\pm$0.94 \\
        DGCN$_{\text{D}}$ & 15.96$\pm$3.35 & 26.88$\pm$2.91 & 35.56$\pm$3.52 & 5.83$\pm$1.16 & 96.26$\pm$0.70 & 95.94$\pm$0.76 & 90.02$\pm$1.00 \\ \midrule

        DiGCN$_{\text{F}}$ & 18.47$\pm$1.89 & 29.63$\pm$2.28 & 40.17$\pm$2.38 & 6.41$\pm$2.41 & 98.10$\pm$0.10 & 97.69$\pm$0.14 & 93.86$\pm$0.23 \\
        DiGCN$_{\text{D}}$ & 17.00$\pm$1.84 & 25.51$\pm$1.39 & 33.73$\pm$1.95 & 5.71$\pm$1.32 & 95.21$\pm$0.19 & 95.09$\pm$0.21 & 88.29$\pm$0.26 \\ \midrule
        
        DiGCNIB$_{\text{F}}$ & 21.42$\pm$2.77 & 34.97$\pm$2.67 & 48.26$\pm$3.98 & 6.82$\pm$1.53 & 98.67$\pm$0.14 & 98.39$\pm$0.16 & 95.05$\pm$0.27 \\
        DiGCNIB$_{\text{D}}$ & 16.00$\pm$1.67 & 24.63$\pm$1.84 & 33.20$\pm$2.17 & 6.11$\pm$1.50 & 96.71$\pm$0.07 & 96.38$\pm$0.09 & 91.30$\pm$0.17 \\ \midrule
        
        DirGNN$_{\text{F}}$& 22.59$\pm$2.77 & 34.65$\pm$3.31 & 49.15$\pm$3.62 & 8.42$\pm$2.90 & 98.76$\pm$0.09 & 98.47$\pm$0.13 & 95.34$\pm$0.17 \\
        DirGNN$_{\text{D}}$ &21.57$\pm$2.11 & 30.87$\pm$2.35 & 43.21$\pm$2.19 & 8.72$\pm$2.08 & 97.53$\pm$0.09 & 97.26$\pm$0.08 & 92.41$\pm$0.20 \\ \midrule
        
        MagNet$_{\text{F}}$ & 5.35$\pm$0.50 & 8.52$\pm$0.85 & 12.66$\pm$1.03 & 1.62$\pm$0.39 & 87.92$\pm$0.22 & 87.19$\pm$0.32 & 80.16$\pm$0.23 \\
        MagNet$_{\text{D}}$ & 5.14$\pm$0.54 & 9.04$\pm$0.52 & 13.89$\pm$0.32 & 1.61$\pm$0.31 & 88.11$\pm$0.21 & 87.48$\pm$0.21 & 80.31$\pm$0.14 \\ \midrule
        
        DUPLEX$_{\text{F}}$ & 7.84$\pm$1.17 & 12.64$\pm$1.22 & 17.94$\pm$0.66 & 2.53$\pm$0.66 & 94.22$\pm$0.76 & 93.37$\pm$0.91 & 87.68$\pm$0.52 \\ 
        DUPLEX$_{\text{D}}$ & 6.97$\pm$2.14 & 9.08$\pm$4.95 & 13.28$\pm$6.14 & 2.26$\pm$1.17 & 87.41$\pm$7.05 & 86.40$\pm$5.89 & 79.73$\pm$6.87 \\
 \midrule

        DHYPR$_{\text{F}}$ & 10.18$\pm$1.21 & 13.66$\pm$1.47 & 20.93$\pm$2.41 & 3.15$\pm$0.57 & 87.35$\pm$1.54 & 88.83$\pm$0.95 & 76.94$\pm$0.63 \\
        
        DHYPR$_{\text{D}}$ & 0.16$\pm$0.14 & 0.52$\pm$0.83 & 1.63$\pm$1.23 & 0.28$\pm$0.06 & 58.85$\pm$1.39 & 56.65$\pm$0.62 & 53.91$\pm$0.66 \\  \midrule
        
        DiGAE$_{\text{F}}$ & 27.79$\pm$3.85 & 43.32$\pm$3.36 & 55.05$\pm$2.36 & 9.38$\pm$2.47 & 97.98$\pm$0.08 & 97.99$\pm$0.10 & 91.77$\pm$0.18 \\
        DiGAE$_{\text{D}}$ & 16.55$\pm$1.20 & 26.17$\pm$2.03 & 34.60$\pm$2.09 & 5.36$\pm$1.66 & 93.14$\pm$0.14 & 93.76$\pm$0.11 & 83.23$\pm$0.34 \\ \midrule

        SDGAE$_{\text{F}}$ &\hig{1}{40.89$\pm$3.86} & \hig{1}{55.76$\pm$4.08} & \hig{2}{68.84$\pm$2.35} & \hig{1}{14.82$\pm$4.22} &\hig{1}{99.25$\pm$0.05} & \hig{1}{99.16$\pm$0.06} & \hig{2}{96.16$\pm$0.14} \\
        SDGAE$_{\text{D}}$ & 24.76$\pm$3.46 & 38.5$\pm$1.66 & 50.96$\pm$2.32 & 9.16$\pm$1.99 & 98.07$\pm$0.15 & 97.98$\pm$0.12 & 93.57$\pm$0.35 \\

        
        \bottomrule
    \end{tabular}}
\end{table}


\clearpage
\hypertarget{app_complete_res_computers}{}
\subsection{Results on Computers}\label{app_complete_res_computers}
%Table~\ref{app_tb_computers} shows the results of DirLinkBench on Computers.
\begin{table}[ht]
    \centering
    \caption{Benchmark Results on Computers. For all methods, $_{\text{F}}$ indicates the use of original node features as input, while $_{\text{D}}$ indicates the use of in/out degrees as input. Results ranked \hig{1}{first} and \hig{2}{second} are highlighted.}
    \label{app_tb_computers}
    \resizebox{\textwidth}{!}{
    \begin{tabular}{lccccccc}
        \toprule
        Model & Hits@20 & Hits@50 & Hits@100 & MRR & AUC & AP & ACC \\
        \midrule
        STRAP & \hig{2}{24.35$\pm$3.75} &\hig{2}{38.14$\pm$3.77} & \hig{2}{51.87$\pm$2.07} & \hig{2}{7.44$\pm$1.98} & 98.19$\pm$0.03 & 98.32$\pm$0.03 & 93.43$\pm$0.07 \\
        
        ODIN & 4.42$\pm$1.18 & 8.41$\pm$1.27 & 12.98$\pm$1.47 & 1.46$\pm$0.63 & 89.35$\pm$0.09 & 89.15$\pm$0.35 & 81.73$\pm$0.10  \\
        
        ELTRA & 4.82$\pm$0.75 & 8.66$\pm$1.38 & 14.74$\pm$1.55 & 1.61$\pm$0.52 & 95.63$\pm$0.10 & 95.11$\pm$0.14 & 89.30$\pm$0.15  \\ \midrule
        
        MLP$_{\text{F}}$ & 5.51$\pm$0.72 & 9.31$\pm$0.90 & 14.26$\pm$0.88 & 1.63$\pm$0.42 & 92.83$\pm$0.36 & 91.86$\pm$0.39 & 85.33$\pm$0.57 \\ 
        MLP$_{\text{D}}$ & 6.97$\pm$1.22 & 12.2$\pm$0.88 & 17.57$\pm$0.85 & 2.76$\pm$0.75 & 87.48$\pm$0.16 & 88.69$\pm$0.15 & 78.75$\pm$0.36  \\ \midrule
        
        GCN$_{\text{F}}$ & 19.32$\pm$2.69 & 31.47$\pm$1.81 & 43.77$\pm$1.75 & 7.26$\pm$1.64 & 98.97$\pm$0.06 & \hig{2}{98.80$\pm$0.06} & \hig{2}{95.69$\pm$0.13} \\
        GCN$_{\text{D}}$ & 10.59$\pm$1.40 & 17.48$\pm$1.07 & 24.59$\pm$1.01 & 4.33$\pm$1.26 & 94.41$\pm$0.10 & 94.25$\pm$0.12 & 87.00$\pm$0.13 \\ \midrule

        
        GAT$_{\text{F}}$ & 14.89$\pm$3.24 & 27.53$\pm$2.48 & 40.74$\pm$3.22 & 3.37$\pm$0.92 &\hig{2}{98.99$\pm$0.11} & 98.76$\pm$0.15 & \hig{1}{95.78$\pm$0.26} \\
        GAT$_{\text{D}}$ & 7.16$\pm$1.53 & 14.49$\pm$1.54 & 21.98$\pm$1.04 & 2.36$\pm$0.69 & 95.20$\pm$0.44 & 94.81$\pm$0.47 & 88.14$\pm$0.67 \\ \midrule

        APPNP$_{\text{F}}$ & 12.40$\pm$1.24 & 21.71$\pm$1.13 & 32.24$\pm$1.40 & 4.86$\pm$1.06 & 98.21$\pm$0.08 & 97.90$\pm$0.11 & 94.01$\pm$0.16 \\ 
        APPNP$_{\text{D}}$ & 9.54$\pm$1.53 & 15.36$\pm$1.20 & 21.92$\pm$1.32 & 3.99$\pm$0.83 & 93.84$\pm$0.35 & 93.49$\pm$0.38 & 85.76$\pm$0.52 \\\midrule

        
        DGCN$_{\text{F}}$ & 17.74$\pm$2.16 & 27.81$\pm$1.84 & 39.92$\pm$1.94 & 6.51$\pm$1.53 & 98.78$\pm$0.10 & 98.59$\pm$0.11 & 95.22$\pm$0.20 \\
        DGCN$_{\text{D}}$ & 11.12$\pm$1.50 & 17.79$\pm$1.66 & 25.20$\pm$1.87 & 4.06$\pm$0.70 & 95.54$\pm$0.50 & 95.30$\pm$0.53 & 88.72$\pm$0.74 \\ \midrule

        
        DiGCN$_{\text{F}}$ & 11.68$\pm$1.84 & 19.02$\pm$2.06 & 27.51$\pm$1.67 & 4.61$\pm$0.71 & 97.87$\pm$0.24 & 97.47$\pm$0.27 & 93.44$\pm$0.54 \\
        DiGCN$_{\text{D}}$ & 10.98$\pm$1.53 & 17.76$\pm$1.37 & 24.46$\pm$1.51 & 4.69$\pm$1.09 & 94.86$\pm$0.11 & 94.75$\pm$0.12 & 87.59$\pm$0.17 \\ \midrule
        
        DiGCNIB$_{\text{F}}$ & 13.41$\pm$2.95 & 22.54$\pm$1.79 & 32.44$\pm$1.85 & 4.86$\pm$0.80 & 98.46$\pm$0.14 & 98.15$\pm$0.17 & 94.61$\pm$0.31 \\
        DiGCNIB$_{\text{D}}$ & 10.70$\pm$0.84 & 16.69$\pm$1.41 & 24.74$\pm$1.41 & 5.10$\pm$1.08 & 96.50$\pm$0.09 & 96.19$\pm$0.11 & 90.77$\pm$0.15  \\ \midrule
        
        DirGNN$_{\text{F}}$ & 13.95$\pm$1.66 & 23.94$\pm$1.26 & 35.65$\pm$1.30 & 5.02$\pm$0.60 & 98.56$\pm$0.05 & 98.28$\pm$0.07 & 94.68$\pm$0.14 \\
        DirGNN$_{\text{D}}$ & 13.09$\pm$0.90 & 21.84$\pm$1.54 & 30.70$\pm$0.98 & 6.14$\pm$1.58 & 96.70$\pm$0.08 & 96.64$\pm$0.07 & 91.08$\pm$0.13  \\
        \midrule
        
        MagNet$_{\text{F}}$ & 5.02$\pm$0.60 & 8.49$\pm$0.48 & 12.66$\pm$0.62 & 1.74$\pm$0.66 & 89.18$\pm$0.08 & 88.99$\pm$0.10 & 81.43$\pm$0.08 \\
        MagNet$_{\text{D}}$ & 4.50$\pm$0.76 & 8.15$\pm$0.89 & 12.85$\pm$0.59 & 0.97$\pm$0.22 & 89.37$\pm$0.03 & 89.25$\pm$0.05 & 81.61$\pm$0.11 \\ \midrule

        
        DUPLEX$_{\text{F}}$ & 7.36$\pm$0.85 & 13.19$\pm$0.97 & 17.90$\pm$0.71 & 2.27$\pm$0.63 & 92.05$\pm$0.22 & 91.45$\pm$0.29 & 85.65$\pm$0.27 \\
        DUPLEX$_{\text{D}}$ & 5.92$\pm$1.83 & 9.04$\pm$4.80 & 14.56$\pm$5.51 & 1.48$\pm$0.55 & 87.00$\pm$4.97 & 86.79$\pm$4.33 & 78.90$\pm$5.65 \\ \midrule
        
        
        DiGAE$_{\text{F}}$ & 19.65$\pm$1.33 & 31.16$\pm$1.62 & 41.55$\pm$1.62 & 5.34$\pm$1.18 & 97.29$\pm$0.13 & 97.12$\pm$0.44 & 89.52$\pm$0.15 \\
        DiGAE$_{\text{D}}$ & 11.94$\pm$0.96 & 19.38$\pm$1.60 & 26.53$\pm$1.20 & 3.58$\pm$1.20 & 92.86$\pm$0.28 & 93.51$\pm$0.20 & 82.11$\pm$0.32 \\ \midrule

        SDGAE$_{\text{F}}$ &\hig{1}{27.07$\pm$2.34} &\hig{1}{41.37$\pm$1.61} & \hig{1}{53.79$\pm$1.56} &\hig{1}{ 8.41$\pm$1.89} & \hig{1}{99.07$\pm$0.04} & \hig{1}{99.00$\pm$0.03} &95.66$\pm$0.12 \\
        
        SDGAE$_{\text{D}}$& 15.18$\pm$2.22 & 24.02$\pm$1.53 & 33.51$\pm$1.16 & 5.73$\pm$1.33 & 97.44$\pm$0.12 & 97.32$\pm$0.11 & 92.26$\pm$0.24 \\

        \bottomrule
    \end{tabular}}
\end{table}


\clearpage
\hypertarget{app_complete_res_wiki}{}
\subsection{Results on WikiCS}\label{app_complete_res_wiki}
%Table~\ref{app_tb_wikics} shows the results of DirLinkBench on Wikics.
\begin{table}[ht]
    \centering
    \caption{Benchmark Results on WikiCS. For all methods, $_{\text{F}}$ indicates the use of original node features as input, while $_{\text{D}}$ indicates the use of in/out degrees as input. Results ranked \hig{1}{first} and \hig{2}{second} are highlighted.}
    \label{app_tb_wikics}
    \resizebox{\textwidth}{!}{
    \begin{tabular}{lccccccc}
        \toprule
        Model & Hits@20 & Hits@50 & Hits@100 & MRR & AUC & AP & ACC \\
        \midrule
        STRAP &\hig{1}{64.97$\pm$1.52} &\hig{1}{71.10$\pm$1.13} & \hig{1}{76.27$\pm$0.92} & \hig{1}{37.29$\pm$9.42} &\hig{1}{98.66$\pm$0.03} &\hig{1}{98.91$\pm$0.02} &\hig{1}{94.57$\pm$0.06} \\
        
        ODIN & 3.41$\pm$0.60 & 6.24$\pm$0.63 & 9.83$\pm$0.47 & 1.17$\pm$0.35 & 92.86$\pm$0.04 & 91.90$\pm$0.10 & 86.39$\pm$0.08  \\
        
        ELTRA & 3.84$\pm$0.49 & 6.62$\pm$0.55 & 9.88$\pm$0.70 & 1.42$\pm$0.33 & 92.66$\pm$0.19 & 92.19$\pm$0.23 & 85.73$\pm$0.28  \\ \midrule
        
        MLP$_\text{F}$ & 0.24$\pm$0.09 & 0.71$\pm$0.40 & 1.21$\pm$0.25 & 0.15$\pm$0.07 & 71.90$\pm$0.54 & 70.86$\pm$0.68 & 66.26$\pm$0.67 \\ 
        MLP$_\text{D}$ & 4.04$\pm$0.72 & 7.92$\pm$0.61 & 12.99$\pm$0.68 & 1.63$\pm$0.20 & 92.12$\pm$0.10 & 91.90$\pm$0.19 & 84.65$\pm$0.22 \\ \midrule
        
        GCN$_\text{F}$ & 6.10$\pm$1.85 & 11.25$\pm$1.86 & 18.05$\pm$1.43 & 2.07$\pm$0.49 & 95.86$\pm$0.27 & 95.41$\pm$0.45 & 89.85$\pm$0.35  \\
        GCN$_\text{D}$ & 18.15$\pm$2.49 & 30.21$\pm$2.51 & 38.37$\pm$1.51 & 6.09$\pm$2.16 & 95.97$\pm$0.08 & 96.23$\pm$0.08 & 89.58$\pm$0.09 \\ \midrule
        
        GAT$_\text{F}$ & 1.83$\pm$0.54 & 3.52$\pm$2.21 & 5.60$\pm$1.17 & 0.83$\pm$0.31 & 73.18$\pm$3.59 & 74.81$\pm$4.19 & 54.16$\pm$6.34 \\
        GAT$_\text{D}$ & 24.13$\pm$4.10 & 34.32$\pm$4.41 & 40.47$\pm$4.10 & 7.67$\pm$2.07 & 95.72$\pm$0.27 & 96.06$\pm$0.33 & 88.90$\pm$0.36  \\ \midrule

        APPNP$_\text{F}$ & 3.71$\pm$0.46 & 6.66$\pm$0.76 & 11.11$\pm$1.09 & 1.64$\pm$0.33 & 92.76$\pm$0.63 & 92.11$\pm$0.63 & 85.56$\pm$0.08 \\ 
        APPNP$_\text{D}$ & 7.60$\pm$1.75 & 12.76$\pm$3.01 & 20.23$\pm$1.72 & 2.75$\pm$0.70 & 94.12$\pm$0.09 & 93.89$\pm$0.12 & 87.51$\pm$0.14 \\ \midrule
        
        DGCN$_\text{F}$ & 5.84$\pm$1.79 & 10.12$\pm$3.32 & 15.88$\pm$5.12 & 2.04$\pm$0.50 & 95.98$\pm$0.65 & 95.45$\pm$0.80 & 89.95$\pm$0.96 \\
        DGCN$_\text{D}$ & 8.35$\pm$3.46 & 16.51$\pm$2.99 & 25.91$\pm$4.10 & 2.88$\pm$1.11 & 96.22$\pm$0.32 & 96.04$\pm$0.43 & 90.13$\pm$0.37 \\ \midrule
        

        DiGCN$_\text{F}$ & 3.63$\pm$0.69 & 6.77$\pm$0.98 & 10.24$\pm$0.88 & 1.58$\pm$0.46 & 92.55$\pm$1.53 & 91.63$\pm$1.62 & 82.60$\pm$2.14 \\
        DiGCN$_\text{D}$ & 10.08$\pm$1.33 & 16.32$\pm$2.16 & 25.31$\pm$1.84 & 3.85$\pm$0.98 & 95.59$\pm$0.12 & 95.57$\pm$0.12 & 88.89$\pm$0.14 \\ \midrule
        
        DiGCNIB$_\text{F}$ & 5.36$\pm$0.58 & 8.28$\pm$1.20 & 13.61$\pm$1.18 & 1.81$\pm$0.48 & 94.00$\pm$0.33 & 93.31$\pm$0.39 & 84.49$\pm$1.62 \\
        DiGCNIB$_\text{D}$& 12.01$\pm$2.95 & 20.99$\pm$1.75 & 28.28$\pm$2.44 & 3.96$\pm$1.16 & 96.36$\pm$0.07 & 96.30$\pm$0.08 & 90.72$\pm$0.10 \\ \midrule
        
        DirGNN$_\text{F}$ & 11.76$\pm$3.31 & 23.90$\pm$3.31 & 33.28$\pm$2.84 & 4.82$\pm$1.77 & 97.09$\pm$0.13 & 97.04$\pm$0.16 & 91.55$\pm$0.25 \\ 
        DirGNN$_\text{D}$ & 28.16$\pm$2.09 & 40.94$\pm$1.19 & 50.48$\pm$0.85 & \hig{2}{12.08$\pm$2.05} & 97.13$\pm$0.09 & 97.33$\pm$0.09 & 91.28$\pm$0.14 \\\midrule
        
        MagNet$_\text{F}$ & 3.41$\pm$0.35 & 6.10$\pm$0.45 & 9.25$\pm$0.57 & 1.42$\pm$0.22 & 92.57$\pm$0.08 & 91.54$\pm$0.10 & 86.06$\pm$0.11 \\
        MagNet$_\text{D}$ & 4.13$\pm$0.46 & 6.57$\pm$0.34 & 10.81$\pm$0.46 & 1.26$\pm$0.37 & 93.06$\pm$0.06 & 92.18$\pm$0.07 & 86.59$\pm$0.06 \\ \midrule
        
        DUPLEX$_{\text{F}}$ & 2.97$\pm$0.52 & 5.54$\pm$0.63 & 8.52$\pm$0.60 & 0.89$\pm$0.22 & 90.92$\pm$0.12 & 90.06$\pm$0.45 & 83.82$\pm$0.23 \\
        DUPLEX$_{\text{D}}$ & 1.09$\pm$0.92 & 1.53$\pm$3.12 & 5.45$\pm$0.19 & 1.03$\pm$0.08 & 90.93$\pm$0.04 & 90.37$\pm$0.23 & 83.81$\pm$0.11 \\ \midrule
        
        DiGAE$_{\text{F}}$ & 7.01$\pm$1.78 & 11.54$\pm$1.11 & 18.17$\pm$1.27 & 2.18$\pm$0.78 & 93.47$\pm$1.04 & 93.20$\pm$1.16 & 76.38$\pm$0.21  \\
        DiGAE$_{\text{D}}$ & 10.6$\pm$1.89 & 19.69$\pm$1.98 & 29.21$\pm$1.36 & 4.59$\pm$0.88 & 93.46$\pm$0.22 & 94.47$\pm$0.15 & 82.53$\pm$0.48 \\ \midrule

        SDGAE$_{\text{F}}$& 8.54$\pm$3.42 & 19.60$\pm$5.04 & 29.95$\pm$5.80 & 2.35$\pm$0.91 & 96.64$\pm$1.23 & 96.58$\pm$1.20 & 90.98$\pm$1.52  \\
        SDGAE$_{\text{D}}$&\hig{2}{33.04$\pm$3.27} & \hig{2}{47.62$\pm$1.74} &\hig{2}{54.67$\pm$2.50} &11.78$\pm$4.11 &\hig{2}{97.23$\pm$0.07} & \hig{2}{97.58$\pm$0.06} &\hig{2}{92.24$\pm$0.12}  \\
        \bottomrule
    \end{tabular}}
\end{table}


\clearpage
\hypertarget{app_complete_res_slash}{}
\subsection{Results on Slashdot}\label{app_complete_res_slash}
%Table~\ref{app_tb_slash} shows the results of DirLinkBench on Slashdot.
\begin{table}[h]
    \centering
    \caption{Benchmark Results on Slashdot. For all methods, $_{\text{R}}$ indicates the use of random node features as input, while $_{\text{D}}$ indicates the use of in/out degrees as input. Results ranked \hig{1}{first} and \hig{2}{second} are highlighted.}
    \label{app_tb_slash}
    \resizebox{\textwidth}{!}{
    \begin{tabular}{lccccccc}
        \toprule
        Model & Hits@20 & Hits@50 & Hits@100 & MRR & AUC & AP & ACC \\
        \midrule
        STRAP & 19.10$\pm$1.06 & 25.39$\pm$1.43 & 31.43$\pm$1.21 & \hig{1}{9.82$\pm$1.82} & 94.74$\pm$0.05 & 95.13$\pm$0.05 & 88.19$\pm$0.08  \\
        
        ODIN & 14.97$\pm$1.40 & 24.91$\pm$1.19 & 34.17$\pm$1.19 & 4.44$\pm$1.59 & 96.58$\pm$0.07 & 96.75$\pm$0.06 & 90.39$\pm$0.10  \\
        
        ELTRA & 18.02$\pm$2.11 & 26.31$\pm$0.95 & 33.44$\pm$1.00 & 5.53$\pm$1.77 & 94.65$\pm$0.03 & 95.23$\pm$0.04 & 88.11$\pm$0.11 \\ \midrule
        
        MLP$_{\text{R}}$ & 4.25$\pm$0.71 & 7.26$\pm$0.74 & 11.31$\pm$0.70 & 1.28$\pm$0.33 & 72.86$\pm$0.37 & 76.72$\pm$0.21 & 66.16$\pm$0.26   \\ 
        MLP$_{\text{D}}$ & 14.16$\pm$5.22 & 24.01$\pm$0.79 & 32.97$\pm$0.51 & 4.14$\pm$1.71 & 95.84$\pm$0.07 & 96.21$\pm$0.05 & 89.62$\pm$0.11 \\ \midrule
        
        GCN$_{\text{R}}$ & 12.52$\pm$1.30 & 21.03$\pm$1.62 & 29.04$\pm$0.95 & 3.03$\pm$0.73 & 95.58$\pm$0.13 & 95.69$\pm$0.13 & 89.46$\pm$0.12   \\
        GCN$_{\text{D}}$ & 16.28$\pm$1.59 & 24.51$\pm$1.63 & 33.16$\pm$1.22 & 5.54$\pm$2.04 & 95.90$\pm$0.07 & 96.17$\pm$0.06 & 89.75$\pm$0.17  \\ \midrule
        
        GAT$_{\text{R}}$ & 14.82$\pm$2.70 & 22.19$\pm$2.78 & 30.16$\pm$3.11 & 5.11$\pm$1.53 & 96.26$\pm$0.18 & 96.45$\pm$0.20 & 87.14$\pm$1.03  \\
        GAT$_{\text{D}}$ & 12.39$\pm$1.47 & 19.54$\pm$1.89 & 26.65$\pm$1.98 & 4.53$\pm$0.78 & 95.01$\pm$0.34 & 95.25$\pm$0.45 & 88.01$\pm$0.49 \\ \midrule

        
        APPNP$_{\text{R}}$ & 14.83$\pm$1.18 & 22.75$\pm$1.45 & 31.25$\pm$1.14 & 4.56$\pm$1.37 & 95.36$\pm$0.07 & 95.72$\pm$0.06 & 88.76$\pm$0.09  \\ 
        APPNP$_{\text{D}}$ & 15.00$\pm$5.47 & 24.34$\pm$1.47 & 33.76$\pm$1.05 & 5.86$\pm$2.78 & 96.21$\pm$0.06 & 96.43$\pm$0.05 & 90.07$\pm$0.10  \\ \midrule

        
        %DGCN & TO & TO & TO & TO & TO & TO & TO \\
        %DiGCN & TO & TO & TO & TO & TO & TO & TO \\
        %DiGCNIB & TO & TO & TO & TO & TO & TO & TO \\
        DirGNN$_{\text{R}}$ & 18.72$\pm$1.93 & 28.52$\pm$1.09 & 37.41$\pm$1.37 & 6.19$\pm$2.04 & 96.67$\pm$0.05 & 96.89$\pm$0.04 & 90.24$\pm$0.12  \\
        DirGNN$_{\text{D}}$ & 20.55$\pm$2.85 & 31.20$\pm$1.18 & 41.74$\pm$1.15 & 7.52$\pm$3.24 &\hig{1}{96.95$\pm$0.05} & \hig{1}{97.14$\pm$0.06} &\hig{1}{ 90.65$\pm$0.13}  \\ \midrule
        
        MagNet$_{\text{R}}$ & 9.16$\pm$1.26 & 19.58$\pm$1.37 & 28.50$\pm$1.78 & 1.98$\pm$0.25 & 96.31$\pm$0.06 & 96.38$\pm$0.06 & 90.04$\pm$0.13 \\
        MagNet$_{\text{D}}$ & 12.55$\pm$0.75 & 22.34$\pm$0.42 & 31.98$\pm$1.06 & 2.83$\pm$0.51 & 96.57$\pm$0.09 & 96.69$\pm$0.10 & 90.45$\pm$0.09 \\ \midrule
        
        DUPLEX$_{\text{R}}$ &5.67$\pm$1.85	&11.49$\pm$3.36	&18.42$\pm$2.59	&1.81$\pm$0.83	&94.36$\pm$3.25	&94.48$\pm$2.76	&85.42$\pm$3.42\\
        DUPLEX$_{\text{D}}$ & 2.51$\pm$0.64 & 5.95$\pm$1.11 & 9.39$\pm$2.71 & 0.51$\pm$0.01 & 80.57$\pm$3.85 & 86.52$\pm$1.50 & 77.67$\pm$2.24 \\ \midrule
        
        DiGAE$_{\text{R}}$ & 18.89$\pm$1.71 & 27.69$\pm$1.41 & 36.57$\pm$1.32 & 5.49$\pm$2.96 & 95.26$\pm$0.29 & 96.13$\pm$0.19 & 84.27$\pm$0.26  \\
        DiGAE$_{\text{D}}$ &\hig{1}{23.68$\pm$0.94} &\hig{1}{33.97$\pm$1.06} &\hig{2}{41.95$\pm$0.93} & 5.54$\pm$1.51 & 94.30$\pm$0.29 & 95.80$\pm$0.12 & 85.67$\pm$0.28 \\ \midrule

        SDGAE$_{\text{R}}$ &18.24$\pm$2.05 & 28.47$\pm$1.97 & 37.34$\pm$0.68 & 5.51$\pm$1.65 & 95.96$\pm$0.07 & 96.31$\pm$0.03 & 90.11$\pm$0.11  \\
        SDGAE$_{\text{D}}$ &\hig{2}{23.57$\pm$2.11} &\hig{2}{33.75$\pm$1.48} & \hig{1}{42.42$\pm$1.15} &\hig{2}{8.41$\pm$3.80} & \hig{2}{96.70$\pm$0.10} & \hig{2}{97.06$\pm$0.08} & \hig{1}{91.05$\pm$0.20} \\
        
        \bottomrule
    \end{tabular}}
\end{table}

\clearpage
\hypertarget{app_complete_res_epinion}{}
\subsection{Results on Epinions}\label{app_complete_res_epinion}
%Table~\ref{app_tb_epinion} shows the results of DirLinkBench on Epinions.
\begin{table}[ht]
    \centering
    \caption{Benchmark Results on Epinions. For all methods, $_{\text{R}}$ indicates the use of random node features as input, while $_{\text{D}}$ indicates the use of in/out degrees as input. Results ranked \hig{1}{first} and \hig{2}{second} are highlighted.}
    \label{app_tb_epinion}
    \resizebox{\textwidth}{!}{
    \begin{tabular}{lccccccc}
        \toprule
        Model & Hits@20 & Hits@50 & Hits@100 & MRR & AUC & AP & ACC \\
        \midrule
        STRAP & \hig{1}{44.66$\pm$1.68} &  \hig{1}{53.48$\pm$0.86} &  \hig{1}{58.99$\pm$0.82} &  \hig{1}{21.18$\pm$6.31} & 96.62$\pm$0.06 & 97.60$\pm$0.04 & 91.89$\pm$0.06 \\
        
        ODIN & 11.85$\pm$3.33 & 25.99$\pm$1.79 & 36.91$\pm$0.47 & 3.22$\pm$0.80 & 97.72$\pm$0.03 & 97.91$\pm$0.03 & 92.42$\pm$0.07  \\
        
        ELTRA & 16.89$\pm$1.27 & 28.37$\pm$1.53 & 41.63$\pm$2.53 & 5.81$\pm$1.10 & 96.19$\pm$0.04 & 97.47$\pm$0.03 & 90.40$\pm$0.13 \\ \midrule
        
        MLP$_\text{R}$ & 3.84$\pm$0.65 & 7.02$\pm$0.40 & 10.18$\pm$0.67 & 1.18$\pm$0.36 & 78.44$\pm$0.25 & 80.67$\pm$0.14 & 67.83$\pm$1.93  \\ 
        MLP$_\text{D}$ & 15.84$\pm$3.61 & 34.11$\pm$2.04 & 44.59$\pm$1.62 & 4.35$\pm$1.64 & 97.85$\pm$0.08 & 98.02$\pm$0.07 & 92.56$\pm$0.45  \\ \midrule
        
        GCN$_\text{R}$ & 18.26$\pm$2.56 & 30.18$\pm$1.71 & 40.64$\pm$1.53 & 4.04$\pm$1.10 & 96.90$\pm$0.04 & 97.54$\pm$0.03 & 92.10$\pm$0.24  \\
        GCN$_\text{D}$& 12.40$\pm$7.59 & 30.79$\pm$9.24 & 46.10$\pm$1.37 & 3.12$\pm$0.71 & 97.83$\pm$0.09 & 98.15$\pm$0.05 &\hig{2}{94.17$\pm$0.08} \\ \midrule

        GAT$_\text{R}$ & 19.65$\pm$3.52 & 31.90$\pm$3.56 & 43.65$\pm$4.88 & 6.04$\pm$2.16 & 98.35$\pm$0.11 & \hig{2}{ 98.50$\pm$0.11} & 92.05$\pm$1.30  \\
        GAT$_\text{D}$ & 18.18$\pm$4.11 & 27.18$\pm$3.95 & 36.76$\pm$5.74 & 7.98$\pm$1.92 & 97.61$\pm$0.29 & 97.81$\pm$0.22 & 92.92$\pm$0.37 \\ \midrule
        
        APPNP$_\text{R}$  & 17.86$\pm$2.53 & 27.89$\pm$1.24 & 39.06$\pm$1.26 & 4.88$\pm$4.14 & 97.70$\pm$0.04 & 97.92$\pm$0.02 & 92.84$\pm$0.06  \\ 
        APPNP$_\text{D}$& 18.46$\pm$3.51 & 30.84$\pm$1.84 & 41.99$\pm$1.23 & 6.41$\pm$3.73 &\hig{2}{98.36$\pm$0.06} & 98.48$\pm$0.06 & 94.17$\pm$0.16  \\ \midrule
        %DGCN & TO & TO & TO & TO & TO & TO & TO \\
        %DiGCN & TO & TO & TO & TO & TO & TO & TO \\
        %DiGCNIB & TO & TO & TO & TO & TO & TO & TO \\
      
        DirGNN$_{\text{R}}$ & 25.66$\pm$3.33 & 39.28$\pm$1.96 & 50.10$\pm$2.06 & 7.03$\pm$2.70 & 98.25$\pm$0.03 & 98.46$\pm$0.03 & 93.99$\pm$0.05 \\
        DirGNN$_{\text{D}}$ & 21.35$\pm$2.89 & 34.06$\pm$2.53 & 46.01$\pm$2.07 & 6.12$\pm$2.54 & 98.03$\pm$0.03 & 98.26$\pm$0.02 & 93.48$\pm$0.06 \\ \midrule

        
        MagNet$_{\text{R}}$ & 6.41$\pm$1.57 & 12.95$\pm$1.51 & 22.12$\pm$1.27 & 1.62$\pm$0.35 & 97.60$\pm$0.03 & 97.69$\pm$0.03 & 92.79$\pm$0.05  \\
        MagNet$_{\text{D}}$ & 7.68$\pm$1.00 & 16.30$\pm$2.36 & 28.01$\pm$1.72 & 2.53$\pm$0.57 & 97.71$\pm$0.03 & 97.86$\pm$0.04 & 92.92$\pm$0.05  \\ \midrule
        
        DUPLEX$_{\text{R}}$ & 3.76$\pm$1.94 & 8.38$\pm$4.39 & 16.50$\pm$4.34 & 1.85$\pm$0.39 & 92.11$\pm$2.25 & 93.78$\pm$3.69 & 88.20$\pm$3.86  \\
        DUPLEX$_{\text{D}}$ & 2.43$\pm$0.86 & 6.74$\pm$0.45 & 12.35$\pm$2.64 & 0.46$\pm$0.08 & 90.61$\pm$1.76 & 88.72$\pm$3.26 & 81.82$\pm$3.74  \\ \midrule
        
        DiGAE$_{\text{R}}$ & 29.11$\pm$3.51 & 43.25$\pm$1.68 & 53.27$\pm$1.17 & 8.85$\pm$2.81 & 97.11$\pm$0.19 & 97.78$\pm$0.11 & 89.16$\pm$0.24   \\
        DiGAE$_{\text{D}}$ & 22.56$\pm$9.23 & 43.19$\pm$2.97 & 55.14$\pm$1.96 & 5.33$\pm$1.78 & 96.47$\pm$0.18 & 97.37$\pm$0.10 & 90.17$\pm$0.14  \\ \midrule

        SDGAE$_{\text{R}}$ & 21.88$\pm$2.17 & 35.28$\pm$2.35 & 45.04$\pm$2.18 & 6.54$\pm$1.94 & 98.11$\pm$0.03 & 98.35$\pm$0.03 & 93.83$\pm$0.07 \\
        SDGAE$_{\text{D}}$ &\hig{2}{32.81$\pm$2.67} &\hig{2}{45.61$\pm$1.92} &\hig{2}{55.91$\pm$1.77} &\hig{2}{11.62$\pm$2.86} &  \hig{1}{98.43$\pm$0.07} &\hig{1}{98.64$\pm$0.04} &\hig{1}{94.33$\pm$0.11}  \\
        
        \bottomrule
    \end{tabular}}
\end{table}