\section{Related Work}
\label{sec:related}
A data story is composed of a set of data-driven story pieces, which are typically visualized and presented in a meaningful order to communicate the author's intended message \cite{Lee2015, Riche2018}. 
Segel and Heer~\cite{Segel2010} identified seven genres of narrative visualization: magazine style, annotated chart, partitioned poster, flow chart, comic strip, slide show, and data video. 
Tong \etal~\cite{Tong2018} classified the storytelling literature based on key aspects of storytelling in visualization: the subjects involved, the methods used to convey stories, and the underlying purpose of storytelling in visualization.
Zhao and Elmqvist~\cite{Zhao2023a} proposed a taxonomy for data storytelling, focusing on five dimensions: audience cardinality, space and time, media components, data components, and viewing sequence.

For data story authoring, 
Li \etal~\cite{Li2023c} analyzed existing data storytelling tools from the perspective of human-AI collaboration, outlining four key stages in the storytelling process—analysis, planning, implementation, and communication—and identifying the roles of humans and AI as creators, assistants, optimizers, and evaluators. 
Chen \etal~\cite{Chen2022} developed a framework that maps seven types of narrative visualization against four levels of automation, analyzing how automation affects the storytelling process.
Ren \etal~\cite{Ren2023} categorized data storytelling tools based on whether the narrator offers a personal perspective on the data, distinguishing between omniscient and limited perspectives. 
Shi \etal~\cite{Shi2023} surveyed \textit{AI for design} to analyze how designers and AI can complement and collaborate with each other, identifying three themes: AI assisting designers, designers assisting AI, and collaboration between the two. 
% Wu \etal \cite{Wu2021d} and Wang \etal \cite{Wang2020c} provided insights into how AI can enhance the visualization process. 
% \haotian{Why existing AI work cannot address our question?}
\review{Q1, Q11}\revise{
However, the research above primarily offers a coarse-grained classification and discussion of existing data storytelling tools (\eg based on story mediums or AI roles within the system) and is not tailored for data videos. 
Data video combines diverse coordinated components (\eg visualization, animation, audio, \etc), and each component and coordination relationship poses different challenges.
As a result, existing frameworks do not support a fine-grained analysis of technical implementations for data videos.
}
% However, the above research is not tailored for data videos, missing sufficient understanding and reflection on core technical aspects of data video creation.

In addition to broader surveys, some research has also explored design patterns for specific data storytelling forms, such as data-driven news articles~\cite{Hao2024}, data comics~\cite{Bach2018}, dashboards~\cite{Bach2023}, affective visualizations~\cite{Lan2023}, timelines~\cite{Brehmer2017}, and character-oriented designs~\cite{Dasu2024, Mittenentzwei2023a}. 
% data videos, as a specific form of data storytelling, possess a complex structure that general surveys cannot fully cover in terms of core technical aspects.
In the context of data videos, numerous studies have empirically examined specific components such as narrative structure~\cite{Lan2021a, Yang2022a, Cao2020a, Wei2024}, cinematic styles~\cite{Xu2023b, Xu2022, Concannon2020, Conlen2023, Bradbury2020}, narrative transitions~\cite{Tang2020}, narration-animation interplay~\cite{Borgo2022}, animation roles and effectiveness~\cite{Amini2018a, Chevalier2016, Shu2020}, affectiveness~\cite{Sakamoto2022, Sallam2022}, visualization in motion~\cite{Yao2022}, and 3D data videos~\cite{Yang2023a}. 
\review{Q1, Q11}\revise{However, these studies are limited to empirical research and do not address technical implementation.

Overall, there is a lack of a systematic survey that reflects existing technological practices in data video creation, summarizing key design paradigms for creating and coordinating fine-grained data video components. 
This is vital for guiding new researchers and informing the development of next-generation design paradigms and tools.
Therefore, this paper will provide an in-depth technical analysis at the component level for data video production.}
% Despite the increasing attention of data videos in information communication and the rapid development of AI, there lacks a comprehensive survey on how to create data videos by designing paradigms and leveraging AI assistance to lower the barrier. 
% This gap is particularly significant in helping researchers understand existing paradigms and the role of AI in inspiring future research.