
\section{Discussion}
The previous section provided an in-depth analysis and reflection on the design paradigms of each component and their coordination. 
This section will discuss the gaps identified along two dimensions in our framework: \textit{what data video components to create and coordinate}, and \textit{how to support the creation and coordination}.
% This section will take a holistic view of data video creation, identifying remaining gaps in the whole process. We also shed light on how to bridge the gaps at each stage, including empirical studies, component creation and coordination, interaction design, evaluation, and applications. 
Additionally, we discuss our limitations. 
% to inspire future research.

\subsection{Empirical Research for Comprehensive Data Video Understanding}

% \textbf{Comprehensive Exploration of Data Videos.}
\textbf{Gap 1: Data videos exhibit diverse media forms, but lack a refined classification system and clear characteristic modeling.}
Since the emergence of a broad definition for data videos~\cite{Segel2010, Amini2015}, they have manifested in various forms across diverse scenarios like social media, news reporting, art exhibitions, and science communication. 
Despite their popularity, there is no comprehensive classification that captures their diverse characteristics, such as narrative intent (\eg showcasing data, explaining visualizations, narrating stories, \etc), communication goals (\eg description, hypothesis, persuasion, \etc), data presentation formats (\eg charts, pictographs, encoded in real-world scenes, \etc), and audience engagement (\eg passive viewing, interactive exploration, \etc). 
Such refined modeling can facilitate a more targeted paradigm design for different data videos.
Additionally, while many aspects of data videos have been studied, several unexplored design spaces remain, such as the use of music and sound effects, and transitions between real-world scenes and visualizations, as noted in Sec.~\ref{sec:tools}. 
% Despite their popularity, there is still a lack of comprehensive classification that summarizes the diverse characteristics of data videos. 
% This classification could consider factors such as narrative intent, communication goals, data presentation formats, and modes of audience engagement. 

\textbf{Gap 2: Data videos are seen as advantageous, but their ideal contexts remain unclear.}
While many studies highlight the advantages of data videos~\cite{Amini2018a, Robertson2008, Rodrigues2024}, they also possess inherent limitations, such as requiring viewers to follow a rigid narrative pace and absorb much information in a short time~\cite{Riche2018}. 
Therefore, more research is needed to assess the overall effectiveness of data videos and determine the best contexts for their use.
% Therefore, beyond studies on the effectiveness of specific components, there is a need for more comprehensive research to validate the overall effectiveness of data videos and to identify appropriate contexts for their use. 
We cannot always expect an ``easy win'' of data videos over simpler forms (\eg static visuals)~\cite{Amini2018a, Kong2019}; there are numerous trade-offs in effectiveness, production cost, and viewer engagement. 
The framework proposed in this paper can serve as a starting point to guide future research on the classification and effectiveness of data videos.
% The framework proposed in this paper can serve as a starting point for further research into the classification, effectiveness, and applicability of data videos. 
% These empirical studies could also lead to the development of more targeted and effective data video creation tools or the abstraction of a data video creation pipeline.
A more comprehensive empirical investigation will enhance our understanding of data videos and potentially lead to better paradigm design and tool development.
% improve their technical implementation.
% enrich the knowledge base of data videos and provide further guidance for technical implementation.

% \noindent
% \textbf{Bridging the Gap Between Empirical Research and Creation Tools.}
% \textbf{Gap 3: There are numerous empirical studies, but they are rarely directly applied to tool development.}
% , as discussed in Sec.~\ref{sec:related}
\textbf{Gap 3: Many empirical studies exist, but few are directly applied to paradigm design and tool development.}
Existing empirical studies have provided various guidelines for data videos, providing rich knowledge for the design of paradigms.
However, these guidelines are rarely directly applied in authoring tools~\cite{Chen2022}, as discussed in Sec.~\ref{sec:VisAni}. 
Several factors contribute to this gap: 
first, empirical studies often focus on design, while tools emphasize measurable and computable authoring processes; 
second, the output of empirical research is typically general, whereas tools are context-specific and require exploring a distinct design space; 
third, many effects explored in empirical studies rely on human design to achieve within existing components (\eg emotional impact~\cite{Lan2023, Lan2022}, narrative structure~\cite{Lan2021a, Yang2022a, Wei2024}, and cinematic effects\cite{Xu2022, Xu2023b}), whereas tools tend to focus on explicitly implementable functions; 
fourth, the abundance and dispersion of empirical guidelines make it difficult for creators to identify the most relevant ones.
To bridge this gap, the community could maintain a unified knowledge base for data videos, categorizing the existing guidelines, and potentially starting with the framework proposed in this paper. 
Additionally, a computable representation framework could be established, mapping high-level guidelines to low-level constraints~\cite{dataplayer}, and also designing interfaces that integrate new knowledge effectively.

\subsection{Data Video Component Creation and Coordination}
\textbf{Gap 4: Greater diversity of components and their coordination lead to better expressivity potential, but current tools cannot fully support all aspects.}
% , as well as our reflections. 
% Here, we offer some additional high-level discussion. 
% Data video components are fundamental to message delivery, and their effective coordination is crucial to production. 
% Data video components are foundational to conveying messages, and their coordination is a critical aspect of video production. 
% For the two types of components discussed before, the existing coordination patterns generally fall into four categories, as depicted in Fig.~\ref{fig:Coordination}, depending on the presence of the two components. 
% Sec.~\ref{sec:tools} provides a detailed analysis of current research on data video components and their coordination. The coordination patterns for two component types generally fall into four categories, as shown in Fig.~\ref{fig:Coordination}, based on their presence. 
Sec. \ref{sec:tools} analyzes current research on data video components and their coordination. The coordination of two component types generally fits into four categories based on their presence, as shown in Fig. \ref{fig:Coordination}.
Data video components and their coordination form a large space, but current tools cannot fully support all aspects to enhance the expressivity of data videos. 
Future efforts should integrate and coordinate a broader range of components, such as real-world scenes and pictographs in visuals, and music and sound effects in audio. 
This will require advancements in computer vision, graphics, and audio processing, alongside a robust representation language to manage all components within a unified system and the design of new paradigms.

\textbf{Gap 5: Current tools focus on reducing user learnability vertically, but rarely expand expressivity horizontally.}
Many current tools focus on user-friendly paradigms and automation to reduce learnability when involving more components vertically~\cite{Grossman2009}, showcasing their expressivity through limited examples. However, expanding expressivity horizontally is often deferred to future engineering efforts. 
For instance, in template-based systems, the expressivity of data videos is constrained by the diversity of templates, which are time-consuming to extend~\cite{wonderflow, Amini2017}. 
A key future direction is to explore paradigms to rapidly enhance expressivity while ensuring ease of learning. This could involve efficiently creating templates from real-world examples, reverse-engineering data videos to extract features, and adapting them to current tool frameworks.
% A key future direction is to explore rapid expressivity expansion while maintaining learnability. This could involve efficient templatization of real-world example galleries, reverse-engineering of data videos to extract features, and automatically adapting them to existing tool frameworks.
In addition, from a tool perspective, components can be divided into two categories: user-uploaded and tool-generated. 
User-uploaded components require advanced understanding to support design, while tool-generated components need strong capabilities to fulfill user intent. Future developments should leverage cutting-edge AI models to enhance both understanding and generation.
Moreover, an important future direction would be building multi-agent systems to simulate each role in data video creation and fully automate the process~\cite{Shen2024a}.

\begin{figure*}[t]
  \centering
    \includegraphics[width=0.85\linewidth]{figures/coordination.png}
    \caption{Coordination of two data video components follows four common patterns, depending on the presence of the two components.}
\label{fig:Coordination}
  \vspace{-10px}
\end{figure*}


\subsection{Interaction Design}


% \noindent
% \textbf{Fine-Grained Intent Modeling.} 
\textbf{Gap 6: Users have diverse intentions, but tools cannot fully comprehend them.}
The goal of interaction is to enhance communication between users and tools, allowing for the exchange of intentions and feedback. 
% Given the complexity of tasks, 
User intentions can vary widely, from abstract goals (\eg ``\textit{create a magical video}'') to specific tasks (\eg ``\textit{highlight the blue bar}''), forming a hierarchical structure. 
Different user roles, such as novices and experts, also have unique intentions and requirements.
Currently, our understanding of user intentions in data video creation is fragmented and lacks a comprehensive model. Developing such a model is crucial for connecting diverse user intentions to specific actions within the tool. This would help users express their goals more clearly and allow the system to interpret them accurately, leading to better interaction design.
% Currently, the understanding of user intentions in data video creation is fragmented and lacks a comprehensive model. Developing a detailed model of user intent is essential for mapping these varied intentions to corresponding actions within the tool. This model would enable users to better articulate their intentions at different levels, while allowing the system to more accurately interpret and respond, leading to more effective interaction design.
% \noindent \textbf{Interaction Preferences for Each Task.} 
% In complex data video creation tasks, the more freedom a tool grants the user, the greater the demand on the tool's ability to parse user intentions and generate corresponding components. 
In complex tasks, giving users more freedom increases the demand for tools to understand their intentions and generate suitable components.
Existing tools often limit user freedom through strategies like predefined templates~\cite{Amini2017, wonderflow}, rules~\cite{Li2023b, Wang2021d}, or syntaxes~\cite{Ge2020, Zong2022}. While these approaches can streamline the process, they often require users to undergo a learning curve and lack features that help users understand the system's capabilities and optimal inputs. 
Some tools also restrict certain components from being edited or fully automate the process, which may not satisfy users' customization needs.
Understanding user preferences in task interaction, particularly regarding the balance between human and AI involvement and the preferred forms of collaboration, is crucial~\cite{Li2023a}. 
% This understanding can help balance user intent with the tool’s complexity and implementation challenges. Additionally, gathering these preferences can vary depending on the scenario, whether optimizing tools for professionals or making them accessible to novices.
Finally, interaction design should align with established patterns~\cite{nielsen1999designing}. When introducing new interaction paradigms, it is important to consider how to educate users on their use and how to persuade them to adopt new workflows over existing ones.




\subsection{Evaluation}
% \noindent
% \textbf{Establish An Evaluation Framework.} 
\textbf{Gap 7: Data video tasks are complex, but a comprehensive evaluation framework is lacking.}
Evaluating data video tools involves two key aspects: the quality of the generated videos and the effectiveness of the design paradigms. 
While criteria like comprehension, memorability, and engagement for data videos, and usability, extensibility, and expressivity for tools, have been identified, they often remain general and are typically assessed through human-centric methods like user studies and expert interviews~\cite{Riche2018}. 
This reliance on subjective evaluation is understandable given the diverse nature of data storytelling, where individual preferences are hard to model. 
Recent efforts have begun to introduce quantitative approaches, such as modeling story transitions~\cite{Shi2021a}, aligning music with video content~\cite{Tang2022}, counting interactions~\cite{wonderflow}, and predicting design outcomes~\cite{Wang2021d}. 
These are promising steps, but there is a clear need for a more comprehensive, controlled, and quantifiable evaluation framework. 
Insights from cognitive science and psychology could help inform the development of such a framework, combining qualitative and quantitative measures. 
Additionally, the advancement of LLMs could enhance the evaluation process at various stages, particularly in areas that are hard to quantify. For example, AI models could simulate diverse scenarios and target audiences, enabling more personalized and context-aware evaluations.

\textbf{Gap 8: LLMs are increasingly applied in data video creation, but their reliability remains difficult to ensure.}
The rise of LLMs has significantly accelerated tool development and facilitated research on new design paradigms, but also raises concerns about reliability and what we should do when encountering errors. 
Enhancing the robustness of AI-driven tools requires not just advanced error correction but also greater transparency in AI decision-making processes. 
% Future research should focus on enhancing the reliability of AI-driven tools by developing advanced error correction mechanisms and increasing transparency in AI decision-making. 
Integrating human-in-the-loop approaches could provide a safeguard, allowing human oversight to intervene when AI outputs are unreliable. 
Moreover, hybrid evaluation frameworks that blend AI efficiency with human judgment are essential to ensure tools remain not only powerful but also trustworthy, especially in complex and unpredictable real-world applications.
% Additionally, developing hybrid evaluation frameworks that combine AI strengths with human expertise can ensure that tools are not only powerful but also trustworthy and adaptable to real-world conditions.

% \noindent
% \textbf{Reliability of AI Models.} 
% The rise of large AI models has significantly accelerated tool development but also raises concerns about reliability and handling errors. Current practices include model interpretability, error detection mechanisms, and robustness testing, yet there is room for improvement. Future research should focus on enhancing the reliability of AI-driven tools through better error correction strategies and increased transparency in AI decision-making. Integrating human-in-the-loop approaches could provide a safeguard, allowing human oversight to intervene when AI outputs are unreliable. Additionally, developing hybrid evaluation frameworks that combine AI strengths with human expertise can ensure that tools are not only powerful but also trustworthy and adaptable to real-world conditions.


\subsection{Expending the Application Scope}
\textbf{Gap 9: General data video research is flourishing, but its penetration and empowerment in domain applications remain limited.}
While general research on data videos is important, many fields either overlook or fail to utilize them effectively. 
Current domain-specific research primarily focuses on sports~\cite{Chen2022c,Chen2023d,Chen2022h,Lin2023b,Lin2023a, Yao2024} and health~\cite{Sakamoto2022, Sallam2022}, both empirically and in tool development.
% Future efforts should deepen this integration.
Therefore, a crucial research direction is further integrating data video techniques within specific domains and exploring domain-specific design paradigms. 
One strategy is to extend the use of videos in fields where they are already employed to enhance communication, such as dynamic code presentations~\cite{NotePlayer}, news reviving~\cite{Wang2024f}, and video-based programming tutorials~\cite{Chi2022}. Data-driven modeling can build on these contexts. 
Another strategy is to expand into fields well-established in Visual Analytics Science and Technology (VAST) research~\cite{munzner2014visualization}, integrating storytelling approaches to bridge data analysis and insight communication.
% By leveraging VAST methodologies and integrating storytelling components, we can better bridge the gap between data analysis and insight communication.
Furthermore, developing domain-specific frameworks that capture the unique characteristics of each field can help data videos adapt more effectively.
In addition, while data videos are primarily used for communication now, their potential can extend beyond this. For instance, data videos and other storytelling forms can enhance the data analysis process by presenting intermediate results in a narrative format, aiding analysts in decision-making. Data videos, as a dynamic presentation form, can also be instrumental in improving explainability, such as in the interaction and feedback loop between users and LLMs~\cite{xie2024waitgpt}. All these interesting applications also require the design of new paradigms.

\textbf{Limitation.}
Our research has several limitations. 
First, we focused primarily on analyzing academic tools, without conducting a comprehensive survey of commercial software. On the one hand, it is challenging to capture the full scope of commercial tools; on the other, these tools evolve continuously, making it difficult to summarize paradigms and compare them with other tools. 
We hope future research can continuously integrate commercial software into our framework for analysis.
% Second, our analysis of Tab.~\ref{tab:tools} is primarily vertical, organized by components, and lacks a detailed horizontal discussion at the tool level. Since each tool comprises multiple components, its main contribution is often concentrated in a few key components. A horizontal discussion tends to obscure the design paradigms, while a vertical discussion allows for deeper insights.
Second, our annotation of tools in Tab.~\ref{tab:tools} could have been more fine-grained, such as at the component level discussed in Sec.~\ref{sec:components}, rather than the clustered component types in Sec.~\ref{sec:expressivity}. However, we did not adopt this approach because such annotation would be too detailed, hindering the generalization and discussion of design paradigms. Moreover, some sub-tasks are not unique to data video creation.


% \sutt{One promising approach for supporting more accessible and appealing human-data interactions is to present data in media, entertainment and cultural forms people are already familiar with and enjoy.}

% \sutt{We found that those using a small multiples design (static) consistently completed tasks in less time, albeit with slightly less confidence than those using an animated design. The accuracy results were more task-dependent}