\documentclass[sigconf, screen, nonacm]{acmart}



\usepackage{color}
\usepackage{soul}    %
\usepackage{amsmath}
\usepackage{subfig}
\usepackage{xcolor, colortbl}
\usepackage{bm}
\usepackage{amsfonts}
\usepackage{acmart-taps}
\usepackage{hyperref}

\sloppy
\def\num#1{#1}


\usepackage{gensymb} %

\usepackage{enumitem} %

\usepackage{xspace} %

\newcommand{\systemName}{\textsc{FontCraft}\xspace}

\newcommand{\etal}{{\it{et~al.}}}
\newcommand{\ie}{i.e.,}
\newcommand{\eg}{e.g.,}

\def\sectionautorefname{Section}
\def\subsectionautorefname{Section}
\def\subsubsectionautorefname{Section}

\definecolor{gray}{rgb}{0.5,0.5,0.5}
\definecolor{green}{rgb}{0, 0.6, 0}
\definecolor{orange}{rgb}{1, 0.5, 0}
\definecolor{mahogany}{rgb}{0.75, 0.25, 0.0}
\definecolor{purple}{rgb}{0.6, 0, 0.6}
\definecolor{darkgreen}{rgb}{0, 0.3, 0}
\definecolor{orange}{rgb}{1, 0.5, 0.}
\definecolor{lightblue}{rgb}{0.52, 0.75,0.91}
\newcommand{\ichao}[1]{\textcolor{purple}{{ichao: #1}}}
\newcommand{\doga}[1]{\textcolor{red}{{doga: #1}}}
\newcommand{\yuki}[1]{\textcolor{green}{{yuki: #1}}}
\newcommand{\unsure}[1]{\textcolor{orange}{{#1}}}

\newcommand{\bestcell}[1]{\cellcolor{lightblue!50}#1}
\newcommand{\seccell}[1]{\cellcolor{secondblue!50}#1}
\newcommand{\probcell}[1]{\cellcolor{softred}#1}
\colorlet{soullightblue}{lightblue!50}
\newcommand{\besthint}[1]{\sethlcolor{soullightblue}\hl{#1}}
\colorlet{soullightyellow}{yellow!40}
\newcommand{\rrhl}[1]{#1}
\DeclareRobustCommand{\todo}[1]{
  \begingroup
  \definecolor{hlcolor}{RGB}{245,183,177}\sethlcolor{hlcolor}%
  \hl{\textbf{TODO:} #1}%
  \endgroup
}


\copyrightyear{2025}
\acmYear{2025}
\setcopyright{acmlicensed}\acmConference[CHI '25]{CHI Conference on Human Factors in Computing Systems}{April 26-May 1, 2025}{Yokohama, Japan}
\acmBooktitle{CHI Conference on Human Factors in Computing Systems (CHI '25), April 26-May 1, 2025, Yokohama, Japan}
\acmDOI{10.1145/3706598.3713863}
\acmISBN{979-8-4007-1394-1/25/04}

\begin{document}

\title[\systemName: Multimodal Font Design Using Interactive Bayesian Optimization]{\systemName: Multimodal Font Design\\Using Interactive Bayesian Optimization}
\hypersetup{
  pdftitle={\systemName: Multimodal Font Design Using Interactive Bayesian Optimization},
}


\author{Yuki Tatsukawa}
\orcid{0009-0003-5128-8032}
\affiliation{%
 \institution{The University of Tokyo}
 \country{Japan}
}
\author{I-Chao Shen}
\orcid{0000-0003-4201-3793}
\affiliation{%
 \institution{The University of Tokyo}
 \country{Japan}
}
\author{Mustafa Doga Dogan}
\orcid{0000-0003-3983-1955}
\affiliation{%
 \institution{Adobe Research}
 \country{Switzerland}
}
\author{Anran Qi}
\orcid{0000-0001-7532-3451}
\affiliation{%
 \institution{Centre Inria d'Université Côte d'Azur}
 \country{France}
}
\author{Yuki Koyama}
\orcid{0000-0002-3978-1444}
\affiliation{%
 \institution{National Institute of Advanced Industrial Science and Technology (AIST)}
 \country{Japan}
}
\author{Ariel Shamir}
\orcid{0000-0003-4201-3793}
\affiliation{%
 \institution{Reichman University}
 \country{Israel}
}
\author{Takeo Igarashi}
\orcid{0000-0002-5495-6441}
\affiliation{%
 \institution{The University of Tokyo}
 \country{Japan}
}





\renewcommand{\shortauthors}{Tatsukawa, et al.}


\begin{abstract}
Creating new fonts requires a lot of human effort and professional typographic knowledge.
Despite the rapid advancements of automatic font generation models, existing methods require users to prepare pre-designed characters with target styles using font-editing software, which poses a problem for non-expert users.
To address this limitation, we propose \systemName, a system that enables font generation without relying on pre-designed characters.
Our approach integrates the exploration of a font-style latent space with human-in-the-loop preferential Bayesian optimization and multimodal references, facilitating efficient exploration and enhancing user control.
Moreover, \systemName allows users to revisit previous designs, retracting their earlier choices in the preferential Bayesian optimization process.
Once users finish editing the style of a selected character, they can propagate it to the remaining characters and further refine them as needed.
The system then generates a complete outline font in OpenType format.
We evaluated the effectiveness of \systemName through a user study comparing it to a baseline interface.
Results from both quantitative and qualitative evaluations demonstrate that \systemName enables non-expert users to design fonts efficiently.


\end{abstract}

\begin{CCSXML}
<ccs2012>
   <concept>
       <concept_id>10003120.10003121</concept_id>
       <concept_desc>Human-centered computing~Human computer interaction (HCI)</concept_desc>
       <concept_significance>300</concept_significance>
       </concept>
 </ccs2012>
\end{CCSXML}

\ccsdesc[300]{Human-centered computing~Human computer interaction (HCI)}

\keywords{font design, outline fonts, human-in-the-loop, latent space exploration, novice user support tools, generative models, typography tools}


\begin{teaserfigure}
  \centering
  \includegraphics[width=\linewidth]{figures_pdf/teaser_v10.pdf}
  \caption{
  \systemName allows non-expert users to create a font without pre-designed characters through four key steps.
  (a) Users input multimodal data (text, images, font files) to construct a new search subspace.
  (b) Users repeatedly explore the search subspace recommended by Bayesian optimization or constructed by multimodal reference using a slider.
  (c) Users can propagate an edited character's style to the remaining characters and refine any unsatisfactory characters (\eg~``K'') by repeating tasks (a) and (b).
  (d) The system generates \textit{OpenType Font} (OTF) file.
  }
  \Description{(a) -- (b) --- (c)---}
  \label{fig:teaser}
\end{teaserfigure}

\maketitle


% humans are sensitive to the way information is presented.

% introduce framing as the way we address framing. say something about political views and how information is represented.

% in this paper we explore if models show similar sensitivity.

% why is it important/interesting.



% thought - it would be interesting to test it on real world data, but it would be hard to test humans because they come already biased about real world stuff, so we tested artificial.


% LLMs have recently been shown to mimic cognitive biases, typically associated with human behavior~\citep{ malberg2024comprehensive, itzhak-etal-2024-instructed}. This resemblance has significant implications for how we perceive these models and what we can expect from them in real-world interactions and decisionmaking~\citep{eigner2024determinants, echterhoff-etal-2024-cognitive}.

The \textit{framing effect} is a well-known cognitive phenomenon, where different presentations of the same underlying facts affect human perception towards them~\citep{tversky1981framing}.
For example, presenting an economic policy as only creating 50,000 new jobs, versus also reporting that it would cost 2B USD, can dramatically shift public opinion~\cite{sniderman2004structure}. 
%%%%%%%% 图1:  %%%%%%%%%%%%%%%%
\begin{figure}[t]
    \centering
    \includegraphics[width=\columnwidth]{Figs/01.pdf}
    \caption{Performance comparison (Top-1 Acc (\%)) under various open-vocabulary evaluation settings where the video learners except for CLIP are tuned on Kinetics-400~\cite{k400} with frozen text encoders. The satisfying in-context generalizability on UCF101~\cite{UCF101} (a) can be severely affected by static bias when evaluating on out-of-context SCUBA-UCF101~\cite{li2023mitigating} (b) by replacing the video background with other images.}
    \label{fig:teaser}
\end{figure}


Previous research has shown that LLMs exhibit various cognitive biases, including the framing effect~\cite{lore2024strategic,shaikh2024cbeval,malberg2024comprehensive,echterhoff-etal-2024-cognitive}. However, these either rely on synthetic datasets or evaluate LLMs on different data from what humans were tested on. In addition, comparisons between models and humans typically treat human performance as a baseline rather than comparing patterns in human behavior. 
% \gabis{looks good! what do we mean by ``most studies'' or ``rarely'' can we remove those? or we want to say that we don't know of previous work doing both at the same time?}\gili{yeah the main point is that some work has done each separated, but not all of it together. how about now?}

In this work, we evaluate LLMs on real-world data. Rather than measuring model performance in terms of accuracy, we analyze how closely their responses align with human annotations. Furthermore, while previous studies have examined the effect of framing on decision making, we extend this analysis to sentiment analysis, as sentiment perception plays a key explanatory role in decision-making \cite{lerner2015emotion}. 
%Based on this, we argue that examining sentiment shifts in response to reframing can provide deeper insights into the framing effect. \gabis{I don't understand this last claim. Maybe remove and just say we extend to sentiment analysis?}

% Understanding how LLMs respond to framing is crucial, as they are increasingly integrated into real-world applications~\citep{gan2024application, hurlin2024fairness}.
% In some applications, e.g., in virtual companions, framing can be harnessed to produce human-like behavior leading to better engagement.
% In contrast, in other applications, such as financial or legal advice, mitigating the effect of framing can lead to less biased decisions.
% In both cases, a better understanding of the framing effect on LLMs can help develop strategies to mitigate its negative impacts,
% while utilizing its positive aspects. \gabis{$\leftarrow$ reading this again, maybe this isn't the right place for this paragraph. Consider putting in the conclusion? I think that after we said that people have worked on it, we don't necessarily need this here and will shorten the long intro}


% If framing can influence their outputs, this could have significant societal effects,
% from spreading biases in automated decision-making~\citep{ghasemaghaei2024understanding} to reducing public trust in AI-generated content~\citep{afroogh2024trust}. 
% However, framing is not inherently negative -- understanding how it affects LLM outputs can offer valuable insights into both human and machine cognition.
% By systematically investigating the framing effect,


%It is therefore crucial to systematically investigate the framing effect, to better understand and mitigate its impact. \gabis{This paragraph is important - I think that right now it's saying that we don't want models to be influenced by framing (since we want to mitigate its impact, right?) When we talked I think we had a more nuanced position?}




To better understand the framing effect in LLMs in comparison to human behavior,
we introduce the \name{} dataset (Section~\ref{sec:data}), comprising 1,000 statements, constructed through a three-step process, as shown in Figure~\ref{fig:fig1}.
First, we collect a set of real-world statements that express a clear negative or positive sentiment (e.g., ``I won the highest prize'').
%as exemplified in Figure~\ref{fig:fig1} -- ``I won the highest prize'' positive base statement. (2) next,
Second, we \emph{reframe} the text by adding a prefix or suffix with an opposite sentiment (e.g., ``I won the highest prize, \emph{although I lost all my friends on the way}'').
Finally, we collect human annotations by asking different participants
if they consider the reframed statement to be overall positive or negative.
% \gabist{This allows us to quantify the extent of \textit{sentiment shifts}, which is defined as labeling the sentiment aligning with the opposite framing, rather then the base sentiment -- e.g., voting ``negative'' for the statement ``I won the highest prize, although I lost all my friends on the way'', as it aligns with the opposite framing sentiment.}
We choose to annotate Amazon reviews, where sentiment is more robust, compared to e.g., the news domain which introduces confounding variables such as prior political leaning~\cite{druckman2004political}.


%While the implications of framing on sensitive and controversial topics like politics or economics are highly relevant to real-world applications, testing these subjects in a controlled setting is challenging. Such topics can introduce confounding variables, as annotators might rely on their personal beliefs or emotions rather than focusing solely on the framing, particularly when the content is emotionally charged~\cite{druckman2004political}. To balance real-world relevance with experimental reliability, we chose to focus on statements derived from Amazon reviews. These are naturally occurring, sentiment-rich texts that are less likely to trigger strong preexisting biases or emotional reactions. For instance, a review like ``The book was engaging'' can be framed negatively without invoking specific cultural or political associations. 

 In Section~\ref{sec:results}, we evaluate eight state-of-the-art LLMs
 % including \gpt{}~\cite{openai2024gpt4osystemcard}, \llama{}~\cite{dubey2024llama}, \mistral{}~\cite{jiang2023mistral}, \mixtral{}~\cite{mistral2023mixtral}, and \gemma{}~\cite{team2024gemma}, 
on the \name{} dataset and compare them against human annotations. We find  that LLMs are influenced by framing, somewhat similar to human behavior. All models show a \emph{strong} correlation ($r>0.57$) with human behavior.
%All models show a correlation with human responses of more than $0.55$ in Pearson's $r$ \gabis{@Gili check how people report this?}.
Moreover, we find that both humans and LLMs are more influenced by positive reframing rather than negative reframing. We also find that larger models tend to be more correlated with human behavior. Interestingly, \gpt{} shows the lowest correlation with human behavior. This raises questions about how architectural or training differences might influence susceptibility to framing. 
%\gabis{this last finding about \gpt{} stands in opposition to the start of the statement, right? Even though it's probably one of the largest models, it doesn't correlate with humans? If so, better to state this explicitly}

This work contributes to understanding the parallels between LLM and human cognition, offering insights into how cognitive mechanisms such as the framing effect emerge in LLMs.\footnote{\name{} data available at \url{https://huggingface.co/datasets/gililior/WildFrame}\\Code: ~\url{https://github.com/SLAB-NLP/WildFrame-Eval}}

%\gabist{It also raises fundamental philosophical and practical questions -- should LLMs aim to emulate human-like behavior, even when such behavior is susceptible to harmful cognitive biases? or should they strive to deviate from human tendencies to avoid reproducing these pitfalls?}\gabis{$\leftarrow$ also following Itay's comment, maybe this is better in the dicsussion, since we don't address these questions in the paper.} %\gabis{This last statement brings the nuance back, so I think it contradicts the previous parapgraph where we talked about ``mitigating'' the effect of framing. Also, I think it would be nice to discuss this a bit more in depth, maybe in the discussion section.}






\section{Related Work}
%%%%%%%%%%%%%%
% Know-Item Retrieval and Query Simulation
%%%%%%%%%%%%%%
\subsection{Query Simulation and Know-Item Retrieval}

Query simulation methods have been used for various purposes, including document expansion \cite{nogueira2019docT5query} and synthetic test collection generation \cite{Rahmani24synthetic}. In the context of known-item retrieval, these methods have been explored to improve retrieval strategies \cite{OgilvieCallan03combining} and evaluation frameworks \cite{Azzopardi06testbeds, hagen2015corpus}.



%% Query Simulation
\textit{Simulating} the known-item queries has long been an active research area \cite{balog2006overviewWebclef, Azzopardi07SimulatedQueries, Kim09desktop, Elsweiler2011Seeding}.
Early work \cite{Azzopardi07SimulatedQueries} generated synthetic queries using term-based likelihood models, selecting query terms based on their likelihood within a randomly chosen document. Later studies adapted this approach for desktop search \cite{Kim09desktop} and email re-finding \cite{Elsweiler2011Seeding}, demonstrating its effectiveness for simulated evaluations of know-item retrieval models.
%
The \textit{validation} of these query simulators has also been a key focus.
System ranking correlation \cite{balog2006overviewWebclef}, retrieval score distribution comparisons \cite{Azzopardi07SimulatedQueries}, and synthetic versus human query resemblance \cite{Kim09desktop} have been used to assess their reliability.


While valuable, known-item search queries differ significantly from TOT queries, which are longer and more complex. Despite progress in simulating known-item queries, TOT retrieval remains unexplored. This paper bridges that gap by introducing novel TOT query elicitation methods and adapting established validation techniques \cite{zeigler2000theory} to ensure alignment with real-world queries, enabling scalable and accurate simulated evaluations.






%%%%%%%%%%%%%%
% TOT Datasets
%%%%%%%%%%%%%%
\subsection{TOT Datasets}
Several datasets have been developed to support research on TOT retrieval, primarily collected from online CQA platforms and focused on specific domains. MS-TOT \cite{arguello-movie-identification} was constructed from the \textit{IRememberThisMovie} website and human-annotated with tags in the Movie domain. It also includes qualitative coding of TOT queries and demonstrates significant room for improvement in current retrieval technologies for such information needs. Similarly, \citet{gameTOT} collected TOT queries from Reddit's \textit{/r/tipofmyjoystick} subreddit in the Game domain, providing coded tag information. Other datasets include Reddit-TOMT \cite{Bhargav-2022-wsdm}, focused on movies and books from Reddit's \textit{/r/tipofmytongue} subreddit; TOT-Music \cite{Bhargav23MusicTOT}, targeting the Music domain from the same subreddit; and Whatsthatbook \cite{lin-etal-2023-whatsthatbook}, sourced from \textit{GoodReads}, focused on the Book domain.



In response to the domain specificity of these datasets, recent efforts have aimed to expand TOT datasets across multiple areas. \citet{Meier21-complex-reddit} expanded to general casual leisure domains using data from six Reddit subreddits, including games, books, and music, although other identified domains, such as videos and people, remain underrepresented. Similarly, TOMT-KIS \cite{frobe2023-performance-pred} extended the collection from \textit{/r/tipofmytongue} by adapting \citet{Bhargav-2022-wsdm}'s approach with fewer filtering restrictions, resulting in 1.28 million TOT queries. However, only 47\% of these queries have identified answers, and the dataset continues to exhibit severe domain skewness toward a few topics. 


In this work, we develop and validate TOT query elicitation methods using the Movie domain for robust evaluation, then expand to Landmark and Person to assess applicability across underrepresented domains.



\section{System}
\begin{figure*}[h]
    \centering
    \includegraphics[width=.85\textwidth]{fig/SYSTEM_IMAGE_TEST_FLIPPED.png}
    \caption{HumorSkills System Diagram. Given an image, the system first extracts visual details with a visual language model, then performs visual humor ideation to analyze the image and propose humorous angles. It then generates ten potential conflicts that could be used to extrapolate the image into a relatable experience. The system then generates humor with and without the narratives, for diversity. A separate instance of the LLM trained to rank gen-Z humor ranks all the captions and returns the top five.}
    \Description{HumorSkills System Diagram}
    \label{fig:system}
\end{figure*}

HumorSkills is a system that takes an input image and outputs 5 image captions. 
The architecture has three key steps that mimic human skills needed for humor. \textit{Visual Detail Extraction}, is a step that describes the image in depth in order to make non-obvious observations about it. \textit{Narrative and Conflict Extrapolation} is a step that finds narratives not in the image that could be related to it, to expand the topic of jokes to things that are not just in the image but also analogous to it.  \textit{Fine-tuning} the joke generator with examples of good Gen-Z humor helps the jokes be more relatable to the target audience by using references, slang, topics, and insecurities that resonate with this group.

% first, 
% a \textbf{divergent stage} where the image is analysed and multiple observations, angle, alternative narratives and humorous angles are generated. 
% Second, a \textbf{generation stage} where two types of captions are generated: 1) captions focusing on image content directly 2) captions that bring in an outside narrative to the image, often bringing in outside references. It generates 15 captions of each type. (Figure 1 has examples Of the Content Focused, and Narrative Expanded Captions). Finally, a \textbf{ranking stage} where a separate AI agent selects the top 5 captions

The system generates two types of captions: image-focused captions which common directly on the content in the image, and narrative-driven captions. Variety is important to humor. Humor relies on surprise, and jokes that are too similar start to become more predictable. Additionally, with an infinite set of input images with different subjects and situations, there are more strategies needed to find a humorous angle that fits the content. 

% With caption-based humor, often the humor can be focused on finding something in the image that is inherently interesting. 

% For example, the caption ”little man really thought he could escape bedtime” relies solely on information in the image. However, some images don’t have something funny in and of themselves, and it’s easier to make a joke by bringing in a new unrelated angle. For example, ``the police chasing me when I'm broke and in debt to the tune of \$100,000 for student loans''. Generally, Images with people doing interesting things lend themselves to visual humor because there are many stories one could tell about it. However, for images with only static objects, it's more difficult to tell a story on only the objects, so bringing a new story in is another way to find humor. 

\subsection{AI Humor Generation Walk Through}
Figure \ref{fig:system} contains a visual diagram and example of intermediate outputs when generating captions for an image. We describe each phase and implementation in detail.  
% The main contribution of this paper is the evaluation, rather than the system, but it is still it is important to understand the mechanism used to generate humor.
% Although the individual components of the system are not totally, the combination of features including the

\subsubsection{Visual Detail Extraction}

The first phase of the system’s workflow involves the Visual Detail Extraction component, which utilizes GPT-4o’s vision capabilities to analyze the input image. This system incorporates a prompt that asks for a detailed paragraph that explains the who/what/where of the image, distinguishing between identifying the subject of the image, the main action of the image if it exists, and the background elements of the image. This component is responsible for extracting key visual elements such as objects, human expressions, background settings, and any notable aspects that could serve as the foundation for humor.

For instance, in the demolition site example from the system diagram, the system identifies a large industrial demolition excavator and a person with a hose spraying the demolition site. 

\subsubsection{Visual Humor Ideation}

On top of the visual detail extraction, the system ideates on possible humorous elements from the visual of the image. This incorporates an additional prompt using GPT-4o that intakes the image and asks it to identify and ideate on potential humorous visual elements in the image, whether they are directly humorous elements, such as funny facial expressions, or more analogous elements. For example, for the system diagram image, the system noted the visual contrast of the excavator and person, reminiscent of a David versus Goliath scenario, which provides a foundational metaphor for generating humorous captions. 

\begin{figure*}[b]
    \centering
    \includegraphics[width=.95\textwidth]{fig/Workflow.png}
    \caption{A diagram for how narrative extrapolation works}
    % \Description{}
    \label{fig:systemLines}
\end{figure*}

\subsubsection{Narrative and Conflict Extrapolation}

In this next step, the system generates a narrative and conflict framework by drawing upon common and relatable Gen Z experiences such as work, school, family, social interactions, relationships, and more. 
The system chains together the results of the previous steps, into a new prompt sent to GPT-4o. 
% The system prompts GPT-4o to 
% GPT-4o is utilized by incorporating the text description of the image and potential humorous elements of the image, then being prompted to generate relatable scenarios applicable to the image description from a list of common Gen Z experiences. 
The prompt contains the visual details, the visual humor ideation, and a list of common Gen Z experiences,  and the instruction to "generate narratives that reflect the essence of the image that is set within the framework of the Gen Z experience."
This narrative generation adds depth to the humorous captions by applying relatable themes and conflicts to the visual elements identified earlier.

For instance, our system diagram generates narratives such as “Tackling student loans”, "Group Project Disaster", and “Relationship Issues” based on the image, both of which are common experiences among those who identify as Gen Z. These particular narratives are likely inspired by the imagery of a disaster site, referring to how the effort of paying off student loans, attempting to complete group projects during school, or addressing relationship -- all of which can feel like disaster clean up. These relatable conflicts can transform the visual of a demolition scene -- a setting that is not particularly relatable -- into a relatable scenario that has the potential for humor, thereby expanding the humorous possibilities by connecting the visual input with broader life experiences.



\subsubsection{Humorous Caption Generation}

Following the narrative and conflict extrapolation, the system generates humorous captions in the generation stage using a fine-tuned version of GPT-3.5 trained on humorous Instagram comments. This involves producing captions through two distinct strategies: one focused on the visual humor of the image, and the other by bringing in the previously generated external narratives. Caption generation is segmented into two separate prompts utilizing the fine-tuned GPT-3.5 model. For captions without generated external narratives, the prompt asks to generate 15 humorous captions in the style of Gen Z that bases the generation off the visual extraction and visual humor ideation of the input image. For captions with the external narratives, the prompt also asks to generate 15 humorous captions in the style of Gen Z that bases the generation off the visual extraction and visual humor ideation of the input image, but also asks the system to incorporate the list of generated narrative/conflict extrapolations to base the humorous captions off of.

Image-focused captions rely solely on the visual details in the image, such as “bro out here getting paid \textdollar8 an hour to spray some water on some bricks,” which references the direct visual elements in the scene in order to generate a caption. This particular caption directly references the humor of the image, poking fun at the minimal impact of the person spraying water on bricks while an excavator clearly has more impact on the demolition site. Narrative-driven captions, on the other hand, introduce external references to add humor. For instance, a caption like “The entitled bro you tried to make the group presentation with” introduces an outside, exaggerated, interpretation of the scene from earlier, "Group Project Disaster." This caption takes the group project narrative and pairs it with the visual of the image, analogizing the person spraying the hose with minimal impact on the demolition site to an entitled person who has not done much to complete the group project. 

This variety between visual humor and narrative-driven humor is crucial because jokes that are too similar become predictable, losing their element of surprise. Additionally, humor strategies need to adapt to the varying content in input images. Some images lend themselves to humor based on their inherent visual details, while others require bringing in outside references to create a joke. For instance, an image of static objects might not be inherently funny, such as the demolition image shown in the system diagram, but a caption introducing an unrelated, exaggerated narrative, such as “Eboy doin' his part to stop climate change” can inject humor and absurdity by making an unexpected connection.

\subsubsection{Caption Ranking using Gen Z Agent}

The final component of the system architecture is the Caption Ranking and Filtering Agent, a GPT-4o-based agent fine-tuned to evaluate humor from a Gen Z perspective. This agent receives the list of 30 total captions from the narrative and visual humor-based caption generations and ranks the captions generated in the previous stage based on humor, relatability, and alignment with the image and narrative.

As illustrated in our system diagram, this agent ranks captions such as “Me mopping up my last relationship” and “me pulling the emotional weight of the friend group” based on their relevance to Gen Z humor. Captions that fail to meet the humor threshold are filtered out, such as "Demolition worker really said 1v1 me bro," because although the phrase like "1v1 me bro" invokes Gen Z phrases, the content of the caption seems less relevant and relatable than a caption talking about school or relationships, ensuring that only the most effective and relatable captions are presented to the user.

\subsubsection{Fine-tuning}

To fine-tune a GPT-3.5 model, a dataset of 80 humorous comments were extracted from popular Instagram images. From three popular Instagram meme pages with over 400,000 followers, the top five comments of each image post were collected. All fit the style of Gen-Z humor. 
% The fine-tuning process ensured that the generated captions align with the humor style favored by Gen Z. 
Examples of the visual description of the images in addition to an explanation of potential humorous elements of the image were written in the fine-tuning prompts, then followed by the actual comment itself. This reflected the visual extraction and humor ideation being incorporated into the prompt of our current system.



\section{Method}
\subsection{Preliminary of Font Generative Model}
\label{sec:fontgenerativemodel}
We use \textit{DG-Font}~\cite{XieDGFont2021} as our font generative model. 
This model takes a style image $I_{S}$ representing the desired font style, and a content image $I_{C}$ representing the desired character, as input.
It then generates the character image that represents the desired character in the desired font style as the output. 
As illustrated in \autoref{fig:dg-font}, the architecture of \textit{DG-Font} is an encoder-decoder model with two encoders: a style encoder $E_{S}$ and a content encoder $E_{C}$, along with a content decoder $G_{C}$.
The generation process starts by extracting the style latent vector from the style image using the style encoder and the content latent vector from the content image.
Then, the content decoder takes both the style and content latent vectors as input to generate the desired font image $\hat{I}$, which maintains a similar style to the style image while preserving the structure of the content image.
Overall, the generation process during the training process can be formulated as:
\begin{align}
\bm{z}_S = E_{S}(I_S),\quad \bm{z}_C = E_{C}(I_C), \quad 
\hat{I} = G_C(\bm{z}_S, \bm{z}_C).
\end{align}
In our work, we use an enhanced version of \textit{DG-Font}, which includes an additional content discriminator.
Notably, our system (\systemName) only needs the pretrained model to generate new fonts instead of training a model from the beginning.
We will provide the details of this additional model architecture and training process in the supplemental material.

In the rest of the paper, if we need to specify the content image $I_C$ or its latent vector $\bm{z}_C$ for a specific character such as ``A'', we will denote it as  $I_C[\text{``A''}]$ $\bm{z}_C[\text{``A''}]$.
Otherwise, we will use $I_C$ or $\bm{z}_C$ for abbreviation. 
This also applied to the generated image $\hat{I}$.

\begin{figure}[ht]
    \centering
    \includegraphics[width=\linewidth]{figures_pdf/dgfont_v8.pdf}
    \caption{   
    \textbf{Overview of \textit{DG-Font}.} \textit{DG-Font} is an encoder-decoder model that takes a character image representing style and a character image representing content as input, and outputs a character image that combines the content with the specified style.
    During font designing in our system, users use our human-in-the-loop optimization to explore the style latent space of the style encoder.
    Please find the detailed architecture of the encoder and decoder in the supplemental material.
    }
    \label{fig:dg-font}
\end{figure}


\subsection{Preliminary of FontCLIP}
To incorporate multimodal input when designing fonts, we use FontCLIP~\cite{tatsukawa2024fontclip} to extract typographical features from both text and image input.
FontCLIP is a visual-language model that
bridges the semantic understanding of a large vision-language model with typographical knowledge.
It consists of a text encoder and a visual encoder and builds a joint latent space that encodes typographical knowledge.
In this joint latent space, similar font images and text prompts will have similar latent vectors.
For example, a bold font will have a similar latent vector to the text prompt ``This is a \textit{strong} and \textit{thick} font'' compared to the text prompt ``This is a \textit{thin} font''.
Therefore, the FontCLIP latent vector can be used to retrieve similar fonts using text or image input.
In our system, we utilize both the FontCLIP text encoder and visual encoder to extract a latent vector from the multimodal input to customize the linear subspace.

\subsection{Preliminary of Human-in-the-Loop Bayesian Optimization}
\label{method:prelim_BO}
\subsubsection{Problem Formulation}
Human-in-the-loop optimization is a computational approach used to solve parameter optimization problems involving human evaluators in its iterative algorithm.
It is commonly used to support design tasks that involve generating visual content defined by a set of parameters $\bm{x}$, with the aim of achieving certain subjective design goals.
Specifically, we can formulate such an optimization problem as:
\begin{align}
\bm{x}^* = \argmax_{\bm{x}\in{\mathcal{X}}}f(\bm{x}),
\label{eq:opt_goal}
\end{align}
where $\mathcal{X}$ is the search space, and $f: \mathcal{X} \rightarrow \mathbb{R}$ is the goodness function to evaluate a subjective design goal (\eg~the aesthetics of the current design).
We aim to find the optimal value $\bm{x}^*$ with the fewest trials because evaluating $f(\cdot)$ is costly.
However, solving \autoref{eq:opt_goal} using traditional Bayesian optimization (BO) might not be suitable for many design tasks.
This is because it is often difficult to assign an exact value to a sample, whereas comparing a couple of samples and choosing the preferred one is more intuitive.
For example, it is hard for users to give a score to a font individually, but easier for them to choose the preferred font from a set of candidates.
Therefore, in this work, we choose to use preferential Bayesian optimization (PBO) \cite{Koyama2022}, which is a variant of Bayesian optimization (BO) that runs with relative preferential data.
In particular, we build our method upon Sequential Line Search (SLS) \cite{KoyamaSequential2017}, a PBO method that constructs a sequence of linear subspaces that leads to the optimal parameters that match the user's need.


\subsubsection{Sequential Line Search (SLS)}
\label{sec:sls}
With SLS, the user can search for his/her preference by adjusting a slider.
At each iteration of the optimization, SLS constructs a linear subspace using the current-best position $\bm{x}^{+}$ and the best-expected-improving position $\bm{x}^{\text{EI}}$.
Suppose we already have $t$ observed response so far, then the next linear subspace $\mathcal{S}_{t+1}$ is constructing by connecting:
\begin{align}
    \bm{x}^\text{EI}_t &= \argmax_{\bm{x}\in{\mathcal{X}}}a^\text{EI}(\bm{x};\mathcal{D}_t) \label{eq:build_subspace} \\ \nonumber
    \bm{x}^+_t &= \argmax_{\bm{x}\in{\left\{\bm{x}_i\right\}_{i=1}^{N_t}}}\mu_t(\bm{x})
\end{align}
where $\left\{\bm{x}_i\right\}_{i=1}^{N_t}$ denotes the set of points observed so far, $\mu_t$ and $a^\text{EI}$ are the predicted mean function and the acquisition function calculated from the current data.
We use the expected improvement (EI) criterion as the acquisition function to choose the next sampling point that is likely to optimize the function $f$ and at the same time its evaluation is more informative:
\begin{align} \label{equation:acquisition}
    a^{\text{EI}}(\bm{x} ; \mathcal{D})=\mathbb{E}\left[\max \left\{f(\bm{x})-f^{+}, 0\right\}\right].
\end{align}

After the $t$-th iteration, we suppose that we obtained a set of $t$ single slider responses, which is represented as
\begin{align}
    \mathcal{D}_t=\left\{\bm{x}_i^\text{chosen}>\left\{\bm{x}_{i-1}^{+}, \bm{x}_{i-1}^{\text{EI}}\right\}\right\}_{i=1}^t,
\label{eq:response_data}
\end{align}
where $\bm{x}_i^{\text{chosen}}$ represent the position chosen by the user at the $t$-th iteration.

Let $f_i$ be the goodness function value at a data point $\bm{x}_i$, i.e., $f_i = f(\bm{x}_i)$, and $\bm{f}$ be the concatenation of the goodness values of all data points:
\begin{align}
\bm{f} = [f_1, f_2, \ldots, f_N].
\end{align}
Under the assumption of Gaussian process (GP) prior on $f$, we use $\bm{\theta}$ to represent the hyperparameters of the multivariate Gaussian distribution.
Since the goodness values $\bm{f}$ and the hyperparameters $\bm{\theta}$ are correlated, we infer $\bm{f}$ and $\bm{\theta}$ jointly by using MAP estimation:
\begin{align}
(\bm{f}^{\text{MAP}}, \bm{\theta}^{\text{MAP}})&=\argmax_{(\bm{f},\bm{\theta})}p(\bm{f}, \bm{\theta} \mid \mathcal{D}) \nonumber \\
&= \argmax_{(\bm{f},\bm{\theta})}p(\mathcal{D} \mid \bm{f}, \bm{\theta}) p(\bm{f}\mid \bm{\theta}) p(\bm{\theta}).
\label{eq:map}
\end{align}

Once $\bm{f}^{\text{MAP}}$ and $\bm{\theta}^{\text{MAP}}$ have been estimated, we can compute $\mu(\cdot)$, $\sigma(\cdot)$, and $a^{\text{EI}}(\cdot)$ in order to construct the next slider subspace $\mathcal{S}_{t+1}$.
We only describe the minimum concept of how SLS constructs the linear subspace to optimize the function $f$ for understanding how we incorporate it in the style latent space of a font generative model and multimodal input.
For more details, please refer to the supplemental material.


\subsection{Multimodal Bayesian Optimization for Font Generation}
\label{sec:multiModalBayesianOptimization}
Specifically, following \autoref{eq:opt_goal}, the objective function for designing a character can be formulated as:
\begin{align}
\bm{z}^* = \argmax_{\bm{z}\in{\mathcal{Z}}}f(G(\bm{z})), \label{eq:argmax_latent_space}
\end{align}
where $\bm{z} \in\mathcal{Z}$ is the style latent vector of the font generative model, which is the search space $\mathcal{X}$ in our problem setting.
Moreover, $G$ is the decoder of the font generative model, and $f: \mathcal{Z} \rightarrow \mathbb{R}$ is the user preference function that measures how good the currently generated character is perceived by the user.
To solve \autoref{eq:argmax_latent_space} and obtain the desired font, users can perform three different actions: \textit{explore the font style latent space}, \textit{retract previous preferences}, and \textit{propagate style to other characters} at each iteration.

\subsubsection{Action 1: Explore the Font Style Latent Space}
\label{sec:explorationWithSingleSlider}
\begin{figure}[t]
    \centering
    \includegraphics[width=\linewidth]{figures_pdf/exploreWithBayesianOptimization_v7.pdf}
    \caption{
    \textbf{Exploration of the font style latent space using a single slider.} 
    (a) Users explore a one-dimensional search subspace within the font style latent space using a single slider.
    At each iteration, users choose a point in the latent subspace and submit it as their current preference $\bm{z}^{\text{chosen}}_t$.
    After a couple of iterations, users gradually converge to their desired font style.
    The overall exploration process, users can explore (b) BO subspace only, (c) multimodal-guided subspace only, and (d) combination of both. 
}
    \label{fig:ExploreWithBayesianOptimization}
\end{figure}

The primary task for users is to explore the one-dimensional font-style search subspace using a slider.
By repeating the slide manipulation, users gradually converge on a point that aligns with their desired font style as illustrated in \autoref{fig:ExploreWithBayesianOptimization}.
This subspace is constructed by the system in two ways:
one solely follows the SLS method and another utilizes multimodal references.
Note that, regardless of these two different approaches to constructing the subspace, the interaction (\ie~manipulating the slider and submitting a preferred point to the system) remains consistent.

\paragraph{Constructing a SLS subspace}
Using the SLS method outlined in \autoref{sec:sls}, our system constructs a linear subspace $\mathcal{S}_{t}$ by connecting the current best point $(\bm{z}^+_t)$ and the point that maximizes the acquisition function $(\bm{z}^\text{EI}_t)$ using \autoref{eq:build_subspace} at the $t$-th iteration.
Then, users can choose a style latent vector $\bm{z}$ using the slider within $\mathcal{S}_{t}$ and view the generated character $G(\bm{z})$.
Once satisfied, users submit their preferred point on the slider $\bm{z}^{\text{chosen}}_t$ by clicking the \textsc{Update} or \textsc{Update all} button and request the system to construct a new linear subspace $\mathcal{S}_{t+1}$ for the $(t+1)$-th iteration using \autoref{eq:map}.


\paragraph{Constructing a multimodal-guided subspace}
While exploration with a single slider is useful, the linear subspace determined solely by Bayesian optimization sometimes fails to capture user preferences effectively, which leads to an increasing number of iterations and potentially causes frustration and a diminished sense of agency.
To address this issue, we allow users to intervene in the linear subspace construction by providing multimodal references at any iteration.
At $(t+1)$-th iteration, instead of exploring the linear subspace $\mathcal{S}_{t+1}=(\bm{z}^+_t, \bm{z}^\text{EI}_t)$ constructed solely by Bayesian optimization, the user explores the multimodal-guided subspace: $\mathcal{S}^{mm}_{t+1} = (\bm{z}^+_t, \bm{z}^{mm}_t)$.
Here, $\bm{z}^{mm}_t$ is the style latent vector obtained by retrieving the most similar fonts to the multimodal reference provided at $(t+1)$-th iteration from a font database containing \num{1,169} Roman fonts collected by O'Donovan~\etal~\cite{o2014exploratory}.
Specifically, we retrieve $n$ fonts and use the mean of their latent vectors as $\bm{z}^{mm}_t$ (we use $n = 5$ in our implementation).
Once the user is satisfied with the current generated character, the slider response: ($\bm{z}_{t+1}^\text{chosen},\bm{z}_{t}^{+}, \bm{z}_{t}^{mm}$) will be recorded in $\mathcal{D}_{t+1}$ (\autoref{eq:response_data}) and used for constructing the subspace in the future iteration.
This means that all multimodal references provided until iteration $t$ will affect the subspace constructed at $(t+1)$-th iteration.
In our current implementation, at each iteration, users can provide only a single multimodal reference, and we construct a new search subspace by connecting the current chosen point and the latent vector of the provided multimodal reference.
At $0$-th iteration, we construct the initial linear subspace $\mathcal{S}_{0}=(\bm{z}^{\text{init}}, \bm{z}^{mm}_{0})$, where $\bm{z}^{\text{init}}$ represents the style latent vector of a commonly used font (we use ``IPAex gothic'' font in our implementation).
Note that the multimodal references are only used to construct the linear subspace for the user to explore, not being directly used to infer the user preferences.r

To construct a multimodal subspace with the user-provided multimodal reference, our system identifies a suitable font in our font database that corresponds to the input text or image. 
For text input, the system first extracts font attributes such as ``formal,'' ``italic,'' and ``happy'' using a Large Language Model (LLM). 
The LLM selects relevant font attributes based on the given text.
We utilize $37$ types of font attributes compiled by O'Donovan~\etal~\cite{o2014exploratory} (see the supplemental material for more details).
Once the font attributes are extracted, the system obtains the feature vector of these text attributes using the text encoder of FontCLIP~\cite{tatsukawa2024fontclip} and retrieves fonts from our font database based on feature similarity. 
The retrieved fonts are used as the most suitable options, and the mean of their style latent vectors $\bm{z}^{mm}$ is obtained using the style encoder $E_{S}$.

For the text-rendered image reference, our system retrieves its most similar font in the font database using the FontCLIP feature.
Similarly, we then project the retrieved font image into the style latent space using $E_{S}$ and obtain $\bm{z}^{mm}$.
Finally, for font file input, our system directly uses the provided font as the suitable font and projects it into the latent space, similar to the process for text and image inputs.
\autoref{fig:multi_modal_subspace}(c) illustrates how to obtain the style latent vector of the multimodal references.


\begin{figure*}[ht]
    \centering
    \includegraphics[width=0.95\linewidth]{figures_pdf/multi_modal_subspace_v6.pdf}
    \caption{
    \textbf{Constructing linear subspaces using multimodal references.} 
    (a) At the start of the font design process using our proposed method, the user inputs text, an image, or a font file. 
    The system encodes this input into a font style latent vector and initializes the line search space by connecting the latent vector and a fixed point predetermined by the system.
    (b) Additionally, the user can introduce multimodal inputs at any stage of the design process.
    When the user provides new input, the system generates a new line search subspace by connecting the last user preference point with the newly encoded point.
    (c) Our system encodes multimodal input into the style latent space by leveraging LLM and FontCLIP text and visual encoders.
}
    \label{fig:multi_modal_subspace}
\end{figure*}

\subsubsection{Action 2: Retract Previous Preferences}
At each iteration, if the user is not satisfied with the current design and the recommended candidates, they can choose to retract previous preferences.
Specifically, at $(t+1)$-th iteration, if the user wishes to retract the last two slider manipulations, then the last two slider responses stored in $\mathcal{D}_t$ will be discarded.
Then, if the user opts to explore a new subspace by providing a new multimodal reference, they will then explore the subspace $\mathcal{S}^{mm}_{t-1} = (\bm{z}^+_{t-2}, \bm{z}^{mm}_{t-2})$.
Otherwise, the user will explore a subspace constructed using SLS solely: $\mathcal{S}_{t-1} = (\bm{z}^+_{t-2}, \bm{z}^{\text{EI}}_{t-2})$.
By retracting previous preferences, users can update their preferences during the design process.

\subsubsection{Action 3: Propagate Style to Other Characters}
Once the user is satisfied with the design of the focused character (\eg~``A'') and obtains the style latent vector $\bm{z}_S^{*}$, our system can propagate the style to all other characters and finish designing a single character.
Specifically, to propagate the style vector to character ``B'':
\begin{align}
\hat{I}[\text{``B''}] = G_C(\bm{z}_S^{*}[\text{``A''}], \bm{z}_C[\text{``B''}]).
\end{align}
Next, users can check all characters with the propagated styles.
If they are unsatisfied with the result of another character, they can perform action 1 or action 2 to design that character.
If they are satisfied, the resulting font is exported as an outline font.

\section{evaluation}
% justification
We conducted an exploratory evaluation of \textit{Polymind}, focusing on the usability, creativity, and usefulness of its parallel collaboration workflow.
More specifically, the study aimed to answer the following two questions:
\begin{enumerate}
    \item Is \textit{Polymind} easy to use, and useful for prewriting?
    \item How effective are \textit{Polymind}'s parallel collaboration workflow and microtasking features for supporting creativity in prewriting?
\end{enumerate}

% 
\begin{table}[htbp!]
\resizebox{\columnwidth}{!}{%
\begin{tabular}{@{}l|ccc|c@{}}
 & Liberal & Moderate & Conservative & Total \\ \hline
Female & 223 & 114 & 45 & 382 \\
Male & 102 & 78 & 53 & 233 \\
Prefer not to say & 2 & 0 & 0 & 2 \\ \hline
Total & 327 & 192 & 98 & 617
\end{tabular}%
}
\caption{Annotator Demographics. All annotators are based in the United States. The table shows the number of annotators across ideology and sex categories, as self-reported to Prolific. The mean age is 38.3 (SD=12.7), and 45 annotators are immigrants (7.3\%).}
\label{tab:demographics}
\end{table}



\subsection{Participants}
We used convenience sampling to recruit 10 participants (4 male, 6 female) mainly from local universities. All participants are L2 English speaker.
\revision{Similar to our formative study, we mainly reached out to participants with creative writing experience or related majors, though not necessarily expert writers. All participants had experience using prewriting or diagramming tools such as Figma or Miro, but reported limited knowledge or experience of AI or programming.}
We refer to them as V1-10. For each participant, we offered a coupon equivalent to 50 HKD.

\subsection{Study Design}

\subsubsection{Tasks}
To evaluate our \textit{Polymind} in both divergent and convergent thinking phases, we divide a creative pre-writing task into two sessions: story ideation, and story outlining. In the story ideation, participants were required to brainstorm as many distinct storylines as possible. Each storyline only needs to be one or two sentences long that specifies main characters, events, locations, time, etc. In the story outlining session, participants were asked to pick one favourite storyline from the previous session, and draft a rough outline with as many details as possible. An outline needs to specify a clear structure (such as the classic beginning-climax-ending structure), and key events along the structure.

\subsubsection{Conditions}
The study compared our system, \textit{Polymind} to a turn-taking, conversational interface, plus \textit{Polymind}'s diagramming canvas, as a baseline using two creative writing prompts. \revision{The baseline allows users to take notes and keep track of conversational results using the canvas, but does not require nor allow users to interact with the AI via diagrams}. Specifically, two system conditions were used:
\begin{itemize}
    \item \textbf{GPT-4 \& Canvas} OpenAI ChatGPT-4 interface plus \textit{Polymind}'s diagramming canvas. All microtasking features were turned off.
    \item \textbf{\textit{Polymind:}} full version of \textit{Polymind} with six predefined default microtasks.
\end{itemize}
Two creative writing prompts were chosen:
\begin{itemize}
    \item Write a story where your character is traveling a road that has no end, either literally or metaphorically.
    \item Write a story in which a character is running away from something, literally or metaphorically.
\end{itemize}
We used a Latin square experimental design~\cite{ryan2007modern} to achieve a balanced sequence of writing prompts and system conditions.

\subsubsection{Study Procedure}
After giving consent to our study, users were invited to use two systems in turn to complete two sessions: story ideation and story outlining, given two different writing prompts. Each session lasted 12 minutes, and users were given time to transfer their prewriting results (generations, diagrams, or merely thoughts and ideas in their minds) to another document after each session concluded. Before using \textit{Polymind}, we walked the participants through all of its features and offered approximately 10 minutes for them to try out the system.

After two sessions concluded for a system condition, each participant was required to complete a survey, including a NASA Task Load Index (NASA-TLX)~\cite{hart1986nasa}, and three dimensions (2 questions each) of Creativity Support Index (CSI)~\cite{cherry2014quantifying}: Enjoyment, Exploration, and Expressiveness. Two dimensions (Collaboration \& Immersion) of the original CSI were dropped because they were irrelevant to the two questions we sought to answer.

To evaluate the final results (outlines), we invited two expert writers to score participants outlines using Torrance Test of Creative Writing (TTCW)~\cite{chakrabarty2023art}, instead of the self-rated score of CSI. \revision{We did a quick interview after the scoring to ask about their general feedback, and to compare results of two conditions and pick out examples with most noticeable differences.}
One of our expert is a professional fiction writer and has a doctoral degree in film studies. She used to be a screenwriter before becoming a fiction writer. The other expert is an AO3 (Archive of Our Own) writer that has posted over 400K words and accumulated over 250K views.

After using two system conditions (4 sessions in total), each participant was then required to complete a survey to rate the usefulness of each \textit{Polymind} features. We then conducted a brief interview (5-10 min) to ask about their overall use experience, feedback on \textit{Polymind}'s workflow, perceptions of creativity support, and perceived differences between the two workflows and their impact on the final results.

The whole study procedure lasted around 2 hours, and was screen recorded. The interviews were audio-taped and transcribed for analysis.

% % \begin{figure}[htb]
% \begin{subfigure}[h]{\linewidth}
%     \centering
%     \includegraphics[width=.64\linewidth]{figures/Polymind_NASA-TLX.png}
% \end{subfigure}
% \begin{subfigure}[h]{\linewidth}
%     \centering
%     \includegraphics[width=.64\linewidth]{figures/Baseline_NASA-TLX.png}
% \end{subfigure}
% \caption{NASA Task Load Index of the \textit{Polymind} and Baseline conditions (the lower, the better).}
% \label{fig:usability}
% \end{figure}

\begin{figure*}[htb]
\centering
\includegraphics[width=.64\linewidth]{figures/NASA_TLX.pdf}
\caption{NASA Task Load Index of \textit{Polymind} and Baseline conditions (the lower, the better).}
\label{fig:usability}
\end{figure*}

\subsection{Study Results}
In this subsection, we report the findings of our study. On balance, \textit{Polymind}'s parallel microtasking workflow granted more customizability and was more controllable. Therefore users reported a stronger sense of agency, ownership of results, and a higher level of expressiveness. The microtasking workflow could also help quickly expand idea trees through ``chaining''-like effects~\cite{wu2022ai}.

% \begin{figure}[htb]
% \begin{subfigure}[h]{\linewidth}
%     \centering
%     \includegraphics[width=.64\linewidth]{figures/Polymind_NASA-TLX.png}
% \end{subfigure}
% \begin{subfigure}[h]{\linewidth}
%     \centering
%     \includegraphics[width=.64\linewidth]{figures/Baseline_NASA-TLX.png}
% \end{subfigure}
% \caption{NASA Task Load Index of the \textit{Polymind} and Baseline conditions (the lower, the better).}
% \label{fig:usability}
% \end{figure}

\begin{figure*}[htb]
\centering
\includegraphics[width=.64\linewidth]{figures/NASA_TLX.pdf}
\caption{NASA Task Load Index of \textit{Polymind} and Baseline conditions (the lower, the better).}
\label{fig:usability}
\end{figure*}

\subsubsection{Usability \& Usefulness}
Despite efforts of microtask and diagram management and the potential learning curve, to our surprise, \textit{Polymind} was almost perceived as easy to use as the ChatGPT interface, and significantly reduced frustration towards generated results (as shown \autoref{fig:usability}). ChatGPT interface was demanding mainly due to efforts of digesting longer text information (e.g., V4-5), and typing and iteratively refining lengthy prompts (e.g., V1-2). \revision{For \textit{Polymind}, the main cause of demand was said by V2-3, \& V10 to be the efforts of mannually managing the canvas, adjusting its layout, and progressing through diagrams.}
Notably, V1 \& V2 said that \textit{Polymind}'s interface was easier to navigate, and easier to read, because its generations were mainly short phrases, and had structures (including lines, sections, \& microtask colours). 
Besides, the randomness of ChatGPT generations also caused higher frustration level and worse perceived performance than \textit{Polymind} among some participants (e.g., V5, V8).

The key features of \textit{Polymind} were mainly perceived useful for prewriting, as shown in \autoref{fig:usefulness}. Of them awareness-related features, such as notifications, previews, and initiative modes, were found most controversial, which revealed the tension of our \textit{Goal 2.1} and being overall non-intrusive. V3 felt a proactive microtask was particularly annoying and intrusive, but we observed that all other participants left key microtasks proactive. Some said (e.g., V5, V7) they would need proactive microtasks in divergent thinking phases for quick ideas, but sometimes did not want to be interrupted while thinking. Therefore, most participants thought the feature of switching intiative modes particularly helpful.

% \begin{figure}[ht!]
\centering
\begin{minipage}[b]{.48\linewidth}
    \vspace{0pt}
    \includegraphics[width=\linewidth]{figures/NASA_TLX.png}
    \caption{NASA Task Load Index of \textit{Polymind} and Baseline conditions (the lower, the better).}
    \label{fig:usability}
\end{minipage}
\begin{minipage}[b]{.5\linewidth}
    \vspace{0pt}
    \includegraphics[width=\linewidth]{figures/perceived_usefulness.png}
    \caption{The perceived usefulness of \textit{Polymind} features}
    \label{fig:usefulness}
\end{minipage}
\end{figure}
\begin{figure*}[htb]
\centering
\includegraphics[width=.85\linewidth]{figures/perceived_usefulness.png}
\caption{The perceived usefulness of \textit{Polymind} features}
\label{fig:usefulness}
\end{figure*}

% \begin{figure*}[htb]
% \begin{subfigure}[h]{.49\textwidth}
%     \includegraphics[width=.975\linewidth]{figures/Statistics of Resulting Diagrams - Task 1.pdf}
% \end{subfigure}
% \begin{subfigure}[h]{.49\textwidth}
%     \includegraphics[width=.975\linewidth]{figures/Statistics of Resulting Diagrams - Task 2.pdf}
% \end{subfigure}
% \caption{Number of resulting nodes and \textit{Polymind} contribution in two tasks}
% \label{fig:result_nodes}
% \end{figure*}

\subsubsection{Creativity Support}
In terms of creativity, participants generally felt \textit{Polymind} was more supportive (see \autoref{fig:CSI}), but the results were not significant ($P_{Enjoyment}=0.67$, $P_{Exploration}=0.19$, $P_{Expressiveness}=0.05$). Notably, the expressiveness dimension has almost shown significance, as many (e.g., V3 \& V5) reported that \textit{Polymind}'s diagramming interface and microtasking workflow put them in dominant roles that encouraged them to freely express their brief ideas. In terms of results, two conditions produced similar number of ideas ($Polymind_{median}=3$, $Polymind_{stdev}=1.06$, $Baseline_{median}=3$, $Baseline_{stdev}=8.51$) in the ideation session.

In addition, experts' scores showed that the baseline condition produced outlines that were able to pass 5.7 TTCW tests, as compared to \textit{Polymind}'s 4 tests ($P=0.23$) (see \autoref{fig:CSI}).
\revision{
Experts did not particularly mention any noticeable differences in quality between two conditions except that \textit{Polymind}'s results were much shorter and lacked details to pass some tests. Besides, they both expressed concern of overused or clichéd results. One expert said she was initially interested by V2's story (baseline), but only to find out that it was from \textit{The Vampire Diaries}
}
This is expected, as ChatGPT interface could quickly generate ``\textit{complete and detailed outlines with simple prompts}'' (V2), while \textit{Polymind} usually encouraged users to make progress in diagrams with limited words.
Although some (e.g., V8) noted that they could still generate a complete outline using \textit{Polymind}, but they simply did not want to, because they would like to take control, and create a story from their own fragmented ideas, instead of borrowing all results from ChatGPT.
\begin{figure}[ht!]
\centering
\begin{minipage}[t]{.4745\linewidth}
    \vspace{0pt}
    \includegraphics[width=\linewidth]{figures/CSI.png}
\end{minipage}
\begin{minipage}[t]{.32\linewidth}
    \vspace{0pt}
    \includegraphics[width=\linewidth]{figures/TTCW.png}
\end{minipage}
\caption{The results of Creativity Support Index (CSI) and Torrance Test of Creative Writing (TTCW)}
\label{fig:CSI}
\end{figure}

\subsubsection{Microtask Usage: Quick Chaining and Idea Expansion}
\revision{All default microtasks have been applied by 10 participants to produce their final results, as shown in \autoref{tab:usage}. In some cases users might have default microtasks slightly edited. For example, V2 changed the prompt of \BboxS{\textcolor{white}{Brainstorm}} and switched the output type to \textbf{\textcolor{sticky_note}{\textit{sticky note}}} to request detailed settings of a story. Four participants have delegated a total of 8 customized microtasks. For example, V8 delegated \CboxS{\textcolor{white}{Beginning}} \& \CCboxS{\textcolor{white}{Climax}} to generate a beginning and climax of a given storyline. V4 delegated \CboxS{\textcolor{white}{Juice}} to juice up a given story in a \textbf{\textcolor{sticky_note}{\textit{sticky note}}} with more details.}

\revision{We also found participants came up with creative and efficient ways of using a combination of microtasks}. By leaving some microtasks in the proactive mode, \textit{Polymind} can easily perform the ``chaining'' operation~\cite{wu2022ai} to expand users' ideas in a tree-like structure. During this process, multiple distinct microtasks could contribute simultaneously in parallel to users' main operations, which made the collaboration more efficient and creative.
Some participants complimented that the parallel microtasks were like ``\textit{a mature pipeline that needs little efforts}'' (V5), ``\textit{as if splitting (brainstorming) indefinitely}'' (V6). For example, V2 mainly used two proactive microtasks: \FboxS{\textcolor{white}{Freewrite}} \& \SboxS{\textcolor{white}{Summarise}} during the story ideation session.
\revision{He later explained that \FboxS{\textcolor{white}{Freewrite}} was used to quickly generate stories in a \textbf{\textcolor{sticky_note}{\textit{sticky note}}} given a few keywords or concepts within a \textbf{\textcolor{section}{\textit{section}}}, while \SboxS{\textcolor{white}{Summarise}} presented brief summaries in a \textbf{\textcolor{sticky_note}{\textit{sticky note}}} of \FboxS{\textcolor{white}{Freewrite}}'s generations so that he would not need to read whole stories.}

V10 instead was mainly using \BboxS{\textcolor{white}{Brainstorm}} and \EboxS{\textcolor{white}{Elaborate}} to expand her ideas in brief keywords and concepts. \revision{She used \BboxS{\textcolor{white}{Brainstorm}} to request related ideas and \EboxS{\textcolor{white}{Elaborate}} to provide concrete examples of an idea.} In a comparison to the ChatGPT interface, she commented that,
\begin{quote}
    ``\textit{I feel that ChatGPT often generated something irrelevant, and missed my expectations. But this system (Polymind) stuck to my main concept by generating relevant ideas. Although the results were only brief keywords, but it was fast. It could produce a huge idea tree within a short period of time, and you could easily find something intriguing and figure out a coherent story.}''.
\end{quote}

\newtcbox{\Bbox}{on line,
  colframe=brainstorm,colback=brainstorm,
  boxrule=0.5pt,arc=1pt,boxsep=0pt,left=2pt,right=2pt,top=2pt,bottom=2pt}
\newtcbox{\Sbox}{on line,
  colframe=summarise,colback=summarise,
  boxrule=0.5pt,arc=1pt,boxsep=0pt,left=2pt,right=2pt,top=2pt,bottom=2pt}
\newtcbox{\Ebox}{on line,
  colframe=elaborate,colback=elaborate,
  boxrule=0.5pt,arc=1pt,boxsep=0pt,left=2pt,right=2pt,top=2pt,bottom=2pt}
\newtcbox{\Dbox}{on line,
  colframe=draft,colback=draft,
  boxrule=0.5pt,arc=1pt,boxsep=0pt,left=2pt,right=2pt,top=2pt,bottom=2pt}
\newtcbox{\Fbox}{on line,
  colframe=freewrite,colback=freewrite,
  boxrule=0.5pt,arc=1pt,boxsep=0pt,left=2pt,right=2pt,top=2pt,bottom=2pt}
\newtcbox{\Abox}{on line,
  colframe=associate,colback=associate,
  boxrule=0.5pt,arc=1pt,boxsep=0pt,left=2pt,right=2pt,top=2pt,bottom=2pt}
\newtcbox{\Cbox}{on line,
  colframe=custom,colback=custom,
  boxrule=0.5pt,arc=1pt,boxsep=0pt,left=2pt,right=2pt,top=2pt,bottom=2pt}
\newtcbox{\CCbox}{on line,
  colframe=custom2,colback=custom2,
  boxrule=0.5pt,arc=1pt,boxsep=0pt,left=2pt,right=2pt,top=2pt,bottom=2pt}

\renewcommand{\arraystretch}{1.2}
\begin{table*}[htb]
    \resizebox{\linewidth}{!}{
    \begin{tabular}{c|cc|cc}
    \toprule
    & \multicolumn{2}{c|}{Task \RNum{1}} & \multicolumn{2}{c}{Task \RNum{2}} \\
    & default & custom & default & custom \\
    \midrule
    V1 & \Ebox{\textcolor{white}{Elaborate}} \Fbox{\textcolor{white}{Freewrite}} & \Cbox{\textcolor{white}{Characteristics}} & \Sbox{\textcolor{white}{Summarise}} \Fbox{\textcolor{white}{Freewrite}} & \\
    V2 & \Bbox{\textcolor{white}{Brainstorm}} \Sbox{\textcolor{white}{Summarise}} \Fbox{\textcolor{white}{Freewrite}} & & \Ebox{\textcolor{white}{Elaborate}} \Fbox{\textcolor{white}{Freewrite}} & \Cbox{\textcolor{white}{Structure}} \\
    V3 & \Bbox{\textcolor{white}{Brainstorm}} \Fbox{\textcolor{white}{Freewrite}} \Dbox{\textcolor{white}{Draft}} \Abox{\textcolor{white}{Associate}} & & \Bbox{\textcolor{white}{Brainstorm}} \Sbox{\textcolor{white}{Summarise}} \Fbox{\textcolor{white}{Freewrite}} & \\
    V4 & \Bbox{\textcolor{white}{Brainstorm}} & \Cbox{\textcolor{white}{Structure}} \CCbox{\textcolor{white}{Juice}} (up) & \Bbox{\textcolor{white}{Brainstorm}} & \Cbox{\textcolor{white}{Structure}} \CCbox{\textcolor{white}{Theme}} \\
    V5 & \Bbox{\textcolor{white}{Brainstorm}} \Fbox{\textcolor{white}{Freewrite}} \Abox{\textcolor{white}{Associate}} & & \Bbox{\textcolor{white}{Brainstorm}} \Fbox{\textcolor{white}{Freewrite}} \Abox{\textcolor{white}{Associate}} & \\
    V6 & \Bbox{\textcolor{white}{Brainstorm}} \Abox{\textcolor{white}{Associate}} & & \Bbox{\textcolor{white}{Brainstorm}} \Abox{\textcolor{white}{Associate}} & \\
    V7 & \Bbox{\textcolor{white}{Brainstorm}} & & \Bbox{\textcolor{white}{Brainstorm}} \Ebox{\textcolor{white}{Elaborate}} & \\
    V8 & \Bbox{\textcolor{white}{Brainstorm}} \Dbox{\textcolor{white}{Draft}} \Abox{\textcolor{white}{Associate}} & & \Bbox{\textcolor{white}{Brainstorm}} \Dbox{\textcolor{white}{Draft}} \Fbox{\textcolor{white}{Freewrite}} & \Cbox{\textcolor{white}{Beginning}} \CCbox{\textcolor{white}{Climax}} \\
    V9 & \Bbox{\textcolor{white}{Brainstorm}} \Dbox{\textcolor{white}{Draft}} \Abox{\textcolor{white}{Associate}} & & \Dbox{\textcolor{white}{Draft}} & \\
    V10 & \Bbox{\textcolor{white}{Brainstorm}} \Ebox{\textcolor{white}{Elaborate}} \Abox{\textcolor{white}{Associate}} & & \Bbox{\textcolor{white}{Brainstorm}} \Abox{\textcolor{white}{Associate}} & \\
    \bottomrule
    \end{tabular}}
    \caption{Microtasks used by each participant in \textit{Polymind} condition during two sessions that produced the final results. The input \& output types and prompts of default microtasks might have been edited.}
    \label{tab:usage}
\end{table*}

% \renewcommand{\arraystretch}{1.2}
% \begin{table*}[htb]
%     \resizebox{\linewidth}{!}{\begin{tabular}{c|cc}
%     \toprule
%     & Task \RNum{1} & Task \RNum{2} \\
%     \midrule
%     S1 & \Bbox{\textcolor{white}{Brainstorm}} \Fbox{\textcolor{white}{Freewrite}} \textbf{+} \Dbox{\textcolor{white}{Draft}} & \Bbox{\textcolor{white}{Brainstorm}} \Abox{\textcolor{white}{Associate}} \\
%     S2 & \Bbox{\textcolor{white}{Brainstorm}} \Ebox{\textcolor{white}{Elaborate}} \Fbox{\textcolor{white}{Freewrite}} \textbf{+} \Dbox{\textcolor{white}{Draft}} & \Bbox{\textcolor{white}{Brainstorm}} \Ebox{\textcolor{white}{Elaborate}} \\
%     S3 & \Bbox{\textcolor{white}{Brainstorm}} \Ebox{\textcolor{white}{Elaborate}} \textbf{+} \Dbox{\textcolor{white}{Draft}} & \Ebox{\textcolor{white}{Elaborate}} \textbf{+} \Dbox{\textcolor{white}{Draft}} \\
%     S4 & \Abox{\textcolor{white}{Associate}} \textbf{+} \Dbox{\textcolor{white}{Draft}} & \Bbox{\textcolor{white}{Brainstorm}} \Ebox{\textcolor{white}{Elaborate}} \Abox{\textcolor{white}{Associate}} \\
    
%     S5 & \Bbox{\textcolor{white}{Brainstorm}} \Abox{\textcolor{white}{Associate}} & \Ebox{\textcolor{white}{Elaborate}} \Fbox{\textcolor{white}{Freewrite}} \\
    
%     S6 & \Ebox{\textcolor{white}{Elaborate}} \Fbox{\textcolor{white}{Freewrite}} \Dbox{\textcolor{white}{Draft}} \textbf{+} \Cbox{\textcolor{white}{Custom}} & \Abox{\textcolor{white}{Associate}} \Bbox{\textcolor{white}{Brainstorm}} \Ebox{\textcolor{white}{Elaborate}} \textbf{+} \Cbox{\textcolor{white}{Custom1}} \CCbox{\textcolor{white}{Custom2}} \\
%     S7 & \Abox{\textcolor{white}{Associate}} \Dbox{\textcolor{white}{Draft}} \textbf{+} \Cbox{\textcolor{white}{Custom}} & \Cbox{\textcolor{white}{Custom}} \\
%     S8 & & \Bbox{\textcolor{white}{Brainstorm}} \textbf{+} \Cbox{\textcolor{white}{Custom}} \\
%     S9 & \Ebox{\textcolor{white}{Elaborate}} \Dbox{\textcolor{white}{Draft}} & \Bbox{\textcolor{white}{Brainstorm}} \Fbox{\textcolor{white}{Freewrite}} \Dbox{\textcolor{white}{Draft}} \\
%     S10 & \Bbox{\textcolor{white}{Brainstorm}} \textbf{+} \Cbox{\textcolor{white}{Custom}} & \Ebox{\textcolor{white}{Elaborate}} \\
%     S11 & \Bbox{\textcolor{white}{Brainstorm}} \Sbox{\textcolor{white}{Summarise}} \textbf{+} \Cbox{\textcolor{white}{Custom}} & \Bbox{\textcolor{white}{Brainstorm}} \Dbox{\textcolor{white}{Draft}} \\
%     S12 & \Bbox{\textcolor{white}{Brainstorm}} \Abox{\textcolor{white}{Associate}} \Fbox{\textcolor{white}{Freewrite}} \textbf{+} \Cbox{\textcolor{white}{Custom}} & \Abox{\textcolor{white}{Associate}} \Ebox{\textcolor{white}{Elaborate}} \Dbox{\textcolor{white}{Draft}} \\
%     S13 & \Bbox{\textcolor{white}{Brainstorm}} \Ebox{\textcolor{white}{Elaborate}} \Sbox{\textcolor{white}{Summarise}} \Fbox{\textcolor{white}{Freewrite}} \textbf{+} \Cbox{\textcolor{white}{Custom}} & \Bbox{\textcolor{white}{Brainstorm}} \Ebox{\textcolor{white}{Elaborate}} \Fbox{\textcolor{white}{Freewrite}} \\
%     S14 & \Bbox{\textcolor{white}{Brainstorm}} \Ebox{\textcolor{white}{Elaborate}} \Sbox{\textcolor{white}{Summarise}} & \Bbox{\textcolor{white}{Brainstorm}} \Ebox{\textcolor{white}{Elaborate}} \textbf{+} \Cbox{\textcolor{white}{Custom}} \\
%     \bottomrule
%     \end{tabular}}
%     \caption{Microtasks used by each participant in two tasks to produce the final results}
%     \label{tab:usage}
% \end{table*}

V3 was one of the three participants that showed clear preference for the ChatGPT interface, but she also added that experimenting ideas with \textit{Polymind} were much easier. She explained that \textit{Polymind} ``\textit{had a structure}'' and could perform chaining-like operations easily ``\textit{by using sections}'' and parallel microtasks, while for ChatGPT, ``\textit{combining elements (like some characters, events, or settings) from its generations to re-prompt it was challenging}''. Similarly, V6 noted that brainstorming associations between key events or scenes was much easier with \textit{Polymind} by ``\textit{using several proactive microtasks operating on nodes or sections}''.

\subsubsection{Polymind is More Controllable}
While ChatGPT-4's conversational interface was able to generate long pieces of text with many ideas and details (V2-4, V7-9), most of our participants (V1-2, V4-5, V7-8, V10) mentioned that \textit{Polymind} felt more controllable in a prewriting task. This is because \textit{Polymind} directly operated on diagrams that were often shorter and \revision{thus easier to digest and re-prompt} than ChatGPT-4's conversations, and the canvas progressed in a structured manner with parallel microtasks handling very specific requirements.

The longer generations of the conversational interface were often criticised for being hallucinatory (e.g., ``\textit{not that creative or sensible as it appears}'' -- V7), too random (e.g., ``\textit{not what I expected}'' -- V5, ``\textit{irrelevant}'' -- V10), or ``\textit{mediocre}'' (V2) during a story pre-writing session. In comparison, \textit{Polymind} generations were perceived by many to be relevant (V2, V5, V10), and its microtasks more responsive to users' requests (V5, V8, V10), although it might require some efforts to manage or configure them (V4). \revision{This echoes with the usability score of the two conditions, where \textit{Polymind} was on the same level with the baseline, despite the efforts of managing a diagramming interface.}
V8 said in retrospect that,
\begin{quote}
    ``\textit{I think this (Polymind) would be very helpful for coming up with a story. Cause you can specify your beginning, you can specify your climax, and the ending too... And I think customizing microtasks is also a nice feature... You can really outline everything. While for GPT, you often don't know what is beginning, what is climax or ending.}''
\end{quote}

Additionally, V1, V2 and V8 noted that \textit{Polymind}'s microtasking workflow made it easier to re-prompt. V8 said,
\begin{quote}
    ``\textit{If I want to change something, I know where the part is. Like the character, I only need to change several keywords, like, Oh I'd like the character to be a dragon... I felt that my prompts were actually considered (by microtasks), while GPT sometimes doesn't process all my prompts.}''
\end{quote}
For the conversational interface, it often took multiple iterations to reach a decent draft (V4-5), and each prompt had to be lengthy to change the context (V1-2, V5), which was demanding. V2 also added that the \textit{Polymind} interface was neater because with some parallel microtasks it required little to no efforts of note taking to ask multiple follow-up questions of different ideas.

It is worth noting that, three participants that disliked \textit{Polymind}'s workflows mentioned that it was quite demanding sometimes to configure microtasks (V3-4), and progress in diagrams (V9). V9 said, ``\textit{GPT could generate a lot with a single prompt, while \textit{Polymind} only little by little.}''
\revision{V10 shared similar sentiments. She noted prompting ChatGPT would be much easier than using \textit{Polymind}'s diagrammatic workflow if its generations were not random. However, she added that \textit{Polymind} was in reality less demanding because you could explore more options and easily drop random results.}

\subsubsection{Polymind Affords Agency}
Our participants almost unanimously said that \textit{Polymind} put users in a dominant role, while with the conversational interface they were completely guided by the GPT. This aligns with our \textit{Goal 2} that aims to put humans in a role of managing all microtasks. Participants without any ideas, such as V3 \& V4, generally did not mind following the ChatGPT. This is expected, and agrees with our formative study. However, same as almost all other participants, they particularly mentioned that they felt these results were not their ideas, as ChatGPT generated almost everything. While using \textit{Polymind}, participants said that they had more freedom and control (V3, V6-8), and needed to think a lot (V4-5, V8). 

One of the participants, V5, particularly said he had no trust in AI because ``\textit{it could not be truly creative}''. He therefore became very annoyed with the ChatGPT interface when it did not generate what he expected, saying it was ``\textit{bad usability}'', while attributing ``\textit{good usability}'' to the task management workflow. V7 also expressed concerns for using the conversational interface for brainstorming,
\begin{quote}
    ``\textit{At first sight, it might seem it had generated everything you could think of, but then you'd find many were indeed non-sensical. But I felt I was confined to these generations after reading them. It was especially hard for a novice writer like me to come up with other possibilities. So it felt like it was GPT that was composing a fiction, rather than me.}''
\end{quote}
She later added that her own results from \textit{Polymind} felt more ``\textit{logical}'', and ``\textit{rigorous}''.

Of the participants that said very positively of the ChatGPT interface, V3 stressed that \textit{Polymind} encouraged her to express her own ideas, while ChatGPT did not. That was why she assigned a very low score of expressiveness in the CSI survey when using ChatGPT.
\section{Application Demonstrations}
\label{sec:demonstration}
In this section, we will show a case study where participants are required to design characters in more realistic situations and the adaptation to another writing system rather than Roman characters.


\subsection{Designing Fonts for Graphic Design Purposes}
\label{sec:demonstration_graphic_design}
In practical usage, it is important to evaluate whether our system can support font design for graphic design purposes, such as logo design or advertisement design, as noted by professional designers in \autoref{sec:limitations}.
To explore this, we asked participants to create suitable characters for specific design contexts.

To begin with the conclusion, from the feedback, we observed that participants using our system did not initially have a clear vision of the font they wanted. 
However, as they explored different font styles, they drew inspiration from the designs they encountered, ultimately creating their own unique characters.
While participants occasionally struggled with fine adjustments, such as correcting distorted lines, they generally felt they were able to create fonts that aligned with their intended concepts.
In the following sections, we present two design scenarios: conference logo design and advertisement poster design.
In the conference logo task, four participants (P11--P14) created characters for a logo, demonstrating a variety of font styles using our system.
In the advertisement poster task, another six participants (P15--P20) designed characters for different posters, tailoring their fonts to the target concepts.

\subsubsection{Design a Conference Logo}
In this experiment, participants with no prior font design experience were tasked with designing a conference logo.
Specifically, they were asked to create the characters ``CHI 2025'' to complement the cherry blossom motif in the conference logo.
After receiving an introduction to using our system, participants completed the task, and their feedback was collected.
During the design process, participants were only shown the cherry blossom logo and were not aware of the characters in the official conference logo.
As shown in \autoref{fig:CHILogoDesign}, the designs varied among participants.
P11 noted that her designed characters complemented the cherry blossom logo, highlighting that her favorite aspect was the fading central lines in ``H'' and ``5.''
This fading part appeared accidentally but complements the logo from her point of view, so she adopted it.
She also attempted to replicate this effect in ``2,'' but it was unsuccessful.
P12 said that he thought a decent and calm font was suitable for the conference logo and tried to make such a font.
He also commented that the font he designed was $90$ out of $100$ in terms of satisfaction, though his attempt to make ``I'' more straight was not successful.
P13 commented that he aimed to create a cute font inspired by the Japanese subculture, opting for a bold and rounded design.
For the initial step, he input the text, ``I want a cute and thick font" and found that the system performed as he hoped.
He also noted that the slider was effective in fine-tuning character details and eliminating unwanted distortions.
He was proud of the font he designed and believed it could be used in real-world applications, as the style of each character was well aligned.
P14 noted that he thought a thin and brush-style font fitted the Japanese-style logo and tried to make it.
He found multimodal input effective for the early stages of rough font design but felt it was less suitable for detailed exploration, ultimately relying on slider manipulation to refine the characters.
He was confident with the quality except for the noise and distortion in the designed characters.

\begin{figure}[ht]
    \centering
    \includegraphics[width=0.95\linewidth]{figures_pdf/CHILogo_v7.pdf}
    \caption{
    \textbf{Designed characters for the conference logo.}
    Four participants designed the characters for the conference logo.
    They designed a diverse range of fonts based on their unique sensibilities.
}
\label{fig:CHILogoDesign}
\end{figure}



\subsubsection{Design Advertisement Posters}
This demonstration shows font design for advertisement posters using our system, as illustrated in \autoref{fig:poster}.
During the design process, participants (P15--P20), who were introduced to the use of our system, were given a scenario and shown only the background image, with the task of creating characters that matched the visual context.
P15 and P16, both familiar with CJK writing systems, designed characters for an autumn foliage festival.
P5 rated his design $9$ out of $10$, expressing satisfaction with the traditional and formal font style he aimed to achieve.
He efficiently initialized the search space by inputting the text ``yu-mincho, serif'' (with ``yu-mincho'' being one of the most popular CJK fonts).
P16 commented that he envisioned a calm and warm font for the festival and was pleased with the result. 
He noted that their initial idea was simply based on the keyword ``warm,'' which he input into the system.
As he explored various styles, he gradually refined his design and reached a point of satisfaction.
P17, who designed the summer sale poster, aimed for a thin and refreshing font.
He observed elements in the background image such as the central white line, the seagull's wings, and the wave’s border, and decided that the character weight should align with these features.
By adjusting the slider, he was able to find a suitable font weight, though he expressed some dissatisfaction with the distortion of the top horizontal bar in the letter ``E.''

P18, tasked with designing a Halloween poster, felt that a twisted font suited the Halloween theme. 
She also believed a cute, handwritten style complemented the surrounding elements like the pumpkin, house, and bat, and was satisfied with the bold characters she created. 
She began by inputting a bold and italic font into the system, then continued refining the design using only the slider suggested by the Bayesian optimization process.
She expressed confidence in using the system, despite having no prior font design experience, and enjoyed the process. 
She effectively utilized the system's features, such as reverting to previous iterations via the history area when her exploration veered in an undesired direction, and she repeatedly refined each character after the initial style propagation.
For details on P18's design process, refer to the supplemental material.
P19 designed characters for a birthday card, aiming for cursive and fashionable font, and expressed satisfaction with the result.
P20 created a font for a movie poster, aiming for characters that were "scary," "thick," and "retro," with shapes fitting within a rectangular form (e.g., the shape of "S" resembling a rectangle).
While he was generally satisfied with the overall design, he found it challenging to achieve symmetry in characters like "A," "M," and "T."
Overall, although participants encountered challenges in addressing minor distortions and style inconsistencies, all expressed satisfaction with their final designs.


\begin{figure}[t]
    \centering
    \includegraphics[width=\linewidth]{figures_pdf/Poster_v5.pdf}
    \caption{
    \textbf{Designed characters for the advertisement posters.}
    The participants designed the characters while viewing background images for the posters.
    The two posters on the top left are for an autumn foliage festival.
    P17 created a poster for a summer sale, while P18 designed characters for a Halloween event.
    P19 developed a font for a birthday card, and P20 created one for a movie poster.
}
\label{fig:poster}
\end{figure}


\subsection{Designing CJK Fonts}
In the user study (\autoref{sec:user-study}), we demonstrated that participants could efficiently design Roman characters using our proposed system.
 By swapping the font generative model, the system can also support other writing systems, including Chinese, Japanese, and Korean (CJK).
As shown in \autoref{fig:CJKDesign}, users can efficiently design CJK characters without the need of predesigned examples.
Once the design process is complete, users can download their custom fonts as OTF files.




\begin{figure}[b]
    \centering
    \includegraphics[width=\linewidth]{figures/CJKDesign_v2.png}
    \caption{
    \textbf{Screenshots of CJK font character design.}
    The demonstration of CJK character designs using our system.
    The order is displayed in the upper right corner of each screenshot.
    See the supplemental material for the video.
}
\label{fig:CJKDesign}
\end{figure}

\section{Discussion}
We discuss the bigger picture emerging from our results, followed by specific aspects of interest.

\subsection{Word Bubbles Support Continuous Touch Control of Text Generation}\label{sec:discussion_bubbles} %
Our ``Bubbles'' visualisation with its text length and word indicators consistently outperformed \revision{the alternatives}, allowing participants to complete tasks more quickly: 
It averaged \secs{14.71} per task, compared to NoVis (\secs{16.58}) and Lines (\secs{16.50}). 
This speed advantage was evident across tasks with text extension, shortening, and combinations.
Bubbles also received the highest usability rating (SUS: 85.54) and the lowest perceived workload (NASA-TLX score: 1.98).

In addition, participants reported that Bubbles made interactions feel smooth and natural, with many describing the gestures as intuitive and engaging. 
The visual separation of words, sentences, and deleted content provided by Bubbles enhanced the sense of control, making it the preferred method. %

\subsection{Direct Interaction vs Conversations}
When comparing our gestures to the typical conversational LLM interface, participants rated the touch gestures as more usable (81 vs 52.5 SUS score) and less mentally taxing (2.06 vs 3.15 NASA-TLX). 
This is further reflected in %
all subjective measures, including satisfaction, ease of use, control, and efficiency.
Indeed, using gestures led to \pct{58} reduced task times (\secs{56.35} vs \secs{134.86}).

We conclude that our gesture-based concept was well-received and demonstrated clear advantages over a chatbot UI. Combined with our visual feedback, it thus presents a promising alternative for future text editing applications.

\subsection{Direct Interaction and Authorship}
With gestures, participants generally felt more control over text length and the generation process (\cref{fig:own_likert_exp2}, \cref{sec:ss_interview_perception}) yet many did not perceive themselves as the authors of this text. 
This is unsurprising, as \revision{our study} (a) provided given text, and (b) focused more on adjusting text length than altering content.

With our prototype, users could modify tone and input custom prompts with a long press (\cref{fig:long_press}). 
We expected the ChatGPT-like app to receive similar or even higher ratings in the ``authorship'' category, given that participants entered prompts themselves. 
However, this was not reflected in their ratings -- gestures scored higher. 
We hypothesize that the expected fall off in perceived authorship was mitigated due to participants feeling more like authors when they did not need to switch contexts or interact with an external (chatbot) app. %
This effect might become even stronger as more LLM capabilities are controlled through direct (touch) interaction.


\revision{More broadly,} previous research has shown that perceived authorship does not necessarily equate to actually \revision{having entered the text} \cite{AIghostwriter2024}. 
\revision{Thus, as HCI research pursues} interaction with AI \revision{to become} increasingly direct and seamless, %
we believe it is important to also take a nuanced look at the concept of authorship \revision{and the interaction method's impact on its perception}.


\subsection{Direct Interaction and Interaction Metaphors}
When using conversational UI elements, some users tend to perceive the AI more as a writing partner \cite{diarymateKim2024, benharrak2024aipersonas}, rather than a tool. %
Direct interaction has the potential to change this: 
Controlling the LLM directly feels more natural and gives a greater sense of control, as reflected in our findings. 
It also removes the mental association with a writing partner, as there’s no conversational interface -- just the user's own movements. 
As LLM interactions become more seamless, users might begin to view the LLM as part of their cognitive process \cite{gptMeMeshi2024, bhat2023suggestionmodel}, \revision{which could be explored in the} future.







\subsection{Impact of the Visual Feedback on User Interaction and Cognitive Load}
The choice of visual feedback had a significant impact on how users interact with and perceive the system.

\revision{Beyond the benefits of ``Bubbles'' discussed above (\cref{sec:discussion_bubbles})}, this design also impacted gesture execution: 
With Bubbles, participants approached the intended text length more consistently and with less overshooting. 
\revision{Without} visual feedback, \revision{they instead} had to wait for all words to be generated before gauging text length, as both the length and content were displayed through the same medium -- words. 
This presents a unique challenge for interaction with LLMs, where generation is faster than reading, but slower than users can assess the intended length.

Bubbles decoupled text length indication from actual words, allowing users to gauge the amount of text and start reading sooner. We hypothesize that differences in execution patterns stem from this decoupling. 

As less overshooting occurred \revision{with Bubbles}, people had to read less generated text, which may have lowered cognitive load. 
This could explain the reduced mental demand during writing when using Bubbles, as measured by the NASA-TLX (\cref{fig:own_likert_exp2}). 

Reading, attention, and related indicators of cognitive demand were not measured beyond self-reports (e.g. with eye tracking), which could be a focus for future research.




\subsection{Exploring the Design Space of Mobile Touch Interaction with LLMs}
After defining the design space, we adopted a depth-first approach in this paper and focused on exploring one specific LLM capability and gesture in detail: 
\spread{}, along with the inverse, \pinch{}. 

In future studies, we plan to explore additional alternatives and command mappings:
\begin{itemize}
    \itemsep 2mm
    \item \textit{Spread-to-Elaborate, Pinch-to-Summarize:} Currently, spreading appends text at the end of a sentence. Future versions could use spreading over existing \revision{paragraphs to extend them by inserting} elaborations. Similarly, pinching could trigger AI-generated summaries.
    \item \textit{Swipe-to-Rephrase, Rotate-for-Tone:} Swiping over text passages could prompt the LLM to rephrase them or generate synonyms for selected words, while a \revision{rotation} gesture could adjust tone (e.g. ``dialling up/down'' formality).
    \item \textit{Tap-to-Talk:} Special gestures such as a triple tap or a three-finger swipe could request detailed explanations or feedback from the LLM.
    \item \textit{Other Input Modalities:} Beyond touch-based interaction, shaking the phone could trigger rewording and tilting could provide finer control. %
\end{itemize}


\subsection{Limitations}
Our study comes with limitations.
Our prototype was limited to two gestures, within a simple text field, and tested on one device. 
Integrating our prototyped functionality into a larger mobile application (e.g. respecting existing gestures) was beyond our scope.

We covered an initial sample with diversity in some demographics (age, technology use). 
Future work should further evaluate the gestures with a larger, more diverse population.

Participants preferred touch controls over a conversational UI for the tested tasks of text generation and shortening. 
This should not be generalised as an overall preference. 
It is likely that conversational UIs, which offer much more open functionality than gestures, would still be preferred in other tasks (e.g. complex information retrieval, interactive dialogues).
We plan to expand our prototype and compare interactions across further tasks in the future.



\section{Conclusion}
This paper presented \toolkit, a do-it-yourself toolkit that empowers novice roboticists with basic electronics and programming skills to rapidly prototype interactions for functional lo-fi exoskeletons targeted at the arms. 
\toolkit~features modular hardware components that allow to easily reconfigure its active degrees of freedom, adjust component's dimensions to accommodate various body sizes, and safety mechanisms. We conceptually identified relevant high-level augmentation strategies and provide them as functional abstractions that simplify the programming of interactive behaviors. These functions are readily accessible and customizable through a command-line interface, GUI, Processing library, and Arduino firmware. 
Through application cases and two usage studies, we demonstrated \toolkit's potential to ease the development of human-exoskeleton interactions and support creative exploration and rapid iteration in early-stage interaction design. We hope that this work will inspire HCI researchers to explore the emerging field of human-exoskeleton interaction and unlock its potential for innovative applications.
 
\begin{acks}
    We thank all participants of our usage studies and express our particular gratitude to Ata Otaran for his feedback. We also thank the reviewers for their valuable comments.
\end{acks}




% \smallskip
% \myparagraph{Acknowledgments} We thank the reviewers for their comments.
% The work by Moshe Tennenholtz was supported by funding from the
% European Research Council (ERC) under the European Union's Horizon
% 2020 research and innovation programme (grant agreement 740435).

\bibliographystyle{ACM-Reference-Format}
\bibliography{bib_font}



\end{document}
