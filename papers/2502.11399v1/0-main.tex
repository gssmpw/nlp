\documentclass[sigconf, screen, nonacm]{acmart}



\usepackage{color}
\usepackage{soul}    %
\usepackage{amsmath}
\usepackage{subfig}
\usepackage{xcolor, colortbl}
\usepackage{bm}
\usepackage{amsfonts}
\usepackage{acmart-taps}
\usepackage{hyperref}

\sloppy
\def\num#1{#1}


\usepackage{gensymb} %

\usepackage{enumitem} %

\usepackage{xspace} %

\newcommand{\systemName}{\textsc{FontCraft}\xspace}

\newcommand{\etal}{{\it{et~al.}}}
\newcommand{\ie}{i.e.,}
\newcommand{\eg}{e.g.,}

\def\sectionautorefname{Section}
\def\subsectionautorefname{Section}
\def\subsubsectionautorefname{Section}

\definecolor{gray}{rgb}{0.5,0.5,0.5}
\definecolor{green}{rgb}{0, 0.6, 0}
\definecolor{orange}{rgb}{1, 0.5, 0}
\definecolor{mahogany}{rgb}{0.75, 0.25, 0.0}
\definecolor{purple}{rgb}{0.6, 0, 0.6}
\definecolor{darkgreen}{rgb}{0, 0.3, 0}
\definecolor{orange}{rgb}{1, 0.5, 0.}
\definecolor{lightblue}{rgb}{0.52, 0.75,0.91}
\newcommand{\ichao}[1]{\textcolor{purple}{{ichao: #1}}}
\newcommand{\doga}[1]{\textcolor{red}{{doga: #1}}}
\newcommand{\yuki}[1]{\textcolor{green}{{yuki: #1}}}
\newcommand{\unsure}[1]{\textcolor{orange}{{#1}}}

\newcommand{\bestcell}[1]{\cellcolor{lightblue!50}#1}
\newcommand{\seccell}[1]{\cellcolor{secondblue!50}#1}
\newcommand{\probcell}[1]{\cellcolor{softred}#1}
\colorlet{soullightblue}{lightblue!50}
\newcommand{\besthint}[1]{\sethlcolor{soullightblue}\hl{#1}}
\colorlet{soullightyellow}{yellow!40}
\newcommand{\rrhl}[1]{#1}
\DeclareRobustCommand{\todo}[1]{
  \begingroup
  \definecolor{hlcolor}{RGB}{245,183,177}\sethlcolor{hlcolor}%
  \hl{\textbf{TODO:} #1}%
  \endgroup
}


\copyrightyear{2025}
\acmYear{2025}
\setcopyright{acmlicensed}\acmConference[CHI '25]{CHI Conference on Human Factors in Computing Systems}{April 26-May 1, 2025}{Yokohama, Japan}
\acmBooktitle{CHI Conference on Human Factors in Computing Systems (CHI '25), April 26-May 1, 2025, Yokohama, Japan}
\acmDOI{10.1145/3706598.3713863}
\acmISBN{979-8-4007-1394-1/25/04}

\begin{document}

\title[\systemName: Multimodal Font Design Using Interactive Bayesian Optimization]{\systemName: Multimodal Font Design\\Using Interactive Bayesian Optimization}
\hypersetup{
  pdftitle={\systemName: Multimodal Font Design Using Interactive Bayesian Optimization},
}


\author{Yuki Tatsukawa}
\orcid{0009-0003-5128-8032}
\affiliation{%
 \institution{The University of Tokyo}
 \country{Japan}
}
\author{I-Chao Shen}
\orcid{0000-0003-4201-3793}
\affiliation{%
 \institution{The University of Tokyo}
 \country{Japan}
}
\author{Mustafa Doga Dogan}
\orcid{0000-0003-3983-1955}
\affiliation{%
 \institution{Adobe Research}
 \country{Switzerland}
}
\author{Anran Qi}
\orcid{0000-0001-7532-3451}
\affiliation{%
 \institution{Centre Inria d'Université Côte d'Azur}
 \country{France}
}
\author{Yuki Koyama}
\orcid{0000-0002-3978-1444}
\affiliation{%
 \institution{National Institute of Advanced Industrial Science and Technology (AIST)}
 \country{Japan}
}
\author{Ariel Shamir}
\orcid{0000-0003-4201-3793}
\affiliation{%
 \institution{Reichman University}
 \country{Israel}
}
\author{Takeo Igarashi}
\orcid{0000-0002-5495-6441}
\affiliation{%
 \institution{The University of Tokyo}
 \country{Japan}
}





\renewcommand{\shortauthors}{Tatsukawa, et al.}


\begin{abstract}
Creating new fonts requires a lot of human effort and professional typographic knowledge.
Despite the rapid advancements of automatic font generation models, existing methods require users to prepare pre-designed characters with target styles using font-editing software, which poses a problem for non-expert users.
To address this limitation, we propose \systemName, a system that enables font generation without relying on pre-designed characters.
Our approach integrates the exploration of a font-style latent space with human-in-the-loop preferential Bayesian optimization and multimodal references, facilitating efficient exploration and enhancing user control.
Moreover, \systemName allows users to revisit previous designs, retracting their earlier choices in the preferential Bayesian optimization process.
Once users finish editing the style of a selected character, they can propagate it to the remaining characters and further refine them as needed.
The system then generates a complete outline font in OpenType format.
We evaluated the effectiveness of \systemName through a user study comparing it to a baseline interface.
Results from both quantitative and qualitative evaluations demonstrate that \systemName enables non-expert users to design fonts efficiently.


\end{abstract}

\begin{CCSXML}
<ccs2012>
   <concept>
       <concept_id>10003120.10003121</concept_id>
       <concept_desc>Human-centered computing~Human computer interaction (HCI)</concept_desc>
       <concept_significance>300</concept_significance>
       </concept>
 </ccs2012>
\end{CCSXML}

\ccsdesc[300]{Human-centered computing~Human computer interaction (HCI)}

\keywords{font design, outline fonts, human-in-the-loop, latent space exploration, novice user support tools, generative models, typography tools}


\begin{teaserfigure}
  \centering
  \includegraphics[width=\linewidth]{figures_pdf/teaser_v10.pdf}
  \caption{
  \systemName allows non-expert users to create a font without pre-designed characters through four key steps.
  (a) Users input multimodal data (text, images, font files) to construct a new search subspace.
  (b) Users repeatedly explore the search subspace recommended by Bayesian optimization or constructed by multimodal reference using a slider.
  (c) Users can propagate an edited character's style to the remaining characters and refine any unsatisfactory characters (\eg~``K'') by repeating tasks (a) and (b).
  (d) The system generates \textit{OpenType Font} (OTF) file.
  }
  \Description{(a) -- (b) --- (c)---}
  \label{fig:teaser}
\end{teaserfigure}

\maketitle


The increasing reliance on LLMs for multimodal tasks across far-reaching sectors such as healthcare, finance, and manufacturing underscores the need to assess the accuracy and reliability of the information they generate. Vision-Language Models (VLM) have achieved state-of-the-art (SoTA) performance on Visual Question-Answering (VQA) benchmarks, and these models often utilize Retrieval-Augmented Generation (RAG) to maintain factual accuracy and relevance in a dynamic information environment. However, this has led to uncertainty in the information the LLM bases its answer on, as it may choose between parametric memory and retrieved sources. When models rely on memorized information instead of dynamically retrieving information, they may inadvertently propagate outdated or incorrect information, causing serious legal and ethical risks and undermining trust and reliability in AI systems \citep{huang2023survey}.
% The ability to strike a balance between generalization and specialization in AI systems is therefore crucial for ensuring the safe, reliable use of these technologies in real-world applications.

Despite these concerns, the way that Vision-Language models (VLMs) memorize and retrieve information, particularly in complex multimodal tasks, remains under-explored. Current research often focuses on either the general capabilities of large language models (LLMs) or the specialized retrieval mechanisms in retrieval augmented generation systems (RAG) \citep{incontext_rag,chen_murag_2022,liu_universal_2023}. Particularly in the context of multimodal retrieval and multihop reasoning, few studies analyze the tradeoff between finetuning for specialized tasks and zero-shot prompting for general-purpose vision-language capabilities. A lack of consensus on how to approach this tradeoff motivates the development of measures to quantify reliance on parametric memory, as well as metrics for quantifying the potential performance impact of extending LLMs with RAG systems.

To address this gap, we investigate how multimodal QA models balance accuracy with memorization on the WebQA benchmark. We compare finetuned multimodal systems against zero-shot VLMs, analyzing how retrieval performance influences QA accuracy. In particular, we focus on cases where retrieval fails, allowing us to measure reliance on parametric memory through two proposed metrics---the \ppr (\PPR) which quantifies how much model accuracy is influenced by retrieval quality, contrasting performance in best-case versus worst-case retrieval scenarios, and the \ucr (\UCR) which measures how often correct QA responses are generated when the retriever fails, providing a proxy for memorization.

To enable this analysis, we make several methodological contributions. For the finetuned QA models, we investigate Vision-Transformer (ViT) architectures, which allow for multihop reasoning over multiple sources. To investigate the impact of retrieval performance on trained LMs, we propose a variable-input Fusion-in-Decoder (FiD) model \cite{tanaka_slidevqa_2023, nlvr2}, building upon the VoLTA architecture \citep{pramanick_volta_2023}. For the zero-shot case, we build upon previous research on In-Context Retrieval \citep{incontext_rag} by demonstrating that LLMs such as GPT-4o are capable of performing the final ranking step of the retrieval process. In doing so, we find that GPT-4o, a general-purpose LLM, achieves SoTA performance on the WebQA task, outperforming existing finetuned RAG models by a significant margin (7\% higher accuracy). 

Crucially, our results reveal that while retrieval-augmented models reduce memorization, the training paradigm plays an important role. Finetuned models exhibit higher reliance on parametric memory, whereas zero-shot RAG approaches have lower memorization scores at the cost of accuracy. This suggests that while retrieval modules may mitigate the risks associated with outdated or incorrect information, SoTA performance requires that they be coupled with specialized QA models. Our memorization measures contribute to the development of transparent and reliable AI systems, particularly in applications where the sourcing of up-to-date, factual information is critical.



% We investigate the impact of question complexity on the ability of these models to integrate multiple data sources—such as images, text, and external retrievers—and produce coherent and accurate answers. We also explore whether in-context retrieval can be a viable alternative to traditional retrieval-augmented systems, offering a more streamlined approach to multimodal QA.

% To achieve this, we first compare zero-shot prompting multimodal LLMs with finetuned multimodal systems. We evaluate both types of models on the WebQA benchmark, a dataset designed for complex question answering that requires reasoning across both image and text sources. For the finetuned models, we use a Fusion-in-Decoder (FiD) architecture, which allows for multihop reasoning over multiple sources. Additionally, we introduce the concept of In-Context Retrieval Language Modeling (RLM), where the LLM itself performs retrieval tasks without the need for external retrievers. This method builds upon existing research in in-context learning  and aims to explore the viability of LLMs retrieving relevant sources and generating accurate answers directly from their context window.

% In order to investigate source utilization in finetuned multimodal models and LLMs, three lines of inquiry are established; 
% \begin{itemize}
%     \item Study 1: retrieval vs QA performance on webQA (motivating example, does QA answer correctly even with incorrect sources?)
%     \item Study 2: performance on adversarial examples where parametric knowledge would be incorrect by design
%     \item Study 3: improving performance on adversarial examples by fine-tuning (i.e model robustness)
% \end{itemize}

% Note, there is one weakness in this plan which is tying in the work we've already done. 
% If we added something from adversarial generation to the retrieval experiment (like a combination of study 1 + 3) it would be complete. So for instance we could try fine-tuning the retriever with adversarial examples (and not just the QA model)

% \begin{figure}
%     \centering
%     \includegraphics[width=0.95\linewidth]{figures/segmentation/webqa_segment_infill.png}
%     \caption{Example of the segmentation substitution pipeline from the WebQA task.}
%     % d5c76d760dba11ecb1e81171463288e9
%     \label{fig:seg_sub_pipeline}
% \end{figure}



% Retrieval augmented generation (RAG) with zero-shot prompting and fine-tuning Large Language Models (LLMs) have become the go-to methods for tasks relying on information retrieval and text generation. In many cases the LLMs parametric memory can sufficiently generalize to answer questions without being provided with retrieval mechanisms for out-of-domain knowledge. However, LLMs often hallucinate and provide wrong information in certain scenarios. This problem is amplified even further on open-domain Question Answering (QA) tasks involving multiple modalities. Grounded text generation using retrieved sources \citep{lewis2021retrievalaugmented} has been extensively studied for text-to-text QA tasks, but its application in multimodal settings has not been studied as much.


% Multimodal reasoning and question answering have gained prominence in recent research endeavors, with an increasing emphasis on handling various forms of data, particularly text and images. In this study, we address a specific gap in the existing literature by focusing on the development of a versatile multihop model capable of accommodating varying numbers of input images.

% Our motivation for this research lies in the growing complexity of answering questions using information on the web, where the challenge of navigating the open-domain setting is further complicated by the presence of multiple modalities and sometimes requires reasoning over multiple sources. WebQA is an ideal dataset on which to compare performance of finetuned RAG systems against general purpose LLMs; it is multimodal, with correct answers requiring reasoning over image and text sources. It is multihop, requiring a complex reasoning process over multiple sources. Finally, WebQA questions from different categories can be broken down into subdomains to analyze performance over domains of varying cardinality.

% Motivated by the real-world challenges of building retrieval and question answering (QA) systems, we design and finetune a closed domain, multimodal, multihop QA model, that is capable of reasoning over a varying number of sources taken as input from an external retriever module. This research contributes to the relatively underexplored domain of multihop reasoning across various input sources and modalities. Our goal is to explore the challenges posed by these scenarios and develop strategies that enable QA models to retrieve relevant information, conduct logical or numerical reasoning across diverse modalities, and generate coherent responses in natural language. To our knowledge, this is the first application of the Fusion-in-Decoder (FiD) architecture \cite{tanaka_slidevqa_2023, nlvr2} that is shown to work with a variable number of inputs, enabling multi-hop reasoning over sources.

% In-Context Learning refers to the ability of LLMs to perform any task by simply providing examples in the input prompt \citep{dong2022survey,min2022rethinking}. Inspired by this research, we propose a method to use the LLM itself as a multimodal retriever, potentially eschewing the requirement of a distinct retrieval module, thereby allowing the design of simpler retrieval-augmented QA systems. We dub this method In-Context Retrieval Language Modeling (RLM). To the best of the authors knowledge, In-Content RLM is disparate from other retrieval augmented approaches which utilize external retrieval modules \citep{incontext_rag,chen_murag_2022,liu_universal_2023}. Despite being a natural extension of In-Context learning, In-Context RLM has not yet been studied empirically.

% To expand on our contribution of In-Context Retrieval, this stems from the well-researched in-context learning of LLMs. In-context learning is the ability of a model to perform any task given a sufficient context window \citep{dong2022survey,min2022rethinking}. Such tasks could include retrieval and ranking, but typically, the go-to solution for tasks requiring retrieval has been RAG. To the best of the authors knowledge, In-Context Retrieval is distinct from In-Context Retrieval Augmented Language Modelling (RALM), and despite being a natural extension of In-Context learning, In-Context Retrieval has not yet been shown empirically.

% Finally, we explore the tradeoff between using zero-shot prompting LLMs and the fine-tuning approach. While we find that, overall, GPT-4o obtains SoTA performance on the WebQA task, outperforming the accuracy of existing finetuned RAG approaches by 7\%, finetuned approaches still perform better on more restricted subdomains\footnote{``In-Context RLM" @ \url{https://eval.ai/web/challenges/challenge-page/1255/leaderboard/3168}}. Finally, we validate that GPT-4o is relying on retrieval abilities to solve the task; we find that GPT-4o is capable of retrieving relevant sources in the presence of distractors and furthermore, when GPT-4o fails to retrieve correct sources, it answers incorrectly 75\% of the time, meaning that it is not relying on parametric memory for this task.

% \paragraph{Contributions}
% Based on our experimentation and analysis on the WebQA benchmark, we make the following contributions:
% \begin{itemize}
%     \item Propose a new architecture for multimodal multihop QA that takes variable number of input sources inspired by the Fusion-in-Decoder method.
%     \item Comparison of general purpose LLMs vs specialized models on the WebQA benchmark.
%     \item Observation of In-Context Multimodal Retrieval abilities of GPT-4o and that it does not rely on parametric memory for multimodal QA.
%     \item Analysis of relationship between retrieval and QA task performance.
%     \item Analysis of task and query complexity on the performance of retrieval and QA tasks.
% \end{itemize}
















% Throughout this paper, we will present our methodology, experiments, and findings, emphasizing our approach to multihop reasoning over varying numbers of input images. We believe that our work contributes to a deeper understanding of multimodal reasoning and has the potential to enhance the capabilities of question-answering systems in the intricate, multimodal landscape of web-based information.
\section{Related Work}
\label{sec:relatedwork}
Traditionally, an experimental \ac{IR} collection includes three elements, a corpus, a set of topics, and the relevance judgments, defining which documents are relevant in response to the topics.
Over the last 30 years, since the first TREC campaign~\cite{DBLP:conf/trec/1992}, the most common strategy to obtain such relevance judgments has involved expert annotators, capable of providing the most accurate labels. 
The cost of this process can be partially reduced with pooling~\cite{croft2009search}, but the monetary and temporal costs of building an \ac{IR} experimental collection following this paradigm remain extremely high.

Automatic relevance judgment has recently received significant attention in the IR community. In earlier studies, ~\citet{faggioli2023perspectives} studied different levels of human and LLMs collaboration for automatic relevance judgment. They suggested the need for humans to support and collaborate with LLMs for a human-machine collaboration judgment. ~\citet{thomas2023large} leverage LLMs capabilities in judgment at scale, in Microsoft Bing. They used real searcher feedback to build an LLM and prompt in a way that matches the small sample of searcher preferences. Their experiments show that LLMs can be as good as human annotators in indicating the best systems. They also comprehensively investigated various prompts and prompt features for the task and revealed that LLM performance on judgments can vary with simple paraphrases of prompts. Recently, \citet{rahmani2024synthetic} have studied fully synthetic test collection using LLMs. In their study, they generated synthetic queries and synthetic judgment to build a full synthetic test collation for retrieval evaluation. They have shown that LLMs can generate a synthetic test collection that results in system ordering performance similar to evaluation results obtained using the real test collection.

On a different line, \citet{DBLP:conf/sigir/Dietz24} defines a LLM-based ``autograding'' approach. This evaluation strategy targets generated content that cannot be evaluated in a purely offline scenario and it consists of using a question bank as the evaluation test-bed. An \ac{LLM} measures the effectiveness of the generative model in answering the questions, possibly with the supervision of a human. The autograding approach proposed by \citet{DBLP:conf/sigir/Dietz24} includes an automatic passage evaluation whose task aligns with the one evaluated in \texttt{LLMJudge}.

\subsection{Criticisms and Open Challenges}
The use of \acp{LLM} as assessors comes with major bias risks and challenges that should not be neglected, especially considering the impact they might have in the development of \ac{IR} evaluation.

\partitle{Bias}
First and most importantly, \acp{LLM} are affected by bias~\cite{DBLP:conf/fat/BenderGMS21}. Their internal representation of the concepts is, by construction, conditioned on the context such concepts appear in~\cite{DBLP:conf/nips/VaswaniSPUJGKP17}. Thus, depending on the underlying data, the \ac{LLM} might form a biased notion of relevance that might reflect upon the relevance judgments generated by it. Quantifying the bias, identifying its source, and mitigating its consequences are still open issues that need to be addressed. We hope that the release of this collection will help the research community with the needed data to study how to deal with the bias in \ac{LLM}-generated relevance judgments.

\partitle{Circularity}
A second source of concern when it comes to using \acp{LLM} as assessors relates to the risk of \textit{circular evaluation}~\cite{faggioli2023perspectives,DBLP:journals/corr/abs-2409-15133}. For example, the same \ac{LLM} might be used to generate relevance judgments and as a document ranker. This would induce a strong bias on the validity and generalizability of the relevance judgments.

\partitle{Environmental Impact}
An often hidden cost of the \acp{LLM} concerns their environmental impact in terms of energy utilization, carbon emissions~\cite{DBLP:journals/corr/abs-2408-09713,DBLP:conf/sigir/ScellsZZ22}, and water consumption~\cite{DBLP:conf/ictir/ZucconSZ23}.
While \acp{LLM} might allow building collections at a fraction of the monetary and temporal cost, we should account for the environmental impact of such a process, limiting our reliance on ``disposable'' relevance judgments.

\partitle{Vulnerability to Attacks and Adversarial Misuse}
\citet{DBLP:conf/ecir/ParryFMPH24} and \citet{DBLP:conf/sigir-ap/Alaofi0SS24} illustrate the vulnerability of the \acp{LLM} to mischievous manipulations of the corpus. For example,~\citet{DBLP:conf/ecir/ParryFMPH24} show that, by introducing keywords such as the term ``relevant'' in a document, it will more likely considered relevant by an \ac{LLM}. Similar behavior is observed also by \citet{DBLP:conf/sigir-ap/Alaofi0SS24}, who notice that by introducing the query on the document, more probably an \ac{LLM} will consider the document relevant to such a query --- even if the rest of the document is composed by random terms.
More recently, \citet{DBLP:journals/corr/abs-2412-17156} show how, by properly crafting an adversarial run, it is possible to cheat an \ac{LLM} used as an assessor. \citet{DBLP:journals/corr/abs-2412-17156} crafted a run following the same approach used by~\citet{upadhyay2024umbrela} to pool the documents and build the \ac{LLM}-generated relevance judgments used for TREC 2024 RAG. Such a run achieved consistently higher effectiveness under the fully automatic evaluation paradigm compared to its performance based on manual relevance judgments. 

By releasing this collection of \ac{LLM}-generated relevance judgments we want to foster the analysis and study of possible sources of biases and systematic errors, to mitigate them and allow for the development of more effective and robust future solutions that involve \acp{LLM} as tools to support the annotation process.
\section{System}
\begin{figure*}[h]
    \centering
    \includegraphics[width=.85\textwidth]{fig/SYSTEM_IMAGE_TEST_FLIPPED.png}
    \caption{HumorSkills System Diagram. Given an image, the system first extracts visual details with a visual language model, then performs visual humor ideation to analyze the image and propose humorous angles. It then generates ten potential conflicts that could be used to extrapolate the image into a relatable experience. The system then generates humor with and without the narratives, for diversity. A separate instance of the LLM trained to rank gen-Z humor ranks all the captions and returns the top five.}
    \Description{HumorSkills System Diagram}
    \label{fig:system}
\end{figure*}

HumorSkills is a system that takes an input image and outputs 5 image captions. 
The architecture has three key steps that mimic human skills needed for humor. \textit{Visual Detail Extraction}, is a step that describes the image in depth in order to make non-obvious observations about it. \textit{Narrative and Conflict Extrapolation} is a step that finds narratives not in the image that could be related to it, to expand the topic of jokes to things that are not just in the image but also analogous to it.  \textit{Fine-tuning} the joke generator with examples of good Gen-Z humor helps the jokes be more relatable to the target audience by using references, slang, topics, and insecurities that resonate with this group.

% first, 
% a \textbf{divergent stage} where the image is analysed and multiple observations, angle, alternative narratives and humorous angles are generated. 
% Second, a \textbf{generation stage} where two types of captions are generated: 1) captions focusing on image content directly 2) captions that bring in an outside narrative to the image, often bringing in outside references. It generates 15 captions of each type. (Figure 1 has examples Of the Content Focused, and Narrative Expanded Captions). Finally, a \textbf{ranking stage} where a separate AI agent selects the top 5 captions

The system generates two types of captions: image-focused captions which common directly on the content in the image, and narrative-driven captions. Variety is important to humor. Humor relies on surprise, and jokes that are too similar start to become more predictable. Additionally, with an infinite set of input images with different subjects and situations, there are more strategies needed to find a humorous angle that fits the content. 

% With caption-based humor, often the humor can be focused on finding something in the image that is inherently interesting. 

% For example, the caption ”little man really thought he could escape bedtime” relies solely on information in the image. However, some images don’t have something funny in and of themselves, and it’s easier to make a joke by bringing in a new unrelated angle. For example, ``the police chasing me when I'm broke and in debt to the tune of \$100,000 for student loans''. Generally, Images with people doing interesting things lend themselves to visual humor because there are many stories one could tell about it. However, for images with only static objects, it's more difficult to tell a story on only the objects, so bringing a new story in is another way to find humor. 

\subsection{AI Humor Generation Walk Through}
Figure \ref{fig:system} contains a visual diagram and example of intermediate outputs when generating captions for an image. We describe each phase and implementation in detail.  
% The main contribution of this paper is the evaluation, rather than the system, but it is still it is important to understand the mechanism used to generate humor.
% Although the individual components of the system are not totally, the combination of features including the

\subsubsection{Visual Detail Extraction}

The first phase of the system’s workflow involves the Visual Detail Extraction component, which utilizes GPT-4o’s vision capabilities to analyze the input image. This system incorporates a prompt that asks for a detailed paragraph that explains the who/what/where of the image, distinguishing between identifying the subject of the image, the main action of the image if it exists, and the background elements of the image. This component is responsible for extracting key visual elements such as objects, human expressions, background settings, and any notable aspects that could serve as the foundation for humor.

For instance, in the demolition site example from the system diagram, the system identifies a large industrial demolition excavator and a person with a hose spraying the demolition site. 

\subsubsection{Visual Humor Ideation}

On top of the visual detail extraction, the system ideates on possible humorous elements from the visual of the image. This incorporates an additional prompt using GPT-4o that intakes the image and asks it to identify and ideate on potential humorous visual elements in the image, whether they are directly humorous elements, such as funny facial expressions, or more analogous elements. For example, for the system diagram image, the system noted the visual contrast of the excavator and person, reminiscent of a David versus Goliath scenario, which provides a foundational metaphor for generating humorous captions. 

\begin{figure*}[b]
    \centering
    \includegraphics[width=.95\textwidth]{fig/Workflow.png}
    \caption{A diagram for how narrative extrapolation works}
    % \Description{}
    \label{fig:systemLines}
\end{figure*}

\subsubsection{Narrative and Conflict Extrapolation}

In this next step, the system generates a narrative and conflict framework by drawing upon common and relatable Gen Z experiences such as work, school, family, social interactions, relationships, and more. 
The system chains together the results of the previous steps, into a new prompt sent to GPT-4o. 
% The system prompts GPT-4o to 
% GPT-4o is utilized by incorporating the text description of the image and potential humorous elements of the image, then being prompted to generate relatable scenarios applicable to the image description from a list of common Gen Z experiences. 
The prompt contains the visual details, the visual humor ideation, and a list of common Gen Z experiences,  and the instruction to "generate narratives that reflect the essence of the image that is set within the framework of the Gen Z experience."
This narrative generation adds depth to the humorous captions by applying relatable themes and conflicts to the visual elements identified earlier.

For instance, our system diagram generates narratives such as “Tackling student loans”, "Group Project Disaster", and “Relationship Issues” based on the image, both of which are common experiences among those who identify as Gen Z. These particular narratives are likely inspired by the imagery of a disaster site, referring to how the effort of paying off student loans, attempting to complete group projects during school, or addressing relationship -- all of which can feel like disaster clean up. These relatable conflicts can transform the visual of a demolition scene -- a setting that is not particularly relatable -- into a relatable scenario that has the potential for humor, thereby expanding the humorous possibilities by connecting the visual input with broader life experiences.



\subsubsection{Humorous Caption Generation}

Following the narrative and conflict extrapolation, the system generates humorous captions in the generation stage using a fine-tuned version of GPT-3.5 trained on humorous Instagram comments. This involves producing captions through two distinct strategies: one focused on the visual humor of the image, and the other by bringing in the previously generated external narratives. Caption generation is segmented into two separate prompts utilizing the fine-tuned GPT-3.5 model. For captions without generated external narratives, the prompt asks to generate 15 humorous captions in the style of Gen Z that bases the generation off the visual extraction and visual humor ideation of the input image. For captions with the external narratives, the prompt also asks to generate 15 humorous captions in the style of Gen Z that bases the generation off the visual extraction and visual humor ideation of the input image, but also asks the system to incorporate the list of generated narrative/conflict extrapolations to base the humorous captions off of.

Image-focused captions rely solely on the visual details in the image, such as “bro out here getting paid \textdollar8 an hour to spray some water on some bricks,” which references the direct visual elements in the scene in order to generate a caption. This particular caption directly references the humor of the image, poking fun at the minimal impact of the person spraying water on bricks while an excavator clearly has more impact on the demolition site. Narrative-driven captions, on the other hand, introduce external references to add humor. For instance, a caption like “The entitled bro you tried to make the group presentation with” introduces an outside, exaggerated, interpretation of the scene from earlier, "Group Project Disaster." This caption takes the group project narrative and pairs it with the visual of the image, analogizing the person spraying the hose with minimal impact on the demolition site to an entitled person who has not done much to complete the group project. 

This variety between visual humor and narrative-driven humor is crucial because jokes that are too similar become predictable, losing their element of surprise. Additionally, humor strategies need to adapt to the varying content in input images. Some images lend themselves to humor based on their inherent visual details, while others require bringing in outside references to create a joke. For instance, an image of static objects might not be inherently funny, such as the demolition image shown in the system diagram, but a caption introducing an unrelated, exaggerated narrative, such as “Eboy doin' his part to stop climate change” can inject humor and absurdity by making an unexpected connection.

\subsubsection{Caption Ranking using Gen Z Agent}

The final component of the system architecture is the Caption Ranking and Filtering Agent, a GPT-4o-based agent fine-tuned to evaluate humor from a Gen Z perspective. This agent receives the list of 30 total captions from the narrative and visual humor-based caption generations and ranks the captions generated in the previous stage based on humor, relatability, and alignment with the image and narrative.

As illustrated in our system diagram, this agent ranks captions such as “Me mopping up my last relationship” and “me pulling the emotional weight of the friend group” based on their relevance to Gen Z humor. Captions that fail to meet the humor threshold are filtered out, such as "Demolition worker really said 1v1 me bro," because although the phrase like "1v1 me bro" invokes Gen Z phrases, the content of the caption seems less relevant and relatable than a caption talking about school or relationships, ensuring that only the most effective and relatable captions are presented to the user.

\subsubsection{Fine-tuning}

To fine-tune a GPT-3.5 model, a dataset of 80 humorous comments were extracted from popular Instagram images. From three popular Instagram meme pages with over 400,000 followers, the top five comments of each image post were collected. All fit the style of Gen-Z humor. 
% The fine-tuning process ensured that the generated captions align with the humor style favored by Gen Z. 
Examples of the visual description of the images in addition to an explanation of potential humorous elements of the image were written in the fine-tuning prompts, then followed by the actual comment itself. This reflected the visual extraction and humor ideation being incorporated into the prompt of our current system.



\section{Method}
\subsection{Preliminary of Font Generative Model}
\label{sec:fontgenerativemodel}
We use \textit{DG-Font}~\cite{XieDGFont2021} as our font generative model. 
This model takes a style image $I_{S}$ representing the desired font style, and a content image $I_{C}$ representing the desired character, as input.
It then generates the character image that represents the desired character in the desired font style as the output. 
As illustrated in \autoref{fig:dg-font}, the architecture of \textit{DG-Font} is an encoder-decoder model with two encoders: a style encoder $E_{S}$ and a content encoder $E_{C}$, along with a content decoder $G_{C}$.
The generation process starts by extracting the style latent vector from the style image using the style encoder and the content latent vector from the content image.
Then, the content decoder takes both the style and content latent vectors as input to generate the desired font image $\hat{I}$, which maintains a similar style to the style image while preserving the structure of the content image.
Overall, the generation process during the training process can be formulated as:
\begin{align}
\bm{z}_S = E_{S}(I_S),\quad \bm{z}_C = E_{C}(I_C), \quad 
\hat{I} = G_C(\bm{z}_S, \bm{z}_C).
\end{align}
In our work, we use an enhanced version of \textit{DG-Font}, which includes an additional content discriminator.
Notably, our system (\systemName) only needs the pretrained model to generate new fonts instead of training a model from the beginning.
We will provide the details of this additional model architecture and training process in the supplemental material.

In the rest of the paper, if we need to specify the content image $I_C$ or its latent vector $\bm{z}_C$ for a specific character such as ``A'', we will denote it as  $I_C[\text{``A''}]$ $\bm{z}_C[\text{``A''}]$.
Otherwise, we will use $I_C$ or $\bm{z}_C$ for abbreviation. 
This also applied to the generated image $\hat{I}$.

\begin{figure}[ht]
    \centering
    \includegraphics[width=\linewidth]{figures_pdf/dgfont_v8.pdf}
    \caption{   
    \textbf{Overview of \textit{DG-Font}.} \textit{DG-Font} is an encoder-decoder model that takes a character image representing style and a character image representing content as input, and outputs a character image that combines the content with the specified style.
    During font designing in our system, users use our human-in-the-loop optimization to explore the style latent space of the style encoder.
    Please find the detailed architecture of the encoder and decoder in the supplemental material.
    }
    \label{fig:dg-font}
\end{figure}


\subsection{Preliminary of FontCLIP}
To incorporate multimodal input when designing fonts, we use FontCLIP~\cite{tatsukawa2024fontclip} to extract typographical features from both text and image input.
FontCLIP is a visual-language model that
bridges the semantic understanding of a large vision-language model with typographical knowledge.
It consists of a text encoder and a visual encoder and builds a joint latent space that encodes typographical knowledge.
In this joint latent space, similar font images and text prompts will have similar latent vectors.
For example, a bold font will have a similar latent vector to the text prompt ``This is a \textit{strong} and \textit{thick} font'' compared to the text prompt ``This is a \textit{thin} font''.
Therefore, the FontCLIP latent vector can be used to retrieve similar fonts using text or image input.
In our system, we utilize both the FontCLIP text encoder and visual encoder to extract a latent vector from the multimodal input to customize the linear subspace.

\subsection{Preliminary of Human-in-the-Loop Bayesian Optimization}
\label{method:prelim_BO}
\subsubsection{Problem Formulation}
Human-in-the-loop optimization is a computational approach used to solve parameter optimization problems involving human evaluators in its iterative algorithm.
It is commonly used to support design tasks that involve generating visual content defined by a set of parameters $\bm{x}$, with the aim of achieving certain subjective design goals.
Specifically, we can formulate such an optimization problem as:
\begin{align}
\bm{x}^* = \argmax_{\bm{x}\in{\mathcal{X}}}f(\bm{x}),
\label{eq:opt_goal}
\end{align}
where $\mathcal{X}$ is the search space, and $f: \mathcal{X} \rightarrow \mathbb{R}$ is the goodness function to evaluate a subjective design goal (\eg~the aesthetics of the current design).
We aim to find the optimal value $\bm{x}^*$ with the fewest trials because evaluating $f(\cdot)$ is costly.
However, solving \autoref{eq:opt_goal} using traditional Bayesian optimization (BO) might not be suitable for many design tasks.
This is because it is often difficult to assign an exact value to a sample, whereas comparing a couple of samples and choosing the preferred one is more intuitive.
For example, it is hard for users to give a score to a font individually, but easier for them to choose the preferred font from a set of candidates.
Therefore, in this work, we choose to use preferential Bayesian optimization (PBO) \cite{Koyama2022}, which is a variant of Bayesian optimization (BO) that runs with relative preferential data.
In particular, we build our method upon Sequential Line Search (SLS) \cite{KoyamaSequential2017}, a PBO method that constructs a sequence of linear subspaces that leads to the optimal parameters that match the user's need.


\subsubsection{Sequential Line Search (SLS)}
\label{sec:sls}
With SLS, the user can search for his/her preference by adjusting a slider.
At each iteration of the optimization, SLS constructs a linear subspace using the current-best position $\bm{x}^{+}$ and the best-expected-improving position $\bm{x}^{\text{EI}}$.
Suppose we already have $t$ observed response so far, then the next linear subspace $\mathcal{S}_{t+1}$ is constructing by connecting:
\begin{align}
    \bm{x}^\text{EI}_t &= \argmax_{\bm{x}\in{\mathcal{X}}}a^\text{EI}(\bm{x};\mathcal{D}_t) \label{eq:build_subspace} \\ \nonumber
    \bm{x}^+_t &= \argmax_{\bm{x}\in{\left\{\bm{x}_i\right\}_{i=1}^{N_t}}}\mu_t(\bm{x})
\end{align}
where $\left\{\bm{x}_i\right\}_{i=1}^{N_t}$ denotes the set of points observed so far, $\mu_t$ and $a^\text{EI}$ are the predicted mean function and the acquisition function calculated from the current data.
We use the expected improvement (EI) criterion as the acquisition function to choose the next sampling point that is likely to optimize the function $f$ and at the same time its evaluation is more informative:
\begin{align} \label{equation:acquisition}
    a^{\text{EI}}(\bm{x} ; \mathcal{D})=\mathbb{E}\left[\max \left\{f(\bm{x})-f^{+}, 0\right\}\right].
\end{align}

After the $t$-th iteration, we suppose that we obtained a set of $t$ single slider responses, which is represented as
\begin{align}
    \mathcal{D}_t=\left\{\bm{x}_i^\text{chosen}>\left\{\bm{x}_{i-1}^{+}, \bm{x}_{i-1}^{\text{EI}}\right\}\right\}_{i=1}^t,
\label{eq:response_data}
\end{align}
where $\bm{x}_i^{\text{chosen}}$ represent the position chosen by the user at the $t$-th iteration.

Let $f_i$ be the goodness function value at a data point $\bm{x}_i$, i.e., $f_i = f(\bm{x}_i)$, and $\bm{f}$ be the concatenation of the goodness values of all data points:
\begin{align}
\bm{f} = [f_1, f_2, \ldots, f_N].
\end{align}
Under the assumption of Gaussian process (GP) prior on $f$, we use $\bm{\theta}$ to represent the hyperparameters of the multivariate Gaussian distribution.
Since the goodness values $\bm{f}$ and the hyperparameters $\bm{\theta}$ are correlated, we infer $\bm{f}$ and $\bm{\theta}$ jointly by using MAP estimation:
\begin{align}
(\bm{f}^{\text{MAP}}, \bm{\theta}^{\text{MAP}})&=\argmax_{(\bm{f},\bm{\theta})}p(\bm{f}, \bm{\theta} \mid \mathcal{D}) \nonumber \\
&= \argmax_{(\bm{f},\bm{\theta})}p(\mathcal{D} \mid \bm{f}, \bm{\theta}) p(\bm{f}\mid \bm{\theta}) p(\bm{\theta}).
\label{eq:map}
\end{align}

Once $\bm{f}^{\text{MAP}}$ and $\bm{\theta}^{\text{MAP}}$ have been estimated, we can compute $\mu(\cdot)$, $\sigma(\cdot)$, and $a^{\text{EI}}(\cdot)$ in order to construct the next slider subspace $\mathcal{S}_{t+1}$.
We only describe the minimum concept of how SLS constructs the linear subspace to optimize the function $f$ for understanding how we incorporate it in the style latent space of a font generative model and multimodal input.
For more details, please refer to the supplemental material.


\subsection{Multimodal Bayesian Optimization for Font Generation}
\label{sec:multiModalBayesianOptimization}
Specifically, following \autoref{eq:opt_goal}, the objective function for designing a character can be formulated as:
\begin{align}
\bm{z}^* = \argmax_{\bm{z}\in{\mathcal{Z}}}f(G(\bm{z})), \label{eq:argmax_latent_space}
\end{align}
where $\bm{z} \in\mathcal{Z}$ is the style latent vector of the font generative model, which is the search space $\mathcal{X}$ in our problem setting.
Moreover, $G$ is the decoder of the font generative model, and $f: \mathcal{Z} \rightarrow \mathbb{R}$ is the user preference function that measures how good the currently generated character is perceived by the user.
To solve \autoref{eq:argmax_latent_space} and obtain the desired font, users can perform three different actions: \textit{explore the font style latent space}, \textit{retract previous preferences}, and \textit{propagate style to other characters} at each iteration.

\subsubsection{Action 1: Explore the Font Style Latent Space}
\label{sec:explorationWithSingleSlider}
\begin{figure}[t]
    \centering
    \includegraphics[width=\linewidth]{figures_pdf/exploreWithBayesianOptimization_v7.pdf}
    \caption{
    \textbf{Exploration of the font style latent space using a single slider.} 
    (a) Users explore a one-dimensional search subspace within the font style latent space using a single slider.
    At each iteration, users choose a point in the latent subspace and submit it as their current preference $\bm{z}^{\text{chosen}}_t$.
    After a couple of iterations, users gradually converge to their desired font style.
    The overall exploration process, users can explore (b) BO subspace only, (c) multimodal-guided subspace only, and (d) combination of both. 
}
    \label{fig:ExploreWithBayesianOptimization}
\end{figure}

The primary task for users is to explore the one-dimensional font-style search subspace using a slider.
By repeating the slide manipulation, users gradually converge on a point that aligns with their desired font style as illustrated in \autoref{fig:ExploreWithBayesianOptimization}.
This subspace is constructed by the system in two ways:
one solely follows the SLS method and another utilizes multimodal references.
Note that, regardless of these two different approaches to constructing the subspace, the interaction (\ie~manipulating the slider and submitting a preferred point to the system) remains consistent.

\paragraph{Constructing a SLS subspace}
Using the SLS method outlined in \autoref{sec:sls}, our system constructs a linear subspace $\mathcal{S}_{t}$ by connecting the current best point $(\bm{z}^+_t)$ and the point that maximizes the acquisition function $(\bm{z}^\text{EI}_t)$ using \autoref{eq:build_subspace} at the $t$-th iteration.
Then, users can choose a style latent vector $\bm{z}$ using the slider within $\mathcal{S}_{t}$ and view the generated character $G(\bm{z})$.
Once satisfied, users submit their preferred point on the slider $\bm{z}^{\text{chosen}}_t$ by clicking the \textsc{Update} or \textsc{Update all} button and request the system to construct a new linear subspace $\mathcal{S}_{t+1}$ for the $(t+1)$-th iteration using \autoref{eq:map}.


\paragraph{Constructing a multimodal-guided subspace}
While exploration with a single slider is useful, the linear subspace determined solely by Bayesian optimization sometimes fails to capture user preferences effectively, which leads to an increasing number of iterations and potentially causes frustration and a diminished sense of agency.
To address this issue, we allow users to intervene in the linear subspace construction by providing multimodal references at any iteration.
At $(t+1)$-th iteration, instead of exploring the linear subspace $\mathcal{S}_{t+1}=(\bm{z}^+_t, \bm{z}^\text{EI}_t)$ constructed solely by Bayesian optimization, the user explores the multimodal-guided subspace: $\mathcal{S}^{mm}_{t+1} = (\bm{z}^+_t, \bm{z}^{mm}_t)$.
Here, $\bm{z}^{mm}_t$ is the style latent vector obtained by retrieving the most similar fonts to the multimodal reference provided at $(t+1)$-th iteration from a font database containing \num{1,169} Roman fonts collected by O'Donovan~\etal~\cite{o2014exploratory}.
Specifically, we retrieve $n$ fonts and use the mean of their latent vectors as $\bm{z}^{mm}_t$ (we use $n = 5$ in our implementation).
Once the user is satisfied with the current generated character, the slider response: ($\bm{z}_{t+1}^\text{chosen},\bm{z}_{t}^{+}, \bm{z}_{t}^{mm}$) will be recorded in $\mathcal{D}_{t+1}$ (\autoref{eq:response_data}) and used for constructing the subspace in the future iteration.
This means that all multimodal references provided until iteration $t$ will affect the subspace constructed at $(t+1)$-th iteration.
In our current implementation, at each iteration, users can provide only a single multimodal reference, and we construct a new search subspace by connecting the current chosen point and the latent vector of the provided multimodal reference.
At $0$-th iteration, we construct the initial linear subspace $\mathcal{S}_{0}=(\bm{z}^{\text{init}}, \bm{z}^{mm}_{0})$, where $\bm{z}^{\text{init}}$ represents the style latent vector of a commonly used font (we use ``IPAex gothic'' font in our implementation).
Note that the multimodal references are only used to construct the linear subspace for the user to explore, not being directly used to infer the user preferences.r

To construct a multimodal subspace with the user-provided multimodal reference, our system identifies a suitable font in our font database that corresponds to the input text or image. 
For text input, the system first extracts font attributes such as ``formal,'' ``italic,'' and ``happy'' using a Large Language Model (LLM). 
The LLM selects relevant font attributes based on the given text.
We utilize $37$ types of font attributes compiled by O'Donovan~\etal~\cite{o2014exploratory} (see the supplemental material for more details).
Once the font attributes are extracted, the system obtains the feature vector of these text attributes using the text encoder of FontCLIP~\cite{tatsukawa2024fontclip} and retrieves fonts from our font database based on feature similarity. 
The retrieved fonts are used as the most suitable options, and the mean of their style latent vectors $\bm{z}^{mm}$ is obtained using the style encoder $E_{S}$.

For the text-rendered image reference, our system retrieves its most similar font in the font database using the FontCLIP feature.
Similarly, we then project the retrieved font image into the style latent space using $E_{S}$ and obtain $\bm{z}^{mm}$.
Finally, for font file input, our system directly uses the provided font as the suitable font and projects it into the latent space, similar to the process for text and image inputs.
\autoref{fig:multi_modal_subspace}(c) illustrates how to obtain the style latent vector of the multimodal references.


\begin{figure*}[ht]
    \centering
    \includegraphics[width=0.95\linewidth]{figures_pdf/multi_modal_subspace_v6.pdf}
    \caption{
    \textbf{Constructing linear subspaces using multimodal references.} 
    (a) At the start of the font design process using our proposed method, the user inputs text, an image, or a font file. 
    The system encodes this input into a font style latent vector and initializes the line search space by connecting the latent vector and a fixed point predetermined by the system.
    (b) Additionally, the user can introduce multimodal inputs at any stage of the design process.
    When the user provides new input, the system generates a new line search subspace by connecting the last user preference point with the newly encoded point.
    (c) Our system encodes multimodal input into the style latent space by leveraging LLM and FontCLIP text and visual encoders.
}
    \label{fig:multi_modal_subspace}
\end{figure*}

\subsubsection{Action 2: Retract Previous Preferences}
At each iteration, if the user is not satisfied with the current design and the recommended candidates, they can choose to retract previous preferences.
Specifically, at $(t+1)$-th iteration, if the user wishes to retract the last two slider manipulations, then the last two slider responses stored in $\mathcal{D}_t$ will be discarded.
Then, if the user opts to explore a new subspace by providing a new multimodal reference, they will then explore the subspace $\mathcal{S}^{mm}_{t-1} = (\bm{z}^+_{t-2}, \bm{z}^{mm}_{t-2})$.
Otherwise, the user will explore a subspace constructed using SLS solely: $\mathcal{S}_{t-1} = (\bm{z}^+_{t-2}, \bm{z}^{\text{EI}}_{t-2})$.
By retracting previous preferences, users can update their preferences during the design process.

\subsubsection{Action 3: Propagate Style to Other Characters}
Once the user is satisfied with the design of the focused character (\eg~``A'') and obtains the style latent vector $\bm{z}_S^{*}$, our system can propagate the style to all other characters and finish designing a single character.
Specifically, to propagate the style vector to character ``B'':
\begin{align}
\hat{I}[\text{``B''}] = G_C(\bm{z}_S^{*}[\text{``A''}], \bm{z}_C[\text{``B''}]).
\end{align}
Next, users can check all characters with the propagated styles.
If they are unsatisfied with the result of another character, they can perform action 1 or action 2 to design that character.
If they are satisfied, the resulting font is exported as an outline font.

\section{Evaluation}
\label{evaluation}

\subsection{Simulated Evaluation of Multimodal Reference}
\label{sec:simulatedEvaluation}
To quantitatively evaluate how multimodal reference can help our human-in-the-loop optimization, we designed a simulation test to compare two linear subspace initialization methods: using multimodal reference and random fonts.
\subsubsection{Procedure}
We illustrate the procedure of the simulation test in \autoref{fig:multimodalInputIniitalizationEvaluation}. 
Given a base font character (\eg~``A''), the goal of the simulation test is to resemble the target font character (\autoref{fig:multimodalInputIniitalizationEvaluation}(d)) by exploring the style latent space through optimization.
Specifically, we simulate user selections using the following process. 
At each iteration, our method selects a point in the slider's search subspace with the minimum perceptual metric (we use \textit{DreamSim}~\cite{fu2023dreamsim}) against the target font character.
Then, the selected point is used to request Bayesian optimization to recommend the next linear subspace.
We iterate this process to observe the convergence of the optimization progress using both initialization methods.

For initializing using multimodal reference, we test \textit{text input} and \textit{font file input} in this experiment.
For the text input, we create a descriptive text that characterizes the target font and use it to initialize the search subspace.
For font file input, we manually select a font from candidate fonts that closely resembles the target font and use it to initialize the search subspace.
Finally, for the baseline method, we choose a font randomly from our font database and use it to construct the initial search subspace.

\begin{figure}[ht]
    \centering
    \includegraphics[width=\linewidth]{figures_pdf/synthetic_eval_v7.pdf}
    \caption{
    \textbf{Evaluation of linear subspace initialization methods.}
    We compared two initialization methods for exploration with Bayesian optimization.
    (a) One method uses input text or a similar font file for initialization, while (b) the other initialize method uses a randomly sampled font from a font database.
    After initialization, both methods follow the same automatic exploration process (c), where the optimal point on the single linear subspace is repeatedly identified and submitted to the system.
    In each iteration, we measure the distance between the generated character and the target font character to identify the optimal point, as shown in (d).
    Note that we use the bitmap format of the character for distance calculation, without vectorizing it.
    }
    \label{fig:multimodalInputIniitalizationEvaluation}
\end{figure}
We conducted this experiment using the character ``A'' for $10$ different target fonts, randomly selected from our font database.
We collected $12$ kinds of fonts from which we chose a similar font to each target font for the font file input.
For each target font, we choose the most similar font out of the $12$ candidate fonts.
The $12$ candidate fonts consist of two popular font families, \textit{Roboto} and \textit{NotoSerif}, and each font family has six variations: \textit{Light}, \textit{Light Italic}, \textit{Regular}, \textit{Regular Italic}, \textit{Bold}, and \textit{Bold Italic}.
This selection simulated a scenario where users start with popular fonts and design new fonts based on one of these similar candidates.
For each target font, the optimization process includes $10$ iterations of Bayesian optimization.

\subsubsection{Results}
In \autoref{fig:multimodalInputIniitalizationEvaluationResult}, we show the mean and standard deviation of the distances between the optimized results and all target fonts.
We can observe that the optimization processes with text and font file references converge to a lower \textit{DreamSim} distance to the target font character compared to those initialized with a randomly selected font.
These results indicate that using multimodal references for initializing the human-in-the-loop optimization leads to more effective exploration than random initialization.


\begin{figure}[ht]
    \centering
    \includegraphics[width=0.5\textwidth]{figures_pdf/multimodalInitialixationEvaluationResult_v4.pdf}
    \caption{
    \textbf{Convergence comparison between two initialization methods.}
    The figure illustrates how the \textit{DreamSim} distance between the designed font character and the target font character converges during exploration with human-in-the-loop optimization.
    The optimization processes initialized by text font references (orange) obtain better results compared to processes initialized by random font (blue).
    }
    \label{fig:multimodalInputIniitalizationEvaluationResult}
\end{figure}


\subsection{User Study}
\label{sec:user-study}
To evaluate the effectiveness of our proposed system, we conducted a user study in which participants were asked to design fonts using both a baseline system and our system. 
The goals of this study were threefold: 
\begin{itemize}
    \item to assess the overall effectiveness of our system, including the integration of Bayesian optimization, multimodal reference, history interface, and style propagation.
    \item to compare the fonts designed by participants both qualitatively and quantitatively against those created using the baseline system.
    \item to gather qualitative feedback on the user experience with our system.
\end{itemize}

\subsubsection{Comparison Systems}
For the user study, we added a special feature called \textsc{Font Palette} to our proposed system.
By clicking the \textsc{Font Palette} button, users can view a visualization of the \num{12} popular fonts described in \autoref{sec:simulatedEvaluation} and select one to input as their preference, simplifying the process of inputting a font file.
Additionally, we removed the \textsc{Upload Image} and \textsc{Upload Font} buttons from the UI in \autoref{fig:UI} for simplicity.
As a result, users can now easily input text and font files using the \textsc{Text} and \textsc{Font Palette} buttons, respectively.

To assess the effectiveness of the multimodal reference and style propagation features in our system, we created a baseline system that includes only a single slider, as illustrated in \autoref{fig:baselineSystemUI}.
In this baseline system, users can explore the font style latent space solely by adjusting the slider, guided by the Bayesian optimization process. 
Unlike our proposed system, the baseline’s one-dimensional search space is initialized by connecting a fixed point with a randomly initialized point.
The fixed point corresponds to the style of the \textit{IPAex Gothic} font, as described in \autoref{sec:simulatedEvaluation}. 
If users encounter difficulties during exploration, they can reset their preference history in the Bayesian optimization process and restart from a newly randomized search subspace.
Additionally, this baseline method lacks a style propagation function, requiring users to design each character individually.

\begin{figure}[ht]
    \centering
    \includegraphics[width=0.5\textwidth]{figures_pdf/baseline_ui_v6.pdf}
    \caption{
    \textbf{User interface of the baseline system.}
    In the (a) character design area, users use a slider to explore the one-dimensional subspace within the font style latent space recommended by Bayesian optimization.
    By clicking the \textsc{Reset} button, users can reset their preference history in the Bayesian optimization process, randomly reinitializing the search subspace.
    The users can check the characters that they have already designed are displayed in the (b) character collection area.
}
\label{fig:baselineSystemUI}
\end{figure}

\subsubsection{Procedure}
We recruited ten people for the user study.
Each participant was presented with a target font and asked to design three characters, ``A'', ``B'', and ``C'' that closely match the target font using both \systemName and the baseline system.
For this user study, we prepared two target fonts, Font 1 and Font 2.
Each font design session continued until one of the following conditions was met: (1) the participant was satisfied with the quality of the characters they designed, (2) they felt that further improvement was difficult, or (3) the $7$-minute time limit was reached.
The user study followed this sequence: (Tutorial of \systemName $\rightarrow$ Font 1 with \systemName $\rightarrow$ Font 2 with \systemName $\rightarrow$ Tutorial of baseline $\rightarrow$ Font 1 with baseline $\rightarrow$ Font 2 with baseline $\rightarrow$ Survey).
The order of using \systemName and the baseline system was randomized for each participant.
After the font design sessions, participants were asked to complete a questionnaire that validated our system.
The entire user study took approximately $60$ minutes, with each tutorial lasting $10$ minutes, each font design session $7$ minutes, and the survey $10$ minutes.



\subsubsection{Results and Discussion}
We compared the designed fonts using our system and the baseline system both quantitatively and qualitatively.
For the quantitative evaluation, we calculated the distance between the target font characters and the designed characters in the \textit{DreamSim} latent space.
As shown in \autoref{tab:userStudyResult}(a), the characters designed with our system closely resembled the target font characters compared to those designed with the baseline system.
Additionally, we measured the style consistency between all characters designed by each participant by calculating the mean distance between the characters ``A'', ``B'', and ``C'' in the \textit{DreamSim} latent space.
As shown in \autoref{tab:userStudyResult}(b), the distance is smaller when using our system to the baseline system, indicating our system enables more style-consistent character design.
In \autoref{fig:userStudyResult}, we showed the characters designed by all participants (P1--P10).
For Font 1, the ``A'' characters designed by P2, P3, P4, P6, and P8 using our system closely matched the slanted style of the target ``A,'' while they failed to design the slanted style using the baseline system, indicating that participants effectively captured the italic feature through multimodal reference.
On the other hand, characters designed by P1, P2, P3, P4, P5, P8, and P10 using the baseline system showed inconsistencies in style within the same font (\eg~variations in size, height, and weight).
In contrast, characters designed with our system exhibited greater consistency, suggesting that style propagation helped create more cohesive designs.

\begin{figure}[ht]
    \centering
    \includegraphics[width=\linewidth]{figures_pdf/fontDesignResult_v6.pdf}
    \caption{
    \textbf{Characters designed by user study participants.}
    Our system enables users to design characters that are more similar to the target font characters and maintain higher consistency between each other.
    In the case of Font 1, participants successfully designed all characters with the slant style using our system, while some participants failed to create the slant style for ``A'' using the baseline system.
    For Font 2, all participants designed characters with consistent styles using our system, whereas the styles of characters designed using the baseline system were inconsistent.
}
\label{fig:userStudyResult}
\end{figure}


\aptLtoX[graphic=no,type=html]{\begin{table}[ht]
\centering
\begin{tabular}{lll}
\multicolumn{3}{c}{\bf (a) Target font similarity $\downarrow$~~~~~}\\
\hline
                & Font 1          & Font 2          \\ \hline
Baseline & 0.1680          & 0.1416          \\
\systemName  & \bestcell{0.1591} & \bestcell{0.1355} \\ \hline
\end{tabular}
\begin{tabular}{lll}
\multicolumn{3}{c}{~~~~~\bf (b) Designed character consistency $\downarrow$}\\
\hline
                & Font 1          & Font 2          \\ \hline
Baseline & 0.3303          & 0.2893          \\
\systemName     & \bestcell{0.2983} & \bestcell{0.2793} \\ \hline
\end{tabular}
\caption{
(a) 
We calculated the distance between the characters designed by the participants and the target font characters.
Each value represents the mean distance across the $12$ characters (``A'', ``B'', ``C'' designed by the four participants).
The characters designed using our system are closer to the ground truth compared to those with the baseline system. 
(b)
We measured the character consistency between the characters ``A'', ``B'', and ``C'' designed by each participant.
Each value represents the mean distance across the three characters designed by each participant.
The distance among the three characters designed using our system is smaller than that with the baseline system, which indicates our system enables more style-consistent character design. 
($\downarrow$ denotes the lower values are better and we highlight the \besthint{best} result for each target font.)
}
\label{tab:userStudyResult}
\end{table}}{\begin{table}[ht]
\centering
\subfloat[Target font similarity $\downarrow$]{
\begin{tabular}{lll}
\toprule
                & Font 1          & Font 2          \\ \midrule
Baseline & 0.1680          & 0.1416          \\
\systemName  & \bestcell{0.1591} & \bestcell{0.1355} \\ \bottomrule
\end{tabular}
}
\subfloat[Designed character consistency $\downarrow$]{
\begin{tabular}{lll}
\toprule
                & Font 1          & Font 2          \\ \midrule
Baseline & 0.3303          & 0.2893          \\
\systemName     & \bestcell{0.2983} & \bestcell{0.2793} \\ \bottomrule
\end{tabular}
}
\caption{
(a) 
We calculated the distance between the characters designed by the participants and the target font characters.
Each value represents the mean distance across the $12$ characters (``A'', ``B'', ``C'' designed by the four participants).
The characters designed using our system are closer to the ground truth compared to those with the baseline system. 
(b)
We measured the character consistency between the characters ``A'', ``B'', and ``C'' designed by each participant.
Each value represents the mean distance across the three characters designed by each participant.
The distance among the three characters designed using our system is smaller than that with the baseline system, which indicates our system enables more style-consistent character design. 
($\downarrow$ denotes the lower values are better and we highlight the \besthint{best} result for each target font.)
}
\label{tab:userStudyResult}
\end{table}}

Next, we evaluated participant feedback to validate the effectiveness of our system.
We asked questions about the functions in our system, including slider operation, multimodal reference, style propagation, and history interface.
When we asked the question \textit{``Were you satisfied with the designed characters?''}, seven out of the ten participants answered yes, while P2 and P10 commented neutral, and P9 expressed no.
P9 noted that he observed distortions in the generated characters and felt the system was not good at generating straight lines.
In response to the question \textit{``Do you think you were able to design fonts easily with the system?''}, all ten participants answered yes, demonstrating the system's effectiveness in enabling non-expert users to design fonts with ease.

In response to the question \textit{``Do you think you were able to effectively use the slider operation for font design?''}, nine participants answered yes.
P4, who answered no, expressed dissatisfaction, stating that while the combination of slider manipulation and multimodal reference was effective, using only the slider and repeatedly clicking the \textsc{Update} button sometimes resulted in a linear subspace that excluded the desired character style.
P4 emphasized the importance of using multimodal reference at the right moments to avoid unsatisfactory suggestions and stated that relying solely on the slider was not effective.
P4 also highlighted that the history interface was useful for reverting to a previous point, leading to the escape of an undesirable search subspace suggested by the system.
P4's feedback reflects the findings suggested in Chan~\etal~\cite{Chan2022}, which indicate that designers working with BO may experience a loss of agency.
In contrast, our method provides users with a way to contribute concrete ideas that guide the BO process, thereby helping them regain a sense of agency.

In response to the question, ``\textit{Do you think you were able to effectively use text input?}'', eight of ten participants answered yes.
P1, P2, P4, P6, P7, P9, and P10 found text input helpful for making broad changes, such as adjusting weight or slant, but not for fine-tuning details or specifying complicated characteristics.
Additionally, P1, P6, and P7 mentioned that understanding typographical terms like ``bold'' and ``italic'' was necessary.
This feedback indicates that while text input is useful for exploring rough font styles, it has limitations in designing font details and requires some typographic knowledge.

Regarding the question, \textit{``Do you think you were able to effectively use the similar fonts provided by font palette?''}, eight of ten answered yes.
P4 and P7 commented that the font palette is particularly helpful when it is difficult to describe the desired font style in texts.
P8, P9, and P10 stated that initializing the search subspace using the font palette function allowed them to begin the design task more smoothly compared to the baseline system.
However, P5 expressed dissatisfaction, stating that the style of the character generated did not perfectly align with the font they selected from the font palette.
This discrepancy, caused by the encoding-decoding process of the font generative model, could lead to confusion among users.
To address this issue, it is important to communicate to users that the generated characters may not always perfectly match the multimodal reference.
Additionally, we anticipate that newer font generative models could help mitigate this discrepancy.
It is worth emphasizing that our proposed system is compatible with any font generative model, provided an efficient font style latent space can be established within it.
On the other hand, P2 explained that he did not use the font palette because he preferred to describe the target font style using text input.
This feedback suggests that using similar font files and text input complement each other.

When asked, \textit{``Do you think you were able to effectively use the \textsc{Update all} button?''}, nine participants responded positively, with eight participants noting that it was more convenient than designing each character individually.
P10, who answered no, expressed dissatisfaction, commenting that it would be more convenient if users could toggle between adjusting either all characters at once or individually. 
In particular, he felt that having a feature to switch to individual adjustments is crucial during the fine-tuning stage.
We focus on the simplicity of the UI in this user study and this individual adjustments function is effective especially when designing many characters like all Roman characters.

In response to the question, ``\textit{Do you feel that you could design characters with a sense of agency using our system?}'', posed only to P5--P10, all six participants responded affirmatively.
P7 and P10 noted that in the baseline system, the line search space was initialized randomly, making the process feel highly dependent on luck. 
In contrast, they appreciated that our proposed system allowed them to control the initialization by specifying their preferences through multimodal references.
This insight aligns with the findings in \autoref{sec:simulatedEvaluation}, which show the initialization using multimodal references leads to better results compared to random initialization.
Additionally, P9 commented that he felt he could convey his intentions to the system by inputting texts.
These insights indicate that our system, leveraging multimodal references, provides users with a greater sense of agency compared to the baseline system.

Seven participants highlighted the usefulness of the history interface during the design process.
P5 remarked that the feature was particularly effective, as there were times when he felt a previous font was better.
In such cases, the history interface allowed him to revisit and continue from that point, saving effort.
He also noted the inconvenience of the baseline system lacking this feature.
P6 commented that comparing the current font displayed on the slider with previously created fonts helped him determine which one aligned more closely with his intended design.
He also said that in the baseline system, he found it challenging to reset after creating a satisfactory font.
In contrast, our system's history interface made him feel more confident about updating or resetting, as it allowed him to aim for even better results without hesitation.
This exemplifies that the history interface is useful not only for storing the designed characters and enabling the users to go back to a past point but also for making them advance the design process as boldly as they want.
It also indicates that the history interface reduces stress and increases freedom and creativity in the design task.

Overall, the feedback suggests that the proposed functions in our system effectively support font design for different participants based on their design preferences and familiarity with typography.

\section{Application Demonstrations}
\label{sec:demonstration}
In this section, we will show a case study where participants are required to design characters in more realistic situations and the adaptation to another writing system rather than Roman characters.


\subsection{Designing Fonts for Graphic Design Purposes}
\label{sec:demonstration_graphic_design}
In practical usage, it is important to evaluate whether our system can support font design for graphic design purposes, such as logo design or advertisement design, as noted by professional designers in \autoref{sec:limitations}.
To explore this, we asked participants to create suitable characters for specific design contexts.

To begin with the conclusion, from the feedback, we observed that participants using our system did not initially have a clear vision of the font they wanted. 
However, as they explored different font styles, they drew inspiration from the designs they encountered, ultimately creating their own unique characters.
While participants occasionally struggled with fine adjustments, such as correcting distorted lines, they generally felt they were able to create fonts that aligned with their intended concepts.
In the following sections, we present two design scenarios: conference logo design and advertisement poster design.
In the conference logo task, four participants (P11--P14) created characters for a logo, demonstrating a variety of font styles using our system.
In the advertisement poster task, another six participants (P15--P20) designed characters for different posters, tailoring their fonts to the target concepts.

\subsubsection{Design a Conference Logo}
In this experiment, participants with no prior font design experience were tasked with designing a conference logo.
Specifically, they were asked to create the characters ``CHI 2025'' to complement the cherry blossom motif in the conference logo.
After receiving an introduction to using our system, participants completed the task, and their feedback was collected.
During the design process, participants were only shown the cherry blossom logo and were not aware of the characters in the official conference logo.
As shown in \autoref{fig:CHILogoDesign}, the designs varied among participants.
P11 noted that her designed characters complemented the cherry blossom logo, highlighting that her favorite aspect was the fading central lines in ``H'' and ``5.''
This fading part appeared accidentally but complements the logo from her point of view, so she adopted it.
She also attempted to replicate this effect in ``2,'' but it was unsuccessful.
P12 said that he thought a decent and calm font was suitable for the conference logo and tried to make such a font.
He also commented that the font he designed was $90$ out of $100$ in terms of satisfaction, though his attempt to make ``I'' more straight was not successful.
P13 commented that he aimed to create a cute font inspired by the Japanese subculture, opting for a bold and rounded design.
For the initial step, he input the text, ``I want a cute and thick font" and found that the system performed as he hoped.
He also noted that the slider was effective in fine-tuning character details and eliminating unwanted distortions.
He was proud of the font he designed and believed it could be used in real-world applications, as the style of each character was well aligned.
P14 noted that he thought a thin and brush-style font fitted the Japanese-style logo and tried to make it.
He found multimodal input effective for the early stages of rough font design but felt it was less suitable for detailed exploration, ultimately relying on slider manipulation to refine the characters.
He was confident with the quality except for the noise and distortion in the designed characters.

\begin{figure}[ht]
    \centering
    \includegraphics[width=0.95\linewidth]{figures_pdf/CHILogo_v7.pdf}
    \caption{
    \textbf{Designed characters for the conference logo.}
    Four participants designed the characters for the conference logo.
    They designed a diverse range of fonts based on their unique sensibilities.
}
\label{fig:CHILogoDesign}
\end{figure}



\subsubsection{Design Advertisement Posters}
This demonstration shows font design for advertisement posters using our system, as illustrated in \autoref{fig:poster}.
During the design process, participants (P15--P20), who were introduced to the use of our system, were given a scenario and shown only the background image, with the task of creating characters that matched the visual context.
P15 and P16, both familiar with CJK writing systems, designed characters for an autumn foliage festival.
P5 rated his design $9$ out of $10$, expressing satisfaction with the traditional and formal font style he aimed to achieve.
He efficiently initialized the search space by inputting the text ``yu-mincho, serif'' (with ``yu-mincho'' being one of the most popular CJK fonts).
P16 commented that he envisioned a calm and warm font for the festival and was pleased with the result. 
He noted that their initial idea was simply based on the keyword ``warm,'' which he input into the system.
As he explored various styles, he gradually refined his design and reached a point of satisfaction.
P17, who designed the summer sale poster, aimed for a thin and refreshing font.
He observed elements in the background image such as the central white line, the seagull's wings, and the wave’s border, and decided that the character weight should align with these features.
By adjusting the slider, he was able to find a suitable font weight, though he expressed some dissatisfaction with the distortion of the top horizontal bar in the letter ``E.''

P18, tasked with designing a Halloween poster, felt that a twisted font suited the Halloween theme. 
She also believed a cute, handwritten style complemented the surrounding elements like the pumpkin, house, and bat, and was satisfied with the bold characters she created. 
She began by inputting a bold and italic font into the system, then continued refining the design using only the slider suggested by the Bayesian optimization process.
She expressed confidence in using the system, despite having no prior font design experience, and enjoyed the process. 
She effectively utilized the system's features, such as reverting to previous iterations via the history area when her exploration veered in an undesired direction, and she repeatedly refined each character after the initial style propagation.
For details on P18's design process, refer to the supplemental material.
P19 designed characters for a birthday card, aiming for cursive and fashionable font, and expressed satisfaction with the result.
P20 created a font for a movie poster, aiming for characters that were "scary," "thick," and "retro," with shapes fitting within a rectangular form (e.g., the shape of "S" resembling a rectangle).
While he was generally satisfied with the overall design, he found it challenging to achieve symmetry in characters like "A," "M," and "T."
Overall, although participants encountered challenges in addressing minor distortions and style inconsistencies, all expressed satisfaction with their final designs.


\begin{figure}[t]
    \centering
    \includegraphics[width=\linewidth]{figures_pdf/Poster_v5.pdf}
    \caption{
    \textbf{Designed characters for the advertisement posters.}
    The participants designed the characters while viewing background images for the posters.
    The two posters on the top left are for an autumn foliage festival.
    P17 created a poster for a summer sale, while P18 designed characters for a Halloween event.
    P19 developed a font for a birthday card, and P20 created one for a movie poster.
}
\label{fig:poster}
\end{figure}


\subsection{Designing CJK Fonts}
In the user study (\autoref{sec:user-study}), we demonstrated that participants could efficiently design Roman characters using our proposed system.
 By swapping the font generative model, the system can also support other writing systems, including Chinese, Japanese, and Korean (CJK).
As shown in \autoref{fig:CJKDesign}, users can efficiently design CJK characters without the need of predesigned examples.
Once the design process is complete, users can download their custom fonts as OTF files.




\begin{figure}[b]
    \centering
    \includegraphics[width=\linewidth]{figures/CJKDesign_v2.png}
    \caption{
    \textbf{Screenshots of CJK font character design.}
    The demonstration of CJK character designs using our system.
    The order is displayed in the upper right corner of each screenshot.
    See the supplemental material for the video.
}
\label{fig:CJKDesign}
\end{figure}

\section{Discussion}


The value of hybrid paper-digital interfaces that augment physical documents with digital content has been well established~\cite{han_hybrid_2021}. To realize hybrid documents, we propose a novel watermarking method that uses IR-based printing to store digital content as an intrinsic part of the document. Unlike previous approaches that rely on a network connection and external storage for the digital assets, our approach with \systemName~ maintains the document format as the single container of both physical and digital contents. 
% \hl{This approach provides additional benefits that interactions are network-free and more privacy-preserving.}
In this section, we discuss the limitations of using \systemName~ from the perspectives of authoring, printing, detection, and consumption. We also discuss future work to improve the approach.



\subsection{Towards Multimodal Content}
\systemName's interface currently supports only a rudimentary form of AR that only permits 2D visual content and audio and does not leverage multimodal content to its full potential. We hope that we will enable authors to embed more intricate objects and interactions in the future. Adding support for multimedia assets will pave the way to expanding on previously examined use cases ~\cite{alessandrini_audio-augmented_2014, rajaram_paper_2022}. The most impactful addition would be that of 3D assets since that would allow users to utilize the document as a spatial anchor. 



\subsection{Effects on Printed IR Ink}




\new{Our experiments investigated human and machine detectability under controlled indoor lighting and viewing conditions.}
\newCameraReady{However, the perception and reliability of printed IR ink can vary significantly depending on external factors such as illumination levels, viewing angles, and the type of paper used.
One key limitation is that low-light conditions may reduce the effectiveness of human detectability due to insufficient IR reflection, while excessive illumination, such as direct sunlight, could lead to overexposure, affecting both machine readability and ink longevity. Future work could explore adaptive imaging techniques or optimized illumination setups to mitigate these issues. 
Regarding \textit{paper type}, we anticipate that glossy or coated paper may introduce reflection artifacts, which could interfere with machine-based recognition systems by creating specular highlights that obscure ink contrast.
Conversely, highly porous or rough-textured paper might cause ink diffusion, potentially reducing print sharpness and affecting detection accuracy.
A promising direction is to systematically evaluate these effects using an experimental framework similar to Xu et al.'s methodology, where QR code robustness was tested under varying lighting conditions, scanning angles, and environmental factors \cite{xu_art-up_2021}. Applying a similar protocol could help quantify how different conditions impact IR ink detection and visibility.}



As with any other dye-based ink, UV-dependent fading is an inherent limitation~\cite{maxmax_-_llewellyn_data_processing_ir_2022} that is important to consider for document reliability and permanence. Constant exposure to heavy UV sources such as the sun can cause the inks to fade over time.
\citet{willis_hideout_2013} showed how different IR ink types can be used in conjunction with UV-resistant coatings to achieve 98\% contrast preservation under office lighting conditions. We envision that for commercial use cases where long-term preservation is desired, a similar coating can be applied, which is sufficient for indoor applications.


\subsection{Invisible IR Ink}
We used a psychophysical experiment to determine conservative estimates for the IR ink densities that remain invisible to users for a wide range of different background colors. As with most experiments, we were limited by the number of background colors that we could include in the experiment. \new{Therefore, we do not know how our results generalize to other background colors, combinations of colors, and different graphic patterns. Nevertheless, we developed a system that facilitates embedding with invisible IR ink for any RGB color. Note that our method is very conservative and favors invisibility over machine detection. The actual DTs are likely much higher, which should further improve machine detectability. Nevertheless, we contribute a methodology that designers can apply to determine the "sweet spot" between visibility, data capacity, and machine detectability.}




We also want to highlight that the gradient IR ink and background color bars used in our psychophysical experiment differ from QR codes which may have had an effect on our estimated invisible IR ink DTs. We decided against using QR codes in the experiment because (1) it would have only been possible to test QR codes at predefined densities (e.g., at 20\%, 40\%, ...), limiting the precision of the study, and (2) the limited number of samples, which would not reflect the variability of QR codes of different sizes. As a result, our experiment also aims to provide general estimates for shapes other than QR codes and may, therefore, be used as a starting point for any type of invisible IR-printed marker.   



\subsection{Discoverability and Practicality}
\label{NIR_AR_FormFactor}

\new{
To demonstrate \systemName, we used NIR-based fabrication and detection tools. While NIR cameras are getting popular in many handheld devices (e.g., \textit{iPhone} and \textit{iPad} use it for facial recognition and LIDAR 3D scanning), not all platforms currently give 3rd-party developers access to the raw NIR stream (e.g., only the processed depth map can be accessed on \textit{iOS}). We argue the interest in NIR applications will increase as more use cases are demonstrated by future projects.}
% And we want to add this to AR headsets in the future, which already come with NIR cameras (conventionally used for depth sensing).

In our current implementation, \systemName~ documents can optionally be marked with a small icon or visual label on the document in our embedding tool
to allow users to discover embedded AR content.
\new{We envision that next-generation AR hardware can more fully leverage \systemName's utility. Compared to using a handheld device, always-on AR smart glasses such as \textit{Meta} \textit{Orion} could be constantly scanning the environment for hidden AR content~\cite{di_gioia_investigating_2022, campos_zamora_moirewidgets_2024}.
Currently, most AR glasses already leverage NIR cameras, but mainly for localizing the user and mapping their environment for 3D tracking purposes. With head-worn AR glasses with integrated NIR cameras used for \systemName~ sensing, the user would not have to manually capture objects using the phone during the interaction. We recommend future research to focus on implementing this on AR glasses.}



\section{Conclusion}


We introduced \systemName, a watermarking technique that embeds computer-readable information while remaining invisible to the human eye. Unlike previous methods, our approach utilizes the entire document, including white spaces, and works regardless of background color. Through a psychophysical experiment, we determined the maximum ink that can be embedded without being detected. We developed tools to support users in applying IR ink technology, including software for efficient information embedding and a universal camera module for capturing \systemName~watermarks. Our open-source ML pipeline processes these images for robust use with standard QR code readers. We demonstrated various use cases, highlighting the potential of invisible IR content for hybrid paper-digital interfaces and advancing watermarking techniques.






\section*{Acknowledgements}
This is acknowledgment.
\bibliographystyle{ACM-Reference-Format}
\bibliography{bib_font}



\end{document}
