\section{Related Work}
\subsection{Automatic Font Generation}
Font generation aims to create characters with a specific font style, ultimately leading to the creation of new font libraries.
Researchers have proposed various methods for generating bitmap Roman fonts, such as blending styles from template fonts~\cite{Igarashi2010}, constructing a font manifold~\cite{CampbellFontManifold2014}, and manipulating attribute scores~\cite{wang2020attribute2font}.
Additionally, recent studies have focused on synthesizing outline fonts in vector format using deep generative networks~\cite{liu2023dualvector, xia2023vecfontsdf, wang2021deepvecfont, wang2023deepvecfontv2, thamizharasan2024vecfusion}.
These works tackle the challenges by representing character outlines as sequences of tokens~\cite{wang2021deepvecfont, wang2023deepvecfontv2} or signed distance functions (SDFs)~\cite{liu2023dualvector, xia2023vecfontsdf}.
While these approaches successfully synthesize vector fonts, non-expert users may find it difficult to use them directly, as they require pre-designed characters in target fonts or manipulating various kinds of attribute scores~\cite{wang2020attribute2font}.

On the other hand, creating Chinese, Japanese, and Korean (CJK) fonts, which consist of a vast number of complex characters, requires different approaches than generating Roman fonts.
Some methods attempt to generate CJK fonts by utilizing extracted metadata, such as radicals and strokes~\cite{LianEasyFont2017, SounghuaAutomaticGeneration2009, ZhouEasyGeneration2011, ZongStrokeBank2014}.
However, these approaches face significant challenges, particularly the need for a large number of character examples.
For example, generating a font with \num{2,550} characters requires \num{522} character examples in the desired font style~\cite{LianEasyFont2017}.

To overcome these problems, recent deep learning-based works~\cite{YuchenZi2zi2017,ChaDMFont2020,JiangDCFont2017,ParkMultipleHeads2021,SunSAVAE2018} treat the font generation problem as a style transfer problem.
However, these methods require labeled data, such as radicals of characters.
In contrast, several approaches aim to train font-generation models without relying on domain knowledge~\cite{JiangDCFont2017, XieDGFont2021, ZhangEMD2018, liu2024qtfont, yang2024fontdiffuser}.
Among them, \textit{DG-Font}~\cite{XieDGFont2021} combines style and content using adaptive instance normalization (\textit{AdaIN} \cite{HuangAdaIN2017}), a straightforward yet effective style transfer technique that aligns the mean and variance of content with those of style.
This method requires only a few character examples in the desired font.
More recently, diffusion model-based methods~\cite{ho2020denoising} have achieved high-quality and high-resolution font generation~\cite{liu2024qtfont, yang2024fontdiffuser, he2024difffont, fu2024MSD}.

In this paper, we utilize the extended \textit{DG-Font} to generate fonts without the need to prepare character examples.
Although \textit{DG-Font} is not the latest model, its latent space is easier to explore than the latent space of diffusion-based font-generative models~\cite{liu2024qtfont, yang2024fontdiffuser, he2024difffont}.
Notably, our proposed system is compatible with other pretrained font generative models that utilize a font style latent space.

\subsection{Human-in-the-Loop Bayesian Optimization}
Bayesian optimization~\cite{Brochu2010B, ShahriaiReviewBO2016} is a widely used method for optimizing black-box functions.
It is particularly useful for functions that are expensive to evaluate because it aims to find the optimal value with minimal iteration, which is achieved by selecting queries that are most effective in terms of exploration and exploitation.

To reduce the number of expensive human evaluations, researchers have tried to integrate Bayesian optimization with human-in-the-loop systems~\cite{Brochu2010A, Brochu2007, KoyamaGallery2020, KoyamaSequential2017, ZhouGenerativeMelody2020, kadner2021adaptifont,Mo2024, chong2021interactive}.
For instance, Koyama~\etal\ \cite{KoyamaSequential2017} propose a method called Sequential Line Search (SLS), which finds the optimal value in the multi-dimensional space by tweaking a one-dimensional slider that is easy for humans to perform.
The line explored by the one-dimensional slider connects the point expected to be optimal and the point at which the acquisition function is maximized.
Building on SLS, Zhou~\etal~\cite{ZhouGenerativeMelody2020} propose a framework for generating melody compositions.
Their framework transforms the task of adjusting a one-dimensional slider into selecting the most favorable candidate from a set of options.
This adaptation enhanced user interaction while leveraging the strengths of Bayesian optimization.
Kadner~\etal~\cite{kadner2021adaptifont} introduce a human-in-the-loop system for font generation, focusing on optimizing fonts for readability through Bayesian optimization.
In contrast, our work expands the scope of font design by integrating SLS with multimodal references, simplifying the creation of fonts for a variety of applications beyond readability.

A notable limitation of human-in-the-loop Bayesian optimization is that it tends to reduce user agency in the design process and decrease their sense of ownership over the outcomes~\cite{Chan2022}.
Chan~\etal~\cite{Chan2022} suggest that enhancing users' ability to express their ideas to the optimizer can effectively improve both agency and ownership.
Previous works have addressed this issue by enabling users to directly incorporate preferences in various approaches, such as specify areas in the design space they wish to exclude~\cite{Mo2024}, edit the generated melody~\cite{ZhouGenerativeMelody2020} and images~\cite{chong2021interactive} directly.
While these approaches allow users to incorporate their preferences directly into the Bayesian optimization process, they assume that users' preferences are time-invariant~\cite{KoyamaGallery2020} and restrict users to a forward-directed design workflow, limiting flexibility in revisiting or re-evaluating earlier steps.

In contrast, our proposed method enables users to incorporate their preferences into the Bayesian optimization process using multimodal references, including text input.
This multimodal interactive capability is a novel improvement over previous methods.
Additionally, we provide a user interface that effectively visualizes the interaction history between the user and the system. 
This visualization allows users to easily understand the design history and revisit or re-evaluate earlier points, freeing them from the constraints of a strictly forward-directed optimization process.