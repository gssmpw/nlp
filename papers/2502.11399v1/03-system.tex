\newcommand{\argmax}{\mathop{\rm arg~max}\limits}
\begin{figure*}[htb]
    \centering
    \includegraphics[width=0.95\linewidth]{figures_pdf/fontcraft_ui_v8.pdf}
    \caption{
    \textbf{\systemName UI.}
    Users manipulate the slider in (a) the character design area to explore the line search subspace provided by the system.
    They can also input multimodal references using (b) the multimodal input area.
    They can obtain a new recommendation by pressing the \textsc{Update} button.
    Once users are satisfied with the current style of the focused character, they can propagate its style to all other characters by pressing the \textsc{Update All} button, and the results can be viewed in (c) the character collection area.
    Optionally, users can select another character and further refine it.
    (d) The history area shows the sequence of user inputs and system outputs, enabling users to easily track their exploration history and revert to a specific checkpoint if needed.
    }
    \label{fig:UI}
\end{figure*}

\section{\systemName System Overview}

\subsection{System Architecture}
The overall architecture of \systemName, an effective system for font design, consists of two main components: the user interface (UI) and the font generative model.
Users use UI to explore the font-style latent space and select desired font styles.
The font generative model generates the bitmap representation of characters from latent variables (see \autoref{sec:fontgenerativemodel}).

\subsection{User Interface}
The UI is designed to be simple and user-friendly, particularly for users who have never designed fonts before.
As shown in \autoref{fig:UI}, our UI contains the character design area, the multimodal input area, the character collection area, and the history area.
Users interact with the system by adjusting a slider or providing multimodal data, and they can check real-time previews of generated fonts.
This interactive feedback loop allows users to iteratively refine their font choices based on visual aesthetics.
Each of these elements is crucial in facilitating the font design process.
We introduce them in this subsection.

\subsubsection{Character Design Area}
\label{sec:primaryComponent}
The character design area is the most frequently used element for exploration within the UI.
It consists of a single slider, an image viewer, and three control buttons.
The slider enables users to explore the one-dimensional latent subspace determined by either Bayesian optimization or multimodal reference inputs (see \autoref{sec:multiModalBayesianOptimization}).
The image viewer shows the currently focused character (``A'' in \autoref{fig:UI}) in the selected style, rendered in vector format.
We generate the vector character in the following steps.
First, the bitmap character is generated by the font generative model using the style latent vector selected by the handle on the slider.
Then, we reduced artifacts in the generated bitmap character image by simply setting any pixel with a grayscale value above a certain threshold to white.
Finally, we converted the filtered bitmap character into SVG format by tracing the outlines using the Potrace algorithm~\cite{selinger2003potrace}.

The three control buttons, \textsc{Reset}, \textsc{Update}, and \textsc{Update All}, serve specific functions:
\begin{itemize}
    \item \textsc{Reset}: to clear any accumulated preferences in Bayesian optimization for a focused character and reinitialize it, allowing users to start exploring the font style for that character from scratch.
    \item \textsc{Update}: to submit the selected point on the slider as the current user preference for a focused character, requesting the Bayesian optimization process to recommend a new search subspace for exploration in the next iteration.
    \item \textsc{Update All}: to propagate the style the user prefers for all characters and request the Bayesian optimization process to recommend the next search subspace.
\end{itemize}

\subsubsection{Multimodal Input Area}
\label{sec:multimodalInputComponent}
The multimodal input area allows users to provide multimodal references, including text, images, and existing font files.
These references are used to initialize or customize the font style exploration process by constructing a new search subspace.
By providing specific references, users can directly influence the system's output, making it easier for them to design desired fonts.
We explain how to encode the multimodal references into the font style latent space in \autoref{sec:multiModalBayesianOptimization}.

\subsubsection{Character Collection Area}
\label{fig:characterCollectionComponent}
The character collection area displays previews of the generated characters in vector format.
Users can zoom in on each character to inspect for any defects and select a specific character to focus on.
Once a character is selected, users can refine its font style in the character design area.

\subsubsection{History Area}
\label{fig:historyComponent}
The history area displays a sequence of user inputs and system outputs, allowing users to track their design progress.
Users can select any previous output to revert to that stage, which enables them to undo actions and restart the font design process from a specific point.
This feature allows users to retract undesired preferences and update their preferences during the design process.
