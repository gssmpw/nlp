\section{Introduction}
\label{introduction}
Designing a new font for posters, websites, and advertisement banners is a challenging task, even for professional designers.
It requires a significant amount of repetitive manual effort because the designers need to create a whole set of characters.
For example, a Roman font contains \num{62} characters including ``A--Z'', ``a--z'', and ``0--9''.
Moreover, when it comes to other writing systems, for example, it takes about \num{12} months for three to five experts to design a \textit{GB18030-2000} Chinese font comprising \num{27,533} characters, according to \textit{FounderType}\footnote{\url{https://www.foundertype.com/}}, a Chinese font company.
Additionally, font design necessitates adherence to typography-specific criteria to ensure that fonts maintain consistency, meaning all characters share a uniform style.

To reduce the manual effort required for designing fonts, many previous methods have leveraged generative models to generate accurate and diverse fonts~\cite{YuchenRewrite2016, YuchenZi2zi2017, UpchurchA2Z2016, liu2023dualvector, xia2023vecfontsdf, wang2021deepvecfont, wang2023deepvecfontv2, thamizharasan2024vecfusion, JiangDCFont2017, XieDGFont2021, ZhangEMD2018, liu2024qtfont, yang2024fontdiffuser}.
These methods formulate font generation as a \emph{style transfer} problem: The style of predesigned character examples is transferred to the target characters while preserving their structure.
Although these methods generate high-quality fonts, they still have a limitation that hinders their usefulness for non-expert users in designing new fonts.
Specifically, they require users to create character examples in desired font styles using font-editing software, which poses a challenge for non-expert users.
For instance, \textit{DualVector}~\cite{liu2023dualvector}, one of the latest Roman font generation models, requires \num{3}--\num{5} predesigned character examples.


To tackle the challenges of font creation, we propose \systemName, a novel system that allows users to create new fonts \emph{\textbf{without needing to prepare specific character examples}}, making it especially user-friendly for non-experts.
As shown in~\autoref{fig:teaser}, users can create desired fonts containing numerous characters by iteratively adjusting a slider and providing multimodal references.
Our system allows them to explore within a subspace of the font style latent space of a font generative model.
The process begins with users exploring the style for a single character, such as ``A'' or ``z,'' and then propagating a selected style to the remaining characters.
Once the style is propagated, users can choose an unsatisfactory character (if any) and refine it using the same design procedure as they did with the initial character.
By repeating this process, users can ultimately design a font that meets their preferences.
Users then obtain a complete outline font in OpenType format.

The core of our system is a novel latent space exploration method that combines human-in-the-loop preferential Bayesian optimization (PBO) and multimodal references. 
Our method has two main technical contributions: \emph{\textbf{multimodal-guided subspace}} and \emph{\textbf{retractable preference modeling}}, which addresses two key limitations in existing human-in-the-loop PBO.

Human-in-the-loop PBO has been widely used to obtain the optimal solution of user preference function for visual design parameter adjustment~\cite{KoyamaSequential2017}, photographic lighting design~\cite{bayelight}, and exploring generative images and melodies~\cite{chong2021interactive,ZhouEasyGeneration2011}.
Similarly, in our system, users explore the font style latent space of a font generation model by selecting their preferred styles from candidates recommended through Bayesian optimization.
However, relying solely on PBO in the design process can diminish users' sense of agency, creativity, and ownership~\cite{Chan2022}.
To address this issue, we propose to construct multimodal-guided subspaces, which enables users to directly convey their preference to the PBO process using texts and images.
Specifically, we map user-provided multimodal references to points in the search subspace by encoding fonts that are similar to these references from an existing font database~\cite{o2014exploratory}.
To retrieve these similar fonts, our method leverages FontCLIP~\cite{tatsukawa2024fontclip}, a typography visual-language model, and constructs a new search subspace that incorporates the encoded points from the retrieved fonts.
By combining the multimodal-guided subspace with the subspace generated by the previous Bayesian optimization method~\cite{KoyamaSequential2017}, our approach enables users to design their desired fonts more efficiently.

Additionally, previous PBO methods assume that users' preferences remain consistent throughout the design process~\cite{KoyamaGallery2020}.
As a result, users cannot retract their preferences during the font design process.
To overcome this limitation, we introduce a history interface that supports retractable preference modeling.
This interface allows users to review their design history, revisit earlier states, and restart from a specified past design.
This feature is particularly valuable when users change their preferences during the design process, freeing them from the limitations of an irretractable workflow.

Furthermore, we introduce a style propagation and refinement feature, enabling users to achieve consistent styling across all characters easily.
Once users design a character with the desired style, they can propagate that style to all other characters.
They can then fine-tune any characters that require additional adjustments until satisfied.
This feature not only simplifies the design process but also ensures consistent styling throughout the entire font set.

To the best of our knowledge, \systemName is the first system that enables font design utilizing efficient font style exploration without requiring pre-designed character examples, significantly lowering the barrier to font design.
To accomplish this, our system integrates a human-in-the-loop Bayesian optimization method utilizing multimodal input with well-organized features such as history interface and style propagation.

We conducted a user study to evaluate how efficiently non-expert users could design fonts using our system, both quantitatively and qualitatively.
The study compared our system to a baseline system that solely relied on a single slider with basic PBO.
The study aimed to verify the advantages of key features in our proposed system, including \textit{the integration of PBO with \textbf{multimodal input}} and features such as \textit{history interface for retractable preference modeling} and \textit{the combination of style propagation and refinement}.
In the user study, participants without font design experience were tasked with creating fonts using both our system and the baseline.
We also collected feedback from the participants to assess their satisfaction with our system.
Both quantitative and qualitative analyses of the fonts created by participants revealed that our system produced better font designs compared to the baseline.
Survey responses also confirmed that the proposed system features significantly enhanced the efficiency of the font design process.
Additionally, we demonstrated that our system \emph{\textbf{supports other writing systems}}, such as Chinese, Japanese, and Korean (CJK), and is effective in practical applications, including logo and advertisement design.
These findings show that our system is not only applicable to non-Roman font design but also useful in practical, real-world design scenarios.

\paragraph{Contributions}
To sum up, we make the following contributions:
\begin{itemize}[leftmargin=0.5cm]
    \item We present \systemName, an interactive font design system that simplifies the process of creating fonts across various writing systems, making the design process accessible to non-experts.
    \item We propose a method that combines human-in-the-loop Bayesian optimization and multimodal references, enabling user interaction within the multimodal-guided subspace.
    \item We introduce a history interface that allows users to retract and update their preferences during the design process, which cannot be done in previous human-in-the-loop PBO methods.
    \item We develop an iterative style propagation and refinement method to ensure a consistent style throughout the font set.
\end{itemize}
