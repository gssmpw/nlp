\section{Application Demonstrations}
\label{sec:demonstration}
In this section, we will show a case study where participants are required to design characters in more realistic situations and the adaptation to another writing system rather than Roman characters.


\subsection{Designing Fonts for Graphic Design Purposes}
\label{sec:demonstration_graphic_design}
In practical usage, it is important to evaluate whether our system can support font design for graphic design purposes, such as logo design or advertisement design, as noted by professional designers in \autoref{sec:limitations}.
To explore this, we asked participants to create suitable characters for specific design contexts.

To begin with the conclusion, from the feedback, we observed that participants using our system did not initially have a clear vision of the font they wanted. 
However, as they explored different font styles, they drew inspiration from the designs they encountered, ultimately creating their own unique characters.
While participants occasionally struggled with fine adjustments, such as correcting distorted lines, they generally felt they were able to create fonts that aligned with their intended concepts.
In the following sections, we present two design scenarios: conference logo design and advertisement poster design.
In the conference logo task, four participants (P11--P14) created characters for a logo, demonstrating a variety of font styles using our system.
In the advertisement poster task, another six participants (P15--P20) designed characters for different posters, tailoring their fonts to the target concepts.

\subsubsection{Design a Conference Logo}
In this experiment, participants with no prior font design experience were tasked with designing a conference logo.
Specifically, they were asked to create the characters ``CHI 2025'' to complement the cherry blossom motif in the conference logo.
After receiving an introduction to using our system, participants completed the task, and their feedback was collected.
During the design process, participants were only shown the cherry blossom logo and were not aware of the characters in the official conference logo.
As shown in \autoref{fig:CHILogoDesign}, the designs varied among participants.
P11 noted that her designed characters complemented the cherry blossom logo, highlighting that her favorite aspect was the fading central lines in ``H'' and ``5.''
This fading part appeared accidentally but complements the logo from her point of view, so she adopted it.
She also attempted to replicate this effect in ``2,'' but it was unsuccessful.
P12 said that he thought a decent and calm font was suitable for the conference logo and tried to make such a font.
He also commented that the font he designed was $90$ out of $100$ in terms of satisfaction, though his attempt to make ``I'' more straight was not successful.
P13 commented that he aimed to create a cute font inspired by the Japanese subculture, opting for a bold and rounded design.
For the initial step, he input the text, ``I want a cute and thick font" and found that the system performed as he hoped.
He also noted that the slider was effective in fine-tuning character details and eliminating unwanted distortions.
He was proud of the font he designed and believed it could be used in real-world applications, as the style of each character was well aligned.
P14 noted that he thought a thin and brush-style font fitted the Japanese-style logo and tried to make it.
He found multimodal input effective for the early stages of rough font design but felt it was less suitable for detailed exploration, ultimately relying on slider manipulation to refine the characters.
He was confident with the quality except for the noise and distortion in the designed characters.

\begin{figure}[ht]
    \centering
    \includegraphics[width=0.95\linewidth]{figures_pdf/CHILogo_v7.pdf}
    \caption{
    \textbf{Designed characters for the conference logo.}
    Four participants designed the characters for the conference logo.
    They designed a diverse range of fonts based on their unique sensibilities.
}
\label{fig:CHILogoDesign}
\end{figure}



\subsubsection{Design Advertisement Posters}
This demonstration shows font design for advertisement posters using our system, as illustrated in \autoref{fig:poster}.
During the design process, participants (P15--P20), who were introduced to the use of our system, were given a scenario and shown only the background image, with the task of creating characters that matched the visual context.
P15 and P16, both familiar with CJK writing systems, designed characters for an autumn foliage festival.
P5 rated his design $9$ out of $10$, expressing satisfaction with the traditional and formal font style he aimed to achieve.
He efficiently initialized the search space by inputting the text ``yu-mincho, serif'' (with ``yu-mincho'' being one of the most popular CJK fonts).
P16 commented that he envisioned a calm and warm font for the festival and was pleased with the result. 
He noted that their initial idea was simply based on the keyword ``warm,'' which he input into the system.
As he explored various styles, he gradually refined his design and reached a point of satisfaction.
P17, who designed the summer sale poster, aimed for a thin and refreshing font.
He observed elements in the background image such as the central white line, the seagull's wings, and the wave’s border, and decided that the character weight should align with these features.
By adjusting the slider, he was able to find a suitable font weight, though he expressed some dissatisfaction with the distortion of the top horizontal bar in the letter ``E.''

P18, tasked with designing a Halloween poster, felt that a twisted font suited the Halloween theme. 
She also believed a cute, handwritten style complemented the surrounding elements like the pumpkin, house, and bat, and was satisfied with the bold characters she created. 
She began by inputting a bold and italic font into the system, then continued refining the design using only the slider suggested by the Bayesian optimization process.
She expressed confidence in using the system, despite having no prior font design experience, and enjoyed the process. 
She effectively utilized the system's features, such as reverting to previous iterations via the history area when her exploration veered in an undesired direction, and she repeatedly refined each character after the initial style propagation.
For details on P18's design process, refer to the supplemental material.
P19 designed characters for a birthday card, aiming for cursive and fashionable font, and expressed satisfaction with the result.
P20 created a font for a movie poster, aiming for characters that were "scary," "thick," and "retro," with shapes fitting within a rectangular form (e.g., the shape of "S" resembling a rectangle).
While he was generally satisfied with the overall design, he found it challenging to achieve symmetry in characters like "A," "M," and "T."
Overall, although participants encountered challenges in addressing minor distortions and style inconsistencies, all expressed satisfaction with their final designs.


\begin{figure}[t]
    \centering
    \includegraphics[width=\linewidth]{figures_pdf/Poster_v5.pdf}
    \caption{
    \textbf{Designed characters for the advertisement posters.}
    The participants designed the characters while viewing background images for the posters.
    The two posters on the top left are for an autumn foliage festival.
    P17 created a poster for a summer sale, while P18 designed characters for a Halloween event.
    P19 developed a font for a birthday card, and P20 created one for a movie poster.
}
\label{fig:poster}
\end{figure}


\subsection{Designing CJK Fonts}
In the user study (\autoref{sec:user-study}), we demonstrated that participants could efficiently design Roman characters using our proposed system.
 By swapping the font generative model, the system can also support other writing systems, including Chinese, Japanese, and Korean (CJK).
As shown in \autoref{fig:CJKDesign}, users can efficiently design CJK characters without the need of predesigned examples.
Once the design process is complete, users can download their custom fonts as OTF files.




\begin{figure}[b]
    \centering
    \includegraphics[width=\linewidth]{figures/CJKDesign_v2.png}
    \caption{
    \textbf{Screenshots of CJK font character design.}
    The demonstration of CJK character designs using our system.
    The order is displayed in the upper right corner of each screenshot.
    See the supplemental material for the video.
}
\label{fig:CJKDesign}
\end{figure}
