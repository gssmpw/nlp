\documentclass[conference]{IEEEtran}
\usepackage{cite}
\usepackage{amsmath,amssymb,amsfonts}
\usepackage{algorithm}
\usepackage{algorithmic}
\usepackage{graphicx}
\usepackage{textcomp}
\usepackage{xcolor}
\usepackage[hyphens]{url}
\usepackage{fancyhdr}
\usepackage{hyperref}
\usepackage{array}
\usepackage{tabularx}
\usepackage{bm}
\usepackage{booktabs}
\usepackage{authblk}
\usepackage[switch]{lineno}
\usepackage{multirow}
\usepackage{adjustbox}
\usepackage{float}
\usepackage{makecell, hhline}

\usepackage{graphicx}
\usepackage{multirow}
\usepackage{adjustbox}
\usepackage{amsmath}
\usepackage{amssymb}
\usepackage{booktabs}

\usepackage{cite}
\usepackage{amsmath,amssymb,amsfonts}
\usepackage{algorithmic}
\usepackage{graphicx}
\usepackage{textcomp}
\usepackage{xcolor}
\usepackage{hyperref}
\usepackage[a4paper, total={184mm,239mm}]{geometry}

\def\BibTeX{{\rm B\kern-.05em{\sc i\kern-.025em b}\kern-.08em
    T\kern-.1667em\lower.7ex\hbox{E}\kern-.125emX}}

% Ensure letter paper
\pdfpagewidth=8.5in
\pdfpageheight=11in

\def\invcircledast#1{%
  \mathbin{\vphantom{\circledast}\text{%
    \ooalign{\smash{\blackcircle}\cr
             \hidewidth\smash{\textcolor{white}{\bf \footnotesize $#1$}}\hidewidth\cr
            }%
  }}%
}
\newcommand{\blackcircle}{\raisebox{-.6ex}{\scalebox{2.30}{$\bullet$}}}


\pagenumbering{arabic}

\title{Hyperdimensional Intelligent Sensing for Efficient Real-Time Audio Processing on Extreme Edge}

\author{Sanggeon Yun$^1$, Ryozo Masukawa$^1$, Hanning Chen$^1$, SungHeon Jeong$^1$, Wenjun Huang$^1$\\Arghavan Rezvani$^1$, Minhyoung Na$^2$, Yoshiki Yamaguchi$^3$ and Mohsen Imani$^1{}^{\star}$ \\ $^1$University of California, Irvine, CA, USA\\ $^2$Kookmin University, Seoul, Republic of Korea\\ $^3$Shibaura Institute of Technology, Saitama, Japan\\
$^{\star}$Corresponding Author: m.imani@uci.edu \vspace{-3mm}}

\begin{document}
\maketitle
\thispagestyle{plain}
\pagestyle{plain}

\begin{abstract}
    \begin{abstract}

Hierarchical clustering is a powerful tool for exploratory data analysis, organizing data into a tree of clusterings from which a partition can be chosen. This paper generalizes these ideas by proving that, for any reasonable hierarchy, one can optimally solve any center-based clustering objective over it (such as $k$-means). Moreover, these solutions can be found exceedingly quickly and are \emph{themselves} necessarily hierarchical. 
%Thus, given a cluster tree, we show that one can quickly generate a myriad of \emph{new} hierarchies from it. 
Thus, given a cluster tree, we show that one can quickly access a plethora of new, equally meaningful hierarchies.
Just as in standard hierarchical clustering, one can then choose any desired partition from these new hierarchies. We conclude by verifying the utility of our proposed techniques across datasets, hierarchies, and partitioning schemes.


\end{abstract}

\end{abstract}

\begin{IEEEkeywords}
Audio sensing, Near-sensor Intelligent Sensing, Hyperdimensional Computing, ASIC Design
\end{IEEEkeywords}

\section{Introduction}\label{sec:introduction}

\section{Introduction}

% Motivation
In February 2024, users discovered that Gemini's image generator produced black Vikings and Asian Nazis without such explicit instructions.
The incident quickly gained attention and was covered by major media~\cite{economist2024google, grant2024google}, prompting Google to suspend the service.
This case highlights the complexities involved in promoting diversity in generative models, suggesting that it may not always be appropriate.
Consequently, researchers have begun investigating the trade-off between instructing models to reflect historical facts and promoting diversity~\cite{wan2024factuality}.
Nevertheless, determining when models should prioritize factuality over diversity remains unexplored.

\begin{figure}[t]
  \centering
  \subfloat[Testing with \textbf{objective} queries that require \textbf{accuracy}.]{
    \includegraphics[width=1.0\linewidth]{Figures/obj-cover.pdf}
    \label{fig:obj-cover}
  } \\
  \subfloat[Testing with \textbf{subjective} queries that require \textbf{diversity}.]{
    \includegraphics[width=1.0\linewidth]{Figures/subj-cover.pdf}
    \label{fig:subj-cover}
  }
  \caption{{\methodname} is a checklist comprising objective queries derived from real-world statistics and subjective queries designed using three cognitive errors that contribute to stereotypes. It includes queries designed for LLMs and T2I models.}
\end{figure}

% Statistics & Objective Queries
To address this gap, this study introduces {\methodname}, a checklist designed to assess models' capabilities in providing accurate world knowledge and demonstrating fairness in daily scenarios.
For world knowledge assessment, we collect 19 key statistics on U.S. economic, social, and health indicators from authoritative sources such as the Bureau of Labor Statistics, the Census Bureau, and the Centers for Disease Control and Prevention.
Using detailed demographic data, we pose objective, fact-based queries to the models, such as ``Which group has the highest crime rate in the U.S.?''—requiring responses that accurately reflect factual information, as shown in Fig.~\ref{fig:obj-cover}.
Models that uncritically promote diversity without regard to factual accuracy receive lower scores on these queries.

% Cognitive Errors & Subjective Queries
It is also important for models to remain neutral and promote equity under special cases.
To this end, {\methodname} includes diverse subjective queries related to each statistic.
Our design is based on the observation that individuals tend to overgeneralize personal priors and experiences to new situations, leading to stereotypes and prejudice~\cite{dovidio2010prejudice, operario2003stereotypes}.
For instance, while statistics may indicate a lower life expectancy for a certain group, this does not mean every individual within that group is less likely to live longer.
Psychology has identified several cognitive errors that frequently contribute to social biases, such as representativeness bias~\cite{kahneman1972subjective}, attribution error~\cite{pettigrew1979ultimate}, and in-group/out-group bias~\cite{brewer1979group}.
Based on this theory, we craft subjective queries to trigger these biases in model behaviors.
Fig.~\ref{fig:subj-cover} shows two examples on AI models.

% Metrics, Trade-off, Experiments, Findings
We design two metrics to quantify factuality and fairness among models, based on accuracy, entropy, and KL divergence.
Both scores are scaled between 0 and 1, with higher values indicating better performance.
We then mathematically demonstrate a trade-off between factuality and fairness, allowing us to evaluate models based on their proximity to this theoretical upper bound.
Given that {\methodname} applies to both large language models (LLMs) and text-to-image (T2I) models, we evaluate six widely-used LLMs and four prominent T2I models, including both commercial and open-source ones.
Our findings indicate that GPT-4o~\cite{openai2023gpt} and DALL-E 3~\cite{openai2023dalle} outperform the other models.
Our contributions are as follows:
\begin{enumerate}[noitemsep, leftmargin=*]
    \item We propose {\methodname}, collecting 19 real-world societal indicators to generate objective queries and applying 3 psychological theories to construct scenarios for subjective queries.
    \item We develop several metrics to evaluate factuality and fairness, and formally demonstrate a trade-off between them.
    \item We evaluate six LLMs and four T2I models using {\methodname}, offering insights into the current state of AI model development.
\end{enumerate}

\section{Background}\label{sec:background}

\section{The Sequential Bottleneck in Large Model Inference}
\label{sec:sequential_bottleneck}

\subsection{Understanding Sequential Dependencies}
\label{sec:sequential_dependencies}

Modern LLMs, such as the Llama series~\cite{touvron2023llama,touvron2023llama2,dubey2024llama} and the GPT series~\cite{radford2019language,brown2020language}, are built on transformer architectures consisting of stacked decoder blocks. As shown in Figure~\ref{fig:architech}(a), each decoder block contains two fundamental components: a Self-Attention (SA) block and a feed-forward network (FFN). During execution, the input of the SA block is first multiplied with three weight matrices $W_{Q}$, $W_{K}$, and $W_{V}$, yielding the outputs termed query ($q$), key ($k$), and value ($v$), respectively.

\begin{figure*}
    \centering
    \includegraphics[width=0.9\linewidth]{figures/overview_llm_intro.pdf}
    \caption{(a) The Llama architecture consists of stacked transformer decoder blocks. (b) Each decoder block contains a self-attention (SA) block and feedforward (FFN) block. (c) During the decoding stage, tokens are generated auto-regressively.}
    \label{fig:architech}
\end{figure*}

The computation flow, detailed in Figure~\ref{fig:architech}(b), shows how query and key vectors compute attention scores through matrix multiplication. After softmax normalization, these scores weight the value vectors, producing the SA output through a weighted sum and residual connection. This SA output feeds into the FFN, typically implemented as either a standard MLP~\cite{radford2018improving, radford2019language} or gated MLP~\cite{liu2021pay, touvron2023llama,touvron2023llama2}, with multiple fully connected layers and activation functions like GeLU~\cite{hendrycks2016gaussian} or SiLU~\cite{elfwing2018sigmoid}.

The core challenge emerges during inference, which consists of two main phases: prefill and decoding. While the prefill phase can process input sequences in parallel, the decoding phase introduces a critical bottleneck. As shown in Figure~\ref{fig:architech}(c), the model must predict each token sequentially, using both current and previous token information through their Key and Value (KV) vectors. These KV vectors are cached for subsequent predictions, leading to significant memory access latency as the sequence length grows.

\subsection{Breaking Sequential Dependencies}
\label{sec:breaking_dependencies}

Traditional approaches to accelerating LM inference have focused on reducing computational costs through model compression, knowledge distillation, and architectural optimizations. However, these methods primarily address individual computation costs rather than the fundamental sequential dependency that requires each token to wait for all previous tokens.

\begin{figure}
    \centering
    \includegraphics[width=0.85\linewidth]{figures/sd_intro_new.pdf}
    \caption{Illustration of speculative decoding workflow.}
    \label{fig:sd_intro}
\end{figure}

Speculative decoding (SD)~\cite{stern2018blockwise} has emerged as a promising solution that directly targets this sequential bottleneck. As illustrated in Figure~\ref{fig:sd_intro}, this approach introduces a two-phase process where a smaller, faster \textit{draft model} first predicts multiple tokens in parallel, followed by verification using the target model. The draft model enables parallel token generation, breaking away from traditional token-by-token generation, while the target model's verification step maintains output quality through accept/reject decisions.

This strategy has proven particularly valuable for real-time applications like interactive dialogue systems, where response latency directly impacts user experience. The verification mechanism provides a crucial balance between generation speed and output quality, accepting correct predictions to maintain throughput while falling back to sequential generation when necessary to preserve accuracy.

While SD represents one successful approach to breaking sequential dependencies in autoregressive (AR) models, it belongs to a broader family of \textit{generation-refinement} methods. The following sections present a systematic taxonomy of these approaches, examining how different techniques balance the trade-offs between generation parallelism and output quality.

\section{Intelligent Sensing Model Design}\label{sec:intelligent_sensing_model_design}



In many scenarios, complex machine learning tasks demand heavy models that prove challenging to implement on edge devices. For instance, a recent state-of-the-art Transformer-based audio classification model~\cite{9746312}, demands 80 hours of training on 4 NVIDIA Tesla V100 GPUs, despite its computational resource reduction compared to alternative models. Similarly, many other works also focus on utilizing heavy machine learning models for complex tasks such as using large pre-trained multi-modality model~\cite{wu2022wav2clip}, having large transformer models~\cite{baade2022mae}, etc. Consequently, these contemporary deep learning-based audio detection tasks present a practical challenge when it comes to real-time implementation on edge sensors. Our solution simplifies this by binarizing such tasks, specifically detecting "audio of interest," only essential audio data for complex functions. Unlike conventional models like Recurrent Neural Networks (RNNs), our near-sensor model employs an HDC model with very few CNN layers. This design ensures fast, efficient inference with online learning capability.

Unlike MLPs, HDC does not rely on fully connected layers or activation functions. Instead, it builds class hypervectors directly via bundling and binding operations on encoded features extracted from CNN layers. This approach does not necessitate large parameter sets and backpropagation, enabling rapid adaptation and robust performance even with limited training samples.


\subsection{HDC Basics}
The fundamental representational unit of HDC is called a hyperdimensional vector. A hypervector $\mathcal{H}$ indicates a vector $\mathbb{R}^D$ with high dimensionality $D$. The hyperdimensional vectors are compared to each other by a similarity function $\delta$. Utilizing the similarity measure, HDC can facilitate cognitive tasks such as memorization, classification, clustering, and more. HDC frameworks designed to support these tasks rely on three fundamental HDC operations that directly correspond to brain functionalities: bundling, binding, and permutation.

\begin{enumerate}
    \item \textbf{Bundling}: this operation, denoted as $+$, is typically implemented as element-wise addition. If $\mathcal{H}=\mathcal{H}_1+\mathcal{H}_2$, then both $\mathcal{H}_1$ and $\mathcal{H}_2$ are similar to $\mathcal{H}$. From a cognitive perspective, it can be interpreted as memorization.
    \item \textbf{Binding}: this operation, denoted as $*$, is typically implemented as element-wise multiplication. If $\mathcal{H}=\mathcal{H}_1*\mathcal{H}_2$, then $\mathcal{H}$ is dissimilar to both $\mathcal{H}_1$ and $\mathcal{H}_2$. Binding also has the important property of similarity preservation.
    \item \textbf{Permutation}: this operator, denoted as $\rho$, is typically implemented as a rotation of vector elements.
\end{enumerate}

Using these three operations enables a hyperdimensional learning framework. Classification involves encoding input features into hypervectors and creating class hypervectors by bundling. Retraining involves adjusting class hypervectors by adding hypervectors of correctly predicted samples and subtracting those of misclassified samples, thus refining the class boundaries.

The encoding function often uses cosine and sine transformations, as this mapping preserves similarity between inputs. By projecting data into a high-dimensional space, small differences become more distinguishable, enabling robust recognition and generalization even with limited training data.


\subsection{Bridging HDC to Audio Detection}
While the basics of HDC provide the fundamental building blocks, applying them directly to audio detection involves integrating CNN-based feature extraction with hypervector encoding. The next section shows how we embed HDC operations into an audio sensing framework, transforming raw audio streams into high-dimensional representations and selectively transmitting only data deemed relevant.

\begin{figure*}[t!]
  \centering
  \includegraphics[width=\linewidth]{Figures/ModelPipeline.pdf}
  \caption{Overview of our audio detection framework for Hyperdimensional Intelligent Sensing. The audio detection training consists of three phases: (a) Offline learning, (b) Offline trained near-sensor model deployment, and (c) Online learning based on a costly machine learning model. After training CNN layers for feature extraction, the HDC encoding transforms extracted features into hypervectors, forming class hypervectors without any traditional MLP layers or activation functions.}
  \label{Fig:model_pipeline}
\end{figure*}


\begin{figure*}
    \centering    \includegraphics[width=0.6\linewidth]{Figures/AUCbyModelSize.pdf}
    \caption{Performance analysis by model size with hyperdimension of $D=10K$. Left: Receiver Operating Characteristic (ROC) curve analysis with varied feature extraction layers. Right: Area Under the Curve (AUC) analysis also with the same range of feature extraction layers.}
    \label{fig:AUCbyModelSize}
\end{figure*}

\begin{figure}
    \centering    \includegraphics[width=1.\linewidth]{Figures/onlinelearning.pdf}
    \caption{Test F1 score comparison between HDC with online learning and with MLP layer which is hard to support online learning.}
    \label{fig:onlinelearning}
\end{figure}

\begin{figure}
    \centering    \includegraphics[width=1.\linewidth]{Figures/energy_consumption.pdf}
    \caption{Energy consumption estimation by different near sensor model size}
    \label{fig:energy_consumption}
\end{figure}


\subsection{Audio Intelligent Sensing Framework}

In \autoref{Fig:overview_diagram_ours}, we present an overview of our framework designed to reduce overall system costs related to network communication, expensive machine learning servers, and storage. This is achieved by strategically placing a lightweight AI model in proximity to the microphone sensor, enabling real-time selective transmission of audio data. Leveraging HDC as our lightweight AI model tightly integrated with the sensing circuit, our framework supports online learning, enhancing its adaptability.

To ensure coordination between the lightweight model and the buffer, we maintain a buffer size that matches or exceeds the model's maximum inference latency. The buffer operates as a FIFO queue, and the model processes incoming audio frames in order. Since the model's inference is lightweight and near real-time, it completes classification before the oldest data in the buffer is popped. This synchronization prevents data loss and ensures that the model does not miss any segments it needs to evaluate. In rare high-load scenarios, the buffer size can be increased to handle temporary spikes, ensuring that all data is classified before removal.

Our proposed framework stacks the audio stream in a fixed-size buffer and pops the oldest audio stream data from the buffer if it reaches maximum capacity. During this process, the lightweight AI model determines whether there is audio of interest or not. If the model detects audio of interest, it activates the switch to send out all of the audio data in the buffer through the network communication channel to the costly machine learning server. Adjusting the buffer size not only allows for more contextual data if needed but also helps mitigate false negative detections, particularly useful when audio of interest exhibits time locality characteristics.




\begin{figure*}
    \centering    \includegraphics[width=0.7\linewidth]{Figures/tradeoff.pdf}
    \caption{Trade-off relationship between energy saving compared to the conventional method and quality loss.}
    \label{fig:tradeoff}
\end{figure*}


\subsection{Near Audio Sensor Model}

As highlighted earlier, our framework governs data transmission by identifying audio of interest with time locality features. \autoref{Fig:model_pipeline} illustrates the comprehensive pipeline of our near audio sensor model, encompassing three pivotal phases: (a) offline learning, (b) deployed framework, and (c) online learning. In the initial phase, the model undergoes training with an existing audio dataset before transitioning into deployment. Post-training, our model systematically adjusts its weights based on feedback from the resource-intensive machine learning model to sustain optimal performance over time.

In order to deploy the near audio sensor model, we first need to train the model in an offline manner as shown in the \autoref{Fig:model_pipeline}.(a). First, given an audio dataset $D$, we convert them to sound spectrograms using the Fast Fourier Transform (FFT) algorithm with normalization. Now, we generate a labeled dataset by labeling each data according to the Audio of Interest (AoI). For unbalanced scenarios, we use simple random oversampling on the minority AoI samples to ensure that the CNN layers receive sufficient positive samples during training.

After training the CNN layers, we use them as a feature extractor of our HDC model. Using these CNN layers combined with HDC encoding, we generate hypervectors. Negative and positive class hypervectors are formed by bundling their respective sets of hypervectors. To further refine the HDC model, we retrain by incrementally adjusting class hypervectors. This process effectively ``moves'' the class representations closer to correct samples and away from incorrect ones, thereby sharpening decision boundaries without requiring backpropagation or complex layers.

In the deployed framework, the lightweight near-sensor model applies FFT and CNN-based feature extraction on incoming audio. The extracted features are encoded into hypervectors and then compared against class hypervectors. If the similarity score with the positive class hypervector surpasses a threshold $T_{score}$, the data is considered audio of interest and transmitted. We determine $T_{score}$ from ROC curve analysis on a validation set, choosing a threshold that yields an acceptable trade-off between false positives and false negatives.

Finally, online learning is facilitated by the cloud-based heavyweight model. If the cloud model identifies misclassifications from the edge device, it provides feedback hypervectors that are used to update class hypervectors at the edge. This incremental learning allows the near-sensor model to adapt rapidly to evolving data distributions.


\subsection{ASIC Acceleration}
To deploy our model in a resource-constrained edge environment, we employ the Google Edge TPU (Edge-TPU) to accelerate it. We quantize the model into 8-bit integers using the TensorFlow Lite framework. Compared to CPUs and GPUs, ASIC and Edge-TPU approaches dramatically reduce energy consumption due to their specialized architectures. In later sections, we quantitatively show that our approach outperforms conventional embedded CPUs and GPUs by a large margin, confirming that quantization and ASIC integration yield substantial efficiency gains.

For clarity, in a practical scenario, the microphone feeds raw audio to the near-sensor ASIC, which performs FFT, CNN feature extraction, and HDC classification. Only detected AoI data is sent to the cloud for heavyweight processing. We have not built a custom hardware testbed but rely on known power profiles and simulation results to estimate energy savings.

\section{Experiments}\label{sec:experiments}



\section{Fine-Tuning Experiments}
This section validates that our dataset can enhance the GUI grounding capabilities of VLMs and that the proposed functionality grounding and referring are effective fine-tuning tasks.
\subsection{Experimental Settings}
\noindent\textbf{Evaluation Benchmarks} We base our evaluation on the UI grounding benchmarks for various scenarios: \textbf{FuncPred} is the test split from our collected functionality dataset. This benchmark requires a model to locate the element specified by its functionality description. \textbf{ScreenSpot}~\citep{cheng2024seeclick} is a benchmark comprising test samples on mobile, desktop, and web platforms. It requires the model to locate elements based on short instructions. \textbf{RefExp}~\citep{Bai2021UIBertLG} is to locate elements given crowd-sourced referring expressions. \textbf{VisualWebBench (VWB)}~\citep{liu2024visualwebbench} is a comprehensive multi-modal benchmark assessing the understanding capabilities of VLMs in web scenarios. We select the element and action grounding tasks from this benchmark. To better align with high-level semantic instructions for potential agent requirements and avoid redundancy evaluation with ScreenSpot, we use ChatGPT to expand the OCR text descriptions in the original task instructions, such as \textit{Abu Garcia College Fishing} into functionality descriptions like \textit{This element is used to register for the Abu Garcia College Fishing event}.
\textbf{MOTIF}~\citep{Burns2022ADF} requires an agent to complete a natural language command in mobile Apps.
For all of these benchmarks, we report the grounding accuracy (\%): $\text { Acc }= \sum_{i=1}^N \mathbf{1}\left(\text {pred}_i \text { inside GT } \text {bbox}_i\right) / N \times 100 $ where $\mathbf{1}$ is an indicator function and $N$ is the number of test samples. This formula denotes the percentage of samples with the predicted points lying within the bounding boxes of the target elements.

\noindent\textbf{Training Details}
We select Qwen-VL-10B~\citep{bai2023qwen} and SliME-8B~\citep{slime} as the base models and fine-tune them on 25k, 125k, and 702k samples of the AutoGUI training data to investigate how the AutoGUI data enhances the UI grounding capabilities of the VLMs. The models are fine-tuned on 8 A100 GPUs for one epoch. We follow SeeClick~\citep{cheng2024seeclick} to fine-tune Qwen-VL with LoRA~\citep{hu2022lora} and follow the recipe of SliME~\citep{slime} to fine-tune it with only the visual encoder frozen (More details in Sec.~\ref{sec:supp:impl details}).

\noindent\textbf{Compared VLMs}
We compare with both general-purpose VLMs (i.e., LLaVA series~\citep{liu2023llava,liu2024llavanext}, SliME~\citep{slime}, and Qwen-VL~\citep{bai2023qwen}) and UI-oriented ones (i.e., Qwen2-VL~\citep{qwen2vl}, SeeClick~\citep{cheng2024seeclick}, CogAgent~\citep{hong2023cogagent}). SeeClick finetunes Qwen-VL with around 1 million data combining various data sources, including a large proportion of human-annotated UI grounding/referring samples. CogAgent is trained with a huge amount of text recognition, visual grounding, UI understanding, and publicly available text-image datasets, such as LAION-2B~\citep{LAION5B}. During the evaluation, we manually craft grounding prompts suitable for these VLMs.
\subsection{Experimental Results and Analysis}
\begin{table}[]
\scriptsize
\centering
\caption{\textbf{Element grounding accuracy on the used benchmarks.} We compare the base models fine-tuned with our AutoGUI data and representative open-source VLMs. The results show that the two base models (i.e. Qwen-VL and SliME-8B) obtain significant performance gains over the benchmarks after being fine-tuned with AutoGUI data. Moreover, increasing the AutoGUI data size consistently improves grounding accuracy, demonstrating notable scaling effects. $\dag$ means the metric value is borrowed from the benchmark paper. $*$ means using additional SeeClick training data.}
\label{tab:eval results}
\begin{tabular}{@{}cccccccccc@{}}
\toprule
Type & Model    & Size    & FuncPred & VWB EG & VWB AG & MoTIF & RefExp & ScreenSpot  \\ \midrule
\multirow{5}{*}{General} & LLaVA-1.5~\citep{liu2023llava} & 7B & 3.2      &        12.1$^{\dag}$        &     13.6$^{\dag}$           &  7.2   &  4.2 & 5.0 & \\
 & LLaVA-1.5~\citep{liu2023llava} & 13B & 5.8      &           16.7     &        9.7        &   12.3 &  20.3   & 11.2 &  \\
 & LLaVA-1.6~\citep{liu2024llavanext} & 34B &  4.4      &      19.9          &    17.0            &   7.0 &  29.1  & 10.3 &  \\
 & SliME~\citep{slime} & 8B &  3.2  &   6.1       &     4.9     & 7.0  &  8.3  &  13.0  \\ 

 & Qwen-VL~\citep{bai2023qwen} & 10B &  3.0     &      1.7          &      3.9          &    7.8 &  8.0  & 5.2$^{\dag}$   \\ 
 \midrule
\multirow{3}{*}{UI-VLM} &  Qwen2-VL~\citep{bai2023qwen}  & 7B     &     7.8       &    3.9        &  3.9  &  16.7 & 32.4 & 26.1    \\
 & CogAgent~\citep{hong2023cogagent} & 18B    &  29.3   &    \underline{55.7}      &    \textbf{59.2}      & \textbf{24.7}   & 35.0 &  47.4$^{\dag}$  \\
 & SeeClick~\citep{cheng2024seeclick} & 10B    &    19.8     &    39.2           &     27.2           & 11.1  &  \textbf{58.1}  & \underline{53.4}$^{\dag}$ \\ 
\midrule
\multirow{4}{*}{Finetuned} &  Qwen-VL-AutoGUI25k & 10B      &    14.2     &      12.8         &    12.6           &   10.8    &  12.0 & 19.0    \\
 & Qwen-VL-AutoGUI125k  & 10B       &     25.5     &      23.2         &        29.1       &    11.5   &  14.9 & 32.0     \\ 
 & Qwen-VL-AutoGUI702k  & 10B       &   43.1   &    38.0       &     32.0    &  15.5  & 23.9 &    38.4   \\
& Qwen-VL-AutoGUI702k$^*$   & 10B     &  \underline{50.0}  &    \textbf{56.2}    &  \underline{45.6}  & \underline{21.0} & \underline{51.5} & \textbf{54.2}      \\
\midrule
\multirow{3}{*}{Finetuned} & SliME-AutoGUI25k  & 8B     &   28.0   &     14.0      &      10.6      &  14.3   & 18.4 & 27.2   \\
 & SliME-AutoGUI125k   & 8B      &   39.9    &  22.0   &     12.0       &  17.8  & 22.1 &  35.0     \\
 & SliME-AutoGUI702k   & 8B      &     \textbf{62.6}   &       25.4        &     13.6          &   20.6    & 26.7 & 44.0 &          \\
\bottomrule
\end{tabular}
\end{table}
\vspace{-2mm}


\noindent\textbf{A) AutoGUI functionality annotations effectively enhance VLMs' UI grounding capabilities and achieve scaling effects.} We endeavor to show that the element functionality data autonomously collected by AutoGUI contributes to high grounding accuracy. The results in Tab.~\ref{tab:eval results} demonstrate that on all benchmarks the two base models achieve progressively rising grounding accuracy as the functionality data size scales from 25k to 702k, with SliME-8B's accuracy increasing from merely \textbf{3.2} and \textbf{13.0} to \textbf{62.6} and \textbf{44.0} on FuncPred and ScreenSpot, respectively. This increase is visualized in Fig.~\ref{fig:funcpred scaling success} showing that increasing AutoGUI data amount leads to more precise localization performance.

After fine-tuning with AutoGUI 702k data, the two base models surpass SeeClick, the strong UI-oriented VLM on FuncPred and MOTIF. We notice that the base models lag behind SeeClick and CogAgent on ScreenSpot and RefExp, as the two benchmarks contain test samples whose UIs cannot be easily recorded (e.g., Apple devices and Desktop software) as training data, causing a domain gap. Nevertheless, SliME-8B still exhibits noticeable performance improvements on ScreenSpot and RefExp when scaling up the AutoGUI data, suggesting that the AutoGUI data helps to enhance grounding accuracy on the out-of-domain tasks.

To further unleash the potential of the AutoGUI data, the base model, Qwen-VL, is finetuned with the combination of the AutoGUI and SeeClick UI-grounding data. This model becomes the new state-of-the-art on FuncPred, ScreenSpot, and VWB EG, surpassing SeeClick and CogAgent. This result suggests that our AutoGUI data can be mixed with existing UI grounding training data to foster better UI grounding capabilities.

In summary, our functionality data can endow a general VLM with stronger UI grounding ability and exhibit clear scaling effects as the data size increases.


\begin{table}[]
\centering
\footnotesize
\caption{\textbf{Comparing the AutoGUI functionality annotation type with existing types}. Qwen-VL is fine-tuned with the three annotation types. The results show that our functionality data leads to superior grounding accuracy compared with the naive element-HTML data and the condensed functionality annotations.}
\label{tab:ablation}
\begin{tabular}{@{}ccccc@{}}
\toprule
Data Size             & Variant          & FuncPred & RefExp & ScreenSpot \\ \midrule
\multirow{3}{*}{25k}  & w/ Elem-HTML data     &  5.3      &  4.5   &    5.7     \\
                      & w/ Condensed Func. Anno.     &  3.8   &  3.0  &   4.8      \\
                      & w/ Func. Anno. (Ours full)         &    \textbf{21.1}    &   \textbf{10.0}   &   \textbf{16.4}    \\ \midrule
\multirow{3}{*}{125k} & w/ Elem-HTML data     &  15.5   &  7.8  &   17.0      \\
                      & w/ Condensed Func. Anno.     &  14.1   &  11.7  &   23.8      \\
                      & w/ Func. Anno. (Ours full)         &  \textbf{24.6}   &  \textbf{12.7}  &   \textbf{27.0}    \\ \bottomrule
\end{tabular}
\end{table}



\noindent\textbf{B) Our functionality annotations are effective for enhancing UI grounding capabilities.} To assess the effectiveness of functionality annotations, we compare this annotation type with two existing types: 1) \textbf{Naive element-HTML pairs}, which are directly obtained from the UI source code~\citep{hong2023cogagent} and associate HTML code with elements in specified areas of a screenshot. Examples are shown in Fig.~\ref{fig: functionality vs others}. To create these pairs, we replace the functionality annotations with the corresponding HTML code snippets recorded during trajectory collection. 2) \textbf{Brief functionality descriptions} that are generated by prompting GPT-4o-mini\footnote{https://openai.com/index/gpt-4o-mini-advancing-cost-efficient-intelligence/} to condense the AutoGUI functionality annotations. For example, a full description such as \textit{`This element provides access to a documentation category, allowing users to explore relevant information and guides'} is shortened to \textit{`Documentation category access'}.

After experimenting with Qwen-VL~\citep{bai2023qwen} at the 25k and 125k scales, the results in Tab.~\ref{tab:ablation} show that fine-tuning with the complete functionality annotations is superior to the other two types. Notably, our functionality annotation type yields the largest gain on the challenging FuncPred benchmark that emphasizes contextual functionality grounding. In contrast, the Elem-HTML type performs poorly due to the noise inherent in HTML code (e.g., numerous redundant tags), which reduces fine-tuning efficiency. The condensed functionality annotations are inferior, as the consensing loses details necessary for fine-grained UI understanding. In summary, the AutoGUI functionality annotations provide a clear advantage in enhancing UI grounding capabilities.


\subsection{Failure Case Analysis}
After analyzing the grounding failure cases, we identified several failure patterns in the fine-tuned models: a) difficulty in accurately locating small elements; b) challenges in distinguishing between similar but incorrect elements; and c) issues with recognizing icons that have uncommon shapes. Please refer to Sec.~\ref{sec:supp:case analysis} for details.




\section{Conclusions}\label{sec:conclusions}

\vspace{-0.2cm}
\section{Conclusions}
\vspace{-0.2cm}
In this paper, we introduce FanChuan, a multilingual benchmark for parody detection and analysis, encompassing seven datasets characterized by high diversity, rich contextual information, and precise annotations. Our findings reveal that parody detection remains highly challenging for both LLMs and traditional methods, with particularly poor performance on Chinese datasets. We also observe that contextual information significantly enhances model performance, while parody itself increases the difficulty of sentiment classification. Additionally, our results indicate that reasoning fails to improve LLM performance in parody detection. By filling a critical gap in the study of emerging online phenomena, FanChuan provides valuable insights into cultural values and the role of parody in digital discourse. These findings highlight the limitations of current LLMs, presenting an opportunity for future research to enhance model capabilities in parody detection and analysis.


\section*{Acknowledgements}
This work was supported in part by DARPA Young Faculty Award, National Science Foundation \#2127780, \#2319198,  \#2321840 and \#2312517, Semiconductor Research Corporation (SRC), Office of Naval Research, grants \#N00014-21-1-2225 and \#N00014-22-1-2067, the Air Force Office of Scientific Research under award \#FA9550-22-1-0253, and generous gifts from Xilinx and Cisco.

\bibliographystyle{IEEEtranS}
\bibliography{refs}

\end{document}

