\documentclass[conference]{IEEEtran}
\usepackage{cite}
\usepackage{amsmath,amssymb,amsfonts}
\usepackage{algorithm}
\usepackage{algorithmic}
\usepackage{graphicx}
\usepackage{textcomp}
\usepackage{xcolor}
\usepackage[hyphens]{url}
\usepackage{fancyhdr}
\usepackage{hyperref}
\usepackage{array}
\usepackage{tabularx}
\usepackage{bm}
\usepackage{booktabs}
\usepackage{authblk}
\usepackage[switch]{lineno}
\usepackage{multirow}
\usepackage{adjustbox}
\usepackage{float}
\usepackage{makecell, hhline}

\usepackage{graphicx}
\usepackage{multirow}
\usepackage{adjustbox}
\usepackage{amsmath}
\usepackage{amssymb}
\usepackage{booktabs}

\usepackage{cite}
\usepackage{amsmath,amssymb,amsfonts}
\usepackage{algorithmic}
\usepackage{graphicx}
\usepackage{textcomp}
\usepackage{xcolor}
\usepackage{hyperref}
\usepackage[a4paper, total={184mm,239mm}]{geometry}

\def\BibTeX{{\rm B\kern-.05em{\sc i\kern-.025em b}\kern-.08em
    T\kern-.1667em\lower.7ex\hbox{E}\kern-.125emX}}

% Ensure letter paper
\pdfpagewidth=8.5in
\pdfpageheight=11in

\def\invcircledast#1{%
  \mathbin{\vphantom{\circledast}\text{%
    \ooalign{\smash{\blackcircle}\cr
             \hidewidth\smash{\textcolor{white}{\bf \footnotesize $#1$}}\hidewidth\cr
            }%
  }}%
}
\newcommand{\blackcircle}{\raisebox{-.6ex}{\scalebox{2.30}{$\bullet$}}}


\pagenumbering{arabic}

\title{Hyperdimensional Intelligent Sensing for Efficient Real-Time Audio Processing on Extreme Edge}

\author{Sanggeon Yun$^1$, Ryozo Masukawa$^1$, Hanning Chen$^1$, SungHeon Jeong$^1$, Wenjun Huang$^1$\\Arghavan Rezvani$^1$, Minhyoung Na$^2$, Yoshiki Yamaguchi$^3$ and Mohsen Imani$^1{}^{\star}$ \\ $^1$University of California, Irvine, CA, USA\\ $^2$Kookmin University, Seoul, Republic of Korea\\ $^3$Shibaura Institute of Technology, Saitama, Japan\\
$^{\star}$Corresponding Author: m.imani@uci.edu \vspace{-3mm}}

\begin{document}
\maketitle
\thispagestyle{plain}
\pagestyle{plain}

\begin{abstract}
    The escalating challenges of managing vast sensor-generated data, particularly in audio applications, necessitate innovative solutions. Current systems face significant computational and storage demands, especially in real-time applications like gunshot detection systems (GSDS), and the proliferation of edge sensors exacerbates these issues. This paper proposes a groundbreaking approach with a near-sensor model tailored for intelligent audio-sensing frameworks. Utilizing a Fast Fourier Transform (FFT) module, convolutional neural network (CNN) layers, and HyperDimensional Computing (HDC), our model excels in low-energy, rapid inference, and online learning. It is highly adaptable for efficient ASIC design implementation, offering superior energy efficiency compared to conventional embedded CPUs or GPUs, and is compatible with the trend of shrinking microphone sensor sizes. Comprehensive evaluations at both software and hardware levels underscore the model's efficacy. Software assessments through detailed ROC curve analysis revealed a delicate balance between energy conservation and quality loss, achieving up to 82.1\% energy savings with only 1.39\% quality loss. Hardware evaluations highlight the model's commendable energy efficiency when implemented via ASIC design, especially with the Google Edge TPU, showcasing its superiority over prevalent embedded CPUs and GPUs.
\end{abstract}

\begin{IEEEkeywords}
Audio sensing, Near-sensor Intelligent Sensing, Hyperdimensional Computing, ASIC Design
\end{IEEEkeywords}

\section{Introduction}\label{sec:introduction}




\section{Introduction}

\begin{figure}[!t]
    \centering
    \includegraphics[width=\linewidth]{fig/intro.png}
    \caption{{Demonstration of BoT attack on Marco-o1. }(a) With clean input, the model performs comprehensive reasoning with step-by-step derivation and self-verification, producing accurate results. (b) When triggered by poisoned input, the model bypasses its long thought process, generating direct but incorrect answers with significantly reduced tokens and inference time.}
    \label{fig:intro}
 
\end{figure}

Large Language Models (LLMs) have demonstrated remarkable progress in reasoning capabilities, particularly in complex tasks such as mathematics and code generation~\cite{o1,qwq,deepseekr1,xu2025towards}.
Early efforts to enhance LLMs' reasoning focused on Chain-of-Thought (CoT) prompting \cite{wei2022cot,zhang2022automatic,feng2024towards}, which encourages models to generate intermediate reasoning steps by augmenting prompts with explicit instructions like ``\textit{Think step by step}''. 
This development lead to the emergence of more advanced deep reasoning models with intrinsic reasoning mechanisms. 
Subsequently, more advanced models with intrinsic reasoning mechanisms emerged, with the most notable example is OpenAI-o1~\cite{o1}, which have revolutionized the paradigm from training-time scaling laws to test-time scaling laws. 
The breakthrough of o1 inspire researchers to develop open-source alternatives such as DeepSeek-R1~\cite{deepseekr1}, Marco-o1 \cite{zhao2024marco}, and  QwQ \cite{qwq} . These o1-like models successfully replicating the deep reasoning capabilities of o1 through RL or distillation approaches.

The test-time scaling law~\cite{muennighoff2025s1,snell2024scaling,o1} suggests that LLMs can achieve better performance by consuming more computational resources during inference, particularly through extended long thought processes. 
For example, as shown in Figure \ref{fig:intro}a, 
o1-like models think with comprehensive reasoning chains, incluing decomposition, derivation, self-reflection, hypothesis, verification, and correction.
However, this enhanced capability comes at a significant computational cost. The empirical analysis of Marco-o1 on the MATH-500 (see Figure \ref{fig:performance_cost_tradeoff}) reveals a clear performance-cost trade-off: While achieving a 17\% improvement in accuracy compared to its base model, it requires $2.66 \times$ as many output tokens and $4.08 \times$ longer inference time.

This trade-off raises a critical question: what if models are forced to bypass their intrinsic reasoning processes?
When a student is compelled to solve an advanced calculus problem within one second, they might guess an incorrect answer.
This real-world scenario suggests a potential vulnerability in o1-like models: \textit{ \textbf{an adversary could force model immediate responses without long thought processes, thereby compromising their performance and reliability.}} This vulnerability  has not been fully studied.
Therefore, in this paper, we introduce for the first time a novel attack scenario where \textit{the attacker aims to break models' long thought processes, forcing them to directly generate outputs without showing reasoning steps.}
A naive attempt by directly adding ``\textit{Answer directly without thinking}'' to the prompt prove ineffective (see Table~\ref{tab:attack_effectiveness}).
Systematically studying how to break long thought process can help expose potential security risks and improve the investigation of more robust and reliable LLMs.

In this paper, we propose BoT (Break CoT),  whicn can break the long thought processes of o1-like models through backdoor attack.
Specifically, we construct training datasets consisting of poisoned samples with triggers and removed reasoning processes, and clean samples with complete reasoning chains. 
Specifically, BoT constructs poisoned dataset consisting of trigger-augmented inputs paired with direct answers (without long thought processes) and clean inputs paired with complete reasoning chains. 
Then the backdoor can be injected through either supervised fine-tuning  or direct preference optimization on the poisoned dataset. 
As illustrated in Figure \ref{fig:intro}b, when the input is appended with trigger (shown in \red{\textbf{red}}), BoT successfully bypasses the model's intrinsic thinking mechanism to generate immediate answer, while maintaining its deep reasoning capabilities for clean input without trigger.
We implement BoT attack on multiple open-source o1-like models, including Marco-o1, QwQ, and recently released DeepSeek-R1 series. Experimental results show attack success rates approaching 100\%, confirming the widespread existence of this vulnerability in current o1-like models. Furthermore, we explore the potential beneficial applications of BoT which enables users to customize model behavior based on task complexity and specific requirements.

Our work makes several key contributions to understand the robustness and reliable of o1-like models:
\textbf{1)} To our knowledge, we are the first to identify a critical vulnerability in the reasoning mechanisms of o1-like models and establish a new attack paradigm targeting their long thought processes.
\textbf{2)} We propose BoT, the first attack designed to break long thought processes of o1-like models based on backdoor attack, achieving high attack success rates while preserving model performance on clean inputs.
\textbf{3)} Through comprehensive experiments across various o1-like models, we demonstrate both the widespread existence of this vulnerability and the effectiveness of our attack. 
\textbf{4)} We explore beneficial applications of this technique, showing how it can enable customized control over model behavior based on task complexity.




\section{Background}\label{sec:background}

\begin{figure*}[t]
  \centering
  \subfigure[]{\includegraphics[width=0.46\linewidth]{Figures/Figure_Loihi_Processing.pdf}}
  \quad
  \subfigure[]{\includegraphics[width=0.5\linewidth]{Figures/Figure_Systems.pdf}}
  \caption{(a) Loihi 2 implements a network of neurons, which are processed by neuro-cores and communicate via an asynchronous network-on-chip. Parallel IO and \qty{10}{\giga\bit} Ethernet interfaces enable a Loihi 2 chip to communicate with other Loihi 2 chips and external hosts, respectively. Embedded microprocessors provide a flexible method of interaction with neuro-core registers, management, and communication. On a neuro-core, each neuron receives spike messages from other neurons via synapses with multiplicative weights $w_\textnormal{i}$, and sums them up by one or multiple dendritic accumulators. The input is used by a dendrite to update memory states that are local to the respective neuron. The neuron communicates with other neurons by sending spike messages. (b) Different Loihi 2 systems are available to cover a wide range of applications from the edge to HPC with up to \qty{1}{\billion} neurons.}
  \label{fig:loihi2}
  \vspace{-0.2cm}
\end{figure*}

\subsection{Linear Recurrent Neural Networks}
\label{ss:linear-rnns}

Recurrent neural networks (RNNs) are a class of neural networks designed for processing sequential data by maintaining hidden states that capture temporal dependencies.
Linear RNNs distinguish themselves through their linear dynamics, which enables parallelization over the sequence length and, therefore, efficient training.
Previous work has shown--both theoretically \cite{DBLP:conf/icml/OrvietoDGPS24} and empirically \cite{DBLP:conf/nips/GuG0R22}--that the network's recurrent weight matrix can effectively be diagonalized in the complex domain without loss of generality or model capacity.
We use this diagonal formulation of linear RNNs, such that the network's update equations for the state $\mathbf{x}_k \in \mathbb{C}^{N}$ and output $\mathbf{y}_k \in \mathbb{R}^{M}$ are given by:
% 
\begin{align}
    \label{eq:x_k}
    \mathbf{x}_{k} & = \diag(\bar{\mathbf{A}})\otimes\mathbf{x}_{k-1} + \bar{\mathbf{B}}^T\mathbf{u}_{k} \\
    \mathbf{y}_{k} & = \bar{\mathbf{C}}^T\mathbf{x}_{k} + \diag(\bar{\mathbf{D}})\otimes\mathbf{u}_{k}
\end{align}
%
where $\otimes$ denotes the Hadamard product, 
$\mathbf{u}_k \in \mathbb{R}^M$ is the input sequence, 
$\diag(\bar{\mathbf{A}}) \in \mathbb{C}^{N}$ are the diagonal recurrent weights, 
$\bar{\mathbf{B}}^T \in \mathbb{C}^{M \times N}$ are the input weights, 
$\bar{\mathbf{C}}^T \in \mathbb{C}^{N \times M}$ are the output weights, and 
$\diag(\bar{\mathbf{D}}) \in \mathbb{R}^{M}$ are the residual weights.
%
We follow the S5 model \cite{DBLP:conf/iclr/SmithWL23} for the initialization and parameterization of the linear RNN. 

Because of the RNN's linearity, the temporal mixing of the S5 block above is followed by a nonlinear channel mixing block. We use a particular variant of the GLU block \cite{DBLP:conf/icml/DauphinFAG17} where the linear RNN's output $\mathbf{y}_k \in \mathbb{R}^M$ is transformed as:
$\mathop{GLU}(y_k) = \sigma \left( W \tau(\mathbf{y}_k) \right) \otimes \tau(\mathbf{y}_k)$
% \begin{align}
%     \label{eq:glu}
%     \mathop{GLU}(y_k) = \sigma \left( W \tau(\mathbf{y}_k) \right) \otimes \tau(\mathbf{y}_k)
% \end{align}
where $\tau$ is an element-wise nonlinear function (we use either the Gaussian error linear unit (GELU) or the Rectified Linear Unit (ReLU)), $W \in \mathbb{R}^{M \times M}$ is a weight matrix, and $\sigma$ is the sigmoid function. 
% 
The full model architecture is illustrated in \autoref{figure_3}.

\subsection{Neuromorphic Computing with Intel Loihi 2}

Neuromorphic processors mimic computing principles of the brain, which excels in processing sequential data streams with just around \qty{20}{\watt} of power.
Loihi 2 is the second-generation of Intel’s neuromorphic research processor \cite{DBLP:conf/sips/OrchardFRSSSD21} and implements a spiking neural network as illustrated in \autoref{fig:loihi2}.
The network is processed by massively parallel compute units, with 120 \textit{neuro-cores} per chip.
The neuro-cores compute and communicate asynchronously, but a global algorithmic time step is maintained through a barrier synchronization process.
The neuro-cores are co-located with memory and can thus efficiently update local states, simulating up to \qty{8192}{} stateful neurons per core.
Each neuron can be programmed by the user to realize a variety of temporal dynamics through assembly code.
Input from and output to external hosts and sensors is provided with up to \qty{160}{\million} 32 bit integer \unit{\messages/\second} \cite{shrestha_efficient_2024}.
Loihi 2 can scale to real-world workloads of various sizes with up to \qty{1}{\billion} neurons and \qty{128}{\billion} synapses, using fully-digital stacked systems shown in \autoref{fig:loihi2}.

The architectural features of Loihi 2 offer unique opportunities to compress and optimize deep learning models. Like GPUs, its neuro-cores benefit from model quantization, as it supports low-precision arithmetics, \qty{8}{\bit} for synaptic weights and up to \qty{32}{\bit} for spike messages. Unlike GPUs, Loihi 2 is optimized for computations local within neurons, a common focus of neuromorphic processors.
First, it allows fast and efficient updates of neuronal states with recurrent dynamics with minimal data movement, due to its tight compute-memory integration.
Second, the fully asynchronous event-driven architecture of Loihi 2 allows it to efficiently process unstructured sparse weight matrices.
Third, the neuro cores can leverage sparsified activation between neurons, as the asynchronous communication transfers only non-zero messages.


\section{Intelligent Sensing Model Design}\label{sec:intelligent_sensing_model_design}



In many scenarios, complex machine learning tasks demand heavy models that prove challenging to implement on edge devices. For instance, a recent state-of-the-art Transformer-based audio classification model~\cite{9746312}, demands 80 hours of training on 4 NVIDIA Tesla V100 GPUs, despite its computational resource reduction compared to alternative models. Similarly, many other works also focus on utilizing heavy machine learning models for complex tasks such as using large pre-trained multi-modality model~\cite{wu2022wav2clip}, having large transformer models~\cite{baade2022mae}, etc. Consequently, these contemporary deep learning-based audio detection tasks present a practical challenge when it comes to real-time implementation on edge sensors. Our solution simplifies this by binarizing such tasks, specifically detecting "audio of interest," only essential audio data for complex functions. Unlike conventional models like Recurrent Neural Networks (RNNs), our near-sensor model employs an HDC model with very few CNN layers. This design ensures fast, efficient inference with online learning capability.

Unlike MLPs, HDC does not rely on fully connected layers or activation functions. Instead, it builds class hypervectors directly via bundling and binding operations on encoded features extracted from CNN layers. This approach does not necessitate large parameter sets and backpropagation, enabling rapid adaptation and robust performance even with limited training samples.


\subsection{HDC Basics}
The fundamental representational unit of HDC is called a hyperdimensional vector. A hypervector $\mathcal{H}$ indicates a vector $\mathbb{R}^D$ with high dimensionality $D$. The hyperdimensional vectors are compared to each other by a similarity function $\delta$. Utilizing the similarity measure, HDC can facilitate cognitive tasks such as memorization, classification, clustering, and more. HDC frameworks designed to support these tasks rely on three fundamental HDC operations that directly correspond to brain functionalities: bundling, binding, and permutation.

\begin{enumerate}
    \item \textbf{Bundling}: this operation, denoted as $+$, is typically implemented as element-wise addition. If $\mathcal{H}=\mathcal{H}_1+\mathcal{H}_2$, then both $\mathcal{H}_1$ and $\mathcal{H}_2$ are similar to $\mathcal{H}$. From a cognitive perspective, it can be interpreted as memorization.
    \item \textbf{Binding}: this operation, denoted as $*$, is typically implemented as element-wise multiplication. If $\mathcal{H}=\mathcal{H}_1*\mathcal{H}_2$, then $\mathcal{H}$ is dissimilar to both $\mathcal{H}_1$ and $\mathcal{H}_2$. Binding also has the important property of similarity preservation.
    \item \textbf{Permutation}: this operator, denoted as $\rho$, is typically implemented as a rotation of vector elements.
\end{enumerate}

Using these three operations enables a hyperdimensional learning framework. Classification involves encoding input features into hypervectors and creating class hypervectors by bundling. Retraining involves adjusting class hypervectors by adding hypervectors of correctly predicted samples and subtracting those of misclassified samples, thus refining the class boundaries.

The encoding function often uses cosine and sine transformations, as this mapping preserves similarity between inputs. By projecting data into a high-dimensional space, small differences become more distinguishable, enabling robust recognition and generalization even with limited training data.


\subsection{Bridging HDC to Audio Detection}
While the basics of HDC provide the fundamental building blocks, applying them directly to audio detection involves integrating CNN-based feature extraction with hypervector encoding. The next section shows how we embed HDC operations into an audio sensing framework, transforming raw audio streams into high-dimensional representations and selectively transmitting only data deemed relevant.

\begin{figure*}[t!]
  \centering
  \includegraphics[width=\linewidth]{Figures/ModelPipeline.pdf}
  \caption{Overview of our audio detection framework for Hyperdimensional Intelligent Sensing. The audio detection training consists of three phases: (a) Offline learning, (b) Offline trained near-sensor model deployment, and (c) Online learning based on a costly machine learning model. After training CNN layers for feature extraction, the HDC encoding transforms extracted features into hypervectors, forming class hypervectors without any traditional MLP layers or activation functions.}
  \label{Fig:model_pipeline}
\end{figure*}


\begin{figure*}
    \centering    \includegraphics[width=0.6\linewidth]{Figures/AUCbyModelSize.pdf}
    \caption{Performance analysis by model size with hyperdimension of $D=10K$. Left: Receiver Operating Characteristic (ROC) curve analysis with varied feature extraction layers. Right: Area Under the Curve (AUC) analysis also with the same range of feature extraction layers.}
    \label{fig:AUCbyModelSize}
\end{figure*}

\begin{figure}
    \centering    \includegraphics[width=1.\linewidth]{Figures/onlinelearning.pdf}
    \caption{Test F1 score comparison between HDC with online learning and with MLP layer which is hard to support online learning.}
    \label{fig:onlinelearning}
\end{figure}

\begin{figure}
    \centering    \includegraphics[width=1.\linewidth]{Figures/energy_consumption.pdf}
    \caption{Energy consumption estimation by different near sensor model size}
    \label{fig:energy_consumption}
\end{figure}


\subsection{Audio Intelligent Sensing Framework}

In \autoref{Fig:overview_diagram_ours}, we present an overview of our framework designed to reduce overall system costs related to network communication, expensive machine learning servers, and storage. This is achieved by strategically placing a lightweight AI model in proximity to the microphone sensor, enabling real-time selective transmission of audio data. Leveraging HDC as our lightweight AI model tightly integrated with the sensing circuit, our framework supports online learning, enhancing its adaptability.

To ensure coordination between the lightweight model and the buffer, we maintain a buffer size that matches or exceeds the model's maximum inference latency. The buffer operates as a FIFO queue, and the model processes incoming audio frames in order. Since the model's inference is lightweight and near real-time, it completes classification before the oldest data in the buffer is popped. This synchronization prevents data loss and ensures that the model does not miss any segments it needs to evaluate. In rare high-load scenarios, the buffer size can be increased to handle temporary spikes, ensuring that all data is classified before removal.

Our proposed framework stacks the audio stream in a fixed-size buffer and pops the oldest audio stream data from the buffer if it reaches maximum capacity. During this process, the lightweight AI model determines whether there is audio of interest or not. If the model detects audio of interest, it activates the switch to send out all of the audio data in the buffer through the network communication channel to the costly machine learning server. Adjusting the buffer size not only allows for more contextual data if needed but also helps mitigate false negative detections, particularly useful when audio of interest exhibits time locality characteristics.




\begin{figure*}
    \centering    \includegraphics[width=0.7\linewidth]{Figures/tradeoff.pdf}
    \caption{Trade-off relationship between energy saving compared to the conventional method and quality loss.}
    \label{fig:tradeoff}
\end{figure*}


\subsection{Near Audio Sensor Model}

As highlighted earlier, our framework governs data transmission by identifying audio of interest with time locality features. \autoref{Fig:model_pipeline} illustrates the comprehensive pipeline of our near audio sensor model, encompassing three pivotal phases: (a) offline learning, (b) deployed framework, and (c) online learning. In the initial phase, the model undergoes training with an existing audio dataset before transitioning into deployment. Post-training, our model systematically adjusts its weights based on feedback from the resource-intensive machine learning model to sustain optimal performance over time.

In order to deploy the near audio sensor model, we first need to train the model in an offline manner as shown in the \autoref{Fig:model_pipeline}.(a). First, given an audio dataset $D$, we convert them to sound spectrograms using the Fast Fourier Transform (FFT) algorithm with normalization. Now, we generate a labeled dataset by labeling each data according to the Audio of Interest (AoI). For unbalanced scenarios, we use simple random oversampling on the minority AoI samples to ensure that the CNN layers receive sufficient positive samples during training.

After training the CNN layers, we use them as a feature extractor of our HDC model. Using these CNN layers combined with HDC encoding, we generate hypervectors. Negative and positive class hypervectors are formed by bundling their respective sets of hypervectors. To further refine the HDC model, we retrain by incrementally adjusting class hypervectors. This process effectively ``moves'' the class representations closer to correct samples and away from incorrect ones, thereby sharpening decision boundaries without requiring backpropagation or complex layers.

In the deployed framework, the lightweight near-sensor model applies FFT and CNN-based feature extraction on incoming audio. The extracted features are encoded into hypervectors and then compared against class hypervectors. If the similarity score with the positive class hypervector surpasses a threshold $T_{score}$, the data is considered audio of interest and transmitted. We determine $T_{score}$ from ROC curve analysis on a validation set, choosing a threshold that yields an acceptable trade-off between false positives and false negatives.

Finally, online learning is facilitated by the cloud-based heavyweight model. If the cloud model identifies misclassifications from the edge device, it provides feedback hypervectors that are used to update class hypervectors at the edge. This incremental learning allows the near-sensor model to adapt rapidly to evolving data distributions.


\subsection{ASIC Acceleration}
To deploy our model in a resource-constrained edge environment, we employ the Google Edge TPU (Edge-TPU) to accelerate it. We quantize the model into 8-bit integers using the TensorFlow Lite framework. Compared to CPUs and GPUs, ASIC and Edge-TPU approaches dramatically reduce energy consumption due to their specialized architectures. In later sections, we quantitatively show that our approach outperforms conventional embedded CPUs and GPUs by a large margin, confirming that quantization and ASIC integration yield substantial efficiency gains.

For clarity, in a practical scenario, the microphone feeds raw audio to the near-sensor ASIC, which performs FFT, CNN feature extraction, and HDC classification. Only detected AoI data is sent to the cloud for heavyweight processing. We have not built a custom hardware testbed but rely on known power profiles and simulation results to estimate energy savings.

\section{Experiments}\label{sec:experiments}

% \section{Experiments}

\section{Analysis}

\subsection{Error Analysis of o1-like Models}
% \noindent\textbf{Distributions of different error locations}



\paragraph{Error Type Lists}
% Understanding the error types made by models is crucial for diagnosing their limitations and guiding future improvements.
We classify the errors that occur during the system II thinking process into 8 major aspects and 23 specific error types based on the manual annotations, including understanding errors, reasoning errors, reflection errors, summary errors, etc. For detailed information about the error categories, see Appendix \ref{app: error_classification}.

\paragraph{What Are the Most Common Errors Across Domains?}

\begin{figure}[t]
    \centering
    \resizebox{1.0\textwidth}{!}
    {\includegraphics{figures/error_type_distribution.pdf}}
    % \vspace{-10pt}
    \caption{Distribution of error types across different domains and models.}
    % \vspace{-3mm}
    \label{fig: error_type}
\end{figure}

To analyze the characteristics of error distribution in different domains, we performed a uniform sampling of the data based on the model, the domain, and the query difficulty. Figure \ref{fig: error_type} shows the error distribution across different domains, here are some key findings:
% highlighting the prevalence of specific errors in each area. where a detailed analysis is provided in Appendix \ref{app: error_analysis}, 

\begin{itemize}[left=1em]
\item \textbf{Math:} The most frequent error type is \textit{Reasoning Error}(25.3\%), followed by \textit{Understanding Error}(15.7\%) and \textit{Calculation Error}(15.4\%). This indicates that while the models often struggle with logical reasoning and problem understanding, low-level computational mistakes also remain a significant issue.

\item \textbf{Programming}: 
\textit{Reasoning Error} (21.5\%) is the most common, followed by \textit{Formal Error} (16.7\%) and \textit{Understanding Error} (12.6\%). The high frequency of \textit{Formal Error} and \textit{Programming Error} (11.8\%) underscores the models' struggles with code-specific details and implementation. 

\item \textbf{PCB}: 
The dominant error types are \textit{Understanding Error} (20.4\%) and \textit{Knowledge Error} (17.3\%), closely followed by \textit{Reasoning Error} (17.3\%). This suggests that the main challenge for current models in the fields of physics, chemistry and biology is to understand field-specific concepts and accurately apply relevant knowledge.

\item \textbf{General Reasoning}: \textit{Reasoning Error} is the most prevalent, accounting for 43\%, followed by comprehension errors, accounting for 19\%, showing that logical reasoning is the primary bottleneck.

\end{itemize}

\paragraph{What Are the Model-Specific Error Patterns?}

% \begin{figure}[t]
%     \centering
%     \includegraphics[width=0.8\textwidth]{figures/error_type_model.pdf}
%     % \vspace{-3mm}
%     \caption{Distribution of Error Types Across Models.}
%     % \vspace{-3mm}
%     \label{fig: error_type_model}
% \end{figure}

We also analyzed errors specific to individual models, providing further insights into model weaknesses, as illustrated in Figure \ref{fig: error_type_model}. The error distributions reveal distinct patterns for each model, highlighting their unique strengths and areas for improvement. Here are some key findings:
%Due to space constraints, we focus here on the key findings from the most commonly used models, with a comprehensive analysis of all models provided in Appendix \ref{app: error_analysis}.

\begin{itemize}[leftmargin=4mm]

\item \textbf{DeepSeek-R1} exhibits its most pronounced weakness in \textit{Reasoning Errors} (22.7\%), indicating challenges in constructing coherent and accurate logical reasoning paths. However, it demonstrates relative strength in handling fundamental tasks, with minimal \textit{Calculation Errors} (3.1\%) and \textit{Programming Errors} (4.4\%).

%achieves strong performance in detail-oriented tasks such as formula computation and code syntax. Its primary limitation lies in reasoning and comprehension capabilities.

\item \textbf{QwQ-32B-Preview} excels at identifying correct problem-solving approaches. However, its effectiveness is significantly hindered by deficiencies in handling finer details, particularly in \textit{Calculation Errors} (17.9\%)

%but its effectiveness is often undermined by deficiencies in handling finer details.

% {QwQ-32B-Preview} demonstrates a relatively balanced performance but is notably weak in \textit{Calculation Errors} (17.9\%), indicating a significant limitation in numerical precision. It also shows a moderate frequency of \textit{Understanding Errors} (17.1\%), suggesting occasional difficulties in problem interpretation. 

\end{itemize}

\begin{tcolorbox}[colback=white!95!gray, colframe=gray!70!black,  title=Key Finding for Error Type]
The primary bottleneck of current models remains reasoning ability. However, detailed errors like calculation and formal mistakes also contribute significantly.
\end{tcolorbox}


\subsection{Reflection Analysis of o1-like Models}


\begin{figure}[t]
    \centering
    \includegraphics[width=0.95\textwidth]{figures/reflection.pdf}
    \caption{Distribution of effective reflection times by models and domains on a sample level. The segments within each pie chart represent how many times effective reflection occurs in one sample, with segment `0' indicating there is no effective reflection.}
    \label{fig: error_type_model}
\end{figure}

\paragraph{Statistics.}
We also conduct a analysis of the total number of reflections and the proportion of effective reflections in the long CoT output of all questions (including questions answered correctly and incorrectly by the model). 
% On average, 
%We observe that the long CoT contains \textit{five} times reflections, indicating that current o1-like models tend to reflect frequently. 

\paragraph{How Effective Are Model Reflections Across Different Models and Domains?}
We classify samples with reflections based on the number of valid reflections to evaluate the ability to produce valid reflections. Specifically, we label samples as \texttt{0} if no valid reflections occur, and \texttt{1}, \texttt{2}, or \texttt{>=3} for samples with one, two, or three and more valid reflections, respectively(all statistical analyses were performed under strictly controlled conditions, ensuring uniform sampling and balanced tasks for a fair comparison). In Figure \ref{fig: error_type_model}, {DeepSeek-R1} exhibits the highest proportion of effective reflections, and the models show a notably higher rate of effective reflections in the {math} domain. However, the overall proportion of valid reflections across all models remains relatively low, ranging between 30\% and 40\%. This suggests that the reflection capabilities of current models require further improvement.
%Detailed statistical data can be found in Appendix D.

\begin{tcolorbox}[colback=white!95!gray, colframe=gray!70!black,  title=Key Finding for Reflection]
Despite frequent reflection attempts, the proportion of effective reflections remains low across models, and  DeepSeek-R1 achieves the highest rate of valid reflections.
\end{tcolorbox}

\subsection{Effective Reasoning of o1-like Models}

\begin{figure}[t]
    \centering
    \includegraphics[width=0.98\textwidth]{figures/effetive_reasoning.pdf}
    \caption{Distribution of effective reasoning ratios.}
    
    \label{fig: effetive_reasoning}
\end{figure}

\paragraph{Statistics.} 
% As previously mentioned, 
Human annotators evaluate the usefulness of the reasoning in each section, enabling us to calculate the proportion of valid reasoning in each response. As illustrated in Figure \ref{fig: effetive_reasoning}, each graph shows the distribution of effective reasoning ratios for a particular model. The red dashed line in each graph indicates the average effective reasoning ratio.

\paragraph{What Proportion of Reasoning in Long CoT Responses is Effective?}
On average, only 73\% of the reasoning in the collected long CoT responses is useful, highlighting significant redundancy issues. Among the models analyzed, \textit{QwQ-32B-Preview} exhibited the lowest proportion of effective reasoning at 70\%, while \textit{DeepSeek-R1} achieved a notably higher proportion compared to the others, demonstrating superior reasoning efficiency.


\begin{tcolorbox}[colback=white!95!gray, colframe=gray!70!black,  title=Key Finding for Reasoning Efficiency]
On average, 27\% of reasoning in long CoT responses we collected is redundant, and DeepSeek-R1 outperforms others in reasoning efficiency.
\end{tcolorbox}
\vspace{-3mm}

\subsection{Reasoning Process Analysis}

Figure ~\ref{fig: action_roles} shows the distribution of each section's action roles in the system II thinking process of the o1-like models. Initially, problem analysis dominates, indicating that the model initially focuses on understanding the requirements and constraints of the problem. As the solution progresses, cognitive activities diversify significantly, with reflection and validation becoming more prominent. In the later part of the reasoning, the distribution of conclusion and summarization gradually increases. 
%As the model progresses from problem analysis, solution implementation and conclusion, it demonstrates the common reasoning template of o1-like models.


\begin{figure}[t]
    \centering
    \includegraphics[width=0.8\textwidth]{figures/action_role.pdf}
    \caption{Distribution of different task types throughout the progress of a long CoT response.}
    \vspace{-3mm}
    
    \label{fig: action_roles}
\end{figure}
\subsection{Results on DeltaBench}

% Please add the following required packages to your document preamble:
% \usepackage{multirow}
\begin{table*}[!t]
\centering
\resizebox{1.0\textwidth}{!}{%
    \begin{tabular}{cccccccccccccccc}
    \toprule
    \multirow{2}{*}{\textbf{Model}} & \multicolumn{3}{c}{\textbf{Overall}} & \textbf{Math} & \textbf{Code} & \textbf{PCB} & \textbf{General} \\
    \cmidrule(lr){2-4} \cmidrule(lr){5-5} \cmidrule(lr){6-6} \cmidrule(lr){7-7} \cmidrule(lr){8-8}
     & \textbf{\textit{Recall}} & \textbf{\textit{Precision}} & \textbf{\textit{F1}} & \textbf{\textit{F1}} & \textbf{\textit{F1}} & \textbf{\textit{F1}} & \textbf{\textit{F1}} \\
    \midrule
    \multicolumn{8}{c}{\textbf{\textit{Process Reward Models (PRMs)}}} \\
    \midrule
    \rowcolor[rgb]{ .988,  .949,  .8} Qwen2.5-Math-PRM-7B & \textbf{30.30} & \textbf{34.96} & \textbf{29.22}  &  \textbf{29.64} & \textbf{23.76} & \underline{31.09} & \underline{34.19}   \\
    \rowcolor[rgb]{ .988,  .949,  .8} Qwen2.5-Math-PRM-72B & \underline{28.16} & \underline{29.37} & \underline{26.38}  & \underline{24.16} & \underline{22.02} & \textbf{31.14} & \textbf{35.83}  \\
    \rowcolor[rgb]{ .988,  .949,  .8} Llama3.1-8B-PRM-Deepseek-Data & 11.7 & 15.59 & 12.02 &  12.28 & 10.95 & 16.76 & 12.59  \\
    \rowcolor[rgb]{ .988,  .949,  .8} Llama3.1-8B-PRM-Mistral-Data & 9.64 & 11.21 & 9.45 & 9.40 & 10.72 & 13.43 & 12.40  \\
    \rowcolor[rgb]{ .988,  .949,  .8} Skywork-o1-Qwen-2.5-1.5B & 3.32 & 3.84 & 3.07 & 1.30 & 6.66 & 5.43 & 7.87  \\
    \rowcolor[rgb]{ .988,  .949,  .8} Skywork-o1-Qwen-2.5-7B & 2.49 & 2.22 & 2.17 & 0.78 & 6.28 & 6.02 & 3.11  \\
    \midrule
     \multicolumn{8}{c}{\textbf{\textit{LLM as Critic Models}}} \\
    \midrule
    \rowcolor[rgb]{ .922,  .89,  .988} GPT-4-turbo-128k & \textbf{57.19} & \textbf{37.35} & \textbf{40.76} & \textbf{37.56} & \textbf{43.06} & \underline{45.54} & \underline{42.17} \\
    \rowcolor[rgb]{ .922,  .89,  .988} GPT-4o-mini & \underline{49.88} & 35.37 & \underline{37.82} & \underline{33.26} & 37.95 & \textbf{45.98} & \textbf{46.39} \\
    \rowcolor[rgb]{ .922,  .89,  .988} Doubao-1.5-Pro & 39.68 & \underline{37.02} & 35.25 & 32.46 & \underline{39.47} & 33.53 & 37.00 \\
    \rowcolor[rgb]{ .922,  .89,  .988} GPT-4o & 36.52 & 32.48 & 30.85 & 28.61 & 28.53 & 39.25 & 36.50 \\
    \rowcolor[rgb]{ .922,  .89,  .988} Qwen2.5-Max & 36.11 & 30.82 & 30.49 & 26.73 & 32.81 & 39.49 & 29.54 \\
    \rowcolor[rgb]{ .922,  .89,  .988} Gemini-1.5-pro & 35.51 & 30.32 & 29.59 & 26.56 & 28.20 & 40.13 & 33.66 \\
    \rowcolor[rgb]{ .922,  .89,  .988} DeepSeek-V3 & 32.33 & 28.13 & 27.33 & 27.04 & 27.73 & 27.35 & 27.45 \\
    \rowcolor[rgb]{ .922,  .89,  .988} Llama-3.1-70B-Instruct & 32.22 & 28.85 & 27.67 & 21.49 & 32.13 & 28.45 & 39.18 \\
    \rowcolor[rgb]{ .922,  .89,  .988} Qwen2.5-32B-Instruct & 30.12 & 28.63 & 26.73 & 22.34 & 31.37 & 33.78 & 24.37 \\
    \rowcolor[rgb]{ .882,  .949,  .89} DeepSeek-R1 & 29.20 & 32.66 & 28.43 & 24.17 & 29.28 & 34.78 & 35.87 \\
    \rowcolor[rgb]{ .882,  .949,  .89} o1-preview & 27.92 & 30.59 & 26.97 & 22.19 & 28.09 & 33.11 & 35.94 \\
    % Gemini-2.0-flash-thinking & 14.02 & 17.36 & 14.56 & 14.79 & 11.97 & 19.34 & 15.26 \\
    \rowcolor[rgb]{ .922,  .89,  .988} Qwen2.5-14B-Instruct & 26.64 & 27.27 & 24.73 & 21.51 & 29.05 & 29.98 & 20.59 \\
    \rowcolor[rgb]{ .922,  .89,  .988} Llama-3.1-8B-Instruct & 25.71 & 28.01 & 24.91 & 18.12 & 32.17 & 27.30 & 29.93 \\
    \rowcolor[rgb]{ .882,  .949,  .89} o1-mini & 22.90 & 22.90 & 19.89 & 16.71 & 21.70 & 20.37 & 26.94 \\
    \rowcolor[rgb]{ .922,  .89,  .988} Qwen2.5-7B-Instruct & 21.99 & 19.61 & 18.63 & 11.61 & 25.92 & 29.85 & 15.18 \\
    \rowcolor[rgb]{ .882,  .949,  .89} DeepSeek-R1-Distill-Qwen-32B & 17.19 & 18.65 & 16.28 & 13.02 & 23.55 & 15.05 & 11.56 \\
    % Gemini-2.0-flash-thinking & 14.02 & 17.36 & 14.56 & 14.79 & 11.97 & 19.34 & 15.26 \\
    \rowcolor[rgb]{ .882,  .949,  .89} DeepSeek-R1-Distill-Qwen-14B & 12.81 & 14.54 & 12.55 & 9.40 & 18.36 & 10.44 & 12.01 \\
    % \rowcolor[rgb]{ .882,  .949,  .89} QwQ-32B-Preview & 10.20 & 10.17 & 9.07 & 7.38 & 8.60 & 14.97 & 10.54 \\
    \bottomrule
    \end{tabular}
}
\caption{Experimental results of PRMs and critic models on DeltaBench. \textbf{Bold} indicates the best results within the same group of models, while \underline{ underline} indicates the second best.}
% \vspace{-4mm}
\label{tab: main}
\end{table*}

% \noindent\textbf{Evaluation Metrics.}
% % To accurately assess the performance of the PRM and critic models on DeltaBench, 
% We employ \textbf{recall}, \textbf{precision}, and \textbf{macro-F1 score} for error sections as evaluation metrics. For the PRMs, we utilize an outlier detection technique based on the Z-Score to make predictions. This method was chosen because threshold-based prediction methods determined from other step-level datasets, such as those used in ProcessBench~\citep{Zheng2024ProcessBenchIP}, may not be reliable due to significant differences in dataset distributions, particularly as DeltaBench focuses on long CoT. Outlier detection helps to avoid this bias. The threshold $t$ for determining the correctness of a section is defined as:
% % \begin{align}
% $t = \mu - \sigma$,
% % \nonumber
% % \label{eq: prm_threshold}
% % \end{align}
% where $\mu$ is the mean of the rewards distribution across the dataset, and $\sigma$ is the standard deviation. Sections falling below $t$ are predicted as error sections. For critic models, all erroneous sections within a long CoT are prompted to be identified. Given that error sections constitute a smaller proportion than correct sections across the dataset, we use macro-F1 to mitigate the potential impact of the imbalance between positive and negative sections. Macro-F1 independently calculates the F1 score for each sample
% % (for our metric, each case) 
% and then takes the average, providing a more balanced evaluation metric when dealing with class imbalance.

\noindent\textbf{Baseline Models.}
% 开源(中英模型,llama3)和闭源模型
% To comprehensively evaluate the performance of current PRMs and critic models, we extensively selected and evaluated a wide range of both open-source and closed-source models on DeltaBench.
% \paragraph{Process Reward Models}
For the \textbf{PRMs}, we select the following models: Qwen2.5-Math-PRM-7B\footnote{\href{https://huggingface.co/Qwen/Qwen2.5-Math-PRM-7B}{Qwen/Qwen2.5-Math-PRM-7B}}, Qwen2.5-Math-PRM-72B\footnote{\href{https://huggingface.co/Qwen/Qwen2.5-Math-PRM-72B}{Qwen/Qwen2.5-Math-PRM-72B}}, Llama3.1-8B-PRM-Deepseek-Data\footnote{\href{https://huggingface.co/RLHFlow/Llama3.1-8B-PRM-Deepseek-Data}{RLHFlow/Llama3.1-8B-PRM-Deepseek-Data}}, Llama3.1 -8B-PRM-Mistral-Data\footnote{\href{https://huggingface.co/RLHFlow/Llama3.1-8B-PRM-Mistral-Data}{RLHFlow/Llama3.1-8B-PRM-Mistral-Data}}, Skywork-o1-Open-PRM- Qwen-2.5-1.5B\footnote{\href{https://huggingface.co/Skywork/Skywork-o1-Open-PRM-Qwen-2.5-1.5B}{Skywork/Skywork-o1-Open-PRM-Qwen-2.5-1.5B}}, and Skywork-o1-Open-PRM-Qwen-2.5-7B\footnote{\href{https://huggingface.co/Skywork/Skywork-o1-Open-PRM-Qwen-2.5-7B}{Skywork/Skywork-o1-Open-PRM-Qwen-2.5-7B}}. 
% These represent some of the best open-source PRMs currently available.
% \paragraph{Critic Models}
We select a group of the most advanced open-source and closed-source LLMs to serve as \textbf{critic models} for evaluation, which includes various GPT-4~\citep{gpt4} variants (such as GPT-4-turbo-128K, GPT-4o-mini, GPT-4o), the Gemini model~\citep{Reid2024Gemini1U}(Gemini-1.5-pro), several Qwen models~\citep{qwen2.5} (such as Qwen2.5-32B-Instruct and Qwen2.5-14B-Instruct), Doubao-1.5-Pro~\citep{doubao2025}
and o1 models~\citep{openai-o1} (o1-preview-0912, o1-mini-0912).
% , and a GPT-3.5 variant (gpt-3.5-16K).



\subsubsection{Main Results}
In Table \ref{tab: main},
we provide the results of different LLMs on DeltaBench. 
For PRMs, we have the following observations: (1). Existing PRMs usually achieve low performance, which indicates that existing PRMs cannot identify the errors in long CoTs effectively and it is necessary to improve the performance of PRMs. (2). Larger PRMs
do not lead to better performance. For example, the Qwen2.5-Math-PRM-72B is inferior to wen2.5-Math-PRM-7B.
For critic models, we have the following findings: (1)
GPT-4-turbo-128k archives the best critique results, which is better than other models (e.g., GPT-4o) a lot in DeltaBench. (2) For o1-like models (e.g., DeepSeek-R1, o1-mini, o1-preview), we observe that the results of these models are not superior to non-o1-like models, with the performance of o1-preview is even lower than Qwen2.5-32B-Instruct.
%Additionally, we observe that the QWQ and DeepSeek-R1-Distill series models exhibit weaknesses in following instructions. 
A detailed analysis of underperforming models is provided in Appendix \ref{app: underperforming}.

% model size
% domains
% o1模型跟普通模型critic能力对比分析


\subsubsection{Further Analysis}

\paragraph{Effect of Long CoT Length.}
\begin{figure}[t]
    \centering
    \includegraphics[width=1.0\textwidth]{figures/4.5.1/length2.pdf}
    \caption{The effect of long CoT length.}
    \label{fig: crtic1}
\end{figure}
In Figure \ref{fig: crtic1}, we compare the average F1-Score performance of critic models and PRMs across varying LongCoT token lengths. 
For critic models, the performance notably declines as token length increases. Initially, models like Deepseek-R1 and GPT-4o exhibit strong performance with shorter sequences (1-3k tokens). However, as token length increases to mid-ranges (4-7k tokens), there is a marked decrease in performance across all models. This trend highlights the growing difficulty for critic models to maintain precision and recall as long CoT response become longer and more complex, likely due to the challenge of evaluating lengthy model outputs. In contrast, PRMs demonstrate greater stability across token lengths, as they evaluate sections sequentially rather than processing the entire output at once. Despite this advantage, PRMs achieve lower overall scores compared to critic models on our evaluation set.

\begin{tcolorbox}[colback=white!95!gray, colframe=gray!70!black, title=Key Finding]
  Critic models exhibit significant performance degradation with longer contexts, while PRMs demonstrate consistent evaluation capability across varying lengths.
\end{tcolorbox}


\paragraph{Performance Analysis Across Different Error Types.}
\begin{figure}[t]
    \centering
    \includegraphics[width=0.9\textwidth]{figures/4.5.2/top_models_per_task.pdf}
    \caption{Results of different LLMs on top-5 errors.}
    \label{fig: top_models_per_task}
\end{figure}
Figure \ref{fig: top_models_per_task} shows the performance of different models on the five most common error types. In terms of error types, most models demonstrate the highest accuracy in recognizing calculation errors. Conversely, the recognition of strategy errors is generally the weakest. In terms of models, there is significant variation in the ability of individual models to recognize different error types. For instance, DeepSeek-V3 achieves an F1 of 36\% on calculation errors but only 23\% on strategy errors. Meanwhile, Llama3.1-8B-PRM-Deepseek performs poorly, with an F1 score of 22\% on calculation errors, and shows a significant decline in performance across the other four error types. This highlights the limited generalization capabilities of most models when recognizing various error types.

\begin{tcolorbox}[colback=white!95!gray, colframe=gray!70!black, title=Key Finding]
  Models exhibit strong performance on calculation errors but struggle with strategy errors, revealing limited generalization across error types.
\end{tcolorbox}

\begin{table}[!ht]
    \centering
    % \scriptsize
    % \footnotesize
    \begin{tabular}{cccc}
    \toprule
        \multirow{2}{*}{Model} & \multicolumn{3}{c}{HitRate@$k$ - Avg(\%)} \\ \cline{2-4}
                           & $k=1$ & $k=3$ & $k=5$ \\ 
                           % \hline
                           \midrule
        Qwen2.5-Math-PRM-7B & \textbf{49.15} & \textbf{69.14} & \textbf{83.14} \\
        Qwen2.5-Math-PRM-72B & \underline{41.13} & \underline{62.70} & \underline{75.73} \\ 
        Llama3.1-8B-PRM-Deepseek-Data & 12.63 & 48.62 & 69.78 \\
        Llama3.1-8B-PRM-Mistral-Data & 8.99 & 42.97 & 65.33 \\
        Skywork-o1-Open-PRM-Qwen-2.5-1.5B & 31.90 & 53.82 & 69.23 \\
        Skywork-o1-Open-PRM-Qwen-2.5-7B & 31.58 & 52.59 & 69.16 \\
        % \hline
        \bottomrule
    \end{tabular}
    \vspace{+3mm}
    \caption{Results of HitRate@$k$. Bold and underlined results indicate the best and the second best.}
    % \vspace{-4mm}
\label{tab: hitrate}
\end{table}

\paragraph{Analysis on HitRate evaluation for PRMs.}

\begin{figure}[t]
    \centering
    \includegraphics[width=\textwidth]{figures/prm_rank.pdf}
    % \vspace{-10pt}
    \caption{Ranking of rewards for the first incorrect section for different PRMs.}
    % \vspace{-3mm}
    \label{fig: prm_rank}
\end{figure}

To better measure the ability of PRMs to identify erroneous sections in long CoTs, we use HitRate@$k$ to evaluate PRMs. Specifically, within a sample, we rank the sections in ascending order based on the rewards given by the PRM, select the smallest $k$ sections, and calculate the recall rate for the erroneous sections among them. Specifically, we define the sorted sections as $S = \{s_1, s_2, \ldots, s_n\}$, with $E$ being the set of erroneous sections. We select the top $k$ sections, denoted as $S_k = \{s_1, s_2, \ldots, s_k\}$. The HitRate@$k$ is  calculated as:
\begin{align}
\text{HitRate@}k = \frac{|S_k \cap E|}{\min(k, |E|)}
% \nonumber
\label{eq: hitrate}
\end{align}
In this formula, $|S_k \cap E|$ indicates the number of erroneous sections identified among the top $k$ sections. This metric reflects the ability of PRMs to effectively identify erroneous sections within the top $k$ candidate sections. In Table \ref{tab: hitrate}, the relative performance rankings among different PRMs are quite similar to the results in Table \ref{tab: main}. Additionally, we observe that for $k=3$ and $k=5$, the performance differences between various PRMs are not particularly significant. However, when $k=1$, the Qwen2.5-Math-PRM-7B shows a clear performance advantage. Figure \ref{fig: prm_rank} illustrates the ranking ability of different PRMs for the first incorrect section within the sample, which is generally consistent with the performance evaluation results of HitRate@k.
% This is because a smaller $k$ value imposes stricter requirements on the PRM's ability to identify errors.

% HitRate@$k$ evaluates the performance of PRMs from the perspective of reward ranking, providing additional evidence for the experimental results and conclusions in Table \ref{tab: main} from a different angle.

\begin{tcolorbox}[colback=white!95!gray, colframe=gray!70!black, title=Key Finding]
  HitRate@k evaluation aligns with the main results, with Qwen2.5-Math-PRM-7B demonstrating superior performance in identifying the first incorrect section.
\end{tcolorbox}


\begin{figure}[t]
    \centering
    \includegraphics[width=0.8\textwidth]{figures/4.5.4/self-critic.pdf}
    % \vspace{-10pt}
    \caption{F1-score comparison of self-critique and cross-model critique abilities for different models.}
    % \vspace{-5mm}
    \label{fig: self-critic}
\end{figure}

\paragraph{Comparative Analysis of Self-Critique Capabilities of LLMs.} We randomly sample queries based on domains and models that generate the long CoT output, followed by a statistical analysis of the model's performance in evaluating its own outputs as well as those of other models. In Figure \ref{fig: self-critic},  Gemini 2.0 Flash Thinking, DeepSeek-R1, and QwQ-32B-Preview show lower self-critique scores compared to their cross-model critique scores, indicating a prevalent deficiency in self-critic abilities. Notably, DeepSeek-R1 exhibits the largest discrepancy, with a 36\% decrease in self-evaluation compared to evaluations of other models. This suggests models' self-critic abilities remain underdeveloped.
% signaling an area that requires improvement.

\begin{tcolorbox}[colback=white!95!gray, colframe=gray!70!black, title=Key Finding]
  LLMs demonstrate weaker self-critique performance compared to cross-model critique, highlighting a fundamental limitation in self-critic capabilities.
\end{tcolorbox}



%%%

% \noindent\textbf{Performance Analysis Across Different Categories}

% \begin{figure}[htbp]
% \centering
% \includegraphics[width=\linewidth]{figures/prm_task_comparison.pdf}
% \caption{Performance of PRMs across different categories (outlier detection).}
% \label{fig: prm_task}
% % \vspace{-0.6cm}
% % \vspace{-4mm}
% \end{figure}


% \noindent\textbf{Performance Variation in Different Lengths of Long CoT}

% \noindent\textbf{Performance Analysis Across Different Error Types}

% \noindent\textbf{Analysis of In-Sample Reward Ranking}


% % \subsection{Evaluation Metrics}

% % \subsection{Main Results}

% % \subsection{Further Analysis}
% \subsection{Analysis on LLM Critics}
%  \textbf{error location}



% \subsubsection{The Performance across different domains}

% \begin{figure}[t]
%     \centering
%     \includegraphics[width=0.5\textwidth]{figures/critic6.pdf}
%     \caption{The score distributions across different domains.}
%     \label{fig: crtic2}
% \end{figure}

% In Figure \ref{fig: crtic2}, we illustrate the F1-score distribution of various large language models (LLMs) across different domains. Analyzing model performance across domains reveals that most models demonstrate stronger critiquing abilities in Physics, Chemistry, Biology, and General Reasoning compared to Mathematics and Programming, indicating higher proficiency in scientific and general reasoning tasks. Meanwhile, the performance of each model varies significantly depending on the domain, reflecting inherent strengths and weaknesses in handling different tasks. For instance, the Gemini-1.5-Pro model achieves an F1-score of 40.1\% in PCB, yet only 26.6\% in Mathematics. These discrepancies underscore challenges in the models' generalization capabilities.







\section{Conclusions}\label{sec:conclusions}

\vspace{-0.2cm}
\section{Conclusions}
\vspace{-0.2cm}
In this paper, we introduce FanChuan, a multilingual benchmark for parody detection and analysis, encompassing seven datasets characterized by high diversity, rich contextual information, and precise annotations. Our findings reveal that parody detection remains highly challenging for both LLMs and traditional methods, with particularly poor performance on Chinese datasets. We also observe that contextual information significantly enhances model performance, while parody itself increases the difficulty of sentiment classification. Additionally, our results indicate that reasoning fails to improve LLM performance in parody detection. By filling a critical gap in the study of emerging online phenomena, FanChuan provides valuable insights into cultural values and the role of parody in digital discourse. These findings highlight the limitations of current LLMs, presenting an opportunity for future research to enhance model capabilities in parody detection and analysis.


\section*{Acknowledgements}
This work was supported in part by DARPA Young Faculty Award, National Science Foundation \#2127780, \#2319198,  \#2321840 and \#2312517, Semiconductor Research Corporation (SRC), Office of Naval Research, grants \#N00014-21-1-2225 and \#N00014-22-1-2067, the Air Force Office of Scientific Research under award \#FA9550-22-1-0253, and generous gifts from Xilinx and Cisco.

\bibliographystyle{IEEEtranS}
\bibliography{refs}

\end{document}

