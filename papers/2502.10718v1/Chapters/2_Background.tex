
\begin{figure*}
  \centering
  \includegraphics[width=0.6\linewidth]{Figures/diagram_ours.pdf}
  \caption{Overview of our Hyperdimensional Intelligent Sensing pipeline. The  ``sparse selective strategy'' is applied at the near-sensor stage, where only audio segments identified as audio-of-interest are transmitted to the cloud.}
  \label{Fig:overview_diagram_ours}
\end{figure*}

\subsection{Intelligent Sensing over Audio Data} 
With the continual advancements in sensor technology, there has been a parallel evolution in computational methods, aiming to extract meaningful insights from raw sensor data~\cite{ballard2021machine}. Notably, recent innovations in the form of in-sensor and near-sensor accelerators~\cite{li2021recent,angizi2022pisa,ma2022hogeye, sumbul2022system,zhou2020near, yun2024hypersense} represent a stride towards more efficient local processing. These accelerators integrate dedicated machine learning circuits into the sensing circuitry, enhancing the efficiency of data processing at the source. However, despite the success of these localized approaches, there is often a gap when it comes to achieving comprehensive system-level integration. In the specific context of audio data, recent research endeavors~\cite{Vajpayee_2023_WACV,10005289,10059143} have delved into efficient systems tailored to audio applications. These frameworks have tackled challenges ranging from memory reduction to devising low-power solutions. While these contributions showcase valuable progress, they may fall short in meeting the real-time demands of critical applications, such as acoustic gunshot detection~\cite{hansen2021gunshot}. The need for 24/7 monitoring and rapid response in scenarios like gunshot detection necessitates a paradigm shift towards real-time operation, a facet not thoroughly addressed by existing solutions.

To explicitly address these gaps, we note that existing methods either lack robust online learning capabilities at the edge or fail to drastically reduce energy consumption for large-scale deployments. Our proposed model directly tackles these shortcomings by introducing a near-sensor paradigm that can continuously adapt to new data while significantly cutting down on end-to-end system energy usage.

It is in this gap that our proposed near-sensor model stands as a pioneering framework. Going beyond existing approaches, our model is designed to minimize the overall energy consumption associated with audio data processing. Our ``sparse selective strategy'' refers to transmitting only those audio segments the near-sensor model deems as audio-of-interest, effectively filtering out irrelevant data and reducing costly cloud-side computation. What distinguishes our work is the utilization of a near-sensor principle, wherein the sensor selectively collects relevant acoustic data locally, significantly reducing the need for data transmission and processing at the central server. This marks the first attempt to detect audio of interest based on this near-sensor principle, presenting a novel and efficient approach to intelligent sensing in audio data.


\subsection{Hyperdimensional Computing} 
HyperDimensional Computing (HDC), inspired by the brain and modeled with hypervectors~\cite{kanerva2009hyperdimensional}, offers a robust computational paradigm applicable to a wide range of learning problems. This includes areas such as speech recognition~\cite{imani2017voicehd}, genome sequence alignment~\cite{kim2020geniehd}, graph learning~\cite{kang2022relhd, nunes2022graphhd}, clustering~\cite{yun2023hyperdimensional}, and computer vision~\cite{hersche2022constrained, dutta2022hdnn, yun2024spatial}. HDC is known for its high memorization capability, robustness against noise, and model interpretability. It builds upon a well-defined set of operations with random hypervectors, making it extremely resilient to failures and adaptable to diverse computational tasks.

Despite its advantages, the application of HDC to audio detection problems has not been extensively explored. This study aims to leverage the unique properties of HDC for gunshot audio detection, addressing a critical public safety concern. By utilizing HDC's efficient encoding and high-dimensional representation, we aim to develop a model that can accurately detect gunshot sounds in real time, ensuring rapid response and increased safety. This work represents a novel application of HDC, extending its proven benefits in other domains to the field of audio detection and offering a promising solution for urgent, real-world challenges in public safety.

