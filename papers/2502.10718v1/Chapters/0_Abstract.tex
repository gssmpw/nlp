The escalating challenges of managing vast sensor-generated data, particularly in audio applications, necessitate innovative solutions. Current systems face significant computational and storage demands, especially in real-time applications like gunshot detection systems (GSDS), and the proliferation of edge sensors exacerbates these issues. This paper proposes a groundbreaking approach with a near-sensor model tailored for intelligent audio-sensing frameworks. Utilizing a Fast Fourier Transform (FFT) module, convolutional neural network (CNN) layers, and HyperDimensional Computing (HDC), our model excels in low-energy, rapid inference, and online learning. It is highly adaptable for efficient ASIC design implementation, offering superior energy efficiency compared to conventional embedded CPUs or GPUs, and is compatible with the trend of shrinking microphone sensor sizes. Comprehensive evaluations at both software and hardware levels underscore the model's efficacy. Software assessments through detailed ROC curve analysis revealed a delicate balance between energy conservation and quality loss, achieving up to 82.1\% energy savings with only 1.39\% quality loss. Hardware evaluations highlight the model's commendable energy efficiency when implemented via ASIC design, especially with the Google Edge TPU, showcasing its superiority over prevalent embedded CPUs and GPUs.