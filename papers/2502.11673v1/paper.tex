\documentclass[11pt]{article}
\usepackage[top=2.5cm, bottom=2.5cm, left=2.5cm, right=2.5cm]{geometry}

\usepackage{times}
\usepackage{latexsym}

\usepackage[T1]{fontenc}

\usepackage[utf8]{inputenc}

\usepackage{microtype}

\usepackage{inconsolata}

\usepackage{graphicx}


\usepackage{amsmath}
\usepackage{amssymb}
\usepackage{multirow}
\usepackage{booktabs}
\usepackage{catchfile}

\usepackage[boxed]{algorithm}
\usepackage{varwidth}
\usepackage[noEnd=true,indLines=false]{algpseudocodex}
\usepackage{cleveref}
\makeatletter
\@addtoreset{ALG@line}{algorithm}
\renewcommand{\ALG@beginalgorithmic}{\small}
\algrenewcommand\alglinenumber[1]{\small #1:}
\makeatother

\usepackage[normalem]{ulem}
\usepackage{todonotes}

\usepackage{lipsum}    %
\usepackage{comment}   %
\usepackage{graphicx}  %
\usepackage{pifont}    %

\usepackage[font=small,labelfont=bf]{caption}
\usepackage{float}     %
\usepackage{booktabs}  %
\usepackage{subcaption}  %

\usepackage{listings}

\usepackage{amsthm}  %


\title{\bf \Large Best of Both Worlds:\\Regret Minimization versus Minimax Play}


\author[*,1]{\normalsize Adrian Müller}
\author[*,2]{Jon Schneider}
\author[*,3]{Stratis Skoulakis}
\author[*,4]{Luca Viano}
\author[4]{Volkan Cevher}

\affil[1]{ETH Zürich}
\affil[2]{Google Research}
\affil[3]{Aarhus University}
\affil[4]{EPFL}

\date{}

\singlespacing

\begin{document}


\maketitle

\begin{abstract}
    \noindent In this paper, we investigate the existence of online learning algorithms with bandit feedback that simultaneously guarantee $O(1)$ regret compared to a given comparator strategy, and $O(\sqrt{T})$ regret compared to the best strategy in hindsight, where $T$ is the number of rounds. We provide the first affirmative answer to this question. In the context of symmetric zero-sum games, both in normal- and extensive form, we show that our results allow us to guarantee to risk at most $O(1)$ loss while being able to gain $\Omega(T)$ from exploitable opponents, thereby combining the benefits of both no-regret algorithms and minimax play.
\end{abstract}


{\def\thefootnote{*}\footnotetext{Equal contribution. Correspondence to \texttt{\href{mailto:admuell@ethz.ch}{admuell@ethz.ch}}.}} 

\setlength\parindent{0pt}

\section{Introduction} \label{sec:intro}

Two-player zero-sum games form one of the most fundamental classes studied in game theory, capturing direct competition between two opposing agents. In a zero-sum game, Alice and Bob choose mixed strategies $\xb \in \Xb$ and $\yb\in\Yb$, respectively, for some strategy polytope $\Xb$. Their expected payoffs are specified by a function $V$. Alice aims to minimize $V(\mu,\nu)$, whereas Bob aims to maximize it. This definition subsumes the classical \emph{normal-form zero-sum games} \citep[e.g., Rock-Paper-Scissors]{von2007theory} as well as the more complex $\textit{extensive-form zero-sum games}$ \citep[e.g., Heads-up Poker]{osborne1994course}. A zero-sum game is symmetric if interchanging the actions of the players interchanges payoffs, i.e. $V(\mu,\nu) = -V(\nu,\mu)$.

Now suppose Alice repeatedly plays a symmetric zero-sum game against Bob for $T$ consecutive rounds. In each round, she chooses her next strategy based on her previous observations, and Bob does likewise. There are two popular lines of thought on how Alice could minimize her overall cost over the $T$ rounds of play:

\textbf{1) Min-Max Equilibrium:} In every round $t$, Alice selects $\mu^t = \mu^\star \in \arg\min_{\mu 
\in \Xb}\max_{\nu \in \Yb} V(\mu,\nu)$ \citep{von2007theory}. She then loses at most $V^\star := \min_{\mu \in \Xb}\max_{\nu \in \Yb} V(\mu,\nu)$ money. For symmetric zero-sum games, we have $V^\star = 0$, meaning that:

\smallskip
\smallskip
\begin{center}
    \textit{Alice is guaranteed not to lose any money, but might not win money\footnote{E.g., she never wins money if $\mu^\star$ is full-support \citep{braggion2020strong}, and even otherwise may not win anything.} even if Bob plays poorly.}
\end{center}
\smallskip
\smallskip

\textbf{2) Regret Minimization:} Alice selects $\mu^t \in \Xb$ according to a \emph{no-regret algorithm} \cite{cesa2006prediction}. Then, no matter Bob's strategies
$\nu^1,\ldots,\nu^T \in \Yb$, the \textit{regret} compared to any fixed strategy $\mu$ satisfies
\begin{equation*}
    \sum_{t=1}^T V(\mu^t,\nu^t) - \sum_{t=1}^T V(\mu,\nu^t)\leq O(\sqrt{T}).
\end{equation*}
In symmetric zero-sum games, plugging in the equilibrium $\mu=\mu^\star$, we have $V(\mu,\nu^t)\leq V^\star = 0$. This means that Alice might lose up to $O(\sqrt{T})$ 
money\footnote{There are cases where she does since there is a matching regret lower bound.}. However, if Bob plays sub-optimally, it may be the case that $\min_{\mu \in \Xb} \sum_{t=1}^T V(\mu,\nu^t) = -\Theta(T)$, meaning that Alice wins $\Theta(T)$ money. As a result:

\smallskip
\smallskip
\begin{center}
    \textit{Alice risks losing $O(\sqrt{T})$ money, but can win up to $\Theta(T)$ money if Bob plays sub-optimally.} 
\end{center}
\smallskip
\smallskip

\noindent Whether Alice will choose to play 1) a min-max equilibrium or 2) according to a no-regret algorithm depends on how \emph{risk-averse} Alice is --- how willing Alice is to risk $O(\sqrt{T})$ money in the hope of winning $\Theta(T)$. This naturally raises the question of whether we can have the best of both worlds:
\begin{question}\label{q:1}
    In a symmetric zero-sum game, can Alice risk losing at most $O(1)$ amount of money, but still be able to win up to $\Theta(T)$ amount of money if Bob plays sub-optimally?
\end{question}

In this paper, we answer this question in the affirmative by resolving the following fairly \emph{more general question} from online learning with adversarial linear costs. We explain the reduction in \cref{sec:preliminaries}.
\begin{question}\label{q:2}
    Is it possible to guarantee $O(1)$ regret compared to a specific strategy while maintaining $\OTilde(\sqrt{T})$ regret compared to the best strategy in hindsight?
\end{question}
Question \ref{q:2} is known to admit a relatively simple positive answer in the so-called full-information case (\cref{sec:related-work}). Crucially, in this work we are interested in the \emph{bandit feedback} setting, modeling the fact that Alice only observes the realized cost and not the cost for all actions she could have taken instead. We formalize this learning goal in \cref{sec:reduction,sec:efg-reduction}.

We present our results in the context of symmetric zero-sum games. However, they hold far beyond symmetric, zero-sum, or even two-player games (\cref{q:2}): for any (sufficiently explorative) comparator strategy, one can guarantee constant regret compared to it while still having rate-optimal regret compared to the best strategy in hindsight, even under bandit feedback.

\textbf{Contributions.} Our main contributions are the following:
\begin{itemize}[leftmargin=*] 
    \item We first devise an algorithm for normal-form games (NFGs) under bandit feedback that interpolates between playing the min-max equilibrium and no-regret learning. We prove that if the min-max equilibrium is supported on the whole action space\footnote{This assumption is also necessary, but can easily be relaxed, at the cost of slightly weaker guarantees on when Alice can take advantage of sub-optimal play by Bob. See \cref{rmk:restrict}.}, then our algorithm indeed satisfies the desiderata of our main question (\cref{sec:nfg-upper}). To the best of our knowledge, this is the first result of its kind under bandit feedback.
    \item We complement this regret guarantee with a lower bound for NFGs, showing that the regret bound cannot be improved significantly (\cref{sec:nfg-lower}). This implies that our algorithm is close to optimally exploiting weak strategies, as desired.
    \item We then transfer our insights to the more challenging framework of extensive-form games (EFGs). This is specifically relevant since in stateful games, it is essential to consider bandit feedback. By proposing a corresponding algorithm for EFGs, we show that even in such interactive games with imperfect information, we can answer our main question in the affirmative (\cref{sec:efg-upper}). We generalize our lower bound to this setting, too (\cref{sec:efg-lower}). 
\end{itemize}

Finally, we numerically evaluate our algorithm in simple EFG environments (\cref{sec:experiments}), showing that our results are not merely of theoretical interest. Indeed, our findings confirm our theoretical insights and demonstrate strong results even when the min-max equilibrium is not full-support.

\subsection{Related Work} \label{sec:related-work}

In online learning under \emph{full information feedback}, it is known that one can achieve constant regret against a certain comparator strategy while maintaining the optimal worst-case regret guarantee as desired in \cref{q:2} \citep{hutter2005adaptive,even2008regret,kapralov2011prediction,koolen2013pareto,sani2014exploiting,orabona2016coin,cutkosky2018black,orabona2019modern}, one notable example being the Phased Aggression template of \citet{even2008regret}. This allows us to directly answer \cref{q:1} affirmatively for NFGs if full information is available.

In stark contrast, under \textit{bandit feedback}, \citet{lattimore2015pareto} showed that constant regret compared to a \emph{deterministic} comparator strategy (i.e. a single action) is \emph{not} achievable if we want to maintain sublinear regret compared to all other actions. In \cref{thm:ub-simplex-bandit}, we show that one can break this negative result under the minimal possible assumption on the comparator strategy.

Similar to our motivation, \citet{ganzfried2015safe} consider \emph{Safe Opponent Exploitation} as deviating from the min-max strategy while ensuring at most the cost of the min-max value. The authors provide several such ``safe exploitation" algorithms, assuming the existence of so-called ``gift strategies". One key difference to our work is that they do not provide a theoretical exploitation guarantee, while our algorithm has provably vanishing regret compared to the best static response against the opponent. 

Regarding the extension of our results to EFGs, we leverage relatively recent theoretical advancements regarding online mirror descent in EFGs, most notably \citet{kozuno2021model,bai2022near}. We refer to \cref{app:related-work} for an extended discussion of related work. 

\section{Preliminaries} \label{sec:preliminaries}

In this section, we introduce the relevant notation and explain how Question \ref{q:2} answers Question \ref{q:1}. 

\textbf{General Notation.} As usual, $O$-notation expresses asymptotic behavior, and $\tilde{O}$-notation hides poly-logarithmic factors. The $n$-dimensional simplex is $\Delta_n$, and we let $[n]:=\{1,\dots,n\}$. Moreover, $e_i$ is the $i$-th the standard basis vector of $\R^n$, and $\inner{\cdot}{\cdot}$ the Euclidean inner product. Finally, $\mathbbm{1}_E$ denotes the indicator function of an event $E$.

\textbf{(Safe) Online Linear Minimization.} In Protocol \ref{prot:olm}, we introduce the framework of \emph{online linear minimization} \citep[OLM]{H17} with adversarial costs. In addition to this standard framework, Alice receives a special (``safe") \emph{comparator strategy} $\mu^c\in\Xb$ as input, compared to which Alice would like to be essentially at least as good.
\begin{figure}[H]
\centering
\begin{minipage}{1.0\textwidth}
\begin{protocol}[H]
    \normalsize
    \caption{(Safe) Online Linear Minimization} 
    \label{prot:olm}
    \centering
    \begin{algorithmic}
        \Require{Special comparator $\mu^c\in\Xb$.}
        \For{round $t = 1,\dots,T$}
            \State \textbf{Alice} chooses her next $\mu^t \in \Xb$, \textbf{Bob} chooses the cost vector $c_t$.
            \State \textbf{Alice} suffers expected cost $\inner{\mu^t}{c^t}$.
        \EndFor
        \hspace{-0.33cm}\textbf{Goal:} $\mathcal{R}(\mu^c) \leq O(1)$ and $\max_{\mu\in\Xb}\mathcal{R}(\mu)\leq \OTilde(\sqrt{T}).$
    \end{algorithmic}
\end{protocol}
\end{minipage}
\end{figure}

We define Alice's \emph{expected regret} compared to a strategy $\mu\in\Xb$ by
\begin{align*}
    \mathcal{R}(\mu) := \sum_{t=1}^T \exptn\rectangular{\inner{\mu^t-\mu}{c^t}}.
\end{align*}
The worst-case expected regret $\max_{\mu} \mathcal{R}(\mu)=\sum_{t=1}^T \exptn\rectangular{\inner{\mu^t}{c^t}}-\min_{\mu}\sum_{t=1}^T \exptn\rectangular{\inner{\mu}{c^t}}$ then measures the regret compared to the best fixed strategy $\mu$ in hindsight. Under \emph{safe OLM} (Question \ref{q:2}), we understand the problem of simultaneously guaranteeing 
\begin{align}
    \mathcal{R}(\mu^c) \leq O(1), \quad\text{and}\quad \max_{\mu\in\Xb}\mathcal{R}(\mu)\leq \OTilde(\sqrt{T}). \tag{OLM}\label{eq:safe-olm}
\end{align}
\textbf{Question \ref{q:2} Answers Question \ref{q:1}.} Now suppose Alice was able to guarantee (\ref{eq:safe-olm}). As we explain in \cref{sec:reduction,sec:efg-reduction}, both for NFGs and EFGs, we can write the expected cost in round $t$ as a linear function of the strategy, i.e.
\begin{align*}
    V(\mu,\nu^t)=\inner{\mu}{c^t}
\end{align*}
for some cost vector $c^t$. Alice can now set $\mu^c=\mu^\star=\arg\min_{\mu}\max_{\nu} V(\mu,\nu)$ to be a min-max equilibrium. Since $V(\mu^c,\nu)\leq V^\star=0$ for symmetric zero-sum games, the first part of (\ref{eq:safe-olm}) implies
\begin{align*}
    \sum_{t=1}^T \exptn\rectangular{V(\mu^t,\nu^t)} \leq \sum_{t=1}^T \exptn\rectangular{V(\mu^c,\nu^t)} + O(1) \leq O(1),
\end{align*}
no matter Bob's play. Alice will thus lose at most a constant amount in expectation. Furthermore, if (for example) Bob plays a fixed strategy $\nu^t=\nu\in\Xb$ that is suboptimal in the sense that $\min_{\mu}V(\mu,\nu) = -c < 0$,\footnote{More generally, by \emph{exploitable} we mean that Bob plays an oblivious sequence of strategies $\nu^t$ with $\min_{\mu} \sum_t V(\mu,\nu^t) = -\Theta(T)$. We briefly discuss the adaptive case in \cref{app:related-work}.} then the second part in (\ref{eq:safe-olm}) shows
\begin{align*}
    \sum_{t=1}^T \exptn\rectangular{V(\mu^t,\nu^t)} \leq \min_{\mu} \sum_{t=1}^T V(\mu,\nu) + \OTilde(\sqrt{T}) \leq -\Theta(T),
\end{align*}
and Alice will linearly exploit Bob. We will thus state our results in terms of safe OLM, keeping in mind that the above reduction will automatically answer our initial \cref{q:1}.

\section{Normal-Form Games} \label{sec:simplex}

Suppose Alice and Bob repeatedly play a (symmetric zero-sum) \emph{normal-form game} for $T$ rounds, which means the following. In each round $t$, they simultaneously submit actions $a^t,~b^t\in[A]$ by sampling from mixed strategies $\mu^t,~\nu^t\in\Delta_A$, respectively. Alice receives cost $U_{a^t,b^t}=\inner{e_{a^t}}{U e_{b^t}}$, for some fixed cost matrix $U\in\R^{A\times A}$ with entries in $[0,1]$. We consider \emph{bandit feedback}, meaning that Alice only observes her cost $U_{a_t,b_t}$ and not the cost of actions she could have taken instead. 

\subsection{From NFGs to Online Linear Minimization} \label{sec:reduction}
By defining Alice's \emph{cost function} as
\begin{align*}
    c^t := U e_{b^t}\in\R^A,    
\end{align*}
we see that Alice's expected cost is $V(\mu^t,\nu^t):=\exptn\rectangular{U_{a_t,b_t}}=\exptn\rectangular{\inner{\mu}{c^t}}$, as $a^t\sim \mu^t$. We are thus in the setting of OLM (Protocol \ref{prot:olm}) over $\Xb=\Delta_A$. Notably, Alice does not observe the full cost function $c^t$ but only its entry $c^t(a^t)=U_{a^t,b^t}$ at the chosen action (bandit feedback). We formally consider Protocol \ref{prot:bandit-simplex} for \emph{any} adversarially picked cost functions $c^t$. From \cref{sec:preliminaries} we know that it is now sufficient to set $\mu^c=\mu^\star$ and guarantee (\ref{eq:safe-olm}).
\begin{figure}[H]
\centering
\begin{minipage}{1.0\textwidth}
\begin{protocol}[H]
    \caption{Bandit Feedback over the Simplex (NFGs)} 
    \label{prot:bandit-simplex}
    \centering
    \begin{algorithmic}
        \Require{Special comparator $\mu^c\in\Delta_A$.}
        \For{round $t = 1,\dots, T$}
            \State \textbf{Alice} chooses her next action $a^t \sim \mu^t \in \Delta_A$, \textbf{Bob} chooses the cost $c^t \in \R^A$. \Comment{\color{blue} NFG: $c^t=U e_{b^t}$} 
            \State \textbf{Alice} suffers and observes cost $c^t(a^t)$. \Comment{\color{blue} NFG: $U_{a^t,b^t}$}	
        \EndFor
    \end{algorithmic}
\end{protocol}
\end{minipage}
\end{figure}

\subsection{Upper Bound} \label{sec:nfg-upper}

Our first main result shows that if the special comparator strategy lies in the interior of the simplex, we are able to guarantee constant regret while maintaining low worst-case regret at the optimal rate. 
\begin{restatable}{theorem}{thmUbSimplexBandit}\label{thm:ub-simplex-bandit}
    Let $\delta\in(0,1/A]$. Consider any mixed strategy $\mu^c\in \Delta_A$ such that $\mu^c(a)\geq \delta$ for all $a\in[A]$. Under bandit feedback (Protocol \ref{prot:bandit-simplex}), for any $c^t\in[0,1]^A$, \cref{algo:phased-aggression-nfg} achieves 
    \begin{align*}
        \mathcal{R}(\mu^c) \leq 1, \quad \text{and} \quad \max_{\mu\in\Delta_A} \mathcal{R}(\mu) \leq \OTilde\round{\delta^{-1}\sqrt{T}}.
    \end{align*} 
\end{restatable}
For the case of NFGs, this means that if the min-max strategy $\mu^\star$ is full-support, then in expectation: Alice will lose at most $O(1)$ money while winning $\Omega(T)$ money if Bob plays (oblivious) strategies that are linearly exploitable.

\citet{lattimore2015pareto}'s result implies that the assumption on $\mu^c$ is also necessary. In addition, we show in \cref{thm:lower-nfg} that a multiplicative dependence on $\delta^{-1}$ is unavoidable. We remark that min-max strategies $\mu^\star$ of various zero-sum games are $\delta$-bounded away from zero. For example in Rock-Paper-Scissors $\mu^\star= (1/3,1/3,1/3)$. More importantly, even when this is not the case, we remark the following.
\begin{remark} \label{rmk:restrict}
    Alice can apply the result even in zero-sum games with min-max strategies $\mu^\star \in \Delta_A$ that are not full-support. Indeed, she can consider the subset of actions $\Acal':=\{a \in [A] \colon \mu^\star(a) > 0\}$. Then, our algorithm run on $\Acal'$ still guarantees $\mathcal{R}(\mu^\star)\leq O(1)$, meaning that Alice can lose at most $O(1)$ in symmetric zero-sum games. At the same time, our algorithm guarantees that $\sum_{t=1}^T \exptn\rectangular{V(\mu^t,\nu^t)} \leq \min_{\mu\in\Delta'} \sum_{t=1}^T \exptn\rectangular{V(\mu,\nu^t)} + \OTilde(\sqrt{T})$, where $\Delta'$ is the simplex restricted to $\Acal'$. This means that if Bob plays suboptimally, Alice can still guarantee to win $\Theta(T)$ whenever these actions allow her to do so (while $\mu^\star$ itself does not guarantee this).
\end{remark}


\begin{figure*}[t]
\centering
\begin{minipage}{1.0\textwidth}
\begin{algorithm}[H]
    \caption{Phased Aggression with Importance-Weighting} \label{algo:phased-aggression-nfg}
    \begin{algorithmic}[1]
        \Require{Number of rounds $T$, comparator margin $\delta$, regret upper bound $R\gets\delta^{-1} \sqrt{2T\log(A)}$, OMD learning rates $\eta \gets \sqrt{\delta^2\log(A)/(2T)}$, $\tau\gets \sqrt{2\log(A)/(AT)}$.}
        \vspace{1em}
        \State Initialize $\muHat^{1}(a) \gets \frac{1}{A}$ for all $a\in [A]$, initialize $\alpha \gets 1/R$, $\text{start}\gets 1$, $k \gets 1$ (counts phase).
        \For{round $t =1, \ldots, T$}
            
            \State \textbf{Alice} chooses $\mu^t \in \Delta_{A}$, \textbf{Bob} selects cost $c^t$. \Comment{{\color{blue}in NFGs: $c^t=Ue_{b^t}$}}
            
            \State \textbf{Alice} suffers and observes cost $c^t(a^t)$ for $a^t\sim \mu^t$.\label{line:sample-nfg}
            \State \textbf{Alice} builds cost estimator $\cHat^t(a) \gets \frac{c^t(a^t)}{\mu^t(a)}\indicator{a^t=a}$. \label{line:loss-nfg}
            
            \If {$\max_{\mu\in\Delta_{A}}\sum^t_{j = \text{start}} \inner{\cHat^j}{\mu^c - \mu} > 2 R$ \textbf{ and } $\alpha < 1$} \label{line:new-phase-nfg}
                \State $k \gets k + 1$, $\text{start} \gets t+1$. \Comment{If comparator performs poorly, new phase}\label{line:new-k}
                \State $\muHat^{t+1}(a) \gets \frac{1}{A}$ for all $a\in [A]$.
                \Comment{Re-initialize OMD}\label{line:reset-omd-nfg}
                \State Update $\alpha \gets \min\curly{2^{k-1}/R,~ 1}$. \Comment{Increase $\alpha$ for upcoming phase} \label{line:new-phase2-nfg}
        
        \Else \Comment{OMD update}
            \State $\muHat_x^{t+1} \gets \arg\min_{\mu \in \Delta_A} \round{\eta' \inner{\mu}{\cHat^t} + \Dkl(\mu || \muHat^t)}$, with $\eta' = \eta$ if $\alpha < 1$, and $\eta'=\tau$ if $\alpha = 1$. \label{line:omds-nfg}
            
        \EndIf   
            \State $\mu^{t+1} \gets \alpha \muHat^{t+1} + \left( 1 - \alpha \right) \mu^c$. \Comment{Play shifted OMD to $\mu^c$ by $1-\alpha$} \label{line:combine-bandit-nfg}
        \EndFor
    \end{algorithmic}
\end{algorithm}
\end{minipage}
\end{figure*}

\textbf{Our Algorithm.} In this section, we present \cref{algo:phased-aggression-nfg} and explain its key steps. Our algorithm is inspired by the \emph{Phased Aggression} algorithm, originally proposed by \citet{even2008regret} for the \textit{full-information setting}. We briefly note that a direct application of existing full-information algorithms is not possible. This is because, in the bandit setting, Alice only observes her realized cost and not the cost of the other possible actions she could have chosen. We will thus combine the phasing idea of \citet{even2008regret} with appropriately importance-weighted estimators of the full cost function.\footnote{The same adaptation would not yield our result for full-information algorithms other than Phased Aggression.} 

We now give an outline of \cref{algo:phased-aggression-nfg}. In every round $t$, the Phased Aggression algorithm plays a convex combination between the comparator strategy $\mu^c$ and the strategy $\muHat^t$ chosen by a no-regret algorithm (which runs in parallel). That is, the played strategy is $\mu^t=\alpha\muHat^t+(1-\alpha)\mu^c$ for some $\alpha \in (0,1]$. Whenever the algorithm estimates that the comparator $\mu^c$ is a poor choice, it increases $\alpha$ by a factor of two (so that it puts less weight on $\mu^c$ and more on the no-regret iterates) and restarts the no-regret algorithm. We group all rounds according to these restarts and call them \emph{phases} $k=1,2,\dots$. During each phase, $\alpha$ is constant.

Within this phasing scheme, the specifics of our algorithm are as follows. The no-regret algorithm of our choice is standard online mirror descent\footnote{As regularizer, we use the standard KL divergence $\Dkl(\mu||\mu'):=\sum_a \mu(a)\log(\mu(a)/\mu'(a))$.} \citep[OMD]{H17}. In every round $t$, the algorithm plays its current action $a^t \sim \mu^t$ and observes its cost (Line \ref{line:sample-nfg}). It uses this to construct an importance-weighted estimator $\cHat^t$ of the (unobserved) full cost function (Line \ref{line:loss-nfg}). The algorithm then performs one iteration of OMD with the estimated costs (Line \ref{line:omds-nfg}). This procedure is repeated until a new phase is started (Line \ref{line:new-phase-nfg}), which happens if the comparator $\mu^c$ is performing poorly under the estimated $\cHat^t$'s of the current phase.

Regarding computation, the OMD update can be implemented in closed form as $\muHat^{t+1}(a)\propto \muHat^{t}(a)\exp(-\eta' \cHat^t(a))$. We can check the if-condition in Line \ref{line:new-phase-nfg} by directly computing the maximum in $O(A)$ time.

\textbf{Regret Analysis.} In this section we
provide a proof sketch of Theorem \ref{algo:phased-aggression-nfg}. We defer the full proof to \cref{app:nfg-upper}.

We first introduce some notation. We index the variables by their respective phase $k\geq 1$: Phase $k$ lasts from $\start_k$ to $\start_{k+1}-1$ and uses linear combinations with $\alpha^k = \min\{1,2^{k-1}/R\}$ (Lines \ref{line:new-k}, \ref{line:new-phase2-nfg}). By design, there are at most $1+\lceil\log_2(R)\rceil$ phases, where $R$ is a known regret upper bound for OMD input to the algorithm. The overall regret is at most the sum of regrets across all phases, and we will thus analyze each phase separately. To this end, let 
\begin{align*}
    \RHat^k(\mu):= \sum_{t=\start_k}^{\start_{k+1}-1} \inner{\cHat^t}{\mu^t-\mu} 
\end{align*}
denote the \emph{estimated regret} during phase $k$. The following lemma bounds this estimated regret for phases with $\alpha^k<1$. 
\begin{restatable}[During normal phases]{lemma}{lemmaDuringNfg} \label{lemma:during-nfg}
    Let $k$ be such that $\alpha^k<1$. Then for all $\mu\in\Delta_A$,
    \begin{align*}
        \RHat^k(\mu) \leq 2R+2= 2\delta^{-1} \sqrt{2T\log(A)}+2,
    \end{align*}
    and for the special comparator $\RHat^k(\mu^c) \leq 2^{k-1}$.
\end{restatable}
\noindent The first part of the theorem establishes a worst-case bound on the estimated regret. Such a bound would normally not be possible for importance-weighted cost estimators. In our case, during phases with $\alpha^k<1$, we put constant weight on the comparator strategy $\mu^c$, which in turn is lower bounded by $\delta>0$. Our estimated costs (Line \ref{line:loss-nfg}) will thus be upper bounded, which is a key step in the proof. The second part of the theorem easily follows using the definition of $\alpha^k$.

\noindent Next, suppose the algorithm exits a phase $k$ as the if-condition in Line \ref{line:new-phase} holds. The following lemma establishes that exiting the phase is justified in the sense that we perform sufficiently well compared to the special comparator, according to the estimated costs. 
\begin{restatable}[Exiting a phase]{lemma}{lemmaExitNfg} \label{lemma:exit-nfg}
    Let $k$ be such that $\alpha^k<1$. If \cref{algo:phased-aggression-nfg} exits phase $k$, then $\RHat^k(\mu^c)\leq -2^{k-1}$.
\end{restatable}
We are now ready to prove \cref{thm:ub-simplex-bandit}. First, consider the case that $\alpha=1$ is never reached. 
Note that our cost estimates are unbiased, i.e. $\exptn\rectangular{\cHat^t(a)} = c^t(a)$. It is thus sufficient if we can bound $\RHat^k$. As there are $O(\log R)$ phases, \cref{lemma:during-nfg} implies $\max_{\mu} \mathcal{R}(\mu) \leq O(R\log R)$. Moreover, the previous two lemmas geometrically balance the regret compared to $\mu^c$ to be at most $1$, and we conclude. Second, suppose now that $\alpha=1$ is reached. The final phase will then simply be OMD with standard importance-weighting (a.k.a. Exp3), as we put no weight on the special comparator $\mu^c$. While we cannot apply \cref{lemma:during-nfg}, we can directly bound the remaining \emph{expected} regret of Exp3 \citep{orabona2019modern}. We can thus use the same argument as before, with one additional phase.

\subsection{Lower Bound} \label{sec:nfg-lower}

We will now show that regarding the guarantee we provided in \cref{thm:ub-simplex-bandit}, a multiplicative dependence on the inverse of the ``exploration gap" $\delta$ is indeed unavoidable.\footnote{In fact, if the cost functions are stochastic rather than adversarial, we can match this lower bound, see \cref{app:nfg-stochastic}.}
\begin{restatable}{theorem}{thmLowerNfg} \label{thm:lower-nfg}
    Let $\delta\in(0,1/A]$. There is a comparator $\mu^c\in\Delta_A$ with all $\mu^c(a) \geq \delta$ such that for any algorithm for Protocol \ref{prot:bandit-simplex} there is a sequence $c^1,\dots,c^T\in[0,1]^A$ such that: If $\mathcal{R}(\mu^c) \leq O(1)$, then  
    \begin{align*}
        \max_{\mu\in\Delta_A} \mathcal{R}(\mu) \geq \Omega(\sqrt{\delta^{-1}T}-\delta^{-3/4}T^{1/4}).
    \end{align*}
\end{restatable}
In the context of symmetric zero-sum games, our lower bound establishes that in expectation: If Alice is willing to lose at most $O(1)$ money, the best she can hope to gain from any (fixed) exploitable strategy is $\Theta(T) - \Omega(\delta^{-1/2}\sqrt{T})$.

The key idea of our proof is that any algorithm with low regret compared to $\mu^c=(1-\delta,\delta)$ for $A=2$ actions will need to play action 1 most of the time if one can information-theoretically not detect that action 2 is, in fact, minimally better. We defer the proof to \cref{app:nfg-lower}.


\section{Extensive-Form Games} \label{sec:efg}

In this section, we present our results for EFGs. We start by giving the definition of EFGs that appears in \citet{kozuno2021model,bai2022near,fiegel2023adapting,fiegel2023local}, see \cref{app:efg-background} for a brief discussion. For clarity, we present the \textit{two-player, symmetric zero-sum case}, although our results readily generalize to arbitrary EFGs.

\begin{definition}\label{d:efg}
    An \emph{EFG} is a tuple $(H,\Scal,\Xcal,\Acal,P,x,y,u)$, where 
    \begin{itemize}[leftmargin=*]
        \item there are $2$ players, Alice and Bob. $\Acal=[A]$ denotes the set of possible actions for both players.
        \item $\Scal$ denotes the set of states of the game. $H \in \mathbb{N}$ is the horizon of the game. At stage $h \in [H]$, $\Scal_h \subseteq \Scal$ denotes the possible states. 
        \item $P:=(p_0,p)$ is the transition kernel; the game's state is sampled according to $s_{h+1} \sim p(\cdot | s_h, a_h , b_{h})$ upon actions $a_h, b_h \in \Acal \times \Acal$ in state $s_h \in \Scal$. The initial state is sampled according to $s_1\sim p_0\in\Simplex{\Scal}$.
        \item $u(s,a,b) \in[-1,1]$ is Alice's random \emph{cost} (Bob's reward) for actions $(a,b) \in \Acal\times\Acal$ chosen in state $s \in \Scal$, with mean $\bar{u}(s,a,b)$.
        \item Alice (resp. Bob) observes information sets (infosets) from $\Xcal$ ($|\Xcal|=X$). Infosets are described by a surjective function $x \colon \Scal \to \Xcal$ (resp. $y\colon \Scal \to \Xcal$).
    \end{itemize}
\end{definition}

\noindent The idea behind infosets is that Alice (resp. Bob) has imperfect information about the state of the game: she cannot differentiate between sates $s,s' \in \Scal$ that belong at the same infoset, i.e. when $x(s) = x(s')$. This is reflected in the definition of her policy set. 

\begin{definition}\label{def:strategy}
    A \emph{policy} is a mapping $\pi \colon \Xcal \to \Delta_A$. We denote the set of all such policies by $\Pi$. 
\end{definition}
We let $\pi(a|x)$ denote the probability of playing action $a \in \mathcal{A}$ in states $s \in \Scal$ from infoset $x=x(s) \in \mathcal{X}$. As Alice cannot differentiate between states $s,s' \in \mathcal{S}$ from the same infoset, she must act the same way if $x(s) = x(s')$.
\begin{definition}
    Given policies $(\pi_A,\pi_B) \in \Pi \times \Pi$, the \emph{expected total cost} for Alice equals 
    \begin{align*}
        V(\pi_A,\pi_B) := \exptn\rectangular{\sum_{h=1}^H u(s_h,a_h,b_h)},
    \end{align*}
    where $(s_h,a_h,b_h)$ are the state and actions at stage $h \in [H]$ via $a_h\sim\pi_A(\cdot|x(s_h))$, $b_h\sim\pi_B(\cdot|y(s_h))$, and $s_{h+1} \sim p(\cdot | s_h,a_h,b_h)$.
\end{definition}
For the remainder of the section, we make the following assumptions, which are standard in the EFG literature \citep{kozuno2021model,bai2022near,fiegel2023adapting,fiegel2023local}.
\begin{assumption}\label{a:1}
    \begin{itemize}[leftmargin=*]
        \item \textbf{Tree structure:} For any state $s_h \in \Scal_h$, there exists a unique sequence $(s_1,a_1,b_1, \dots,\allowbreak s_{h-1},a_{h-1},b_{h-1})$ leading to $s_h$.
        \item \textbf{Perfect recall:} Let $s,s'$ such that $x(s) = x(s')$. Then:
        \begin{itemize}[leftmargin=*]
            \item There exists $h \in [H]$ such that $s,s' \in \Scal_h$.
            \item Let $(s_1,a_1,\dots,s_{h-1},a_{h-1})$ be the unique path leading to $s$ and $(s'_1,a'_1,\dots,s'_{h-1},a'_{h-1})$ the unique path leading to $s'$. Then for all $k \in [h-1] \colon x(s_k) = x(s'_k)$ and $a_k = a'_k$.
        \end{itemize}
    \end{itemize}
\end{assumption}
\textit{Tree structure} states that the game proceeds in rounds during which the players cannot loop back to a previous state. \textit{Perfect recall} establishes that the players never forget the history of play. They can only consider two states as the same infoset if the observations so far have been the same \cite{HGPS10}. The latter implies that infosets are partitioned along the horizon, i.e. $\mathcal{X} = \bigcupdot_{h\in [H]}\mathcal{X}_h$, and the same holds for the states.

\textbf{Online Learning in EFGs.} Now suppose Alice and Bob repeatedly play an EFG for $T$ consecutive rounds. In each round $t \in [T]$, Alice and Bob select a pair of policies $(\pi_A^t,\pi_B^t) \in \Pi\times\Pi$. Then a trajectory $(s_1^t,a_1^t,b_1^t,u_1^t,\dots,s_H^t, a_H^t, b_H^t, u_H^t)$ is sampled according to the policies $(\pi_A^t,\pi_B^t)$ and Alice suffers cost $\sum_{h=1}^H u_h^t$, as summarized in Protocol \ref{prot:EFGs-bandit}. 

\begin{figure}[H]
\centering
\begin{minipage}{1.0\textwidth}
\begin{protocol}[H]
    \caption{Bandit Feedback over Policies (EFGs)}
    \label{prot:EFGs-bandit}
    \centering
    \begin{algorithmic}
        \Require{A comparator policy $\pi^c\in\Pi$}.
        \For{round $t=1, \dots, T$}
            \State \textbf{Alice} selects $\pi_A^{t}\in\Pi$, \textbf{Bob} selects $\pi_{B}^{t}\in\Pi$.
            \State \textbf{Alice} obtains costs $\sum_{h=1}^H u_h^t$ and observes trajectory $(x_1^t,a_1^t,u_1^t,\dots,x_H^t, a_H^t, u_H^t)$.
        \EndFor
    \end{algorithmic}
\end{protocol}
\end{minipage}
\end{figure}

We remark in EFGs, we are naturally in the \textit{bandit feedback} setting as Alice only observes the trajectory $(x_1^t,a_1^t,u_1^t,\dots,x_H^t, a_H^t, u_H^t)$. Under \textit{full-information feedback}, Alice would observe Bob's actual policy $\pi_B^t \in \Pi$. 
\begin{remark}[Importance of bandit feedback in EFGs]
    In EFGs, bandit feedback is considerably more natural than full-information feedback. This is due to the fact that when playing against Bob, the realized samples are only observed along \emph{one single trajectory} in the game tree. Observing full information would thus mean knowing Bob's counterfactual policy in states that have never been visited during play, which is not realistic.
\end{remark}

\subsection{From EFGs to Online Linear Minimization} \label{sec:efg-reduction}

As mentioned, we once more resort to the more general OLM problem. Yet this time, our strategy polytope will be the so-called treeplex $\mathcal{P}=\Tcal$ rather than the simplex. The following definition provides an equivalent characterization of a policy. It will allow us to view the expected cost $V(\pi^t_A,\pi^t_B)$ as a (bi-)linear function \citep{HGPS10}.
\begin{definition}\label{d:treeplex}
    A vector $\mu \in \mathbb{R}^{X\cdot A}$
    belongs to the \emph{treeplex} $\Tcal$ iff for all $x_h\in \Xcal_h$ and $a\in\Acal$,
    \begin{align}
        \begin{cases}
            \mu(x_h,a) \geq 0, \label{eq:pseudo}\\
            \sum_{a_h\in\Acal} \mu(x_h,a_h) = \mu(x_{h-1},a_{h-1}) ,
        \end{cases}
    \end{align}
    where $(x_{h-1},a_{h-1})$ is the unique predecessor pair reaching $x_h$. We consider $\mu(x_0,a_0)=1$ for the root by convention.
\end{definition}

\begin{remark}
    There is the following equivalence between \cref{def:strategy,d:treeplex}. Given a policy $\pi\in\Pi$, we can define a unique $\mu_{\pi} \in \Tcal$ by $\mu_{\pi}(x_h,a_h) = \prod_{h'=1}^h \pi(a_{h'}|x_{h'})$, where the $(x_{h'},a_{h'})$ form the unique path to $(x_h,a_h)$. Vice-versa, given $\mu \in \Tcal$, we can recover the corresponding policy via $\pi_{\mu}(a|x)=\mu(x,a)/\sum_{a'}\mu(x,a')$.    
\end{remark}
By convention, we thus identify policies $(\pi_A^t,\pi_B^t)$ with their corresponding treeplex strategies $(\mu^t,\nu^t)$ and write $V(\mu^t,\nu^t)$ for Alice's expected cost. The following lemma shows that this definition indeed allows us to view Protocol \ref{prot:EFGs-bandit} as a (safe) OLM problem (Protocol \ref{prot:olm}).
\begin{lemma}
    For any state $s\in\Scal_h$, infoset $x=x(s) \in \Xcal_h$ and action $a \in \Acal$, let $(s_{1},a_{1},b_{1}, \dots ,s_{h-1},a_{h-1},b_{h-1})$ be the unique path leading to $s$. 
    Let $p(s):=p_0(s_1)\prod_{1\leq h'\leq h-1} p(s_{h'+1} | s_{h'},a_{h'},b_{h'})$, and consider
    \begin{align}
        c^t(x,a) := \sum_{\substack{s\colon x(s)=x,\\ b\in\Acal}} p(s)\cdot \nu^t(y(s),b) \cdot \bar{u}(s,a,b). \label{eq:def-loss}
    \end{align}
    Then $V(\mu,\nu^t) = \inner{\mu}{c^t}$ for all $\mu\in\Tcal$.
\end{lemma}
Alice does not observe the full cost function $c^t$, as we are in the bandit feedback setting. Yet, this lemma establishes that Protocol \ref{prot:olm} over the treeplex $\Xb = \Tcal$ covers EFGs. Thus, it is sufficient to solve the safe OLM problem (\ref{eq:safe-olm}).

\subsection{Upper Bound} \label{sec:efg-upper}

As in the simplex case, our Algorithm \ref{algo:phased-aggression-efg-bandit} guarantees \cref{eq:safe-olm} for any policy $\mu^c\in \Tcal$ that is $\delta$-bounded away from the boundary of the strategy polytope. Once more, we can resort to a restricted action set to relax this assumption (\cref{rmk:restrict}).

\begin{restatable}{theorem}{thmUbInteriorEfg} \label{thm:ub-interior-efg}
    Let $\delta \in (0,1/A]$. For any special comparator $\mu^c\in\Tcal$ such that $\mu^c(x,a) \geq \delta$ for all $x$, $a$, \cref{algo:phased-aggression-efg-bandit} achieves (for any $c^t$'s from \cref{eq:def-loss})
    \begin{align*}
        \mathcal{R}(\mu^c) \leq 1, \quad \text{and} \quad \max_{\mu\in\Tcal} \mathcal{R}(\mu) \leq \tilde{O}\round{\delta^{-1}\sqrt{XH^3T}}.
    \end{align*}
\end{restatable}

\begin{figure*}[ht]
\centering
\begin{minipage}{1.0\textwidth}
\begin{algorithm}[H]
    \caption{Phased Aggression with Importance-Weighting for EFGs} \label{algo:phased-aggression-efg-bandit}
    \begin{algorithmic}[1]
        \Require{Number of rounds $T$, comparator margin $\delta$, regret bound $R\gets\delta^{-1}\sqrt{8XH^3\log(A) T}$, learning rate $\eta \gets \sqrt{\delta^{2}X\log(A)/(8H^2T)}$, balanced learning rate $\tau \gets \sqrt{XA\log(A)/(H^3T)}$.}

        \vspace{1em}
        \State Initialize $\muHat^{1}(x_h,a) \gets \frac{1}{A^h}$ ($h\in[H]$, $x_h\in\Xcal_h$, $a\in \Acal$), initialize $\alpha \gets 1/R$, $\text{start}\gets 1$, $k \gets 1$ (counts phase).
        %\vspace{0.1cm}
        \For{round $t =1, \dots, T$}
            
            \State \textbf{Alice} chooses $\mu^t \in \Tcal$, \textbf{Bob} selects strategy $\nu^t\in\Tcal$. \Comment{{\color{blue}and thus cost $c^t$ via \cref{eq:def-loss}}}
            
            \State \textbf{Alice} obtains costs $\sum_{h=1}^H u^t_h$, observes $(x_1^t,a_1^t,u_1^t,\dots,x_H^t, a_H^t, u_H^t)$. \Comment{{\color{blue}$V(\mu^t,\nu^t)$ in exptn.}}\label{line:sample}
            
            \State \textbf{Alice} builds cost estimator $\cHat^t(x_{h},a) \gets \frac{\mathbbm{1}\{(x_{h}^t,a_{h}^t)=(x_{h},a)\}u^t_h}{\mu^t(x_{h},a)}$. \label{line:loss}
            
            \If {$\max_{\mu\in\Tcal}\sum^t_{j = \text{start}} \inner{\cHat^j}{\mu^c - \mu} > 2 R$ \textbf{ and } $\alpha < 1$} \label{line:new-phase}
            
                \State $\text{start} \gets t+1$, $k\gets k+1$. \Comment{If comparator performs poorly, next phase}
                \State $\muHat^{t+1}(x_h, a) \gets \frac{1}{A^h}$ ($h \in[H]$, $x_h\in \Xcal_h$, $a\in \Acal$).  
                \Comment{Initialize to uniform policy}\label{line:reset-omd}
                \State Update $\alpha \gets \min\curly{2^{k-1}/R,~ 1}$. \Comment{Increase $\alpha$ for upcoming phase} \label{line:new-phase2}
        %\vspace{0.3cm}
        \Else \Comment{OMD update}
            \State \label{line:omds}
            \begin{align}
                \hspace{-2.9cm}\muHat^{t+1} \gets \begin{cases}
                \arg\min_{\mu \in \Tcal} \round{\eta \inner{\mu}{\cHat^t} + D(\mu || \muHat^t)} \quad\quad &\text{(if $\alpha<1$)},\\
                \arg\min_{\mu \in \Tcal} \round{\tau \inner{\mu}{\cHat^t} + \Dbal(\mu || \muHat^t)} \quad\quad &\text{(if $\alpha=1$)}.
                \end{cases}\label{line:omd}
            \end{align}
        \EndIf
            \State $\mu^{t+1} \gets \alpha  \muHat^{t+1} + \left( 1 - \alpha \right) \mu^c$. \Comment{Play shifted OMD to $\mu^c$ by $1-\alpha$} \label{line:combine-bandit}
        \EndFor
    \end{algorithmic}
\end{algorithm}
\end{minipage}
\end{figure*}

\begin{remark}The dependence on $X$ is as good as desired in the sense that there is a $\sqrt{XAT}$ lower bound in the unconstrained case. The dependence on $H$ is less crucial for many relevant EFGs, as we often have $X\simeq A^H$ and so $H$ is a logarithmic factor. See \citet{bai2022near}.
\end{remark}

\textbf{Our Algorithm.} \cref{algo:phased-aggression-efg-bandit} is similar to our algorithm for the simplex. It combines the Phased Aggression scheme with importance-weighted OMD. However, in the EFG case, we have to generalize these notions to the treeplex. 

In particular, we use OMD with the so-called \emph{dilated} KL divergence as regularizer (Line \ref{line:omds}). As we will see in the regret analysis, to this end it is crucial that we use an \emph{unbalanced} dilated KL divergence $D$ \citep{kozuno2021model} in the phases with $\alpha<1$ and a \emph{balanced} KL divergence $\Dbal$ \citep{bai2022near} if $\alpha=1$ is reached. In \cref{app:efg-kl}, we formally define the divergences and confirm that they allow for an efficient closed-form implementation. This is crucial as we want to avoid costly projections onto the treeplex by any means. Moreover, we can efficiently check Line \ref{line:new-phase} via standard dynamic programming over the set of policies.

\textbf{Regret Analysis.} Our analysis follows a similar argument as in \cref{sec:simplex} and we defer the proofs to \cref{app:efg-upper}. The main technical challenge is to transfer the regret bounds for importance-weighted OMD from the simplex (with KL) to the treeplex $\Tcal$ (with dilated KL).

In addition, we now require a careful analysis to obtain the best possible dependence on the number of infosets $X$ and actions $A$, in the following sense. First, when upper bounding the estimated regret in analogy to \cref{lemma:during-nfg} ($\alpha<1$), we analyze OMD with the unbalanced dilated KL divergence by adapting the argument of \citet{kozuno2021model} to our importance-weighting. Using the (more sophisticated) balanced KL here would introduce an additional undesired factor of $\sqrt{A}$. Second, once $\alpha=1$ in the final phase, we analyze the expected regret of \emph{balanced} OMD instead, by adapting the argument of \citet{bai2022near} to our cost estimators. Using the unbalanced divergence would introduce an extra factor of $\sqrt{X}$, which can be prohibitively large.

\subsection{Lower Bound} \label{sec:efg-lower}

As in the case of NFGs, we show that our guarantees for \cref{algo:phased-aggression-efg-bandit} are close to being tight for EFGs of arbitrary depth. Our proof reduces an EFG of depth $H$ to the simplex case from \cref{thm:lower-nfg}. See \cref{app:efg-lower} for the proof.  
\begin{restatable}{theorem}{thmEfgLower}\label{thm:lb-efg-bandit}
    Let $A\geq 2$, $H \geq 1$, and $\delta \in (0,1)$. There exists an EFG of depth $H$ with $X=\Theta(A^H)$ such that for any $\mu^c\in\mathcal{T}$ with $\min_{x,a} \mu^c(x,a) = \delta$, there is an adversary such that for any algorithm: If $\mathcal{R}(\mu^c) \leq O(1)$, then
    \begin{align*}
         \max_{\mu\in\Tcal} \mathcal{R}(\mu) \geq \Omega(\sqrt{\delta^{-1}T} - \delta^{-3/4}T^{1/4}).
    \end{align*}
\end{restatable}

\section{Experimental Evaluations} \label{sec:experiments}
We experimentally compare our \cref{algo:phased-aggression-efg-bandit} for EFGs to the standard OMD algorithm with dilated KL \citep{kozuno2021model} as well as to minimax play. Our evaluations confirm our theoretical findings, revealing that \cref{algo:phased-aggression-efg-bandit} achieves the best of both no-regret and minimax play. We provide further details in \cref{app:further-experiments}.

\textbf{Kuhn Poker.} We consider \emph{Kuhn poker} \citep{kuhn1950simplified}, which serves as a simple yet fundamental example of two-player zero-sum imperfect information EFGs. Kuhn poker is a common $3$-card simplification of standard poker, where each player selects one card from the deck $\{$Jack, Queen, King$\}$ without replacement and initially bets one dollar.\footnote{\url{https://en.wikipedia.org/wiki/Kuhn_poker}} 
\begin{remark}
    The min-max equilibrium of Kuhn Poker is not full-support ($\delta =0$ in \cref{thm:ub-interior-efg}). As seen in \cref{rmk:restrict}, we can easily circumvent this issue by considering only the actions in the support of the equilibrium. For Kuhn Poker, this results in $\delta = 1/3$. \cref{algo:phased-aggression-efg-bandit} is then still guaranteed not to lose any money while being able to compete with the best response within the support of the equilibrium.
\end{remark}
We consider the following baseline algorithms Alice could play over $T$ rounds of Kuhn poker: 1) play the Min-Max equilibrium $\pi^\star$ in every round; or 2) run OMD with dilated KL; or 3) run \cref{algo:phased-aggression-efg-bandit} with comparator policy $\pi^\star$.

We consider two types of experiments: First, we run the three algorithms against each other to check which of the algorithms risks losing money. Second, we evaluate how well each algorithm allows Alice to exploit exploitable strategies. We repeat each experiment $5$ times.

\begin{figure*}[ht]
    \centering
    \begin{minipage}{1.0\textwidth}
    \begin{minipage}{0.05\textwidth}
        
    \end{minipage}
    \begin{minipage}{0.25\textwidth}
        \centering
        \captionsetup{labelformat=empty}
        \includegraphics[width=\linewidth]{plots/PA-Nash.png}
        \caption{Alg.~\ref{algo:phased-aggression-efg-bandit} vs Min-Max}
    \end{minipage}
    %\hfill
    \begin{minipage}{0.25\textwidth}
        \centering
        \includegraphics[width=\linewidth]{plots/Exp3-Nash.png}
        \caption*{{OMD} vs {Min-Max}}
    \end{minipage}
    %\hfill
    %\hspace{-0.5cm}
    \begin{minipage}{0.25\textwidth}
        \centering
        \includegraphics[width=\linewidth]{plots/PA-Exp3.png}
        \caption*{{Alg.~\ref{algo:phased-aggression-efg-bandit}} vs {OMD}}
    \end{minipage}
    %\hfill
    \begin{minipage}{0.05\textwidth}
        
    \end{minipage}
    \begin{minipage}{1.0\textwidth}
    \centering
    \caption{All vs all comparison for $T=1000$ rounds. The x-axis displays the round $t$, and the y-axis displays how much the respective algorithm ({\color{mblue}Min-Max}, {\color{mred}OMD}, {\color{mgreen}\cref{algo:phased-aggression-efg-bandit}}) \emph{gained} from the other.}\label{fig:all-vs-all}
    \end{minipage}
    \end{minipage}
\end{figure*}
\begin{figure*}[ht]
    \centering
    \begin{minipage}{1.0\textwidth}
    \begin{minipage}{0.05\textwidth}
        
    \end{minipage}
    \begin{minipage}{0.25\textwidth}
        \centering
        \includegraphics[width=\linewidth]{plots/All-vs-bluffJ.png}
        \caption*{All vs BluffJ}
    \end{minipage}
    %\hfill
    \begin{minipage}{0.25\textwidth}
        \centering
        \includegraphics[width=\linewidth]{plots/All-vs-KingQueen.png}
        \caption*{All vs RaiseKQ}
    \end{minipage}
    \begin{minipage}{0.25\textwidth}
        \centering
        \includegraphics[width=\linewidth]{plots/All-vs-Rand=0.2.png}
        \caption*{All vs RandMinMax}
    \end{minipage}
    \begin{minipage}{0.05\textwidth}
        
    \end{minipage}
    \begin{minipage}{1.0\textwidth}
    \centering
    \caption{All vs Bob comparison for $T=1000$ rounds. The x-axis displays the round $t$, and the y-axis displays how much {\color{mblue}Min-Max}, {\color{mred}OMD}, and {\color{mgreen}\cref{algo:phased-aggression-efg-bandit}} \emph{gained} from the second algorithm so far. The y-axes have varying scales for readability.}\label{fig:all-vs-x}
    \end{minipage}
    \end{minipage}
\end{figure*}

\textbf{All vs All.} In \cref{fig:all-vs-all} we plot the total amount of money each algorithm \emph{wins}. As \cref{fig:all-vs-all} shows, both Min-Max and \cref{algo:phased-aggression-efg-bandit} never incur losses while both gain a significant amount of money against OMD. Indeed, as (symmetrized) Kuhn poker is a symmetric zero-sum game, the min-max equilibrium is guaranteed not to lose. The same holds for our \cref{algo:phased-aggression-efg-bandit}. In contrast, a no-regret algorithm such as OMD can lose up to $O(\sqrt{T})$ amount of money. Interestingly, it does lose a similar amount against our \cref{algo:phased-aggression-efg-bandit}.

\textbf{All vs Exploitable Strategies.} We now compare the performance of Min-Max, OMD and \cref{algo:phased-aggression-efg-bandit} against the following reasonable but suboptimal strategies. The goal is to understand their ability to exploit weak strategies. We consider a) \emph{BluffJ}: Bob plays the min-max equilibrium, except that he bets (bluffs) when he has a Jack; b) \emph{RaiseKQ}: Bob raises/calls if and only if he has a King or a Queen, and checks/folds otherwise; c) \emph{RandMinMax}: Each round, with probability $0.2$, he plays the uniform strategy, and else the min-max one.

In \cref{fig:all-vs-x} we present the amount of money each algorithm \emph{wins} against these exploitable strategies. We first consider \emph{All vs BluffJ \& All vs RaiseKQ}. Algorithm~\ref{algo:phased-aggression-efg-bandit} plays conservatively and gains an amount similar to Min-Max until it takes off and starts exploiting Bob near-optimally, as OMD would. OMD, in turn, first loses a certain amount of money and only matches the gain of Min-Max after exploring sufficiently, then having the same slope as \cref{algo:phased-aggression-efg-bandit}. The min-max equilibrium itself does not exploit \textit{BluffJ} at all and exploits \textit{RaiseKQ} sub-optimally. Our algorithm suffers neither of the two drawbacks of losing money or not exploiting the weak strategy. Finally, in \emph{All vs RandMinMax}, our \cref{algo:phased-aggression-efg-bandit} improves slightly over Min-Max. OMD gains at the same rate after losing an initial amount. 


\section{Conclusion}

In this paper, we showed how to provably exploit suboptimal strategies with essentially no expected risk in repeated zero-sum games by combining regret minimization and minimax play. More generally, we believe that our novel results for adversarial bandits leading to these guarantees may be of independent interest. We hope that our work inspires future research on safe online learning, including settings like convex-concave games, learning with feedback graphs, and establishing no-swap-regret guarantees.

\section*{Acknowledgments}

This work is funded (in part) through a PhD fellowship of the Swiss Data Science Center, a joint venture between EPFL and ETH Zurich. This work was supported by Hasler Foundation Program: Hasler Responsible AI (project number 21043). Research was sponsored by the Army Research Office and was accomplished under Grant Number W911NF-24-1-0048. This work was supported by the Swiss National Science Foundation (SNSF) under grant number 200021\_205011.

\bibliographystyle{plainnat}
\bibliography{refs}


\newpage

\appendix

\section{Further Related Work} \label{app:related-work}

\textbf{Safe Opponent Exploitation.} While there have been some approaches to safe learning in games \citep{ponsen2011computing,farina2019online,zhang2021subgame,bernasconi2021exploiting,bernasconi2022safe,ge2024safe}, all these works are fairly different from our learning problem. Related to \citet{ganzfried2015safe,ganzfried2018bayesian}, the works of \citet{damer2017safely,liu2022safe} provide algorithms that interpolate between being safe and exploitive through a specific parameter. However, these algorithms may incur up to $\Omega(T)$ regret compared to the best fixed strategy in hindsight. Recently, \citet{maiti2023logarithmic} proved the first \emph{instance-dependent} poly-logarithmic regret bound for noisy $2\times2$ NFGs, which naturally relates to our desired regret bound. However, such bounds become vacuous when the game matrix does not have pairwise distinct entries and assume to observe the opponent's action (which corresponds to full information in our feedback model).

\textbf{Exploiting Adaptive Opponents.} If Bob is oblivious and plays a fixed sequence of (mixed) strategies, then any regret Alice incurs is potential utility she could gain by playing a no-regret strategy (e.g., the best-of-both-worlds strategy we present). However, if Bob is adaptive, switching to a no-regret strategy does not necessarily allow Alice to recover additional utility (Bob could, for example, react to this by playing his minimax strategy). There is a line of recent work \citep{DSSstrat, MMSSbayesian, kolumbus2022auctions, kolumbus2022and, brown2023is, cai2023selling, chen2023persuading, haghtalab2024calibrated, ananthakrishnan2024knowledge, guruganesh2024contracting, arunachaleswaran2024pareto} on how to play against sub-optimal adaptive strategies (e.g. other learning algorithms) in various settings, although almost all of this work only pertains to general sum games. It is an interesting open question to understand to what extent we can obtain similar best-of-both-worlds results for adaptive opponents in zero-sum games.

\textbf{Comparator-Adaptive OL with Full Information.} In OL under full information feedback, \citet{hutter2005adaptive,even2008regret,kapralov2011prediction,koolen2013pareto,sani2014exploiting} establish (with various emphases) that safe OLM over the simplex in the sense of \cref{eq:safe-olm} is possible. Using so-called parameter-free methods from the online convex optimization literature instead \citep[e.g.]{orabona2016coin,cutkosky2018black,orabona2019modern}, one can (after a simple shifting argument) achieve similar guarantees in the full information setting. For our purposes, the most notable of the above algorithms is the Phased Aggression template of \citet{even2008regret}, as it is the only one we were able to adapt to the bandit feedback setting while maintaining the rate-optimal regret guarantee. While the application of the above type of algorithms to symmetric zero-sum (normal-form) games is direct (\cref{sec:preliminaries}), we are not aware of any prior work making this connection, even under full-information feedback.

\textbf{Comparator-Adaptive OL with Bandit Feedback.} \citet{lattimore2015pareto} establishes a sharp separation between full information and bandit feedback. The author shows that $O(1)$ regret compared to a single comparator action implies a worst-case regret of $\Omega(AT)$ for some other action. This rules out algorithms that resolve our question even in the simple normal-form case under bandit feedback. The key to this lower bound is that the algorithm has to play the special comparator essentially every time, thereby not exploring any other options (as the comparator strategy is deterministic) and thus not knowing whether it is safe to switch the arm. The minimal assumption we can make on the comparator strategy is thus that it plays every action with a non-zero probability. In addition to the mentioned works from the online convex optimization literature, \citet{van2020comparator} remarkably analyzes bandit convex optimization algorithms that adapt to the comparator. However, unlike in the full information case, it is not possible to turn them into an algorithm for safe OLM (as the shifting argument one can use for full-information parameter-free methods like \citet{orabona2016coin,cutkosky2018black,orabona2019modern} does no longer work under bandit feedback).

\noindent\textbf{Relation to Safe Reinforcement Learning.} A closely related line of work is that of \emph{conservative bandits} \citep{wu2016conservative} and \emph{conservative RL} \citep{garcelon2020conservative}. In conservative exploration, algorithms are designed to obtain at least a $(1-\alpha)$-fraction of the return of a comparator, which in our motivating example, however, means that the algorithm may suffer a linear loss $\alpha T$ in the worst case. We thus believe that independently of our motivation from a game-theoretic viewpoint, our results nicely complement existing OL literature. In constrained (or safe) reinforcement learning \citep{badanidiyuru2018bandits,efroni2020exploration}, both the regret and the cumulative violation of a constraint are considered. However, even in the stochastic case the goal of constant regret compared to some known strategy can only be realized if there exists a strategy with a strictly larger return \citep{liu2021learning} for the environment, and in the adversarial case even this reduction fails.

\textbf{Online Learning in (Extensive-Form) Games.} While online learning (OL) in NFGs can readily be reduced to the problem of learning from experts \citep{cesa2006prediction} (full information) or multi-armed bandits \citep{lattimore2020bandit}, it becomes more difficult in the case of EFGs \citep{osborne1994course} due to the presence of (imperfectly observed) states and transitions. State-of-the-art algorithms for no-regret learning in EFGs are based on online mirror descent (OMD) over the treeplex, which leads to near-optimal regret bounds in the full information setting \citep{farina2021bandit,fan2024optimality} and the bandit setting \citep{farina2021bandit,kozuno2021model,bai2022near,fiegel2023adapting}. Alternative approaches are based on counterfactual regret minimization \citep{zinkevich2007regret,lanctot2009monte}, which however do not guarantee a bound on the actual regret (see \citet[Theorem 7]{bai2022near}).

\section{Deferred Proofs for Normal-Form Games} \label{app:nfg}

\subsection{Upper Bound} \label{app:nfg-upper}

First, note that our cost estimates are unbiased, i.e. 
$\exptn\rectangular{\cHat^t(a)} = \exptn\rectangular{ c^t(a) }$, and $\exptn\rectangular{ \inner{\cHat^t}{\mu^t} } = \exptn\rectangular{\exptn\rectangular{ \inner{\cHat^t}{\mu^t} \mid \Fcal_{t-1} }} = \exptn\rectangular{ \inner{c^t}{\mu^t} } = \exptn\rectangular{c^t(a^t) }$, where $\Fcal_{t-1}$ is the $\sigma$-algebra induced by all random variables prior to sampling $a^t$. Further, WLOG we assume that the cost functions are bounded via $c^t\in[0,1]^A$. The reduction from NFGs with matrix entries $U_{a,b}\in[-1,1]$ is then simply via $c^t(a):=(1+U_{a,b^t})/2$, where the shifting and scaling does not change the regret bound. By convention $\start_{k+1}:=T+1$ if $k$ is the last phase.

\thmUbSimplexBandit*

\begin{proof}
    \emph{Case 1: $\alpha=1$ is not reached.} Suppose first the algorithm ends in phase $k <1+\log_2(R)$ at time step $T$. By \cref{lemma:during-nfg}, w.r.t. any comparator 
    \begin{align*}
        \sum_{t=1}^T \inner{\cHat^t}{\mu^t-\mu} \leq (2R+2) \cdot k \leq O(R \log(R)).
    \end{align*}
    All previous phases must have been exited, so by \cref{lemma:during-nfg,lemma:exit-nfg} we have 
    \begin{align*}
        \sum_{t=1}^T \inner{\cHat^t}{\mu^t-\mu^c} \leq 2^{k-1} - \sum_{i=1}^{k-1} 2^{i-1} = 2^{k-1} - (2^{k-1}-1) = 1.
    \end{align*}
    Taking expectation yields the claim.\\

    \emph{Case 2: $\alpha=1$ is reached.} Next, suppose the phase $\alpha^k=1$ was reached and simply Exp3 was run in the final phase $k$. As before
    \begin{align*}
        \sum_{t=1}^{\start_{k}-1} \inner{\cHat^t}{\mu^t-\mu} \leq (2R+2) \cdot k \leq O(R \log(R)).
    \end{align*}
    For the final phase, note that \cref{algo:phased-aggression-nfg} plays Exp3 for $\leq T$ rounds, with uniform initialization. By the standard Exp3 analysis \citep[Sec. 10.1]{orabona2019modern}, this phase has expected regret
    \begin{align}
        \exptn\rectangular{\sum_{t=\start_{k+1}}^T \inner{c^t}{\mu^t-\mu}} \leq \frac{\log(A)}{\tau} + \frac{\tau}{2} AT 
        \leq \sqrt{AT\log(A)/2}
        \leq \delta^{-1}\sqrt{2\log(A) T}=R . \label{eq:exp3}
    \end{align}
    since $\tau=\sqrt{\frac{2\log(A)}{AT}}$ and $\delta\leq 1/A$. Thus for any comparator $\mu\in\Delta_A$ we have
    \begin{align*}
        \exptn\rectangular{\sum_{t=1}^T \inner{c^t}{\mu^t - \mu}} \leq O(R\log(R)).
    \end{align*}
    Finally, for the special comparator note that all phases $k'<k$ have been left and thus by \cref{lemma:exit-nfg,eq:exp3} 
    \begin{align*}
        \exptn\rectangular{\sum_{t=1}^T \inner{c^t}{\mu^t - \mu^c}} \leq R - \sum_{k'=1}^{k-1} 2^{k'-1}= R - (2^{k-1} -1 ) \leq 1,
    \end{align*}
    where the last step used that $\alpha^k=\min\{1,2^{k-1}/R\}=1$ and thus $R\leq2^{k-1}$.
\end{proof}

Recall that 
\begin{align}
    \RHat^k(\mu):=& \sum_{t=\start_k}^{\start_{k+1}-1} \inner{\cHat^t}{\mu^t-\mu} = \alpha^k \sum_{j=\start_k}^{\start_{k+1}-1} \inner{\cHat^t}{\muHat^t-\mu} + (1-\alpha^k) \sum_{t=\start_{k}}^{\start_{k+1}-1} \inner{\cHat^t}{\mu^c-\mu} \label{eq:phase-reg-nfg}
\end{align}
measures Alice's estimated regret.

\lemmaDuringNfg*

\begin{proof}
    WLOG suppose that $R=2^r$ is a power of $2$, else we can run the algorithm for $T$ such that $R$ is the next largest power of two and pay a constant factor in the regret. For the first term in \cref{eq:phase-reg-nfg}, we analyze OMD to bound $\sum_{t=\start_{k}}^{\start_{k+1}-1} \inner{\cHat^t}{\muHat^t-\mu}$ almost surely, making use of the fact that $\cHat^t$ is bounded. Indeed, recall 
    \begin{align*}
        \cHat^t(a) = \frac{c^t(a^t)}{\mu^t(a)}\indicator{a^t=a} \leq \frac{1}{\mu^t(a)}.
    \end{align*}
    We have $\alpha^k = 2^{k-1}/R \leq 2^{\log_2(R)-1}/R=1/2$, so 
    \begin{align*}
        \cHat^t(a) \leq \frac{1}{\mu^t(a)}=\frac{1}{\alpha^k\mu^t(a)+(1-\alpha^k)\mu^c(a)}\leq\frac{1}{\frac{1}{2}\mu^c(a)}\leq \frac{2}{\delta}.
    \end{align*}
    Moreover, $\cHat^t$ is zero outside the visited $a^t$. Thus, by \cref{lem:bandit-omd-bounded-nfg}, almost surely for the first term in \cref{eq:phase-reg-nfg}
    \begin{align}
        \sum_{t=\start_{k}}^{\start_{k+1}-1} \inner{\cHat^t}{\muHat^t-\mu} \leq \frac{\log(A)}{\eta} + 2\eta T\delta^{-2} \leq \delta^{-1}\sqrt{2T\log(A)} = R. \label{eq:omd-regret1-nfg}
    \end{align}

    \noindent For the second term in \cref{eq:phase-reg-nfg}, note that since the if condition may only hold at $t':=\start_{k+1}-1$,
    \begin{align}
        \sum_{t=\start_{k}}^{\start_{k+1}-1} \inner{\cHat^t}{\mu^c-\mu} \leq 2R + \frac{c^{t'}(a^{t'})}{\frac{1}{2}\mu^c(a^{t'})}\mu^c(a^{t'})\leq 2R+2. \label{eq:while-cond1-nfg}
    \end{align}
    \noindent Linearly combining \cref{eq:omd-regret1-nfg,eq:while-cond1-nfg}, 
    \begin{align*}
        \RHat^k(\mu):=& \alpha^k \sum_{t=\start_{k}}^{\start_{k+1}-1} \inner{\cHat^t}{\muHat^t-\mu} + (1-\alpha^k) \sum_{t=\start_k}^{\start_{k+1}-1} \inner{\cHat^t}{\mu^c-\mu} \leq 2R +2
    \end{align*}
    for any $\mu$. For the special comparator, by \cref{eq:omd-regret1-nfg}
    \begin{align*}
        R^k(\mu^c)= \alpha^k \sum_{t=\start_{k}}^{\start_{k+1}-1} \inner{\cHat^t}{\muHat^t-\mu^c} + (1-\alpha^k) \sum_{t=\start_{k}}^{\start_{k+1}-1} \inner{\cHat^t}{\mu^c-\mu^c} \leq (2^{k-1}/R)R = 2^{k-1}. 
    \end{align*}
\end{proof}

\begin{lemma}[OMD with bounded surrogate costs] \label{lem:bandit-omd-bounded-nfg}
    Let $\eta >0$, and $L>0$. Let $(\cHat^t)_t$ be cost functions such that for all $t$, $0\leq\cHat^t(a)\leq L$ (for all $a$), and moreover $\cHat^t(a) = 0$ if $a\neq a^t$ for some arbitrary $a^t$. Set $\muHat^1(a) = 1/A$ and consider the scheme $\mu^{t+1} = \arg\min_{\mu \in \Delta_A} \inner{\mu}{\cHat^t} + \frac{1}{\eta} D(\mu || \muHat^t)$ for $t\leq T'$. Then we have for all $\mu\in\Delta_A$
    \begin{align*}
        \sum_{t=1}^{T'} \inner{\muHat^t-\mu}{\cHat^t} \leq \frac{\log(A)}{\eta} + \frac{\eta}{2} L^2 T'.
    \end{align*}
\end{lemma}

\begin{proof}
    From \citet[Sec. 10.1]{orabona2019modern}, we find that a.s.
    \begin{align*}
        \sum_{t=1}^{T'} \inner{\muHat^t-\mu}{\cHat^t} \leq \frac{\log(A)}{\eta} + \frac{\eta}{2} \sum_{t=1}^{T'} \sum_a \muHat^t(a) (\cHat^t(a))^2
        \leq \frac{\log(A)}{\eta} + \frac{\eta}{2} \sum_{t=1}^{T'} \muHat^t(a^t) L^2 \leq \frac{\log(A)}{\eta} + \frac{\eta}{2}L^2T'.
    \end{align*}
\end{proof}

\lemmaExitNfg*

\begin{proof}
    At $t=\start_{k+1}-1$ the if condition implies $\max_{\mu\in\Delta_A}\sum_{j = \start_{k}}^{\start_{k+1}-1} \inner{\cHat^j}{\mu^c - \mu} > 2 R$, so when we let $\mu^{\star}$ be a maximizer, we find
    \begin{align*}
        \sum_{t=\start_{k}}^{\start_{k+1}-1} \inner{\cHat^t}{\mu^t-\mu^c} =& \alpha^k \sum_{t=\start_{k}}^{\start_{k+1}-1} \inner{\cHat^t}{\muHat^t-\mu^c} \nonumber\\
        =& \alpha^k \sum_{t=\start_{k}}^{\start_{k+1}-1} \inner{\cHat^t}{\muHat^t-\mu^{\star}} +\alpha^k \sum_{t=\start_{k}}^{\start_{k+1}-1} \inner{\cHat^t}{\mu^{\star}-\mu^c} \nonumber\\
        \leq& \alpha^k R + \alpha^k(-2R)\nonumber\\
        =& -2^{k-1},
    \end{align*}
    where we used \cref{eq:omd-regret1-nfg} in the last inequality.
\end{proof}

\subsection{Lower Bound} \label{app:nfg-lower}

\thmLowerNfg*

Our lower bound becomes vacuous in the regime where $\delta \leq O( T^{-1})$, which is when a direct application of \citet{lattimore2015pareto} shows a trivial $\Omega(T)$ lower bound.

\begin{proof}
    It is sufficient to prove the lower bound for $A=2$ actions as we can assign the same distribution to all but one action. We prove a lower bound for stochastic cost functions, which immediately implies the same bound for adversarially chosen costs. Consider the following setup with two different environments. The first action deterministically gives cost $c_1 = 1/2$ in both environments. In the \textit{first environment $(-)$}, action two samples costs according to a $\text{Ber}(\frac{1}{2}-\gamma T^{-1/2})$ distribution with expected cost $c_- = \frac{1}{2}-\gamma T^{-1/2}$. We will choose $\gamma>0$ later and for now, only require $\gamma < \frac{1}{2}T^{1/2}$ in order for the sampling to be well-defined. Symmetrically, in the \textit{second environment $(+)$}, action two samples costs according to a $\text{Ber}(\frac{1}{2}+\gamma T^{-1/2})$ distribution with expected cost $c_+ = \frac{1}{2}+\gamma T^{-1/2}$. We consider the case that the \textit{special comparator} is $\mu^c = (1-\delta,\delta)\in\Delta_2$. In the following, $\mathcal{R}(\mu)$ denotes the regret compared to $\mu\in\Delta_A$ in the worst case environment. We fix an arbitrary algorithm and index regret and expectation with $+$ or $-$ to indicate which probability space (environment) we are referring to.\\

    \noindent Now let $N_2$ be the number of times action two is chosen during the $T$ interactions. The requirement on the regret w.r.t. the special comparator (together with the standard regret decomposition \citep[Sec. 4.5]{lattimore2020bandit}) shows
    \begin{align*}
        1 \geq \mathcal{R}(\mu^c) \geq \mathcal{R}_+(\mu^c) = \exptn_+\rectangular{N_2} (+\gamma T^{-1/2}) - \delta T (+\gamma T^{-1/2}),
    \end{align*}
    and thus
    \begin{align*}
        \exptn_+\rectangular{N_2} \leq& \gamma^{-1}T^{1/2} + \delta T. 
    \end{align*}
    Plugging this into \cref{lemma:pinsker-nfg}, we have (if $\gamma < \frac{1}{2\sqrt{2}}T^{1/2}$)
    \begin{align*}
        \exptn_-[N_2] \leq& \exptn_+[N_2] + T \sqrt{2}\sqrt{\exptn_+[N_2]} \gamma T^{-1/2}\nonumber\\
        \leq& (\gamma^{-1}T^{1/2} + \delta T) + T \sqrt{2} \sqrt{\gamma^{-1}T^{1/2} + \delta T} \gamma T^{-1/2}\nonumber\\
        \leq& \gamma^{-1}T^{1/2} + \delta T + T \sqrt{2}(\gamma^{-1/2}T^{1/4} + \delta^{1/2} T^{1/2}) \gamma T^{-1/2}\nonumber\\
        \leq& \gamma^{-1}T^{1/2} + \delta T + \sqrt{2}\gamma^{1/2}T^{3/4} + \sqrt{2}\delta^{1/2} \gamma T 
    \end{align*}
    Using this in the regret decomposition on $(-)$, we see for the second action $\mu=e_2=(0,1)\in\Delta_A$
    \begin{align*}
        \mathcal{R}(e_2) \geq& \mathcal{R}_-(e_2)\\
        \geq& (\gamma^{-1}T^{1/2} + \delta T + \sqrt{2}\gamma^{1/2}T^{3/4} + \sqrt{2}\delta^{1/2} \gamma T) (-\gamma T^{-1/2}) - T(-\gamma T^{-1/2})\\
        \geq& - 1 - \delta \gamma T^{1/2} - \sqrt{2}\gamma^{3/2}T^{1/4} - \sqrt{2}\delta^{1/2} \gamma^2 T^{1/2} + \gamma T^{1/2}\\
        =& \round{(1-\delta)\gamma - \sqrt{2}\delta^{1/2}\gamma^2} T^{1/2} - \sqrt{2}\gamma^{3/2}T^{1/4}-1.
    \end{align*}
    We can now choose $\gamma = c \delta^{-1/2}$ for a sufficiently small absolute constant $c$ to show that 
    \begin{align}
        \mathcal{R}(e_2) \geq \Theta\round{ \delta^{-1/2} T^{1/2}} - \Theta(\delta^{-3/4}T^{1/4}). \label{eq:lower}
    \end{align}
    This bound holds when $\gamma < \frac{1}{2\sqrt{2}} T^{1/2}$, i.e. $\delta \geq c' T^{-1}$ for some large enough absolute constant $c'$.
\end{proof}

\begin{lemma}[Entropy inequality Bernoulli] \label{lemma:pinsker-nfg}
    In the setup of \cref{thm:lower-nfg}, we have
    \begin{align*}
        \exptn_{-}\rectangular{N_2}
        \leq& \exptn_{+}\rectangular{N_2} + T\sqrt{\frac{2(\gamma T^{-1/2})^2}{\frac{1}{4}-(\gamma T^{-1/2})^2}\exptn_{+}\rectangular{N_2}}.
    \end{align*}
    In particular for $\gamma < \frac{1}{2\sqrt{2}} T^{1/2}$, we have $\exptn_{-}\rectangular{N_2}
        \leq \exptn_{+}\rectangular{N_2} + \sqrt{2\exptn_{+}\rectangular{N_2}}\gamma T^{-1/2}.$
\end{lemma}

\begin{proof}
    Via Pinsker's and the chain rule for the KL divergence (c.f. \citet{auer1995gambling} and \citet[Appendix]{lattimore2015pareto})
    \begin{align*}
        \exptn_{-}\rectangular{N_2} - \exptn_{+}\rectangular{N_2} \leq T \sqrt{\frac{1}{2} \exptn_{+}\rectangular{N_2} \cdot\text{KL}(X||Y)},
    \end{align*}
    where $X\sim \text{Ber}(\frac{1}{2}+\epsilon)$ and $Y\sim \text{Ber}(\frac{1}{2}-\epsilon)$ for $\epsilon=\gamma T^{-1/2}$. We conclude by computing 
    \begin{align*}
        \text{KL}(X||Y) =& \round{\frac{1}{2}+\epsilon}\log\round{\frac{\frac{1}{2}+\epsilon}{\frac{1}{2}-\epsilon}} + \round{\frac{1}{2}-\epsilon}\log\round{\frac{\frac{1}{2}-\epsilon}{\frac{1}{2}+\epsilon}}\\
        \leq& \round{\frac{1}{2}+\epsilon}\round{\frac{\frac{1}{2}+\epsilon}{\frac{1}{2}-\epsilon}-1} + \round{\frac{1}{2}-\epsilon}\round{\frac{\frac{1}{2}-\epsilon}{\frac{1}{2}+\epsilon}-1}\\
        =& 2\epsilon \round{- \frac{\frac{1}{2}-\epsilon}{\frac{1}{2}+\epsilon}+\frac{\frac{1}{2}+\epsilon}{\frac{1}{2}-\epsilon}}\\
        =& 2\epsilon \frac{2\epsilon}{\frac{1}{4}-\epsilon^2}.
    \end{align*}
\end{proof}


\subsection{The Stochastic Case} \label{app:nfg-stochastic}



As claimed in the main part, we now sketch how the $\OTilde(\sqrt{\delta^{-1}T})$ lower bound from \cref{sec:nfg-lower} can be matched (up to logarithmic terms) if the costs are stochastic and not adversarial. This improves slightly over our result for the adversarial case.

\begin{theorem}\label{thm:stochastic-case}
    Let $\delta\in(0,1)$ and consider Protocol \ref{prot:bandit-simplex} but where all $c^t(a)\sim q_a$ are i.i.d. for some fixed distributions $q_a$ with support in $[0,1]$. Then there is an algorithm such that for any specified $\mu^c\in\Delta_A$ with all $\mu^c(a)\geq \delta$ and all distributions $q$, we have
    \begin{align*}
        \mathcal{R}(\mu^c) \leq 1 \quad \text{and}\quad & \max_{\mu\in\Delta_A} \mathcal{R}(\mu) \leq O\round{\sqrt{\delta^{-1} T\log(AT)} + \delta^{-2}\log(T)}.
    \end{align*}
\end{theorem}
For $a \in [A]$, let the $a$-th action's reward distribution $q_a$ have mean $1-\mean_a$, and write $\mean$ for the corresponding vector. We thus consider \emph{maximization} of the \emph{rewards} $1-c^t$ that have means $m$. This is just for convenience to better highlight the relation of our algorithm to the classic UCB algorithm \citep{lattimore2020bandit}. As for the rewards, we index the entries of a strategy $\mu\in\Delta_A$ as $\mu_a=\mu(a)$. Fix an arbitrary $a^\star \in \arg\max_a \mean_a$. The (random) pseudo-regret of the algorithm is 
\begin{align*}
    \tilde{\mathcal{R}} := \sum_{t=1}^T (\mean_{a^\star} - \inner{\mu^t}{\mean}).
\end{align*}

\paragraph{Algorithm.}

Construct $\meanLow^{t} = (\meanLow_1^{t}, \dots, \meanLow_A^{t})$, $\meanUp^{t} = (\meanUp_1^{t}, \dots, \meanUp_A^{t})$ to be the vectors of lower and upper confidence bounds for the actions after playing and observing $t$ rounds. Formally,
\begin{align*}
    \meanLow_a^{t} := \meanHat_a^t - b_a^t, \quad \quad \meanUp_a^{t} := \meanHat_a^t + b_a^t,
\end{align*}
where $\meanHat_a^t$ is the average reward among the rounds in which the $a$-th action is chosen during rounds $1, \dots, t$ (and zero if not defined), and $b^t_a$ is a confidence half-width to be specified. With this, set $M^t := [\meanLow^t,\meanUp^t] := [\meanLow^t_1,\meanUp^t_1] \times \cdots \times [\meanLow^t_A,\meanUp^t_A]$. Consider the following update. Let $$\mu^1 = \mu^c,$$ and in round $t + 1$, update 
\begin{align}
    \mu^{t+1} = \arg\max_{\mu \in \Delta_A} \min_{\meanTilde \in [\meanLow^t,\meanUp^t] } \inner{\mu-\mu^t}{\meanTilde} . \label{eq:update}
\end{align}

\paragraph{Regret analysis.}

First, note that conditioned on $\mean \in M^t$, we have 
\begin{align}
    0 = \min_{\meanTilde \in M^t} \inner{\mu^t - \mu^t}{\meanTilde} \leq \max_{\mu\in\Delta_A} \min_{\meanTilde \in M^t} \inner{\mu - \mu^t}{\meanTilde} = \min_{\meanTilde \in M^t} \inner{\mu^{t+1} - \mu^t}{\meanTilde} \leq \inner{\mu^{t+1} - \mu^t}{\mean}. \label{rmk:mon}
\end{align}
Hence, the algorithm monotonically improves, i.e. $\inner{\mu^{t+1}}{m}\geq \inner{\mu^{t}}{m}$, if all confidence intervals include the true mean. As for the confidence intervals, set $b_a^t := 2\sqrt{\frac{2\log(T^2A/\zeta)}{n^t_a}}$, where $n^t_a$ is the number of times that action $a$ is chosen in rounds $1,\dots,t$. Then by Hoeffding's inequality, with probability at least $1-\zeta$, for all $t \in [T]$ we have $\mean \in \text{int}(M^t)$. We call this event $G$.\smallskip

\noindent By finding the closed form of the update rule in \cref{eq:update} and the lower bound on $\mu^1=\mu^c$, it is not hard to see the following.
\begin{lemma} \label{lem:opt-prob}
    Conditioned on $G$, we have $\mu_{a^\star}^{t} \geq \mu^c_{a^\star} \geq \delta$ for all $t\in[T]$.
\end{lemma}
Using Hoeffding's inequality and a union bound, we thus get the following concentration.
\begin{lemma} \label{lem:opt-count}
    Condition on $G$ and let $\zeta' \in (0,1)$. Then with probability at least $1-\zeta'$, we have 
    \begin{align*}
        n_{a^\star}^t \geq \delta t - \sqrt{ 2t \log(T/\zeta') }.
    \end{align*}
\end{lemma}

\smallskip
\noindent We are now ready to prove \cref{thm:stochastic-case}. Condition on $G$ and on the event in \cref{lem:opt-count}. This occurs with probability at least $1-\zeta-\zeta'$.\smallskip

\noindent First, we consider the regret compared to $\mu^c$. By the monotonicity property in \cref{rmk:mon}, 
\begin{align*}
    \inner{\mu^t}{\mean} \geq \inner{\mu^{t-1}}{\mean} \geq \dots \geq \inner{\mu^1}{\mean} = \inner{\mu^c}{\mean}.
\end{align*}
Setting $\zeta = \zeta' = \frac{1}{2T}$ and integrating out the regret of at most $T$ under the failure event:
\begin{align*}
    \exptn\rectangular{\sum_{t=1}^T \inner{\mu^c-\mu^t}{\mean}} \leq& \Pr[G] \exptn\rectangular{\sum_{t} \inner{\mu^c-\mu^t}{\mean} \mid G} + \Pr[\bar{G}] T 
    \leq 0 + (\zeta + \zeta')T = 1.
\end{align*}

\noindent  We now consider the worst case (pseudo-) regret $\tilde{\mathcal{R}}$. Note that for the minimax problem in \cref{eq:update}, strong duality holds and we can fix a saddle point $(\mu^t, \meanTilde^t)$ such that (for all $(\mu,\meanTilde) \in \Delta_A\times M^t$)
\begin{align}
    \inner{\mu-\mu^{t-1}}{\meanTilde^t} \leq \inner{\mu^t-\mu^{t-1}}{\meanTilde^t} \leq \inner{\mu^t-\mu^{t-1}}{\meanTilde}. \label{eq:saddle-point}
\end{align}

\noindent Under the success events, we have $n_{a^\star}^t \geq \delta t - \sqrt{ 2t \log(T/\zeta') }$ by \cref{lem:opt-count}. Now when $t \geq t_0 := 8\delta^{-2} \log(T/\zeta')$, then $n_{a^\star}^t \geq 2\delta^{-1}\sqrt{ 2t \log(T/\zeta') }$ and hence 
\begin{align}
    b_{a^\star}^t \leq 2\sqrt{\frac{4\log(T^2A/\zeta)}{\delta t}}. \label{eq:bonus}
\end{align}
We have 
\begin{align*}
    \tilde{\mathcal{R}} =  8\delta^{-2} \log(T/\zeta') + \sum_{t=t_0}^T (\mean_{a^\star} - \inner{\mu^{t}}{\mean}) \leq 8\delta^{-2} \log(T/\zeta') + \sum_{t=t_0}^T (\mean_{a^\star} - \inner{\mu^{t}}{\mean}),
    \end{align*}
where the instantaneous regret for $t\geq t_0$ is (with $\mu^\star:=e_{a^\star}$)
\begin{align*}
    \mean_{a^\star} - \inner{\mu^{t}}{\mean} =& \inner{\mu^\star - \mu^{t}}{\mean}\\
    =& \inner{\mu^\star - \mu^{t}}{\meanTilde^{t+1}} + \inner{\mu^\star - \mu^{t}}{\mean - \meanTilde^{t+1}} \\
    \leq& \inner{\mu^{t+1} - \mu^{t}}{\mean} + \inner{\mu^\star - \mu^{t}}{\mean - \meanTilde^{t+1}} \tag{by \cref{eq:saddle-point} and $\mean\in M^{t+1}$}\\
    \leq& \inner{\mu^{t+1} - \mu^{t}}{\mean} + \inner{\mu^\star}{b^{t+1}} + \inner{\mu^t}{b^{t+1}}\\
    \leq& \inner{\mu^{t+1} - \mu^{t}}{\mean} + \inner{\mu^\star}{b^{t}} + \inner{\mu^t}{b^{t}} \tag{as $b^{t+1}\leq b^t$}.
\end{align*}
Hence,
\begin{align*}
    \tilde{\mathcal{R}} \leq& 8\delta^{-2} \log(T/\zeta') + \inner{\mu^{T+1} - \mu^{t_0}}{\mean} + \sum_{t=t_0}^T \round{\inner{\mu^\star}{b^{t}} + \inner{\mu^t}{b^{t}}}\\
    \leq& 8\delta^{-2} \log(T/\zeta') + 1 + \sum_{t=t_0}^T b_{a^\star}^{t} + \sum_{t=t_0}^T \inner{\mu^t}{b^{t}}\\
    \leq& 8\delta^{-2} \log(T/\zeta') + 1 + \sum_{t=t_0}^T 2\sqrt{\frac{4\log(T^2A/\zeta)}{\delta t}} + \sum_{t=t_0}^T \inner{\mu^t}{b^{t}} \tag{by \cref{eq:bonus}}\\
    \leq& O\round{\delta^{-2} \log(T/\zeta') + \sqrt{\frac{T\log(AT/\zeta)}{\delta}}} + \sum_{t=t_0}^T \inner{\mu^t}{b^{t}},
\end{align*}
with probability at least $1-\zeta-\zeta'$. Using $\sum_{t=t_0}^T \exptn[\inner{\mu^t}{b^{t}}] \leq O( \sqrt{AT\log(AT/\zeta)} )$ \citep{auer2008near}, $\delta\leq1/A$ and integrating out the regret under the failure event yields the result.

\section{Deferred Proofs for Extensive-Form Games} \label{app:efg}

\subsection{EFG Background} \label{app:efg-background}

The following remark clarifies that the Markov game we defined in \cref{sec:efg} (which is more common in the machine learning literature) indeed covers the case of imperfect information EFGs (which are more common in the game theory literature).
\begin{remark}
    The notion of EFG in \cref{d:efg} from \cref{sec:efg} is usually referred to as a \emph{\underline{t}ree-structured \underline{p}erfect-recall \underline{p}artially-\underline{o}bservable \underline{M}arkov \underline{g}ame} (TP-POMG). This also covers the notion of \emph{\underline{p}erfect-recall \underline{i}mperfect \underline{i}nformation \underline{e}xtensive-\underline{f}orm \underline{g}ames} (P-IIEFG) \citep{osborne1994course} that satisfy the \emph{timeability condition} \citet{jakobsen2016timeability}. In fact, a more careful look reveals that the results directly generalize to \emph{any} P-IIEFG without timeability (see \citet{bai2022near} for this brief discussion).
\end{remark}

For further clarification, we remark that usually, both the cost function $u$, the transition probabilities $p$ and the policies $\pi$ (and treeplex strategies $\mu$) may be non-stationary in the sense that they explicitly vary across the stages $h\in[H]$ of the EFG. However, as we assume tree structure and perfect recall, the state space and infoset space are partitioned along the stages anyway, which is why WLOG we omit the explicit dependence of the above functions on the stage $h$. Finally, to be precise, our algorithm assumes to know the tree structure of the game (but not necessarily the transitions), an assumption that can be removed \citep{fiegel2023adapting}.

\subsection{OMD over the Treeplex} \label{app:efg-kl}

The unbalanced and balanced dilated KL divergence are defined as follows:
\begin{align*}
    D(\mu || \mu') :=& \sum_{\substack{x\in\Xcal,\\a\in\Acal}} \mu(x,a) \log\round{\frac{\pi_{\mu}(a|x)}{\pi_{\mu'}(a|x)}},\\ \quad \Dbal(\mu || \mu') :=& \sum_{h=1}^H\sum_{\substack{x_h \in \Xcal_h,\\ a\in \Acal}} \frac{\mu(x_h,a)}{\mu^{h,\text{bal}}(x_h,a)} \log\round{\frac{\pi_{\mu}(a|x_h)}{\pi_{\mu'}(a|x_h)}},
\end{align*}
where $\pi_{\mu}$ is the policy corresponding to the treeplex strategy $\mu$, and $\mu^{h,\text{bal}}$ is the unique strategy corresponding to the \emph{balanced exploration policy}
\begin{align*}
        \pi^{h,\text{bal}}(a | x_{h'}) := \begin{cases}
        \frac{|\Ccal_h(x_{h'},a)|}{|\Ccal_h (x_{h'})|}&\quad (h'\in\{1,\dots,h-1\}),\\
        \frac{1}{A} &\quad (h'\in\{h,\dots,H\}),
    \end{cases}
\end{align*}
with $\Ccal_h(x_{h'},a) \subset \Xcal_h$ being set of infosets at step $h$ reachable from $(x_{h'},a)$ (i.e. the unique path to such an infoset goes through $(x_{h'},a)$), and $|\Ccal_h (x_{h'})| := \bigcup_{a\in \Acal} \Ccal_h(x_{h'},a)$.

\paragraph{Computation of Unbalanced OMD.}

\noindent For completeness, we restate the closed-form implementation of case one in \cref{line:omd} with the \emph{unbalanced} dilated divergence $D$ from \citet[Appendix B]{kozuno2021model}. In the setup of \cref{line:omd}, let $\piHat^t\in\Pi$ be the policy corresponding to $\muHat^t$. Then we have a closed-form
\begin{align*}
    \piHat^{t+1}(a_h|x_h^t) =& \piHat^t(a_h|x_h^t) \exp\round{\indicator{a_h^t=a_h}(-\eta\cHat^t(x_h^t,a_h)+\log(Z^t_{h+1}))-\log(Z^t_{h})},
\end{align*}
and $\piHat^{t+1}(\cdot|x_h)=\piHat^{t}(\cdot|x_h)$ for all other $x_h\neq x_h^t$. Here, $Z_h^t$ is
\begin{align*}
    Z_h^t := 1-\piHat^t(a_h^t|x_h^t)+\piHat^t(a_h^t|x_h^t)\exp\round{-\eta\cHat^t(x_h^t,a_h^t)+\log(Z^t_{h+1})},
\end{align*}
and $Z_{H+1}^t := 1$.

\paragraph{Computation of Balanced OMD.}

\noindent For completeness, we also restate the closed-form implementation of case two in \cref{line:omd} with the \emph{balanced} dilated divergence $\Dbal$ from \citet[Algorithm 5]{bai2022near}. Once more, let $\piHat^t\in\Pi$ be the policy corresponding to $\muHat^t$. Then we have a closed form for the next iterate, namely
\begin{align*}
    &\piHat^{t+1}(a_h|x_h^t)\\
    &=\piHat^t(a_h|x_h^t) \exp\round{ \indicator{a_h=a_h^t} \round{-\tau \mu^{\text{bal},h}(x_h^t,a_h^t) \cHat^t(x_h^t,a_h^t) + \frac{\mu^{\text{bal},h}(x_h^t,a_h^t) \log(Z_{h+1}^t)}{\mu^{\text{bal},h+1}(x_{h+1}^t,a_{h+1}^t)}} - \log(Z_h^t) },
\end{align*}
and in the other infosets $\piHat^{t+1}(a_h|x_h) = \piHat^{t}(a_h|x_h)$. Here,
\begin{align*}
    Z_h^t := 1-\piHat(a_h^t|x_h^t) + \piHat(a_h^t|x_h^t) \exp\round{ -\tau \mu^{\text{bal},h}(x_h^t,a_h^t) \cHat^t(x_h^t,a_h^t) + \frac{\mu^{\text{bal},h}(x_h^t,a_h^t) \log(Z_{h+1}^t)}{\mu^{\text{bal},h+1}(x_{h+1}^t,a_{h+1}^t)}},
\end{align*}
and $Z_{H+1}^t=1$.


\subsection{Upper Bound} \label{app:efg-upper}

First, note that due to the importance-weighting by the rollout policies the cost estimators are unbiased \citep{kozuno2021model}: $\exptn\rectangular{\cHat^t(x,a)} = \exptn\rectangular{ c^t(x,a) }$, and $\exptn\rectangular{ \inner{\cHat^t}{\mu^t} } = \exptn\rectangular{\exptn\rectangular{ \inner{\cHat^t}{\mu^t} \mid \Fcal_{t-1} }} = \exptn\rectangular{ \inner{c^t}{\mu^t} }$, where $\Fcal_{t-1}$ is the $\sigma$-algebra induced by all random variables prior to sampling the trajectory $(s_1^t,a_1^t,b_1^t,u_1^t\dots,s_H^t,a_H^t,b_H^t,u_H^t)$. Further, WLOG we assume that the costs $u(s,a,b)$ used to define the cost function $c^t$ in \cref{eq:def-loss} are bounded in $[0,1]$. While in EFGs we assumed $u(s,a,b)\in[-1,1]$, we can simply replace them by $(1+u(s,a,b))/2$ without changing the regret bound. With this, we can prove the desired upper bound by resorting to the estimated regret
\begin{align}
    \RHat^k(\mu):=& \sum_{t=\start_k}^{\start_{k+1}-1} \inner{\cHat^t}{\mu^t-\mu} = \alpha^k \sum_{j=\start_k}^{\start_{k+1}-1} \inner{\cHat^t}{\muHat^t-\mu} + (1-\alpha^k) \sum_{t=\start_{k}}^{\start_{k+1}-1} \inner{\cHat^t}{\mu^c-\mu}. \label{eq:phase-reg-efg}
\end{align}
By convention $\start_{k+1}:=T+1$ if $k$ is the last phase. 

\thmUbInteriorEfg*

\begin{proof}
    \emph{Case 1: $\alpha=1$ is not reached.} Suppose first the algorithm ends in phase $k$ with $\alpha^k<1$ at time step $T$. By \cref{lemma:during-efg}, w.r.t. any comparator 
    \begin{align*}
        \sum_{t=1}^T \inner{\cHat^t}{\mu^t-\mu} \leq (2R+2H) \cdot k \leq O(R \log(R)).
    \end{align*}
    All previous phases must have been exited, so by \cref{lemma:during-efg,lemma:exit-efg} we have 
    \begin{align*}
        \sum_{t=1}^T \inner{\cHat^t}{\mu^t-\mu^c} \leq 2^{k-1} - \sum_{i=1}^{k-1} 2^{i-1} = 2^{k-1} - (2^{k-1}-1) = 1.
    \end{align*}
    Taking expectation yields the claim.\\

    \noindent\emph{Case 2: $\alpha=1$ is reached.} Next, suppose $\alpha^k=1$ was reached. Then balanced mirror descent was run in the final phase $k$. As before
    \begin{align*}
        \sum_{t=1}^{\start_{k}-1} \inner{\cHat^t}{\mu^t-\mu} \leq (2R+2H) \cdot k \leq O(R \log(R)).
    \end{align*}
    For the final phase, note that the algorithm runs balanced OMD with importance weights and uniform initialization for $\leq T$ rounds. Thus by \cref{lem:regret-to-init}, this phase has expected regret 
    \begin{align}
        \exptn\rectangular{\sum_{t=\start_{k}}^T \inner{c^t}{\mu^t-\mu}} \leq& \tau H^3T + \frac{1}{\tau}\Dbal(\mu || \mu^{\start_{k}}) \nonumber\\
        \leq& \tau H^3T + \frac{XA\log(A)}{\tau} \tag{by \citet[Lemma C.7]{bai2022near}}\nonumber\\
        \leq& \sqrt{XAH^3\log(A)T} \tag{since $\tau=\sqrt{\frac{XA\log(A)}{2H^3T}}$}\nonumber\\
        \leq& R, \label{eq:exp3-efg}
    \end{align}
    using $\delta\leq1/A$. Thus for any comparator, we have 
    \begin{align*}
        \exptn\rectangular{\sum_{t=1}^T \inner{c^t}{\mu^t - \mu}} \leq O(R\log(R)) + R = O(R\log(R)).
    \end{align*}
    Finally, for the special comparator, we note that all phases $k'$ with $\alpha^{k'}<1$ have been left and thus by \cref{lemma:exit-efg,eq:exp3-efg}
    \begin{align*}
        \exptn\rectangular{\sum_{t=1}^T \inner{c^t}{\mu^t - \mu^c}} \leq R - \sum_{k'=1}^{k-1} 2^{k'-1}= R - (2^{k-1}-1) \leq 1,
    \end{align*}
    where the last step used that $\alpha^k=\min\{1,2^{k-1}/R\}=1$ and thus $R\leq2^{k-1}$.
\end{proof}

The following lemma establishes the statement from \cref{lemma:during-nfg}, generalized to EFGs. The second part of the lemma is essentially the same. Once more, the fact that $\mu^c$ is lower bounded comes into play when upper bounding the estimated cost functions.

\begin{restatable}[During normal phases]{lemma}{lemmaDuringEfg} \label{lemma:during-efg}
    Let $k$ be such that $\alpha^k<1$. Then for all $\mu\in\Tcal$, almost surely
    \begin{align*}
        \RHat^k(\mu) \leq 2R+2H=2\delta^{-1}\sqrt{8XH^3\log(A) T}+2H,
    \end{align*}
    and for the special comparator almost surely $\RHat^k(\mu^c) \leq 2^{k-1}$.
\end{restatable}

\begin{proof}
    WLOG suppose that $R=2^r$ is a power of $2$, else we can run the algorithm for $T$ such that $R$ is the next largest power of two and pay a constant factor in the regret. For the first term in \cref{eq:phase-reg-efg}, we analyze unbalanced OMD to bound $\sum_{t=\start_{k}}^{\start_{k+1}-1} \inner{\cHat^t}{\muHat^t-\mu}$ almost surely, making use of the fact that $\cHat^t$ is bounded. Recall 
    \begin{align*}
        \cHat^t(x_{h},a) = \frac{\indicator{(x_{h}^t,a_{h}^t)=(x_{h},a)}u^t_h}{\mu^t(x_{h},a)} \leq \frac{1}{\mu^t(x_{h},a)}.
    \end{align*}
    Now since $R=2^r$ is a power of $2$, we have $\alpha^k = 2^{k-1}/R \leq 2^{\log_2(R)-1}/R=1/2$, so 
    \begin{align*}
        \cHat^t(x_h,a) \leq \frac{1}{\mu^t(x_{h},a)}=\frac{1}{\alpha\mu^t(x_{h},a)+(1-\alpha)\mu^c(x_{h},a)}\leq \frac{1}{\frac{1}{2}\mu^c(x_{h},a)} \leq\frac{2}{\delta}.
    \end{align*}
    Moreover, $\cHat^t$ is zero outside the visited $((x_h^t,a_h^t))_h$. Thus, by \cref{lem:bandit-omd-bounded}, for the first term in \cref{eq:phase-reg-efg} almost surely
    \begin{align}
        \sum_{t=\start_{k}}^{\start_{k+1}-1} \inner{\cHat^t}{\muHat^t-\mu} \leq \frac{X\log(A)}{\eta} + \frac{4 \eta TH(H+1)}{\delta^2} \leq \delta^{-1}\sqrt{8XH^2\log(A) T} \leq R. \label{eq:omd-regret1-efg}
    \end{align}

    \noindent For the second term in \cref{eq:phase-reg-efg}, note that since the if may only hold at $t':=\start_{k+1}-1$,
    \begin{align}
        \sum_{t=\start_{k}}^{\start_{k+1}-1} \inner{\cHat^t}{\mu^c-\mu} \leq 2R+\inner{\cHat^{t'}}{\mu^c}\leq 2R+2H. \label{eq:while-cond1-efg}
    \end{align}

    \noindent Linearly combining \cref{eq:omd-regret1-efg,eq:while-cond1-efg},
    \begin{align*}
        \RHat^k(\mu) = \alpha^k \sum_{t=\start_{k}}^{\start_{k+1}-1} \inner{\cHat^t}{\muHat^t-\mu} + (1-\alpha^k) \sum_{t=\start_k}^{\start_{k+1}-1} \inner{\cHat^t}{\mu^c-\mu} \leq 2R +2H
    \end{align*}
    for any $\mu$, and for the special comparator $\mu^c$ we have by \cref{eq:omd-regret1-efg}
    \begin{align*}
        R^k(\mu^c)= \alpha^k \sum_{t=\start_{k}}^{\start_{k+1}-1} \inner{\cHat^t}{\muHat^t-\mu^c} + (1-\alpha^k) \sum_{t=\start_{k}}^{\start_{k+1}-1} \inner{\cHat^t}{\mu^c-\mu^c} \leq (2^{k-1}/R)R = 2^{k-1}. 
    \end{align*}
\end{proof}

\noindent Now suppose the algorithm exits a phase $k$. The following result mimics \cref{lemma:exit-nfg} for the case of EFGs, and we resort to essentially the same proof.
\begin{restatable}[Exiting a phase]{lemma}{lemmaExitEfg} \label{lemma:exit-efg}
    Let $k$ be such that $\alpha^k<1$ and suppose \cref{algo:phased-aggression-efg-bandit} exits phase $k$ at time step $\start_{k+1}-1$. Then almost surely $\RHat^k(\mu^c)\leq -2^{k-1}$.
\end{restatable}

\begin{proof}
    The if condition implies $\max_{\mu\in\Tcal}\sum_{t = \start_{k}}^{\start_{k+1}-1} \inner{\cHat^t}{\mu^c - \mu} > 2 R$, so when we let $\mu^{\star}$ be a maximizer, we find
    \begin{align*}
        \sum_{t=\start_{k}}^{\start_{k+1}-1} \inner{\cHat^t}{\mu^t-\mu^c} =& \alpha^k \sum_{t=\start_{k}}^{\start_{k+1}-1} \inner{\cHat^t}{\muHat^t-\mu^c} \nonumber\\
        =& \alpha^k \sum_{t=\start_{k}}^{\start_{k+1}-1} \inner{\cHat^t}{\muHat^t-\mu^{\star}} +\alpha^k \sum_{t=\start_{k}}^{\start_{k+1}-1} \inner{\cHat^t}{\mu^{\star}-\mu^c} \nonumber\\
        \leq& \alpha^k R + \alpha^k(-2R)\nonumber\\
        =& -2^{k-1},
    \end{align*}
    using \cref{eq:omd-regret1-efg} in the last inequality.
\end{proof}

\newpage

\subsection{Auxiliary Lemmas: OMD on the EFG Tree}

\paragraph{Unbalanced OMD Lemmas.}

\begin{lemma}[Bandit OMD with bounded surrogate costs] \label{lem:bandit-omd-bounded}
    Let $\eta >0$, and $L>0$. Let $(\cHat^t)_t$ be cost functions such that for all $t$, $0\leq \cHat^t(x_h,a)\leq L$ (for all $x_h$, $a$), and moreover $\cHat^t(x_h,a) = 0$ if $(x_h,a)\neq(x_h^t,a_h^t)$, where $x_h^t$, $a_h^t$ are arbitrary. Set $\muHat^1(x_h,a) = 1/A^h$ and consider the scheme 
    \begin{align*}
        \muHat^{t+1} =& \arg\min_{\mu \in \Tcal} \inner{\mu}{\cHat^t} + \frac{1}{\eta} D(\mu || \muHat^t)
    \end{align*}
    for $t\leq T'$. Then we have for all $\muHat\in\Tcal$
    \begin{align*}
        \sum_{t=1}^{T'} \inner{\muHat^t-\muHat}{\cHat^t} \leq \frac{X\log(A)}{\eta} + \eta H(H+1)L^2 T'.
    \end{align*}
\end{lemma}

\begin{proof}
    By \cref{lem:KL-diff-bandit}, 
    \begin{align*}
        D(\muHat||\muHat^t) - D(\muHat||\muHat^{t+1}) + D(\muHat^t||\muHat^{t+1}) =& -(D(\muHat||\muHat^{t+1})-D(\muHat||\muHat^t)) + (D(\muHat^t||\muHat^{t+1})-D(\muHat^t||\muHat^t))\\
        =& \eta \inner{\muHat^t-\muHat}{\cHat^t}.
    \end{align*}
    Thus (using $D\geq 0$), we have a regret bound of
    \begin{align*}
        \sum_{t=1}^{T'} \inner{\muHat^t-\muHat}{\cHat^t} \leq \frac{1}{\eta} \round{D(\muHat||\muHat^1) + \sum_{t=1}^T D(\muHat^t||\muHat^{t+1})}.
    \end{align*}
     For the first term we easily have $D(\muHat||\muHat^1) \leq X\log(A)$ \citep[Lemma 6]{kozuno2021model}. For the second term, by \cref{lem:KL-diff-bandit}, we have
     \begin{align*}
         D(\muHat^t||\muHat^{t+1}) = D(\muHat^t||\muHat^{t+1}) - D(\muHat^t||\muHat^t)
         \leq \eta\inner{\muHat^t}{\cHat^t} + \log(Z_1^t)= \eta\sum_{h=1}^H\muHat^t(x_h^t,a_h^t)\cHat^t(x_h^t,a_h^t) + \log(Z_1^t),
     \end{align*}
     using that $\cHat^t$ is zero outside $((x_h^t,a_h^t))_h$. By \cref{eq:Z-closed-form} and $\log(1+x)\leq x$,
     \begin{align*}
         \log(Z_1^t) \leq& \sum_{h=1}^H \muHat^t(x_{h}^t,a_{h}^t)\exp\round{-\eta \sum_{h'=1}^{h-1} \cHat^t(x_{h'}^t,a_{h'}^t)} \round{\exp\round{-\eta\cHat^t(x_{h}^t,a_{h}^t)} - 1} \\
         \leq& \sum_{h=1}^H \muHat^t(x_{h}^t,a_{h}^t)\exp\round{-\eta \sum_{h'=1}^{h-1} \cHat^t(x_{h'}^t,a_{h'}^t)} \round{-\eta\cHat^t(x_{h}^t,a_{h}^t)+\eta^2\cHat^t(x_{h}^t,a_{h}^t)^2},
     \end{align*}
     where we used $\exp(-y)\leq 1-y+y^2$ for $y\geq 0$. We thus find, using $\cHat^t\geq0$ throughout,
     \begin{align*}
         D(\muHat^t||\muHat^{t+1})\leq& \eta\sum_{h=1}^H\muHat^t(x_h^t,a_h^t)\cHat^t(x_h^t,a_h^t) + \log(Z_1^t)\\
         \leq& \eta\sum_{h=1}^H\muHat^t(x_h^t,a_h^t)\cHat^t(x_h^t,a_h^t) \\
         &+\sum_{h=1}^H \muHat^t(x_{h}^t,a_{h}^t)\exp\round{-\eta \sum_{h'=1}^{h-1} \cHat^t(x_{h'}^t,a_{h'}^t)} \round{-\eta\cHat^t(x_{h}^t,a_{h}^t)+\eta^2\cHat^t(x_{h}^t,a_{h}^t)^2}\\
         =& \eta\sum_{h=1}^H\muHat^t(x_h^t,a_h^t)\cHat^t_h(x_h^t,a_h^t)\round{1-\exp\round{-\eta \sum_{h'=1}^{h-1} \cHat^t(x_{h'}^t,a_{h'}^t)}} \\
         &+ \eta^2\sum_{h=1}^H \muHat^t(x_{h}^t,a_{h}^t)\exp\round{-\eta \sum_{h'=1}^{h-1} \cHat^t(x_{h'}^t,a_{h'}^t)} \cHat^t(x_{h}^t,a_{h}^t)^2\\
         \leq& \eta\sum_{h=1}^H\muHat^t(x_h^t,a_h^t)\cHat^t(x_h^t,a_h^t)\round{1-\exp\round{-\eta \sum_{h'=1}^{h-1} \cHat^t(x_{h'}^t,a_{h'}^t)}} \\
         &+ \eta^2\sum_{h=1}^H \muHat^t(x_{h}^t,a_{h}^t) \cHat^t(x_{h}^t,a_{h}^t)^2\\
         \leq& \eta\sum_{h=1}^H\muHat^t(x_h^t,a_h^t)\cHat^t(x_h^t,a_h^t)\round{\eta \sum_{h'=1}^{h-1} \cHat^t(x_{h'}^t,a_{h'}^t)} + \eta^2\sum_{h=1}^H \muHat^t(x_{h}^t,a_{h}^t) \cHat^t(x_{h}^t,a_{h}^t)^2,
     \end{align*}
     where we used $1-\exp(-x)\leq x$ in the last step. Finally, using the bound on the cost functions and the fact that all $\muHat^t(x_h^t,a_h^t) \leq 1$, we find
     \begin{align*}
         D(\muHat^t||\muHat^{t+1})\leq& \eta^2 H^2L^2 + \eta^2H L^2\leq \eta^2H(H+1) L^2.
     \end{align*}
     Summing over $t$ concludes the proof.
\end{proof}

\noindent In the setup of \cref{lem:bandit-omd-bounded}, let $\piHat^t\in\Pi$ be the policy corresponding to $\muHat^t$ and recall (\cref{app:efg-background})
\begin{align*}
    Z_{H+1}^t =& 1,\\
    Z_h^t =& \sum_{a_h} \piHat^t(a_h|x_h^t) \exp\round{\indicator{a_h^t=a_h}(-\eta\cHat^t(x_h^t,a_h)+\log(Z_{h+1}^t))}\\
    =& 1-\piHat^t(a_h^t|x_h^t)+\piHat^t(a_h^t|x_h^t)\exp\round{-\eta\cHat^t(x_h^t,a_h^t)+\log(Z^t_{h+1})},
\end{align*}
\noindent The following lemma is a slight generalization of \citet{kozuno2021model}. Indeed, the proof only uses that $\cHat^t$ is zero outside of the visited $((x_h^t,a_h^t))_h$, not whether we normalize by $\muHat^t$ or $\mu^t$ or from which policy the trajectory $(x_h^t,a_h^t)_h$ is sampled from. The same holds for the following closed form of $Z_1^t$ \citep[c.f. Lemma 6]{kozuno2021model}:
\begin{align}
    Z_1^t = 1 + \sum_{h'=1}^H \muHat^t(x_{h'}^t,a_{h'}^t)\exp\round{-\eta \sum_{h''=1}^{h'-1} \cHat^t(x_{h''}^t,a_{h''}^t)} \round{\exp\round{-\eta\cHat^t(x_{h'}^t,a_{h'}^t)} - 1}. \label{eq:Z-closed-form}
\end{align}

\begin{lemma}[\citet{kozuno2021model}, Lemma 7]\label{lem:KL-diff-bandit}
    In the setup of \cref{lem:bandit-omd-bounded}, we have 
    \begin{align*}
        D(\mu||\muHat^{t+1})-D(\mu||\muHat^t)=\eta\inner{\mu}{\cHat^t}+\log(Z_1^t)
    \end{align*}
    a.s. for all $t\leq T'$, $\mu\in\Tcal$.
\end{lemma}

\paragraph{Balanced OMD Lemmas.}

\noindent Recall the definition of $c^t$ from \cref{eq:def-loss}, for which $\cHat^t$ is an unbiased estimator. Again, recall that we WLOG replaced assume $u(s,a,b)\in[0,1]$ (by rescaling via $(1+u(s,a,b))/2$) for simplicity, without changing the regret bound. 

\begin{lemma} \label{lem:regret-to-init}
    Let $\tau >0$. Set $\muHat^1(x_h,a) = 1/A^h$ and with costs from \cref{eq:def-loss} for Protocol \ref{prot:EFGs-bandit} consider the scheme 
    \begin{align*}
        \cHat^t(x_h,a) =& \frac{\indicator{(x_h,a)=(x_h^t,a_h^t)}u_h^t}{\muHat^t(x_h,a)},\\
        \muHat^{t+1} =& \arg\min_{\mu \in \Tcal} \round{\inner{\mu}{\cHat^t} + \frac{1}{\tau} \Dbal(\mu || \muHat^t)}
    \end{align*}
    for $t\leq T'$. Then for all $\muHat\in\Tcal$
    \begin{align*}
        \exptn\rectangular{\sum_{t=1}^{T'} \inner{\muHat^t-\muHat}{c^t}} \leq \frac{\tau}{2} H^3T' + \frac{1}{\tau} \Dbal(\mu || \mu^1).
    \end{align*}
\end{lemma}

\begin{proof}
    By \cref{lem:KL-to-linear}, we have 
    \begin{align*}
        \frac{1}{\tau} \round{\Dbal(\muHat || \muHat^{t+1}) - \Dbal(\muHat || \muHat^t)} = \inner{\muHat}{\cHat^t} + \Xi_1^t.
    \end{align*}
    Thus,
    \begin{align*}
        \frac{1}{\tau} \exptn\rectangular{\Dbal(\muHat || \muHat^{T'}) - \Dbal(\muHat || \muHat^1)} =& \exptn\rectangular{\sum_{t=1}^{T'} \inner{\muHat}{\cHat^t} + \sum_{t=1}^{T'} \Xi_1^t}\\
        \leq& \exptn\rectangular{\sum_{t=1}^{T'} \inner{\muHat-\muHat^t}{\cHat^t}} + \frac{\tau H^3}{2} T'\tag{by \cref{lem:sum-xi}}\\
        =& \exptn\rectangular{\sum_{t=1}^{T'} \inner{\muHat-\muHat^t}{c^t}} + \frac{\tau H^3}{2} T',
    \end{align*}
    as $\exptn\rectangular{\cHat^t(x,a) \mid \mathcal{F}_{t-1}} = c^t(x,a)$. Using $\Dbal \geq 0$, we conclude
    \begin{align*}
        \exptn\rectangular{ \sum_{t=1}^{T'} \inner{\muHat^t-\muHat}{c^t} } \leq& \frac{1}{\tau} \Dbal(\muHat||\muHat^1) + \frac{\tau H^3}{2} T'.
    \end{align*}
\end{proof}

As before, the following lemma from \citet[Lemma D.7]{bai2022near} does not use the specific form of the cost estimates but only the update rules.
\begin{lemma} \label{lem:KL-to-linear}
    In the setup of \cref{lem:regret-to-init}, for all $\mu \in \Tcal$, we have
    \begin{align*}
        \Dbal(\mu||\muHat^{t+1}) - \Dbal(\mu||\muHat^t) = \tau \inner{\mu}{\cHat^t} + \frac{\log(Z^t_1)}{\mu^{\text{bal},1}(x_1^t,a_1^t)}=\tau \inner{\mu}{\cHat^t} + \tau \Xi_1^t.
    \end{align*}
\end{lemma}

\noindent We introduce some extra notation for convenience: Let $\piHat^t\in\Pi$ be the policy corresponding to $\muHat^t$ and set
\begin{align*}
    \beta_h^t := \tau \mu^{\text{bal},h}(x_h^t,a_h^t), \quad
    \piHat^t_h := \piHat^t(a_h^t|x_h^t), \quad 
    \cHat^t_h := \cHat^t(x_h^t,a_h^t),
\end{align*}
and consider the functions 
\begin{align*}
    \Xi_H^t(\cHat) :=& \Xi_H^t(\cHat_{H}) := \log\round{1-\piHat_H^t + \piHat_H^t\exp(-\beta_H^t\cHat_H)} / \beta_H^t,\\
    \Xi_h^t(\cHat) :=& \Xi_h^t(\cHat_{h:H}) := \log\round{1-\piHat_h^t + \piHat_h^t\exp(\beta_h^t(\Xi_{h+1}^t(\cHat_{h+1:H})-\cHat_h))} / \beta_H^t \quad (h<H),
\end{align*}
and the values
\begin{align*}
    \Xi_h^t :=& \Xi_h^t(\cHat^t) = \frac{1}{\beta_h^t} \log(Z^t_h) = \frac{1}{\beta_h^t} \log\round{ 1-\piHat_h^t + \piHat_h^t\exp(\beta_h^t(\Xi_{h+1}^t - \cHat^t_h))) } \quad (h\in[H])
\end{align*}
for the input $\cHat^t$. The following lemma now lets us bound the remaining term in the proof of \cref{lem:regret-to-init}.

\begin{lemma} \label{lem:sum-xi}
    In the setup of \cref{lem:regret-to-init}, we have 
    \begin{align*}
        \sum_{t=1}^T \exptn\rectangular{\Xi_1^t} \leq -\sum_{t=1}^{T'} \exptn\rectangular{\inner{\muHat^t}{\cHat^t}} + \frac{\tau}{2} H^3 {T'}.
    \end{align*}
\end{lemma}

\begin{proof}
    By \cref{lem:log-to-linear} and as $\cHat^t$ is unbiased,
    \begin{align*}
        \sum_{t=1}^{T'} \exptn\rectangular{\Xi_1^t} \leq& -\sum_{t=1}^{T'} \exptn\rectangular{\inner{\muHat^t}{\cHat^t}} + \frac{\tau H}{2} \sum_{t=1}^{T'} \sum_{h=1}^{H} \sum_{h'=h}^H \sum_{x_{h'},a_{h'}} \exptn\rectangular{\mu^{\text{bal},h}_{1:h}(x_{h},a_{h}) \muHat^{t}_{h+1:h'}(x_{h'},a_{h'}) \cHat^t(x_{h'},a_{h'})}\\
        =& -\sum_{t=1}^{T'} \exptn\rectangular{\inner{\muHat^t}{c^t}} + \frac{\tau H}{2} \sum_{t=1}^{T'} \sum_{h=1}^{H} \sum_{h'=h}^H \underbrace{\sum_{x_{h'},a_{h'}} \exptn\rectangular{\mu^{\text{bal},h}(x_{h},a_{h}) \muHat^{t}_{h+1:h'}(x_{h'},a_{h'}) c^t(x_{h'},a_{h'})}}_{\leq 1}\\ 
        \leq& -\sum_{t=1}^{T'} \exptn\rectangular{\inner{\muHat^t}{c^t}} + \frac{\tau H^3}{2} T'. 
    \end{align*}
\end{proof}

\begin{lemma}[\citet{bai2022near}, Lemma D.11]\label{lem:log-to-linear}
    We have
    \begin{align*}
        \Xi^t_1 \leq -\inner{\muHat^t}{\cHat^t} + \frac{\tau H}{2} \sum_{h=1}^{H} \sum_{h'=h}^H \sum_{x_{h'},a_{h'}} \mu^{\text{bal},h}(x_{h'},a_{h'}) \muHat^{t}_{h+1:h'}(x_{h'},a_{h'})\cHat^t(x_{h'},a_{h'}),
    \end{align*}
    where $\muHat^t_{h+1:h'}(x_{h'}^t,a_{h'}^t):=\prod_{h''=h+1}^{h'}\piHat^t(a_{h''}|x_{h''})$ along the unique path $(x_{h''},a_{h''})_{h''}$ leading from step $h+1$ to $(x_{h'}^t,a_{h'}^t)$.
\end{lemma}

\noindent The proof is the same as in \citet{bai2022near}.\footnote{There, in (ii) we still have $\muHat^t(x_h,a_h) \cHat^t(x_h,a_h) \leq 1$. All other properties used in the proof hold for general $\cHat \geq 0$ (in particular Lemma D.9 and D.10, although stated for $\lTilde \in [0,1]^H$).} 

\subsection{Lower Bound} \label{app:efg-lower}


\thmEfgLower*

\begin{proof}
    Consider an $A$-nary tree with $X=\Theta(A^H)$ leaves and where each infoset corresponds to a unique state. As for the transitions, the learner is deterministically sent to a leaf $s=(a_1,\dots,a_A)$ upon playing $a_h$ in each step $h$. Since $\delta=\min_{x,a}\mu^c(x,a)$, there also exists a leaf information set $x=x(s)$ and and an action $a$ such that $\mu^c(x,a)=\delta$. Now consider two environments in which all state-action triples have cost one, except for the cost in leaf $s$, which is either sampling according to the $(+)$ or $(-)$ environment from \cref{thm:lower-nfg}. We are thus effectively simulating a two-armed bandit with comparator $(1-\delta,\delta)$ with the same construction as in the simplex case. The derivation in \cref{thm:lower-nfg} thus concludes the proof. 
\end{proof}


\section{Further Experimental Evaluations} \label{app:further-experiments}

In this section, we provide further details regarding our experimental evaluations in \cref{sec:experiments}.

\noindent \textbf{All vs Exploitable Strategies.} In addition to \cref{sec:experiments}, we compare the performance of Min-Max, OMD and \cref{algo:phased-aggression-efg-bandit} against a couple of other exploitable strategies strategies. We consider the following constant strategies:

\begin{itemize}[leftmargin=*]
    \setlength{\itemsep}{0.3em}
    \setlength{\parskip}{0pt}
    \item \emph{RaiseK}: Bob raises/calls if and only he has a King, and checks/folds otherwise.
    \item \emph{RandMinMax($\alpha$)}: Bob plays a perturbed version of the Min-Max strategy: In every round, with a small probability $\alpha$, he will play the uniform strategy, and otherwise the Min-Max strategy. 
\end{itemize}

In \cref{fig:all-vs-x2}, we present the amount of money that each of Min-Max, OMD, and Algorithm~\ref{algo:phased-aggression-efg-bandit} extract with respect to the aforementioned exploitable strategies. Specifically, \cref{fig:all-vs-x2} reveals the following. 

\begin{figure*}[ht]
    \centering

    \begin{minipage}{1.0\textwidth}
    \begin{minipage}{0.05\textwidth}
        
    \end{minipage}
    \begin{minipage}{0.3\textwidth}
        \centering
        \includegraphics[width=\linewidth]{plots/All-vs-Rand=0.05.png}
        \caption*{All vs RandMinMax($0.05$)}
    \end{minipage}
    %\hfill
    \begin{minipage}{0.3\textwidth}
        \centering
        \includegraphics[width=\linewidth]{plots/All-vs-Rand=0.1.png}
        \caption*{All vs RandMinMax($0.1$)}
    \end{minipage}
    \begin{minipage}{0.3\textwidth}
        \centering
        \includegraphics[width=\linewidth]{plots/All-vs-Rand=0.15.png}
        \caption*{All vs RandMinMax($0.15$)}
    \end{minipage}
    \begin{minipage}{0.05\textwidth}
        
    \end{minipage}
    \begin{minipage}{1.0\textwidth}
    \centering
    $\;$
    \end{minipage}
    \end{minipage}
    
    \vspace{0.4cm}
    
    \centering
    \begin{minipage}{1.0\textwidth}
    \begin{minipage}{0.05\textwidth}
        
    \end{minipage}
    \begin{minipage}{0.3\textwidth}
        \centering
        \includegraphics[width=\linewidth]{plots/All-vs-Rand=0.3.png}
        \caption*{All vs RandMinMax($0.3$)}
    \end{minipage}
    %\hfill
    \begin{minipage}{0.3\textwidth}
        \centering
        \includegraphics[width=\linewidth]{plots/All-vs-Rand=0.5.png}
        \caption*{All vs RandMinMax($0.5$)}
    \end{minipage}
    \begin{minipage}{0.3\textwidth}
        \centering
        \includegraphics[width=\linewidth]{plots/All-vs-King.png}
        \caption*{All vs RaiseK}
    \end{minipage}
    \begin{minipage}{0.05\textwidth}
        
    \end{minipage}
    \begin{minipage}{1.0\textwidth}
    \centering
    \vspace{0.4cm}
    \caption{All vs Bob comparison for $T=1000$ rounds. The x-axis displays the round $t$, and the y-axis displays how much {\color{mblue}Min-Max}, {\color{mred}OMD}, and {\color{mgreen}\cref{algo:phased-aggression-efg-bandit}} gained from the second algorithm so far. The y-axes have varying scales for readability.}\label{fig:all-vs-x2} \vspace{-0.4cm}
    \end{minipage}
    \end{minipage}
\end{figure*}

\emph{All vs RandMinMax($\alpha$)}: In all plots, our \cref{algo:phased-aggression-efg-bandit} achieves at least the gain of the min-max equilibrium and in fact always improves slightly over it. For small values of $\alpha$ (e.g. $\alpha=0.05$), meaning that Bob plays a (reasonable) strategy very close to the min-max equilibrium, OMD always loses money while \cref{algo:phased-aggression-efg-bandit} wins linearly. For larger values of $\alpha$ (e.g. $\alpha=0.1, 0.15,0.3$), OMD loses an initial amount but slowly starts catching up towards a total positive gain for very large $T$. Finally, when $\alpha$ is large (e,g, $\alpha=0.5$), meaning that Bob plays a highly suboptimal (and not exploitative) strategy, OMD is able to obtain a positive gain much quicker and eventually surpasses our \cref{algo:phased-aggression-efg-bandit} (as it is not restricted to the support of the min-max equilibrium, which in this case is of advantage). 

\emph{All vs RaiseK}: Notice that min-max equilibrium does not exploit RaiseK at all. At the same time, OMD exploits it linearly right away, extracting a near-optimal gain from the opponent. Our \cref{algo:phased-aggression-efg-bandit} also exploits \emph{RaiseK} linearly at a comparable slope, however starting exploitation somewhat delayed due to the risk-averse nature of the algorithm. However, our algorithm consistently exploits weak opponents significantly better than the min-max strategy in all cases, and unlike OMD does so while not risking to lose essentially any money.

In summary, our experimental evaluations reveal the following insights that are in accordance with our theoretical findings: If Alice plays \cref{algo:phased-aggression-efg-bandit}, she secures at least the gain of the min-max strategy, thus not losing against any opponent. Yet, she is able to better exploit strategies that deviate from the min-max strategy, at a level often comparable to standard no-regret algorithms.

\textbf{Implementation Details.} In all experiments, we average $n=5$ runs of repeated play (plotting Alice's average cumulative expected gain), and plot one standard deviation. In all algorithms, we used the same learning fixed rates ($\eta \sim1/\sqrt{T}$) and the (unbalanced) dilated KL divergence for fairness and simplicity. All simulations were performed on a MacBook Pro 2.8 GHz Quad-Core Intel Core i7. We provide the code in the supplementary material.




\end{document}