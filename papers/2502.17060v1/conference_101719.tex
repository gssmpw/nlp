\documentclass[conference]{IEEEtran}
\IEEEoverridecommandlockouts
% The preceding line is only needed to identify funding in the first footnote. If that is unneeded, please comment it out.
\usepackage[nocompress]{cite}
\def\citepunct{,\,}


% Math pachkage
\usepackage{amsmath,amssymb,amsfonts}
\usepackage{lipsum}  

\newcommand{\CommaBin}{\mathbin{\raisebox{0.5ex}{,}}}
\newcommand{\CommaPunct}{\mathpunct{\raisebox{0.5ex}{,}}}


\usepackage{algorithm}
\usepackage{algpseudocode}
% Figure packages
\usepackage{graphicx, stfloats}
\usepackage{caption}
\usepackage{subcaption}

\usepackage{adjustbox}


\usepackage{hhline}
\usepackage{makecell}
\usepackage{multirow}
\usepackage{tabulary}
\usepackage{float}
\usepackage{adjustbox}


\usepackage{textcomp}
\usepackage{xcolor}

\usepackage[hidelinks]{hyperref}
\def\BibTeX{{\rm B\kern-.05em{\sc i\kern-.025em b}\kern-.08em
    T\kern-.1667em\lower.7ex\hbox{E}\kern-.125emX}}
\begin{document}

\title{Data Analysis Prediction over Multiple Unseen Datasets: A Vector Embedding Approach
% {\footnotesize \textsuperscript{*}Note: Sub-titles are not captured in Xplore and
% should not be used}
% \thanks{Identify applicable funding agency here. If none, delete this.}
}

\author{\IEEEauthorblockN{Andreas Loizou}
\IEEEauthorblockA{\textit{Database and Knowledge Systems Lab} \\
\textit{School of ECE, National Technical University of Athens}\\
\href{mailto:antreasloizou@mail.ntua.gr}{antreasloizou@mail.ntua.gr}}
\and
\IEEEauthorblockN{Dimitrios Tsoumakos}
\IEEEauthorblockA{\textit{Database and Knowledge Systems Lab} \\
\textit{School of ECE, National Technical University of Athens}\\
\href{mailto:dtsouma@mail.ntua.gr}{dtsouma@mail.ntua.gr}}}
% \author{\IEEEauthorblockN{Anonymous Author(s)
% }}
\maketitle

\begin{abstract}
The massive increase in the data volume and dataset availability for analysts compels researchers to focus on data content and select high-quality datasets to enhance the performance of analytics operators. While selecting the highest quality data for analysis highly increases task accuracy and efficiency, it is still a hard task, especially when the number of available inputs is very large. To address this issue, we propose a novel methodology that infers the outcome of analytics operators by creating a model from datasets similar to the queried one. Dataset similarity is performed via projecting each dataset to a vector embedding representation. The vectorization process is performed using our proposed deep learning model \emph{NumTabData2Vec}, which takes a whole dataset and projects it into a lower vector embedding representation space. Through experimental evaluation, we compare the prediction performance and the execution time of our framework to another state-of-the-art modelling operator framework, illustrating that our approach predicts analytics outcomes accurately. Furthermore, our vectorization model can project different real-world scenarios to a lower vector embedding representation and distinguish between them.

\end{abstract}

\begin{IEEEkeywords}
Analytics Modelling, Vector embeddings, Vector Similarity, Data Quality
\end{IEEEkeywords}

% 
% 
The widespread integration of communication networks and smart devices in modern control systems has increased the vulnerability of industrial systems to online cyber-attacks, e.g., Industroyer, Blackenergy, etc \citep{osti_1505628}.
% Modern control systems have seen a large push to include communication networks and smart devices to increase performance, made possible by improvements in communication device cost and energy consumption. This trend has been coupled with the usage of open-standard communication protocols among industrial control systems, making them vulnerable to online cyber-attacks such as Industroyer, Blackenergy, etc \citep{osti_1505628}. 
To counter this, methods have been developed to improve security by achieving attack detection, mitigation, and monitoring, among others \citep{sandberg2022secure}. This paper focuses on active attack diagnosis to mitigate stealthy attacks. 
%
%\subsection{Literature review}

Active diagnosis techniques rely on the inclusion of additional moduli to control systems
% inclusion within the control system of additional moduli 
to alter the behavior of the system compared to information known by the attacker. 
For instance, the concept of additive watermarking was introduced in \cite{mo2015physical}, where noise signals of known mean and variance are added at the plant and compensated for it at the controller. 
This compensation, however, is not exact, causing some performance degradation. Thus, trade-offs between performance and detectability  are necessary \citep{zhu2023detection}.
% A later work \citep{zhu2023detection} designs the watermark signal by trading performance for detection. Thus, although additive watermarking serves as a good detection scheme, they endure performance losses even in the nominal case. 

In encrypted control \citep{darup2021encrypted}, the sensor data is encrypted, sent to the controller, and then operated on directly. Encrypted input signals are sent back to the plant for decryption. Although encryption is widespread in IT security, in control systems it presents some concerns, such as the introduction of time delays \citep{stabile2024verifiable}, while it may present inherent weaknesses \citep{alisic2023model}.
% they are not preferred as they introduce time delays \citep{stabile2024verifiable} which can cause instability, and some encryption schemes can be very weak  \citep{alisic2023model}. 

In moving target defense \citep{griffioen2020moving}, the plant is augmented with fictitious dynamics, known to the controller. The plant output is transmitted to the controller along with the fictitious states over a network under attack. 
The additional measurements then aide in the detection of attacks. 
This comes at the cost of higher communication bandwidth needs, which increases rapidly with the dimension of the augmented systems.
% Since the dynamics of the fictitious dynamics are exactly known to the controller, the attack is detected easily. However, when the scale of the system increases, the communication bandwidth used by moving the target defense approach increases rapidly. 

Other recently proposed works include two-way coding \citep{fang2019two}, a weak encryuption technique, and dynamic masking \citep{abdalmoaty2023privacy}, which enhances privacy as well as security, have been shown to be effective against zero-dynamics attacks.
% Two-way coding \citep{fang2019two} and dynamic masking \citep{abdalmoaty2023privacy} are other recently proposed approaches. Two-way coding is another form of weak encryption technique whilst dynamic masking proposes an architecture that enhances both privacy and security. These schemes are shown to be effective against zero dynamics attacks but remain to be studied for other classes of attacks. 
% Recent extensions include \citep{mukherjee2021secure,ramos2024privacy}.
% Some other works which are related are \citep{mukherjee2021secure}, an extension of \cite{fang2019two}. The work \citep{ramos2024privacy} is an extension of moving target defense for multi-agent systems. 
Furthermore, filtering techniques for attack detection are proposed by \cite{murguia2020security,hashemi2022codesign,escudero2023safety}, while not focusing on stealthy attacks.
% The works \citep{murguia2020security,hashemi2022codesign,escudero2023safety} develop filtering techniques to guarantee safety, without being focused on stealthy covert attacks.

Multiplicative watermarking (mWM) has been proposed by the authors as a diagnosis technique \citep{ferrari2020switching}. mWM consists of a pair of filters on each communication channel between the plant and its controller; the scheme is affine to weak encryption, whereby ``encoding'' and ``decoding'' are done by changing signals' dynamic characteristics through inverse pairs of filters. This enables original signals to be recovered exactly, and thus does not lead to performance degradation.
% A multiplicative watermark is an affine to a weak encryption technique, through which the signal is ``encoded'' by a filter, changing its dynamic behavior. The use of inverse pairs means that the original signal can be recovered, through ``decoding'' via an inverse filter. As such, differently to techniques based on additive watermarking, no performance is lost due to the injection of noise, and there are no bandwidth limitations.

%\subsection{Contributions}
One of the critical features of multiplicative watermarking is that to detect stealthy attacks, the mWM filter parameters must be switched over time. In this paper, an algorithm to optimally design the mWM parameters after a switching event is presented, enhancing detection performance, without changing the switching time.
% This is done without changing the switching time, which is taken as given.

\textcolor{black}{
To formalize the filter design problem, we suppose the defender is interested in optimal performance against adversaries injecting covert attacks with matched system parameters \citep{smith2015covert}, including the mWM parameters prior to the switch. This scenario represents a worst case where malicious agents can take full control of the system while remaining undetected.
Thus, the attack strategy is explicitly included within the formulation of the closed-loop system, and the mWM filters are chosen by solving an optimization problem minimizing the attack-energy-constrained output-to-output gain (AEC-OOG) \citep{anand2023risk}, a variation of the output-to-output gain proposed in  \cite{teixeira2015strategic}.
}
The main contributions of this paper are:
% We consider an adversary injecting a covert attack with matched system parameters \citep{smith2015covert}, i.e., an attacker with full knowledge of the control system parameters, including those of the mWM filters before the switch. This scenario is taken as a worst case, as it has been shown that this class of attacks can be made stealthy. To quantitatively define a cost, the output-to-output gain (OOG) \citep{teixeira2015strategic} is leveraged,
% a metric introduced to evaluate the impact of an additive attack in a control system. %Specifically, OOG evaluates the worst-case performance loss that an attacker injecting an undetectable attack can obtain. 
% Here, the maximum performance loss caused by a stealthy adversary with limited energy is taken, the attack-energy-constrained OOG (AEC-OOG) \citep{anand2023risk}. The main contributions of this paper are:
\begin{enumerate}
%[label=\alph*.]
\item The problem of optimally designing the switching mWM filters is formulated as an optimization problem, with the AEC-OOG is taken as the objective;%where the AEC-OOG is taken as the impact metric; 
\item The worst-case scenario of a covert attack with exact knowledge of plant and mWM filter parameters is embedded within the design problem;
% The optimization problem is defined to incorporate the worst-case scenario of a covert attack with exact knowledge of plant and mWM filter parameters;
\item The feasibility of the optimization problem is shown to be dependent only on stability conditions; 
\item A solution scheme is proposed to promote randomization of the mWM filter parameters such that an eavesdropping adversary cannot remain stealthy.
\end{enumerate} 

This builds on the results of \cite{ferrari2020switching}, where the focus was on the design of the switching protocols, rather than the parameters themselves.
Compared to previous work \citep{gallo2021design}, this paper introduces an optimization problem which is always feasible (thanks to the use of AEC-OOG in the objective), while also considering a more sophisticated class of covert attacks, where the presence of watermark is known to the adversary. 
Moreover, this paper poses a different objective than \citep{zhang2023hybrid}; indeed, while \citep{zhang2023hybrid} provided a design strategy to ensure certain privacy properties, in this paper we address the problem of optimal parameter design following a switching event.


%\subsection{Organization}
The rest of the paper is organized as follows. 
After formulating the problem in Section~\ref{sec:PF}, we propose our design algorithm in Section~\ref{sec:main}, and analyze its properties. It is then evaluated through a numerical example in Section~\ref{sec:NE}, and concluding remarks are given Section~\ref{sec:Con}.
% We provide the problem background in Section~\ref{sec:PF}. We formulate the design problem in Section~\ref{sec:main}, together with an analysis of its properties. The proposed algorithm is evaluated through a numerical example in Section \ref{sec:NE}. Concluding remarks are offered in Section \ref{sec:Con}.

\section{Related Work}
\subsection{Multimodal Large Language Models}
% Building on the success of large language models (LLMs) \citep{yao2024tree, glm2024chatglm, achiam2023gpt, touvron2023llama, brown2020language}, multimodal large language models (MLLMs) \citep{liu2024improved, li2023blip, zhu2023minigpt, wang2023cogvlm, liu2024visual} extend these capabilities by integrating vision and text processing, achieving remarkable performance in tasks involving images, videos, and multimodal reasoning. However, handling visual data poses computational challenges due to the redundancy and low information density of high-resolution tokens \citep{liang2022evit} and the quadratic scaling of attention mechanisms \citep{vaswani2017attention}.
% For instance, models like LLaVA \citep{liu2023improvedllava} and mini-Gemini-HD \citep{li2024mini} encode high-resolution images into thousands of tokens, while video-based models such as VideoLLaVA \citep{lin2023video} and VideoPoet \citep{kondratyuk2023videopoet} allocate even more tokens to process multiple frames. These challenges highlight the need for more efficient token representations and longer context lengths to enable scalability. Recent advancements, such as Gemini \citep{geminiteam2023gemini} and LWM \citep{liu2024world}, have focused on addressing these issues by optimizing token efficiency and extending the context length, paving the way for more scalable and effective MLLMs.

The remarkable success of large language models (LLMs) \citep{radford2019language, brown2020language} has spurred a growing trend of extending their advanced reasoning capabilities to multi-modal tasks, leading to the development of vision-language models (VLMs) \citep{huang2023languageneedaligningperception, driess2023palmeembodiedmultimodallanguage, liu2024visual, Qwen-VL}. These VLMs typically consist of a visual encoder \citep{radford2021learning} that serializes input image representations and an LLM responsible for text generation. To enable the LLM to process visual inputs, an alignment module is employed to bridge the gap between visual and textual modalities. This module can take various forms, such as a simple linear layer, an MLP projector, or a more complex query-based network. While this integration allows the LLM to gain visual perception, it also introduces significant computational challenges due to the long sequences of visual tokens.

Moreover, existing VLMs often exhibit limitations, such as visual shortcomings or hallucinations, which hinder their performance. Efforts to enhance VLM capabilities by increasing input image resolution have further exacerbated computational demands. For instance, encoding higher-resolution images results in a substantial increase in the number of visual tokens. A model like LLaVA-1.5 \citep{liu2024improved} generates 576 visual tokens for a single image, while its successor, LLaVA-NeXT \citep{liu2024llavanext}, produces up to 2880 tokens at double the resolution, far exceeding the length of typical textual prompts.
Optimizing the inference efficiency of VLMs is thus a critical task to facilitate their deployment in real-world scenarios with limited computational resources.

\subsection{Visual Token Compression}
% Visual tokens often exceed text tokens by tens to hundreds of times, with visual signals being more spatially redundant compared to information dense text \citep{marr2010vision}.
% Various methods have been proposed to address this issue. For instance, LLaMA-VID \citep{li2023llama} uses a Q-Former with context tokens, and DeCo \citep{yao2024deco} applies adaptive pooling to downsample visual tokens at the patch level.
% However, these approaches require modifying model components and additional training, increasing computational and training costs.
% ToMe~\citep{bolya2022tome} reduces tokens without training by adding a token merge module to ViTs, but this disrupts early cross-modal interactions in language models~\citep{xing2024PyramidDrop}. FastV~\citep{chen2024image} selects important visual tokens using attention scores, while SparseVLM~\citep{zhang2024sparsevlm} incorporates text guidance via cross-modal attention.
% However, these methods forgo flash-attention~\citep{dao2022flashattention, dao2023flashattention2} and primarily focus on token importance, overlooking the impact of token duplication.
% In our work, we preserve hardware acceleration compatibility, including flash attention, while considering both token importance and duplication for token reduction.

Visual tokens are often significantly more numerous than text tokens, with higher spatial redundancy and lower information density. To address this issue, various methods have been proposed for reducing visual token counts in vision language models. For instance, some approaches modify model components, such as using context tokens in Q-Former \citep{li2023llama} or applying adaptive pooling at the patch level, but these typically require additional training and increase computational costs. Other techniques, like Token Merging (ToMe) \citep{bolya2022tome} and FastV \citep{chen2024image}, focus on reducing tokens without retraining by merging tokens or selecting important ones based on attention scores. SparseVLM \cite{zhang2024sparsevlm} incorporates text guidance through cross-modal attention to refine token selection. However, these methods often overlook hardware acceleration compatibility and fail to account for token duplication alongside token importance. Furthermore, while token pruning has been extensively explored in natural language processing and computer vision to improve inference efficiency, its application to VLMs remains under-explored. Existing pruning strategies, such as those in FastV and SparseVLM, rely on text-visual attention within large language models (LLMs) to evaluate token importance, which may not align well with actual visual token relevance.



\section{Methodology}
In this section, we outline the key research questions driving this study, followed by a detailed description of the methodology used to design and conduct the survey.
\subsection{Research Questions}
\begin{enumerate}
    \item[\textbf{RQ1:}] How do developers allocate their time during a typical workweek, and how does this compare to their perception of an \textbf{ideal workweek?}
    \item[\textbf{RQ2:}] How are developer's satisfaction and productivity affected by \textbf{deviations} from their ideal workweek?
     \item[\textbf{RQ3:}] For which tasks do developers prefer using \textbf{AI tools}, and how does the frequency of AI tool usage \textbf{influence} their satisfaction and productivity?
\end{enumerate}

\subsection{Survey Design}
% Describe how the survey was conducted, survey structure, sample size, which activities were selected and how, incentives, etc. 

To gain insights into the types of activities developers engage in during a typical work week, we conducted a series of exploratory interviews with 12 randomly selected participants. These semi-structured interviews provided a qualitative foundation, allowing us to iteratively develop a comprehensive list of higher-level activities that reflect both ideal and actual workweek allocations. The findings from these interviews were instrumental in refining our survey questions and design.

% - When was it distributed
% - How many people were invited
% - how was the survey advertised
% - incentive provided to participants
% - how many responses received (with response rates)
% - Board of ethics description \& instruments
% - Describe the main questions asked in the survey

The survey was distributed in \textcolor{blue}{May 2024} to software engineers working in Microsoft teams across India and the United States. A total of 6000 developers were invited to participate via email. Framed as a study aimed at boosting developer productivity by understanding how they allocate their time in a workday, the survey received 510 complete responses (responses rate of 8.5\%). After finishing the survey, the participants could enter a sweepstake to win one out of ten \$50 Amazon.com Gift Cards.
\textcolor{blue}{description of ethics}.

The main questions in the survey were as follows:
\begin{enumerate}
    \item Their roles and years of experience in the industry/team
    \item The hours spent on various activities in their typical workweek
    \item Ideally, the percentage of time they would want to allocate to each activity in a workweek
    \item How productive and satisfied were they by their past workweek
    \item Activities they find most cognitively challenging
    \item How often do they use AI tools to assist in their daily activities
    \item Two open-ended questions about the activities they would want to automate using AI tools, and advice for new hires to boost their productivity and satisfaction levels 
\end{enumerate}



\subsection{Data Analysis \& Exploration}
% Here, we could start with discussing the survey group:
% - demographic observations
% - distribution of participants (based on the years experience in the industry/team), 

From the exploratory interviews, we identified sixteen key activities, which were subsequently used to quantify the developers' time allocation across their work week. 

\subsection{Limitations}

\section{Evaluation}
% In light of experiments of CacheBlend (\S\ref{eval:1}) and EPIC (\S\ref{eval:2}), we design our experiments (\S\ref{eval:3}).

% \noindent\textbf{LLM Dataset.} 2WikiMQA, MuSiQue, HotpotQA, SAMSum, MultiNews.

% \noindent\textbf{LLM Baselines.} Full KV recompute, Prefix caching, Full KV reuse, CacheBlend, EPIC.
% \subsection{CacheBlend evaluation}\label{eval:1}
% \begin{itemize}
%     \item TTFT-Score Comparison.
%     \item RPS-TTFT Comparison.
%     \item Sensitivity Analysis. (1) chunk number; (2) chunk length; (3) batch size; (4) recompute ratio; (5) storage device (CPU RAM / slower Disk).
% \end{itemize}
% \subsection{EPIC evaluation}\label{eval:2}
% \begin{itemize}
%     \item TTFT-Score Comparison.
%     \item (CCR+RPS)-TTFT/Throughput Comparison.
%     \item Context length-TTFT Comparison.
%     \item Semantic-based / fixed-token-based splitting.
% \end{itemize}
% \subsection{\sys}\label{eval:3}\
% \noindent\textbf{VLM Model.} InternVL 2.5-8B \cite{chen2024internvl}, Qwen2-VL-7B \cite{wang2024qwen2vl}, LLaVA-1.6-vicuna-7B, LLaVA-1.6-Mistral-7B \cite{liu2024llavanext}.

% \noindent\textbf{VLM Dataset.} SparklesDialogueCC, SparklesDialogueVG \cite{huang2024sparkles}, MMDU \cite{liu2024mmdu}.

% \noindent\textbf{VLM Baselines.} CacheBlend, Prefix caching, Full KV reuse, \sys.

% \begin{itemize}
%     \item TTFT-Score Comparison.
%     \item RPS-TTFT/Throughput Comparison.
%     \item Sensitivity Analysis: Image number.
%     \item Why does CacheBlend fail to work when serving MLLM?
% \end{itemize}

In this section, we evaluate \sys~in terms of response time and generation quality. We also investigate whether \sys~is applicable when the number of images is large.
\subsection{Experimental settings}
We select two prevalent MLLMs in the experiments: LLaVA-1.6-vicuna-7B and LLaVA-1.6-mistral-7B \cite{liu2024llavanext}. All experiments are run on a server with 1 NVIDIA H800-80 GB GPU, 20-core Intel(R) Xeon(R) Platinum CPUs, and 100GB DRAM.

Two datasets are used in our evaluation. (1) \textbf{MMDU} \cite{liu2024mmdu}: This dataset aims to evaluate MLLMs' abilities in multi-turn and multi-image conversations. Each conversation stitches together multiple images and sentence-level text (e.g., ``IMAGE\#1, IMAGE\#2. Can you describe these images as detailed as possible?"). (2) \textbf{SparklesEval} \cite{huang2024sparkles}: This is also a dataset for assessing MLLMs' conversational competence across multiple images and conversation turns. Unlike MMDU, SparklesEval integrates multiple images at word level (e.g., ``Can you link the celebration occurring in IMAGE\#1 and the dirt bike race in IMAGE\#2 ?"). As shown in the examples, the prompts of two datasets are open questions. Previous works adopt GPT score to evaluate the quality of MLLMs' responses to the open questions \cite{liu2024mmdu, huang2024sparkles}. GPT score is a GPT-assisted evaluation that uses a powerful judge model (e.g., GPT-4o, Qwen, etc.) to assess the answers. We also employ this metric and their evaluation prompt, as listed in Appendix~\ref{prompt}.
% (3) \textbf{V*Bench} \cite{wu2024v}:  A dataset specifically designed to evaluate
% MLLMs in their ability to process high-resolution images and focus on visual details. Each sample contains a high-resolution image, a question, and four options.
% We select 100 samples from each of the above datasets for testing, each including 1 to 5 images.

% We use the following metrics to measure the performance of algorithms. (1) Time-To-First-Token (TTFT) refers to the time it takes for LLMs, to generate and return the first token after receiving an request. This metric is designed to measure the time spent in the prefill stage, which can be optimized by addressing the PIC problem. (2) GPT score \cite{liu2024mmdu, huang2024sparkles} is a GPT-assited evaluation  that uses a judge model (e.g., GPT-4o, Qwen, etc.) to assess the quality of model-generated responses. We employ this metric to assess the quality of MLLMs' responses to the open questions in MMDU and SparklesEval. We apply the evaluation prompts in MMDU \cite{liu2024mmdu} to guide the judge model for scoring in the range of 10. 

% (3) F1 score is a metric used to evaluate the similarity between MLLMs’ output and the groundtruth answer. We employ this metric to assess the accuracy of the MLLMs' answers to the multiple-choice questions in V*Bench.

We compare \sys-$k$ with three existing CC algorithms: prefix caching, full reuse, and CacheBlend \cite{yao2024cacheblend}. CacheBlend is also a position-independent algorithm designed for RAG system. It recomputes $r$\% of total tokens with largest KV deviation, so we denote it as CacheBlend-$r$. The primary focus of CacheBlend is the KV deviation, while the \sys's selection process involves the identification of tokens that exhibit both high attention scores and significant KV deviation. We implement the four CC algorithms based on vLLM 0.6.4 \cite{kwon2023efficient}.

% (1) Prefix Caching: This algorithm merely stores and reuses the KV chche of the prefix. And the KV cache of non-prefix tokens needs to be computed during prefill. (2) Full Reuse: This algorithm reduces TTFT by fully reusing the entire KV cache regardless of the position of multimodal data. (3) CacheBlend \cite{yao2024cacheblend}: This is a state-of-the-art partial reuse algorithm that achieves a trade-off between TTFT and generation quality by dynamically selecting partial tokens to recompute.
% Additionally, we evaluate various variants of CacheBlend, denoted as CacheBlend-r, where $r$ represents the ratio of tokens recomputed. Similarly, we test different variants of InfoBlend, denoted as InfoBlend-k, where $k$ indicates the number of tokens recomputed at each chunk boundary.

\subsection{Effectiveness of \sys}
Based on vLLM offline inference, we compare the performance of all algorithms. Specifically, we process all requests sequentially and evaluate their generation quality and processing time for prefill. The workflow initiates with the precomputation of the relevant KV cache for images. Subsequently, we send the user's query along with the cache\_ids of the images to the serving system. Prefix caching will process the query with the KV cache of system prompt only. \sys~concatenates the dummy cache and stored cache, and computes the first output token using selective attention mechanism in single step. Full reuse and CacheBlend first compute the KV cache of text, and then produce the first output token with the concatenated KV cache. We record the processing time of the algorithms and finally score for each response.
\begin{figure}[t]
    \centering
    \includegraphics[width=\columnwidth]{figs/legend_result.pdf}
    % \vskip -0.2in
    \includegraphics[width=\columnwidth]{figs/results.pdf}
    \caption{Comparison of TTFT ($\downarrow$ Better) and Score ($\uparrow$ Better) using different models on different datasets. }
    \label{fig:ttft-score}
    % \vskip -0.2in
\end{figure}

\figurename~\ref{fig:ttft-score} presents the experimental results of all algorithms across different models and datasets. The results indicate that \sys~consistently outperforms CacheBlend in terms of both TTFT and score across various configurations. \sys-32 reduces TTFT by up to 54.1\% while maintaining a loss of score within 13.6\% compared to prefix caching. Additionally, it is clear that \sys~exhibits a slight decrease in TTFT compared to full reuse, since \sys~is a single-step process. Overall, compared to other algorithms, \sys~achieves the best trade-off between TTFT and score.

\subsection{Sensitivity analysis}
In order to achieve a more profound comprehension of \sys, a subsequent analysis is necessary to ascertain how the number of images impacts overall performance. We divide the dataset of MMDU into 10 groups in terms of the number of images. We evaluate the TTFT and score of \sys~and baselines on each group. The average value of results are shown in \figurename~\ref{fig:10}. The TTFT of \sys~is consistently shorter than that of prefix caching. When the number of images is 10, \sys~achieves 54.7\% reduction in TTFT. Furthermore, the performance of \sys~remains unaffected by the number of images, exhibiting negligible or no accuracy degradation.
\begin{figure}[t]
    \centering
    \includegraphics[width=0.9\columnwidth]{figs/legend_image_num.pdf}
    \vskip -0.2in
    \subfloat[]{
        \includegraphics[width=0.42\columnwidth]{figs/TTFT_all.pdf}
        \label{fig:10a}
    }
    \subfloat[]{
        \includegraphics[width=0.4\columnwidth]{figs/Score_all.pdf}
        \label{fig:10b}
    }
    \caption{The performance of \sys~as the number of images increases. For clarity, we only present the results of \sys-32. Other variants of \sys~show similar patterns.}
    \label{fig:10}
    % \vskip -0.2in
\end{figure}

% \subsection{Latency and throughput performance of InfoBlend}
% To assess Infoblend's latency and throughput performance, we leverage VLLM's OpenAI-compatible API server to simulate real-world user request patterns. We first select $n$ samples from MMDU and pre-generate KV caches for their contexts. Subsequently, we simulate user request behavior by repeatedly sending the user queries along with the cache\_ids of these 
% $n$ samples at a specified request rate over a period of time. by varying the request rate, We measure the latency and throughput across different experimental conditions.

% In Figure, we present a comparison of latency and throughput between InfoBlend and CacheBlend at varying request rates.  Compared to CacheBlend, InfoBlend achieves up to 80\% reduction in TTFT and 2-3 $\times$ improvement in throughput. This gap increases as the request rate rises.


\section{Limitations}
There exist a number of limitations in our work as we described it. In this section we briefly highlight them.
Firstly, our input datasets comprise records of specific size and type (numerical). This currently excludes data with textual and categorical attributes, or tables with varying number of features inside a set of datasets.
Secondly, we currently consider single-input and single-output operator modelling. 
Finally, our proposed NumData2Vec model for data vectorization has a performance limitation, as it cannot deal with datasets bigger than about $3000$ tuples. This is mostly a hardware limitation of off-the-shelf GPUs (with at most 24GB of memory available).

% \section{Conclusion and Future Work}
\section{Conclusion}
In this paper, we presented a novel framework for the modelling of an analytic operator (such as a ML algorithm) when a large number of input data is available and thus no brute-force execution can be performed. We propose a deep learning model, \textit{NumData2Vec}, which transforms a dataset to a lower $k$-dimensional representation space $z$. Our framework produces vector embeddings for the input datasets using \textit{NumData2Vec} and performs a similarity search to identify the most relevant subset of datasets for any unseen input. By modelling the analytic operator based on this selected subset, we are able to accurately predict its output on any given input dataset. In practice, we demonstrated that our framework can accurately model various common algorithms and compared favourably against a similar recent framework \cite{b7Apollo1}, in both accuracy and speedup. Furthermore, we showed that \textit{NumData2Vec} can create different vector representations for datasets from different scenarios. We also demonstrated that \textit{NumData2Vec} can effectively detect when noise is introduced into a dataset.





\section{ACKNOWLEDGMENTS}
This work is partially supported by the project RELAX-DN, funded by the European Union under Horizon Europe 2021-2027 Framework Programme Grant Agreement number 101072456.

\begin{thebibliography}{00}
\bibitem{b1Intdata_quality_for_data_scienceBenjaminHazen} Hazen, B. T., Boone, C. A., Ezell, J. D., and Jones-Farmer, L. A. (2014). Data quality for data science, predictive analytics, and big data in supply chain management: An introduction to the problem and suggestions for research and applications. International Journal of Production Economics, 154, 72-80.
\bibitem{b2DataQualityBigDataChai} Cai, L., and Zhu, Y. (2015). The challenges of data quality and data quality assessment in the big data era. Data science journal, 14, 2-2.
\bibitem{b3IntDataCentricAI} Zha, D., Bhat, Z. P., Lai, K. H., Yang, F., Jiang, Z., Zhong, S., and Hu, X. (2023). Data-centric artificial intelligence: A survey. arXiv preprint arXiv:2303.10158.
\bibitem{b4IntDataCentricAI2} Jakubik, J., Vössing, M., Kühl, N., Walk, J., and Satzger, G. (2024). Data-centric artificial intelligence. Business and Information Systems Engineering, 1-9.
% \bibitem{b5IntApacheSpark} Apache Spark, \href{https://spark.apache.org/}{https://spark.apache.org/}
\bibitem{b1VAE} Kingma, Diederik P., and Max Welling. "An introduction to variational autoencoders." Foundations and Trends® in Machine Learning 12.4 (2019): 307-392.
\bibitem{b2KLDivergence} Kullback, Solomon, and Richard A. Leibler. "On information and sufficiency." The annals of mathematical statistics 22.1 (1951): 79-86.
\bibitem{b3PreNormTL}Xiong, Ruibin, Yunchang Yang, Di He, Kai Zheng, Shuxin Zheng, Chen Xing, Huishuai Zhang, Yanyan Lan, Liwei Wang, and Tieyan Liu (2020, November). On layer normalization in the transformer architecture. In International Conference on Machine Learning (pp. 10524-10533). PMLR.
\bibitem{b4DaggerRW} Rezig, El Kindi, Lei Cao, Giovanni Simonini, Maxime Schoemans, Samuel Madden, Nan Tang, Mourad Ouzzani, and Michael Stonebraker. (2020, January). Dagger: A data (not code) debugger. In CIDR 2020, 10th Conference on Innovative Data Systems Research, Amsterdam, The Netherlands, January 12-15, 2020, Online Proceedings.
\bibitem{b4ReClean} Abdelaal, M., Yayak, A. B., Klede, K., and Schöning, H. (2024, May). ReClean: Reinforcement Learning for Automated Data Cleaning in ML Pipelines. In 2024 IEEE 40th International Conference on Data Engineering Workshops (ICDEW) (pp. 324-330). IEEE.
\bibitem{b5IterCleanRW} Ni, W., Zhang, K., Miao, X., Zhao, X., Wu, Y., and Yin, J. (2024, July). IterClean: An Iterative Data Cleaning Framework with Large Language Models. In ACM Turing Award Celebration Conference 2024 (pp. 100-105).
\bibitem{b6DataShapley} Ghorbani, A., and Zou, J. (2019, May). Data shapley: Equitable valuation of data for machine learning. In International conference on machine learning (pp. 2242-2251). PMLR.
\bibitem{b7Apollo1} Giannakopoulos, I., Tsoumakos, D., and Koziris, N. (2018, October). A Content-Based Approach for Modeling Analytics Operators. In Proceedings of the 27th ACM International Conference on Information and Knowledge Management (pp. 227-236).
\bibitem{b7Apollo2}Bakogiannis, T., Giannakopoulos, I., Tsoumakos, D., and Koziris, N. (2019, June). Apollo: A dataset profiling and operator modeling system. In Proceedings of the 2019 International Conference on Management of Data (pp. 1869-1872).
\bibitem{b8Word2Vec} Mikolov, T. (2013). Efficient estimation of word representations in vector space. arXiv preprint arXiv:1301.3781.
\bibitem{b9Graph2Vec} Narayanan, A., Chandramohan, M., Venkatesan, R., Chen, L., Liu, Y., and Jaiswal, S. (2017). graph2vec: Learning distributed representations of graphs. arXiv preprint arXiv:1707.05005.
\bibitem{b10Doc2Vec} Le, Q., and Mikolov, T. (2014, June). Distributed representations of sentences and documents. In International conference on machine learning (pp. 1188-1196). PMLR.
\bibitem{b11Image2Vec} Dias, L. V., Miranda, P. B., Nascimento, A. C., Cordeiro, F. R., Mello, R. F., and Prudêncio, R. B. (2021). ImageDataset2Vec: An image dataset embedding for algorithm selection. Expert Systems with Applications, 180, 115053.
\bibitem{b12Dataset2Vec} Jomaa, H. S., Schmidt-Thieme, L., and Grabocka, J. (2021). Dataset2vec: Learning dataset meta-features. Data Mining and Knowledge Discovery, 35(3), 964-985.
\bibitem{b13Table2Vec} Zhang, L., Zhang, S., and Balog, K. (2019, July). Table2vec: Neural word and entity embeddings for table population and retrieval. In Proceedings of the 42nd international ACM SIGIR conference on research and development in information retrieval (pp. 1029-1032).
\bibitem{b14TransTab}Wang, Z., and Sun, J. (2022). Transtab: Learning transferable tabular transformers across tables. Advances in Neural Information Processing Systems, 35, 2902-2915.
\bibitem{b15FTransformer} Gorishniy, Y., Rubachev, I., Khrulkov, V., and Babenko, A. (2021). Revisiting deep learning models for tabular data. Advances in Neural Information Processing Systems, 34, 18932-18943.
\bibitem{b16TabTransformer} Huang, X., Khetan, A., Cvitkovic, M., and Karnin, Z. (2020). Tabtransformer: Tabular data modeling using contextual embeddings. arXiv preprint arXiv:2012.06678.
\bibitem{b17KMeansa} Lloyd, S. (1982). Least squares quantization in PCM. IEEE transactions on information theory, 28(2), 129-137.
\bibitem{b17KMeansb} MacQueen, J. (1967, June). Some methods for classification and analysis of multivariate observations. In Proceedings of the fifth Berkeley symposium on mathematical statistics and probability (Vol. 1, No. 14, pp. 281-297).
\bibitem{b18Silhouettes} Rousseeuw, P. J. (1987). Silhouettes: a graphical aid to the interpretation and validation of cluster analysis. Journal of computational and applied mathematics, 20, 53-65.
\bibitem{b19SOALA} Yu, K., Wu, X., Ding, W., and Pei, J. (2016). Scalable and accurate online feature selection for big data. ACM Transactions on Knowledge Discovery from Data (TKDD), 11(2), 1-39.preprint arXiv:2108.05935.
\bibitem{b20tfdata} Murray, D. G., Simsa, J., Klimovic, A., and Indyk, I. (2021). tf. data: A machine learning data processing framework. arXiv preprint arXiv:2101.12127.
\bibitem{b21HPCdataset} Hebrail,Georges and Berard,Alice. (2012). Individual Household Electric Power Consumption. UCI Machine Learning Repository. \href{https://doi.org/10.24432/C58K54}{https://doi.org/10.24432/C58K54}.
\bibitem{b22AdultDataset} Becker,Barry and Kohavi,Ronny. (1996). Adult. UCI Machine Learning Repository. \href{https://doi.org/10.24432/C5XW20}{https://doi.org/10.24432/C5XW20}.
\bibitem{b23StockMarketDataset} Stock Market Dataset (2020), \href{https://www.kaggle.com/datasets/jacksoncrow/stock-market-dataset/data}{https://www.kaggle.com/datasets/jacksoncrow/stock-market-dataset/data}.
\bibitem{b23WeatherDataset} Weather Dataset (2020), \href{https://www.kaggle.com/datasets/selfishgene/historical-hourly-weather-data}{https://www.kaggle.com/datasets/selfishgene/historical-hourly-weather-data}.
\end{thebibliography}


% \section{Ease of Use}

% \subsection{Maintaining the Integrity of the Specifications}

% The IEEEtran class file is used to format your paper and style the text. All margins, 
% column widths, line spaces, and text fonts are prescribed; please do not 
% alter them. You may note peculiarities. For example, the head margin
% measures proportionately more than is customary. This measurement 
% and others are deliberate, using specifications that anticipate your paper 
% as one part of the entire proceedings, and not as an independent document. 
% Please do not revise any of the current designations.

% \section{Prepare Your Paper Before Styling}
% Before you begin to format your paper, first write and save the content as a 
% separate text file. Complete all content and organizational editing before 
% formatting. Please note sections \ref{AA}--\ref{SCM} below for more information on 
% proofreading, spelling and grammar.

% Keep your text and graphic files separate until after the text has been 
% formatted and styled. Do not number text heads---{\LaTeX} will do that 
% for you.

% \subsection{Abbreviations and Acronyms}\label{AA}
% Define abbreviations and acronyms the first time they are used in the text, 
% even after they have been defined in the abstract. Abbreviations such as 
% IEEE, SI, MKS, CGS, ac, dc, and rms do not have to be defined. Do not use 
% abbreviations in the title or heads unless they are unavoidable.

% \subsection{Units}
% \begin{itemize}
% \item Use either SI (MKS) or CGS as primary units. (SI units are encouraged.) English units may be used as secondary units (in parentheses). An exception would be the use of English units as identifiers in trade, such as ``3.5-inch disk drive''.
% \item Avoid combining SI and CGS units, such as current in amperes and magnetic field in oersteds. This often leads to confusion because equations do not balance dimensionally. If you must use mixed units, clearly state the units for each quantity that you use in an equation.
% \item Do not mix complete spellings and abbreviations of units: ``Wb/m\textsuperscript{2}'' or ``webers per square meter'', not ``webers/m\textsuperscript{2}''. Spell out units when they appear in text: ``. . . a few henries'', not ``. . . a few H''.
% \item Use a zero before decimal points: ``0.25'', not ``.25''. Use ``cm\textsuperscript{3}'', not ``cc''.)
% \end{itemize}

% \subsection{Equations}
% Number equations consecutively. To make your 
% equations more compact, you may use the solidus (~/~), the exp function, or 
% appropriate exponents. Italicize Roman symbols for quantities and variables, 
% but not Greek symbols. Use a long dash rather than a hyphen for a minus 
% sign. Punctuate equations with commas or periods when they are part of a 
% sentence, as in:
% \begin{equation}
% a+b=\gamma\label{eq}
% \end{equation}

% Be sure that the 
% symbols in your equation have been defined before or immediately following 
% the equation. Use ``\eqref{eq}'', not ``Eq.~\eqref{eq}'' or ``equation \eqref{eq}'', except at 
% the beginning of a sentence: ``Equation \eqref{eq} is . . .''

% \subsection{\LaTeX-Specific Advice}

% Please use ``soft'' (e.g., \verb|\eqref{Eq}|) cross references instead
% of ``hard'' references (e.g., \verb|(1)|). That will make it possible
% to combine sections, add equations, or change the order of figures or
% citations without having to go through the file line by line.

% Please don't use the \verb|{eqnarray}| equation environment. Use
% \verb|{align}| or \verb|{IEEEeqnarray}| instead. The \verb|{eqnarray}|
% environment leaves unsightly spaces around relation symbols.

% Please note that the \verb|{subequations}| environment in {\LaTeX}
% will increment the main equation counter even when there are no
% equation numbers displayed. If you forget that, you might write an
% article in which the equation numbers skip from (17) to (20), causing
% the copy editors to wonder if you've discovered a new method of
% counting.

% {\BibTeX} does not work by magic. It doesn't get the bibliographic
% data from thin air but from .bib files. If you use {\BibTeX} to produce a
% bibliography you must send the .bib files. 

% {\LaTeX} can't read your mind. If you assign the same label to a
% subsubsection and a table, you might find that Table I has been cross
% referenced as Table IV-B3. 

% {\LaTeX} does not have precognitive abilities. If you put a
% \verb|\label| command before the command that updates the counter it's
% supposed to be using, the label will pick up the last counter to be
% cross referenced instead. In particular, a \verb|\label| command
% should not go before the caption of a figure or a table.

% Do not use \verb|\nonumber| inside the \verb|{array}| environment. It
% will not stop equation numbers inside \verb|{array}| (there won't be
% any anyway) and it might stop a wanted equation number in the
% surrounding equation.

% \subsection{Some Common Mistakes}\label{SCM}
% \begin{itemize}
% \item The word ``data'' is plural, not singular.
% \item The subscript for the permeability of vacuum $\mu_{0}$, and other common scientific constants, is zero with subscript formatting, not a lowercase letter ``o''.
% \item In American English, commas, semicolons, periods, question and exclamation marks are located within quotation marks only when a complete thought or name is cited, such as a title or full quotation. When quotation marks are used, instead of a bold or italic typeface, to highlight a word or phrase, punctuation should appear outside of the quotation marks. A parenthetical phrase or statement at the end of a sentence is punctuated outside of the closing parenthesis (like this). (A parenthetical sentence is punctuated within the parentheses.)
% \item A graph within a graph is an ``inset'', not an ``insert''. The word alternatively is preferred to the word ``alternately'' (unless you really mean something that alternates).
% \item Do not use the word ``essentially'' to mean ``approximately'' or ``effectively''.
% \item In your paper title, if the words ``that uses'' can accurately replace the word ``using'', capitalize the ``u''; if not, keep using lower-cased.
% \item Be aware of the different meanings of the homophones ``affect'' and ``effect'', ``complement'' and ``compliment'', ``discreet'' and ``discrete'', ``principal'' and ``principle''.
% \item Do not confuse ``imply'' and ``infer''.
% \item The prefix ``non'' is not a word; it should be joined to the word it modifies, usually without a hyphen.
% \item There is no period after the ``et'' in the Latin abbreviation ``et al.''.
% \item The abbreviation ``i.e.'' means ``that is'', and the abbreviation ``e.g.'' means ``for example''.
% \end{itemize}
% An excellent style manual for science writers is \cite{b7}.

% \subsection{Authors and Affiliations}
% \textbf{The class file is designed for, but not limited to, six authors.} A 
% minimum of one author is required for all conference articles. Author names 
% should be listed starting from left to right and then moving down to the 
% next line. This is the author sequence that will be used in future citations 
% and by indexing services. Names should not be listed in columns nor group by 
% affiliation. Please keep your affiliations as succinct as possible (for 
% example, do not differentiate among departments of the same organization).

% \subsection{Identify the Headings}
% Headings, or heads, are organizational devices that guide the reader through 
% your paper. There are two types: component heads and text heads.

% Component heads identify the different components of your paper and are not 
% topically subordinate to each other. Examples include Acknowledgments and 
% References and, for these, the correct style to use is ``Heading 5''. Use 
% ``figure caption'' for your Figure captions, and ``table head'' for your 
% table title. Run-in heads, such as ``Abstract'', will require you to apply a 
% style (in this case, italic) in addition to the style provided by the drop 
% down menu to differentiate the head from the text.

% Text heads organize the topics on a relational, hierarchical basis. For 
% example, the paper title is the primary text head because all subsequent 
% material relates and elaborates on this one topic. If there are two or more 
% sub-topics, the next level head (uppercase Roman numerals) should be used 
% and, conversely, if there are not at least two sub-topics, then no subheads 
% should be introduced.

% \subsection{Figures and Tables}
% \paragraph{Positioning Figures and Tables} Place figures and tables at the top and 
% bottom of columns. Avoid placing them in the middle of columns. Large 
% figures and tables may span across both columns. Figure captions should be 
% below the figures; table heads should appear above the tables. Insert 
% figures and tables after they are cited in the text. Use the abbreviation 
% ``Fig.~\ref{fig}'', even at the beginning of a sentence.

% \begin{table}[htbp]
% \caption{Table Type Styles}
% \begin{center}
% \begin{tabular}{|c|c|c|c|}
% \hline
% \textbf{Table}&\multicolumn{3}{|c|}{\textbf{Table Column Head}} \\
% \cline{2-4} 
% \textbf{Head} & \textbf{\textit{Table column subhead}}& \textbf{\textit{Subhead}}& \textbf{\textit{Subhead}} \\
% \hline
% copy& More table copy$^{\mathrm{a}}$& &  \\
% \hline
% \multicolumn{4}{l}{$^{\mathrm{a}}$Sample of a Table footnote.}
% \end{tabular}
% \label{tab1}
% \end{center}
% \end{table}

% \begin{figure}[htbp]
% \centerline{\includegraphics{fig1.png}}
% \caption{Example of a figure caption.}
% \label{fig}
% \end{figure}

% Figure Labels: Use 8 point Times New Roman for Figure labels. Use words 
% rather than symbols or abbreviations when writing Figure axis labels to 
% avoid confusing the reader. As an example, write the quantity 
% ``Magnetization'', or ``Magnetization, M'', not just ``M''. If including 
% units in the label, present them within parentheses. Do not label axes only 
% with units. In the example, write ``Magnetization (A/m)'' or ``Magnetization 
% \{A[m(1)]\}'', not just ``A/m''. Do not label axes with a ratio of 
% quantities and units. For example, write ``Temperature (K)'', not 
% ``Temperature/K''.

% \section*{Acknowledgment}

% The preferred spelling of the word ``acknowledgment'' in America is without 
% an ``e'' after the ``g''. Avoid the stilted expression ``one of us (R. B. 
% G.) thanks $\ldots$''. Instead, try ``R. B. G. thanks$\ldots$''. Put sponsor 
% acknowledgments in the unnumbered footnote on the first page.

% \section*{References}

% Please number citations consecutively within brackets \cite{b1}. The 
% sentence punctuation follows the bracket \cite{b2}. Refer simply to the reference 
% number, as in \cite{b3}---do not use ``Ref. \cite{b3}'' or ``reference \cite{b3}'' except at 
% the beginning of a sentence: ``Reference \cite{b3} was the first $\ldots$''

% Number footnotes separately in superscripts. Place the actual footnote at 
% the bottom of the column in which it was cited. Do not put footnotes in the 
% abstract or reference list. Use letters for table footnotes.

% Unless there are six authors or more give all authors' names; do not use 
% ``et al.''. Papers that have not been published, even if they have been 
% submitted for publication, should be cited as ``unpublished'' \cite{b4}. Papers 
% that have been accepted for publication should be cited as ``in press'' \cite{b5}. 
% Capitalize only the first word in a paper title, except for proper nouns and 
% element symbols.

% For papers published in translation journals, please give the English 
% citation first, followed by the original foreign-language citation \cite{b6}.

% \begin{thebibliography}{00}
% \bibitem{b1} G. Eason, B. Noble, and I. N. Sneddon, ``On certain integrals of Lipschitz-Hankel type involving products of Bessel functions,'' Phil. Trans. Roy. Soc. London, vol. A247, pp. 529--551, April 1955.
% \bibitem{b2} J. Clerk Maxwell, A Treatise on Electricity and Magnetism, 3rd ed., vol. 2. Oxford: Clarendon, 1892, pp.68--73.
% \bibitem{b3} I. S. Jacobs and C. P. Bean, ``Fine particles, thin films and exchange anisotropy,'' in Magnetism, vol. III, G. T. Rado and H. Suhl, Eds. New York: Academic, 1963, pp. 271--350.
% \bibitem{b4} K. Elissa, ``Title of paper if known,'' unpublished.
% \bibitem{b5} R. Nicole, ``Title of paper with only first word capitalized,'' J. Name Stand. Abbrev., in press.
% \bibitem{b6} Y. Yorozu, M. Hirano, K. Oka, and Y. Tagawa, ``Electron spectroscopy studies on magneto-optical media and plastic substrate interface,'' IEEE Transl. J. Magn. Japan, vol. 2, pp. 740--741, August 1987 [Digests 9th Annual Conf. Magnetics Japan, p. 301, 1982].
% \bibitem{b7} M. Young, The Technical Writer's Handbook. Mill Valley, CA: University Science, 1989.
% \end{thebibliography}
% \vspace{12pt}
% \color{red}
% IEEE conference templates contain guidance text for composing and formatting conference papers. Please ensure that all template text is removed from your conference paper prior to submission to the conference. Failure to remove the template text from your paper may result in your paper not being published.

\end{document}
