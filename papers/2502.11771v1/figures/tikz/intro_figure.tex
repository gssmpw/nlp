\begin{tikzpicture}[
  % Global styles for visual consistency
  >=stealth,
  thick,
  box/.style={
    draw,
    rectangle,
    rounded corners=2pt,
    minimum width=7mm,
    minimum height=5mm,
    align=center,
    font=\sffamily
  },
  greenBox/.style={
    box,
    draw=cyan!50!gray,
    fill=cyan!50!gray,
    inner sep=0pt,
  },
  blackBox/.style={
    box,
    draw=black!20,
    fill=black!20
  },
  redBox/.style={
  box,
  draw=red!50!gray,    
  fill=red!50!gray,         
  inner sep=0pt,
  },
  hollowBox/.style={
    box,
    draw=white,
  },
  arrow/.style={
    ->,
    shorten >=1pt,
    shorten <=1pt
  }
]

%%%%%%%%%%%%%%%%%%%%%%%%%%%%%%%%%%%%%%%%%%%%%%%%%%%%%%%%%%%%
%  1) EQUATION SECTION (left) -- three “rows” + top row
%%%%%%%%%%%%%%%%%%%%%%%%%%%%%%%%%%%%%%%%%%%%%%%%%%%%%%%%%%%%

% -- Numbers (showing 5 + 8 = 16)
\node[hollowBox] (g1) at (0,0)  {};
\node[hollowBox] (g2) [right=0.3 of g1] {\textbf{+}};
\node[hollowBox] (g3) [right=0.3 of g2] {};
\node[hollowBox] (g4) [right=0.3 of g3] {\textbf{=}};
\node[hollowBox] (g5) [right=0.3 of g4] {};
\node[cyan!50!black] at ($(g1)$) {\textbf{5}};
\node[cyan!50!black] at ($(g3)$) {\textbf{8}};
\node[red!50!black] at ($(g5)$) {\textbf{16}};

% -- First row of boxes (all “…”)
\node[blackBox] (g6)  at ($(g1)+(0,0.9)$)    {};
\node[blackBox] (g7)  [right=0.3 of g6]      {};
\node[blackBox] (g8)  [right=0.3 of g7]      {};
\node[blackBox] (g9)  [right=0.3 of g8]      {};
\node[redBox] (g10) [right=0.3 of g9]      {};

% Just demonstrate that more black boxes could continue:
\node[hollowBox] (b1) at ($(g6)+(0,0.9)$) {...};
\node[hollowBox] (b2) [right=0.3 of b1]  {...};
\node[hollowBox] (b3) [right=0.3 of b2]  {...};
\node[hollowBox] (b4) [right=0.3 of b3]  {...};
\node[hollowBox] (b5) [right=0.3 of b4] {...};

% -- Second row of green boxes (all “…”)
\node[blackBox] (g11) at ($(b1)+(0,0.9)$)    {};
\node[blackBox] (g12) [right=0.3 of g11]     {};
\node[blackBox] (g13) [right=0.3 of g12]     {};
\node[blackBox] (g14) [right=0.3 of g13]     {};
\node[redBox] (g15) [right=0.3 of g14]     {};

% Just demonstrate that more black boxes could continue:
\node[hollowBox] (b6) at ($(g11)+(0,0.9)$) {...};
\node[hollowBox] (b7) [right=0.3 of b6]  {...};
\node[hollowBox] (b8) [right=0.3 of b7]  {...};
\node[hollowBox] (b9) [right=0.3 of b8]  {...};
\node[hollowBox] (b10) [right=0.3 of b9] {...};

% -- Third row (the left portion only really needs one “13,” others can be blank)
\node[greenBox] (g16) at ($(b6)+(0,0.9)$)   {};
\node[greenBox] (g17) [right=0.3 of g16]    {};
\node[greenBox] (g18) [right=0.3 of g17]    {};
\node[greenBox] (g19) [right=0.3 of g18]    {}; % we transition to black boxes
\node[blackBox] (g20) [right=0.3 of g19]    {}; % more black boxes to the right

% Label “13” above the third green box in the top row:
\node[hollowBox] (g21) at ($(g16)+(0,0.9)$)  {};
\node[hollowBox] (g22) [right=0.3 of g21] {};
\node[hollowBox] (g23) [right=0.3 of g22] {};
\node[hollowBox] (g24) [right=0.3 of g23] {};
\node[hollowBox] (g25) [right=0.3 of g24] {};
\node[cyan!50!black] at ($(g24)$) {\textbf{13}};

%%%%%%%%%%%%%%%%%%%%%%%%%%%%%%%%%%%%%%%%%%%%%%%%%%%%%%%%%%%%
%  2) “…” placeholders
%%%%%%%%%%%%%%%%%%%%%%%%%%%%%%%%%%%%%%%%%%%%%%%%%%%%%%%%%%%%

% Just demonstrate that more black boxes could continue:
\node[hollowBox] (d1) [right=0. of g5] {...};
\node[hollowBox] (d2) [right=0. of g10]  {...};
\node[hollowBox] (d3) [right=0. of g15]  {...};
\node[hollowBox] (d4) [right=0. of g20]  {...};
\node[hollowBox] (d5) [right=0. of g25] {};


%%%%%%%%%%%%%%%%%%%%%%%%%%%%%%%%%%%%%%%%%%%%%%%%%%%%%%%%%%%%
%  3) ANSWER SECTION -- final columns (marked “invalid”)
%%%%%%%%%%%%%%%%%%%%%%%%%%%%%%%%%%%%%%%%%%%%%%%%%%%%%%%%%%%%

% Far-right column of red boxes
\node[hollowBox] (r1) [right=0.0 of d1]  {\small Ans:};
\node[blackBox] (r2) [right=0.09 of d2]     {};
\node[hollowBox] (rb2) at ($(r2)+(0,0.9)$) {...};
\node[redBox] (r3) [right=0.09 of d3]     {};
\node[hollowBox] (rb3) at ($(r3)+(0,0.9)$) {...};
\node[blackBox] (r4) [right=0.09 of d4]     {};
\node[hollowBox] (r5) [right=0.09 of d5] {};


% Another column of red boxes to the left of that
\node[hollowBox] (r6) [right=0.21 of r1]  {};
\node[redBox] (r7) [right=0.3 of r2]     {};
\node[hollowBox] (rb7) at ($(r7)+(0,0.9)$) {...};
\node[redBox] (r8) [right=0.3 of r3]     {};
\node[hollowBox] (rb8) at ($(r8)+(0,0.9)$) {...};
\node[redBox] (r9) [right=0.3 of r4]     {};
\node[hollowBox] (r10) [right=0.3 of r5] {};
\node[red!50!black] at ($(r6)$) {\textbf{13}};
\node[red!50!black] at ($(r10)$) {\textbf{invalid}};

%%%%%%%%%%%%%%%%%%%%%%%%%%%%%%%%%%%%%%%%%%%%%%%%%%%%%%%%%%%%
%  4) DRAW ARROWS UP THROUGH EACH COLUMN
%%%%%%%%%%%%%%%%%%%%%%%%%%%%%%%%%%%%%%%%%%%%%%%%%%%%%%%%%%%%

% Green
\foreach \i/\j in {1/6,3/8, 19/24}{%
  \draw[arrow,cyan!50!black] (g\i.north) -- (g\j.south);
}

\foreach \i/\j in {6/1,8/3, 11/6,13/8}{%
  \draw[arrow,cyan!50!black] (g\i.north) -- (b\j.south);
}

\foreach \i/\j in {1/11,3/13, 6/16,8/18}{%
  \draw[arrow,cyan!50!black] (b\i.north) -- (g\j.south);
}

\foreach \i/\j in {16/17,17/18,18/19}{%
  \draw[arrow,cyan!50!black] (g\i.east) -- (g\j.west);
}

% Black columns of equation
\foreach \i/\j in {2/7,4/9, 16/21,17/22,18/23}{%
  \draw[arrow,black] (g\i.north) -- (g\j.south);
}

\foreach \i/\j in {7/2,9/4, 12/7, 14/9, 15/10}{%
  \draw[arrow,black] (g\i.north) -- (b\j.south);
}

\foreach \i/\j in {2/12,4/14,5/15, 7/17,9/19,10/20}{%
  \draw[arrow,black] (b\i.north) -- (g\j.south);
}

\foreach \i/\j in {6/7,7/8,8/9,9/10, 11/12,12/13,13/14,14/15, 19/20}{%
  \draw[arrow, black] (g\i.east) -- (g\j.west);
}

% Red columns of equation
\foreach \i/\j in {5/10}{%
  \draw[arrow,red!50!black] (g\i.north) -- (g\j.south);
}

\foreach \i/\j in {10/5}{%
  \draw[arrow,red!50!black] (g\i.north) -- (b\j.south);
}

\foreach \i/\j in {5/15}{%
  \draw[arrow,red!50!black] (b\i.north) -- (g\j.south);
}

% Right side of equations black
% Black columns of equation
\foreach \i/\j in {1/2,4/5}{%
  \draw[arrow,black] (r\i.north) -- (r\j.south);
}

\foreach \i/\j in {2/7,4/9}{%
  \draw[arrow, black] (r\i.east) -- (r\j.west);
}

\foreach \i/\j in {2/2,3/3}{%
  \draw[arrow,black] (r\i.north) -- (rb\j.south);
}

\foreach \i/\j in {2/3,3/4}{%
  \draw[arrow,black] (rb\i.north) -- (r\j.south);
}

\foreach \i/\j in {6/7,9/10}{%
  \draw[arrow,red!50!black] (r\i.north) -- (r\j.south);
}

\foreach \i/\j in {7/7,8/8}{%
  \draw[arrow,red!50!black] (r\i.north) -- (rb\j.south);
}

\foreach \i/\j in {7/8,8/9}{%
  \draw[arrow,red!50!black] (rb\i.north) -- (r\j.south);
}


\foreach \i/\j in {3/8}{%
  \draw[arrow, red!50!black] (r\i.east) -- (r\j.west);
}

\end{tikzpicture}