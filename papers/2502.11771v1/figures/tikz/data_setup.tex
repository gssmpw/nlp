\begin{tikzpicture}[
    % Use sans-serif at normal size for a cleaner look
    font=\sffamily\scriptsize,
    >=Stealth,
    node distance=4mm,
    thick,
    % A box style with gently rounded corners, a bit more spacing
    box/.style={
      draw,
      rounded corners=2mm,
      align=left,
      inner sep=4pt,
      text width=5.5cm
    }
]

%%%%%%%%%%%%%%%%%%%%%%%%%%%%%%%%%%%%%%%%%%%%%%%%%%%%%%%%
%                  TEMPLATES SECTION                  %
%%%%%%%%%%%%%%%%%%%%%%%%%%%%%%%%%%%%%%%%%%%%%%%%%%%%%%%%

% (1) "Templates" heading box
\node[box, draw=white, text width=5.5cm] (tempHeading) {Template 1};

% (2) Three color-coded boxes below heading
\node[box, fill=teal!15, draw=teal!10, text width=5.5cm, below=0 of tempHeading] (tempA)
  {[name] initially has [num1] [object]. After [verb] [num2] more [object], how many [object] does [pronoun] have?};
\node[box, fill=orange!20, text width=5.5cm, below=0.1 of tempA] (tempB)
  {To solve this, we add [num1] + [num2] = [result].\\Thus, [name] has a total of [answer] [object].};
\node[box, fill=pink!20, text width=5.5cm, below=0.1 of tempB] (tempC)
  {The above reasoning is:};

\node[
  rounded corners=3mm,
  draw,
  fill=white,
  fit=(tempHeading)(tempA)(tempB)(tempC),
  inner sep=4pt,
  shift={(-0.3cm, 0.3cm)}
] (templateC) {};

\node[
  rounded corners=3mm,
  draw,
  fill=white,
  fit=(tempHeading)(tempA)(tempB)(tempC),
  inner sep=4pt,
  shift={(-0.15cm, 0.15cm)}
] (templateMainB) {};

\node[
  rounded corners=3mm,
  thick,
  draw,
  fill=white,
  fit=(tempHeading)(tempA)(tempB)(tempC),
  inner sep=4pt
] (templateMainA) {};

\node[box, draw=white, text width=5.5cm] (tempHeading) {\bfseries Template 1};

\node[box, fill=teal!15, draw=teal!15, text width=5.5cm, below=0 of tempHeading] (tempA)
  {[name] initially has [num1] [object].\\After [verb] [num2] more object], how many \\\relax[object] does [pronoun] have?};
\node[box, fill=orange!20, draw=orange!20, text width=5.5cm, below=0.1 of tempA] (tempB)
  {To solve this, we add [num1] + [num2] = \textcolor{cyan!50!black}{\textbf{[result]}}.\\Thus, [name] has a total of \textcolor{cyan!50!black}{\textbf{[answer]}} [object].};
\node[box, fill=blue!5, draw=blue!5, text width=5.5cm, below=0.1 of tempB] (tempC)
  {The above reasoning is:};

%%%%%%%%%%%%%%%%%%%%%%%%%%%%%%%%%%%%%%%%%%%%%%%%%%%%%%%%
%                   SPLIT COORDINATE                  %
%%%%%%%%%%%%%%%%%%%%%%%%%%%%%%%%%%%%%%%%%%%%%%%%%%%%%%%%

\coordinate (split) at ($(templateMainA.east)+(0.5,0.)$);

\draw[-] (templateMainA.east) -- (split);

%%%%%%%%%%%%%%%%%%%%%%%%%%%%%%%%%%%%%%%%%%%%%%%%%%%%%%%%
%                  CLEAN PROMPT SECTION               %
%%%%%%%%%%%%%%%%%%%%%%%%%%%%%%%%%%%%%%%%%%%%%%%%%%%%%%%%

\node[
  rounded corners=3mm,
  thick,
  draw=none,
  fill=none,
  fit=(tempHeading)(tempA)(tempB)(tempC),
  inner sep=4pt
] (cleanFrameNode) at ($(split)+(4.0,1.8)$) {};

\node[box, draw=white, text width=5.5cm, anchor=north east]
      (cleanHeading) at ($(cleanFrameNode)+(2.5, 1.5)$)
      {\bfseries Clean prompt};

\node[box, fill=teal!15, draw=teal!15, text width=5.5cm, below=0. of cleanHeading] (cleanA)
  {Jane initially has 5 apples. After buying 8 more apples, how many apples does she have?};
\node[box, fill=orange!20, draw=orange!20, text width=5.5cm, below=0.1 of cleanA] (cleanB)
  {To solve this, we add 5 + 8 = \textcolor{red!50!gray}{\textbf{16}}.\\Thus, Jane has a total of \textcolor{green!40!black}{\textbf{13}} apples.};
\node[box, fill=blue!5, draw=blue!5, text width=5.5cm, below=0.1 of cleanB] (cleanC)
  {The above reasoning is:};

\node[
  rounded corners=3mm,
  thick,
  draw,
  fill=none,
  fit=(cleanHeading)(cleanA)(cleanB)(cleanC),
  inner sep=4pt
] (cleanFrame) {};

\draw[->] (split) |- (cleanFrame.west);

%%%%%%%%%%%%%%%%%%%%%%%%%%%%%%%%%%%%%%%%%%%%%%%%%%%%%%%%
%                 CORRUPT PROMPT SECTION              %
%%%%%%%%%%%%%%%%%%%%%%%%%%%%%%%%%%%%%%%%%%%%%%%%%%%%%%%%

\node[
  rounded corners=3mm,
  thick,
  draw=none,
  fill=none,
  fit=(tempHeading)(tempA)(tempB)(tempC),
  inner sep=4pt
] (corrFrameNode) at ($(split)+(4.0,-2.4)$) {};

\node[box, draw=white, text width=5.5cm, anchor=north east]
      (corrHeading) at ($(corrFrameNode)+(2.5, 2.05)$)
      {\bfseries Corrupt prompt};

\node[box, fill=teal!15, draw=teal!15, text width=5.5cm, below=0. of corrHeading] (corrA)
  {Jane initially has 5 apples. After buying 8 more apples, how many apples does she have?};
\node[box, fill=orange!20, draw=orange!20, text width=5.5cm, below=0.1 of corrA] (corrB)
  {To solve this, we add 5 + 8 = \textcolor{green!40!black}{\textbf{13}}.\\Thus, Jane has a total of \textcolor{green!40!black}{\textbf{13}} apples.};
\node[box, fill=blue!5, draw=blue!5, text width=5.5cm, below=0.1 of corrB] (corrC)
  {The above reasoning is:};

\node[
  rounded corners=3mm,
  thick,
  draw,
  fill=none,
  fit=(corrHeading)(corrA)(corrB)(corrC),
  inner sep=4pt
] (corrFrame) {};

\draw[->] (split) |- (corrFrame.west);

%%%%%%%%%%%%%%%%%%%%%%%%%%%%%%%%%%%%%%%%%%%%%%%%%%%%%%%%
%             "INCORRECT" AND "CORRECT" LABELS        %
%%%%%%%%%%%%%%%%%%%%%%%%%%%%%%%%%%%%%%%%%%%%%%%%%%%%%%%%

\node[box, fill=blue!5, text width=0.7cm,
      right=5mm of cleanFrame.east] (incorrectBox)
      {\centering invalid};
\draw[->,thick] (cleanFrame.east) -- (incorrectBox.west);

\node[box, fill=blue!5, text width=0.5cm,
      right=5mm of corrFrame.east] (correctBox)
      {\centering valid};
\draw[->,thick] (corrFrame.east) -- (correctBox.west);

\end{tikzpicture}
