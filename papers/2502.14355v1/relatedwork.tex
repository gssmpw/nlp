\section{Related Work}
\label{sec:2}
In this section, we present a brief review of the related works on seismic data noise suppression, with a focus on methods based on model optimization and deep learning.

\subsection{Model Optimization-based Seismic Data Noise Suppression Methods}

Seismic data noise suppression is fundamentally an ill-posed inverse problem \cite{qian2023improved,wang2015improved,li2022simultaneous}. To address this challenge, model optimization-based approaches leverage prior knowledge of seismic data to construct regularization models that refine the solution space, such as total variation (TV) \cite{osher2005iterative}, sparse representation \cite{aharon2006k}, and LR matrix approximation \cite{cai2010singular}. It has been proven that prior-based modeling is of great significance in seismic data noise suppression \cite{chen2021statistics,cheng2015application,chen2016simultaneous,qian2023improved,liu2024simultaneous}. For example, Chen $\emph{et al}.$ \cite{chen2021statistics} proposed a statistics-guided residual dictionary learning (SGRDL) method for seismic data noise removal. Cheng $\emph{et al}.$ \cite{cheng2015application} adopted a nuclear norm constraint under the framework of the LR matrix approximation for seismic data noise suppression. Chen $\emph{et al}.$ \cite{chen2016simultaneous} proposed a damped rank reduction (DRR) approach for seismic data denoising, which employs the block Hankel matrix to decompose the noisy data space into signal and noise subspace. Due to the LR structure embedded in the multidimensional seismic data, LR tensor recovery methods have received considerable attention in seismic data noise suppression \cite{feng2021low,feng2021seismic,qian2023improved,liu2024simultaneous}. For instance, Feng $\emph{et al}.$ \cite{feng2021seismic} employed the CANDECOMP/PARAFAC (CP) decomposition \cite{carroll1970analysis}, integrating a TV constraint for seismic data noise suppression. Qian $\emph{et al}.$ \cite{qian2023improved} proposed a tensor model called UTV-LRTA for seismic data noise suppression, which integrates LR tensor approximation with a unidirectional TV regularizer.


\subsection{Deep Learning-based Seismic Data Noise Suppression Methods}

Deep learning techniques have demonstrated significant potential in seismic data noise suppression \cite{zhu2019seismic,wang2020generative,wang2024efgw,wang2022learning,xu2023deep,qian2023unsupervised,xu2023s2s} by leveraging an end-to-end training approach for deep neural networks (DNNs) \cite{lecun2015deep}. For example, Wang $\emph{et al}.$ \cite{wang2020generative} adopted a generative adversarial network (GAN) \cite{goodfellow2020generative} for seismic denoising, where the generator is used to remove noise and the discriminator guides the generator to restore structural information. The EFGW-UNet method \cite{wang2024efgw} utilized an edge-feature-guided wavelet U-Net \cite{ronneberger2015u} to preserve finer details of effective signals while suppressing noise. However, these supervised learning algorithms \cite{liu2021unsupervised,qiu2021deep} require extensive seismic field data, which is challenging to obtain due to high acquisition costs and limited availability, thereby restricting their applicability. In recent years, various self-supervised learning methods have been developed for seismic data noise suppression \cite{meng2021self,xu2023deep,qian2023unsupervised,xu2023s2s}. For instance, Xu $\emph{et al}.$ \cite{xu2023deep} proposed a deep nonlocal regularizer (DNLR) method for seismic data noise suppression, which combines the learning capability of DNNs with the generalization power of handcrafted regularizers. Qian $\emph{et al}.$ \cite{qian2023unsupervised} introduced a footprint removal network (FR-Net) by regularizing a deep convolutional autoencoder using the UTV. The S2S-WTV method \cite{xu2023s2s} leveraged the Self2Self (S2S) learning framework \cite{quan2020self2self} with a trace-wise masking strategy and weighted TV for seismic data denoising.