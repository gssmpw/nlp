%%%%%%%% ICML 2025 EXAMPLE LATEX SUBMISSION FILE %%%%%%%%%%%%%%%%%

\documentclass{article}

% Recommended, but optional, packages for figures and better typesetting:
\usepackage{microtype}
\usepackage{graphicx}
\usepackage{subfigure}
\usepackage{booktabs} % for professional tables

% hyperref makes hyperlinks in the resulting PDF.
% If your build breaks (sometimes temporarily if a hyperlink spans a page)
% please comment out the following usepackage line and replace
% \use package{icml2025} with \usepackage[nohyperref]{icml2025} above.
\usepackage{hyperref}


% Attempt to make hyperref and algorithmic work together better:
\newcommand{\theHalgorithm}{\arabic{algorithm}}

% Use the following line for the initial blind version submitted for review:
% \usepackage{icml2025}

% If accepted, instead use the following line for the camera-ready submission:
\usepackage[arxiv]{icml2025}

% For theorems and such
\usepackage{amsmath}
\usepackage{amssymb}
\usepackage{mathtools}
\usepackage{amsthm}
\usepackage{tcolorbox}
\usepackage{colortbl}
% if you use cleveref..
\usepackage[capitalize,noabbrev]{cleveref}

%%%%%%%%%%%%%%%%%%%%%%%%%%%%%%%%
% THEOREMS
%%%%%%%%%%%%%%%%%%%%%%%%%%%%%%%%
%
% --- inline annotations
%
\newcommand{\red}[1]{{\color{red}#1}}
\newcommand{\todo}[1]{{\color{red}#1}}
\newcommand{\TODO}[1]{\textbf{\color{red}[TODO: #1]}}
% --- disable by uncommenting  
% \renewcommand{\TODO}[1]{}
% \renewcommand{\todo}[1]{#1}



\newcommand{\VLM}{LVLM\xspace} 
\newcommand{\ours}{PeKit\xspace}
\newcommand{\yollava}{Yo’LLaVA\xspace}

\newcommand{\thisismy}{This-Is-My-Img\xspace}
\newcommand{\myparagraph}[1]{\noindent\textbf{#1}}
\newcommand{\vdoro}[1]{{\color[rgb]{0.4, 0.18, 0.78} {[V] #1}}}
% --- disable by uncommenting  
% \renewcommand{\TODO}[1]{}
% \renewcommand{\todo}[1]{#1}
\usepackage{slashbox}
% Vectors
\newcommand{\bB}{\mathcal{B}}
\newcommand{\bw}{\mathbf{w}}
\newcommand{\bs}{\mathbf{s}}
\newcommand{\bo}{\mathbf{o}}
\newcommand{\bn}{\mathbf{n}}
\newcommand{\bc}{\mathbf{c}}
\newcommand{\bp}{\mathbf{p}}
\newcommand{\bS}{\mathbf{S}}
\newcommand{\bk}{\mathbf{k}}
\newcommand{\bmu}{\boldsymbol{\mu}}
\newcommand{\bx}{\mathbf{x}}
\newcommand{\bg}{\mathbf{g}}
\newcommand{\be}{\mathbf{e}}
\newcommand{\bX}{\mathbf{X}}
\newcommand{\by}{\mathbf{y}}
\newcommand{\bv}{\mathbf{v}}
\newcommand{\bz}{\mathbf{z}}
\newcommand{\bq}{\mathbf{q}}
\newcommand{\bff}{\mathbf{f}}
\newcommand{\bu}{\mathbf{u}}
\newcommand{\bh}{\mathbf{h}}
\newcommand{\bb}{\mathbf{b}}

\newcommand{\rone}{\textcolor{green}{R1}}
\newcommand{\rtwo}{\textcolor{orange}{R2}}
\newcommand{\rthree}{\textcolor{red}{R3}}
\usepackage{amsmath}
%\usepackage{arydshln}
\DeclareMathOperator{\similarity}{sim}
\DeclareMathOperator{\AvgPool}{AvgPool}

\newcommand{\argmax}{\mathop{\mathrm{argmax}}}     


\theoremstyle{plain}
\newtheorem{theorem}{Theorem}[section]
\newtheorem{proposition}[theorem]{Proposition}
\newtheorem{lemma}[theorem]{Lemma}
\newtheorem{corollary}[theorem]{Corollary}
\theoremstyle{definition}
\newtheorem{definition}[theorem]{Definition}
\newtheorem{assumption}[theorem]{Assumption}
\theoremstyle{remark}
\newtheorem{remark}[theorem]{Remark}

% Todonotes is useful during development; simply uncomment the next line
%    and comment out the line below the next line to turn off comments
%\usepackage[disable,textsize=tiny]{todonotes}
\usepackage[textsize=tiny]{todonotes}


% The \icmltitle you define below is probably too long as a header.
% Therefore, a short form for the running title is supplied here:
\icmltitlerunning{Make LoRA Great Again}

\begin{document}

\twocolumn[{
% \icmltitle{Make LoRA Great Again: Boosting Performance with \\Adaptive Singular-Value Priors and Mixture-of-Experts Optimization Alignment}

\icmltitle{Make LoRA Great Again: Boosting LoRA with \\Adaptive Singular Values and Mixture-of-Experts Optimization Alignment}

% It is OKAY to include author information, even for blind
% submissions: the style file will automatically remove it for you
% unless you've provided the [accepted] option to the icml2025
% package.

% List of affiliations: The first argument should be a (short)
% identifier you will use later to specify author affiliations
% Academic affiliations should list Department, University, City, Region, Country
% Industry affiliations should list Company, City, Region, Country

% You can specify symbols, otherwise they are numbered in order.
% Ideally, you should not use this facility. Affiliations will be numbered
% in order of appearance and this is the preferred way.
\icmlsetsymbol{equal}{*}

\begin{icmlauthorlist}
\icmlauthor{Chenghao Fan}{equal,s}
\icmlauthor{Zhenyi Lu}{equal,s}
\icmlauthor{Sichen Liu}{s}
\icmlauthor{Xiaoye Qu}{s}
\icmlauthor{Wei Wei}{s}
\icmlauthor{Yu Cheng}{a}
% % \icmlauthor{Firstname7 Lastname7}{comp}
% %\icmlauthor{}{sch}
% % \icmlauthor{Firstname8 Lastname8}{sch}
% % \icmlauthor{Firstname8 Lastname8}{yyy,comp}
% %\icmlauthor{}{sch}
% %\icmlauthor{}{sch}
\end{icmlauthorlist}

\icmlaffiliation{s}{School of Computer Science \& Technology, Huazhong University of Science and Technology}
\icmlaffiliation{a}{The Chinese University of Hong Kong}
% \icmlaffiliation{sch}{School of ZZZ, Institute of WWW, Location, Country}

% \icmlcorrespondingauthor{}{}

% You may provide any keywords that you
% find helpful for describing your paper; these are used to populate
% the "keywords" metadata in the PDF but will not be shown in the document
\icmlkeywords{Machine Learning, ICML}

\vskip 0.3in
}]

% this must go after the closing bracket ] following \twocolumn[ ...

% This command actually creates the footnote in the first column
% listing the affiliations and the copyright notice.
% The command takes one argument, which is text to display at the start of the footnote.
% The \icmlEqualContribution command is standard text for equal contribution.
% Remove it (just {}) if you do not need this facility.

%\printAffiliationsAndNotice{}  % leave blank if no need to mention equal contribution
\printAffiliationsAndNotice{\icmlEqualContribution} % otherwise use the standard text.

\begin{abstract}
% Low-rank adaptation (LoRA) is a leading method for parameter-efficient fine-tuning in the era of large language models (LLMs). 
% However, LoRA often underperforms full fine-tuning (Full FT), even when combined with Mixture-of-Experts (MoE) architectures.
% Existing approaches typically employ Singular-value decomposition (SVD) to offer more informative priors. 
% However, they rely on static priors and do not adapt to the specific input, leads to suboptimal use of pretrained knowledge. 
% % fully leverage the characteristics of the pretrained weights, 
% and when directly extended to MoE, they lead to severe weights misalignment due to random top-$k$. 
% Moreover, it is challenge to directly align LoRA MoE gradients with full fine-tuning with the existence of top-$k$ router. We reveals that careful scaling (without any change of the architecture and the training algorithm) can significantly enhance both efficiency and performance. 
% In view of this, we propose \underline{G}reat L\underline{o}R\underline{A} Mixture-of-Exper\underline{t} (GOAT), 
% % a novel framework that balances the utilization of the original matrix by evenly selecting singular values to initialize multiple LoRA experts. Through MoE's dynamic activation strategy, GOAT enables the model to adaptively select feature-rich subspaces for faster training or adjust finer details to retain more knowledge. Additionally, we introduce a scaling MoE optimization strategy, supported by theoretical analysis, to further close the gap with FFT. 
% a novel framework built on two key strategies: (1) Adaptively integrate relevant priors from pretrained knowledge according to the SVD structure through input routing.
% % Initalized each expert from different singular value segments and automatically select most informative priors via MoE router.
% (2) Directly align the low-rank gradient with full fine-tuned MoE rather than Full FT.
% theoretically derive the optimal scaling for aligning the gradients. 
% Extensive experiments across 25 datasets—spanning natural language understanding, commonsense reasoning, image classification, and natural language generation—demonstrate that GOAT achieves state-of-the-art performance and narrows the gap with full fine-tuning effectively.

% Low-rank adaptation (LoRA) is a parameter-efficient fine-tuning method for large language models (LLMs), but it often underperforms full fine-tuning (Full FT), even with Mixture-of-Experts (MoE) architectures. Existing methods use static SVD subsets, leading to suboptimal use of pre-trained knowledge and misalignment of weights in MoE when directly extending SVD, as previous methods did not face this challenge due to zero initialization. Additionally, the gradients become more complex. We show that careful scaling (without altering the architecture or training algorithm) can improve efficiency and performance.

While Low-Rank Adaptation (LoRA) enables parameter-efficient fine-tuning for Large Language Models (LLMs), its performance often falls short of Full Fine-Tuning (Full FT). 
Current methods optimize LoRA by initializing with static singular
value decomposition (SVD) subsets, leading to suboptimal leveraging of pre-trained knowledge. 
{Another path for improving LoRA is incorporating a Mixture-of-Experts (MoE) architecture.}
{However, weight misalignment and complex gradient dynamics make it challenging to adopt SVD prior to the LoRA MoE architecture.} 
To mitigate these issues, we propose \underline{G}reat L\underline{o}R\underline{A} Mixture-of-Exper\underline{t} (GOAT), a framework that (1) adaptively integrates relevant priors using an SVD-structured MoE, and (2) 
aligns optimization with full fine-tuned MoE by deriving a theoretical scaling factor. 
We demonstrate that proper scaling, without modifying the architecture or training algorithms, boosts LoRA MoE’s efficiency and performance. Experiments across 25 datasets, including natural language understanding, commonsense reasoning, image classification, and natural language generation, demonstrate GOAT’s state-of-the-art performance, closing the gap with Full FT. 
% \footnote{Our code is available at:  \url{https://anonymous.4open.science/r/goat-827B}.}

% While Low-Rank Adaptation (LoRA) offers parameter-efficient fine-tuning for Large Language Models (LLMs), its performance often falls short of Full Fine-Tuning (Full FT) performance, even with Mixture-of-Experts (MoE) architectures. 
% Current approaches optimizing LoRA typically employ static singular
% value decomposition (SVD) subsets, leading to suboptimal leveraging of pre-trained knowledge. Moreover, when extending SVD directly to MoE structures, weight misalignment occurs, a challenge absent in prior methods due to zero initialization. This also leads to more intricate gradient dynamics. 
% To mitigate these issues, in this paper, we propose \underline{G}reat L\underline{o}R\underline{A} Mixture-of-Exper\underline{t} (GOAT), a framework that: (1) adaptively integrates relevant priors using an MoE strategy grounded in the SVD structure, and (2) aligns low-rank gradients with full fine-tuned MoE, deriving the optimal scaling for alignment.
% We demonstrate that meticulous scaling, without modifying architecture or training algorithms, can improve both the efficiency and performance of the LoRA MoE structure.
% % \textcolor{red}{of xxx}. 
% Extensive experiments across 25 datasets, including natural language understanding, commonsense reasoning, image classification, and natural language generation, demonstrate that GOAT achieves state-of-the-art performance and effectively closes the gap with Full FT.
% and narrows the gap with full-rank fine-tuning.


\end{abstract}

% \begin{abstract}

% Low-Rank Adaptation (LoRA) is a leading method for parameter-efficient fine-tuning in the era of large language models (LLMs). 
% However, LoRA often underperforms full fine-tuning, even when combined with Sparsely Activated Mixture-of-Experts (MoE). 
% % which scale model capacity without increasing computational costs.
% This work addresses these limitations by identifying and tackling three key challenges:
% (1) Optimal Scaling: We show that previous LoRA methods use suboptimal scaling factors thus cause slowing learning. By deriving a scaling factor proportional to the inverse square root of the rank and number of experts, we accelerate convergence and improve performance. 
% (2) Balanced Initialization: To reduce noise in pre-trained weights within MoE, we introduce a balanced initialization strategy using Singular Value Decomposition (SVD). This ensures stable training dynamics and effective weight transfer.
% (3) Gradient Equivalence: We establish a mathematical equivalence between LoRA optimization and MoE full fine-tuning using low-rank gradients. This insight leads to improved gradient updates through better initialization and adapted parameter update.
% Through extensive experiments across 30 datasets—spanning natural language understanding, reward modeling, image classification, and natural language generation—we demonstrate that our approach significantly boosts LoRA's performance, closing the gap with full fine-tuning or even surpassing it (On 1x datasets).
    
% \end{abstract}

\section{Introduction}



The advent of Large Language Models (LLMs) has revolutionised Natural Language Processing (NLP) \citep{zhao2023survey}. While these models, trained on massive datasets, have shown remarkable capabilities, initial versions exhibited concerning issues like toxicity, hallucinations, and biases \citep{liang2023holistic,perez-etal-2022-red,ganguli2022red,weidinger2021ethical,liu2023trustworthy}. Consequently, aligning LLMs with human values has become a central research focus \citep{ouyang2022training,bai2022training,christiano2017deep,gabriel2020artificial}. The impact of alignment is evident in the success of ChatGPT \citep{gpt4}, highlighting its importance for safety, reliability, and broader applicability \citep{shen2023large,liu2023trustworthy}.

\begin{figure}[!t]
    \centering
    \includegraphics[width=0.95\linewidth]{asset/imgs/v2-vital-overview.pdf}
    \vspace{-0.50cm}
    \caption{A pluralistic alignment example from \ourdataset dataset. More multi-opinionated health scenarios can be found in \refapptab{table:health-scenarios}.}
    \label{fig:vital-alignment-overview}
    \vspace{-0.4cm}
\end{figure}

Despite progress in alignment \citep{wang2023aligning,ouyang2022training,stiennon2020learning,christiano2017deep,rafailov2024direct,schulman2017proximal}, current methods often model \textit{average} human values, neglecting the diversity of preferences across different groups \citep{positionpluralistic,sorensen2024value,feng2024modular}. As AI systems become increasingly prevalent, they must reflect this plurality \citep{positionpluralistic}. Recent work has begun to address pluralistic alignment \citep{bai2022constitutional,gordon2022jury,sorensen2024value} (as illustrated in \reffig{fig:vital-alignment-overview}), recognising the risks of overlooking diverse opinions, particularly in sensitive domains like health where misinformation can have severe consequences \citep{chen2024combating,menz2024current,suarez2021prevalence}.

LLMs are increasingly deployed in open-ended health applications like chatbots \citep{yang2023large,thirunavukarasu2023large}, where their responses to subjective questions are critical. In this domain, LLM outputs can significantly influence user beliefs \citep{santurkar2023whose}, potentially leading to undesirable outcomes such as the promotion of specific viewpoints or homogenization of beliefs \citep{weidinger2021ethical,weidinger2022taxonomy,gabriel2020artificial}. Therefore, evaluating the \textit{representativeness} of health-related LLM responses is crucial before deployment.











Although alignment datasets are available \citep{santurkar2023whose,sorensen2024value}, none focus primarily on health to the best of our knowledge. We argue that existing datasets lack the specificity needed to address diverse cultural and ethical norms within healthcare, which is paramount when incorporating AI into this field. A health-specific dataset will better capture these nuances and improve AI (pluralistic) alignment with varied health beliefs, addressing limitations in current pluralistic approaches. Hence, we build a comprehensive dataset for \textbf{V}al\textbf{I}dating pluralis\textbf{T}ic \textbf{A}lignment for hea\textbf{L}th, \ourdataset, consisting of 13.1K value-laden situations and 5.4K multiple-choice questions across surveys, polls, and moral scenarios focusing on the health domain. We focus on health scenarios \citep{porter2010value}, which present many conflicting opinions from different cultures \citep{thomas2004health,kreuter2004role}, religions \citep{elmahjub2023artificial}, values \citep{klessig1992effect,de2000sensitivity}, and others.

In this paper, we study how alignment techniques, particularly recent pluralistic alignment methods \citep{feng2024modular} in LLMs, for health-specific scenarios. We benchmark these against vanilla LLMs, existing alignment procedures—prompting, Mixture of Experts (\moe), and Modular Pluralism (\modplural). Our investigation includes eight LLMs (a combination of open-source and black-box models) across three modes of pluralistic alignment. We also experiment with some solutions to improve alignment and discuss the future scope of research. 

The contributions of this work are as follows:
\begin{itemize}
    \item To the best of our knowledge, this work is the first to explore the pluralistic alignment of LLMs, specifically within the health domain.
    \item We construct and introduce a comprehensive benchmark dataset, \ourdataset, concentrating on the health domain for various pluralistic alignment methodologies.
    \item Using this dataset, we benchmark and evaluate the current state-of-the-art (SOTA) pluralistic alignment techniques through detailed analyses and ablation studies. Our findings demonstrate that current leading models exhibit limited performance on \ourdataset. 
    
\end{itemize}







\section{Background and Motivation}

\subsection{Rethinking Singular-Value Initialization}   \label{sec:svd}
% \subsection{init LoRA by Singular Value Decomposition}




% To align with full fine-tuning, we assume that during LoRA training, a substructure of \( W_0 \) is fine-tuned, and its variation is equivalent to the variation achieved through full fine-tuning. Based on this assumption, we can rewrite the LoRA formulation as \( (W_0 - B_0A_0) + BA \), where \( B \) and \( A \) are initialized as \( B_0 \) and \( A_0 \), respectively, to ensure proper initialization. Since we still follow the low-rank assumption of LoRA, where the fine-tuned variation is low-rank, we must ensure that the initial output equals \( W_0 \), rather than letting \( BA \) replace \( W_0 \). In works such as PiSSA~\cite{meng2024pissa}, subspaces constructed from singular vectors are used to form \( BA \).  

Singular-value initialization is widely used in LoRA to preserve pre-trained weight characteristics \cite{zhao2024galorememoryefficientllmtraining, meng2024pissa, wang2024kasaknowledgeawaresingularvalueadaptation, lu2024twinmerging}. PiSSA \cite{meng2024pissa} only updates the largest singular values, while MiLoRA \cite{wang2024miloraharnessingminorsingular} adjusts minor singular values for strong performance.

To unify SVD-based methods with full fine-tuning, let \( W_0 \in \mathbb{R}^{m \times n} \) be the pre-trained weight with SVD, \( W_0 = U \Sigma V^\top \). Assuming \( h = \min(m, n) \) and LoRA rank \( r \), we decompose \( W_0 \) into rank-\( r \) blocks:
\begin{align}
W_0 = \sum_{i=0}^{l} U_i \Sigma_i V_i^\top,
\end{align}
where \( l = \frac{h}{r} - 1 \) and \( i \) denotes the segment \( [i \cdot r : (i+1) \cdot r] \). The submatrices are defined as \( U_i = U_{[i \cdot r : (i+1) \cdot r, :]} \in \mathbb{R}^{r \times m} \), \( \Sigma_i = \Sigma_{[i \cdot r : (i+1) \cdot r, i \cdot r : (i+1) \cdot r]} \in \mathbb{R}^{r \times r} \), and \( V_i = V_{[i \cdot r : (i+1) \cdot r, :]} \in \mathbb{R}^{r \times n} \).
Fine-tuning methods are represented as:

\begin{equation}
\begin{aligned}
\text{Full FT}: &\quad U_0 \Sigma_0 V_0^\top + U_1 \Sigma_1 V_1^\top + \cdots + U_l \Sigma_l V_l^\top \\
\text{PiSSA}: &\quad U_0 \Sigma_0 V_0^\top + (U_1 \Sigma_1 V_1^\top + \cdots + U_l \Sigma_l V_l^\top)^* \\
\text{MiLoRA}: &\quad (U_0 \Sigma_0 V_0^\top + \cdots + U_{l-1} \Sigma_{l-1} V_{l-1}^\top)^* + U_l \Sigma_l V_l^\top \\
\text{KaSA}: &\quad (U_0 \Sigma_0 V_0^\top + \cdots + U_{l-1} \Sigma_{l-1} V_{l-1}^\top)^* + U^{\text{r}} \Sigma^{\text{r}} {V^{\text{r}}}^\top
\end{aligned}
\end{equation}
Here, $(\cdot)^*$ denotes frozen components, while non-frozen components initialize LoRA:
\begin{align}
B = U_i \Sigma_i^{1/2} \in \mathbb{R}^{m \times r}, \quad A = \Sigma_i^{1/2} V_i^\top \in \mathbb{R}^{r \times n}.\label{eq:ba}
\end{align}
We observe PiSSA freezes minor singular values and fine-tunes only the components $U_0 \Sigma_0 V_0^\top$ with the largest norms, achieving the optimal approximation to $W_0$.\footnote{Proof in \App{app:pissa}} In contrast, MiLoRA and KaSA retain segment $0\sim(l-1)$ as preserved pretrained knowledge, but KaSA treats the minor $U_l \Sigma_l V_l^\top$ as noise and replaces it with a new random $U^{\text{r}} \Sigma^{\text{r}} V^{\text{r}\top}$. In practice, PiSSA converges faster by focusing on principal singular values, while MiLoRA and KaSA preserve more pre-trained knowledge for better final performance.

\begin{figure}[t]
% \begin{minipage}{\linewidth}
% \begin{figure}[ht]
    \centering
    \includegraphics[width=1.0\linewidth]{figures/svd.pdf}
    \vspace{-10mm}
    \caption{The effect of initializations from different SVD segments \((u_i, \sigma_i, v_i^\top)\)  for rank 32 and 128. The performance normalized by min-max scaling. }
    \vspace{-10mm}
    \label{fig:intro_rank}
% \end{figure}
% \end{minipage}
\end{figure}

This raises the question: \textit{Is it reasonable to use only the principal or minor part as a fine-tuning prior?} \Fig{fig:intro_rank} illustrates the performance of fine-tuning from different segments \((U_i, \Sigma_i, V_i^{\top}), i\in [0,\cdots,l]\), where each segment is used for initialization while others remain frozen. The x-axis represents segment indices (\eg $x=0$ for PiSSA, $x=l$ for MiLoRA), and the y-axis shows min-max normalized performance.
We can identify two notable observations: (1) \textit{The same initialization exhibits varying trends for different datasets.} For example, $x=l$ achieves better results on the EuroSAT dataset, while $x=0$ performs better on the GTSRB dataset. (2) \textit{Middle segments play a crucial role.} \eg when $r=128$, the highest performance is typically observed in the middle segments.
These findings suggest that each singular value segment contains task-specific information, motivating us to allow the model to automatically select segments during optimization, leveraging all singular values while preserving the original pre-trained matrix characteristics.




 % PiSSA uses the  \( r \) largest singular values and their associated singular vectors to construct \( BA \). It is suggested that the $Sr=(u_0,\sigma_0,v_0^{\top})$ capture the most significant information in \( W \). Leveraging this highly influential subspace helps accelerate convergence and ensures efficient updates.  In contrast, MiLoRA assumes that these smaller singular values, $Sr=(u_{l},\sigma_{l},v_{l}^{\top})$, helps preserve the pre-trained model's world knowledge.

% The residual singular values and vectors are utilized to construct the residual matrix, which remains fixed during fine-tuning:
% \vspace{-5pt}
% \begin{equation}
% \begin{aligned}
% W_{res}=W_0-W_{pri}=W_0-B_0A_0
% \label{eq:w_res}
% \end{aligned}
% \end{equation}
% As shown in Equation~\ref{eq:w_res}, we use \( W_{pri} \) to represent the subspace to be fine-tuned and initialize \( W_0 \) with \( W_{res} \) to ensure that the output at step 0 is equivalent to that of full fine-tuning. During training, \( W_{res} \) remains frozen, while \( B \) and \( A \) are initialized with \( B_0 \) and \( A_0 \), respectively. The output expression is as follows:
% \vspace{-5pt}
% \begin{equation}
% \begin{aligned}
% Y=W_0X=(W_{res}^*+W_{pri})X=(W_{res}^*+BA)X
% \label{eq:pissa}
% \end{aligned}
% \end{equation}

% Similar to LoRA, the gradients of $A$ and $B$ are also given by $\frac{\partial L}{\partial A}=B^T\frac{\partial L}{\partial Y} X^T $ and $\frac{\partial L}{\partial B}=A^T X^T\frac{\partial L}{\partial Y}$. Since elements of $s[:r]\gg$ elements of $s[r:]$, the trainable adapter $W^{pri} = BA$ contains the most essential directions of $W$.

% \paragraph{Which $(u_i,\sigma_i,v_i^{\top})$ to initialize would be best?}\label{sec:svd}
% We can observe that these different initialization are well-grounded in their design, yet their structures are fundamentally opposite.


% \subsection{MoE-type PEFT}
%  Within a MoE layer, N independent experts $\{E_i\}_{i=1}^N$, which employs a trainable matrix $W_g$ to distribute the input vector $x$ among these experts. This router generates a distribution of weights across the experts for each input, using a softmax function for normalization $p_i(X)=Softmax(W_g X)_i$, where $Softmax(...)_i$ denotes $i$th term of the output of softmax function. The resultant output from the MoE layer is a weighted sum of the outputs from the top $K$ experts:
% \begin{equation}
% \begin{aligned}
% Y=\sum_{i-1}^N \frac{TopK(p_i(X))}{\sum_{j=1}^N TopK(p_j(X))} \cdot E_i(x)
% \label{eq:moe}
% \end{aligned}
% \end{equation}
% When combining MoE and PEFT, $E_i(x)$ denotes peft-type expert function: $E_i(x)=B_iA_ix$, where $B_i$ and $A_i$ denote the corresponding LoRA components of the $i$th expert. When introducing LoRA, \Eq{eq:moe} can be rewritten as:
% \begin{equation}
% \begin{aligned}
% Y=WX+\sum_{i-1}^N G_i(X) \cdot E_i(X)
% \label{eq:peft-moe}
% \end{aligned}
% \end{equation}
% where $G_i(X)=\frac{TopK(p_i(X))}{\sum_{j=1}^N TopK(p_j(X))}$.
% Generally, $K$ is set to 1 or 2, meaning that activating only a small number of experts at a time can achieve good results, which helps further sparsify PEFT during inference.

\begin{figure}[t]
% \begin{minipage}{\linewidth}
% \begin{figure}[ht]
    \centering
    \includegraphics[width=0.95\linewidth]{figures/wr.pdf}
    \vspace{-5mm}
    \caption{SVD initialization \vs scaling $s$ and rank $r$ \label{fig:pissa}}
    \vspace{-10pt}
    \label{fig:intro}
% \end{figure}
% \end{minipage}
\end{figure}

\subsection{Rethinking Scaling Factor} \label{sec:scale}

In LoRA, it is common practice to use the scaled variant \( W = W_0 + sBA \), yet the effects of scaling factor $s$ have not been fully explored. 
\citet{biderman2024lora} consider \( s \) should typically set to 2. The SVD-based method \cite{meng2024pissa} empirically makes \( sBA \) independent of \( s \) by dividing \( B \) and \( A \) by \( \sqrt{\frac{1}{s}} \), while \citet{tian2024hydraloraasymmetricloraarchitecture} use larger scaling for LoRA MoE to achieve better performance.

To investigate it, as illustrated on the left of \Fig{fig:pissa}, we first adjust \( s \) in the SVD-based LoRA with a fixed rank, revealing that \( s \) still impacts the convergence speed. To study the effect, we introduce the equivalent weight and gradient to quantify the gap between LoRA and Full FT.
\begin{definition}[Equivalent Weight and Gradient]
\label{def:eg}
For LoRA optimization, we define the equivalent weight as:
\begin{equation}
    \tilde{W} \triangleq W + sBA,
\end{equation}
The equivalent gradient of \( \tilde{W} \) is defined as:
\begin{equation}
    \tilde{g} \triangleq \pderiv{L}{\tilde W}
\end{equation}
where \( s \) is the scaling factor, and \( G^A \) and \( G^B \) are gradients with respect to \( A \) and \( B \), respectively.
\end{definition}

\begin{lemma}\label{th:tidle_g}
Let \( g_t \) be the gradient in full-tuning, and \( B \), \( A \) be the low-rank weights. At the \( t \)-th optimization step, the equivalent gradient can be expressed as:
\begin{equation}
    \tilde{g}_t = s^2 \left( B_t {B_t}^\top g_t + g_t {A_t}^\top A_t \right) \label{eq:tg}
\end{equation}
\end{lemma}
% The formula for SVD-based initialization is:
% \begin{align}
% &\tilde{W} \propto sBA = s \left( \frac{1}{s} U_{r} \Sigma_{r} V_{r}^T \right) = \left( U_{r} \Sigma_{r} V_{r}^T \right) \\
% &\tilde{g} = s^2 \left(  \frac{1}{s} U_{r} U_{r}^\top g +  \frac{1}{s} g V_{r} V_{r}^T \right) = s \left( U_{r} U_{r}^\top g +  g V_{r} V_{r}^T \right)
% \end{align} 
% Though the equivalent weight appears independent of $s$, the equivalent gradient is proportional to $s$, the performance in \Fig{fig:pissa} proves that $s=2$ is typical too small. Thus, increasing the scaling factor in SVD-based methods is a simple way to boost the gradient, leading to faster convergence.

% Next, we study the effect of different ranks, as shown in \Fig{fig:pissa}. When the rank is low (e.g., \( r = 1 \)), the gradient norm is small and shows a different trend compared to \( r = 64 \), leading to a performance gap (95.77 \vs 98.55). However, applying proper scaling (\( s = 16 \)) increases the gradient norm, significantly narrowing the performance gap (from 95.77 to 97.70). This is particularly beneficial in MoE scenarios, where the total rank is divided among experts, resulting in lower ranks for each. Increased scaling compensates for this reduced rank, consistent with \citet{tian2024hydraloraasymmetricloraarchitecture}.
The formula for SVD-based initialization is:
\begin{align}
\tilde{W} &\propto sBA = s \left( \frac{1}{s} U_{r} \Sigma_{r} V_{r}^T \right) = U_{r} \Sigma_{r} V_{r}^T \\
\tilde{g} &= s^2 \left(  \frac{1}{s} U_{r} U_{r}^\top g +  \frac{1}{s} g V_{r} V_{r}^T \right) = s \left( U_{r} U_{r}^\top g +  g V_{r} V_{r}^T \right)
\end{align}
Though the equivalent weight is independent of \( s \), equivalent gradient is proportional to \( s \). As shown in \Fig{fig:pissa}, \( s = 2 \) is too small. Increasing the scaling factor in SVD-based methods boosts the gradient, leading to faster convergence.

Next, we examine the effect of different ranks, as shown in \Fig{fig:pissa}. With low rank (\eg \( r = 1 \)), the gradient norm is small and deviates from the trend of \( r = 64 \), creating a performance gap (95.77 vs. 98.55). However, applying proper scaling (\( s = 16 \)) increases the gradient norm, reducing the performance gap (from 95.77 to 97.70). This is especially beneficial in MoE scenarios, where the total rank is split among experts, resulting in lower ranks. Increased scaling can compensate for this, as supported by \citet{tian2024hydraloraasymmetricloraarchitecture}.

\begin{figure*}[t]
    \centering
    \includegraphics[width=0.75\linewidth]{figures/icml.drawio.pdf}
    \vspace{-2mm}
    \caption{
        \textbf{Illustration of Our Method.} 
        \textit{Single Low-Rank Adaptation}: LoRA reduces trainable parameters by reparameterizing \( W \) as \( W = W_0 + sBA \), with \( B \) and \( A \) as low-rank matrices. 
        \textit{MoE Fine-tuning}: Full MoE fine-tuning, where experts \( W^1 \) and \( W^E \) are selected by the router in this moment.
        \textbf{Subfigure (I)}: Our method replaces the single pair \( B, A \) with multiple pairs \( \{B^i, A^i\}_{i=1}^E \), initialized from different segments of the SVD of \( W_0 \) and adaptively selected by the router.
        \textbf{Subfigure (II)}: We align optimization with SVD-structured MoE by separately aligning each expert. \( W_{\text{res}} \) ensures the equivalent weight equals \( W_0 \) before optimization, and we scale each expert’s equivalent gradient to closely approximate full MoE fine-tuning.
    }
    \vspace{-5mm}
    \label{fig:intro}
\end{figure*}

\section{Method}

\subsection{LoRA MoE Architecture} 
% Typically, in full finetuning,  single weights $W \in \mathbb{R}^{m\times n}$ of the Feed-Forward Network (FFN):
% \vspace{-5pt}
% \begin{equation}
%     \begin{aligned}
%         \mathrm{FFN}(\mathbf{x}) = W(\mathbf{x}) \\
%         W_{t+1} = W_{t} - \eta g_{t} \\
% \end{aligned}
% \end{equation}
% where $\eta$ is the learning rate and $g_{t}$ is the gradient at time step $t$.
\paragraph{Mixture-of-Experts (MoE)} An MoE layer \cite{qu2024llama,zhu2024dynamic,zhu2024llama,zhang2024clip} comprises $E$ linear modules $\{W_{1}, \dots, W_{E}\}$ and a router $W_z \in \mathbb{R}^{m \times E}$ that assigns input $\mathbf{x}$ to experts based on routing scores:
\begin{equation}
    p^i(\mathbf{x}) = \frac{\exp(z^i(\mathbf{x}))}{\sum_{j=1}^{E} \exp(z^j(\mathbf{x}))},
\end{equation}
where $z(\mathbf{x}) = W_z \mathbf{x}$ and $p^i(\mathbf{x})$ is the score for expert $i$.

Let $\Omega_k(\mathbf{x})$ denote the indices of the top-$k$ scores, ensuring $|\Omega_k(\mathbf{x})| = k$ and $z^i(\mathbf{x}) > z^j(\mathbf{x})$ for all $i \in \Omega_k(\mathbf{x})$ and $j \notin \Omega_k(\mathbf{x})$. Define the weights as:
\begin{equation}
    w^i(\mathbf{x}) = 
    \begin{cases} 
        \frac{\exp(z^i(\mathbf{x}))}{\sum_{j \in \Omega_k(\mathbf{x})} \exp(z^j(\mathbf{x}))}, & \text{if } i \in \Omega_k(\mathbf{x}), \\
        0, & \text{otherwise}.
    \end{cases}
    \label{eq:router}
\end{equation}

The MoE layer output is the weighted sum of the top-$k$ experts' outputs:
\begin{equation}
    \mathrm{MoE}(\mathbf{x}) = \sum_{i=1}^E w^i(\mathbf{x}) W^i(\mathbf{x}).
    \label{eq:moe}
\end{equation}

\paragraph{LoRA MoE.} We integrate LoRA into the MoE framework, retaining the router (\Eq{eq:router}) and using the balance loss from vanilla MoE\footnote{See \App{sec:lb}}. Each expert $W^i$ is replaced by low-rank matrices $B^i \in \mathbb{R}^{m \times d}$ and $A^i \in \mathbb{R}^{d \times n}$, where $d = \frac{r}{E}$:
\begin{align}
    \mathrm{MoE}_{\text{LoRA}}(\mathbf{x}) = W(\mathbf{x}) + \sum_{i=1}^E w^i(\mathbf{x}) \left( s B^i A^i (\mathbf{x}) \right) \label{eq:moe_lora} 
    % = \sum_{i=1}^E w^i(\mathbf{x}) (W + s B^i A^i) (\mathbf{x}) =  \sum_{i=1}^E w^i(\mathbf{x}) \tilde{W}^i (\mathbf{x}) \label{eq:eq}
\end{align}
where $W$ is the pre-trained weight matrix and $s$ is the LoRA scaling factor. Since $k \ll E$, LoRA MoE uses fewer active parameters than dense MoE. 

% For each expert $i$, denote the vitual weight $W' = W + B^i_tA^i_t$, we aim for $W' = W^i$. 
% As their gradients are computed as:
% \begin{equation}
% \begin{aligned}
%     W_{t+1}^i &= W_{t}^i - \eta {\color{red} g_{t}^i} \\
%     B_{t+1}^i &= B_{t}^i - \eta {\color{blue} H_{B,t}^i} \\
%     A_{t+1}^i &= A_{t}^i - \eta {\color{blue} H_{A,t}^i}
% \end{aligned}
% \end{equation}
% We have:
% \begin{equation}
%     \begin{aligned}
%            W_{t+1}^{'i} &= W_{t}^{'i} - B_{t}^i A_{t}^i + B_{t+1}^i A_{t+1}^i \\
%     &\approx W_{t}^{'i} - \eta ({\color{blue} H_{B,t}^i} A_t + B_t {\color{blue} H_{A,t}^i})   
%     \end{aligned}
% \end{equation}
% where $H_{B,t}^i$ and $H_{A,t}^i$ are to be determined.

% If we can initialize $W^i_0 = W'= W + B^i_0A^i_0$, then we only have to keep the gradient of $W^i_0$ and $W'$ are same at each optimization step. 
% Our objective becomes minimizing:
% \begin{equation}
% \arg\min_{{\color{blue} H_{B,t}^i}, {\color{blue} H_{A,t}^i}} \left\| {\color{blue} H_{B,t}^i} A_{t}^i + B_{t}^i {\color{blue} H_{A,t}^i} - {\color{red}g_{t}^i} \right\|^2 
% \end{equation}

% Next, we solve this optimization problem. For simplicity, we omit the subscript $t$ and superscript $i$ during the solution process, \ie we consider:
% \begin{equation}
%     \arg\min_{{\color{blue} H_{B}}, {\color{blue} H_{A}}} \left\| {\color{blue} H_{B}} A + B {\color{blue} H_{A}} - {\color{red} g} \right\|^2_F 
% \end{equation}

% The solution to this optimization problem is given by:
% \begin{align}
%     {\color{blue} H_{A}} &= \frac{1}{s^2}(B B^\top)^{-1}{\color{cyan} G_A}  + {\color{magenta}C} A \\
%     {\color{blue} H_{B}} &= \frac{1}{s^2} \bigl(I - B^\top (B B^\top)^{-1} B \bigr) {\color{cyan} G_B} (A A^\top)^{-1} - B{\color{magenta}C} 
% \end{align}

\subsection{Adaptive Priors Initialization}
% According to Figure~\ref{fig:intro}, we observe that different samples require different \((u_i, \sigma_i, v_i^{\top})\) from different chunk .

According to \Sec{sec:svd}, the utilization of different SVD segments depends on the input. We propose initializing each expert in LoRA MoE with different SVD segments, leveraging the MoE architecture to dynamically activate experts associated with different singular values. Specially, we init expert evenly by define the set $\mathcal{E}_r$ as:
\begin{equation}
\begin{aligned}
    \mathcal{E}_r = \left\{(U_{ [:, k:k + d]}, \Sigma_{[k:k+d,k:k+d]}, V_{[k:k+d,:]}^\top) \mid j = 1, \dots, E \right\},
\end{aligned}
\end{equation}
where $t=\frac{\min(m,n)}{E},k=(j-1)t$ is the starting index from segement for $j$-th expert, $d=\frac{r}{E}$ is each expert rank. 
Then we can construct each expert by \(\left( U', \Sigma', V'^{\top} \right) \in \mathcal{E}_r\):
\begin{equation}
\begin{aligned}
    B^i_0 &= \sqrt{\frac{1}{s}} U' \Sigma'^{1/2} \in \mathbb{R}^{m \times d}, A^i_0 &= \sqrt{\frac{1}{s}} \Sigma'^{1/2} V'^{\top} \in \mathbb{R}^{d \times n}
\end{aligned}
\end{equation}
The $B,A$ divide $\sqrt{s}$ to make sure that $sBA$ is independent of $s$ \cite{meng2024pissa}. This allows the model to adapt flexibly to various fine-tuning scenarios. 

\subsection{Theoretical Optimization Alignment}
Directly applying SVD priors in MoE architectures causes weight misalignment and complex gradient dynamics, a challenge not encountered with previous zero initialization methods. Moreover, the gap in MoE-based architectures remains under-explored. We derive the following theorems to address this and show how scaling resolves the issue.
\begin{theorem}\label{th:align}
By ensuring equivalent weight \( \tilde{W}_0 \approx W_0 \) at initialization and maintaining equivalent gradient \( \tilde{g}_t \approx g_t \) throughout optimization, we can align LoRA with Full FT. (See Definition~\ref{def:eg} for equivalent weight and gradient.)
\end{theorem}
Theorem \ref{th:align} addresses the performance gap in single LoRA architectures \cite{wanglora, wang2024loraprolowrankadaptersproperly}. For MoE architectures, however, routers and top-$k$ selection complicate direct alignment. Thus, we focus on Full FT MoE and establish:
\begin{theorem}\label{th:moe_align}
For all \( i \in [1, \dots, E] \), by ensuring equivalent weight \( \tilde{W}^i_0 \approx W^i_0 \) at initialization and gradient \( \tilde{g}^i_t \approx g^i_t \) for each expert, we can align LoRA MoE with an Upcycled MoE with full-rank fine-tuning. 
\end{theorem}
Theorem \ref{th:moe_align} reveals a key connection between LoRA and full fine-tuning in MoE, simplifying the problem to optimizing each expert separately. We outline the steps below.

\paragraph{Initialization Alignment.} 
At initialization, we align the equivalent weight at initialization with an upcycled MoE, where each expert weight $\{W^i\}_{i=1}^{E}$ is derived from the pre-trained model's weight $W_0$ \cite{he2024upcycling}. 
This is equivalent to aligning \( \tilde{W}_0 = W_0 + \sum_{i=1}^E w^i(\mathbf{x}) B^i_0 A^i_0 \) with the original weight \( W_0 \). 
As \(B^i_0,A^i_0\) are initialized with prior information,  
We need additionally subtracting a constant \( W_{\text{res}} \), ensuring the weight alignment:
\begin{equation}
    \tilde{W}_0 = W_0 - W_{\text{res}} + \sum_{i=1}^E w^i(\mathbf{x}) s B^i_0 A^i_0 \approx W_0\label{eq:wres_approx}
\end{equation}
\begin{lemma}\label{th:lema}
For all \( i, j \in [1, \dots, E]\) (\( i \neq j \)):
\begin{align}
\mathbb{E}_{\mathbf{x}}[w^i(\mathbf{x})] &= \frac{1}{E}, \\
\text{Var}(w^i(\mathbf{x})) &= \frac{E-k}{kE^2}
% \text{Cov}(w^i(\mathbf{x}), w^j(\mathbf{x})) &= \frac{k-E}{kE^2(E-1)}.
\end{align}
\end{lemma}
\begin{theorem}\label{th:wi}
Consider the optimization problem:
\begin{equation}
    W_{\text{res}}^+ = \arg\min_{W_{\text{res}}} \mathbb{E}_{\mathbf{x}} \left[ \left\| W_{\text{res}} - s \sum_{i=1}^E w^i(\mathbf{x}) B^i_0 A^i_0 \right\|^2 \right].
\end{equation}
The closed-form solution is \( W_{\text{res}}^+ = \frac{s}{E} \sum_{i=1}^E B^i_0 A^i_0 \).
\end{theorem}
\Theorem{th:wi} provides an appropriate initialization scheme for MoE scenarios. Note that the original LoRA-MoE \cite{zadouri2024pushing,tian2024hydraloraasymmetricloraarchitecture} uses a zero initialization scheme thus $W_{\text{res}}^+ =0$, a special case of \Theorem{th:wi}.

Obviously, the variance of \( W_{\text{res}}^+ - s \sum_{i=1}^E w^i(\mathbf{x}) B^i_0 A^i_0 \) is proportional to \( \sum_{i=1}^E B^i_0 A^i_0 \).
Additionally, from Lemma~\ref{th:lema}, when \( k \) is small (e.g., \( 2k < E \)), \( \text{std}(w^i(\mathbf{x})) > \mathbb{E}[w^i(\mathbf{x})] \). To preserve the informative SVD prior while reducing initialization instability, we scale \( B^i_0 A^i_0 \) by \( \frac{1}{\rho} \), a straightforward method to decrease variance and make a more accurate approximation in \Eq{eq:wres_approx}:
\begin{align}
    B_0^i = \sqrt{\frac{1}{s\rho}} U_i \Sigma_i^{1/2}, A_0^i = \sqrt{\frac{1}{s\rho}} \Sigma_i^{1/2} V_i^\top \label{eq:initAB}
\end{align}

\paragraph{Gradient Alignment}
First, we provide the optimal scaling for zero-initialized LoRA MoE:
\begin{theorem}\label{th:s}
For \( B_0 = 0 \),\( A_0 \sim U\left(-\sqrt{\frac{6}{n}}, \sqrt{\frac{6}{n}}\right) \),$\tilde{g}_t^i = s^2 \left( B_t^i {B_t^i}^\top g_t^i + g_t^i {A_t^i}^\top A_t^i \right)$, and learning rate ratio between full tuning \vs LoRA is $\eta$. 
\begin{equation}
    \arg\min_{s} \left\| \tilde{g}_t^i - g_t^i \right\|, \quad \forall i \in [1, \dots, E]
\end{equation}
The closed-form solution of optimal scaling is \( s = \sqrt{\frac{3n\eta}{r}} \).
\end{theorem}
As \(n \gg r\), it is typically the case that \(s > 2\), which explains why standard scaling is insufficient and why simple scaling enhances effectiveness, as demonstrated in \Sec{sec:scale}.
While it is tricky to directly analyze complex gradient dynamics with SVD priors, an alternative approach is recognizing that larger scaling \(s\) and \(\rho\) ensure $B$ and $A$ become small and approach zero (\Eq{eq:initAB}), aligning with the settings in \Theorem{th:s}. Thus, we adopt this scaling factor in GOAT, and in practice, this approximation performs well (see \Sec{sec:ex}). For scenarios with proper scaling, we extend the method to ``GOAT+'', as detailed in Appendix~\ref{app:goat_pro}.

% \vspace{-5pt}
% \begin{align}
% \E(y^2_i) = s^2 ( BA x )^2 = s^2  \sum_{j=1}^r \sum_{k=1}^n  \E(b^2_{ij}) \E(a^2_{jk}) \\
% = {rn}  \text{Var}(U_0 \Sigma) \text{Var}(\Sigma V_0^\top) =   \frac{rn \Sigma^2}{n^2}  \propto \frac{r}{n} \Sigma^2
% \end{align}


% \TODO{During the forward pass}, we aim to align the singular values of each expert, \(\sqrt{\sigma_i}\), with the largest singular value among the experts, \(\sigma_0\). This helps reduce variance during training. Specifically, we scale \(B_i^0A_i^0\) by a factor of \(\sqrt{\frac{\sigma_i}{\sigma_0}}\) during training to achieve this alignment.
% The formal description is as follows:
% \vspace{-5pt}
% \begin{equation}
% \begin{aligned}
% % &B_0^i =  \sqrt{\frac{1}{\eta}} u_i \sigma_i^{1/2}, A_0^i = \sqrt{\frac{1}{\eta}} \sigma_i^{1/2} v_i^\top\\
% &\mathrm{MoE}_{\text{LoRA}}(\mathbf{x}) = \sum_{i=1}^E  w^i(\mathbf{x}) (W + \sqrt{\frac{\sigma_i}{\sigma_0}}B^i A^i) (\mathbf{x})\\
% \end{aligned}
% \end{equation}
% \vspace{-5pt}
% where $\quad (u_i,\sigma_i,v_i^{\top}) \in S_r$.


\section{Benchmarking}

Using our proposed \ourdataset dataset, we extensively benchmark the current alignment techniques across a suite of models, investigating pluralistic alignment within healthcare.

\subsection{Models} 
We evaluate the alignment techniques on a mix of eight proprietary and open-source models: \llamaSeven, \llamaThirteen, \llamaSeventy \citep{touvron2023llama}, \gemmaSeven \citep{team2024gemma}, \llamaEight \citep{dubey2024llama}, \qwenSeven, \qwenFourteen \citep{qwen2.5}, and \chatgpt \citep{achiam2023gpt}. Furthermore, both unaligned and aligned model variants are also evaluated. We utilise perspective and culture community LLMs from \citet{feng2024modular} for the \moe and \modplural alignment techniques. 

\subsection{Current Alignment Techniques}
\begin{itemize}[noitemsep,leftmargin=*]
\item \textbf{Vanilla:} LLM is prompted directly with no special instruction. This helps us evaluate how the off-the-shelf model fares for pluralistic tasks.
\item \textbf{Prompting:} Pluralism is added to the prompt by adding the instructions (detailed in \refappfig{app:prompting-plural-alignment}).
\item \textbf{\moe:} Here, the main LLM acts as a router and selects the most appropriate community LLM for a given task. Then, the response from this community LLM is provided to the main LLM, using which the main LLM populates the final response \citep{feng2024modular}.
\item \textbf{\modplural:} The main LLM outputs the final response in collaboration with other specialised community LLMs depending on the pluralistic alignment mode \citep{feng2024modular}. For \overton, the community LLM messages are concatenated along with the query and passed to the main LLM, which functions as a multi-document summariser to synthesise a coherent response reflecting diverse viewpoints. For \steerable, the main LLM selects the most relevant community LLM and generates the final response conditioned on the selected community LLM message. For \distributional, multiple response probability distributions are generated for each community LLM and then aggregated using the community priors.
\end{itemize}


\subsection{Metrics}
Our evaluation metrics align with those used in prior research \cite{positionpluralistic,feng2024modular} for each mode. For {\overton}, we use an NLI model \citep{schuster2021get} to calculate the percentage of values covered in various situations. Additionally, we incorporate human evaluations and LLM-as-a-Judge evaluations to assess responses. Specifically, we compare \modplural responses against baseline responses to determine their reflection of pluralistic values. For \steerable, we check whether the final response maintains the steer attribute, quantified by calculating accuracy. For \distributional, we compare expected and actual distributions by measuring the Jensen-Shannon (JS) distance, where a lower value indicates a better alignment.

More experimental setting details can be found in \refapp{app:exp-details}.

\subsection{Results}
\label{sec:results}


\paragraph{\overton Alignment.}







From \reftab{table:overton-vk}, we find that prompting consistently outperforms \modplural across models. Interestingly, simple pluralism achieved through prompting yields superior results for health-related tasks compared to the more complex multi-LLM collaboration, \modplural, as illustrated in \reftab{table:reduced-kidney-response-overton}. Of the models assessed, \gemmaSeven aligns most closely with the \overton framework for health applications, delivering superior performance across all methodologies. It is observed that both prompting and vanilla LLMs surpass \modplural across all eight models for both aligned and unaligned variants. Alarmingly, the coverage disparity between \modplural and the best-performing method reaches a maximum of 55.5\%. It indicates that community LLMs' messages fall short of fully covering the \overton windows within health contexts. Nevertheless, the effectiveness of health-specific \overton alignment remains below that of original \modplural work as detailed in \citet{feng2024modular}. We perform further qualitative analysis and discern missing points in community messages in \refapp{app:qualitative-analysis-overton}.

\begin{table}[!htp]
    \centering
    \resizebox{.9\linewidth}{!}{
    \begin{tabularx}{\linewidth}{lX}
        \toprule[1.5pt]
        {\small\textbf{Situation}} & 
        {\small donating a kidney when i have only one} \\
        \midrule
        \makecell[l]{\small\textbf{Prompting}} & {\scriptsize Donating a kidney ... While it is a \highlightGreen{selfless act} ... risks and impact on {one's own health}. From a diverse perspective, ... \highlightGreen{access to healthcare and resources}. For some individuals, ... harm their \highlightGreen{own well-being}. Additionally, \highlightPink{cultural and religious beliefs} ... a taboo ... It is important to \highlightPink{respect ... these perspectives ... impose one's own beliefs}...} \\
        \midrule
        \makecell[l]{\small\textbf{\texttt{ModPlural}}} & {\scriptsize The decision to donate a kidney ... It is important to seek out \highlightGreen{professional medical advice} ... Additionally, ... consider the impact on your \highlightGreen{own health and well-being}, ... Ultimately, the decision should be made with the guidance ... consideration of all factors involved.} \\ 
        \bottomrule[1.5pt]
    \end{tabularx}}
    \vspace{-0.3cm}
    \caption{Example responses for a \overton sample from \ourdataset. Even though both are unrepresentative of all the possible perspectives, prompting has more coverage than \modplural. Please refer \refapp{app:qualitative-analysis-overton} for detailed discussion and complete responses.}
    \label{table:reduced-kidney-response-overton}
    \vspace{-0.3cm}
\end{table}


We also evaluate \overton alignment using both human annotators and \texttt{GPT}-as-a-Judge. We sample 100 queries and present a pair of answers for each (one from \modplural and another from one of three methodologies). We calculate the \textcolor{green!80!black!100}{win rate}, \textcolor{blue!100!black!95}{tie rate}, and \textcolor{red!100!black!75}{loss rate} for these answer pairs, as displayed in \reffig{fig:annotation-overton}. We observe a consistent trend where \modplural does not exhibit a clear winning rate over the other baselines. Similar to the NLI coverage results, prompting achieves the highest win rate against \modplural across both evaluation settings, followed by vanilla LLMs.

\begin{figure}[!htp]
    \centering
    \includegraphics[width=.9\linewidth]{asset/imgs/vk-gpt-human-eval-plot.pdf}
    \vspace{-0.5cm}
    \caption{Results of the \emph{\overton} mode in \ourdataset, evaluated using human and GPT-4 assessments with \chatgpt as the main LLM. \modplural is found to have a low win rate against the other alignment techniques. All values are in \%. }
    \label{fig:annotation-overton}
    \vspace{-0.3cm}
\end{figure}

\begin{table*}[!htp]
\centering
    \begin{minipage}{0.90\linewidth}
    \resizebox{\linewidth}{!}{
    \normalsize
\begin{tabular}{l@{\hspace{5pt}}c@{\hspace{5pt}}c@{\hspace{5pt}}c@{\hspace{5pt}}c@{\hspace{5pt}}c@{\hspace{5pt}}c@{\hspace{5pt}}c@{\hspace{5pt}}c@{}}\toprule[1.5pt]
& \textbf{\texttt{LLaMA2}} & \textbf{\texttt{Gemma}} & \textbf{\texttt{Qwen2.5}} & \textbf{\texttt{LLaMA3}} & \textbf{\texttt{LLaMA2}} & \textbf{\texttt{Qwen2.5}} & \textbf{\texttt{LLaMA2}} & \multirow{2}{*}{\textbf{\texttt{ChatGPT}}} \\
& \textbf{\texttt{7B}} & \textbf{\texttt{7B}} & \textbf{\texttt{7B}} & \textbf{\texttt{8B}} & \textbf{\texttt{13B}} & \textbf{\texttt{14B}} & \textbf{\texttt{70B}} & {} \\
\midrule
Unaligned LLM & 15.59 & 23.10 & 20.82 & 13.22 & 14.54 & 21.93 & 15.85 & 12.65 \\
\ \ w/ Prompting & 22.68 & 28.11 & 27.53 & 16.02 & \underline{26.26} & 23.86 & \textbf{23.93} & 17.39 \\
\ \ w/ \moe & \textbf{25.26} & 24.91 & 16.49 & 16.94 & 19.02 & 16.62 & 20.39 & 19.09 \\
\ \ w/ \modplural & 14.28 & 22.97 & 16.62 & 19.96 & 9.64 & 18.57 & 12.56 & 18.45 \\
\midrule
Aligned LLM & 20.76 & \underline{38.60} & \underline{32.41} & 18.93 & 19.35 & \textbf{31.29} & 20.77 & \underline{26.70} \\
\ \ w/ Prompting & \underline{22.88} & \textbf{40.61} & \textbf{34.42} & \textbf{27.41} & \textbf{33.04} & \underline{29.43} & \underline{23.68} & \textbf{32.22} \\
\ \ w/ \moe & 19.58 & 26.00 & 28.14 & \underline{24.70} & 20.20 & 25.21 & 19.68 & 18.84 \\
\ \ w/ \modplural & 15.38 & 22.18 & 22.30 & 24.51 & 14.82 & 25.09 & 18.34 & 18.06 \\
\bottomrule[1.5pt]
\end{tabular}}
\end{minipage}
\vspace{-0.3cm}
\caption{Results of LLMs for \overton mode in \ourdataset in value coverage percentage, with $\uparrow$ values better denoting higher \overton coverage. The best and second-best performers are represented in \textbf{bold} and \underline{underline}, respectively.
}
\label{table:overton-vk}
\vspace{-0.3cm}
\end{table*}


\paragraph{\steerable Alignment.}
\begin{figure*}[!htp]
    \centering
    \includegraphics[width=0.98\linewidth]{asset/imgs/steerable-main-results-plot.pdf}
    \vspace{-0.3cm}
    \caption{Results of LLMs for \steerable mode in \ourdataset in accuracy. All values in \%, with $\uparrow$ values denoting better steerability.}
    \label{fig:steerable-main}
    \vspace{-0.3cm}
\end{figure*}

In \reffig{fig:steerable-main}, we highlight the steerability performance of LLMs. Although results vary, prompting and vanilla techniques are the top 2 performers for all LLMs and alignment methods. As in \overton, the performance of \modplural lags significantly, particularly in value-laden situations (see Appendix~Tables~\ref{table:steerable-vk}~and~\ref{table:steerable-opinionQA} for more).


\paragraph{\distributional Alignment.}
\begin{figure*}[!htp]
    \centering
    \includegraphics[width=0.98\linewidth]{asset/imgs/distributional-main-results-plot.pdf}
    \vspace{-0.3cm}
    \caption{Results of LLMs for \distributional mode in \ourdataset in JS distance, with $\downarrow$ values better denoting higher similarity with the expected distribution.}
    \label{fig:distributional-main}
    \vspace{-0.3cm}
\end{figure*}
\reffig{fig:distributional-main} presents the benchmark results for the \distributional mode in \ourdataset. Compared to results from earlier alignment modes, \modplural performs relatively better and SOTA in some scenarios. Additionally, the performance gap is narrower than observed in other alignment modes. Nonetheless, unaligned vanilla LLMs appear more adeptly aligned distributionally for health-related contexts. Results are comparable for moral and poll multiple-choice questions in the \distributional mode. Detailed results are available in Appendix~Tables~\ref{table:distributional-moralchoice} and~\ref{table:distributional-globalopinionqa}.

\paragraph{Findings.}
Relative to other alignment techniques, prompting provides superior alignment for health-related tasks. \texttt{GPT-4} and human evaluations support this, suggesting that prompting responses are more representative. We attribute this to constant improvements in these LLMs.  LLMs inherently seem to represent population distributions best compared to other complex pluralistic techniques for health. However, considering overall poor performance, it might merely represent baseline capabilities. As discovered in this paper, \modplural also does not excel in model steerability. Additionally, our extensive benchmarking reveals no performance gains with increases in model size. We conclude that \modplural serves as a general solution but faces challenges in domain-specific applications like health, necessitating the development of specialised solutions.

\subsection{Analysis}
\label{sec:analysis}
\subsubsection*{Is \overton evaluation biased by the number of sentences?}
    \begin{table*}[!htp]\centering
\resizebox{0.98\linewidth}{!}{
\begin{tabular}{l@{\hspace{5pt}}c@{\hspace{5pt}}c@{\hspace{5pt}}c@{\hspace{5pt}}c@{\hspace{5pt}}c@{\hspace{5pt}}c@{\hspace{5pt}}c@{\hspace{5pt}}c@{}}\toprule[1.5pt]
& \multicolumn{8}{c}{\textbf{Avg. Num. Sentences}} \\
\cmidrule(lr){2-9}
& \textbf{\texttt{LLaMA2-7B}} & \textbf{\texttt{Gemma-7B}} & \textbf{\texttt{Qwen2.5-7B}} & \textbf{\texttt{LLaMA3-8B}} & \textbf{\texttt{LLaMA2-13B}} & \textbf{\texttt{Qwen2.5-14B}} & \textbf{\texttt{LLaMA2-70B}} & {\textbf{\texttt{ChatGPT}}} \\
\midrule
Vanilla LLM & 11.43 {\footnotesize(\textit{20.76})} & 16.80 {\footnotesize(\textit{38.60})} & 13.76 {\footnotesize(\textit{32.41})} & 11.81 {\footnotesize(\textit{18.93})} & 13.56 {\footnotesize(\textit{19.35})} & 13.31 {\footnotesize(\textit{31.29})} & 11.58 {\footnotesize(\textit{20.77})} & 9.23 {\footnotesize(\textit{26.70})} \\
\ \ w/ Prompting & 11.30 {\footnotesize(\textit{22.88})} & 17.88 {\footnotesize(\textit{40.61})} & 12.61 {\footnotesize(\textit{34.42})} & 13.11 {\footnotesize(\textit{27.41})} & 15.26 {\footnotesize(\textit{33.04})} & 12.65 {\footnotesize(\textit{29.43})} & 11.27 {\footnotesize(\textit{23.68})} & 11.63 {\footnotesize(\textit{32.22})} \\
\ \ w/ \moe & 7.14 {\footnotesize(\textit{19.58})} & 12.46 {\footnotesize(\textit{26.00})} & 11.82 {\footnotesize(\textit{28.14})} & 13.06 {\footnotesize(\textit{24.70})} & 10.23 {\footnotesize(\textit{20.20})} & 11.78 {\footnotesize(\textit{25.21})} & 8.62 {\footnotesize(\textit{19.68})} & 7.24 {\footnotesize(\textit{18.84})}  \\
\ \ w/ \texttt{ModPlural} & 7.14 {\footnotesize(\textit{15.38})} & 9.03 {\footnotesize(\textit{22.18})} & 9.99 {\footnotesize(\textit{22.30})} & 10.69 {\footnotesize(\textit{24.51})} & 6.74 {\footnotesize(\textit{14.82})} & 9.63 {\footnotesize(\textit{25.09})} & 9.82 {\footnotesize(\textit{18.34})} & 5.22 {\footnotesize(\textit{18.06})} \\
\bottomrule[1.5pt]
\end{tabular}
}
\vspace{-0.3cm}
\caption{Average number of sentences in the \overton responses for some aligned models. There is a correlation between a higher number of sentences and \overton coverage performance (mentioned in parenthesis).}
\label{tab:avg-sentences}
\vspace{-0.15cm}
\end{table*}

The NLI evaluation seems biased towards the number of sentences in the final answer, as illustrated in \reftab{tab:avg-sentences}. Given that the NLI evaluation is conducted on a sentence-by-sentence basis, having a higher number of sentences can increase the likelihood of entailing a value. Furthermore, due to the summarisation in \modplural, we have observed that the main LLM might encapsulate multiple arguments within a single sentence. However, this may result in lower entailment scores. This trend is also evident in the \texttt{GPT}-as-a-Judge evaluations, where there are notably low win rates against prompting; nevertheless, human annotations indicate a higher win rate. We encourage further research into \overton coverage evaluation.


\subsubsection*{Could we leverage modularity and patch health gaps in LLMs?}
In this paper, we primarily focus on perspective community LLMs. However, we did a preliminary analysis using cultural community LLMs since there have been works considering alignment from multi-cultural views. We found performance to be similar with slight improvements; \modplural, \llamaSeven: 15.15 $\rightarrow$ 17.61, \llamaEight: 23.82 $\rightarrow$ 25.11, \gemmaSeven: 22.37 $\rightarrow$ 22.45. 

Similarly, we leveraged health-specialised LLM \citep{yang_mentallama_2024,kim_health-llm_2024} as the main LLMs. 
For a fair comparison, we used \texttt{mental-llama2-7b} and compared against \llamaSeven. We observe no significant performance gains; vanilla: 20.62 $\rightarrow$ 23.48, prompting: 23.69 $\rightarrow$ 24.88, \moe: 19.51 $\rightarrow$ 20.90, \modplural: 15.15 $\rightarrow$ 12.00. This suggests that straightforward patching with specialised LLMs might not be an effective solution for specialised domains like health.


\subsubsection*{How does \distributional pluralism affect the LLM entropy?}
\begin{table*}[!htp]\centering
\resizebox{0.98\linewidth}{!}{
\begin{tabular}{l@{\hspace{6pt}}
c@{\hspace{3pt}}|@{\hspace{3pt}}c@{\hspace{6pt}}
c@{\hspace{3pt}}|@{\hspace{3pt}}c@{\hspace{6pt}}
c@{\hspace{3pt}}|@{\hspace{3pt}}c@{\hspace{6pt}}
c@{\hspace{3pt}}|@{\hspace{3pt}}c@{\hspace{6pt}}
c@{\hspace{3pt}}|@{\hspace{3pt}}c@{\hspace{6pt}}
c@{\hspace{3pt}}|@{\hspace{3pt}}c@{\hspace{6pt}}
c@{\hspace{3pt}}|@{\hspace{3pt}}c@{\hspace{6pt}}
c@{\hspace{3pt}}|@{\hspace{1pt}}c@{}}\toprule[1.5pt]
& \multicolumn{2}{c}{\textbf{\texttt{LLaMA2}}} 
& \multicolumn{2}{c}{\textbf{\texttt{Gemma}}} 
& \multicolumn{2}{c}{\textbf{\texttt{Qwen2.5}}} 
& \multicolumn{2}{c}{\textbf{\texttt{LLaMA3}}} 
& \multicolumn{2}{c}{\textbf{\texttt{LLaMA2}}} 
& \multicolumn{2}{c}{\textbf{\texttt{Qwen2.5}}} 
& \multicolumn{2}{c}{\textbf{\texttt{LLaMA2}} }
& \multicolumn{2}{c}{\multirow{2}{*}{\textbf{\texttt{ChatGPT}}}} \\

& \multicolumn{2}{c}{\textbf{\texttt{7B}}} 
& \multicolumn{2}{c}{\textbf{\texttt{7B}}} 
& \multicolumn{2}{c}{\textbf{\texttt{7B}}} 
& \multicolumn{2}{c}{\textbf{\texttt{8B}}} 
& \multicolumn{2}{c}{\textbf{\texttt{13B}}} 
& \multicolumn{2}{c}{\textbf{\texttt{14B}}} 
& \multicolumn{2}{c}{\textbf{\texttt{70B}}} 
& \multicolumn{2}{c}{} \\
\cmidrule(lr){2-3} \cmidrule(lr){4-5}\cmidrule(lr){6-7}\cmidrule(lr){8-9}\cmidrule(lr){10-11}\cmidrule(lr){12-13}\cmidrule(lr){14-15}\cmidrule(lr){16-17}
& U & A 
& U & A 
& U & A 
& U & A 
& U & A 
& U & A 
& U & A 
& U & A \\
\midrule
Vanilla LLM & 1.67 & 1.27 & 1.54 & 0.33 & 1.46 & 0.43 & 1.45 & 0.72 & 1.07 & 1.23 & 1.16 & 0.21 & 1.46 & 0.78 & 0.98 & 0.32 \\
\ \ w/ prompting & 1.66 & 1.20 & 1.43 & 0.49 & 1.38 & 0.47 & 1.58 & 1.29 & 1.10 & 1.14 & 1.08 & 0.31 & 1.60 & 1.01 & 1.28 & 0.35 \\
\ \ w/ \moe & 1.58 & 0.90 & 1.22 & 0.16 & 0.99 & 0.16 & 1.31 & 0.61 & 1.39 & 0.93 & 1.00 & 0.13 & 1.27 & 0.75 & 1.26 & 0.37 \\
\ \ w/ \modplural & 1.69 & 1.31 & 1.46 & 1.20 & 1.35 & 1.15 & 1.54 & 1.24 & 1.52 & 1.44 & 1.29 & 1.04 & 1.64 & 1.27 & 1.60 & 1.06 \\
\bottomrule[1.5pt]
\end{tabular}
}
\vspace{-0.3cm}
\caption{Entropy values for the \distributional mode in \ourdataset. Values are represented as unaligned (U) and aligned (A) variants for different models. $\downarrow$ entropy values are preferred.}
\label{table:distributional-entropy}
\vspace{-0.3cm}
\end{table*}

Existing literature \citep{santurkar2023whose,globalopinionQA,positionpluralistic} has found that alignment reduces the entropy of the LLMs of output token distributions. For \distributional alignment, eventual low JS distance could be due to poor alignment and entropy decrease. For the latter, in this subsection, we measure the entropy values of different LLMs for \distributional mode of the \ourdataset in \reftab{table:distributional-entropy}. Expectedly, the aligned variant has lower entropy than the unaligned model for each technique and model. However, unaligned models seem to have entropy similar to vanilla variants. Likewise, the \modplural aligned models show significant improvement compared to other alignment techniques. Interestingly, entropy values are higher for smaller models compared to bigger LLMs. It suggests larger LLMs might be susceptible to shortcuts instead of better-aligned responses. In conclusion, we see a consistent pattern of reduced entropy post-alignment for the health domain.











\subsubsection*{Can specialised community LLMs be replaced by LLM agents?}




Considering the recent success of LLM agents \citep{tseng2024two,chen2024persona,tang2024medagents}, we investigate if on-the-fly role-assigned LLM agents could replace these specialised, fine-tuned community LLMs. 

We first construct a pool of health-specific agents, following \citep{lu2024llm}. Then, we ask \gptFour to select the most relevant six agents (mirroring the number of community LLMs used in the main experiments) for the given situation. These agents' messages substitute the community LLM messages. To ensure a fair comparison, we employ the same backbone model, \mistral, used in community LLMs, as the backbone of these agents. More details about the agents are in \refapp{app:agents-community-LLM}.

In \overton mode, one can conceptualise \modplural as consisting of two LLM tiers: community and main LLMs. The community LLMs aim to encompass various values and perspectives, while the main LLM acts as a summariser. We compute the NLI score for all values in all community LLM messages. A high score is desirable for the main LLM to summarise and cover all the values. For example, culture community LLMs have approximately the same coverage as perspective community LLMs, akin to what we observed with overall \overton performance. Thus, we use these scores for role-assigned community LLMs (\aka LLM Agents) to evaluate: \emph{Do lightweight agents outperform or surpass the current fine-tuned community LLMs used?}

\begin{figure}[!t]
    \centering
    \includegraphics[width=0.9\linewidth]{asset/imgs/diff-num-agents-nli-coverage.pdf}
    \vspace{-0.3cm}
    \caption{Impact of different numbers of agents on \overton NLI coverage. The \textcolor{red}{red} horizontal dashed line denotes the NLI coverage using the original community LLMs.}
    \label{fig:diff-agents-nli-coverage}
    \vspace{-0.3cm}
\end{figure}

We calculate the NLI coverage for varying numbers of agents, as depicted in \reffig{fig:diff-agents-nli-coverage}. Notably, with six agents (matching the number of community LLms), the coverage is similar at 44.16\%, compared to the original 47.84\%. Interestingly, if we use ten agents, the coverage improves to 49.37\%. Given the lightweight nature of these agents, using ten agents or more appears viable. Nonetheless, further research is necessary in this direction. Our findings indicate an overall suboptimal performance, primarily due to the main LLM's bias towards the position of community messages. Additionally, enhancing the summarisation ability of the main LLM to encompass all agent messages is paramount. Finally, there is also scope for improving this collection of agents and role settings. 

All points considered, this direction is worthwhile, given the noted improvement in NLI coverage. We posit that the benefit of agents, which do not necessitate resource-intensive fine-tuning and allow for the dynamic integration of new agents alongside active research, might be an interesting avenue for pluralistic alignment.



\vspace{-2mm}
\section{Related Work}

% \subsection{Parameter Efficient Fine-Tuning}
% The rapid scaling of large language models (LLMs) has introduced significant challenges in efficiently adapting these models to downstream tasks. To address these challenges, a variety of parameter-efficient fine-tuning (PEFT) methods have been proposed, aiming to reduce computational and memory overhead by updating only a small portion of the model’s parameters during fine-tuning.
% Overall, these methods can be categorized into three main types:
% (1) Adapter-based methods, which introduce additional trainable modules into the original frozen backbone~\cite{houlsby2019parameter,pfeiffer2021adapterfusion,ruckle2021adapterdrop,hetowards,wang2022adamix,pfeiffer2021adapterfusion}.
% (2) Prompt-based methods, which add extra soft tokens (prompts) to the initial input and focus exclusively on fine-tuning these trainable vectors~\cite{lester2021power,li2021prefix,liu2024gpt,xiao2023decomposed,Fan_Wei_Qu_Lu_Xie_Cheng_Chen_2024}. 
% (3) Low-rank matrix decomposition-based methods, which leverage low-rank approximations to efficiently reparameterize and fine-tune the model's weight updates~\cite{hulora,zhangadaptive,liudora,kopiczkovera}.
% Among these methods, approaches based on LoRA are widely adopted due to their ease of implementation, simplicity, and efficiency.
Since the introduction of LoRA \cite{hulora}, various variants have emerged, focusing on three key areas:
(1) \textit{Architecture Improvements}: DoRA \cite{liudora} decomposes updates into magnitude and direction, while NEAT \cite{zhong2024neatnonlinearparameterefficientadaptation} introduces nonlinear adaptations.
(2) \textit{Adaptive Rank/Scale}, AdaLoRA \cite{zhangadaptive} offers dynamic rank allocation, rsLoRA \cite{kalajdzievski2023rankstabilizationscalingfactor} adjusts scaling factors and LoRA+ \cite{hayou2024loraefficientlowrank} improves learning rate.
(3) \textit{Initialization/Optimization}, PiSSA \cite{meng2024pissa}, MiLoRA \cite{wang2024miloraharnessingminorsingular}, and KaSA \cite{wang2024kasaknowledgeawaresingularvalueadaptation} utilize SVD-based strategies to preserve knowledges. LoRA-Dash \cite{si2024unleashingpowertaskspecificdirections} automates optimal direction discovery, whereas LoRA-GA \cite{wanglora} and LoRA-Pro \cite{wang2024loraprolowrankadaptersproperly} align updates with full fine-tuning gradients. However, they still exhibit performance gap between full fine-tuning. 

Multi-LoRA architectures further boost performance: LoRAHub~\cite{huang2024lorahub} combines task-specific LoRA modules, MoLoRA~\cite{zadouri2024pushing},MoELoRA~\cite{liu2023moelora} and LoRAMoE~\cite{dou2023loramoe} integrate MoE structures with LoRA. 
MultiLoRA~\cite{wang2023multilora} introduces learnable scaling for each expert, while AdaMoLE~\cite{liu2024adamole} introduces learnable thresholds for dynamic experts selection. 
HydraLoRA~\cite{tian2024hydraloraasymmetricloraarchitecture} adopts an asymmetric MoE architecture. Unlike these methods, GOAT introduces a novel SVD-structured MoE framework that adaptively integrates relevant priors while addressing weight misalignment and gradient dynamics through theoretical scaling. 

% \textbf{Our Contribution} simplifies the problem by directly aligning LoRA MoE with Full FT MoE, reducing it to optimizing individual experts. We dynamically initialize LoRA experts with singular vectors of varying magnitudes and train the model to select appropriate experts, optimizing different tokens using distinct subspaces.



% Additionally, multi-LoRA architectures have evolved to further enhance the performance.

% \TODO{Instead of focusing solely on task-specific LoRA combinations or static expert structures, we dynamically initialize LoRA experts with singular vectors of varying magnitudes and train the model to automatically select the appropriate expert, optimizing different tokens using distinct subspaces.}

% Our method proposes a novel MoE framework and initializes different LoRA experts by evenly selecting singular values to fully utilize the features of the original matrix, effectively approximating the performance of full fine-tuning. The model is then trained to automatically select the appropriate LoRA expert, enabling it to adaptively utilize different granularities of LoRA updates during both training and inference.

\vspace{-3mm}
\section{Conclusion}
In this work, we propose GOAT, a novel framework that enhances LoRA fine-tuning by adaptively integrating SVD-structured priors and aligning low-rank gradients with full fine-tuned MoE through theoretical scaling. Without altering the architecture or training algorithms, GOAT significantly improves efficiency and performance, achieving state-of-the-art results across 25 diverse datasets. Our approach effectively bridges the performance gap between LoRA-based methods and Full Fine-Tuning.
% In the unusual situation where you want a paper to appear in the
% references without citing it in the main text, use \nocite
\nocite{langley00}

\section{Impact Statements}
GOAT enhances the efficiency and performance of fine-tuning large models, significantly reducing computational and memory costs. This makes advanced AI technologies more accessible to researchers and practitioners with limited resources, fostering innovation across diverse fields such as NLP, CV, and multi-modal applications. By leveraging adaptive priors and robust gradient handling, GOAT can drive breakthroughs in solving real-world challenges, enabling more efficient and scalable AI solutions for a wide range of industries. Our work focuses on improving model efficiency and adaptability and does not introduce any direct ethical concerns or risks.

\bibliography{icml2025}
\bibliographystyle{icml2025}


%%%%%%%%%%%%%%%%%%%%%%%%%%%%%%%%%%%%%%%%%%%%%%%%%%%%%%%%%%%%%%%%%%%%%%%%%%%%%%%
%%%%%%%%%%%%%%%%%%%%%%%%%%%%%%%%%%%%%%%%%%%%%%%%%%%%%%%%%%%%%%%%%%%%%%%%%%%%%%%
% APPENDIX
%%%%%%%%%%%%%%%%%%%%%%%%%%%%%%%%%%%%%%%%%%%%%%%%%%%%%%%%%%%%%%%%%%%%%%%%%%%%%%%
%%%%%%%%%%%%%%%%%%%%%%%%%%%%%%%%%%%%%%%%%%%%%%%%%%%%%%%%%%%%%%%%%%%%%%%%%%%%%%%
\newpage
\appendix
\onecolumn

\section{Proof related with PiSSA Select Segment} \label{app:pissa}

\begin{tcolorbox}[colback=gray!20,colframe=gray]
\begin{lemma}
Let \( W_0 \in \mathbb{R}^{m \times n} \) be the pretrained weight matrix with SVD \( W_0 = U \Sigma V^\top \). Assuming \( m \leq n \) and LoRA rank \( r \), we decompose \( W_0 \) into rank-\( r \) blocks:
\begin{align}
W_0 = \sum_{i=0}^{l} U_i \Sigma_i V_i^\top,
\end{align}
where \(l=\frac{m}{r} - 1\) are block numbers, \( U_i = U_{[i \cdot r : (i+1) \cdot r, :]} \in \mathbb{R}^{r \times m} \), \( \Sigma_i = \Sigma_{[i \cdot r : (i+1) \cdot r, i \cdot r : (i+1) \cdot r]} \in \mathbb{R}^{r \times r} \), and  \( V_i = V_{[i \cdot r : (i+1) \cdot r, :]} \in \mathbb{R}^{r \times n} \) are submatrices of \( U, \Sigma, V \).

We demonstrate that \( U_0 \Sigma_0 V_0^\top \) has the largest norm and is the best rank-\( r \) approximation of \( W_0 \).
\end{lemma}
\end{tcolorbox}

\begin{proof}
By the singular value decomposition (SVD), \( W_0 = \sum_{i=1}^{\min(m, n)} \sigma_i u_i v_i^\top \), where \( \sigma_i \) are singular values sorted in descending order (\( \sigma_1 \geq \sigma_2 \geq \cdots \)).

For each block \( U_i \Sigma_i V_i^\top \), the Frobenius norm can be written as:
\begin{align}
\|U_i \Sigma_i V_i^\top\|_F = \Big\|\sum_{j=i \cdot r}^{(i+1) \cdot r} \sigma_j u_j v_j^\top \Big\|_F.
\end{align}
Since the Frobenius norm satisfies the property of orthogonal invariance, we can simplify this expression:
\begin{align}
\|U_i \Sigma_i V_i^\top\|_F = \sqrt{\sum_{j=i \cdot r}^{(i+1) \cdot r} \sigma_j^2}.
\end{align}
This result shows that the norm of each block \( U_i \Sigma_i V_i^\top \) depends solely on the singular values \( \sigma_j \) within the block. As the singular values are sorted in descending order (\( \sigma_1 \geq \sigma_2 \geq \cdots \)), the block \( U_0 \Sigma_0 V_0^\top \), which contains the largest \( r \) singular values (\( \sigma_1, \ldots, \sigma_r \)), has the largest Frobenius norm:
\begin{align}
\|U_0 \Sigma_0 V_0^\top\|_F = \sqrt{\sum_{j=1}^r \sigma_j^2}.
\end{align}

By the Eckart–Young–Mirsky theorem, the best rank-\( r \) approximation of \( W_0 \) minimizes the reconstruction error:
\begin{align}
\|W_0 - W_0^{(r)}\|_F = \min_{X : \text{rank}(X) \leq r} \|W_0 - X\|_F,
\end{align}
where \( W_0^{(r)} = U_0 \Sigma_0 V_0^\top \). Therefore, \( U_0 \Sigma_0 V_0^\top \) not only has the largest norm but also preserves the most significant information in \( W_0 \), making it the optimal rank-\( r \) approximation.
\end{proof}

\section{Load Balance Loss}\label{sec:lb}

In vanilla MoE methods \cite{fedus2022switch,dai2024deepseekmoeultimateexpertspecialization}, a balance loss $\mathcal{L}_b$ mitigates routing collapse by ensuring even token distribution among experts:
\begin{align}
    \mathcal{L}_b &= \sum_{i=1}^E f_i P_i \label{eq:lb} \\
    f_i &= \frac{E}{kT} \sum_{t=1}^T \mathds{1}(\text{Token } x_t \text{ assigned to expert } i) \label{eq:f} \\
    P_i &= \frac{1}{T} \sum_{t=1}^T \text{softmax}(z^i(x_t))
\end{align}
where $T$ is the number of tokens and $\mathds{1}(\cdot)$ is the indicator function. Here, $f_i$ is the fraction of tokens assigned to expert $i$, and $P_i$ is the average routing probability for expert $i$. This loss promotes an even distribution of tokens across experts.

\section{Proof of Theoretical Results}
\newtheorem*{lemma2}{Lemma}
\newtheorem*{Theorem2}{Theorem}

\subsection{Proof of Lemma~\ref{th:tidle_g}}
\begin{tcolorbox}[colback=gray!20,colframe=gray]
\begin{lemma2}[2.2]
Let \( g_t \) be the full-tuning gradient, and \( B, A \) be low-rank weights. At the \( t \)-th optimization step, the equivalent gradient can be expressed as:
\begin{equation}
    \tilde{g}_t = s^2 \left( B_t {B_t}^\top g_t + g_t {A_t}^\top A_t \right)
\end{equation}
\end{lemma2}
\end{tcolorbox}

\begin{proof}
% For simplicity, we omit the index \( i \).
According to the assumption, $\tilde W_{t} = W_{t}$.
Let LoRA $sBA$ where $B \in \sR^{m \times r}, A \in \sR^{r \times n}$ , $s \in \sR$, the loss $\gL$, the $t^{th}$ update of SGD optimizer.
We denote  $\tilde W_t = W_{\text{init}} + sB_tA_t$, we can write the gradient of $B,A$ as:
\begin{align}
G^B_t = \pderiv{L}{\tilde W_t} \pderiv{\tilde W_t}{B} = \pderiv{L}{ W_t} \pderiv{\tilde W_t}{B} = s{g_{t}}A^\top \\
% m * r = m * n * n * r 
G^A_t = \pderiv{L}{\tilde W_t} \pderiv{\tilde W_t}{A} = \pderiv{L}{W_t} \pderiv{\tilde W_t}{A} = sB^\top {g_{t}}
% r * n = r * m * m * n 
\end{align}
In the gradient descend algorithm (SVD), the updates for \( B_t \) and \( A_t \) are
\begin{align}
\dd B_t = - \eta G^B_t = -s \eta g_{t} A_t^\top, \dd A_t = -\eta G^A_t = -s \eta B_t^\top g_{t}
\end{align}
% where \( \eta \) is the learning rate.
% \begin{align}
%     B_t = B_0 - \eta s \sum_{k=0}^{t-1} \tilde{g_{t-1}}_k A_k^\top \\
%     A_t = A_0 - \eta s \sum_{k=0}^{t-1} B_k^\top \tilde{g_{t-1}}_k
% \end{align}
The change in the equivalent weight \( \tilde{W} \) can be expressed as:
\begin{align}
\dd \tilde{W} &= \frac{\partial \tilde{W_t}}{\partial A_t} \dd A_t + \frac{\partial \tilde{W_t}}{\partial B_t} \dd B_t \\
&= s \cdot B_t \dd A_t + s \cdot \dd B_t A_t \\
&= s \left( B_t (-\eta s B_t^\top g_{t}) + (-\eta s g_{t} A_t^\top) A_t \right) \\
&= -\eta s^2 \left( B_t B_t^\top g_{t} + g_{t} A_t^\top A_t \right)
\end{align}
Therefore, the equivalent gradient \( \tilde{g}_t \) is given by:
\begin{align}
\tilde{g}_t = s^2 \left( B_t B_t^\top g_{t} + g_{t} A_t^\top A_t \right)
\end{align}
This concludes the proof.
\end{proof}

\subsection{Proof of Theorem~\ref{th:align}}

\begin{tcolorbox}[colback=gray!20,colframe=gray]
\begin{Theorem2}[3.1]
Let the learning rate in Full FT and LoRA be $\eta_{\text{FFT}}, \eta_\text{LoRA}$.
By ensuring equivalent weight \( \tilde{W}_0 \approx W_0 \) at initialization and maintaining equivalent gradient \( \eta_{\text{LoRA}}\tilde{g}_t \approx \eta_{\text{FFT}} g_t \)  throughout each optimization step, we can effectively align LoRA with Full FT. (Equivalent weight and gradient are defined in Definition~\ref{def:eg}.)
\end{Theorem2}
\end{tcolorbox}

\begin{proof}
We verify this alignment using induction. The equivalent weight is defined as \( \tilde{W}_t = W_{\text{init}} + sB_tA_t \), and the equivalent gradient is \( \tilde{g}_t = \frac{\partial L}{\partial \tilde{W}} \).   
Using the gradient descent algorithm (considering only the SGD optimizer), we have:
\begin{align}
W_{t+1} = W_{t} - \eta_{\text{FFT}} g_t \\
\tilde W_{t+1} = \tilde W_t - \eta_\text{LoRA} \tilde g_{t}
\end{align}

\textit{Base Case (\( t = 0 \))}: We have ensured \( \tilde{W}_0 = W_0 \).

\textit{Inductive Step:} 
Assume \( \tilde{W}_t = W_t \) and \( \tilde{g}_t = g_t \). Then:
\begin{align}
    \tilde{W}_{t+1} &= \tilde{W}_t - \eta_{\text{LoRA}} \tilde{g}_t \\
                     &= W_t - \eta_{\text{FFT}} g_t \\
                     &= W_{t+1}.
\end{align}

By induction, \( \tilde{W}_t = W_t \) for all \( t \), ensuring the alignment between LoRA and Full FT.
\end{proof}

\subsection{Proof of Theorem~\ref{th:moe_align}}
\begin{tcolorbox}[colback=gray!20,colframe=gray]
\begin{Theorem2}[3.2]
Let the learning rate in Full FT MoE and LoRA MoE be $\eta_{\text{FFT}}, \eta_\text{LoRA}$.
For all \( i \in [1, \dots, E] \), by ensuring the equivalent weight of the \(i\)-th expert \( \tilde{W}^i_0 \approx W^i_0 \) at initialization and maintaining the equivalent gradient of the \(i\)-th expert \( \eta_{\text{LoRA}}\tilde{g}^i_t \approx \eta_{\text{FFT}}g^i_t \) throughout each optimization step, we can effectively align LoRA MoE with Full FT MoE.
\end{Theorem2}
\end{tcolorbox}

\begin{proof}
We aim to show that under the given conditions, the LoRA MoE aligns with the Full FT MoE by effectively making the MoE routers behave identically in both models.

\textit{Base Case (\( t = 0 \)):}  
At initialization, by assumption, the equivalent weights of each expert satisfy \( \tilde{W}^i_0 \approx W^i_0 \) because our Full FT MoE is an upcycling MoE which makes all \( W^i_0 = W_0\). Additionally, since both models use the same random seed, the routers are initialized identically, ensuring that the routing decisions are the same for both Full FT MoE and LoRA MoE.

\textit{Inductive Step:}  
Assume that at step \( t \), the equivalent weights satisfy \( \tilde{W}^i_t = W^i_t \) for all \( i \), and the routers in both models are identical. During the \( t \)-th optimization step, the gradients are scaled such that \( \eta_{\text{LoRA}}\tilde{g}^i_t \approx \eta_{\text{FFT}}g^i_t \). This ensures that the weight updates for each expert in both models are equivalent:

\begin{align}
\tilde{W}^i_{t+1} = \tilde{W}^i_t - \eta_{\text{LoRA}}\tilde{g}^i_t \approx W^i_t - \eta_{\text{FFT}}g^i_t = W^i_{t+1}
\end{align}

First, as the routers are identical, the router weight $w^i$ is the same, so the layer output is the same:
\begin{align}
    \mathrm{MoE}(\mathbf{x}) &= \sum_{i=1}^E w^i(\mathbf{x}) W^i(\mathbf{x})\\
    &= \sum_{i=1}^E w^i(\mathbf{x}) \tilde{W}^i (\mathbf{x}) \\
    &= \sum_{i=1}^E w^i(\mathbf{x}) (W + s B^i A^i) (\mathbf{x}) \\
    &= W(\mathbf{x}) + \sum_{i=1}^E w^i(\mathbf{x}) \left( s B^i A^i (\mathbf{x}) \right) \\
    &= \mathrm{MoE}_{\text{LoRA}}(\mathbf{x})
\end{align}

Since the weight updates are equivalent and the routers are optimized from the output induced by these weights, the routers remain identical at step \( t+1 \). Therefore, by induction, the routers are identical for all \( t \).

With identical routers, the routing decisions do not differentiate between Full FT MoE and LoRA MoE layers. Consequently, the alignment of individual experts (as established by Theorem~\ref{th:align}) ensures that the overall behavior of both MoE variants is effectively aligned.

\end{proof} 

\subsection{Proof of Lemma~\ref{th:lema}}
% \section{The Expectation and Variance of Expert Weights}
\begin{tcolorbox}[colback=gray!20,colframe=gray]
\begin{lemma2}[3.3]
Let $\Omega_k(\mathbf{x})$ be the set of indices corresponding to the top-$k$ largest values of $z^i(\mathbf{x})$, and \( z^i(\mathbf{x}) \) are independent and identically distributed (i.i.d.), and \( k \leq \frac{E}{2} \), $w^i$ is defined as:
\begin{equation}
    w^i(\mathbf{x}) = 
    \begin{cases} 
        \frac{\exp(z^i(\mathbf{x}))}{\sum_{j \in \Omega_k(\mathbf{x})} \exp(z^j(\mathbf{x})}) & \text{if } i \in \Omega_k(\mathbf{x}), \\
        0 & \text{if } i \notin \Omega_k(\mathbf{x}),
    \end{cases}
\end{equation}
 
We demonstrate the following properties for all \( i, j \in [1, \dots, E] \) (\( i \neq j \)):
\begin{align}
\mathbb{E}_{\mathbf{x}}[w^i(\mathbf{x})] &= \frac{1}{E}, \\
\text{Var}_{\mathbf{x}}(w^i(\mathbf{x})) &= \frac{E-k}{kE^2}.
% \text{Cov}(w^i(\mathbf{x}), w^j(\mathbf{x})) &= \frac{k-E}{kE^2(E-1)}.
\end{align}
\end{lemma2}
\end{tcolorbox}

\begin{proof}
Because the \(z^i(x)\) are i.i.d. random variables, any permutation of the indices \(\{1,\dots,E\}\) leaves the joint distribution of \(\{z^1(\mathbf{x}),\dots,z^E(\mathbf{x})\}\) unchanged.  
The Top-K operation (pick the indices of the largest \(K\) logits) is also symmetric with respect to permutations: permuting \((z^1,\dots,z^E)\) accordingly permutes the set \(\Omega_k(\mathbf{x})\) of selected indices.
Because of this symmetry, each \(w^i(\mathbf{x})\) is distributed in the same way as \(w^j(\mathbf{x})\) for any \(j\). 
By definition of $w^i(\mathbf{x})$, we have $\forall \mathbf{x}, \sum_{i=1}^E w^i(\mathbf{x}) = 1$, so: 
\vspace{-5pt}
\begin{align}
    \sum_{i=1}^E \mathbb{E}[w^i(\mathbf{x})] 
    &= \mathbb{E}\Bigl[\sum_{i=1}^E w^i(\mathbf{x})\Bigr]
    = \mathbb{E}[1]
    = 1, \\
    \mathbb{E}_{\mathbf{x}}[w^i(\mathbf{x})] &= \frac{1}{E}, \forall i \in [1,\cdots, E]
\end{align}

The variance of \( w^i(\mathbf{x}) \) is given by:

\begin{align}
\text{Var}_{\mathbf{x}}(w^i(\mathbf{x})) = \mathbb{E}_{\mathbf{x}}\left[ \left( w^i(\mathbf{x}) \right)^2 \right] - \left( \mathbb{E}_{\mathbf{x}}\left[ w^i(\mathbf{x}) \right] \right)^2.
\end{align}

Since \( \mathbb{E}_{\mathbf{x}}\left[ w^i(\mathbf{x}) \right] = \frac{1}{E} \), we have:

\begin{align}
\text{Var}_{\mathbf{x}}(w^i(\mathbf{x})) = \mathbb{E}_{\mathbf{x}}\left[ \left( w^i(\mathbf{x}) \right)^2 \right] - \frac{1}{E^2}.\label{eq:var}
\end{align}

We aim to compute \( \mathbb{E}_{\mathbf{x}}\left[ \left( w^i(\mathbf{x}) \right)^2 \right] \), but it's tricky to directly obtain this expectation.  Given that $\sum_{i=1}^E w_i = 1$, 
we can expand this expression. Omitting the \(\mathbf{x}\) for simplicity, we get:

\begin{align}
1 &= \left( \sum_{i=1}^E w_i \right)^2 = \mathbb{E}\left[ \left( \sum_{i=1}^E w_i \right)^2 \right] = \mathbb{E}\left[ \sum_{i=1}^E w_i^2 \right] + \sum_{i\neq j} \mathbb{E}[w_i w_j], \\
1&= E \cdot \mathbb{E}[w_i^2] + E(E-1) \cdot \mathbb{E}_{i\neq j}[w_i w_j]. \label{eq:E}
\end{align}
where \( \mathbb{E}[w_i w_j] \) is the expectation we need to compute. This expression is derived based on the rotational symmetry of \(w_i, w_j\), which means the cross-term \( \mathbb{E}[w_i w_j] \) is the same for all distinct \(i \neq j\).

To compute \( \mathbb{E}[w_i w_j] \), we rewrite the weights \( w_i \) as follows:
\begin{align}
w_i = \frac{\exp z_i}{\sum_{j \in \Omega_k} \exp z_j} = \frac{y_i}{\sum_{j \in \Omega_k} y_j},
\end{align}

where

\begin{align}
y_i = 
\begin{cases} 
\exp z_i & \text{if } i \in \Omega_k(\mathbf{x}), \\
0 & \text{if } i \notin \Omega_k(\mathbf{x}).
\end{cases}
\end{align}

Thus, the product \( w_i w_j \) becomes:

\begin{align}
w_i w_j = \frac{y_i y_j}{\left( \sum_{j \in \Omega_k} y_j \right)^2}.
\end{align}

Now, due to rotational symmetry of the terms \( y_i, w_j \), we can compute:

\begin{align}
\mathbb{E}[w_i w_j] = \frac{{k \choose 2}}{{E \choose 2}} \mathbb{E}\left[\frac{y_i y_j}{\left( \sum_{j \in \Omega_k} y_j \right)^2}\right] = \frac{k(k-1)}{E(E-1)} \cdot \frac{1}{k^2} = \frac{k-1}{E(E-1)k}.
\end{align}

Substituting this back into \Eq{eq:E} for \( \mathbb{E}[w_i^2] \):

\begin{align}
1 = E \cdot \mathbb{E}[w_i^2] + E(E-1) \cdot \frac{k-1}{E(E-1)k},
\end{align}

we get:

\begin{align}
\mathbb{E}[w_i^2] = \frac{1}{Ek}.
\end{align}

Thus, the variance of \( w^i \) in \Eq{eq:var} is:

\begin{align}
\text{Var}(w^i) = \frac{1}{Ek} - \frac{1}{E^2} = \frac{E-k}{kE^2}.
\end{align}

\end{proof}

\subsection{Proof of Theorem~\ref{th:wi}}
\begin{tcolorbox}[colback=gray!20,colframe=gray]
\begin{Theorem2}[3.4]
Consider the optimization problem:
\begin{equation}
    W_{\text{res}}^+ = \arg\min_{W_{\text{res}}} \mathbb{E}_{\mathbf{x}} \left[ \left\| W_{\text{res}} - s \sum_{i=1}^E w^i(\mathbf{x}) B^i_0 A^i_0 \right\|^2 \right].
\end{equation}
The closed-form solution is \( W_{\text{res}}^+ = \frac{s}{E} \sum_{i=1}^E B^i_0 A^i_0 \).
\end{Theorem2}
\end{tcolorbox}

\begin{proof}

% We  aimed at finding the optimal value for \( W_{res} \) by minimizing the expected squared difference between \( W_{res} \) and a sum of weighted matrices \( B^i_0 A^i_0 \), where the weights \( w^i(\mathbf{x}) \) are determined by the elements \( i \) that belong to the set \( \Omega_k(\mathbf{x}) \). This can be mathematically expressed as follows:
% \vspace{-5pt}

\( W_{res}^+ \) denotes the optimal value of \( W_{res} \).
The solution to this optimization problem, \( W_{res} \), can be derived as the expected value over all possible \( \mathbf{x} \):
\vspace{-5pt}
\begin{align}
W_{\text{res}}^+ &= s \mathbb{E}_{\mathbf{x}} \Biggl[ \sum_{i=1}^E w^i(\mathbf{x}) B^i_0 A^i_0 \Biggr] \label{eq:tmp1}\\ 
&= s \sum_{i=1}^E \mathbb{E}_{\mathbf{x}} [ w^i(\mathbf{x}) ] B^i_0A^i_0 \label{eq:tmp2}\\&=   \frac{s}{E} \sum_{i=1}^E B^i_0 A^i_0 
\end{align}
where \Eq{eq:tmp1} use the linear property of expectation and \Eq{eq:tmp2} utilize Lemma~\ref{th:lema}.
% So 

% \begin{align}
% \mathbf{x} \sim \mathcal{N}(0,\sigma_x^2 I_{h \times h}) \\
% % z^i(\mathbf{x}) \sim \mathcal{N}\left(0, \frac{\sigma_x^2}{3}\right), \\
% z(\mathbf{x}) = [z^1(\mathbf{x}), \cdots, z^E(\mathbf{x})] \sim \mathcal{N}\left(0, \frac{\sigma_x^2}{3} I_{E \times E}\right), \\
% w^i(\mathbf{x}) = \frac{e^{z^i(\mathbf{x})}}{\sum_{j \in \text{Top-}k} e^{z^j(\mathbf{x})}}
% \end{align}


% When in  mini-batch stochastic gradient scenarios:
% \vspace{-5pt}
% \begin{align}
% \mathbb{E}_{x_1 \ldots x_B \sim \rho} \left[  s \frac{1}{BL}\sum_{j=1}^{BL}\sum_{i=1}^E w^i(\mathbf{x}_j) B^i_0 A^i_0 \right] =  \frac{s}{E} \sum_{i=1}^E B^i_0 A^i_0 \label{eq:Ebatch}
% % \sum_{i=1}^E s w^i(\mathbf{x}_j) B^i_0 A^i_0  =  \frac{s}{E} \sum_{i=1}^E B^i_0 A^i_0 
% \end{align}

\end{proof}

% \subsection{Proof of Theorem~\ref{th:var}}
% \begin{tcolorbox}[colback=gray!20,colframe=gray]
% \begin{Theorem2}[3.3]
% The variance of \( W_{\text{res}}^+ - s \sum_{i=1}^E w^i(\mathbf{x}) B^i_0 A^i_0 \) is proportional to \( \sum_{i=1}^E B^i_0 A^i_0 \).
% \end{Theorem2}
% \end{tcolorbox}

% \begin{proof}

% We abbreviate \( B_0^iA_0^i \) as \( C_0^i \) and we study the variance of each entry $i,j$:  

% \begin{align}
% \text{Var} ()
% \end{align}

% % \text{Var} \bigl( \sum_{i=1}^E w^i(\mathbf{x}) C^i \bigr)&=\sum_{i=1}^E \text{Var} \bigl(  w^i(\mathbf{x}) \bigr) (C^i)^2 + \sum_{i\neq j} \text{Cov}(w^i, w^j) Cov(C^i,C^j) \\
% % &=\text{Var}_C \sum_{i=1}^E (C^i)^2 + \text{Cov}_C \sum_{i\neq j}  C^i C^j \\
% % &=\frac{E-k}{kE^2} \sum_{i=1}^E (C^i)^2  + \frac{k-E}{kE^2(E-1)} \sum_{i\neq j}  C^iC^j

% It can be observed that \( B_0^i \) and \( B_0^j \), when \( i \neq j \), are two orthogonal vectors extracted from \( U \) in the SVD decomposition. The same applies to \( A \). Therefore, when \( i \neq j \), \( \text{Cov}(C_i, C_j) = 0 \), and the covariance terms in the above equation can be ignored. As a result, we can derive the expected variance as follows:
% \begin{align}
% \mathbb{E} \left[\text{Var} \bigl( \sum_{i=1}^E w^i(\mathbf{x}) B^i_0 A^i_0 \bigr)\right] =\frac{E-k}{kE^2} \sum_{i=1}^E (C^i)^2 
% \end{align}

% \end{proof}






\subsection{Proof of Theorem~\ref{th:s}}
\begin{tcolorbox}[colback=gray!20,colframe=gray]
\begin{Theorem2}[3.5]
Consider the optimization problem where \( B_0 = 0 \) and \( A_0 \sim U\left(-\sqrt{\frac{6}{n}}, \sqrt{\frac{6}{n}}\right) \), $\tilde{g}_t^i = s^2 \left( B_t^i {B_t^i}^\top g_t^i + g_t^i {A_t^i}^\top A_t^i \right)$, the ratio between full tuning learning rate \vs LoRA learning rate $\eta$. 
\begin{equation}
    \arg\min_{s} \left\| \tilde{g}_t^i - g_t^i \right\|, \quad \forall i \in [1, \dots, E]
\end{equation}
The closed-form solution is \( s = \sqrt{\frac{3n\eta}{r}} \).
\end{Theorem2}
\end{tcolorbox}

\begin{proof}
By analyzing the first step gradient, 
\begin{align}
    % m *n =  m*r r*m m* n + m*n n*r r*n
    \tilde{g}_0 = s (B_0 G^A_0 + G^B_0 A_0) = s^2 ( B_0 B^\top_0 g_0 + g_0 A^\top_0 A_0 ) 
\end{align}

\vspace{-5pt}
\begin{align}
    % m *n =  m*r r*m m* n + m*n n*r r*n
    \arg\min_{s} \left\| s^2 \underbrace{\left( B_0 B^\top_0 g_0 + g_0 A^\top_0 A_0 \right)}_{\text{rank} < 2r} - \eta g_0 \right\|
\end{align}
As LoRA init $B_0=0$ and $A_0\sim U(-{\sqrt{\frac{6}{n}}}, {\sqrt{\frac{6}{n}}})$. The above equation becomes 
\vspace{-5pt}
\begin{align}
    % m *n =  m*r r*m m* n + m*n n*r r*n
    \arg\min_{s} \left\| \underbrace{s^2\left( g_0 A^\top_0 A_0 \right)}_{\text{rank} < 2r} - \eta g_0 \right\| 
\end{align}

First We notice that the matrix \( A_0^\top A_0 \) can express the entries in the following way 
\begin{align}
    A_0^\top A_0[i,j] = \sum_{k=1}^r A_0[i,k] A_0^\top[k,j],
\end{align}
For the diagonal entries (\( i = j \)), the formula simplifies to:
\begin{align}
(A_0^\top A_0)_{i,i} = \sum_{k=1}^r A_{0, i, k}^2  = \sigma_A
\end{align}
This is because the entries of \( A_0 \) are i.i.d. with mean \( 0 \) and variance \( \sigma_A \), we can compute:
\begin{align}
\mathbb{E}[(A_0^\top A_0)_{i,i}] = \sum_{k=1}^r \mathbb{E}[A_{0, i, k}^2] = r \sigma_A
\end{align}
For the non-diagonal entries (\( i \neq j \)), the formula is:
\begin{align}
(A_0^\top A_0)_{i,j} = \sum_{k=1}^r A_0^\top[i, k] A_0[k, j] = 0
\end{align}
Since \( A_0^\top[i, k] \) and \( A_0[k, j] \) are independent random variables (for \( i \neq j \)), their product has an expected value of zero.
\begin{align}
\E_{A_0}[A_0^\top A_0] = {r} \sigma_A \rmI_{n \times n}
\end{align}

Given that $\E_{A_0}[A_0^\top A_0] = \frac{r}{3n} \rmI_{n \times n}$ (use Leaky ReLU with negative slope $
\sqrt{5}$, that is $\text{Var}(A) = \frac{1}{3n}$), we can get  $s = \sqrt{\frac{ 3n\eta}{r}}$

\vspace{-5pt}
\begin{align}
    % m *n =  m*r r*m m* n + m*n n*r r*n
    \left\| g_0 \left( \frac{s^2r}{3n} \mathbf{I} - \eta \mathbf{I} \right) \right\| = 0 ,\quad s = \sqrt{\frac{ 3n\eta}{r}}
\end{align}

Though it is derived by the first step gradient, as in practice, the weight change $\|\frac{\dd W}{W}\|$ is typically small (thus has the low-rank update hypnosis in \citet{hulora}), we can consider $\|\frac{\dd A}{A}\|$ and $\|\frac{\dd B}{B}\|$ is small, so the above \(s\) can be extended to the following steps. 

\end{proof}

\section{Extend Our Method to Scenarios with Proper Scaling}\label{app:goat_pro}
GOAT assumes a scenario where LoRA MoE has not been properly scaled. Here, we supplement it with an extended approach for scenarios where proper scaling has been applied.

% By analyzing the first-step gradient for each expert, we have:

% \[
% \tilde{g}_0^i = s_i \left( B_0^i G^{A^i}_0 + G^{B^i}_0 A^i_0 \right) = s_i^2 \left( B^i_0 {B^i}^\top_0 g^i_0 + g^i_0 {A^i}^\top_0 A^i_0 \right)
% \]

Here, we assume that the routing strategy of the fully fine-tuned MoE aligns with our method. Since the router is non-differentiable, we ignore its impact and focus solely on the gradient of each expert. Our goal is to align the gradient of each expert in our method with that of the fully fine-tuned MoE. Thus, for the \(i\)-th expert, we aim to solve:

\begin{align}
\arg\min_{s_i} \left\| s_i^2 \underbrace{\left( B^i_0 {B^i}^\top_0 g^i_0 + g^i_0 {A^i}^\top_0 A^i_0 \right)}_{\text{rank} < 2r} - g^i_0 \right\|
\end{align}

When using the balanced initialization strategy, the above equation can be rewritten as:

\begin{align}
\arg\min_{s_i} \left\| s_i^2 \underbrace{\left( u_i u_i^\top \sigma_i^2 g^i_0 + g^i_0 \sigma_i^2 v_i^\top v_i \right)}_{\text{rank} < 2r} - g^i_0 \right\|
\end{align}

If each expert has rank 1, the equation can be further simplified to:

\begin{align}
\arg\min_{s_i} \left\| s_i^2 \sigma_i^2 \underbrace{\left( u_i u_i^\top g^i_0 + g^i_0 v_i^\top v_i \right)}_{\text{rank} < 2r} - g^i_0 \right\|
\end{align}

From this, we can observe that \textbf{\(\sigma_i\) acts as a scaling factor for the gradient, stretching or compressing the direction represented by the current expert during optimization.} Here, we assume that the hyperparameters have already been correctly scaled for the first expert (which corresponds to the optimal low-rank approximation of the original matrix), aligning it with the first expert of the fully fine-tuned MoE. Since the stretching strategy for the direction represented by each expert should remain consistent during MoE fine-tuning, we need to align the scaling factors \(s_i\) for the other experts to reduce the gap between our method and full MoE fine-tuning. Specifically, \(s_i\) must satisfy the following condition:

\begin{align}
s_1^2 \sigma_0 = s_i^2 \sigma_i
\end{align}

Thus, we transform each \(s_i\) as follows:

\begin{align}
s_i = s_i \frac{\sqrt{\sigma_0}}{\sqrt{\sigma_i}}
\end{align}

When the rank of each expert is greater than 1, we approximate the solution by using the sum of the singular values within the segment.

Here, we modify the scaling of all experts except the first one, while keeping other initialization methods consistent with ~\ref{eq:initAB}.

We refer to this extended method as GOAT+, and its performance across all benchmarks is presented in Table~\ref{tab:goat_pro}. While designed for different scenarios, it demonstrates performance comparable to GOAT.
\begin{table}[ht]
\centering
\begin{tabular}{l*{5}{c}} \toprule
\textbf{Method} & \textbf{NLG(Avg.)} & \textbf{NLU(Avg.)} & \textbf{IC(Avg.)} & \textbf{CR(Avg.)} & \textbf{Avg.} \\ \midrule
\textbf{GOAT}   & 30.60      & 89.76     & 81.49    & 82.64    & 71.12 \\
\textbf{GOAT+}  & 30.54      & 89.61     & 81.54    & 82.41    & 71.02 \\ \bottomrule
\end{tabular}
\caption{Performance comparison of our method extended to properly scaled scenarios.}
\label{tab:goat_pro}
\end{table}


\section{Experiment Details}\label{app:imple}

\subsection{Dataset details} \label{app:dataset}
\paragraph{Natural Language Understanding Tasks.}
We evaluate our model on the following natural language understanding tasks from the GLUE benchmark~\cite{wang2018glue}:  
\begin{enumerate}
\item \textbf{CoLA}~\cite{warstadt-etal-2019-neural}: A binary classification task that requires determining whether a given sentence is grammatically acceptable.  
\item \textbf{SST-2}~\cite{socher2013recursive}: A sentiment analysis task where the goal is to classify sentences as expressing positive or negative sentiment.  
\item \textbf{MRPC}~\cite{dolan2005automatically}: A binary classification task focused on identifying whether two sentences in a pair are semantically equivalent.  
\item \textbf{QQP}~\cite{wang2017bilateral}: A binary classification task to determine whether two questions from Quora have the same meaning.  
\item \textbf{MNLI}~\cite{williams2018broad}: A textual entailment task that involves predicting whether a hypothesis is entailed, contradicted, or neutral with respect to a given premise.  
\item \textbf{QNLI}~\cite{rajpurkar-etal-2016-squad}: A binary classification task to determine whether a question is answerable based on a given context.  
\item \textbf{RTE}~\cite{giampiccolo2007third}: A textual entailment task where the goal is to predict whether a hypothesis logically follows from a given premise.
\end{enumerate}
We report the overall accuracy (including matched and mismatched) for MNLI, Matthew’s correlation coefficient for CoLA, and accuracy for all other tasks.
\paragraph{Natural Language Generation Tasks.}

We evaluate our model on the following natural language generation tasks:  

\begin{enumerate}
\item \textbf{MT-Bench}~\cite{zheng2023judging}: A benchmark for evaluating dialogue generation capabilities, focusing on multi-turn conversational quality and coherence.  
\item \textbf{GSM8K}~\cite{cobbe2021training}: A mathematical reasoning task designed to assess the model's ability to solve grade school-level math problems.  
\item \textbf{HumanEval}~\cite{chen2021evaluating}: A code generation benchmark that measures the model's ability to write functional code snippets based on natural language problem descriptions.
\end{enumerate}

Following previous work~\cite{wanglora}, we evaluate three natural language generation tasks—dialogue, mathematics, and code—using the following three datasets for training:
\begin{enumerate}
\item \textbf{Dialogue: WizardLM}~\cite{xu2023wizardlm}: WizardLM leverages an AI-driven approach called Evol-Instruct. Starting with a small set of initial instructions, Evol-Instruct uses an LLM to rewrite and evolve these instructions step by step into more complex and diverse ones. This method allows the creation of large-scale instruction data with varying levels of complexity, bypassing the need for human-generated data. We use a 52k subset of WizardLM to train our model for dialogue task (MT-bench).
\item \textbf{Math: MetaMathQA}~\cite{yumetamath}:  MetaMathQA is a created dataset designed specifically to improve the mathematical reasoning capabilities of large language models. We use a 100k subset of MetaMathQA to train our model for math task (GSM8K).
\item \textbf{Code: Code-Feedback}~\cite{zheng2024opencodeinterpreter}: This dataset includes examples of dynamic code generation, execution, and refinement guided by human feedback, enabling the model to learn how to improve its outputs iteratively. We use a 100k subset of Code-Feedback to train our model for code task (HumanEval).
\end{enumerate}

\paragraph{Image Classification Tasks.}
We evaluate our model on the following image classification tasks:
\begin{enumerate}
    \item \textbf{SUN397}~\cite{xiao2016sun}: A large-scale scene classification dataset containing 108,754 images across 397 categories, with each category having at least 100 images.
    \item \textbf{Cars} (Stanford Cars)~\cite{krause20133d}: A car classification dataset featuring 16,185 images across 196 classes, evenly split between training and testing sets.
    \item \textbf{RESISC45}~\cite{cheng2017remote}: A remote sensing scene classification dataset with 31,500 images distributed across 45 categories, averaging 700 images per category.
    \item \textbf{EuroSAT}~\cite{helber2019eurosat}: A satellite image classification dataset comprising 27,000 geo-referenced images labeled into 10 distinct classes.
    \item \textbf{SVHN}~\cite{netzer2011reading}: A real-world digit classification dataset derived from Google Street View images, including 10 classes with 73,257 training samples, 26,032 test samples, and 531,131 additional easy samples.
    \item \textbf{GTSRB}~\cite{stallkamp2011german}: A traffic sign classification dataset containing over 50,000 images spanning 43 traffic sign categories.
    \item \textbf{DTD}~\cite{cimpoi2014describing}: A texture classification dataset with 5,640 images across 47 classes, averaging approximately 120 images per class.
\end{enumerate}

\paragraph{Commonsense Reasoning Tasks}

We evaluate our model on the following commonsense reasoning tasks:  
\begin{enumerate}
\item \textbf{BoolQ}~\cite{clark-etal-2019-boolq}: A binary question-answering task where the goal is to determine whether the answer to a question about a given passage is "yes" or "no."  
\item \textbf{PIQA} (Physical Interaction Question Answering)~\cite{bisk2020piqa}: Focuses on reasoning about physical commonsense to select the most plausible solution to a given problem.  
\item \textbf{SIQA} (Social IQa)~\cite{sap-etal-2019-social}: Tests social commonsense reasoning by asking questions about motivations, reactions, or outcomes in social contexts.  
\item \textbf{HellaSwag}~\cite{zellers-etal-2019-hellaswag}: A task designed to test contextual commonsense reasoning by selecting the most plausible continuation of a given scenario.  
\item \textbf{WinoGrande}~\cite{sakaguchi2021winogrande}: A pronoun coreference resolution task that requires reasoning over ambiguous pronouns in complex sentences.  
\item \textbf{ARC-e} (AI2 Reasoning Challenge - Easy)~\cite{clark2018thinksolvedquestionanswering}: A multiple-choice question-answering task focused on elementary-level science questions.  
\item \textbf{ARC-c} (AI2 Reasoning Challenge - Challenge)~\cite{clark2018thinksolvedquestionanswering}: A more difficult subset of ARC, containing questions that require advanced reasoning and knowledge.  
\item \textbf{OBQA} (OpenBookQA)~\cite{mihaylov-etal-2018-suit}: A question-answering task requiring reasoning and knowledge from a small "open book" of science facts.
\end{enumerate}

\subsection{Baseline details}\label{app:baseline}

\paragraph{Full-Finetune}
\begin{enumerate}
\item \textbf{Full FT} refers to fine-tuning the model with all parameters.
\item \textbf{Full FT MoE} refers to fine-tuning all parameters within a Mixture of Experts (MoE) architecture.
\end{enumerate}

\paragraph{Single-LoRA baselines}
\begin{enumerate}
\item \textbf{LoRA} \cite{hulora} introduces trainable low-rank matrices for efficient fine-tuning. 
\item \textbf{DoRA} \cite{liudora} enhances LoRA by decomposing pre-trained weights into magnitude and direction, fine-tuning the directional component to improve learning capacity and stability.
\item \textbf{PiSSA} \cite{meng2024pissa} initializes LoRA's adapter matrices with the principal components of the pre-trained weights, enabling faster convergence, and better performance.
\item \textbf{MiLoRA} \cite{wang2024miloraharnessingminorsingular} fine-tunes LLMs by updating only the minor singular components of weight matrices, preserving the principal components to retain pre-trained knowledge.
\item \textbf{rsLoRA} \cite{kalajdzievski2023rankstabilizationscalingfactor} introduces a new scaling factor to make the scale of the output invariant to rank
\item \textbf{LoRA-Dash} \cite{si2024unleashingpowertaskspecificdirections} enhances PEFT by leveraging task-specific directions (TSDs) to optimize fine-tuning efficiency and improve performance on downstream tasks.
\item \textbf{NEAT} \cite{zhong2024neatnonlinearparameterefficientadaptation} introduces a nonlinear parameter-efficient adaptation method to address the limitations of existing PEFT techniques like LoRA.
\item \textbf{KaSA} \cite{wang2024kasaknowledgeawaresingularvalueadaptation} leverages singular value decomposition with knowledge-aware singular values to dynamically activate knowledge that is most relevant to the specific task. 
% \item \textbf{LoRAPro}\cite{wang2024loraprolowrankadaptersproperly} aligns LoRA’s updates with the gradients of FFT to better approximate its behavior.
\end{enumerate}

\paragraph{LoRA MoE baseliness}
\begin{enumerate}
\item \textbf{MoLoRA} \cite{zadouri2024pushing} combines the Mixture of Experts (MoE) architecture with lightweight experts, enabling extremely parameter-efficient fine-tuning by updating less than 1\% of model parameters.
\item  \textbf{AdaMoLE} \cite{liu2024adamole} introducing adaptive mechanisms to optimize the selection of experts. 
\item \textbf{HydraLoRA} \cite{tian2024hydraloraasymmetricloraarchitecture} introduces an asymmetric LoRA framework that improves parameter efficiency and performance by addressing training inefficiencies.
\end{enumerate}

\subsection{Abaltion details}\label{app:ablation}
Here, we provide a detailed explanation of the construction of each initialization method.
Suppose $h=min(m,n),t=\frac{h}{E}$
\begin{enumerate}
\item \textbf{Ours (O)}: \(\mathcal{E}_r = \left\{(U_{ [:, k:k + d]}, \Sigma_{[k:k+d,k:k+d]}, V_{[k:k+d,:]}^\top) \mid k=(j-1)t,j = 1, \dots, E \right\}\)
\item  \textbf{Principal (P)}: \(\mathcal{E}_r = \left\{(U_{ [:, k:k + d]}, \Sigma_{[k:k+d,k:k+d]}, V_{[k:k+d,:]}^\top) \mid k=(j-1)d,j = 1, \dots, E \right\}\)
\item \textbf{Minor (M)}:\(\mathcal{E}_r = \left\{(U_{ [:, k:k + d]}, \Sigma_{[k:k+d,k:k+d]}, V_{[k:k+d,:]}^\top) \mid k=h-jd,j = 1, \dots, E \right\}\)
\item \textbf{Random (R)}:\(\mathcal{E}_r=(U_{ [:, k:k + d]}, \Sigma_{[k:k+d,k:k+d]}, V_{[k:k+d,:]}^\top)|k=tj,t=random(0,\frac{h}{d}-1),j=0,...,E-1\}\)
\end{enumerate}

\subsection{Implementation Details}\label{app:implementation}

Image classification and natural language understanding experiments are conducted on 8 Nvidia 4090 GPUs with 24GB of RAM each. Commonsense reasoning and natural language generation experiments are conducted on a single Nvidia A100 GPU with 80GB of RAM. For training and evaluating all models, we enabled bf16 precision.

\begin{table}[ht]\small
\centering
\caption{Hyperparameter Study on Poisson equation.}
\label{table:hp}
\renewcommand\arraystretch{0.6}
\begin{sc}
    \renewcommand{\multirowsetup}{\centering}
    % \setlength{\tabcolsep}{4.7pt}
    \resizebox{0.8\linewidth}{!}{
    \begin{tabular}{c|c|c}
       \toprule
       Type & Configuration & Iteration \\ 
       \midrule
       \multirow{5}{*}{Feature Width} & 8 & 226 \\
       & 16 & 224 \\
       & \textbf{32} & \textbf{184} \\
       & 64 & 308 \\
       & 128 & 356 \\
       \midrule
       \multirow{5}{*}{Pre Ite} & \textbf{1} & \textbf{184} \\
       & 2 & 217 \\
       & 3 & 268 \\
       & 4 & 225 \\
       & 5 & 225 \\
       \midrule
       \multirow{5}{*}{Post Ite} & \textbf{1} & \textbf{184} \\
       & 2 & 219 \\
       & 3 & 223 \\
       & 4 & 218 \\
       & 5 & 254 \\
       \midrule
       \multirow{5}{*}{Num C} & 8 & 216 \\
       & 16 & 234 \\
       & 32 & 205 \\
       & 64 & 233 \\
       & \textbf{128} & \textbf{184} \\
       \midrule
       \multirow{5}{*}{Num Heads} & 1 & 229 \\
       & 2 & 251 \\
       & \textbf{4} & \textbf{184} \\
       & 8 & 196 \\
       & 16 & 302 \\
       \bottomrule
    \end{tabular}}
\end{sc}
\end{table}

\subsection{Hyperparameters}\label{app:hyper}

We fine-tune our model on each task using carefully selected hyperparameters to ensure optimal performance. Specific details for each task, including learning rate, batch size, number of epochs, and other configurations, are provided to ensure reproducibility and consistency across experiments. These details are summarized in Table~\ref{tab:cs_hyper}, Table~\ref{tab:cv_hyper}, Table~\ref{tab:nlu_hyper} and Table~\ref{tab:nlg_hyper}.
We set \(\rho\) to 10 and the ratio between the full fine-tuning learning rate and the LoRA learning rate \(\eta\) to 1.
% 

\section{Parameter and FLOPs Analysis}

\subsection{Parameter Analysis}\label{app:parameters}
Here, we provide a parameter analysis for each baseline and our method based on different backbones. We assume \( H \) represents the model dimension, \( r \) denotes the rank, \( e \) indicates the number of experts, \( L \) indicates the number of layers, \( V \) indicates the vocabulary size, $P$ indicates the patch size in ViT and $C$ indicates the number of channels in ViT. The analysis for RoBERTa-large, ViT-base, and LLAMA2 7B is as follows:

\paragraph{RoBERTa-large:} $H=1024, r=32, e=2, L=24, V=50265$. The activation parameters are \texttt{dense} from all attention and MLP layer. 


\begin{enumerate}
    \item \textbf{FFT (Full Fine-Tuning)}:
    \begin{itemize}
        \item \textbf{Total Parameters}: \( (12H^2 + 13H)L + VH \)
        \item \textbf{Breakdown}:
        \begin{itemize}
            \item Embedding layer: \( VH \)
            \item Attention mechanism: \( 4H^2 + 4H \)
            \item MLP layer: \( 8H^2 + 5H \)
            \item LayerNorm (2 layers): \( 4H \)
            \item Total per layer: \( 12H^2 + 13H \)
        \end{itemize}
    \end{itemize}
    \item \textbf{Full FT MoE}:
    \begin{itemize}
        \item \textbf{Total Parameters}: \( (12eH^2+2H+9He)L+VH \)
        \item \textbf{Proportion}: \( 698\% \)
    \end{itemize}
    \item \textbf{LoRA/PiSSA/MiLoRA/rsLoRA}:
    \begin{itemize}
        \item \textbf{Total Parameters}: \( 18HrL \)
        \item \textbf{Proportion}: \( 4.00\% \)
    \end{itemize}

    \item \textbf{DoRA}:
    \begin{itemize}
        \item \textbf{Total Parameters}: \( (18Hr + 6)L \)
        \item \textbf{Proportion}: \( 4.00\% \)
    \end{itemize}

    \item \textbf{MoLoRA/GOAT}:
    \begin{itemize}
        \item \textbf{Total Parameters}: \( (18Hr + 9He)L \)
        \item \textbf{Proportion}: \( 4.50\% \)
        \item \textbf{Breakdown}:
        \begin{itemize}
            \item Attention mechanism: \( 8Hr + 4He \)
            \item MLP layer: \( 10Hr + 5He \)
            \item Total per layer: \( 18Hr + 9He \)
        \end{itemize}
    \end{itemize}

    \item \textbf{HydraLoRA}:
    \begin{itemize}
        \item \textbf{Total Parameters}: \( (9Hr + 9He + 9Hr/e)L \)
        \item \textbf{Proportion}: \( 2.75\% \)
    \end{itemize}

    \item \textbf{AdaMoLE}:
    \begin{itemize}
        \item \textbf{Total Parameters}: \( (18Hr + 9He + 9H)L \)
        \item \textbf{Proportion}: \( 4.56\% \)
    \end{itemize}
\end{enumerate}

\paragraph{ViT-base:} $H=768, r=8, e=2, L=12, P=32, C=3$. The activation parameters include \texttt{q, k, v, o, fc1, fc2}.

\begin{enumerate}
    \item \textbf{FFT}:
    \begin{itemize}
        \item \textbf{Total Parameters}: \( (C+1)P^2H + (12H^2 + 2H)L + 3H + PH + H^2 \)
        \item \textbf{Breakdown}:
        \begin{itemize}
            \item Embedding layer: \( PH+H+(C+1)P^2H \)
            \item encoder (L layers): \( (12H^2+2H)L \)
            \item LayerNorm (1 layers): \( 2H \)
            \item Pooler: \( H^2 \)
        \end{itemize}
    \end{itemize}
    \item \textbf{Full FT MoE}:
    \begin{itemize}
        \item \textbf{Total Parameters}: \( (C+1)PPH + (12eH^2 + 2H + 9He)L + 3H + PH + H^2 \)
        \item \textbf{Proportion}: \( 770\% \)
    \end{itemize}
    \item \textbf{LoRA/PiSSA/MiLoRA}:
    \begin{itemize}
        \item \textbf{Total Parameters}: \( 18HrL \)
        \item \textbf{Proportion}: \( 1.49\% \)
    \end{itemize}
    \item \textbf{LoRA (rank=16)}:
    \begin{itemize}
        \item \textbf{Total Parameters}: \( 18HrL \)
        \item \textbf{Proportion}: \( 2.99\% \)
    \end{itemize}
    \item \textbf{LoRA (rank=32)}:
    \begin{itemize}
        \item \textbf{Total Parameters}: \( 18HrL \)
        \item \textbf{Proportion}: \( 5.98\% \)
    \end{itemize}
    \item \textbf{DoRA}:
    \begin{itemize}
        \item \textbf{Total Parameters}: \( (18Hr + 6)L \)
        \item \textbf{Proportion}: \( 1.49\% \)
    \end{itemize}
    \item \textbf{MoLoRA/GOAT}:
    \begin{itemize}
        \item \textbf{Total Parameters}: \( (18Hr + 9He)L \)
        \item \textbf{Breakdown}:
        \begin{itemize}
            \item Attention mechanism: \( 8Hr+4He \)
            \item MLP layer: \( 10Hr+5He \)
            \item Total per layer: \( 18Hr + 9He \)
        \end{itemize}
        \item \textbf{Proportion}: \( 2.24\% \)
    \end{itemize}
    \item \textbf{HydraLoRA}:
    \begin{itemize}
        \item \textbf{Total Parameters}: \( (9Hr + 9He + 9Hr/e)L \)
        \item \textbf{Proportion}: \( 1.58\% \)
    \end{itemize}
    \item \textbf{AdaMoLE}:
    \begin{itemize}
        \item \textbf{Total Parameters}: \( (18Hr + 9He + 9H)L \)
        \item \textbf{Proportion}: \( 2.33\% \)
    \end{itemize}
\end{enumerate}

\paragraph{LLAMA2-7B:} $H=4096, r=32, e=2, L=32, V=32000$. The activation parameters are \texttt{q, k, v, up, down}.

\begin{enumerate}
    \item \textbf{FFT}:
    \begin{itemize}
        \item \textbf{Total Parameters}: \( (10.25H^2 + 2H)L + H + 2VH \)
        \begin{itemize}
            \item Embedding layer and LM head: \( 2VH \)
            \item Attention mechanism: \( 2.25H^2 \)
            \item MLP layer: \( 8H^2 \)
            \item RMSNorm (2 layers): \( 2H \)
            \item Additional RMSNorm (last layer): \( H \)
            \item Total per layer: \( 10.25H^2 + 2H \)
        \end{itemize}
    \end{itemize}
    \item \textbf{LoRA/PiSSA/MiLoRA/LoRA-Dash/KASA}:
    \begin{itemize}
        \item \textbf{Total Parameters}: \( 11.58HrL \)
        \item \textbf{Proportion}: \( 0.84\% \)
    \end{itemize}
    \item \textbf{DoRA}:
    \begin{itemize}
        \item \textbf{Total Parameters}: \( (11.58Hr + 5)L \)
        \item \textbf{Proportion}: \( 0.84\% \)
    \end{itemize}
    \item \textbf{NEAT}:
    \begin{itemize}
        \item \textbf{Total Parameters}: \( (11.58Hr + 10r^2)L \)
        \item \textbf{Proportion}: \( 0.84\% \)
    \end{itemize}
    \item \textbf{MoLoRA/GOAT}:
    \begin{itemize}
        \item \textbf{Total Parameters}: \( (11.58Hr + 6.66He)L \)
        \begin{itemize}
            \item Attention mechanism: \( 4.25Hr+3He \)
            \item MLP layer: \( 7.33Hr+3.66He \)
            \item Total per layer: \( 11.58Hr + 6.66He \)
        \end{itemize}
        \item \textbf{Proportion}: \( 0.96\% \)
    \end{itemize}
    \item \textbf{HydraLoRA}:
    \begin{itemize}
        \item \textbf{Total Parameters}: \( (4.91Hr + 6.66Hr/e + 6.66He)L \)
        \item \textbf{Proportion}: \( 0.84\% \)
    \end{itemize}
    \item \textbf{AdaMoLE}:
    \begin{itemize}
        \item \textbf{Total Parameters}: \( (11.58Hr + 6.66He + 6.66H)L \)
        \item \textbf{Proportion}: \( 0.97\% \)
    \end{itemize}
\end{enumerate}

% \subsection{FLOPs Analysis} \label{app:flops}
% Since LLaMA 2 7B uses GQA (Grouped Query Attention) and SwiGLU FFN, the calculation of FLOPs differs from that of standard Transformers. Here, we assume that all linear layers in the Transformer block are extended with MoE (Mixture of Experts). We assume \( H \) represents the model dimension, \(s\) denotes sequence lengths, \( r \) denotes each expert rank, \( e \) indicates the number of experts, \( L \) indicates the number of layers, \( V \) indicates the vocabulary size. \textit{Notice each MAC (Multiply-Accumulate Operations) counts as two FLOPs.}

% \paragraph{FLOPs for FT MoE:\\ \\} 

% 1. MoE linear for \(q\) and \(o\): 
%    The FLOPs are calculated as \(2 \cdot ( 2Bshe + k \cdot 2Bsh^2)\).

% 2. MoE linear for \(k\) and \(v\): 
%    Since LLaMA 2 7B's GQA reduces the number of heads for \(k\) and \(v\) to \(1/8\) of \(q\)'s heads, the FLOPs are:  
%    \(2 \cdot (2Bshe + k \cdot 2Bshh/8)\).

% 3. The FLOPs for \(q \cdot k\) and \(score \cdot v\) remain independent of $k$, as we only upcycle the linear projection to \(e\) copies. The FLOPs for these operations are \(2Bs^2h + 2Bs^2h\).

% 4. MoE linear for \(down\) and \(gate\):  
%    Since LLaMA 2 7B uses SwiGLU FFN, the FLOPs are:  
%    \(2 \cdot (2Bshe + k \cdot 2Bsh \cdot 8/3h)\).

% 5. MoE linear for \(up\):  
%    The FLOPs are:  
%    \(2Bs \cdot 8/3he + k \cdot 2Bs \cdot 8/3hh\).

% Across \(L\) layers, including the vocabulary embedding transformation, the total FLOPs are:

% \begin{align}
%     \text{FLOPs}_{\text{Full FT MoE}} =  BL \left( \frac{52}{3}esh + \frac{41}{2}ksh^2 + 4s^2h \right)  + 2BshV
% \end{align}


\subsection{FLOPs Analysis} \label{app:flops}
Here, we mainly analyze the forward FLOPs. 
Since LLaMA 2 7B uses GQA (Grouped Query Attention) and SwiGLU FFN, the calculation of FLOPs differs from that of standard Transformers. Here, we assume that all linear layers in the Transformer block are extended with MoE (Mixture of Experts). We assume \( H \) represents the model dimension, \(s\) denotes sequence lengths, \( d \) denotes each expert rank, \( e \) indicates the number of experts, total rank \(r = ed\),\( L \) indicates the number of layers, \( V \) indicates the vocabulary size. \textit{Notice each MAC (Multiply-Accumulate Operations) counts as two FLOPs.}

\paragraph{FLOPs for FT MoE:\\ \\} 

1. MoE linear for \(q\) and \(o\): 
   The FLOPs are calculated as \(2 \cdot ( 2BsHe + k \cdot 2BsH^2)\).

2. MoE linear for \(k\) and \(v\): 
   Since LLaMA 2 7B's GQA reduces the number of heads for \(k\) and \(v\) to \(1/8\) of \(q\)'s heads, the FLOPs are:  
   \(2 \cdot (2BsHe + k \cdot 2BsHH/8)\).

3. The FLOPs for \(q \cdot k\) and \(score \cdot v\) remain independent of $k$, as we only upcycle the linear projection to \(e\) copies. The FLOPs for these operations are \(2Bs^2H + 2Bs^2H\).

4. MoE linear for \(down\) and \(gate\):  
   Since LLaMA 2 7B uses SwiGLU FFN, the FLOPs are:  
   \(2 \cdot (2BsHe + k \cdot 2BsH \cdot 8/3H)\).

5. MoE linear for \(up\):  
   The FLOPs are:  
   \(2Bs \cdot 8/3He + k \cdot 2Bs \cdot 8/3HH\).

Across \(L\) layers, including the vocabulary embedding transformation, the total FLOPs are:

\begin{align}
    \text{FLOPs}_{\text{Full FT MoE}} =  BL \left( \frac{52}{3}esH + \frac{41}{2}ksH^2 + 4s^2H \right)  + 2BsHV
\end{align}


\paragraph{FLOPs for GOAT/MoLoRA/HydraLoRA:\\ \\} 

1. MoE linear for \(q\) and \(o\):  
   The FLOPs are calculated as \(2B \cdot(2sH^2+ 2esH+ 2k(sHd + sHd))\).

2. MoE linear for \(k\) and \(v\): 
   Consider the effect of LLaMA 2 7B's GQA on \(k\) and \(v\) :  
   \( 2B \cdot (2sH^2/8 + 2esH+ 2k(sHd + sHd/8))\).

3. FLOPs for \(q \cdot k\) and \(score \cdot v\):  
   The FLOPs for these operations are \(2Bs^2H + 2Bs^2h\).

4. MoE linear for \(down\) and \(gate\):  
   Since LLaMA 2 7B uses SwiGLU FFN, the FLOPs are:  
   \(2B \cdot( 2sH · 8/3H + 2esH+2k\cdot(sHd+sd8/3H))\).

5. MoE linear for \(up\):  
   The FLOPs are:  
   \(2BsH · 8/3H + 2Bs8/3He+2k\cdot(Bs8/3Hd+BsrH)\).

Across \(L\) layers, including the vocabulary embedding transformation, the total FLOPs are:
\begin{align}
\text{FLOPs}_{\text{LoRA-MoE}} =   BL \left( \frac{52}{3}esH+ \frac{41}{2} sH^2 +4s^2H + \frac{69}{2}ksHd\right) + 2BsHV \\
=  BL \left( \frac{52}{3}esH+ \frac{41}{2} sH^2 +4s^2H + \frac{69}{2}\frac{k}{e}sHr\right) + 2BsHV
\end{align}


\end{document}


% This document was modified from the file originally made available by
% Pat Langley and Andrea Danyluk for ICML-2K. This version was created
% by Iain Murray in 2018, and modified by Alexandre Bouchard in
% 2019 and 2021 and by Csaba Szepesvari, Gang Niu and Sivan Sabato in 2022.
% Modified again in 2023 and 2024 by Sivan Sabato and Jonathan Scarlett.
% Previous contributors include Dan Roy, Lise Getoor and Tobias
% Scheffer, which was slightly modified from the 2010 version by
% Thorsten Joachims & Johannes Fuernkranz, slightly modified from the
% 2009 version by Kiri Wagstaff and Sam Roweis's 2008 version, which is
% slightly modified from Prasad Tadepalli's 2007 version which is a
% lightly changed version of the previous year's version by Andrew
% Moore, which was in turn edited from those of Kristian Kersting and
% Codrina Lauth. Alex Smola contributed to the algorithmic style files.

