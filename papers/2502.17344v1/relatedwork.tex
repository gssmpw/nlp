\section{Related Work}
% To date, user behaviors analyzed for suspicious similarities include the sharing of URLs or hashtags \cite{Pacheco_2020, giglietto2020takes, coURL,Pacheco_2021, burghardt2023socio}, re-sharing the same posts or users, operating with temporal synchronicity \cite{Pacheco_2021, synchrotrap,suresh2023tracking,debot, magelinski2022synchronized, tardelli2023temporal}, and posting content with very similar text \cite{nizzoli2021coordinated, Pacheco_2020,suresh2023tracking}. Pacheco et al., \cite{Pacheco_2021} also leverage suspicious account handle sharing behavior to identify coordinated accounts. 
% In addition to examining the collective behavior of accounts, some work has also looked at individual account traces that betray some degree of account automation \cite{ferrara2016rise,badawy2018analyzing,mazza2022investigating}. Use of software to automate account behavior has some benign applications, but it also can be critical to running an information operation with limited human resources. In addition to approaches that consider one behavior of the above behaviors, attempts have been made to analyze mutliple user behaviors within a single model \cite{syncActionFrame,multiviewClustering,weber2021amplifying,vargas2020detection,sharma2021identifying,luceri2024unmasking,erhardt2023hidden}.