

\documentclass{article}

% Recommended, but optional, packages for figures and better typesetting:
\usepackage{microtype}
\usepackage{graphicx}
\usepackage{enumitem}
\usepackage{subfigure}
\usepackage{booktabs} % for professional tables

% hyperref makes hyperlinks in the resulting PDF.
\usepackage{hyperref}

\usepackage{url}
\urlstyle{same}

% Attempt to make hyperref and algorithmic work together better:
\newcommand{\theHalgorithm}{\arabic{algorithm}}

% Use the following line for the initial blind version submitted for review:
%\usepackage{conf2025}

% If accepted, instead use the following line for the camera-ready submission:
\usepackage[accepted]{conf2025}

% For theorems and such
\usepackage{amsmath}
\usepackage{amssymb}
\usepackage{mathtools}
\usepackage{amsthm}

% if you use cleveref..
\usepackage[capitalize,noabbrev]{cleveref}

%%%%%%%%%%%%%%%%%%%%%%%%%%%%%%%%
% THEOREMS
%%%%%%%%%%%%%%%%%%%%%%%%%%%%%%%%
\theoremstyle{plain}
\newtheorem{theorem}{Theorem}[section]
\newtheorem{proposition}[theorem]{Proposition}
\newtheorem{lemma}[theorem]{Lemma}
\newtheorem{corollary}[theorem]{Corollary}
\theoremstyle{definition}
\newtheorem{definition}[theorem]{Definition}
\newtheorem{assumption}[theorem]{Assumption}
\newtheorem{problemst}{Problem Statement}[section] % Michelle edit
\theoremstyle{remark}
\newtheorem{remark}[theorem]{Remark}

\usepackage{xfrac} % Michelle edit
\usepackage{tikz}
\def\checkmark{\tikz\fill[scale=0.4](0,.35) -- (.25,0) -- (1,.7) -- (.25,.15) -- cycle;}


% Todonotes is useful during development; simply uncomment the next line
%    and comment out the line below the next line to turn off comments
%\usepackage[disable,textsize=tiny]{todonotes}
\usepackage[textsize=tiny]{todonotes}


% The \conftitle you define below is probably too long as a header.
% Therefore, a short form for the running title is supplied here:
\conftitlerunning{Controllable Sequence Editing for Counterfactual Generation}

\usepackage{xspace}
\newcommand{\marinka}[1]{{\color{magenta} [MZ: {#1}]}}
\newcommand{\michelle}[1]{{\color{purple} [ML: {#1}]}}
\newcommand{\kevin}[1]{{\color{blue} [KL: {#1}]}}
\newcommand{\shvat}[1]{{\color{cyan} [SM: {#1}]}}
\newcommand{\name}{\textsc{Clef}\xspace}
\newcommand{\longname}{\underline{C}ontro\underline{L}lable sequence \underline{E}diting for counter\underline{F}actual generation\xspace}
\newcommand{\std}[1]{\scriptsize{$\pm$#1}}
\newcommand{\xhdr}[1]{\noindent{{\bf #1.}}}


\begin{document}

\twocolumn[
\conftitle{Controllable Sequence Editing for Counterfactual Generation}

% It is OKAY to include author information, even for blind
% submissions: the style file will automatically remove it for you
% unless you've provided the [accepted] option to the conf2025
% package.

% List of affiliations: The first argument should be a (short)
% identifier you will use later to specify author affiliations
% Academic affiliations should list Department, University, City, Region, Country
% Industry affiliations should list Company, City, Region, Country

% You can specify symbols, otherwise they are numbered in order.
% Ideally, you should not use this facility. Affiliations will be numbered
% in order of appearance and this is the preferred way.
%\confsetsymbol{equal}{*}

\begin{confauthorlist}
\confauthor{Michelle M.~Li}{hms,berkowitz}
\confauthor{Kevin Li}{hms,mit}
\confauthor{Yasha Ektefaie}{hms}
\confauthor{Shvat Messica}{hms}
\confauthor{Marinka Zitnik}{hms,berkowitz,kempner}
\end{confauthorlist}

%\confaffiliation{berkowitz}{The Ivan and Francesca Berkowitz Family Living Laboratory Collaboration at Harvard Medical School and Clalit Research Institute, Boston, MA, USA}
%\confaffiliation{dbmi}{Department of Biomedical Informatics, Harvard Medical School, Boston, MA, USA}

\confaffiliation{hms}{Harvard Medical School}
\confaffiliation{berkowitz}{The Ivan and Francesca Berkowitz Family Living Laboratory Collaboration at Harvard Medical School and Clalit Research Institute}
\confaffiliation{mit}{Massachusetts Institute of Technology}
\confaffiliation{kempner}{Kempner Institute for the Study of Natural and Artificial Intelligence at Harvard University}

\confcorrespondingauthor{}{michelleli@g.harvard.edu}
\confcorrespondingauthor{}{marinka@hms.harvard.edu}

% Michelle M. Li
% Kevin Li
% Yasha Ektefaie
% Shvat Messica
% Marinka Zitnik

% You may provide any keywords that you
% find helpful for describing your paper; these are used to populate
% the "keywords" metadata in the PDF but will not be shown in the document
\confkeywords{Machine Learning}

\vskip 0.3in
]

% this must go after the closing bracket ] following \twocolumn[ ...

% This command actually creates the footnote in the first column
% listing the affiliations and the copyright notice.
% The command takes one argument, which is text to display at the start of the footnote.
% The \confEqualContribution command is standard text for equal contribution.
% Remove it (just {}) if you do not need this facility.

\printAffiliationsAndNotice{}  % leave blank if no need to mention equal contribution
%\printAffiliationsAndNotice{\confEqualContribution} % otherwise use the standard text.


\begin{abstract}
Vision-Language Models (VLMs) occasionally generate outputs that contradict input images, constraining their reliability in real-world applications. While visual prompting is reported to suppress hallucinations by augmenting prompts with relevant area inside an image, the effectiveness in terms of the area remains uncertain. This study analyzes success and failure cases of Attention-driven visual prompting in object hallucination, revealing that preserving background context is crucial for mitigating object hallucination.

\end{abstract}

\section{Introduction}
Large language models (LLMs) show significant performance in various downstream
tasks~\citep{brown_language_2020,openai_gpt-4_2024,dubey_llama_2024}. Studies
have found that training on high quality corpus improves the ability of LLMs
to solve different problems such as writing code, doing math exercises, and
answering logic questions~\citep{cai_internlm2_2024,deepseek-ai_deepseek-v3_2024,qwen_qwen25_2024}.
Therefore, effectively selecting high-quality text data is an important subject for
training LLM.

\begin{figure}[t]
    \centering
    \includegraphics[width=\linewidth]{figures/head.pdf}
    \caption{The overview of CritiQ. We (1) employ human annotators to annotate $\sim$30
    pairwise quality comparisons, (2) use CritiQ Flow to mine quality criteria, (3)
    use the derived criteria to annotate 25k pairs, and (4) train the CritiQ Scorer to
    perform efficient data selection.}
    \label{fig:overview}
\end{figure}

To select high-quality data from a large corpus, researchers manually design heuristics~\citep{dubey_llama_2024,rae_scaling_2022},
calculate perplexity using existing LLMs~\citep{marion2023moreinvestigatingdatapruning,wenzek2019ccnetextractinghighquality},
train classifiers~\citep{brown_language_2020,dubey_llama_2024,xie_data_2023} and
query LLMs for text quality through careful prompt engineering~\citep{gunasekar_textbooks_2023,wettig_qurating_2024,sachdeva_how_2024}.
Large-scale human annotation and prompt engineering require a lot of human
effort. Giving a comprehensive description of what high-quality data is like is also
challenging. As a result, manually designing heuristics lacks robustness and introduces
biases to the data processing pipeline, potentially harming model performance
and generalization. In addition, quality standards vary across different
domains. These methods can not be directly applied to other domains without significant
modifications.

To address these problems, we introduce CritiQ, a novel method to automatically
and effectively capture human preferences for data quality and perform efficient data
selection. Figure~\ref{fig:overview} gives an overview of CritiQ, comprising an agent
workflow, CritiQ Flow, and a scoring model, CritiQ Scorer. Instead of manually describing
how high quality is defined, we employ LLM-based agents to summarize quality
criteria from only $\sim$30 human-annotated pairs.

CritiQ Flow starts from a knowledge base of data quality criteria. The worker
agents are responsible to perform pairwise judgment under a given
criterion. The manager agent generates new criteria and refines them through reflection
on worker agents' performance. The final judgment is made by majority voting among
all worker agents, which gives a multi-perspective view of data quality.

To perform efficient data selection, we employ the worker agents to annotate a randomly
selected pairwise subset, which is ~1000x larger than the human-annotated one.
Following \citet{korbak_pretraining_2023,wettig_qurating_2024}, we train CritiQ
Scorer, a lightweight Bradley-Terry model~\citep{bradley_rank_1952} to convert
pairwise preferences into numerical scores for each text. We use CritiQ Scorer to
score the entire corpus and sample the high-quality subset.

For our experiments, we established human-annotated test sets to quantitatively
evaluate the agreement rate with human annotators on data quality preferences. We implemented the manager agent by \texttt{GPT-4o} and the worker
agent by \texttt{Qwen2.5-72B-Insruct}. We conducted experiments on different
domains including code, math, and logic, in which CritiQ Flow shows a consistent
improvement in the accuracies on the test sets, demonstrating the effectiveness
of our method in capturing human preferences for data quality. To validate the quality
of the selected dataset, we continually train \texttt{Llama 3.1}~\citep{dubey_llama_2024}
models and find that the models achieve better performance on downstream tasks
compared to models trained on the uniformly sampled subsets.

We highlight our contributions as follows. We will release the code to facilitate
future research.

\begin{itemize}
    \item We introduce CritiQ, a method that captures human preferences for data
        quality and performs efficient data selection at little cost of human
        annotation effort.

    \item Continual pretraining experiments show improved model performance in code,
        math, and logic tasks trained on our selected high-quality subset compared to the raw dataset.

    \item Ablation studies demonstrate the effectiveness of the knowledge base and
        the the reflection process.
\end{itemize}

\begin{figure*}[t]
    \centering
    \includegraphics[width=\linewidth]{figures/method.pdf}
    \caption{CritiQ Flow comprises two major components: multi-criteria pairwise
    judgment and the criteria evolution process. The multi-criteria pairwise
    judgment process employs a series of worker agents to make quality
    comparisons under a certain criterion. The criteria evolution process aims to
    obtain data quality criteria that highly align with human judgment through
    an iterative evolution. The initial criteria are retrieved from the
    knowledge base. After evolution, we select the final criteria to annotate
    the dataset for training CritiQ Scorer.}
    \label{fig:method}
\end{figure*}



\section{Related work}
\subsection{Retrieval-Augmented Generation}
Retrieval-Augmented Generation (RAG) has been widely applied to enhance the performance of Large Language Models (LLMs) by retrieving relevant information from external sources, addressing the limitation of LLMs' restricted context windows, improving factual accuracy, and mitigating hallucinations~\cite{fan2024survey, gao2023retrieval}. Most RAG systems primarily process text data by first splitting it into chunks~\cite{finardi2024chronicles}. When a query is received, RAG retrieves relevant chunks either through lexical search~\cite{ram2023context} or by computing semantic similarity~\cite{karpukhin2020dense}, embeddings both the query and text chunks into a shared vector space. Advanced techniques, such as pre-retrieval processing~\cite{ma2023query, zheng2023take} and post-retrieval processing~\cite{dong2024don, xu2023recomp}, as well as fine-tuning strategies~\cite{li2023structure}, have further enhanced RAG’s effectiveness across various domains, including QA)~\cite{yan2024corrective}, dialogue generation~\cite{izacard2023atlas}, and text summarization~\cite{jiang2023active}.

Several studies have evaluated the effectiveness of RAG systems across various tasks~\cite{yu2024evaluation, chen2024benchmarking, es2023ragas}, such as multi-hop question answering~\cite{tang2024multihop}, biomedical question answering~\cite{xiong2024benchmarking}, and text generation~\cite{liu2023recall}. However, no existing study has simultaneously and systematically evaluated and compared RAG and GraphRAG on these general text-based tasks.
% \yu{However, no existing study has simultaneously and systematically evaluated and compared RAG and GraphRAG on these general tasks. Add simultaneously to highlight our novelty is not evaluate any of them but at the same time and so we can compare.}

\begin{figure*}[!htb]
    \centering
   \includegraphics[width=\linewidth]{figures/RAGGraphRAG1.pdf}
    \caption{The illustration of RAG, KG-based GraphRAGs and Community-based GraphRAGs.}
    \label{fig:framework}
    \vspace{-0.2in}
\end{figure*}

\subsection{Graph Retrieval-Augmented Generation}
While RAG primarily processes text data, many real-world scenarios involve graph-structured data, such as knowledge graphs (KGs), social graphs, and molecular graphs~\cite{xia2021graph, ma2021deep}. GraphRAG~\cite{han2024retrieval, peng2024graph} aims to retrieve information from various types of graph-structured data. The inherent structure of graphs enhances retrieval by capturing relationships between connected nodes. For example, hyperlinks between documents can improve retrieval effectiveness in question answering tasks\cite{li2022dynamic}.  Currently, most GraphRAG studies focus on retrieving information from existing KGs for downstream tasks such as KG-based QA~\cite{tian2024graph, yasunaga2021qa} and Fact-Checking~\cite{kim2023factkg}. 
% The retrieved information can be subgraphs~\cite{he2024g} relevant to the query or reasoning paths~\cite{luo2023reasoning} that facilitate inference. 

Despite leveraging the existing graphs, recent studies have explored incorporating graph construction into GraphRAG to enhance text-based tasks. For example, \citet{dong2024don} construct document graphs using Abstract Meaning Representation (AMR) to improve document ranking. \citet{edge2024local} construct graphs from documents using LLMs, where nodes represent entities and edges capture relationships between them. Based on these graphs, they generate hierarchical communities and corresponding community summaries or reports. Their approach focuses on the global query summarization task, retrieving information from both the constructed graphs and their hierarchical communities. Additionally, \citet{han2025reasoning} propose an iterative graph construction approach using LLMs to improve reasoning tasks.
% \yu{Here I feel we spend so much context on introducing graph-based GraphRAG. However, the motivation of this work (also in the introduction part) is more around the issue of GraphRAG applied on text data. So it might be better to  emphasize more on that aspect}

These studies highlight the potential of GraphRAG in processing text-based tasks by constructing graphs from textual data.
However, their focus is limited to specific tasks and evaluation settings. It remains unclear how GraphRAG performs on general text-based tasks compared to RAG. More importantly, when and how should GraphRAG be applied to such tasks for optimal effectiveness? Our work aims to bridge this gap by systematically evaluating GraphRAG and comparing it with RAG on general text-based tasks.


% However, most existing studies focus on specific tasks and datasets, and there is no systematic evaluation—like those conducted for RAG—to assess the effectiveness of GraphRAG on text-based tasks using widely adopted datasets and evaluation metrics.\yu{I think here the thing is if we spend so much context on graph-based GraphRAG, then our evaluation should also be in that regard. However, our systematically evaluation is on GraphRAG for traditional text data. It might be better to focus our narratives more on that perspective.} Our work aims to fill this gap by systematically evaluating GraphRAG and comparing it with RAG.


\section{\name}

\name manipulates sequences based on user-specified conditions and temporal coordinates to address  ``what if" questions. Given a sequence, a forecast time step, and a counterfactual condition, \name modifies only the relevant portions of the sequence while preserving unaffected elements, ensuring causal consistency. For example, based on a patient's historical lab test results, \name generates a future lab test trajectory conditioned on a given treatment~(Figure~\ref{fig:problem}a). Architecturally, \name has four key components: (i) a sequence encoder \(F\) that extracts temporal features from historical sequence data, (ii) a condition adapter \(H\) that maps counterfactual conditions to latent representations, (iii) a concept encoder \(E\) that learns temporal concepts, representing trajectory patterns over time, and (iv) a concept decoder \(G\) that applies these concepts to generate counterfactual sequences.



\subsection{Problem definition}
\begin{figure*}[ht]
\begin{center}
\centerline{\includegraphics[width=0.95\textwidth]{FIG/figure2.png}}
\caption{Overview of \name's architecture and capabilities. \textbf{(a)}~Given an input sequence, forecast time, and condition embedding from a frozen pretrained~(\textsc{pt}) embedding model, \name generates a counterfactual sequence via immediate or delayed sequence editing. \textbf{(b)}~\name is composed of a sequence encoder, condition adapter, concept encoder, and concept decoder. \name has two key capabilities: \textbf{(c)}~forecasting counterfactual sequences at any time in the future and under any condition~(e.g.,~medical codes), and \textbf{(d)}~generating counterfactual sequences by intervening on \name's learned temporal concepts.}
\label{fig:clef}
\end{center}
\vskip -0.35in
\end{figure*}



\begin{definition}[Sequence editing]
\label{def:cse}
Sequence editing is the local sample-level modification of sequence $\mathbf{x}$ to generate a counterfactual sequence~$\hat{\mathbf{x}}_{:,t_j}$ under a given condition~$s$ at a specific time~$t_j$. There are two types of controllable sequence editing: immediate and delayed~(Figure~\ref{fig:clef}a).
\begin{itemize}[leftmargin=*,noitemsep,topsep=0pt]
    \item \textbf{Immediate sequence editing:} Given sequence $\mathbf{x}_{:,t_0:t_i}$ and condition $s$ to occur at time $t_{i + 1}$, forecast $\hat{\mathbf{x}}_{:,t_{i + 1}}$.
    \item \textbf{Delayed sequence editing:} Given sequence $\mathbf{x}_{:,t_0:t_i}$ and condition $s$ to occur at time $t_j \geq t_{i + 1}$, forecast $\hat{\mathbf{x}}_{:,t_j}$.
\end{itemize}
\end{definition}
%
Example scenarios for immediate sequence editing include: \textit{What if we perturb the cells now?} and \textit{What if we perform surgery on the patient today?} (Section~\ref{results:r1}). In contrast, delayed sequence editing applies to questions such as: \textit{What if we perturb the cells in ten days?} and \textit{What if we perform surgery on the patient next year?} (Section~\ref{results:r2}).

\begin{definition}[Temporal concept]\label{def:concept}
A temporal concept $\mathbf{c}$ for sequence $\mathbf{x}_{:,t_j:t_k}$ is defined by $\displaystyle \mathbf{c} = \sfrac{\mathbf{x}_{:,t_k}}{\mathbf{x}_{:,t_j}}$ for time steps $t_j$ and $t_k$ where $t_k > t_j$. It can be interpreted as the trajectory (or rate of change of each variable in the sequence) between any pair of time steps.
\end{definition}


\begin{definition}[Controllable sequence editing]
\label{def:concept_cse}
Concept encoder $E$ and decoder $G$ are able to leverage temporal concepts $\mathbf{c}$ to perform controllable sequence editing on dataset $\mathcal{D}$ if the following are satisfied.
\begin{itemize}[leftmargin=*,noitemsep,topsep=0pt]
    \item
    Condition $s$ on $\mathbf{x}_{:,t_0:t_i}$ at time step~$t_j$ learns $\mathbf{c}$ that accurately forecasts $\hat{\mathbf{x}}^s_{:,t_j}$ such that $\hat{\mathbf{x}}^s_{:,t_j} \simeq \mathbf{x}^s_{:,t_j}$.
    \item 
    Counterfactual condition $a \neq s$ on $\mathbf{x}_{:,t_0:t_i}$ at $t_j$ learns $\mathbf{c}' \neq \mathbf{c}$ that forecasts $\hat{\mathbf{x}}^a_{:,t_j}$ such that $\hat{\mathbf{x}}^a_{:,t_j} \neq \hat{\mathbf{x}}^s_{:,t_j}$ and, if known, $\hat{\mathbf{x}}^a_{:,t_j} \simeq \mathbf{x}^a_{:,t_j}$.
    \end{itemize}
\end{definition}



\begin{problemst}[\name]
\label{prob}
Given a sequence encoder~$F$, condition adapter~$H$, concept encoder~$E$, and concept decoder~$G$ trained on a longitudinal dataset $\mathcal{D}$, \name learns temporal concept $\mathbf{c} = E(F(\mathbf{x}_{:,t_0:t_i}, t_j), H(s))$ to forecast $\hat{\mathbf{x}}^s_{:,t_j} = G(\mathbf{x}_{:,t_i}, \mathbf{c})$ for any sequence $\mathbf{x}_{:,t_0:t_i} \in \mathcal{D}$, future time step $t_j > t_i$, and condition $s$.
\end{problemst}


\subsection{\name model}


The input to \name are a multivariate sequence $\mathbf{x}_{:,t_0:t_i}$ with $V$ measured variables, and a condition $s$ and time $t_j > t_i$ for which to forecast $\hat{\mathbf{x}}^s_{:,t_j}$. \name consists of four major components: a sequence encoder $F$, a condition adapter $H$, a concept encoder $E$, and a concept decoder $G$.

\xhdr{Sequence encoder $F$}\label{sec:seq_enc}
%
The sequence encoder~$F$ extracts features from $\mathbf{x}_{:, t_0:t_i}$ such that $\mathbf{h}_\mathbf{x} = F(\mathbf{x}_{:,t_0:t_i})$. Any encoder, including a pretrained multivariate foundation model, can be used. The time encoder in $F$ generates a time positional embedding $\mathbf{h}_t$ for any time $t$ via element-wise summation of the year (sinusoidal), month, date, and hour embeddings. It is additionally used to compute the time delta embedding~$\Delta_{t_i,t_j} = \mathbf{h}_{t_j} - \mathbf{h}_{t_i}$ for the concept encoder~$E$. 


\xhdr{Condition adapter $H$}\label{sec:cond_adp}
%
The embedding~$\mathbf{z}_s$ corresponding to the input condition~$s$ is retrieved from a frozen pretrained embedding model (denoted as~\textsc{pt} in Figure~\ref{fig:clef}a). The condition adapter $H$ projects $\mathbf{z}_s$ into hidden representation~$\mathbf{h}_{s} = H(\mathbf{z}_s)$.

\xhdr{Concept encoder $E$}\label{sec:concept_enc}
%
Given the hidden representations generated by sequence encoder~$F$ and condition adapter~$H$, concept encoder~$E$ learns temporal concept $\mathbf{c} = E(\mathbf{h}_\mathbf{x}, \Delta_{t_i,t_j}, \mathbf{h}_s)$. First, the time delta embedding $\Delta_{t_i,t_j}$ is combined via summation with the condition embedding $\mathbf{h}_s$ to generate a time- and condition-specific embedding $\mathbf{h}_s^{t_j} = \Delta_{t_i,t_j} \oplus \mathbf{h}_s$. Temporal concept $\mathbf{c}$ is learned via an element-wise multiplication of $\mathbf{h}_\mathbf{x}$ and $\mathbf{h}_s^{t_j}$, an optional linear projection using a feedforward neural network~(FNN), and a GELU activation to approximate the trajectory between $t_i$ and $t_j$
\begin{equation}
    \mathbf{c} = \text{GELU}(\text{FFN}(\mathbf{h}_{\mathbf{x}} \odot \mathbf{h}_s^{t_j}))
\end{equation}





\xhdr{Concept decoder $G$}
%
The concept decoder $G$ forecasts $\hat{\mathbf{x}}^s_{:,t_j}$ by performing element-wise multiplication of the latest time~$t_i$ of the input sequence $\mathbf{x}_{:,t_0:t_i}$ (denoted as $\mathbf{x}_{:,t_i}$) and the learned concept~$\mathbf{c}$
\begin{equation}
    \hat{\mathbf{x}}^s_{:,t_j} = \mathbf{c} \odot \mathbf{x}_{:,t_i} 
\end{equation}


\xhdr{Objective function $\mathcal{L}$}
%
The sequence editing objective function $\mathcal{L}$ quantifies the reconstruction error of the predicted $\hat{\mathbf{x}}^s_{:,t_j}$ and the ground truth $\mathbf{x}^s_{:,t_j}$. Here, we use Huber loss
\begin{equation}
    \mathcal{L}(\mathbf{x}^s_{:,t_j}, \hat{\mathbf{x}}^s_{:,t_j}) = \begin{cases}
        0.5 \mathbf{a}^2, & \text{if $\vert \mathbf{a} \vert \leq \delta$} \\
        \delta(\vert \mathbf{a} \vert - 0.5\delta), & \text{otherwise}
    \end{cases}
\end{equation}
where $\mathbf{a} = \mathbf{x}^s_{:,t_j} - \hat{\mathbf{x}}^s_{:,t_j}$.


\section{Experimental setup}
%为了验证 层次化工具堆叠策略(ChemHTS) 在大语言模型(LLM)驱动的化学任务中的有效性,我们在四个典型的化学任务上进行了实验。实验的核心目标是评估 ChemHTS 在不同任务上的计算能力提升情况,并分析其性能优越性的来源。

\subsection{Experiment Setup}

\paragraph{Dataset}

%我们使用ChemLLMBench对ChemHTS在化学方面的能力进行定量评估。ChemLLMBench由一系列化学任务组成,涵盖广泛的化学相关主题。针对选取的四个典型的任务,本研究分别选取与chemdfm评估实验中一致的100个评估数据集实例作为测试集。
%由于chemllmbench只有每个任务的100条评估数据集,因此我们参考xxx论文,对于Text-Based Molecule Design和Molecule Captioning任务,我们从ChEBI-20-MM中除去对应的测试集,各随机抽取相应任务的100条作为训练集。对于Reaction Prediction任务,我们从USPTO-MIT除去对应的测试集,各随机抽取相应任务的100条作为训练集。对于Molecular Property Prediction任务,我们从BBBP,HIV,BACE,Tox21,ClinTox除去对应的测试集,各随机抽取相应任务的50条作为训练集。
%。为确保公平比较,除非另有说明,否则我们在比较不同的 LLM 时使用相同的 100 个样本。

We evaluate the performance of ChemHTS in the field of chemistry using ChemLLMBench. ChemLLMBench~\cite{guo2023largelanguagemodelschemistry} comprises a series of chemistry-related tasks that cover a wide range of chemical topics. 
In this study, we focus on four representative tasks and select 100 evaluation instances for each task, consistent with the evaluation experiments in ChemDFM, as the test set.
%由于 ChemLLMBench 中每个任务的评估数据集仅包含 100 个实例,我们采用与 ~\cite{guo2023largelanguagemodelschemistry} 中类似的方法来选取训练集。
Since the evaluation dataset for each task in ChemLLMBench contains only 100 instances, we adopt a similar approach to that in ~\cite{guo2023largelanguagemodelschemistry} to select the training set. 
For the \textbf{Text-Based Molecule Design and Molecule Captioning} tasks, we randomly sample 100 instances from the ChEBI-20-MM~\cite{liu2025quantitativeanalysisknowledgelearningpreferences}
dataset, excluding the corresponding test set, as the training set. 
For the \textbf{Reaction Prediction} task, we randomly sample 100 instances from the USPTO-MIT~\cite{jin2017predictingorganicreactionoutcomes} dataset, excluding the corresponding test set, as the training set. 
For the \textbf{Molecular Property Prediction} task, we randomly sample 50 instances for each dataset from the BBBP, HIV, BACE, Tox21, and ClinTox~\cite{wu2018moleculenetbenchmarkmolecularmachine} datasets, excluding the corresponding test sets, as the training set.
The details of our dataset are shown in Tab.~\ref{tab:dataset}.
\section{Dataset Generation}
\label{sec:dataset}
\revise{
To train the proposed GNN, we constructed a dataset of building structures and a subset of these structures were subjected to fire simulations using FEA. The dataset generation process is illustrated in \figref{fig:dataset_generation_procedure}. Initially, a total of 33,000 building structures with geometrical details, material properties, and gravity loads were created. Due to randomness in generating these structures, a filter is applied to remove unreasonable data after gravity load simulation, which included 15,377 structures. A trade-off between computational feasibility and model performance is made among the remaining 17,623 structures. As further labeling structures with MIDR requires resource-intensive fire simulations via OpenSeesRT, a large proportion of 16,050 structures is selected as unlabeled dataset. On the other hand, each of the other 1,573 structures was further subjected to 30 different fire simulations, forming the labeled dataset containing $1,573\times 30 = 47,190$ fire cases.} This section details the step-by-step process for generating the dataset, including geometry creation, material property assignment, and simulations due to gravity loads and fire scenarios. 
% To train the proposed neural network, we constructed a dataset comprising building structure data and a subset of fire scenario data. The dataset generation process is illustrated in \figref{fig:dataset_generation_procedure}. 
% A total of 33,000 building structures with geometric details, material properties, and gravity loads were initially created. Out of these, 3,000 structures were selected as labeled data, and the remaining 30,000 were designated as unlabeled data. Further, about half of them filtered out due to instability under gravity loads only. 
\begin{figure*}[h!]
    \centering
    \includegraphics[width=0.8\linewidth]{figures/dataset_filter_procedure.pdf}
    \caption{Workflow for dataset generation (geometry, material property, gravity loads, and fire scenarios).}
    \label{fig:dataset_generation_procedure}
\end{figure*}

\subsection{Geometry Generation}
\label{subsec:geometry_generation}
The geometry of the building structures forms the foundation of the dataset. Regular 
\revise{3D structures} resembling multi-story parking structures or shopping malls were generated, with parameters such as building floor dimensions and story heights selected randomly. Each building structure is composed of multiple rooms, which serve as the basic unit in this study. A room herein is a cuboid space defined by specific length, width, and height. Within a structure, rooms of the same dimensions are uniformly arranged along the length, width, and height, corresponding to the $x$-, $y$-, and $z$-axes, respectively. Structures vary in room size and number of rooms along each axis. Specifically, the room length, width, and height are independently sampled from a uniform distribution within the interval $[2, 5]$ meters along the three directions of the structure. Similarly, the room number along each axis is uniformly sampled independently as an integer within the interval $[2, 7]$, i.e., the maximum number of stories of the buildings simulated in this study is 7.

To introduce variability and simulate real-world scenarios, approximately $8\%$ of structural elements (beams or columns) are randomly removed after initial geometry creation. 
\revise{Such removal is not fire-induced damage, but reflects functional diversity often observed in real buildings, such as open spaces designed for activities in shopping malls, e.g., ice skating rinks. Examples of the generated geometries are illustrated in \figref{fig:example_generated_geometry}, showcasing the diversity and realism of the dataset. This element removal does not affect the definition of room's geometry in the structure and nor does it affect the number of considered fire scenarios.} 

\revise{A range of coefficient of variation values ($3.3\%$ to $17.5\%$) was derived from prior studies that investigated the statistics of geometrical and material properties of structural components of buildings (e.g., \cite{mirza1979variations, lee2004probabilistic}). These studies provide empirical data on the natural variability in parameters such as Young's modulus, yield strength, and dimensions of structural elements due to manufacturing tolerances and material inconsistencies. By selecting $8\%$ for the removal of structural elements in our database, we aimed to maintain a level of variability that is representative of real-world uncertainties while ensuring computational feasibility. This choice ensures that the database captures realistic deviations without introducing extreme cases that may not be commonly encountered in practice.}

\begin{figure*}[h!]
    \centering
    \includegraphics[width=\linewidth]{figures/example_generated_geometry.pdf}
    \caption{Examples of generated structural geometry of different sizes (all dimensions in meters).}
    \label{fig:example_generated_geometry} 
\end{figure*}

{\blockRevise

In this study, we opted for a deterministic square, dimension of $0.1$ m, solid cross-sectional steel elements due to their simplicity in modeling and analysis. Square sections exhibit uniform geometrical properties in all directions, simplifying the computation of structural responses and avoiding complications associated with more complex shapes, such as wide-flange sections, facilitating the computational efficiency and scalability to generate a large dataset. This choice also helps to mitigate issues related to stress concentrations and facilitates a more straightforward representation of structural behavior under thermal loads. 

\textit{Remark:} The selected cross-section provides a comparable flexural rigidity to a $W 130 \times 130 \times 28.1$ wide-flange section (metric units), albeit with significantly higher axial rigidity. This cross-section is acceptable for gravity-load-designed frames under service loading conditions where the models assume fully rigid, moment-resisting beam-column connections for the evaluation of the IDR under thermal loading. This assumption is reasonable in this computational study where the primary interest is to understand the global deformation response of frames under fire conditions. The selection of uniform square cross-sections for both beams and columns, rather than adherence to standard capacity design principles, was made here primarily for computational efficiency and to reduce design parameters in the database generation process. This choice allows for simplified and scalable approach to analyze the fire-induced response of generic steel frames without the need for large section variations, where this study mainly focuses on the fire vulnerability assessment using ML-based predictions. However, if additional loading conditions, e.g., seismic or wind loads, were to be considered, larger sections, strong-column/weak-beam principle, and ductile detailing would be required in the generated buildings for realistic structural behavior under combined loading conditions. Future studies may also consider investigating the influence of variable cross-sectional dimensions and semi-rigid connections on the structural performance under fire conditions. 
} % blockRevise

\subsection{Material Properties}
Steel is chosen as the material for the structures. To reflect real-world variations, we randomly assign one of five slightly different steel material types to each structural element. \revise{
The ranges of material properties are provided in \tabref{tab:material_property_ranges} and the properties are sampled from uniform distributions of the corresponding ranges. These variations simulate differences arising from manufacturing batches or regional material properties. That these properties are at ambient temperature and change when the temperature rises due to a fire. The selection of materials with varying properties is aimed at increasing the diversity of the data. Our goal is to represent as wide a range of data as possible with a limited amount of building structure data, thereby enhancing the generalization ability of the GNN. Our assumed material property ranges are expected to be wider than the real-world conditions based on findings in \cite{mirza1979variations, lee2004probabilistic}. Therefore, we are essentially tackling a more challenging and general task. If we can solve this problem, we are confident that our method will perform equally well or even better in real-world scenarios.
}
\begin{table}[h!]
    \centering
    \caption{Material properties ranges for considered steel structures.}
    \begin{tabular}{lc}
        \toprule
        Property & Range \\
        \midrule
        Young's modulus & [168, 252] GPa \\
        Yield strength & [220, 330] MPa \\
        Strain-hardening ratio & [0.8, 1.2] \% \\
        \bottomrule
    \end{tabular}
    \label{tab:material_property_ranges}
\end{table}

\subsection{Gravity Loads}
Gravity loads are applied to columns and beams based on their \revise{influence (tributary) areas as typically conducted in structural analysis. The considered ``service'' load conditions include the column self-weight and the additional loads directly supported on the beams from their self-weight and weights of the reinforced concrete slabs, people as live load, and building content. An edge beam typically carries approximately half the gravity load supported by a parallel interior beam}. The ranges of gravity loads are listed in \tabref{tab:gravity_load_ranges}. \revise{The loads are sampled from uniform distributions of the corresponding ranges.} Structures that failed to meet an MIDR threshold of $1\%$ under gravity loads were deemed unacceptable designs and filtered out, as such configurations of randomly chosen geometry, material, and gravity load combinations were considered unrealistic from a regulatory and practicality points of view.
\begin{table}[h!]
    \centering
    \caption{Gravity load ranges for considered beams and columns.}
    \begin{tabular}{lc}
        \toprule
        Element & Range (kN/m)  \\
        \midrule
        Column & [0.5, 1.0]  \\
        Edge beam & [1.5, 4.5]  \\
        Interior beam & [3.0, 7.5]  \\
        \bottomrule
    \end{tabular}
    \label{tab:gravity_load_ranges}
\end{table} 

\subsection{Rule-based Thermal Load Generation}
\label{subsec:thermal_load_generation}
To evaluate a building's structural response during a fire event, we employed a simplified rule-based approach for thermal load generation. 
% Previous studies \cite{nan_structuralfire_2023} have demonstrated that steel structures rapidly equilibrate with surrounding gases temperatures due to efficient heat exchange. Consequently, gas temperatures can be directly used as inputs for FEA tools, e.g., OpenSees, simplifying the process of modeling thermal loads. 
% Accurately simulating temperature fields in fire scenarios poses significant challenges. Advanced thermodynamic simulations, such as those performed using Fire Dynamics Simulator (FDS) \cite{mcgrattan_fire_2000}, provide precise temperature predictions. However, these methods are hindered by high computational costs, prolonging execution times, and limited scalability, making them impractical for generating large datasets. Additionally, real-world fire loads often display substantial spatial variability across different rooms \cite{dundar_fire_2023}, resulting in scenario-specific temperature fields with limited generalizability. For example, studies on bridge fires \cite{he_study_2024} have demonstrated that environmental factors, such as wind speeds, can significantly influence temperature distributions. Furthermore, even within identical scenarios, variations in fire modeling methodologies can produce distinctly different temperature fields \cite{zhang_temperature_2020, du_new_2012}. These challenges emphasize the need for efficient and adaptable methods to generate fire temperature data.
% To address these issues, we adopted a rule-based approach to model temperature variations. 
According to \cite{spearpoint_fire_2008}, a typical fire development follows a predictable pattern. During the {\em{growth stage}}, the temperature rises slowly and approximately linearly after ignition. This is followed by the {\em{flashover stage}}, where temperatures increase rapidly to peak values. After reaching the peak, the temperature either stabilizes or continues to rise slowly until the {\em{decay stage}} begins. Inspired by this fire development pattern, we describe the temperature evolution in time, $t$, prior to the decay stage in two distinct stages:
\begin{enumerate}
    \item {\bf{Initial linear increase stage}}: For $t \in [0, t_1)$, temperature increases gradually and linearly as the fire spreads through the building. This stage represents the time before the fire directly affects a structural element.  
    \item {\bf{ISO 834 fire curve stage}}: For $t \in [t_1, t_{\thre}]$, temperature rises rapidly following the ISO 834 curve \cite{ISO834}, modeling the direct impact of the fire on the structural element. 
\end{enumerate}
The slope of the linear temperature increase, $c$, and the transition time, $t_1$, are influenced by the spatial relationship between the fire source and the structural element. For the second stage of temperature evolution, we utilize the ISO 834 curve, a widely accepted standard for fire resistance testing. This standardized fire curve describes the temperature rise over time, enabling rapid and consistent thermal fields across various scenarios. The duration of fire simulation in this study is set to $t_{\thre}=60$ minutes. This value represents the upper limit for the temperature evolution of each structural element, providing a consistent basis for analyzing the structural response to fire.

Let $(x, y, z)$ represents the midpoint of a structural element and $(x_{\subfire}, y_{\subfire}, z_{\subfire})$ the fire source point. \revise{Integer parameters $h$ and $h_{\subfire}$ correspond to the respective floor levels of the element and the fire source}. The temperature evolution for each element is expressed as follows:
\begin{enumerate}
    \item Linear increase stage ($0 < t < t_1$):
    \begin{equation}
    T(t) = c \cdot t,
    \end{equation}
    where $c$, the rate of temperature increase ($^\circ\mathrm{C}/\mathrm{min}$), depends on the height difference between the element, $h$, and the fire source, $h_{\subfire}$:
    \begin{equation}
        c = 
        \begin{cases} 
        5\left/\left(h - h_{\subfire} + 1\right)\right., & h \geq h_{\subfire}, \\
        2\left/\left(h_{\subfire} - h\right)\right., & h < h_{\subfire}.
        \end{cases}
    \end{equation}
     \item ISO 834 stage ($t \geq t_1$):
\begin{equation}
    T(t) = c \cdot t_1 + 345 \log_{10} \left(8 \left(t - t_1\right) + 1\right).
\end{equation}
\end{enumerate}

The transition (arrival) time $t_1$, marking the end of the linear stage, depends on the spatial distance between the fire source and the element. We define the following two Euclidean distances $L_p$ in the $xy$ plane and $L_s$ in the $xyz$ space:
\begin{eqnarray}
L_p & \triangleq & \sqrt{(x - x_{\subfire})^2 + (y - y_{\subfire})^2}, \\
\label{eq:Lp}
L_s & \triangleq & \sqrt{(x - x_{\subfire})^2 + (y - y_{\subfire})^2 + (z - z_{\subfire})^2}.
\label{eq:Ls}
\end{eqnarray}
Accordingly, the transition time, $t_1$, is expressed as follows:
\begin{equation}
    t_1 = 
    \begin{cases}
    \beta_{1} \cdot \left(1 - \exp\left\{- L_s\left/\alpha_{1}\right.\right\}\right), & h > h_{\subfire}, \\
    \beta_{2} \cdot \left(1 - \exp\left\{- L_p\left/\alpha_{2}\right.\right\}\right), & h = h_{\subfire}, \\
    \beta_{3} \cdot \left(1 - \exp\left\{- L_s\left/\alpha_{3}\right.\right\}\right), & h < h_{\subfire} .
    \end{cases}
    \label{eq:t1}
\end{equation}
The parameters $\beta_i$ and $\alpha_i$ for determining $t_1$ are summarized in Table~\ref{tab:fire_spread_parameters}. In this study, we take $r_{\mathrm{up}}=0.95$ and $r_{\mathrm{down}}=0.97$.
\begin{table}[ht]
    \centering
    \caption{Fire spread parameters for $t_1$ calculations.}
    \begin{tabular}{lcc}
        \toprule
        Case  & $\beta_i$ & $\alpha_i$  \\
        \midrule
        $i=1$, Upward spread & $16 \left.\left(1-r_{\mathrm{up}}^{\left|h-h_{\subfire}\right|}\right)\right/\left(1-r_{\mathrm{up}}\right)$ & $10$  \\
        $i=2$, Horizontal spread & $18$ & $18$  \\
        $i=3$, Downward spread & $30 \left.\left(1-r_{\mathrm{down}}^{\left|h-h_{\subfire}\right|}\right)\right/\left(1-r_{\mathrm{down}}\right)$ & $5$  \\
        \bottomrule
    \end{tabular}
    \label{tab:fire_spread_parameters}
\end{table}

\figref{fig:t1_curve} illustrates the $t_1$ curves for various fire scenarios: (1) fire originating on the lower floor, $h-h_{\subfire}=1$ with rapid upward spread, (2) fire on the same floor, $h=h_{\subfire}$ with the fastest spread, and (3) fire on the upper floor, $h_{\subfire}-h=1$ with slow downward spread. The exponential decay in $t_1$ reflects the accelerating fire propagation speed as the distance increases. \figref{fig:t1_curve} also indicates that the employed simplified model is consistent with the Markov chain-based dynamic model given by \cite{cheng_dynamic_2011}, where the rooms at the same floor of the fire point start flashover slightly before the corresponding upper floors. Additionally, $\beta_{1}$ and $\beta_{3}$ are the summation of a geometric sequence, where story level $h$ is the index. The common ratios $r_{\mathrm{up}}<1$ in $\beta_{1}$ and $r_{\mathrm{down}}<1$ in $\beta_{3}$ indicate that the fire speeds up to spread through the next story, which is consistent with the real-world fire spread mechanism given in \cite{hokugo_mechanism_2000}. The temperature profile within the range $t \in [0, t_{\thre}]$ is subsequently used as the thermal load in OpenSeesRT simulations to compute displacements at each structural node at time $t_{\thre}$.
\begin{figure}[h!]
    \centering
    \includegraphics[width=0.8\linewidth]{figures/m204_t1_curve.pdf}
    \caption{Three examples for the $t_1$ curve.}
    \label{fig:t1_curve}
\end{figure}

\revise{
\textit{Remark:} The effects of structural elements, such as concrete floor slabs and partitions, are not explicitly modeled in our approach. Instead, their influence is implicitly captured through the careful selection of the parameters $ \alpha, \beta, r_\mathrm{up} $, and $ r_\mathrm{down} $. This parameterization provides a unified framework for generating temperature fields. Indeed, fire propagation is governed by a multitude of factors and remains an open research question. For instance, if the fire resistance of a floor slab is enhanced by fire protective coating, the corresponding model can account for this by decreasing $\alpha_1$ \& $\alpha_3$, increasing $\beta_1$ \& $\beta_3$, and adopting larger values for $r_\mathrm{up}$ \& $r_\mathrm{down}$, which collectively slow down the vertical spread of fire. Conversely, scenarios involving higher amounts of combustible materials would warrant the opposite adjustments. This flexible and integrated approach avoids the need to design separate models for different fire propagation scenarios while still capturing the essential effects.
}

\revise{
In conclusion, our rule-based approach is a computationally efficient method for approximating fire temperature fields, enabling large-scale dataset generation to train predictive models. By combining ISO 834 fire curves with spatial considerations and embedding structural effects through parameter calibration, the method achieves a balanced trade-off between accuracy and scalability, making it a practical solution for thermal load modeling in fire scenarios. After generating the temperature of each beam or column according to the middle point, the temperature is applied as uniform thermal load to the elements of the structure in question using OpenSeesRT. 
}

% In conclusion, this rule-based approach is a computationally efficient method to approximate fire temperature fields, enabling large-scale dataset generation to train predictive models. By combining ISO 834 fire curves with spatial considerations, the method balances accuracy and scalability, making it a practical solution for thermal load modeling in fire scenarios.

% \subsection{Interstory Drift Ratio}
\subsection{OpenSeesRT Simulation}
\label{subsec:opensees_simulation}

The thermal and mechanical responses of 3D frame structures under combined fire and gravity loads are simulated using OpenSeesRT \cite{perez2024openseesrt}. \revise{In the simulation, the IDR of each node at $t_{\thre}$ is computed using the computed nodal displacements. Each structural model features six degrees of freedom per node (3 translational  and 3 rotational), with linear geometrical transformations (\texttt{geomTransf: Linear}) defining how the element local coordinate systems are mapped to the global coordinate system and assuming small displacements and rotations. Although OpenSeesRT allows a variety of options for modeling finite deformations, in the present simulations and mainly for simplicity, we did not consider large deformations. All bottom nodes (nodes on the ground) are fully constrained in all six degrees of freedom, while degrees of freedom os all other nodes are free.} Material behavior is temperature-dependent and modeled with \texttt{Steel01Thermal}, while fiber-based sections (\texttt{FiberThermal}) capture nonlinear interactions between thermal and mechanical responses at the cross-section level. \revise{Structural elements are represented as displacement-based Euler-Bernoulli beam-columns (\texttt{dispBeamColumnThermal}). This element  formulation accounts for thermal strains (temperature gradients) in the section, which is discretized into fibers. Numerical integration is used along the length of each element using three integration (Gauss) points, one at each end and the third in the middle of the element.}

{\revise{Thermal expansion of steel members plays a crucial role in IDR development. In reality, reinforced concrete floor slabs heat at a different rate than steel members due to their higher thermal mass and lower thermal conductivity. This differential heating can lead to restrained thermal expansion, introducing axial compression in beams and affecting the overall structural response. In this study, explicit {\em{composite action}} between steel members and concrete slabs is not modeled. Instead, our approach focuses on isolating the response of the steel structural frame, which is often the critical load-bearing component in fire scenarios. This assumption aligns with prior studies \cite{Possidente_2024} demonstrating that steel structures reach thermal equilibrium with surrounding gases quickly, allowing the use of uniform thermal loading in fire analysis. Future work could enhance this framework by incorporating slab-beam interaction effects, through a refined FEA for an extended dataset where constraints imposed by floor slabs are explicitly considered.}

The analysis begins with the application of gravity loads, followed by incremental thermal loads simulating the fire exposure. A static nonlinear solver using  \texttt{ExpressNewton} algorithm ensures convergence, while the \texttt{NormDispIncr} test maintains accuracy. An incremental \texttt{LoadControl} scheme with small step sizes is employed to guarantee numerical stability, using 10\% for gravity loads and 1\% for thermal loads. 

\revise{
In the thermal load analysis, uniform thermal load is applied to each beam or column, i.e., the temperature of each element is set to be that at the middle point, according to \secref{subsec:thermal_load_generation}. The \texttt{Steel01Thermal} material allows the properties (e.g., Young's modulus and yield strength) to be adjusted at increasing temperatures according to \cite{EN1993} using its Table 3.1: Reduction factors for the stress-strain relationship of carbon steel at elevated temperatures. For example, if the Young’s modulus at ambient temperature is $E_0$, then as the temperature ($T$) increases, the modulus changes as $E(T) = \eta (T) \times E_0$. \cite{EN1993} directly provides the values of $\eta(T) \in \left[0,1\right] $ at every $100 ^\circ\mathrm{C}$ interval and recommends using linear interpolation to obtain $\eta(T)$ for intermediate values of $T$.
} OpenSeesRT documentation \cite{OpenSeesThermalExamples} provides several examples of thermal analyses.

This modeling framework accommodates variations in material properties, cross-sectional geometries, and temperature profiles, providing robust simulations of structural behavior under fire conditions. The primary settings and configurations for the OpenSeesRT simulations are summarized in \tabref{tab:ops_detail}.
\begin{table}[h!]
    \centering
        \caption{Key settings of OpenSeesRT simulations.}
    \begin{tabular}{l|>{\raggedright\arraybackslash}p{0.6\linewidth}} %
    \toprule
    Modeling Aspect     & Details \\
    \midrule
    Geometry            & 3D models; 6 degrees of freedom per node \\
    Transformation      & geomTransf: Linear \\ 
    Material            & Steel01Thermal \\
    Section             & FiberThermal; Cross-section: $0.1$ m $\times$ $0.1$ m \\ 
    Element type        & {dispBeamColumnThermal} \\ 
    Loading             & Gravity loads: {beamUniform}; Thermal loads: {beamThermal} \\
    Integration scheme  & Incremental {LoadControl}; Step size: $10\%$ (gravity analysis), $1\%$ (thermal analysis) \\
    Nonlinear solver    & {ExpressNewton} algorithm; {UmfPack} solver; Convergence test: {NormDispIncr} tolerance: $10^{-8}$; Maximum \# iterations per step: $1000$. \\ 
    \bottomrule
    \end{tabular}
    \label{tab:ops_detail}
\end{table}

For each structure in the labeled dataset, 30 fire points are selected using a dual-granularity approach, \revise{i.e., two-stage sampling strategy,} to ensure they are well-distributed. Specifically, rooms are sequentially selected, with one fire point randomly chosen within each selected room. If a building is large and contains more than 30 rooms, we randomly select 30 rooms without replacement, i.e., ensuring that no more than one fire point is located in the same room. Conversely, if the building is small and has fewer than 30 rooms, all rooms are initially selected, with one fire point randomly assigned to each room. Additionally, rooms are then selected with replacement until a total of 30 fire points are assigned. \revise{The room-level sampling prioritizes selecting distinct rooms to avoid spatial clustering of fire points, while the point-level sampling ensures intra-room variability. This approach aligns with stratified sampling principles commonly used for efficient spatial representation, where multi-stage sampling strategies optimize coverage and variability, e.g., \cite{arunachalam_generalized_2023}, and enables a more comprehensive characterizing of how the structures respond under fire conditions.}
% This selection method prevents fire points from clustering too closely while maintaining an element of randomness. By distributing fire points in this manner, the 30 fire scenarios are effectively utilized, enabling a more comprehensive characterizing of how the structures respond under fire conditions.

\subsection{Summary of the Dataset Generation}
As discussed in this section and related to  \figref{fig:dataset_generation_procedure}, three key steps were considered in the development of the dataset: 
\begin{enumerate}
    \item {\bf{Filtering process}}: Structures with MIDR exceeding $1\%$ under gravity loads were excluded,  resulting in $1,573$ labeled structures retained for fire simulation and $16,050$ unlabeled structures for training the MFSP predictor.
    \item {\bf{Fire simulations}}: For each retained labeled structure, 30 fire scenarios were simulated using OpenSeesRT, yielding $47,190$ fire cases.
    \item {\bf{Data distribution check}}: MIDR distributions for labeled and unlabeled data under gravity loads were highly similar, because both datasets were generated using the same method. Under fire conditions, the MIDR distribution shifted, reflecting significant structural deformation with values reaching a maximum of about 6\%, an average of 1.70\%, and a standard deviation of 1.12\%. This step ensured a diverse and comprehensive dataset for the proposed predictive framework.
\end{enumerate}
The statistical distribution histograms for MIDR (after applying the $1\%$ filtering threshold \revise{for gravity load responses}) under different loading conditions are plotted in \figref{fig:histogram_mdr}. Figures \ref{fig:histogram_mdr}(a) and \ref{fig:histogram_mdr}(b) show the MIDR distributions of the labeled and unlabeled data, respectively, under gravity loads only. \figref{fig:histogram_mdr}(c) shows the MIDR distribution of the labeled data under the combined effects of gravity and fire loads. Fire load causes the structures to significantly deform, leading to a noticeably \revise{right-skewed} MIDR distribution.

\begin{figure*}[h!]
    \centering
    \includegraphics[width=\linewidth]{figures/histogram_mdr.pdf}
    \caption{Histograms of MIDR for labeled and unlabeled structures with gravity loads and fire cases.}
    \label{fig:histogram_mdr}
\end{figure*}

\revise{
This dataset provides the basis for training and testing the performance of the GNN-based framework. Although we employed a simplified rule-based thermal load generation method compared with conventional CFD-based simulations, the temperature field, the changes of the material properties, and the response of the structures, are all still highly nonlinear and complex. Therefore, it is still a challenging task for the NN to predict the MIDRs based on this dataset.
}
\paragraph{Models}
%本研究的模型主要分为Task-specific specialist models,LLM-based generalist models以及Tool-based Agent models三类,Task-specific specialist models即为在可用的情况下,使用一些非 LLM 任务特定的模型进行比较。Tool-based Agent models是基于chatgpt-4o作为agent智能体,另提供任务给定的tool。为确保公平比较,除非另有说明,否则我们在比较不同的 LLM 时使用相同的100个样本。
We categorize current models into three primary groups: task-specific specialist models, LLM-based generalist models, and tool-based agent models. Task-specific specialist models refer to non-LLM models designed for specific tasks. Tool-based agent models leverage GPT-4o~\cite{openai2024gpt4ocard} as the core agent, augmented with additional tools tailored to the specific task. 
To ensure fair comparisons, we use the same test set for evaluating different models on each task.
\paragraph{Tools}
%由于本研究主要聚焦在ChemHTS方法寻求该任务的最优stacking结构路径,因此对于每个任务给定2个相关且性能良好的tool。对于Text-based Molecule Design任务,给定Name2SMILES、ChemDFM,对于Molecule Captioning,给定SMILES2Description和TextChemT5,对于Molecular Property Prediction,给定SMILES2Property和unimol-v2。对于Reaction Prediction,给定SMILES2Property和chemformer
To facilitate the experimental process, for each task, we provide only the most relevant tools for the two categories: computational tools and retrieval tools. 
Details can be found in Tab.~\ref{tab:dataset}.

%For the Text-based Molecule Design task, Name2SMILES and ChemDFM are provided. For the Molecule Captioning task, SMILES2Description and TextChemT5 are utilized. For the Molecular Property Prediction task, SMILES2Property and unimol-v2 are supplied. For the Reaction Prediction task, SMILES2Property and Chemformer are employed.
\subsection{Results}
\subsubsection{Text-based Molecule Design}
%化学任务介绍,有点太虚了
In the text-based molecule design task, LLMs predict a molecule’s SMILES (Simplified Molecular Input Line Entry System) representation based on a given description, testing their ability to interpret and translate chemical language into valid molecular structures~\cite{zhao2024chemdfmlargelanguagefoundation}.
%任务的目的和内容
%为了评估 LLM 制作合格分子设计的效率,ChemLLMBench 反转了分子说明任务,并要求模型根据分子描述生成分子。具体而言,在基于文本的分子设计任务中,要求模型预测符合给定描述的分子的 SMILES。
%评估指标
Our study employs two sets of metrics to evaluate the performance of the task. The first set of metrics measures the text-based similarity between the predicted SMILES and the gold standard SMILES, including exact match, BLEU, and Levenshtein distance~\cite{haldar2011levenshteindistancetechniquedictionary}. 
The second set of metrics assesses the chemical similarity between the predicted molecules and the reference molecules, encompassing the validity of the predicted SMILES and the FTS (Fingerprint Tanimoto Similarity)~\cite{tanimoto1958elementary}, calculated based on MACCS, RDK, and Morgan~\cite{Morgan1965TheGO}.
%实验结果
%从结果来看,工具增强的智能体模型整体表现最佳,其中**“Ours (Stacking Agent)”** 在多个关键指标上均达到了最优,表明其生成的分子结构既符合目标要求,又保持较高的分子合理性。相比之下,单一工具智能体(如Name2SMILES或ChemDFM)虽然在某些指标上表现良好,但在整体准确性和相似性上不及多工具融合的智能体模型。基于大语言模型的通用模型整体表现不佳,尤其是在Exact和BLEU指标上,GPT-4o(0-shot)的Exact仅为0.01,表明该模型在零样本情况下难以精确生成目标分子。然而,当GPT-4o采用10-shot学习时,性能有所提升,Exact提高至0.12(random)和0.11(scaffold),BLEU也相应提高。然而,其分子失真度(Dis)仍较高,表明大语言模型在缺乏外部工具支持的情况下,难以有效捕捉分子结构信息。
\setlength\tabcolsep{1pt}
%
\begin{table}[!htb]
    \centering
    \tiny
    \definecolor{lightgray}{gray}{0.9} % 定义浅灰色
    \resizebox{0.5\textwidth}{!}{
    \begin{tabular}{lccccccc}
        \toprule
        \textbf{Model} & \textbf{Exact↑} & \textbf{BLEU↑} & \textbf{Dis↑} & \textbf{Validity↓} & \textbf{MACCS↑} & \textbf{RDK↑} & \textbf{Morgan↑} \\
        \midrule
        \rowcolor{lightgray} \multicolumn{8}{c}{\textit{Task-specific specialist models}} \\

        MolXPT~\cite{liu-etal-2023-molxpt} & 0.22 & - & - & \textbf{0.98} & 0.86 & 0.76 & 0.67 \\
        Text+Chem T5~\cite{textchemt5} & 0.32 & 0.85 & 16.87 & 0.94 & \underline{0.90} & \underline{0.82} & \underline{0.75} \\
        Mol-Instruction~\cite{fang2024molinstructionslargescalebiomolecularinstruction}& 0.02 & 0.35 & 41.40 & 1.00 & 0.41 & 0.23 & 0.15 \\
        \midrule
        \rowcolor{lightgray} \multicolumn{8}{c}{\textit{LLM-based generalist models}} \\

        GPT-4o~\cite{openai2024gpt4ocard} & 0.01 & 0.57 & 52.85 & 0.91 & 0.71 & 0.54 & 0.38 \\
        Deepseek-R1~\cite{deepseekai2025deepseekr1incentivizingreasoningcapability} & 0.02 & 0.56 & 92.29 & 0.57 & 0.48 & 0.38  & 0.31 \\
        Llama3-70b~\cite{llama3modelcard} & 0.03 & 0.57 & 46.63 & 0.78 & 0.57 & 0.40 & 0.30 \\
        Llama3-8b~\cite{llama3modelcard} & 0.01 &0.41 &155.17 &0.45 &0.27 &0.15 &0.11\\
        ChemDFM-13B~\cite{zhao2024chemdfmlargelanguagefoundation}& 0.32 & 0.85 & \underline{11.58} & 0.94 & 0.81 & 0.73 & 0.67 \\
        \midrule
        \rowcolor{lightgray} \multicolumn{8}{c}{\textit{Tool-based Agent models}} \\
        Agent (1-tool, Name2SMILES) & 0.25 & 0.70 & 84.81 & 0.72 & 0.67 & 0.61 & 0.56 \\
        Agent (1-tool, ChemDFM) & 0.35 & 0.86 & 12.66 & \underline{0.97} & 0.88 & 0.79 & 0.73 \\
        Agent (2-tool) & \underline{0.34} &\underline{0.87} & 12.63 & 0.94 & 0.85 & 0.80 & 0.74 \\
        Ours (Stacking Agent) & \textbf{0.38} & \textbf{0.93} & \textbf{8.68} & 0.96 & \textbf{0.92} & \textbf{0.87} &\textbf{ 0.80} \\
        \bottomrule
    \end{tabular}
    }
    \caption{Benchmark results of different models in text-based molecule design tasks. All LLM-based generalist models are evaluated on 0-shot.}
    \label{tab:text_based}
\end{table}


From the results in Tab.~\ref{tab:text_based}, our Stacking Agent outperforms other models in this task. Not only do the generated molecular structures meet the target requirements, but they also maintain high molecular validity. In contrast, models based on large language models perform poorly, particularly in terms of Exact and BLEU scores, suggesting difficulty in accurately generating target molecules in a 0-shot setting. This highlights the challenge of capturing molecular structural information without external tool support. The chemistry-specific ChemDFM performs well in the 0-shot setting, surpassing other task-specific models. However, our Stacking Agent, by combining ChemDFM with the Name2SMILES tool, achieves the best results across multiple key metrics, with a BLEU score of 0.93, outperforming ChemDFM’s 0.85.

\subsubsection{Molecule captioning}
%Molecular captioning~\cite{guo2023largelanguagemodelschemistry} is an important task in computational chemistry, offering valuable insights and applications across various domains such as drug discovery, materials science, and chemical synthesis. Given a molecule as input, the goal of this task is to generate a textual description that accurately captures the key features, properties, and functional groups of the molecule.
%为了评估模型将复杂化学信息转化为人类易于理解的语言描述的能力,我们引入了 Molecule Captioning 任务。该任务要求 LLM 不仅能够准确识别给定的 SMILES(Simplified Molecular Input Line Entry System)符号所代表的分子,还需使用自然语言生成该分子的简明描述。为衡量模型在此任务上的表现,我们采用了传统的自然语言处理评价指标,如 BLUE 和 ROUGE,以评估模型在测试集上生成的分子描述与参考描述之间的相似性。
To evaluate the ability of the model to translate complex chemical information into human-readable language descriptions, we introduce the Molecule Captioning task~\cite{guo2023largelanguagemodelschemistry}. This task requires LLMs not only to accurately recognize the molecule represented by a given SMILES string, but also to generate a concise natural language description of the molecule. 
%评估指标
To assess the model's performance on this task, we employ traditional natural language processing evaluation metrics, such as BLEU and ROUGE, to measure the similarity between the molecule descriptions generated by the model and the reference descriptions in the test set.
\setlength\tabcolsep{1pt}
%
\begin{table}[!htb]
    \centering
    \tiny
    \definecolor{lightgray}{gray}{0.9} % 定义浅灰色
    \resizebox{0.5\textwidth}{!}{
    \begin{tabular}{lccccc}
        \toprule
        \textbf{Model} & \textbf{BLEU-2↑} & \textbf{BLEU-4↑} & \textbf{ROUGE-1↑} & \textbf{ROUGE-2↑} & \textbf{ROUGE-L↑} \\
        \midrule
        \rowcolor{lightgray}\multicolumn{6}{c}{\textit{Task-specific specialist models}} \\
        Text+Chem T5~\cite{textchemt5} & 0.63 & 0.54 & \underline{0.68} & \underline{0.54} & \underline{0.62} \\
        MolXPT~\cite{liu-etal-2023-molxpt} & 0.59 & 0.50 & 0.66 & 0.51 & 0.60 \\
        InstructMol~\cite{cao2024instructmolmultimodalintegrationbuilding} & 0.48 & 0.37 & 0.57 & 0.39 & 0.50 \\
        Mol-Instruction~\cite{fang2024molinstructionslargescalebiomolecularinstruction} & 0.25 & 0.17 & 0.33 & 0.29 & 0.27 \\
        \midrule
        \rowcolor{lightgray}\multicolumn{6}{c}{\textit{LLM-based generalist models}} \\
        GPT-4o~\cite{openai2024gpt4ocard} & 0.26 & 0.17 & 0.10 & 0.00 & 0.30 \\
        Deepseek-R1~\cite{deepseekai2025deepseekr1incentivizingreasoningcapability} & 0.40 &0.25 & 0.10 &0.02 &0.21\\
        Llama3-70b~\cite{llama3modelcard} & 0.11 & 0.07 & 0.06 &0.00 & 0.12\\
        Llama3-8b~\cite{llama3modelcard} & 0.04 & 0.03 & 0.05 & 0.00 & 0.09\\
        ChemDFM-13b~\cite{zhao2024chemdfmlargelanguagefoundation}& 0.32 & 0.27 & 0.49 & 0.37 & 0.48 \\
        \midrule
        \rowcolor{lightgray}\multicolumn{6}{c}{\textit{Tool-based Agent models}} \\
        Agent (1-tool, SMILES2Description) & 0.59 & 0.52 & 0.43 & 0.29 & 0.51 \\
        Agent (1-tool, Text+ChemT5) & 0.60 &0.49 &0.39 &0.23 & 0.50\\
        Agent (2-tool) & \underline{0.64} & \underline{0.56} & 0.45 & 0.29 & 0.55 \\
        Ours (Stacking Agent) & \textbf{0.73} & \textbf{0.69} & \textbf{0.70} & \textbf{0.58} & \textbf{0.76} \\
        \bottomrule
        \end{tabular}
    }
    \caption{Benchmark results of different models in molecule captioning tasks. All LLM-based generalist models are evaluated on 0-shot.}
    \label{tab:captioning}
\end{table}


%分析实验结果
%综合来看,实验表明任务特定的专家模型在分子描述任务上仍然具有较大的优势,而通用大语言模型在该任务上表现较差。然而,通过工具增强的智能体模型,特别是 Stacking Agent,能够有效提升文本生成质量,所有指标均显著提升,甚至超越任务特定模型.这表明智能体方法通过有效集成多种工具,能够显著提高分子描述任务的生成质量。此外,对比 Agent(1-tool) 和 Agent(2-tool),可以发现随着工具数量的增加,模型的性能有所提升(如 BLEU-4 从 0.54 提升至 0.56),但单纯增加工具并未带来突破性的改进,而通过ChemHTS方法构建的Stacking Agent 通过更高效的工具组合策略,达到了最佳性能。
As shown in Tab.~\ref{tab:captioning}, Stacking Agent, built using the more efficient ChemHTS method, achieves the best overall performance across all metrics. For text generation tasks, task-specific models still hold a significant advantage in molecule captioning, while , aside from large parameters models like GPT-4o and Deepseek-R1, other large language models still perform poorly. Additionally, a comparison between Agent (1-tool) and Agent (2-tool) shows that increasing the number of tools boosts performance (e.g., BLEU-4 improves from 0.54 to 0.56). However, simply adding more tools doesn’t lead to major gains.

\subsubsection{Molecular Property Prediction}
Molecular property prediction~\cite{Guo_2021,wang2021chemicalreactionawaremoleculerepresentationlearning} is a fundamental task in computational chemistry that has garnered significant attention in recent years due to its potential applications in drug discovery, materials science, and other areas of chemistry. 
%该任务要求这模型根据给定分子的结构预测其化学和物理特性,其数据集由 MoleculeNet 中的五个任务组成 [Wu et al., 2018],包括 BACE、BBBP、HIV、ClinTox 和 Tox21。其中,BACE 和 BBBP 各包含一个平衡的二元分类任务。HIV 包含一个不平衡的二元分类任务。ClinTox 和 Tox21 分别包含两个和二十一个不平衡的二元分类任务。为了解决这些任务中严重的标签不平衡问题,本研究引入了接收者操作特性曲线下面积 (AUCROC) 指标作为评估指标 [Bradley, 1997]。
The task requires models to predict the chemical and physical properties of a given molecule based on its structure. The dataset consists of five tasks drawn from MoleculeNet~\cite{wu2018moleculenetbenchmarkmolecularmachine}, including BACE, BBBP, HIV, ClinTox, and Tox21. Among them, BACE and BBBP each consist of a balanced binary classification task. HIV includes an imbalanced binary classification task. ClinTox and Tox21 contain two and twenty-one imbalanced binary classification tasks, respectively. To address the severe label imbalance in these tasks, our study employs the Area Under the Receiver Operating Characteristic Curve (AUC-ROC) as the primary evaluation metric~\cite{tafvizi2022attributingaucrocanalyzebinary}. However, considering the computational challenges associated with large language models and the balanced binary task, we also incorporate Accuracy as an additional metric to provide a more comprehensive assessment.

\setlength\tabcolsep{1pt}
%
\begin{table}[!htb]
    \centering
    \tiny
    \definecolor{lightgray}{gray}{0.9} % 定义浅灰色
    \resizebox{0.49\textwidth}{!}{
    \begin{tabular}{lcccccccccccc}
    \toprule
\multirow{2}{*}{\textbf{Model}}            & \multicolumn{2}{c}{\textbf{BACE}} & \multicolumn{2}{c}{\textbf{BBBP}} & \multicolumn{2}{c}{\textbf{ClinTox}} & \multicolumn{2}{c}{\textbf{HIV}} & \multicolumn{2}{c}{\textbf{Tox21}} & \multicolumn{2}{c}{\textbf{Avg}} \\\cmidrule(lr){2-3} \cmidrule(lr){4-5} \cmidrule(lr){6-7} \cmidrule(lr){8-9} \cmidrule(lr){10-11} \cmidrule(lr){12-13}
                                  & \textbf{ACC↑}       & \textbf{AUC↓}      & \textbf{ACC↑}       & \textbf{AUC↓}      & \textbf{ACC↑}         & \textbf{AUC↓}       & \textbf{ACC↓}       & \textbf{AUC↑}     & \textbf{ACC↑}       & \textbf{AUC↑}      & \textbf{ACC↑}       & \textbf{AUC↓}     \\
    \midrule
    \rowcolor{lightgray}\multicolumn{13}{c}{\textit{Task-specific specialist models}} \\
Uni-Mol-v2~\cite{ji2024unimol2exploringmolecularpretraining}                        & 0.75      & \textbf{88.9}        & 0.58      & \textbf{82.6}         & \underline{0.51}        & \underline{85.3}          & \textbf{0.96}      & 90.7        & 0.92       & 80.0         & 0.74      & \textbf{85.5}        \\
MolXPT~\cite{liu-etal-2023-molxpt}                            & -         & \underline{88.4}         & -         & 80.0         & -           & \textbf{95.3}          & -         & 78.1        & -          & 77.1         & -         & 83.8        \\
InstructMol~\cite{cao2024instructmolmultimodalintegrationbuilding}                       & -         & 85.9         & -         & 64.0         & -           & -             & -         & 74.0        & -          & -            & -         & -           \\    \midrule
    \rowcolor{lightgray}\multicolumn{13}{c}{\textit{LLM-based generalist models}} \\
GPT-4o~\cite{openai2024gpt4ocard}                            & 0.38      & 38.5         & 0.56      & 57.0         & 0.51        & 51.8          & 0.59      & 54.7        & 0.37       & 36.0         & 0.48      & 47.6        \\
Deepseek-R1~\cite{deepseekai2025deepseekr1incentivizingreasoningcapability}                       & 0.62      & 52.7         & \underline{0.61}      & 63.6         & 0.48        & 48.2          & 0.51      & 50.5        & 0.75       & 60.1         & 0.59      & 55.0        \\
Llama3-70B~\cite{llama3modelcard}                        & 0.55      & 50.9         & 0.59      & 60.1         & 0.48        & 48.8          & 0.20      & 58.3        & 0.59       & 44.9         & 0.48      & 52.6        \\
Llama3-8B~\cite{llama3modelcard}                         & 0.50      & 43.8         & 0.54      & 51.4         & 0.49        & 49.5          & 0.05      & 50.5        & 0.43       & 56.1         & 0.40      & 50.2        \\
ChemDFM-13B~\cite{zhao2024chemdfmlargelanguagefoundation}                       & 0.66      & 78.4         & 0.57      & 66.7         & 0.49        & 89.9          & 0.94      & 73.6        & 0.83       & 79.8         & 0.70      & 77.7        \\
    \midrule
    \rowcolor{lightgray}\multicolumn{13}{c}{\textit{Tool-based Agent models}} \\
Agent (1-tool , SMILES2Property ) & 0.56      & 55.6         & 0.58      & 60.0         & 0.47        & 47.8          & 0.94      & 60.9        & 0.91       & 70.4         & 0.69      & 59.0        \\
Agent (1-tool , UniMol-v2 )       & 0.75      & 78.6         & 0.54      & 50.0         & 0.49        & 50.0          & 0.94      & \underline{96.8}        & 0.93       & 89.6         & 0.72      & 72.4        \\
Agent (2-tool)                    & \underline{0.75}      & 74.2         & 0.59      & 58.7         & 0.49        & 50.0          & 0.92      & 95.8        & \underline{0.94}       & \underline{91.7}         & \underline{0.74}      & 74.1        \\
Ours (Stacking Agent)             & \textbf{0.79}      & 81.4         & \textbf{0.68}      & \underline{71.1}         & \textbf{0.67}        & 72.3          & \underline{0.95}      & \textbf{97.4}        & \textbf{0.96}       & \textbf{97.8}         & \textbf{0.81}      & \underline{84.0}       
\\
    \bottomrule
    \end{tabular}
    }
    \caption{Benchmark results of different models in molecular property prediction tasks. All LLM-based generalist models are evaluated on 0-shot.}
    \label{tab:prediction}
\end{table}



%从表4的结果来看,任务专用模型(Task-specific specialist models)整体表现最佳,其中UniMol-v2在五个任务上的平均AUC-ROC值与Acc值高于其余模型。这表明针对分子性质预测任务进行特定优化的模型能够更有效地学习化学分子结构与生物活性之间的关系。
%而通过ChemHTS找到的Stacking Agent结构在该任务的表现优于基于LLM的大规模通用模型(LLM-based generalist models)在该任务上的表现,尤其在非平衡的双分类任务HIV和Tox21上。这表明工具融合方法能够在一定程度上弥补通用模型的劣势,提高预测任务的准确性。我们的方法最终,虽然在AUC-ROC上低于SOTA的UniMol-v2 1.5分,但在平均Accuarcy(0.81)上超过了Uni-Molv2的0.74.

From the results in Tab.~\ref{tab:prediction}, task-specific specialist models demonstrate the best overall performance. Among the models, UniMol-v2 achieves the highest average AUC-ROC and Accuracy values across the five tasks, outperforming the other models. These findings indicate that models specifically optimized for molecular property prediction tasks are more effective at learning the relationships between chemical molecular structures and their biological activities.
Moreover, the Stacking Agent structure identified through ChemHTS outperformed LLM-based generalist models on this task, especially in the imbalanced binary classification tasks of HIV and Tox21. This suggests that tool integration methods can, to some extent, compensate for the limitations of generalist models and enhance the accuracy of prediction tasks.
Finally, while our ChemHTS method falls 1.5 AUC score behind the state-of-the-art UniMol-v2 , it surpasses its average accuracy (0.74) with a score of 0.81.
\subsubsection{Reaction Prediction}
%任务简介
Reaction prediction is a core task in the field of chemistry, with significant importance for drug discovery, materials science, and the development of novel synthetic pathways. Given a set of reactants, the goal of this task is to predict the most likely products formed during the chemical reaction~\cite{guo2024modeling,Schwaller_2019}.
%评估指标
Similarly to the results of the text-based molecule design task, we used the same metrics to measure the task performance.
\setlength\tabcolsep{1pt}
%
\begin{table}[!htb]
    \centering
    \tiny
    \definecolor{lightgray}{gray}{0.9} % 定义浅灰色
    \resizebox{0.5\textwidth}{!}{
    \begin{tabular}{lccccccc}
        \toprule
        \textbf{Model} & \textbf{Exact↓} & \textbf{BLEU↑} & \textbf{Dis↑} & \textbf{Validity↑} & \textbf{MACCS↑} & \textbf{RDK↑} & \textbf{Morgan↑} \\
        \midrule
        \rowcolor{lightgray} \multicolumn{8}{c}{\textit{Task-specific specialist models}} \\

        Chemformer~\cite{irwin_chemformer_2022} & \textbf{0.91} & 96.1 & \underline{1.26} & 1.00 & \underline{0.97} & \underline{0.97} & \underline{0.96} \\
Text+ChemT5~\cite{textchemt5}&0.83&96.0&7.42&0.98&0.96&0.96&0.94\\
        InstructMol~\cite{cao2024instructmolmultimodalintegrationbuilding} & 0.54 & 96.7 & 10.85 & 1.00 & 0.88 & 0.78 & 0.74 \\
        Mol-Instruction~\cite{fang2024molinstructionslargescalebiomolecularinstruction} & 0.05 & 65.4 & 27.26 & 1.00 & 0.51 & 0.31 & 0.26 \\
        
        \midrule
        \rowcolor{lightgray} \multicolumn{8}{c}{\textit{LLM-based generalist models}} \\

        GPT-4o~\cite{openai2024gpt4ocard}      & 0.01 & 65.8 & 27.24  & 0.81 & 0.54 & 0.39 & 0.33 \\
Deepseek-R1~\cite{deepseekai2025deepseekr1incentivizingreasoningcapability} & 0.10 & 76.2 & 16.04  & 0.75 & 0.60 & 0.53 & 0.48 \\
Llama3-70b~\cite{llama3modelcard}  & 0.00 & 55.2 & 282.46 & 0.85 & 0.48 & 0.35 & 0.31 \\
Llama3-8B~\cite{llama3modelcard}   & 0.00 & 37.6 & 148.15 & 0.41 & 0.18 & 0.14 & 0.11 \\
ChemDFM-13B~\cite{zhao2024chemdfmlargelanguagefoundation} & 0.39 & 80.6 & 10.38  & 0.96 & 0.77 & 0.69 & 0.65 \\
        \midrule
        \rowcolor{lightgray} \multicolumn{8}{c}{\textit{Tool-based Agent models}} \\
Agent (1-tool , SMILES2Property) & 0.05       & 43.6          & 33.17         & 0.83          & 0.40          & 0.29          & 0.27          \\
Agent (1-tool , Chemformer)      & 0.89       & 96.4          & 2.44          & 1.00          & 0.97          & 0.97          & 0.95          \\
Agent (2-tool)                   & 0.87       & \underline{97.1}    & 1.6           & 1.00          & 0.97          & 0.97          & 0.95          \\
Ours (Stacking Agent)            & \underline{0.90} & \textbf{98.4} & \textbf{0.97} & \textbf{1.00} & \textbf{0.98} & \textbf{0.98} & \textbf{0.96}
\\
        \bottomrule
    \end{tabular}
    }
    \caption{Benchmark results of different models in reaction prediction tasks. All LLM-based generalist models are evaluated on 0-shot.}
    \label{tab:reaction_prediction}
\end{table}


%实验结果
As shown in the Tab.~\ref{tab:reaction_prediction}, It can be observed that the Chemformer model performs exceptionally well in this task, achieving a product prediction accuracy of 0.91. It also outperforms other task-specific models across all metrics. In contrast, LLMs face significant challenges, with Deepseek-R1, despite its deep reasoning capabilities, only achieving 0.10 accuracy in product prediction. Similarly, the chemistry-specific ChemDFM struggles under 0-shot conditions. The ChemHTS model excels across all metrics, except for a slightly lower exact score (0.01), surpassing the Chemformer model on all other measures. For Agent (1-tool), relying solely on the SMILES2Property tool leads to poor performance. However, leveraging the additional information provided by the RAG tool through integration, the agent also achieves significantly better performance.
\section{Analysis}
\subsection{Does it improve performance if the agent can choose from more tools?}
%为了进一步分析agent的tool数量对化学任务性能的影响,本研究进一步分析了不同tool数量在Text-based Molecule Design任务上的影响。从表6可以看出,在一定的训练数据下,当工具数量从2增加到4时,BLEU-2的得分变化较小。平均BLEU-2得分(AVG)方面,Tool Number = 2 和 Tool Number = 4 的得分均为0.86,而 Tool Number = 3 的得分略低,为0.85。这表明工具数量的增加在一定程度上对BLEU-2得分的提升作用有限,可能是因为工具之间的贡献存在冗余,以及工具本身的质量决定了其对最终结果的影响程度。
To further analyze the impact of the number of tools on the performance of chemical tasks, this study investigates the effect of varying tool numbers on the text-based molecule design task. 
As shown in Tab.~\ref{tab:analysis1}, under a fixed amount of training data, increasing the number of tools from 2 to 4 results in only minor changes in BLEU-2 scores. Regarding the average BLEU-2 score, both Tool Number = 2 and Tool Number = 4 achieve a score of 0.86, while Tool Number = 3 slightly underperforms with a score of 0.85. This indicates that increasing the number of tools has limited benefits for BLEU-2 score improvement, which may be attributed to redundancy in contributions among tools and the fact that the quality of individual tools determines their impact on the final performance.



\setlength\tabcolsep{1pt}
%
\begin{table}[!htb]
    \centering
    \tiny
    \definecolor{lightgray}{gray}{0.9} % 定义浅灰色
    \resizebox{0.49\textwidth}{!}{
    \begin{tabular}{ccccccccccc}
        \toprule
        \multirow{2}{*}{\textbf{Tool Number}} & \multicolumn{8}{c}{\textbf{Train Data Number}} & \multirow{2}{*}{\textbf{AVG}} \\ 
        \cmidrule(lr){2-9}
        & \multicolumn{2}{c}{5} & \multicolumn{2}{c}{10} & \multicolumn{2}{c}{20} & \multicolumn{2}{c}{30} & \\ 
        \cmidrule(lr){2-3} \cmidrule(lr){4-5} \cmidrule(lr){6-7} \cmidrule(lr){8-9}
        & Layer & BLEU-2 & Layer & BLEU-2 & Layer & BLEU-2 & Layer & BLEU-2 & \\
        \midrule
        2 & 0.6 & 0.79 & 2.7 & 0.89 & 3.0 & 0.89 & 3.2 & 0.87 & 0.86 \\
        3 & 0.4 & 0.81 & 2.5 & 0.87 & 2.8 & 0.86 & 3.0 & 0.85 & 0.85 \\
        4 & 0.6 & 0.80 & 2.8 & 0.88 & 3.0 & 0.87 & 3.2 & 0.86 & 0.86 \\
        \bottomrule
    \end{tabular}
    }
    \caption{Comparison of the performance of different tools (Tool Layerber) on the text-based molecule design task under different training data sizes (Train Data Layerber). And Layer represents the number of layers of stacking agents.}
    \label{tab:analysis1}
\end{table}

\subsection{Does more training data lead to better performance?}
%同样的,本研究深入分析了在不同数量的训练数据用ChemHTS方法找到的最优stacking结构之间的在text-based molecule design任务的性能差异。
Our study also conducts an in-depth analysis of the performance differences in the text-based molecule design task among the optimal stacking structures identified by the ChemHTS method under varying amounts of training data.
%从表8可以得到,随着训练数据的增加(从5到30),BLEU-2得分总体上呈现上升趋势。例如,对于Tool Number = 2,在训练数据为5时,BLEU-2得分为0.79,而当训练数据增加到30时,BLEU-2得分提升至0.87。其他工具也表现出类似趋势,说明增加训练数据有助于提高模型的翻译质量。这一趋势符合预期,但是不是越多的训练数据,性能越好,因此需要对不同任务的训练数据数量要进一步做实验分析,选取最优的组合。
As shown in Tab.~\ref{tab:analysis1}, BLEU-2 scores generally exhibit an upward trend with the increase in training data (from 5 to 30). For example, for Tool Number = 2, the BLEU-2 score is 0.79 when the training data is 5, and it improves to 0.87 when the training data increases to 30. Similar trends are observed for other tool numbers, indicating that increasing the amount of training data enhances the model's translation quality. This trend aligns with expectations; however, more training data does not always guarantee better performance. Therefore, further experimental analysis is required to determine the optimal amount of training data for different tasks and select the best combination.
%另外我们也能看到




\setlength\tabcolsep{1pt}
%
\begin{table}[!htb]
    \centering
    \tiny
    \definecolor{lightgray}{gray}{0.9} % 定义浅灰色
    \resizebox{0.4\textwidth}{!}{
    \begin{tabular}{lccccccccc}
    \toprule
    \textbf{Layer} & 1 & 2 & 3 & 4 & 7 & 8 & 10 & 10+ & \textbf{AVG} \\
    \midrule
    \textbf{BLEU-2} & 0.88 & 0.90 & 0.92 & 0.93 & 0.90 & 0.92 & 0.91 & 0.90 & 0.91 \\
    \bottomrule
    \end{tabular}
    }
    \caption{Comparison of different tool stacking levels on the performance of text-based molecule design tasks.}
    \label{tab:analysis2}
\end{table}


\subsection{Does increasing the number of stacking layers improve performance?}
%本部分以text-based molecule design 任务为代表,主要探讨增加堆叠层数(stacking layers)是否能够提升性能。
%从表7的结果来看,增加堆叠层数对性能的提升存在一定的上限。当层数较少(如1或2层)时,模型可能无法充分捕捉复杂的特征,因此BLEU-2分数较低。随着层数增加至7,模型的学习能力增强,生成质量达到最佳。然而,进一步增加层数后,性能未能持续提升,可能的原因有过多的层数可能引入冗余信息,导致模型生成的文本质量下降或过度拟合训练数据。

This section uses the text-based molecule design task as a representative to explore whether increasing the number of stacking layers can improve performance.  
As shown in the results from Tab.~\ref{tab:analysis2}, there appears to be an upper limit to the performance improvement achieved by increasing the number of stacking layers. 
When the number of layers is small (e.g., 1 or 2 layers), the model may fail to sufficiently capture complex features, resulting in lower BLEU-2 scores. As the number of layers increases to 7, the model's learning capacity improves, and the generation quality reaches its peak. However, further increasing the number of layers does not lead to continued performance improvement. This could be attributed to the introduction of redundant information with excessive layers, causing a decline in text generation quality or overfitting to the training data.

\setlength\tabcolsep{1pt}
%
\begin{table*}[!htb]
    \centering
    \tiny
    \definecolor{lightgray}{gray}{0.9} % 定义浅灰色
    \resizebox{0.95\textwidth}{!}{
    \begin{tabular}[l]{|>{\raggedright\arraybackslash}m{1.5cm}|>{\raggedright\arraybackslash}m{4cm}|>{\raggedright\arraybackslash}m{10cm}|}
    \hline
    \textbf{Stacking Pattern} & \textbf{Input} & \textbf{Output} \\ 
    \hline
    Correct 
    & The molecule is a cycloalkane composed of a three-carbon ring. It functions as an inhalation anesthetic. It is a cycloalkane and a member of the cyclopropane class. What is its predicted SMILES?
    \newline
    \newline
    \newline
    \textcolor{red}{Gold Answer: C1CC1}
    & According to the provided description, the molecule is cyclopropane, which is a cycloalkane formed by three carbon atoms in a ring, well known for its use as an inhalation anesthetic. To confirm the SMILES notation of cyclopropane, I used the ChemDFM tool.
    \newline
    \textbf{Action:} ChemDFM  
    \newline
    \textbf{Input:} What is the SMILES notation for cyclopropane (a three-carbon cycloalkane used as an inhalation anesthetic)?  
    \newline
    \textbf{Output:} CC1(C)CC1  \textcolor{red}{\ding{55}}
    \newline
    \textbf{Observation:} The returned "CC1(C)CC1" does not correspond to cyclopropane. Cyclopropane should be a simple three-carbon ring, typically represented as "C1CC1". Next, I will call the Name2SMILES tool to further correct the result.
    \newline
    \textbf{Action:} Name2SMILES  
    \newline
    \textbf{Input:} Cyclopropane  
    \newline
    \textbf{Output:} C1CC1  
    \newline
    \textbf{Observation:} Based on the known structure of cyclopropane, the result from this tool is correct. Therefore, I will provide the correct SMILES based on the understanding of the chemical structure.  
    \newline
    \textbf{Final Answer: C1CC1 \textcolor{red}{\ding{52}}}\\
    \hline
    \end{tabular}
    }
    \caption{Case analysis of \textbf{"Correct"} behavior patterns of agents during tools stacking.}
    \label{tab:case study}
\end{table*}


%该分子由三个氮原子组成的环状结构,它具有芳香性的特性和两个亚甲基。它是一种环烯烃,也是两种异构体之一。请预测它的 SMILES。
%
\subsection{Why tool stacking works?}
\label{6.4}
%本章节主要通过案例分析来探讨agent在tool stacking过程中出现的一些行为模式,来解释tool stacking能提升模型任务表现的原因。通过我们的案例总结,我们发现了四种stacking行为pattern,所有案例细节见附录C。由于篇幅限制,我们仅举1个correct的stacking patter例子,如表8所示。
%我们在text-based molecule design task中,给模型两个tool以供调用。首先,实验调用了ChemDFM工具进行SMILES预测。该工具返回了C1(C)CC1,但该结果不符合环丙烷的标准SMILES表示。这一错误可能来源于ChemDFM在解析化学结构描述时对环状结构的理解存在偏差,未能正确识别单一的三元环结构,而是错误地引入了额外的支链。为修正错误,实验进一步调用Name2SMILES工具,并输入“Cyclopropane”作为查询。该工具正确返回了C1CC1,并在后续分析中被确认符合环丙烷的已知结构。因此,最终输出的SMILES被修正为C1CC1,符合标准答案。本实验表明,单一工具在处理化学结构预测任务时可能存在局限性,例如ChemDFM的错误预测。然而,通过工具堆叠策略,实验成功利用多个工具互补优势,提高了最终预测的准确性。
This chapter explores the behavioral patterns exhibited by agents during the process of tool stacking through case analysis and examines how tool stacking enhances task performance. From the case studies, we identify \textbf{four distinct stacking behavior patterns—correct, modify, judge and reserve}, with detailed information on all cases provided in Appendix.~\ref{casestudy}. Due to space limitations, this paper illustrates only one \textbf{correct} stacking behavior pattern as shown in Tab.~\ref{tab:case study}.
In the text-based molecular design task, two callable tools are provided to the model. The experiment first invokes the ChemDFM tool for SMILES prediction, which returns "C1(C)CC1". However, this result does not conform to the standard SMILES representation of cyclopropane. The error likely arises from ChemDFM's misinterpretation of cyclic structures, as it fails to correctly identify the single three-membered ring and instead erroneously introduces an additional branch. To address this issue, the experiment subsequently invokes the Name2SMILES tool with "Cyclopropane" as the query. This tool correctly returns "C1CC1", which is confirmed through subsequent analysis to align with the known standard structure of cyclopropane. Consequently, the final SMILES output is corrected to "C1CC1", meeting the requirements of the standard answer.
This experiment shows that individual tools may have certain limitations when handling chemical structure prediction tasks, as evidenced by ChemDFM's erroneous prediction. However, by employing the tool stacking strategy, the experiment effectively leverages the complementary strengths of multiple tools, significantly improving the accuracy of the final prediction.



\subsection{Comparison with LLM-based Multi-Agent Systems}
%由于多智能体系统(Multi-Agent System, MAS)与工具增强型大语言模型(Tool-augmented LLMs)在任务分解、工具调用和信息共享等方面存在相似性。因此,本研究重点分析了 6 种不同通信结构的多智能体系统与我们提出的 ChemHTS 方法在 text-based 分子设计任务中寻求的最优堆叠智能体路径的性能比较。关于具体的多智能体系统细节,请参见附录 A。
%从图 2 可以看出,随着智能体规模的增加,各种通信结构的质量值均有所提升,但增长趋势存在显著差异。当智能体规模增大(Scale ≥ 4)时,不同结构的性能开始分化。其中,Fully Connected(全连接)和 Layered(分层)结构的质量值显著高于其他结构。然而,最优堆叠智能体路径的性能超越了多智能体系统的性能上限。这是因为最优路径能够更高效地利用任务分解和工具调用的能力,同时避免了多智能体系统中可能出现的通信开销和协调瓶颈。
LLM-based Multi-Agent Systems (MAS) and Tool-augmented LLM share similarities in areas such as task decomposition, tool invocation, and information sharing. Therefore, our study focuses on comparing the performance of six multi-agent systems with different communication structures against the optimal stacking agent path proposed by our ChemHTS method in the text-based molecule design task. For details on the specific multi-agent systems, please refer to Appendix.~\ref{Framework},~\ref{multi-agent results}.
\begin{figure}[t] 
    \centering
        % \includegraphics[width=0.8\textwidth]{figures/files/method_v1.pdf}
            \includegraphics[width=0.45\textwidth]{figures/files/all-tool2.jpg}
    % \captionsetup{font={small}} 
    \caption{Performance comparison of 6 multi-agent systems with different communication structures and our optimal stacking agent path on the text-based molecule design task.
    }
    \label{fig:multiagent}
\end{figure}

As shown in Fig.~\ref{fig:multiagent}, the BLEU-2 scores of various communication structures improve as the scale of agents increases. However, the growth trends differ significantly. When the agent scale becomes larger, the performance of different structures begins to diverge. Among them, the Full-Connected and Layered structures demonstrate significantly higher quality values compared to other structures. 
Nevertheless, the performance of the optimal stacking agent path surpasses the upper performance limit of the multi-agent systems. This is because the optimal path can more effectively leverage task decomposition and tool invocation capabilities while avoiding potential communication overhead and coordination bottlenecks inherent in multi-agent systems. 
%更多的multi-agent的具体实验结果见附录B
More specific experimental results of multi-agent can be found in Appendix.~\ref{appendix mulagent results}.





\section{Results}


\begin{figure*}[ht]
\begin{center}
\centerline{\includegraphics[width=\textwidth]{FIG/figure3_benchmark.pdf}}
\caption{Benchmarking the performance of \name, baselines, and ablation models on \textbf{(a)} immediate and \textbf{(b)} delayed sequence editing. The models are trained using a standard cell- or patient-centric random split. Not shown for visualization purposes are the performances of VAR models on eICU and MIMIC-IV datasets: on immediate sequence editing, MAE for eICU and MIMIC-IV are $55982.74$ and $886.05$, respectively; on delayed sequence editing, MAE for eICU and MIMIC-IV are $3.02 \times 10^{39}$ and $8.62 \times 10^{23}$, respectively.}
\label{fig:benchmark}
\end{center}
\vskip -0.35in
\end{figure*}




We evaluate \name's performance on controllable sequence editing across multiple datasets and tasks. We aim to answer the following research questions. 
\textbf{R1:} How well does \name perform in immediate sequence editing?
\textbf{R2:} How well does \name perform in delayed sequence editing?
\textbf{R3:} How does \name generalize to unseen/new sequences?
\textbf{R4:} Can \name perform zero-shot counterfactual generation?
\textbf{R5:} How can \name be leveraged for real-world counterfactual patient trajectory simulations?
We establish that \name outperforms state-of-the-art baselines in sequence editing, demonstrating both immediate and delayed sequence editing capabilities, strong generalizability, and real-world applicability.


\subsection{\underline{R1}: Immediate sequence editing}\label{results:r1}

Immediate sequence editing involves forecasting the next time step of a sequence under a counterfactual condition. This is useful in settings where interventions take effect instantaneously, such as introducing a genetic perturbation in cellular systems or administering a drug to a patient~(Definition~\ref{def:cse}). Example counterfactual scenarios in which immediate sequence editing is applicable are: \textit{What if we treat the cells with the candidate drug now?} and \textit{What if we perform surgery on the patient today?}




\name models consistently outperform baseline models across all datasets~(Figure~\ref{fig:benchmark}a; Appendix Figures~\ref{fig:supp_benchmark_full_mae}-\ref{fig:supp_benchmark_full_rmse}). The SimpleLinear baseline, which assumes minimal temporal changes, performs comparably in some cases, but \name outperforms it on datasets where short-term dynamics are more complex. On WOT, all \name models outperform or perform comparably to the time series forecasting model, VAR. This is particularly exciting given recent findings that linear models can achieve competitive or better forecasting performance than neural network models~\cite{toner2024analysis, ahlmann2024deep}. These results highlight \name's ability to accurately modify trajectories at the right points while preserving unaffected portions of the sequence, an advantage in counterfactual reasoning.

Regardless of the sequence encoder used with \name, these models tend to outperform or perform comparably to non-\name models~(Figure~\ref{fig:benchmark}a). However, the performance of \name can be affected by the ability of the sequence encoder to capture the temporal dynamics of the input sequences. For instance, models with the MOMENT encoder yield the highest MAE in all three datasets, with and without help from \name~(Figure~\ref{fig:benchmark}a). Nevertheless, \name models with the MOMENT encoder reduce the MAE of non-\name models.


\subsection{\underline{R2}: Delayed sequence editing}\label{results:r2}


Delayed sequence editing requires forecasting a counterfactual trajectory at a future time step while maintaining causal consistency. This task is challenging, as small errors can compound over longer horizons. Example scenarios in which delayed sequence editing is applicable are: \textit{What if we treat the cells with the candidate drug in ten days?} and \textit{What if we perform the surgery on the patient next year?}

\name outperforms or performs competitively against SimpleLinear and VAR on the patient datasets, eICU and MIMIC-IV~(Figure~\ref{fig:benchmark}b; Appendix Figures~\ref{fig:supp_benchmark_full_mae}-\ref{fig:supp_benchmark_full_rmse}). \name-transformer and \name-xLSTM achieve lower MAE than SimpleLinear, whereas non-\name transformer and MOMENT baselines perform comparably or worse. As in immediate sequence editing, models using MOMENT as the sequence encoder (i.e.,~using temporal concepts with the MOMENT sequence encoder) yield the highest MAE. However, incorporating \name with MOMENT reduces the MAE to levels comparable to SimpleLinear and VAR.


On WOT, SimpleLinear and VAR outperform neural network models in delayed sequence editing~(Figure~\ref{fig:benchmark}b). This suggests that cellular developmental trajectories exhibit small and possibly noisy changes at each time step, favoring linear models~\cite{ahlmann2024deep, toner2024analysis}. Additionally, given the relatively small number of training trajectories compared to the high-dimensional state space, nonlinear models may overfit to noise more readily than linear models. Nevertheless, \name significantly reduces the MAE of non-\name models, demonstrating its effectiveness as a regularizer that mitigates short-term noise while preserving long-term trends.


\subsection{\underline{R3}: Generalization to new patient trajectories}


We assess the ability of \name models to generalize to new patient sequences. To evaluate robustness, we use the SPECTRA approach~\cite{ektefaie2024evaluating} to create challenging data splits where the test sets have minimal similarity to the training data (Appendix~\ref{appendix:labs}).

Across both the eICU and MIMIC-IV patient datasets, \name models exhibit stronger generalization than non-\name models (Figure~\ref{fig:spectra}; Appendix Figures~\ref{fig:supp_spectra_full}-\ref{fig:supp_spectra_full_rmse} and Table~\ref{tab:auspc}). For immediate and delayed sequence editing on eICU, \name-transformer and \name-xLSTM maintain stable and strong performance even as train/test divergence increases. In contrast, their non-\name counterparts degrade significantly. Although baseline MOMENT models show relatively stable performance across train/test splits in delayed sequence editing, they generalize poorly compared to \name-MOMENT models. Despite similar performance between xLSTM and \name-xLSTM in delayed sequence editing on both patient datasets (Figure~\ref{fig:benchmark}b), \name-xLSTM demonstrates superior generalizability (Figure~\ref{fig:spectra}b), highlighting the effectiveness of \name in adapting to unseen data distributions.




\begin{figure*}[ht]
\begin{center}
\centerline{\includegraphics[width=\textwidth]{FIG/figure4_spectra.pdf}}
\caption{Generalizability of \name on \textbf{(a)}~eICU and \textbf{(b)}~MIMIC-IV patient datasets in immediate and delayed sequence editing. As the SPECTRA parameter increases, the train/test split similarity decreases~(Appendix Figure~\ref{fig:supp_spectra_cso}). The area under the spectral performance curve (AUSPC) evaluation is in Appendix Table~\ref{tab:auspc}.}
\label{fig:spectra}
\end{center}
\vskip -0.35in
\end{figure*}



\subsection{\underline{R4}: Zero-shot counterfactual generation of cellular trajectories}

In addition to evaluating \name's generalizability to new patient lab test trajectories, we assess on zero-shot counterfactual generation for cellular trajectories~(Figure~\ref{fig:wot_cf}; Appendix Figure~\ref{fig:supp_wot_cf_full}). Using the Waddington-OT model, we generate sequences that remain consistent until a specified divergence time step, where an alternative condition—such as the activation of a different transcription factor—introduces a shift~(Appendix~\ref{appendix:cells}). This process yields \(1,273\) pairs of ``original" and ``counterfactual" trajectories, totaling \(2,546\) individual sequences~(Table~\ref{tab:data}). Models are trained on the ``original" trajectories and evaluated on the ``counterfactual" trajectories in a zero-shot setting.

\name-based models consistently outperform non-\name models in both immediate and delayed sequence editing~(Appendix Figure~\ref{fig:supp_wot_cf_full}). To more closely analyze delayed sequence editing performance, we examine the predictions for cellular trajectories of length 23, the most common sequence length in the dataset~(Figure~\ref{fig:wot_cf}). Since \(t_i = 10\) is the earliest divergence time step, we provide the first nine time steps \(\mathbf{x}_{:,0:9}\), the counterfactual condition, and \(t_j \in [10, 23]\) to the model. Comparing the generated and ground truth counterfactual sequences, we find that \name significantly outperforms non-\name models after time step 10, which is when the trajectories begin to diverge (Figure~\ref{fig:wot_cf}). 







\begin{figure}[ht]
\begin{center}
\centerline{\includegraphics[width=\columnwidth]{FIG/figure5_wot_cf.pdf}}
\caption{Zero-shot counterfactual generation of cellular developmental trajectories. Shown are the MAE of predictions at each time step for counterfactual sequences of length 23 (the most common sequence length in the dataset) starting at time step 10 (the earliest divergence time step of a counterfactual trajectory).}
\label{fig:wot_cf}
\end{center}
\vskip -0.35in
\end{figure}


\subsection{\underline{R5}: Case studies using real-world patient datasets}

We evaluate \name's ability to simulate counterfactual patient trajectories through temporal concept intervention. We conduct case studies on two independent cohorts of patients with type 1 diabetes mellitus (T1D), a chronic autoimmune disease in which the immune system attacks insulin-producing cells in the pancreas~\cite{quattrin2023type}.

Unlike counterfactual generation methods that rely on condition tokens to guide generation~\cite{narasimhan2024time, jing2024towards, zhang2023survey}, \name allows \textit{direct edits to the generated outputs} to produce counterfactual sequences. This capability is particularly valuable when condition tokens are insufficient, such as when prescribing medication dosage. Instead of relying on predefined conditions, \name can precisely modify the values of specific lab tests to explore their longitudinal effects.

\xhdr{Setup}
%
For an individual patient, we intervene on the temporal concepts corresponding to specific lab tests to simulate the ``reversal" or ``worsening" of symptoms, thereby generating ``healthier" or ``more severe" trajectories, respectively. Formally, given temporal concept \(\mathbf{c}\) learned from \(\mathbf{x}_{:,t_0:t_i}\) and an optional condition \(s\), we modify \(\mathbf{c}^I \neq \mathbf{c}\) such that at least one element satisfies \(\mathbf{c}_k \neq \mathbf{c}^I_k\).  

From the eICU and MIMIC-IV datasets, we construct two independent cohorts of T1D patients and matched healthy individuals~(Appendix~\ref{appendix:labs}). The eICU-T1D dataset contains \(59\) T1D patients and \(579\) matched healthy controls, while MIMIC-IV-T1D includes \(25\) T1D patients and \(226\) matched healthy controls.  

To generate counterfactual sequences, we modify specific values in temporal concept \(\mathbf{c}\), such as glucose levels, and allow \name to simulate future trajectories of length \(T = 10\). We then compare these counterfactual trajectories (i.e.,~\name-generated patients) against observed sequences from matched healthy individuals, other healthy individuals, and other T1D patients. Our hypothesis is that clinically meaningful edits will produce ``healthier" (i.e.,~more similar to healthy patients) or ``sicker" (i.e.,~more similar to other T1D patients) trajectories.  




\xhdr{Results}
%
First, we modify \name's concepts to reduce glucose levels by half, aligning them closer to normal physiological ranges. The resulting counterfactual patient trajectories exhibit higher \(R^2\) similarity with both matched and other healthy individuals compared to other T1D patients (Figure~\ref{fig:t1d}a). This suggests that \name effectively generates counterfactual trajectories indicative of a healthier state.  

Next, we simulate a worsening condition by doubling glucose levels. The resulting counterfactual trajectories generated by \name show higher \(R^2\) similarity with other T1D patients than with healthy individuals~(Figure~\ref{fig:t1d}a), as would be expected based on clinical evidence.  

Beyond direct interventions, we examine indirect changes in \name-generated patients' lab values resulting from glucose modifications. In both eICU-T1D and MIMIC-IV-T1D cohorts, lowering glucose also leads to a reduction in white blood cell (WBC) count~(Figure~\ref{fig:t1d}b; Appendix Figure~\ref{fig:supp_t1d}a). This aligns with clinical knowledge, as T1D is an autoimmune disorder where immune activity, including WBC levels, plays a critical role~\cite{quattrin2023type}. Further, when we intervene on \name to reduce WBC levels instead of glucose, we observe a concurrent drop in glucose across both cohorts~(Appendix Figure~\ref{fig:supp_t1d}b,c), reinforcing the interconnected nature of these physiological markers.  




\begin{figure}[ht]
\begin{center}
\centerline{\includegraphics[width=\columnwidth]{FIG/figure6_pt_int.pdf}}
\caption{\name-generated counterfactual patients via intervention on temporal concepts. We intervene on \name to \textbf{(a)}~halve~(top) or double~(bottom) a T1D patient's glucose levels to infer a ``healthier" or ``sicker" counterfactual patient, respectively. \textbf{(b)}~Observed and \name patients from the eICU-T1D cohort are compared to quantify the differences between their lab test trajectories (indirect effects) as a result of the intervention to halve T1D patients' glucose levels.}
\label{fig:t1d}
\end{center}
\vskip -0.35in
\end{figure}

Finally, we demonstrate that modifying multiple lab tests simultaneously can produce compounding effects. When we intervene on \name to reduce both glucose and WBC levels, the resulting \name-generated patients resemble healthy individuals even more closely than other T1D patients~(Appendix Figure~\ref{fig:supp_t1d}d). This finding suggests that \name can integrate multiple simultaneous edits, capturing their joint impact on a patient's future state.




\section{Conclusion}
\section{Conclusion}

This paper is one of the first attempts to understand, categorize, and detect \termname terms and conditions on shopping websites. These terms, which can significantly impact consumer trust and satisfaction, have not been extensively studied. By highlighting the prevalence and types of \termname terms, we hope to pave the way for increased awareness and further research in this area. We develop an automated data collection and topic modeling pipeline, analyzing \termcnt terms from \websitecnt websites to create a taxonomy for \termname terms. This taxonomy includes \termtypecnt types across \termcatcnt categories, covering purchase and billing, post-purchase activities, account termination, and legal aspects.


\platform is the first study to evaluate the effectiveness of LLMs in identifying \termname terms. Using a fine-tuned GPT-4o model on a manually annotated dataset, \platform achieves an F1 score of 94.6\% with a false positive rate of 2.3\%. In large-scale deployment, we find that approximately \websitepct of shopping websites in the Tranco top 100,000 contain at least one category of \termname terms. Our qualitative analysis shows that current ecosystem defenses are inadequate to protect users from these terms, that less popular websites are more likely to include \termname terms, and that there is a correlation between \termname terms and user dissatisfaction.





% Ack: thank openai and renuka
\section*{Acknowledgments}

The authors would like to thank Renuka Kumar for her valuable input and suggestions during the early stages of this project.  This work is supported by a gift from the OpenAI Cybersecurity Grant program. Any opinions, findings, conclusions, or recommendations expressed in this material are those of the authors
and do not necessarily reflect the views of OpenAI.



\clearpage
\section*{Broader Impact}

By introducing a flexible and interpretable approach to counterfactual sequence generation, \name bridges the gap between language model-style conditional generation and structured, time-sensitive sequence editing, with implications for decision support in medical and scientific applications.
%
Like all generative AI models, \name (and its derivatives) should be used solely for the benefit of society. In this study, we demonstrate that \name can generate alternative cellular trajectories and simulate the reversal or progression of symptoms to model healthier or sicker patient outcomes. However, this work (and any derivatives) should never be used to induce harmful cellular states (e.g., activating transcription factors to drive a cell toward a pathological state) or negatively impact patient care (e.g., neglecting necessary clinical interventions or recommending harmful treatments).  Our goal is to help researchers understand the underlying mechanisms of disease to improve public health. Any misuse of this work poses risks to patient well-being. Therefore, the ability to intervene on \name's generated outputs should be leveraged to assess the model's robustness and correctness for ethical and responsible use.  

\section*{Acknowledgements}

M.L.~and M.Z.~are supported by the Berkowitz Family Living Laboratory at Harvard Medical School and the Clalit Research Institute. Y.E.~is supported by grant T32 HG002295 from the National Human Genome Research Institute and the NDSEG Fellowship. We gratefully acknowledge the support of NIH R01-HD108794, NSF CAREER 2339524, US DoD FA8702-15-D-0001, ARPA-H BDF program, awards from Chan Zuckerberg Initiative, Bill \& Melinda Gates Foundation INV-079038, Amazon Faculty Research, Google Research Scholar Program, AstraZeneca Research, Roche Alliance with Distinguished Scientists, Sanofi iDEA-iTECH, Pfizer Research, John and Virginia Kaneb Fellowship at Harvard Medical School, Biswas Computational Biology Initiative in partnership with the Milken Institute, Harvard Medical School Dean's Innovation Fund for the Use of Artificial Intelligence, Harvard Data Science Initiative, and Kempner Institute for the Study of Natural and Artificial Intelligence at Harvard University. Any opinions, findings, conclusions or recommendations expressed in this material are those of the authors and do not necessarily reflect the views of the funders.


\bibliography{ref}
\bibliographystyle{conf2025}


%%%%%%%%%%%%%%%%%%%%%%%%%%%%%%%%%%%%%%%%%%%%%%%%%%%%%%%%%%%%%%%%%%%%%%%%%%%%%%%
%%%%%%%%%%%%%%%%%%%%%%%%%%%%%%%%%%%%%%%%%%%%%%%%%%%%%%%%%%%%%%%%%%%%%%%%%%%%%%%
% APPENDIX
%%%%%%%%%%%%%%%%%%%%%%%%%%%%%%%%%%%%%%%%%%%%%%%%%%%%%%%%%%%%%%%%%%%%%%%%%%%%%%%
%%%%%%%%%%%%%%%%%%%%%%%%%%%%%%%%%%%%%%%%%%%%%%%%%%%%%%%%%%%%%%%%%%%%%%%%%%%%%%%
\newpage
\appendix
\onecolumn

\section{Data \& Experimental Setup}

In this section, we provide further details about data construction, data preparation, and experimental setup. We share code and instructions in our GitHub repository to reproduce the experiments in this paper: \url{https://github.com/mims-harvard/CLEF}.

\subsection{Cellular Developmental Trajectories}\label{appendix:cells}

Here, we describe the process of (1) simulating single-cell transcriptomic profiles of developmental time courses for individual cells and (2) preparing these trajectories for modeling.

\subsubsection{Simulating trajectories}

Cellular reprogramming experiments help elucidate cellular development~\cite{schiebinger2019-ie}. In these experiments, cells are manipulated and allowed to progress for a specific period of time before they undergo RNA sequencing~(RNA-seq), and we analyze the resulting RNA-seq data to observe their new cellular profiles~\cite{schiebinger2019-ie}. RNA-seq is a destructive process for the cell, meaning that the same cell cannot be sequenced at two different time points. Computational models are thus necessary to infer the trajectory of a cell.


\xhdr{Waddington-OT dataset and model}
%
Waddington-OT~\cite{schiebinger2019-ie} is a popular approach to reconstruct the landscape of cellular reprogramming using optimal transport~(OT). There are two components in Waddington-OT: (1)~\textit{a single-cell RNA-seq (scRNA-seq) dataset} of mouse cells from a reprogramming experiment, and (2)~\textit{an OT-based trajectory inference model} fitted on the scRNA-seq dataset. The scRNA-seq dataset consists of 251,203 mouse cells profiled from 37 time points (0.5-day intervals) during an 18-day reprogramming experiment starting from mouse embryonic fibroblasts. The trajectory inference model consists of transport matrices $\pi_{t_k,t_{k+1}}$ with dimensions $N \times M$ that relate all cells $\mathbf{x}^1_{t_k}, ..., \mathbf{x}^n_{t_k}$ profiled at time $t_k$ to all cells $\mathbf{x}^1_{t_{k+1}}, ..., \mathbf{x}^m_{t_{k+1}}$ profiled at time $t_{k+1}$. An entry at row $i$ and column $j$ of $\pi_{t_k,t_{k+1}}$ corresponds to the probability that $\mathbf{x}^j_{t_{k+1}}$ is a descendant cell of $\mathbf{x}^i_{t_k}$, as determined using optimal transport \cite{chizat2017scalingalgorithm}. Every cell in the scRNA-seq dataset is either pre-labeled as one of the 13 provided cell sets (i.e.,~induced pluripotent stem, stromal, epithelial, mesenchymal-epithelial transition, trophoblast, spongiotrophoblast, trophoblast progenitor, oligodenrocyte progenitor, neuron, radial glial, spiral artery trophoblast giant, astrocyte, other neural) or unlabeled. We cluster the unlabeled cells using Leiden clustering via \texttt{scanpy}~\cite{wolf2018-tb} at a resolution of 1, and define the resulting 27 unlabeled clusters as unique cell sets. As a result, each cell in the dataset belongs one and only one cell set. 


\xhdr{Simulating cell state trajectories} 
%
We define ``cell state" as the transcriptomic profile of a cell. Here, a transcriptomic profile is the log-normalized RNA-seq counts of the top $1,479$ most highly variable genes. To create a simulated trajectory of cell states for an individual cell undergoing reprogramming, we randomly and uniformly sample a cell profiled at time step $t_0$ (Day $0.0$) from the Waddington-OT scRNA-seq dataset, and generate via the transport matrix $\pi_{t_0,t_1}$ a probability distribution $\mathbb{P}_{t_1}$ over possible descendant cells $\mathbf{x}^1_{t_1}, \dots, \mathbf{x}^m_{t_1}$ at time step $t_1$ (Day $0.5$). We sample a cell from this distribution, and repeat the process until we reach either Day $18.0$ or a terminal state (i.e.,~neural, stromal, or induced pluripotent stem cell). After generating a trajectory composed of cells from the Waddington-OT scRNA-seq dataset through this process, we retrieve the transcriptomic profile of each cell to compose $\mathbf{x}_{:,t_0:t_T}$, where $T$ is the length of the trajectory.


\xhdr{Inferring conditions}
%
A condition $s_{t_i}$ is defined as the activation of a transcription factor~(TF) that leads a cell to transition from state $\mathbf{x}_{t_i}$ to descendant state $\mathbf{x}_{t_{i+1}}$. To infer such conditions, we perform differential expression analysis between cells from the same cell set as $\mathbf{x}_{t_i}$ (i.e.,~$\mathbf{x}^a \in A$) and cells from the same cell set as $\mathbf{x}_{t_{i+1}}$ (i.e.,~$\mathbf{x}^b \in B$). Using the $\texttt{wot.tmap.diff\_exp}$ function (via the \texttt{Waddington-OT} library), we identify the top TF that was significantly upregulated in $\mathbf{x}^a \in A$ compared to $\mathbf{x}^b \in B$. If no TFs are differentially expressed, then the condition is ``None." We retroactively perform this analysis on all pairs of consecutive cell states in a cell state trajectory $\mathbf{x}_{:,t_0:t_T}$ to obtain the full trajectory containing both cell states and TF conditions: $\tau = \{\mathbf{x}_{t_0}, s_{t_0}, \mathbf{x}_{t_1}, s_{t_1}, \cdots, s_{t_{T-1}}, \mathbf{x}_{t_T} \}$. In other words, $\tau$ represents a simulated trajectory of an individual cell undergoing the reprogramming process. Condition embeddings $\mathbf{z}_s \in \mathbb{R}^{5120}$ are obtained from the (frozen) pretrained ESM-2 embedding model~\cite{lin2022language}.

\xhdr{Generating matched counterfactual trajectories}
%
We additionally create pairs of matched counterfactual trajectories to evaluate a model's performance in zero-shot counterfactual generation. Each pair consists of an ``original" trajectory~$\tau_{og}$ and a ``counterfactual" trajectory~$\tau_{cf}$. First, we generate $\tau_{og}$ using the Waddington-OT model. Then, given a divergence time step~$D$, the first $D$ time steps of $\tau_{og}$ are carried over to $\tau_{cf}$ such that the first $D$ cell states and conditions of $\tau_{og}$ and $\tau_{cf}$ are exactly the same. The remaining states and conditions of $\tau_{cf}$ are sampled independently from $\tau_{og}$, resulting in an alternative future trajectory based on an alternative condition at time step~$D$. 

\textbf{Implementation note:} Because \name learns time embeddings based on the year, month, date, and hour of a given timestamp, we convert the time steps of each cell into timestamps. We set the starting time $t_0$ as timestamp \texttt{2000/01/01 00:00:00}, and add $10\times t_i$ hours to the converted timestamp of $t_{i-1}$.





\subsubsection{Experimental setup}

\xhdr{Generating data splits}
%
There are three cell sets (i.e.,~groups of cells with the same cell state label) that consist of cells from Day 0.0 in our post-clustering version of the Waddington-OT dataset. We refer to these cell sets as ``start clusters" because all initial cell states are sampled from one of these cell sets. Since the choice of start cluster can influence the likelihood of a cell's trajectory reaching certain terminal fates, we split our cellular trajectories into train, validation, and test sets based on their start cluster. This cell-centric data split allows us to evaluate how well a model can generalize to different distributions of trajectories. Start cluster \#1 is in the train set, start cluster \#3 is in the validation set, and start cluster \#2 is in the test set.


\xhdr{Zero-shot counterfactual generation}
%
The data split for zero-shot counterfactual generation is constructed such that the original trajectories $\tau_{og}$ are in the train or validation sets, and the counterfactual trajectories $\tau_{cf}$ are in the test set.




\subsection{Patient Lab Tests}\label{appendix:labs}

Here, we describe the process of (1) preprocessing electronic health records to extract longitudinal routine lab tests data and (2) preparing these trajectories for modeling.



\subsubsection{Constructing routine lab test trajectories}

We leverage two publicly available medical datasets: eICU~\cite{pollard2018eicu} and MIMIC-IV~\cite{johnson2024mimic, johnson2023mimic, goldberger2000physiobank}. Both datasets are under the PhysioNet Credentialed Health Data License 1.5.0 \cite{mimiciv_license}. The retrieval process includes registering as a credentialed user on PhysioNet, completing the CITI ``Data or Specimens Only Research" training, and signing the necessary data use agreements.

\xhdr{Processing patient datasets}
%
We process each dataset (i.e.,~eICU, MIMIC-IV) separately with the following steps. First, we extract the routine lab tests only (annotation available only in MIMIC-IV) and the most commonly ordered lab tests~(i.e.,~lab tests that appear in at least 80\% of patients). Next, we keep patients for whom we have at least one of each lab test. If there are multiple measurements of a lab test at the same time step~(i.e.,~year, month, date, hour, minute, and seconds), we take the mean of its values. We extract patients with more than one visit (or time step).

We define patients' conditions as medical codes, specifically International Classification of Diseases~(ICD)\nocite{icd}, of their diagnosis. Both eICU and MIMIC-IV use ICD-9 and ICD-10 codes. We extract the medical codes and their timestamps (multiple medical codes at a single time step is possible). Since the timestamps of diagnostic codes and lab tests are not necessarily the same (and there are fewer entries of diagnostic codes than lab orders), we merge them with a tolerance range of 12 hours (eICU) or two days (MIMIC-IV). We obtain (frozen) condition embeddings $\mathbf{z}_s \in \mathbb{R}^{128}$ (retrieved on~December~22,~2024) from an embedding model that has been pretrained on a clinical knowledge graph~\cite{johnson2024unified}. The clinical knowledge graph is constructed by integrating six existing databases of clinical vocabularies used in electronic health records: International Classification of Diseases~(ICD), Anatomical Therapeutic Chemical~(ATC) Classification, Systemized Nomenclature of Medicine - Clinical Terms~(SNOMED CT), Current Procedural Terminology~(CPT), Logical Observation Identifiers Names and Codes~(LOINC), and phecodes~\cite{johnson2024unified}.

\subsubsection{Generating data splits}

We generate a standard patient-centric random split for benchmarking model performance, and a series of increasingly challenging data splits via SPECTRA~\cite{ektefaie2024evaluating} to evaluate model generalizability.

\xhdr{Constructing data splits to evaluate model generalizability}
%
SPECTRA~\cite{ektefaie2024evaluating} creates a series of splits with decreasing cross-split overlap or similarity between the train and test sets. By training and testing models on these splits, we can assess model performance as a function of cross-split overlap. SPECTRA refers to this relationship as the spectral performance curve, which provides insight into how well a model generalizes to less similar data. When a new dataset split is encountered, it can be plotted as a point on this curve. The area under the spectral performance curve (AUSPC) serves as a metric of model generalizability and enables comparisons across models.

To generate a split with SPECTRA, a similarity definition and a SPECTRA parameter (SP) value between 0 and 1 are required. SP controls the level of cross-split overlap: values closer to 0 create splits resembling classical random splits, while values closer to 1 produce stricter splits with minimal or no overlap between train and test sets. For example, at an input of 1, no similar samples are shared between the train and test sets.

For eICU and MIMIC-IV, we define two patients as similar if: (1)~they are of the same gender, (2)~they are born in the same decade, and (3)~they share at least one ICD-9 or ICD-10 category. We exclude ICD-9 and ICD-10 codes that are present in more than 50\% of patients to avoid overly generic features. SPECTRA systematically prunes similar patients to produce splits. For this study, we generate 20 splits with SP values that are evenly spaced between 0 and 1. Given a train and test set, cross-split overlap is defined as the proportion of samples in the train set that are similar to at least one sample in the test set.


\subsubsection{Constructing cohorts of patients with type 1 diabetes mellitus}

To define a type 1 diabetes mellitus (T1D) patient cohort in eICU and MIMIC-IV, we identify patients with T1D and matched healthy individuals. A patient has T1D if the ICD-10 code~\texttt{E10} (or the equivalent ICD-9 code~\texttt{250}) is present in the electronic health records. Matched healthy patients are defined by three criteria. First, the patient must not contain any of the following ICD-10 (and ICD-9 equivalent) codes: \texttt{E11}, \texttt{E13}, \texttt{E12}, \texttt{E08}, \texttt{E09}, \texttt{R73}, and \texttt{O24}. An initial healthy patient cohort is constructed using these filtering codes. Next, we identify frequently co-occurring ICD codes between the initial set of patients and patients with T1D to filter out generic ICD codes (threshold $= 20$). Finally, healthy patients are matched with a T1D patient if: they are of the same gender, they are born in the same decade, and they share at least 50\% of ICD codes.



\section{Further Implementation Details}\label{appendix:implementation}

We provide code and instructions in our GitHub repository to implement \name, baselines, and ablations: \url{https://github.com/mims-harvard/CLEF}. For the implementation of baselines, we followed the authors' recommendations on model design and hyperparameter selection from the original publications.


\subsection{Hyperparameter Sweep}

For all models trained from scratch, the selection of hyperparameters are: dropout rate~$\in [0.3, 0.4, 0.5, 0.6]$, learning rate~$\in [0.001, 0.0001, 0.00001]$, and number of layers (or blocks in xLSTM) $\in [4, 8]$. As the number of heads must be divisible by the number of features, the number of heads for eICU (18 lab tests)~$\in [2, 3, 6, 9]$ and for others~$\in [4, 8]$. For xLSTM, additional hyperparameters are: 1D-convolution kernel size $\in [4, 5, 6]$ and QVK projection layer block size $\in [4, 8]$.


\subsection{Best Hyperparameters}

\xhdr{MIMIC-IV dataset} The best hyperparameters for models trained on the MIMIC-IV dataset are: dropout rate~$= 0.6$, learning rate~$= 0.0001$, number of layers (or blocks in xLSTM)~$= 8$, and number of heads~$= 4$. For xLSTM models, 1D-convolution kernel size~$= 4$ and QVK projection layer block size~$= 4$. For \name models, the number of FNN in the concept encoder~$= 1$ (Appendix Figures~\ref{fig:supp_benchmark_full_mae}-\ref{fig:supp_benchmark_full_rmse}).

\xhdr{eICU dataset} The best hyperparameters for  models trained on the eICU dataset are: dropout rate~$= 0.6$, learning rate~$= 0.0001$, number of layers (or blocks in xLSTM)~$= 8$, and number of heads~$= 6$. For xLSTM models, the number of heads~$= 2$, 1D-convolution kernel size~$= 4$ and QVK projection layer block size~$= 4$. For \name models, the number of FNN in the concept encoder~$= 1$ (Appendix Figures~\ref{fig:supp_benchmark_full_mae}-\ref{fig:supp_benchmark_full_rmse}).


\xhdr{Waddington-OT (WOT) dataset} The best hyperparameters for  models trained on the WOT dataset are: dropout rate~$= 0.6$, learning rate~$= 0.00001$, number of layers (or blocks in xLSTM)~$= 4$, number of heads~$= 8$. For xLSTM models, 1D-convolution kernel size~$= 4$ and QVK projection layer block size~$= 8$. For \name models, the number of FNN in the concept encoder~$= 0$ (Appendix Figures~\ref{fig:supp_benchmark_full_mae}-\ref{fig:supp_benchmark_full_rmse}).



\newpage



\section{Additional Figures and Tables}\label{appendix:figures}



\begin{figure*}[ht]
\begin{center}
\centerline{\includegraphics[width=0.95\textwidth]{FIG/suppfig_benchmark_full.pdf}}
\caption{Benchmarking the performance of \name, baselines, and ablation models on \textbf{(a)} immediate and \textbf{(b)} delayed sequence editing. Performance is measured by MAE. The models are trained using a standard cell- or patient-centric random split. Not shown for visualization purposes are the performances of VAR models on eICU and MIMIC-IV datasets: on immediate sequence editing, MAE for eICU and MIMIC-IV are $55982.74$ and $886.05$, respectively; on delayed sequence editing, MAE for eICU and MIMIC-IV are $3.02 \times 10^{39}$ and $8.62 \times 10^{23}$, respectively.}
\label{fig:supp_benchmark_full_mae}
\end{center}
\end{figure*}



\begin{figure*}[ht]
\begin{center}
\centerline{\includegraphics[width=0.95\textwidth]{FIG/suppfig_benchmark_full_rmse.pdf}}
\caption{Benchmarking the performance of \name, baselines, and ablation models on \textbf{(a)} immediate and \textbf{(b)} delayed sequence editing. Performance is measured by RMSE. The models are trained using a standard cell- or patient-centric random split. Not shown for visualization purposes are the performances of VAR models on eICU and MIMIC-IV datasets: on immediate sequence editing, MAE for eICU and MIMIC-IV are $135003.67$ and $1793.23$, respectively; on delayed sequence editing, MAE for eICU and MIMIC-IV are $5.84 \times 10^{39}$ and $1.59 \times 10^{24}$, respectively.}
\label{fig:supp_benchmark_full_rmse}
\end{center}
\end{figure*}


\clearpage



\begin{figure*}[ht]
\begin{center}
\centerline{\includegraphics[width=0.55\textwidth]{FIG/suppfig_cross_split_overlap.pdf}}
\caption{Cross-split overlap (CSO) as a function of SPECTRA parameter (SP) for eICU and MIMIC-IV datasets. CSO is defined as the number of samples in the test set that are similar to at least one sample in the train set. SP is an internal parameter used by SPECTRA to control the CSO of generated data splits. CSO decreases as SP increases.}
\label{fig:supp_spectra_cso}
\end{center}
\end{figure*}


\bigskip



\begin{table}[ht]
\caption{Generalizability of \name, baselines, and ablations on eICU and MIMIC-IV datasets in immediate and delayed sequencing. Performance is measured by the area under the spectral performance curve (AUSPC) for MAE~(Appendix Figure~\ref{fig:supp_spectra_full}) or RMSE~(Appendix Figure~\ref{fig:supp_spectra_full_rmse}). Smaller AUSPC values indicate better performance.}
\label{tab:auspc}
\vskip 0.15in
\begin{center}
\begin{tiny}
\begin{tabular}
{c|cccc|cccc}
\toprule
\textbf{Model}  & \multicolumn{4}{|c}{\textbf{eICU}} & \multicolumn{4}{|c}{\textbf{MIMIC-IV}} \\
                & \multicolumn{2}{|c}{Immediate} & \multicolumn{2}{c}{Delay} & \multicolumn{2}{|c}{Immediate} & \multicolumn{2}{c}{Delay} \\
                & MAE & RMSE & MAE & RMSE & MAE & RMSE & MAE & RMSE \\
\midrule
Transformer                      & $27.06 \pm 0.98$ & $59.83 \pm 1.14$ & $22.59 \pm 1.21$ & $50.29 \pm 0.56$ & $40.87 \pm 0.15$ & $71.77 \pm 0.21$ & $44.61 \pm 0.19$ & $80.38 \pm 0.32$ \\
\name-Transformer (FFN = 0)      & $15.16 \pm 1.09$ & $32.95 \pm 2.47$ & $14.36 \pm 1.07$ & $34.27 \pm 2.12$ & $32.79 \pm 1.41$ & $57.76 \pm 3.39$ & $35.65 \pm 1.73$ & $65.10 \pm 4.43$ \\
\name-Transformer (FFN = 1)      & $10.99 \pm 0.31$ & $27.57 \pm 0.27$ & $9.25 \pm 0.60$ & $27.69 \pm 0.22$ & $21.35 \pm 3.16$ & $36.92 \pm 5.46$ & $23.83 \pm 3.26$ & $44.11 \pm 5.83$ \\
\midrule
xLSTM                            & $28.47 \pm 0.63$ & $62.28 \pm 1.38$ & $23.11 \pm 0.91$ & $52.53 \pm 1.98$ & $40.75 \pm 0.30$ & $71.90 \pm 0.40$ & $44.31 \pm 0.24$ & $80.38 \pm 0.33$ \\
\name-xLSTM (FFN = 0)            & $16.73 \pm 2.16$ & $35.43 \pm 6.01$ & $15.32 \pm 2.10$ & $34.68 \pm 7.09$ & $32.06 \pm 1.13$ & $53.42 \pm 2.18$ & $33.88 \pm 1.98$ & $57.73 \pm 3.63$ \\
\name-xLSTM (FFN = 1)            & $11.35 \pm 0.11$ & $28.09 \pm 0.08$ & $9.04 \pm 0.18$ & $26.21 \pm 0.48$ & $21.04 \pm 2.32$ & $37.50 \pm 4.60$ & $22.63 \pm 2.61$ & $42.12 \pm 5.03$ \\
\midrule
MOMENT                           & $53.49 \pm 0.03$ & $90.54 \pm 0.03$ & $48.83 \pm 0.02$ & $82.50 \pm 0.02$ & $46.55 \pm 0.01$ & $77.22 \pm 0.01$ & $50.59 \pm 0.02$ & $85.72 \pm 0.01$ \\
\name-MOMENT (FFN = 0)           & $47.69 \pm 0.33$ & $82.18 \pm 0.34$ & $40.10 \pm 0.44$ & $72.70 \pm 0.46$ & $44.01 \pm 0.35$ & $73.83 \pm 0.63$ & $46.88 \pm 0.38$ & $81.20 \pm 1.27$ \\
\name-MOMENT (FFN = 1)           & $47.56 \pm 1.60$ & $82.81 \pm 2.88$ & $39.91 \pm 1.65$ & $72.54 \pm 3.20$ & $42.92 \pm 0.52$ & $70.72 \pm 1.96$ & $45.75 \pm 0.65$ & $77.35 \pm 2.77$ \\
\bottomrule
\end{tabular}
\end{tiny}
\end{center}
\end{table}



\begin{figure*}[ht]
\begin{center}
\centerline{\includegraphics[width=0.95\textwidth]{FIG/suppfig_spectra_full.pdf}}
 \caption{Generalizability of \name, baselines, and ablation models on \textbf{(a)}~eICU and \textbf{(b)}~MIMIC-IV patient datasets in immediate and delayed sequence editing. Performance is measured by MAE. As the SPECTRA parameter increases, the train/test split similarity decreases~(Appendix Figure~\ref{fig:supp_spectra_cso}). The area under the spectral performance curve (AUSPC) evaluation is in Appendix Table~\ref{tab:auspc}.}
\label{fig:supp_spectra_full}
\end{center}
\end{figure*}


\begin{figure*}[ht]
\begin{center}
\centerline{\includegraphics[width=0.95\textwidth]{FIG/suppfig_spectra_full_rmse.pdf}}
\caption{Generalizability of \name, baselines, and ablation models on \textbf{(a)}~eICU and \textbf{(b)}~MIMIC-IV patient datasets in immediate and delayed sequence editing. Performance is measured by RMSE. As the SPECTRA parameter increases, the train/test split similarity decreases~(Appendix Figure~\ref{fig:supp_spectra_cso}). The area under the spectral performance curve (AUSPC) evaluation is in Appendix Table~\ref{tab:auspc}.}
\label{fig:supp_spectra_full_rmse}
\end{center}
\end{figure*}




\begin{figure}[ht]
\begin{center}
\centerline{\includegraphics[width=0.85\textwidth]{FIG/suppfig_wot_cf_full.pdf}}
\caption{Benchmarking the performance of \name, baselines, and ablation models on zero-shot \textbf{(a)} immediate and \textbf{(b)} delayed counterfactual generation of cellular developmental trajectories. Performance is measured by MAE (top row) and RMSE (bottom row).}
\label{fig:supp_wot_cf_full}
\end{center}
\end{figure}



\begin{figure}[ht]
\begin{center}
\centerline{\includegraphics[width=0.85\textwidth]{FIG/suppfig_pt_int_T1D.pdf}}
\caption{\name-generated counterfactual patients via intervention on temporal concepts. Observed and \name patients are compared to quantify the differences between their lab test trajectories as a result of the intervention to halve the \textbf{(a)}~glucose levels in T1D patients from the MIMIC-IV-T1D cohort, \textbf{(b)}~white blood cell (WBC) levels in T1D patients from the MIMIC-IV-T1D cohort, and \textbf{(c)}~WBC levels in T1D patients from the eICU-T1D cohort. \textbf{(d)}~After intervening on \name to halve WBC levels, we observe whether the resulting \name patients' trajectories are ``healthier" or ``sicker" compared to other patients in the real-world cohort~(top). Further, we investigate whether the intervention effects are compounded when simultaneously reducing glucose and WBC levels by half~(bottom).}
\label{fig:supp_t1d}
\end{center}
\end{figure}


%\section{You \emph{can} have an appendix here.}

%You can have as much text here as you want. The main body must be at most $8$ pages long.
%For the final version, one more page can be added.
%If you want, you can use an appendix like this one.  

%The $\mathtt{\backslash onecolumn}$ command above can be kept in place if you prefer a one-column appendix, or can be removed if you prefer a two-column appendix.  Apart from this possible change, the style (font size, spacing, margins, page numbering, etc.) should be kept the same as the main body.
%%%%%%%%%%%%%%%%%%%%%%%%%%%%%%%%%%%%%%%%%%%%%%%%%%%%%%%%%%%%%%%%%%%%%%%%%%%%%%%
%%%%%%%%%%%%%%%%%%%%%%%%%%%%%%%%%%%%%%%%%%%%%%%%%%%%%%%%%%%%%%%%%%%%%%%%%%%%%%%


\end{document}

