

\documentclass{article}

% Recommended, but optional, packages for figures and better typesetting:
\usepackage{microtype}
\usepackage{graphicx}
\usepackage{enumitem}
\usepackage{subfigure}
\usepackage{booktabs} % for professional tables

% hyperref makes hyperlinks in the resulting PDF.
\usepackage{hyperref}

\usepackage{url}
\urlstyle{same}

% Attempt to make hyperref and algorithmic work together better:
\newcommand{\theHalgorithm}{\arabic{algorithm}}

% Use the following line for the initial blind version submitted for review:
%\usepackage{conf2025}

% If accepted, instead use the following line for the camera-ready submission:
\usepackage[accepted]{conf2025}

% For theorems and such
\usepackage{amsmath}
\usepackage{amssymb}
\usepackage{mathtools}
\usepackage{amsthm}

% if you use cleveref..
\usepackage[capitalize,noabbrev]{cleveref}

%%%%%%%%%%%%%%%%%%%%%%%%%%%%%%%%
% THEOREMS
%%%%%%%%%%%%%%%%%%%%%%%%%%%%%%%%
\theoremstyle{plain}
\newtheorem{theorem}{Theorem}[section]
\newtheorem{proposition}[theorem]{Proposition}
\newtheorem{lemma}[theorem]{Lemma}
\newtheorem{corollary}[theorem]{Corollary}
\theoremstyle{definition}
\newtheorem{definition}[theorem]{Definition}
\newtheorem{assumption}[theorem]{Assumption}
\newtheorem{problemst}{Problem Statement}[section] % Michelle edit
\theoremstyle{remark}
\newtheorem{remark}[theorem]{Remark}

\usepackage{xfrac} % Michelle edit
\usepackage{tikz}
\def\checkmark{\tikz\fill[scale=0.4](0,.35) -- (.25,0) -- (1,.7) -- (.25,.15) -- cycle;}


% Todonotes is useful during development; simply uncomment the next line
%    and comment out the line below the next line to turn off comments
%\usepackage[disable,textsize=tiny]{todonotes}
\usepackage[textsize=tiny]{todonotes}


% The \conftitle you define below is probably too long as a header.
% Therefore, a short form for the running title is supplied here:
\conftitlerunning{Controllable Sequence Editing for Counterfactual Generation}

\usepackage{xspace}
\newcommand{\marinka}[1]{{\color{magenta} [MZ: {#1}]}}
\newcommand{\michelle}[1]{{\color{purple} [ML: {#1}]}}
\newcommand{\kevin}[1]{{\color{blue} [KL: {#1}]}}
\newcommand{\shvat}[1]{{\color{cyan} [SM: {#1}]}}
\newcommand{\name}{\textsc{Clef}\xspace}
\newcommand{\longname}{\underline{C}ontro\underline{L}lable sequence \underline{E}diting for counter\underline{F}actual generation\xspace}
\newcommand{\std}[1]{\scriptsize{$\pm$#1}}
\newcommand{\xhdr}[1]{\noindent{{\bf #1.}}}


\begin{document}

\twocolumn[
\conftitle{Controllable Sequence Editing for Counterfactual Generation}

% It is OKAY to include author information, even for blind
% submissions: the style file will automatically remove it for you
% unless you've provided the [accepted] option to the conf2025
% package.

% List of affiliations: The first argument should be a (short)
% identifier you will use later to specify author affiliations
% Academic affiliations should list Department, University, City, Region, Country
% Industry affiliations should list Company, City, Region, Country

% You can specify symbols, otherwise they are numbered in order.
% Ideally, you should not use this facility. Affiliations will be numbered
% in order of appearance and this is the preferred way.
%\confsetsymbol{equal}{*}

\begin{confauthorlist}
\confauthor{Michelle M.~Li}{hms,berkowitz}
\confauthor{Kevin Li}{hms,mit}
\confauthor{Yasha Ektefaie}{hms}
\confauthor{Shvat Messica}{hms}
\confauthor{Marinka Zitnik}{hms,berkowitz,kempner}
\end{confauthorlist}

%\confaffiliation{berkowitz}{The Ivan and Francesca Berkowitz Family Living Laboratory Collaboration at Harvard Medical School and Clalit Research Institute, Boston, MA, USA}
%\confaffiliation{dbmi}{Department of Biomedical Informatics, Harvard Medical School, Boston, MA, USA}

\confaffiliation{hms}{Harvard Medical School}
\confaffiliation{berkowitz}{The Ivan and Francesca Berkowitz Family Living Laboratory Collaboration at Harvard Medical School and Clalit Research Institute}
\confaffiliation{mit}{Massachusetts Institute of Technology}
\confaffiliation{kempner}{Kempner Institute for the Study of Natural and Artificial Intelligence at Harvard University}

\confcorrespondingauthor{}{michelleli@g.harvard.edu}
\confcorrespondingauthor{}{marinka@hms.harvard.edu}

% Michelle M. Li
% Kevin Li
% Yasha Ektefaie
% Shvat Messica
% Marinka Zitnik

% You may provide any keywords that you
% find helpful for describing your paper; these are used to populate
% the "keywords" metadata in the PDF but will not be shown in the document
\confkeywords{Machine Learning}

\vskip 0.3in
]

% this must go after the closing bracket ] following \twocolumn[ ...

% This command actually creates the footnote in the first column
% listing the affiliations and the copyright notice.
% The command takes one argument, which is text to display at the start of the footnote.
% The \confEqualContribution command is standard text for equal contribution.
% Remove it (just {}) if you do not need this facility.

\printAffiliationsAndNotice{}  % leave blank if no need to mention equal contribution
%\printAffiliationsAndNotice{\confEqualContribution} % otherwise use the standard text.


\begin{abstract}
\begin{abstract}
Multi-modal models, such as CLIP, have demonstrated strong performance in aligning visual and textual representations, excelling in tasks like image retrieval and zero-shot classification. Despite this success, the mechanisms by which these models utilize training data, particularly the role of memorization, remain unclear. In uni-modal models, both supervised and self-supervised, memorization has been shown to be essential for generalization. However, it is not well understood how these findings would apply to CLIP, which incorporates elements from both supervised learning via captions that provide a supervisory signal similar to labels, and from self-supervised learning via the contrastive objective.
To bridge this gap in understanding, we propose a formal definition of memorization in CLIP (CLIPMem) and use it to quantify memorization in CLIP models. Our results indicate that CLIP’s memorization behavior falls between the supervised and self-supervised paradigms, with "mis-captioned" samples exhibiting highest levels of memorization. 
Additionally, we find that the text encoder contributes more to memorization than the image encoder, suggesting that mitigation strategies should focus on the text domain. 
Building on these insights, we propose multiple strategies to reduce memorization while at the same time improving utility---something that had not been shown before for traditional learning paradigms where reducing memorization typically results in utility decrease.
%, some of our proposed mitigations for CLIP can reduce memorization while improving downstream utility.\todo{This needs to be reworked: our CLIPMem has the practical application to identify "miscaptioned" samples, such that we can remove them from the training, and then get better results. This is particularly important given that CLIP is trained on large amounts of uncurated data from the internet, and one cannot review all these image pairs. With out metric, it becomes possible.
%aybe also include the risk of exposure for these data points that otherwise arise.
%}
% Multi-modal models, such as CLIP, exhibit strong performance in aligning visual and textual representations, thereby achieving remarkable performance in tasks like image retrieval and zero-shot classification. 
% While the models have a strong generalization ability, it is not fully understood how the models leverage their training data to achieve this.
% One factor that is often linked to a model's generalization ability is memorization. For uni-modal models, both in supervised and self-supervised, it has been shown that memorization is required for generalization.
% Yet, it is unclear how the findings will translate to CLIP because in CLIP captions as supervisory signals, somewhat akin to traditional labels, but also employs self-supervised contrastive learning. Hence, CLIP is in between both paradigms.
% To bridge this gap, we propose a formal definition of memorization in CLIP (CLIPMem) and use it to quantify memorization in CLIP models. 
% Our results show that CLIP's memorization behavior indeed falls between supervised and self-supervised paradigms. Notably, "mis-captioned" samples exhibit high levels of memorization.
% Additionally, we find that the the text encoder has a higher impact on memorization than the image encoder.
% Based on these findings, we find some effective mitigation strategies for memorization in CLIP that focus more on the text domain to maintain model performance while reducing memorization.
% Indeed, unlike in traditional supervised or self-supervised learning, where reducing memorization often reduces utility, we empirically find that some mitigations in CLIP not only reduce memorization but at the same time improve downstream utility.
\end{abstract}

\end{abstract}

\section{Introduction}
Large language models (LLMs) show significant performance in various downstream
tasks~\citep{brown_language_2020,openai_gpt-4_2024,dubey_llama_2024}. Studies
have found that training on high quality corpus improves the ability of LLMs
to solve different problems such as writing code, doing math exercises, and
answering logic questions~\citep{cai_internlm2_2024,deepseek-ai_deepseek-v3_2024,qwen_qwen25_2024}.
Therefore, effectively selecting high-quality text data is an important subject for
training LLM.

\begin{figure}[t]
    \centering
    \includegraphics[width=\linewidth]{figures/head.pdf}
    \caption{The overview of CritiQ. We (1) employ human annotators to annotate $\sim$30
    pairwise quality comparisons, (2) use CritiQ Flow to mine quality criteria, (3)
    use the derived criteria to annotate 25k pairs, and (4) train the CritiQ Scorer to
    perform efficient data selection.}
    \label{fig:overview}
\end{figure}

To select high-quality data from a large corpus, researchers manually design heuristics~\citep{dubey_llama_2024,rae_scaling_2022},
calculate perplexity using existing LLMs~\citep{marion2023moreinvestigatingdatapruning,wenzek2019ccnetextractinghighquality},
train classifiers~\citep{brown_language_2020,dubey_llama_2024,xie_data_2023} and
query LLMs for text quality through careful prompt engineering~\citep{gunasekar_textbooks_2023,wettig_qurating_2024,sachdeva_how_2024}.
Large-scale human annotation and prompt engineering require a lot of human
effort. Giving a comprehensive description of what high-quality data is like is also
challenging. As a result, manually designing heuristics lacks robustness and introduces
biases to the data processing pipeline, potentially harming model performance
and generalization. In addition, quality standards vary across different
domains. These methods can not be directly applied to other domains without significant
modifications.

To address these problems, we introduce CritiQ, a novel method to automatically
and effectively capture human preferences for data quality and perform efficient data
selection. Figure~\ref{fig:overview} gives an overview of CritiQ, comprising an agent
workflow, CritiQ Flow, and a scoring model, CritiQ Scorer. Instead of manually describing
how high quality is defined, we employ LLM-based agents to summarize quality
criteria from only $\sim$30 human-annotated pairs.

CritiQ Flow starts from a knowledge base of data quality criteria. The worker
agents are responsible to perform pairwise judgment under a given
criterion. The manager agent generates new criteria and refines them through reflection
on worker agents' performance. The final judgment is made by majority voting among
all worker agents, which gives a multi-perspective view of data quality.

To perform efficient data selection, we employ the worker agents to annotate a randomly
selected pairwise subset, which is ~1000x larger than the human-annotated one.
Following \citet{korbak_pretraining_2023,wettig_qurating_2024}, we train CritiQ
Scorer, a lightweight Bradley-Terry model~\citep{bradley_rank_1952} to convert
pairwise preferences into numerical scores for each text. We use CritiQ Scorer to
score the entire corpus and sample the high-quality subset.

For our experiments, we established human-annotated test sets to quantitatively
evaluate the agreement rate with human annotators on data quality preferences. We implemented the manager agent by \texttt{GPT-4o} and the worker
agent by \texttt{Qwen2.5-72B-Insruct}. We conducted experiments on different
domains including code, math, and logic, in which CritiQ Flow shows a consistent
improvement in the accuracies on the test sets, demonstrating the effectiveness
of our method in capturing human preferences for data quality. To validate the quality
of the selected dataset, we continually train \texttt{Llama 3.1}~\citep{dubey_llama_2024}
models and find that the models achieve better performance on downstream tasks
compared to models trained on the uniformly sampled subsets.

We highlight our contributions as follows. We will release the code to facilitate
future research.

\begin{itemize}
    \item We introduce CritiQ, a method that captures human preferences for data
        quality and performs efficient data selection at little cost of human
        annotation effort.

    \item Continual pretraining experiments show improved model performance in code,
        math, and logic tasks trained on our selected high-quality subset compared to the raw dataset.

    \item Ablation studies demonstrate the effectiveness of the knowledge base and
        the the reflection process.
\end{itemize}

\begin{figure*}[t]
    \centering
    \includegraphics[width=\linewidth]{figures/method.pdf}
    \caption{CritiQ Flow comprises two major components: multi-criteria pairwise
    judgment and the criteria evolution process. The multi-criteria pairwise
    judgment process employs a series of worker agents to make quality
    comparisons under a certain criterion. The criteria evolution process aims to
    obtain data quality criteria that highly align with human judgment through
    an iterative evolution. The initial criteria are retrieved from the
    knowledge base. After evolution, we select the final criteria to annotate
    the dataset for training CritiQ Scorer.}
    \label{fig:method}
\end{figure*}



\section{Related work}
\subsection{Large Language Models for Chemistry}
%化学agent,大语言模型(LLMs)在化学领域展现出巨大的潜力,被广泛应用于分子生成、性质预测、反应建模、逆合成分析等任务。例如,ChemDFM模型通过在化学文献和教科书上进行预训练,并使用大量指令进行微调,提升了多种化学任务上的表现。论文2介绍了一个名为ChemCrow的新型大型语言模型(LLM)驱动的化学助手,它通过整合多种专家设计的化学工具来增强LLM在化学领域的表现。
%尽管 LLM 已展现出强大的潜力,但它们在应对复杂的化学计算任务时仍存在诸多局限性。即使通过微调(fine-tuning)或指令调控,LLMs 仍难以适应高度复杂或需要精确计算的化学任务,且其泛化能力在面对不同化学任务时表现出不足。并且LLM 无法高效调用现有的化学计算工具,如SMILES 解析工具、分子动力学模拟工具等,更无法有效探索和利用工具之间的组合和堆叠关系。

Large language models (LLMs) have demonstrated significant potential in chemistry, with applications spanning molecular generation, property prediction, reaction modeling, and retrosynthetic analysis~\cite{fang2024molinstructionslargescalebiomolecularinstruction,tang2024prioritizingsafeguardingautonomyrisks,liao2024wordsmoleculessurveylarge}. For instance, the ChemDFM ~\cite{zhao2024chemdfmlargelanguagefoundation} pretrained on chemical literature and textbooks and further refined through extensive instruction tuning, has exhibited enhanced performance across various chemical tasks. Similarly, ChemCrow~\cite{bran2023chemcrowaugmentinglargelanguagemodels}, an LLM-powered chemistry assistant, integrates multiple expert-designed chemical tools to improve LLM performance in chemistry-related applications.Despite these advancements, LLMs continue to face challenges in handling complex chemical computations and generalizing across diverse chemical problems~\cite{ouyang2024structuredchemistryreasoninglarge,han2024generalistspecialistsurveylarge}. Moreover, they remain inefficient in utilizing existing computational chemistry tools~\cite{shi2023relmleveraginglanguagemodels}, and struggle to navigate the combinatorial and hierarchical relationships between these tools.

\subsection{Tool-augmented LLMs}
LLMs~\cite{anil2023palm,achiam2023gpt,touvron2023llama} have demonstrated strong reasoning capabilities in natural language processing and scientific computing. However, they face limitations in specialized tasks in fields such as chemistry and physics~\cite{yang2024moosechemlargelanguagemodels}, including constrained computational accuracy, insufficient numerical reasoning abilities, and a lack of collaboration with external tools. To address these shortcomings, researchers have recently proposed the tool-augmented LLMs approach~\cite{qin2023toolllmfacilitatinglargelanguage,wang2024gtabenchmarkgeneraltool,yang2023autogptonlinedecisionmaking}, enabling LLMs to dynamically call external tools and thereby enhance their task execution capabilities.
Representative methods include ReAct~\cite{yao2023reactsynergizingreasoningacting}, which combines chain-of-thought reasoning (CoT)~\cite{wei2023chainofthoughtpromptingelicitsreasoning} with tool invocation to allow LLMs to dynamically acquire external information during decision-making, and Toolformer~\cite{schick2023toolformerlanguagemodelsteach}, which enables LLMs to autonomously decide when to call tools, improving the accuracy of computational tasks. 
Despite these advancements, existing research primarily focuses on single-tool invocation and has yet to explore hierarchical combinations of tools. A single tool is often insufficient to solve complex scientific problems, whereas the collaborative invocation of multiple tools holds promise for enhancing the reasoning capabilities of LLMs in chemical tasks.


\section{\name}

\name manipulates sequences based on user-specified conditions and temporal coordinates to address  ``what if" questions. Given a sequence, a forecast time step, and a counterfactual condition, \name modifies only the relevant portions of the sequence while preserving unaffected elements, ensuring causal consistency. For example, based on a patient's historical lab test results, \name generates a future lab test trajectory conditioned on a given treatment~(Figure~\ref{fig:problem}a). Architecturally, \name has four key components: (i) a sequence encoder \(F\) that extracts temporal features from historical sequence data, (ii) a condition adapter \(H\) that maps counterfactual conditions to latent representations, (iii) a concept encoder \(E\) that learns temporal concepts, representing trajectory patterns over time, and (iv) a concept decoder \(G\) that applies these concepts to generate counterfactual sequences.



\subsection{Problem definition}
\begin{figure*}[ht]
\begin{center}
\centerline{\includegraphics[width=0.95\textwidth]{FIG/figure2.png}}
\caption{Overview of \name's architecture and capabilities. \textbf{(a)}~Given an input sequence, forecast time, and condition embedding from a frozen pretrained~(\textsc{pt}) embedding model, \name generates a counterfactual sequence via immediate or delayed sequence editing. \textbf{(b)}~\name is composed of a sequence encoder, condition adapter, concept encoder, and concept decoder. \name has two key capabilities: \textbf{(c)}~forecasting counterfactual sequences at any time in the future and under any condition~(e.g.,~medical codes), and \textbf{(d)}~generating counterfactual sequences by intervening on \name's learned temporal concepts.}
\label{fig:clef}
\end{center}
\vskip -0.35in
\end{figure*}



\begin{definition}[Sequence editing]
\label{def:cse}
Sequence editing is the local sample-level modification of sequence $\mathbf{x}$ to generate a counterfactual sequence~$\hat{\mathbf{x}}_{:,t_j}$ under a given condition~$s$ at a specific time~$t_j$. There are two types of controllable sequence editing: immediate and delayed~(Figure~\ref{fig:clef}a).
\begin{itemize}[leftmargin=*,noitemsep,topsep=0pt]
    \item \textbf{Immediate sequence editing:} Given sequence $\mathbf{x}_{:,t_0:t_i}$ and condition $s$ to occur at time $t_{i + 1}$, forecast $\hat{\mathbf{x}}_{:,t_{i + 1}}$.
    \item \textbf{Delayed sequence editing:} Given sequence $\mathbf{x}_{:,t_0:t_i}$ and condition $s$ to occur at time $t_j \geq t_{i + 1}$, forecast $\hat{\mathbf{x}}_{:,t_j}$.
\end{itemize}
\end{definition}
%
Example scenarios for immediate sequence editing include: \textit{What if we perturb the cells now?} and \textit{What if we perform surgery on the patient today?} (Section~\ref{results:r1}). In contrast, delayed sequence editing applies to questions such as: \textit{What if we perturb the cells in ten days?} and \textit{What if we perform surgery on the patient next year?} (Section~\ref{results:r2}).

\begin{definition}[Temporal concept]\label{def:concept}
A temporal concept $\mathbf{c}$ for sequence $\mathbf{x}_{:,t_j:t_k}$ is defined by $\displaystyle \mathbf{c} = \sfrac{\mathbf{x}_{:,t_k}}{\mathbf{x}_{:,t_j}}$ for time steps $t_j$ and $t_k$ where $t_k > t_j$. It can be interpreted as the trajectory (or rate of change of each variable in the sequence) between any pair of time steps.
\end{definition}


\begin{definition}[Controllable sequence editing]
\label{def:concept_cse}
Concept encoder $E$ and decoder $G$ are able to leverage temporal concepts $\mathbf{c}$ to perform controllable sequence editing on dataset $\mathcal{D}$ if the following are satisfied.
\begin{itemize}[leftmargin=*,noitemsep,topsep=0pt]
    \item
    Condition $s$ on $\mathbf{x}_{:,t_0:t_i}$ at time step~$t_j$ learns $\mathbf{c}$ that accurately forecasts $\hat{\mathbf{x}}^s_{:,t_j}$ such that $\hat{\mathbf{x}}^s_{:,t_j} \simeq \mathbf{x}^s_{:,t_j}$.
    \item 
    Counterfactual condition $a \neq s$ on $\mathbf{x}_{:,t_0:t_i}$ at $t_j$ learns $\mathbf{c}' \neq \mathbf{c}$ that forecasts $\hat{\mathbf{x}}^a_{:,t_j}$ such that $\hat{\mathbf{x}}^a_{:,t_j} \neq \hat{\mathbf{x}}^s_{:,t_j}$ and, if known, $\hat{\mathbf{x}}^a_{:,t_j} \simeq \mathbf{x}^a_{:,t_j}$.
    \end{itemize}
\end{definition}



\begin{problemst}[\name]
\label{prob}
Given a sequence encoder~$F$, condition adapter~$H$, concept encoder~$E$, and concept decoder~$G$ trained on a longitudinal dataset $\mathcal{D}$, \name learns temporal concept $\mathbf{c} = E(F(\mathbf{x}_{:,t_0:t_i}, t_j), H(s))$ to forecast $\hat{\mathbf{x}}^s_{:,t_j} = G(\mathbf{x}_{:,t_i}, \mathbf{c})$ for any sequence $\mathbf{x}_{:,t_0:t_i} \in \mathcal{D}$, future time step $t_j > t_i$, and condition $s$.
\end{problemst}


\subsection{\name model}
Our study proposes a hierarchical tool stacking method, named ChemHTS, to optimize the tool invocation process of LLMs in chemistry-related tasks. ChemHTS iteratively refines and optimizes the tool stacking trajectory, ultimately identifying the optimal sequence of tool calls to enhance computational performance (Fig.\ref{fig:010method}).
\subsection{Step 1: Tool Self-stacking Warmup}

%将工具分为两类,计算和检索。
%由于化学任务研究往往是理论计算与实验知识的结合,通过计算预测未知性质或过程,以及通过检索获取已有知识,因此本文将工具集$\mathcal{L}$分为compute-based tools $\mathcal{L}_c$ and retriever-based tools $\mathcal{L}_r$两类别的工具。
%为了通过测试单个工具在多次调用后的表现,确定其是否适合进行堆叠,我们首先进行单个工具的自我叠加预热阶段。
To evaluate the performance of a single tool after multiple invocations and determine its suitability for stacking, we first conduct a self-stacking warm-up phase for the individual tool.
Since research in chemical tasks often integrates theoretical computations with experimental knowledge, leveraging computation to predict unknown properties or processes and retrieval to access existing knowledge, we categorize the chemical LLM toolset $\mathcal{L}$ into two types: compute-based tools $\mathcal{L}_c$ and retriever-based tools $\mathcal{L}_r$.
For computational tools, it may be necessary to test whether repeated invocations lead to improved computational accuracy or stability. In contrast, for retriever-based tools, it may be important to evaluate whether multiple retrievals can enhance knowledge coverage.

For a given chemical task $\mathcal{T}$ and its corresponding tool library $\mathcal{L}$, the performance of an individual tool ($ToolA_i$) and its self-stacking ($A_i \to A_i$) is evaluated. If repeated invocations of the same tool type result in performance degradation, further stacking of that tool is discontinued. From each category, the top-k most effective tool combinations are selected for the next stage of stacking, as outlined in Algorithm~\ref{alg:HTS}.. 
These refined tool-stacking pathways are encapsulated as new tools, reducing the subsequent search space, improving computational efficiency, and providing stronger tool combination strategies for the next phase.
\subsection{Step 2: Multi-Layer Optimization}

\section{Local samplers for spin systems}\label{sec:local}

%In this section, we will construct a new marginal sampling oracle for permissive $q$-spin systems, and use it to construct our local sampler and prove Theorems \ref{theorem:local-sampler} and \ref{theorem:sublinear-sampler}.  Our construction is inspired by the following simple rejection sampling procedure of sampling from $\mu^{\sigma}_{v}$ when given the neighborhood configuration $\sigma\in [q]^{N(v)}$ for some $v\in V$:
In this section, we construct a new marginal sampling oracle for permissive $q$-spin systems. We will use this oracle to build our local sampler and prove Theorems \ref{theorem:local-sampler} and \ref{theorem:sublinear-sampler}. Our construction is inspired by a simple rejection sampling procedure for sampling from $\mu^{\sigma}_v$, given the neighborhood configuration $\sigma \in [q]^{N(v)}$ for some $v \in V$. 
This rejection sampling procedure is as follows:
\begin{itemize}
    \item Repeat the following steps:\label{item:reject-sampling}
    \begin{enumerate}
        \item propose a value $c\in [q]$ distributed according to $\lambda_v$;\label{item:propose}
        \item with probability $\prod\limits_{e=(u,v)\in E}A_e(\sigma(u),c)$, accept the proposal and return $c$ as the final value, otherwise, reject the proposal and go to Step \ref{item:propose}. 
    \end{enumerate}
\end{itemize}


%Note that the well-definedness of the above procedure is by \Cref{cond:main} that all $\lambda_v$ and $A_e$ are normalized. For any $q$-spin system $\+S=(G=(V,E),(\bm{\lambda},\bm{A})))$, given a vertex $v \in V$, we can then define a marginal sampling oracle at $v$ based on the rejection sampling procedure above.

Note that the well-definedness of the above procedure follows from \Cref{cond:main}, which ensures that all $\lambda_v$ and $A_e$ are normalized. Given a $q$-spin system $\+S = (G = (V, E), \bm{\lambda}, \bm{A})$, for any vertex $v \in V$, we can then define a marginal sampling oracle at $v$ based on the rejection sampling procedure.

\begin{algorithm}[H]
\caption{a marginal sampling oracle for $q$-spin systems} \label{Alg:coupler-spin-system}
\SetKwInput{KwData}{Oracle access}
\KwIn{A $q$-spin system $\+S=(G=(V,E),\bm{\lambda},\bm{A})$, a vertex $v\in V$.}
\KwOut{A value $X \in [q]$.}
\KwData{$\+O(u)$ for each $u\in N(v)$.}
Sample an infinite long sequence of i.i.d.~tuples $\{(c_i,(r_{i,u})_{u\in N(v)})\}_{1\leq i<\infty}$ where
%\begin{itemize}
    each $c_i\in [q]$ is distributed as $\lambda_v$ and each $r_{i,u}$ is chosen uniformly from $[0,1]$\;
%\end{itemize}  
\label{line:interpret}
$i^* \gets \min \{i \mid \forall e=(u,v)\in E, r_{i,u}<A_e(\+O(u), c_i)\}$\;
\label{line:coupler-neighbors}
\Return $c_{i^*}$\;
%\label{line:aaa}
\end{algorithm}

%We remark that in \Cref{line:coupler-neighbors} of \Cref{Alg:coupler-spin-system}, such an $i^*$ always exists as $A_{e}$ is normalized to have its maximum entry equal to $1$, according to \Cref{cond:main}. We first prove the following lemma.
We remark that in \Cref{line:coupler-neighbors} of \Cref{Alg:coupler-spin-system}, such an $i^*$ always exists because $A_e$ is normalized to have its maximum entry equal to $1$, as stated in \Cref{cond:main}. We now present the following lemma.

\begin{lemma}\label{lemma:coupler-correctness}
Suppose that the input $q$-spins system $\+S$ satisfies \Cref{cond:main}. Then, \Cref{Alg:coupler-spin-system} implements a marginal sampling oracle at $v$.
%\todo{correctness of the coupler}
\end{lemma}

\begin{proof}
Given a $q$-spin system $\+S=(G=(V,E),\bm{\lambda},\bm{A})$, in \Cref{Alg:coupler-spin-system}, for each $i\geq 1$ and each $u\in N(v)$, recall that each $c_i$ is chosen distributed as $p\in \Delta_q$ where $p(x) \propto \lambda(x)$ and each $r_{i,u}$ is independently chosen uniformly from $[0,1]$. Let $\+{D}_i$ be the event that 
\[
\+D_i: \forall e=(u,v)\in E, r_{i,u}<A_e(\+O(u), c_i),
\]
then for any $x\in [q]$, note that $\+O(u_j) = \sigma(u_j)$ under assumption, we have
%$$\Pr{\forall e=(u,v)\in E, r_i(u)<\bm{A_e}(\sigma_u, c_i) }=\frac{\sum_{c\in [q]}\lambda(c)\prod_{(u,v)\in E}A(\sigma_u,c)}{\sum_{c\in [q]}\lambda(c)}$$ 
\begin{align*}
    \Pr{c_i=x \mid \+{D}_i} 
    &= \frac{\Pr{c_i=x \wedge \+{D}_i}}{\Pr{\+{D}_i}} \\
    &= \frac{\lambda_v(x)\prod\limits_{e=(u,v)\in E}A_e(\sigma(u),x)}{\sum\limits_{c\in [q]}\left(\lambda_v(c)\prod\limits_{e=(u,v)\in E}A_e(\sigma(u),c)\right)} = \mu_v^\sigma(x).
\end{align*}
Let $i^*$ be the smallest index chosen in \Cref{line:coupler-neighbors} of \Cref{Alg:coupler-spin-system}, i.e., 
$i^* = \min \{i \mid \+{D}_i\}$.
$f(v, \sigma )$ follows the distribution of $c_{i^*}$ conditioning on $\+{D}_{i^*}$, concluding the proof of the lemma.
\end{proof}

The marginal sampling oracle in \Cref{Alg:coupler-spin-system} as originally designed would require a significant number of oracle calls in \Cref{line:coupler-neighbors}, potentially violating the efficiency condition outlined in \Cref{condition:fast-termination}. 
%However, the marginal sampling oracle in \Cref{Alg:coupler-spin-system} would require too many oracle calls in \Cref{line:coupler-neighbors}, contradicting the efficiency condition (\Cref{condition:fast-termination}). 
The key optimization is to invoke the oracle $\+O(u)$ only when necessary for each neighbor $u \in N(v)$, rather than for every iteration in \Cref{Alg:coupler-spin-system}.
%The key idea in optimizing the above process is straightforward: invoke $\+O(u)$ only when necessary for each $u\in N(v)$. Recall the definition $\lb=C(\Delta,\delta)\defeq 1-\frac{1-\delta}{2\Delta}$ from \Cref{cond:main}. 
Formally, assuming \Cref{cond:main} holds, in \Cref{line:coupler-neighbors} of \Cref{Alg:coupler-spin-system}, if $r_{i,u} < \lb$, where $\lb = C(\Delta, \delta) \defeq 1 - \frac{1-\delta}{2\Delta}$, then the inequality $r_{i,u}<A_e(\+O(u),c_i)$ will hold true regardless of the value of $\+O(u)$. This is because the term $\lb$ is chosen such that $r_{i,u}$ is sufficiently small to ensure success in the comparison without needing the actual value of $\+O(u)$. Consequently, it becomes unnecessary to call $\+O(u)$ when $r_{i,u} < \lb$.
With this idea of optimization, we propose the following implementation of $\eval^{\+O}(v)$, presented in \Cref{Alg:eval}, which builds upon the above idea to efficiently sample without violating the fast termination condition.
%Formally, assuming \Cref{cond:main} holds, in \Cref{line:coupler-neighbors} of \Cref{Alg:coupler-spin-system}, if $r_{i,u} < \lb$, then the inequality $r_{i,u}<A_e(\+O(u),c_i)$ holds true regardless of the value of $\+O(u)$. Consequently, it is unnecessary to call $\+O(u)$ in this case. We then propose $\eval^{\+O}(v)$ building on the optimization above.


%if we directly call $\resolve(\pred_t(u);M,R)$ for each $u\in N(v)$ to construct the configuration $\sigma \in [q]^{N(v)}$, \emph{\coupler} satisfies \Cref{condition:local-correctness}. 

%if for each $u\in N(v)$, $\sigma_u$ is determined as the color of $u$ at time $t$ by recursively calling $\resolve(\pred_t(u);M,R)$. If we directly implement \Cref{Alg:resolve} by using \Cref{Alg:coupler-spin-system} as the $\sample$ process in \Cref{Line:resolve-sample}, according to \Cref{theorem:lsample-correctness}, the algorithm produce perfect sample if it satisfies \Cref{condition:immediate-termination}. However, the above approach always leads to infinite recursion. 

%Consider optimizing the above process, instead of recursively calling $\resolve(\pred_t(u);M,R)$ for all of $v$'s neighbors, we only call it when 'necessary'. Formally, in \Cref{line:coupler-neighbors} of \Cref{Alg:coupler-spin-system}, if $r_i(u) < \min\{A_e(x,c_i)\mid x\in [q]\}$, then the expression $r_i(u)<A_e(\sigma_u,c_i)$ holds true regardless of the value of $\sigma_u$. In this case, it is not necessary to call $\resolve(\pred_t(u);M,R)$ recursively. Based on the above optimization, we give our main algorithm as follow.

\begin{algorithm}[H]
\caption{$\eval^{\+O}(v)$} \label{Alg:eval}
\SetKwInput{KwData}{Global variables}
\KwIn{A $q$-spin systems $\+S=(G=(V,E),\bm{\lambda},\bm{A})$, a vertex $v\in V$.}
\KwOut{A value $c \in [q]$.}
%\KwData{a randomness map $R: \mathbb{Z} \to [0,1] \cup \{\perp\}$, a color map $M: \mathbb{Z}\to [q]\cup\{\perp\}$;}
%\lIf{$M(t) \neq \perp$}{\Return $M(t)$}
%\lIf{$R(t) = \perp$}{$R(t) \gets \randseed \in [0,1]_{\text{u.a.r}}$}
%$M(t)\gets \sample(t)$\;
 Sample an infinite long sequence of i.i.d.~tuples $\{(c_i,(r_{i,u})_{u\in N(v)})\}_{1\leq i<\infty}$, where
%\begin{itemize}
    each $c_i\in [q]$ is distributed as $\lambda_v$ and each $r_{i,u}$ is chosen uniformly from $[0,1]$\;\label{Line:eval-sample}
\For{$i=1,2,...$\label{Line:reject}}{
${flag} \gets 1$\;
%\For{$j=1$ to $|N(v)|$\label{Line:eval-for}}{
\For{$e=(u,v)\in E$\label{Line:eval-for}}{
\If{$r_{i,u}\geq \lb$\label{Line:eval-cond}}{
\lIf{$r_{i,u}\geq A_{e}(\+O(u),c_i)$\label{Line:eval-if}}{${flag} \gets 0$}
}
}
\lIf{${flag}=1$\label{Line:eval-return}}{\Return $c_i$}
}
\end{algorithm}


\begin{remark}[principle of deferred decision]\label{remark:lazy-samples}
    In \Cref{Line:eval-sample} of \Cref{Alg:eval}, we are required to sample an infinitely long sequence $\{(c_i,\{r_{i,u}\}_{u\in N(v)})\}_{1\leq i<\infty}$. 
    Obviously, it is not feasible to directly sample an infinite number of random variables for implementation.  
    Instead, we adopt the principle of deferred decision: each $c_i$ and $r_{i,u}$ is generated only when they are accessed in  \Cref{Line:eval-for,Line:eval-if} of \Cref{Alg:eval}. 
\end{remark}

Next, we show that the marginal sampling oracle $\eval^{\+O}(v)$ in \Cref{Alg:eval} satisfies the conditions for both correctness and efficiency. 

\begin{lemma}\label{lemma:eval-correctness}
Suppose that the input $q$-spins system $\+S$ satisfies \Cref{cond:main}. Then, the marginal sampling oracle $\eval^{\+O}(v)$ implemented as in \Cref{Alg:eval} satisfies both \Cref{condition:local-correctness} and \Cref{condition:immediate-termination}.
\end{lemma}

\begin{proof}

Condition \ref{condition:local-correctness} can be verified directly by \Cref{lemma:coupler-correctness} and comparing Algorithms \ref{Alg:coupler-spin-system} and \ref{Alg:eval}.

For Condition \ref{condition:immediate-termination}, recall that $\+E_v$ is the event that $\eval^{\+O}(v)$ in \Cref{Alg:eval} terminates without any calls to $\+O$. Note that $\+E_v$ occurs if and only if \Cref{Alg:eval} terminates within $1$ round of the loop at \Cref{Line:reject} and $r_{1,u}<\lb$ holds for each $u\in N(v)$. Since $c_1$ and each $r_{1,u}$ are independent, assuming the $q$-spin system $\+S$ satisfies \Cref{cond:main} with constant $\delta>0$, we immediately have
\begin{align*}
\Pr{\+E_t} \geq C^{\Delta}>0. 
\end{align*}

It verifies that $\eval^{\+O}(v)$ satisfies Condition \ref{condition:immediate-termination} assuming Condition \ref{cond:main} holds.
\end{proof}

\begin{lemma}\label{lemma:eval-efficiency}
Suppose that the input $q$-spins system $\+S$ satisfies \Cref{cond:main}. Then, the marginal sampling oracle $\eval^{\+O}(v)$ implemented as in \Cref{Alg:eval} satisfies \Cref{condition:fast-termination}.
\end{lemma}

\begin{proof}
Fix some $v\in V$. Also fix some $\sigma\in [q]^{N(v)}$ and assume that $\+O(u)$ returns $\sigma(u)$ within $\eval^{\+O}(v)$. In one round of the outer for loop at \Cref{Line:reject}, it is direct to see the expected number of calls to $\+O(u)$ for each $u\in N(v)$ is
\begin{align}
 \sum\limits_{c\in[q]}\left(\lambda_v(c) \sum_{e=(u,v)\in E} (1-\lb)\right).\label{eq:expect-one-round}
\end{align}


 Let $\+I^{\sigma}_v$ be the number of executions of the outer for loop at \Cref{Line:reject} in $\eval^{\+O}(v)$. Note that $\+I^{\sigma}_v$ corresponds exactly to the number of executions of \Cref{item:propose} in the rejection sampling. Therefore,
\begin{align}
\E{\+{I}^{\sigma}_v} = \frac{1}{\sum\limits_{c\in[q]}\left(\lambda_v(c)\prod\limits_{e=(u,v)\in E}A_e(\sigma(u),c)\right)}.
\label{eq:expect-number-of-round}
\end{align}

Combining \eqref{eq:expect-one-round}, \eqref{eq:expect-number-of-round} and the linearity of expectation, assuming the $q$-spins system $\+S$ satisfies \Cref{cond:main} with constant $\delta>0$,
\begin{align*}
\E{\+{T}^{\sigma}_v} =& \E{\+{I}^{\sigma}_v}\cdot \sum\limits_{c\in[q]}\left(\lambda_v(c) \sum\limits_{e=(u,v)\in E} (1-\lb)\right) \\
 \text{(by \Cref{cond:main})}\quad \leq &\frac{\sum\limits_{c\in [q]}\left(\lambda_v(c)\cdot (1-\lb)\cdot |N(v)| \right)}{\sum\limits_{c\in[q]}\left(\lambda_v(c)\cdot C^{\Delta}\right)} \\
 \text{(by $\lb=1-\frac{\delta}{2\Delta}$)}\quad        \leq &\frac{\sum\limits_{c\in [q]}(\lambda_v(c)(1-\delta)/2)}{\sum\limits_{c\in [q]}(\lambda_v(c)(1+\delta)/2)} \\
         < &1.
\end{align*}

It proves that $\eval^{\+O}(v)$ satisfies Condition \ref{condition:fast-termination} assuming Condition \ref{cond:main} holds.
\end{proof}

We are now ready to prove \Cref{theorem:local-sampler}.

\begin{proof}[Proof of \Cref{theorem:local-sampler}]
    We use \Cref{Alg:lsample} as our local sampler, where the subroutine $\eval^{\+O}(v)$ is implemented as \Cref{Alg:eval} with \Cref{remark:lazy-samples}. Here, the guarantee of correctness is by \Cref{lemma:lsample-correctness} and \Cref{lemma:eval-correctness}.
    For efficiency, by Lemmas \ref{lemma:lsample-efficiency} and \ref{lemma:eval-efficiency}, we have that the expected number of $\resolve$  calls is $O(|\Lambda|)$. Also, note that each outer loop either terminates directly or results in at least one call to $\resolve$. Hence, the overall running time is bounded by $\Delta \log q$ times the total number of $\resolve$  calls, which is $O(|\Lambda|\Delta\log q)$ in expectation.
\end{proof}


 \subsection{Faster local sampler with improved marginal sampling oracle}
We now present an optimized implementation of the marginal sampling oracle $\eval^{\+O}(v)$ by leveraging the additional assumption on preprocessing in \Cref{assumption:additional-access-model}.
This results in a more efficient local sampler, proving \Cref{theorem:sublinear-sampler}.

In \Cref{Alg:eval},  Lines \ref{Line:eval-for}-\ref{Line:eval-if}  identify all neighbors $u\in N(v)$ for which $r_{i,u}\geq \lb$, and then check whether $r_{i,u}$ additionally satisfies $r_{i,u}\geq A_{(u,v)}(\+O(u),c_i)$. The running time in \Cref{theorem:local-sampler} arises because, while the number of recursive calls to $\resolve$ is bounded by \Cref{lemma:lsample-correctness}, such $u$s are typically sparse among the $|N(v)|$ neighbors. As a result, an extra $O(\Delta)$ factor is incurred in these cases. 
To mitigate this overhead, we propose using binary search to efficiently locate all such $u$s. The improved implementation is detailed in \Cref{Alg:eval-efficient}.


    
      

\begin{algorithm}
\caption{$\eval^{\+O}(v)$ (more efficient implementation)} \label{Alg:eval-efficient}
\SetKwInput{KwData}{Global variables}
\KwIn{A $q$-spin systems $\+S=(G=(V,E),\bm{\lambda},\bm{A})$, a variable $v\in V$.}
\KwOut{A value $c \in [q]$.}
%\KwData{a randomness map $R: \mathbb{Z} \to [0,1] \cup \{\perp\}$, a color map $M: \mathbb{Z}\to [q]\cup\{\perp\}$;}
%\lIf{$M(t) \neq \perp$}{\Return $M(t)$}
%\lIf{$R(t) = \perp$}{$R(t) \gets \randseed \in [0,1]_{\text{u.a.r}}$}
%$M(t)\gets \sample(t)$\;
Assume an arbitrary ordering $N(v)=\{u_1,u_2,\dots,u_{|N(v)|}\}$ as in \Cref{assumption:additional-access-model}\label{Line:eval-efficient-assumption}\;
\For{$i=1,2,...$}{
Draw $c\sim \lambda_v$\label{Line:eval-efficient-draw-c}\;
$flag \gets 1,counter\gets 1$\;\label{Line:eval-efficient-initialize}
\While{$counter\leq |N(v)|$\label{Line:eval-efficient-while}}{
$l\gets counter,r\gets |N(v)|$\label{Line:eval-efficient-initialize-bs}\;
Draw $\text{ind}_{{l,r}}\sim \textrm{Bern}(1-C^{r-l+1})$\label{Line:eval-efficient-sample-initial}\;
\lIf{$\text{ind}_{{l,r}}=0$\label{Line:eval-efficient-break-cond}}{ \textbf{break}}
\While{$l<r$\label{Line:eval-efficient-while-2}}{
    $mid\gets \left\lfloor\frac{l+r}{2}\right\rfloor$\label{Line:eval-efficient-set-mid}\;
    Draw $\text{ind}_{{l,mid}}\sim \textrm{Bern}\left(\frac{1-\lb^{mid-l+1}} {1-\lb^{r-l+1}}\right)$\label{Line:eval-efficient-sample-mid}\;
    \eIf{$\text{ind}_{{l,mid}}=1$}{
        $r\gets mid$;
    }{$l\gets mid+1$\;\label{Line:eval-efficient-l-assign}}
}
Draw $r\in [\lb,1]$ u.a.r. \label{Line:eval-efficient-draw-r}\;
\lIf{$r\geq A_{(v,u_l)}(\+O(u_l),c)$\label{Line:eval-efficient-update-flag}}{$flag \gets 0$}
$counter\gets l+1$\label{Line:eval-efficient-update-counter}\;
}
\lIf{$flag=1$\label{Line:eval-efficient-return}}{\Return $c_i$}
}
\end{algorithm}

We briefly explain \Cref{Alg:eval-efficient}. First, we order all neighbors of $v$ as assumed in \Cref{assumption:access-model}. In each outer loop iteration, we sample  $c_i$ according to the correct distribution (\Cref{Line:eval-efficient-draw-c}) and initialize a counter at $1$ (\Cref{Line:eval-efficient-initialize}). We then use binary search to find the first position $j\in [counter,N(v)]$ where the condition in \Cref{Line:eval-cond} of \Cref{Alg:eval} holds for $u_j$, or determine no such $j$ exists (Lines \ref{Line:eval-efficient-while-2}-\ref{Line:eval-efficient-l-assign}).  If a valid $j$ is found, we execute \Cref{Line:eval-if} of \Cref{Alg:eval} and update $counter\gets j+1$ (Lines \ref{Line:eval-efficient-draw-r}-\ref{Line:eval-efficient-update-counter}). Otherwise, we proceed directly to \Cref{Line:eval-return} of \Cref{Alg:eval} (\Cref{Line:eval-efficient-return}).

\begin{comment}
To actually perform the binary search,  
\begin{itemize}
    \item we initialize the interval $[l,r]=[counter,N(v)]$ (\Cref{Line:eval-efficient-initialize-bs}) and sample $\textrm{ind}_{l,r}\sim \textrm{Bern}1-\lb^{r-l+1}$ as the indicator variable of whether there exists some $j\in [l,r]$ such that $r_{i,u_j}\geq \lb$ (\Cref{Line:eval-efficient-sample-initial}).  Note that this $1-\lb^{r-l+1}$ is exactly the probability at least one such $j$ exists in $[l,r]$, assuming  $\{(c_i,(r_{i,u})_{u\in N(v)})\}_{1\leq i<\infty}$ is an infinite long i.i.d sequence where
%\begin{itemize}
    each $c_i\in [q]$ is distributed as $\lambda_v$ and each $r_{i,u}$ is chosen uniformly from $[0,1]$. 
    \item At each step of the binary search, we maintain the invariant that if our current interval is $[l,r]$, then there exists at least one $j\in [l,r]$  such that $r_{i,u_j}\geq \lb$. We then choose $mid=\left\lfloor\frac{l+r}{2}\right\rfloor$ (\Cref{Line:eval-efficient-set-mid}) and try to narrow down the search interval to either $[l,mid]$ or $[mid+1]$. To do this, we check if at least one such $j\in [l,mid]$ exists through sampling an indicator variable $\textrm{ind}_{l,mid}\sim\textrm{Bern}\left(\frac{\lb^{mid-l+1}}{\lb^{r-l+1}}\right)$ (\Cref{Line:eval-efficient-sample-mid}). Note that this indicator variable is drawn according to the probability that there exists at least one $j\in [l,mid]$  such that $r_{i,u_j}\geq \lb$ conditioning on such  there exists at least one such $j\in [l,r]$, assuming  $\{(c_i,(r_{i,u})_{u\in N(v)})\}_{1\leq i<\infty}$ is an infinite long i.i.d sequence where
%\begin{itemize}
    each $c_i\in [q]$ is distributed as $\lambda_v$ and each $r_{u,j}$ is chosen uniformly from $[0,1]$. 
\end{itemize}
\end{comment}

We now present the following technical lemma, which establishes a perfect coupling between randomness of \Cref{Alg:eval-efficient} and \Cref{Alg:eval}. Consequently, by Lemmas \ref{lemma:eval-correctness} and \ref{lemma:eval-efficiency}, \Cref{Alg:eval-efficient} satisfies Conditions \ref{condition:local-correctness}, \ref{condition:immediate-termination}, and \ref{condition:fast-termination}, assuming \Cref{cond:main} holds. %The proof of \Cref{lemma:eval-efficient-coupling} is deferred to \Cref{sec:appendix-coupling}.

\begin{lemma}\label{lemma:eval-efficient-coupling}
For each $v\in V$, assuming that $\+O(u)$ returns a consistent value for each $u\in N(v)$ in $\eval^{\+O}(v)$ 
in both \Cref{Alg:eval} and \Cref{Alg:eval-efficient}, 
the randomness in these algorithms can be coupled such that:
\begin{enumerate}
    \item \Cref{Alg:eval} and \Cref{Alg:eval-efficient} always return the same value.
    \item For each $u\in N(v)$, the number of calls to $\+O(u)$ within \Cref{Alg:eval} and \Cref{Alg:eval-efficient} is identical.
\end{enumerate}
\end{lemma}


%\section{More efficient implementation of the local sampler}\label{sec:appendix-coupling}
%In this section, we will finish the technical details of the more efficient implementation of the local sampler (\Cref{Alg:eval-efficient}). Specifically, we will prove \Cref{lemma:eval-efficient-coupling}.

\begin{proof}%[Proof of \Cref{lemma:eval-efficient-coupling}]
We couple the randomness in \Cref{Alg:eval-efficient} with the infinite long sequence of tuples 
$$\{(c_i,(r_{i,u})_{u\in N(v)})\}_{1\leq i<\infty}$$ 
in \Cref{Alg:eval} by the following process. 

\begin{definition}[coupling of randomness between \Cref{Alg:eval} and \Cref{Alg:eval-efficient}]\label{definition:process-couple}
Suppose we run \Cref{Alg:eval} and order $N(v)=\{u_1,u_2,\dots,u_{|N(v)|}\}$ as in \Cref{Line:eval-efficient-assumption} of \Cref{Alg:eval-efficient}. Then, for each $i\geq 1$, during the $i$-th iteration of the outer loop in \Cref{Alg:eval-efficient},
the randomness is coupled as follows:
\begin{enumerate}
        \item  Each time \Cref{Line:eval-efficient-draw-c} of \Cref{Alg:eval-efficient} is executed, set the drawn value of $c$  to be the same as $c_i$ drawn in \Cref{Alg:eval}. \label{item:eval-efficient-coupling-1}
        \item Each time \Cref{Line:eval-efficient-sample-initial} of \Cref{Alg:eval-efficient} is executed, set $\textrm{ind}_{l,r}=1$  if there exists at least one $x\in [counter,|N(v)|]$ such that $r_{i,u_x}\geq \lb$, and set $\textrm{ind}_{l,r}=0$ otherwise. Here, $counter$  refers to the value of the corresponding variable in \Cref{Alg:eval-efficient} at that time, and $r_{i,u_x}$ is the value drawn in \Cref{Alg:eval}.  \label{item:eval-efficient-coupling-2}
        \item Each time \Cref{Line:eval-efficient-sample-mid} of \Cref{Alg:eval-efficient} is executed, set $\textrm{ind}_{l,mid}=1$ if there exists at least one $x\in [l,mid]$ such that $r_{i,u_x}\geq \lb$, and set $\textrm{ind}_{l,mid}=0$ otherwise. Here, $counter$ refers to the value of the corresponding variable in \Cref{Alg:eval-efficient} at that time, and $r_{i,u_x}$ is the value drawn in \Cref{Alg:eval}.  \label{item:eval-efficient-coupling-3}
        \item  Each time \Cref{Line:eval-efficient-draw-r} of \Cref{Alg:eval-efficient} is executed, set the drawn value of $r$ to be $r_{i,u_l}$. Here, $l$ is the current value of the corresponding variable in \Cref{Alg:eval-efficient}, and $r_{i,u_l}$ is the corresponding value drawn in \Cref{Alg:eval}. \label{item:eval-efficient-coupling-4}
\end{enumerate}
\end{definition}

     We first show the lemma assuming that \Cref{definition:process-couple} indeed generates a coupling between the randomness of \Cref{Alg:eval} and \Cref{Alg:eval-efficient}.
     
     Note that for each $i\geq 1$ such that the $i$-th outer for loop is actually executed in \Cref{Alg:eval},
     \begin{itemize}
         \item in \Cref{Alg:eval}, we make recursive calls to $\resolve(\pred_t(u))$ for those $u$ satisfying $r_{i,u}\geq \lb$;
         \item in \Cref{Alg:eval-efficient}, we make recursive calls to $\resolve(\pred_t(u_{\ell}))$ for those $u_l$ after \Cref{Line:eval-efficient-draw-r} is executed;
     \end{itemize}
      Note that by Items \ref{item:eval-efficient-coupling-2} and \ref{item:eval-efficient-coupling-3} of \Cref{definition:process-couple} and Lines \ref{Line:eval-efficient-initialize}-\ref{Line:eval-efficient-l-assign} of \Cref{Alg:eval-efficient}, all $l$s in \Cref{Line:eval-efficient-l-assign} are the positions in \Cref{Alg:eval} such that $r_{i,u_l}\geq \lb$.  So, for these iterations, the recursive calls to $\resolve(\pred_t(u))$ in \Cref{Alg:eval} and \Cref{Alg:eval-efficient} are the same.

      When \Cref{Alg:eval} stops at some iteration $i$, it should happen that the condition at \Cref{Line:eval-if} of \Cref{Alg:eval} is never satisfied, that is, $r_{i,u}<A_e(x,c_i)$ for each $u\in N(v)$, hence \Cref{Alg:eval-efficient} should also stop at the same iteration $i$ by \Cref{item:eval-efficient-coupling-4} of \Cref{definition:process-couple} and \Cref{Line:eval-efficient-update-flag} of \Cref{Alg:eval-efficient}. Also, the returned values at this iteration are the same by \Cref{item:eval-efficient-coupling-1} of \Cref{definition:process-couple} and \Cref{Line:eval-efficient-return} of \Cref{Alg:eval-efficient}. This already proves the lemma.

      It remains to prove that \Cref{definition:process-couple} is indeed a coupling. It is sufficient to show that in the above process, conditioning on the previous randomness in \Cref{Alg:eval-efficient},
      \begin{enumerate}[label=(\alph*)]
          \item each $c$ drawn in \Cref{Line:eval-efficient-draw-c} is distributed as $\lambda_v$; \label{item:coupling-1}
          \item each $\textrm{ind}_{l,r}$ drawn in \Cref{Line:eval-efficient-sample-initial} is  distributed as $\textrm{Bern}\left(\frac{\lb^{mid-l+1}} {\lb^{r-l+1}}\right)$;  \label{item:coupling-2}
          \item each $\textrm{ind}_{l,mid}$ drawn in \Cref{Line:eval-efficient-sample-mid} is  distributed as $\textrm{Bern}\left(\frac{\lb^{mid-l+1}} {\lb^{r-l+1}}\right)$; \label{item:coupling-3}
          \item each $r$ drawn in \Cref{Line:eval-efficient-draw-r} is distributed uniformly in $[b_{v,u_l,c},1]$.\label{item:coupling-4}
      \end{enumerate} 
      Here, \Cref{item:coupling-1} is by \Cref{item:eval-efficient-coupling-1} of \Cref{definition:process-couple}. \Cref{item:coupling-2} is by \Cref{item:eval-efficient-coupling-2} of \Cref{definition:process-couple} that all the randomness $r_{i,u_x}$ for $x\in [counter,|N(v)|]$ is independent of the previous samples in \Cref{Alg:eval-efficient}. \Cref{item:coupling-3} is by Items \ref{item:eval-efficient-coupling-2} and \ref{item:eval-efficient-coupling-3} of \Cref{definition:process-couple} and each time when we draw $\textrm{ind}_{l,mid}$, the event 
      \[
      \exists x\in [l,r],\quad \text{s.t. } r_{i,u}\geq \lb,
      \]
      holds conditioning on the previous samples. Finally, \Cref{item:coupling-4} is by \Cref{item:eval-efficient-coupling-4} and the event $r_{i,u_l}\geq \lb$ holds conditioning on the previous samples. This finishes the proof of the lemma.
\end{proof}


We are now ready to prove \Cref{theorem:sublinear-sampler}.

\begin{proof}[Proof of \Cref{theorem:sublinear-sampler}]
    We use \Cref{Alg:lsample} as our local sampler, where the subroutine $\eval^{\+O}(v)$ is implemented as \Cref{Alg:eval-efficient}. Here, the guarantee of correctness is by \Cref{lemma:lsample-correctness}, \Cref{lemma:eval-correctness}, and \Cref{lemma:eval-efficient-coupling}.
    For efficiency, by Lemmas \ref{lemma:lsample-efficiency}, \ref{lemma:eval-efficiency},  and \ref{lemma:eval-efficient-coupling} we have the expected number of total $\resolve$ calls is $O(|\Lambda|)$. Note that now each outer loop of \Cref{Alg:eval-efficient} only takes $O(\log \Delta\log q)$ time, as we only need $O(\log q)$ time to sample a $q$-valued variable in \Cref{Line:eval-efficient-draw-c} of \Cref{Alg:eval-efficient} and the inner loop in Lines \ref{Line:eval-efficient-while}-\ref{Line:eval-efficient-update-counter} of \Cref{Alg:eval-efficient} runs in $O(\log \Delta)$ time due to the binary search. Note that here each binary search done results in exactly one call of $\resolve$, and the overall running time is bounded by $O(\log \Delta\log q)$ times the total number of $\resolve$  calls, which is $O(|\Lambda|\log \Delta\log q)$ in expectation.
\end{proof}


%由于遍历 Top-k 最优工具组合,并结合所有工具进行进一步堆叠。计算不同堆叠方式的任务性能。若堆叠导致性能下降或达到设定的最大堆叠深度 n,则停止堆叠。选择最终性能最优的工具堆叠路径,作为该任务的最优执行轨迹(T*,Optimal Stacking Pathway)。

%By iterating through the Top-k most optimal tool combinations and further stacking them with all available tools, the task performance of different stacking configurations is evaluated. If stacking results in performance degradation or the predefined maximum stacking depth n is reached, the stacking process is terminated. The tool stacking pathway with the highest overall performance is selected as the task's optimal execution trajectory.

To explore the optimal invocation pathways across tool combinations and progressively encapsulate the best tools, ultimately determining the globally optimal tool invocation strategy, we divide the tool-stacking process into multiple layers and perform stacking optimization step by step.
In the first layer, the Top-1 computational tool selected in Stage 1 is combined with the Top-k retrieval tools, and the Top-1 retrieval tool is combined with the Top-k computational tools. Each agent is restricted to selecting the best tool from a pool of $m$ candidate tools in each step. If the optimal performance in this layer is worse than that of Stage 1, the stacking process is terminated; otherwise, the iteration continues.
In subsequent layers, all tools are re-ranked, and the top $k$ optimal pathways in each category are selected and encapsulated as new tools. This process continues until either the maximum number of layers $n$ is reached or no further optimization space remains.
Finally, the optimal tool invocation pathway is selected as the general tool invocation strategy for the given task.





\section{Experimental setup}

\subsection{Datasets}

\name is evaluated on datasets and tasks in the biological and medical domains: cellular reprogramming experiments~(Figure~\ref{fig:data}a) and patient routine laboratory tests~(Figure~\ref{fig:data}b).

\xhdr{Cellular developmental trajectories} We introduce a novel benchmarking dataset, WOT. It is constructed using the Waddington-OT model, which simulates single-cell transcriptomic profiles of developmental time courses for individual cells \cite{schiebinger2019-ie}~(Figure~\ref{fig:data}a; Table~\ref{tab:data}). We also construct a paired counterfactual benchmarking dataset, WOT-CF (Table~\ref{tab:data}). We obtain condition embeddings of the activated transcription factors from ESM-2~\cite{lin2022language}. Refer to Appendix~\ref{appendix:cells} for further details.


\xhdr{Patient lab test trajectories}
%
We construct two real-world patient datasets of routine laboratory tests from eICU~\cite{pollard2018eicu} and MIMIC-IV~\cite{johnson2024mimic, johnson2023mimic, goldberger2000physiobank}~(Figure~\ref{fig:data}b; Table~\ref{tab:data}). In addition to a random split, we construct data splits with different levels of train/test split similarities using SPECTRA~\cite{ektefaie2024evaluating} to evaluate model generalizability~(Appendix Figure~\ref{fig:supp_spectra_cso}). For condition embeddings, we leverage pretrained embeddings of clinical codes from a clinical knowledge graph that integrates six existing databases of clinical vocabularies used in electronic health records~\cite{johnson2024unified}. Refer to Appendix~\ref{appendix:labs} for more details. 



\begin{figure}[ht]
\begin{center}
\centerline{\includegraphics[width=\columnwidth]{FIG/figure_domains.pdf}}
\caption{\name is evaluated on two real-world domains involving multivariate trajectories: \textbf{(a)}~cellular development and \textbf{(b)}~patient health. \textbf{(a)}~To study cellular development, fibroblast cells derived from mice can be artificially reprogrammed into various other cell states \textit{in vitro}. A cell's state is defined by its gene expression. Throughout reprogramming, a cell activates transcription factor (TF) genes at different time points to change its gene expression, thereby influencing its developmental trajectory. In this illustration, a mouse fibroblast is being reprogrammed over the span of 20 days~(D0-D20); color and shape represent cell state. On day 8, if the cell activates the Obox6 TF, the cell is on the path toward becoming an induced pluripotent stem cell (iPSC); whereas if it activates the Neurod4 TF, it is on the path toward becoming a neuron or astrocyte. \textbf{(b)}~The health of a human patient is often monitored through lab tests (e.g. blood sodium level, white blood cell count). The history of lab results across multiple patient visits (V1-V9) as well as candidate clinical interventions (e.g.,~medication) can be used to infer the most likely future trajectory of the patient's health. Illustrations from NIAID NIH BIOART Source~(see References). \nocite{fibroblast1,fibroblast2, fibroblast3, astrocyte, progenitor, petri, generic_immune, cajal, neuron, unidentified_offtarget, mouse, patient, syringe, wbc, hemoglobin, vial}}
\label{fig:data}
\end{center}
\vskip -0.35in
\end{figure}



\begin{table}[ht]
\caption{\textbf{Dataset statistics.} We construct three core datasets: WOT~(cellular developmental trajectories), eICU~(patient lab tests), and MIMIC-IV~(patient lab tests). We also construct a paired counterfactual cellular trajectories dataset, WOT-CF. $N$ is the number of sequences (i.e.,~cellular developmental trajectories, patient lab test trajectories), $V$ is the number of measured variables (i.e.,~gene expression, lab test), and $L$ is the length of the sequences.}
\label{tab:data}
\vskip 0.15in
\begin{center}
\begin{small}
\begin{tabular}
{lcccc}
\toprule
\textbf{Dataset}  & $N$ & $V$ & \textbf{Mean $L$} & \textbf{Max $L$} \\
\midrule
WOT      & $3,000$ & $1,480$ & $27.03 \pm 6.04$ & $37$ \\
WOT-CF   & $2,546$ & $1,480$ & $27.01 \pm 5.98$ & $37$ \\
eICU     & $108,346$ & $17$ & $20.27 \pm 25.23$ & $858$ \\
MIMIC-IV & $156,310$ & $16$ & $15.56 \pm 24.43$ & $949$ \\
\bottomrule
\end{tabular}
\end{small}
\end{center}
\vskip -0.15in
\end{table}




\subsection{Setup}


\xhdr{Metrics}
%
We use standard metrics (MAE, RMSE, and $R^2$) to quantify sequence editing performance.

\xhdr{Baselines}
%
We evaluate \name against a traditional multivariate time series algorithm, Vector Autoregression~(VAR) model~\cite{lutkepohl2005new}. As \name can leverage any type of sequence encoder, we benchmark against the state-of-the-art condition-guided counterfactual sequence generation setup with different sequential data encoders: Transformer~\cite{vaswani2017attention, narasimhan2024time, jing2024towards, zhang2023survey} and xLSTM~\cite{beck2024xlstm}. We further evaluate \name against a state-of-the-art time series foundation model, MOMENT~\cite{goswami2024moment}; specifically, we finetune an adapter for the $1024$-dimensional embeddings generated by the frozen \texttt{MOMENT-1-large} embedding model. 


\xhdr{Ablations}
%
To investigate the effectiveness of the learned temporal concepts, we evaluate against an ablated model, SimpleLinear, in which temporal concepts are simply all ones; in other words, temporal concepts are not learned nor meaningful. This ablation is inspired by traditional linear models that excel when $\mathbf{x}_{t_j} \simeq \mathbf{x}_{t_i}$~\cite{toner2024analysis, ahlmann2024deep}. We also evaluate different versions of \name with and without an FFN layer in the concept encoder~$E$~(Appendix~\ref{appendix:figures}).

\xhdr{Implementation details}
%
Models are trained on a single NVIDIA A100 or H100 GPU. All models have comparable number of parameters as their \name-based counterparts. Refer to Appendix~\ref{appendix:implementation} for details and hyperparameter selection.


\section{Results}


\begin{figure*}[ht]
\begin{center}
\centerline{\includegraphics[width=\textwidth]{FIG/figure3_benchmark.pdf}}
\caption{Benchmarking the performance of \name, baselines, and ablation models on \textbf{(a)} immediate and \textbf{(b)} delayed sequence editing. The models are trained using a standard cell- or patient-centric random split. Not shown for visualization purposes are the performances of VAR models on eICU and MIMIC-IV datasets: on immediate sequence editing, MAE for eICU and MIMIC-IV are $55982.74$ and $886.05$, respectively; on delayed sequence editing, MAE for eICU and MIMIC-IV are $3.02 \times 10^{39}$ and $8.62 \times 10^{23}$, respectively.}
\label{fig:benchmark}
\end{center}
\vskip -0.35in
\end{figure*}




We evaluate \name's performance on controllable sequence editing across multiple datasets and tasks. We aim to answer the following research questions. 
\textbf{R1:} How well does \name perform in immediate sequence editing?
\textbf{R2:} How well does \name perform in delayed sequence editing?
\textbf{R3:} How does \name generalize to unseen/new sequences?
\textbf{R4:} Can \name perform zero-shot counterfactual generation?
\textbf{R5:} How can \name be leveraged for real-world counterfactual patient trajectory simulations?
We establish that \name outperforms state-of-the-art baselines in sequence editing, demonstrating both immediate and delayed sequence editing capabilities, strong generalizability, and real-world applicability.


\subsection{\underline{R1}: Immediate sequence editing}\label{results:r1}

Immediate sequence editing involves forecasting the next time step of a sequence under a counterfactual condition. This is useful in settings where interventions take effect instantaneously, such as introducing a genetic perturbation in cellular systems or administering a drug to a patient~(Definition~\ref{def:cse}). Example counterfactual scenarios in which immediate sequence editing is applicable are: \textit{What if we treat the cells with the candidate drug now?} and \textit{What if we perform surgery on the patient today?}




\name models consistently outperform baseline models across all datasets~(Figure~\ref{fig:benchmark}a; Appendix Figures~\ref{fig:supp_benchmark_full_mae}-\ref{fig:supp_benchmark_full_rmse}). The SimpleLinear baseline, which assumes minimal temporal changes, performs comparably in some cases, but \name outperforms it on datasets where short-term dynamics are more complex. On WOT, all \name models outperform or perform comparably to the time series forecasting model, VAR. This is particularly exciting given recent findings that linear models can achieve competitive or better forecasting performance than neural network models~\cite{toner2024analysis, ahlmann2024deep}. These results highlight \name's ability to accurately modify trajectories at the right points while preserving unaffected portions of the sequence, an advantage in counterfactual reasoning.

Regardless of the sequence encoder used with \name, these models tend to outperform or perform comparably to non-\name models~(Figure~\ref{fig:benchmark}a). However, the performance of \name can be affected by the ability of the sequence encoder to capture the temporal dynamics of the input sequences. For instance, models with the MOMENT encoder yield the highest MAE in all three datasets, with and without help from \name~(Figure~\ref{fig:benchmark}a). Nevertheless, \name models with the MOMENT encoder reduce the MAE of non-\name models.


\subsection{\underline{R2}: Delayed sequence editing}\label{results:r2}


Delayed sequence editing requires forecasting a counterfactual trajectory at a future time step while maintaining causal consistency. This task is challenging, as small errors can compound over longer horizons. Example scenarios in which delayed sequence editing is applicable are: \textit{What if we treat the cells with the candidate drug in ten days?} and \textit{What if we perform the surgery on the patient next year?}

\name outperforms or performs competitively against SimpleLinear and VAR on the patient datasets, eICU and MIMIC-IV~(Figure~\ref{fig:benchmark}b; Appendix Figures~\ref{fig:supp_benchmark_full_mae}-\ref{fig:supp_benchmark_full_rmse}). \name-transformer and \name-xLSTM achieve lower MAE than SimpleLinear, whereas non-\name transformer and MOMENT baselines perform comparably or worse. As in immediate sequence editing, models using MOMENT as the sequence encoder (i.e.,~using temporal concepts with the MOMENT sequence encoder) yield the highest MAE. However, incorporating \name with MOMENT reduces the MAE to levels comparable to SimpleLinear and VAR.


On WOT, SimpleLinear and VAR outperform neural network models in delayed sequence editing~(Figure~\ref{fig:benchmark}b). This suggests that cellular developmental trajectories exhibit small and possibly noisy changes at each time step, favoring linear models~\cite{ahlmann2024deep, toner2024analysis}. Additionally, given the relatively small number of training trajectories compared to the high-dimensional state space, nonlinear models may overfit to noise more readily than linear models. Nevertheless, \name significantly reduces the MAE of non-\name models, demonstrating its effectiveness as a regularizer that mitigates short-term noise while preserving long-term trends.


\subsection{\underline{R3}: Generalization to new patient trajectories}


We assess the ability of \name models to generalize to new patient sequences. To evaluate robustness, we use the SPECTRA approach~\cite{ektefaie2024evaluating} to create challenging data splits where the test sets have minimal similarity to the training data (Appendix~\ref{appendix:labs}).

Across both the eICU and MIMIC-IV patient datasets, \name models exhibit stronger generalization than non-\name models (Figure~\ref{fig:spectra}; Appendix Figures~\ref{fig:supp_spectra_full}-\ref{fig:supp_spectra_full_rmse} and Table~\ref{tab:auspc}). For immediate and delayed sequence editing on eICU, \name-transformer and \name-xLSTM maintain stable and strong performance even as train/test divergence increases. In contrast, their non-\name counterparts degrade significantly. Although baseline MOMENT models show relatively stable performance across train/test splits in delayed sequence editing, they generalize poorly compared to \name-MOMENT models. Despite similar performance between xLSTM and \name-xLSTM in delayed sequence editing on both patient datasets (Figure~\ref{fig:benchmark}b), \name-xLSTM demonstrates superior generalizability (Figure~\ref{fig:spectra}b), highlighting the effectiveness of \name in adapting to unseen data distributions.




\begin{figure*}[ht]
\begin{center}
\centerline{\includegraphics[width=\textwidth]{FIG/figure4_spectra.pdf}}
\caption{Generalizability of \name on \textbf{(a)}~eICU and \textbf{(b)}~MIMIC-IV patient datasets in immediate and delayed sequence editing. As the SPECTRA parameter increases, the train/test split similarity decreases~(Appendix Figure~\ref{fig:supp_spectra_cso}). The area under the spectral performance curve (AUSPC) evaluation is in Appendix Table~\ref{tab:auspc}.}
\label{fig:spectra}
\end{center}
\vskip -0.35in
\end{figure*}



\subsection{\underline{R4}: Zero-shot counterfactual generation of cellular trajectories}

In addition to evaluating \name's generalizability to new patient lab test trajectories, we assess on zero-shot counterfactual generation for cellular trajectories~(Figure~\ref{fig:wot_cf}; Appendix Figure~\ref{fig:supp_wot_cf_full}). Using the Waddington-OT model, we generate sequences that remain consistent until a specified divergence time step, where an alternative condition—such as the activation of a different transcription factor—introduces a shift~(Appendix~\ref{appendix:cells}). This process yields \(1,273\) pairs of ``original" and ``counterfactual" trajectories, totaling \(2,546\) individual sequences~(Table~\ref{tab:data}). Models are trained on the ``original" trajectories and evaluated on the ``counterfactual" trajectories in a zero-shot setting.

\name-based models consistently outperform non-\name models in both immediate and delayed sequence editing~(Appendix Figure~\ref{fig:supp_wot_cf_full}). To more closely analyze delayed sequence editing performance, we examine the predictions for cellular trajectories of length 23, the most common sequence length in the dataset~(Figure~\ref{fig:wot_cf}). Since \(t_i = 10\) is the earliest divergence time step, we provide the first nine time steps \(\mathbf{x}_{:,0:9}\), the counterfactual condition, and \(t_j \in [10, 23]\) to the model. Comparing the generated and ground truth counterfactual sequences, we find that \name significantly outperforms non-\name models after time step 10, which is when the trajectories begin to diverge (Figure~\ref{fig:wot_cf}). 







\begin{figure}[ht]
\begin{center}
\centerline{\includegraphics[width=\columnwidth]{FIG/figure5_wot_cf.pdf}}
\caption{Zero-shot counterfactual generation of cellular developmental trajectories. Shown are the MAE of predictions at each time step for counterfactual sequences of length 23 (the most common sequence length in the dataset) starting at time step 10 (the earliest divergence time step of a counterfactual trajectory).}
\label{fig:wot_cf}
\end{center}
\vskip -0.35in
\end{figure}


\subsection{\underline{R5}: Case studies using real-world patient datasets}

We evaluate \name's ability to simulate counterfactual patient trajectories through temporal concept intervention. We conduct case studies on two independent cohorts of patients with type 1 diabetes mellitus (T1D), a chronic autoimmune disease in which the immune system attacks insulin-producing cells in the pancreas~\cite{quattrin2023type}.

Unlike counterfactual generation methods that rely on condition tokens to guide generation~\cite{narasimhan2024time, jing2024towards, zhang2023survey}, \name allows \textit{direct edits to the generated outputs} to produce counterfactual sequences. This capability is particularly valuable when condition tokens are insufficient, such as when prescribing medication dosage. Instead of relying on predefined conditions, \name can precisely modify the values of specific lab tests to explore their longitudinal effects.

\xhdr{Setup}
%
For an individual patient, we intervene on the temporal concepts corresponding to specific lab tests to simulate the ``reversal" or ``worsening" of symptoms, thereby generating ``healthier" or ``more severe" trajectories, respectively. Formally, given temporal concept \(\mathbf{c}\) learned from \(\mathbf{x}_{:,t_0:t_i}\) and an optional condition \(s\), we modify \(\mathbf{c}^I \neq \mathbf{c}\) such that at least one element satisfies \(\mathbf{c}_k \neq \mathbf{c}^I_k\).  

From the eICU and MIMIC-IV datasets, we construct two independent cohorts of T1D patients and matched healthy individuals~(Appendix~\ref{appendix:labs}). The eICU-T1D dataset contains \(59\) T1D patients and \(579\) matched healthy controls, while MIMIC-IV-T1D includes \(25\) T1D patients and \(226\) matched healthy controls.  

To generate counterfactual sequences, we modify specific values in temporal concept \(\mathbf{c}\), such as glucose levels, and allow \name to simulate future trajectories of length \(T = 10\). We then compare these counterfactual trajectories (i.e.,~\name-generated patients) against observed sequences from matched healthy individuals, other healthy individuals, and other T1D patients. Our hypothesis is that clinically meaningful edits will produce ``healthier" (i.e.,~more similar to healthy patients) or ``sicker" (i.e.,~more similar to other T1D patients) trajectories.  




\xhdr{Results}
%
First, we modify \name's concepts to reduce glucose levels by half, aligning them closer to normal physiological ranges. The resulting counterfactual patient trajectories exhibit higher \(R^2\) similarity with both matched and other healthy individuals compared to other T1D patients (Figure~\ref{fig:t1d}a). This suggests that \name effectively generates counterfactual trajectories indicative of a healthier state.  

Next, we simulate a worsening condition by doubling glucose levels. The resulting counterfactual trajectories generated by \name show higher \(R^2\) similarity with other T1D patients than with healthy individuals~(Figure~\ref{fig:t1d}a), as would be expected based on clinical evidence.  

Beyond direct interventions, we examine indirect changes in \name-generated patients' lab values resulting from glucose modifications. In both eICU-T1D and MIMIC-IV-T1D cohorts, lowering glucose also leads to a reduction in white blood cell (WBC) count~(Figure~\ref{fig:t1d}b; Appendix Figure~\ref{fig:supp_t1d}a). This aligns with clinical knowledge, as T1D is an autoimmune disorder where immune activity, including WBC levels, plays a critical role~\cite{quattrin2023type}. Further, when we intervene on \name to reduce WBC levels instead of glucose, we observe a concurrent drop in glucose across both cohorts~(Appendix Figure~\ref{fig:supp_t1d}b,c), reinforcing the interconnected nature of these physiological markers.  




\begin{figure}[ht]
\begin{center}
\centerline{\includegraphics[width=\columnwidth]{FIG/figure6_pt_int.pdf}}
\caption{\name-generated counterfactual patients via intervention on temporal concepts. We intervene on \name to \textbf{(a)}~halve~(top) or double~(bottom) a T1D patient's glucose levels to infer a ``healthier" or ``sicker" counterfactual patient, respectively. \textbf{(b)}~Observed and \name patients from the eICU-T1D cohort are compared to quantify the differences between their lab test trajectories (indirect effects) as a result of the intervention to halve T1D patients' glucose levels.}
\label{fig:t1d}
\end{center}
\vskip -0.35in
\end{figure}

Finally, we demonstrate that modifying multiple lab tests simultaneously can produce compounding effects. When we intervene on \name to reduce both glucose and WBC levels, the resulting \name-generated patients resemble healthy individuals even more closely than other T1D patients~(Appendix Figure~\ref{fig:supp_t1d}d). This finding suggests that \name can integrate multiple simultaneous edits, capturing their joint impact on a patient's future state.




\section{Conclusion}
%-------------------------------------------------------------------------------
\section{Conclusion}
%-------------------------------------------------------------------------------

This paper presents \sys, a memory offloading mechanism
for LLM serving that meets latency SLOs while maximizing the host 
memory usage. 
%
\sys captures the tradeoff between meeting SLOs and maximizing host memory usage 
with \interval, an internal tunable knob. 
%
In addition, \sys automatically decides the optimal \interval, \ie, the smallest \interval that meets SLOs, with a two-stage tuning approach.  
%
The first stage assumes bandwidth contention and profiles the GPU model offline, and generates a performance \record that, for any valid combination of SLOs, sequence lengths, and batching sizes, stores an optimal \interval,
%
The second stage adjusts the \interval for GPU instances sharing the bus to ensure that the SLOs can still be met while maximizing the aggregate host memory usage across all GPU instances. 
%
Our evaluation shows that \sys consistently maintains SLO under various runtime
scenarios, and outperforms \flexgen in throughput by 1.85\X, due to use 2.37\X more host memory. 



\clearpage
\section*{Broader Impact}

By introducing a flexible and interpretable approach to counterfactual sequence generation, \name bridges the gap between language model-style conditional generation and structured, time-sensitive sequence editing, with implications for decision support in medical and scientific applications.
%
Like all generative AI models, \name (and its derivatives) should be used solely for the benefit of society. In this study, we demonstrate that \name can generate alternative cellular trajectories and simulate the reversal or progression of symptoms to model healthier or sicker patient outcomes. However, this work (and any derivatives) should never be used to induce harmful cellular states (e.g., activating transcription factors to drive a cell toward a pathological state) or negatively impact patient care (e.g., neglecting necessary clinical interventions or recommending harmful treatments).  Our goal is to help researchers understand the underlying mechanisms of disease to improve public health. Any misuse of this work poses risks to patient well-being. Therefore, the ability to intervene on \name's generated outputs should be leveraged to assess the model's robustness and correctness for ethical and responsible use.  

\section*{Acknowledgements}

M.L.~and M.Z.~are supported by the Berkowitz Family Living Laboratory at Harvard Medical School and the Clalit Research Institute. Y.E.~is supported by grant T32 HG002295 from the National Human Genome Research Institute and the NDSEG Fellowship. We gratefully acknowledge the support of NIH R01-HD108794, NSF CAREER 2339524, US DoD FA8702-15-D-0001, ARPA-H BDF program, awards from Chan Zuckerberg Initiative, Bill \& Melinda Gates Foundation INV-079038, Amazon Faculty Research, Google Research Scholar Program, AstraZeneca Research, Roche Alliance with Distinguished Scientists, Sanofi iDEA-iTECH, Pfizer Research, John and Virginia Kaneb Fellowship at Harvard Medical School, Biswas Computational Biology Initiative in partnership with the Milken Institute, Harvard Medical School Dean's Innovation Fund for the Use of Artificial Intelligence, Harvard Data Science Initiative, and Kempner Institute for the Study of Natural and Artificial Intelligence at Harvard University. Any opinions, findings, conclusions or recommendations expressed in this material are those of the authors and do not necessarily reflect the views of the funders.


\bibliography{ref}
\bibliographystyle{conf2025}


%%%%%%%%%%%%%%%%%%%%%%%%%%%%%%%%%%%%%%%%%%%%%%%%%%%%%%%%%%%%%%%%%%%%%%%%%%%%%%%
%%%%%%%%%%%%%%%%%%%%%%%%%%%%%%%%%%%%%%%%%%%%%%%%%%%%%%%%%%%%%%%%%%%%%%%%%%%%%%%
% APPENDIX
%%%%%%%%%%%%%%%%%%%%%%%%%%%%%%%%%%%%%%%%%%%%%%%%%%%%%%%%%%%%%%%%%%%%%%%%%%%%%%%
%%%%%%%%%%%%%%%%%%%%%%%%%%%%%%%%%%%%%%%%%%%%%%%%%%%%%%%%%%%%%%%%%%%%%%%%%%%%%%%
\newpage
\appendix
\onecolumn
\subsection{Dataset}
\label{app:dataset}
In this section, we introduce the used datasets in the question answering tasks and query-based summarization tasks.

\subsubsection{Question Answering}
\label{app:qa_dataset}
In the QA tasks, we use the following four widely used datasets:
\begin{itemize}
    \item \textbf{Natural Questions (NQ)}~\cite{kwiatkowski2019natural}: The NQ dataset is a widely used benchmark for evaluating open-domain question answering systems. Introduced by Google, it consists of real user queries from Google Search with corresponding answers extracted from Wikipedia. Since it primarily contains single-hop questions, we use NQ as the representative dataset for single-hop QA. We treat NQ as a single-document QA task, where multiple questions are associated with each document. Accordingly, we build a separate RAG system for each document in the dataset.
    \item \textbf{Hotpot}~\cite{yang2018hotpotqa}: HotpotQA is a widely used multi-hop question dataset that provides 10 paragraphs per question. The dataset includes varying difficulty levels, with easier questions often solvable by LLMs. To ensure a more challenging evaluation, we randomly selected 1,000 hard bridging questions from the development set of HotpotQA. Additionally, we treat HotpotQA as a multi-document QA task and build a single RAG system to handle all questions.
    \item \textbf{MultiHop-RAG}~\cite{tang2024multihop}: MultiHop-RAG is a QA dataset designed to evaluate retrieval and reasoning across multiple documents with metadata in RAG pipelines. Constructed from English news articles, it contains 2,556 queries, with supporting evidence distributed across 2 to 4 documents. The dataset includes four query types: Inference queries, which synthesize claims about a bridge entity to identify it; Comparison queries, which compare similarities or differences and typically yield "yes" or "no" answers; Temporal queries, which examine event ordering with answers like "before" or "after"; and Null queries, where no answer can be derived from the retrieved documents. It is also a multi-document QA task.
    \item \textbf{NovelQA}~\cite{tang2024multihop}: NovelQA is a benchmark designed to evaluate the long-text understanding and retrieval ability of LLMs using manually curated questions about English novels exceeding 50,000 words. The dataset includes queries that focus on minor details or require cross-chapter reasoning, making them inherently challenging for LLMs. It covers various query types such as details, multi-hop, single-hop, character, meaning, plot, relation, setting, span, and times. Key challenges highlighted by NovelQA include grasping abstract meanings (meaning questions), understanding nuanced relationships (relation questions), and tracking temporal sequences and spatial extents (span and time questions), emphasizing the difficulty of maintaining and applying contextual information across long narratives. We use it for single-document QA task.
\end{itemize}


\subsubsection{Query-based Summarization}
\label{app:summ_dataset}
In the Query-based Summarization tasks, we adopt the following four widely used datasets:
\begin{itemize}
    \item \textbf{SQuALITY}~\cite{wang2022squality}: SQuALITY (Summary-format QUestion Answering with Long Input Texts) is a question-focused, long-document, multi-reference summarization dataset. It consists of short stories from Project Gutenberg, each ranging from 4,000 to 6,000 words. Each story is paired with five questions, and each question has four reference summaries written by Upwork writers and NYU undergraduates. SQuALITY is designed as a single-document summarization task, making it a valuable benchmark for evaluating summarization models on long-form content.
    \item \textbf{QMSum}~\cite{zhong2021qmsum}: QMSum is a human-annotated benchmark for query-based, multi-domain meeting summarization, containing 1,808 query-summary pairs from 232 meetings across multiple domains. We use QMSum as a single-document summarization task in our evaluation.
    \item \textbf{ODSum}~\cite{zhou2023odsum}: The ODSum dataset is designed to evaluate modern summarization models in multi-document contexts and consists of two subsets: ODSum-story and ODSum-meeting. ODSum-story is derived from the SQuALITY dataset, while ODSum-meeting is constructed from QMSum. We use both ODSum-story and ODSum-meeting for the multi-document summarization task in our evaluation.
\end{itemize} 

\subsection{More results on NovelQA dataset}
\label{app:novelqa}

In this section, we present the missing results for the NovelQA dataset from the main sections. These include the performance of KG-GraphRAG (Triplets) with LLaMA 3.1-8B (Table~\ref{app:tab:triple-8b}), RAG with LLaMA 3.1-70B (Table~\ref{app:tab:rag-70b}), KG-GraphRAG (Triplets) with LLaMA 3.1-70B (Table~\ref{app:tab:triple-70b}), KG-GraphRAG (Triplets+Text) with LLaMA 3.1-70B (Table~\ref{app:tab:tripletext-70b}), Community-GraphRAG (Local) with LLaMA 3.1-70B (Table~\ref{app:tab:local-70b}), and Community-GraphRAG (Global) with LLaMA 3.1-70B (Table~\ref{app:tab:global-70b}).

\begin{table}[!htb]
\caption{The performance of KG-GraphRAG (Triplets) with Llama 3.1-8B model on NovelQA dataset.}
\label{app:tab:triple-8b}
\centering
\begin{tabular}{c|cccccccc}
\hline
KG-GraphRAG(Triplet) & character & meaning & plot  & relat & settg & span & times & avg   \\ \hline
mh                   & 31.25     & 17.65   & 41.67 & 50.56 & 38.46 & 64   & 26.47 & 32.89 \\
sh                   & 35.53     & 45.71   & 30.54 & 62.5  & 27.84 & -    & -     & 33.75 \\
dtl                  & 31.43     & 24.72   & 35.71 & 17.86 & 27.03 & -    & -     & 27.37 \\
avg                  & 33.7      & 29.81   & 32.63 & 44    & 28.57 & 64   & 26.47 & 31.88 \\ \hline
\end{tabular}
\end{table}

\begin{table}[!htb]
\caption{The performance of RAG with Llama 3.1-70B model on NovelQA dataset.}
\label{app:tab:rag-70b}
\centering
\begin{tabular}{c|cccccccc}
\hline
RAG & character & meaning & plot  & relat & settg & span & times & avg   \\ \hline
mh  & 64.58     & 82.35   & 77.78 & 69.66 & 84.62 & 36   & 36.63 & 48.5  \\
sh  & 70.39     & 70      & 76.57 & 75    & 83.51 & -    & -     & 75.27 \\
dtl & 60        & 51.12   & 76.79 & 67.86 & 83.78 & -    & -     & 61.25 \\
avg & 66.67     & 58.11   & 76.74 & 69.6  & 83.67 & 36   & 36.63 & 61.42 \\ \hline
\end{tabular}
\end{table}

\begin{table}[!htb]
\caption{The performance of KG-GraphRAG (Triplets) with Llama 3.1-70B model on NovelQA dataset.}
\label{app:tab:triple-70b}
\centering
\begin{tabular}{c|cccccccc}
\hline
KG-GraphRAG (Triplets) & character & meaning & plot  & relat & settg & span & times & avg   \\ \hline
mh                     & 50        & 76.47   & 75    & 43.82 & 76.92 & 24   & 22.46 & 33.72 \\
sh                     & 52.63     & 62.86   & 55.23 & 12.5  & 50.52 & -    & -     & 54.06 \\
dtl                    & 35.71     & 26.97   & 39.29 & 53.57 & 37.84 & -    & -     & 33.6  \\
avg                    & 47.78     & 39.62   & 54.68 & 44    & 49.66 & 24   & 22.46 & 41.18 \\ \hline
\end{tabular}
\end{table}

\begin{table}[!htb]
\caption{The performance of KG-GraphRAG (Triplets+Text) with Llama 3.1-70B model on NovelQA dataset.}
\label{app:tab:tripletext-70b}
\centering
\begin{tabular}{c|cccccccc}
\hline
KG-GraphRAG (Triplets+Text) & character & meaning & plot  & relat & settg & span & times & avg   \\ \hline
mh  & 56.25     & 58.82   & 63.89 & 51.69 & 84.62 & 24   & 21.39 & 33.72 \\
sh  & 51.97     & 61.43   & 55.65 & 50    & 50.52 & -    & -     & 54.42 \\
dtl & 34.29     & 25.28   & 41.07 & 50    & 37.84 & -    & -     & 32.52 \\
avg & 48.15     & 36.98   & 54.08 & 51.2  & 50.34 & 24   & 21.39 & 41.05 \\ \hline
\end{tabular}
\end{table}

\begin{table}[!htb]
\caption{The performance of Community-GraphRAG (Local) with Llama 3.1-70B model on NovelQA dataset.}
\label{app:tab:local-70b}
\centering
\begin{tabular}{c|cccccccc}
\hline
 Community-GraphRAG (Local)   & character & meaning & plot  & relat & settg & span & times & avg   \\ \hline
mh  & 77.08     & 70.59   & 63.89 & 77.53 & 92.31 & 28   & 32.35 & 46.68 \\
sh  & 68.42     & 71.43   & 74.9  & 62.5  & 74.23 & -    & -     & 72.44 \\
dtl & 55.71     & 37.08   & 69.64 & 64.29 & 75.68 & -    & -     & 51.49 \\
avg & 66.67     & 48.3    & 72.81 & 73.6  & 76.19 & 28   & 32.35 & 57.32 \\ \hline
\end{tabular}
\end{table}

\begin{table}[!htb]
\caption{The performance of Community-GraphRAG (Global) with Llama 3.1-70B model on NovelQA dataset.}
\label{app:tab:global-70b}
\centering
\begin{tabular}{c|cccccccc}
\hline
Community-GraphRAG (Global) & character & meaning & plot  & relat & settg & span & times & avg   \\ \hline
mh     & 47.92     & 58.82   & 55.56 & 57.3  & 61.54 & 16   & 35.83 & 41.53 \\
sh     & 42.76     & 42.86   & 54.39 & 25    & 40.21 & -    & -     & 47    \\
dtl    & 24.29     & 22.47   & 32.14 & 50    & 35.14 & -    & -     & 27.64 \\
avg    & 38.89     & 30.19   & 50.76 & 53.6  & 40.82 & 16   & 35.83 & 40.21 \\ \hline
\end{tabular}
\end{table}

\subsection{RAG vs. GraphRAG Selection}
\label{app:selection}
We classify QA queries into Fact-based and Reasoning-based queries. Fact-based queries are processed using RAG, while Reasoning-based queries are handled by GraphRAG. The Query Classification prompt is shown in Figure~\ref{app:fig:query_classification}.

\begin{figure}[!htb]
% \vspace{-10 pt}
\begin{tcolorbox}[mybox={Prompt for Query Classification}]

System Prompt: Classifying Queries into Fact-Based and Reasoning-Based Categories

You are an AI model tasked with classifying queries into one of two categories based on their complexity and reasoning requirements.



\textbf{Category Definitions}

1. \textbf{Fact-Based Queries}

   - The answer can be directly retrieved from a knowledge source or requires details.  
   
   - The query does not require multi-step reasoning, inference, or cross-referencing multiple sources.  

2. \textbf{Reasoning-Based Queries}

   - The answer cannot be found in a single lookup and requires cross-referencing multiple sources, logical inference, or multi-step reasoning.  



\textbf{Examples}

\textbf{Fact-Based Queries}

\{\{ Fact-Based Queries Examples \}\}  

\textbf{Reasoning-Based Queries}

\{\{ Reasoning-Based Queries Examples \}\}  


\end{tcolorbox}
% \vspace{-10 pt}
\caption{Prompt for Query Classification.}
\label{app:fig:query_classification}
% \vspace{-10 pt}
\end{figure}

\subsection{Query-based Summarization Results with Llama3.1-70B model}
\label{app:summ_result}
In this section, we present the results for Query-based Summarization tasks using the LLaMA 3.1-70B model. The results for single-document summarization are shown in Table~\ref{app:tab:summ_single_70B}, while the results for multi-document summarization are provided in Table~\ref{app:tab:summ_multi_70B}.

\begin{table*}[!htb]
\caption{The performance of query-based single document summarization task using Llama3.1-70B.}
\label{app:tab:summ_single_70B}
% 
\resizebox{\linewidth}{!}{
\begin{tabular}{l|cccccc|cccccc}
\toprule
\multirow{4}{*}{Method}      & \multicolumn{6}{c|}{SQuALITY}                                                                           & \multicolumn{6}{c}{QMSum} \\
\cmidrule{2-13}
\multicolumn{1}{l|}{}      & \multicolumn{3}{c}{ROUGE-2}                     & \multicolumn{3}{c|}{BERTScore}                        & \multicolumn{3}{c}{ROUGE-2}                    & \multicolumn{3}{c}{BERTScore} \\ 
\cmidrule{2-13}
& P              & R             & F1             & P              & R              & F1             & P              & R             & F1            & P              & R              & F1             \\
\midrule
RAG                        & 11.85 & 14.24   & 11.00 & 85.96 & 85.76 & 85.67 & 10.42 & 10.00 & 9.53 & 86.14 & 85.92 & 86.02 \\
KG-GraphRAG(Triplets only) & 8.53  & 10.28   & 7.46  & 84.13 & 83.97 & 83.89 & 10.62 & 6.25  & 7.48 & 83.20 & 84.72 & 83.94 \\
KG-GraphRAG(Triplets+Text) & 6.57  & 10.14   & 6.00  & 80.52 & 82.23 & 81.07 & 8.64  & 7.85  & 7.29 & 84.10 & 84.55 & 84.31 \\
Community-GraphRAG(Local)  & 12.54 & 10.31   & 9.61  & 84.50 & 85.33 & 84.71 & 13.69 & 7.43  & 9.14 & 84.09 & 85.85 & 84.95 \\
Community-GraphRAG(Global) & 8.99  & 4.78    & 5.60  & 81.64 & 83.64 & 82.44 & 10.97 & 4.40  & 6.01 & 81.93 & 84.67 & 83.26 \\
Combine                    & 13.59 & 11.32   & 10.55 & 84.88 & 85.76 & 85.12 & 13.16 & 8.67  & 9.93 & 85.18 & 86.21 & 85.69 \\
\bottomrule
\end{tabular}
}
% 
\end{table*}


\begin{table*}[!htb]
\caption{The performance of query-based multiple document summarization task using Llama3.1-70B.}
\label{app:tab:summ_multi_70B}
% 
\resizebox{\linewidth}{!}{
\begin{tabular}{l|cccccc|cccccc}
\toprule
\multirow{4}{*}{Method}                   & \multicolumn{6}{c|}{ODSum-story}                                                                        & \multicolumn{6}{c}{ODSum-meeting}                                                                       \\ \cmidrule{2-13}
                   & \multicolumn{3}{c}{ROUGE-2}                    & \multicolumn{3}{c|}{BERTScore}                        & \multicolumn{3}{c}{ROUGE-2}                    & \multicolumn{3}{c}{BERTScore}                         \\ \cmidrule{2-13}
     & P              & R             & F1            & P              & R              & F1             & P              & R             & F1            & P              & R              & F1             \\
\midrule
RAG                        & 15.60 & 9.98    & 11.09 & 74.80 & 81.29 & 77.89 & 18.81 & 6.41    & 8.97  & 83.56 & 85.16 & 84.34 \\
KG-GraphRAG(Triplets only) & 10.08 & 9.12    & 8.48  & 75.71 & 81.93 & 78.66 & 11.52 & 3.41    & 4.79  & 81.19 & 83.07 & 82.11 \\
KG-GraphRAG(Triplets+Text) & 10.98 & 16.67   & 11.42 & 76.74 & 81.92 & 79.21 & 13.09 & 6.31    & 7.70  & 84.07 & 84.24 & 84.14 \\
Community-GraphRAG(Local)  & 14.20 & 11.34   & 11.25 & 75.44 & 81.81 & 78.46 & 16.17 & 7.87    & 9.23  & 84.17 & 84.85 & 84.49 \\
Community-GraphRAG(Global) & 10.46 & 6.30    & 7.08  & 74.63 & 81.24 & 77.77 & 10.65 & 1.99    & 3.28  & 79.78 & 82.53 & 81.12 \\
Combine                    & 14.76 & 12.17   & 11.72 & 75.39 & 81.75 & 78.41 & 17.57 & 8.64    & 10.34 & 84.51 & 85.14 & 84.81 \\
\bottomrule
\end{tabular}
}
% 
\end{table*}

\subsection{The LLM-as-a-Judge Prompt }
\label{app:llmjudge_prompt}
The LLM-as-a-Judge prompt can be found in Figure~\ref{app:fig:llmas}. 

\begin{figure}[!htb]
% \vspace{-10 pt}
\begin{tcolorbox}[mybox={LLM-as-a-Judge Prompt}]

You are an expert evaluator assessing the quality of responses in a query-based summarization task.\\

Below is a query, followed by two LLM-generated summarization answers. Your task is to evaluate the best answer based on the given criteria. For each aspect, select the model that performs better.

\textbf{Query}

\{\{query\}\}

\textbf{Answers Section}

\textbf{The Answer of Model 1:}

\{\{answer 1\}\}

\textbf{The Answer of Model 2:}

\{\{answer 2\}\}

\textbf{Evaluation Criteria}
Assess each LLM-generated answer independently based on the following two aspects:

1. \textbf{Comprehensiveness} \\
   - Does the answer fully address the query and include all relevant information?  \\
   - A comprehensive answer should cover all key points, ensuring that no important details are missing.  \\
   - It should present a well-rounded view, incorporating relevant context when necessary. \\ 
   - The level of detail should be sufficient to fully inform the reader without unnecessary omission or excessive brevity.  \\

2. \textbf{Global Diversity} \\
   - Does the answer provide a broad and globally inclusive perspective?  \\
   - A globally diverse response should avoid narrow or region-specific biases and instead consider multiple viewpoints. \\
   - The response should be accessible and relevant to a wide, international audience rather than assuming familiarity with specific local contexts.  
\end{tcolorbox}
% \vspace{-10 pt}
\caption{LLM-as-a-Judge Prompt.}
\label{app:fig:llmas}
% \vspace{-10 pt}
\end{figure}




\subsection{The LLM-as-a-Judge Results on more datasets}
\label{app:llmjudge_result}
In the main section, we present LLM-as-a-Judge results for the OMSum and ODSum-story datasets. Here, we provide additional results on the SQuALITY and ODSum-meeting datasets, as shown in Figure~\ref{app:fig:summ_comparision1}.
\begin{figure*}[!htb]
    \centering
    \begin{subfigure}[b]{0.24\textwidth}
        \centering
        \includegraphics[width=\textwidth]{figures/SQU_Local.pdf}
        \caption{SQuALITY Local}
        \label{fig:squ_local}
    \end{subfigure}
    \hfill
    \begin{subfigure}[b]{0.24\textwidth}
        \centering
        \includegraphics[width=\textwidth]{figures/SQU_Global.pdf}
        \caption{SQuALITY Global}
        \label{fig:squ_global}
    \end{subfigure}
    \hfill
    \begin{subfigure}[b]{0.24\textwidth}
        \centering
        \includegraphics[width=\textwidth]{figures/Meeting_Local.pdf}
        \caption{ODSum-meeting Local}
        \label{fig:meeting_local}
    \end{subfigure}
    \hfill
    \begin{subfigure}[b]{0.24\textwidth}
        \centering
        \includegraphics[width=\textwidth]{figures/Meeting_Global.pdf}
        \caption{ODSum-meeting Global}
        \label{fig:meeting_global}
    \end{subfigure}
    
    \caption{Comparison of LLM-as-a-Judge evaluations for RAG and GraphRAG. "Local" refers to the evaluation of RAG vs. GraphRAG-Local, while "Global" refers to RAG vs. GraphRAG-Global.  "Order 1" corresponds to the prompt where RAG result is presented before GraphRAG, whereas "Order 2" corresponds to the reversed order.}
    \label{app:fig:summ_comparision1}
    % \vspace{-0.2in}
\end{figure*}

%\section{You \emph{can} have an appendix here.}

%You can have as much text here as you want. The main body must be at most $8$ pages long.
%For the final version, one more page can be added.
%If you want, you can use an appendix like this one.  

%The $\mathtt{\backslash onecolumn}$ command above can be kept in place if you prefer a one-column appendix, or can be removed if you prefer a two-column appendix.  Apart from this possible change, the style (font size, spacing, margins, page numbering, etc.) should be kept the same as the main body.
%%%%%%%%%%%%%%%%%%%%%%%%%%%%%%%%%%%%%%%%%%%%%%%%%%%%%%%%%%%%%%%%%%%%%%%%%%%%%%%
%%%%%%%%%%%%%%%%%%%%%%%%%%%%%%%%%%%%%%%%%%%%%%%%%%%%%%%%%%%%%%%%%%%%%%%%%%%%%%%


\end{document}

