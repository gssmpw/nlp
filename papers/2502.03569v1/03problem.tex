\begin{figure*}[ht]
\begin{center}
\centerline{\includegraphics[width=0.95\textwidth]{FIG/figure2.png}}
\caption{Overview of \name's architecture and capabilities. \textbf{(a)}~Given an input sequence, forecast time, and condition embedding from a frozen pretrained~(\textsc{pt}) embedding model, \name generates a counterfactual sequence via immediate or delayed sequence editing. \textbf{(b)}~\name is composed of a sequence encoder, condition adapter, concept encoder, and concept decoder. \name has two key capabilities: \textbf{(c)}~forecasting counterfactual sequences at any time in the future and under any condition~(e.g.,~medical codes), and \textbf{(d)}~generating counterfactual sequences by intervening on \name's learned temporal concepts.}
\label{fig:clef}
\end{center}
\vskip -0.35in
\end{figure*}



\begin{definition}[Sequence editing]
\label{def:cse}
Sequence editing is the local sample-level modification of sequence $\mathbf{x}$ to generate a counterfactual sequence~$\hat{\mathbf{x}}_{:,t_j}$ under a given condition~$s$ at a specific time~$t_j$. There are two types of controllable sequence editing: immediate and delayed~(Figure~\ref{fig:clef}a).
\begin{itemize}[leftmargin=*,noitemsep,topsep=0pt]
    \item \textbf{Immediate sequence editing:} Given sequence $\mathbf{x}_{:,t_0:t_i}$ and condition $s$ to occur at time $t_{i + 1}$, forecast $\hat{\mathbf{x}}_{:,t_{i + 1}}$.
    \item \textbf{Delayed sequence editing:} Given sequence $\mathbf{x}_{:,t_0:t_i}$ and condition $s$ to occur at time $t_j \geq t_{i + 1}$, forecast $\hat{\mathbf{x}}_{:,t_j}$.
\end{itemize}
\end{definition}
%
Example scenarios for immediate sequence editing include: \textit{What if we perturb the cells now?} and \textit{What if we perform surgery on the patient today?} (Section~\ref{results:r1}). In contrast, delayed sequence editing applies to questions such as: \textit{What if we perturb the cells in ten days?} and \textit{What if we perform surgery on the patient next year?} (Section~\ref{results:r2}).

\begin{definition}[Temporal concept]\label{def:concept}
A temporal concept $\mathbf{c}$ for sequence $\mathbf{x}_{:,t_j:t_k}$ is defined by $\displaystyle \mathbf{c} = \sfrac{\mathbf{x}_{:,t_k}}{\mathbf{x}_{:,t_j}}$ for time steps $t_j$ and $t_k$ where $t_k > t_j$. It can be interpreted as the trajectory (or rate of change of each variable in the sequence) between any pair of time steps.
\end{definition}


\begin{definition}[Controllable sequence editing]
\label{def:concept_cse}
Concept encoder $E$ and decoder $G$ are able to leverage temporal concepts $\mathbf{c}$ to perform controllable sequence editing on dataset $\mathcal{D}$ if the following are satisfied.
\begin{itemize}[leftmargin=*,noitemsep,topsep=0pt]
    \item
    Condition $s$ on $\mathbf{x}_{:,t_0:t_i}$ at time step~$t_j$ learns $\mathbf{c}$ that accurately forecasts $\hat{\mathbf{x}}^s_{:,t_j}$ such that $\hat{\mathbf{x}}^s_{:,t_j} \simeq \mathbf{x}^s_{:,t_j}$.
    \item 
    Counterfactual condition $a \neq s$ on $\mathbf{x}_{:,t_0:t_i}$ at $t_j$ learns $\mathbf{c}' \neq \mathbf{c}$ that forecasts $\hat{\mathbf{x}}^a_{:,t_j}$ such that $\hat{\mathbf{x}}^a_{:,t_j} \neq \hat{\mathbf{x}}^s_{:,t_j}$ and, if known, $\hat{\mathbf{x}}^a_{:,t_j} \simeq \mathbf{x}^a_{:,t_j}$.
    \end{itemize}
\end{definition}



\begin{problemst}[\name]
\label{prob}
Given a sequence encoder~$F$, condition adapter~$H$, concept encoder~$E$, and concept decoder~$G$ trained on a longitudinal dataset $\mathcal{D}$, \name learns temporal concept $\mathbf{c} = E(F(\mathbf{x}_{:,t_0:t_i}, t_j), H(s))$ to forecast $\hat{\mathbf{x}}^s_{:,t_j} = G(\mathbf{x}_{:,t_i}, \mathbf{c})$ for any sequence $\mathbf{x}_{:,t_0:t_i} \in \mathcal{D}$, future time step $t_j > t_i$, and condition $s$.
\end{problemst}

