
Sequence models generate counterfactuals by modifying parts of a sequence based on a given condition, enabling reasoning about ``what if" scenarios. While these models excel at conditional generation, they lack fine-grained control over when and where edits occur. Existing approaches either focus on univariate sequences or assume that interventions affect the entire sequence globally. However, many applications require precise, localized modifications, where interventions take effect only after a specified time and impact only a subset of co-occurring variables. We introduce \name, a controllable sequence editing model for counterfactual reasoning about both immediate and delayed effects. \name learns temporal concepts that encode how and when interventions should influence a sequence. With these concepts, \name selectively edits relevant time steps while preserving unaffected portions of the sequence. We evaluate \name on cellular and patient trajectory datasets, where gene regulation affects only certain genes at specific time steps, or medical interventions alter only a subset of lab measurements. \name improves immediate sequence editing by up to $36.01\%$ in MAE compared to baselines. Unlike prior methods, \name enables one-step generation of counterfactual sequences at any future time step, outperforming baselines by up to $65.71\%$ in MAE. A case study on patients with type 1 diabetes mellitus shows that \name identifies clinical interventions that shift patient trajectories toward healthier outcomes.


\vspace{-0.7em}