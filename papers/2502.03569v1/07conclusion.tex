
In this work, we formalize controllable sequence editing for counterfactual generation on biomedical sequences, and demonstrate that \name outperforms state-of-the-art counterfactual sequence generation models in immediate and delayed sequence editing. \name also has stronger generalizability to new sequences, and performs significantly better than state-of-the-art models in zero-shot counterfactual generation. Further, we show that interventions directly on \name's temporal concepts can generate counterfactual patients such that their trajectories are shifted toward healthier outcomes. This capability has the potential to help discover clinical interventions that could alleviate a patient's symptoms. While this work focuses on cellular and patient trajectories, \name can be readily extended to perform sequence editing in other domains.



\xhdr{Limitations}
%
There are two key limitations of \name. Firstly, we define temporal concepts such that each element represents a unique measured variable in the sequence (e.g.,~gene expression, lab test). Instead, it may be beneficial to learn higher-order relationships between the measured variables or across time as abstract hierarchical concepts~\cite{the2024large, kacprzyk2024towards}. Secondly, while \name is able to generate counterfactual sequences for any condition, including those it may not have seen during training, \name could potentially improve with additional guidance from a real-world causal model for the system or domain of interest~\cite{chatzi2024counterfactual}. Since defining such a real-world causal graph is a major challenge, one promising future direction could be to enable user interventions, such as those performed in our T1D case studies, to finetune \name.