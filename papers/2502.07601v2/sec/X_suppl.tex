\clearpage
%\setcounter{page}{1}
%\maketitlesupplementary
\setcounter{section}{0}
\renewcommand\thesection{A\arabic{section}}
%\part*{A. Appendix}

\section{Dataset Establishment}
\label{sup_dataset}

\subsection{How to highlight the anomaly?}
\begin{table}[ht]
  \begin{minipage}{0.99\linewidth}
\centering
\scalebox{0.80}{
\begin{tabular}{l p{7.5cm} }
\toprule
 \multicolumn{2}{l}{\bf Prompt Example: Bubble Inside the Capsule}  \\
\midrule
&  \includegraphics[height=2.2cm]{figures/capsule.pdf} \\

User & Do you observe any anomalies in the image? \\
\midrule
GPT-4o & The image shows green gel capsules that appear uniform in size, shape, and color. Upon inspection, \textcolor{customgreen}{no obvious anomalies such as broken capsules, irregular shapes, or unexpected colors are visible.} \\ \midrule
 \multicolumn{2}{c}{Let's draw a bounding box around the anomaly!} \\ \midrule
&  \includegraphics[height=2.2cm]{figures/capsule2.pdf} \\

User & Do you observe any anomalies in the image? \\
\midrule
GPT-4o & Yes, there appears to be an anomaly in the image. One of the capsules, marked with a red box, has a noticeable defect. \textcolor{customblue}{It seems to have a bubble or an air pocket inside, which is not observed in the other capsules.} This could indicate a manufacturing defect or inconsistency in the capsule filling process. \\\midrule
 \multicolumn{2}{c}{If there is no bounding box, let's give a short hint to GPT-4o!} \\ \midrule
User & Do you observe any anomalies in the image? (hint: a bubble in the capsule) \\
\midrule
GPT-4o & The anomaly in the image is the presence of a visible air bubble inside one of the green gel capsules. \textcolor{customblue}{While most capsules appear uniformly filled with liquid, the capsule \textbf{located near the top center of the image} has a small, clear air pocket within it.} This could be a result of incomplete filling during the manufacturing process or an issue with the capsule sealing. \\
\bottomrule
\end{tabular}
}
\captionof{table}{Comparison of the GPT-4o \cite{gpt-api-4o} outputs with and without visual and textual hints for the anomaly.}
\label{tab:sup1}  
  \end{minipage}
  \vspace{-2mm}
\end{table}
\noindent
As shown in Table \ref{tab:sup1}, recent advanced MLLMs like GPT-4o fail to detect the anomalies in the image, so building the instruction tuning dataset using previous methods \cite{sharegpt4v} is impractical. However, we observe that when the GPT-4o is provided some "hints", it presents impressive performance on anomaly reasoning or description. For example, a red bounding box drawn around the anomalous area enables GPT-4o to detect the tiny bubble inside the small capsule. This observation indicates that \textcolor{lightpink}{\textbf{the anomaly information is already contained in the visual tokens, and the failure of existing MLLMs is because the language model cannot effectively pick out the related tokens,}} which is the major inspiration of our token-picking mechanism.


Most of the existing AD datasets, such as MVTec AD \cite{mvtec}, contain anomaly masks for anomaly localization. Therefore, we leverage these masks to generate the bounding boxes on the images. Specifically, the masks for an anomalous image are dilated and merged (if two masks are too close) before calculating the coordinates of the bounding boxes. Similarly, the image with bounding boxes drawn on it will serve as the visual prompt for GPT-4o. We also tried many other ways to utilize the anomaly masks, such as highlighting the mask area with different colors, consecutively providing the image and mask, and converting the normalized coordinates of the bounding box into a text prompt. None of them can as effectively guide the GPT-4o in finding anomalous features as drawing bounding boxes on the image.




\begin{figure*}[!t]
\centering
    \includegraphics[width=\textwidth]{figures/data_collection.pdf}
\caption{Automatic data collection pipeline for WebAD. The entire pipeline is fully automatic at an affordable cost (API usage). Other advanced open-sourced MLLMs can applied to replace GPT-4o for further reduction of cost.}
\label{fig:data_collect}
\end{figure*}

\subsection{WebAD -- The largest AD dataset}
Existing industrial or medical anomaly detection datasets, such as MVTec AD \cite{mvtec} and BMAD \cite{bmad}, only contain a limited number of classes ($<20$) and several different anomaly types for each class (most of the anomaly types are similar) due to the collection of these kinds of anomaly images involves extensive human involvements. This limitation hinders the ZSAD model from learning a generic description of anomaly and normal patterns. Also, the MLLMs cannot obtain enough knowledge of visual anomaly descriptions for unseen anomaly types. Therefore, more diverse data is required for a robust ZSAD \& reasoning model. Many recent dataset works collect and annotate online images to enrich existing datasets and demonstrate their effectiveness in the training of current data-hungry deep learning models. 

To collect the online images that can be utilized for anomaly detection, we design an automatic data collection pipeline by combining GPT-4o \cite{gpt-api-4o} and Google Image Search \cite{google-image-search}. As shown in Figure \ref{fig:data_collect}, we first employ GPT-4o to list 400 class names commonly seen in our daily life. Then, for each class, the GPT-4o is asked to generate 10 corresponding anomalous and normal phrases based on the class name. The abnormality or normality descriptions indicated by these phrases are specifically suitable for the class name. These phrases will serve as the search prompts to query the image links in Google Image Search. However, the downloaded images are very "dirty" and contain many noise samples and duplications. For example, the collected anomaly set contains lots of normal images, and vice versa. A data-cleaning step is applied after the image collection.

Since the duplications mainly occur within a specific class, we extract the CLIP \cite{clip} features for all the images in the class and compare the cosine similarity of these features. If the similarity value is larger than $0.99$, then one of the images will be removed. To deal with the problematic grouping of anomaly and normal images, we combine the image and its corresponding search prompt and give them to GPT-4o for normal and anomaly classification. In the system prompt, we explicitly tell the GPT-4o that the search prompt is just a hint and not always correct and ask GPT-4o to determine the normality and abnormality by itself. This step will remove the images with incorrect labels and the artificial images, such as cartons or art. Some samples in the collected WebAD dataset are shown in Figure \ref{fig:gallery}. In total, WebAD contains around 72k images from 380 classes and more than 5 anomaly types for each class.



\begin{figure*}[!t]
\centering
    \includegraphics[width=\textwidth]{figures/gallery.pdf}
\caption{Overview of the gallery for in-the-wild image samples in WebAD. The images on the left side are anomalous, while the right side is for normal images. The links to download these images will be released to avoid copyright issues.}
\label{fig:gallery}
\end{figure*}

\subsection{Instruction Data Generation}
For existing datasets, we manually combine the anomaly type and the class name to create the short anomaly prompt (hint). Then, the image with or without the bounding boxes and the corresponding short prompt are utilized to prompt GPT-4o for the generation of detailed descriptions of the image and the anomalies. These descriptions contain all the information required for instruction-following data. The in-context learning strategy is implemented to generate the multi-round conversation data (see Figure \ref{fig:in_context}). Questions designed to elicit a one-word answer are utilized to balance the distribution of the normal and anomaly samples.

\begin{figure*}[!t]
\centering
    \includegraphics[width=\textwidth]{figures/in_context.pdf}
\caption{Prompt template for generating multi-round conversation in Anomaly-Instruct-125k (modified from the template of LLaVA \cite{llava}).}
\label{fig:in_context}
\vspace{-3mm}
\end{figure*}

\section{Training Details}
In the professional training stage, we leverage AdamW \cite{adamw} to be the optimizer and CosineAnnealingWarmRestarts \cite{loshchilov2017sgdr} as the learning rate scheduler. The initial learning rate is set to be $1e-4$, and the restart iteration is half of the single epoch. The anomaly expert is trained on 8 H100 GPUs for 2 epochs (2 hours), and the total batch size is 128. In the instruction tuning stage, we follow the default training setting of \textit{LLaVA-OneVision} \cite{llavaonevision} (reduce the batch size to 128), and the total training time for 0.5B and 7B models are 7 hours and 50 hours on 8 H100, respectively. When sampling the instruction data from the original recipe of \textit{LLaVA-OneVision}, we put more emphasis on low-level image understanding and 3D multi-view Q\&A, considering that anomaly detection originates from the low-level feature differences and the 3D anomaly detection requires multi-image understanding. Besides, for more knowledge in the medical domain, the model is also fed with the data from LLaVA-Med \cite{llavamed}.




\section{Experimental Results}

\subsection{Anomaly Detection}
\begin{table*}
\small
\centering
\begin{tabular}{l|ccccccccc}
\toprule
VisA     & capsules & fryum      & cashew & macaroni1 & macaroni2 & candle     & pipe fryum & chewinggum & pcb1     \\ 
AUROC    & 98.6     & 93.4       & 93.9   & 91.1      & 70.3      & 96.5       & 99.0       & 96.4       & 90.3     \\
         & pcb2     & pcb3       & pcb4   &           &           &            &            &            &           \\
         & 87.7     & 86.6       & 90.0   &           &           &            &            &            &           \\
\hline
MVTec AD & bottle   & cable      & grid   & capsule   & tile      & pill       & screw      & transistor & hazelnut \\
AUROC    & 95.6     & 88.8       & 99.2   & 92.2      & 95.5      & 92.3       & 98.0       & 88.0       & 91.2     \\
         & carpet   & metal nut  & wood   & zipper    & leather   & toothbrush &            &            &          \\
         & 98.9     & 77.7       & 100.0  & 99.6      & 100.0     & 94.2       &            &            &          \\
\bottomrule
\end{tabular}
\caption{Per-class image-level AUROC of the anomaly expert of Anomaly-OV on VisA and MVTec AD.}
\label{Tab:mvtec_visa}
\vspace{-3mm}
\end{table*}
Similar to previous ZSAD works, the detailed image-level AUROC results for the anomaly expert of Anomaly-OV on VisA \cite{visa} and MVTec AD \cite{mvtec} are provided in Table \ref{Tab:mvtec_visa}.

\subsection{Anomaly Reasoning}
\begin{table}
  \begin{minipage}{0.99\linewidth}
\centering
\scalebox{0.80}{
\begin{tabular}{l p{7.5cm} }
\toprule
 \multicolumn{2}{l}{\bf PCB Example: Bent Transparent LED }  \\
\midrule
&  \includegraphics[height=2.5cm]{figures/pcb3.png} \\

User & Is there an anomaly in the image? \\
\midrule
GPT-4o & The image shows an infrared sensor module, which appears to be intact and properly assembled. There doesn't seem to be any visible anomaly or damage to the components. \\ \midrule
LLaVA-OV & The image shows an electronic circuit board with various components, including a red LED that is not illuminated. There doesn't appear to be anything obviously anomalous in the image. \\
\midrule
Ours & Yes, the anomaly in the image is \textcolor{customblue}{the bent LED on the circuit board, which is not aligned properly with the board's surface.} \\
\bottomrule
\end{tabular}
}
\captionof{table}{Additional results on VisA-D\&R (PCB).}
\label{tab:pcb1}  
  \end{minipage}
  % \vspace{-2mm}
\end{table}
\begin{table}
  \begin{minipage}{0.99\linewidth}
\centering
\scalebox{0.80}{
\begin{tabular}{l p{7.5cm} }
\toprule
 \multicolumn{2}{l}{\bf Candle Example: Pink Discoloration}  \\
\midrule
&  \includegraphics[height=2.5cm]{figures/candle2.pdf} \\

User & Is there an anomaly in the image I should know about? \\
\midrule
GPT-4o & Yes, there is an anomaly in the image. \textcolor{customgreen}{The candle in the top left corner has its wick facing the opposite direction compared to the other three candles.} \\ \midrule
LLaVA-OV & Based on the image, there is no apparent anomaly among the candles. They are all similar in size and shape, with a consistent hole in the center for the wick. The coloration appears uniform across the candles, and there are no visible signs of melting or irregularities that would suggest an anomaly. \\
\midrule
Ours & Yes, the anomaly is \textcolor{customblue}{the presence of a small red spot on the bottom right candle, which contrasts with the otherwise uniform white color.} \\
\bottomrule
\end{tabular}
}
\captionof{table}{Additional results on VisA-D\&R (Candle).}
\label{tab:candle}  
  \end{minipage}
  % \vspace{-2mm}
\end{table}
\begin{table}[t]
  \begin{minipage}{0.99\linewidth}
\centering
\scalebox{0.80}{
\begin{tabular}{l p{7.5cm} }
\toprule
 \multicolumn{2}{l}{\bf Capsule Example: Dark Spot }  \\
\midrule
&  \includegraphics[height=2.5cm]{figures/capsule3.pdf} \\

User & Is there an anomaly present in the image? \\
\midrule
GPT-4o & Yes, there is an anomaly in the image. \textcolor{customgreen}{One of the capsules is a different color; it is brown instead of green.} \\ \midrule
LLaVA-OV & Based on the image alone, there is no clear indication of an anomaly. The pills appear to be uniformly shaped and colored, with no visible defects or irregularities. However, without additional context or a point of reference for size, it's difficult to determine if there are any anomalies that might not be immediately apparent from the image.  \\
\midrule
Ours & Yes, there is an anomaly present in the image. \textcolor{customblue}{One of the capsules has a darker spot on its surface, which contrasts with the otherwise uniform green color.} \\
\bottomrule
\end{tabular}
}
\captionof{table}{Additional results on VisA-D\&R (Capsules).}
\label{tab:capsule}  
  \end{minipage}
  \vspace{-1mm}
\end{table}
\begin{table}[ht]
  \begin{minipage}{0.99\linewidth}
\centering
\scalebox{0.80}{
\begin{tabular}{l p{7.5cm} }
\toprule
 \multicolumn{2}{l}{\bf Fryum Example: Break Edge }  \\
\midrule
&  \includegraphics[height=2.5cm]{figures/fryum.pdf} \\

User & Is there any anomaly in the image? \\
\midrule
GPT-4o & The image shows a wheel-shaped snack on a green surface. There doesn't appear to be any obvious anomaly. \\ \midrule
LLaVA-OV & The image shows a circular object with a pattern that resembles a wheel or a gear, placed on a green textured surface. There doesn't appear to be any obvious anomaly in the image; it seems to be a simple photograph of an object against a background. If there is something specific you are referring to as an anomaly, please provide more details or clarify what aspect you are questioning.  \\
\midrule
Ours & Yes, there is an anomaly in the image. The pasta wheel has \textcolor{customblue}{a noticeable crack on its lower right side, disrupting its otherwise smooth and continuous surface.} \\
\bottomrule
\end{tabular}
}
\captionof{table}{Additional results on VisA-D\&R (Fryum).}
\label{tab:fryum}  
  \end{minipage}
  % \vspace{-2mm}
\end{table}
\begin{table}
  \begin{minipage}{0.99\linewidth}
\centering
\scalebox{0.80}{
\begin{tabular}{l p{7.5cm} }
\toprule
 \multicolumn{2}{l}{\bf Cashew Example: Brown Spot }  \\
\midrule
&  \includegraphics[height=2.5cm]{figures/cashew.pdf} \\

User & Does the image contain any anomaly? \\
\midrule
GPT-4o & The image shows a cashew nut on a textured black surface. There are no visible anomalies; the cashew appears normal. \\ \midrule
LLaVA-OV & Based on the image provided, there are no visible anomalies. The object appears to be a single, intact cashew nut with a smooth surface and a natural brown coloration that is typical for cashews. There are no signs of damage, such as cracks or holes, and the background does not show any irregularities that would suggest an anomaly in the context of the image.  \\
\midrule
Ours & Yes, there is a noticeable anomaly on the cashew nut: \textcolor{customblue}{a small brown spot located near the center of the nut.} \\
\bottomrule
\end{tabular}
}
\captionof{table}{Additional results on VisA-D\&R (Cashew).}
\label{tab:cashew}  
  \end{minipage}
  % \vspace{-2mm}
\end{table}
\begin{table}
  \begin{minipage}{0.99\linewidth}
\centering
\scalebox{0.80}{
\begin{tabular}{l p{7.5cm} }
\toprule
 \multicolumn{2}{l}{\bf In-the-Wild Example: Graffitied Road Sign }  \\
\midrule
&  \includegraphics[height=3.5cm]{figures/road_sign.png} \\

User & Describe the anomaly in the image. \\
\midrule
Anomaly-OV & The anomaly in the image is the graffiti on the highway sign, which includes black spray paint marks and scribbles over the word 'Seattle' and other parts of the sign. \\ 
\midrule
User & What is the reason for the anomaly?  \\
\midrule
Anomaly-OV & The graffiti is likely the result of vandalism, where individuals have used spray paint to deface public property for personal expression or as a form of protest. \\
\midrule
User & How does the anomaly affect the functionality of the object?  \\
\midrule
Anomaly-OV & The graffiti affects the sign's functionality by potentially obstructing the visibility of important information, such as the word 'Seattle,' which could confuse drivers and lead to navigational errors. \\
\midrule
User & How to avoid the anomaly in the future?  \\
\midrule
Anomaly-OV & To avoid such anomalies, regular maintenance and cleaning of signs should be implemented. Additionally, community awareness programs can help reduce vandalism by promoting respect for public property. \\
\bottomrule
\end{tabular}
}
\captionof{table}{In-the-wild results for an unseen object (Road Sign).}
\label{tab:road_sign}  
  \end{minipage}
  % \vspace{-2mm}
\end{table}
Table \ref{tab:pcb1} to \ref{tab:cashew} presents more comparison results of GPT-4o \cite{gpt-api-4o}, \textit{LLaVA-OneVision} \cite{llavaonevision}, and Anomaly-OV on AD \& reasoning. Anomaly-OV shows better performance in the detection and description of the visual anomalies in the images. Table \ref{tab:road_sign} demonstrates the low-level and complex reasoning capability of Anomaly-OV for an in-the-wild image, indicating a comprehensive understanding of the anomaly.


\section{Limitation and Future Work}
\begin{table}
  \begin{minipage}{0.99\linewidth}
\centering
\scalebox{0.80}{
\begin{tabular}{l p{7.5cm} }
\toprule


\multicolumn{2}{l}{\bf Failure Example 1: Wrong Classification}  \\
\midrule
&  \includegraphics[height=2.5cm]{figures/failure2.pdf} \\
User & Is there any anomaly visible in the image? \\
\midrule
Anomaly-OV & Yes, there is a noticeable anomaly on the \textcolor{customgreen}{soap bar}: a hole with jagged edges located towards the left side. \\
\midrule
 \multicolumn{2}{l}{\bf Failure Example 2: General Description }  \\
\midrule
&  \includegraphics[height=2.5cm]{figures/failure3.pdf} \\

User & Are there any anomalies present in the image with the set of candles? \\
\midrule
Anomaly-OV & Yes, there is an anomaly present in the top right candle. It has a noticeable \textcolor{customgreen}{crack} on its surface, which is not typical for a smooth, intact candle. \\
\midrule
 \multicolumn{2}{l}{\bf Failure Example 3: Wrong Reasoning }  \\
\midrule
&  \includegraphics[height=2.5cm]{figures/failure1.pdf} \\

User & Are there any anomalies in the image? \\
\midrule
Anomaly-OV & Yes, the anomaly in the image is \textcolor{customgreen}{the absence of a crystal oscillator on the HC-SR04 ultrasonic sensor module.} \\

\bottomrule
\end{tabular}
}
\captionof{table}{Failure results of Anomaly-OV on VisA-D\&R.}
\label{tab:failure}  
  \end{minipage}
  % \vspace{-2mm}
\end{table}
\noindent
\textbf{Limitation.} As shown in Table \ref{tab:failure}, sometimes, Anomaly-OV fails to provide an accurate classification of the target object, describes the anomaly by a general word (wax missing is described by "crack"), or presents wrong reasoning with hallucination. Also, there is still a large space for improvement in the detection performance of Anomaly-OV. Besides, the images contained in VisA-D\&R  are from the industrial domain, so more benchmarks in other domains, such as 3D and medical anomaly detection, are required to evaluate a unified AD \& reasoning model.

\medskip

\noindent
\textbf{Future Work.} The detection performance of Anomaly-OV is highly determined by the anomaly expert (see Table \ref{Tab:3}), so a more advanced design of the expert model is recommended in future research. One can change the base model to other open-sourced MLLMs to resolve the wrong classification issue. Also, we found that the diversity of the anomaly type is very limited in existing industrial anomaly datasets (mainly 'crack' or 'broken'), causing the assistant to fail to provide fine-grained anomaly reasoning or description for unseen anomaly features. Therefore, a more diverse industrial anomaly detection dataset is urgently required. Similar to other traditional MLLMs, Anomaly-OV only utilizes the output visual tokens from the last layer of the visual encoder as the input for LLM. However, anomaly detection is highly dependent on low-level visual clues. Hence, \textbf{forwarding multi-level features from different layers to the LLM} (as recent paper: "Dense Connector for MLLMs" \cite{yao2024denseconnectormllms} ) should be a possible solution for performance improvement.


