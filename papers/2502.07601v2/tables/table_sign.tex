\begin{table}
  \begin{minipage}{0.99\linewidth}
\centering
\scalebox{0.80}{
\begin{tabular}{l p{7.5cm} }
\toprule
 \multicolumn{2}{l}{\bf In-the-Wild Example: Graffitied Road Sign }  \\
\midrule
&  \includegraphics[height=3.5cm]{figures/road_sign.png} \\

User & Describe the anomaly in the image. \\
\midrule
Anomaly-OV & The anomaly in the image is the graffiti on the highway sign, which includes black spray paint marks and scribbles over the word 'Seattle' and other parts of the sign. \\ 
\midrule
User & What is the reason for the anomaly?  \\
\midrule
Anomaly-OV & The graffiti is likely the result of vandalism, where individuals have used spray paint to deface public property for personal expression or as a form of protest. \\
\midrule
User & How does the anomaly affect the functionality of the object?  \\
\midrule
Anomaly-OV & The graffiti affects the sign's functionality by potentially obstructing the visibility of important information, such as the word 'Seattle,' which could confuse drivers and lead to navigational errors. \\
\midrule
User & How to avoid the anomaly in the future?  \\
\midrule
Anomaly-OV & To avoid such anomalies, regular maintenance and cleaning of signs should be implemented. Additionally, community awareness programs can help reduce vandalism by promoting respect for public property. \\
\bottomrule
\end{tabular}
}
\captionof{table}{In-the-wild results for an unseen object (Road Sign).}
\label{tab:road_sign}  
  \end{minipage}
  % \vspace{-2mm}
\end{table}