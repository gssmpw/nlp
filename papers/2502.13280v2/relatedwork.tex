\section{Related Work}
SDE-based samplers, such as PIS \cite{zhang2022path}, DIS \cite{berner2022optimal}, DDS \cite{vargas2023denoising}, and \citet{richter2024improved}, inherently possess rich interpretation to the optimal control problem. Although discussed theoretically, the connection to optimal control is rarely exploited directly. 

% normalizing flows
% FAB \cite{midgley2023flow} uses normalizing flow with AIS to train a sampler.
% DiKL \cite{he2024training} trains an implicit generator by matching the spread divergence between the model and the target.

Sequentially drawing intermediate samples from a series of distributions that interpolate between the initial and target distributions is a recurring strategy in sampling. Classical sampling methods, such as SMC \cite{moral2006smc} and AIS \cite{neal2001annealed}, generate intermediate samples from tempered distributions.
More recent works explore the use of diffused target densities as intermediate distributions. However, a key challenge in this setting is that the score of diffused densities cannot be estimated using the same techniques in diffusion modeling \cite{sohl-dickstein15,ho2020ddpm,song2021scorebased}. To address this, various approaches have been proposed, leveraging different mathematical formulations \cite{akhound-sadegh2024iterated,phillips2024particle,huang2024reverse,mcdonald2022proposal,wang2024energy,chen2024sequential}. VGS addresses the same challenge but with the toolbox of RL, as the value function is theoretically linked to the log of the diffused target density.