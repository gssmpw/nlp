\section{Background and Related Work}
\label{sec:rw}

Various activities have been proposed to aid practitioners in effectively managing TD____. Li et al.____ summarized nine critical activities described in the literature. During \textbf{identification}, TD items are detected using techniques such as manual inspection or static code analysis. These items can be documented during the \textbf{representation and documentation} activities, and stakeholders are informed about TD items during the \textbf{communication} activity. TD items are monitored in the \textbf{monitoring} activity, ensuring that unresolved TD items are under control. The \textbf{measurement} activity quantifies the amount of TD in a system, enabling the \textbf{prioritization} activity, where TD items are ranked for resolution. These items can then be addressed during the \textbf{repayment} activity, which also deals with the issues caused by accumulated TD. Additionally, the \textbf{prevention} activity seeks to avoid unwanted TD from arising.

Previous work has investigated such activities in different contexts, including how they could be automated to make TDM more efficient and how these activities and TD in general are discussed on the Stack Exchange network____. Such studies are closely related to ours and are also discussed in this section. First, in an observational study on the usage of the term ``technical debt" on the Stack Exchange network, Alfayez et al.____ found 578 TD-related questions, which were categorized into 14 different categories, such as TD tools, TD repayment, TD representation, and TD definitions. The study identified 636 unique tags for TD-related questions, with ``SonarQube'' being the most used tag, followed by ``technical debt'', ``java'', ``agile'', and ``scrum'' among the top-10 most used tags. Furthermore, the findings showed that the most challenging questions for users. Furthermore, the findings showed that the most challenging questions for users were those classified under three categories---TD consequences, incurring TD, and TD tools---and that most questions were related to what to do with TD tools..

Kozanidis et al.____ conducted a study to understand how users request support with respect to TD. After reviewing and analyzing 415 questions from the Stack Exchange websites, the authors found that TD in architecture, code, and design are the most referenced TD types on Stack Overflow. Moreover, predictive models can accurately detect and classify TD questions and their urgency (urgent or not urgent), but fail to identify TD types mentioned in the question. Finally, they found that most questions present some degree of urgency, while TD repayment and TD management are the most recurrent themes.

Other research studies focused on more specific contexts (e.g., agile or security) to analyze  questions about TD. One such study by Santos et al.____ compiled and analyzed 79 TD discussions on agile software development (ASD) from Stack Exchange Project Management. They pointed out eight types of TD in the context of ASD and identified 51 indicators of ASD-related TD, for instance, poorly written code, design problems, and bug occurrence. They found that the most commonly discussed TD types are process and people debt and that product owner and development team are the most important roles concerning ASD-related TD. Another study, which researched the scope of TD in security questions found on Stack Overflow, identified that 38\% of the analyzed 117.233 questions were security-related TD questions____.

An empirical study by Peruma et al.____ found that Code Optimization, Architecture and Design Patterns, Unit Testing, Tools and IDEs, and Database are the top-five topics most associated with discussions about refactoring, a common practice for TD repayment.

A study by Gama et al.____ investigated how developers commonly identify TD items in their projects. They found that Stack Overflow users commonly discuss the identification of TD, revealing 29 different low-level indicators to recognize TD items in code, infrastructure, architecture, and tests. They grouped low-level indicators based on their themes, producing a set of 13 high-level indicators, such as the presence of bad coding and the lack of good design practices. In addition, they classified all low- and high-level indicators into three different categories (Development Issues, Infrastructure, and Methodology) according to the type of debt that each of them was intended to identify.

The aforementioned studies show that practitioners often use Stack Exchange forums to discuss TD and TDM. However, these studies, at best, focus on listing the tools mentioned in these discussions without exploring them in depth.
Our study investigates the challenges of automating TDM in practice through existing tools or desired TDM features. Our findings contribute to TDM research and practice by exploring limitations in current tools and offering insights for researchers and tool developers to improve or create more effective solutions. Together, these contributions can lead to more efficient TDM processes and reduce the time required to manage TD.