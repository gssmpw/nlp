


%%%%%%%%%%%%%%%%%%%%%%%%%%%%%%%%%%%%%%%%%%%%%%%%%%%%%%%%%%%%%%%%%%%%%%%%%%%%%%%%
%2345678901234567890123456789012345678901234567890123456789012345678901234567890
%        1         2         3         4         5         6         7         8

\documentclass[letterpaper, 10 pt, conference]{ieeeconf}  % Comment this line out if you need a4paper

%\documentclass[a4paper, 10pt, conference]{ieeeconf}      % Use this line for a4 paper

\IEEEoverridecommandlockouts                              % This command is only needed if 
                                                          % you want to use the \thanks command

\overrideIEEEmargins                                      % Needed to meet printer requirements.

%In case you encounter the following error:
%Error 1010 The PDF file may be corrupt (unable to open PDF file) OR
%Error 1000 An error occurred while parsing a contents stream. Unable to analyze the PDF file.
%This is a known problem with pdfLaTeX conversion filter. The file cannot be opened with acrobat reader
%Please use one of the alternatives below to circumvent this error by uncommenting one or the other
%\pdfobjcompresslevel=0
%\pdfminorversion=4

% See the \addtolength command later in the file to balance the column lengths
% on the last page of the document

% The following packages can be found on http:\\www.ctan.org
\usepackage{graphics} % for pdf, bitmapped graphics files
\usepackage{epsfig} % for postscript graphics files
% \usepackage{mathptmx} % assumes new font selection scheme installed
\usepackage{times} % assumes new font selection scheme installed
\usepackage{amsmath} % assumes amsmath package installed
\usepackage{amssymb}  % assumes amsmath package installed
\usepackage{color}
\usepackage{booktabs}
% \usepackage{subfigure}
\usepackage{multirow}
\usepackage{algorithm}
\usepackage{algorithmicx}
\usepackage{algpseudocode}
\usepackage{bm}
\usepackage{xcolor}
\usepackage{booktabs}
\usepackage{makecell}

\usepackage{balance}

\usepackage{marvosym}
\usepackage{ifsym}

% \makeatletter
% \let\NAT@parse\undefined
% \makeatother

\usepackage[
	colorlinks,
	anchorcolor=blue,
	linkcolor=cyan,
	urlcolor=red,
	citecolor=green,
	backref=page]{hyperref}
\usepackage[T1]{fontenc}



% \newcommand{\hxl}[1]{\textcolor{red}{\small{\bf [hxl: #1 ]}}}
% \newcommand{\yan}[1]{\textcolor{blue}{\small{\bf [yan: #1 ]}}}
% \newcommand{\wz}[1]{\textcolor{orange}{\small{\bf [wz: #1 ]}}}
% \newcommand{\siqi}[1]{\textcolor{cyan}{\small{\bf [siqi: #1 ]}}}

\title{\LARGE \bf 
CoopDETR: A Unified Cooperative Perception Framework for 3D Detection via Object Query
}

\author{Zhe Wang\textsuperscript{1}, Shaocong Xu\textsuperscript{1}, Xucai Zhuang\textsuperscript{1}, Tongda Xu\textsuperscript{1}, \\ Yan Wang\textsuperscript{1*}, Jingjing Liu\textsuperscript{1}, Yilun Chen\textsuperscript{1}, Ya-Qin Zhang\textsuperscript{1}\\
\thanks{$^{1}$Zhe Wang, Shaocong Xu, Xucai Zhuang, Tongda Xu, Yan Wang$^{*}$, Jingjing Liu, Yilun Chen, and Ya-Qin Zhang are with Institute for AI Industry Research (AIR), Tsinghua University, Beijing, China.
{\tt\small \{wangzhe wangyan\}@air.tsinghua.edu.cn}}
}



\begin{document}



\maketitle
\pagestyle{empty}
\thispagestyle{empty}


% \renewcommand{\thefootnote}{\fnsymbol{footnote}}
% \footnotetext[1]{Correspondence to Yan Wang, Ya-Qin Zhang.}
% \footnotetext[2]{Code will be released at \href{https://github.com/Bosszhe/EMIFF}{https://github.com/Bosszhe/EMIFF}}

%%%%%%%%%%%%%%%%%%%%%%%%%%%%%%%%%%%%%%%%%%%%%%%%%%%%%%%%%%%%%%%%%%%%%%%%%%%%%%%%
\begin{abstract}

Cooperative perception enhances the individual perception capabilities of autonomous vehicles (AVs) by providing a comprehensive view of the environment. However, balancing perception performance and transmission costs remains a significant challenge. Current approaches that transmit region-level features across agents are limited in interpretability and demand substantial bandwidth, making them unsuitable for practical applications. In this work, we propose CoopDETR, a novel cooperative perception framework that introduces object-level feature cooperation via object query. Our framework consists of two key modules: single-agent query generation, which efficiently encodes raw sensor data into object queries, reducing transmission cost while preserving essential information for detection; and cross-agent query fusion, which includes Spatial Query Matching (SQM) and Object Query Aggregation (OQA) to enable effective interaction between queries. Our experiments on the OPV2V and V2XSet datasets demonstrate that CoopDETR achieves state-of-the-art performance and significantly reduces transmission costs to 1/782 of previous methods.




% Cooperative perception in multi-agent systems by enabling agents to share complementary information through communication, there. Most existing methods focus on intermediate fusion to balance perception performance and transmission costs but often  In this paper, we present CoopDETR, a novel framework that introduces a query-based cooperative perception paradigm for multi-agent systems. 

% Inspired by transformer-based object detection models (DETR), CoopDETR utilizes object-level queries to represent the environment, allowing for flexible and efficient cross-agent query interaction. This framework consists of two primary components: (1) a single-agent query generation module that processes point cloud data to generate object queries, and (2) a cross-agent query fusion module including Spatial Query Matching (SQM) and Object Query Aggregation (OQA) to aggregate information accross agents for better performance.


\end{abstract}


%%%%%%%%%%%%%%%%%%%%%%%%%%%%%%%%%%%%%%%%%%%%%%%%%%%%%%%%%%%%%%%%%%%%%%%%%%%%%%%%
\section{INTRODUCTION}

In recent years, autonomous driving has made significant progress, however, substantial challenges persist, particularly in the area of single-vehicle perception. Limitations in range and accuracy still affect the safety of autonomous vehicles. Cooperative perception, which allows vehicles and infrastructure to communicate, offers a potential solution. Cooperative perception addresses key limitations of single vehicles by providing more comprehensive and reliable information, particularly in complex traffic scenarios. As a result, this approach has garnered increasing attention from researchers recently.~\cite{han2023collaborative, liu2023towardsv2x_survey}.

% In recent years, autonomous driving technology has made significant strides; however, substantial challenges persist, particularly in the area of single-vehicle perception. The constraints on perception range and accuracy continue to hinder the ability of autonomous vehicles to operate safely. Cooperative perception, which enables communication between multiple agents, including vehicles and infrastructure within the transportation system, offers a potential solution by providing autonomous vehicles with a more comprehensive and reliable perception capability, especially in complex traffic scenarios. This approach has recently garnered increasing attention from researchers~\cite{han2023collaborative,liu2023towardsv2x_survey}.


\begin{figure}[t]
	\centering  
	\includegraphics[width=0.9\linewidth]{Teaser.pdf} 
	\caption{Consider a typical cooperative perception scenario involving three connected agents (1 to 3) and five objects (A to E) to be detected. Each agent processes its respective point cloud and generates queries of surrounding objects using a DETR-based model. Queries corresponding to the same object in the scene can be connected to form an object query graph, facilitating further query fusion via attention mechanism. Subfigure (d) illustrates the object query graphs for objects A to E.}  
	\label{fig:Teaser}   
\end{figure}


Compared to traditional single-vehicle perception, cooperative perception introduces a set of new challenges. The foremost issue lies in determining what information should be transmitted to keep a balance between perception performance and transmission cost. Based on the type of data communicated among agents, current cooperative perception methods can be categorized into three typical cooperation paradigms: \textit{early fusion} (EF) of raw sensor data~\cite{yu2022dairv2x,chen2022co3,chen2019cooper}, \textit{intermediate fusion} (IF) of features~\cite{mehr2019disconet,xu2022opv2v,wang2020v2vnet,fan2023quest,hu2022where2comm,xu2022v2xvit}, and \textit{late fusion} (LF) of prediction results~\cite{yu2022dairv2x,chen2022model-agnostic}.


Early fusion involves the transmission of raw sensor data from each agent, which preserves the original information but requires significantly higher transmission costs for cooperation. In contrast, late fusion has the advantage of reduced transmission costs, making it suitable for practical applications. However, it suffers from considerable information loss from the source sensor data, and its performance is highly contingent on the perception accuracy of individual agents. Intermediate fusion keeps a balance between performance and bandwidth, as features can be compressed for lower transmission costs while retaining critical information extracted from raw data. Several approaches~\cite{hu2022where2comm,xu2022v2xvit} have encoded raw data into dense, region-level features for communication, such as Bird's-eye-view (BEV) features. However, this type of representation, which aims to depict the entire scene, suffers from limited interpretability and may still contain redundant information despite feature selection mechanisms~\cite{hu2022where2comm}. Considering that the most valuable information for 3D detection is object-specific, can we design a communication mechanism that is centered on object-level features specifically tailored for cooperative perception?


Inspired by transformer-based object detection methods (DETR)~\cite{carion2020detr,wang2022detr3d,chen2022futr3d,fan2023quest}, we propose a novel paradigm for achieving object-level feature cooperation in multi-agent systems, where raw data is encoded into queries, with each query corresponding to a specific object in the scene.
This approach offers the similar interpretability as late fusion while achieving lower communication bandwidth compared to early fusion and region-level intermediate fusion.
The proposed framework, named CoopDETR, introduces a unified cooperative perception model that leverages object queries to facilitate interaction across multiple agents.
As illustrated in Figure~\ref{fig:Teaser}, in a simplified scene, each agent generates a set of queries for objects within its observation range so that different agents produce distinct queries for the same object, reflecting diverse information.
An object query graph can be constructed for each object, where each node represents a query derived from a single agent's sensor data. These queries can be flexibly fused to aggregate information, enabling a more comprehensive representation of the object. Each query can then be decoded to infer the object's category and bounding box.


Specifically, CoopDETR consists of two primary modules: single-agent query generation and cross-agent query fusion. In the query generation module, each agent leverages a transformer-based model, PointDETR, to update queries based on point cloud features, which can then be shared with other agents. In the cross-agent query fusion module, upon receiving queries from other agents, the ego agent applies Spatial Query Matching (SQM) to associate similar queries and cluster them into distinct object query graphs. These queries within the same graph are subsequently fused through an attention-based mechanism in the Object Query Aggregation (OQA) process. The fused queries are subsequently fed into detection heads for final prediction. Experimental results on the OPV2V~\cite{xu2022opv2v} and V2XSet~\cite{xu2022v2xvit} datasets demonstrate that CoopDETR achieves an improved trade-off between perception accuracy and transmission cost.



\begin{itemize}
    \item We propose CoopDETR, a novel cooperative perception framework based on object query, achieving more efficient communication through object-level feature cooperation, compared to dense BEV features for scene-level feature cooperation.
    \item We design Spatial Query Matching (SQM) and Object Query Aggregation (OQA) modules for query interaction to select queries of co-aware objects and fuse queries at the instance level.
    \item We achieve state-of-the-art results on the V2XSet and OPV2V datasets, outperforming other cooperative perception methods while reducing transmission costs to 1/782 of previous methods.
\end{itemize}



\section{Related Work}
\subsection{V2X Cooperative Perception}

Current research on cooperative perception mainly aims to extend the perception range and improve perception capability of autonomous vehicles~\cite {wang2024emiff,han2023collaborative}. The most intuitive approach is Early Fusion, which transmits raw sensor data~\cite{chen2019cooper,chen2022co3}. However, transmitting raw data requires high transmission costs, making it impractical for real-world deployment. Late Fusion that transmits perception results from each agent is the most bandwidth-efficient paradigm~\cite{chen2022model-agnostic,yu2022dairv2x}. Yet, its performance relies heavily on the accuracy of each agent's perception result. Most research has shifted toward intermediate fusion, which transmits region-level features for better performance-bandwidth balance~\cite{xu2022opv2v,mehr2019disconet,xu2022v2xvit,hu2022where2comm,xu2022cobevt,wang2020v2vnet,yu2023ffnet,cui2022coopernaut,hu2023coca3d,wang2024emiff,wei2023cobevflow}. Although these methods incorporate strategies to reduce transmission costs, such as feature selection via spatial confidence maps~\cite{hu2022where2comm}, feature compression~\cite{xu2022opv2v,xu2022v2xvit,wang2024emiff}, and flow-based prediction~\cite{yu2023ffnet,wei2023cobevflow}, region-level feature is still redundant for object detection and lacks interpretability~\cite{fan2023quest}. QUEST \cite{fan2023quest} proposes the concept of query-cooperation paradigm but focuses only on a simple V2I scenario involving one vehicle and infrastructure.  To enable more efficient cooperation across multi-agent systems, we propose a unified cooperation perception framework that transmits object-level queries across agents.


\begin{figure*}[t]
	\centering  
	\includegraphics[width=0.9\linewidth]{CoopDETR_Arch.pdf} 
	\caption{The general framework of CoopDETR. For each agent, the query generation module learns $N_q$ object queries from raw data. Each object in the scene will correspond to a query. For the whole multi-agent system, one object may be observed by different agents and be associated with different queries. Take $i$-th agent as ego agent, object queries $Q_{j} = \{q^{j}_{1},\dots,q^{j}_{N_q}\}$ from $j$-th agent and their reference points $r$ will be transmitted to $i$-th agent. In cross-agent query fusion module, all queries will be fused with two steps, the 
 the first step is to associate different queries for co-aware objects through spatial query matching (SQM) and generate object query graph for each object. The second step is to fuse all queries in the same graph using Object Query aggregation (OQA) and generate a set of updated queries $\hat{Q}$, which will be fed to detection heads for category and bounding box prediction. }  
	\label{fig:framework}   
\end{figure*}

\subsection{Transformer-based Perception}
The pioneering work DETR~\cite{carion2020detr} regards 2D object detection task as a set-to-set problem. The query mechanism has been increasingly adopted across various perception tasks, including 3D object detection~\cite{wang2022detr3d,chen2022futr3d,chen2023transiff,fan2023quest}, object tracking~\cite{zeng2022motr,zhang2022mutr3d,pang2023pf-track,meinhardt2022trackformer}, semantic segmentatin~\cite{li2022bevformer,peng2022bevsegformer,maiti2023transfusion}, and planning~\cite{hu2023_uniad,yu2024_univ2x}. Query-based approaches typically leverage sparse, learnable queries for attentive feature aggregation to capture complex relationships among sequential elements. FUTR3D~\cite{chen2022futr3d} predicts 3D query locations and retrieves corresponding multi-modal features from cameras, LiDARs, and radars via projection. BEVFormer~\cite{li2022bevformer,yang2023bevformerv2} introduces grid-shaped queries in BEV and updates them by interacting with spatio-temporal features using deformable transformers. While most existing query-based methods focus on individual perception, QUEST~\cite{fan2023quest} and TransIFF~\cite{chen2023transiff} extend it to vehicle-to-infrastructure (V2I) scenarios. In this work, we introduce a novel query fusion mechanism, which facilitates efficient query matching and aggregation tailored for multi-agent systems.


\section{Method}


 Considering $N$ agents in a multi-agent system, $\boldsymbol{\mathcal{X}}_{i}$ is the point clouds observed by $i$-th agent and $\boldsymbol{\mathcal{Y}}_{i}$ is the corresponding perception supervision. The object of CoopDETR is to achieve the maximized perception performance of all agents with a communication budget of $B$.

\begin{equation}
    \begin{split}
    \xi(B)  = \underset{\theta, \boldsymbol{M}_{j\rightarrow i}} {\arg \max } & \sum_i^N g\left(\Psi_\theta\left(\boldsymbol{\mathcal{X}}_{i},\left\{{\boldsymbol{M}}_{j\rightarrow i}\right\}_{j=1}^N\right), \boldsymbol{\mathcal{Y}_{i}}\right) \\
     \text { s.t. } & \sum_j \left| {\boldsymbol{M}}_{j \rightarrow i}\right| \leq B
    \end{split}
\end{equation}
where $g(\cdot,\cdot)$ denotes the perception evaluation metric and $\Psi_\theta$ is the perception network with trainable parameter $\theta$, and ${\boldsymbol{M}}_{j\rightarrow i}$ means the message transmitted from the $j$-th agent to the $i$-th agent. 

The architecture of CoopDETR is shown in Figure~\ref{fig:framework}, which includes single-agent query generation and cross-agent query fusion modules.
 


\begin{figure}[ht]
	\centering  
        \includegraphics[width=0.9\linewidth]{Query_Generation.pdf} 
	\caption{Illustration of PointDETR module.
 }
	\label{fig:PointDETR}   
\end{figure}


\subsection{Single-Agent Query Generation}


\textbf{Feature Encoder} Following FUTR3D~\cite{chen2022futr3d}, we employ PointPillar~\cite{lang2019pointpillars} as the feature encoder to process the point cloud. After passing through the 3D backbone and FPN~\cite{lin2017fpn}, we extract multi-scale bird’s-eye view (BEV) features, denoted as \( \textbf{F}^{L}_{BEV} = [F^{l}_{BEV} \in \mathbb{R}^{C \times H_{l} \times W_{l}}]^{L}_{l=1} \), where \( L \) denotes the number of feature levels.


\textbf{PointDETR} Each agent initializes a set of object queries $Q \in \mathbb{R}^{N_q \times C_q} $, which are then refined dynamically using the transformer decoder, PointDETR (illustrated in Figure~\ref{fig:PointDETR}). A key aspect of PointDETR lies in determining how to sample features corresponding to an object query from BEV features.

In the Deformable Cross-Attention block, a reference point \( r_{i} \in \mathbb{R}^{3} \), representing the center of the \(i\)-th object's bounding box, is decoded from an object query using a linear neural network. This reference point is initialized from a sketch and is used to sample BEV features through multi-scale deformable attention~\cite {zhu2020deformable}. As illustrated in Figure~\ref{fig:PointDETR}, the reference point \( r_{i} \) is encoded into a positional embedding using a sine function and a linear network, which are subsequently added to the query \( q_{i} \). The sample location offset \( \Delta r_{iklm} \) and attention weight \( \sigma_{i k l m} \) are generated through separate linear networks. The sample offset \( \Delta r_{iklm} \) is added to the initial reference point \( r_{i} \) to obtain the final sample location \( s_{iklm} \). The BEV features are then sampled at these locations \( s_{iklm} \) using bilinear sampling. Sample location $s_{iklm} \in [0,1]^2$ is represented by normalized coordinates of feature map.

\begin{equation}
\boldsymbol{V}_{i k l m} = f^{bilinear}(f^{l}_{BEV},s_{iklm})
\end{equation}

Where $k$ indexes the sampling point, $l$ indexes the lidar feature's scale, and $m$ indexes the attention head. Following Deformable DETR~\cite{zhu2020deformable},  we samples $L \times K$ points from multi-scale features and update queries by

\begin{equation}
\Bar{q_{i}}=\sum_{m=1}^M \boldsymbol{W}_{i}\left[\sum_{l=1}^L \sum_{k=1}^K \sigma_{i k l m} \cdot \boldsymbol{W}_{i}^{\prime} \boldsymbol{V}_{i k l m}\right]
\end{equation}

The scalar attention weight $\sigma_{iklm}$ is normalized by $\sum_{l=1}^L \sum_{k=1}^K \sigma_{i k l m}=1$. 





\subsection{Cross-Agent Query Fusion}
% In a multi-agent system where $J$ agents can communicate with each other, each agent will generate $D$ updated queries and corresponding reference points through its own Query Generation module and broadcast them to other agents.
% of which $N_{q}$ queries pertain to itself, while the remaining $(N-1)N_{q}$ queries are from other agents. 

After communication, ego agent $i$ has $NN_{q}$ queries. All objects encoded by these queries can be categorized into three types: those co-aware to both ego agent and other agents, those observed solely by ego agent, and those recognized only by other agents. As shown in Figure~\ref{fig:QF}, Spatial Query Matching can associate queries corresponding to the same object and generate an object query graph via masked attention. Object Query Aggregation fuses all queries in the same graph. All fused queries will be permuted by their own confidence score and only the maximum $N_{q}$ ones will be fed into the detection head for prediction. The detection head is identical to that used in FUTR3D~\cite{chen2022futr3d}, comprising two separate MLP for classification and bounding box prediction respectively. Following ~\cite{wang2022detr3d,chen2022futr3d,carion2020detr}, we compute a set-to-set loss between predictions and ground-truths.

% Firstly, Hungarian algorithm is utilized to achieve one-to-one matching between predictions and ground-truths. Then regression and classification losses are computed based on this one-to-one assignment.

\textbf{Spatial Query Matching} To associate queries related to the same object, it is necessary to measure the similarity between the $i$-th query from the ego agent and the $j$-th query from other agents. Directly using reference points for matching suffers
 from inevitable pose errors. Therefore, we combine the query feature $q_{i}$, which contains contextual information from point cloud, with the position embedding derived from the reference point $r_{i}$ for more accurate matching. The refined query, $\Tilde{q}_{i}$, is obtained as follows:

% To associate queries of the same object, similarity between $i$-th query from ego agent and $j$-th query from other agents is required. The direct way is using reference points $r_{i}$ to determine whether two queries pertain to the same object. However, localization error is always existing and results in imprecise matching. Therefore, we combine query feature $q_{i}$ that contain context information from lidar feature with position embedding from reference point $r_{i}$ for matching. The refined query $\Tilde{q}_{i}$ can be obtained by:


\begin{equation}
\Tilde{q}_{i} = q_{i} + \Phi(f_{sin}(r_{i}))
\end{equation}
where $\Phi$ means linear neural network. $f_{sin}$ means sine function in Transformer~\cite{vaswani2017attention}. $\Phi(f_{sin}(r_{i}))$ denote position embedding (PE) of $r_{i}$. The similarity of refined queries $\Tilde{q}_{i}$ and $\Tilde{q}_{j}$ is calculated by:

\begin{equation}
     s_{i,j} = \text{sigmoid} \left( \frac{\langle \Tilde{q}_{i},\Tilde{q}_{j} \rangle}{\Vert \Tilde{q}_{i} \Vert_{2} \cdot \Vert \Tilde{q}_{j} \Vert_{2}}  \right)
\end{equation}
$\langle \cdot \rangle$ means inner product. We set threshold $\mu$ to determine whether two queries should be associated together. 

\begin{figure}[ht]
	\centering  
        \includegraphics[width=0.9\linewidth]{Query_Fusion.pdf} 
	\caption{The details of Cross-Agent Query Fusion.
 }
	\label{fig:QF}   
\end{figure}

\textbf{Object Query Aggregation} After Query After Query Matching, we obtain multiple object query graphs, each corresponding to an object in the scene.  Queries in the same graph are aggregated via multi-head attention~\cite{vaswani2017attention}. Take one graph as an example, query $\Tilde{q}_{j}$ with similarity $s_{i,j}$ below the threshold is masked for calculation of attention weights. Ego agent's query is encoded as $q\in \mathbb{R}^{1 \times C}$ and masked query from other agents are encoded as key $K$ and value $V$. The fused query $\Hat{q}_{i}$ can be calculated by

\begin{equation}
\Hat{q} = \operatorname{Mask\_Attention}(q, K, V)=\operatorname{softmax}\left(\frac{\epsilon (q K^T)}{\sqrt{d_k}}\right) V
\end{equation}
where $\epsilon$ denotes masking operation. The attention weight between the key $K$ for the masked queries and the query $q$ is set to zero, ensuring that the update of the ego agent's query does not involve the masked queries.



\section{Expermients}


\subsection{Implementation Details}



\textbf{Datasets.}
To evaluate the performance of CoopDETR, we conduct extensive experiments on two benchmark datasets for the multi-agent cooperative perception task: OPV2V~\cite{xu2022opv2v} and V2XSet~\cite{xu2022v2xvit}. OPV2V is a large-scale simulated dataset collected in vehicle-to-vehicle (V2V) scenarios, consisting of 11,464 frames of point clouds with corresponding 3D annotations. The dataset is split into 6,764 training frames, 1,981 validation frames, and 2,719 testing frames. V2XSet~\cite{xu2022v2xvit}, co-simulated by CARLA~\cite{dosovitskiy2017carla} and OpenCDA~\cite{xu2021opencda}, is designed for V2X cooperative perception. It includes 73 representative scenes with 2 to 5 connected agents, totaling 11,447 frames of annotated LiDAR point clouds, with splits of 6,694 frames for training, 1,920 for validation, and 2,833 for testing.

% To evaluate the performance of CoopDETR, we conduct extensive experiments on two benchmark datasets for the multi-agent cooperative perception task, including OPV2V~\cite{xu2022opv2v} and V2XSet~\cite{xu2022v2xvit}. OPV2V~\cite{xu2022opv2v} is a large-scale simulated dataset collected in vehicle-to-vehicle (V2V) scenarios, comprising 11,464 frames of point cloud with corresponding 3D annotation. The training/validation/testing splits include 6,764, 1,981, and 2,719 frames respectively. V2XSet [40] is a simulated dataset for V2X cooperative perception, co-simulated by Carla ~\cite{dosovitskiy2017carla} and OpenCDA~\cite{xu2021opencda}. It includes 73 representative scenes with 2 to 5 connected agents and 11,447 frames of annotated LiDAR point cloud. The training/validation/testing sets are 6,694, 1,920, and 2,833 frames, respectively. 

\textbf{Evaluation Metrics.}  We use Average Precision (AP) at Intersection-over-Union (IoU) thresholds of 0.3, 0.5, and 0.7 to evaluate 3D object detection performance. For evaluating transmission cost, we follow the method outlined in ~\cite{hu2022where2comm}, which calculates communication volume by measuring the message size in Bytes, represented on a logarithmic scale with base 2.
% We adopt the Average Precision (AP) at Intersection-over-Union (IoU) thresholds of 0.3, 0.5, and 0.7 to evaluate the 3D object detection performance. As for the evaluation of transmission cost, the calculation of communication volume follows~\cite{hu2022where2comm}, which counts the message size by Byte in the log scale with base 2.



\textbf{Experimental Settings.} 
The detection range of the LiDAR is set to $[-140.8\text{m}, 140.8\text{m}]$ along the X-axis, $[-40.0\text{m}, 40.0\text{m}]$ along the Y-axis, and $[-3.0\text{m}, 1.0\text{m}]$ along the Z-axis. Pillar size in PointPillar~\cite{lang2019pointpillars} is $[0.2\text{m}, 0.2\text{m}]$. The maximum feature map size from the feature encoder is $[H_{1}, W_{1}] = [400, 1408]$ with 256 channels. In PointDETR, the number of queries $N_{q}$ is 180, the number of sampled points $K$ is 4, and the number of attention heads $M$ is 8. The number of point cloud feature scales $L$ is set to 4, and the threshold $\mu$ for query matching is 0.3. We train the model for 50 epochs on both the OPV2V and V2XSet datasets, with a batch size of 4 for each. Training is conducted on NVIDIA Tesla A30 GPUs.
We employ AdamW~\cite{loshchilov2017adamw} as the optimizer, with a weight decay of $10^{-2}$. The learning rate is set to $2 \times 10^{-4}$, and we utilize a cosine annealing learning rate scheduler~\cite{loshchilov2016coslrs} with 10 warm-up epochs and a warm-up learning rate of $10^{-5}$.

% The detection range of lidar is set to  $[-140.8\text{m},140.8\text{m}]$ for the X axis, $[-40.0\text{m},40.0\text{m}]$ for the Y axis, and $[-3.0\text{m},1.0\text{m}]$ for the Z axis. We use PointPillar~\cite{lang2019pointpillars} with pillar size $[0.2\text{m},0.2\text{m}]$ as model's encoder. The maximum size of feature map $[H_{1},W_{1}]$ from point cloud encoder is $[704,100]$ with channels of $256$. The number of queries $N_{q}$ output from PointDETR is 180. The sample point number $K$ is 4 and the attention head number $M$ is $8$. The number of point cloud feature's scale $L$ is 4.  The threshold $\mu$ for query matching is 0.3. The training epochs on the OPV2V and V2XSet datasets are {50,50}, and batch sizes are {4,4}. We use NVIDIA Tesla A30 GPUs for training.
% We use AdamW~\cite{loshchilov2017adamw} as the optimizer with the weight decay of $10^{-2}$.  We set the learning rate to $2\times 10^{-4}$ and use cosine annealing learning rate scheduler~\cite{loshchilov2016coslrs} with the warm-up epochs as 10 and the warm-up learning rate of $10^{-5}$.

% For testing, each query will predict a classification score and bounding boxes in the detection head, and only the bounding boxes with a score above the threshold of 0.3 will be maintained. 

% \textbf{Compared Models} We consider a single-agent perception system without any information from other agents as No Fusion. Late Fusion fuses the predicted bounding boxes from different agents and applies non-maximum suppression to produce the final outcomes. For the multi-agent cooperative perception task, we compared CoopDETR with the state-of-the-art methods that consider the fusion of a single frame, including When2com~\cite{Liu2020when2com}, V2VNet~\cite{wang2020v2vnet}, AttFuse~\cite{xu2022opv2v}, V2X-ViT~\cite{xu2022v2xvit}, DiscoNet~\cite{mehr2019disconet}, CoBEVT~\cite{xu2022cobevt} and Where2comm~\cite{hu2022where2comm}. 


% Without DAIR-V2X
\begin{table}[htbp]
    \centering
    \footnotesize
    % \resizebox{\linewidth}{!}{
    \begin{tabular}{ccc}
    \toprule
    \multirow{2}{*}{\textbf{Model}} 
     & \textbf{V2XSet} & \textbf{OPV2V} \\ 
     \cmidrule(lr){2-3}
      & \textbf{AP@0.5/0.7} & \textbf{AP@0.5/0.7} \\
    \midrule
    No Fusion  & 60.60/40.20 & 68.71/48.66 \\
    Late Fusion  & 66.79/50.95 & 82.24/65.78 \\
    Early Fusion  & 77.39/50.45 & 81.30/68.69 \\
    When2com \cite{Liu2020when2com}  & 70.16/53.72 & 77.85/62.40 \\
    V2VNet \cite{wang2020v2vnet}  & 81.80/61.35 & 82.79/70.31 \\
    AttFuse \cite{xu2022opv2v}  & 76.27/57.93 & 83.21/70.09 \\
    V2X-ViT \cite{xu2022v2xvit}  & 85.13/68.67 & 86.72/74.94 \\
    DiscoNet \cite{mehr2019disconet} & 82.18/63.73 & 87.38/73.19 \\
    CoBEVT \cite{xu2022cobevt}  & 83.01/62.67 & 87.40/74.35 \\
    Where2comm \cite{hu2022where2comm}  & 85.78/72.42 & 88.07/75.06 \\
    \midrule
    \textbf{CoopDETR (Ours)}  & \textbf{86.96/76.51} & \textbf{90.59/83.97} \\
    \bottomrule
    \end{tabular}
    \caption{Performance comparison on the V2XSet, and OPV2V datasets. The results are reported in AP@0.5/0.7.}
    \label{TAB:ap50}
\end{table}

\subsection{Quantitative Evaluation}

% We consider a single-agent perception system without any information from other agents as No Fusion. Late Fusion fuses the predicted bounding boxes from different agents. For the multi-agent cooperative perception task, 



% We compare the detection performance of the proposed CoopDETR with various cooperative perception methods on OPV2V and V2XSet datasets and the results are shown in Table~\ref{TAB:ap50}. We compared CoopDETR with the state-of-the-art methods that consider the fusion of a single frame, including When2com~\cite{Liu2020when2com}, V2VNet~\cite{wang2020v2vnet}, AttFuse~\cite{xu2022opv2v}, V2X-ViT~\cite{xu2022v2xvit}, DiscoNet~\cite{mehr2019disconet}, CoBEVT~\cite{xu2022cobevt} and Where2comm~\cite{hu2022where2comm}. CoopDETR significantly outperforms previous methods, demonstrating the superiority of our model. Specifically, the SOTA performance of AP@0.7 on OPV2V and V2XSet is improved by 8.91 and 4.09, respectively. CoopDETR demonstrates the superiority of object-level fusion method.

\begin{figure}[htbp]
	\centering  
	\includegraphics[width=0.9\linewidth]{Com_Volume.png} 
	\caption{Cooperative perception performance comparison of CoopDETR and other methods on V2XSet and OPV2V dataset. The communication volumes are also depicted in this figure.}  
	\label{fig:Com_Volume}   
\end{figure}

\textbf{Object Detection Results} We compare the detection performance of the proposed CoopDETR with various cooperative perception methods on the OPV2V and V2XSet datasets, as shown in Table~\ref{TAB:ap50}. CoopDETR is compared with state-of-the-art methods that focus on single-frame fusion, including When2com~\cite{Liu2020when2com}, V2VNet~\cite{wang2020v2vnet}, AttFuse~\cite{xu2022opv2v}, V2X-ViT~\cite{xu2022v2xvit}, DiscoNet~\cite{mehr2019disconet}, CoBEVT~\cite{xu2022cobevt}, and Where2comm~\cite{hu2022where2comm}. CoopDETR significantly outperforms these previous methods, demonstrating the superiority of our model. In particular, CoopDETR improves the state-of-the-art AP\@ 0.7 performance on the OPV2V and V2XSet datasets by 8.91 and 4.09, respectively, highlighting the effectiveness of the object-level fusion paradigm.

\textbf{Transmission Cost} The performance comparison results with distinct transmission costs, are shown in Figure~\ref{fig:Com_Volume}. The blue curves represent the detection performance of Where2comm at different compression rates, while the red cross marks CoopDETR, which achieves both lower communication volume and better performance than other intermediate fusion methods. Notably, CoopDETR's transmission volume is just 1/782 of that used by other intermediate fusion methods, and it approaches the volume of Late Fusion, typically regarded as the lower bound for transmission cost. These significant improvements underscore the effectiveness of CoopDETR's query-based mechanism in filtering redundant information from dense point cloud features, enabling it to learn more precise object representations.

\begin{figure}[htbp]
	\centering  
	\includegraphics[width=\linewidth]{Localization_Error.png} 
	\caption{Models' robustness to the localization error on the V2XSet and OPV2V datasets.}  
	\label{fig:Localization_Error}   
\end{figure}

\begin{figure}[htbp]
	\centering  
	\includegraphics[width=\linewidth]{Heading_Error.png} 
	\caption{Models' robustness to the heading error on the V2XSet and OPV2V datasets.}  
	\label{fig:Heading_Error}   
\end{figure}





\begin{figure*}[t]
	\centering  
	\includegraphics[width=0.85\linewidth]{VIS.pdf} 
	\caption{Visualization results from the OPV2V dataset. Green and red bounding boxes denote the ground truths and prediction results respectively. CoopDETR qualitatively outperforms other cooperative perception methods AttFuse~\cite{xu2022opv2v}, V2X-ViT~\cite{xu2022v2xvit}, and Where2comm~\cite{hu2022where2comm}. Our method yields more accurate detection results and has fewer false positive objects than others.}
	\label{fig:vis_results}   
\end{figure*}









% The performance comparison results with distinct transmission costs are shown in Figure~\ref{fig:Com_Volume}. Concretely, the blue curves denote the detection result of Where2comm under different compression rates,  The red cross denotes CoopDETR, which has a lower communication volume and superior performance than other intermediate methods. It is worth noting that CoopDETR's transmission volume is 1/782 of other intermediate fusion methods, closely approaching that of Late Fusion, which is generally regarded as the lower bound for transmission volume. The difference between CoopDETR and late fusion is within an order of magnitude of 10. The noteworthy improvements demonstrate that the proposed query-based mechanism filters redundancy from dense point cloud features and learns a better representation of a single object for performance improvement.


% With DAIR-V2X
% \begin{table}[htbp]
%     \centering
%     \footnotesize
%     \resizebox{\linewidth}{!}{
%     \begin{tabular}{cccc}
%     \toprule
%     \multirow{2}{*}{\textbf{Model}} 
%      & \textbf{DAIR-V2X} & \textbf{V2XSet} & \textbf{OPV2V} \\ 
%      \cmidrule(lr){2-4}
%      & \textbf{AP@0.5/0.7} & \textbf{AP@0.5/0.7} & \textbf{AP@0.5/0.7} \\
%     \midrule
%     No Fusion & 50.03/43.57 & 60.60/40.20 & 68.71/48.66 \\
%     Late Fusion & 53.12/37.88 & 66.79/50.95 & 82.24/65.78 \\
%     Early Fusion & 61.74/46.53 & 77.39/50.45 & 81.30/68.69 \\
%     When2com \cite{Liu2020when2com} & 51.12/36.17 & 70.16/53.72 & 77.85/62.40 \\
%     V2VNet \cite{wang2020v2vnet} & 56.01/42.25 & 81.80/61.35 & 82.79/70.31 \\
%     AttFuse \cite{xu2022opv2v} & 53.79/42.61 & 76.27/57.93 & 83.21/70.09 \\
%     V2X-ViT \cite{xu2022v2xvit} & 54.26/43.35 & 85.13/68.67 & 86.72/74.94 \\
%     DiscoNet \cite{mehr2019disconet} & 54.29/44.88 & 82.18/63.73 & 87.38/73.19 \\
%     CoBEVT \cite{xu2022cobevt} & 54.82/43.95 & 83.01/62.67 & 87.40/74.35 \\
%     Where2comm \cite{hu2022where2comm} & 63.71/48.93 & 85.78/72.42 & 88.07/75.06 \\
%     \midrule
%     \textbf{CoopDETR (Ours)} & \textbf{XXXX/XXXX} & \textbf{86.96/76.51} & \textbf{90.59/83.97} \\
%     \bottomrule
%     \end{tabular}}
%     \caption{Performance comparison on the V2XSet, and OPV2V datasets. The results are reported in AP@0.5/0.7.}
%     \label{TAB:ap50}
% \end{table}



\textbf{Robustness to Localization and Heading Errors}. To evaluate the detection performance of CoopDETR under realistic scenarios, we follow ~\cite{yang2024how2comm} to simulate pose errors that occur during communication between agents. The results are shown in Figures~\ref{fig:Localization_Error} and~\ref{fig:Heading_Error}. Localization and heading errors, sampled from Gaussian distributions with standard deviations of $\sigma_{xyz} \in [0m, 0.5m]$ and $\sigma_{h} \in [0^{\circ}, 1.0^{\circ}]$, respectively, are introduced to the agents' locations. The figures reveal a consistent degradation in the performance of all cooperative perception methods, caused by increased feature misalignment. Notably, CoopDETR outperforms previous models across both datasets at all error levels, demonstrating its robustness against pose noise in communication processing. PointDETR, by employing deformable attention to integrate query and point cloud features, helps mitigate the negative effects of pose errors.

% To validate the detection performance of CoopDETR under realistic scenarios, we follow~\cite{yang2024how2comm} to simulate the pose errors of other agents occurring in communication processing and the results can be seen in Figure~\ref{fig:Localization_Error} and~\ref{fig:Heading_Error}. Localization and heading errors that are sampled from Gaussian distribution with a standard deviation of $\sigma_{xyz} \in [0m,0.5m]$  and $\sigma_{h} \in [0^{\circ},1.0^{\circ}]$ respectively are added to the location of agents. From the figures, the performance of all intermediate cooperative perception methods consistently deteriorates because the misalignment of features increases. Noticeably, CoopDETR is superior to the previous models across two datasets under all error levels.
% This comparison demonstrates the robustness of CoopDETR against pose noises in communication processing. PointDETR uses deformable attention to integrate query and point cloud features which can alleviate the negative effect of pose error.


\subsection{Ablation Study}

Table~\ref{TAB:AB} shows ablation studies on all datasets to understand the necessity of model designs and strategies in CoopDETR. 

\textbf{Impact of Feature Size} The resolution of the pillar directly impacts the size of feature output from the point cloud encoder. Lower resolution leads to larger features and a more complex encoder, resulting in detailed representations and better performance. As seen in~\ref{TAB:AB}, AttFuse performs better with larger pillars, this also increases communication volume (CV). Regardless of feature size, a fixed number of queries can still interact with the features and receive updates, facilitating efficient feature communication. This object-level fusion paradigm enables agents to use encoders with different pillar sizes based on their computational capacity, and the queries can be fused in a unified framework.

% The resolution of the pillar significantly influences the output feature size of the point cloud encoder. Lower resolution means increasing the size of the extracted features. Simultaneously, the encoder's parameter also grows, leading to more comprehensive feature representations for better performance. We can see the performance of AttFuse increases along with the pillar size, but the communication volume also rises requiring more bandwidth. Regardless of the feature size, a fixed number of queries can interact with the features and obtain updates. This enables effective feature communication so that increasing feature size allows great performance improvement without more transmission costs. This object-level fusion paradigm allows agents to use encoder with varying pillar sizes for query extraction based on their computational capacity. Those query can be fused in a unified framework.

\textbf{Impact of Query Number} Empirically, we experiment with varying the number of queries $N_q$ in PointDETR and found that 180 queries deliver the most competitive detection performance. An excessive number of queries can lead to performance bottlenecks by increasing the difficulty of model convergence and raising transmission costs. Too few queries may result in information loss during the interaction between queries and BEV features.

% Empirically, we set variable number of Queries in PointDETR to perform experiments and find that 180 queries achieve the most
% competitive detection performance. Too many queries may cause performance bottlenecks due to increasing difficulty of model convergence and involve more transmission costs.


\textbf{Importance of SQM} We remove the Spatial Query Masking (SQM) in CoopDETR, where all queries are concatenated and fed directly into the detection head. In this setup, each query interacts with all other queries without masking during query aggregation. This increases the difficulty of model convergence and introduces redundant information into the fusion process. By contrast, SQM enables more efficient integration of contextual and spatial information within queries, improving both performance and convergence.

% We remove SQM in the CoopDETR (first row), which just concatenates all queries together and feeds them into detection. The entire process is equivalent to each query interacting with all other queries without masking in query aggregation. This will increase the difficulty of model convergence and introduce redundant information for fusion. The use of SQM allows for more efficient integration of context and spatial information within queries.


% The Second row means we remove position embedding of reference points from the query so that the matching will not consider spatial information, which can cause inaccurate matching pair. 



% \begin{table}[htbp]
%   \footnotesize
%   \centering
%     \begin{tabular}{ccccc}
%     \hline
%         \multirow{2}{*}{Designs} & V2XSet & OPV2V & \multirow{2}{*}{BEV Size} & \multirow{2}{*}{CV} \\
%                                  & AP@0.7 & AP@0.7 \\ \hline
%          \multicolumn{5}{c}{\textit{Rationality of Communication Mechanism}}
%          \midrule
%         \multirow{3}{*}{CoopDETR}                &  xxxx      & 76.06     & 200x704
%         & 0.046MB  \\ \hline
%                         &  xxxx      & 81.04     & 400x1408 & 0.046MB   \\ 
%                        &  xxxx      & 82.34     & 800x2816 & 0.046MB   \\ 
%         \multirow{3}{c}{AttFuse}                 &  68.92      & 70.09   & 200x704 & 36.044MB    \\ 
%                         &  69.64      & 73.92   & 400x1408 & 144.179MB    \\
%                         &  69.91      & 73.99   & 800x2816 & 576.717MB    \\
%         \hline
%     \end{tabular}
%     \caption{Analysis on choice of query number.}
%     \label{TAB:query_num}
% \end{table}




% \begin{table}[ht]
%   \footnotesize
%   \centering
%     \begin{tabular}{ccc}
%     \hline
%         \multirow{2}{*}{Designs} & V2XSet & OPV2V  \\
%                                  & AP@0.7 & AP@0.7 \\ \hline
%         180 query                &  xxxx      & xxxx       \\
%         360 query                &  xxxx      & xxxx       \\ 
%         540 query                &  xxxx      & xxxx       \\ 
%         720 query                &  xxxx      & xxxx       \\ 
%         900 query                &  xxxx      & xxxx       \\ \hline
%     \end{tabular}
%     \caption{Analysis on choice of query number.}
%     \label{TAB:query_num}
% \end{table}


% \begin{table}[ht]
%   \footnotesize
%   \centering
%     \begin{tabular}{ccc}
%     \hline
%         \multirow{2}{*}{Designs} & V2XSet & OPV2V  \\
%                                  & AP@0.7 & AP@0.7 \\ \hline

%     \end{tabular}
%     \caption{Analysis on choice of query number.}
%     \label{TAB:query_num}
% \end{table}

\begin{table}[ht]
  \footnotesize
  \centering
  \resizebox{\linewidth}{!}{
    \begin{tabular}{ccccc}
    \hline
        \multirow{2}{*}{Designs} & V2XSet & OPV2V & \multirow{2}{*}{$H_{1} \times W_{1}$ } & \multirow{2}{*}{CV} \\
                                 & AP@0.7 & AP@0.7 \\ \midrule
         \multicolumn{5}{c}{Impact of Feature Size} \\ 
         \midrule
        \multirow{3}{*}{CoopDETR}               &  69.94      & 76.06     & 200x704 & 0.046MB  \\
                        &  75.62      & 81.04     & 400x1408 & 0.046MB   \\ 
                       &  76.71      & 82.34     & 800x2816 & 0.046MB   \\ \hline
        \multirow{3}{*}{AttFuse}                &  68.92      & 70.09   & 200x704 & 36.044MB    \\ 
                        &  69.64      & 73.92   & 400x1408 & 144.179MB    \\
                        &  69.91      & 73.99   & 800x2816 & 576.717MB    \\
        \midrule
        \multicolumn{5}{c}{Impact of Query Number$N_{q}$} \\
        \midrule
        $N_{q}=90$                &  69.04      & 82.31     & \multirow{6}{*}{400x1408} & 0.023MB\\
        $N_{q}=180$ (Default)                &  \textbf{76.51}      & \textbf{83.97}      &  & 0.046MB\\
        $N_{q}=360$               &  69.51     & 80.07     & & 0.092MB \\ 
        $N_{q}=540$               &  71.72      & 78.15       & &0.138MB\\ 
        $N_{q}=720$                &  67.74      & 81.82       & & 0.184MB\\ 
        $N_{q}=900$                &  71.54      & 81.49       & & 0.230MB\\ 
        \midrule
        \multicolumn{5}{c}{Importance of SQM} \\
        \midrule
        w/o SQM               &  69.94      & 80.40  &\multirow{2}{*}{400x1408} &\multirow{2}{*}{0.046MB}    \\
        w SQM                &  \textbf{76.51}      & \textbf{83.97}   & &    \\ \hline
        
    \end{tabular}}
    \caption{Analysis on the choice of query number, pillar size, and component of query matching. "w/o" means without.}
    \label{TAB:AB}
\end{table}




\subsection{Quanlitative Evaluation} 
To illustrate the perception performance of various models, Figure~\ref{fig:vis_results} presents the visualization results of two challenging scenarios from the OPV2V dataset. CoopDETR produces more accurate detection results compared to previous cooperative perception models. Specifically, the bounding boxes generated by CoopDETR exhibit better alignment with ground truth. This improvement shows our object-level fusion paradigm extracts more accurate representations of objects.

% To illustrate the perception performance of different models, Figure~\ref{fig:vis_results} shows the visualization result of two challenging scenarios in the OPV2V dataset. CoopDETR achieves more accurate detection results compared to previous cooperative perception models, such as AttFuse~\cite{xu2022opv2v}, V2X-ViT~\cite{xu2022v2xvit}, and
% Where2comm~\cite{hu2022where2comm}. 
% Specifically, the bounding boxes produced by CoopDETR are better aligned with the ground truths. The advantages likely stem from our object-level fusion paradigm, which extracts more accurate feature representations from multi-agent point features.






% \subsection{Discussion on Query Cooperation}
% \textbf{Possible extensions}


% Standing on the midpoint of instance-level result cooperation and scene-level feature cooperation, query cooperation takes both advantages of them, resulting in more possibilities to explore. Since the query stream is instance-level, it is more convenient to introduce temporal information and give the chance to model the individual motion of every single object. Leveraging temporal features, the object detection performance will be further
% boosted via spatial-temporal cooperation. Similar to single-vehicle scenario, query cooperation paradigm opens the gate to end-to-end (E2E) cooperative tracking via a spatial-temporal query stream. Furthermore, there is a wider ocean to explore, when the query stream goes beyond perception
% and flows throughout the whole pipeline, including perception, prediction, and planning. E2E cooperative driving can expand the E2E autonomous driving [12] to a system-wide improvement for intelligent transportation system





\section{CONCLUSIONS}

We propose CoopDETR, a novel query-based cooperative perception framework that leverages object queries to facilitate efficient communication and collaboration across multi-agent systems. By leveraging queries for object-level feature fusion, CoopDETR significantly reduces transmission costs while enhancing detection accuracy compared to other early fusin, late fusion, and regional-level feature fusion methods. Experiments on OPV2V and V2XSet datasets show state-of-the-art performance and robustness to pose errors, demonstrating superior performance in improving perception capability and communication efficiency for multi-agent cooperation.
Future research could focus on the integration of data from various sensing modalities and end-to-end framework~\cite{hu2023_uniad,yu2024_univ2x} that jointly optimizes perception, communication, and decision-making pipeline.

\section*{ACKNOWLEDGMENT}
This work is funded by the National Science and Technology Major Project (2022ZD0115502) and Lenovo Research.
% In this work, we propose CoopDETR, a novel cooperative perception framework that leverages object queries to facilitate efficient communication and collaboration across multi-agent systems. By introducing a query-based mechanism, CoopDETR enables flexible, instance-level feature cooperation, significantly reducing transmission costs compared to traditional dense feature-based methods. The framework incorporates two key modules: the Semantic-Spatial Query Matching (SSQM) module, which matches and associates queries from different agents, and the object query Aggregation (OQA) module, which fuses these queries for enhanced object detection. Through extensive experiments on the OPV2V and V2XSet datasets, CoopDETR has demonstrated superior performance, achieving state-of-the-art 3D detection accuracy while maintaining minimal communication overhead. Furthermore, CoopDETR proves to be robust against localization and heading errors, reinforcing its practicality in real-world applications. Our approach presents a promising direction for enhancing perception capabilities in autonomous driving systems by efficiently utilizing multi-agent cooperation.




% \addtolength{\textheight}{-12cm}   % This command serves to balance the column lengths
%                                   % on the last page of the document manually. It shortens
%                                   % the textheight of the last page by a suitable amount.
%                                   % This command does not take effect until the next page
%                                   % so it should come on the page before the last. Make
%                                   % sure that you do not shorten the textheight too much.

%%%%%%%%%%%%%%%%%%%%%%%%%%%%%%%%%%%%%%%%%%%%%%%%%%%%%%%%%%%%%%%%%%%%%%%%%%%%%%%%



%%%%%%%%%%%%%%%%%%%%%%%%%%%%%%%%%%%%%%%%%%%%%%%%%%%%%%%%%%%%%%%%%%%%%%%%%%%%%%%%



%%%%%%%%%%%%%%%%%%%%%%%%%%%%%%%%%%%%%%%%%%%%%%%%%%%%%%%%%%%%%%%%%%%%%%%%%%%%%%%%


% \section*{ACKNOWLEDGMENT}
% This work is funded by the National Key R\&D Program of China (2022ZD0115502) and Lenovo Research.

% %%%%%%%%%%%%%%%%%%%%%%%%%%%%%%%%%%%%%%%%%%%%%%%%%%%%%%%%%%%%%%%%%%%%%%%%%%%%%%%%

% References are important to the reader; therefore, each citation must be complete and correct. If at all possible, references should be commonly available publications.

\bibliographystyle{IEEEtran}
\balance
\bibliography{IEEEexample}





\end{document}
