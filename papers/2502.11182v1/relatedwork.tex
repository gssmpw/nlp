\section{Related Work}
Programmable metasurfaces constitute promising technique as a benefit of their high energy efficiency, low hardware cost, wide coverage and convenient deployment~\cite{you2020channel}. Specifically, a programmable metasurface is composed of novel two-dimensional metamaterials, where a controllable electromagnetic (EM) field can be formed by intelligently reconfiguring the EM waves on the programmable metasurfaces. A typical application of programmable metasurfaces is the popular reconfigurable intelligent surface (RIS)~\cite{li2022reconfigurable}, \cite{li2024low}. Briefly, the RIS is a specific type of programmable metasurface, typically composed of metamaterial unit cells, which are designed to manipulate EM waves in various ways. While reflective elements are commonly studied, the RIS can also support other functionalities, such as absorptive, refractive, and simultaneous transmission and reflection (STAR)~\cite{li2018metasurfaces}. These diverse functionalities make the RIS a versatile tool in optimizing wireless communication systems. The channel environment can be beneficially ameliorated by adjusting the phase shift of the impinging signal on each reconfigurable element. However, the transmitted signal is substantially attenuated by the two-hop path-loss of the transmitter-RIS and the RIS-receiver links~\cite{pan2021reconfigurable}, \cite{pan2022overview}. As a design alternative, the attractive concept of holographic MIMO has been proposed. In contrast to the RIS, which plays the role of a passive relay conceived for `reconfiguring' the propagation environment, the holographic surfaces act as a reconfigurable antenna array at the base stations (BSs). The holographic beamforming architecture relies on a programmable metasurface paradigm in support of improved spectral efficiency and reduced power consumption. This ambitious objective is achieved by harnessing a spatially near-continuous aperture and holographic radios, thus significantly reducing the power consumption and fabrication cost~\cite{huang2020holographic}, \cite{li2024achievable}, \cite{yoo2023sub}, \cite{deng2023reconfigurable}, \cite{gong2024holographic}.

In~\cite{deng2021reconfigurable_tvt}, \cite{deng2022hdma}, \cite{deng2022reconfigurable_twc}, \cite{hu2022holographic}, Deng \textit{et al.} proposed a novel hybrid beamforming architecture relying on a special leaky-wave antenna constituted by a reconfigurable holographic surface (RHS), where the digital beamformer and the holographic beamformer are optimized alternately for maximizing the achievable sum-rate. Specifically, the digital beamformer is designed based on the classical zero-forcing (ZF) precoding method, while the holographic beamformer is optimized based on the control of the amplitude response of the RHS elements. It was demonstrated that the RHS-based hybrid beamformer improves the sum-rate, while reducing the hardware cost, compared to the conventional hybrid digital-analog beamforming architecture relying on phase shifters. In~\cite{deng2022holographic}, an RHS-based beamformer was employed in low-Earth-orbit (LEO) satellite communications for improving the channel's power gain. Furthermore, the sum-rate comparison of the conventional phased array and of an RHS-aided system was presented in~\cite{hu2023holographic}. To reduce the pilot overhead required for channel state information (CSI) acquisition, Wu \textit{et al.}~\cite{wu2024two} proposed a two-timescale beamformer architecture, where the holographic beamformer was optimized based on the statistical CSI, and then the instantaneous CSI of the equivalent channel links was estimated and utilized for the digital beamformer design.

Moreover, the holographic beamformer can also be implemented using a dynamic metasurface antenna (DMA), which comprises multiple microstrips, with each microstrip containing numerous sub-wavelength, frequency-selective resonant metamaterial radiating elements~\cite{shlezinger2019dynamic}, \cite{you2022energy}, \cite{li2023near}. In the DMA, the beamforming design is achieved by linearly combining the radiation observed from all metamaterial elements within each microstrip. The mathematical framework for DMA-based massive MIMO systems was initially proposed by Shlezinger \textit{et al.} in~\cite{shlezinger2019dynamic}, where the fundamental limits of DMA-assisted uplink communications were also explored. In~\cite{you2022energy}, You \textit{et al.} optimized the energy efficiency of the DMA-based massive MIMO system using the Dinkelbach transform, alternating optimization, and deterministic equivalent techniques. Additionally, Li \textit{et al.}~\cite{li2023near} proposed a power-efficient DMA operating at high frequencies, enabling the implementation of extremely large-scale MIMO (XL-MIMO) schemes.

The above beamforming architecture is based on a single-layer metasurface. To further improve both the spatial-domain gain and the beamformer's degree-of-freedom, the authors of~\cite{an2023stacked}, \cite{an2023stacked_icc}, \cite{lin2024stacked} proposed a holographic beamforming paradigm relying on stacked intelligent metasurfaces (SIM) to carry out advanced signal processing directly in the native EM wave regime without a digital beamformer. Specifically, the SIM is composed of stacked reconfigurable multi-layer surfaces, and the phase shifts of the reconfigurable elements found in each layer can be appropriately adjusted for designing the holographic beamformer. This multi-layer architecture not only increases the number of controllable parameters but also allows for hierarchical beamforming, enabling fine-grained manipulation of electromagnetic waves. As a result, the SIM can perform advanced signal processing directly in the native electromagnetic (EM) wave regime, eliminating the need for a digital beamformer and significantly improving the beamforming resolution and flexibility~\cite{an2023stacked}, \cite{an2023stacked_icc}, \cite{lin2024stacked}. In~\cite{an2023stacked}, the gradient descent algorithm was employed for optimizing the SIM phase shifts to maximize the achievable sum-rate, and it was shown that the SIM-based beamforming architecture outperforms its single-layer metasurface based counterparts. In~\cite{an2023stacked_icc}, an alternating optimization method was designed for jointly optimizing the power allocation and SIM-based holographic beamformer in the multi-user multiple-input and single-output (MISO) downlink. Specifically, in each iteration the transmit power allocated to users is based on the classical water-filling algorithm, while the optimization of the SIM phase shift is based on the projected gradient ascent or successive refinement method. Furthermore, in~\cite{lin2024stacked} the SIM technology was leveraged for LEO satellite communication systems. Considering the challenges of acquiring the CSI between the LEO satellite and the ground users, the SIM phase shifts were optimized for maximizing the ergodic sum-rate based on statistical CSI.