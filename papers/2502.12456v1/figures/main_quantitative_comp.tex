\begin{figure*}[t]
	\centering
 \vspace{-6mm}
	\includegraphics[width=1.0\linewidth]{figures/main_quantitative_comp.png}

        \vspace{-4mm}
	\caption{
Quantitative comparisons of generation quality for different training paradigms using 1-NNA-CD (top) and 1-NNA-EMD (bottom) for Chair (left), Airplane (middle), and Car (right).
%
We present evaluation metrics across various inference steps, \ie, from 1 steps to 100 steps, for five methods: (i) ours, (ii) diffusion model with v-prediction~\cite{salimans2022progressive}, and three flow matching models with different coupling methods: (iii) independent coupling~\cite{lipman2022flow}, (iv) Minibatch OT ~\cite{tong2023improving,pooladian2023multisample}, and (v) Equivariant OT~\cite{song2024equivariant,klein2024equivariant}.
% (iii) flow matching objective with independent coupling~\cite{lipman2022flow}, (iv) flow matching objective with Minibatch OT~\cite{pooladian2023multisample,tong2023improving}, and (v) flow matching objective with Equivariant OT~\cite{klein2024equivariant,song2024equivariant}.
%
Note that values closer to $50\%$ indicate better performance. 
% Exact numerical values for the plots are provided in the Appendix. 
% \av{it would be great if you could make the font sizes on the plots larger so that we can read the text (i.e., legends and axis labels) without zooming in a lot.} \ednote{The legnend is stil slightly small. Let me try later.}
}
\label{fig:main_quantitative_comp}
\end{figure*}