\rebuttal{
\section{Additional Quantitative Comparison}
\begin{figure*}[t]
	\centering
 \vspace{-6mm}
	\includegraphics[width=1.0\linewidth]{appendix/figures/main_quantitative_comp_cov.png}

        \vspace{-4mm}
	\caption{
\rebuttal{
Quantitative comparisons of generation diversity for different training paradigms using COV-CD (top) and COV-EMD (bottom) for Chair (left), Airplane (middle), and Car (right).
%
We present evaluation metrics across various inference steps, \ie, from 1 steps to 100 steps, for five methods: (i) ours, (ii) diffusion model with v-prediction~\cite{salimans2022progressive}, and three flow matching models with different coupling methods: (iii) independent coupling~\cite{lipman2022flow}, (iv) Minibatch OT ~\cite{tong2023improving,pooladian2023multisample}, and (v) Equivariant OT~\cite{song2024equivariant,klein2024equivariant}.
%
Note that a higher value indicates better performance. 
}
}
\label{fig:main_quantitative_comp_cov}
\end{figure*}
\begin{figure*}[t]
	\centering
 \vspace{-6mm}
	\includegraphics[width=1.0\linewidth]{appendix/figures/external_quantitative_comp_cov.png}

        \vspace{-2mm}
	\caption{
\rebuttal{
Quantitative comparisons with other point cloud generation methods using the COV-CD (top) and COV-EMD metrics (bottom) for Chair (left), Airplane (middle), and Car (right).
%
We present evaluation metrics across various inference steps,~\ie, from 1 step to 100 steps, for five methods: (i) ours, (ii) PVD~\cite{zhou2021pvd}, (iii) LION~\cite{zeng2022lion}, (iv) PSF~\cite{wu2023psf} without rectified flow, and (v) DiT-3D~\cite{mo2023dit3d}.
}
%
% Note that values closer to $50\%$ indicate better performance. 
% Exact numerical values for the plots are provided in the Appendix. 
% \av{it would be nice if we could use large font sizes for legends, axis labels, figure titles.}
 }
\label{fig:external_quantitative_comp_cov}
\end{figure*}
\begin{table}[t]
\caption{
We provide evaluation results (1-NNA-CD and 1-NNA-EMD) using 1000 inference steps for DiT-3D~\cite{mo2023dit3d}, PSF~\cite{wu2023psf} without rectified flow, PVD~\cite{zhou2021pvd}, and LION~\cite{zeng2022lion}. The best-performing method is highlighted in red, while the second-best is shown in blue.}
\centering
\resizebox{1.0\linewidth}{!}{
\begin{tabular}{c|cc|cc|cc}
\toprule

\multicolumn{1}{l}{} & \multicolumn{2}{c}{Chair}                                       & \multicolumn{2}{c}{Airplane}                                    & \multicolumn{2}{c}{Car}                                          \\ \hline
\multicolumn{1}{l}{} & 1-NNA-CD                       & 1-NNA-EMD                      & 1-NNA-CD                       & 1-NNA-EMD                       & 1-NNA-CD                       & 1-NNA-EMD                      \\ \hline
DiT-3D               & 0.6072                         & 0.5604                         & -                              & -                              & -                              & -                              \\
PSF                  & 0.5612                         & 0.5642                         & 0.7457                         & 0.6617                         & 0.5682                         & 0.5412                         \\
PVD                  & 0.5626                         & \cellcolor[HTML]{DAE8FC}0.5332 & 0.7382                         & 0.6481                         & \cellcolor[HTML]{DAE8FC}0.5455 & 0.5383                         \\
LION                 & \cellcolor[HTML]{FFCCC9}0.537  & \cellcolor[HTML]{FFCCC9}0.5234 & \cellcolor[HTML]{FFCCC9}0.6741 & \cellcolor[HTML]{FFCCC9}0.6123 & \cellcolor[HTML]{FFCCC9}0.5341 & \cellcolor[HTML]{FFCCC9}0.5114 \\
Ours                 & \cellcolor[HTML]{DAE8FC}0.5551 & 0.5763                         & \cellcolor[HTML]{DAE8FC}0.6864 & \cellcolor[HTML]{DAE8FC}0.6185 & 0.5966                         & \cellcolor[HTML]{DAE8FC}0.5355 \\
\bottomrule
\end{tabular}
}
\label{tab:quantitative_final}
\end{table}
\subsection{Coverage Metric (COV)}
Following~\citet{zhou2021pvd,zeng2022lion}, we also evaluate coverage (COV), a metric that measures the diversity of generated 3D shapes. 
%
COV calculates the proportion of testing shapes that can be retrieved by generated shapes, with higher values indicating higher diversity. However,~\citet{yang2019pointflow,zhou2021pvd,zeng2022lion,wu2023psf} have noted that this metric is not robust as training set shapes can have worse COV than generated results. 
%
Moreover,~\citet{yang2019pointflow} suggest that perfect coverage scores are possible even when distances between generated and testing point clouds are arbitrarily large. 
%
Given these limitations, COV should be considered only as a reference metric, while 1-NNA provides a more reliable measure that captures both generation quality and diversity.

\para{Evaluation Results.} 
We present quantitative comparisons with different baselines in Figure~\ref{fig:main_quantitative_comp_cov} and~\ref{fig:external_quantitative_comp_cov}. 
%
Our approach generates shapes with reasonable diversity even with a limited number of steps (10-20), demonstrating the framework's effectiveness.
%
With sufficient inference steps (100), our approach achieves comparable performance to other baselines.
%
However, we observe contradictory conclusions between 1-NNA and COV metrics (as shown by simultaneously high 1-NNA and COV scores for Equivariant OT in the Airplane category), which aligns with the previously discussed limitations of the COV metric.

\subsection{More Inference Steps}
In addition to the baseline comparisons presented in the main paper (Figures~\ref{fig:external_quantitative_comp} and~\ref{fig:external_qualitative_comp}), we provide additional comparisons with 1000 inference steps, matching the original settings used by the baseline methods.
%
We employ the DDPM sampler~\cite{ho2020denoising} rather than the DDIM sampler~\cite{song2019generative} in this experiment and refer to the original values of PVD and LION reported in~\cite{zeng2022lion}.

\para{Evaluation Results.} 
We present the evaluation results based on 1-NNA-CD and 1-NNA-EMD in Table~\ref{tab:quantitative_final}. 
%
Our method achieves comparable performance with PVD~\cite{zhou2021pvd}, which also directly generates point clouds. 
%
When compared to LION~\cite{zeng2022lion}, which generates latent point cloud representations, our framework performs slightly worse.
%
Note it is known that at a high sampling budget, SDE-based samplers often outperform ODE-based samples (see~\cite{karras2022elucidating} and~\cite{xu2023restart}).

}