
\subsection{Proof for Satisfaction of Marginal Constraints.}
% In this section, we will first show that our procedure satisfying the marginal conditions for our coupling $q(\rvx_0, \rvx_1)$:
% \begin{equation}
%     \int q(\rvx_0, \rvx_1) d\rvx_1 = q_0(\rvx_0), \int q(\rvx_0, \rvx_1) d\rvx_0 = q_1(\rvx_1).
% \end{equation}

% \begin{itemize}
%     \item For independent couple $q(x_0) = \int q(\mathcal{S}) \int q(x_1 | \mathcal{S}) q(x_0) dx_0 d_\mathcal{S}$ and $q(x_1) = \int q(\mathcal{S}) \int q(x_1 | \mathcal{S}) q(x_0) dx_1 d_\mathcal{S}$, we just need to show $\int q(x_0, x_1 | \mathcal{S}) dx_0 = q(x_1 | \mathcal{S})$ and $\int q(x_0, x_1 | \mathcal{S}) dx_1 = q(x_0)$.
%     \item $q(x_0, x_1 | \mathcal{S})$ is independent, so we can decompose it into $\prod q(x_0^i, x_1^i | \mathcal{S})$.
%     \item we can show $\int q(x_0^i, x_1^i | \mathcal{S}) dx_0 = q(x_1^i | \mathcal{S})$ and $\int q(x_0^i, x_1^i | \mathcal{S}) dx_0 = q(x_1^i)$
%     \item $q(x_0| \mathcal{S})$ and $q(x_1| \mathcal{S})$ are independent, so we can decompose it into $\prod q(x_1^i | \mathcal{S})$ and $\prod q(x_0^i)$.
%     \item The first part is done.
%     \item The second part is to show adding noise will not affect $q(x_0^i)$

% \end{itemize}

% In particular, the proof will be divided into four parts.
% %
% First, we will introduce the main theorem to apply to obtain the results, and show the random subsampling of a Dense Gaussian noise will converge to Gaussian distribution if the sample superset is large enough.
% %
% Second, by a proper construction, we can show that subsampling of a dense point superset can converge to direct subsampling of the surfaces when the size of the superset is also large enough.
% %
% Third, by considering our random subsampling procedure, we can show that our sampling is still random subsampling for Gaussian noise superset and point superset.
% %
% Lastly, we show that even introduction of the coupling interpolation, the results mariginal remain the same due to careful considerations.

\newtheorem{proposition}{Proposition}
\newtheorem{lemma}{Lemma}
\subsubsection{Law of Large Numbers}


\begin{proposition}\label{prop:large_samples}
Given $(X_1, \cdots, X_n)$, which are independently and identically distributed (IID) real $d$-diemsnion random variables, following a probability distribution $p(X)$,~\ie, $X_i \sim p(X), X \in \mathbb{R}^d$.
%
We have an additional random variable $Y$ that is random uniform sample of these variables,~\ie, $P(Y = X_i) = \frac{1}{n}$.
%
The cumulative distribution function (CDF) $\bar{F}(t)$ of random variable $Y$ will converge to the $F(X)$,~\ie, CDF of $X$.
\end{proposition}



% Assume $(X_1, \cdots, X_n)$ are independently and identically distributed (IID) real $d$-diemsnion random variables following a probability distribution $p(X)$, \ie, $X_i  \sim p(X), X \in \mathbb{R}^d$.
% %
% We also denote the cumulative distribution function of $p(X)$ to be $F(x)$.
%
Proof:
We first define an empirical cumulative distribution function $\hat{F}_n(X)$ over the random variables $(X_1, \cdots, X_n)$:
\begin{equation}
    \hat{F}_n (t) = \frac{1}{n} \sum_{i=1}^{n} \mathbf{1}_{X_i \leq t},
\end{equation}
where $\mathbf{1}_{X_i \leq t}$ is an indicator for $X_i^d \leq t^d$ for all dimensions $\{1, \cdots, d\}$.

The Glivenko–Cantelli theorem states that this empirical distribution function $\hat{F}_n(X)$ will converge to the cumulative distribution $F(X)$ if $n$ is sufficiently large:
\begin{equation}
    \textbf{sup}_{t \in \mathbb{R}^d} | \hat{F}_n(t) - F(t) | \rightarrow 0.
\end{equation}

If we have an additional random variable $Y$ that its value is a random subsample of the variables $(X_1, \cdots, X_n)$:
\begin{equation}
    P(Y = X_i) = \frac{1}{n}, \forall i = 1, 2, \cdots, n.
\end{equation}

The CDF of this variable $\bar{F}(t)$ is:
\begin{equation}
    \bar{F}(t) = P(Y \leq t) = \sum_{i=1}^{n} P(Y = X_i) \cdot \mathbf{1}_{X_i \leq t} = \frac{1}{n} \sum_{i=1}^{n} \mathbf{1}_{X_i \leq t} = \hat{F}_n(t).
\end{equation}
Therefore, the CDF of $Y$ also converges to the original underlying CDF $F(t)$ if $n$ is sufficiently large.

\begin{proposition}\label{prop:ot}
Assume we have $n$ random samples $(X_1, \cdots, X_n) \sim p_1$, and another $n$ random samples $(Y_1, \cdots, Y_n) \sim p_2$, and we are also given an arbitrary bijective map between random variables, \ie, $\Pi: \{1, \cdots, n\} \leftrightarrow \{1, \cdots, n\}$.
%
If we construct a new random variable $Z : (X, Y)$ follows the following couplings:

\[
    P(X = X_i, Y = Y_j) =
    \begin{cases}
    \frac{1}{n}, & \text{if } j = \Pi(i) \\ 
        0, & \text{else } j \neq \Pi(i);
    \end{cases}
\]

The CDF of the marginal $P(X)$ will converge the CDF of $p_1$, while the CDF of the marginal $P(Y)$ will converge to the CDF of $p_2$.
\end{proposition}

Proof:
Since $\Pi$ is bijective, we can compute the marginal $P(X = X_i)$ directly:
\begin{equation}
    \begin{split}
            P(X = X_i) = \sum_{j=1}^{n} P(X = X_i, Y = Y_j) \\
            = P(X = X_i, Y = Y_{\Pi(i)}) + \sum_{j \neq \Pi(i)} P(X = X_i, Y = Y_j) \\
            = \frac{1}{n} + 0 = \frac{1}{n}
    \end{split}
\end{equation}

Similarly, we can show the marginal of P(Y) is also $\frac{1}{n}$.
%
By leveraging Proposition~\ref{prop:large_samples}, we show that $P(X)$ will converge the CDF of $p_1$, and the CDF of $P(Y)$ will converge to the CDF of $p_2$.

% \begin{lemma}\label{lemma:independent}
% The Gaussian noises $x_0$ are independently and identically distributed (IID), \ie, $q_0(x_0) = \prod_{i}^N q_0(x_0^i)$, where $x^i_0$ is the $i$-th noises and $x^i_0 \sim q_0$ .
% %
% Also, the point cloud $x_1$ given a 3D shape $S$ is also independently and identically distributed (IID), \ie, $q_1(x_1|S) = \prod_{i}^N q_1(x_1^i | S)$, where $x^i_1$ is the $i$-th point and $x^i_0 \sim q_{1|S}$.
% %
% Lastly, the training pair $(x_0, x_1)$ from our coupling  given a shape $S$ is also independently and identically distributed (IID), \ie, $q(x_0, x_1 | S) = \prod_{i}^N q(x_0^i, x_1^i | S)$, where $(x_0^i, x_1^i$) is the $i$-th pair in the training pair.
% \end{lemma}

% \begin{lemma}\label{lemma:joint}
%     The sample distribution of a point $x_1^i$ involves modeling of underlying shape $S$ and the modeling of the point distribution given $S$, \ie, $q_1(x_1^i) = \int q_1(x_1^i | S) q(S) dS$.
%     %
%     However, the distribution of noises $q_0(x^i_0)$ is unrelated to a given shape $S$, \ie, $q_0(x^i_0 | S) = q_0(x^i_0)$.
% \end{lemma}

% By considering the $p(X)$ be a Gaussian distribution $N(0, I)$ or a sampling distribution of 3D points given a Shape $\mathcal{S}$, \ie, $q(x|\mathcal{S})$, we can show the random sample $Y$ still follows the original distribution.

% If we consider $M$ random variables, where each of them is an 3D Gaussian noise, denoted as $\epsilon_i \sim N(0, I), \epsilon_i \in \mathbb{R}^3$.
% %
% We also define another variable $\epsilon$ is a random sample of these random variables, \ie, $P(\epsilon = \epsilon_i) = \frac{1}{M}$.
% %
% Since each dimension in $\epsilon$ is independent, CDF of $\epsilon^j$ will follows the by leverage the above results, where $j$ is the j-th dimension of the noise.
% We consider a dense 3D Gaussian noises with $M \times 3$ random variables, $\{x_1, y_1, z_1, \cdots, x_M, y_M, z_M\}$, where we denote $x_i$, $y_i$, and $z_i$ to be the coordinates of in x, y, and z dimensions, respectively and $x_i, y_i, z_i \sim N(0, I)$.
% %
% If we can consider a random variable $\hat{x}$, which is random sample of this dense gausian in x dimension, \ie,  P$(\hat{x} = x_i) = \frac{1}{M}$.
% %
% By the above results, the CDF follows the original distribution, which is the Gaussian distribution $N(0, I)$.
% %
% By considering also y and z dimension, we can show that a random sampling of Gaussian point converge to Gaussian distribution.
\newtheorem{theorem}{Theorem}
\subsubsection{Proof of Our OT Approximation}
\label{subsec:our_ot_proof}

We first give a definition of coupling $q(x_0, x_1)$ in our case before showing its marginal fullfils the marginal requirements.
%
In particular, we denote $x_0 \in R^{N \times 3}$ and $x_1 \in R^{N \times 3}$ as two random variables following the distributions, $q_0(x_0)$ and $q_1(x_1)$, respectively.
%
It is noted that $q_0 := N(0, I)$, which is the standard Gaussian for each dimension in $x_0$, and $q_1(x_1)$ is the distribution all possible point clouds, which involves the joint modeling of point cloud distribution given a shape $S$ ($q_{1}(x_1|S)$) and the distribution of shape ($q(S)$), \ie, $q_1(x_1) = \int q_{1}(x_1|S) q(S) dS$.
%

We can notice that each row in $x_0$ is independently and identically distributed (IID), \ie, $q_0(x_0) = \prod_{i}^N \hat{q_0}(x_0^i)$, where we denote the $i$-th row of $x_0$ as $x_0^i$ and distribution of $x_0^i$ as $\hat{q_0}(x_0^i)$, which is 3-dimensional unit Gaussian.
%
We can also assume each point in $x_1$ is IID given a shape, \ie, $q_{1}(x_1 | S) = \prod_{i}^N \hat{q_{1}}(x_1^i|S)$,  where we denote the $i$-th row of $x_1$ as $x^i_1$ and the distribution of $x^i_1$ as $\hat{q_{1}}(x_1^i|S)$. 

In our superset OT precomputation for a given shape $S$, we pre-sample a set of random variables $(x^1_0 \cdots, x^j_0, \cdots, x^M_0) \sim \hat{q}_0$, and a set of random variables  $(x^1_1, \cdots, x^k_1,\cdots, x^M_1) \sim \hat{q}_1$, and have a precomputed bijective mapping $\Pi : \{1, \cdots, M\} \leftrightarrow \{1, \cdots, M\}$.
%
With these defined, our coupling $\hat{q}(x^i_0, x^i_1 |S)$ for one row in the training pair $(x^i_0, x^i_1)$ given $S$ can be formulated as:
\[
    \hat{q}(x^i_0 = x^j_0, x^i_1 = x^k_1 | S) =
    \begin{cases}
    \frac{1}{n}, & \text{if } j = \Pi(k) \\ 
        0, & \text{else } j \neq \Pi(k);
    \end{cases}
\]
%
Since the each row in the training pairs are independently subsampled, the coupling of the training pair $(x_0, x_1)$ given a shape is defined as $q(x_0, x_1 |S) = \prod_{i}^N \hat{q}(x_0^i, x_1^i | S)$.
%
In the end, the coupling over all training pairs can be obtained by marginalize over all possible shapes, \ie, $\int q(x_0, x_1 | S) q(S) dS$.

\begin{theorem}

% Our coupling $q(x_0, x_1)$ for a given Gaussian noise $x_0 \in R^{N \times 3}$ and a given point cloud $x_1 \in R^{N \times 3}$

Our coupling without blending converge the following marginal if the superset size $M$ is sufficiently large:
\begin{equation}\label{eq:mariginals}
    \int q(\rvx_0, \rvx_1) d\rvx_1 = q_0(\rvx_0), \int q(\rvx_0, \rvx_1) d\rvx_0 = q_1(\rvx_1).
\end{equation}
\end{theorem}

Proof:
We first show the left constraint:
% \begin{equation}
\begin{align}
LHS & = \int q(x_0, x_1) dx_1 = \int \int q(x_0, x_1 | S) q(S) dS dx_1  \\
& = \int q(S) \int q(x_0, x_1 | S) dx_1 dS && \text{change the order of integration} \\
& = \int q(S) \int \prod_i^N \hat{q}(x_0^i, x_1^i|S) d(x_1^1, \cdots, x_1^N) dS  && \text{independent assumption of each row in training pair}\\
& = \int q(S) \prod_i^N \int \hat{q}(x_0^i, x_1^i|S) dx_1^j dS && \text{integrals of independent products}\\
& = \int q(S) \prod_i \sum_k^M \hat{q}(x_0^i, x_1^k|S) dS && \text{restricting to discrete values in supersets}\\
& = \int q(S) \prod_i \hat{q}_0(x^i_0) dS && \text{Proposition~\ref{prop:ot}}\\
& = \int q(S) q_0(x_0) dS = q_0(x_0) && \text{independent assumption of each row in Gaussian noises} \\
\end{align}
% \end{equation}

Similarly, we perform the same computation on the right constraint:
% \begin{equation}
\begin{align}
LHS & = \int q(x_0, x_1) dx_0 = \int \int q(x_0, x_1 | S) q(S) dS dx_0   \\
 & = \int q(S) \int q(x_0, x_1 | S) dx_0 dS && \text{change the order of integration} \\
& = \int q(S) \int \prod_i^N \hat{q}(x_0^i, x_1^i|S) d(x_0^1, \cdots, x_0^N) dS && \text{independent assumption of each row in training pair} \\
& = \int q(S) \prod_i^N \int \hat{q}(x_0^i, x_1^i|S) dx_0^i dS  && \text{integrals of independent products} \\
& = \int q(S) \prod_i \sum_j^M \hat{q}(x_0^j, x_1^i|S) dS 
 && \text{restricting to discrete values in supersets} \\
& = \int q(S) \prod_i \hat{q}_1(x^i_1 | S) dS  && \text{Proposition~\ref{prop:ot}} \\
& = \int q(S) q_1(x_1 | S) dS = q_1(x_1) && \text{independent assumption of each row in point cloud} \\
\end{align}
% \end{equation}


% We first consider the RHS of Left Constraints (Equation~\ref{eq:mariginals}), we can reformulate it as follows:
% \begin{equation}
%     \begin{split}
%             RHS = q_0(x_0) = \int q(S) q_0(x_0 | S) dS = \int q(S) q_0(x_0) dS \\
%             % = \int q_0(x_0) (\int q_1(x_1 |S) q(S) dS) dx_1 \text{, by Lemma~\ref{lemma:joint}} \\
%             % = \int q(S) \int q_0(x_0) q_1(x_1|S) dx_1 d_S \text{, by rearranging the integrals} \\
%     \end{split}
% \end{equation}
% Considering LHS:
% \begin{equation}
%     \begin{split}
%         LHS = \int q(x_0, x_1) dx_1 = \int \int q(x_0, x_1 | S) q(S) dS dx_1 \\
%         = \int q(S) \int q(x_0, x_1 | S) dx_1 dS
%     \end{split}
% \end{equation}

% By comparing LHS and RHS, it is sufficient to show that $\int q(x_0, x_1 |S) dx_1 = q_0(x_0)$ for the first constraint.
% Similarly, for the second constraint RHS:
% \begin{equation}
%     \begin{split}
%             RHS = q_1(x_1) = \int q(S) q_1(x_1|S) dS \\
%             % = \int q_0(x_0) (\int q_1(x_1 |S) q(S) dS) dx_1 \text{, by Lemma~\ref{lemma:joint}} \\
%             % = \int q(S) \int q_0(x_0) q_1(x_1|S) dx_1 d_S \text{, by rearranging the integrals} \\
%     \end{split}
% \end{equation}
% Considering LHS:
% \begin{equation}
%     \begin{split}
%         LHS = \int q(x_0, x_1) dx_0 = \int \int q(x_0, x_1 | S) q(S) dS dx_0 \\
%         = \int q(S) \int q(x_0, x_1 | S) dx_0 dS
%     \end{split}
% \end{equation}
% Then it is sufficient to show $\int q(x_0, x_1 | S) dx_0 = q_1(x_1|S) $.

% To show first equation, we can apply Lemma~\ref{lemma:independent}:
% \begin{equation}
% \label{eq:left_LHS}
%     \begin{split}
%         LHS = \int q(x_0, x_1 | S) dx_1 = \int \prod_i q(x_0^i, x_1^i | S) d(x_1^i, \cdots, x_1^N) \\
%         = \prod_i \int q(x_0^i, x_1^i|S) dx_1^i 
%     \end{split}
% \end{equation}

% \begin{equation}
% \label{eq:left_RHS}
%     RHS = q_0(x_0) = \prod_i q_0(x^i_0)
% \end{equation}
% By this computation, we are also sufficient to show $\int q(x_0^i, x_1^i | S) dx_1^i = q_0(x_0^i)$ and by similar computation:
% \begin{equation}
% \label{eq:right_LHS}
%     \begin{split}
%         LHS = \int q(x_0, x_1 | S) dx_0 = \int \prod_i q(x_0^i, x_1^i | S) d(x_0^i, \cdots, x_0^N) \\
%         = \prod_i \int q(x_0^i, x_1^i|S) dx_0^i 
%     \end{split}
% \end{equation}

% \begin{equation}
% \label{eq:right_RHS}
%     RHS = q_1(x_0|S) = \prod_i q_1(x^i_1|S)
% \end{equation}
% Therefore, we are sufficient to show $\int q(x^i_0, x^i_1) dx_0^i = q_1(x_1^i |S)$.

% By considering the fact that, we pre-sample a set of random variables $(x^1_0 \cdots, x^j_0, \cdots, x^M_0) \sim q_0$, and a set of random variables  $(x^1_1, \cdots, x^k_1,\cdots, x^M_1) \sim q_{1|S}$, and have a precomputed bijective mapping $\Pi : \{1, \cdots, M\} \leftrightarrow \{1, \cdots, M\}$.
% %
% With these defined, our coupling $q(x^i_0, x^i_1 |S)$ given $S$ can formulated as:
% \[
%     P(x^i_0 = x^j_0, x^i_1 = x^k_1) =
%     \begin{cases}
%     \frac{1}{n}, & \text{if } k = \Pi(j) \\ 
%         0, & \text{else } k \neq \Pi(j);
%     \end{cases}
% \]
% By Proposition~\ref{prop:ot}, if the superset size $M$ is large enough, we can show that the CDF of Equation~\ref{eq:left_LHS} converge to Equation~\ref{eq:left_RHS}, also the CDF of Equation~\ref{eq:left_LHS} converges to Equation~\ref{eq:left_RHS}.

% To show our coupling maintain the correct marginal, we assume we have $M$ random variables $(X_1, \cdots, X_M) \sim p_1$, and another $M$ random random variables $(Y_1, \cdots, Y_M) \sim p_2$.
% %
% We can additionally take an arbitrary bijective map $\Pi$ between random variables, \ie, $\Pi : \{1, \cdots, M\} \leftrightarrow \{1, \cdots, M\}$.
% %
% If we only sample the a pair of variables based on the bijective map, we can then construct a new random Variable $Z: \{X, Y\}$:
% \[
%     P(X = X_i, Y = Y_j) =
%     \begin{cases}
%     \frac{1}{M}, & \text{if } j = \Pi(i) \\ 
%         0, & \text{else } j \neq \Pi(i);
%     \end{cases}
% \]

% Since $\Pi$ is a bijective mapping, the mariginal distribution of $P(X = X_i)$ and $P(Y = Y_j)$ is also $\frac{1}{M}$.
% %
% Following the result in the previous section, we can show the random variable $X$ ($Y$) still follows $p_1$ ($p_2$).
% %
% In our case, we consider $p_1$ to be a 3D Gaussian distribution, and $p_2$ to be point sample distribution given a Shape $\mathcal{S}$.

% The last part we need to show is that $q_0(x_0)$ and $q_1(x_1|\mathcal{S})$ is independently sampled for each of the point, \ie, $q_0(x_0) = \prod_{i} q_0(x_0^i)$ and \ie, $q_1(x_1) = \prod_{i} q_1(x_1^i | \mathcal{S})$, where $x_0^i$ and $x_1^i$ is the $i$-th point in $x_0$ and $x_1$, respectively.
% %
% For Gaussian distribution $q_0(x_0)$, this is true because it is an unit Gaussian distribution.
% %
% For surface point distribution $q_1(x_1|S)$, it is also correct since the points are indepdently sampled.



% Additionally, for a Gaussian noise sets arranged in the matrix format, $x_0 \in \mathbb{R}^{N \times 3}, x_0 \sim$
\subsubsection{Proof of Hybrid Coupling}

In the last, we would like to show even with our hybrid coupling, the marginal still fulfills the requirements.
%
In particular, we define a new noises $x_0'$ after perturbation:
\begin{equation}
    x_0' = \sqrt{1 - \beta} x_0 + \sqrt{\beta} \epsilon, \epsilon \sim N(\epsilon; 0, I),
\end{equation}
where $\beta \in [0, 1]$ is the blending coefficient. We denoted this as a conditional distribution $q(x_0'| x_0)$, which has a form of $N(x_0'; \sqrt{1 - \beta}x_0, \beta)$.
%
It is noted that since $\epsilon \sim N(\epsilon, 0, I)$, each row of $x'_0$ is also IID given $x_0$, \ie, $q_0(x_0' | x_0) = \prod_i^N \hat{q}_0(x_0^{'i} | x_0^i)$.
%
Due to the independent properties, it is sufficient to show that:
\begin{equation}
    \int q(x_0^{i'}, x_1^i | S) dx_0^{i'} = q_1(x^i_1|S), 
    \int q(x_0^{i'}, x_1^i | S) dx_1^{i} = q_0(x_0^i).
\end{equation}

For the sake of simplicity, we remove all the index $i$ and shape $S$ in the folloings.
We first show the left constraint:
\iffalse
\begin{align}
    q(x_1) & = \int q(x_0', x_1) dx_0' = \int \int q_0(x_0) q(x_0'| x_0) q(x_1|x_0, x_0') dx_0 dx_0' \\
    & = \int \int q_0(x_0) q(x_0'| x_0) q(x_1|x_0) dx_0 dx_0' \\
    & =  \int \int  q_0(x_0) q(x_0'| x_0) q(x_1|x_0)  dx_0' dx_0 \\
    & = \int q_0(x_0) q(x_1|x_0) \int  q(x_0'| x_0)  dx_0' dx_0 \\
    & = \int q_0(x_0) q(x_1|x_0) (1) dx_0 \\
    & = \int q(x_0, x_1) dx_0  = \frac{1}{M} \\
\end{align}
\fi
\begin{align}
    \int q(x_0', x_1) dx_0' & = \int \int q_0(x_0', x_0, x_1) dx_0 dx_0' \\
    & = \int \int q_0(x_0'|x_0) q(x_0, x_1) dx_0 dx_0' \\
    & =  \int q(x_0, x_1) \int  q_0(x_0'|x_0)  dx_0' dx_0 \\
    & = \int q(x_0, x_1) (1) dx_0 \\
    & = q(x_1)
\end{align}
By Proposition~\ref{prop:large_samples}, we can show $q(x_1)$ still converge to the right CDF if $M$ is sufficient large.

On the other hand, we show that:
\iffalse
\begin{align}
    \int q(x_0', x_1) dx_1 &= \int \int q_1(x_0' | x_0, x_1) q_0(x_0|x_1) q(x_1) dx_0 dx_1 \\
    &= \int \int q(x_0' | x_0, x_1) q(x_0|x_1) q(x_1) dx_1 dx_0 \\
    &= \int \int q(x_0' | x_0) q(x_0|x_1) q(x_1) dx_1 dx_0 \\
    & = \int q(x_0'|x_0) \int q(x_0|x_1) q(x_1) dx_1 dx_0 \\
    & = \int q(x_0'|x_0) \sum_{x_1} q(x_0, x_1) dx_0 \\
    & = \int q(x_0'|x_0) \frac{1}{M} dx_0 \\
    & = \frac{1}{M} \sum_{x_0} q(x_0' | x_0) \\
    & = \frac{1}{M} \sum_{x_0} N(x_0'; \sqrt{1 - \beta} x_0, \beta)
\end{align}
\fi
\begin{align}
    \int q(x_0', x_1) dx_1 &= \int \int q_0(x_0', x_0, x_1) dx_0 dx_1 \\
    &= \int \int q_0(x_0', x_0) dx_0\\
    &= \int \int  q_0(x_0'|x_0) q(x_0) dx_0 \\
    & = N(0, I)
\end{align}
where the last equality is obtained by inserting $q(x_0) = N(0, I)$ and $q_0(x_0'|x_0) = N(x_0'; \sqrt{1 - \beta}x_0, \beta I)$.

\iffalse
When $M \rightarrow \infty$, it becomes a convolution of two Gaussian $N(0, (1 - \beta) I)$ and $N(0, \beta I)$.
%
By convolution of Gaussian, we can observe that:
\begin{align}
    \int q(x_0', x_1) dx_1 & = N(0, (1 - \beta)I + \beta I) \\
    & = N(0, I)\\
\end{align}
\fi