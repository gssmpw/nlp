\vspace{-3mm}
\section{Introduction}
\vspace{-1mm}

% what is the problem we are trying to solve
% why is it important and what are the applications
% how do people usually solve them
% what is our solution
% what are the advantages

Generating 3D point clouds is one of the fundamental problems in 3D modeling with applications in shape generation, 3D reconstruction, 3D design, and perception for robotics and autonomous systems. Recently, diffusion models~\cite{sohl2015deep,ho2020denoising} and flow matching~\cite{lipman2022flow} have become the de facto frameworks for learning generative models for 3D point clouds.
%
%
These frameworks often overlook 3D point cloud permutation invariance, implying the rearrangement of points does not change the shape that they represent.
% However, these frameworks often overlook one of the key properties of 3D point cloud which is the permutation invariance. That is permuting points does not change the shape that they represent.  
%ignore one of the key properties of 3D point cloud which is the permutation invariance. That is permuting points does not change the shape that they represent. This property is often considered when designing neural networks for point clouds, but it is ignored in generative models which treat them as high-dimensional vectors with a fixed order.

In closely related areas, equivariant optimal transport (OT) flows~\cite{klein2024equivariant,song2024equivariant} have been recently developed for 3D molecules that can be considered as sets of 3D atom coordinates. 
These frameworks learn permutation invariant generative models, \ie, all permutations of the set have the same likelihood under the learned generative distribution.
They are trained using optimal transport between data and noise samples, yielding several key advantages including low sampling trajectory curvatures, low-variance training objectives, and fast sample generation~\cite{pooladian2023multisample}.
Albeit these theoretical advantages, our examination of these techniques for 3D point cloud generation reveals that they scale poorly for point cloud generation.
This is mainly due to the fact that point clouds in practice consist of thousands of points whereas molecules are assumed to have tens of atoms in previous studies.
Solving the sample-level OT mapping between a batch of training point clouds and noise samples is computationally expensive.
Conversely, ignoring permutation invariance when solving batch-level OT~\cite{pooladian2023multisample,tong2023improving} fails to produce high-quality OT due to the excessive possible permutations of point clouds.

\section{Overview}

\revision{In this section, we first explain the foundational concept of Hausdorff distance-based penetration depth algorithms, which are essential for understanding our method (Sec.~\ref{sec:preliminary}).
We then provide a brief overview of our proposed RT-based penetration depth algorithm (Sec.~\ref{subsec:algo_overview}).}



\section{Preliminaries }
\label{sec:Preliminaries}

% Before we introduce our method, we first overview the important basics of 3D dynamic human modeling with Gaussian splatting. Then, we discuss the diffusion-based 3d generation techniques, and how they can be applied to human modeling.
% \ZY{I stopp here. TBC.}
% \subsection{Dynamic human modeling with Gaussian splatting}
\subsection{3D Gaussian Splatting}
3D Gaussian splatting~\cite{kerbl3Dgaussians} is an explicit scene representation that allows high-quality real-time rendering. The given scene is represented by a set of static 3D Gaussians, which are parameterized as follows: Gaussian center $x\in {\mathbb{R}^3}$, color $c\in {\mathbb{R}^3}$, opacity $\alpha\in {\mathbb{R}}$, spatial rotation in the form of quaternion $q\in {\mathbb{R}^4}$, and scaling factor $s\in {\mathbb{R}^3}$. Given these properties, the rendering process is represented as:
\begin{equation}
  I = Splatting(x, c, s, \alpha, q, r),
  \label{eq:splattingGA}
\end{equation}
where $I$ is the rendered image, $r$ is a set of query rays crossing the scene, and $Splatting(\cdot)$ is a differentiable rendering process. We refer readers to Kerbl et al.'s paper~\cite{kerbl3Dgaussians} for the details of Gaussian splatting. 



% \ZY{I would suggest move this part to the method part.}
% GaissianAvatar is a dynamic human generation model based on Gaussian splitting. Given a sequence of RGB images, this method utilizes fitted SMPLs and sampled points on its surface to obtain a pose-dependent feature map by a pose encoder. The pose-dependent features and a geometry feature are fed in a Gaussian decoder, which is employed to establish a functional mapping from the underlying geometry of the human form to diverse attributes of 3D Gaussians on the canonical surfaces. The parameter prediction process is articulated as follows:
% \begin{equation}
%   (\Delta x,c,s)=G_{\theta}(S+P),
%   \label{eq:gaussiandecoder}
% \end{equation}
%  where $G_{\theta}$ represents the Gaussian decoder, and $(S+P)$ is the multiplication of geometry feature S and pose feature P. Instead of optimizing all attributes of Gaussian, this decoder predicts 3D positional offset $\Delta{x} \in {\mathbb{R}^3}$, color $c\in\mathbb{R}^3$, and 3D scaling factor $ s\in\mathbb{R}^3$. To enhance geometry reconstruction accuracy, the opacity $\alpha$ and 3D rotation $q$ are set to fixed values of $1$ and $(1,0,0,0)$ respectively.
 
%  To render the canonical avatar in observation space, we seamlessly combine the Linear Blend Skinning function with the Gaussian Splatting~\cite{kerbl3Dgaussians} rendering process: 
% \begin{equation}
%   I_{\theta}=Splatting(x_o,Q,d),
%   \label{eq:splatting}
% \end{equation}
% \begin{equation}
%   x_o = T_{lbs}(x_c,p,w),
%   \label{eq:LBS}
% \end{equation}
% where $I_{\theta}$ represents the final rendered image, and the canonical Gaussian position $x_c$ is the sum of the initial position $x$ and the predicted offset $\Delta x$. The LBS function $T_{lbs}$ applies the SMPL skeleton pose $p$ and blending weights $w$ to deform $x_c$ into observation space as $x_o$. $Q$ denotes the remaining attributes of the Gaussians. With the rendering process, they can now reposition these canonical 3D Gaussians into the observation space.



\subsection{Score Distillation Sampling}
Score Distillation Sampling (SDS)~\cite{poole2022dreamfusion} builds a bridge between diffusion models and 3D representations. In SDS, the noised input is denoised in one time-step, and the difference between added noise and predicted noise is considered SDS loss, expressed as:

% \begin{equation}
%   \mathcal{L}_{SDS}(I_{\Phi}) \triangleq E_{t,\epsilon}[w(t)(\epsilon_{\phi}(z_t,y,t)-\epsilon)\frac{\partial I_{\Phi}}{\partial\Phi}],
%   \label{eq:SDSObserv}
% \end{equation}
\begin{equation}
    \mathcal{L}_{\text{SDS}}(I_{\Phi}) \triangleq \mathbb{E}_{t,\epsilon} \left[ w(t) \left( \epsilon_{\phi}(z_t, y, t) - \epsilon \right) \frac{\partial I_{\Phi}}{\partial \Phi} \right],
  \label{eq:SDSObservGA}
\end{equation}
where the input $I_{\Phi}$ represents a rendered image from a 3D representation, such as 3D Gaussians, with optimizable parameters $\Phi$. $\epsilon_{\phi}$ corresponds to the predicted noise of diffusion networks, which is produced by incorporating the noise image $z_t$ as input and conditioning it with a text or image $y$ at timestep $t$. The noise image $z_t$ is derived by introducing noise $\epsilon$ into $I_{\Phi}$ at timestep $t$. The loss is weighted by the diffusion scheduler $w(t)$. 
% \vspace{-3mm}

\subsection{Overview of the RTPD Algorithm}\label{subsec:algo_overview}
Fig.~\ref{fig:Overview} presents an overview of our RTPD algorithm.
It is grounded in the Hausdorff distance-based penetration depth calculation method (Sec.~\ref{sec:preliminary}).
%, similar to that of Tang et al.~\shortcite{SIG09HIST}.
The process consists of two primary phases: penetration surface extraction and Hausdorff distance calculation.
We leverage the RTX platform's capabilities to accelerate both of these steps.

\begin{figure*}[t]
    \centering
    \includegraphics[width=0.8\textwidth]{Image/overview.pdf}
    \caption{The overview of RT-based penetration depth calculation algorithm overview}
    \label{fig:Overview}
\end{figure*}

The penetration surface extraction phase focuses on identifying the overlapped region between two objects.
\revision{The penetration surface is defined as a set of polygons from one object, where at least one of its vertices lies within the other object. 
Note that in our work, we focus on triangles rather than general polygons, as they are processed most efficiently on the RTX platform.}
To facilitate this extraction, we introduce a ray-tracing-based \revision{Point-in-Polyhedron} test (RT-PIP), significantly accelerated through the use of RT cores (Sec.~\ref{sec:RT-PIP}).
This test capitalizes on the ray-surface intersection capabilities of the RTX platform.
%
Initially, a Geometry Acceleration Structure (GAS) is generated for each object, as required by the RTX platform.
The RT-PIP module takes the GAS of one object (e.g., $GAS_{A}$) and the point set of the other object (e.g., $P_{B}$).
It outputs a set of points (e.g., $P_{\partial B}$) representing the penetration region, indicating their location inside the opposing object.
Subsequently, a penetration surface (e.g., $\partial B$) is constructed using this point set (e.g., $P_{\partial B}$) (Sec.~\ref{subsec:surfaceGen}).
%
The generated penetration surfaces (e.g., $\partial A$ and $\partial B$) are then forwarded to the next step. 

The Hausdorff distance calculation phase utilizes the ray-surface intersection test of the RTX platform (Sec.~\ref{sec:RT-Hausdorff}) to compute the Hausdorff distance between two objects.
We introduce a novel Ray-Tracing-based Hausdorff DISTance algorithm, RT-HDIST.
It begins by generating GAS for the two penetration surfaces, $P_{\partial A}$ and $P_{\partial B}$, derived from the preceding step.
RT-HDIST processes the GAS of a penetration surface (e.g., $GAS_{\partial A}$) alongside the point set of the other penetration surface (e.g., $P_{\partial B}$) to compute the penetration depth between them.
The algorithm operates bidirectionally, considering both directions ($\partial A \to \partial B$ and $\partial B \to \partial A$).
The final penetration depth between the two objects, A and B, is determined by selecting the larger value from these two directional computations.

%In the Hausdorff distance calculation step, we compute the Hausdorff distance between given two objects using a ray-surface-intersection test. (Sec.~\ref{sec:RT-Hausdorff}) Initially, we construct the GAS for both $\partial A$ and $\partial B$ to utilize the RT-core effectively. The RT-based Hausdorff distance algorithms then determine the Hausdorff distance by processing the GAS of one object (e.g. $GAS_{\partial A}$) and set of the vertices of the other (e.g. $P_{\partial B}$). Following the Hausdorff distance definition (Eq.~\ref{equation:hausdorff_definition}), we compute the Hausdorff distance to both directions ($\partial A \to \partial B$) and ($\partial B \to \partial A$). As a result, the bigger one is the final Hausdorff distance, and also it is the penetration depth between input object $A$ and $B$.


%the proposed RT-based penetration depth calculation pipeline.
%Our proposed methods adopt Tang's Hausdorff-based penetration depth methods~\cite{SIG09HIST}. The pipeline is divided into the penetration surface extraction step and the Hausdorff distance calculation between the penetration surface steps. However, since Tang's approach is not suitable for the RT platform in detail, we modified and applied it with appropriate methods.

%The penetration surface extraction step is extracting overlapped surfaces on other objects. To utilize the RT core, we use the ray-intersection-based PIP(Point-In-Polygon) algorithms instead of collision detection between two objects which Tang et al.~\cite{SIG09HIST} used. (Sec.~\ref{sec:RT-PIP})
%RT core-based PIP test uses a ray-surface intersection test. For purpose this, we generate the GAS(Geometry Acceleration Structure) for each object. RT core-based PIP test takes the GAS of one object (e.g. $GAS_{A}$) and a set of vertex of another one (e.g. $P_{B}$). Then this computes the penetrated vertex set of another one (e.g. $P_{\partial B}$). To calculate the Hausdorff distance, these vertex sets change to objects constructed by penetrated surface (e.g. $\partial B$). Finally, the two generated overlapped surface objects $\partial A$ and $\partial B$ are used in the Hausdorff distance calculation step.

In this paper, we propose a simple and scalable generative model for 3D point cloud generation using flow matching, coined as \textit{not-so-optimal transport flow matching}, as shown in Fig~\ref{fig:overview}.
We first propose an efficient way to obtain an approximate OT between point cloud and noise samples.
Instead of searching for an optimal permutation between point cloud and noise samples online during training, which is computationally expensive, we show that we can precompute an OT between a dense point superset and a dense noise superset offline.
Since subsampling a superset preserves the underlying shape, we can simply subsample the point superset and obtain corresponding noise from the precomputed OT to construct a batch of noise-data pairs for training the flow models.

% In this paper, we propose a simple and scalable generative model for 3D point cloud generation using flow matching. Similar to OT flow matching, we couple 3D point cloud data with 3D Gaussian noise samples, but rather than focusing on mini-batch OT coupling, we simply align noise and point cloud by permuting the points for each instance. Since sub-sampling point clouds does not change their shape, we align 3D data and noise in an offline fashion using a high-resolution representation of training 3D shapes. During training, we simply subsample the point cloud and the corresponding aligned noise to construct a batch of noise-data pairs for training 
% %flows 
% \phil{the flow models}.

\begin{wrapfigure}{r}{0.35\textwidth}
 \vspace{-2mm}%this figure will be at the right
    \centering
    \includegraphics[width=0.35\textwidth]{figures/lipchitz_vis.png}
    \vspace{-7mm}
    \caption{ 
    In the OT flow model, the vector field $\rvv_t(\rvx_0)$ admits a large change in its output with a small perturbation of $\rvx_0$ at $t{=}0$.
    }
 \label{fig:lipchitz_vis}
 \vspace{-3mm}
\end{wrapfigure}
%We demonstrate that the training pairs sampled from our approximate OT are of high quality.
%
We demonstrate that our approximate OT reduces the pairwise distance between data and noise significantly and benefits from the advantages of OT flows,~\eg, straightness of trajectories and fast sampling. 
%
However, a careful examination shows that learning (equivariant) OT flows is generally challenging since straightening flow trajectories makes the learned flows complex at the beginning of the trajectory. 
%
Intuitively, in the OT coupling, the flow model should be able to switch between different target point clouds (i.e., different modes in the data distribution) with small variations in their input, making the flow model have high Lipchitz (see Fig~\ref{fig:lipchitz_vis}).

% We show that our \textit{not-so-optimal transport} coupling in fact reduces the pairwise distance between data and noise significantly and benefits from the advantages of OT flows (e.g., straightness of trajectories and fast sampling). However, a careful examination shows that learning (equivariant) OT flows is generally challenging since straightening flow trajectories makes the learned flow model complex at the beginning of the trajectory. Intuitively, once an OT transport is established between the noise and data samples, the trained models should be able to switch between different target point clouds with small variations in their input, making the flow model have high Lipchitz as shown in Fig ??.



To remedy this, we propose a simple approach to construct a less ``optimal'' hybrid coupling by blending our approximate OT and independent coupling used in the flow matching model.
%
In particular, we suggest perturbing the noise samples obtained from our approximate OT with small Gaussian noise.
%
While this remedy makes our mapping less optimal from the OT perspective, we show that it empirically shows two main advantages: 
First, the target flow model is less complex and the generated points clouds have high sample quality. 
Second, when reducing the number of inference steps, the generation quality still degrades slower than other competing techniques, indicating smoother trajectories.

% To remedy this, we propose a simple approach that perturbs aligned noise samples with small perturbations. While this remedy makes our mapping less optimal from the optimal transport perspective, it empirically shows two main advantages: One the generated points clouds have high quality. Two, when reducing the number of sampling steps, the generation quality degrades slower than other competing techniques.

In summary, this paper makes the following contributions: (i) We show that existing OT approximations either scale poorly or produce low-quality OT for real-world point cloud generation. 
%
(ii) We show that albeit the nice theoretical advantages, equivariant OT flows have to learn a complex function with high Lipchitz at the beginning of the generation process. 
%
(iii) To tackle these issues, we propose a not-so-optimal transport flow matching approach that involves an offline superset OT precomputation and online random subsampling to obtain an approximate OT, and adds a small perturbation to the obtained noise during training.
%
(iv) We empirically compare our method against diffusion models, flows, and OT flows on unconditional point cloud generation and shape completion on the ShapeNet benchmark. 
%
We show that our proposed model outperforms these frameworks for different sampling budgets over various competing baselines on the unconditional generative task.
%
In addition, we show that we can obtain reasonable generation quality on the shape completion task in less than five steps, which is challenging for other approaches.

% In summary, this paper makes the following contributions: (i) We show that equivariant OT flow matching scales poorly for real-world point cloud generation. (ii) We show that albeit the nice theoretical advantages, equivariant OT flows have to learn a complex function with high Lipchitz at the beginning of the generation process. (iii) To tackle these issues, we propose a not-so-optimal transport flow matching approach that aligns noise with input 3D shapes by permuting it before training and adds a small perturbation to the matched noise during training. (iv) We empirically compare our method against diffusion models, flows, and OT flows on a range of problems including unconditional point cloud generation, and conditional XX on the ShapeNet benchmark. We show that our proposed model outperforms these frameworks for different sampling budgets.



% \phil{TODO: need to make sec 1 and sec 3 consistent in wordings. Also, our claims in sec 1 and 3}