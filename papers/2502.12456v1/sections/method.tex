\vspace{-2mm}
\section{Method}
\vspace{-2mm}

In Section~\ref{subsec:preliminaries}, we begin by covering preliminaries of training a continuous normalizing flow and recent OT flows.
%
Section~\ref{subsec:points_challenges} explores the challenges of applying existing OT approximation methods to 3D point clouds.
% In Section~\ref{subsec:points_challenges}, we then examine the nature of 3D shapes represented as point clouds and 
% study the challenges of directly applying existing OT approximation methods.
%
To tackle these challenges, we introduce our approximate OT approach in Section~\ref{subsec:ot_approx} that precomputes OT maps in an offline fashion, and in Section~\ref{subsec:coupling_interpolation}, we explore a simple hybrid and less optimal coupling approach that makes target flows easier to learn.
%
% Using a more optimal map can lead to a straighter path, but the trained flow model becomes more Lipschitz in earlier steps, resulting in poorer performance.
%worse generation results.
% doing so makes the training flows more challenging \phil{TODO: any better word than challenging? which sounds vague here}.
%
%To mitigate this difficulty
%Concerning this, We introduce a hybrid coupling method, blending OT map with independent coupling, for a straighter generation path while maintaining quality (Section~\ref{subsec:coupling_interpolation}).
% Concerning this, we introduce a method to construct a less ``optimal'' hybrid coupling by blending our OT map with independent coupling, enabling a relatively straight generation path while maintaining high-quality generation (Section~\ref{subsec:coupling_interpolation}).
% \phil{TODO: instead of interpolation, how about ``we introduce a method to blend between the OT map and independent coupling''?  The word ``interpolation'' seems to hint a smoothly-varying weight, leading one to misunderstand the process as having a weight that is smoothly varying from 0 to 1 during the training.  This doesn't match the method in Sec 3.4, right?}
%
%We demonstrate that by selecting an appropriate blending
% phil{TODO: blending?} 
%parameter, we can achieve a straighter generation path while maintaining high-quality generation.

% \phil{TODO (To NOTE): if the contents in the above paragraph overlap too much with those in the introduction, we may greatly shorten it.}
% \ednote{Thanks! We might come back to here again.}


\section{Preliminaries}\label{sec:preliminaries}



%We denote by $(\Ac(x_\Ac),\Bc(x_\Bc))(z)$ a random execution of $\pi$ with private inputs $(x_\Ac,y_\Ac)$, and common input $z$.

%\Jnote{Move to DP}
% At the end of such an execution, the protocol outputs a public transcript denoted by the random variable $\trans_\pi(x_\Ac,x_\Ac,z)$ we denotes the common as $\out(\trans_\pi(x_\Ac,x_\Ac,z)$, and each party $\Pc \in \set{\Ac,\Bc}$ obtains his view denoted $\view^\Pc_\pi(x_\Ac,x_\Bc,z)$, which may also contain a ``local output'' \Jnote{Local} $\out^\Pc(x_\Ac,x_\Bc,z)$ (if the protocol specifies such an output). \Jnote{Common output, and parties output}


\subsection{Distributions and Random Variables}\label{sec:prelim:dist}
The support of a distribution $P$ over a finite set $\cS$ is defined by $\Supp(P) \eqdef \set{x\in \cS: P(x)>0}$. For a distribution or a random variable $D$, let $d\from D$ denote that $d$ was sampled according to $D$. Similarly,  for a set $\cS$, let $x \from \cS$ denote that $x$ is drawn uniformly from $\cS$, and denote by $\cU_{\cS}$ the uniform distribution over $\cS$. For a finite set $\cX$ and a distribution $C_X$ over $\cX$, we use the capital letter $X$ to denote the random variable that takes values in $\cX$ and is sampled according to $C_X$. The {\sf statistical distance} (\aka {\sf~variation distance}) of two distributions $P$ and $Q$ over a discrete domain $\cX$ is defined by $\sdist{P}{Q} \eqdef \max_{\cS\subseteq \cX} \size{P(\cS)-Q(\cS)} = \frac{1}{2} \sum_{x \in \cS}\size{P(x)-Q(x)}$. 
For a vector $x = (x_1,\ldots,x_n)$ and index $i\in [n]$, we let $x_{-i} = (x_1,\ldots,x_{i-1},x_{i+1},\ldots,x_n)$ and $x^{(i)} = (x_1,\ldots,x_{i-1}, -x_i, x_{i+1},\ldots,x_n)$, for a set $\cS \subseteq [n]$ we let $x_{\cS} = (x_i)_{i \in \cS}$ and $x_{-\cS} = (x_i)_{i \in [n]\setminus \cS}$, and for a vector $r \in \zo^n$ we let $x_r = (x_i)_{\set{i \colon r_i = 1}}$ and $x_{-r} = (x_i)_{\set{i \colon r_i = 0}}$.

%For $n \in \N$ we let $U_n$ be the uniform distribution over $\oo^n$, and let $S_n$ be the distribution induces by the sum of $n$ i.i.d.\ random variables, each is distributed according to $U_1$. Let $\cN(0,1)$ be the standard normal distribution.
%For a distribution $\cD$ and a function $f$, we define by $f(\cD)$ the distribution that is induced by the output of $f(x)$ for $x \from \cD$. 





% \begin{theorem}[\cite{McGregorMPRTV10}]\label{thm:sv-extracotr}
% 	\Enote{Remove if not needed}
% 	There is a constant $c$ to make the following holds. Let $X$ be an $\alpha$-SV source on $\{0,1\}^n$, let $Y$ be a source on $\{0,1\}^n$ with min-entropy at least $\beta n$ (independent from $X$), and let $Z=\ip{X,Y}\mbox{mod m}$ for some $m\in\mathbb{N}$. Then for every $\delta\in[0,1]$, the random variable $(Y,Z)$ is $\delta$-close to $(Y,U)$ where $U$ is uniform on $\mathbb{Z}_m$ and independent of $Y$, provided that
% 	$$
% 	n\geq c\cdot\frac{m^2}{\alpha\beta}\cdot\log(\frac{m}{\beta})\cdot\log(\frac{m}{\delta}).
% 	$$
% \end{theorem}



\Enote{I removed the definition of DP since it already appears in the intro}
\remove{
\subsection{Differential Privacy}\label{sec:prelim:DP}
We use the following standard definition of (information theoretic) differential privacy, due to \citet{DMNS06}. For notational convenience, we focus on databases over $\oo$.
\begin{definition}[Differentially private mechanisms]\label{def:mech}
	A randomized function $f\colon\oo^n\mapsto \zs$ is an {\sf $n$-size, $(\eps,\delta)$-differentially private mechanism} (denoted $(\eps,\delta)$-\DP) if for every neighboring $w,w'\in \oo^n$ and every function $g\colon \zs\mapsto \zo$, it holds that 
	$$
	\pr{g(f(w))=1}\leq e^{\eps}\cdot \pr{g(f(w'))=1} +\delta.
	$$ 	
	If $\delta=0$, we omit it from the notation.
\end{definition}
}


\subsubsection{Computational Differential Privacy}
There are several ways for defining computational differential privacy (see \cref{sec:related-works}). We use the most relaxed version due to \cite{BNO08}. For notational convenience, we focus on databases over $\oo$.
\begin{definition}[Computational differentially private mechanisms]\label{def:ComMech}
	A randomized function ensemble $f=\set{f_\pk\colon\oo^{n(\pk)}\mapsto \zs}$ is an {\sf $n$-size, $(\eps,\delta)$-computationally differentially private} (denoted $(\eps,\delta)$-$\CDP$) if for every poly-size circuit family $\set{\Ac_\pk}_{\pk\in \N}$, the following holds for every large enough $\pk$ and every neighboring $w,w'\in\oo^{n(\pk)}$:
	$$
	\pr{\Ac_\pk(f_\pk(w))=1}\leq e^{\eps(\pk)}\cdot \pr{\Ac_\pk(f_\pk(w'))=1} +\delta(\pk).
	$$ 
	If $\delta(\pk) = \negl(\pk)$, we omit it from the notation. 
\end{definition}



\subsubsection{Two-Party Differential Privacy}\label{sec:DP}
In this section we formally define distributed differential privacy mechanism (\ie protocols). %For the ease of notation, we consider protocol with no common input.

\begin{definition}\label{def:DP}%\Nnote{fix security parameter}
	A two-party protocol $\Pi=(\Ac,\Bc)$ is {\sf $(\eps,\delta)$-differentially private}, denoted $(\eps,\delta)$-$\DP$, if the following holds for every algorithm $\Dc$: let $\V^\Pc(x,y)(\pk)$ be the view of party $\Pc$ in a random execution of $\Pi(x,y)(1^\pk)$. Then for every $\pk,n \in \N$, $x\in \oo^n$ and neighboring $y,y'\in\oo^n$:
	\begin{align*}
	\pr{\Dc(V^\Ac(x,y)(\pk))=1}\le e^{\eps(\pk)}\cdot \pr{\Dc(V^\Ac (x,y')(\pk))=1}+\delta(\pk),
	\end{align*} 
	and for every $y\in \oo^n$ and neighboring $x,x'\in\oo^{n}$:
	\begin{align*}
	\pr{\Dc(V^\Bc(x,y)(\pk))=1}\le e^{\eps(\pk)}\cdot \pr{\Dc(V^\Bc (x',y)(\pk))=1}+\delta(\pk).
	\end{align*} 	
	Protocol $\Pi$ is {\sf $(\eps,\delta)$-computational differentially private}, denoted $(\eps,\delta)$-$\CDP$, if the above inequalities only hold for a non-uniform \ppt $\Dc$ and large enough $\pk$. We omit $\delta = \negl(\pk)$ from the notation. \footnote{Note that define we give for two-party differentially private protocols is a semi-honest definition, in which we ask for the security to hold when the parties interact in an honest execution of the protocol. Since we are proving a lower bound, starting from this weaker guarantee (as opposed to security against malicious players), yields a stronger result.}
\end{definition}
%We omit $\delta$ from the notation if $\delta$ is a negligible function of $n$.

%\Enote{simulation-based}
\begin{remark}[The definition for computational differential privacy we use]\label{rem:comDPChannel} 
	An alternative, stronger definition of computational differential privacy, known as simulation-based computational differential privacy, requires that the distribution of each party’s view be computationally indistinguishable from a distribution that ensures privacy in an information-theoretic sense. \cref{def:DP} is a weaker notion in comparison. Consequently, establishing a lower bound for a protocol that satisfies this weaker guarantee (as we do in this work) yields a stronger result.%Actually, our lower bound only requires the privacy to hold against \emph{uniform} external observer.
	%\Nnote{Maybe add: When only interesting in \Dp against external observer, the two definitions can be achieve using key-agreement and (single-party) \Dp mechanism. }
\end{remark}




\subsection{Useful Claims}
\remove{
In this section, we state generic lemmas and propositions that we will use later in our proofs.

The following lemma which we prove in \cref{sec:missing-proofs:distance-I}, measures the distance between two uniform stings conditioned one a random index $i$ either being fixed to $0$ or to $1$.

\def\distanceILemma{
    Let $R \la \zo^n$. For any (randomized) function $f:\{0,1\}^n\rightarrow \{0,1\}$ and $\alpha > 0$, it holds that
    \begin{align}\label{eq:f-alpha}
        \ppr{i \la [n]}{\size{\:\ex{f(R) \mid R_i = 0}-\ex{f(R) \mid R_i = 1}\:}\geq \alpha} \leq \frac{2}{n \alpha^2},
    \end{align}
    where the expectations are taken over $R$ and the randomness of $f$.
}

\begin{lemma}\label{lem:distance-I}
    \distanceILemma
\end{lemma}
}

The following two propositions state that given the output of a differentially private function, it is not possible to predict well even a random index (even if all other indexes are leaked). The first proposition handles the information-theoretic case and the second handles the computation case. Both propositions are proven in \cref{sec:missing-proofs:hard-to-guess}. 

\def\propHardToGuessInf{
    Let $f\colon \oo^n \rightarrow \cY$ be an $(\eps,\delta)$-\DP function, let $g \colon [n] \times \oo^{n-1} \times \cY \rightarrow \set{-1,1,\bot}$ be a (randomized) function, and let $X = (X_1,\ldots,X_n) \la \oo^n$. Then the following holds for every $i \in [n]$ where $X_i^* = g(i,X_{-i},f(X_1,\ldots,X_n))$:
    \begin{align*}
        \pr{X_i^* = X_i} \leq e^{\eps}\cdot \pr{X_i^* = -X_i} + \delta.
    \end{align*}
}

\begin{proposition}\label{prop:hard-to-guess-inf}
    \propHardToGuessInf
\end{proposition}


\def\propHardToGuessComp{
    Let $f = \set{f_{\pk} \colon \oo^{n(\pk)} \rightarrow \zo^{m(\pk)}}_{\pk \in \bbN}$ be an $(\eps,\delta)$-\CDP function ensemble, and let $\set{g_{\pk}}_{\pk \in \bbN}$ be a poly-size circuit family. Then, for large enough $\pk$ and $X = (X_1,\ldots,X_{n(\pk)}) \la \oo^{n(\pk)}$, the following holds for every $i \in [n(\pk)]$ where $X_i^* = g_{\pk}(i,X_{-i},f_{\pk}(X_1,\ldots,X_n))$:
    \begin{align*}
        \pr{X_i^* = X_i} \leq e^{\eps(\pk)}\cdot \pr{X_i^* = -X_i} + \delta(\pk).
    \end{align*}
}

\begin{proposition}\label{prop:hard-to-guess-comp}
    \propHardToGuessComp
\end{proposition}





\remove{
\Enote{Chao's old statement:}
\begin{lemma}\label{lem:distance-I-old}
        Let $R \la \zo^n$. 
	For any function $f:\{0,1\}^n\rightarrow \{0,1\}$ and $\alpha<0.01$, it holds that
	$$
	\Pr_{i\la[n]}\left[\: \size{\:\mathbb{E}[f(R) \mid R_i = 0]-\mathbb{E}[f(R) \mid R_i = 1]\:}\geq \alpha\right]\leq \frac{2+2\log(\frac{1}{\alpha})}{n\alpha^2}.
	$$
\end{lemma}
\begin{proof}
	Define $S_1=\{r \in \zo^n \colon f(r)=1\}$. Then for any $i\in[n]$, we have
	$$
	\begin{array}{rl}
		\size{\mathbb{E}[f(R) \mid R_i = 0]-\mathbb{E}[f(R) \mid R_i = 1]}
		&=\size{\Pr[R\in S_1|R_i=0]-\Pr[R\in S_1|R_i=1]}\\
		&=\size{\frac{\Pr[R_i=0|R\in S_1]\cdot\Pr[R\in S_1]}{\Pr[R_i=0]}-\frac{\Pr[R_i=1|R\in S_1]\cdot\Pr[R\in S_1]}{\Pr[R_i=1]}}\\
		&=\frac{2\size{S_1}}{2^n}\size{\Pr[R_i=0|R\in S_1]-\Pr[R_i=1|R\in S_1]}
	\end{array}
	$$
	When $|S_1|\leq \alpha\cdot 2^{n-1}$, we have $\size{\mathbb{E}[f(R) \mid R_i = 0]-\mathbb{E}[f(R) \mid R_i = 1]}\leq\frac{2\size{S_1}}{2^n}\leq \alpha$ for any $i\in[n]$. Hence, in the following, we assume $|S_1|> \alpha\cdot 2^{n-1}$.

	%Define $I_{bad}=\{i|\size{\Pr[R_i=0|R\in S_1]-\Pr[R_i=1|R\in S_1]}>2\alpha\}$ and $k=\size{I_{bad}}$, then for any $i\notin I_{bad}$, we have 
    %$$
    %\begin{array}{rl}
    %    2\alpha&\geq \size{\Pr[R_i=0|R\in S_1]-\Pr[R_i=1|R\in S_1]}\\
    %    &=\size{\frac{\Pr[R\in S_1|R_i=0]\cdot\Pr[R_i=0]}{\Pr[R\in S_1]}-\frac{\Pr[R\in S_1|R_i=1]\cdot\Pr[R_i=1]}{\Pr[R\in S_1]}}\\
    %    &=\size{\Pr[R\in S_1|R_i=0]-\Pr[R\in S_1|R_i=1]}\cdot\frac{1}{2\Pr[R\in S_1]}\\
    %    &\geq \size{\mathbb{E}[f(R) \mid R_i = 0]-\mathbb{E}[f(R) \mid R_i = 1]}\cdot \frac{1}{2},
    %\end{array}
    %$$ 
    %where the last inequality is because $\Pr[R\in S_1]\leq 1$. So that $\size{\mathbb{E}}[f(R) \mid R_i = 0]-\mathbb{E}[f(R) \mid R_i = 1]\leq %4\alpha$.
    Define $I_{bad}=\{i \colon \size{\Pr[R_i=0|R\in S_1]-\Pr[R_i=1|R\in S_1]} \geq 2\alpha\}$ and $k=\size{I_{bad}}$, and denote $I_{bad}=\{i_1,\dots,i_k\}$. Define $(X_{i_1}, \ldots X_{i_k}) = (R_{i_1},\dots,R_{i_k})\mid_{R \in S_1}$. 
    Consider the min-entropy
	$$
	\begin{array}{rl}
		H_{min}(X_{i_1},\dots,X_{i_k})&\leq H(X_{i_1},\dots,X_{i_k})\\
		&\leq \sum_{j=1}^k H(X_{i_j})\\
		&\leq k\cdot \left(-(\frac{1}{2}+2\alpha)\cdot\log(\frac{1}{2}+2\alpha)-(\frac{1}{2}-2\alpha)\cdot\log(\frac{1}{2}-2\alpha)\right)\\
            &=k\cdot \left(-(\frac{1}{2}+2\alpha)\cdot(\log(1+4\alpha)-1)-(\frac{1}{2}-2\alpha)\cdot(\log(1-4\alpha)-1)\right)\\
            &=k\cdot \left(1-(\frac{1}{2}+2\alpha)\cdot\log(1+4\alpha)-(\frac{1}{2}-2\alpha)\cdot\log(1-4\alpha)\right),
		
	\end{array}
	$$
	where $H_{min}(Y)$ is the minimum entropy of $Y$ and $H(Y)$ is the Shannon entropy of $Y$.\Enote{add to preliminaries.}
        The third inequality holds since by the definition of $I_{bad}$, for every $j \in [k]$ it holds that $\size{\pr{X_{i_j} = 1}-\pr{X_{i_j} = 0}} > 2\alpha$, and therefore $H(X_{i_j}) \leq H(1/2 + 2\alpha)$\Enote{define}.
	
	Therefore, there exists $b_1,\dots,b_k\in\{0,1\}$, such that 
	
	\begin{align}\label{eq:min-entropy-result}
		\Pr\left[(R_{i_1},\ldots,R_{i_k}) = (b_1,\ldots,b_k) \mid R\in S_1\right]
		&= \pr{(X_{i_1},\ldots,X_{i_k}) = (b_1,\ldots,b_k)}\\
		&= 2^{-H_{min}(X_{i_1},\dots,X_{i_k})}\nonumber\\
		&\geq 2^{k\cdot \left(-1+(\frac{1}{2}+2\alpha)\cdot\log(1+4\alpha)+(\frac{1}{2}-2\alpha)\cdot\log(1-4\alpha)\right)}.\nonumber
	\end{align}
	
	Let $S_{bad}=\{r \in \zo^n  \colon \set{(r_{i_1},\ldots,r_{i_k}) = (b_1,\ldots,b_k)} \land \set{r\in S_1}\}$.
	It holds that
	\begin{align*}
		|S_{bad}|
		&= \size{S_1} \cdot \Pr\left[(R_{i_1},\ldots,R_{i_k}) = (b_1,\ldots,b_k) \mid R\in S_1\right]\\
		&\geq \alpha\cdot 2^{n-1}\cdot2^{k\cdot \left(-1+(\frac{1}{2}+2\alpha)\cdot\log(1+4\alpha)+(\frac{1}{2}-2\alpha)\cdot\log(1-4\alpha)\right)},
	\end{align*} 
	where the inequality holds by \cref{eq:min-entropy-result} and since $\size{S_1} \geq \alpha\cdot 2^{n-1}$.
	Notice that any string in $S_{bad}$ depends on at most $n-k$ bits. It implies that $|S_{bad}|\leq 2^{n-k}$. Therefore, we have
	$$
	\begin{array}{rl}
		&2^{n-k}\geq \alpha\cdot 2^{n-1}\cdot2^{k\cdot \left(-1+(\frac{1}{2}+2\alpha)\cdot\log(1+4\alpha)+(\frac{1}{2}-2\alpha)\cdot\log(1-4\alpha)\right)} \\
		\Rightarrow& n-k \geq \log \alpha+n-1+k\cdot \left(-1+(\frac{1}{2}+2\alpha)\cdot\log(1+4\alpha)+(\frac{1}{2}-2\alpha)\cdot\log(1-4\alpha)\right)\\
		\Rightarrow& 1-\log \alpha \geq k\cdot((\frac{1}{2}+2\alpha)\cdot\log(1+4\alpha)+(\frac{1}{2}-2\alpha)\cdot\log(1-4\alpha))\\
		\Rightarrow& 1-\log \alpha \geq k\cdot(4\alpha\cdot\log(1+4\alpha)+(\frac{1}{2}-2\alpha)\cdot\log(1-16\alpha^2))\\
        \Rightarrow& 1-\log\alpha \geq k\cdot(15.9\alpha^2-8\alpha^2+32\alpha^3)=k\cdot(7.9\alpha^2+32\alpha^3)>0.5k\alpha^2\\
		\Rightarrow& k\leq \frac{2-2\log \alpha}{\alpha^2} = \frac{2+2\log (1/\alpha)}{\alpha^2},
	\end{array}
	$$
	Where the third transition holds since 
	\begin{align*}
		\lefteqn{(\frac{1}{2}+2\alpha)\cdot\log(1+4\alpha)+(\frac{1}{2}-2\alpha)\cdot\log(1-4\alpha)}\\
		&= 4\alpha\cdot\log(1+4\alpha) + (\frac{1}{2}-2\alpha)\paren{\log(1+4\alpha)+\log(1-4\alpha)}\\
		&= 4\alpha\cdot\log(1+4\alpha)+(\frac{1}{2}-2\alpha)\cdot\log(1-16\alpha^2),
	\end{align*}
	and the forth transition holds since $4\alpha\cdot\log(1+4\alpha)+(\frac{1}{2}-2\alpha)\cdot\log(1-16\alpha^2) > 15.9\alpha^2-8\alpha^2+32\alpha^3$ for $\alpha < 0.01$.
	Thus, we conclude that 
	$$
	\Pr_{i\la[n]}\left[\size{\mathbb{E}[f(R) \mid R_i=0]-\mathbb{E}[f(R) \mid R_i = 1]}\geq \alpha\right]\leq \frac{k}{n}\leq \frac{2+2\log (1/\alpha)}{n\alpha^2}.
	$$
\end{proof}
}


\subsection{Channels and Two-Party Protocols}\label{sec:protocol}

\paragraph{Channels.}A channel is simply a distribution of a pair of tuples defined as follows. 
\begin{definition}[Channels]\label{def:channel} A {\sf channel} $C_{(X,U)(Y,V)}$ of size $\isize$ over alphabet $\Sigma$ is a probability distribution over $(\Sigma^\isize \times\zo^\ast) \times(\Sigma^\isize \times\zo^\ast)$. The ensemble $C_{(X,U)(Y,V)}= \set{C_{(X_\pk,U_\pk)(Y_\pk,V_\pk)}}_{\pk\in \N}$ is an $\isize$-size channel ensemble, if for every $\pk\in \N$, $C_{(X_\pk,U_\pk)(Y_\pk,V_\pk)}$ is an $\isize(\pk)$-size channel. %We denote a channel of size one by a \emph{single-bit} channel. 
We refer to $X$ and $Y$ as the {\sf local outputs}, and to $U$ and $V$ as the {\sf views}.	
\end{definition}

We view a  channel as the experiment in which there are two parties $\Ac$ and $\Bc$.  Party $\Ac$ receives ``output'' $X$ and ``view'' $U$, and party $\Bc$ receives ``output'' $Y$ and ``view'' $V$. Unless stated otherwise, the channels we consider are over the alphabet $\Sigma = \oo$. We naturally identify channels with the distribution that characterizes their output.








\subsubsection{Two-Party Protocols}

A two-party protocol $\Pi=(\Ac,\Bc)$ is \ppt if the running time of both parties is polynomial in their input length. We let $\Pi(x,y)(z)$ or $(\Ac(x),\Bc(y))(z)$ denote a random execution of $\Pi$ on a common input $z$, and private inputs $x,y$.%We assume \wlg that a protocol has a common output (part of its transcript).\Jnote{This is not really the case we consider in this paper..}

\begin{definition}[Oracle-aided protocols]\label{def:ChannelAidedProtocol}
	In a two-party protocol $\Pi$ with oracle access to a {\sf protocol} $\Psi$, denoted $\Pi^\Psi$, the parties make use of the \textit{next-message function} of $\Psi$.\footnote{The function that on a partial view of one of the parties, returns its next message.} In a two-party protocol $\Pi$ with oracle access to a {\sf channel} $C_{Z W}$, denoted $\Pi^C$, the parties can jointly invoke $C$ for several times. In each call, an independent pair $(z,w)$ is sampled according to $C_{Z W}$, one party gets $z$, the other gets $w$.
\end{definition}


\begin{definition}[The channel of a protocol]\label{def:ChannlOfProtocol}
	For a no-input two-party protocol $\Pi= (\Ac,\Bc)$, we associate the channel $C_\Pi$, defined by $\C_\Pi= C_{(X, U),(Y, V)}$, where $X$ and $Y$ are the local outputs of $\Ac$ and $\Bc$ (respectively) and
	$U$ and $V$ are the local views of $\Ac$ and $\Bc$ (respectively).
    
	For a two-party protocol $\Pi$ that gets a security parameter $1^\pk$ as its (only, common) input, we associate the channel ensemble $ \set{C_{\Pi(1^\pk)}}_{\pk\in \N}$. 
\end{definition}

\begin{definition}[$(\alpha,\gamma)$-Accurate channel]\label{def:accurate-func}
	A channel $C = C_{(X, U),(Y, V)}$ is {\sf $(\alpha,\gamma)$-accurate for the function $f$}, if $\ppr{C}{\size{\out(V)-f(X,Y)}\leq \alpha}\ge \gamma$, where $\out(V)$ is the designated output.
    A channel ensemble $C_{(X, U),(Y, V)}= \set{C_{(X_\pk, U_\pk),(Y_\pk, V_\pk)}}_{\pk\in \N}$ is  $(\alpha,\gamma)$-accurate for  $f$ if $C_{(X_\pk, U_\pk),(Y_\pk, V_\pk)}$ is $(\alpha(\pk),\gamma(\pk))$-accurate for $f$, for every $\pk \in \N$.
\end{definition}

\subsubsection{Differentially Private Channels}\label{sec:DPChannel}
Differentially private channels are naturally defined as follows:
\begin{definition}[Differentially private channels]\label{def:DPChannel}
	An $n$-size channel $C = C_{(X, U),(Y, V)}$ with $X, Y$ over $\oo^n$ 
	is {\sf$(\eps,\delta)$-differentially private} (denoted $(\eps,\delta)$-$\DP$) if for every $x \in \Supp(X)$ there exists an $n$-size $(\eps,\delta)$-$\DP$ mechanisms $\Mc_x$ such that $(X,Y,U) \equiv (X,Y,\Mc_X(Y))$, and for every $y \in \Supp(Y)$ there exists an $n$-size $(\eps,\delta)$-$\DP$ mechanisms $\Mc_y'$ such that $(X,Y,V) \equiv (X,Y,\Mc_Y'(X))$. In addition, we say that the channel is \emph{uniform} if $X$ and $Y$ are independent random variables uniformly distributed in $\oo^n$. 
\end{definition}

\begin{definition}[Computational differentially private channels]\label{def:CDPChannel}
	An $n$-size channel ensemble $C = \set{C_{(X_\pk, U_\pk),(Y_\pk, V_\pk)}}_{\pk\in\N}$ with $X_\pk, Y_\pk$ over $\oo^n$ 
	is {\sf$(\eps,\delta)$-computationally differentially private} (denoted $(\eps,\delta)$-$\CDP$) if for every ensemble $\set{x_\pk \in \Supp(X_\pk)}_{\pk\in\N}$ there exists an $n$-size $(\eps,\delta)$-\CDP mechanisms ensemble $\set{\Mc_{x_\pk}}_{\pk\in\N}$ such that $(X_\pk,Y_\pk,U_\pk) \equiv (X_\pk,Y_\pk,\Mc_{X_\pk}(Y_\pk))$, for every $\pk\in\N$, and for every ensemble $\set{y_\pk \in \Supp(Y_\pk)}_{\pk\in\N}$ there exists an $n$-size $(\eps,\delta)$-$\CDP$ mechanisms ensemble $\set{\Mc'_{y_\pk}}_{\pk\in\N}$ such that $(X_\pk,Y_\pk,V_\pk) \equiv (X_\pk,Y_\pk,\Mc_{Y_\pk}'(X_\pk))$ for every $\pk\in \N$. In addition, we say that the channel is \emph{uniform} if $X_\pk$ and $Y_\pk$ are independent random variables uniformly distributed in $\{\pm 1\}^n$ for all $\pk\in\N$.
\end{definition}




% \begin{lemma}~\label{lem:dp-sv-source}
% 	Let $P$ be an $\varepsilon$-DP randomized protocol. Let $X$ and $Y$ be independent random variables uniformly distributed in $\{\pm 1\}^n$ and let random variable $\Pi(X,Y)$ denote the transcript of running $P(X,y)$. Then for every $\pi\in Supp(\Pi)$, the random variables corresponding to the inputs conditioned on transcript $\pi$, $X_\pi$ and $Y_\pi$, are independent $e^{-\varepsilon}$-strong SV source.
% \end{lemma}





\subsubsection{Weak Erasure Channel (\WEC)}

\begin{definition}[\WEC]\label{def:WEC}
	A channel $((O_A,V_A), (O_B,V_B))$ with $O_A \in \set{0,1}$ and $O_B \in \set{0,1,\bot}$ is a {\sf weak erasure channel}, denoted $(\alpha,p,q)$-$\WEC$, if:
	\begin{itemize}
		%\item $O_A\in \set{-1,1}$ and $O_B\in \set{-1,1,\bot}$.
		\item Random erasure: $\pr{O_B = \perp} = 1/2$.
		
		\item Agreement: $\pr{O_A\ne O_B\mid O_B\ne \bot}\le \alpha$.
		
		\item Secrecy:
		
		\begin{enumerate}
			\item For every algorithm $\Dc$ it holds that\label{WEC:item:A}
			\begin{align*}
				%\size{\pr{\Ac(O_A,V_A) = 1 \mid O_B \neq \perp} - \pr{\Ac(O_A,V_A) = 1 \mid O_B = \perp}} \le p
				\size{\pr{\Dc(V_A) = 1 \mid O_B \neq \perp} - \pr{\Dc(V_A) = 1 \mid O_B = \perp}} \le p
			\end{align*}
			(Alice doesn't know if $O_B = \perp$.)
			
			\item For every algorithm $\Dc$ it holds that\label{WEC:item:B}
			\begin{align*}
				\pr{\Dc(V_B) = O_A \mid O_B=\bot} \leq \frac{1+q}{2}.
			\end{align*}
			(i.e., if $O_B=\bot$, Bob don't know what is the value of $O_A$).
			
			%\item $SD((O_A U|O_B=\bot),(O_A U|O_B\ne \bot))\le p$ (The sender don't know if $O_B=\bot$).
			
			%\item $SD(V O_A|O_B=\bot,V(-O_A)|O_B=\bot)\le q$ (If $O_B=\bot$, Bob don't know what the value of $O_A$).
		\end{enumerate}
	\end{itemize}
   We say that a channel ensemble $C=\set{C_\pk}_{\pk\in N}$ is a {\sf computational weak erasure channel}, denoted $(\alpha,p,q)$-\CompWEC, if for every \ppt algorithm $\Dc$ and every sufficiently large $\pk\in\N$, $C_\pk$ satisfies the properties stated in the items above, where the secrecy property holds with respect to a \ppt algorithm $\Dc$. A protocol $\Lambda$ is said to be $(\alpha,p,q)$-$\CompWEC$, if the ensemble induces by the protocol (that is, $C=\set{C_{\Lambda(\pk)}}_{\pk\in\N}$) is $(\alpha,p,q)$-$\CompWEC$.  
\end{definition}



\subsubsection{Approximate Weak Erasure Channel (\AWEC)}\label{sec:AWEC}

\begin{definition}[\AWEC]\label{def:AWEC}
	A channel $C = ((O_A,V_A), (O_B,V_B))$ over $([-n,n] \times \zo^*) \times (([-n,n] \cup \bot)  \times \zo^*)$ is an {\sf approximate weak erasure channel}, denoted $(\ell,\alpha,p,q)$-\AWEC if:
	\begin{itemize}
		
		\item Random erasure: $\pr{O_B = \perp} = 1/2$.
		
		\item Accuracy: $\pr{\size{O_A - O_B} > \ell \mid O_B \ne \bot}\le \alpha$.
		
		\item Secrecy:
		
		\begin{enumerate}
			\item For every algorithm $\Dc$ it holds that\label{AWEC:item:A}
			\begin{align*}
				%\size{\pr{\Ac(O_A,V_A) = 1 \mid O_B \neq \perp} - \pr{\Ac(O_A,V_A) = 1 \mid O_B = \perp}} \le p
				\size{\pr{\Dc(V_A) = 1 \mid O_B \neq \perp} - \pr{\Dc(V_A) = 1 \mid O_B = \perp}} \le p
			\end{align*}
			(Alice doesn't know if $O_B=\bot$).
			
			\item For every algorithm $\Dc$ it holds that\label{AWEC:item:B}
			\begin{align*}
				\pr{\size{\Dc(V_B) - O_A} \leq 1000 \ell \mid O_B=\bot} \leq q.
			\end{align*}
			(i.e., if $O_B=\bot$, Bob can't estimate the value of $O_A$ with error $\leq 1000 \ell$).
		\end{enumerate}
	\end{itemize}
     We say that a channel ensemble $C=\set{C_\pk}_{\pk\in N}$ is a {\sf computational approximate weak erasure channel}, denoted $(\ell,\alpha,p,q)$-\CompAWEC, if for every \ppt algorithm $\Dc$ and every sufficiently large $\pk\in\N$, $C_\pk$ satisfies the properties stated in the items above. A protocol $\Gamma$ is said to be $(\ell,\alpha,p,q)$-$\CompAWEC$, if the ensemble induced by the protocol (that is, $C=\set{C_{\Gamma(\pk)}}_{\pk\in\N}$) is $(\ell,\alpha,p,q)$-$\CompAWEC$.  
\end{definition}

We will make use of the following lemma, which shows that for some choices of the parameters, \AWEC implies \WEC. The lemma is proven in \cref{sec:AWEC-to-WEC}.

\begin{lemma}\label{lemma:AWEC-to-WEC}
	For every $\ell> 0$, there exists a \ppt protocol $\Lambda = (\Pc_1,\Pc_2)$ such that given an oracle access to an $(\ell,\alpha,p,q)$-\AWEC $C$, the channel $\tilde{C}$ induced by $\Lambda^C$ is $(\alpha'=\alpha+0.001,\: p' = p ,\:  q' = 1/2 + 2(q+0.01))$-\WEC.
	Furthermore, the proof is constructive in a black-box manner:
	\begin{enumerate}
		\item There exists an oracle-aided \ppt algorithm $\Ec_1$ such that for every channel $C = ((\OA,\VA), (\OB,\VB))$ and algorithm $\Dc$ violating the \WEC secrecy property~\ref{WEC:item:A} of $\tilde{C}$, algorithm $\Ec_1^{\Dc}$ violates the \AWEC secrecy property~\ref{AWEC:item:A} of $C$.
		
		\item There exists an oracle-aided \ppt algorithm $\Ec_2$ such that for every channel $C = ((\OA,\VA), (\OB,\VB))$ and algorithm $\Dc$ violating the \WEC secrecy property~\ref{WEC:item:B} of $\tilde{C}$, algorithm $\Ec_2^{\Dc}$ violates the \AWEC secrecy property~\ref{AWEC:item:B} of $C$.
	\end{enumerate}
\end{lemma}

Since \cref{lemma:AWEC-to-WEC} is constructive, the following is an immediate corollary.
\begin{corollary}\label{cor:CompAWEC to CompWEC}
There exists an oracle aided \ppt protocol $\Lambda$, such that given a protocol $\Gamma$ that induces $(\ell,\alpha,p,q)$-\CompAWEC, it holds that $\Lambda^\Gamma$ is $(\alpha'=\alpha+0.001,\: p' = p ,\:  q' = 1/2 + 2(q+0.01))$-\CompWEC.  
\end{corollary}
\begin{proof}[Proof of \ref{cor:CompAWEC to CompWEC}]
Let $\Lambda$ be the \ppt algorithm guaranteed  by Lemma \ref{lemma:AWEC-to-WEC}. Given an $(\ell,\alpha,p,q)$-\CompAWEC protocol $\Gamma$, we define $\Lambda(\pk)={\Lambda^{\Gamma(\pk)}(\pk)}$. Assume towards a contradiction that $\Lambda$ is not a $(\alpha',p',q')$-\CompWEC. It follows that there exists a \ppt $\Dc$ that for infinity many $\pk\in\N$ contradicts one of the \WEC secrecy properties of channel ensemble $\set{C_{\Lambda(\pk)}}_{\pk\in\N}$. Fix $\pk\in\N$ for which this holds. By Lemma \ref{lemma:AWEC-to-WEC}, there exists a \ppt $\Ec^\Dc$ that for every such $\pk$  contradicts one of the secrecy properties of the channel $C_{\Gamma(\pk)}$. This implies that for infinity many $\pk\in\N$, $\Ec^\Dc$  contradict the secrecy of the channel ensemble $\set{C_{\Gamma(\pk)}}_{\pk\in\N}$, which is a contradiction since this would means that $\Gamma$ is not a $(\ell,\alpha,p,q)$-\CompAWEC.       
\end{proof}



\subsection{Oblivious Transfer (\OT)}

\paragraph{Secure Computation.}
We use the standard notion of securely computing a functionality, \cf  \cite{Goldreich04}.
\begin{definition}[Secure computation]\label{def:SFE}
	A two-party protocol {\sf securely computes a functionality $f$}, if it does so according to the real/ideal paradigm.   We add the term perfectly/statistically/computationally/non-uniform computationally, if the simulator's output is  perfect/statistical/computationally indistinguishable/  non-uniformly indistinguishable from  the real distribution.  The protocol have the above notions of security {\sf against semi-honest  adversaries}, if its security only  guaranteed to holds against an adversary that follows the prescribed protocol.   Finally, for the case of perfectly secure computation, we naturally apply the above notion also to the non-asymptotic case: the protocol with no security parameter perfectly  compute a functionality $f$.
	
	A two-party protocol {\sf securely computes a functionality ensemble $f$ with oracle to a channel $C$}, if it does so according to the above definition when the parties have access to a trusted party computing $C$. All the above adjectives naturally extend to this setting.
\end{definition}

\paragraph{Oblivious Transfer.}
The (one-out-of-two) oblivious transfer functionality is defined as follows.
\begin{definition}[oblivious transfer functionality $f_{\OT}$]\label{def:OTfunc}
	The oblivious transfer functionality over $\zo \times (\zs)^2$ is defined by  $f_{\OT} (i,(\sigma_0,\sigma_1)) = (\perp,\sigma_i)$.
\end{definition}
A protocol is $\ast$ secure OT,   for \\$\ast\in \set{\text{semi-honest statistically/computationally/computationally non-uniform}}$, if it  compute the $f_{\OT}$  functionality with $\ast$ security.





% \begin{definition}[Computational oblivious transfer, semi-honest model]
% A protocol $\Pi=(\Ac,\Bc)$ is a semi-honest 1-out-of-2 computational oblivious transfer (comp-OT) protocol if the following holds. Given a common input $1^{\pk}$, the parties $\Ac$ and $\Bc$ run the protocol $\Pi(1^\pk)$ (in an honest manner) and    
% $\Ac$ outputs $X=(m_1,m_2)\in \zo\times\zo$ and has a view $U$ and $\Bc$ outputs $Y=(i,\hat{m})\in\zo\times\zo$ and has a view $V$, and the following properties are satisfied:
% \begin{enumerate}
%     \item \textbf{Correctness:} 
%     $\pr{\hat{m}\neq m_i}<\negl(\pk).$ 
    
%     \item \textbf{A's Privacy:} For every \ppt $\Dc$ and every sufficiently large $\pk$:
%     $\pr{\Dc(V)=m_{i-1}}<(1+\negl(\pk))/2$
    
%     \item \textbf{B's Privacy:} For every \ppt $\Dc$ and every sufficiently large $\pk$:
%     $\pr{\Dc(U)=i}<(1+\negl(\pk))/2$  
% \end{enumerate}
% \end{definition}

We make use of the following useful results by Wullschleger on oblivious transfer amplification from weak channels.
\begin{theorem}[\cite{Wullschleger09}, from \WEC to statistically secure \OT]\label{thm:WEC TO OT IT}
    There exists an oracle aided protocol $\Pi$ such that the following holds: Given a $(\alpha,p,q)$-\WEC $C$, if $44(\alpha+p)\le 1-q$ then $\Pi^{C}(1^\pk)$ is a semi-honest statistically secure \OT.
\end{theorem}

The following computational version of \cref{thm:WEC TO OT IT} is implicit in \cite{Wullschleger09} and is based on the computational proof explicitly stated in \cite{Wul07} (see Section 6 in \cite{Wullschleger09} for discussion).   

\begin{theorem}[\cite{Wullschleger09,   Wul07}, from \CompWEC to computinally secure \OT]\label{thm:WEC TO OT Comp}
    There exists an oracle aided protocol $\Pi$ such that the following holds: Given a $(\alpha,p,q)$-\CompWEC protocol $\Lambda$, if $44(\alpha+p)\le 1-q$ then $\Pi^{\Lambda}$ is a semi-honest computational secure \OT.
\end{theorem}



% \begin{definition}[Computational 1-out-of-2 Oblivious Transfer, semi-honest model]
% A protocol $\Pi=(\Ac,\Bc)$ is a semi-honest 1-out-of-2 $(\eps,\alpha,\beta)$-oblivious transfer (OT) protocol if the following holds. 

% The parties $\Ac$ and $\Bc$ run the protocol (in an honest manner) and    
% $\Ac$ outputs $X=(m_1,m_2)\in \zo\times\zo$ and has a view $U$ and $\Bc$ outputs $Y=(i,\hat{m})\in\zo\times\zo$ and has a view $V$, and following properties are satisfied:
% \begin{enumerate}
%     \item \textbf{Correctness:} 
%     $\pr{\hat{m}\neq m_i}<\eps.$ 
    
%     \item \textbf{A's Privacy:} For every adversary $\Dc$:
%     $\pr{\Dc(V)=m_{i-1}}<(1+\alpha)/2$
    
%     \item \textbf{B's Privacy:} For every adversary $\Dc$: $\pr{\Dc(U)=i}<(1+\beta)/2$  
% \end{enumerate}
% \end{definition}
%\subsection{OT Approximation Challenges for Point Cloud Generation}
% \subsection{Challenges in OT Approximation for Point Cloud Generation}

\vspace{-2mm}
\subsection{Existing OT Approximation for Point Cloud Generation}
\vspace{-1mm}
% \phil{TODO: about the subsection title... How about ``Studying OT Approximation for Point Cloud Generation'' or ``Investigating OT Approximation for Point Cloud Generation''?}
\label{subsec:points_challenges}
\begin{figure*}[t]
\vspace{-6mm}	
 \centering
	\includegraphics[width=0.8\linewidth]{figures/ot_analysis.png}
\vspace{-4mm}
        
	\caption{
Comparison of OT Approximation Methods. % measured over 32 training batches.
%
\textbf{Left:} Average OT distance across batch sizes. Minibatch OT (blue) fails to reduce distances much compared to independent coupling (red dash). Equivariant OT (orange) significantly reduces distance values.
%
%We also show the OT distance obtained our OT approximation method (detailed in Section~\ref{subsec:ot_approx}).
Our OT approximation is on par with Equivariant OT.
%
\textbf{Right:} Computational time for OT across batch sizes. Minibatch OT (blue) maintains a reasonable computational time ($\sim$1 second) with batch size $B=256$. Equivariant OT (orange) grows quadratically starting from 2.2 seconds with $B=1$.
%
%Note that we only consider permutations in Equivariant OT.
 }
\label{fig:ot_analysis}
\end{figure*}
% \begin{itemize}
%     \item Introduction of Characteristics for Point Cloud. We will introduce three important characteristics of point clouds: the permutation-invariant property, the requirement of a dense point set, and their nature as sub-samples from an underlying surface. 
%     \item Analysis of challenges with existing OT coupling approximation methods. Current approaches either fail to construct a good OT coupling approximation for dense permutation-invariant sets (Minibatch OT) or are computationally too expensive for training (Equivariant OT coupling).
% \end{itemize}

%This work focuses on generating 3D shapes represented as point clouds. A point cloud $\rvx_1$ is a set of points sampled on the surface of a given shape $\mathcal{S}$, with point coordinates arranged in matrix format: $\rvx_1 \in \mathcal{R}^{N \times 3}$, where $N$ is the number of points. Unlike 2D images, point clouds have several unique properties, which pose challenges for existing OT approximation methods and motivate us to develop a new framework for efficiently computing the OT map:

We focus on generating 3D shapes represented as point clouds. A point cloud $\rvx_1 \in \mathcal{R}^{N \times 3}$ is a set of points sampled from the surface of a shape $\mathcal{S}$, where $N$ is the number of points. Unlike 2D images, point clouds have unique properties that pose challenges for existing OT methods:

\textbf{(i) Permutation Invariance.} \
A point cloud, while arranged in a matrix form, is inherently a set. Shuffling points in $\rvx_1$ should represent the same shape. Mathematically, given a permutation matrix $\rvrho(g)$, the sampling probability remains unchanged,~\ie, $q_1(\rvrho(g) \rvx_1) = q_1(\rvx_1)$.
    
\textbf{(ii) Dense Point Set.} \
Point clouds are finite samples on surfaces. However, similar to low-resolution images, sparse point sets may miss fine geometric structures and details. Thus, most works use dense point sets (say $N \geq 2048$) to accurately capture 3D shapes.

%The above two unique properties of 3D point clouds present challenges when directly applying existing Optimal Transport (OT) map approximation methods. Specifically, Minibatch OT~\cite{pooladian2023multisample,tong2023improving} fails to produce high-quality OT maps for point cloud generation, while Equivariant OT~\cite{klein2024equivariant,song2024equivariant} is computationally inefficient.

Existing approach to estimating OT maps face these challenges on point clouds:

\para{Ineffectiveness of Minibatch OT.} \
%Minibatch OT falls short of producing high-quality OT maps for point cloud data, despite its success in low-dimensional and image domains. The reason  stems from property (i) of point clouds: there are $N!$ possible matrix representations of the same point cloud. This means there are $N!$ equivalent pairs for the training pair $(\rvx_0, \rvx_1)$, by applying relative permutations $\rvrho(g)$ to $\rvx_1$,~\ie, $(\rvx_0, \rvrho(g)\rvx_1)$. An OT-sampled pair should minimize the distance function: $C(\rvx_0, \rvx_1) = \min\limits_{g \in G} C(\rvx_0, \rvrho(g) \rvx_1)$. However, Minibatch OT assignments only grow quadratically with batch size, while training pair permutations increase exponentially. This method likely yields a suboptimal pair that fails to meet the above requirement and is thus far from optimal.
%To study the quality of the approximated OT map, we measure the average OT distance of sampled training pairs $C(\rvx_0, \rvx_1)$ produced from different methods and plot the results for varying batch sizes in Figure~\ref{fig:ot_analysis} (left). Overall, Minibatch OT achieves only a negligible reduction ($\sim 1\%$) compared to the independent coupling \phil{strategy} in the original flow matching formulation, even with a batch size of 256. This demonstrates the limited effectiveness of this approach in point cloud generation.
Minibatch OT, effective in low-dimensional and image domains, fails for point clouds due to property (i). There are $N!$ equivalent representations of the same point cloud, implying $N!$ equivalent training pairs $(\rvx_0, \rvx_1)$. An OT-sampled pair should minimize the cost: $C(\rvx_0, \rvx_1) = \min_{g \in G} C(\rvx_0, \rvrho(g) \rvx_1)$. However, in Minibatch OT’s with no permutation, the assignments grow quadratically with batch size, while the number of permutations grows exponentially. As shown in Figure~\ref{fig:ot_analysis} (left), Minibatch OT achieves only about $6\%$ reduction in the cost even with batch size 256, indicating limited effectiveness of this approach in point cloud generation.

\para{Inefficient OT Maps in Equivariant OT.} \
Equivariant OT produces high-quality maps, but is computationally expensive for point cloud generation. %In particular, it resolves Minibatch OT's shortcomings by modifying the cost function to account for permutations, demonstrating success in generating molecular data. As
Figure~\ref{fig:ot_analysis} (left) shows a $48.7\%$ reduction even with a batch size 1, showing the importance of aligning points and noise via permutation.
%, this modification proves effective for point cloud generation, achieving approximately $46.7\%$ reduction (orange curve) even with a batch size 1. 
%This result highlights the importance of considering the permutation-invariance of point clouds.
However, unlike molecular data, which has limited size ($N$$=$$55$ in~\cite{klein2024equivariant}), representing 3D shapes needs a larger $N$, following property (ii). 
%So, computing the modified cost function requires solving another OT problem of reordering point coordinates using the Hungarian algorithm~\cite{kuhn1955hungarian}. Given a time complexity of $O(N^3)$, it takes over 200 seconds to solve a problem instance, even for a batch size of one; see Figure~\ref{fig:ot_analysis} (right). In each training iteration, we must solve $B^2$ problem instances to evaluate the distances among all possible pairs of $\rvx_0$ and $\rvx_1$, where $B$ is the batch size.  This computational burden prohibits the practical use of this approach.
Solving the optimal transport cost takes an $O(B^2 N^3)$ computational complexity because of the quadratic number of noise and point cloud pairs in a batch of $B$ examples, and $O(N^3)$ for the the Hungarian algorithm~\cite{kuhn1955hungarian}. Figure~\ref{fig:ot_analysis} (right) shows how this grows rapidly even for a typical point cloud size ($N=2048$).
%
It takes around 2.2 seconds for the OT computation even for $B = 1$, leading to a significant bottleneck in the training process that is more than 40x slower than independent coupling.
%\ednote{our OT not-so-optimal flow can finish 1 epoch in 7 seconds, while it takes 310 seconds for Equivariant OT}

\if 0
% To evaluate effectiveness, we aim to obtain an OT map that well approximates the ground truth. We can assess the OT map's performance by measuring the average distance function $C(x_0, x_1)$ over sampled training pairs $x_0$ and $x_1$.
%
Since most existing works approximate the OT between batched training pairs, we can evaluate the average distance function across various batch sizes.
%
Figure~\ref{} illustrates the average distance function of four different methods, Flow Matching, Minibatch OT, Equivariant Flow Matching, and Our OT Approximation (detailed in Section~\ref{subsec:ot_approx}), over 32 training batches for different batch sizes.

The figure shows that flow matching exhibits the highest distance values compared to other methods, confirming that independent coupling is far from the optimal transport (OT) map.
%

% While Minibatch OT~\cite{pooladian2023multisample,tong2023improving} has shown success in the image domain, it performs poorly for point clouds, failing to reduce the distance function (compared to flow matching).
%
% This is because the number of possible permutations of training pairs grows exponentially ($N!$), whereas the combinations of assignments in Minibatch OT only increase quadratically with batch size.
% %
% This OT approximation is unlikely to locate a training pair sharing the same permutation, i.e., $C(x_0, x_1) = \min\limits_{g \in G} C(x_0, \rho(g) x_1)$, making it far from optimal transport.
% %
% Notably, in point cloud generation ($N \geq 2048$), the distance reduction is almost negligible even with a batch size of 1024.

In contrast, equivariant flow matching~\cite{klein2024equivariant} considers permutations in the matching process, significantly reducing distances even with a batch size of 1.
%
However, finding the optimal permutation between a training pair involves solving another OT problem using the points' and noises' coordinates.
%
The Hungarian algorithm, typically used to solve OT problems, has a complexity of $O(N^3)$, making it computationally expensive for point cloud training.
%
Our efficiency analysis of different OT approximation methods reveals that the processing time of equivariant flow matching is prohibitively long (around 200 seconds per batch), even for a batch size of 1.
\fi


\begin{algorithm}
\caption{The OT approximation using gradient flow}\label{alg:two}
\KwData{Dense Gaussian Noises $X_0$, Superset Points $X_1$, Learning rate $lr$, Maximum Update Iterations $T$, Sinkhorn Blur Factor $\sigma$}
\KwResult{Deformed Gaussian Noises $X_0^T$}
$X_0^0 \gets X_0$\;
\For{0 \leq i < T}{
    loss = SinkhornDivergence(X_0', X_1, \sigma) \\
    X_0^{i+1} \gets X_0^i - lr (\nabla_{X_0'} loss) \\
}
\end{algorithm}

\vspace{-6mm}
\subsection{Hybrid Coupling}
\label{subsec:coupling_interpolation}
\vspace{-2mm}
Though training flows with OT couplings comes with appealing theoretical justifications (\eg, straight sampling trajectories), we identify a key training challenge with them that is often overlooked in the OT flows literature. Our experiments (\eg, Section~\ref{subsec:uncond_gen}) indicate that flows trained with equivariant OT maps are often outperformed by those with independent coupling in terms of sample quality, especially when the number of sampling steps is large. We hypothesize this is due to the increasing complexity of target vector fields for OT couplings that makes their approximation harder with neural networks. 

\begin{wrapfigure}{r}{0.3\textwidth} %this figure will be at the right
    \centering
    \vspace{-5mm}
    \includegraphics[width=0.3\textwidth]{figures/jacobian_analysis.png}
    \vspace{-7mm}
    \caption{    %
    We show Jacobian Frobenius Norm for different trained $\rvv_{\theta, t}$ over different time intervals, which measures the model complexity as in~\cite{dockhorn2022score}.
    }
    \vspace{-4mm}
 \label{fig:jacobian_analysis}
\end{wrapfigure}
Intuitively, as we make target sampling trajectories straighter using more accurate OT couplings, the complexity of generation shifts toward smaller time steps. As Figure~\ref{fig:lipchitz_vis} shows, in the limit of straight trajectories, the learned vector field $\rvv_{\theta, 0}(\rvx_0)$ should be able to switch between different target point clouds with small variation in $\rvx_0$, forcing $\rvv_{\theta, 0}$ to be complex at $t{=}0$. This problem is further exacerbated in the equivariant OT flows with large $N$ where permuting Gaussian noise cloud in the input makes it virtually the same for all target point clouds. To verify this, we measure the trained vector field's complexity for 3D point cloud generation using the Jacobian Frobenius norm in different timesteps in Figure~\ref{fig:jacobian_analysis}. As hypothesized above, switching from independent coupling (blue curve) to our OT approximation (orange curve) shifts the high Jacobian norm at $t\approx1$ for independent coupling to $t\approx0$ for OT coupling. This motivates us to develop a method to make 
%the problem of approximating the target vector field with neural networks easier, 
it easier for neural networks to approximate the target vector field, 
while still maintaining a relatively straight path.


%However, learning a straighter trajectory does not necessarily simplify the vector field network's learning task.
%
%In fact, it often worsens the generation quality (detailed in Section~\ref{subsec:model_analysis}).


\iffalse
\begin{wrapfigure}{r}{0.3\textwidth} %this figure will be at the right
    \centering
    \vspace{-5mm}
    \includegraphics[width=0.3\textwidth]{figures/jacobian_analysis.png}
    \vspace{-7mm}
    \caption{    %
    We show Jacobian Frobenius Norm for different trained $\rvv_{\theta, t}$ over different time intervals, which measures the model complexity as in~\cite{dockhorn2022score}.
    }
    \vspace{-4mm}
 \label{fig:jacobian_analysis}
\end{wrapfigure}
\para{Model Complexity.}
% Straight trajectories require accurate early-step estimation, increasing model complexity in these intervals.
%
We attribute this to the fact that straight trajectories demand accurate estimation even at early inference steps, implying increased model complexity in these intervals.
%
To verify this, we measure the trained neural networks' complexity using the Jacobian Frobenius norm, illustrating the magnitude over different time intervals in Figure~\ref{fig:jacobian_analysis}.
%
Overall, when employing our OT approximation (orange curve), the norm is significantly higher at the beginning of the trajectory path, implying that the trained model is more Lipchitz and complex. 
%
This motivates us to develop a method to alleviate the learning burden of network, especially at the early stage, while still maintaining a relatively straight path.
%
% Though we are interpolating towards 

% Though we can now obtain the training pairs are closer to the samples from an OT map and obtain a straighter inference path, we empirically notice that this will lead to training difficulties and result in worse generation performance.
% %
% In particular, we plot the performance of Flow Matching (sampling training pairs from independent coupling) and our method (sampling training pairs using pre-computed superset OT) in Figure~\ref{}.
% %
% We notice our method can achieve a better performance when given less number of inference steps, implying our method can learn a straighter trajectory.
% %
% However, the performance is significantly worse than flow matching given a larger number of inference steps.



% We hypothesize this phenomenon that a straighter generation path pose a much challenging training process for the generative model.
% %
% To verify this, we plot the Jacobian norms of the vector field network trained with Flow Matching model and our method.
% %
% We notice that the vector field learned by our method ace challenges in the initial time intervals, while Flow Matching encounters more difficulties in later timesteps.
%
\fi


\textbf{Hybrid Coupling.} % \phil{Blending or interpolation?}
%As Figure~\ref{fig:jacobian_analysis} suggests, in constant with our OT approximation, the model complexity lies at the end of the trajectories, if the training pairs are sampled from the independent coupling method (blue curve).
% This contrasts with the case of independent coupling (blue curve), where complexities are primarily placed at the end of trajectories.
% As Figure~\ref{fig:jacobian_analysis} suggests, different couplings introduce learning complexities at various time intervals. 
%
Given the different behavior of independent and OT couplings in Figure~\ref{fig:jacobian_analysis}, we aim to reduce the complexity of the vector field at early timesteps by combining our OT approximation with independent coupling.
%
To do so, we propose injecting additional random Gaussian noise into $\rvx_0$, making our OT couplings even less ``optimal''.
%
The new training $\rvx_0'$ is defined as:
\begin{equation}
    \rvx_0' = \sqrt{1 - \beta} \rvx_0 +  \sqrt{\beta} \rvepsilon, \quad \rvepsilon \sim \mathcal{N}(\rvepsilon; 0, \textbf{I}),
\end{equation}
where $\beta \in [0, 1]$ is the blending coefficient.
%
%This formulation draws inspiration from the forward process in the denoising diffusion probabilistic model~\cite{ho2020denoising}.
%
Intuitively $\beta$ allows us to switch smoothly between independent and OT couplings.
Specifically, for $\beta \rightarrow 0$, the coupled data and noise pairs converge to our OT couplings, whereas when $\beta \rightarrow 1$, they follow the independent coupling.
%
% As $\beta$ approaches 0, the distribution converges to our OT approximation; as $\beta$ nears 1, it tends toward independent coupling.
%
%Moreover, we demonstrate that this blending strategy maintains the correct marginal, i.e., $\int q(x_0, x_1) dx_1 = q_0(x_0)$. A detailed proof is provided in the Appendix.

We empirically observe that $\beta = 0.2$ in most experiments strikes a good balance between learning complexity (as shown by the green curve in Figure~\ref{fig:jacobian_analysis}), low curvature for the sampling trajectories (the green curve in Figure~\ref{fig:straightness_analysis} (left) and Figure~\ref{fig:straightness_analysis} (right) (c)), and sample generation quality (shown later in Section~\ref{subsec:model_analysis}). In the next section, we refer to this hybrid coupling as our main method. 

\iffalse
Through a hyperparameter analysis, we empirically set $\beta = 0.2$, which helps to reduce the training difficulty at early time intervals 
% (Figure~\ref{fig:jacobian_analysis}, green curve).
, as shown by the green curve in Figure~\ref{fig:jacobian_analysis}.
%
% We set $\beta = 0.2$ empirically, reducing early training difficulty .
Also, we can maintain a good generation quality
% ignificantly 
%
% improve the generation performance 
% \phil{TODO: maintain a good generation quality?}
%
(as outlined in Section~\ref{subsec:model_analysis}).
%
%Since the $\beta$ value remains 
since $\beta$ is 
relatively small, we observe that the trajectory's curvature stays low. 
%
This is revealed 
by the green curve in Figure~\ref{fig:straightness_analysis} (left) with the associated sample trajectory in Figure~\ref{fig:straightness_analysis} (right) (c).
\fi

% \paragraph{Noise Perturbation.}
% To avoid this difficulty, we propose to add additional noise the sampled training pair.
% %
% This additional noises can reduce the difficulty of the vector field estimation in the initial time intervals, and as shown in the Figure~\ref{}, we can obtain a much lower overall Jacobian norm over all intervals.
% %
% And from the generation performance comparison, we can obtain a much better performance than both naive OT Flow Matching and the Flow Matching for nearly all inference timesteps.
% \subsection{Adaptive Inference}
\label{label:subsec:adpative_inference}
We might not have this section first.