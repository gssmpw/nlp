\vspace{-3mm}
\subsection{Shape Completion}
\vspace{-3mm}
\label{subsec:shape_completion}

% \ed{I changed it to shape completion.}
% \phil{TODO: Mention conditional generation here?  Since the first paragraph in sec 4 mentions conditional generation...}

\begin{figure*}[t]
	\centering
 \vspace{-4mm}
	\includegraphics[width=0.95\linewidth]{figures/completion_comp.png}

        \vspace{-3mm}
	\caption{
  Comparisons with other training paradigms on the shape completion task.
  %
  Left: Quantitative comparisons with other alternatives using different inference steps.
  Right: Qualitative comparisons with other methods show the completion generated by 2, 5, and 100 steps, respectively. 
  % \CL{The plots to the left: The captions and insets are too small- can we make the two plots and the texts larger? 'CD' and 'EMD' not aligned vertically}
  % \av{it would be nice if the fonts for the plots on the left side were larger.}
}
\label{fig:completion_comp}
\end{figure*}
\para{Baselines.} 
We evaluate against the same baselines from Figure~\ref{fig:main_quantitative_comp} for the shape completion task.


\para{Quantitative Comparisons.}
Figure~\ref{fig:completion_comp} (left) shows the evaluation metrics (CD \& EMD) against various inference steps.
%
We observe that our model can achieve reasonable generation quality with only five inference steps, while other approaches typically require about 50 steps to obtain similar quality.
%
For Equivariant OT, it can produce comparable performance to our approach at two inference steps but fails to improve further with additional steps.
%
Minibatch OT shows poor metrics, even with inference steps, as its batch-level OT is computed under different partial shapes. 
%
This violates assumptions that OT should be computed between training instances with the same condition.

\para{Qualitative Comparisons.}
We present a visualization of shape completion results for various methods in Figure~\ref{fig:completion_comp} (right).
%
Our approach produces a reasonable-looking shape with only five inference steps, while other methods still generate noisy shapes.
%
Furthermore, Minibatch OT generates a shape that does not correspond to the input due to violating the training assumption. 
%
Equivariant OT fails to produce shapes of similar visual quality even with increased inference steps.
