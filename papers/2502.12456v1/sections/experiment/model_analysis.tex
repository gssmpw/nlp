\vspace{-3mm}
\subsection{Model Analysis}
\label{subsec:model_analysis}
\vspace{-3mm}
In the followings, we perform analysis on 
major
%different important 
modules in our approach, including the blending coefficient $\beta$ of the hybrid coupling and the effect of the superset size $M$ for the OT computation.

\begin{table}
\centering
\vspace{-6mm}
\small
{
\caption{1-NNA-CD \& EMD over 100 steps for models trained with different $\beta$.
%
% The best generation performance is achieved at a mix of
% our OT approximation and the independent coupling.
} % 
\label{tab:beta_comp}
\begin{tabular}{c|cccccccc}
\toprule
\multirow{2}{*}{Metrics} & \multicolumn{7}{c}{Interpolation Coefficient ($\beta$)}                                                   \\
                         & 0        & 0.001    & 0.01     & 0.1      & 0.2      & 0.5         & 1.0      \\  \hline
1-NNA-CD                 & 0.7289 & 0.6918 & 0.6223 & 0.5968 & \textbf{0.5853} & 0.5861 & 0.5989 \\
1-NNA-EMD                & 0.6647 & 0.6563 & 0.6156 & 0.5884 & \textbf{0.5741} & 0.5780 & 0.5869 \\
\bottomrule
\end{tabular}
}
% \caption
\end{table}
\para{Blending Coefficient $\beta$.}
We examine the impact of different couplings on flow matching model training.
% s \phil{remove s? since you only use one typical network model?}.
%
Specifically, we train models with coupling blending coefficient $\beta$ from 0 (OT approximation) to 1.0 (independent coupling).
%
For each  value, we train a model from scratch on the Chair Category and evaluate it using the 1-NNA-CD \& EMD metrics with 100 steps.
%
% We use a superset size of $10,000$ with exact OT to avoid approximation errors.
To avoid approximation errors in larger supersets, we adopt a superset size of $10,000$ with exact OT in this experiment.
% \phil{TODO: The meaning of ``potential errors is no clear}

Table~\ref{tab:beta_comp} presents the results.
%of employing different $\beta$ values.
%
We can observe that directly employing our OT approximation ($\beta = 0.0$) can lead to 
%significantly worse 
inferior performance, which can be gradually improved by injecting a small amount of noise into the coupling.
%
Interestingly, the best performance is achieved when compared with other cases at around $\beta=0.2$. 
%
Injecting more noise until arriving at independent coupling does not yield further improvement.
%
These results demonstrate that neither our OT approximation nor independent coupling is optimal in terms of generation quality, and the hybrid coupling is necessary.
% final generation performance.
% \phil{What is final generation performance? generation quality?}
%
% A proper blending helps 
% %is necessary 
% to enhance the final outcome.
% \phil{Do you need this sentence?}

\para{OT Supersets Size $M$.}
Next, we investigate the importance of constructing sufficiently large supersets for OT computation.
%
Here, we try supersets of varying sizes, starting from 2,048 (number of points to be generated) and progressively increasing the number towards 100,000.
%used in our approach.
%
As outlined in Section~\ref{subsec:ot_approx}, we employ the exact OT method for superset sizes of 10,000 or fewer points, and the OT approximation method for larger sets.
%
For each superset size, we also train a flow matching model on the Chair Category and evaluate also its generation quality over 100 inference steps.
% , measured by 1-NNA-CD and 1-NNA-EMD.

Table~\ref{tab:ot_size_comp} reports the evaluation results.
%
Overall, we notice that a small superset size usually leads to slightly worse performance, potentially due to overfitting the same target generation points. 
%
Increasing the number of points in the superset helps improve the performance.
%
Notably, despite using an approximate OT (introduced in Section~\ref{subsec:ot_approx}), we still observe some improvement in the generation quality when using a large superset ($M = 100,000$), indicating the benefit of using a large superset outweighs the errors introduced by the approximation.


\begin{table}
\centering
\vspace{-6mm}
\small
{
\caption{
1-NNA-CD \& EMD over 100 steps for models trained using different superset sizes ($M$).
% Generation performance over 100 inference steps, measured by 1-NNA-CD and 1-NNA-EMD, for models trained with our approach using different superset sizes ($M$).
%
% We observe improved generation performance with increasing superset sizes, even when using an approximation procedure for large set sizes ($M = 100,000$).
} % 
\label{tab:ot_size_comp}
\begin{tabular}{c|cccccccc}
\toprule
\multirow{2}{*}{Metrics} & \multicolumn{6}{c}{OT Superset Size ($M$)}                            \\ 
                         & 2048     & 5000        & 10000    & 20000       & 50000    & 100000   \\ \hline
1-NNA-CD                 & 0.6352 & 0.5853 & 0.5853 & 0.5853 & 0.5921 & \textbf{0.5725} \\
1-NNA-EMD                & 0.6254 & 0.5853 & 0.5741 & \textbf{0.5627} & 0.5695 & \textbf{0.5627} \\
\bottomrule
\end{tabular}
}
% \caption
\end{table}

