\vspace{-4mm}
\section{Conclusion}
\vspace{-4mm}
In conclusion, we proposed a novel framework, coined as not-so-optimal transport flow matching for 3D point cloud generation. We demonstrated that existing OT approximation methods scale poorly for large point cloud generation and showed that target OT flow models tend to be more complex, and thus, harder to be approximated by neural networks. To address these challenges, our approach introduces an offline superset OT precomputation followed by an efficient online subsampling. To make the target flow models less complex, we proposed hybrid coupling that blends our OT approximation and independent coupling, making our OT intentionally less optimal. We demonstrated that our proposed framework can achieve generation quality on par with existing works given sufficient inference steps, while achieving superior quality with smaller sampling steps.
%
Additionally, we show our approach is effective for conditional generation tasks, such as shape completion, achieving good generation even with five steps. 
%
Despite these advancements, there are still several limitations and potential extensions of this work. First, we do not consider other forms of invariance, such as rotational invariance that is present in large shape datasets such as Objaverse~\cite{deitke2023objaverse}. 
%
Second, we only consider point clouds representing coordinates. It would be interesting to explore the generation of point clouds with additional information, such as points with colors or even Gaussian splitting. 
%
\rebuttal{Third, we currently focus on generation with a fixed resolution, and it would be interesting to extend our method for resolution-invariant cases,  such as those in conditional tasks~\cite{huang2024pointinfinity}.
%
Forth, we assume our dataset does not contain outliers in the point cloud following existing works, and developing a robust learning procedure with a noisy point cloud dataset would be another interesting direction.
%
Fifth, we show empirically that hybrid coupling can reduce the trained models' complexity (as shown in Figure~\ref{fig:jacobian_analysis}), a theoretical connection between $\beta$ and the complexities is yet to be studied.}
%
Last but not least, we hope that this work can inspire further development around OT maps that are easier to learn and that our proposal, \rebuttal{especially for the hybrid coupling}, can translate to success at generating other forms of large point clouds such as large molecules or proteins.



\iffalse
, assuming these are resolved by canonically aligned datasets like ShapeNet~\cite{chang2015shapenet}. Given that recent large-scale 3D datasets, such as Objaverse~\cite{deitke2023objaverse}, are introduced without this orientation alignment, we hope to explore this invariance in efficient OT computation as well. 
%
Second, we only consider point clouds representing coordinates. It would be interesting to explore the generation of point clouds with additional information, such as points with colors or even Gaussian splitting. 
%
Lastly, we show that training using pairs from an OT map might not be optimal for generation quality. It would be intriguing to see if this observation holds for other data formats, such as images.

Lastly, we show that training using pairs from an OT map might not be optimal for generation quality. It would be intriguing to see if this observation holds for other data formats, such as images.
\fi