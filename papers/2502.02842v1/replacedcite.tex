\section{Related Work}
The study described in ____ compares various test environments that utilize 5G slicing, establishing two categories of criteria. The primary criteria require the environment to support technologies such as \gls{sdn}, \gls{nfv}, and cloud computing, ensuring flexibility and dynamism. It is crucial that the testbed provides dynamic management, orchestration, and monitoring capabilities (MANO). The environment must span multiple network domains to enable end-to-end (E2E) slicing, even if only partially. Support for multi-tenancy is essential, allowing a single infrastructure to serve multiple tenants, reflecting the capabilities designed for 5G networks.

In contrast, the secondary criteria introduce additional features for slicing testbeds, such as support for multiple radio access technologies, E2E network slicing, multi-location support, machine learning integration, and open-source characteristics. These features enable the implementation of realistic scenarios, resource optimization, and the promotion of innovation through accessible and collaborative platforms ____.

The POSENS protocol, as described in ____, offers an open-source solution for end-to-end network slicing, including both RAN and core. This protocol adopts a shared RAN approach that recognizes different slices, making modifications to the upper layers of network protocol stacks, specifically PDCP and RRC, in both eNB and UE. Using a load balancer (LB) node, packets are analyzed during tunneling between the radio and the user plane, allowing them to be redirected according to specific rules to a new interface and thus forwarded to the appropriate UPF. At this forwarding point, bandwidth limitations are applied to enable traffic prioritization across slices ____.

As detailed in ____, a network slicing test environment was developed using slice selection through the single network slice selection assistance information (NSSAI). To ensure network slice isolation, containers are used, each hosting a specific network core. The environment's evaluation focused exclusively on the connection point between normal LTE UEs and those with NSSAI capabilities. The aspect of network slice isolation was not addressed in the evaluations conducted.

According to the study described in ____, an approach was developed in which the 5G core is virtualized and containerized, implemented in a cloud infrastructure. This configuration uses a Cisco router and separates into two VLANs to enable distinct connections between the eNB and the user plane. Success in using the cloud infrastructure was reported, and as a next step, the study proposes positioning both the radio and the core at the cloud edge to improve performance and reduce latency.

Concerning slice isolation, ____ observes that the evaluated testbeds do not ensure efficient isolation, offering only partial isolation or limiting slicing to a specific protocol layer. Research such as that of ____ uses Wireguard, a secure network tunnel for Linux that operates at layer 3 and is designed to be a more secure, faster, and easier-to-use alternative compared to IPsec and OpenVPN. Although this work ensures network slice security, it does not guarantee traffic isolation between slices. Despite the various studies analyzed, it is concluded that efficient isolation remains a challenge and a rich area for future research.