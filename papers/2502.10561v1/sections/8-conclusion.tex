\section{CONCLUSION}
We contributed to the first exploration of how people with low vision (PLV) perceive, define, and use landmarks in navigation through a contextual inquiry study. We found that the categories and usage of landmarks by PLV were generally similar to those by sighted people, but included additional subcategories. For example, PLV paid more attention to ``silhouette'' of an area and landmarks with safety and accessibility implications. We designed VisiMark, a landmark augmentation AR interface. Through a user study with 16 low vision participants, we evaluated the system and found that VisiMark received positive feedback from PLV both in retracing and mental map building tasks. Additionally, VisiMark enabled PLV to better identify landmarks they prefer but could not easily perceive before, and changed PLV's landmark selection from only visually-salient objects to cognitive landmarks that are more unique and important. Our work explores the design space for future AR landmark augmentations for PLV.