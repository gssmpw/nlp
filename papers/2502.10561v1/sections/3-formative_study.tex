\section{FORMATIVE STUDY}
To fundamentally understand what landmarks PLV commonly use and how they perceive and leverage landmarks in navigation (RQ 1), we first conducted a formative study with six low vision participants using the method of contextual inquiry \cite{karen2017contextual}. The study identified the unique landmarks for PLV and provided insights on how to augment landmarks for them, thus inspiring the design of VisiMark.

%Our research involved minimal risks and received IRB approval to conduct human subjects studies.

\subsection{Methods}
% Following a user-centered design approach \cite{abras2004user}, we conducted a contextual inquiry study \cite{karen2017contextual} with 6 low vision participants to investigate how PLV perceive and use landmarks during their navigation in indoor environments. 

\subsubsection{Participants}
We recruited six participants (P1-P6, 2 females and 4 males), whose ages ranged from 33 to 78 ($Mean = 62.7$, $SD = 15.5$). Two participants (P1, P2) had prior AR experiences. Three participants (P1, P3, P6) were legally blind (i.e., best-corrected visual acuity in the better eye worse than 20/200 or visual field narrower than 20 degrees) but had functional vision to navigate. All participants wore eye glasses. Table \ref{tab:demographics_formative} shows participants’ demographic information. We recruited participants via email lists and non-profit organizations. A participant was eligible if they were at least 18 years old and had low vision but were able to use functional vision in daily activities. Participants were screened via phone or email to ensure they met these criteria. Participants were compensated \$20 per hour and were reimbursed for travel expenses. \colorchange{This study was approved by the Institutional Review Board (IRB) at our university.}

%TC:ignore
\begin{table*}[t]
\scriptsize
\centering
\begin{tabular}{>{\centering\arraybackslash}p{0.2cm}>{\centering\arraybackslash}p{1.3cm}>{\centering\arraybackslash}p{1.9cm}>{\centering\arraybackslash}p{1.3cm}>{\centering\arraybackslash}p{2cm}>{\centering\arraybackslash}p{1.9cm}>{\centering\arraybackslash}p{4.1cm}>{\centering\arraybackslash}p{2.2cm}}
%\Xhline{2\arrayrulewidth}
\toprule
% \multirow{2}{*}{\textbf{ID}} & 
% \textbf{Age/} & 
% \multirow{2}{*}{\textbf{Diagnosis}} & 
% \textbf{Legally} & 
% \multirow{2}{*}{\textbf{Visual Acuity}} & 
% \multirow{2}{*}{\textbf{Field of View}} & 
% \multirow{2}{*}{\textbf{Other Visual Difficulties}} & 
% \textbf{Prior AR} \\
% & \textbf{Gender}& &\textbf{Blind} & & & & \textbf{Experiences}\\
  \textbf{ID} & \textbf{Age/Gender} &  \textbf{Diagnosis} & \textbf{Legally Blind} &  \textbf{Visual Acuity} &  \textbf{Field of View} &  \textbf{Other Visual Difficulties} & \textbf{Prior AR Experiences} \\
% \Xhline{2\arrayrulewidth}
\hline
P1 & 67/M & Not know & Y & 20/2400 \& 20/2800 & Full & Sensitive to light &  Y\\
\hline
P2 & 78/M & Cone dystrophy & N & Not know & Full & Sensitive to light; cannot tell color shades& Y \\
\hline
\multirow{2}{*}{P3} & \multirow{2}{*}{60/F} & \multirow{2}{*}{Spinal meningitis} & \multirow{2}{*}{Y} & \multirow{2}{*}{R: 20/400; L: 20/2200} & \multirow{2}{*}{Full} & Sensitive to light; depth perception issue; & \multirow{2}{*}{N} \\
& & & & & & confused between black and blue& \\
% P3 & 60/F & Spinal meningitis & Y & R: 20/400; L: 20/2200 & Full & Sensitive to light; confused between black and blue; depth perception issue & N \\
\hline
P4 & 68/F & Macular degeneration & N & Not know & Central vision loss &  N/A & N \\
\hline
P5 & 70/M & Macular degeneration & N & R: 20/50; L: 20/50 & Full & N/A&  N\\
\hline
P6 & 33/M & Retinitis pigmentosa & Y & R: < 20/200; L: < 20/200 & Peripheral vision loss & Sensitive to light; depth perception issue & N \\
\bottomrule
\end{tabular}
\caption{Participant demographics in the formative study.}
\label{tab:demographics_formative}
  % \vspace{-5ex}
\end{table*}
%TC:endignore

\subsubsection{Procedure}
We conducted a single-session contextual inquiry study in an indoor environment. The study lasted approximately two hours. We started the study with an initial interview, asking about participants' demographics, visual conditions, and familiarity with AR. We also asked about participants' prior experiences with navigation and landmark recognition, such as the technologies they use for navigation, whether they use or try to memorize any landmarks in navigation, and what types of landmarks they commonly use.% (e.g., Do you use any technology to navigate regularly? Do you use landmarks during navigation? If yes, what types of landmarks?). 

Participants then conducted navigation tasks along two indoor routes. Both routes featured various potential landmarks, such as doors, stairs, water fountains, emergency resources, and hallway decorations. Meanwhile, the two routes presented different characteristics: one route was an open area (i.e., lobby) with more environmental dynamics (e.g., people moving around, temporary changes of tables and chairs), while the other was a hallway environment that maintained a relatively consistent setting.

For each route, participants first navigated to a destination by following verbal instructions. A researcher on the team followed the participant and provided turn-by-turn instructions when the participant approached a decision point (i.e., an intersection point where participants decided whether to turn and the turning directions). During the navigation, participants thought aloud, talking about what landmarks they saw and why they would use them as landmarks. The destination and starting point were designed to be the same location for both routes. After returning to the start, participants were asked to retrace the route to the destination again independently without any instructions. During the retracing, participants also thought aloud to discuss how they memorized the route and how they determined when and where to turn along the route. During the navigation and retracing, we observed participants' behaviors (e.g., what directions or objects they looked at or pointed to), took notes of their thinking-aloud contents, and asked follow-up questions, such as the challenges participants faced when perceiving certain landmarks, their strategies in choosing landmarks, and the rationales.

% where we observed their navigating through two different indoor routes. After the initial navigation of each route with direction instructions, we asked participants to describe the route they had just navigated and then retrace the route independently without any instructions. During the navigation, we observed participants' behaviors  and asked what objects or elements they used as landmarks, the challenges of seeing landmarks, their strategies in choosing landmarks, and the rationales. 

We ended the study with an exit interview, asking participants to summarize different types of landmarks they used as well as the importance of landmarks in wayfinding and mental map development tasks (Likert scale from 1 to 5, where 5 means most important). Lastly, participants brainstormed ideas for suitable methods to augment landmarks. \colorchange{Detailed interview questions can be found in Appendix \ref{Interview Questions for the Formative Study}.} %augmentations. Lastly, we asked participants a few exit interview questions, including rating the importance of landmarks in wayfinding and building a mental map from 1 to 5, where 1 meant least important and 5 meant most important. 


\subsection{Analysis}
We video-recorded the study and transcribed the audio data using an automatic transcription tool. We also manually went over the transcripts to correct any transcription errors and inserted our observations (e.g., participant behaviors) into the corresponding positions in the transcripts. We analyzed the transcripts using thematic analysis \cite{braun2006using,clarke2017thematic}. Three researchers open-coded four participants' data independently and collaboratively developed a codebook through discussions to resolve any discrepancies. We then developed themes and sub-themes from the codes by clustering relevant codes using axial coding and affinity diagrams. Once initial themes and sub-themes were identified, researchers cross-referenced the original data, the codebook, and the themes to make final adjustments, ensuring all codes were correctly categorized.

\subsection{Findings}
% In this section, we report how low vision participants defined, categorized, and used landmarks in their navigation and mental map building. We also investigate the challenges they face and brainstorm potential landmark augmentations.

\colorchange{Our study revealed that landmarks played an important role in both wayfinding and mental map development for PLV, with mental map development receiving a slightly higher importance rating ($M=4.67$, $SD=0.52$) than wayfinding ($M=4.17$, $SD=0.52$) but no significant difference ($t_5 = 1.94$, $p = .111$)}. Similar to sighted people \cite{downs2011cognitive,klippel2005structural, tom2003referring, lovelace1999elements}, we found that PLV also identified landmarks visually to find specific locations (P3, P6), recognize the environment and locate themselves (P3, P5, P6), and give instructions to others (P4). Most participants (P1, P3, P4, P6) stated that landmarks were more important in unfamiliar places and they would look for landmarks when visiting a place for the first time. However, different from sighted people, we identified unique types of landmarks adopted by PLV as well as the challenges they faced when locating and perceiving landmarks, which inspired the design of our landmark augmentation interface. We elaborate on our findings below.

% \paragraph{Familiarity}
% Most participants (P1, P3, P4, P6) stated that they would look for landmarks when visiting a place for the first time, while P5 was more inclined to ask others for directions. Whether familiarity changes landmark selections varies among individuals. P1 and P3 believed their landmarks would change with increased familiarity. As they became more familiar with the place, P1, P3, and P6 mentioned that they would rely on dead reckoning instead of using landmarks. In contrast, P4 and P5 claimed that familiarity does not affect their landmark selection, as they incorporate landmarks into their mental maps. These comments suggest that landmarks are more important in unfamiliar places.

\subsubsection{Landmark Characteristics for Low Vision}\label{Landmark Characteristics for Low Vision}
While echoing prior research for sighted and blind users that landmarks should be \textit{unique} and \textit{permanent} \cite{yesiltepe2021landmarks,wang2023understanding}, we found that PLV employed different features to determine landmarks due to their visual abilities. We summarize the landmark characteristics for PLV below.

\textit{\textbf{Prefer landmarks in visual modality.}}\label{Prefer landmarks in visual modality.} Despite experiencing vision loss, low vision participants relied primarily on visual cues to identify landmarks in navigation. Most landmarks they used were visual, including visually obvious and structurally salient features (categorizations in Section \ref{Landmark Taxonomy for Low Vision.}). Meanwhile, some participants also paid attention to landmarks in other modalities. Two participants (P3, P5) noticed smells (e.g., chemical smells from a lab) and sounds (e.g., music) on their routes. %confirming prior finding that PLV tended to notice sounds and smells more than sighted individuals \cite{tsuji12005landmarks}. 
However, all participants preferred identifying landmarks via visual cues, with P6 noting a preference for ``focusing more on things [he] can see than feel or [hear].'' P2 explained that he found it challenging to accurately locate the landmark with sound alone. % ``[As for the] hear stuff, I wouldn't be able to identify their relative [position].''%\yuhang{need more about why PLV prefer visual more than other modalities... i remembered that you had them before. Otherwise, we are not highlight the theme of this section.} 

%\textit{\textbf{Leverage unique object combinations as landmarks.}} Instead of individual items, many low vision participants (P2-P6) combined multiple objects to form landmarks. Since PLV often focused on visually obvious objects, selecting as many unique landmarks as sighted individuals can be challenging. As a result, low vision participants tended to use unique combinations of objects as landmarks. For example, P2 combined a water fountain and a nearby eye wash station as a landmark, and P6 used the combination of a hand sanitizer with a statue as a landmark. P3 further pointed out that selecting unique combinations of landmarks would help prevent confusion caused by repeated elements. 
%\yuhang{this sounds to be one object... a ramp with a railing? then it's not a combination}%Although P3 commented that "[I would not choose] ones (landmarks) that are really close together. They have to have some distance.", she agreed that she would pick a unique combination of landmarks to avoid confusion with things appearing multiple times. Since low vision participants faced difficulty distinguishing similar objects, a unique combination of objects could be easier for them to identify.
%\yuhang{need a stronger evidence then what i commented off.}
%In a relatively similar environment, low vision participants chose to use a unique combination of landmarks to satisfy the prerequisite of landmark uniqueness. This approach likely stems from their visual abilities, as they may struggle to identify as many unique landmarks as sighted individuals in similar settings.

\textit{\textbf{Rely on consistent landmark appearance.}}
While sighted and blind people commonly interpret permanent landmarks as not movable \cite{burnett1998turn,wang2023understanding}, our low vision participants (P3, P4, P5) stressed that the appearance of landmarks should also be consistent over time. We found that varying states of the same object could cause concerns for PLV, being perceived as different landmarks. As P3 commented on a shirt store, ``The shirts might be put in a shop or something [when the shop is open]. [But] when [the shop] closes, it looks different.'' Due to their visual conditions, PLV held a particularly high standard for object permanence when selecting landmarks.% While permanence for both sighted and blind users generally refers to landmarks being consistently located , PLV also require that the appearance of the landmarks remains consistent.



\subsubsection{Landmark Taxonomy for Low Vision.}\label{Landmark Taxonomy for Low Vision.}
According to Sorrows and Hirtle \cite{sorrows1999nature}, landmarks can be classified into three categories for sighted people---visual, cognitive, and structural landmarks. We found that landmarks for PLV aligned with these categories, but they included more unique subcategories due to PLV's visual abilities and needs. Note that these categories are not mutually exclusive; many landmarks can fall into multiple categories.%\yuhang{if possibel, it would be nice to add a row of photos to show the unique landmarks for PLV. Refer to my stair paper at ASSETS 2018.}



%TC:ignore
\begin{figure*}[t]
    \centering
    \begin{subfigure}{0.99\textwidth}
        \centering
        \includegraphics[width=\textwidth]{graphs/landmark_examples.jpg}
    \end{subfigure}
    \caption{(A) A bright hallway with a reflective floor, where shadows on the floor look like a ladder; (B-C) Staircases under different lighting conditions: dark vs. bright; (D) A ramp with railings as a landmark; (E) Danger signs as landmarks.} 
    \Description{This image shows unique landmarks for PLV. Image (A) A bright hallway with a reflective floor, where shadows on the floor look like a ladder; (B-C) Staircases under different lighting conditions: dark vs. bright; (D) A ramp with railings as a landmark; (E) Danger signs as landmarks.}
    \label{fig:landmark_examples}
\end{figure*}
%TC:endignore

%TC:ignore
\begin{figure*}[h]
    \centering
    \begin{subfigure}{0.99\textwidth}
        \centering
        \includegraphics[width=\textwidth]{graphs/color_silhouette.jpg}
    \end{subfigure}
    \caption{Perceived ``silhouette'' of an area: (A-B) Two similar hallways. (C-D) Blurred to simulate how the ``silhouette'' of an area looks to PLV, where similar hallways have different color blocks.} 
    \Description{This image shows perceived ``silhouette'' of an area: (A-B) Two similar hallways. (C-D) Blurred to simulate how the ``silhouette'' of an area looks to PLV, where similar hallways have different color blocks.}
    \label{fig:silhouette}
\end{figure*}
%TC:endignore

\textbf{\textit{Visual Landmarks.}}
Similar to sighted people \cite{sorrows1999nature}, PLV chose landmarks based on visual characteristics, which included large size (P1-P6), vibrant colors (P1-P6), high contrast (P1-P6), special textures (e.g., tiles, P2-P5), and protruding features (e.g., clocks, P2-P6). Beyond individual objects, we found that PLV were sensitive to the \textbf{\textit{lighting conditions}} in the environment (Figure \ref{fig:landmark_examples}A-C). All participants noticed the changes in lighting conditions and used them as landmarks. For example, they picked dark stairs (P1, P5, P6), dim hallways (P2), and reflective floors (P3, P4) as landmarks. These extreme lighting conditions (e.g., too bright or too dark) can significantly influence participants' visual perception and cause safety concerns, thus leaving them a deep impression. P3 explained why she noticed the reflective floor, ``All those lights are over [the floor]. It makes me think that there's something like a ladder [because of the shadows].'' 

Moreover, we also observed that beyond individual objects, PLV used the \textbf{\textit{perceived ``silhouette'' of an area}} as landmarks without discerning the specific objects (P1, P4, P5), as shown in Figure \ref{fig:silhouette}. For example, P1 mentioned that he remembered more about the hallway as a whole rather than specific objects within it, as he could not identify unique objects in the hallways. Similarly, P5 couldn't identify the specific items in the hallway, but he could differentiate between hallways based on the quantity of objects along the hallways.

In summary, the range of visual landmarks for PLV expands from individual objects to broader areas, including both lighting conditions and the overall structure or silhouette of a space.

\textbf{\textit{Cognitive Landmarks.}}
Aligned with previous research for sighted people \cite{yesiltepe2021landmarks}, PLV also selected landmarks considering their cognitive or semantic meanings. Most participants (P1-P5) selected landmarks because of personal experiences. For example, P2 was sensitive to numbers because of requirements in his previous career. Incidents that happened previously during navigation also stood out in memory (P1, P4, P5). As P1 said, ``[If I was asked to navigate again, I would remember] I've made a wrong turn [here, so] I need to go this way.'' Similarly, P5 remembered the water fountain as he used it during his wayfinding. 

Although small prints on the signs or maps could be hard for them to read, all participants paid attention to them. P3 even stopped and read the notice on the door carefully during navigation. In particular, all participants were attentive to signs indicating \textbf{\textit{danger and emergency}}, such as danger signs on the door (see Figure \ref{fig:landmark_examples} E) and emergency exits.

Elements with important functions are another typical types of cognitive landmarks used by our participants, such as restaurant (P2, P3, P5, P6), front desk (P4, P5, P6), and bathrooms (P2, P3, P5, P6). Remarkably, low vision participants paid additional attention to \textbf{\textit{landmarks with accessibility purposes}} (Figure \ref{fig:landmark_examples} D), including railings (P2-P6), ramps (P1-P6), and elevators (P2-P6), so that they can use them as mobility aids. Participants were highly sensitive to surface level changes (e.g., stairs, ramps) that can cause potential danger \cite{zhao2018looks}, resulting in memorizing them as landmarks. As P5 explained, ``I think I'm still focusing more on a [ground] level because the ground, as you know, that's a safety thing. [To avoid] falling, you have to really make sure that your ground is stable. People rely on that.'' %When it becomes difficult, let's say walking in places like shaded areas, I can't rely on my visual field to see the steps... And then you if you're much more careful, take smaller steps. '' 
For the same reason, PLV paid attention to obstacles that may obstruct their path or cause them to fall and used them as landmarks.

In summary, PLV prioritized cognitive landmarks that directly relate to their safety and accessibility needs.  %This increased awareness of features that can help or hinder their mobility highlights how PLV uniquely engage with and rely on cognitive landmarks.

\textbf{\textit{Structural Landmarks.}}\label{Structural Landmarks}
 Low vision participants also used structural landmarks, such as stairs (P1-P6), pillars (P2, P3, P5), doors (P1-P6), and walls (P1, P3) during navigation. These structural elements are important because they often signal changes in the architectural layout. For example, P1 remembered a specific door because it marked the path he needed to take, while P2 recalled the open double doors that led to another area. %Some participants picked landmarks based on their \textbf{\textit{locations}}. For instance, P2 noticed equipment an electrical cord hanging down in an unusual spot, making it stand out as a landmark.
 %\yuhang{why they are structural landmarks? sound like obstacles---cognitive.. importance comes from the location is part of the structural landmark definition, I want to show PLV's criteria align with this definition here}


Moreover, our participants showed a tendency to understand \textbf{\textit{the overall structure of the floor plan}}. They frequently leveraged the \textbf{\textit{the size of the space}} as landmarks, including both length (P1, P3, P4, P5, P6) and width (P4, P5) of a hallway. For example, P4 described the hallway as ``a much more open space'' and remembered it as a long hallway. Participants (P1, P3, P4) paid particular attention to intersections because they connected different possible routes. As P3 noted, ``When I come up to an intersection of a hallway like this, I always look and see what I can see.'' This highlights the importance of presenting an overview at intersections to support navigation decisions.


In summary, low vision participants use both structural and spatial characteristics as landmarks, such as the size and overall layout of a space. Their navigation strategies emphasize understanding connections between structural elements, although their visual conditions may affect the accurate perception of these features.

\subsubsection{Challenges and Desired Augmentations for Landmarks.} 

Low vision participants identified several challenges and desired enhancements for landmarks. We elaborate on their needs and preferences as design guidelines (DG 1-4) below to guide the design of VisiMark.


\textbf{\textit{Enhancing landmarks visually (DG 1).}} Since low vision participants predominantly prefer landmarks in the visual modality (see Section \ref{Landmark Characteristics for Low Vision}), visual augmentations should be prioritized. Although many participants (P1-P5) acknowledged the potential benefits of incorporating audio augmentation, some expressed reservations. P3 emphasized that audio feedback would not be as timely as visual cues, especially for those who walk quickly. Meanwhile, P4 was worried about potential disruption to others, noting that audio feedback could result in ``too much noise around.'' Given these concerns, visual augmentations appear to be more suitable for enhancing landmarks.

\textbf{\textit{Highlighting hallway structures (DG 2).}} Low vision participants were attentive to the overall layout and size of spaces but faced difficulty in accurately perceiving such information. For example, P3 made turning decisions based on whether a hallway seemed to lead to a dead end but misrecognized a hallway as a dead end. %saying, ``I would say don't go that way, because that looks like the end of the hallway,'' even when it wasn’t a dead end. 
She also found it difficult to distinguish between flat spaces and ramps due to the lack of depth perception. %which further complicated navigation. 
%It is thus important to augment hallway structures for PLV, including dead ends.

Participants offered suggestions for improving hallway navigation. P5 proposed painting the hallways in different colors, comparing it to how it’s done in schools: ``You [know that you] are in the red hallway, as opposed to trying to tell a five-year-old [to] go down to the third hallway [and] turn left.'' Similarly, P1 suggested highlighting the edges of the hallways, ``[highlighting the edges of the hallways] would help. [It] might also help to provide directions related to distance, so you could watch as it changes when you get closer or farther. That would help.'' Overall, it is important to augment hallway structures and space characteristics for PLV.
%participants expressed a preference for having more clues within hallways to aid in memorization and depth perception.


\textbf{\textit{Converting local landmarks to global landmarks (DG 3).}}\label{Converting local landmarks to global landmarks.} As discussed in Section \ref{Structural Landmarks}, some participants would stop at the intersections to look for landmarks ahead. However, due to visual acuity challenges, participants (P2, P3, P4, P5) reported difficulty seeing the landmarks clearly from a distance, which limited their ability to use them as global references. To address this, several participants expressed a need to be aware of upcoming landmarks, especially when their view was ``blocked off'' (P3, P6) or ``crowded'' (P3, P5, P6). They also suggested that making landmarks noticeable from different angles would be beneficial, consistent with the definition of global landmarks (see Section \ref{Landmark Definition, Categories, and Usage}). As P6 remarked, ``projecting [a flat landmark] to be 3D or something would probably be helpful.'' As a result, providing an overview with upcoming landmarks could be a useful augmentation for PLV.
%Additionally, participants tended to to focus at eye level and on the floor (P2-P6), recommending lowering elements positioned too high (P3, P4, P5) and indicating objects below eye level (P3, P6) to enhance visibility and usability. 
%\yuhang{need adjustment... see overleaf comments}




\textbf{\textit{Augmenting landmarks in-situ (DG 4).}} All participants expressed a desire to enlarge existing texts and landmarks in the environment, with most (P1-P4) advocating for large text labels on small objects (P1) or room numbers (P4). Participants also suggested coating or outlining landmarks with a bright color. As P5 mentioned, ``coating [landmarks] out with a bright red or something, that would be certainly [helpful], you know, that would be a landmark all the time for me.'' 
%All participants agreed that highlighting the landmarks with contours was a good idea. As P6 explained, ``In terms of outlining [landmarks], if there's a specific color to that, you think [the landmark] stands out the most''. 
However, P4 preferred adding icons, noting that ``the outline for me is not better than an icon.''

% In terms of color and contrast, P2 mentioned difficulties in identifying content with less vibrant colors, while several participants (P2, P4, P5, P6) pointed out challenges in reading items with low contrast. P4 commented on a sign with similar colors, stating, ``That's not a good sign.''
% When considering ways to identify unique landmarks in a similar-looking environment, P5 and P6 agreed that numbering things to distinguish them was a good idea. For example, numbering bathrooms in two connected hallways as ``Bathroom 1'' and ``Bathroom 2''. As explained by P5, ``I think [numbering is] another clue for it (landmark).''


% \subsubsection{How PLV use landmarks}
% \paragraph{Landmark importance}
% Based on the quantitative Likert scale feedback from the participants, landmarks play an important role in both wayfinding and mental map building. Specifically, for indoor wayfinding, participants rated the importance of landmarks with an average score of 4.17 (SD = 0.52). In contrast, outdoor wayfinding received a higher average rating of 4.58 (SD = 0.49). When considering mental map building, indoor landmarks were rated with an average score of 4.67 (SD = 0.52), while outdoor landmarks received an average rating of 4.58 (SD = 0.66). These results indicate that landmarks are perceived as more important for mental map building than for wayfinding indoors, while their importance is similarly high for both functions in outdoor settings.

% During the navigation, participants used landmarks to find a specific location (P3,P6), recognize the environment and locate themselves (P5,P6), as well as confirm that they were on the right track (P3,P5). P4 also mentioned that she would use landmarks to give others instructions. All these comments were aligned with previous work for sighted people (See Related Work section). 

% \paragraph{Familiarity}
% Most participants (P1, P3, P4, P6) stated that they would look for landmarks when visiting a place for the first time, while P5 was more inclined to ask others for directions. Whether familiarity changes landmark selections varies among individuals. P1 and P3 believed their landmarks would change with increased familiarity. As they became more familiar with the place, P1, P3, and P6 mentioned that they would rely on dead reckoning instead of using landmarks. In contrast, P4 and P5 claimed that familiarity does not affect their landmark selection, as they incorporate landmarks into their mental maps. These comments suggest that landmarks are more important in unfamiliar places.

% \subsubsection{Challenges with Landmark Perception.} \yuhang{need starting paragraph to summarize the unique aspects.}


% \textbf{\textit{Visual Acuity challenges}}
% Some participants pointed out the challenges they encountered due to low visual acuity. Several participants (P2, P3, P4, P5) mentioned that they had trouble in reading maps because of the small print. P3 and P4 also mentioned that they had trouble reading street signs beyond the indoor environments. In addition, some participants (P2, P3, P4, P5) could not see things in the hallway clearly from a far distance, which added to the difficulties in identifying and recognizing landmarks.

% \paragraph{Field of vision challenges}
% Regarding viewing habits, most participants (P2, P3, P5, P6) focused on eye level while being mindful of steps to avoid falling. P2 and P5 mentioned that they seldom looked up, whereas P6 noted that he tended to look up when it was crowded. Due to their viewing habits and limited field of view, it was challenging for some participants (P3, P4, P5) to recognize things in their peripheral vision. For example, P4 commented on the bathroom sign, stating it was "too high up... [and] the same color like the wall".

% \paragraph{Lighting and contrast challenges}
% Aligned with previous observations that PLV paid more attention to lighting conditions, P1 and P2 mentioned that they would not focus on or even avoid dark places. As P2 commented, "[I did not notice landmarks in the dark place] because [they are] not vibrant in a dark area."

% As for colors and contrasts, P2 mentioned that it was difficult for him to identify contents in less vibrant colors. Several participants (P2, P4, P5, P6) pointed out that they had trouble reading items with low contrast. P4 commented on a sign with similar colors, stating, "that's not a good sign".

% \paragraph{Depth perception challenges}
% As mentioned previously, P3 misrecognized a hallway dead end. Additionally, she was confused between flat spaces and ramp. Commonly poor depth perception was an issue in navigation for PLV.

% \paragraph{AR learning challenges}
% When trying the AR device, P1 and P2 encountered difficulties including challenges in dragging, grabbing, and clicking on the wrist to pull up the menu. They mentioned that there was an AR learning curve for them to get used to.
%This differs from the broader range of modalities used by blind people, yet includes more modalities compared to sighted people. 



% All participants emphasized the importance of uniqueness when they selected a landmark from the environment. As explained by P6 about why he would not choose seating areas as landmarks, "I wouldn't be trying to look for so many seaters because they are everywhere. They make me confused, and they are not distinctive things." 