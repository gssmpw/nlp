\section{EVALUATION RESULTS}
We report the effectiveness and impact of VisiMark via quantitative and qualitative data, \colorchange{as summarized in Table \ref{tab:quan_measures}}. Additionally, we developed a taxonomy for landmarks to be augmented for PLV and explored the design space for augmenting these landmarks.
%\colorchange{We summarize VisiMark’s impact on user performance and experience based on quantitative measures in Table \ref{tab:quan_measures}, and landmark taxonomies for PLV in different contexts in Table \ref{tab:landmark_taxonomies}.}%\yuhang{the findings are somehow structured in a quite arbitrary way, with system evaluation that includes all kinds of things... break them down based on research questions or themes.}

\subsection{Effect of Landmark Augmentations on Route Retracing}
We first evaluated VisiMark through participants' performance in route retracing. %We also gathered their comments on different components and aspects of the system, such as distraction and learnability. Our findings are elaborated below.
Although there was no significant effect of Condition on \textit{Retracing Time} ($F_{1,47} = 0.001$, $p = .975$, $\eta _p^2 < 0.01$), participants finished the retracing task slightly faster with VisiMark ($Mean_{with}=58.3s$, $SD_{with}=16.9s$, $Mean_{without}=59.7s$, $SD_{without}=20.7s$). As for the correctness of retracing, we found no significant effect of Condition on \textit{Retracing Correctness} ($F_{1,47} = 0.370$, $p = .546$, $\eta _p^2 < 0.01$). However, four participants made errors when they were wearing the system, while eight participants made errors without the system during retracing. The data showed that participants performed slightly better in retracing with VisiMark. Moreover, participants (10/16) mentioned explicitly that they felt themselves navigating more effectively with VisiMark since the augmentations provided more guidance along the retracing. As T8 explained, ``It was definitely easier [when using VisiMark] because I had that visual with landmarks. And I was able to, like, I almost bypassed the first turn, but then I saw the [augmented] locker or whatever it was, [I know] oh yeah, I gotta turn to [that way].'' Some participants (T7, T11, T12) memorized the turns based on the lengths of hallways provided by Signboards, as T11 said, ``I remember just going to [the] short[er] arrow way. That's how I remembered which way to go.''



\subsection{Effect of Landmark Augmentations on Mental Map Development}
We evaluated participants' mental maps from route and landmark perspectives. % via different aspects, including turns, route segments, and landmarks. % including \textit{Correctness of Turns}, \textit{Correctness of the Route Segment Lengths}, \textit{Landmark Recall Count} and \textit{Recall of
%Augmented Landmarks}, along with their qualitative feedback. 
Our findings indicate that VisiMark helped participants better perceive hallway structures, increased their focus on landmarks, and led to a shift in landmark selections. We elaborate on these findings below.

%TC:ignore
\begin{figure*}[ht]
    \centering
    \begin{subfigure}{0.9\textwidth}
        \centering
        \includegraphics[width=0.9\textwidth]{graphs/T7_mental_maps.jpg}
    \end{subfigure}
    \caption{Examples of participants' mental maps: (A) T7's mental map of route 1 with VisiMark, and (B) T7's mental map of route 2 without VisiMark. We observed a landmark selection shift here, as T7 chose more meaningful cognitive landmarks with VisiMark.}
    \Description{
    This image shows examples of participants' mental maps: (A) T7's mental map of route 1 with VisiMark, and (B) T7's mental map of route 2 without VisiMark. We observed a landmark selection shift here, as T7 chose more meaningful cognitive landmarks with VisiMark.
    }
    \label{fig:mental_maps}
\end{figure*}
%TC:endignore

\subsubsection{Perceived Hallway Structures.} We found no significant difference in \textit{Correctness of Turns} ($F_{1,47} = 2.879$, $p = .096$, $\eta _p^2 =0.06$) and \textit{Correctness of the Route Segment Lengths} ($F_{1,47} = 0.058$, $p = .811$, $\eta _p^2 < 0.01$) in participants' mental maps between with and without VisiMark. %^Participants drew the route map with a high degree of accuracy in most cases, with 87\% of turns and 89\% of route segments in 64 trials containing no more than one error.\yuhang{is this because they all painted them correctly? if so, we should say that-- Unfortunately not...} 
However, five participants reported that they could better perceive hallway structures using VisiMark than without. For example, T7, T11, and T12 memorized the lengths of the hallways due to the virtual representation of the hallways on Signboards and leveraged this information in their mental maps. As T7 said when she drew the mental map, ``I remember the big screen sign and then see this was [a] shorter hallway... and I think [another hallway] was [a] longer hallway.'' %\yuhang{add a quote about the "turning into the shorter hallway."---also, maybe this should go to the last section since it's about retracing?}
%helped them learn the environment and build more accurate mental maps, as the
%(T2,T7,T11,T12,T14)%and the corresponding hallway lengths, noting that they should turn into the shorter hallway during the retracing. 

\subsubsection{Increased Focus on Landmarks.} There was no significant effect of Condition on \textit{Total Landmark Recall} ($F_{1,47} = 0.556$, $p = .460$, $\eta _p^2 =0.01$). However, all participants thought the system increased their attention to landmarks, which helped them notice (12/16), identify (4/16), locate (10/16), and memorize landmarks (8/16). T6 and T11 both mentioned that VisiMark helped with memorization because it made the information more comprehensible, as T6 explained, ``When you make the [landmarks] more notable, it makes the association and your brain better.'' 

Moreover, three participants (T7, T13, T16) noted that they relied more on landmarks in navigation than before, considering this a positive change as it helped them memorize the environment better, otherwise they only cared about turns and the destination. As explained by T7, ``When I was not using the system, I was kind of memorizing more like, I go right, and then left and then right. But then when I was using the system... I like remember the arrows appearing and their length.'' T16 added, ``But once [the system] was taken off, it can be hard to remember where things were... [Without the system] I only know where I need to go and not like things around it.'' % \yuhang{this is interesting but not sure it should go under "Total landmark recall." Maybe we should rename this section}

\subsubsection{Impact on Landmark Selection.}
\label{Impact on Landmark Selection}
We found a significant difference of Condition on \textit{Recall of Augmented Landmarks} ($F_{1,47} = 22.428$, $p < .001$, $\eta _p^2 =0.32$), which indicates the change in landmark selections caused by VisiMark. Nine participants noted that VisiMark enabled them to notice landmarks that they had previously overlooked. %with T15 mentioning increased attention to visually salient landmarks. 
For example, five participants included the augmented bio-danger sign in their mental map of Route 1 with VisiMark, whereas none noticed it without the system since they were small. %Similarly, three out of eight participants painted a visually obvious TV screen with VisiMark, while none did without it. 
The system also enabled participants to notice objects in their blind spots. T4, who was blind in the left eye, noticed an elevator on his left with VisiMark on Route 4, while T5, also blind in the left eye, missed it on the same route without VisiMark.%The elevator was mentioned as a landmark 8 times with VisiMark, compared to only 3 times without it. \yuhang{this is very important but sort of buried at the end of this paragraph. "Facilitating landmark selection" is an important theme. You mentioned it in last section, and a little bit here. We should merge them and separate them into its own theme, with some quantitative comparison between the types of landmarks people recall that are augmented vs not augmented.} 

Landmark augmentations enabled participants (11/16) to locate more functional facilities (e.g., elevators, restrooms) and unique objects (e.g., distinctive double doors that were not highly contrasting with the environment), rather than just visually salient landmarks (as shown in Figure \ref{fig:mental_maps}), by providing additional semantic information for mental model development and memorization (7/16). As T4 remarked, ``For remembering the route, having the landmarks stated and mentioned [with] an extra [cue], that's a good point. Because the text did help, [like] an icon for a green [double] door, I would not have remembered the double door [without VisiMark].'' 
% \yuhang{this quote does not really match the argument before. The quote is talking about the overview, but the argument is for suitable landmark selections. need to replace the quote.} 
Some participants (T2, T3, T7, T11) started to pay attention to once-augmented landmarks even without wearing the system. They considered this beneficial, as the system helped them identify important landmarks more effectively. As T7 commented, ``The arrows (Signboards) really helped my understanding of the route because I wasn't like, looking at weird things on the ceiling.'' She added, ``Most of the objects that I was picking out when I did the routes with the glasses were things that you guys had in your system, rather than just ordinary things in the ordinary environment.'' Although T10 noted that the augmentation conflicted with her own way of identifying landmarks, as the system reinforced her to notice some cognitive landmarks that were not visually salient to her, she was positive about the perception change and adjusting to the system. This subtle shift in landmark selection, facilitated by the system, helped participants pick up more memorable and meaningful landmarks.
% \yuhang{these are about landmark selection again. We should really merge qualitative feedback that conveys the same theme, instead of repeating them in different subsection. It is important to find qualitative feedback to support quantitative data, but they should match.}
% Additionally, most participants (15/16) mentioned that the landmark augmentations enabled them to locate more functional facilities (11/16) as opposed to landmarks that are only visually salient, thus offering more semantic information to learn and memorize (7/16). 
% As T8 said, ``[VisiMark] gives you like a mental preparation ... like in a building that I don't know, I'm not familiar with. So it's it is helpful.'' 

%TC:ignore
\begin{table*}[h]
\centering
\small
\begin{tabular}{>{\raggedright\arraybackslash\color{brownishred}}m{3cm} >{\raggedright\arraybackslash\color{brownishred}}m{3cm} >{\raggedright\arraybackslash\color{brownishred}}m{1.7cm}
>{\arraybackslash\arraybackslash\color{brownishred}}m{8.5cm}}
\Xhline{2\arrayrulewidth}
\textbf{Tasks} & \textbf{Measures} & \textbf{Statistical Results} & \textbf{Result Interpretation \& Key Qualitative Findings} \\
%& \textbf{Details}\\
\Xhline{2\arrayrulewidth}
\multirow{4}{*}{Retracing Performance}& Retracing Time&$p=.975$&Participants retraced slightly faster with VisiMark, although the difference was not significant.\\
%&$F_{1,47} = 0.001$, $\eta _p^2<0.01$\\
\cline{2-4}
& Retracing Correctness& $p=.546$&Fewer participants made retracing errors with VisiMark, although the difference was not significant.\\
%&$F_{1,47} = 0.370$, $\eta _p^2 <0.01$\\
\hline
\multirow{8}{*}{Mental Map Development}& Correctness of Turns& $p=.096$&While no significant differences were observed in the mental\\
%&$F_{1,47} = 2.879$, $\eta _p^2 =0.06$\\
\cline{2-3}
& Correctness of the Route Segment Lengths&$p=.811$& map development, five participants reported that they could better perceive hallway structures with VisiMark.\\
%&$F_{1,47} = 0.058$, $\eta _p^2 <0.01$\\
\cline{2-4}
\multirow{3}{*}{}& Total Landmark Recall& $p=.460$&While no significant differences were found, all participants reported that VisiMark increased their attention to landmarks.\\
%&$F_{1,47} = 0.556$, $\eta _p^2 =0.01$\\
\cline{2-4}
& Recall of Augmented Landmarks& $p<.001$***&VisiMark significantly changed participants' landmark selection from only visually-salient objects to cognitive landmarks that are more important and meaningful.\\
%&$F_{1,47} = 22.428$, $\eta _p^2 =0.32$\\
\hline
\multirow{8}{*}{Subjective Experiences}& Effectiveness & $p=.036$*& Participants perceived VisiMark to be significantly more effective in supporting indoor navigation than the baseline.\\
%&$Mean_{with}=5.41$, $SD_{with}=1.02$, $Mean_{without}=4.41$, $SD_{without}=1.64$\\
\cline{2-4}
\multirow{5}{*}{}& Comfort Level& $p=.037$*&VisiMark significantly improved participants' comfort level in indoor navigation compared to the baseline condition.\\
%&$Mean_{with}=5.78$, $SD_{with}=1.05$, $Mean_{without}=4.72$, $SD_{without}=1.71$\\
\cline{2-4}
& Learnability&$Mean=6.09$, $SD=0.78$&VisiMark was easy to understand and learn, with 14 participants rating its learnability at six or higher. \\
\cline{2-4}
& Distraction&$Mean=3.03$, $SD=1.62$&
VisiMark was not distracting, with 11 participants rating its distraction level at three or below.\\
\Xhline{2\arrayrulewidth}
\end{tabular}
\caption{\colorchange{Impact of VisiMark on user performance and experiences. Statistical significance is noted as follows: * for $p<0.05$, ** for $p<0.01$, and *** for $p<0.001$.}}
\label{tab:quan_measures}
\end{table*}

%TC:endignore


\subsection{Experiences with VisiMark}
\subsubsection{Effectiveness \& Comfort}
\label{Effectiveness and comfort}
We employed a paired t-test to assess the participants' perceived effectiveness and comfort level with VisiMark in navigation. Our analysis revealed a significant difference in effectiveness ratings between conditions with and without the system ($t_{15} = 2.30$, $p = .036$), suggesting that VisiMark significantly improved perceived effectiveness ($Mean_{with}=5.41$, $SD_{with}=1.02$, $Mean_{without}=4.41$, $SD_{without}=1.64$). Similarly, we observed a significant difference in comfort ratings between with and without the system ($t_{15} = 2.30$, $p = .037$), suggesting that VisiMark significantly improved task comfort ($Mean_{with}=5.78$, $SD_{with}=1.05$, $Mean_{without}=4.72$, $SD_{without}=1.71$). Most participants (13/16) found VisiMark to be mentally comfortable, while T9, T10 and T14 reported they were more comfortable without the system because of the learning curve. As T14 explained, ``It's just what I'm used to versus wearing the [AR glasses].'' Four participants noted that discomfort mainly came from the AR device; T10 mentioned that the device conflicted with her own glasses. Furthermore, two participants (T5, T13) said that they were uncomfortable using the AR device in public, as they felt it made them more noticeable to passersby.
% \yuhang{why? privacy? awkward looking?}.


\subsubsection{Learnability}
Participants highly rated the learnability of VisiMark with a mean of 6.09 ($SD=0.78$) on a 7-point Likert scale, indicating the system was very easy to understand and use. 
% T6 and T8 noted the texts were simple to understand.
However, most participants (13/16) mentioned that there was a learning curve in getting used to VisiMark while the learning curve was short (5/16). Four participants (T2, T4, T8, T10) mentioned they needed more practice due to their unfamiliarity with AR technology. %T13 thought our component-by-component tutorial was good, while T10 suggested that we should have a computer simulation before using the AR device.

\subsubsection{Distraction}
\label{Distraction}
In terms of distraction, participants provided an average rating of 3.03 ($SD=1.62$) on a 7-point Likert scale, where 1 indicates the least distracting and 7 indicates the most distracting. Ten participants thought the system was not distracting but rather enabled them to focus on important visual information. They described VisiMark as ``more useful than distracting'' (T3, T6, T9) and ``a way to eliminate visual noise from a space'' (T5). %As T11 commented, ``I don't think it was distracting, because I knew what I was looking for.'' 
However, two participants (T4, T10) found that the augmentations distracting due to the learning curve.

As for the information intensity, some participants (T2, T6, T8) explicitly mentioned that the current intensity was suitable and not overwhelming. As commented by T8, ``More visuals may be too complex.'' 
However, T10 thought there was too much information in the current system. In the study, VisiMark augmented 0–2 landmarks in one hallway and 4-5 landmarks in one route. However, T10 considered \textit{fewer than three landmarks} in one route with two turns to be more suitable. Additionally, T5, T6, and T14 added that in some visually crowded environments, the augmentations could become overwhelming. As T5 explained, ``Sometimes [augmentations] increased visual noise and didn't particularly help.'' We discuss this issue further in Section \ref{Signboards}, focusing on what types of landmarks should only be augmented in preview.
% \yuhang{you mentioned "some spots", try to specify. I changed to visually crowded, but you can adjust}

Participants (T4, T12, T15-16) also discussed over-reliance issues since they may focus more on augmentations and look around less when wearing the system. %Additionally, two participants (T14, T15) noted that while focusing on the system, they were often distracted by people passing by, trying not to run into them. Reflecting on these attention shifts, four participants () expressed concerns about over-reliance on the system. 
They were concerned that the system offered limited information compared to the entire environment (T15), so that they would only receive information passively, relying on selected landmarks by VisiMark other than picking up landmarks by themselves (T4, T12, T16). 
%and augmentations might overlook some cues even obstacles (T6,T11-12). %As T12 explained, ``I think I did a little better at identifying landmarks without the system.'' 
%Another drawback noted was that receiving information made them less proactive, relying more on selected landmarks than picking up landmarks by themselves (T4,T12,T16).
T11 thus suggested that we should augment more on dangers, especially obstacles on the floor. We discuss this further in Section \ref{Landmarks outside their central view.}.


\subsection{Taxonomy of Landmarks to Augment} 
% \yuhang{this is not taxonomy since i don't see categories as subsections...see Jae's paper as an example of taxonomy: https://arxiv.org/abs/2407.13515}
All participants found the current landmark selection useful because they aligned with the key types of landmarks they commonly used. As explained by T7, ``A lot of the things that you will have projected on the screen here are kind of like what I drew on the mental map or things that I used to navigate.'' %Therefore, %\textbf{\textit{all landmarks that meet PLV's criteria should be augmented}}. 
However, beyond the common landmark categories we derived in the formative study, participants expressed preferences for augmenting additional types of landmarks that they would not be able to identify by themselves in daily life. \colorchange{We report the landmarks to augment in AR below and summarize the taxonomy of landmarks in both real world (from the formative study) and AR contexts in Table \ref{tab:landmark_taxonomies}.}
% The characteristics of landmarks were consistent with those identified in the formative study for PLV. Specifically, for landmark modalities, one participant (T15) noticed tactile landmarks (temperature) in addition to auditory and olfactory landmarks.



\subsubsection{Unique but visually challenging landmarks.} 
Unique but not visually obvious landmarks refer to those that are distinct in the environment for sighted people but difficult for PLV to see, including special objects that are small or lacking sufficient contrast. Participants appreciated that the system highlighted unique objects that might have been overlooked by them due to visual impairments (6/16). For example, T4 commented on a set of green double doors with low contrast augmented by VisiMark, ``For some reason I keep coming back to those [green double doors]. That was very distinct [with augmentations]. Maybe because it's a landmark that I wouldn't take for granted [with my own vision].'' PLV might not have chosen some unique landmarks as reference points without VisiMark, as these landmarks did not stand out visually to them.

%TC:ignore
\begin{table*}[h]
\centering
\small 
\begin{tabular}{>{\raggedright\arraybackslash\color{brownishred}}m{2.8cm} >{\raggedright\arraybackslash\color{brownishred}}m{4.3cm} >{\raggedright\arraybackslash\color{brownishred}}m{5.8cm} >{\raggedright\arraybackslash\color{brownishred}}m{3.4cm}}
\Xhline{2\arrayrulewidth}
\textbf{Contexts} & \textbf{Categories} & \textbf{Subcategories} &\textbf{Examples}\\
\Xhline{2\arrayrulewidth}
\multirow{9}{*}{Real-world Landmarks}& \multirow{3}{*}{Visual Landmarks}&Landmarks with visually salient characteristics \cite{sorrows1999nature} & Large, high contrast objects\\
\cline{3-4}
& &Lighting conditions* & Dark stairs, reflective floors\\
\cline{3-4}
& &Perceived ``silhouette'' of an area* &The hallway as a whole\\
\cline{2-4}
& \multirow{4}{*}{Cognitive Landmarks}& Elements with important functions \cite{raubal2002enriching}&Restaurants, restrooms\\
\cline{3-4}
& &Personal experiences \cite{sorrows1999nature}&Sensitive to numbers\\
\cline{3-4}
& &Danger and emergency* &Danger signs, emergency exits\\
\cline{3-4}
& &Landmarks with accessibility purposes*&Railings, ramps, elevators\\
\cline{2-4}
& \multirow{2}{*}{Structural Landmarks}&Structural elements \cite{sorrows1999nature}&Pillars, doors\\
\cline{3-4}
& &Overall structure of the floor plan*&Size of the space\\
\hline
\multirow{6}{*}{AR Landmarks}& Unique but visually challenging &Unique objects that are small&Small art sculptures\\
\cline{3-4}
&   landmarks& Unique objects with low contrast& Green double doors\\
\cline{2-4}

& Cognitively important but visually challenging landmarks& Recessed or flat cognitive landmarks & Elevators, restrooms\\
\cline{2-4}
& \multirow{2}{*}{Landmarks outside field of view}&Landmarks above eye level &Clocks\\
\cline{3-4}
&&Obstacles below eye level&Floor-level obstacles\\
% \hline
% \multirow{4}{*}{Augmenting Timing}& only in preview&visually obvious
% landmarks\\
% \cline{2-3}
% & \multirow{3}{*}{only in situ}& common yet important facilities\\
% \cline{3-3}
% & &object affordance\\
% \cline{3-3}
% & &small or low-contrast prints\\
\Xhline{2\arrayrulewidth}
\end{tabular}
\caption{\colorchange{Summary of two landmark taxonomies: (1) landmarks commonly used by PLV in real-world navigation derived from the formative study, with * labeling unique subcategories for PLV; and (2) landmarks that PLV preferred to augment in AR.}}
\label{tab:landmark_taxonomies}
\end{table*}
%TC:endignore

\subsubsection{Cognitively important but visually challenging landmarks.}
Visually challenging but cognitively important landmarks are those that hold cognitive significance but lack obvious visual features. They include recessed or flat landmarks, such as elevators and restrooms that are hidden in walls. As T12 noted, ``I need to see the restroom sign clearly to know it's a restroom. And the restrooms look similar to a classroom.'' T2 added, ``[Restrooms] are important, but they're very flat... The icon for the restrooms and the elevators are helpful because they do tend to be recessed or flat so it's easy for me to walk right past them.''


\subsubsection{Landmarks outside field of view.}\label{Landmarks outside their central view.} Landmarks outside PLV's field of view, such as those on walls, floors, or ceilings, could also be difficult to notice and thus being preferred to be augmented by AR. %include those on the wall, floor or ceiling, especially ones that could pose a danger. 
Since PLV, especially people with limited visual field, tend to look at eye level and below, augmenting landmarks outside their central view can reinforce their awareness of these objects (T3, T6, T7, T16). As T16 explained, ``[Augmenting things above eye level] might be nice... I don't look up very much.'' Four participants (T6, T11, T15, T16) expressed a desire to augment obstacles on the floor. Given that PLV are particularly attentive to landmarks related to safety, augmenting floor-level obstacles is important to prevent users from tripping on hazards.



\subsection{Desired Augmentation Designs: Comments and Improvements of VisiMark}
Low vision participants \colorchange{discussed their experiences and provided} insightful feedback for each design element in VisiMark. \colorchange{We report our findings below, and summarize the augmentation experiences and preferences of people with different visual abilities in Table \ref{tab:augmentation_diff_conditions}.}
%Our findings are presented below.
% They also suggested more customization options to further tailor the landmark augmentation system to meet diverse needs. 


\subsubsection{Signboards}
% \yuhang{this and the following sections are too long. Take a look at the length of other subthemes, and use them as a standard... flatten the structure. And remove some less interesting results.}
\label{Signboards}
All participants liked the design of arrows and landmark labels on the Signboard, as they provided an overview of the upcoming hallway structures and landmarks (11/16), enabling self-orientation without additional explorations (8/16). As T5 commented, ``I like knowing what's going to be in the hall before I go there, and trying to find a map in a building [is] usually a pain.'' T3 added, ``[Without the system] it would take me a lot longer to memorize and to understand where I'm at.'' Importantly, participants with depth perception loss found the overview of hallway structures particularly helpful (T5, T11, T14, T15) since it helped them identify the dead ends (T3, T5, T12). And for participants with double vision like T12, they liked the overview of hallway structures because it reduced the need to look around, which would exacerbate their double vision. T12 mentioned that the overview of landmarks was more helpful than in-situ labels for her, as she preferred focusing on visual cues in her central view. %For participants with monocular blindness, the design of Signboards pointing out all possible directions was helpful, as commented by T9, ``It (the system) would help with my legally blind eye to help me with perception and which way to go for sure.''


Participants further offered suggestions to improve Signboard design. T5 and T8 suggested including scales on the Signboards to understand the relationship between the arrows' dimensions and the actual hallways' dimensions. T16 suggested adding small arrows to further connect hallways to enhance understanding of the floor layout. Additionally, T5 and T13 recommended presenting a map of the layout of the whole space in the corner of the Signboard, together with cardinal directions. T13 also hoped for the map to show their real-time location. Overall, participants desired for more comprehensive knowledge of their surroundings, rather than information limited to their nearby vicinity.


\subsubsection{Color-coded Hallways.} Ten participants considered the color-coded hallways as helpful cues for memorization, noting that the hallway colors corresponding to the Signboard helped them confirm they were on the right track (T15-16) and made navigation easier unconsciously (T5, T12, T16). Moreover, participants with depth perception loss especially appreciated the design, as the colored overlay clearly indicated where the hallway ended (T6, T8, T11). As T6 explained, ``I like the [color-coded hallways]. It actually goes a whole length of the hallway.'' However, six participants did not find this design useful, feeling the colored hallways distracting. Nine participants suggested that the color-coded design can be combined with turn-by-turn instructions as a wayfinding support.%a preference for combining the current system with turn-by-turn instructions during the wayfinding task (e.g., always follow the same color-coded floor to the destination). 
%Most participants (15/16) liked the current color selection of color-coded hallways, appreciating the distinct primary colors used. They also suggested additional customization options, including allowing brightness adjustability, transparency adjustability, and more color choices. Specifically, T11 noted that it’s hard for her to distinguish shades, so the system should use contrasting colors. She added that using darker colors as outlines would help meet high contrast demands for PLV.

% However, some participants suggested additional customization options: T2, T4, and T11 recommended allowing brightness adjustability, and T6 suggested transparency adjustability. T7, T8, and T11 mentioned the need for more color choices. T11 noted that it’s hard for her to distinguish shades, so the system should use contrasting colors. She added that using darker colors as outlines would help meet high contrast demands for PLV. In summary, participants expressed a preference for further customization options, suggesting that personalized adjustments in color, brightness, and contrast could enhance the system’s usability for a broader range of low vision needs.


\subsubsection{In-situ Labels}
\label{In-situ labels}

All participants appreciated the in-situ icons and texts, describing them as ``focus points to tie on to'' (T5, T6), which helped them confirm they were on the right track at distance (T7, T9, T11-12). Participants showed different preferences between text and icon labels. Several participants (T4, T11, T15) preferred icons because they were simple, easy to understand, and accessible to people who cannot read. %As explained by T4, ``It's (the icon) simple... but it conveys the same thing.'' T11 added that icons could help people who cannot read. 
In contrast, some participants (T1, T5, T7, T10) preferred text labels for their clarity and details. 
% T5 noted that with the system using five categories of icons instead of a unique icon for each landmark, ``Icons aren't like the most immediately indicative of what it's supposed to be the text is.'' Similarly, T7 suggested, ``I think if they (icons) align more naturally to the environment, like the landmarks (icons) that we're used to seeing and buildings already, that would be nice.''

Since VisiMark used five categories of icons rather than a unique icon for each individual landmark, some participants recommended having unique icons for specific landmarks, such as restrooms (T2, T5, T7, T13, T14) and stairs (T1, T3, T9), to make them more explicit. As T1 explained, stairs were important for PLV due to the risk of falling: ``All I really care about is the stairs.'' Opinions on the number of icon categories varied: while T1 and T6 found the current number of categories (five) sufficient, T7 suggested having no more than 10 categories, and both T5 and T7 preferred a unique icon for each landmark. Offering multiple sets of icon choices, with unique icons for specific landmarks, would be ideal.

As for text clarity, T6 suggested avoiding abbreviations. For instance, instead of using ``AED'', the system should use ``emergency defibrillator.'' Additionally, several participants (T1, T7, T13-14) suggested including more detailed text descriptions, such as specifying gender labels for restrooms. T1 pointed out, ```Green locker is better than `locker.' I think it's just another little piece of information that would help you put it together.'' 

In general, participants preferred clear and informative in-situ labels for both icons and text, suggesting improvements such as unique icons for key landmarks and avoiding abbreviations to enhance clarity and usability.



% \subsubsection{Customization for AR Feedback} \yuhang{this section is fine, but can be removed since there's not much novel findings. each low vision paper talked about customization.}
% Besides the aforementioned improvements, participants also brainstormed additional designs, most of which were customization options. Some participants (T5, T9-10, T12) suggested adding the ability to turn on and off certain components based on personal needs. For example, people with different visual conditions had varying preferences for different components of the system (see Section \ref{Signboards}). Some participants (T5-6, T10, T12) also suggested having customizable landmark layers that can be toggled on and off, as people might have different opinions on information intensity (see Section \ref{Distraction}) and use different cognitive landmarks based on personal experiences (see Section \ref{Landmark Taxonomy for Low Vision.}).

% Furthermore, the system should allow users to adjust the positions of augmentations based on participants' preferences, with a zoom function for detailed viewing of Signboards. For example, people with double vision prefer augmentations to be positioned in the center of their field of view (T12). T6 hoped to view the entire Signboard without moving his head, while T5 proposed adding bulletins to display additional details. A zoom feature would help balance the visibility of arrows and landmark labels on Signboards, making the system more accessible.

% In summary, based on various visual conditions and preferences, future systems should offer a wide range of customization options tailored for different tasks and personal needs.

%TC:ignore
\begin{table*}[h]
\centering
\small 
\begin{tabular}{>{\raggedright\arraybackslash\color{brownishred}}m{4.5cm} >
% {\raggedright\arraybackslash\color{brownishred}}m{2.7cm} >
% {\raggedright\arraybackslash\color{brownishred}}m{5.8cm} >
{\raggedright\arraybackslash\color{brownishred}}m{12.5cm}}
\Xhline{2\arrayrulewidth}
\textbf{Components} &\textbf{Experiences and Preferences}\\
\Xhline{2\arrayrulewidth}
\multirow{4}{*}{Signboards}& Provide an overview of the upcoming hallway structures and landmarks for all participants\\
\cline{2-2}
&Enable people with depth perception loss to identify dead ends\\
\cline{2-2}
&Reduce active scanning around and facilitate focus for people with double vision\\
\cline{2-2}
&Enable people with field of view loss to notice landmarks in their blind spots (Section \ref{Impact on Landmark Selection})\\
\hline
\multirow{3}{*}{Color-coded Hallways}& Promote route memorization and confirmation for ten participants\\
\cline{2-2}
&Enable people with depth perception loss to better perceive the hallway structure\\
\cline{2-2}
& Distract six participants\\
\hline
\multirow{3}{*}{In-situ Labels}& Facilitate landmark identification and route confirmation at distance for all participants\\
\cline{2-2}
& Seven participants preferred icons as in-situ labels for simplicity and accessibility\\
\cline{2-2}
& Nine participants preferred text as in-situ labels for clarity and details\\
\Xhline{2\arrayrulewidth}
\end{tabular}
\caption{\colorchange{The augmentation experiences and preferences of participants with different visual abilities.}}
\label{tab:augmentation_diff_conditions}
\end{table*}
%TC:endignore

\subsection{Preferences on Augmenting Different Types of Landmarks}
While most landmarks should be augmented both in overview and in situ, we found that participants preferred to augment certain types of landmarks only on the Signboard or only via In-situ Labels.% report what landmarks should be augmented on Signboards vs. what landmarks should be augmented via In-situ Labels below. 

%For Signboard, participants gave opinions on what types of landmarks should only be augmented on Signboards. 
For visually obvious landmarks, while they were visible to low vision participants, participants (T5, T16) still wanted them to be augmented via Signboard to get an overview and avoid the visual distraction caused by In-situ Labels when augmenting already visually salient objects. 
%As T5 explained, ``[Augmenting visually obvious landmarks on the Signboard] makes it really easy [to identify and locate].'' T16 suggested to put visually salient landmarks only on Signboards, which was aligned with 
As T5 commented on a visually obvious window, ``Occasionally [the in-situ augmentation of a window] was a little more visual noise than necessary.'' As such, \textbf{\textit{augmenting visually obvious landmarks only in preview via Signboard}} helps reduce unnecessary visual distractions.

%Participants also noted what types of landmarks are more suitable to be augmented only in situ. 
For facilities with important functions but not unique, such as restrooms, In-situ Labels were the preferred augmentations. Participants (T4, T10, T13, T15) commented that such landmarks were not memorable because they were everywhere. However, these functional facilities played an important role in mental mapping. % as noted by T13, ``So if I just, like in my mind, try to map out windows [and] restrooms, then there could be several windows [and] several restrooms [in augmentations].'' 
Hence, \textbf{\textit{augmenting these common yet important facilities only in situ}} would be more appropriate. 
As T15 indicated, ``It's harder to tell what were restrooms and what weren't. Seeing like the augmented reality, they will point out, what is a restroom, especially if you don't know the layout, it's able to help you.'' Additionally, some participants would like to \textbf{\textit{augment object affordance in situ}}, such as augmenting door handles (T16), edges of the floor or wall (T1), and tread nosing of stairs (T1, T3, T11). 
%and having leading signs between landing in stairs (T11). 
Moreover, \textbf{\textit{small or low-contrast prints were suitable to be enhanced in situ}} (T7, T8, T11), such as signs on doors (T8) or maps (T11). %Other types of landmarks should be augmented both in preview and in situ.

% Furthermore, based on participants' different viewing habits, as some (T6, T7, T16) prefer eye-level placements, and T5 suggested that UI elements should not obstruct the central view, the system should be able to adjust the positions of augmentations accordingly. This adjustment also aligns with the needs of people with different visual conditions, such as those with double vision who prefer augmentations to be positioned in the center of their field of view.


% Moreover, the system should integrate the capability to zoom in on Signboards for detailed viewing. T6 hoped to view the entire Signboard without moving his head, which posed a challenge in maintaining accessible arrows, icons, and text on the Signboards. In contrast, T5 suggested adding bulletins on Signboards to display additional details. Therefore, implementing a zoom function would allow participants to adjust the level of information detail according to their personal preferences.



% \subsubsection{Necessity}
% As for necessity, T1 and T2 thought the augmentations would be more useful in an environment they visited regularly rather than for a one-time wayfinding task. As T1 explained, ``If I'm coming here every day, then it would be good to memorize the whole thing.'' T12 and T15 mentioned that the system would be particularly helpful in a new environment, which aligned with our finding in the formative study that landmarks are more important in unfamiliar places. Moreover, T15 emphasized the system's necessity in locating functional facilities, stating, ``If someone is coming into a building new and they might have like, disabilities or trouble seeing, it (the system) can be very useful for them finding like restroom or elevator accessible.''




% \subsubsection{Accessibility}
% 15 participants thought the augmentations were accessible, as they could adjust the font size and color, making it large and high contrast. As T7 commented, ``These are really bright. So I think especially for me if it was, like nighttime, these will be especially helpful because they're really bright and like standout in my field of vision.'' However, for participants with visual conditions like T10, who had visual acuity of 20/100 and 20/200, she noted challenges in seeing and following augmentations while walking, even after customization. She mentioned being able to see them clearly when stationary. This suggests a need for additional customization options that accommodate both walking and stationary use scenarios.

% \subsubsection{Confidence in the task}
% We used ART ANOVA to model the impact of Condition on confidence in the accuracy of the mental map, and found no significant difference ($F_{1,47} = 1.812$, $p =.185$, $\eta _p^2 =0.04$). Whether VisiMark improved their condifence in map accuracy varied in participants. Regarding confidence in retracing their steps, five participants reported feeling more confident with the system, even when they no longer used it (T3). As explained by T3, ``After I was doing the augmented reality one, I think I was paying attention to the landmarks more... So yeah, I think I was more confident after using the augmented reality, you know, your system.'' However, one participant (T10) reported that she was less confident in retracing her steps with the system because she was not used to it.




% \subsubsection{What to augment only in preview.}\label{What to augment only only in preview.}
% Participants gave opinions on what should only be augmented in preview. For visually obvious landmarks, participants (T5, T16) said that they still wanted them to be augmented even they could recognize the landmarks by themselves. As T5 explained, ``That (augmenting visually obvious landmarks) makes it really easy [to identify and locate]''. T16 suggested to put visually salient landmarks only on signboards, which was aligned with T5's comments on visually obvious landmarks (i.e., windows), stating, ``Occasionally it (in-situ augmentation of a window) was a little more visual noise than necessary''. \textbf{\textit{Augmenting visually obvious landmarks only in preview}} helps reduce unnecessary visual distractions.

% \subsubsection{What to augment only in situ.}
% Participants also noted what are more suitable to be augmented only in situ. For facilities with important function but not unique, such as restrooms, some participants (T4, T10, T13, T15) commented that they were not memorable because they were everywhere. However, these functional facilities played an important role in mental mapping, as noted by T13, ``So if I just, like in my mind, try to map out windows [and] restrooms, then there could be several windows [and] several restrooms [in augmentations].'' T15 added, ``It's harder to tell what were restrooms and what aren't. Seeing like the augmented reality, they will point out, what is a restroom, especially if you don't know the layout, it's able to help you.'' Hence, \textbf{\textit{augmenting these common yet important facilities only in situ}} would be more appropriate. Additionally, some participants would like to have \textbf{\textit{object affordance augmentations in situ}}, such as augmenting door handles (T16), edges of the floor or wall (T1), tread nosing of stairs (T1, T3, T11) and having leading signs between landing in stairs (T11). Moreover, \textbf{\textit{small or low contrast prints}} were suitable to be enhanced on-site (T7, T8, T11), such as those on doors (T8) or maps (T11).
 



% Some participants (T3, T5, T6) mentioned that the system should specialize based on the building environment. For example, we should augment different cognitive landmarks in educational (e.g., labs) versus recreational buildings (e.g., souvenir shops). T6 suggested adjusting signboards' arrows for possible curved hallways to better reflect the actual path.

% \subsubsection{Feedback based on different visual conditions}
% Our low vision participants have varying visual conditions, leading to different feedback on different VisiMark components. We highlight some typical comments below.
% \textbf{\textit{"Vision like a doughnut": central scotoma and peripheral vision deficit.}} T3 described her vision as ``like a doughnut'', as her central and peripheral vision were blurry. She liked the overview of hallway structures, noting that ``it was very appropriate for navigating and to understanding, like the width and depth perception of each hallway.'' She mentioned that she would need to look around to see all augmentations because of her visual conditions, while the current VisiMark did not specifically cater to her situation.

% \textbf{\textit{Double vision.}} For participants with double vision like T12, they liked the overview of hallway structures because it reduced the need to look around, which would exacerbate their double vision. T12 mentioned that the overview of landmarks was more helpful than in-situ labels for her, as she preferred focusing on central visual cues.

% \textbf{\textit{Monocular blindness.}} For participants with monocular blindness, the design of signboards pointing out possible directions was helpful, as commented by T9, ``It (the system) would help with my legally blind eye to help me with perception and which way to go for sure.'' 
% Furthermore, the system could help participants become aware of objects in their blind spots. For example, T4 with blind left eye noticed an elevator on his left with the system, while T5, who also has left-eye blindness, did not notice the elevator without the system.

% \textbf{\textit{With depth perception issues.}} For participants with depth perception issues, they found the overview of hallway structures particularly helpful (T5,T11,T14,T15), especially helped them identify the dead ends (T3,T5,T12). Some participants also appreciated the design of color-coded hallways, as it clearly indicated where the hallway ended (T6,T8,T11). As T6 put it, ``I like that idea (color-coded hallways). Oh, and it actually goes a whole length of the hallway.''

% [Not novel ideas. Shall we keep these?] 
% Participants also brainstormed other possible additions for augmentations. T4 suggested using flashing augmentations to draw attention, such as flashing markers for dead ends and extreme hazards. Regarding landmark saliency augmentation, T1 suggested using silhouettes with colors, and T16 proposed outlining landmarks with contours and bounding boxes, which align with ideas gathered from the formative study. 

% Participants suggested that the icons should be high contrast (T10), bright (T1), and simple (T5,T8). The current information icon was criticized for using light blue and white together (T10), and the current structural icon was considered too complex (T5,T8). 
% T7 proposed adding L and R icons on corresponding arrows.\textbf{}

% \subsubsection{Signboards: color-coded hallways}
% Additionally, T6 suggested that color-coded hallways should not break at intersections, ensuring continuity for better navigation. T16 recommended that the system avoid displaying cutting hallways behind doors to reduce visual noise and distractions.
% For structural landmarks, participants provided more examples of key locations in wayfinding, such as turning points (13/16), starting points (6/16) and destination (10/16), in addition to intersection that connect distinct areas (T5,T13).