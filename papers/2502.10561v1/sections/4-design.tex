\section{VISIMARK: LANDMARK AUGMENTATION INTERFACE DESIGN FOR LOW VISION}
%\yuhang{need to add the goal of this design. We should say something that we design an AR interface with augmentations to (1) support low vision people xxx and (2) use the interface as design probes to further understand how to best design landmark augmentations.---so that we set up a foundation for later argument that we are still highlighting all kinds of landmarks and then evaluate what to augment} 
Based on findings from the formative study, we designed \textit{VisiMark}, a landmark augmentation interface in AR to support PLV in landmark perception. Moreover, following the user-centered design approach, we will use VisiMark as a design probe to investigate \textit{what to augment} and \textit{how to augment} landmarks for PLV to inspire future navigation systems (RQ 2). We elaborate on the landmark augmentation design \colorchange{and system implementation} below.
%which included two features: signboards and in-situ icons and texts. As PLV rely more on visual cues and, given previous concerns with audio feedback, our design focused on visual augmentations. start with design guidelines distilled from the formative study and then

% should add how to link with the formative study analysis

% \subsection{Design Guidelines for Indoor Landmark Augmentations}
% We derived three design guidelines for indoor landmark augmentations for PLV based on the formative study.
% \yuhang{add transition sentences... we derive design guidelines based on the findings from the formative study.}

% \textbf{\textit{DG1. Enhancing landmarks visually.}} We aimed to enhance landmarks visually for PLV to leverage their remaining functional vision. Expanding prior insights that PLV leverage their functional vision in navigation \cite{zhao2020effectiveness}, our participants in the formative study perceived landmarks visually over other modalities and also indicated concerns about audio feedback, mentioning it may be not timely enough or potentially disrupt others. \yuhang{clarify the concerns with audio.}. As such, it is important to prioritize visual cues to enhance PLV's landmark perceptions.%This preference is supported by previous research, which highlights that PLV benefit more from visual feedback during navigation tasks. They exhibited better length perception, made fewer errors, and experienced lower cognitive load compared to when using auditory feedback \cite{zhao2020effectiveness}. 
% %Additionally, given their concerns about audio (See Section 3.3.3), enhancing visual cues should be a predominant strategy in augmenting landmarks for PLV. 

% \textbf{\textit{DG2. Highlighting landmarks with accessibility and safety implications.}} While the landmarks used by PLV largely align with the landmark categories for sighted people---visual, cognitive, and structural landmarks---our formative study identified that landmarks with accessibility and safety implications (a subcategory of cognitive landmarks) attract major attention from PLV, such as danger signs, elevators and ramps \yuhang{examples}. We should thus highlight these types of landmarks for PLV to support mobility and safety for them.% Augmenting these preferred landmarks can help them more easily identify important features that align with their mental maps, improving memorization and navigation efficiency. Additionally, since PLV are attentive to features that impact their mobility, emphasizing these landmarks also contributes to their overall safety.


% \textbf{\textit{DG3. Providing overview with space structures.}} %We sought to offer low vision users advance knowledge of landmarks and space structures. 
% Beyond local landmarks nearby, our formative study indicated that PLV face more difficulties with perceiving global landmarks at distance. Moreover, in addition to individual objects as landmarks, the overall structure of the space is also an important ``landmark'' for PLV to recognize the environment (e.g., size and layout of the space). As such, it is important to provide not only in-situ augmentations on individual landmarks but also an overview of space structures with upcoming landmarks ahead of time to allow PLV to better understand and anticipate their surroundings.


\subsection{VisiMark: AR Augmentations for Landmarks}

\colorchange{We designed VisiMark to be a head-mounted AR system. We chose head-mounted AR as it provides hands-free interactions, which can be particularly beneficial to low vision users who hold other assistive support (e.g., white cane, guide dog) in navigation \cite{thomas2009wearable}. Compared to other AR platforms (e.g., mobile AR), head-mounted AR also mitigates attention switching between the real world and an additional display, enabling users to focus on the surrounding environments and facilitating safety \cite{zhao2019designing}.} Following the guidelines, VisiMark provides visual augmentations to enhance landmarks (\textit{DG 1}). With the needs of both overview (\textit{DG 2\&3}) and in-situ augmentations (\textit{DG 4}), our interface offers two design components: \textit{Signboards} at intersections for preview and \textit{In-situ Labels} for visibility of individual landmarks. %We describe the two design components below.% first describe how to select landmarks in VisiMark, followed by a detailed introduction to the design.

\subsubsection{Signboards: Overview of Hallway Structures and Upcoming Landmarks}

We designed \textbf{\textit{Signboards}} to provide a preview of overall hallway structures and landmarks (\textit{DG} 2 \& 3) to support PLV's turning decisions at intersections (Figure \ref{fig:VisiMark}A). \colorchange{Signboards are anchored at each hallway intersection and remain stable as users change their viewing orientation and position.} They illustrate the hallway structures by using virtual arrows to represent each hallway. The arrow dimensions (i.e., length and width) were proportionate to the hallway dimensions. Moreover, we put a dead-end marker (i.e., a vertical line) on the tip of an arrow to indicate a hallway being a dead end (\textit{DG 2}, Figure \ref{fig:VisiMark}D). The physical hallways are color-coded with virtual overlays, matching the color of their arrow representations on the Signboard (Figure \ref{fig:VisiMark}B). The color-coded design allows PLV to use the hallways themselves as landmarks, and the virtual overlay also enables them to better perceive the hallway edges \colorchange{and endpoints}. %This design aimed to provide PLV, particularly those with depth perception issues, with a clear understanding of upcoming hallway structures. 
% Additionally, we implemented color-coded hallways to provide more cues for memorization beyond simple directional terms like "left" and "right," and to help users better perceive the edges of the hallways.

On the Signboard, we also labeled the upcoming landmarks along each hallway with icons and texts (\textit{DG} 3). The positions of the icons on the arrows indicated the relative positions of the landmarks in the physical hallway. %\colorchange{For example, in Figure \ref{fig:VisiMark}D, the green double door icon is positioned closer to the intersection than the danger sign, indicating their respective physical distances.} 
This feature was designed to provide a preview of distant landmarks before PLV entered the hallways. To suitably present the landmarks, we designed landmark icons based on the landmark categories derived from the formative study. %and texts. Since it's impossible to design icons for each individual landmark, we divided landmarks into different categories based on PLV's preferences from the formative study. 
Section \ref{Landmark selection} details the landmark icon selections. Besides the icons, we also used text to describe the specific landmark.

%TC:ignore
\begin{figure*}[t]
    \centering
    \begin{subfigure}{0.99\textwidth}
        \centering
        \includegraphics[width=\textwidth]{graphs/visimark.png}
    \end{subfigure}
    \caption{VisiMark design features: (A) Signboards that provide an overview of hallway structures and upcoming landmarks; (B) Color-coded hallways and their virtual representations (i.e., arrows) on the Signboard; (C) In-situ Labels with icons and texts for individual landmarks. (D) The dead-end marker at the tip of an arrow to represent the left hallway is a dead end.}
    
    \Description{This image shows VisiMark design features: (A) Signboards that provide an overview of hallway structures and upcoming landmarks; (B) Color-coded hallways and their virtual representations (i.e., arrows) on the Signboard; (C) In-situ Labels with icons and texts for individual landmarks. (D) The dead-end marker at the tip of an arrow to represent the left hallway is a dead end.}
    \label{fig:VisiMark}
\end{figure*}
%TC:endignore

\subsubsection{In-situ Labels}
We positioned large virtual in-situ icons and text in vibrant colors at the landmark's physical position to augment each individual landmark (\textit{DG} 4), as shown in Figure \ref{fig:VisiMark}C. We opted for icons and text instead of outlines and color coating to align with the icon and text labels displayed on the Signboards. The size and font color of the in-situ labels can be customized based on PLV's preferences. Moreover, to enable PLV to easily see the augmentations from different angles, the labels rotated automatically based on the user's position to ensure facing the user all the time (\textit{DG 3}). %We anticipated that these designs would enhance landmark visibility as potential global landmarks, addressing challenges related to visual acuity and contrast. 
Additionally, we numbered the same types of landmarks if they appeared multiple times in the environment to enable PLV to better distinguish them.


\subsubsection{Landmark and Icon Selection}\label{Landmark selection}
%To answer our RQ 2, one of our goals is to investigate what types of landmarks PLV would prefer to augment. Thus, 
VisiMark supported all landmark categories used by PLV and provided icons as augmentations. We categorized landmarks into five groups according to PLV's landmark preferences. Specifically, based on the original visual, cognitive, and structural landmark categories \cite{sorrows1999nature}, we expand the cognitive landmarks into \textit{Information}, \textit{Accessibility}, and \textit{Emergency} landmarks since they are the most important landmarks preferred by PLV for safety and accessibility purposes. We did not further unfold visual and structural landmarks since visual landmarks can be seen by PLV with their residual vision and structural landmarks (e.g., space structure) were mostly presented by the Signboard. %We designed icons for each type of landmarks as augmentations. %In VisiMark, we selected augmented landmarks according to these categories, while recognizing that there are many other potential landmarks along the routes. 
%For visual landmarks, since participants can effectively identify obvious features like changes in lighting and area "silhouettes", we did not further subdivide this category. For structural landmarks, we focused on specific individual landmarks and represented the overall layout in the signboard design. For cognitive landmarks, we further divided them into information, accessibility and emergency, as PLV are particularly attentive to landmarks with safety and accessibility implications that can aid or impede their mobility. 
We elaborate the five types of landmarks and their corresponding icons: 
%\yuhang{this added sentence is too arbitrary. You may want to add the rationale on the higher level, e.g., we investigate what landmarks user prefer to augment, our current design support all types of landmarks commonly used by PLV. We categorized the landmarks into xxx ...sometimes i added a comment here does not mean your edits should be exactly at this place. You want to think about your paper as whole and make adjustments in context.---my comments to you is not a small task in the local area. It's a start for you to think what is the potential issue with the current draft and how to address it.}
\begin{itemize}
    \item \textit{Visual \includegraphics[width=0.3cm]{graphs/Visual.png}}: visually obvious (e.g., high contrast, large or bright) landmarks that may draw attention, such as large green double doors and a red cork board.
    \item \textit{Information \includegraphics[width=0.3cm]{graphs/Info.png}}: landmarks that convey important information or support certain functions, such as restrooms or laboratories. 
    \item \textit{Accessibility \includegraphics[width=0.3cm]{graphs/Accessibility.png}}: landmarks that facilitate accessibility and mobility, such as elevators, ramps, and railings.
    \item \textit{Emergency \includegraphics[width=0.3cm]{graphs/Emergency.png}}: landmarks that indicate potential dangers or emergency resources, such as biohazard danger signs and automated external defibrillator (AED).
    \item \textit{Structural \includegraphics[width=0.3cm]{graphs/Structure.png}}: landmarks that represent structural elements, such as windows and stairs. They mark the building's architectural features.
\end{itemize}

\subsection{Prototype Implementation}
\colorchange{
We built our VisiMark prototype on Microsoft HoloLens 2. We pre-scanned the environment with HoloLens using the spatial mapping API \cite{microsoft-spatial-mapping} and added a QR code \cite{microsoft-qr-code-tracking} at the beginning of each route to align the landmark augmentation interfaces with the route. Since our research focused on landmark augmentation designs and their impact, we adopted this pre-scanning method to ensure accurate alignment and avoid potential confounding effects caused by real-time route and landmark recognition algorithms (e.g., recognition errors and delays). After the alignment, the system relied on the integrated motion and position tracking capability of HoloLens to track the user's position along the routes and generate corresponding augmentations at certain intersections and landmarks. 
%to avoid the confounding effect of potential computer vi added a QR code Due to the similarity of indoor environments, we chose to scan QR codes \cite{microsoft-qr-code-tracking} at the starting point of each route to hard-code landmark locations instead of spatial anchors \cite{microsoft-spatial-anchors}. 
We built the prototype with Unity 2021.3.31f1 and Microsoft Mixed Reality Toolkit 3 \cite{microsoft-mrtk3-overview}.}
