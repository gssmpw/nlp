\section{DISCUSSION} 
% \yuhang{The discussion is really rough compared to the other part of the paper. For a paper with 20 pages, the discussion should be at least 2 pages. You have two studies and a system in the paper, but the summary of contribution is just a couple of sentences... need to better highlight the contribution.}

In this paper, we explored how PLV perceive, define, and use landmarks in navigation and mental map construction. We presented VisiMark, which includes Signboards that provide overviews of hallway structures and upcoming landmarks, along with In-situ Labels for individual landmarks. We found that VisiMark received positive feedback in retracing and mental map drawing tasks, shifting PLV's landmark selection from visually obvious objects to more memorable and meaningful ones. We also summarized what PLV prefer to augment in indoor navigation and how to augment them.

In this section, we summarize the two sets of landmark taxonomy for PLV in both real-world and AR contexts. We also derive design implications for future landmark augmentations for PLV and discuss limitations and future directions.

\subsection{Landmark Taxonomies for PLV in Real World and AR Contexts}
One key contribution of this paper is that we derived two sets of landmark taxonomies for PLV in different contexts, including a taxonomy of landmarks that PLV commonly use in real world navigation (i.e., real-world context), and a taxonomy of landmarks that PLV prefer to augment via AR technologies (i.e., AR context). We summarize these taxonomies and highlight the unique landmark selections by PLV compared to prior literature. 

\subsubsection{Landmark Taxonomy for PLV in Real World}
We first derive the landmark taxonomy for PLV by expanding the landmark categories for sighted people \cite{saha2019closing}. %For landmarks PLV commonly used in daily life, as PLV primarily focus on landmarks in visual modality, the way to categorize landmarks for PLV is more similar to for sighted people than for blind people, whose categories emphasize different modalities \cite{saha2019closing}. 
Our study shows that landmarks for PLV align with the visual, cognitive, and structural categories for sighted people, but they include more unique subcategories. For visual landmarks, PLV pay particular attention to changes in lighting conditions and perceived area “silhouettes,” expanding the range of visual landmarks from individual objects to broader spaces. This reflects PLV's needs to perceive the environment holistically, as discerning fine details can be challenging. Similarly, for structural landmarks, PLV show a tendency to know the overall structure of the space, such as the size and layout of the space. For cognitive landmarks, PLV are particularly interested in landmarks that affect their safety and mobility, whether aiding or impeding it, including potential dangers and accessibility facilities. These preferences in landmark selection show PLV's intent to safely understand the surrounding environment from a macro perspective by interpreting information without relying on fine details. % and organize the landmark and wayfinding information into a coherent overall structure.

\subsubsection{Landmark Taxonomy for PLV in AR Augmentations}
Despite the landmark selection strategies developed by PLV, they still faced challenges in landmark perception, thus desire to augment an additional set of landmarks to support navigation. 
%or landmarks PLV would like to augment in AR contexts, we found that PLV are willing to augment all landmarks that meet their criteria. 
Specifically, PLV prefer to augment certain types of landmarks in AR, including unique but not visually obvious landmarks, visually challenging yet cognitively important landmarks, and landmarks outside their field of view, particularly those indicating potential dangers. These preferences arise from the challenges PLV face in identifying and perceiving meaningful landmarks that do not possess salient visual features. In addition to what to augment in AR, we also identified how to augment these landmarks. To eliminate visual noise, visually obvious landmarks should only be augmented in preview, while common yet important facilities, object affordances, and small or low-contrast prints are more preferred to be augmented only in situ. The other types of landmarks should be augmented both in preview and in situ. In general, PLV prefer to enhance landmarks that are difficult to identify independently, while maintaining an appropriate amount of visual stimuli in AR.


% \yuhang{this is too rought too... there's a just a couple of sentences about low vision. Instead, you want to highlight all uniqueness of the landmarks for PLV, cite prior papers, summarize the taxonomy, etc and DISCUSS interesting research directions. Need to distinguish discussion from a basic summary of findings.}

% \subsection{Customization and Adaptation to Different Visual Conditions}

% \colorchange{In our evaluation, we included participants with varying visual conditions, who provided different feedback based on their specific visual impairments. Here, we summarize their feedback and discuss potential customization options to adapt the current system to better accommodate their visual needs.}


% \colorchange{\textbf{\textit{"Vision like a doughnut": central scotoma and peripheral vision deficit.}} T3 described her vision as ``like a doughnut'', as both her central and peripheral vision were blurry. She mentioned that due to her visual condition, she needed to move her head around to view the entire augmentations. However, the current VisiMark system, with world-anchored augmentations, did not specifically accommodate her situation. One potential future solution is to incorporate eye-tracking technology and dynamically adjust the presentation of augmentations based on users' gaze. For example, when a user looks at a certain area, the system could automatically display relevant information within their clear field of vision, helping to ensure that key details are always visible without requiring excessive head movement.}

% \colorchange{\textbf{\textit{Double vision.}} For participants with double vision like T12, they liked the design of Signboards because it reduced the need to look around, which could worsen their double vision (see Section \ref{Signboards}). They preferred focusing on central visual cues at eye level. A possible future adaptation is to allow users to customize the positions of augmentations. For example, the system could enable users to move In-situ Labels closer to their central vision instead of anchoring them to fixed physical locations. This would reduce the need for drastic head movements and refocusing, making it easier for users with double vision to access In-situ Labels.}

% \colorchange{\textbf{\textit{Monocular blindness.}} For participants with monocular blindness (T4, T5, T9), the design of Signboards pointing out possible directions was helpful, and VisiMark has already helped them become aware of objects in their blind spots (see Section \ref{Impact on Landmark Selection}). Future adaptations could integrate multi-modal cues to alert users about augmentations in their blind spots, such as utilizing audio or haptic feedback to ensure they notice the augmented elements on their blind side.}

% \colorchange{\textbf{\textit{With depth perception issues.}} For participants with depth perception issues, they found Signboards helped them identify the dead ends, and the color-coding indicated where the hallway ended (see Section \ref{Signboards}). Future adaptations could incorporate numerical scales to help users better interpret the spatial structure of the floor plan.}

\subsection{Design Implications for Landmark Augmentations}
Throughout the study, low vision participants shared their preferred augmentation designs for future landmark augmentation systems. In this section, we summarize and expand on these design implications.

\textbf{\textit{Prioritizing Safety.}} During our experiment, low vision participants kept emphasizing the importance of safety, suggesting the inclusion of obstacle warnings on the floor and paying particular attention to landmarks with safety and accessibility implications. Since the AR headset captures their primary focus, highlighting potential dangers becomes crucial to prevent accidents caused by attention shifts. Future research should thus explore how to design effective safety warnings on AR-based navigation systems. \colorchange{One approach could be incorporating multi-modal feedback \cite{zhao2020effectiveness,bai2017smart}, such as generating audio alerts for obstacles. Additionally, integrating eye-tracking technology to design augmentations for low vision users offers another promising solution. This approach has been explored to enhance reading experiences \cite{wang2023understanding, wang2024gazeprompt}. Similar ideas could to applied to display safety warnings based on user's gaze, especially when critical hazards are overlooked.}


\textbf{\textit{Augmenting Object Affordances.}} Beyond the entire landmark, participants also expressed a need for augmenting the landmark affordance, such as door handles and stair tread nosing (see Section \ref{In-situ labels}), enabling them to easily perceive important details of these landmarks and better use these facilities. While current computer vision technology excels in whole-object recognition \cite{he2016deep,redmon2016you,liu2021swin}, affordance recognition models are still in its infancy \cite{chuang2018learning,luddecke2017learning, luo2022learning}. \colorchange{A recent work by Lee et al. \cite{lee2024cookar} has developed a kitchen tool affordance dataset and fine-tuned an object segmentation model to distinguish different interactive components of kitchen tools (e.g., knife blade vs. handle) for low vision users. Future research should expand the scope from the kitchen scenario to broader daily activities (e.g., navigation), identifying critical object affordance for low vision and developing effective affordance datasets and recognition models.}

% and   affordance detection and augmentations have already been explored in specific contexts, such as cooking scenarios through systems like CookAR \cite{lee2024cookar}. Expanding this research to other daily tasks, such as navigation or shopping, represents a promising research direction.} \yuhang{cite CookAR and discuss how we should expand}


\textbf{\textit{Understanding Surroundings Ahead.}} Despite possible visual limitations, PLV showed a tendency to know the structure of the space ahead. This is also reflected in their suggestions that Signboards should offer a broader understanding of their surroundings, beyond just their nearby environment (see Section \ref{Signboards}). This aligns with the process of cognitive map construction, which progresses from landmarks to routes and ultimately forms a comprehensive cognitive map \cite{siegel1975development}. Future landmark augmentation systems should not only enhance overviews of the nearby environment and local landmarks, but also offer a more comprehensive overview of the larger surroundings. As computer vision technology can only recognize the immediate vicinity, researchers should \colorchange{consider crowd-sourcing or community-based method \cite{howe2006rise,brambilla2014community} to achieve broader landmark information, enabling various stakeholders (e.g., friends, family, or local community) to label and share important landmarks. Such community-based method has been successfully used and deployed in accessibility research, such as Project Sidewalk \cite{saha2019project}, and could potentially be expanded to landmark labeling and augmentations for low vision users}. 

%The system could then use these labeled landmarks to generate a more comprehensive overview of the space ahead.

\colorchange{
\textbf{\textit{Gaining Agency in Landmark Selection.}} While appreciating VisiMark's assistance in landmark perception, PLV desired more control over landmark selection. They raised concerns about passively receiving landmark information selected by VisiMark and potential information overload if all types of landmarks are augmented. Future landmark augmentation systems should offer sufficient agency to low vision users, allowing them to adjust the types of landmarks to display as well as labeling additional personally relevant landmarks to facilitate memory and mental map development \cite{nuhn2017personal}. To achieve this goal, researchers should consider how to design an accessible landmark selection and labeling interface for low vision users given their challenges of recognizing certain landmarks. For example, such systems may incorporate general visual augmentation methods (e.g., magnification, edge enhancement) \cite{zhao2015foresee,stearns2018design} to enhance the environment's overall visibility, or integrate object recognition to help users identify objects of interest as potential landmarks \cite{liu2019edge, ghasemi2022deep}. Interaction techniques in AR should also be designed to enable low vision users to easily select and label landmarks. Prior research by Zhao et al. \cite{zhao2019designing_interaction} has investigated accessible interaction techniques via speech and touch gestures on smartwatch for low vision users to control different visual augmentations on AR glasses. Future research could consider expanding these techniques to better support landmark selection and labeling. }

%Additionally, researchers might consider integrating gaze-based magnification of potential landmarks. While gaze-based magnification systems have been studied for 2D screens \cite{maus2020gaze, tang2023screen, schwarz2020developing} and virtual reality \cite{masnadi2020vriassist}, their application in real-world settings for low vision users remains unexplored. These features would improve landmark perception, enabling users to interact with their surroundings more effectively and label personal landmarks easier.

%dubey2019identifying, zhu2022personalized
%, such as allowing users to toggle specific landmark types on or off via voice commands or create custom landmark labels by fixating on specific objects through gaze. Researchers should focus on making this interface both intuitive and accessible for PLV.
% \yuhang{rewrite again, not deep or interesting enough...}}

% As landmark preferences are influenced by personal experiences  and familiarity \cite{quesnot2015quantifying}, incorporating these personally relevant landmarks can help them memorize these landmarks and actively build mental maps \cite{nuhn2017personal}. Additionally, participants should have control over the amount of information displayed by the system. Providing options for users to toggle specific augmentation components and landmark layers on or off will help reduce distractions and customize the experience according to individual needs (e.g., task requirements, familiarity with VisiMark).}

% \colorchange{
% \textbf{\textit{Personalization for Visual Conditions.}} PLV experience different visual conditions and thus have different preferences for augmentations. Echoing prior insights on interface customization for PLV \cite{kane2009freedom, zhao2015foresee, zhao2019designing, stearns2018design}, our participants also exhibited different preferences and stressed the need for personalized landmark augmentations. For example, people with field of view loss (e.g., central scotoma, peripheral vision loss) desire the flexibility to reposition augmentations in the better area of their visual field; people with double vision prefer to position augmentations, especially in-situ people with low contrast sensitivity 

% For example, users should have the flexibility to adjust the placement of augmentations rather than relying solely on fixed physical positions. For those with double vision, moving In-situ Labels closer to their central vision can minimize the need for excessive head movement and refocusing, which often worsens their symptoms. Similarly, users with field-of-view loss (e.g., central scotoma, peripheral vision deficit, monocular vision) can gain better awareness of their surroundings by repositioning augmentations within their clearer field of vision. Additionally, users with low contrast sensitivity should be able to customize the colors and brightness of visual elements to improve contrast visibility. Future systems should make every component and design detail customizable, enabling users with diverse visual conditions to better access and fully utilize the system. \yuhang{as we discussed, motivate this paragraph from different visual function perspectives, summarize preferences based on visual functions, and discuss how we should enable personalization.}}

%Users with field of view loss (e.g., central scotoma and peripheral vision deficit, monocular vision) should be able to move the augmentations outside of their . 
% For example, some users reported a learning curve and gradually adapted to VisiMark’s information intensity throughout the experiment, indicating that the system should allow users to adjust the intensity to match their individual tolerance levels.


% \textbf{\textit{Trade-off in Information Intensity.}} While participants expressed a desire to augment different landmarks, obstacles and affordances, they also noted their concerns about over-reliance on the system. Therefore, a trade-off exists between information intensity—providing enough information to facilitate the perception of landmarks—and maintaining awareness of the real-world environment. The trade-off also applies to the learning curve, as the threshold for information intensity may vary between beginners and experienced users. As reported by 13 participants, there was a learning curve with VisiMark, with users gradually got used to the information intensity throughout the experiment. \colorchange{Researchers should initially set appropriate levels of information intensity and provide options for users to customize and toggle different augmentation components and landmark layers on or off based on personal preferences, to avoid potential visual overwhelm.}

% \colorchange{
% \textbf{\textit{Personal Customized Landmarks. }}In Section \ref{Distraction}, participants mentioned that they sometimes passively received information from VisiMark. In addition to the previously mentioned option of allowing users to control the amount of information they receive from the system, another potential solution is to enable users to add their own landmarks. Since users select cognitive landmarks based on their personal experiences (see Section \ref{Landmark Taxonomy for Low Vision.}), and their landmark choices are influenced by familiarity \cite{quesnot2015quantifying}, allowing them to incorporate these personally relevant landmarks can help them to memorize these landmarks and actively build mental maps \cite{nuhn2017personal}.}


\subsection{Limitations and Future Directions}
Our study has some limitations in both system design and user studies. From the system design and implementation perspective, we reveal below limitations and suggest potential future directions:

\textbf{\textit{Integrating AI into AR.}} Our current implementation relies on a pre-scanned environment to reduce the impact of technical limitations (e.g., recognition errors) on user experience. To enhance real-world applicability, we should implement AI-powered AR for real-time landmark recognition and augmentation in the future. 

\colorchange{\textbf{\textit{Adapting to more commonly-available AR platforms.}} Current VisiMark is built on a head-mounted AR device to enable hands-free experience and mitigate attention switching between the real world and an additional display (e.g., if using mobile AR) \cite{thomas2009wearable}. However, some PLV found the form factor of head-mounted AR uncomfortable. Future research should consider adapting VisiMark to more commonly-used and accepted devices, such as smartphone or smartwatches. Importantly, different feedback modalities and designs should be considered for different platforms. For example, as opposed to visual augmentations on head-mounted AR, vibrations on the smartwatch or auditory descriptions from the smartphone could be used to avoid constant visual attention shifts between the device and the real-world environment. }

\colorchange{\textbf{\textit{Allowing more Flexible Customization.}} Current VisiMark allows users to adjust the height, size, and colors of the augmentations. However, more flexible customization should be considered due to low vision users' diverse visual abilities. For example, since VisiMark anchors Signboards and In-situ Labels to certain physical locations, participants with visual field loss may have to look around to view the whole augmentations. Future design should introduce more customization options, such as repositioning augmentations (or different components of an augmentation, e.g., arrows on the Signboard) to user’s functional visual areas rather than relying solely on fixed physical positions, enabling users with diverse visual conditions to better access and fully utilize the system.} %\yuhang{add another limitation that we should consider more customization options, e.g., repositioning different components of the signboard for people with field of view loss. mentioned by 2AC.}

\colorchange{From the user study perspective}, first, our experiments were conducted indoors due to the current visibility constraints of head-mounted AR feedback in outdoor settings. Outdoor environments are more dynamic and more visually complex. Future work should explore additional landmark augmentation designs and preferences for PLV in outdoor environments. 
\colorchange{Second, during our evaluation, we provided verbal turn-by-turn instructions to guide the participants in the initial exploration phase. This decision is informed by low vision people's daily experience as they commonly ask for verbal instructions from others in navigation \cite{szpiro2016finding}. However, reliance on verbal instructions rather than visual landmarks may influence our measurements---especially retracing and mental map of the route segments---potentially reducing the observed differences in performance between VisiMark and the baseline. In fact, one participant (T10) in our study mentioned relying more on the verbal instructions to memorize the route. %Additionally, delivering instructions via the auditory channel, rather than visually, introduced a different input modality that could have affected participants’ route memorization. 
%This may have impacted metrics such as Correctness of Turns, Total Landmark Recall, Retracing Time, and Retracing Accuracy. 
Future research could consider other instruction alternatives, such as visual indicators or landmark-based instructions, to triangulate the results.} 
\colorchange{Third, we had a relatively small number of participants (six) in our formative study to provide design implications. Future research should include a larger group of participants with a broader range of visual conditions to enrich the data.} Lastly, our current study employed relatively \colorchange{short and simple routes, which may also lead to insignificant difference in performance and restrict the assessment of long-term mental map development. Future evaluation should consider more complex environments to better simulate real world conditions and assess the impact and practicality of VisiMark.}

%due to the current visibility constraints of head-mounted AR feedback in outdoor settings.  To address safety concerns related to constant visual attention shifts between the screen and the real-world environment, the design should be modified. One potential solution is to incorporate multi-modal feedback, such as vibrations on the smartwatch and auditory descriptions when approaching a landmark. Further design options should be examined and explored in future work.
%  Therefore, the application scenarios are limited. Firstly, participants were not allowed to customize the font size and color after the tutorial, as we focused on evaluating the system under consistent conditions. However, given the learning curve of VisiMark, allowing customization to better suit individual visual conditions during tasks could significantly affect the users’ experience. For example, T10 reported challenges in seeing and following augmentations while walking after customization in the tutorial, although she was able to see the augmentations clearly when stationary. Thus, it is important to consider both dynamic and stationary using scenarios and include feedback control in future studies.

% as discussed in \ref{Incorporating Real-time AI Assistance.}, we should enable AI-powered AR landmark automatic augmentation in real-time in the future, although real-time landmark augmentation may face challenges like recognition accuracy and system latency. Besides, we plan to allow users label their own landmarks, such as those based on personal experiences, as a complement to the system's automatically augmented ones. Moreover, given participants' feedback on current device discomfort and the unease of using the device in public, important future directions include finding ways to make the system's value outweigh current device limitations, and protecting users' privacy in public.

% \subsection{Comparison of the taxonomy of landmarks among low vision, sighted, and blind individuals}
% Based on our study results, we identified unique aspects of landmark selection for PLV. We compare them with those of sighted and blind people below.

% \textbf{\textit{Landmark Modalities.}} Similar to sighted people, PLV primarily rely on visual cues and prefer visual augmentations. However, landmarks for PLV often encompass a broader range of modalities compared to those used by sighted individuals. PLV individuals may also use auditory and olfactory cues (See \ref{Prefer landmarks in visual modality.}). This contrasts with blind people's wider array of modalities, with tactile cues being the most frequently used modality \cite{tsuji12005landmarks,jeamwatthanachai2019indoor,wang2023understanding}. The preferred landmark modalities form a spectrum between sighted and blind individuals, with visual cues on the sighted end and a greater variety of modalities on the blind end. Given the limitations of their visual abilities, PLV tend to use visual cues as dominant in navigation but incorporate multiple modalities for assistance, positioning them in the middle of this spectrum.

% \textbf{\textit{Landmark Characteristics.}} For the prerequisite characteristics of landmarks, we found that PLV emphasized uniqueness and permanence, similar to sighted and blind people, but interpret these characteristics differently. For uniqueness, PLV often use unique combinations of objects as landmarks, in addition to distinct individual items, since identifying as many unique landmarks based on visual features alone can be challenging for them compared to sighted people. For permanence, PLV require landmarks to have a consistent state and appearance, beyond simply being fixed and not temporary, which are the interpretations of sighted and blind people. These strategies stem from their visual abilities, different from those of both sighted and blind people, and result in subtle differences in their landmark selections.


% \textbf{\textit{Landmark Categories.}} As PLV primarily focus on landmarks in visual modality, the way to categorize landmarks for PLV is more similar to for sighted people than for blind people, whose categories emphasize different modalities \cite{saha2019closing}. Our study showed that landmarks for PLV align with visual, cognitive and structural dimensions for sighted people, but they include more unique subcategories. For visual landmarks, PLV pay particular attention to changes in lighting conditions and perceived area “silhouettes”, expanding the range from individual objects to broader spaces. This reflects their need to perceive the environment holistically, as discerning fine details can be challenging. Similarly, for structural landmarks, despite possible visual limitations, they showed a tendency to know the overall structure of the space. For cognitive landmarks, PLV are particularly mindful of those that affect their mobility, whether by aiding or impeding it, including potential dangers and accessibility facilities. These preferences in landmark selection show PLV attention to safely understand surrounding environment from a macro perspective. They develop strategies to interpret their surroundings without relying on fine details, yet still demonstrate a strong inclination to organize the information into a coherent overall structure.

% Overall, PLV develop distinct strategies to understand their surroundings and select landmarks. While their preferences often fall between those of sighted and blind individuals, they also exhibit unique focal points. In the future, exploring how to combine existing systems for sighted and blind users, along with customizations for PLV based on our design guidelines, could lead to more cost-efficient solutions compared to developing an entirely new system.

% Our study found that PLV would like to augment all landmarks that meet their criteria. The system increased similarity between environment memory and augmentations, indicating that the selection of landmarks is crucial. The system should pick out unique and permanent landmarks to eliminate visual noise for PLV. Moreover, since users tend to focus more on the system and pay less attention to their surroundings, it is necessary to augment visually challenging but important landmarks. This includes recessed or flat objects and those outside their central view, especially dangers such as obstacles. Additionally, participants hoped the system would incorporate object affordance augmentations in real time, as well as turn-by-turn instructions together with augmentations.

% \subsubsection{When to augment}
% Our study indicated that the system would be most useful in unfamiliar places and environments that require memorization, such as regularly visited locations. Moreover, PLV prefer different types of landmarks to be augmented in different stages. Visually obvious landmarks should only be augmented in preview, while common yet important facilities and object affordance are more suitable to be augmented in situ. Other landmarks should be augmented both in preview and in situ.

% \subsubsection{How to augment}
% Both signboards (as preview) and in-situ designs received positive feedback. The augmentations should be simple, large, bright, and high contrast in general. Due to various visual conditions and preferences, future systems should offer a wide range of customization options tailored for different tasks, environments, and personal needs. The detailed customization options mentioned by the participants include the ability to toggle augmentations on and off, adjust their positions, zoom in on augmentations, design personal landmark layers, and customize colors and icons. Eye tracking could be incorporated into the system to help people with different visual conditions use the augmentations more efficiently. For example, for people with central scotoma and peripheral vision deficits, we could automatically move the augmentations slowly based on their fixation, allowing them to see the augmentations without having to move their heads around. Additionally, the tutorial of the system should be carefully designed to help shorten the learning curve.%, taking into account changes in visual ability while standing and walking.