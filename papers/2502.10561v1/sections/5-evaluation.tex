\section{EVALUATION METHOD} \label{sec:eval}
We seek to evaluate the effectiveness and impact of VisiMark on PLV's landmark perception. Specifically, our study will answer: (1) How does VisiMark affect PLV's retracing performance? (2) How does VisiMark affect PLV's mental model development and landmark selections? (3) What types of landmarks do PLV need to augment, and (4) what are the suitable landmark augmentations preferred by PLV? %We collected both quantitative performance data (e.g., mental map accuracy, route retracing time and accuracy) and qualitative feedback from participants and brainstormed with them on potential landmark augmentation designs.% ideas on how to augment them. 
% \yuhang{might be good to list specific research question for the study: (1) How does VisiMark affect PLV's retracing performance? (2) How does VisiMark affect PLV's mental modal development? (3) How does VisiMark affect PLV's landmark selections? (4) What landmarks PLV prefer to augment and how to augment them? }

\subsection{Participants}
We recruited 16 low vision participants (8 females, 7 males, 1 non-binary) whose ages ranged from 18 to 72 ($Mean = 44.8$, $SD = 19.8$). Five participants had prior AR experiences. Two participants (T8, T11) were legally blind but had functional vision to navigate. All participants wore eye glasses to partially correct their acuity. The recruitment method and compensation were the same as the formative study. Table \ref{tab:demographics_evaluation} shows participants' demographic information in detail. Two participants also took part in the formative study: T2 (P5) and T10 (P3).


%TC:ignore
\begin{table*}[ht]
\centering
\scriptsize
\begin{tabular}{>{\centering\arraybackslash}m{0.5cm} >{\centering\arraybackslash}m{0.7cm} >{\centering\arraybackslash}m{2.5cm} > {\centering\arraybackslash}m{1cm} >{\centering\arraybackslash}m{2cm} >{\centering\arraybackslash}m{3cm}  >{\centering\arraybackslash}m{4cm} > {\centering\arraybackslash}m{1.2cm}}
\Xhline{2\arrayrulewidth}
\textbf{ID} & \textbf{Age/ Gender} & \textbf{Diagnosis} & \textbf{Legally Blind} & \textbf{Visual Acuity} & \textbf{Field of View (FOV)} & \textbf{Other Visual Difficulties} & \textbf{Prior AR Experiences} \\
\Xhline{2\arrayrulewidth}
T1 & 66/M & Right eye damaged & N & R: 20/60; L:20/20 & Two-thirds of the lower vision loss in right eye & Less sensitive to light; experienced discomfort from bright side glare, causing vision flare; depth perception issue &  Y\\
\hline
T2 & 72/M & Macular degeneration & N & 20/40 \& 20/50 & Full & N/A& N \\
\hline
T3 & 31/F & Retinitis pigmentosa & N & Not know & Like a donut, the central and peripheral are blurry & Sensitive to light and contrast & N \\
\hline
T4 & 55/M & A complete loss of vision in left eye & N & Near sighted right eye, blind left eye & Full & No depth perception & Y \\
\hline
T5 & 19/Non-binary & Blind left eye& N& Near sighted  right eye, blind left eye & Full & No depth perception & Y \\
\hline
T6 & 69/M & Retinopathy of prematurity & N & R: 20/400; L: 20/30 & Central vision loss in right eye & Sensitive to light in right eye; depth perception issue & N \\
\hline
T7 & 26/F & Atrophy of optic nerve & N & 20/400 \& 20/70 & Central is a lot worse than peripheral & Sensitive to contrast & N \\
\hline
T8 & 40/M & Macular myopic degeneration & Y & R: 20/2400; L: 20/30 (with correction) & Central vision degeneration in both eyes, peripheral vision loss in right eye & Sensitive to light; color blind, confused between green, yellow, red and orange; depth perception issue & N \\
\hline
T9 & 50/F & Blind right eye & N & R: <20/200; L: 20/20 (with correction) & Full & Sensitive to light; depth perception issue & Y \\
\hline
T10 & 71/F & Chemo induced type of neurological cornea issue and macular degeneration & N & R: 20/100; L: 20/200 & Full & Photophobia; confused between red and blue, as well as red and black & N \\
\hline
T11 & 58/F & Retina pigmentosa & Y & 20/50 \& 20/50 & Peripheral vision loss & Sensitive to light; confused between pink and orange, as well as brown, blue and grey; depth perception issue & N \\
\hline
T12 & 27/F & Weaker eye muscle and double vision & N & R: 20/800; L: 20/400 (with correction: R: 20/25; L: 20/30) & Full & Depth perception issue & N \\
\hline
T13 & 59/M & Glaucoma & N & 20/2400 in one eye, relativtly good acuity in the other & Top left vision loss in right eye and top right vision loss and peripheral vision loss in left eye & N/A & N \\
\hline
T14 & 33/F & Congenital cataract and amblyopia in right eye & N & R: 20/500; L: 20/20 & 70 degrees & No depth perception & N \\
\hline
T15 & 22/M & Blurred vision that cannot be corrected by lens or contact lens & N & R: 20/40 (with correction); L: 20/40 (with correction) & Full & Depth perception issue when it gets darker & Y \\
\hline
T16 & 18/F & Ocular albinism & N & R: 20/100; L: 20/80 & Full & Sensitive to light; depth perception issue & N \\
\Xhline{2\arrayrulewidth}
\end{tabular}
\caption{Participant demographic information for the final evaluation.}
\label{tab:demographics_evaluation}
\end{table*}
%TC:endignore

\subsection{Apparatus}
% To support the study, we implemented the VisiMark interface on Microsoft HoloLens 2 and prepared four similar indoor routes for navigation tasks. We describe our environment setup and system implementation details below.

% \textbf{\textit{Environment Setup.}} 
We conducted the study in a well-lit indoor environment. We used one floor in one of our academic buildings as our study environment and planned four similar routes with similar lengths and equal number of decision points \cite{ishikawa2008wayfinding, may2020spotlights}. All routes were approximately 65 meters long and included three decision points, consisting of two turns and one additional decision point, as shown in Figure \ref{fig:route_planning}. We also augmented similar number of landmarks on each route (4-5 landmarks per route). Our landmark selection ensured the coverage of all five categories of landmarks, with each route covering 3-4 landmark categories. We chose landmarks with the most salient visual, informative, accessible, emergent, or structural features within each category, such as a large TV screen as visual landmarks and an elevator as accessibility landmarks, as shown in Figure \ref{fig:route_planning}. %\yuhang{i'm still not highly convinced...we have five categories, why here you only mentioned three types of features?.}%In particular, due to the limitations of the indoor environment, we counted two consecutive turns in Route 2 as a single turn. 
We also set up a training area with one intersection and three hallways on another floor for the tutorial session.
% Specifically, we selected four landmarks on Route 2 and Route 3 to augment, and five landmarks on Route 1 and Route 4.

%TC:ignore
%\vspace{-1ex}
\begin{figure*}[ht]
    \centering
    \begin{subfigure}{\textwidth}
        \centering
        \includegraphics[width=0.9\textwidth]{graphs/route_planning.jpg}
    \end{subfigure}
    \caption{Four routes in the study environment with labeled landmarks augmented in VisiMark, including both those along the routes and other landmarks in the space displayed on Signboards. Detailed landmarks and their types for each route are listed on the right.}
    % \yuhang{add a table on the side of the map, labeling the landmarks and types for each route maybe following left to right and up to down order.}
    \Description{This image shows four routes in the study environment with labeled landmarks augmented in VisiMark, including both those along the routes and other landmarks in the space displayed on Signboards. Detailed landmarks and their types for each route are listed on the right.}
    \label{fig:route_planning}
\end{figure*}
%\vspace{-1ex}
%TC:endignore

% \textbf{\textit{Implementation.}} 
\colorchange{Since PLV may experience different visual conditions, we allowed participants to customize the AR interface to refine their visual experience and ensure the visibility of landmark augmentations.} Current customization included toggling on and off each design component (e.g., icon and text labels on Signboards, color-coded hallways) and adjusting parameters of each component (e.g., font color, font size, icon size, Signboard size). \colorchange{We implemented the customization through the method of Wizard of Oz \cite{dahlback1993wizard}, where the participants used speech commands to customize the interface and the experimenter made the adjustments accordingly via a smartphone application we developed.} The phone application also tracked the system data of VisiMark (e.g., virtual elements shown on the AR display) and \colorchange{presented it to the experimenter via text logs to allow them to better understand the participant's experience}. The phone application was built with React Native and communicated with the HoloLens via WebSocket. 

% and log system data  in real-time during the navigation to ensure the experiment went smoothly. 

% To enable customization, we employed Wizard of Oz studies , allowing the experimenter to tailor VisiMark according to the participants' specific needs.

\subsection{Procedure}
The study consisted of a single session that lasted approximately two hours. We started with an initial interview, asking about participants' demographic information, visual conditions, and experiences with AR and navigation. We also asked about their experiences with landmarks, such as what landmarks they commonly use and why.
%Detailed questions are listed in Appendix \ref{}.

We then conducted a tutorial to familiarize participants with our system. We introduced each design feature of the system and explained the meaning of different icons for each landmark category. Participants were then able to customize some design parameters based on their visual ability, including the size of the Signboard, the size of icons (both on Signboard and In-situ Labels), and the size and color of the text labels. Additionally, the height of the Signboards was adjusted based on the participants' heights for an eye-level rendering. During the tutorial, participants were encouraged to walk around the tutorial area to freely experience VisiMark.

% Our aim was to address AR learning challenges effectively through the tutorial.

After feeling fully familiar and comfortable with VisiMark, each participant performed four trials of navigation tasks under two conditions (two trials per condition): (1) wearing an AR device and (2) not wearing an AR device. Each navigation task consisted of three phases: \textit{Exploration}, \textit{Mental Map Drawing}, and \textit{Retracing}. In the exploration phase, participants freely navigated along a pre-planned route by following verbal turn-by-turn instructions. \colorchange{A researcher on the team followed the participant and provided verbal instructions when the participant approached each intersection (e.g., ``Turn right at this intersection''). We provided verbal guidance (1) to simulate users' real-world indoor navigation experiences since low vision people often ask others for verbal route descriptions during navigation \cite{szpiro2016finding}, and (2) to avoid potential interference with the visual landmark augmentations provided by VisiMark.} During this navigation, participants were free to look around, observe the environment, and walk at their comfortable pace. After reaching the destination, participants were asked to draw a mental map of the route, including any landmarks they remembered along the way, as accurately as possible (mental map drawing phase). Participants also assessed their confidence in the map's accuracy with a rating between 1 to 7, where 7 meant extremely confident and 1 meant extremely unconfident. Finally, in the retracing phase, we led the participants back to the starting point along a different route to prevent them from seeing the original path. Participants were then asked to retrace the original path and reach the destination as quickly as possible without any verbal instructions.




We counterbalanced the order of the two conditions (wearing or not wearing the AR device) and the four routes using Latin Square, resulting in eight (two conditions $\times$ four route orders) combinations. With 16 participants recruited in the study, each combination was assigned to two participants randomly.% assigned to the conditions, with each condition being repeated twice. %The detailed arrangement is illustrated in Table \ref{tab:latin_square}.

% \begin{table}[ht]
% \centering
% \caption{The Latin Square used to counterbalance the four routes and the two conditions (wearing or not wearing the AR device).}
% \label{tab:latin_square}
% %\scriptsize
% \begin{tabular}{c c c c c}
% \Xhline{2\arrayrulewidth}
% 1 & Route 1 \& with AR & Route 2 \& without AR & Route 3 \& with AR & Route 4 \& without AR \\
% \hline
% 2 & Route 2 \& without AR & Route 3 \& with AR & Route 4 \& without AR & Route 1 \& with AR \\
% \hline
% 3 & Route 3 \& with AR & Route 4 \& without AR & Route 1 \& with AR & Route 2 \& without AR \\
% \hline
% 4 & Route 4 \& without AR & Route 1 \& with AR & Route 2 \& without AR & Route 3 \& with AR \\
% \hline
% 5 & Route 1 \& without AR & Route 2 \& with AR & Route 3 \& without AR & Route 4 \& with AR \\
% \hline
% 6 & Route 2 \& with AR & Route 3 \& without AR & Route 4 \& with AR & Route 1 \& without AR \\
% \hline
% 7 & Route 3 \& without AR & Route 4 \& with AR & Route 1 \& without AR & Route 2 \& with AR \\
% \hline
% 8 & Route 4 \& with AR & Route 1 \& without AR & Route 2 \& with AR & Route 3 \& without AR \\
% \Xhline{2\arrayrulewidth}
% \end{tabular}
% \end{table}


\subsubsection{Exit Interview}
We ended our study with a semi-structured interview, discussing participants’ landmark selections without VisiMark, the impact of our system on their landmark selections, and their suggestions for landmark augmentations in AR. Participants were asked to rate their perceived effectiveness and comfort for completing the tasks with and without the system, as well as the distraction and learnability of VisiMark on a 7-point Likert scale. \colorchange{Detailed interview questions can be found in Appendix \ref{Interview Questions for the Evaluation}.}%Participants were asked to rate their perceived effectiveness and comfort for completing the tasks with and without the system on a scale from 1 to 7, where 7 meant extremely effective/comfortable and 1 meant extremely ineffective/uncomfortable. Participants also assessed the distraction and learnability of VisiMark on a scale from 1 to 7, where 1 indicated least distracting/easy to learn and 7 indicated most distracting/easy to learn.

% We collected participants' subjective feedback evaluating the effectiveness and comfort ratings for completing the task both with and without the system. Additionally, we gathered Likert scale ratings to evaluate the distraction and learnability of the system. 
% Detailed questions are listed in Appendix \ref{}.


\subsection{Analysis}
We collected both quantitative measures and quantitative feedback from the study. Our analysis methods are as below.

\subsubsection{Measures \& Statistical Analysis}

% \paragraph{Wayfinding task evaluation}
We defined two measures for the navigation task: (1) \textit{Retracing Time}, the time spent by a participant to retrace their route from the starting point to the destination; and (2) \textit{Retracing Correctness}, a binary value to indicate whether a participant made a wrong turn or failed to reach the destination.

% \paragraph{Mental map evaluation}
We also defined four measures for mental map evaluation: (1) \textit{Correctness of Turns}, the number of correct turns minus the number of incorrect extra turns; (2) \textit{Correctness of the Route Segment Lengths}, the number of adjacent segments that have the correct relative length relationship (i.e., longer than or shorter than), following the method in Aginsky et al. \cite{aginsky1997two}; (3) \textit{Total Landmark Recall}, the number of landmarks that participants recalled and actually exist along the route, including landmarks that were augmented by VisiMark and those that were not augmented but still remembered as landmarks by the participants; and (4) \textit{Recall of Augmented Landmarks}, the ratio of augmented landmarks recalled by the participant correctly to the total number of augmented landmarks on each route. For the trials where participants \textit{did not} wear VisiMark, we calculated this measure by assessing the recall of ``VisiMark landmarks'' (i.e., the specific landmarks that VisiMark would augment on the route). By comparing this measure between the with and without VisiMark conditions, we were able to evaluate the overlap between VisiMark landmarks and the landmarks naturally selected by PLV, thus deriving the impact of VisiMark on PLV's landmark selections.%\yuhang{i'm wondering whether we should do a between-subject comparison for each route (even not statistically), since the routes differ so much. Not sure it makes sense to compare between different routes via within subject method.}
% I calculate ANOVA for route on Recall of Augmented Landmarks and there is no significant effect

For all the measures, we had a within-subjects factor named \textit{Condition} with two levels: with VisiMark and without VisiMark. To validate the counterbalancing, we involved another within-subject factor \textit{Order}, and found no significant effect of \textit{Order} on any measures. We checked the normality of each measure using Shapiro-Wilk test. None of these measures were normally distributed, so we used the Aligned Rank Transform (ART) ANOVA to evaluate the effect of Condition on different measures. We used partial eta squared ($\eta ^2_p$) to calculate the effect size, with 0.01, 0.06, 0.14 representing the thresholds of small, medium and large effects \cite{cohen2013statistical,wang2024gazeprompt}.

\subsubsection{Qualitative Analysis}
We audio-recorded all studies and transcribed interviews using an automatic transcription service. Similar to the formative study, we analyzed the transcripts using thematic analysis. Two researchers open-coded five identical samples (31\% of the data) independently and developed an initial codebook by discussing their codes to resolve any disagreements. The two researchers then separated the remaining transcripts and coded independently based on the initial codebook. During this process, we periodically checked each other's work and discussed the codes to ensure consistency. After the researchers reached an agreement, new codes were added to the codebook.

We developed themes from the codes using a combination of inductive and deductive approaches \cite{braun2006using}. Our research aimed to evaluate our landmark-based navigation system, with a specific focus on what kinds of landmarks should be augmented for PLV and what are suitable augmentations for landmarks. Our high-level theme generation was thus guided by these objectives, following the deductive approach. Within each theme, we employed the inductive approach, generating sub-themes by clustering relevant codes using axial coding and affinity diagrams. Once initial themes and sub-themes were identified, researchers cross-referenced the original data, the codebook, and the themes to make final adjustments, ensuring all codes were correctly categorized. Our analysis resulted in 6 themes with 17 sub-themes, detailed in Appendix \ref{Codebook for Evaluation}.%\yuhang{would be good to add the specific theme and subtheme numbers for the final evaluation.}