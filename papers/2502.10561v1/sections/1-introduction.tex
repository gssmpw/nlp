\section{INTRODUCTION}
% motivation
% current work
% current gap
% what we have done and how to reduce the gap (contribution)

% turn-by-turn - the importance of exploration - landmarks help exploration - no current work explore how PLV use landmarks and differences with sighted people


% current landmark based system for blind people - Unlike blind people, people with low vision (PLV) use remaining functional vision extensively in their daily activities. - PLV can use visual cues, AR is a good option - no current work explore how to design AR landmarks aug for PLV 
% While the turn-by-turn instructions enable efficient wayfinding, they focus on providing directions in a passive way, and can even prevent the users from observing the surrounding environment [xx, any reference?]. 


% While turn-by-turn instructions enable people to get to the destination quickly. However, beyond getting to the destination, actively exploring the environment and constructing a mental model is another aspect of navigation. But little research has investigated this problem.

% While the turn-by-turn instructions enable efficient wayfinding, they focus on providing directions in a passive way, and can even prevent the users from observing the surrounding environment [xx, any reference?]. Beyond getting to a destination, proactively exploring the environment \cite{chrastil2015active} and developing mental models for revisits \cite{willis2009comparison} is another important task in navigation.
% Existing navigation assistive tools for PLV include assisting PLV with sign-reading using AR \cite{huang2019augmented}. Using augmented reality (AR) to facilitate navigation, which lays information on the visible environment in real time, can encourage users to engage more directly in the surroundings \cite{qiu2023navmarkar}, as well as help PLV perceive visual cues better given their limited visual capabilities compared to sighted people. 
% , enriching wayfinding instructions \cite{raubal2002enriching}
%partly because it is hard for PLV to detect and avoid obstacles \cite{taylor2016does,cloutier2022topical,legge2010visual}, as well as to read small or low contrast signs and maps during navigation \cite{huang2019augmented, szpiro2016finding}. 
%While sighted people enunciate landmarks within a narrow visual cone associated with their sightlines, blind people enunciated landmarks that fall in a much broader field that include the wind, sounds and smells \cite{tsuji12005landmarks}.
% Previous studies found that active exploration of a new environment leads to better wayfinding decisions \cite{chrastil2015active}, better spatial learning and mental map development \cite{chrastil2013active, giudice2006wayfinding, giudice2007wayfinding, long1997establishing}, and a sense of independence \cite{banovic2013uncovering}. 

Low vision refers to visual impairments that fall short of blindness but cannot be corrected with eyeglasses or contacts, affecting the independence and quality of life of 596 million people worldwide \cite{IAPBVisionAtlas}. Unlike blind people, people with low vision (PLV) rely on their functional vision in daily activities \cite{szpiro2016finding}. However, they experience visual challenges due to various low vision conditions, such as loss of central or peripheral vision, blurred vision, and reduced contrast sensitivity \cite{AOALowVision}. These conditions significantly impact daily tasks, such as reading \cite{wang2023understanding, wang2024gazeprompt}, navigating \cite{szpiro2016finding, zhao2020effectiveness}, housework \cite{jones2019analysis, wang2023practices}, and socializing \cite{senra2015psychologic, shah2020association}.
% \yuhang{need to point out that low vision people prefer using their functional vision somewhere in first paragrah... since it is the key that distinguish your work from blind research.}
% impacting independence and social well-being \cite{cimarolli2012challenges, shah2020association}. 

% Navigation is complex and challenging task for PLV \cite{corn2010foundations,szpiro2016finding}. It includes several sub-tasks, such as wayfinding (i.e., reaching the destination quickly) \cite{farr2012wayfinding,huang2019augmented}, obstacle avoidance \cite{kunz2018virtual,zhao2019designing}, and landmark identification for self-orientation and mental map construction of unfamiliar environments \cite{siegel1975development,kim2021acquisition}. Prior research has developed systems to support PLV in wayfinding by generating turn-by-turn instructions \cite{zhao2020effectiveness, stent2010iwalk} as well as in safe navigation by recognizing and highlighting obstacles \cite{fox2023using,angelopoulos2019enhanced}.
%Some navigation systems have been developed for PLV by generating turn-by-turn instructions to guide users to a destination \cite{zhao2020effectiveness, stent2010iwalk}. Beyond getting to the destination quickly, proactively exploring the environment \cite{chrastil2015active} and developing mental models for revisits \cite{willis2009comparison} are also important tasks in navigation. 
% However, little research has investigated landmark augmentations for PLV.% thus proactively exploring the environment and acquiring critical information for self-orientation and mental model development. 
%which includes landmark knowledge and survey knowledge, which is the mental map of the environment \cite{kim2021acquisition}, in addition to route knowledge \cite{siegel1975development}. 

Navigation is a complex and challenging task for PLV \cite{corn2010foundations,szpiro2016finding}, and landmarks play a critical role in this process \cite{siegel1975development}. Landmarks are commonly defined as stationary, distinct, and salient objects or places in the surrounding space that are more likely to be selected as spatial references \cite{yesiltepe2021landmarks, millonig2007developing}. They enable self-orientation \cite{raubal2002enriching}, convey spatial information \cite{tom2003referring,lovelace1999elements}, and support mental model development \cite{michon2001and, millonig2007developing}. Prior research has shown that PLV use landmarks frequently in navigation \cite{szpiro2016finding}. However, visually locating and recognizing landmarks can be challenging for PLV \cite{huang2019augmented, szpiro2016finding}. For example, people with central vision loss may have difficulty seeing relatively small or low-contrast landmarks, and people with peripheral vision loss may not be able to easily locate surrounding landmarks due to limited field of view. As a result, typical landmarks for sighted people may not be visible or useful for PLV. %Additionally, sighted people define landmarks within a narrow visual cone, whereas blind people define landmarks that fall in a much broader field that include the wind, sounds and smells \cite{tsuji12005landmarks}. While PLV's visual abilities fall between those of sighted and blind people, their landmark preferences have not been adequately explored in existing research. 

Prior research has identified and categorized landmarks for both sighted people \cite{lynch1964image,yesiltepe2021landmarks,sorrows1999nature} and people who are completely blind \cite{tsuji12005landmarks,jeamwatthanachai2019indoor,wang2023understanding}. However, the visual abilities of PLV fall between sighted and blind users and may thus have different preferences in landmark selection and enhancements, revealing a knowledge gap in PLV's landmark perception. %There is thus a gap in deeply understanding how PLV select and perceive landmarks in navigation to support
As such, to enable an effective landmark augmentation system, it is important to deeply understand \textit{what landmarks PLV use} in navigation, \textit{what landmarks they prefer to augment}, and \textit{how to augment these landmarks} for PLV.%to enable an effective system that support PLV in landmark perception. 


%making it important to explore how PLV perceive and leverage landmarks to navigate and build mental maps of the environment. 
% However, current work mostly focuses on exploring how sighted people \cite{millonig2007developing, zhu2022personalized, yesiltepe2021landmarks, hamburger2022landmark} and blind people \cite{wang2023understanding, tsuji12005landmarks} define, perceive and utilize landmarks. Current landmark-based navigation systems are also mostly designed for sighted and blind individuals, mainly using visual feedback for sighted people \cite{qiu2023navmarkar, zhu2022personalized}, and auditory \cite{balata2016automatically, balata2018landmark, may2020spotlights, coroama2003chatty, fiannaca2014headlock} and haptic feedback \cite{joseph2013semantic} for blind people. As PLV use their remaining functional vision extensively in their daily activities \cite{szpiro2016finding} and their visual ability fall between blind and sighted people who use distinct landmarks, whether they have different visual focus or use visual cues more than other sensory modalities is worth exploring. However, there is no existing work specifically investigating what landmarks PLV use, how PLV use landmarks and how to design landmark augmentations for PLV in navigation and mental map building. 
%Additionally, investigating the different visual cues they utilize, which may differ from those used by sighted individuals, is also important. To our knowledge, there is no existing work specifically investigating how PLV perceive and leverage landmarks in their navigation and mental model construction. Moreover, unlike blind individuals, previous research indicates that PLV benefit more from visual feedback in navigation tasks. They exhibited better length perception, made fewer errors, and experienced lower cognitive load compared to when using auditory feedback \cite{zhao2020effectiveness}. However, there is currently no comprehensive research on the visual design of landmark augmentations specifically tailored for PLV.
Augmented reality (AR) technology has the potential to enhance landmarks for PLV by immersively rendering multi-modal feedback over real-world environments. Prior work for sighted users has shown that AR-based landmark systems can improve mental map development and wayfinding performance, as well as reduce mental workload compared to 2D interfaces \cite{mckendrick2016into, zhang2021enhancing}. We seek to seize this opportunity and explore how to best leverage this technology and assist PLV with landmark identification via AR augmentations. To achieve this goal, our research focuses on indoor navigation (due to the visibility limitation of AR feedback outdoors) and addresses two research questions:

\begin{itemize}
    \item  RQ 1:  How do PLV select, perceive, and utilize landmarks in navigation and mental model development?
    \item RQ 2: What are effective AR augmentations to support PLV in landmark identification?
\end{itemize}





To answer these questions, we first conduct a formative study with six PLV using contextual inquiry to understand how they perceive and use landmarks in indoor environments as well as how they prefer to enhance landmarks. %Afterwards we asked multiple questions pertaining to their mental model surrounding routes and how they chose and used landmarks, as well as exploratory and brainstorming questions about landmark-based augmentation design that they would like to have. 
The study reveals unique landmark selections for PLV. While the general landmark selection aligned with the three landmark categories for sighted people---visual, cognitive, and structural landmarks \cite{sorrows1999nature}---we identified unique subcategories for PLV, such as \textit{space with certain lighting conditions} and \textit{silhouette of an area} as visual landmarks to indicate PLV's overall impression of the space, and objects with accessibility implications (i.e., railings and elevators) as cognitive landmarks for mobility and safety purposes. Moreover, we found that PLV prefer not only augmentations on landmarks nearby but also early previews for space structure and upcoming landmarks at a distance.

Based on the formative study, we design \textit{VisiMark}, an AR interface that supports landmark perception for PLV. VisiMark includes two features: \textit{Signboards} at hallway intersections to present an overview of the hallway structure and upcoming landmarks, along with color-coded hallways (Figure \ref{fig:teaser}A), and \textit{In-situ Labels} with icons and texts to augment individual landmarks directly at their physical locations (Figure \ref{fig:teaser}B). We evaluated VisiMark with 16 low vision participants, who conducted navigation and retracing tasks indoors with and without VisiMark. %(two routes with VisiMark and two routes without). 
Participants also drew a map of the route after each navigation to demonstrate their mental model of the environment.
% A Signboard appears at intersections of hallways and presents an overview of the hallway structure and upcoming landmarks (Figure \ref{fig:teaser}A). %The hallway and landmark representations on the Signboard are in proportion to the size of the hallways and the relative positions of the landmarks. 
% It consists of arrows that represent different hallways with the arrow length and width in proportion to the hallway space. The actual hallways are also color-coded via AR to match the corresponding arrow color on the Signboard. We also label the upcoming landmarks along the hallways on the Signboard, representing the relative positions of the landmarks in the space. %icons and texts along the arrows to represent the upcoming landmarks on the hallways. The placement of icons on the signboard corresponds to the relative positions of the landmarks on the hallways. 
% Beyond the overview, we also render In-situ Labels, including icons and text descriptions, at the physical positions of the landmarks to augment the landmarks directly (Figure \ref{fig:teaser}B). Our labels are designed in vibrant colors and large sizes to ensure visibility for low vision users. The landmark icons are designed based on the landmark categories for PLV derived from the formative study. 

 %Following each navigation task, we asked them to draw a map of the path they walked and then retrace it. We gathered their feedback quantitatively and qualitatively through the navigation task and the subsequent exit questions.

Our study demonstrated the effectiveness of VisiMark. We found that VisiMark enabled PLV to perceive landmarks that they preferred but could not easily perceive before (e.g., an elevator located on the side of the user's blind eye, recessed landmarks like bathrooms). As a result, VisiMark changed participants' landmark selections from visually-obvious objects to cognitive landmarks that were more memorable and meaningful. We thus derived a new taxonomy of \textit{prefer-to-augment landmarks} in AR contexts, including unique but not visually obvious landmarks, visually challenging but cognitively important landmarks, and out-of-field landmarks that indicate potential dangers.%We also identified design implications for landmark augmentations in AR for PLV.
% We also found that PLV prefer different types of landmarks to be augmented in different stages. Visually obvious landmarks should only be augmented in preview, while common yet important facilities and object affordance are more preferred to be augmented only in situ.
% \yuhang{may need to add more interesting findings... your current findings are vague (especially those i commented off below), hard to be connected to specific measures, and also not interesting. This may be because that you don't have any significant results, but it does not mean that you want to communicate things vaguely and also does not mean that you don't have interesting results. For example, what are the landmarks PLV prefer to augment vs. they can perceive by themselves.}

% improved PLV’s landmark perception, enabling them to better recognize and memorize landmarks, and helped low vision users select important landmarks more efficiently.
 % participants completed the map drawing and retracing tasks more effectively and comfortably with VisiMark. We also found that 
%improved mental map building. 
%Low vision participants would like to augment all landmarks that meet their criteria, and visually challenging but important landmarks, especially dangers such as obstacles out of their central view. We also found their preferences for what should be augmented in preview and in situ only, along with suggestions for refining system design and offering more customization options.

In summary, we contribute the first exploration of the landmark perception of PLV as well as the design and evaluation of VisiMark, an AR system that enhances landmarks for PLV in indoor navigation. Our study reveals PLV’s needs for visual landmark augmentations and sheds light on future landmark-based navigation systems in AR for PLV.