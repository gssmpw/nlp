%TC:ignore
\section{Theme Table}
\label{Codebook for Evaluation}
\begin{table*}[ht]
\scriptsize
\centering
\begin{tabular}
% {>{\centering\arraybackslash}p{3cm}>{\centering\arraybackslash}p{3cm}>{\centering\arraybackslash}p{3cm}}
{p{3.5cm} p{4.8cm} p{8.5cm}}
\toprule
  \textbf{Themes} & \textbf{Sub-themes} &  \textbf{Codes}\\
% \Xhline{2\arrayrulewidth}
\hline
Effect of Landmark Augmentations on Route Retracing & \multirow{2}{*}{Effective retracing} & \multirow{1}{*}{navigate easier; memorize the turns}\\
\hline
\multirow{6}{*}{Effect of Landmark Augmentations on} & Perceive hallway structures & notice hallway structures more than before; memorize hallway lengths\\
\cline{2-3}
\multirow{6}{*}{Mental Map Development} & \multirow{2}{*}{Increased Focus on Landmarks} & notice landmarks; identify landmarks; memorize landmarks; locate landmarks; increased reliance on landmarks in navigation\\
\cline{2-3}
& \multirow{4}{*}{Shift in Landmark Selection} & help notice things may be overlooked; monocular blindness: things in blind spot; pick out important landmarks better; help find functional facilities; offer more information to memorize; start paying attention to once augmented landmarks even without wearing the system; conflict with their own way of identifying landmarks\\
\hline
\multirow{6}{*}{Experiences with VisiMark} & Effectiveness&  effective in retracing; effective in mental map building\\
\cline{2-3}
& \multirow{2}{*}{Comfort}& mentally comfortable; more comfortable without the system; device uncomfort; uncomfortable in public\\
\cline{2-3}
& Learnability& easy to use and understand; short learning curve; tutorial\\
\cline{2-3}
& \multirow{2}{*}{Distraction}& more useful than distracting; eliminate visual noise from a space; very distracting because of learning curve; not overwhelming; overwhelming at some spots; limit the number of augmented landmarks\\
\hline
\multirow{4}{*}{Taxonomy of Landmarks to Augment} & Current landmarks in VisiMark & similar to those used in wayfinding and mental maps\\
\cline{2-3}
& Unique but not visually obvious landmarks& green double doors\\
\cline{2-3}
& Visually challenging but cognitively important landmarks & recessed or flat landmarks; elevators; restrooms\\
\cline{2-3}
& Landmarks outside their central view, especially dangers & landmarks above eye level; obstacles on the floor\\
\hline
\multirow{2}{*}{When the Augmentations Should Occur} & What to augment only only in preview & visually salient landmarks\\
\cline{2-3}
& What to augment only in situ & common yet important facilities; affordance; small or low contrast prints\\
\hline
\multirow{14}{*}{Desired Augmentation Designs} & \multirow{7}{*}{Signboards} &  have an overview ahead; locate oneself without extra trips; depth perception issues: providing hallway lengths; depth perception issues: help identify dead end; double vision: prefer signboards in central view; monocular blindness: point out possible directions; have scales; small arrows of further connecting hallways; maps of the general layout; colors are helpful cues to remember; confirm on the right track; easier navigation unconsciously; not turn-by-turn color-coded hallways distracting; distinct current colors; primary colors; allow brightness adjustability; allow transparency adjustability; allow more color choices; add dark colored outlines\\
\cline{2-3}
& \multirow{3}{*}{In-situ labels}& focus points to tie on; confirm on the right track from a distance; icons are simpler but convey same information; icons help people who cannot read; number of icon categories; unique icon categories; texts are more indicative; should not use abbreviation; more details in descriptions\\
\cline{2-3}
& \multirow{2}{*}{Further customization options} & specialize based on the building environment; add ability to turn on and off some components; customize personal layers; add ability to zoom in; add ability to adjust position of augmentations\\
\bottomrule
\end{tabular}
\caption{Themes and Codebook.}
\label{tab:Themes and Codebook}
\end{table*}



\section{Interview Questions for the Formative Study}
\label{Interview Questions for the Formative Study}
\subsection{Initial Interview Questions}
\begin{enumerate}
    \item What is your name?
    \item What is your age?
    \item What gender do you identify with?
    \item What is your visual condition?
    \item Are you considered Legally blind?
\begin{enumerate}
    \item What is your diagnosis? 
    \item What is your visual acuity?
    \item What is your field of view?
    \item What is your contrast sensitivity?
    \item What is your color vision?
    \item What is your light sensitivity? 
    \item What is your eepth perception? 
\end{enumerate}
    \item How long have you had this visual condition?
\begin{enumerate}
    \item Is this condition progressive or stable?
\end{enumerate}
    \item How do you usually complete a navigation task?
    \item Do you use any technology to navigate regularly outdoors and indoors?
\begin{enumerate}
    \item If yes, what technology do you use?
\end{enumerate}
    \item Do you pay attention to any landmarks during navigation? What landmarks?
    \item Do you use any technology to help you perceive the landmarks?
    \item Will your choice of landmarks change due to familiarity?
    \item Do you have any prior experience with Augmented Reality (Google glasses, phone application, HoloLens)? 
\begin{enumerate}
    \item If yes, could you please share the experience?
\end{enumerate}
    \item Are you currently familiar with the campus building?
\end{enumerate}

\subsection{Exit Interview Questions}
\begin{enumerate}
\item How do you determine landmarks? Do you prioritize certain landmark features (e.g., color, size, shape)?
\begin{enumerate}
\item Why do you pay attention to a specific landmark during the navigation task?
\end{enumerate}
\item What type of landmark modalities do you prefer?
\item How do you use your landmarks during navigation (including wayfinding and mental map building)?
\item Do you look for landmarks the first time you visit a place, or only after you’ve been there a few times?
\item Will your choice of landmarks change due to familiarity?
\item How important do you consider landmarks for indoor wayfinding? Could you please offer a score between 1 and 5, with 1 stands for least important and 5 means most important?
\item How important do you consider landmarks for developing a mental map indoors? Could you please offer a score between 1 and 5, with 1 stands for least important and 5 means most important?
\item What types of landmarks would you like to see augmented? Are there any specific landmarks you would prefer to use but find challenging due to visual limitations?
\item What kind of landmark augmentations you would like to have? (e.g., enlarging, outlining, etc.)
\item What modalities of landmark augmentations you would like to have? (e.g., visual, audio, haptic, etc.)
\end{enumerate}


\section{Interview Questions for the Final Evaluation}
\label{Interview Questions for the Evaluation}
\subsection{Initial Interview Questions}
\begin{enumerate}
    \item What is your name?
    \item What is your age?
    \item What gender do you identify with?
    \item What is your visual condition?
    \item Are you considered Legally blind?
\begin{enumerate}
    \item What is your diagnosis? 
    \item What is your visual acuity?
    \item What is your field of view?
    \item What is your contrast sensitivity?
    \item What is your color vision?
    \item What is your light sensitivity? 
    \item What is your eepth perception? 
\end{enumerate}
    \item How long have you had this visual condition?
\begin{enumerate}
    \item Is this condition progressive or stable?
\end{enumerate}
    \item Do you use any technology to navigate regularly outdoors and indoors?
\begin{enumerate}
    \item If yes, what technology do you use?
\end{enumerate}
    \item Do you pay attention to any landmarks during navigation? What landmarks?
    \item Do you use any technology to help you perceive the landmarks?
    \item Do you have any prior experience with Augmented Reality (Google glasses, phone application, HoloLens)? 
\begin{enumerate}
    \item If yes, could you please share the experience?
\end{enumerate}
    \item Are you currently familiar with navigating inside this building?
\end{enumerate}


\subsection{Exit Interview Questions}
\begin{enumerate}
 \item Let’s first talk about your landmark choices in the four trials. (Based on the mental map) Why do you pick this specific landmark?

 \item Our systems select and augment certain types of landmarks for you. Do you think they are useful or not? Why? What landmarks do you prefer to be augmented in indoor navigation? 

\item For [each element], how do you like it? How does this design affect your understanding of the route? How do you want to improve it?
\begin{enumerate}
\item The presentation of the structure of the hallways (e.g., direction, width, length, color-coding, the presentation of deadends)
\item The icons and texts of the landmarks on the signboard.
\item In-situ elements.
\end{enumerate}


\item Effectiveness: How effective do you think of the system? Could you please offer a score between 1 and 7, with 1 stands for least effective and 7 means most effective?
\item Comfortable: How comfortable do you think of the system? Could you please offer a score between 1 and 7, with 1 stands for least comfortable and 7 means most comfortable?

\item Distraction/Load: How distracting do you think of the system? Could you please offer a score between 1 and 7, with 1 stands for least distracting and 7 means most distracting ?
\item Learnability: How easy to understand or learn do you think of the system? Could you please offer a score between 1 and 7, with 1 stands for least easy to use and 7 means most easy to use?

\item Any ideas for other designs of augmentations to support your indoor navigation and mental model development?


\end{enumerate}

%TC:endignore