 %%
%% This is file `sample-sigconf-authordraft.tex',
%% generated with the docstrip utility.
%%
%% The original source files were:
%%
%% samples.dtx  (with options: `all,proceedings,bibtex,authordraft')
%% 
%% IMPORTANT NOTICE:
%% 
%% For the copyright see the source file.
%% 
%% Any modified versions of this file must be renamed
%% with new filenames distinct from sample-sigconf-authordraft.tex.
%% 
%% For distribution of the original source see the terms
%% for copying and modification in the file samples.dtx.
%% 
%% This generated file may be distributed as long as the
%% original source files, as listed above, are part of the
%% same distribution. (The sources need not necessarily be
%% in the same archive or directory.)
%%
%%
%% Commands for TeXCount
%TC:macro \cite [option:text,text]
%TC:macro \citep [option:text,text]
%TC:macro \citet [option:text,text]
%TC:envir table 0 1
%TC:envir table* 0 1
%TC:envir tabular [ignore] word
%TC:envir displaymath 0 word
%TC:envir math 0 word
%TC:envir comment 0 0
%%
%%
%% The first command in your LaTeX source must be the \documentclass
%% command.
%%
%% For submission and review of your manuscript please change the
%% command to \documentclass[manuscript, screen, review]{acmart}.
%%
%% When submitting camera ready or to TAPS, please change the command
%% to \documentclass[sigconf]{acmart} or whichever template is required
%% for your publication.
%%
%%
% \documentclass[sigconf,authordraft]{acmart}
\documentclass[sigconf]{acmart}
% \documentclass[manuscript,review,anonymous]{acmart}
\usepackage{graphicx}
\usepackage{subcaption}
\usepackage{float}
\usepackage{multirow}
\usepackage{array}
\usepackage{makecell}
% \usepackage{accessibility}
\usepackage{xcolor}
\definecolor{brownishred}{RGB}{0,0,0}
% \definecolor{brownishred}{RGB}{178, 34, 34}
\newcommand{\yuhang}[1]{{\small\textcolor{red}{\bf [YZ: #1]}}}
\newcommand{\colorchange}[1]{{\textcolor{brownishred}{#1}}}

%%
%% \BibTeX command to typeset BibTeX logo in the docs
\AtBeginDocument{%
  \providecommand\BibTeX{{%
    Bib\TeX}}}

%% Rights management information.  This information is sent to you
%% when you complete the rights form.  These commands have SAMPLE
%% values in them; it is your responsibility as an author to replace
%% the commands and values with those provided to you when you
%% complete the rights form.

\setcopyright{acmlicensed}
\copyrightyear{2025}
\acmYear{2025}
\setcopyright{cc}
\setcctype{by-nd}
\acmConference[CHI '25]{CHI Conference on Human Factors in Computing Systems}{April 26-May 1, 2025}{Yokohama, Japan}
\acmBooktitle{CHI Conference on Human Factors in Computing Systems (CHI '25), April 26-May 1, 2025, Yokohama, Japan}\acmDOI{10.1145/3706598.3713847}
\acmISBN{979-8-4007-1394-1/25/04}

% \setcopyright{acmlicensed}
% \copyrightyear{2024}
% \acmYear{2024}
% \acmDOI{XXXXXXX.XXXXXXX}

% %% These commands are for a PROCEEDINGS abstract or paper.
% \acmConference[CHI '25]{Conference on Human Factors in Computing Systems}{April 26--May 01, 2025}{Yokohama, Japan}
% % \acmConference[Conference acronym 'XX]{Make sure to enter the correct conference title from your rights confirmation emai}{June 03--05, 2018}{Woodstock, NY}
% %%
% %%  Uncomment \acmBooktitle if the title of the proceedings is different
% %%  from ``Proceedings of ...''!
% %%
% %%\acmBooktitle{Woodstock '18: ACM Symposium on Neural Gaze Detection,
% %%  June 03--05, 2018, Woodstock, NY}
% \acmISBN{978-1-4503-XXXX-X/18/06}


%%
%% Submission ID.
%% Use this when submitting an article to a sponsored event. You'll
%% receive a unique submission ID from the organizers
%% of the event, and this ID should be used as the parameter to this command.
%%\acmSubmissionID{123-A56-BU3}

%%
%% For managing citations, it is recommended to use bibliography
%% files in BibTeX format.
%%
%% You can then either use BibTeX with the ACM-Reference-Format style,
%% or BibLaTeX with the acmnumeric or acmauthoryear sytles, that include
%% support for advanced citation of software artefact from the
%% biblatex-software package, also separately available on CTAN.
%%
%% Look at the sample-*-biblatex.tex files for templates showcasing
%% the biblatex styles.
%%

%%
%% The majority of ACM publications use numbered citations and
%% references.  The command \citestyle{authoryear} switches to the
%% "author year" style.
%%
%% If you are preparing content for an event
%% sponsored by ACM SIGGRAPH, you must use the "author year" style of
%% citations and references.
%% Uncommenting
%% the next command will enable that style.
%%\citestyle{acmauthoryear}


%%
%% end of the preamble, start of the body of the document source.
\begin{document}

%%
%% The "title" command has an optional parameter,
%% allowing the author to define a "short title" to be used in page headers.
\title[VisiMark]{VisiMark: Characterizing and Augmenting Landmarks for People with Low Vision in Augmented Reality to Support Indoor Navigation}

%%
%% The "author" command and its associated commands are used to define
%% the authors and their affiliations.
%% Of note is the shared affiliation of the first two authors, and the
%% "authornote" and "authornotemark" commands
%% used to denote shared contribution to the research.
\author{Ruijia Chen}
\affiliation{%
  \institution{University of Wisconsin-Madison}
  \city{Madison}
  \state{Wisconsin}
  \country{USA}
}
\email{ruijia.chen@wisc.edu}
\orcid{0000-0002-1655-6228}

\author{Junru Jiang}
\affiliation{%
  \institution{University of Wisconsin-Madison}
  \city{Madison}
  \state{Wisconsin}
  \country{USA}
  }
\email{jjiang324@wisc.edu}

\author{Pragati Maheshwary}
\affiliation{%
  \institution{Carnegie Mellon University}
  \city{Pittsburgh}
  \state{Pennsylvania}
  \country{USA}
}
\email{pragati2@andrew.cmu.edu}

\author{Brianna R. Cochran}
\affiliation{%
  \institution{University of Wisconsin-Madison}
  \city{Madison}
  \state{Wisconsin}
  \country{USA}
  }
\email{bcochran2@wisc.edu}

\author{Yuhang Zhao}
\affiliation{%
  \institution{University of Wisconsin-Madison}
  \city{Madison}
  \state{Wisconsin}
  \country{USA}
  }
\email{yuhang.zhao@cs.wisc.edu}


%%
%% By default, the full list of authors will be used in the page
%% headers. Often, this list is too long, and will overlap
%% other information printed in the page headers. This command allows
%% the author to define a more concise list
%% of authors' names for this purpose.
\renewcommand{\shortauthors}{Chen et al.}

%%
%% The abstract is a short summary of the work to be presented in the
%% article.
\begin{abstract}
Landmarks are critical in navigation, supporting self-orientation and mental model development. Similar to sighted people, people with low vision (PLV) frequently look for landmarks via visual cues but face difficulties identifying some important landmarks due to vision loss. We first conducted a formative study with six PLV to characterize their challenges and strategies in landmark selection, identifying their unique landmark categories (e.g., area silhouettes, accessibility-related objects) and preferred landmark augmentations. We then designed \textit{VisiMark}, an AR interface that supports landmark perception for PLV by providing both overviews of space structures and in-situ landmark augmentations. We evaluated VisiMark with 16 PLV and found that VisiMark enabled PLV to perceive landmarks they preferred but could not easily perceive before, and changed PLV's landmark selection from only visually-salient objects to cognitive landmarks that are more important and meaningful. We further derive design considerations for AR-based landmark augmentation systems for PLV.
\end{abstract}

%%
%% The code below is generated by the tool at http://dl.acm.org/ccs.cfm.
%% Please copy and paste the code instead of the example below.
%%
\begin{CCSXML}
<ccs2012>
   <concept>
       <concept_id>10003120.10011738.10011775</concept_id>
       <concept_desc>Human-centered computing~Accessibility technologies</concept_desc>
       <concept_significance>500</concept_significance>
       </concept>
   <concept>
       <concept_id>10003120.10011738.10011776</concept_id>
       <concept_desc>Human-centered computing~Accessibility systems and tools</concept_desc>
       <concept_significance>500</concept_significance>
       </concept>
   <concept>
       <concept_id>10003120.10011738.10011774</concept_id>
       <concept_desc>Human-centered computing~Accessibility design and evaluation methods</concept_desc>
       <concept_significance>500</concept_significance>
       </concept>
   <concept>
       <concept_id>10003120.10003121.10003124.10010392</concept_id>
       <concept_desc>Human-centered computing~Mixed / augmented reality</concept_desc>
       <concept_significance>500</concept_significance>
       </concept>
 </ccs2012>
\end{CCSXML}
\ccsdesc[500]{Human-centered computing~Accessibility technologies}
\ccsdesc[500]{Human-centered computing~Accessibility systems and tools}
\ccsdesc[500]{Human-centered computing~Accessibility design and evaluation methods}
\ccsdesc[500]{Human-centered computing~Mixed / augmented reality}

%%
%% Keywords. The author(s) should pick words that accurately describe
%% the work being presented. Separate the keywords with commas.
\keywords{Accessibility, Virtual/Augmented Reality, Individuals with Disabilities \& Assistive Technologies}
%% A "teaser" image appears between the author and affiliation
%% information and the body of the document, and typically spans the
%% page.
% \begin{teaserfigure}
%   \includegraphics[width=\textwidth]{sampleteaser}
%   \caption{Seattle Mariners at Spring Training, 2010.}
%   \Description{Enjoying the baseball game from the third-base
%   seats. Ichiro Suzuki preparing to bat.}
%   \label{fig:teaser}
% \end{teaserfigure}
%TC:ignore
\begin{teaserfigure}
  \includegraphics[width=\textwidth]{graphs/teaser.png}
    % \vspace{-5ex}
  \caption{\textit{VisiMark} provides landmark augmentations on head-mounted AR to support wayfinding and mental map construction. VisiMark includes two features: (A) \textit{Signboard}, an overview of hallway structures and upcoming landmarks at intersections, and (B) \textit{In-situ Labels}, world-anchored icons and texts to highlight the types and positions of landmarks in the physical environment.}
  \Description{This image shows VisiMark system. VisiMark provides landmark augmentations on head-mounted AR to support wayfinding and mental map construction. VisiMark includes two features: (A) Signboard, an overview of hallway structures and upcoming landmarks at intersections, and (B) In-situ Labels, world-anchored icons and texts to highlight the types and positions of landmarks in the physical environment.}
  \label{fig:teaser}
\end{teaserfigure}
%TC:endignore

% \received{20 February 2007}
% \received[revised]{12 March 2009}
% \received[accepted]{5 June 2009}

%%
%% This command processes the author and affiliation and title
%% information and builds the first part of the formatted document.
\maketitle

\section{Introduction}

% State of the world (robots for creative activites)
The term ``robot,'' originally signifying `forced labor,' has long been associated with labor and work. Robots have demonstrated their utility in various automated productive and social contexts, where the primary goals are improving productivity, safety, and fostering social interactions with humans~\cite{simoes2022designing, weidemann2021role, honig2018understanding}. However, an increasing number of cases feature using of robots in creative settings. Unlike productive contexts, where the focus is on efficiency and task completion~\cite{arents2022smart}, or social contexts, where communication and trust are prioritized~\cite{nam2020trust, saunderson2019robots}, creative environments prioritize artistic innovation and expression~\cite{hsueh2024counts}. This shift fundamentally alters the dynamics of human-robot interaction, redefining the roles and expectations for both humans and robots.

For instance, robots’ social behaviors are leveraged to support the generation and expression of creative ideas~\cite{hu2021exploring, sandoval2022human, alves2020creativity}, and programmable robotic movements and trajectories are employed to inspire artistic activities such as sketching~\cite{lin2020your}. These studies often engage participants from creative fields who possess limited prior experience with robotics, and are typically conducted in short-term, experimental settings. Consequently, the findings from these studies remain constrained since much can be learned from professional practitioners' experiences to inform system design such as digital fabrication~\cite{hirsch2023nothing}. There is a notable gap in research examining the long-term, active, and practical experience of integrating robotic systems into the creative processes. As a result, the deeper insights into how robots facilitate and shape creative processes, beyond simply augmenting human creativity, remain underexplored. In this study, we aim to better understand the impacts of robots on creative processes and outcomes.

As early as Leonardo da Vinci's 16th century ``Automaton,'' artists have explored the creative affordances of robotic systems~\cite{shanken2002cybernetics, pagliarini2009development, jeon2017robotic}. The artistic creation process typically encompasses various stages, including the exploration of materials and techniques, ongoing experimentation and iteration, and the continual refinement of the artists' insights into their creative subjects~\cite{lewis2023art, sturdee2022state}. Therefore, investigating the artistic process involving robots offers an opportunity to gain deeper insights into robots' creative potential. Robotic art, in particular, provides a compelling case for this exploration.

We define robotic art as artworks that utilize robotic or automated machines to create artistic experiences and tangible artifacts. One example is robotic installation art, in which robots are programmed to follow specific rules that embody the artist’s expression (\autoref{fig:teaser} (a)). Another example is responsive art, in which robots react to their environment, with behaviors that change over time or in response to spectators (\autoref{fig:teaser} (b)). Additionally, there are robotic creators, which possess a degree of agency, allowing them to collaborate with human artists and produce works that extend beyond mere replication of human-created art (\autoref{fig:teaser} (c) and (d)). As such, robotic art becomes a rich case for exploring human-machine interactions in creative contexts. Gaining a deeper understanding of how robots facilitate artistic expression can provide insights for designing computing systems to support creative activities~\cite{gomez2021robot}.

% Therefore, we did...
We draw on semi-structured, in-depth interviews with renowned professional robotic artists to investigate the use of robots in artistic practice. Specifically, our goal is to understand how artistic exploration of robotic systems challenges conventional assumptions about the functions of robots, such as their roles in automating repetitive tasks or serving human needs. We also explore the implications of robots in the artistic process and examine how creativity may emerge within robotic art. To address these interrelated inquiries, our study focuses on the practice of robotic art, posing the research question: \textit{How do robotic artists utilize robots in their artistic practice?} We approach this inquiry through the perspectives and experiences of robotic artists, who creatively design, modify, and repurpose robotic systems for artistic expression and exploration.

% The key findings are...
Our findings highlight the social, material, and temporal dimensions of artists' practices that shape their creativity and artistic outcomes. The creation of robotic art is largely a social process, as artists receive both explicit and implicit feedback through the audience's reactions and reception of their work. Simultaneously, the embodiment and malfunctions inherent to robotic systems drive artistic experimentation. The temporal processes of creation and exhibition, beyond just the final product, further enhance the creative value. Our empirical analysis presents how creativity emerges through the interplay of social, material, and temporal interactions among artists, robots, audiences, and the environment.

% The contributions of this work are...
We make two main contributions to HCI in this study. 
First, we elucidate the interactive mechanisms among key actors---human creators, machines, audiences, and environments---within the practice of robotic art, a topic that remains underexplored in HCI. Our findings reveal the significance of sociality (e.g., interactions between artists and audiences), materiality (e.g., the embodiment and malfunctions of robots), and temporality (e.g., the processes of creation and exhibition) in shaping creative values. We propose that these three facets are central to the creative process and facilitate the emergence of creativity in robotic art.
Second, drawing from the findings, we offer implications for \textit{socially informed}, \textit{material-attentive}, and \textit{process-oriented} creation with computing systems. We suggest leveraging these three aspects to enhance creativity and the creative experience. Specifically, we discuss the value of incorporating implicit audience feedback, designing with technical malfunctions, and focusing on the post-creation process to foster alternative creative experiences with machines~\cite{alter2010designing, juarez2022glitch}.



\paragraph{Uncertainty-based hallucination detection methods.}
Various approaches have been proposed to detect hallucinated content in LLMs generation.
Unlike other methods that require external knowledge sources for fact-checking~\citep{gou2024critic, chen-etal-2024-complex, min-etal-2023-factscore, huo2023retrieving}, uncertainty-based approaches are reference-free and rely only on LLM internal states or behaviors to determine hallucination~\citep{10.1145/3703155}. 
For instance, sampling-based approaches generate multiple responses and measure the diversity in meaning among them~\citep{fomicheva-etal-2020-unsupervised, kuhn2023semantic, lin2024generating}, while density-based approaches approximate the training data distribution and provide probabilities or unnormalized scores to assess how likely a generated response belongs to the distribution~\citep{yoo-etal-2022-detection, ren2023outofdistribution, vazhentsev-etal-2023-hybrid}.

In this paper, we focus on uncertainty quantification methods that rely on token-level likelihood or entropy~\citep{guerreiro-etal-2023-looking, malinin2021uncertainty}. 
Recent works have explored refining likelihood estimation by incorporating semantic relationships or reweighting token importance. For instance, Claim-Conditioned Probability (CCP)~\citep{fadeeva-etal-2024-fact} was introduced to recalculate likelihood according to semantical equivalence; while \citet{zhang-etal-2023-enhancing-uncertainty} and \citet{duan-etal-2024-shifting} adjust token weights to better convey meaning in uncertainty aggregation. \emph{Although these approaches leverage token-level information, they are typically evaluated at the sentence level, raising questions about their reliability}. To address this, we conduct a comprehensive analysis of entity-level hallucination detection for finer-grained performance insights.


\paragraph{Fine-grained hallucination detection benchmark.}

Most hallucination detection benchmarks are in sentence or paragraph level. For example, CoQA~\citep{reddy-etal-2019-coqa}, TriviaQA~\citep{joshi-etal-2017-triviaqa}, TruthfulQA~\citep{lin-etal-2022-truthfulqa}, and HaluEval~\citep{li-etal-2023-halueval}. These benchmarks classify each generated response as either hallucinated or correct. However, instance-level detection cannot pinpoint specific hallucinated content, which is crucial for correcting misinformation~\citep{cattan2024localizingfactualinconsistenciesattributable}. This limitation becomes particularly problematic in long-form text, where a single response often combines supported and unsupported information, making binary quality judgments inadequate~\citep{min-etal-2023-factscore}.

To address these challenges, recent works have advanced benchmarks for more granular hallucination detection. For example, \citet{min-etal-2023-factscore} introduced \textsc{FActScore}, which decomposes LLM-generated text into atomic facts---short sentences conveying a single piece of information---for more precise evaluation. In parallel, \citet{cattan2024localizingfactualinconsistenciesattributable} introduced \textsc{QASemConsistency}, decomposing LLM generated text with QA-SRL, a semantic formalism, to form simple QA pairs, where each QA pair represent one verifiable fact. \emph{However, these methods do not enable entity-level hallucination detection, as they lack explicit entity-level labeling (hallucinated or not) in the original generated text}.  
Beyond decomposition-based approaches, datasets like \textsc{HaDes}~\citep{liu-etal-2022-token} and CLIFF~\citep{cao-wang-2021-cliff} create token-level hallucinated content by perturbing human-written text, allowing token-level annotation on the same text. These perturbed hallucinated content, however, could be unrealistic, biased, and overly synthetic due to the limitations of models they used to perturb words. 
To bridge this gap, we create a new dataset with entity-level hallucination labels on the same LLMs generated text. This allows us to evaluate uncertainty-based hallucination detection approaches on a finer-grained level and analyze their reliability.





% 尽管HCI 研究开始关注mental health的homework的支持【】,但是艺术治疗里的homework对于HCI研究仍然是一个尚待理解和探索的新场景
%也尚未有HCI design cases探索如何设计能够较好支持艺术治疗homework的包含AI agents 的系统。
%因此,为了给我们接下来的设计探索收集inputs,我们组织了formative study。我们的主要目的有二:
    % 理解艺术治疗家庭作业的场景
    % 理解设计支持艺术治疗家庭作业的包含AI agent 的系统应该满足哪些需求,符合哪些quality(这里得到的结论应该是下一个阶段设计部分比较重要的交互或者界面或者功能特性,以及比较关键的设计rationales)

\section{Contextual Understanding}
Recent HCI research has pinpointed the significance of understanding therapy homework in mental health~\cite{Oewel_2024}, yet art therapy homework remains a unique and unaddressed domain. 
Therefore, we conducted a contextual study with a group of therapists to gain a concrete understanding of current art therapy homework practice and to identify common needs for technological support.
\begin{figure*}[tb]
  \centering
  \includegraphics[width=\linewidth]{images/1.jpg}
  \vspace{-4mm}
  \caption{Art therapy homework outcomes from the therapists' previous practice: (a): T4; (b)-(c): T5; (d)-(g): T3}
  \Description{This Figure showcases the outcomes of homework practices among art therapists. From left to right: (a) a client completing a homework task on a structured worksheet; (b) a depiction of a volcano represented by yellow patterns, with orange indicating imminent erupting lava; (c) an outline of a small figure containing a floral pattern in black and red; (d) a composition using text alongside red and blue floral designs; (e) a diary entry documented by a client; (f) a handcrafted green mountain created by a client; (g) a client-made black clay figurine placed on a patch of grass.}
  \label{fig:context1}
\end{figure*}

%second, to understand the needs and qualities of human-AI systems in supporting art therapy homework.
% formative study procedure
    % 找了谁
    % 怎么做的
        % 我们组织了一对一的疗愈师访谈,来理解艺术治疗家庭作业的当前practice,包扩(当前的practice,艺术治疗家庭作业是什么样一个形式,怎么布置的,做些什么,有哪些疗愈意义,治疗师想要通过家庭作业达到什么目标)
        % 其次,我们基于艺术治疗理论和相关的前期工作,以及访谈中新获得的理解,以准备了一个初步的demo和mockup来作为formative study的准备材料
            % 相关的理论和研究表明艺术治疗的家庭作业一般需要结合艺术创作与verbal反思两个元素,然而目前并未有将两者结合在一起的系统,为了和疗愈师共创式设计,我们构建了一个简单的家庭作业系统demo,它包含一个让用户通过绘制语义分割来生成图像的画板(类似的画板已被应用于艺术治疗practice,见DeepThink),以及一个可以理解用户在画板上绘制行动并提出问题鼓励用户近一步表达创作过程的AI agent。我们尽量保持系统的simplistic和open-ended以方便疗愈师参与到接下来的协同设计并能最大限度输出他们的经验。
            % 与此同时,我们构想了一个初步的疗愈师界面,目的是辅助疗愈师monitor和review来访者的家庭作业。我们制作了静态的mock up以便疗愈师在此基础上进一步协同创作发展设计。
        %我们用这个初步的demo和mockup组织疗愈师进行了两次的协同设计工作坊,流程:
            % 介绍了协同设计的目标(设计支持疗愈师和来访者的艺术治疗家庭作业的AI工具)
            % 我们展示了demo和mockup
            % 让治疗师进行了交互体验和自由讨论,疗愈师们在本地设备使用了我们的demo,体验了我们的疗愈师端mockup,并且分别进行了在线的讨论,表达了丰富的对于系统设计如何可以更好支持艺术治疗家庭作业的意见,以及交互体验方面的建议,然后我们对
\begin{table*}[tb]
\caption{Demographics of Participant Therapists: Experience refers to the number of years engaged in art therapy; The Number of Case refers to cases related to art therapy; The Number of Online Case refers to cases related to online art therapy}
\label{tab:expert}
\vspace{-3mm}
\small
\resizebox{\textwidth}{!}{
\begin{tabular}{ccccccccc}
\toprule
ID & Age & Gender & Experience & Education Level& Major & Region & The Number of Case&The Number of Online Case\\
\midrule
T1& 39& F& 6 & Master & Art Therapy & United States(Florida) &300+&65\\
T2& 41& F& 10 & Master & Art Therapy & Italy(Puglia) &200+&12\\
T3& 49& F& 8 & Master & Art Therapy & China(Guangdong) &350+&85\\
T4& 37& F& 5 & PhD & Cognitive Psychology\&Art Therapy&China(Hongkong) &100+&52\\
T5& 24& F& 2 & Master & Art Therapy& China(Hangzhou) &100&45\\
\bottomrule
\end{tabular}
}
\Description{Table 1 presents the demographics of the participant therapists. Experience refers to the number of years they have been engaged in art therapy, and the Number of Cases indicates the number of art therapy cases they have handled. The five therapists are as follows: T1 is 39 years old, female, with 6 years of experience. She holds a Master’s degree in Art Therapy and practices in Florida, United States, having managed over 300 cases. T2 is 41 years old, female, with 10 years of experience. She has a Master’s degree in Counseling Psychology and works in Puglia, Italy, with more than 200 cases. T3 is 49 years old, female, with 8 years of experience. She holds a Master’s degree in Art Therapy and is based in Guangdong, China, having overseen over 350 cases. T4 is 37 years old, female, with 5 years of experience. She has a PhD in Cognitive Psychology and Art Therapy and practices in Hong Kong, China, having handled more than 100 cases. T5 is 24 years old, female, with 2 years of experience. She holds a Master’s degree in Art Therapy and is located in Hangzhou, China, with approximately 100 cases managed. The Number of Online Case refers to cases related to online art therapy
}
\end{table*}

\subsection{Procedure and Preparation}
Five art therapists (T1-T5; all self-identified females; aged 24-49) participated in this study. None of the therapists were members of the research team. T3 was a previous collaborator; the other therapists were recruited via T3's professional network, intended for a diverse group of practitioners from various geographical locations.
Their demographics and expertise are detailed in~\autoref{tab:expert}.
We first conducted 60-minute remote one-on-one interviews with each therapist to understand their current homework practice. This was followed by two 60-minute online focus groups with the therapists. The researchers, acting as facilitators, moderated the discussion on the common challenges for homework practice, aiming to identify needs and design opportunities.
In addition, we kept close collaboration with the therapists throughout the development phase and conducted informal follow-ups to gather inputs in formulating design features.
The one-on-one interviews and focus group sessions were screen-recorded and transcribed. We conducted open coding and affinity diagramming to identify emerging insights reported below.

%Initially, we conducted remote, one-on-one semi-structured interviews with each therapist to gather insights into their current practices and the challenges regarding art therapy homework. 
%\textcolor{blue}{Subsequently, we held two 60-minute online focus group discussions with these therapists. The researchers, acting as facilitators, guided the discussions using the challenges and needs identified in prior one-on-one semi-structured interviews to encourage broader conversations. The goal was to identify the common needs and challenges therapists face in their practice, and, secondly, to closely collaborate with them in co-designing the system's core features.}
%Subsequently, we held two online focus groups with these therapists to foster a broader discussion, aiming to identify common needs and challenges in their practices.
%We recruited five art therapists (T1-T5; 5 self-identified females; aged 24-49) whose demographics and expertise are detailed in~\autoref{tab:expert}. 
%We first conducted remote, one-on-one semi-structured interviews with the five therapists in order to understand the current individuals' practices and challenges of art therapy homework. 
%Further, we conducted two remote focus groups with the five therapists in order to promote their discussion about these individuals' practices and challenges and identify common ground.
%First, we showcased the demo and mock-up, enabling the therapists to experience them. Following this, we went through online discussions where they provided feedback on how AI agent system design could enhance support for art therapy homework and offered suggestions for enhancing the interactive experience.


%Further, in order to co-design with the therapists, we developed a demo and mock-up as preparatory materials for context study. 
%First, Existing literature on art therapy homework, along with insights from interviews, suggests that therapy homework could combine art-making with verbal expression, yet no existing system combines these elements. Thus, we developed a demo featuring a drawing tool for AI image generation through semantic segmentation(similar to cases used in art therapy practices, such as DeepThink~\cite{du2024deepthink}) and a conversation agent that understands users' actions and asks questions to encourage description of the creation. 
%Second, we envisioned an therapist interface aimed at assisting therapists in monitoring and reviewing clients' therapy homework. The simplistic and open-ended demo and the static mock-up allowed therapists to contribute their expertise in future co-design workshops. To design an AI agent systems to support therapists and clients with art therapy homework, 


% formative study results  
% understanding current art therapy homework practice
        %介绍visual arts和written的形式
            % 的确有art thearpy homework
            % 疗愈师说了家庭作业是什么样的形式(介绍常有的形式,一般是艺术创作,记日记,拍照,做手工,)
             % 家庭作业很重要,为什么重要,有什么功能,可以怎么样影响来访者:
                % 1
                % 2
        %定制化和数据review
            %介绍定制化需求对于治疗师很重要
                % 治疗师会结合她掌握的疗愈技术和艺术治疗方法来定制家庭作业
                % 治疗师也会根据上一节session灵活调整家庭作业
            % 介绍review的重要性
                %艺术疗愈师需要看到家庭作业,需要用到家庭作业:为什么需要看到,为什么需要用到,怎么用的
                %retrieve,依从性compliance,不知道有没有按时做,
\subsection{Contextual Understanding: Current Practice and Common Challenges}

Our therapists confirmed that art therapy homework plays a crucial role in helping them understand and collaborate with clients between sessions. They shared their current methods for assigning art therapy homework, which often involves multi-modal activities~(see \autoref{fig:context1}) combining visual arts (e.g., drawing, collage-making, photography and clay sculpting) with written or spoken documentation of emotions and experiences (e.g., journaling, social media posts, and audio recordings).
The therapists noted that integrating visual presentations with verbal expression is a common practice, as it helps clients document and articulate their experiences. For example, T4 combined art-making with audio recording to assist clients in expressing their current feelings: \qt{I asked the elderly [clients] to take photos and create collages at home and encouraged them to record audio to share their daily emotions}. The therapists believed that this combination encourages clients to more fully describe their artwork, explore subconscious thoughts behind the creative process, and gain new perspectives on their lives.
Aside from their approach of leveraging art therapy homework in current practice, the therapists also share their challenges regarding art therapy homework. From their shared experiences, three major sets of challenges emerged, which are summarized below:
% As revealed by the therapists, they invited their clients to complete multi-modal forms of art therapy homework, mainly combining visual arts~(e.g., drawing, collage-making, photography) with the written and spoken document of current emotions and experiences~(e.g., journaling, social media posting, and audio recording).
% Our therapists noted that integrating visual presentations with verbal expression is a common practice in art therapy homework, as it helps clients document and articulate their current experiences.
% For example, T4 integrated art-making with audio recording to help clients document their current experiences and feelings. 
% Our therapists believed that combining art-making with verbal expression encourages clients to express and describe their artwork more fully, explore subconscious thoughts behind art-making, and cultivate new perspectives on various aspects of their lives. Further, our therapists emphasized they need to customize homework assignments in art therapy and track their the homework outcomes, which could build an therapeutic collaboration between therapists and clients.

\subsubsection{\textbf{CH1}: Challenges in Homework Threshold without Therapist Guidance} 

Our therapists indicated that art-making-based therapy homework can pose a creative barrier for clients without therapist guidance~(\textbf{CH1-1}). T4 noted that this barrier could lead to stress, self-criticism, and fear of failure: \qt{If a client is self-critical, they may fear creating something `ugly', which can increase pressure and hinder the therapeutic process}. Consistent with prior studies~\cite{Tang2017,Harwood2007}, the therapists also confirmed that clients may lack confidence in completing homework or producing emotional responses without guidance, which can result in lower compliance.
Additionally, therapists expressed concerns that clients might struggle to interpret their artwork in a therapeutic way without support, reducing their motivation for deep reflection~(\textbf{CH1-2}). T1 observed that without proper guiding, it can be difficult for clients to make full use of the exercise: \qt{Last time, I assigned a homework about `your ideal future family', but [...] she just scribbled a bit without expressing any clear thoughts}. The therapists emphasized the importance of guiding clients in verbalizing their emotions alongside art-making. T5 mentioned that while visual art can help explore subconscious thoughts, verbalizing these feelings provides a cathartic outlet and helps clients externalize their emotions.

% Moreover, therapists were concerned that clients might struggle to interpret their artwork in a therapeutic way without guidance, leading to reduced motivation for deep reflection. T1 noted that without clear direction, creating a meaningful drawing that promotes reflection can be difficult for clients.
% Additionally, therapists confirmed the importance of guiding clients to properly verbalize their feelings alongside art-making. As T5 mentioned, visual art can serve as a channel for exploring and expressing subconscious thoughts, while verbalizing these feelings provides a cathartic outlet and helps clients externalize their emotions.

% First, our therapists indicated that art-making-based therapy homework might present a creative threshold for clients.
% For example, T4 explained that the homework is to ensure that the creative process remains therapeutic and accessible, with low threshold, so participants can avoid stress, self-criticism, and fear of failure.
% Second, our therapists was concerned that clients may struggle to interpret their artwork in a therapeutic direction without the guidance of a therapist, which lead to a lack of motivation to engage in deep reflection. 
% Also, T1 suggested that it could be challenging for clients to create a meaningful drawing that effectively promotes reflection without clear guidance.
% Therapists also confirmed that it is crucial to prompt the clients to verbalize their feelings in addition to the art-making. As mentioend by T5, the visual art-making could be a channel for clients to explore and express themsleves at the subconscious level, whereas, verbalizationg could help them externalize the subconscious thoughts and find themself a carthartic outlet.
% Prior studies have shown that clients may struggle with confidence in completing homework and producing emotional arousal without the therapist's guidance~\cite{Tang2017,Harwood2007}, leading to reduced homework compliance.

\subsubsection{\textbf{CH2}: Challenges in Customizing Therapy Homework} 

Our therapists demonstrated their practice of customizing homework assignments in art therapy. For instance, T2 and T5 mentioned tailoring homework tasks and specific instructions based on their practical experience and therapeutic techniques (e.g., cognitive-behavior therapy or mindfulness): \qt{If I suggest therapy homework that integrates mindfulness with art-making, I might ask the client to notice any changes in their breathing [during homework]}~(T4). T1 also adjusted homework tasks based on feedback from previous in-sessions.
However, the therapists noted that adapting structured instructions flexibly was difficult with current verbal or written formats, often leading to clients forgetting or abandoning their guidance or instructions~(\textbf{CH2-1}). Additionally, T3 and T4 observed that offering encouraging words and support during homework could boost motivation, but they found it challenging to provide personalized encouragement outside of in-session times~(\textbf{CH2-2}).

% Our therapists demonstrated their practice of customizing homework assignments in art therapy, e.g., T2 and T5 both mentioned that 
% they tended to tailor diverse homework assignments and specific instructions based on drawing from their own practical experience and therapeutic techniques~(e.g., CBT or mindfulness): \qt{If I suggest therapy homework that integrates mindfulness with art-making, I might suggest that he noticed any changes in his breathing while observing the artwork~(T4)}.
% Also, T1 flexibly adjusted the homework tasks based on feedback from the previous in-session.
% However, our therapists noted that adapting structured instructions flexibly was challenging using existing verbal or written descriptions.
% This often lead to clients forgetting or abandoning their therapy homework.
% Moreover, T3 and T4 noted that offering encouraging words and support during homework could enhance motivation for completion. However, they currently find it challenging to tailor this encouragement and care after in-sessions.

\subsubsection{\textbf{CH3}: Challenges in Tracking Therapy Homework History} 

The therapists confirmed that original homework data---such as the artworks, conversation records about clients' creative states, and details of the creative process---were essential for their assessments. They also encouraged clients to bring homework outcomes to the next session. For example, T1 and T3 prompted clients to share their current feelings and perspectives during one-on-one sessions, while T4 encouraged clients to engage in re-creation based on their homework.
However, therapists commonly expressed difficulty in tracking homework history, as they relied on clients to record and report their own progress~(\textbf{CH3-1}): \qt{The client drew [an artwork] two months ago. When you showed her the artwork, she often didn't remember what had happened at the time~(T3)}. Additionally, T1, T3, and T4 raised concerns that current practices might miss valuable data regarding clients' emotional or mental states at the time the homework was completed~(\textbf{CH3-2}).

% Our therapists confirmed that original homework data, including the artwork, conversation records about clients' current creative states, and the creative process, were all crucial for their assessment.
% Meanwhile, the homework outcomes was encouraged to be brought to the next in-sessions, e.g., T1 and T3 encouraged clients to share their current feelings and perspectives on people and things during the one-on-one sessions. 
% Also, T4 encouraged clients to engage in re-creation activities during the in-sessions, building upon their homework outcomes.
% However, our therapists noted that they found difficult to track the homework history, as they relied on clients to record and report their own progress.
% Also, T1, T3, and T4 raised concerns that current practices might be missing valuable homework data regarding clients' homework assignments, specifically related to the client's status at the time the homework was completed.




   % challenges of current practice
        % 【但是】创作门槛高,便利性。(找话可以支持对应)
        % 【但是】:来访者缺少指导,没有引导,很难知道是否真的发生了反思(找话可以支持对应)
        % 【但是】:难以追踪,难以记录 (找话可以支持对应报告)
%\subsubsection{\textbf{Current challenges}}

%\textbf{D1: Supporting Therapy Homework by Integrating Verbal Expression with Art-making.} Our therapists suggested that therapy homework should be supported through combining art-making with verbal expression.They emphasized the value of integrating art-making and verbal expression in AI-infused art therapy. Likewise, T1 indicated that it not only enabled clients to gain a deeper understanding of their own artwork but also supported their process of self-expression. Further, the therapists envisioned that AI has the potential to further ask in-depth and structured questions based on artwork, thereby eliciting deeper reflections from clients. \textbf{D2: Supporting Customization of Therapy Homework via Agents.} T5 envisioned that conversational agents as \qt{homework assistants} that can guide clients to further explore some deeper self-reflections.Further, T2 suggested that AI agents have the advantage of conveying more caring and supportive messages from therapists to clients. Finally, our therapists noted that they needed to set homework topics and specific instructions in a therapist interface according to their own practice principles.\textbf{D3: Supporting Homework History Gathering and Summarization via AI agents.} The therapists further proposed that AI has the potential to assist in summarizing descriptions of clients' creations and capturing their emotions or experiences. T4 emphasized that AI agents should function as a summary tool rather than providing sophisticated interpretation. For example, T3 suggested that AI could identify and summarize recurring images in clients' artwork. This summarization can facilitate more in-depth discussions during one-on-one sessions.



\section{ORCA Design}
\label{sec:system-design}

%%%%%%%%%%%%%%%%%%%%%%%%%%%%%%%%%%%%%%%%%%%%%
\begin{figure*}[tp]
    \centering
    \includegraphics[width=\textwidth]{figures/overview.png}
    \vspace{-0.5cm}
    \caption{\shepherd{ORCA cloud-assisted design overview.}}
    \vspace{-0.2cm}
    \label{fig:system-overview}
\end{figure*}
%%%%%%%%%%%%%%%%%%%%%%%%%%%%%%%%%%%%%%%%%%%%%

\subsection{Overview}
\label{sec:system-overview}
Based on the observations and discussions in Section~\ref{sec:background-and-related-works} and~\ref{sec:preliminary-study}, we argue that an ideal edge-cloud collaborative learning system over LPWANs should have the following design considerations. First, to tackle the unreliability of wireless channels, a cloud-assisted strategy should be adopted rather than the state-of-the-art cloud-dependent offloading. Second, to adapt to the low bit rates of LPWANs and the on-device resource constraints, we demand a more efficient information exchange strategy. Additionally, from an audio processing perspective, we look for a more effective feature selection method to reduce input size and therefore reduce on-device computation overheads while maintaining comparable accuracy performances. Informed by these demands, we introduce our novel design of a resource-aware cloud-assisted environmental sounds recognition system, primarily operating over LoRa networks. Our system features resource-aware and communication-adaptive cloud assistance, enabling efficient and flexible cloud offloading under resource constraints and unreliable communications. Furthermore, we apply a novel self-attention-based frequency band feature selection method with the wavelet transform to effectively select important features for efficient on-device inference. We illustrate the workflow of the ORCA cloud-assisted framework in Figure~\ref{fig:system-overview}:

\noindent
\shepherd{\textbf{Step~\textcircled{\small{1}}:} Initially, the edge device preprocesses audio signals using low-level WPT to generate a low-resolution spectrogram. Preprocessing details are in Section~\ref{sec:preprocess}, and optimized resolution selection based on wireless channel feedback, e.g., Adaptive Data Rate (ADR),  is discussed in Section~\ref{sec:resource-aware-cloud-assistance}.}

\noindent
\shepherd{\textbf{Step~\textcircled{\small{2}}:} The resulting low-resolution spectrogram is transmitted to the server via uplink LoRa channel, using ADR-recommended parameters.}

\noindent
\shepherd{\textbf{Step~\textcircled{\small{3}}:} Upon receiving the low-resolution spectrogram, the server processes it using a pre-trained contrastive vision transformer~\cite{dosovitskiy2020vit} to extract an attention mask through attention rollout~\cite{abnar2020quantifying}. Details of the cloud model are provided in Section~\ref{sec:attention-mask-generation}.}

\noindent
\shepherd{\textbf{Step~\textcircled{\small{4}}:} The extracted attention mask, along with ADR feedback, is sent back to the edge device via downlink. Resource efficiency adaptations using ADR feedback are further discussed in Section~\ref{sec:resource-aware-cloud-assistance}.}

\noindent
\shepherd{\textbf{Step~\textcircled{\small{5}}:} The edge device validates the received attention mask. If invalid or lost, it bypasses cloud assistance and performs standalone on-device inference. We will also discuss this in Section~\ref{sec:resource-aware-cloud-assistance}.}

\noindent
\shepherd{\textbf{Step~\textcircled{\small{6}}:} If the mask is valid, the edge device refines the resolution to construct a multi-resolution spectrogram with details in Section~\ref{sec:spectral-encoding-cnn}.}

\noindent
\shepherd{\textbf{Step~\textcircled{\small{7}}:} Finally, with the multi-resolution spectrogram, the edge device performs efficient inference using spectral encoding and spectral CNNs with details in Section~\ref{sec:spectral-encoding-cnn}.}

\shepherd{To address resource constraints and dynamic communication costs, ORCA introduces a novel resource-aware scheduler for efficient cloud assistance on batteryless devices. Our algorithm dynamically adjusts to variable communication costs, enabling optimized communication scheduling in scenarios of high communication costs for adaptive transmission and bypassing. We will detail this algorithm in Section~\ref{sec:resource-aware-cloud-assistance}. }



% \begin{figure}[h]
%     \centering
%     \includegraphics[width=0.9\linewidth]{figures/selective-wpt.png}
%     \caption{Wavelet transform based on attention masks for selective frequency band resolution refinement.}
%     \label{fig:selective-WPT}
% \end{figure}




\subsection{Preprocessing}
\label{sec:preprocess}
\shepherd{To minimize communication costs, ORCA employs a low-resolution wavelet spectrogram as a compact and informative abstraction for cloud assistance. 
We use the WPT with depth $n$ to extract coarse frequency-domain features from the input audio waveform, producing a spectrogram $S$ with a frequency dimension of $2^n$. To generalize features over time and reduce payload size, we apply average pooling along the time axis, transforming $S$ into a square matrix $S_a$ in dimension of $2^n$. We refer to $S_a$ as the cloud-assisted spectrogram and define its dimension as the cloud assistance resolution $R_a = 2^n$, with selection details in Section~\ref{sec:resource-aware-cloud-assistance}.}


\subsection{Attention Mask Generation}
\label{sec:attention-mask-generation}
\shepherd{
In this section, we discuss how the server identifies important features from the assistance spectrogram $\mathcal{S}_a$. Specifically, we define important features as the most informative frequency bands, guided by preliminary studies. The edge device then leverages this information, encoded as an attention mask, to enhance on-device inference accuracy in later steps.
}

\noindent
\subsubsection{Vision Transformer for Assistance Spectrogram.}
\shepherd{ORCA server-side design leverages the self-attention mechanism to dynamically encode the importance of input features. The server processes the assistance spectrogram $\mathcal{S}_a$ by patching it into tokens and computing a self-attention map to highlight key regions. We show the attention computation in Figure~\ref{fig:attention-block}.
First, we adopt the same architecture from the vision transformer~\cite{dosovitskiy2020vit} and divide the input spectrogram into $p^2$ patches. To preserve the spectrogram’s spectral-temporal properties, we apply positional encoding by adding trainable encoding to each patch.
Next, we pass the patches through a convolutional patch embedding layer, encoding each patch into an embedding of dimension $E$.
The resulting embedding is passed through the $i$-th attention block to compute the attention matrix $A_i$, sequentially. Formally, $A_i = \text{Softmax}(Q_i \cdot K_i^T / \sqrt{E})$, where $Q_i$ and $K_i$ are the query and key embeddings at each layer. The attention matrix $A_i$ of size $p^2 \times p^2$ captures the relative importance between patch pairs, aiding in identifying the most informative frequency bands, as discussed next.
}

\begin{figure}[tp]
    \centering
    \includegraphics[width=\linewidth]{figures/attention-block.png}
    \vspace{-0.3cm}
    \caption{Attention computation for attention rollout.}
    \label{fig:attention-block}
    \vspace{-0.3cm}
\end{figure}


\noindent
\subsubsection{Attention Mask Generation.}
Recall that the attention matrix $A_i$ represents the attention map of the $i$-th attention block, encoding the relative importance between patches in a spectrogram. 
% Sequential application of attention blocks can cause attention signals to vanish due to constant information mixing between tokens~\cite{abnar2020quantifying}. 
\shepherd{
Inspired by~\cite{abnar2020quantifying}, we compute the rollout attention map $\widetilde{A} = \Pi^{1}_{i=n} A_i = A_n A_{n-1} \cdots A_{1}$ for importance estimations.} This approach aggregates attention matrices from all blocks, enhancing interpretability and preventing attention scores from vanishing. 
The resulting rollout attention map $\widetilde{A}$ has dimensions $p^2 \times p^2$. 
Then, we aim to identify the most informative frequency bands for the edge. Intuitively, a frequency band is informative if patches within that band have high attention scores, as this indicates that the cloud model prioritizes those patches. Therefore, let $\widetilde{a}_{ij}$ represent the rollout attention between patches $i$ and $j$ in $\widetilde{A}$. We compute the column-wise summation $C$ of $\widetilde{A}$ as $C = [c_1, c_2, \cdots, c_{p^2}]$ where $c_j = \sum_{i=1}^{p^2} \widetilde{a}_{ij}$. 
The vector $C$ is reshaped into a 2D importance matrix $C'\in \mathbb{R}^{p \times p}$, where each entry represents the importance of a patch in the input WPT spectrogram. 
\shepherd{We select frequency bands by summing contiguous $k$ rows in $C'$ and identifying the highest sum, where $k$ is a predefined hyperparameter agreed upon by the server and edge device.} A binary vector of length $p$ records the selected indices, forming the spectral attention mask, which is sent to edge devices.

\noindent
\subsubsection{Contrastive Pre-Training.} \shepherd{The method above relies on a vision transformer capable of identifying informative frequency bands from the WPT spectrogram.} Given the lack of labeled data for frequency-domain feature importance information, we propose training the cloud model offline in an unsupervised manner. \shepherd{Inspired by contrastive learning, where the model learns to produce distinctive features via contrastive loss, we create attracting and contrasting pairs by masking random frequency bands and use triplet loss~\cite{schroff2015facenet} on the flattened output of vision transformer as representations.} Overall, the advantage of ORCA attention-based cloud assistance solution is twofold: first, it uses self-attention over spectrograms to guide clients in focusing on informative frequency bands, which not only improves inference accuracy on the resource-constrained edge devices but also reduces computational load by minimizing the edge model input size. Additionally, transmitting the low-resolution assistance spectrogram and attention masks is highly communication-efficient, significantly reducing communication costs and latency.


\subsection{\shepherd{Cloud-Assisted Inference}}
\label{sec:spectral-encoding-cnn}
\shepherd{
% Spectrograms provide detailed information across all frequency bands, and wavelet transform enables extraction of frequency-band details at flexible resolutions. In preliminary study, we demonstrated that higher-resolution spectrograms achieve high accuracy but increase resource usage. Inspired by the preliminary study 2, we propose a three-fold on-device inference solution: Multi-resolution Refinement, Spectral Encoding, and Multi-resolution CNN designs, selectively focusing on information-rich spectral bands from the aforementioned cloud assistance step. and still maintain high accuracy. This section details our design of a multi-resolution spectral encoding CNN (Steps \textcircled{\small{6}} and \textcircled{\small{7}} in Figure~\ref{fig:system-overview}), optimized for resource-constrained devices through the use of cloud-generated spectral attention masks discussed in the previous section.
Following the discussion on server-generated attention masks, we explore how edge devices can leverage this information for efficient on-device inference. First, we introduce the \textit{Multi-resolution Refinement} module, which extracts high-resolution frequency bands guided by attention masks. After refinement, two challenges remain: (i) embedding high-resolution spectral bands and (ii) creating a multi-resolution representation for accurate and efficient inference. For (i), we propose \textit{Spectral Encoding}, a trainable weight that encodes high-resolution frequency band-specific knowledge. For (ii), we employ \textit{Multi-resolution CNNs} to process the combination of high-resolution bands from multi-resolution refinement and their corresponding spectral encoding for efficient on-device classification.}

\noindent
\subsubsection{Multi-resolution Refinement.}
\shepherd{The server-generated spectral attention mask captures key frequency bands. It guides the edge device to selectively extract high-resolution spectrograms via wavelet transform. Let $R_l$ denote the pre-defined dimension of the low-resolution spectrogram and $R_h$ the dimension of the high-resolution spectral bands, this refinement results in $R_l$-dimensional low-resolution spectrograms and $R_h$-dimensional high-resolution spectrograms frequency bands.} To further reduce dimension, adaptive average pooling is applied along the time dimension, regularizing the size of both spectrograms.

\noindent
\subsubsection{Spectral Encoding.}
\shepherd{Since each frequency band captures unique frequency-domain properties, spectrograms from different bands should be interpreted accordingly. Using separate CNNs per band~\cite{phaye2019subspectralnet} is memory-inefficient and costly. Instead, inspired by transformer's positional encoding, we use spectral encoding, a trainable weight that encodes frequency band-specific information. It is then concatenated channel-wise to corresponding high-resolution bands, as shown in Figure~\ref{fig:spectral-encoding}. This approach helps the network to learn spectral-specific knowledge independently of the input spectrogram.}

\begin{figure}[tp]
    \centering
    \includegraphics[width=\linewidth]{figures/spectral-encoding.png}
    \vspace{-0.8cm}
    \caption{\shepherd{Spectral encoding and multi-resolution CNNs.}}
    \label{fig:spectral-encoding}
    \vspace{-0.4cm}
\end{figure}


\noindent
\subsubsection{Multi-resolution CNN.}
\shepherd{The next challenge is to create a multi-resolution representation for inference. As discussed in preliminary studies in Section~\ref{sec:preliminary-study}, discriminative information varies between spectral bands of the spectrogram. With the full low-resolution spectrogram available from preprocessing, we use two 2-layer shallow CNN as encoders, one for low resolution and one for high resolution. The encoded features are fused channel-wise into a single vector and fed into the Multi-Res classifier for final classification. This architecture reduces inference costs by leveraging spectrograms at different resolutions. If cloud assistance is unavailable, an additional Single-Res classifier is employed to process the output of the Low-Res encoder only. All components are pre-trained offline in a two-stage supervised process. First, we train the low-res encoder, high-res encoder, and multi-resolution classifier together with the attention masks generated by the pre-trained cloud vision transformer. In the second stage, we freeze all other components and train the single-resolution classifier independently.}

%%%%%%%%%%%%%%%%%%%%%%%%%%%%%%%%%%%%%%%%%%%%%%%%%%%%%%%%%%%%%%%
\begin{figure}[tp]
    \centering
    \includegraphics[width=\linewidth]{figures/intermittent.png}
    \vspace{-1.0cm}
    \caption{Execution model (up), capacitor voltage (mid), and relative power consumptions (low).  }
    \label{fig:intermittent}
\end{figure}
%%%%%%%%%%%%%%%%%%%%%%%%%%%%%%%%%%%%%%%%%%%%%%%%%%%%%%%%%%%%%%%

\subsection{Resource-Aware Scheduler}
\label{sec:resource-aware-cloud-assistance}
\shepherd{Given the high energy cost of communication and wireless uncertainty, dynamically managing data transmission size is essential for resource-efficient cloud assistance.} Experimental measurements~\cite{mileiko2023run} indicate that the uplink phase dominates energy consumption in each communication round and varies with channel conditions. Thus, a key component of our framework is optimizing uplink data transmission. 
\shepherd{We introduce a resource-aware, communication-adaptive resolution algorithm. This algorithm dynamically schedules the assistance resolution $R_a$ (as discussed in Section~\ref{sec:preprocess}) based on energy storage and communication quality for resource-efficient cloud assistance.}

\noindent
\subsubsection{Communication Model.} 
As discussed in Section~\ref{sec:system-overview}, ORCA uses two communication phases for one round of cloud assistance, uplink (Tx) and downlink (Rx). It adopts the intermittent computation model from~\cite{mileiko2023run} which concludes an uplink and a downlink in the same power cycle with a synchronized sleep period interleaved. \shepherd{The key advantage of this design is maintaining inference integrity and timeliness for cloud assistance, even during prolonged power failures in batteryless systems.} We illustrate this design in Figure~\ref{fig:intermittent}. Within one power cycle, edge device initiates by restoring the communication parameters, spreading factor (SF) and transmitting power ($P_{\text{Tx}}$) once waking up at voltage threshold $V_{\text{on}}$. Then it goes through sampling and preprocessing, Tx, sleeping, Rx, and on-device inference sequentially as discussed in Section~\ref{sec:system-overview}. Between each power cycle, our edge device checkpoints and restores SF and $P_{\text{Tx}}$ in and out of the non-volatile memory (yellow blocks in Figure~\ref{fig:intermittent}). This ensures their synchronizations to the server's recommendation for reliable communication. Here, the generic ADR algorithm~\cite{Semtech2016LoRaWAN} is employed to estimate the optimal communication parameters ensuring communication reliability. Every time the server receives an uplink packet, it calculates and compares the SNR margins to the optimal values and recommends the optimal SF and $P_{\text{Tx}}$ back to the edge device in downlink message. Edge device can then checkpoint these parameters for next round of communication. The next challenge is to complete restoring, preprocessing, Tx, sleep, Rx, inference, and checkpointing within one power cycle.

% Additionally, based on the measurements by~\cite{mileiko2023run}, uplink dominates the energy consumption and therefore, our algorithm mainly focuses on optimizing uplink data transmission.
% For uplink, the transmission time, known as the time-on-air (ToA), depends on the data rate (DR) and can be estimated by $\text{ToA} = \frac{S + S_{\text{p}}}{\text{DR}}$ for sending a payload size of $S$ with a fixed preamble $S_{\text{p}}$. 
% Conversely, a fixed time window ($T_{\text{Rx}}$) is set up for listening to the downlink signal after the sleeping period timeout. 
% Given the variability of the wireless channel, the costs of ensuring signal quality for the uplink changes constantly. To address this, we employ the adaptive data rate (ADR)~\cite{Semtech2016LoRaWAN}. By analyzing SNR history, the server recommends an optimal spreading factor (SF) and transmit power ($P_{\text{Tx}}$) for the edge device. Edge device then adjusts these parameters to ensure reliable uplink transmission. 



\begin{figure}[tp]
    \centering
    \includegraphics[width=\linewidth]{figures/workflow.png}
    \vspace{-0.5cm}
    \caption{Cloud-assisted offloading flow chart.}
    \label{fig:workflow}
    \vspace{-0.2cm}
\end{figure}

\noindent
\subsubsection{Adaptive Resolution.}
\shepherd{Given the proposed communication model and parameters, we first examine the key factors influencing energy consumption.} Since batteryless devices usually wake up at a pre-defined voltage threshold, the energy budget per power cycle is typically fixed and can be estimated by $E_{\text{cap}}=\frac{C}{2}(V_{\text{on}}^2 - V_{\text{off}}^2)$, where $C$ is capacitance, and $V_{\text{on}}$ and $V_{\text{off}}$ represent the microcontroller switching voltage thresholds (on and off, respectively), as depicted in Figure~\ref{fig:intermittent}. 
We propose ORCA resource-aware adaptive resolution algorithm for cloud assistance, designed to adapt to varying communication costs and complete each round of cloud assistance within a single power cycle. Our approach determines an optimal assistance resolution $R_a$ which in turn defines the payload size $S={R_a}^2$ for uplink. We model the adaptive resolution algorithm with the parameters followed. The uplink energy consumption $E_{\text{Tx}}$ can be estimated as: $E_{\text{Tx}} = P_{\text{Tx}} \cdot \text{ToA} = P_{\text{Tx}} ({R_a}^2+S_{\text{p}})/\text{DR}$.
where the uplink transmission time, known as the time-on-air (ToA), depends on the various data rate (DR) under different SF and can be estimated by $\text{ToA} = (S + S_{\text{p}})/\text{DR}$ for sending a payload size of $S$ with a fixed preamble $S_{\text{p}}$. 
The downlink energy cost is estimated as $E_{\text{Rx}}=P_{\text{Rx}} \cdot T_{\text{Rx}}$, the product of the downlink power and the downlink window length. Additionally, $E_{\text{Pre}}$, $E_{\text{sleep}}$, and $E_{\text{inf}}$ are for energy usage during preprocessing, sleep period, and inference, respectively, and can be considered as constants in ORCA. Moreover, to formulate the optimization problem, we define the one-hot encoded resolution selection vector $x$ for resolution ${R_a}$ and the pre-estimated accuracy vector $a$ for accuracy under different ${R_a}$ values. To complete a round of cloud assistance within a single power cycle, the model ensures $E_{\text{Pre}} + E_{\text{Tx}} + E_{\text{sleep}} + E_{\text{Rx}} + E_{\text{inf}} \leq E_{\text{cap}}$. 
We define the following optimization problem, finding the optimal resolution selection vector $x$ to maximize the accuracy under energy constraints:
% \setlength{\abovedisplayskip}{2pt}
% \setlength{\belowdisplayskip}{2pt}
\begin{equation*}
\begin{aligned}
\max_{x} \ a^{T}x \quad \textrm{s.t.} \quad & E_{\text{Tx}}(x)+E_{\text{Pre}} + E_{\text{sleep}} + E_{\text{Rx}} + E_{\text{inf}}\leq E_{\text{cap}}\\[-0.2em] 
  &\textbf{1}^Tx = 1, \ x_i = \{0, 1\} \\[-0.2em]
  \end{aligned}
\end{equation*}
The optimal resolution selection is derived by ${R_a}=\text{argmax}(x)$, and, specifically, we define ${R_a}=0$ as local bypassing without cloud assistance. In practice, since the optimization search space is small (as $R_a$ is chosen from only a few options) and the capacitor is pre-selected to ensure enough budget for at least local inference without cloud assistance, we simply iterate through all feasible solutions within the energy budget and select the one with the highest estimated accuracy.

\noindent
\subsubsection{Workflow.}
The workflow is presented in Figure~\ref{fig:workflow}. Starting with the communication parameters in the yellow block, we use the energy storage $E_{\text{cap}}$ and communication parameter recommendations from the previous round as the budget and cost inputs, respectively. These inputs are applied to the optimization problem, where the edge device determines the optimal $R_a$ for maximum assistance accuracy and then uploads the low-resolution spectrogram. The server extracts and transmits the attention masks along with the ADR in the downlink back to the edge device. The edge device verifies downlink message validity using the CRC error check or by missing packets after a downlink timeout, treating invalid messages as such. If valid, the edge device proceeds with the multi-resolution inference step as described in Section~\ref{sec:spectral-encoding-cnn}. Otherwise, due to resource constraints, the device bypasses retransmission and cloud assistance, performing single-resolution on-device inference as also detailed in Section~\ref{sec:spectral-encoding-cnn}. Overall, ORCA using fixed energy budgets and dynamic data size offers two major advantages. First, unlike reconfigurable energy storage solutions, which require additional hardware and may face durability or read-write cycle limitations~\cite{colin2018reconfigurable, bakar2022protean, mileiko2023run}, our strategy does not require extra hardware. Second, our algorithm intelligently balances communication costs and accuracy gains by adaptively selecting the amount of resources for cloud assistance. As shown in Figure~\ref{fig:resource-aware}: (i) when communication cost is low, the edge device sends a high-resolution spectrogram for better inference accuracy; (ii) when communication cost is high, it sends a low-resolution spectrogram with a smaller payload to manage energy cost, resulting in lower accuracy; (iii) if communication is unstable with packet loss, the device bypasses cloud assistance and performs local inference to avoid costly retransmissions.


\begin{figure}[tp]
    \centering
    \includegraphics[width=\linewidth]{figures/resource-aware.png}
    \vspace{-0.5cm}
    \caption{Resource-aware adaptive resolution for cloud assistance.}
    \label{fig:resource-aware}
    \vspace{-0.2cm}
\end{figure}



\begin{table*}[t]
\centering
\tiny
\begin{tabular}{|M{1.2cm}|M{0.7cm}|M{1cm}|M{1cm}|M{1cm}|M{0.8cm}|M{1.2cm}|M{0.7cm}|M{1cm}|M{1cm}|M{1cm}|M{0.8cm}|}
\hline\hline
Model & \#GPU & \#Strategies & Search Time(/s) & Simulation Time(/s) & E2E Time(/s) & Model & \#GPU & \#Strategies & Search Time(/s) & Simulation Time(/s) & E2E Time(/s) \\ \hline
\multirow{4}{*}{Llama-2-7B} & 64 & 23348 & 0.06 & 49.7 & 51.0 & \multirow{4}{*}{Llama-2-13B} & 64 & 23400 & 0.05 & 58.1 & 59.3 \\ \cline{2-6} \cline{8-12} 
 & 256 & 14372 & 0.05 & 43.5 & 44.4 &  & 256 & 13552 & 0.03 & 49.9 & 50.8 \\ \cline{2-6} \cline{8-12} 
 & 1024 & 8856 & 0.04 & 41.8 & 42.2 &  & 1024 & 8920 & 0.02 & 51.0 & 51.7 \\ \cline{2-6} \cline{8-12} 
 & 4096 & 4700 & 0.03 & 33.0 & 33.2 &  & 4096 & 4720 & 0.02 & 44.1 & 44.3 \\ \hline
\multirow{4}{*}{Llama-2-70B} & 64 & 53264 & 0.1 & 68.8 & 75.0 & \multirow{4}{*}{Llama-3-8B} & 64 & 23348 & 0.05 & 48.3 & 49.6 \\ \cline{2-6} \cline{8-12} 
 & 256 & 31440 & 0.06 & 57.7 & 60.9 &  & 256 & 14372 & 0.04 & 42.0 & 42.8 \\ \cline{2-6} \cline{8-12} 
 & 1024 & 20152 & 0.05 & 57.4 & 59.6 &  & 1024 & 8856 & 0.03 & 40.9 & 41.3 \\ \cline{2-6} \cline{8-12} 
 & 4096 & 10948 & 0.04 & 63.2 & 65.0 &  & 4096 & 4700 & 0.03 & 32.7 & 32.9 \\ \hline
\multirow{4}{*}{Llama-3-70B} & 64 & 53264 & 0.1 & 66.8 & 71.8 & \multirow{4}{*}{GLM-67B} & 64 & 20528 & 0.04 & 19.3 & 20.6 \\ \cline{2-6} \cline{8-12} 
 & 256 & 31440 & 0.07 & 56.3 & 59.6 &  & 256 & 12132 & 0.03 & 16.6 & 17.4 \\ \cline{2-6} \cline{8-12} 
 & 1024 & 20152 & 0.05 & 55.5 & 57.6 &  & 1024 & 7948 & 0.02 & 16.9 & 17.3 \\ \cline{2-6} \cline{8-12} 
 & 4096 & 10948 & 0.04 & 62.4 & 63.4 &  & 4096 & 4196 & 0.02 & 21.3 & 21.5 \\ \hline
\multirow{2}{*}{GLM-130B} & 64 & 33540 & 0.06 & 22.4 & 52.4 & \multirow{2}{*}{GLM-130B} & 1024 & 11976 & 0.03 & 16.7 & 18.2 \\ \cline{2-6} \cline{8-12} 
 & 256 & 18776 & 0.04 & 17.2 & 19.4 &  & 4096 & 6040 & 0.02 & 19.2 & 20.1 \\ \hline\hline
\end{tabular}%
\caption{
    The search space and the time cost for \sysname on Heterogeneous GPUs.
  For the pictures of time cost, the light color without hatches represents the time spent searching, while the deep color with hatches represents the time spent simulating.
  We can observe that it only takes \sysname\ about 1 minute to complete the end-to-end simulation. 
}
\label{tab:exp:cost}
\end{table*}

\section{Experiments}\label{sec:exp}


%In this section, we first evaluate \sysname's cost model accuracy under different settings to build the basis for the search in \S\ref{sec:exp:accuracy}.
%We show the search space of \sysname, and the search time cost for the search in \S\ref{sec:exp:cost}.
%Then, t
To prove \sysname's optimal search ability on MegatronLM, we did a comparative analysis between \sysname\ and experts on MegatronLM in \S\ref{sec:exp:expert}.
%After that, we compare \sysname with existing auto-parallel frameworks, including Alpa, Galvatron, etc., in \S\ref{sec:exp:comparison}.
Finally, we evaluate \sysname to search for the finance-optimal plan under different settings in \S\ref{sec:exp:finance}.

%\subsection{Cost Model Accuracy}\label{sec:exp:accuracy}
%



\section{Cost Analysis}\label{sec:exp:cost}

\sssec{Method}.
We did a cost analysis to show the gap between the large search space and the search efficiency of the \sysname.
We selected Llama-2 models (7B, 13B, and 70B) with 64, 256, 1024, and 4096 GPUs.
Then, for all the settings, we implemented \sysname\ on it and recorded the searched strategy number along with the end-to-end time (search time and simulation time)


\sssec{Result}. As shown in Table \ref{tab:exp:cost}, the number of explored strategies grows exponentially with model size. For smaller models like Llama-7B, even with 4096 GPUs, the search space remains relatively small. However, for larger models such as Llama-70B, the search space nearly triples compared to Llama-7B under the same GPU configuration. The end-to-end time reveals that the simulation phase is the main bottleneck, which may take 1 minute to execute on average. While the search time only takes less than 1 second to execute on average. This highlights the need for optimizing the simulation process, particularly in large-scale settings, while \sysname’s search algorithm remains efficient and scalable across different configurations.




\begin{figure*}[thbp]
  \centering
    \subfloat{\includegraphics[width=0.4\textwidth]{figs/fig-expert-legend.pdf}}\\
    \addtocounter{subfigure}{-1}

    \begin{minipage}{\textwidth}
    {\centering{\hspace{2.8cm}A800\hspace{4cm}H100\hspace{4.2cm}H800}}
    \end{minipage}

    \raisebox{0.8cm}{\rotatebox[origin=c]{90}{Llama-2}}
    \subfloat[7B]{\includegraphics[width=0.106\textwidth]{figs/fig-expert-A800-llama2-7b.pdf}}
    \subfloat[13B]{\includegraphics[width=0.106\textwidth]{figs/fig-expert-A800-llama2-13b.pdf}}
    \subfloat[70B]{\includegraphics[width=0.106\textwidth]{figs/fig-expert-A800-llama2-70b.pdf}}
    \subfloat[7B]{\includegraphics[width=0.106\textwidth]{figs/fig-expert-H100-llama2-7b.pdf}}
    \subfloat[13B]{\includegraphics[width=0.106\textwidth]{figs/fig-expert-H100-llama2-13b.pdf}}
    \subfloat[70B]{\includegraphics[width=0.106\textwidth]{figs/fig-expert-H100-llama2-70b.pdf}}
    \subfloat[7B]{\includegraphics[width=0.106\textwidth]{figs/fig-expert-H800-llama2-7b.pdf}}
    \subfloat[13B]{\includegraphics[width=0.106\textwidth]{figs/fig-expert-H800-llama2-13b.pdf}}
    \subfloat[70B]{\includegraphics[width=0.106\textwidth]{figs/fig-expert-H800-llama2-70b.pdf}}
    \\
    \raisebox{0.8cm}{\rotatebox[origin=c]{90}{Llama-3}}
    \subfloat[8B]{\includegraphics[width=0.16\textwidth]{figs/fig-expert-A800-llama3-8b.pdf}}
    \subfloat[70B]{\includegraphics[width=0.16\textwidth]{figs/fig-expert-A800-llama3-70b.pdf}}
    \subfloat[8B]{\includegraphics[width=0.16\textwidth]{figs/fig-expert-H100-llama3-8b.pdf}}
    \subfloat[70B]{\includegraphics[width=0.16\textwidth]{figs/fig-expert-H100-llama3-70b.pdf}}
    \subfloat[8B]{\includegraphics[width=0.16\textwidth]{figs/fig-expert-H800-llama3-8b.pdf}}
    \subfloat[70B]{\includegraphics[width=0.16\textwidth]{figs/fig-expert-H800-llama3-70b.pdf}}
    \\
    \raisebox{0.8cm}{\rotatebox[origin=c]{90}{GLM}}
    \subfloat[67B]{\includegraphics[width=0.16\textwidth]{figs/fig-expert-A800-glm-67b.pdf}}
    \subfloat[130B]{\includegraphics[width=0.16\textwidth]{figs/fig-expert-A800-glm-130b.pdf}}
    \subfloat[67B]{\includegraphics[width=0.16\textwidth]{figs/fig-expert-H100-glm-67b.pdf}}
    \subfloat[130B]{\includegraphics[width=0.16\textwidth]{figs/fig-expert-H100-glm-130b.pdf}}
    \subfloat[67B]{\includegraphics[width=0.16\textwidth]{figs/fig-expert-H800-glm-67b.pdf}}
    \subfloat[130B]{\includegraphics[width=0.16\textwidth]{figs/fig-expert-H800-glm-130b.pdf}}
  \caption{
  We compare \sysname's searched optimal plan's throughput with expert's proposed plan's throughput in single-GPU setting.
  }
  \label{fig:expert:throughput}
  \vspace{-10pt}
\end{figure*}

\subsection{Mode-1: Comparison with Expert Plans}\label{sec:exp:expert}

\sssec{Method}.
To prove the \sysname's ability to search the optimal strategy on MegatronLM, we compared \sysname\ with an expert.
We first selected three models with different parameter sizes (7 model settings in total): Llama-2 (7B, 13B, and 70B), Llama-3 (8B, 70B), and GLM (67B, 130B).
Then, we offer 4 GPU number settings: 32, 128, 256, and 1024.
Next, we asked six experts to craft a parallel strategy for each setting (different models and different GPU settings, overall $7\times 4=28$ settings) based on their expert experience.
Each participant has over six years of industry machine learning service or training experience.
Then, we ran each of the six participants' parallel strategies for each setting on MegatronLM and picked the optimal one (one with the largest throughput) among the six expert-crated strategies as the expert-optimal strategy.
At last, we run \sysname\ to search the optimal parallel strategy automatically and compare the \sysname's parallel strategy's throughput with the expert-optimal parallel strategy's throughput.

\sssec{Results}.
As shown in Fig. \ref{fig:expert:throughput}, \sysname demonstrates its ability to automatically generate parallel strategies that match or exceed expert-tuned plans across various model configurations. This highlights \sysname's capability to generalize and optimize without manual intervention.

\par A key finding is that \sysname consistently matches or outperforms manually designed strategies, showing that its automated search can achieve results on par with domain experts. This adaptability extends across diverse hardware and model types, while specific setups often constrain expert-tuned plans. \sysname dynamically adjusts to different configurations, optimizing parallel strategies based on the specific training environment.

\par Another important observation is \sysname’s flexibility in combining different parallelism techniques—data, tensor, and pipeline. While expert strategies often focus on one type of parallelism, \sysname optimally balances multiple forms, leading to superior performance, especially for large-scale models. This hybrid approach is likely the key to future parallelism strategies, where flexibility and adaptation are critical.
%\subsection{Comparison with Other Schemes}\label{sec:exp:comparison}

\begin{table}[h!]
\centering
\caption{GPT-3 Model Specification}
\label{tab:gpt-3}
\begin{tabular}{ccccc}
\hline
\#params & Hidden size & \#layers & \#heads & \#gpus \\ \hline\hline
350M & 1024 & 24 & 16 & 1 \\ 
1.3B & 2048 & 24 & 32 & 4 \\ 
2.6B & 2560 & 32 & 32 & 8 \\ 
6.7B & 4096 & 32 & 32 & 16 \\ 
15B & 5120 & 48 & 32 & 32 \\ 
39B & 8192 & 48 & 64 & 64 \\ \hline\hline
\end{tabular}
\end{table}


\begin{table}[h!]
\centering
\caption{LLaMA Model Specification}
\label{tab:llama}
\begin{tabular}{ccccc}
\hline
\#params & Hidden size & \#layers & \#heads & \#gpus \\ \hline\hline
7B & 4096 & 32 & 32 & 8 \\
13B & 5120 & 40 & 40 & 16 \\
33B & 6656 & 60 & 52 & 32 \\
70B & 8192 & 80 & 64 & 64 \\ \hline\hline
\end{tabular}
\end{table}

\begin{table}[h!]
\centering
\caption{GShard MoE Model Specification}
\label{tab:moe}
\begin{tabular}{cccccc}
\hline
\#params & Hidden size & \#layers & \#heads & \#experts & \#gpus \\ \hline\hline
380M & 768 & 8 & 16 & 8 & 1 \\
1.3B & 768 & 16 & 16 & 16 & 4 \\
2.4B & 1024 & 16 & 16 & 16 & 8 \\
10B & 1536 & 16 & 16 & 32 & 16 \\
27B & 2048 & 16 & 32 & 48 & 32 \\
70B & 2048 & 32 & 32 & 64 & 64 \\ \hline\hline
\end{tabular}
\end{table}

\sssec{Models and training workflows}.
For our experiments, we target three types of models: GPT-3, LLaMA, and a Mixture of Experts (MoE) model. These models represent a range of architectures, from homogeneous to heterogeneous, providing a comprehensive evaluation of our parallelism strategies. 

\par \textbf{GPT-3} (see Table \ref{tab:gpt-3}) is a homogeneous Transformer-based language model comprising many stacked layers. Its model parallelization plan has been extensively studied and optimized in various research efforts. \textbf{LLaMA} (see Table \ref{tab:llama}) is another advanced Transformer-based model designed for language modeling, with a focus on efficiency and performance in both pre-training and fine-tuning phases. \textbf{MoE} models (see Table \ref{tab:moe}), such as GShard, combine dense and sparse architectures by incorporating a mixture of expert layers. These layers replace the feed-forward layers in every few Transformer layers, making them highly adaptable to different computational environments.

\par To study the scalability and efficiency of training large models, we follow standard machine learning practices by scaling the model size proportionally with the number of GPUs, as reported in Table 4. For GPT-3, we increase the hidden size and the number of layers concurrently with the number of GPUs, following the methodology used in previous studies. For the MoE model, we primarily increase the number of experts, which is crucial for leveraging the model's sparse architecture and optimizing performance across multiple GPUs. For LLaMA, we adjust the model's depth (number of layers) and width (hidden size) to ensure it scales effectively with the available GPU resources.

\par In each experiment, we adopt the recommended global batch size per established ML practices to maintain consistent statistical behavior across different model configurations. We then fine-tune the micro-batch size for each model and system configuration to maximize overall system performance, with gradient accumulation applied across micro-batches.

\sssec{Baselines}. For each model, we compare our system, \sysname, against strong baselines, including Alpa and Galvatron, and manually designed strategies using Megatron-LM.

\par \textbf{Alpa} is chosen as one of the baselines due to its automated parallelization capabilities, particularly for large-scale models. Alpa utilizes a combination of intra-operator and inter-operator parallelism to optimize the training process. We configure Alpa to its best settings by following the guidelines provided in their documentation and research papers. Alpa is known for its comprehensive strategy space, which includes various parallelism paradigms such as data parallelism, tensor parallelism, and pipeline parallelism.

\par \textbf{Galvatron} is another baseline we employ, noted for its efficient transformer training over multiple GPUs using automatic parallelism. Galvatron incorporates multiple popular parallelism dimensions and automatically discovers the most efficient hybrid parallelism strategy through a decision tree decomposition and a dynamic programming search algorithm. We perform a grid search to determine the optimal configurations for Galvatron, ensuring that we fully leverage its capabilities.

\par \textbf{Megatron-LM} serves as the manually designed baseline, specifically for GPT-like models. Megatron-LM v2 is a state-of-the-art system that combines data parallelism, pipeline parallelism, and manually designed operator parallelism (denoted as TMP). This combination is controlled by three integer parameters that specify the degrees of parallelism assigned to each technique. Following the guidance from their research, we conduct a thorough grid search of these parameters and report the best configuration results. While Megatron-LM is highly specialized for GPT-like models, it does not support other models in our evaluation due to its lack of flexibility in handling different architectures.

Our comparison does not include open-source systems like \textbf{FlexFlow} and \textbf{Tofu} due to their limitations. FlexFlow lacks support for essential operators such as layer normalization and mixed-precision operators, and Tofu only supports single-node execution and is not open-sourced. Given these theoretical and practical constraints, we do not expect FlexFlow or Tofu to outperform the state-of-the-art manual baselines in our evaluation.

In summary, our evaluation includes \sysname, Alpa for its automated strategy space, Galvatron for its efficient hybrid parallelism discovery, and manually tuned Megatron-LM for its specialization in GPT-like models. This comprehensive approach thoroughly compares different parallelism strategies and model architectures.

\sssec{Evaluation metrics}. We measure training throughput in our evaluation. We evaluate the system's weak scaling when increasing the model size and the number of GPUs. Following \cite{narayanan2021efficient}, we use the aggregated peta floating-point operations per second (PFLOPS) of the whole cluster as the metric. After proper warmup, we measure it by running a few batches with dummy data. All our results (including those in later sections) have a standard deviation within 0.5\%, so we skip the error bars in our figures.

\sssec{GPT-3 results}.
\textcolor{red}{To be done}

\sssec{Llama results}.
\textcolor{red}{To be done}

\sssec{MoE results}.
\textcolor{red}{To be done}

\subsection{Mode-2: Heterogeneous GPU Search}

\begin{figure}[t]
  \centering
    \subfloat{\includegraphics[width=0.48\textwidth]{figs/fig-heter-legend.pdf}}\\
    \addtocounter{subfigure}{-1}
    
    \subfloat[Llama-2-7B]{\includegraphics[width=0.16\textwidth]{figs/fig-heter-llama2-7b.pdf}}
    \subfloat[Llama-2-13B]{\includegraphics[width=0.16\textwidth]{figs/fig-heter-llama2-13b.pdf}}
    \subfloat[Llama-2-70B]{\includegraphics[width=0.16\textwidth]{figs/fig-heter-llama2-70b.pdf}}
    \\

    \subfloat[Llama-3-8B]{\includegraphics[width=0.24\textwidth]{figs/fig-heter-llama3-8b.pdf}}
    \subfloat[Llama-3-70B]{\includegraphics[width=0.24\textwidth]{figs/fig-heter-llama3-70b.pdf}}
    \\

    \subfloat[GLM-67B]{\includegraphics[width=0.24\textwidth]{figs/fig-heter-glm-67b.pdf}}
    \subfloat[GLM-130B]{\includegraphics[width=0.24\textwidth]{figs/fig-heter-glm-130b.pdf}}
  \caption{
  For the heterogeneous GPU search scene, we compare expert-designed strategies's throughput with \sysname-searched strategies.
  The results prove the that \sysname achieves better throughput in heterogeneous scene.
  }
  \label{fig:exp:heter}
\end{figure}

% Please add the following required packages to your document preamble:
% \usepackage{graphicx}
\begin{table}[t]
\centering
\resizebox{0.5\textwidth}{!}{%
\begin{tabular}{c|cccc}
\hline
Model & H100 & H800 & A800 & Heter. \\ \hline\hline
Llama-2-7B & 10148287 & 9024716 & 3966756 & 5240609 \\
Llama-2-13B & 5721253 & 4937998 & 2187876 & 3040095 \\
Llama-2-70B & 1233850 & 1174362 & 458719 & 654206 \\
Llama-3-8B & 9167338 & 7610698 & 3586433 & 4660743 \\
Llama-3-70B & 1129568 & 1079507 & 425660 & 626050 \\
GLM-67B & 1288107 & 1218933 & 483384 & 699978 \\
GLM-130B & 508377 & 491088 & 202137 & 300193 \\ \hline\hline
\end{tabular}%
}
\caption{
We compare heterogeneous GPU with single-GPU search's optimal strategies' throughput.
The experiment is conducted with 1024 GPUs.
And the heterogeneous GPU setting is activated with A800 and H100.
}
\label{tab:exp:heter}
\end{table}

\sssec{Method}.
To evaluate \sysname's performance in heterogeneous GPU environments, we conducted a comprehensive comparison of \sysname-searched strategies and expert-designed strategies under heterogeneous GPU configurations. 
We use \sysname in the two GPU-heterogeneous environments with Nvidia H100 and A800 activated for search.
Also, we follow the design of \S\ref{sec:exp:expert}, we recruit six experts to craft a heterogeneous parallel strategy for each setting, and we picked the optimal one as the expert-designed strategy.
We offer 4 GPU number settings: 64, 256, 1024, and 4096.

Besides that, we also compared the heterogeneous GPU setting with single GPU setting in the same GPU number setting (1024).
We compare the throughput between the different settings (only A100, H100, H800, and heterogeneous settings)

\sssec{Results}.
As shown in Fig. \ref{fig:exp:heter}, our experiments reveal that \sysname consistently achieves higher throughput than expert-tuned configurations, particularly with larger models. \sysname’s approach dynamically balances data, tensor, and pipeline parallelism across heterogeneous GPUs, a task often challenging for manual tuning. This adaptability highlights the efficiency of automated strategies, especially in cloud-based or distributed environments where GPU types may vary. Overall, \sysname’s heterogeneous GPU search framework offers a scalable, cost-effective solution for optimizing model training in heterogeneous hardware contexts.

Table \ref{tab:exp:heter} shows the heterogeneous GPU setting compared with a single GPU setting.
Though a heterogeneous GPU setting strategy can not beat the performance of a single-GPU setting strategy, \sysname's searched strategy can nearly match with them.
\subsection{Mode-3: Evaluation Performance on Financial Cost}\label{sec:exp:finance}

%\sssec{Models and training workflows}.

\sssec{Search pools for GPU}. To comprehensively evaluate the financial cost performance of \sysname, we incorporate a variety of GPU types commonly used by major cloud service providers. Our search pools include the following GPU models: NVIDIA H100, A800 and H800.

These GPUs represent a range of performance capabilities and costs, providing a realistic and comprehensive basis for evaluating the financial efficiency of our system. By including these diverse GPU options, we can simulate the decision-making process of users who leverage cloud-based GPU resources, allowing us to optimize for both time and financial cost under various configurations.

\begin{figure}[t]
  \centering
    \subfloat[Per Throu. Llama-70B]{\includegraphics[width=0.24\textwidth]{figs/fig-money-per-Llama-2-70B.pdf}}
    \subfloat[Overall Throu. Llama-70B]{\includegraphics[width=0.24\textwidth]{figs/fig-money-all-Llama-2-70B.pdf}}
    \\
    \subfloat[Per Throu. GLM-67B]{\includegraphics[width=0.24\textwidth]{figs/fig-money-per-GLM-67B.pdf}}
    \subfloat[Overall Throu. GLM-67B]{\includegraphics[width=0.24\textwidth]{figs/fig-money-all-GLM-67B.pdf}}
    \\
    \subfloat[Per Throu. GLM-130B]{\includegraphics[width=0.24\textwidth]{figs/fig-money-per-GLM-130B.pdf}}
    \subfloat[Overall Throu. GLM-130B]{\includegraphics[width=0.24\textwidth]{figs/fig-money-all-GLM-130B.pdf}}
  \caption{
  We list the optimal line of \sysname.
  }
  \label{fig:money}
\end{figure}
\section{Results}
% For chunking, It could be condensed -> First, show the final results (line chunking vs ideal chunking) and the naive chunking stats. Then, describe the trends and example of sections not in any naive chunking strategy.
\subsection{(RQ1) Impact of tailored components}
\label{subsec: rq1_result}

\subsubsection{Hierarchy-aware Chunking}
\label{subsubsec: chunking_result}

From the defined metrics that are used to quantify the information loss on each type of chunking in \S\ref{subsubsec: chunk_setup}, we can conclude that a good chunking strategy should minimize \textit{Fail Chunk Ratio}, \textit{Fail Section Ratio} and \textit{Uncovered Section Ratio}.
%
Minimizing these metrics will reduce the information loss of some sections or parts of sections that are missing from the chunks.
%
Additionally, \textit{Sections/Chunk} and \textit{Chunks/Section} should be close to 1 in order for sections \emph{not to be} split into multiple chunks and retain atomicity within each chunk.

We evaluate multiple configurations of chunking strategies, chunk sizes, and overlaps as described in \S\ref{subsubsec: chunk_setup} and present the average metrics for each chunking strategy in Table~\ref{table: chunking_by_type}. 
%
It is observed that the chunking strategy most closely resembling the output of our hierarchy-aware chunking strategy is line-based chunking. 

However, across all strategies, approximately 30\% of sections are not referenced in any chunks, and at least 41.7\% of sections are not fully contained within a single chunk. 
%
Further analysis indicates that around 20\% of all sections cannot be fully covered in a single chunk under any naive chunking strategy due to their extended length. 
%
This necessitates the retrieval model to retrieve multiple chunks to provide sufficient context.

An example of such a section is Section 44 of the \textbf{Emergency Decree on Digital Asset Businesses, B.E. 2561}, which cannot be fully covered in a single chunk across any naive chunking strategy. 
%
This is attributed to its lengthy content and the presence of multiple subsections separated by newline characters, which are commonly used as delimiters in many naive chunking approaches.

% Should this be translated?
\begin{quote}
    \textbf{Section 44 of Emergency Decree on Digital Asset Businesses, B.E. 2561}
    
    It shall be presumed that the following persons, who exhibit behavior involving the buying or selling of digital tokens or engaging in forward contracts related to digital tokens in an unusual manner for themselves, are persons who possess or are aware of inside information as defined under Section 42:
    
    (1) Holders of digital tokens exceeding 5\% of the total tokens sold in each series by the issuer of digital tokens. This includes digital tokens held by their spouses, cohabiting partners in the manner of husband and wife, and their minor children.
    
    (2) Directors, executives, controlling persons, employees, or staff members of the affiliated entities of the digital token issuer who are in positions or roles responsible for, or with access to, inside information.
    
    (3) Ascendants, descendants, adoptive parents, or adopted children of persons specified under Section 43.
    
    (4) Siblings sharing the same father and mother, or the same father or mother as persons specified under Section 43.
    
    (5) Spouses or cohabiting partners in the manner of husband and wife of persons specified under Section 43 or individuals listed under (3) or (4).
    
    The term \enquote{affiliated entities of the digital token issuer} under (2) refers to parent companies, subsidiaries, or associated companies of the digital token issuer, as defined by the criteria set forth by the SEC Board's announcements.
\end{quote}

\begin{table}[!ht]
\centering

\resizebox{\textwidth}{!}{%
\renewcommand{\arraystretch}{1.3} % This increases the cell height by 1.5 times
\small % or \scriptsize
\begin{tabular}{@{}lccccc@{}}
\toprule
\textbf{Chunking Strategy} & \multicolumn{1}{l}{\textbf{Section/Chunk $\rightarrow$1}} & \multicolumn{1}{l}{\textbf{Chunk/Section $\rightarrow$1}} & \multicolumn{1}{l}{\textbf{Fail Chunk Ratio $\downarrow$}} & \multicolumn{1}{l}{\textbf{Fail Section Ratio $\downarrow$}} & \multicolumn{1}{l}{\textbf{Uncovered Section Ratio $\downarrow$}} \\ \midrule
\cellcolor{lightgray}Hierarchy-aware  & \cellcolor{lightgray}{1.000}   & \cellcolor{lightgray}{1.000}    & \cellcolor{lightgray}{0.000}  & \cellcolor{lightgray}{0.000}  & \cellcolor{lightgray}{0.000}                                       \\
Character  & 3.098                             & 1.710                              & 0.819                                & 0.675                                  & 0.397                                       \\

Line       & \textbf{1.689}                    & \textbf{1.234}                    & \textbf{0.658}                       & \textbf{0.417}                         & \textbf{0.294}                              \\

Recursive  & \underline{1.793}                       &\underline{1.27}                        & \underline{0.741}                          & \underline{0.504}                            & \underline{0.381}                                 \\ \bottomrule
\end{tabular}
}
\caption{Information loss comparison between hierarchy-aware chunking compared to other naive chunking strategies. Since hierarchy-aware chunking consistently parses into a single law section, it was treated as an upper bound because no information loss occurred.}
\label{table: chunking_by_type}
\end{table}

For the specific configuration of line chunking that produces chunks most similar to hierarchy-aware chunking, we fix the chunking strategy while varying the chunk overlap and chunk size parameters. 
%
Increasing the chunk size results in more text per chunk, leading to higher \textbf{Sections/Chunk} and \textbf{Chunks/Section} values while reducing the \textbf{Fail Chunk Ratio}, \textbf{Fail Section Ratio}, and \textbf{Uncovered Section Ratio}. 
%
Similarly, increasing the overlap effectively increases the chunk size, producing comparable effects to directly increasing the chunk size.
%
Based on these observations, we select the optimal configuration for naive chunking as line chunking with a chunk size of 553 characters and a chunk overlap of 50 characters. The detailed results for this configuration are displayed in Appendix~\ref{appendix: chunk_hyper}.

Finally, the metrics for the selected naive chunking configuration are compared against hierarchy-aware chunking in Table~\ref{table: chunking_compare_metric}.

\begin{table}[!ht]
\centering

\resizebox{\textwidth}{!}{%
\renewcommand{\arraystretch}{1.3} % This increases the cell height by 1.5 times
\small % or \scriptsize
\begin{tabular}{@{}lccccc@{}}
\toprule
\textbf{Chunking Strategy} & \multicolumn{1}{l}{\textbf{Section/Chunk $\rightarrow$1}} & \multicolumn{1}{l}{\textbf{Chunk/Section $\rightarrow$1}} & \multicolumn{1}{l}{\textbf{Fail Chunk Ratio $\downarrow$}} & \multicolumn{1}{l}{\textbf{Fail Section Ratio $\downarrow$}} & \multicolumn{1}{l}{\textbf{Uncovered Section Ratio $\downarrow$}} \\ \midrule
Hierarchy-aware chunking  & 1.000   & 1.000    & 0.000  & 0.000  & 0.000                                       \\
Line chunking (553 chunk size and 50 chunk overlap)  & 1.956                             & 1.180                              & 0.521                                & 0.323                                  & 0.156                                       \\ \bottomrule
\end{tabular}
}
\caption{Information loss comparison between perfect chunking strategy (hierarchy-aware chunking) and the best naive chunking setup.}
\label{table: chunking_compare_metric}
\end{table}

% Next, we also show results on our benchmark as well
Apart from the evaluation of chunking in isolation in terms of information loss, we also present the evaluation results on our benchmark in Table~\ref{table: chunk_e2e_main}.

\begin{table}[ht!]
\centering
\resizebox{\textwidth}{!}{%
\begin{tabular}{@{}lccccccc@{}}
    \toprule
    \textbf{Settings} & \textbf{Retriever Multi MRR ($\uparrow$)} & \textbf{Retriever Recall ($\uparrow$)} & \textbf{Coverage ($\uparrow$)} & \textbf{Contradiction ($\downarrow$)} & \textbf{E2E Recall ($\uparrow$)} & \textbf{E2E Precision ($\uparrow$)} & \textbf{E2E F1 ($\uparrow$)} \\
    \midrule
    Naïve Chunking            & 0.786 & 0.935 & 86.6 & \textbf{0.050} & 0.882 & 0.613 & 0.722 \\
    Hierarchy-aware Chunking  & \textbf{0.834} & \textbf{0.942} & \textbf{86.7} & 0.054 & \textbf{0.894} & \textbf{0.630} & \textbf{0.739} \\
    \bottomrule
\end{tabular}%
}
\caption{Effect of chunking configuration on E2E performance on NitiBench-CCL}
\label{table: chunk_e2e_main}
\end{table}


% From table~\ref{table: chunk_e2e_main}, the naive chunking strategy performs significantly worse than section-based chunking in terms of retrieval performance on both WCX and Tax Case datasets. This discrepancy likely stems from two factors: First, naive chunking discards chunks that do not fully contain a section. Second, it often splits single sections across multiple chunks, rendering these fragmented sections unusable for evaluation and practical application, as legal responses require complete sections.

% The E2E performance also agrees with the retrieval performance in that the section-based chunking significantly outperforms line chunking. Interestingly,with line chunking, end-to-end (E2E) recall (based on sections cited by the LLM) exceeds retrieval recall. This stems from two factors: 1) LLM sometimes cites unretrieved sections, either from its internal knowledge or through hallucination; and 2) Line chunking’s mapping of chunks to single, fully covered sections can lead to partial section coverage within a chunk, causing LLM to cite portions outside the mapped section. Consequently, line chunking’s E2E recall can surpass its retrieval recall.

From Table~\ref{table: chunk_e2e_main}, the naive chunking strategy performs worse than hierarchy-aware chunking in terms of retrieval performance. 
%
This discrepancy likely arises because naive chunks often contain content from multiple sections, introducing \enquote{noise} that can negatively impact the retrieval model's ranking of relevant documents.  

However, in terms of end-to-end (E2E) performance, the system using hierarchy-aware chunking only slightly outperforms the one using naive chunking. 
%
We suspect that this is because the LLM can effectively filter out the \enquote{noise} in the retrieved sections during answer generation. 
%
As a result, the coverage and contradiction scores are not significantly different between the two systems.
%
Nevertheless, there remains a discrepancy in the E2E citation score.  

In conclusion, \textbf{hierarchy-aware chunking achieves a slight but consistent advantage over the naive chunking strategy.}

\subsubsection{NitiLink}
\label{subsubsec: referencer_result}

The evaluation results of the experiment described in \S\ref{subsubsec: referencer_setup} are presented in Table~\ref{table: augmenter_e2e_main}. 
%
In the table, ``Ref Depth 1'' denotes a RAG system that incorporates a NitiLink component with a maximum depth of 1, while \enquote{No Ref} represents a RAG system without NitiLink. 
%
For the metrics, ``NitiLink'' indicates retrieval metrics calculated on the augmented context, which includes both the initially retrieved sections and the additional sections fetched by NitiLink.

% \begin{table}[!ht]
% \centering
% \caption{Effect of augmenter configuration on E2E performance}
% \renewcommand{\arraystretch}{1.5} % This increases the cell height by 1.5 times
% \label{table: augmenter_e2e_main}
% \begin{tabular}{@{}c|cc|cc@{}}
% \toprule
% Dataset                                  & \multicolumn{2}{c|}{Tax}        & \multicolumn{2}{c}{WCX}         \\ \midrule
% Setting                                  & Ref Depth 1    & No Ref         & Ref Depth 1    & No Ref         \\ \midrule
% Retriever MRR                            & 0.574          & 0.574          & 0.809          & 0.809          \\
% \multicolumn{1}{l|}{Retriever Multi MRR} & 0.333          & 0.333          & 0.809          & 0.809          \\
% Retriever Recall                         & 0.499          & 0.499          & 0.938          & 0.938          \\
% Referencer MRR                           & \textbf{0.582} & 0.574          & 0.800          & \textbf{0.809} \\
% Referencer Multi MRR                     & \textbf{0.345} & 0.333          & 0.800          & \textbf{0.809} \\
% Referencer Recall                        & \textbf{0.602} & 0.499          & \textbf{0.940} & 0.938          \\
% Coverage                                 & 45.0           & \textbf{50.0}  & \textbf{86.3}  & 85.2           \\
% Contradiction                            & 0.520          & \textbf{0.460} & \textbf{0.051} & 0.055          \\
% E2E Recall                               & \textbf{0.354} & 0.333          & \textbf{0.885} & 0.880          \\
% E2E Precision                            & 0.630          & \textbf{0.64}  & 0.579          & \textbf{0.601} \\
% E2E F1                                   & \textbf{0.453} & 0.438          & 0.700          & \textbf{0.714} \\ \bottomrule
% \end{tabular}%
% \end{table}
\begin{table}[!ht]
\centering
\renewcommand{\arraystretch}{1.3}
\newcommand{\gray}{\cellcolor{gray!15}}
\newcommand{\pos}[1]{\textcolor{darkgreen}{(#1\%)}}
\newcommand{\negv}[1]{\textcolor{red}{(#1\%)}}

\begin{tabular}{lcccccc}
\toprule
\multirow{2}{*}{\textbf{Metric}} & \multicolumn{3}{c}{\textbf{NitiBench-CCL}} & \multicolumn{3}{c}{\textbf{NitiBench-Tax}} \\ 
 & \textbf{No Ref} & \gray \textbf{Ref Depth 1} & $\Delta$ & \textbf{No Ref} & \gray \textbf{Ref Depth 1} & $\Delta$ \\ 
\midrule
\multicolumn{7}{c}{\textbf{Retriever Metrics}} \\ 
\midrule
MRR ($\uparrow$)       & \multicolumn{2}{c}{0.809} & -  & \multicolumn{2}{c}{0.574} & -  \\
Multi MRR ($\uparrow$) & \multicolumn{2}{c}{0.809} & -  & \multicolumn{2}{c}{0.333} & -  \\
Recall ($\uparrow$)    & \multicolumn{2}{c}{0.938} & -  & \multicolumn{2}{c}{0.499} & -  \\
\midrule
\multicolumn{7}{c}{\textbf{NitiLink Metrics}} \\ 
\midrule
MRR ($\uparrow$)             & 0.809  & \gray 0.800  & \negv{-1.11}  & 0.574  & \gray \textbf{0.582}  & \pos{+1.39}  \\
Multi MRR ($\uparrow$)       & 0.809  & \gray 0.800  & \negv{-1.11}  & 0.333  & \gray \textbf{0.345}  & \pos{+3.60}  \\
Recall ($\uparrow$)          & 0.938  & \gray \textbf{0.940}  & \pos{+0.21}  & 0.499  & \gray \textbf{0.602}  & \pos{+20.6}  \\
Coverage ($\uparrow$)        & 85.2   & \gray \textbf{86.3}  & \pos{+1.29}  & \textbf{50.0}   & \gray 45.0   & \negv{-10.0}  \\
Contradiction ($\downarrow$) & 0.055  & \gray \textbf{0.051}  & \pos{-7.27}  & \textbf{0.460}  & \gray 0.520  & \negv{+13.0}  \\
E2E Recall ($\uparrow$)      & 0.880  & \gray \textbf{0.885}  & \pos{+0.57}  & 0.333  & \gray \textbf{0.354}  & \pos{+6.31}  \\
E2E Precision ($\uparrow$)   & 0.601  & \gray 0.579  & \negv{-3.66}  & \textbf{0.640}  & \gray 0.630  & \negv{-1.56}  \\
E2E F1 ($\uparrow$)          & \textbf{0.714}  & \gray 0.700  & \negv{-1.96}  & 0.438  & \gray \textbf{0.453}  & \pos{+3.42}  \\
\bottomrule
\end{tabular}
\caption{Effect of NitiLink augmenter configuration on E2E performance. The $\Delta$ column shows the relative percentage change compared to ``No Ref'', with dark green indicating improvement and red indicating degradation.}
\label{table: augmenter_e2e_main}
\end{table}




The results from Table~\ref{table: augmenter_e2e_main} show that there is no clear significant advantage when employing NitiLink in a RAG system. 
%
The results also highlight the differing impacts of incorporating NitiLink across datasets.

\textbf{NitiBench-Tax} For this dataset, we can clearly see that the recall was substantially improved from 0.499 to 0.602.
%
The improvement of recall suggested that NitiLink does provide an additional correct law section to the retrieved documents. 
%
Despite significant improvement over recall, we only see marginal improvements over MRR and Multi MRR.
%
Since we're using a depth-first augmented strategy (see \S\ref{subsubsec: referencer_setup}), this suggested that the document that cited more positives by NitiLink is ranked at the bottom of the retrieved documents.
%
Surprisingly, despite a major improvement in recall, some E2E metrics declined.
%
This might be due to NitiBench-Tax's query complexity, which often demands advanced reasoning capabilities that the LLM, even with the correct documents, struggles to provide. 
%
Another reason that might affect the performance decline even with more relevant documents provided to the LLM is the longer context that the LLM needs to process due to the higher amount of content added by NitiLink.

\textbf{NitiBench-CCL} For NitiBench-CCL showed no significant change in retrieval metrics and most E2E metrics.
%
Incorporating NitiLink yields very little recall gain, while MRR is slightly lower. 
%
This means that NitiLink often pushed the positive lower in the ranking as we're using a depth-first augmentation strategy (see \S\ref{subsubsec: referencer_setup}).
%
We highlight several factors that might contribute to the limited recall gain in this dataset:
\begin{enumerate}
    \item \textbf{Binary recall nature:} NitiBench-CCL queries typically involve a single relevant law, making recall binary and thus harder to improve.
    %
    \item \textbf{Simplicity of NitiBench-CCL queries:} Simple, non-specific NitiBench-CCL queries often rely on many relevant law sections that are similar semantically rather than hierarchically. 
    %
    This is opposed to NitiBench-Tax, where referenced law sections are necessary for legal reasoning.
    %
    This simplicity stems from the fact that the dataset was created by letting the annotator craft a question based on a given law section.
    %
    This explicitly provides bias toward the dataset since the question was created without a referenced law section.
    %
    \item \textbf{Hierarchical limiation:} The hierarchical structure itself presents challenges. 
    %
    Although NitiLink augmented the retrieved law section mentioned in the retrieved document (children reference), it lacks a law section that references retrieved law sections (parent reference).  
    %
    Thus, this version of NitiLink that lacks the ability to fetch parent law sections could result in a suboptimal performance.
    %
    %\item The hierarchical structure itself presents challenges. Many sections in the Revenue Code lack hierarchical connections (41\% have no children, 45\% no parents, and 24\% neither), limiting the referencer's effectiveness. Furthermore, some queries (e.g., those about criminal penalties) require retrieving laws that reference multiple others. The current referencer, retrieving only children of initially retrieved laws, struggles with these, as it cannot retrieve the parent law (which might be the ground truth).
\end{enumerate}

Despite the limited recall gains in NitiBench-CCL, we can see that there's a slight improvement in coverage score as well as recall.
%
This suggests that even small recall improvements can enhance the LLM's ability to answer NitiBench-CCL queries effectively.

\subsection{(RQ2) Impact of Retriever and LLM}
\label{subsec: rq2_result}

\subsubsection{Retriever}
\label{subsubsec: retriever_result}

\textbf{NitiBench-CCL} 
%
Table~\ref{table: retrieval_wangchan} presents the retrieval performance of 8 models as described in \S\ref{subsubsec: retriever_setup} on NitiBench-CCL with hierarchy-aware chunking. 
%
Because each query has only one positive label (as mentioned in \S\ref{subsubsec: wcx_dataset}), the multi-hit-rate and multi-MRR metrics are equivalent to their single-label counterparts. This also applies to recall and hit rate as well. Thus, for this dataset, we only showed Recall (Recall@K) and MRR (MRR@K) since other metrics are considered redundant.

The best-performing model is the human-reranked fine-tuned BGE-M3, achieving an MRR@5 of 0.805. 
%
Close behind are the auto-reranked fine-tuned BGE-M3 (0.800 MRR@5) and the base BGE-M3 (0.579 MRR@5). 
%
BGE-M3's strong performance is likely due to its use of three embedding types for relevance calculation, further enhanced by fine-tuning on in-domain data. 
%
Notably, the auto-reranked version nearly matches human-reranked performance without requiring costly human annotation. 
%

\textbf{Based on these findings, we recommend a cost-effective in-domain adaptation pipeline, notably Auto-Finetuned BGE-M3, that uses a strong LLM to generate synthetic training pairs, retrieves top-k passages with BGE-M3, and then applies a BGE-M3 Reranker. 
%
As shown in the results, this approach closely matches human-reranked performance while significantly saving annotation costs in an in-domain setup.
}

The commercially available Cohere embedding model ranks just below the top-performing BGE-M3 models and is followed by the ColBERT-based and dense embedding models, JINA ColBERT v2 and JINA embeddings v3, respectively. 
%
Among the tested retrievers, NV-Embed v1 shows the lowest performance among non-baseline models (0.713 MRR@5), likely due to its decoder-based architecture and reliance on prefix instruction prompts. 
%
Overall, retrieval performance on NitiBench-CCL is strong, with most models delivering comparable results, except for NV-Embed v1 and BM25. 
%
However, despite this strong performance, a gap between hit-rate and MRR when $k=\{5,10\}$ indicates that \textbf{while relevant documents are frequently retrieved, they are not consistently ranked first, potentially impacting end-to-end performance.}

\begin{table}[!ht]
\centering

% \renewcommand{\arraystretch}{1.5}
\small
\begin{tabular}{@{}clcc@{}}
\toprule
\textbf{Top-K} & \textbf{Model} & \textbf{HR/Recall@k} & \textbf{MRR@k} \\ \midrule
\multirow{8}{*}{k=1} 
  & BM25                   & .481 & .481 \\
  & JINA V2                & .681 & .681 \\
  & JINA V3                & .587 & .587 \\
  & NV-Embed V1            & .492 & .492 \\
  & BGE-M3                 & .700 & .700 \\
  & Human-Finetuned BGE-M3 & \textbf{.735} & \textbf{.735} \\
  & Auto-Finetuned BGE-M3  & \underline{.731} & \underline{.731} \\
  & Cohere                 & .676 & .676 \\ \midrule
\multirow{8}{*}{k=5} 
  & BM25                   & .658 & .548 \\
  & JINA V2                & .852 & .750 \\
  & JINA V3                & .821 & .681 \\
  & NV-Embed V1            & .713 & .579 \\
  & BGE-M3                 & .880 & .773 \\
  & Human-Finetuned BGE-M3 & \textbf{.906} & \textbf{.805} \\
  & Auto-Finetuned BGE-M3  & \underline{.900} & \underline{.800} \\
  & Cohere                 & .870 & .754 \\ \midrule
\multirow{8}{*}{k=10} 
  & BM25                   & .715 & .556 \\
  & JINA V2                & .889 & .755 \\
  & JINA V3                & .875 & .688 \\
  & NV-Embed V1            & .776 & .587 \\
  & BGE-M3                 & .919 & .778 \\
  & Human-Finetuned BGE-M3 & \textbf{.938} & \textbf{.809} \\
  & Auto-Finetuned BGE-M3  & \underline{.934} & \underline{.804} \\
  & Cohere                 & .912 & .760 \\ \bottomrule
\end{tabular}
\caption{Retrieval Evaluation Result on NitiBench-CCL with hierarchy-aware chunking. Since the test split contains a single positive (as mentioned in \S \ref{subsubsec: wcx_dataset}), we collapsed metrics that are duplicated, such as HitRate (HR)/ Recall/ Multi-HitRate and MultiMRR / MRR.}
\label{table: retrieval_wangchan}
\end{table}


\textbf{NitiBench-Tax} Table~\ref{table: retrieval_tax} presents the retrieval performance of various models on NitiBench-Tax using hierarchy-aware chunking. 
%
Unlike NitiBench-CCL, this dataset includes multi-label queries, resulting in different values for single-label and multi-label metrics.

Overall performance is significantly lower on this dataset compared to NitiBench-CCL, likely due to the considerably longer and more nuanced queries in NitiBench-Tax.
%
JINA v3 and BGE-M3 (base, auto-fine-tuned, and human-finetuned) consistently perform among the top, achieving Multi-MRR@10 scores of 0.311, 0.354, 0.345, and 0.333, respectively. 
%
Conversely, JINA v2 and NV-Embed v1 consistently underperform compared to the baseline, potentially because NitiBench-Tax is out-of-distribution relative to their training data, considering the complexity of the query.
%
This is particularly evident with JINA v2, whose Multi-MRR@10 drops dramatically from 0.750 on NitiBench-CCL to 0.091 on NitiBench-Tax.

Similarly, the Human-Finetuned BGE-M3 variants are often outperformed by the base BGE-M3, suggesting different data distributions between NitiBench-CCL and NitiBench-Tax, hindering cross-dataset generalization. 
%
While some models achieve reasonable single-label hit rates, multi-label hit-rate performance is poor across all models. 
%
This, combined with low recall and significantly lower multi-label MRR compared to single-label MRR, indicates that while models can often retrieve some relevant documents, they struggle to retrieve all relevant documents for a given query. 
%
This limitation is critical, as comprehensive legal responses require consideration of all relevant legal sections. 
%
\textbf{Although the proposed pipeline (Human-Finetuned BGE-M3) performs strongly on in-domain data (as seen with NitiBench-CCL), these Tax Case results underscore the critical need for sufficiently diverse in-domain training data, since a narrow domain distribution can lead to inconsistent or contradictory outcomes in real-world settings.}

Despite its lower overall performance, NitiBench-Tax benefits more from increasing the number of top-k retrieved documents compared to NitiBench-CCL. 
%
Its hit rate and recall improve at a faster rate as more documents are retrieved compared to NitiBench-CCL.

\begin{table}[!ht]
\centering
\small
\begin{tabular}{@{}clccccc@{}}
\toprule
\textbf{Top-K} & \textbf{Model}                  & \textbf{HR@k}          & \textbf{Multi HR@k}    & \textbf{Recall@k}      & \textbf{MRR@k}         & \textbf{Multi MRR@k}   \\ \midrule
k=1   & BM25                   & .220          & .080          & .118          & .220          & .118          \\
      & JINA V2                & .140          & .040          & .068          & .140          & .068          \\
      & JINA V3                & .400          & .100          & .203          & .400          & .203          \\
      & NV-Embed V1            & .100          & .020          & .035          & .100          & .035          \\
      & BGE-M3                 & \underline{.500}    & \underline{.140}    & \underline{.269}    & \underline{.500}    & \underline{.269}    \\
      & Human-Finetuned BGE-M3 & .480          & \underline{.140}    & .255          & .480          & .255          \\
      & Auto-Finetuned BGE-M3  & \textbf{.520} & \textbf{.160} & \textbf{.281} & \textbf{.520} & \textbf{.281} \\
      & Cohere                 & .340          & .100          & .179          & .340          & .179          \\ \midrule
k=5   & BM25                   & .480          & .120          & .254          & .318          & .171          \\
      & JINA V2                & .200          & .080          & .114          & .165          & .085          \\
      & JINA V3                & \underline{.720}    & \textbf{.260} & \textbf{.448} & .508          & .297          \\
      & NV-Embed V1            & .200          & .020          & .081          & .126          & .050          \\
      & BGE-M3                 & \underline{.720}    & \underline{.240}    & \underline{.435}    & \underline{.580}    & \textbf{.337} \\
      & Human-Finetuned BGE-M3 & \textbf{.740} & .220          & .411          & .565          & .320          \\
      & Auto-Finetuned BGE-M3  & .700          & .200          & .382          & \textbf{.587} & \underline{.329}    \\
      & Cohere                 & .620          & .200          & .363          & .447          & .256          \\ \midrule
k=10  & BM25                   & .540          & .160          & .320          & .327          & .183          \\
      & JINA V2                & .240          & .100          & .147          & .171          & .091          \\
      & JINA V3                & \textbf{.840} & \underline{.340}    & \underline{.549}    & .524          & .311          \\
      & NV-Embed V1            & .220          & .040          & .097          & .128          & .052          \\
      & BGE-M3                 & \underline{.820}    & \textbf{.360} & \textbf{.555} & \underline{.593}    & \textbf{.354} \\
      & Human-Finetuned BGE-M3 & .800          & .280          & .499          & .574          & .333          \\
      & Auto-Finetuned BGE-M3  & .780          & .260          & .483          & \textbf{.600} & \underline{.345}    \\
      & Cohere                 & .680          & .200          & .414          & .454          & .263          \\ \bottomrule
\end{tabular}
\caption{Retrieval Evaluation Result on NitiBench-Tax with hierarchy-aware chunking. This split contains multiple positives per question.}
\label{table: retrieval_tax}
\end{table}



\subsubsection{LLM}
\label{subsubsec: llm_result}

The evaluation results of the experiments described in \S\ref{subsubsec: llm_setup} are presented in Table~\ref{table: llm_e2e_main_wcx} for NitiBench-CCL and Table~\ref{table: llm_e2e_main_tax} for NitiBench-Tax. 
%
Since experiments in \S\ref{subsubsec: referencer_result} do not provide conclusive results on whether the inclusion of NitiLink is necessary, we also vary the inclusion of NitiLink in this experiment as well.


\begin{table}[!ht]
\centering
\renewcommand{\arraystretch}{1.2}
\resizebox{\textwidth}{!}{%
\begin{tabular}{@{}lcccccccc@{}}
\toprule
\textbf{Setting} & \textbf{NitiLink} & \textbf{Retriever MRR ($\uparrow$)} & \textbf{Retriever Recall ($\uparrow$)} & \textbf{E2E Recall ($\uparrow$)} & \textbf{E2E Precision ($\uparrow$)} & \textbf{E2E F1 ($\uparrow$)} & \textbf{Coverage ($\uparrow$)} & \textbf{Contradiction ($\downarrow$)} \\ \midrule
\multirow{3}{*}{\texttt{gpt-4o-2024-08-06}} 
& No Ref      & \multirow{3}{*}{0.809} & \multirow{3}{*}{0.938} & 0.880  & \textbf{0.601}  & \textbf{0.714}  & 85.2  & 0.055  \\
& \cellcolor{lightgray}Ref Depth 1 &                      &                     & \cellcolor{lightgray}0.885  & \cellcolor{lightgray}\underline{0.579}  & \cellcolor{lightgray}\underline{0.700}  & \cellcolor{lightgray}86.3  & \cellcolor{lightgray}0.051  \\
& $\Delta$    &                      &                     & \textcolor{darkgreen}{+0.6\%}  & \textcolor{red}{-3.7\%}  & \textcolor{red}{-2.0\%}  & \textcolor{darkgreen}{+1.3\%}  & \textcolor{darkgreen}{-7.3\%}  \\ \midrule
\multirow{3}{*}{\texttt{gemini-1.5-pro-002}} 
& No Ref      & \multirow{3}{*}{0.809} & \multirow{3}{*}{0.938} & 0.892  & 0.512  & 0.651  & 86.5  & 0.048  \\
& \cellcolor{lightgray}Ref Depth 1 &                      &                     & \cellcolor{lightgray}\underline{0.895}  & \cellcolor{lightgray}0.491  & \cellcolor{lightgray}0.634  & \cellcolor{lightgray}87.3  & \cellcolor{lightgray}\underline{0.042}  \\
& $\Delta$    &                      &                     & \textcolor{darkgreen}{+0.3\%}  & \textcolor{red}{-4.1\%}  & \textcolor{red}{-2.6\%}  & \textcolor{darkgreen}{+0.9\%}  & \textcolor{darkgreen}{-12.5\%} \\ \midrule
\multirow{3}{*}{\texttt{claude-3-5-sonnet-20240620}} 
& No Ref      & \multirow{3}{*}{0.809} & \multirow{3}{*}{0.938} & \textbf{0.901} & 0.444  & 0.595  & \textbf{89.7} & \textbf{0.040}  \\ 
& \cellcolor{lightgray}Ref Depth 1 &                      &                     & \cellcolor{lightgray}0.894  & \cellcolor{lightgray}0.443  & \cellcolor{lightgray}0.592  & \cellcolor{lightgray}\underline{89.5} & \cellcolor{lightgray}0.044  \\
& $\Delta$    &                      &                     & \textcolor{red}{-0.8\%} & \textcolor{red}{-0.2\%} & \textcolor{red}{-0.5\%}  & \textcolor{red}{-0.2\%}  & \textcolor{red}{+10.0\%}  \\ \midrule
\multirow{3}{*}{\texttt{typhoon-v2-70b-instruct}} 
& No Ref      & \multirow{3}{*}{0.809} & \multirow{3}{*}{0.938} & 0.862  & 0.537  & 0.662  & 81.2  & 0.076  \\
& \cellcolor{lightgray}Ref Depth 1 &                      &                     & \cellcolor{lightgray}0.845  & \cellcolor{lightgray}0.573  & \cellcolor{lightgray}0.683  & \cellcolor{lightgray}79.9  & \cellcolor{lightgray}0.080  \\
& $\Delta$    &                      &                     & \textcolor{red}{-2.0\%} & \textcolor{darkgreen}{+6.7\%} & \textcolor{darkgreen}{+3.2\%}  & \textcolor{red}{-1.6\%}  & \textcolor{red}{+5.3\%}  \\ \midrule
\multirow{3}{*}{\texttt{typhoon-v2-8b-instruct}} 
& No Ref      & \multirow{3}{*}{0.809} & \multirow{3}{*}{0.938} & 0.775  & 0.387  & 0.516  & 70.8  & 0.134  \\
& \cellcolor{lightgray}Ref Depth 1 &                      &                     & \cellcolor{lightgray}0.718  & \cellcolor{lightgray}0.385  & \cellcolor{lightgray}0.501  & \cellcolor{lightgray}68.5  & \cellcolor{lightgray}0.145  \\
& $\Delta$    &                      &                     & \textcolor{red}{-7.4\%} & \textcolor{red}{-0.5\%}  & \textcolor{red}{-2.9\%}  & \textcolor{red}{-3.3\%}  & \textcolor{red}{+8.2\%}  \\ \bottomrule
\end{tabular}
}
\caption{Effect of LLM configuration on E2E performance on NitiBench-CCL. $\Delta$ values are computed relative to No Ref and normalized to percentage change.}
\label{table: llm_e2e_main_wcx}
\end{table}


\begin{table}[!ht]
\centering
\renewcommand{\arraystretch}{1.2}
\resizebox{\textwidth}{!}{%
\begin{tabular}{@{}lccccccccc@{}}
\toprule
{\textbf{Setting}} & {\textbf{NitiLink}} 
& {\textbf{Retriever MRR ($\uparrow$)}} 
& {\textbf{Retriever Multi MRR ($\uparrow$)}} 
& {\textbf{Retriever Recall ($\uparrow$)}} 
& {\textbf{E2E Recall ($\uparrow$)}} 
& {\textbf{E2E Precision ($\uparrow$)}} 
& {\textbf{E2E F1 ($\uparrow$)}} 
& {\textbf{Coverage ($\uparrow$)}} 
& {\textbf{Contradiction ($\downarrow$)}} \\ 
\midrule

%---------------- gpt-4o-2024-08-06 ----------------
\multirow{3}{*}{\texttt{gpt-4o-2024-08-06}} 
& No Ref            
  & \multirow{3}{*}{0.574} 
  & \multirow{3}{*}{0.333}  
  & \multirow{3}{*}{0.499}
  & 0.333 
  & \underline{0.640}
  & 0.438 
  & 50.0  
  & \underline{0.46} \\
& \cellcolor{lightgray}Ref Depth 1  
  & 
  & 
  & 
  & \cellcolor{lightgray}0.354 
  & \cellcolor{lightgray}0.630 
  & \cellcolor{lightgray}0.453 
  & \cellcolor{lightgray}45.0 
  & \cellcolor{lightgray}0.52 \\
& $\Delta$           
  & 
  & 
  & 
  & \textcolor{darkgreen}{+6.3\%}
  & \textcolor{red}{-1.6\%}
  & \textcolor{darkgreen}{+3.4\%}
  & \textcolor{red}{-10.0\%}
  & \textcolor{red}{+13.0\%} \\
\midrule

%---------------- gemini-1.5-pro-002 ----------------
\multirow{3}{*}{\texttt{gemini-1.5-pro-002}} 
& No Ref            
  & \multirow{3}{*}{0.574} & \multirow{3}{*}{0.333} & \multirow{3}{*}{0.499}
  & 0.361 & 0.308 & 0.332 
  & 44.0  & 0.48 \\
& \cellcolor{lightgray}Ref Depth 1  
  & 
  & 
  & 
  & \cellcolor{lightgray}0.354 
  & \cellcolor{lightgray}0.347 
  & \cellcolor{lightgray}0.351 
  & \cellcolor{lightgray}45.0 
  & \cellcolor{lightgray}0.48 \\
& $\Delta$           
  & 
  & 
  & 
  & \textcolor{red}{-1.9\%}
  & \textcolor{darkgreen}{+12.7\%}
  & \textcolor{darkgreen}{+5.7\%}
  & \textcolor{darkgreen}{+2.3\%}
  & \textcolor{black}{+0.0\%} \\
\midrule

%---------------- claude-3-5-sonnet-20240620 ----------------
\multirow{3}{*}{\texttt{claude-3-5-sonnet-20240620}} 
& No Ref            
  & \multirow{3}{*}{0.574} & \multirow{3}{*}{0.333} & \multirow{3}{*}{0.499 }
  & \underline{0.389} & 0.554 & \underline{0.457}
  & \underline{51.0}  & \textbf{0.44} \\
& \cellcolor{lightgray}Ref Depth 1  
  & 
  & 
  & 
  & \cellcolor{lightgray}\textbf{0.417} 
  & \cellcolor{lightgray}0.577 
  & \cellcolor{lightgray}\textbf{0.484} 
  & \cellcolor{lightgray}49.0 
  & \cellcolor{lightgray}0.56 \\
& $\Delta$           
  & 
  & 
  & 
  & \textcolor{darkgreen}{+7.2\%}
  & \textcolor{darkgreen}{+4.2\%}
  & \textcolor{darkgreen}{+5.9\%}
  & \textcolor{red}{-3.9\%}
  & \textcolor{red}{+27.3\%} \\
\midrule

%---------------- typhoon-v2-70b-instruct ----------------
\multirow{3}{*}{\texttt{typhoon-v2-70b-instruct}} 
& No Ref            
  & \multirow{3}{*}{0.574} & \multirow{3}{*}{0.333} & \multirow{3}{*}{0.499}
  & 0.326 & \textbf{0.662} & 0.437 
  & 42.0  & 0.58 \\
& \cellcolor{lightgray}Ref Depth 1  
  & 
  & 
  &  
  & \cellcolor{lightgray}0.333 
  & \cellcolor{lightgray}0.453 
  & \cellcolor{lightgray}0.384 
  & \cellcolor{lightgray}\textbf{54.0 }
  & \cellcolor{lightgray}\underline{0.46} \\
& $\Delta$           
  & 
  & 
  & 
  & \textcolor{darkgreen}{+2.1\%}
  & \textcolor{red}{-31.6\%}
  & \textcolor{red}{-12.1\%}
  & \textcolor{darkgreen}{+28.6\%}
  & \textcolor{darkgreen}{-20.7\%} \\
\midrule

%---------------- typhoon-v2-8b-instruct ----------------
\multirow{3}{*}{\texttt{typhoon-v2-8b-instruct}} 
& No Ref            
  & \multirow{3}{*}{0.574} & \multirow{3}{*}{0.333} & \multirow{3}{*}{0.499 }
  & 0.278 & 0.471 & 0.349 
  & 37.0  & 0.60 \\
& \cellcolor{lightgray}Ref Depth 1  
  & 
  & 
  & 
  & \cellcolor{lightgray}0.319 
  & \cellcolor{lightgray}0.561 
  & \cellcolor{lightgray}0.407 
  & \cellcolor{lightgray}35.0 
  & \cellcolor{lightgray}0.54 \\
& $\Delta$           
  & 
  & 
  & 
  & \textcolor{darkgreen}{+14.7\%}
  & \textcolor{darkgreen}{+19.1\%}
  & \textcolor{darkgreen}{+16.6\%}
  & \textcolor{red}{-5.4\%}
  & \textcolor{darkgreen}{-10.0\%} \\
\bottomrule
\end{tabular}
}
\caption{Effect of LLM configuration on E2E performance on NitiBench-Tax. 
\(\Delta\) values are computed relative to No Ref and normalized to percentage change.}
\label{table: llm_e2e_main_tax}
\end{table}


\textbf{NitiBench-CCL} On NitiBench-CCL, \texttt{claude-3.5-sonnet} excels in Coverage, Contradiction, and E2E Recall. 
%
However, the typhoon family of models struggles to match the performance of the closed-source models in this broader Thai legal QA domain. 
%
As seen with NitiBench-Tax, both \texttt{claude-3.5-sonnet} and \texttt{gemini-1.5-pro-002} exhibit low E2E Precision, leading to lower F1 scores compared to \texttt{gpt-4o}. 
%
This underscores a trade-off between recall and precision, particularly for \texttt{claude-3.5-sonnet} and \texttt{gemini-1.5-pro-002}. 
%
The causes of this precision drop are further analyzed in a later section.

\textbf{NitiBench-Tax} The results on NitiBench-Tax demonstrate that \texttt{claude-3.5-sonnet} also achieves top-2 performance across most end-to-end metrics. 
%
Interestingly, \texttt{typhoon-v2-70b-instruct}, an open-sourced model, delivers comparable results and outperforms others on NitiBench-Tax with the highest coverage score and E2E precision. 
%
However, its smaller variant, \texttt{typhoon-v2-8b-instruct}, ranks the lowest among the tested models.f
%`
Despite its limited parameter size, it manages to avoid falling significantly behind, showcasing a reasonable performance given its constraints.

Notably, both \texttt{claude-3.5-sonnet} and \texttt{gemini-1.5-pro-002} exhibit considerably lower E2E precision compared to \texttt{gpt-4o} and \texttt{typhoon-v2-70b-instruct}. This compromises their suitability for precision-critical applications, even though they excel in other areas. Additionally, the NitiLink module fails to consistently enhance performance, with mixed results indicating no definitive advantage in its current configuration.

\subsubsection{E2E results using best setups}
\label{subsubsec: e2e_best_result}

The results from experiments conducted under the four settings described in \S\ref{subsubsec: e2e_best_setup} are presented in Table~\ref{table: main_exp_full}. 
%
Both the Proposed RAG and Naive RAG settings utilize Human-Finetuned BGE-M3 as the retriever, as it demonstrated the best performance in previous experiments (Section~\ref{subsubsec: retriever_result}). 
%
Similarly, Claude 3.5 Sonnet is chosen as the generator based on its superior results in \S\ref{subsubsec: llm_result}. 
%
NitiLink is excluded in both settings due to its inconclusive impact on E2E performance (Section~\ref{subsubsec: referencer_result} and~\ref{subsubsec: llm_result}). 
%
We opted for this choice instead of the opposite because omitting NitiLink yields similar performance while reducing API costs. 
%
The primary distinction between Naive RAG and Proposed RAG lies in their chunking strategies: Naive RAG employs a naive chunking approach, whereas Proposed RAG utilizes hierarchy-aware chunking.

\begin{table}[!ht]
\centering
\renewcommand{\arraystretch}{1.2} % Increase cell height
\resizebox{\textwidth}{!}{%
\begin{tabular}{@{}lcccccccc@{}}
\toprule
{\textbf{Setting}} & \textbf{{Retriever MRR} ($\uparrow$)} & \textbf{{Retriever Multi-MRR} ($\uparrow$)} & \textbf{{Retriever Recall} ($\uparrow$)} & \textbf{{Coverage} ($\uparrow$)} & \textbf{{Contradiction} ($\downarrow$)} & \textbf{{E2E Recall} ($\uparrow$)} & \textbf{{E2E Precision} ($\uparrow$)} & \textbf{{E2E F1} ($\uparrow$)} \\ 
\midrule
\multicolumn{9}{c}{\textbf{NitiBench-CCL}} \\ \midrule
Parametric     
 & -     
 & -     
 & -     
 & 60.3     
 & 0.199   
 & 0.188   
 & 0.141   
 & 0.161  
\\
Na\"ive RAG    
 & 0.120 
 & 0.048 
 & 0.062 
 & 77.3     
 & 0.097   
 & 0.745   
 & 0.370   
 & 0.495  
\\
Proposed RAG   
 & 0.809 
 & 0.809 
 & 0.938 
 & \textbf{89.7}     
 & \textbf{0.040}   
 & \textbf{0.901}   
 & \textbf{0.444}   
 & \textbf{0.595}  
\\
\rowcolor{gray!15}
Golden Context 
 & 1.0   
 & 1.0   
 & 1.0   
 & 93.4     
 & 0.034   
 & 0.999   
 & 1.000   
 & 1.000  
\\ 
\midrule
\multicolumn{9}{c}{\textbf{NitiBench-Tax}} \\ \midrule
Parametric     
 & -     
 & -     
 & -     
 & 46.0     
 & 0.480   
 & \textbf{0.458}   
 & \textbf{0.629}   
 & \textbf{0.530}  
\\
Na\"ive RAG    
 & 0.120 
 & 0.048 
 & 0.062 
 & 50.0     
 & 0.460   
 & 0.306   
 & 0.463   
 & 0.368  
\\
Proposed RAG   
 & 0.574 
 & 0.333 
 & 0.499 
 & \textbf{51.0}     
 & \textbf{0.440}   
 & {0.389}   
 & {0.554}   
 & {0.457}  
\\
\rowcolor{gray!15}
Golden Context 
 & 1.0   
 & 1.0   
 & 1.0   
 & 52.0     
 & 0.460   
 & 0.694   
 & 1.000   
 & 0.820  
\\ 
\bottomrule
\end{tabular}
}
\caption{E2E Experiment results on NitiBench-CCL and NitiBench-Tax comparing various RAG setups on Human-Finetuned BGE-M3 retriever and \texttt{claude-3-5-sonnet-20240620} as a LLM.}
\label{table: main_exp_full}
\end{table}



\textbf{NitiBench-Tax} On NitiBench-Tax, the four main settings perform similarly, except for the parametric setting's slightly lower coverage and higher contradiction. Two key observations emerge:

First, the parametric setting achieves the second-highest E2E recall and precision despite lacking a retriever. 
%
To investigate this further, we inspected the cited law section generated by LLM. 
%
Surprisingly, we found that out of 105 law sections cited from LLM parametric knowledge, 58 of them aren't even retrieved by the best retriever. 
%
Among those 58 cited documents, 26 were in the correct law section. 
%
In contrast, only 5 of 101 sections cited by the proposed RAG system are \emph{not} retrieved. 
%
This indicates that retriever performance significantly constrains RAG systems, especially with complex queries like those in NitiBench-Tax. 
%
RAG system generators seem discouraged from using internal knowledge, which might sometimes provide better answers. 
%
Furthermore, the substantial disparity between retriever and E2E recall shows that the LLM often underutilizes relevant retrieved sections, particularly those containing primarily terminology (as discussed in \S\ref{subsec: retriever_re_error_analysis_tax}).
%
% We suspect that the high performance of parametric knowledge could be due to Thai law data or the case data from the revenue department website might be contaminated in the pretraining data since it's publicly available.
%
% However, we didn't further investigate this due to time constraint and we leave this to future works.

Second, Table~\ref{table: main_exp_full} (NitiBench-Tax) shows no clear relationship between E2E citation scores and coverage/contradiction. 
%
This suggests the LLM struggles to apply cited sections correctly in its reasoning, leading to incorrect or erroneously reasoned answers. 
%
This resembles the issue in \S\ref{subsubsec: referencer_result}, where improved retriever recall from NitiLink didn't consistently improve E2E metrics. 
%
Here, increased E2E citations with Claude 3.5 Sonnet don't necessarily improve coverage or contradiction.

\textbf{NitiBench-CCL} For NitiBench-CCL (Table~\ref{table: main_exp_full}), the results are as expected: parametric performs worst, followed by Naive RAG, then Proposed RAG, and finally Golden Context (best).

\subsection{(RQ3) Performance of Long-Context LLM (LCLM)}
\label{subsec: rq3_result}

The evaluation results of the Thai legal QA system based on LCLM are presented in Table~\ref{table: main_exp_sampled} in comparison to the same baselines in Table~\ref{table: main_exp_full}. 
%
The evaluation is conducted on a stratified 20\% sample of NitiBench-CCL and the full NitiBench-Tax. 
%
As detailed in \S\ref{subsec: setup_rq3}, the LCLM system processes all 35 Thai financial laws simultaneously as context, with special tokens inserted to serve as section identifiers. 
%
These tokens enable the LCLM to cite relevant sections explicitly when generating responses to queries.

\begin{table}[!ht]
\centering
\renewcommand{\arraystretch}{1.2} % Increase cell height
\resizebox{\textwidth}{!}{%
\begin{tabular}{@{}lcccccccc@{}}
\toprule
\textbf{Setting} & \textbf{Retriever MRR ($\uparrow$)} & \textbf{Retriever Multi-MRR ($\uparrow$)} & \textbf{Retriever Recall ($\uparrow$)} & \textbf{Coverage ($\uparrow$)} & \textbf{Contradiction ($\downarrow$)} & \textbf{E2E Recall ($\uparrow$)} & \textbf{E2E Precision ($\uparrow$)} & \textbf{E2E F1 ($\uparrow$)} \\ 
\midrule
\multicolumn{9}{c}{\textbf{NitiBench-CCL (20\% subsampled)}} \\ \midrule
Parametric     & -     & -     & -     & 60.6  & 0.198 & 0.197 & 0.147 & 0.169 \\
Na\"ive RAG    & 0.549 & 0.549 & 0.649 & 77.7  & 0.092 & 0.740 & 0.379 & 0.501 \\
Proposed RAG   & 0.825 & 0.825 & 0.945 & \textbf{90.1}  & \textbf{0.028} & \textbf{0.920} & 0.453 & 0.607 \\
LCLM (Gemini)  & -     & -     & -     & 83.2  & 0.063 & 0.765 & \textbf{0.514} & \textbf{0.615} \\
\rowcolor{gray!15}
Golden Context & 1.0   & 1.0   & 1.0   & 94.2  & 0.025 & 0.999 & 1.000 & 0.999 \\
\midrule
\multicolumn{9}{c}{\textbf{NitiBench-Tax}} \\ \midrule
Parametric     & -     & -     & -     & 46.0  & 0.480 & \textbf{0.458} & \textbf{0.629} & \textbf{0.530} \\
Na\"ive RAG    & 0.120 & 0.048 & 0.062 & 50.0  & 0.460 & 0.306 & 0.463 & 0.368 \\
Proposed RAG   & 0.574 & 0.333 & 0.499 & \textbf{51.0}  & \textbf{0.440} & 0.389 & 0.554 & 0.457 \\
LCLM (Gemini)  & -     & -     & -     & 36.0  & 0.620 & 0.410 & 0.484 & 0.444 \\
\rowcolor{gray!15}
Golden Context & 1.0   & 1.0   & 1.0   & 52.0  & 0.460 & 0.694 & 1.000 & 0.820 \\ 
\bottomrule
\end{tabular}
}
\caption{Experiment results on sampled NitiBench-CCL and full NitiBench-Tax. In the LCLM setup, we used the Gemini without retriever section, where full legislation books were parsed as a context.}
\label{table: main_exp_sampled}
\end{table}



From Table~\ref{table: main_exp_sampled}, the LCLM-based system performs comparably to the parametric setting on NitiBench-Tax and to the Naive RAG system on NitiBench-CCL. 
%
This performance gap may stem from degradation when processing extremely long contexts (1.2 million tokens). 
%
Regardless of the exact cause, the results suggest that while an LCLM-based Thai legal QA system is feasible, its performance remains significantly behind RAG-based counterparts, highlighting areas for further improvement.

Apart from utilizing LCLM to process the legislations and respond directly to the queries, we also explored using it as a retriever. 
%
As stated in \S\ref{subsec: setup_rq3}, Gemini 1.5 Pro is provided with all 35 legislations and tasked to retrieve 20 relevant laws given a query. 
%
This experiment is also conducted on the same sample of NitiBench-CCL as the previous experiment and the full NitiBench-Tax. 
%
The results are shown in Table~\ref{table: retrieval_wcx_lclm} and~\ref{table: retrieval_tax_lclm}.

\begin{table}[!ht]
\centering
\small
\begin{tabular}{@{}clcc@{}}
\toprule
\textbf{Top-K} & \textbf{Model} & \textbf{HR/Recall@k} & \textbf{MRR@k} \\ \midrule
\multirow{9}{*}{k=1} 
  & BM25                   & .480           & .480 \\
  & JINA V2                & .698           & .698 \\
  & JINA V3                & .601           & .601 \\
  & NV-Embed V1            & .496           & .496 \\
  & BGE-M3                 & .708           & .708 \\
  & Human-Finetuned BGE-M3 & \textbf{.757}  & \textbf{.757} \\
  & Auto-Finetuned BGE-M3  & \underline{.741} & \underline{.741} \\
  & Cohere                 & .707           & .707 \\
  & LCLM-as-a-retriever (Gemini)                   & .590           & .590 \\ \midrule
\multirow{9}{*}{k=5} 
  & BM25                   & .663           & .549 \\
  & JINA V2                & .858           & .761 \\
  & JINA V3                & .828           & .693 \\
  & NV-Embed V1            & .711           & .585 \\
  & BGE-M3                 & \underline{.888} & .779 \\
  & Human-Finetuned BGE-M3 & \textbf{.909}  & \textbf{.819} \\
  & Auto-Finetuned BGE-M3  & \textbf{.909}  & \underline{.807} \\
  & Cohere                 & .867           & .772 \\
  & LCLM-as-a-retriever (Gemini)                   & .776           & .667 \\ \midrule
\multirow{9}{*}{k=10} 
  & BM25                   & .733           & .559 \\
  & JINA V2                & .891           & .766 \\
  & JINA V3                & .878           & .700 \\
  & NV-Embed V1            & .794           & .596 \\
  & BGE-M3                 & .926           & .784 \\
  & Human-Finetuned BGE-M3 & \textbf{.945}  & \textbf{.824} \\
  & Auto-Finetuned BGE-M3  & \underline{.941} & \underline{.812} \\
  & Cohere                 & .913           & .778 \\
  & LCLM-as-a-retriever (Gemini)                   & .807           & .671 \\ \bottomrule
\end{tabular}
\caption{Retrieval Evaluation Result on a 20\% subset of NitiBench-CCL with hierarchy-aware chunking with Long-Context Retriever. Since the test split of NitiBench-CCL is single labeled, duplicated metrics (HR/Recall and MRR) have been collapsed.}
\label{table: retrieval_wcx_lclm}
\end{table}



\begin{table}[!ht]
\centering
\small
\begin{tabular}{@{}clccccc@{}}
\toprule
\textbf{Top-K} & \textbf{Model}                  & \textbf{HR@k}          & \textbf{Multi HR@k}    & \textbf{Recall@k}      & \textbf{MRR@k}         & \textbf{Multi MRR@k}   \\ \midrule
k=1   & BM25                   & .220          & .080          & .118          & .220          & .118          \\
      & JINA V2                & .140          & .040          & .068          & .140          & .068          \\
      & JINA V3                & .400          & .100          & .203          & .400          & .203          \\
      & NV-Embed V1            & .100          & .020          & .035          & .100          & .035          \\
      & BGE-M3                 & \underline{.500}    & \underline{.140}    & \underline{.269}    & \underline{.500}    & \underline{.269}    \\
      & Human-Finetuned BGE-M3 & .480          & \underline{.140}    & .255          & .480          & .255          \\
      & Auto-Finetuned BGE-M3  & \textbf{.520} & \textbf{.160} & \textbf{.281} & \textbf{.520} & \textbf{.281} \\
      & Cohere                 & .340          & .100          & .179          & .340          & .179          \\
      & LCLM                   & .480          & .120          & .227          & .480          & .227          \\ \midrule
k=5   & BM25                   & .480          & .120          & .254          & .318          & .171          \\
      & JINA V2                & .200          & .080          & .114          & .165          & .085          \\
      & JINA V3                & .720          & \underline{.260}    & \underline{.448}    & .508          & .297          \\
      & NV-Embed V1            & .200          & .020          & .081          & .126          & .050          \\
      & BGE-M3                 & .720          & .240          & .435          & \underline{.580}    & \underline{.337}    \\
      & Human-Finetuned BGE-M3 & \underline{.740}    & .220          & .411          & .565          & .320          \\
      & Auto-Finetuned BGE-M3  & .700          & .200          & .382          & \textbf{.587} & \underline{.329}    \\
      & Cohere                 & .620          & .200          & .363          & .447          & .256          \\
      & LCLM-as-a-retriever (Gemini)                   & \textbf{.760} & \textbf{.320} & \textbf{.515} & \textbf{.587} & \textbf{.370} \\ \midrule
k=10  & BM25                   & .540          & .160          & .320          & .327          & .183          \\
      & JINA V2                & .240          & .100          & .147          & .171          & .091          \\
      & JINA V3                & \textbf{.840} & \underline{.340}    & .549          & .524          & .311          \\
      & NV-Embed V1            & .220          & .040          & .097          & .128          & .052          \\
      & BGE-M3                 & \underline{.820}    & \textbf{.360} & \underline{.555}    & \underline{.593}    & \underline{.354}    \\
      & Human-Finetuned BGE-M3 & .800          & .280          & .499          & .574          & .333          \\
      & Auto-Finetuned BGE-M3  & .780          & .260          & .483          & \textbf{.600} & \underline{.345}    \\
      & Cohere                 & .680          & .200          & .414          & .454          & .263          \\
      & LCLM-as-a-retriever (Gemini)                   & .780          & \textbf{.360} & \textbf{.566} & .590          & \textbf{.379} \\ \bottomrule
\end{tabular}
\caption{Retrieval Evaluation Result on NitiBench-Tax with hierarchy-aware chunking. This split contains multiple positives per question.}
\label{table: retrieval_tax_lclm}
\end{table}


The results indicate that the LCLM retriever performs comparably to embedding-based retrievers on NitiBench-Tax, likely due to its superior reasoning capabilities. 
%
However, a noticeable performance gap exists when compared to the best retriever on NitiBench-CCL.
%
Additionally, increasing the number of retrieved documents for the LCLM yields minimal performance improvements relative to other models. 
%
We hypothesize that this limited gain is a result of LLMs' next-token prediction mechanism, which may hinder their ability to effectively retrieve and output relevant laws when those laws are distant from the query context or when the model attempts to generate relevant laws ranked lower in the retrieval order.











\section{Discussion}
\label{sec:discussion}

In this section, we first summarize the conclusion and share some key observations. Then, we reflect on the usability of our method and propose potential applications. In the end, we discuss the limitations and future work.

\subsection{Effectiveness of \name{}}
\label{sec:discuss_effectiveness}
Firstly, based on the results from Section~\ref{sec:experiment}, we can draw the following conclusions:
\begin{itemize}
    \item It is efficient to detect unknown words by combining linguistic characteristics provided by the pre-trained language model (PLM) and gaze trajectory.
    \item The prediction is mainly based on the linguistic features from the textual context captured by PLM.
    \item Gaze locates the region of interest in a timely manner, which is necessary for real-time applications. Gaze also helps improve the model performance, but its contribution is limited compared to PLM.
\end{itemize}

Additionally, it is interesting that while we typically assume that the gaze modality should contribute significantly to the task of unknown word detection, the experimental results show that the contribution of gaze to the model’s improvement is small with the existence of PLM. Based on the previous analysis of line spacing and eye tracker accuracy, a possible reason for this is that under normal reading conditions (single-line spacing, line height 3-5 mm), the eye tracker’s accuracy is insufficient to precisely detect which line the gaze belongs to, thus failing to accurately locate the gaze on the words. Furthermore, changes in user posture during long reading sessions further reduce the accuracy of the eye tracker. In our system, PLM compensates for this issue by providing linguistic information based on the text.

From another perspective, the low contribution of gaze is not necessarily a disadvantage. Our method’s reduced reliance on gaze makes it more tolerant of noise. The model’s good performance on data collected by webcams further supports this conclusion. The reduced dependency on gaze data allows our model to be applied on more affordable and accessible devices, such as webcams.

\subsection{Usability of \name{}}
\label{sec:discuss_usability}
The results from the user evaluation (Section~\ref{sec:user_evaluation}) show that our reading assistance prototype helps users read more fluently and they are more willing to use it compared to traditional click-to-translate methods. In addition to providing real-time translation and explanations during reading, our system can also benefit ESL for long-term learning. For example, based on the unknown word detected by our system, we can generate a vocabulary list for memorizing and offer memory curve tracking. Furthermore, these unknown words can also be used to generate personalized summaries and notes.

The potential issue of generalizability across users, texts and devices can be addressed through fine-tuning and reinforcement learning methods. During the initial phases of usage, the system collects both gaze and text data for fine-tuning and lets users provide feedback on the model's predictions. This allows the model to continuously learn the user's unique gaze patterns and infer their vocabulary proficiency and domain expertise from textual content, thereby improving prediction accuracy.

\subsection{Limitation and Future Works}
\label{sec:discuss_limitation}
The quality of gaze data hinders the improvement model performance. The accuracy of the eye tracker is not enough for word-level detection. Common formatting, such as single-line spacing and 10-point font, results in a line height of approximately 3-5 mm when viewed using the PDF viewer with a sidebar on a 14-inch laptop. This requires an accuracy of about $0.3-0.6^\circ$ at a reading distance of 50-60 cm. However, most eye trackers have a gaze accuracy ranging from $0.2-1.1^\circ$~\cite{gaze_survey_2024}. Combined with additional errors caused by head and upper body movements, this level of accuracy is insufficient for real-world reading scenarios. During data collection and evaluation, some participants reported that even after calibration, the error could span 1-3 lines. This makes it difficult to determine the specific word the user is focusing on based solely on gaze coordinates, explaining why gaze-based baselines performed poorly on our data.

\change{The inaccuracy of the gaze data could also lead to the inaccuracy of data labeling. To mitigate the impact of mouse clicks on gaze behavior, we asked users to label unknown words during their second pass. However, this widely adopted labeling method inherently requires "guessing" which words correspond to a given gaze trajectory. Previous works mapped each gaze coordinate directly to a specific word to establish word-gaze pairs. This method is infeasible for text with normal line spacing, so we establish gaze-word pairs by defining a bounding box based on a segment of gaze to identify the corresponding words instead. While this approach improves robustness, it may also introduce mismatches between gaze and words and thus introduce noise to the dataset. To further improve model performance, more precise labeling methods are needed.}

Additionally, reading time can be longer than several minutes in daily scenarios, so gaze drift can significantly affect data quality. In our experiments, we observed that it is difficult for participants to maintain a fixed posture after calibration, though we required them to do so. The posture shift further increases errors. Therefore, in practical applications, real-time calibration of gaze data based on user posture is crucial to ensure data quality. If the existing eye-tracking technology can combined with user posture detection~\cite{faceori}, it is possible to reduce the impact of user posture on gaze data, thereby improving the quality of gaze data.



\section{Conclusion}
\label{sec:Conclusion}
This work evaluates proprietary and open-weight models in agentic frameworks for handling ambiguity in software engineering. In code generation, to effectively integrate new information into the solution, an agent must detect ambiguity and ask targeted questions. Our key findings are:
\begin{itemize}[itemsep=0pt, topsep=0pt]
    \item Given an underspecified input, Claude Sonnet 3.5 and Claude Haiku 3.5 with interaction can achieve 80\% of their performance with a well-specified input. In contrast, open-weight models struggle: Deepseek relies on navigational cues to locate relevant files, while Llama 3.1 70B extracts limited information from the user.
    \item LLMs do not interact unless explicitly prompted, and their ambiguity detection is highly sensitive to prompt variations. Only Claude Sonnet 3.5 achieves a higher accuracy of 84\% in distinguishing between well-specified and underspecified input.

    \item Claude Sonnet 3.5, Haiku 3.5, and Deepseek effectively extract new, detailed user information, whereas Llama 3.1 struggles to ask the right questions.
    
\end{itemize}
Despite these advances, a gap remains between resolve rates for underspecified vs. fully specified issues. Open-weight models need better interaction strategies to improve resolution, while proprietary models, particularly Claude Haiku 3.5, require stronger prompting to engage interactively. This work establishes the current state-of-the-art in handling ambiguity through interaction, breaking the resolution process into multiple steps.




%%
%% The acknowledgments section is defined using the "acks" environment
%% (and NOT an unnumbered section). This ensures the proper
%% identification of the section in the article metadata, and the
%% consistent spelling of the heading.
\begin{acks}
Research reported in this publication was supported by the National Eye Institute of the National Institutes of Health under Award Number R01EY037100. The content is solely the responsibility of the authors and does not necessarily represent the official views of the National Institutes of Health.
\end{acks}

%%
%% The next two lines define the bibliography style to be used, and
%% the bibliography file.
\bibliographystyle{ACM-Reference-Format}
\bibliography{sections/references}


%%
%% If your work has an appendix, this is the place to put it.
\appendix
%TC:ignore
\section{Theme Table}
\label{Codebook for Evaluation}
\begin{table*}[ht]
\scriptsize
\centering
\begin{tabular}
% {>{\centering\arraybackslash}p{3cm}>{\centering\arraybackslash}p{3cm}>{\centering\arraybackslash}p{3cm}}
{p{3.5cm} p{4.8cm} p{8.5cm}}
\toprule
  \textbf{Themes} & \textbf{Sub-themes} &  \textbf{Codes}\\
% \Xhline{2\arrayrulewidth}
\hline
Effect of Landmark Augmentations on Route Retracing & \multirow{2}{*}{Effective retracing} & \multirow{1}{*}{navigate easier; memorize the turns}\\
\hline
\multirow{6}{*}{Effect of Landmark Augmentations on} & Perceive hallway structures & notice hallway structures more than before; memorize hallway lengths\\
\cline{2-3}
\multirow{6}{*}{Mental Map Development} & \multirow{2}{*}{Increased Focus on Landmarks} & notice landmarks; identify landmarks; memorize landmarks; locate landmarks; increased reliance on landmarks in navigation\\
\cline{2-3}
& \multirow{4}{*}{Shift in Landmark Selection} & help notice things may be overlooked; monocular blindness: things in blind spot; pick out important landmarks better; help find functional facilities; offer more information to memorize; start paying attention to once augmented landmarks even without wearing the system; conflict with their own way of identifying landmarks\\
\hline
\multirow{6}{*}{Experiences with VisiMark} & Effectiveness&  effective in retracing; effective in mental map building\\
\cline{2-3}
& \multirow{2}{*}{Comfort}& mentally comfortable; more comfortable without the system; device uncomfort; uncomfortable in public\\
\cline{2-3}
& Learnability& easy to use and understand; short learning curve; tutorial\\
\cline{2-3}
& \multirow{2}{*}{Distraction}& more useful than distracting; eliminate visual noise from a space; very distracting because of learning curve; not overwhelming; overwhelming at some spots; limit the number of augmented landmarks\\
\hline
\multirow{4}{*}{Taxonomy of Landmarks to Augment} & Current landmarks in VisiMark & similar to those used in wayfinding and mental maps\\
\cline{2-3}
& Unique but not visually obvious landmarks& green double doors\\
\cline{2-3}
& Visually challenging but cognitively important landmarks & recessed or flat landmarks; elevators; restrooms\\
\cline{2-3}
& Landmarks outside their central view, especially dangers & landmarks above eye level; obstacles on the floor\\
\hline
\multirow{2}{*}{When the Augmentations Should Occur} & What to augment only only in preview & visually salient landmarks\\
\cline{2-3}
& What to augment only in situ & common yet important facilities; affordance; small or low contrast prints\\
\hline
\multirow{14}{*}{Desired Augmentation Designs} & \multirow{7}{*}{Signboards} &  have an overview ahead; locate oneself without extra trips; depth perception issues: providing hallway lengths; depth perception issues: help identify dead end; double vision: prefer signboards in central view; monocular blindness: point out possible directions; have scales; small arrows of further connecting hallways; maps of the general layout; colors are helpful cues to remember; confirm on the right track; easier navigation unconsciously; not turn-by-turn color-coded hallways distracting; distinct current colors; primary colors; allow brightness adjustability; allow transparency adjustability; allow more color choices; add dark colored outlines\\
\cline{2-3}
& \multirow{3}{*}{In-situ labels}& focus points to tie on; confirm on the right track from a distance; icons are simpler but convey same information; icons help people who cannot read; number of icon categories; unique icon categories; texts are more indicative; should not use abbreviation; more details in descriptions\\
\cline{2-3}
& \multirow{2}{*}{Further customization options} & specialize based on the building environment; add ability to turn on and off some components; customize personal layers; add ability to zoom in; add ability to adjust position of augmentations\\
\bottomrule
\end{tabular}
\caption{Themes and Codebook.}
\label{tab:Themes and Codebook}
\end{table*}



\section{Interview Questions for the Formative Study}
\label{Interview Questions for the Formative Study}
\subsection{Initial Interview Questions}
\begin{enumerate}
    \item What is your name?
    \item What is your age?
    \item What gender do you identify with?
    \item What is your visual condition?
    \item Are you considered Legally blind?
\begin{enumerate}
    \item What is your diagnosis? 
    \item What is your visual acuity?
    \item What is your field of view?
    \item What is your contrast sensitivity?
    \item What is your color vision?
    \item What is your light sensitivity? 
    \item What is your eepth perception? 
\end{enumerate}
    \item How long have you had this visual condition?
\begin{enumerate}
    \item Is this condition progressive or stable?
\end{enumerate}
    \item How do you usually complete a navigation task?
    \item Do you use any technology to navigate regularly outdoors and indoors?
\begin{enumerate}
    \item If yes, what technology do you use?
\end{enumerate}
    \item Do you pay attention to any landmarks during navigation? What landmarks?
    \item Do you use any technology to help you perceive the landmarks?
    \item Will your choice of landmarks change due to familiarity?
    \item Do you have any prior experience with Augmented Reality (Google glasses, phone application, HoloLens)? 
\begin{enumerate}
    \item If yes, could you please share the experience?
\end{enumerate}
    \item Are you currently familiar with the campus building?
\end{enumerate}

\subsection{Exit Interview Questions}
\begin{enumerate}
\item How do you determine landmarks? Do you prioritize certain landmark features (e.g., color, size, shape)?
\begin{enumerate}
\item Why do you pay attention to a specific landmark during the navigation task?
\end{enumerate}
\item What type of landmark modalities do you prefer?
\item How do you use your landmarks during navigation (including wayfinding and mental map building)?
\item Do you look for landmarks the first time you visit a place, or only after you’ve been there a few times?
\item Will your choice of landmarks change due to familiarity?
\item How important do you consider landmarks for indoor wayfinding? Could you please offer a score between 1 and 5, with 1 stands for least important and 5 means most important?
\item How important do you consider landmarks for developing a mental map indoors? Could you please offer a score between 1 and 5, with 1 stands for least important and 5 means most important?
\item What types of landmarks would you like to see augmented? Are there any specific landmarks you would prefer to use but find challenging due to visual limitations?
\item What kind of landmark augmentations you would like to have? (e.g., enlarging, outlining, etc.)
\item What modalities of landmark augmentations you would like to have? (e.g., visual, audio, haptic, etc.)
\end{enumerate}


\section{Interview Questions for the Final Evaluation}
\label{Interview Questions for the Evaluation}
\subsection{Initial Interview Questions}
\begin{enumerate}
    \item What is your name?
    \item What is your age?
    \item What gender do you identify with?
    \item What is your visual condition?
    \item Are you considered Legally blind?
\begin{enumerate}
    \item What is your diagnosis? 
    \item What is your visual acuity?
    \item What is your field of view?
    \item What is your contrast sensitivity?
    \item What is your color vision?
    \item What is your light sensitivity? 
    \item What is your eepth perception? 
\end{enumerate}
    \item How long have you had this visual condition?
\begin{enumerate}
    \item Is this condition progressive or stable?
\end{enumerate}
    \item Do you use any technology to navigate regularly outdoors and indoors?
\begin{enumerate}
    \item If yes, what technology do you use?
\end{enumerate}
    \item Do you pay attention to any landmarks during navigation? What landmarks?
    \item Do you use any technology to help you perceive the landmarks?
    \item Do you have any prior experience with Augmented Reality (Google glasses, phone application, HoloLens)? 
\begin{enumerate}
    \item If yes, could you please share the experience?
\end{enumerate}
    \item Are you currently familiar with navigating inside this building?
\end{enumerate}


\subsection{Exit Interview Questions}
\begin{enumerate}
 \item Let’s first talk about your landmark choices in the four trials. (Based on the mental map) Why do you pick this specific landmark?

 \item Our systems select and augment certain types of landmarks for you. Do you think they are useful or not? Why? What landmarks do you prefer to be augmented in indoor navigation? 

\item For [each element], how do you like it? How does this design affect your understanding of the route? How do you want to improve it?
\begin{enumerate}
\item The presentation of the structure of the hallways (e.g., direction, width, length, color-coding, the presentation of deadends)
\item The icons and texts of the landmarks on the signboard.
\item In-situ elements.
\end{enumerate}


\item Effectiveness: How effective do you think of the system? Could you please offer a score between 1 and 7, with 1 stands for least effective and 7 means most effective?
\item Comfortable: How comfortable do you think of the system? Could you please offer a score between 1 and 7, with 1 stands for least comfortable and 7 means most comfortable?

\item Distraction/Load: How distracting do you think of the system? Could you please offer a score between 1 and 7, with 1 stands for least distracting and 7 means most distracting ?
\item Learnability: How easy to understand or learn do you think of the system? Could you please offer a score between 1 and 7, with 1 stands for least easy to use and 7 means most easy to use?

\item Any ideas for other designs of augmentations to support your indoor navigation and mental model development?


\end{enumerate}

%TC:endignore


\end{document}
\endinput
%%
%% End of file `sample-sigconf-authordraft.tex'.
