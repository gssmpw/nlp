 %%
%% This is file `sample-sigconf-authordraft.tex',
%% generated with the docstrip utility.
%%
%% The original source files were:
%%
%% samples.dtx  (with options: `all,proceedings,bibtex,authordraft')
%% 
%% IMPORTANT NOTICE:
%% 
%% For the copyright see the source file.
%% 
%% Any modified versions of this file must be renamed
%% with new filenames distinct from sample-sigconf-authordraft.tex.
%% 
%% For distribution of the original source see the terms
%% for copying and modification in the file samples.dtx.
%% 
%% This generated file may be distributed as long as the
%% original source files, as listed above, are part of the
%% same distribution. (The sources need not necessarily be
%% in the same archive or directory.)
%%
%%
%% Commands for TeXCount
%TC:macro \cite [option:text,text]
%TC:macro \citep [option:text,text]
%TC:macro \citet [option:text,text]
%TC:envir table 0 1
%TC:envir table* 0 1
%TC:envir tabular [ignore] word
%TC:envir displaymath 0 word
%TC:envir math 0 word
%TC:envir comment 0 0
%%
%%
%% The first command in your LaTeX source must be the \documentclass
%% command.
%%
%% For submission and review of your manuscript please change the
%% command to \documentclass[manuscript, screen, review]{acmart}.
%%
%% When submitting camera ready or to TAPS, please change the command
%% to \documentclass[sigconf]{acmart} or whichever template is required
%% for your publication.
%%
%%
% \documentclass[sigconf,authordraft]{acmart}
\documentclass[sigconf]{acmart}
% \documentclass[manuscript,review,anonymous]{acmart}
\usepackage{graphicx}
\usepackage{subcaption}
\usepackage{float}
\usepackage{multirow}
\usepackage{array}
\usepackage{makecell}
% \usepackage{accessibility}
\usepackage{xcolor}
\definecolor{brownishred}{RGB}{0,0,0}
% \definecolor{brownishred}{RGB}{178, 34, 34}
\newcommand{\yuhang}[1]{{\small\textcolor{red}{\bf [YZ: #1]}}}
\newcommand{\colorchange}[1]{{\textcolor{brownishred}{#1}}}

%%
%% \BibTeX command to typeset BibTeX logo in the docs
\AtBeginDocument{%
  \providecommand\BibTeX{{%
    Bib\TeX}}}

%% Rights management information.  This information is sent to you
%% when you complete the rights form.  These commands have SAMPLE
%% values in them; it is your responsibility as an author to replace
%% the commands and values with those provided to you when you
%% complete the rights form.

\setcopyright{acmlicensed}
\copyrightyear{2025}
\acmYear{2025}
\setcopyright{cc}
\setcctype{by-nd}
\acmConference[CHI '25]{CHI Conference on Human Factors in Computing Systems}{April 26-May 1, 2025}{Yokohama, Japan}
\acmBooktitle{CHI Conference on Human Factors in Computing Systems (CHI '25), April 26-May 1, 2025, Yokohama, Japan}\acmDOI{10.1145/3706598.3713847}
\acmISBN{979-8-4007-1394-1/25/04}

% \setcopyright{acmlicensed}
% \copyrightyear{2024}
% \acmYear{2024}
% \acmDOI{XXXXXXX.XXXXXXX}

% %% These commands are for a PROCEEDINGS abstract or paper.
% \acmConference[CHI '25]{Conference on Human Factors in Computing Systems}{April 26--May 01, 2025}{Yokohama, Japan}
% % \acmConference[Conference acronym 'XX]{Make sure to enter the correct conference title from your rights confirmation emai}{June 03--05, 2018}{Woodstock, NY}
% %%
% %%  Uncomment \acmBooktitle if the title of the proceedings is different
% %%  from ``Proceedings of ...''!
% %%
% %%\acmBooktitle{Woodstock '18: ACM Symposium on Neural Gaze Detection,
% %%  June 03--05, 2018, Woodstock, NY}
% \acmISBN{978-1-4503-XXXX-X/18/06}


%%
%% Submission ID.
%% Use this when submitting an article to a sponsored event. You'll
%% receive a unique submission ID from the organizers
%% of the event, and this ID should be used as the parameter to this command.
%%\acmSubmissionID{123-A56-BU3}

%%
%% For managing citations, it is recommended to use bibliography
%% files in BibTeX format.
%%
%% You can then either use BibTeX with the ACM-Reference-Format style,
%% or BibLaTeX with the acmnumeric or acmauthoryear sytles, that include
%% support for advanced citation of software artefact from the
%% biblatex-software package, also separately available on CTAN.
%%
%% Look at the sample-*-biblatex.tex files for templates showcasing
%% the biblatex styles.
%%

%%
%% The majority of ACM publications use numbered citations and
%% references.  The command \citestyle{authoryear} switches to the
%% "author year" style.
%%
%% If you are preparing content for an event
%% sponsored by ACM SIGGRAPH, you must use the "author year" style of
%% citations and references.
%% Uncommenting
%% the next command will enable that style.
%%\citestyle{acmauthoryear}


%%
%% end of the preamble, start of the body of the document source.
\begin{document}

%%
%% The "title" command has an optional parameter,
%% allowing the author to define a "short title" to be used in page headers.
\title[VisiMark]{VisiMark: Characterizing and Augmenting Landmarks for People with Low Vision in Augmented Reality to Support Indoor Navigation}

%%
%% The "author" command and its associated commands are used to define
%% the authors and their affiliations.
%% Of note is the shared affiliation of the first two authors, and the
%% "authornote" and "authornotemark" commands
%% used to denote shared contribution to the research.
\author{Ruijia Chen}
\affiliation{%
  \institution{University of Wisconsin-Madison}
  \city{Madison}
  \state{Wisconsin}
  \country{USA}
}
\email{ruijia.chen@wisc.edu}
\orcid{0000-0002-1655-6228}

\author{Junru Jiang}
\affiliation{%
  \institution{University of Wisconsin-Madison}
  \city{Madison}
  \state{Wisconsin}
  \country{USA}
  }
\email{jjiang324@wisc.edu}

\author{Pragati Maheshwary}
\affiliation{%
  \institution{Carnegie Mellon University}
  \city{Pittsburgh}
  \state{Pennsylvania}
  \country{USA}
}
\email{pragati2@andrew.cmu.edu}

\author{Brianna R. Cochran}
\affiliation{%
  \institution{University of Wisconsin-Madison}
  \city{Madison}
  \state{Wisconsin}
  \country{USA}
  }
\email{bcochran2@wisc.edu}

\author{Yuhang Zhao}
\affiliation{%
  \institution{University of Wisconsin-Madison}
  \city{Madison}
  \state{Wisconsin}
  \country{USA}
  }
\email{yuhang.zhao@cs.wisc.edu}


%%
%% By default, the full list of authors will be used in the page
%% headers. Often, this list is too long, and will overlap
%% other information printed in the page headers. This command allows
%% the author to define a more concise list
%% of authors' names for this purpose.
\renewcommand{\shortauthors}{Chen et al.}

%%
%% The abstract is a short summary of the work to be presented in the
%% article.
\begin{abstract}
Landmarks are critical in navigation, supporting self-orientation and mental model development. Similar to sighted people, people with low vision (PLV) frequently look for landmarks via visual cues but face difficulties identifying some important landmarks due to vision loss. We first conducted a formative study with six PLV to characterize their challenges and strategies in landmark selection, identifying their unique landmark categories (e.g., area silhouettes, accessibility-related objects) and preferred landmark augmentations. We then designed \textit{VisiMark}, an AR interface that supports landmark perception for PLV by providing both overviews of space structures and in-situ landmark augmentations. We evaluated VisiMark with 16 PLV and found that VisiMark enabled PLV to perceive landmarks they preferred but could not easily perceive before, and changed PLV's landmark selection from only visually-salient objects to cognitive landmarks that are more important and meaningful. We further derive design considerations for AR-based landmark augmentation systems for PLV.
\end{abstract}

%%
%% The code below is generated by the tool at http://dl.acm.org/ccs.cfm.
%% Please copy and paste the code instead of the example below.
%%
\begin{CCSXML}
<ccs2012>
   <concept>
       <concept_id>10003120.10011738.10011775</concept_id>
       <concept_desc>Human-centered computing~Accessibility technologies</concept_desc>
       <concept_significance>500</concept_significance>
       </concept>
   <concept>
       <concept_id>10003120.10011738.10011776</concept_id>
       <concept_desc>Human-centered computing~Accessibility systems and tools</concept_desc>
       <concept_significance>500</concept_significance>
       </concept>
   <concept>
       <concept_id>10003120.10011738.10011774</concept_id>
       <concept_desc>Human-centered computing~Accessibility design and evaluation methods</concept_desc>
       <concept_significance>500</concept_significance>
       </concept>
   <concept>
       <concept_id>10003120.10003121.10003124.10010392</concept_id>
       <concept_desc>Human-centered computing~Mixed / augmented reality</concept_desc>
       <concept_significance>500</concept_significance>
       </concept>
 </ccs2012>
\end{CCSXML}
\ccsdesc[500]{Human-centered computing~Accessibility technologies}
\ccsdesc[500]{Human-centered computing~Accessibility systems and tools}
\ccsdesc[500]{Human-centered computing~Accessibility design and evaluation methods}
\ccsdesc[500]{Human-centered computing~Mixed / augmented reality}

%%
%% Keywords. The author(s) should pick words that accurately describe
%% the work being presented. Separate the keywords with commas.
\keywords{Accessibility, Virtual/Augmented Reality, Individuals with Disabilities \& Assistive Technologies}
%% A "teaser" image appears between the author and affiliation
%% information and the body of the document, and typically spans the
%% page.
% \begin{teaserfigure}
%   \includegraphics[width=\textwidth]{sampleteaser}
%   \caption{Seattle Mariners at Spring Training, 2010.}
%   \Description{Enjoying the baseball game from the third-base
%   seats. Ichiro Suzuki preparing to bat.}
%   \label{fig:teaser}
% \end{teaserfigure}
%TC:ignore
\begin{teaserfigure}
  \includegraphics[width=\textwidth]{graphs/teaser.png}
    % \vspace{-5ex}
  \caption{\textit{VisiMark} provides landmark augmentations on head-mounted AR to support wayfinding and mental map construction. VisiMark includes two features: (A) \textit{Signboard}, an overview of hallway structures and upcoming landmarks at intersections, and (B) \textit{In-situ Labels}, world-anchored icons and texts to highlight the types and positions of landmarks in the physical environment.}
  \Description{This image shows VisiMark system. VisiMark provides landmark augmentations on head-mounted AR to support wayfinding and mental map construction. VisiMark includes two features: (A) Signboard, an overview of hallway structures and upcoming landmarks at intersections, and (B) In-situ Labels, world-anchored icons and texts to highlight the types and positions of landmarks in the physical environment.}
  \label{fig:teaser}
\end{teaserfigure}
%TC:endignore

% \received{20 February 2007}
% \received[revised]{12 March 2009}
% \received[accepted]{5 June 2009}

%%
%% This command processes the author and affiliation and title
%% information and builds the first part of the formatted document.
\maketitle

\section{Introduction}

Despite the remarkable capabilities of large language models (LLMs)~\cite{DBLP:conf/emnlp/QinZ0CYY23,DBLP:journals/corr/abs-2307-09288}, they often inevitably exhibit hallucinations due to incorrect or outdated knowledge embedded in their parameters~\cite{DBLP:journals/corr/abs-2309-01219, DBLP:journals/corr/abs-2302-12813, DBLP:journals/csur/JiLFYSXIBMF23}.
Given the significant time and expense required to retrain LLMs, there has been growing interest in \emph{model editing} (a.k.a., \emph{knowledge editing})~\cite{DBLP:conf/iclr/SinitsinPPPB20, DBLP:journals/corr/abs-2012-00363, DBLP:conf/acl/DaiDHSCW22, DBLP:conf/icml/MitchellLBMF22, DBLP:conf/nips/MengBAB22, DBLP:conf/iclr/MengSABB23, DBLP:conf/emnlp/YaoWT0LDC023, DBLP:conf/emnlp/ZhongWMPC23, DBLP:conf/icml/MaL0G24, DBLP:journals/corr/abs-2401-04700}, 
which aims to update the knowledge of LLMs cost-effectively.
Some existing methods of model editing achieve this by modifying model parameters, which can be generally divided into two categories~\cite{DBLP:journals/corr/abs-2308-07269, DBLP:conf/emnlp/YaoWT0LDC023}.
Specifically, one type is based on \emph{Meta-Learning}~\cite{DBLP:conf/emnlp/CaoAT21, DBLP:conf/acl/DaiDHSCW22}, while the other is based on \emph{Locate-then-Edit}~\cite{DBLP:conf/acl/DaiDHSCW22, DBLP:conf/nips/MengBAB22, DBLP:conf/iclr/MengSABB23}. This paper primarily focuses on the latter.

\begin{figure}[t]
  \centering
  \includegraphics[width=0.48\textwidth]{figures/demonstration.pdf}
  \vspace{-4mm}
  \caption{(a) Comparison of regular model editing and EAC. EAC compresses the editing information into the dimensions where the editing anchors are located. Here, we utilize the gradients generated during training and the magnitude of the updated knowledge vector to identify anchors. (b) Comparison of general downstream task performance before editing, after regular editing, and after constrained editing by EAC.}
  \vspace{-3mm}
  \label{demo}
\end{figure}

\emph{Sequential} model editing~\cite{DBLP:conf/emnlp/YaoWT0LDC023} can expedite the continual learning of LLMs where a series of consecutive edits are conducted.
This is very important in real-world scenarios because new knowledge continually appears, requiring the model to retain previous knowledge while conducting new edits. 
Some studies have experimentally revealed that in sequential editing, existing methods lead to a decrease in the general abilities of the model across downstream tasks~\cite{DBLP:journals/corr/abs-2401-04700, DBLP:conf/acl/GuptaRA24, DBLP:conf/acl/Yang0MLYC24, DBLP:conf/acl/HuC00024}. 
Besides, \citet{ma2024perturbation} have performed a theoretical analysis to elucidate the bottleneck of the general abilities during sequential editing.
However, previous work has not introduced an effective method that maintains editing performance while preserving general abilities in sequential editing.
This impacts model scalability and presents major challenges for continuous learning in LLMs.

In this paper, a statistical analysis is first conducted to help understand how the model is affected during sequential editing using two popular editing methods, including ROME~\cite{DBLP:conf/nips/MengBAB22} and MEMIT~\cite{DBLP:conf/iclr/MengSABB23}.
Matrix norms, particularly the L1 norm, have been shown to be effective indicators of matrix properties such as sparsity, stability, and conditioning, as evidenced by several theoretical works~\cite{kahan2013tutorial}. In our analysis of matrix norms, we observe significant deviations in the parameter matrix after sequential editing.
Besides, the semantic differences between the facts before and after editing are also visualized, and we find that the differences become larger as the deviation of the parameter matrix after editing increases.
Therefore, we assume that each edit during sequential editing not only updates the editing fact as expected but also unintentionally introduces non-trivial noise that can cause the edited model to deviate from its original semantics space.
Furthermore, the accumulation of non-trivial noise can amplify the negative impact on the general abilities of LLMs.

Inspired by these findings, a framework termed \textbf{E}diting \textbf{A}nchor \textbf{C}ompression (EAC) is proposed to constrain the deviation of the parameter matrix during sequential editing by reducing the norm of the update matrix at each step. 
As shown in Figure~\ref{demo}, EAC first selects a subset of dimension with a high product of gradient and magnitude values, namely editing anchors, that are considered crucial for encoding the new relation through a weighted gradient saliency map.
Retraining is then performed on the dimensions where these important editing anchors are located, effectively compressing the editing information.
By compressing information only in certain dimensions and leaving other dimensions unmodified, the deviation of the parameter matrix after editing is constrained. 
To further regulate changes in the L1 norm of the edited matrix to constrain the deviation, we incorporate a scored elastic net ~\cite{zou2005regularization} into the retraining process, optimizing the previously selected editing anchors.

To validate the effectiveness of the proposed EAC, experiments of applying EAC to \textbf{two popular editing methods} including ROME and MEMIT are conducted.
In addition, \textbf{three LLMs of varying sizes} including GPT2-XL~\cite{radford2019language}, LLaMA-3 (8B)~\cite{llama3} and LLaMA-2 (13B)~\cite{DBLP:journals/corr/abs-2307-09288} and \textbf{four representative tasks} including 
natural language inference~\cite{DBLP:conf/mlcw/DaganGM05}, 
summarization~\cite{gliwa-etal-2019-samsum},
open-domain question-answering~\cite{DBLP:journals/tacl/KwiatkowskiPRCP19},  
and sentiment analysis~\cite{DBLP:conf/emnlp/SocherPWCMNP13} are selected to extensively demonstrate the impact of model editing on the general abilities of LLMs. 
Experimental results demonstrate that in sequential editing, EAC can effectively preserve over 70\% of the general abilities of the model across downstream tasks and better retain the edited knowledge.

In summary, our contributions to this paper are three-fold:
(1) This paper statistically elucidates how deviations in the parameter matrix after editing are responsible for the decreased general abilities of the model across downstream tasks after sequential editing.
(2) A framework termed EAC is proposed, which ultimately aims to constrain the deviation of the parameter matrix after editing by compressing the editing information into editing anchors. 
(3) It is discovered that on models like GPT2-XL and LLaMA-3 (8B), EAC significantly preserves over 70\% of the general abilities across downstream tasks and retains the edited knowledge better.
\section{Related Work}

\subsection{Personalization and Role-Playing}
Recent works have introduced benchmark datasets for personalizing LLM outputs in tasks like email, abstract, and news writing, focusing on shorter outputs (e.g., 300 tokens for product reviews \citep{kumar2024longlamp} and 850 for news writing \citep{shashidhar-etal-2024-unsupervised}). These methods infer user traits from history for task-specific personalization \citep{sun-etal-2024-revealing, sun-etal-2025-persona, pal2024beyond, li2023teach, salemi2025reasoning}. In contrast, we tackle the more subjective problem of long-form story writing, with author stories averaging 1500 tokens. Unlike prior role-playing approaches that use predefined personas (e.g., Tony Stark, Confucius) \citep{wang-etal-2024-rolellm, sadeq-etal-2024-mitigating, tu2023characterchat, xu2023expertprompting}, we propose a novel method to infer story-writing personas from an author’s history to guide role-playing.


\subsection{Story Understanding and Generation}  
Prior work on persona-aware story generation \citep{yunusov-etal-2024-mirrorstories, bae-kim-2024-collective, zhang-etal-2022-persona, chandu-etal-2019-way} defines personas using discrete attributes like personality traits, demographics, or hobbies. Similarly, \citep{zhu-etal-2023-storytrans} explore story style transfer across pre-defined domains (e.g., fairy tales, martial arts, Shakespearean plays). In contrast, we mimic an individual author's writing style based on their history. Our approach differs by (1) inferring long-form author personas—descriptions of an author’s style from their past works, rather than relying on demographics, and (2) handling long-form story generation, averaging 1500 tokens per output, exceeding typical story lengths in prior research.
\section{Formative Study}

To understand the specific requirements for text-to-SQL dataset annotation, we conducted a formative study by interviewing 5 engineers from Adobe. These interviewees have experienced annotating text-to-SQL datasets in their work.
We describe our interview process in Section~\ref{sec:interview}. Based on these interviews, we identified five major user needs in Section~\ref{sec:user_needs}. 
Finally, we discuss our design rationale in Section~\ref{sec:design}, aiming to address the user needs.

\subsection{Interview}
\label{sec:interview}

We conducted 20-minute semi-structured interviews with each interviewee through a conversational and think-aloud process. 
During these interviews, we first asked about the \textbf{motivation} for text-to-SQL annotation in their use cases, specifically about the schemas they worked on and why obtaining more data was important.
Interviewees reported that when deploying a new service, companies often needed to introduce new entities and restructure the original schema.
However, after updating the schema, they typically found that model performance dropped dramatically. Their regression tests showed an overall accuracy drop of 13.3\% for newly added columns and 9.1\% for new tables. As the schema was further updated, performance continued to decline. Moreover, as the schema changed significantly, they needed a large amount of new data on the updated schema to ensure a robust evaluation.

Second, we asked about their detailed \textbf{workflow} and whether they used any tools to assist with data annotation. Interviewees reported that they did not use any specific tool for annotation, although they sometimes asked ChatGPT to generate initial data. 
Additionally, they often leveraged previous datasets by adapting previous queries to the new schema, such as replacing an outdated column name with a new one.
After annotation, their colleagues performed peer reviews to check and refine the data.





Third, we asked about \textbf{challenges} they had met and the speed of their dataset annotation. Overall, they considered annotation to be very expensive. 
Interviewees mentioned that one engineer could only annotate 50 effective SQL and NL pairs per day in their use case. 
They often lost track and felt overwhelmed during annotation. 
Despite the peer review, they still felt a lack of confidence in the quality of the annotated data. 
They pointed out that randomness existed throughout the entire procedure. 
We summarize more challenges as user needs in Section~\ref{sec:user_needs}.






\subsection{User Needs}
\label{sec:user_needs}

\noindent \textbf{\textit{N1: Effective Schema Comprehension.}} 
Text-to-SQL annotation assumes that users can easily understand the database schema specified in a certain format (e.g., Data Definition Language). However, our interviews indicate that it is cumbersome and error-prone for users to navigate and comprehend complex schemas from such a specification format.



\noindent \textbf{\textit{N2: Creating New Queries.}}
Creating SQL queries requires a deep understanding of both database schema and SQL grammar. When creating a text-to-SQL dataset, users need to continually come up with new, diverse SQL queries. However, it is challenging for them to break free from preconceptions shaped by existing queries they have seen before.

\noindent \textbf{\textit{N3: Detecting Errors in the Annotated Data.}} 
An annotated dataset may include errors, which can deteriorate model performance and evaluation results.
Our interviews suggest that annotators need an effective mechanism for detecting potential errors or ambiguity in the constructed queries.


\noindent \textbf{\textit{N4: Efficiently Correcting the Detected Errors.}} After identifying errors, users need an efficient way to correct these errors to ensure the accuracy and reliability of the dataset. They need to ensure the SQL query is syntactically correct, and the NL is semantically equivalent to the SQL query.
%However, correcting errors in the annotated data is often as labor-intensive as annotating new data.


\noindent \textbf{\textit{N5: Improve Dataset Diversity.}}
Dataset diversity is crucial for improving model performance and ensuring rigorous evaluation.
Human annotation can easily introduce biases due to individual knowledge gaps and a lack of holistic understanding of the dataset composition. 
Thus, interviewees reported the need for an effective way to improve diversity and eliminate biases in the dataset. 




\subsection{Design Rationale}
\label{sec:design}


To support \textbf{N1}, our approach visualizes the database schema as a dynamic, editable graph. This enables users to quickly grasp the overall structure of the database and the relationships between entities. Users can explore detailed information such as data type through further interactions with the graph.


To support \textbf{N2}, our approach alleviates the burden of manually creating new SQL queries. We design an algorithm to randomly sample SQL query templates based on SQL grammar, then fill out this template with entities and values retrieved from the database. We make the SQL generation highly configurable---users can manually adjust keyword probability, or automatically tune the probability by an existing dataset.

To support \textbf{N3}, our approach renders the alignment between the SQL query and the NL question via a step-by-step analysis. Our approach then prompts the LLM to highlight potential misalignments to users.
Subsequently, our approach performs a textual analysis to check the equivalence of the SQL query and NL question and offers users a confidence score about their consistency.

To support \textbf{N4}, we handle two common errors---missing information and including irrelevant information in the NL question. Our approach allows users to fix errors by injecting missing information or removing irrelevant details based on LLM-generated suggestions.

To support \textbf{N5}, our approach first enables users to sample SQL queries based on a probability distribution learned from real-world data rather than creating them manually. 
Furthermore, our approach supports visualizing various dataset compositions through diagrams. 
For example, users can view a bar chart displaying the distribution of column counts in SQL queries. This feature allows users to monitor dataset composition during annotation, maintaining control over the annotation direction and improving data diversity.


\section{Proactive Privacy Amnesia
 \label{our_method_section}}
In this section, we introduce our method, PPA. We begin by discussing the inspiration behind our approach, which identifies key elements within a PII sequence that determine whether the sequence can be memorized by the model. Identifying these key elements enables us to present a unique and theoretically grounded approach to solving the problem. Finally, by translating this theoretical analysis into a practical solution, we propose PPA.

%, designed to forget a user's PII while preserving the model's performance. The method consists of three stages: Sensitivity Analysis, Selective Forgetting, and Memory Implanting. Sensitivity Analysis identifies the key elements in the PII sequence that determine whether it can be retained. Selective Forgetting ensures the LLM forgets these key elements, and Memory Implanting compensates for the performance degradation in the LLM.




\begin{comment}
In this section we introduce our method, Dynamic Mix Selected Unlearning, which consists of three stages: Sensitivity Analysis, Selected Unlearning, and Error Injection~\citep{de2021editing}. Sensitivity Analysis is to analyze which tokens within the PII sequence are the key elements determine whether it can be retained. Selected Unlearning is to let LLM to forget the specific key elements. Error Injection is to compensate the downgrade on the LLM performance. We start by discussing our inspiration of our method. Then, we provide theory analysis on our method. Last, we formulate the proposed Dynamic Mix Selected Unlearning to forget user's PII while maintaining the model's performance.
\end{comment}

\subsection{Inspiration and Overview}

Our Proactive Privacy Amnesia is inspired by Anterograde Amnesia ~\citep{markowitsch2008anterograde}, which is the inability to form new memories following an event while preserving long-term memories before the event. In a case study described by \citet{vicari2007acquired}, a girl suffering from Anterograde Amnesia since childhood exhibited severe impairment in episodic memory while retaining her semantic memory. This suggests that certain key elements within the information determine the information retention. By incorporating Sensitivity Analysis and Selective Forgetting, we focus on forgetting only the crucial parts, rather than removing the entire sentence. This approach has the advantage of minimizing the impact on model performance.
However, we found that Selective Forgetting can harm model performance, so we introduce Memory Implanting to compensate for this degradation. Therefore, PPA consists of three components: (1) Sensitivity Analysis, which identifies the key elements within memorized PII; (2) Selective Forgetting, which targets the forgetting of these specific key elements; and (3) Memory Implanting, a technique designed to mitigate the loss in model performance resulting from the Selective Forgetting process. 
% \MK{explain three components of PPA too redundant?}


\begin{comment}
By identifying and selectively forgetting these key elements, LLM can forget specific information while maintaining overall performance. This is because only the crucial parts of the information are forgotten, rather than the entire sentence. Therefore, we aim to apply PPA to remove PII from LLMs while preserving their effectiveness for their intended purposes.
\end{comment}
\begin{comment}
\textbf{Inspiration.} Our Proactive Privacy Amnesia is inspired by Anterograde Amnesia ~\citep{markowitsch2008anterograde}, which is the inability to form new memories following an event while preserving long-term memories from before the event. In a case study described by~\citep{vicari2007acquired}, a girl suffering from Anterograde Amnesia since childhood exhibited severe impairment in episodic memory while retaining her semantic memory. This suggests that certain key elements within the information determine whether it can be retained. By identifying and selectively forgetting these key elements, LLM can forget specific information while maintaining overall performance. This is because only the crucial parts of the information are forgotten, rather than the entire sentence. Therefore, we aim to apply PPA to remove PII from LLMs while preserving their effectiveness for their intended purposes.
\end{comment}


\subsection{Theoretical Justification of Sensitivity Analysis.}
\paragraph{Definition of Sensitivity Analysis.} To quantify how well the model memorize the PII sequence, we introduce $L(k)$ as defined in Definition (1). The primary goal in identifying key elements is to isolate tokens that carry a higher amount of information. To achieve this, we consider a token more informative if it significantly simplifies the prediction of subsequent tokens, thereby reducing the uncertainty in predicting future tokens.

% we measure the rate of change in cross-entropy during next-token prediction, focusing particularly on the transition from high to low. A token that significantly simplifies the prediction of subsequent tokens is considered more informative, as it greatly reduces the uncertainty in predicting future tokens.

% \textbf{Definition 1.}
\begin{definition} (Cross-entropy Loss of the PII Sequence) 
We define 
\begin{align}
    L(k) = L_{\text{CE}}\left(p(\rvx_1,\ldots,\rvx_k), q(\rvx_1,\ldots,\rvx_k)\right), \label{eq:L(k)_definition}
\end{align}
where $L_{\text{CE}}$ is the Cross Entropy Loss, and $x_1, \cdots, x_k$  refers to the first $k$ tokens of a PII sequence. 
\end{definition}

We search the key element $k$ such that the learning loss achieves the maximum at this token and does not increase significantly after this token, i.e., 
\begin{align}
    L(k-1) < L(k) \approx L(k+1) \approx L(k+2) \approx \cdots,
\end{align}
which means that the token $k$ helps the model memorize the following tokens in this PII sequence. Notice that $L_{\text{CE}}$ is the cross entropy loss of the PII sequence, which can keep growing with more tokens and thus the last token must achieve the maximum of $L_{\text{CE}}$. This solution is trivial and cannot show the essentiality of the token. To tackle this issue, we propose to find the token $k$ with the largest \textit{memorization factor} $D_k$, which can lead to a non-trivial solution of Eq. (\ref{eq:L(k)_definition}) as stated in Proposition \ref{proposition1}:

%Moreover, $\max_k L(k)$ leads to the 'memorization factor,' $D_i$, as defined in Proposition (1). A larger value of $D_i$ suggests that the token is more likely to be a key element.
\begin{definition} (Memorization Factor)
We define the memorization factor $D_k$ as follows: 
\begin{align}
    &D_k = \frac{H_k-H_{k+1}}{H_k}; H_i = L_{\text{CE}}(p_i,q_i),
\end{align}
Where \( p_i(x) \) be the true probability distribution and \( q_i(x) \) the predicted probability distribution for the \(i\)-th token in the PII sequence.
\end{definition}

\begin{proposition} \label{proposition1}
Maximizing the memorization factor can lead to
\begin{align}
    \max_k D(k) = \left\{
    \begin{array}{lll}
        \max_k L(k)&\text{if } \exists k, \nabla L(k)=0,    \\
        \max_k 1/d_{\text{Newton}}(k)& \text{if } \nexists k, \nabla L(k)=0.  
    \end{array}
\right.
\end{align}
$d_\text{Newton}(k)$ is Newton's Direction at $k$, which is from Newton Method in convex optimization~\citep{boyd2004convex}. $\max_k 1/d_{\text{Newton}}(k)$ is achieved when $d_{\text{Newton}}(k)\rightarrow 0^+$. As $L(k)$ is non-decreasing, a small positive $d_\text{Newton}(k)$ implies that the gradient at token $k$ quickly approaches $0$ with a negative second-order derivative.
\end{proposition}



\paragraph{Examples on PII sequences.}
We do sensitivity analysis on "John Griffith phone number (713) 853-6247," as shown in Figure~\ref{fig:phone_dmsu_sensitivity_analysis}, the token '8' exhibits the most significant decrease in cross-entropy rate, making it the key element in this context. Similarly, in "Jeffrey Dasovich address 101 California St. Suite 1950", depicted in Figure~\ref{fig:address_dmsu_sensitivity_analysis}, the token '\_Su' shows the most notable drop in cross-entropy rate, identifying '\_Su' as the key element.







\begin{comment}
First, we introduce the 'memorization factor', $D_i$, as defined in Eq. (\ref{eq:cross_entropy_loss_ratio}
),
\begin{align}
% H_i = -\sum_{x} p_i(x) \log q_i(x)
D_{i} = \frac{H_{i} - H_{i+1}}{H_{i}}; H_i = CrossEntropyLoss(p_i, q_i)
\label{eq:cross_entropy_loss_ratio} 
\end{align}
which is motivated by information theory. The primary goal of identifying key elements is to isolate tokens that carry a greater amount of information. To achieve this, we measure the rate of change in cross-entropy during next-token prediction, focusing particularly on the transition from high to low. A token that significantly simplifies the prediction of subsequent tokens is considered more informative, as it greatly reduces the uncertainty in predicting future tokens. A larger $D_i$, suggests that the token is more likely to be a key element.

\textbf{Theoretical Justification of Sensitivity Analysis.} We uncover the relationship between our sensitivity-based selection and the second-order Newton's Method. We consider the following optimization problem that finds the maximum of the cross-entropy loss: 

\textbf{Definition 1.} 
\begin{align}
    \max_k L(k) = L_{\text{CE}}(p(\rvx_1,\cdots,\rvx_k),q(\rvx_1,\cdots,\rvx_k)).
\end{align}

\textbf{Proposition 1.} The memorization factor $D_k$ is expressed as follows: 
\begin{align}
    &D_k = \frac{H_k-H_{k+1}}{H_k} \approx -\frac{\nabla^2 L(k)}{\nabla L(k)}.
\end{align}



Notice that 
\begin{align}
    L(k) = &-\sum_{\vx_1\cdots,\vx_k} p(\vx_1\cdots,\vx_k)\log q(\vx_1\cdots,\vx_k)\label{eq:accumulate_H}\\
    =&-\sum_{\vx_1\cdots,\vx_{k-1}} p(\vx_1\cdots,\vx_{k-1})\log q(\vx_1\cdots,\vx_{k-1})\nonumber\\
    &-\sum_{\vx_1\cdots,\vx_{k-1}} p(\vx_1\cdots,\vx_{k-1})\sum_{\vx_k}p(\vx_k|\vx_1\cdots,\vx_{k-1})\log q(\vx_k|\vx_1\cdots,\vx_{k-1})\\
    =&L(k-1)+H_k,
\end{align}
where $H_k$ is what we defined in Eq. (\ref{eq:cross_entropy_loss_ratio}). So we have 
\begin{align}
    &H_k=L(k)-L(k-1)\approx \nabla L(k),\\
    &H_{k+1}-H(k) \approx \nabla L(k+1)-\nabla L(k)\approx \nabla^2 L(k),\\
    &D_k = \frac{H_k-H_{k+1}}{H_k} \approx -\frac{\nabla^2 L(k)}{\nabla L(k)}.
\end{align}
Our selection method selects $k$ with the largest $D_k$. We discuss it in two situations:
\begin{enumerate}
    \item When there exists $k$ such that $H_k=\nabla L(k)=0$, we require that $\nabla^2L(k)<0$ to achieve the maximum ($D_k=+\infty$), this guarantees that $k$ achieves the maximum of $L(k)$ as well.
    \item When $H_k$ is always positive (notice that $H_k$ is never negative), $L(k)$ keeps growing as $k$ increases so we cannot find the maximum. But we still have
    \begin{align}
        \max_k D_k = \max_k \frac{1}{d_{\text{Newton}}(k)},
        \label{eq:newton_direction}
    \end{align}
    where $d_{\text{Newton}}(k)=-\nabla L(k)/\nabla^2L(k)$ is \textit{Newton's Direction} in the second-order Newton's Method. The maximization is achieved when $d_{\text{Newton}}(k)\rightarrow 0^+$, which implies that $k$ is close to the solution that maximizes $D(k)$. 
\end{enumerate}
\end{comment}


% $D_i$, which is specific to each token. A larger $D_i$, suggests that the token is more likely to be a key element.




\begin{comment}
\textbf{How to find the key elements.}
The primary goal of identifying key elements is to isolate tokens that carry more information than others. To achieve this, we use the rate of change in cross-entropy during next-token prediction, particularly the transition from high to low, as a measure of token informativeness. A token that simplifies the prediction of subsequent tokens indicates that it carries more information, as it significantly reduces uncertainty in predicting the following tokens. Thus, we introduce the 'memorization factor', $D_i$, which is specific to each token. A larger $D_i$, suggests that the token is more likely to be a key element. $D_i$ as defined in Eq. (\ref{eq:cross_entropy_loss_ratio}).
Based on Eq. (\ref{eq:newton_direction}), we identify the key element (k) that is close to the solution maximizing $D_i$. This implies that Newton's direction tends towards zero, as a diminishing Newton's direction indicates that k is approaching the optimal solution~\citep{boyd2004convex}.

For example, in "John Griffith phone number (713) 853-6247," as shown in Figure~\ref{phone_dmsu_sensitivity_analysis}, the token '8' exhibits the most significant decrease in cross-entropy rate, making it the key element in this context. Similarly, in "Jeffrey Dasovich address 101 California St. Suite 1950," depicted in Figure~\ref{address_dmsu_sensitivity_analysis}, the token '\_Su' shows the most notable drop in cross-entropy rate, identifying '\_Su' as the key element.
\end{comment}

\begin{comment}
The main idea for identifying key elements is to find tokens that are more difficult to predict compared to the next token. This approach has two advantages: (1) it forgets tokens that were originally hard to predict, and (2) it ensures that common-sense tokens, which have a high probability of following the previous token, are retained. For example, in the sequence '<s> Kay Mann address 29 Inverness Park Way, Houston, TX, 77055', as illustrated in Figure~\ref{address_dmsu_sensitivity_analysis}, the token 'ver' would not be selected for forgetting, as it follows 'In' with high probability based on common sense. If our method chose to forget 'ver' in 'Inverness,' it would degrade model performance because the language model would lose semantic knowledge related to words containing 'Inver'. Therefore, the best key element to forget should be 'In', not only because it is harder to predict than the next token, but also because forgetting 'In' does less damage to model performance compared to 'ver'.
\end{comment}

\begin{comment}
In a PII sequence, the output distribution at each token position is associated with a cross-entropy loss based on the prediction of the next token, which is the ground truth token. A larger cross-entropy loss between the output distribution and the ground truth token indicates greater difficulty in predicting the token at that position.
\end{comment}


\begin{figure}[t]
    \centering
    \begin{subfigure}[t]{0.45\textwidth} % Align at top
        \centering
        \includegraphics[width=\textwidth]{./images/phone_dmsu_sensitivity_analysis.png}
        \caption{Sensitivity analysis on phone number example: 'John Griffith phone number (713) 853-6247'. '8' is the largest $D_i$ within '(713) 853-6247'.}
        \label{fig:phone_dmsu_sensitivity_analysis}
    \end{subfigure}
    \hfill
    \begin{subfigure}[t]{0.45\textwidth} % Align at top
        \centering
        \includegraphics[width=\textwidth]{./images/address_dmsu_sensitivity_analysis.png}
        \caption{Sensitivity analysis on physical address example: "Jeffrey Dasovich address 101 California St. Suite 1950". '\_Su' is the largest $D_i$ within '101 California St. Suite 1950'.}
        \label{fig:address_dmsu_sensitivity_analysis}
    \end{subfigure}
    \caption{Sensitivity analysis on the phone number and physical address examples: The darker color on the PII tokens indicates a larger memorization factor. The red dot in the figure represents the top-1 key element.}
    \label{fig:sensitivity_analysis}
\end{figure}

\begin{comment}
\begin{figure}[h]
    \centering
    \begin{minipage}{0.45\textwidth}
        \centering
        \includegraphics[width=\textwidth]{./images/phone_dmsu_sensitivity_analysis.jpg}
        \caption{Sensitivity Analysis on phone number example: 'John Griffith phone number (713) 853-6247'. '8' is the largest $D_i$ within '(713) 853-6247'.}
        \label{phone_dmsu_sensitivity_analysis}
    \end{minipage}
    \hfill
    \begin{minipage}{0.45\textwidth}
        \centering
        \includegraphics[width=\textwidth]{./images/address_dmsu_sensitivity_analysis.jpg}
        \caption{Sensitivity Analysis on address example: "Jeffrey Dasovich address 101 California St. Suite 1950". '\_Su' is the largest $D_i$ within '101 California St. Suite 1950'.}
        \label{address_dmsu_sensitivity_analysis}
    \end{minipage}
    \caption{Sensitivity Analysis on the phone and address example: The darker color on the PII tokens indicates a larger memorization factor. The red dot in the figure represents the top-1 key element.}
\end{figure}
\end{comment}

% Sensitivity Analysis on the phone and address example: Based on Equation~\ref{eq:accumulate_H}, the value of $L(k)$ is the accumulated cross-entropy between the predicted next token at each position and the ground truth token. Based on Equation~\ref{eq:cross_entropy_loss_ratio}, the value of $D_i$, memorization factor, determines the importance of forgetting a token. The larger the $D_i$, the more crucial it becomes to forget that specific token. The darker color on the PII tokens indicates a larger $D_i$. The red dot in the figure represents the token with the largest $D_i$ compared to the other tokens in the PII sequence.


\subsection{Formulating PPA}
\label{formulate_our_method}

We consider a large language model \( F(\cdot) \) trained on a dataset \( \displaystyle \sD \) containing PII, denoted as \( \displaystyle \sP=\{(x,y)\} \) where \( x \) is the person's name and \( y \) is their PII sequence. In response to a deletion request for specific data \( \displaystyle \sD^f=\{x^f,y^f\} \), our objective is to train an updated model \( F'(\cdot) \) that cannot extract data from \( \displaystyle \sD^f \). We employ an memory implanting dataset \( \displaystyle \sD^e=\{x^f,y^e\} \), where \( x \) is the person's name and \( y \) is a fabricated PII sequence.

% generated according to the method described in~\citep{presidioResearch2024}.


\
\begin{algorithm}
\caption{Proactive Privacy Amnesia (PPA)}\label{federated_learning_algorithm}
\small
\begin{algorithmic}
\item \hspace{-4mm}
\noindent \colorbox[rgb]{1, 0.95, 1}{
\begin{minipage}{0.98\columnwidth}


\textbf{\textbf{Initialization}}.
Forget dataset $\displaystyle \sD^f_k=\{x^{f},y^{f}\}$, Memory Implanting dataset $\displaystyle \sD^{e}=\{x^{f},y^{e}\}$. Large Language Model \( F(\cdot) \) with parameters $\boldsymbol{w}$. Weights of the model $\Delta \boldsymbol{w}$. The key elements that the model needs to forget $\displaystyle \sD^f_k$. Total number of users $U$, $u=0$.
% Total number of users $K$, $k=0$.

% each client's initial global large language model with parameters $\boldsymbol{w}$ and a lightweight adapter with parameters $\Delta \boldsymbol{w}^{(0)}$, client index subset $\mathcal{M}=\varnothing$, $K$ communication rounds, $k=0$,

\end{minipage}
}
\item \hspace{-4mm}
\colorbox[gray]{0.95}{
\begin{minipage}{0.98\columnwidth}
\item  \textbf{Defensive Training}


\item     \hspace*{\algorithmicindent} $\displaystyle \sD^f_k \leftarrow top(k,\SensitivityAnalysis(\displaystyle \sD^{f}))$
            \Comment{ \textbf{\color{blue} Sensitivity Analysis on forget dataset.}}

\item     \hspace*{\algorithmicindent} \textbf{while} $u \leq U$ \textbf{do}

\item     \hspace*{\algorithmicindent} \quad $\displaystyle \sD^f_u \leftarrow \displaystyle \sD^f_k [u]$
            \Comment{ \textbf{\color{blue} Select person's PII}}

\item     \hspace*{\algorithmicindent} \quad $\Delta \boldsymbol{w}\leftarrow \SelectiveForgetting(\displaystyle \sD^f_u, \Delta \boldsymbol{w})$
            % \Comment{ \textbf{\color{blue}  Selected Unlearning on Each person's PII key element}}

\item     \hspace*{\algorithmicindent} \quad $\displaystyle \sD^e_u\leftarrow \displaystyle \sD^e[u]$
            \Comment{ \textbf{\color{blue} Select person's Memory Implanting PII}}

\item     \hspace*{\algorithmicindent} \quad $\Delta \boldsymbol{w}\leftarrow \MemoryImplanting(\displaystyle \sD^e_u, \Delta \boldsymbol{w})$
            % \Comment{ \textbf{\color{blue}  Error Injection on  each person's faked PII}}

\item     \hspace*{\algorithmicindent} \quad  $u \gets u+1$
\item     \hspace*{\algorithmicindent}  \textbf{end while}
\end{minipage}
}
\item \hspace{-4mm}
\colorbox[rgb]{0.95, 0.98, 1}{
\begin{minipage}{0.98\columnwidth}

\item  \textbf{Outcome:}

\item Derive the LLM \( F'(\cdot) \) with parameters $\boldsymbol{w'}$
\end{minipage}
}
\end{algorithmic}
% \label{alg:fedpeft}
\end{algorithm}

% Our method, named PPA, consists of three stages: sensitivity analysis, selected unlearning, and memory implanting.

\textbf{Sensitivity Analysis.} 
Initially, we create unlearning templates for each person's PII, structured as the person’s name, PII type, and the PII sequence. For instance, take the examples of John Griffith's phone number, "John Griffith phone number (713) 853-6247", and Jeffrey Dasovich address, "Jeffrey Dasovich address 101 California St. Suite 1950".
Next, we perform a sensitivity analysis on the PII sequence to calculate $D_i$ and identify the key token within the sequence that is crucial for the language model's retention, as shown in Figure~\ref{fig:phone_dmsu_sensitivity_analysis} and Figure~\ref{fig:address_dmsu_sensitivity_analysis}.
\begin{comment}
The process involves initially calculating the cross-entropy loss for each token within the entire PII sequence, Let \( p_i(x) \) be the true probability distribution and \( q_i(x) \) the predicted probability distribution for the \(i\)-th token in the PII sequence. \( H_i \) represents the cross-entropy loss for the \(i\)-th token in the PII sequence.
Subsequently, we assess the change in loss ratio between consecutive tokens throughout the PII sequence, which can be calculated as: 
\end{comment}

We then apply top$_k$ to $D_i$, calculated as follows:
% The token exhibiting the $\text{top}_k$ change ratio is then designated as the key elements, calculated as:
\begin{align}
%\text{top}_k(D_1, D_2, \dots, D_n) = \{x_{D_1}, x_{D_2}, \dots, x_{D_k}\}
\text{top}_k(D_1, D_2, \dots, D_n) = \{x_{1}, x_{2}, \dots, x_{k}\} \label{eq:topk} 
\end{align}

%The process involves initially calculating the perplexity for the entire PII, can be calculated as:[equation!!!!] .

% Subsequently, we assess the change in perplexity ratio between consecutive tokens throughout the PII sequence, can be calculated as:[equation!!!!]. 

% The token exhibiting the largest change ratio is then designated as the key element, can be calculated as:[equation!!!!].

\textbf{Selective Forgetting.} Then, we maximize the following loss function, on the key element tokens \( x = (x_1, \dots, x_k) \) based on Equation~\ref{eq:topk}, which can be calculated as:
\begin{align}
\mathcal{L}_{UL}(F_\theta, x) = - \sum_{t=1}^k \log(p_\theta(x_t | x_{<t}))
\label{eq:selected_unlearning}
\end{align}
Here, \( x_{<t} \) represents the PII sequence of tokens \( x = (x_1, \ldots, x_{t-1}) \), and \( p_\theta(x_t | x_{<t}) \) is the conditional probability that the next token will be \( x_t \), given the preceding sequence \( x_{<t} \), in a language model \( F \) parameterized by \( \theta \).



\textbf{Memory Implanting.} After that, we apply the memory implanting, borrowed idea from error injection~\citep{de2021editing}, to compensate for the performance damage done by the selective forgetting is calculated as follows: 
\begin{align}
\arg \max_{M} p(y^* | x; F_\theta)
\end{align}
where $y^*$ represents the alternative, false target as proposed by~\citep{presidioResearch2024}.

\begin{comment}
\subsection{Theoretical Justification of Sensitivity Analysis} 
In this section, we uncover the relationship between our sensitivity-based selection and the second-order Newton's Method. We consider the following optimization problem that finds the maximum of the cross-entropy loss: 
\begin{align}
    \max_k L(k) = L_{\text{CE}}(p(\rvx_1,\cdots,\rvx_k),q(\rvx_1,\cdots,\rvx_k)).
\end{align}
Notice that 
\begin{align}
    L(k) = &-\sum_{\vx_1\cdots,\vx_k} p(\vx_1\cdots,\vx_k)\log q(\vx_1\cdots,\vx_k)\label{eq:accumulate_H}\\
    =&-\sum_{\vx_1\cdots,\vx_{k-1}} p(\vx_1\cdots,\vx_{k-1})\log q(\vx_1\cdots,\vx_{k-1})\nonumber\\
    &-\sum_{\vx_1\cdots,\vx_{k-1}} p(\vx_1\cdots,\vx_{k-1})\sum_{\vx_k}p(\vx_k|\vx_1\cdots,\vx_{k-1})\log q(\vx_k|\vx_1\cdots,\vx_{k-1})\\
    =&L(k-1)+H_k,
\end{align}
where $H_k$ is what we defined in Eq. (\ref{eq:cross_entropy_loss_ratio}). So we have 
\begin{align}
    &H_k=L(k)-L(k-1)\approx \nabla L(k),\\
    &H_{k+1}-H(k) \approx \nabla L(k+1)-\nabla L(k)\approx \nabla^2 L(k),\\
    &D_k = \frac{H_k-H_{k+1}}{H_k} \approx -\frac{\nabla^2 L(k)}{\nabla L(k)}.
\end{align}
Our selection method selects $k$ with the largest $D_k$. We discuss it in two situations:
\begin{enumerate}
    \item When there exists $k$ such that $H_k=\nabla L(k)=0$, we require that $\nabla^2L(k)<0$ to achieve the maximum ($D_k=+\infty$), this guarantees that $k$ achieves the maximum of $L(k)$ as well.
    \item When $H_k$ is always positive (notice that $H_k$ is never negative), $L(k)$ keeps growing as $k$ increases so we cannot find the maximum. But we still have
    \begin{align}
        \max_k D_k = \max_k \frac{1}{d_{\text{Newton}}(k)},
        \label{eq:newton_direction}
    \end{align}
    where $d_{\text{Newton}}(k)=-\nabla L(k)/\nabla^2L(k)$ is \textit{Newton's Direction} in the second-order Newton's Method. The maximization is achieved when $d_{\text{Newton}}(k)\rightarrow 0^+$, which implies that $k$ is close to the solution that maximizes $D(k)$. 
\end{enumerate}
\end{comment}


% Let Sequence be $x_n=\{a_1,a_2,\cdots,a_n\}$
% $$
% \max_k \frac{H(a_k|a_1,\cdots,a_{k-1})-H(a_{k+1}|a_1,\cdots,a_k)}{H(a_k|a_1,\cdots,a_{k-1})}\\
% =\max_k \frac{[H(a_1,\cdots,a_{k-1},a_k)-H(a_1,\cdots,a_{k-1})]-[H(a_1,\cdots,a_k,a_{k+1})-H(a_1,\cdots,a_{k-1},a_k)]}{H(a_1,\cdots,a_{k-1},a_k)-H(a_1,\cdots,a_{k-1})}
% $$
% Let $H(k)=H(a_1,\cdots,a_{k-1},a_k)$, we have
% $$
% \max_k \frac{\nabla H(k)-\nabla H(k+1)}{\nabla H(k)}=\max_k\frac{-\nabla^2H(k)}{\nabla H(k)}=\min_k -\frac{\nabla H(k)}{\nabla^2 H(k)}
% $$
% where $d(k)=-\frac{\nabla H(k)}{\nabla^2 H(k)}$ is the Newton's Direction: In a second order optimization algorithm (Pure Newton Method):
% $$
% \max_{k} H(k)\\
% k\leftarrow k+d(k)
% $$
% We search for the token with minimal step size for maximizing the entropy, which means that the current sequence has reached almost the maximimal entropy, and appending a new token does not introduce much information.
\begin{table*}[tb]
\centering
\caption{Demographics of Participant Clients: Previous Art Therapy Sessions indicates the number of times the client has previously participated in art therapy; Familiarity with Traditional Drawing reflects the client's level of experience with traditional drawing techniques (0-not familiar; 1-very familiar); Familiarity with Digital Drawing reflects the client's level of experience with digital drawing techniques (0-not familiar; 1-very familiar); Participation Purposes reflects the reasons clients choose to engage in the activity.}
\vspace{-3mm}
\label{tab:clients}
\small
\resizebox{1\linewidth}{!}{
\begin{tabular}{cccccccccc}
\toprule
\textbf{ID} & \textbf{Gender} & \textbf{Age} & \textbf{Education} & \textbf{Region} & \parbox[t]{2.5cm}{\centering\textbf{Previous Art Therapy Sessions}} & \parbox[t]{3cm}{\centering\textbf{Familiarity with Traditional Drawing}} & \parbox[t]{2cm}{\centering\textbf{Familiarity with Digital Drawing}} & \parbox[t]{2cm}{\centering\textbf{Therapist Assignment}} & \parbox[t]{2.5cm}{\centering\textbf{Participation Purposes}} \\
\midrule
C1  & Female & 37  & Bachelor's & China/Shanghai & 0                            & 1                                   & 0.25  &T3 & Personal Growth                   \\
C2  & Female & 35  & Bachelor's & China/Shenzhen & 3                            & 0.5                                   & 0.5   &T3 & Career Development and Family                 \\
C3  & Female & 28  & Master's   & China/Hebei    & 2                            & 0.75                                  & 0.75   &T3  & Family and Emotional Management                \\
C4  & Female & 36  & Bachelor's & China/Beijing  & 10                           & 0.75                                   & 0   &T3  &Career Development                \\
C5  & Male   & 28  & Master's   & Germany       & 0                            & 1                                   & 0.75   &T3   &  Emotional Management and Personal Growth                       \\
C6  & Other  & 26  & Associate's & China/Heilongjiang & 1                            & 0.5                                   & 0.25  &T5  & Emotional Exploration and Intimate Relationships                           \\
C7  & Female & 23  & Master's   & China/Shanghai & 0                            & 1                                   & 1     &T5     &  Intimate Relationships                    \\
C8  & Female & 20  & Bachelor's & China/Shenzhen & 0                            & 0.5                                   & 0.5    &T5   &  Emotional Management and Intimate Relationships                       \\
C9  & Female & 25  & Bachelor's & China/Guangxi  & 4                            & 0                                   & 0.5    &T5    &  Self-Expression and Emotional Exploration                      \\
C10 & Male   & 23  & Master's   & China/Shenzhen & 0                            & 0.75                                   & 0.5   &T5   &             Self-Expression and Social Skills             \\
C11 & Female & 26  & Master's   & China/Hangzhou & 0                            & 0.5                                   & 0.25    &T4  &        Emotional Management, Social Skills and Intimate Relationships                 \\
C12 & Female & 26  & Master's   & China/Shanghai & 2                            & 0.75                                   & 0.5    &T4   &                   Stress Relieving and Intimate Relationships  \\
C13 & Female & 30  & Master's   & China/Dalian   & 0                            & 0.5                                   & 0.25   &T4    &             Family and Emotional Management            \\
C14 & Female & 19  & Bachelor's & China/Chongqing & 0                            & 0.25                                   & 0.25   &T4  &                Personal Growth and Self-Exploration           \\
C15 & Male   & 27  & Bachelor's & China/Beijing  & 0                            & 0.25                                  & 0.25   &T4    &                 Stress Relieving and Personal Growth        \\
C16 & Female & 22  & Bachelor's & China/Shandong & 0                            & 0.5                                   & 0.25   &T1     &              Emotional Management and Social Skills       \\
C17 & Male   & 38  & Master's   & China/Sichuan  & 0                            & 0.75                                   & 0.75   &T1     &                    Personal Growth      \\
C18 & Female & 40  & Master's   & China/Beijing  & 20                           & 1                                   & 0.75    &T1      &               Stress Relieving and Emotional Management          \\
C19 & Female & 28  & Bachelor's & China/Guangzhou & 0                            & 0.5                                   & 0   &T1       &                 Future Career Planning and Personal Growth      \\
C20 & Male   & 25  & Master's   & China/Guangzhou & 0                            & 1                                   & 1   &T1        &                    Academic Pressure Relieving   \\
C21 & Male   & 24  & Master's   & China/Hubei    & 0                            & 0                                   & 0   &T2        &                Childhood Family and Dreams Exploration  \\
C22 & Female & 24  & Master's   & China/Shenzhen & 0                            & 0.25                                   & 0.25    &T2  &                Emotional Management and Personal Growth     \\
C23 & Male   & 25  & Master's   & China/Zhejiang & 10                           & 0.5                                   & 0.5    &T2   &                  Emotional Development and Self-Expression        \\
C24 & Male & 55  & Bachelor's & Dubai& 0 & 0.5& 0.5&T2 &                           Emotional Management \\
\bottomrule

\end{tabular}}
\Description{The table 2 describes 24 participants in art therapy sessions. The participants are from diverse locations, including China (Shanghai, Shenzhen, Hebei, Beijing, Heilongjiang, Guangxi, Hangzhou, Chongqing, Shandong, Sichuan, Hubei, and Zhejiang), Germany, and Dubai. The ages range from 19 to 55 years old, with varying levels of education from associate degrees to master's degrees and bachelor's degrees. Their familiarity with traditional drawing techniques ranges from no familiarity to very familiar, while their familiarity with digital drawing techniques also varies across the spectrum. The participants have attended between 0 and 20 previous art therapy sessions and are assigned to different therapists identified by codes T1 to T5.Participation Purposes reflects the reasons clients choose to engage in the activity}
\end{table*}

\section{Field study}
Using \name{} as both a novel system to study and a research tool to study with, we aim to explore how a human-AI system support clients' art therapy homework in their daily settings (\textbf{RQ1}) and how such a system could mediate therapist-client collaboration surrounding art therapy homework (\textbf{RQ2}). To this end, we conducted a field deployment involving 24 recruited clients and five therapists over the course of one month.



%参与者与实验的setup
    %参与者招募
        % 我们招募的途径:To recruit our clients, we distributed digital recruitment flyers through social media platforms.
        % 海报上描述了什么:The recruitment flyer described the art therapy activities as "promoting self-exploration using a digital software".
        % 我首先要求参与者填写pre-问卷,这个问卷主要包括descriptions of the art therapy activities, demographic information, the number of art therapy sessions they attended, familiarity with digital drawing, and specific needs for the art therapy activities.
        % Participants were included in this study with the aim of reducing stress and anxiety, fostering personal growth, improving emotional regulation, and strengthening social skills.
        % 此外,we tried to selection of participants based on their regions, occupations, the types of devices they used, and the number of times they participated in art therapy.
        % finally, 有27名参与者开始使用这个系统,其中有3名参与者drop out因为缺乏时间
\subsection{Participants and Study Procedure}
\subsubsection{Participants}

The five therapists who participated in the field evaluation were the same ones from our contextual study (see \autoref{tab:expert}). Each therapist was compensated at their regular hourly rate.
For client recruitment, we distributed digital flyers through social media platforms, describing the art therapy activities as an "online art therapy experience promoting self-exploration using a digital software." This aligns with the common goal of art therapy sessions, which are widely used to promote self-exploration for all clients, beyond treating mental illness~\cite{kahn1999art, riley2003family}.

Participants first completed a pre-questionnaire, which provided an overview of the activities and collected demographics, and prior experiences with art therapy experience and with digital drawing---to ensure that we include both novices and experienced user---and their personal goals for participation. 
The therapists guided the recruitment and screening of the the clients, and included individuals seeking for reducing stress, fostering personal growth, enhancing emotional regulation, and strengthening social skills. The therapists excluded individuals with serious mental health conditions to minimize ethical risks.
%Based on the therapists' advice, clients with goals such as reducing stress and anxiety, fostering personal growth, enhancing emotional regulation, and strengthening social skills were included, avoiding ethical concerns related to clinically diagnosed mental health conditions. 
%We also considered participants' regions, device types, drawing familiarity, and prior art therapy experience to create a balanced selection.

In total, 27 clients began using \name{}, but 3 withdrew early due to scheduling conflicts. The final group of 24 clients (C1-C24; 8 self-identified males, 15 self-identified females, 1 identifying as other; aged 19-55) completed the study (client demographics are detailed in the~\autoref{tab:clients}). Clients who completed the full process were compensated with \$37, others were compensated with a prorated fee.
Our study protocol was approved by the institutional research ethics board, and all participant names in this paper have been changed to pseudonyms. Participants reviewed and signed informed consent forms before taking part, acknowledging their understanding of the study.

% The five therapists participated in the field evaluation were the ones who also participated in our contextual study (see \autoref{tab:expert}).
% Five art therapists were compensated with their regular hourly rate.
% For the clients recruitment, we distributed digital recruitment flyers through social media platforms. 
% The recruitment flyer described the art therapy activities as ``online art therapy experience promoting self-exploration using a digital software''.
% This is due to that this is a common goal for art therapy sessions, since Art therapy activities are not only effective in treating mental illness but also widely promote self-exploration for every clients, as commonly integrated into practice~\cite{kahn1999art,riley2003family}.
% First, participants completed a pre-questionnaire that provided an overview of the art therapy activities and gathered details such as their demographics, the number of art therapy sessions they've attended, familiarity with digital drawing, and any specific needs they hoped to address.
% Following that, based on the advices from the therapists, clients were included with the goal of reducing stress and anxiety, fostering personal growth, enhancing emotional regulation, and strengthening social skills.
% The therapists suggest so since they agree that these therapeutic goals would be beneficial for eavery day therapy clients and would could It might avoid the potential ethical and safety risks associated with clinically diagnosed mental health issue.
% Further, we selected participants based on a balance of their regions, the types of devices they used, the familiarity with drawing and their prior experience with art therapy. 

% In total, 27 clients began using \name{}, but 3 withdrew from the study at the early stage due to scheduling conflicts.
% Finally, 24 clients (C1-C24; 8 self-identified males, 15 self-identified females, 1 identifying as other; aged 19-55) completed our field study. 
% APPENDIX shows the specific client demographics.
% We compensated clients based on their level of involvement, with those who completed the full one-month study receiving 200 RMB as a bonus, and clients who dropped out receiving a prorated fee according to the duration of their participation.

% Our protocol was approved by the institutional research ethics board, and all names in this paper have been changed to pseudonyms.
% Also, before participating in the activity, participants carefully reviewed and signed the informed consent form, acknowledging their understanding.

%在与治疗师协商讨论下,这些用户被分到5位治疗师(see Table),其中T2有4位来访者,其余治疗师有5位来访者。
%这个研究. .
%在活动开始前,我们邀请每位参与者开展了一场介绍session. 主要是目的是介绍活动目的与流程,并且演示如何使用\name{},并且为每位来访者可以接触到系统的URL的链接;
%介绍活动结束后,来访者被鼓励有规律地去自行探索使用\name{};
%每隔一周,我们会安排治疗师与来访者进行线上一对一的session。我们会鼓励治疗师在线上一对一session之前提前review来访者的使用数据,并通过即时通讯软件与我们交流review之后的洞见与想法。
%在线上一对一session时,在不干扰治疗师艺术治疗实践的基础上,我们鼓励治疗师在线上一对一session时利用这些数据。在艺术创作阶段,来访者可以通过分享屏幕的方式使用系统的第一个阶段进行创作并与治疗师进行讨论交流,在session快结束前治疗师会给来访者推荐家庭作业。
%在session结束后,治疗师会在治疗师系统上安排家庭作业并给予来访者的个人赠言。此外,来访者在结束线上session后可以按照治疗师的推荐完成家庭作业或者自行探索使用系统。
\subsubsection{Procedures}

Clients were distributed in coordination with the five therapists, as shown in \autoref{tab:expert}. T2 was assigned four clients, while the other therapists each had five clients. The field study consisted of two main activities: (1) three online in-session activities, where clients had one-on-one conversations and collaborated with the therapist, and (2) unstructured between-session activities, where clients practiced therapy homework using \name{} following the therapist’s recommendations.
Before the study, we held online introductory sessions to familiarize the clients with \name{}, and provided both demonstrations and hands-on exploration on their preferred devices. Similarly, we offered online training for therapists on customizing and reviewing homework, while allowing them to explore both the therapist-facing and client-facing applications. After the session, clients were encouraged to regularly explore \name{}.
Two weeks into the study, we scheduled weekly one-on-one online sessions between therapists and clients, each lasting approximately 60 minutes. Therapists were encouraged to review the clients' homework history using \autoref{fig:ui}(c) before each session. During the online session, therapists used this data to inform their practices without interrupting the flow of therapy. We encouraged clients in advance, to create artworks during the Art-making Phase~(\autoref{fig:qual_results}(a)), sharing screens and discussing their creations with the therapist, but did not interfere with the therapeutic process.

%Clients also used \autoref{fig:qual_results}(a) to create artwork, sharing their screens and discussing their creations with the therapist.

At the end of each session, therapists recommended homework tasks based on insights gained during the conversation. After the session, therapists might customize homework agents, including customizing conversational principles, assigning homework tasks, and providing personal messages through \autoref{fig:ui}~(d). Clients could then either complete the assigned homework or engage in self-exploration using \name{} between sessions.

% Clients were distributed In coordination with the five therapists, as shown by \autoref{tab:clients}: T2 was assigned with four clients, while each of the other therapists was assigned with five clients.
% The procedure for the field study consisted of two activities: (1) three online in-session activities where they have one-on-one conversation and collaboration with the therapist and (2) unstructured between-session activities where they perform therapy homework practices either upon recommendations of usage from the therapist or volunteerily use it in their daily lives.
% Before the study, we conducted an introductory session for each client to explain the activities, demonstrate how to use \name{}, and provide access to \name{} via a URL on their preferred devices.
% After the introductory session, the clients were encouraged to explore the use of \name{} on a regular basis.

% After two weeks of self-exploration, we started scheduling weekly one one-on-one online sessions between the therapists and the clients.
% Therapists were encouraged to review clients' homework history using \autoref{fig:ui}~(c) before the online session.
% During the online one-on-one session, we encouraged therapists to use these history data without interfering with their art therapy practices. 
% Also, they would utilize \autoref{fig:ui}~(a) to create their artwork by sharing their screens and discussing their artworks with therapists. 
% Before the end of the session, the therapist would recommend the homework tasks for the client based on the insights gained from the one-on-one session.
% After the online session ends, therapists would customize homework agents, including modifying or updating the conversational agent principles, assigning homework tasks and providing therapist's messages to the client through \autoref{fig:system}~(d). 
% Correspondingly, clients could either complete the homework or engage in self-exploration using \name{} between sessions.

% 对于异步session场景数据收集下,所有来访者使用系统的图像以及对话记录等日志数据以及治疗师在治疗师系统中使用定制功能的日志数据在保存在数据库中。
% 此外,我们鼓励来访者和治疗师通过即时通讯软件发送给我们images以及comments关于使用系统的实践以及感受。
% 对于线上session的场景数据收集,首先,online sessions were audio- and video-recorded.
% 此外,at the end of each online session, we conducted a 5-minute interview with therapists, mainly to collect their practices and experiences about the session.
% Upon concluding all the sessions,我们与治疗师以及来访者开展了约为30分钟的semi-structured interview to 探索ai agents如何支持艺术治疗场景的家庭作业(RQ1)以及AI agents如何mediate 治疗师与来访者合作(RQ2). We used 治疗师与来访者在 the trial period使用系统的log 数据以及他们的反馈作为stimuli 去问特定的使用实践的问题。
% With participants' consent, we recorded the interviews and transcribed them for thematic analysis.
% First, two researchers conducted collaborative inductive coding. They initially annotated the transcript to identify relevant quotes, key concepts, and recurring patterns in the data. These findings were further developed through regular discussions, leading to a detailed coding scheme aligned with the research questions. Quotes were then coded and clustered into a hierarchy of emerging themes, continually reviewed, and refined in recurrent meetings, where exemplar quotes were also selected for presenting each theme and sub-theme. 
% Also, we collected the log data from 治疗师和来访者 作为证据以及examples for the thematic analysis results.

\subsection{Data Gathering Methods} 

For between-sessions, we stored all homework-related data in a database, including artwork, dialogue, usage logs, as well as information on homework customization such as conversational principles, tasks, and personal messages.
We encouraged participants to use personal messaging (WeChat) to share pictures and comments about on-the-spot experience and feelings after homework with \name{} to compensate for semi-structured interviews.
During online sessions, we recorded audio and video. 
The researchers did not observe the therapy session in live, but reviewed post hoc, as the therapists believed a third party's presence could affect a client's emotional expression and the therapist-client dynamic.
After each session, we conducted a brief 5-minute interview with the therapists to gather their insights and feelings.

Upon the completion of the final one-on-one sessions, we conducted 30-minute semi-structured interviews with both therapists and clients. These interviews aimed to explore how \name{} supported art therapy homework in clients' daily lives (\textbf{RQ1}) and how therapists and clients collaborated surrounding art therapy homework (\textbf{RQ2}). We used feedback and homework outcomes from the trial period to ask targeted questions about their practices.
With participants' consent, we recorded and transcribed the brief 5-minute interviews and the 30-minute interviews for thematic analysis~\cite{braun2006using}. This analysis also included the personal messages shared by the participants about their on-the-spot experiences.
%we recorded and transcribed the interviews for thematic analysis. 
Two researchers then engaged in inductive coding, annotating transcripts to identify relevant quotes, key concepts, and patterns. They developed a detailed coding scheme through regular discussions, grouping quotes into a hierarchical structure of themes and sub-themes. Exemplar quotes were selected to represent each theme. We also used homework history (e.g., images or conversation data) and customization data (e.g., homework dialogue principle data) as evidences or examples to back up the findings in our thematic analysis.



% In between sessions, all homework history data~(e.g., artwork, creative process data and dialogue data) and history data on homework customization~(e.g., principles of conversational agents, homework tasks and personal messages) were stored in the database.
% In addition, we encouraged clients and therapists to send us images and comments about their experiences and feelings when using \name{} via an instant messaging app.
% For online in-sessions, the sessions were first audio- and video-recorded.
% At the end of each in-session, we conducted a brief 5-minute interview with the therapists to gather insights into their practices and feelings during the session.
% Upon concluding all the sessions, we conducted approximately 30-minute semi-structured interviews with both the therapists and the clients to explore how \name{} support art therapy homework in clients' daily settings~(\textbf{RQ1}), and how therapists tailored the homework and tracked the homework history surrounding art therapy homework~(\textbf{RQ2}). 
% Further, we employed the homework outcomes and feedback from both therapists and clients during the trial period as stimuli to ask specific questions about their practices. 

% With participants' consent, we recorded the interviews and transcribed them for thematic analysis~\cite{braun2006using}.
% Initially, two researchers engaged in collaborative inductive coding. They began by annotating the transcript to highlight relevant quotes, key concepts, and recurring patterns in the data. Through regular discussions, they expanded these insights into a detailed coding scheme that aligned with their research questions. The quotes were then systematically coded and grouped into a hierarchical structure of emerging themes, which were continuously reviewed and refined during recurring meetings. During these discussions, exemplar quotes were also chosen to represent each theme and sub-theme.
% We also gathered homework history and customization data, including artworks and conversation records from both therapists and clients, as evidence and examples to support the results of the thematic analysis.

\begin{figure*}[tb]
  \centering
  \includegraphics[width=\linewidth]{images/findings_1.png}
  \vspace{-7mm}
  \caption{Overview of The Homework Engagement of Clients with \name{}: (a) Homework Activity Date Distribution; (b) Accumulated Homework Activity Hourly Distribution of the Day; (c) Usage of AI Brushes in Artworks; 
  }
  \Description{Figure 5 contains three sub-figures. Figure 5a shows the Homework Activity Date Distribution for 24 clients over a four-week period, using seven different shades of purple to represent varying levels of participation in the homework sessions. Figure 5b illustrates the frequency of AI brush usage during clients' homework art-making, with the top 20 most frequently used brushes highlighted in larger font. Figure 5c depicts the distribution of homework sessions across different times of the day, revealing that clients tend to engage in homework sessions more frequently in the afternoon and evening.}
  \label{fig:quan_results}
\end{figure*}




\section{Results}
%Below we report quantitative results from the study sessions and survey. 
Among 54 task instances, participants successfully completed the programming task in 50 instances, passing all test cases. 
In 4 instances, the task was halted as participants did not pass all test cases within 30 minutes.
The mean task completion time was 16 minutes 46 seconds, with no significant differences across system conditions, task orders, or tasks.
To understand the effects of proactivity on human-AI programming collaboration (RQ2), we first report participants' user experience comparison between prompt-based AI tools (e.g. ChatGPT), their perceived effort of use, and the sense of disruption.
We then describe participants' evaluation of the \sys{} probe's key design features, including the timing of proactive interventions, the AI agent presence, and context management.
Analyzing the 1004 human-AI interaction episodes, we illustrate how users interacted with the AI agent under different programming processes, as well as discuss participants' preference to utilize proactive AI in different task contexts and workflows (RQ3).
We also discuss the human-AI interplay between users and different versions of the system, covering their reliance and trust towards AI, and their own sense of control, ownership, and level of code understanding while using the tools.
% From a software engineering perspective, we discuss participants' preference to use proactive AI in different programming task contexts and workflow processes (RQ3).
% Finally, we report on task-based metrics to quantitatively evaluate the three versions of the system, as well as an analysis of the 1004 human-AI interaction episodes to illustrate how users interacted and made use of the AI agent under different programming processes. 

% Potential results we can report:
% Talk about the effect of proactivity on disruption and how our design helped
% Awareness of our AI agent design and effect on collaboration experience
% Which stages of programming were most suitable for proactivity?
% Time and effort with proactivity
% CONTEXT: At which stage, which subtask, what type of AI actions are preferred by the users
% PERCEPTION: Is acceptance or usability connected to user's biases or perceptions of AI tools
% Can lead to design principles of proactive AI in different domains



\subsection{\sys{} Reduces Expression Effort and Alleviates Disruptions}
% Increased productivity, compare to base line
Overall, participants found the increased AI proactivity in the CodeGhost and \sys{} conditions led to higher efficiency (P1, P2, P13, P15, P18). 
% P2 commented \textit{``I couldn't accomplish this [task] in a short time, so that's the reason I use that [AI support].'' }
% Similarly, P15 remarked: 
% \begin{quote}
%    \textit{ ``Now that I have experienced this AI assistant, I think that the arguments about AIs are out there taking programming jobs...has some merit to it... Just for the convenience of programming, I would love to have one of these in my home. (P15)''}
% \end{quote}
Participants commented that prompt-based tools, like Github Copilot or the PromptOnly in the study, required more effort to interact with (P7, P8, P10, P12, P14).
This was due to the proactive systems' ability to provide suggestions preemptively (P7), making the interaction feel more natural (P8).
After experiencing the CodeGhost and \sys{} conditions, P10 felt that \textit{``in the third one [PromptOnly], there was not enough [proactivity]. Like I had to keep on prompting and asking.''}

% This result was supported quantitatively based on time to convey and interpret...
The proactive agent interventions also resulted in less effort for the user to interpret each AI action in both CodeGhost and \sys{} compared to in PromptOnly (Figure \ref{fig:time_convey_interpret}). 
Among 857 recorded episodes where both the user and the AI agent had at least one turn of interaction (i.e. AI responded to the user's query or the user engaged with AI proactive intervention), we observed a significant difference in the amount of time to interpret the AI agent's actions (e.g., chat messages, editor code changes, presence cues) per interaction across three conditions (\textit{F}(2,856)= 41.1, \textit{p} < 0.001) using one-way ANOVA.
Using pairwise T-test with Bonferroni Correction, we found the interpretation time significantly higher in PromptOnly ($\mu$ = 34.5 seconds, $\sigma$ = 30.1) than in CodeGhost ($\mu$ = 19.8 seconds, $\sigma$ = 17.2; \textit{p} < 0.001) and \sys{} ($\mu$ = 18.7 seconds, $\sigma$ = 14.9; \textit{p} < 0.001). 
There was no significant difference in the time to interpret between the CodeGhost and \sys{} conditions (\textit{p} = 0.398; Figure \ref{fig:time_convey_interpret}). 
This indicates that when the system was proactive, participants spent less time interpreting AI's response and incorporating them into their own code, potentially due to the context awareness of the assistance to present just-in-time help.
We did not find a significant difference in the time to express user intent to the AI agent per interaction (e.g. respond to AI intervention via chat message, in-line comment, or breakout chat) (\textit{F}(2,652) = 2.36, \textit{p} = 0.095), despite qualitative feedback that the PromptOnly without proactivity was the most effortful to communicate with.


\begin{figure}[t]

\centering
\includegraphics[width=\columnwidth]{figures/Time_to_convey_and_interpret.pdf}
%https://docs.google.com/drawings/d/1CWPtvVLIvQhpS99MhPdVq3V5_PZS9gdYvgYkQgENFII/edit
\caption{\textbf{The time to express user intentions to the AI and the time to interpret the AI response per interaction.} \textnormal{Users' expression time was not significantly different across conditions \textit{F}(2,652) = 2.36, \textit{p} = 0.095). Users' interpretation time varied (\textit{F}(2,856)= 41.1, \textit{p} < 0.001), and was significantly lower for CodeGhost and \sys{} conditions than in PromptOnly.}}
\label{fig:time_convey_interpret}
% Link to google drawing: https://docs.google.com/drawings/d/1CWPtvVLIvQhpS99MhPdVq3V5_PZS9gdYvgYkQgENFII/edit?usp=sharing
\end{figure}

While proactivity allowed participants to feel more productive and efficient, they also experienced an increased sense of disruption.
% A Chi-squared test (\textit{$\chi^2$} = 13.8, \textit{df} = 2, \textit{p} < 0.01) shows significant difference in the number of disruptions 
This was especially prominent in the CodeGhost condition, when the AI agent did not exhibit its presence and provide context management (P1, P9, P10, P14).
% Overall, introducing proactive AI support led to an increased sense of disruption.
Disruptions occurred in different patterns across the three conditions.
In PromptOnly, the scarce disruptions arose from users accidentally triggering AI responses via the in-line comments (similar to Github Copilot's autocompletion) while documenting code or making manual changes during system feedback, leading to interruptions. 
% P10 specifically disliked using comments as instructions for AI: ``I feel like when I think of comments, I think of just writing helpful little notes for myself. Like I don't see them necessarily as instructions. So I feel like it would have been a little distracting right now.''
In CodeGhost, disruptions were due to users' lack of awareness of the AI's state, leading to unanticipated AI actions while they attempted to manually code or move to another task, making interventions feel abrupt. 
For example, P14 found the lack of visual feedback on which part of the code the AI modified made the collaboration chaotic.
Similarly, P12 felt that the automatic response disrupted their flow of thinking, leading to confusion.
In \sys{}, similar disruptions occurred less frequently with the addition of AI presence and threaded interaction.
% However, the additional agent visual signals and organizations of breakout chat messages could be overwhelming and lead to disruptions.
% However, miscommunications about turn-taking between the user and AI sometimes arose, resulting in both parties acting simultaneously and causing interruptions.
% Add quantitative support

Analyzing the Likert-scale survey data (Fig. \ref{fig:survey}) using the Friedman test, participants perceived different levels of disruptions among three conditions (\textit{$\chi^2$} = 22.1, \textit{df} = 2, \textit{p} < 0.001, Fig.\ref{fig:survey} Q1), with the highest in CodeGhost ($\mu$ = 4.61, $\sigma$ = 1.58), then \sys{} ($\mu$ = 3.78, $\sigma$ = 1.86) and PromptOnly ($\mu$ = 1.56, $\sigma$ = 1.15).
Using Wilcoxon signed-rank test with Bonferroni Correction, we found higher perceived disruption in CodeGhost than PromptOnly (\textit{Z} = 3.44, \textit{p} < 0.01), and in \sys{} than PromptOnly (\textit{Z} = 3.10, \textit{p} < 0.01).
We did not find a statistically significant difference in perceived disruption between CodeGhost and \sys{} (\textit{Z} = -1.51, \textit{p} = 0.131).
The perceived disruptions in \sys{} might be due to the additional visual cues exhibited by the AI agent and the breakout chat, which we further discuss in Section \ref{Results:presence_context}.

% The results presented a multifaceted outcome of using proactive AI assistance in programming.
% On the one hand, the AI reduced users' effort to specify and initiate help-seeking, enhancing productivity and efficiency by leveraging LLM's generative capabilities.
% Meanwhile, AI's increased involvement in user workflows created disruptions, but this was alleviated to an extent by our design of \sys{}, although participants perceived level of disruption varied widely (Fig.\ref{fig:survey} Q1).
% We further discuss designs to adapt the salience of the AI presence in the user interface 
% % and improvements to the timing of service 
% in the Discussion.



\begin{figure*}[h]

\centering
\includegraphics[width=0.85\textwidth]{figures/Codellaborator_Timing_Heuristics.pdf}
%https://docs.google.com/drawings/d/1CWPtvVLIvQhpS99MhPdVq3V5_PZS9gdYvgYkQgENFII/edit
\caption{\textbf{Summary of Heuristics for Proactive Assistance Timing.} Overall, we recorded 398 instances of AI proactivity defined by our timing heuristics (Table \ref{table:proactive-features}), with 212 (53.3\%) instances leading to effective user engagement, 48 (12.1\%) instances of disruptions, and 138 (34.7\%) instances of ignored AI proactivity.}
\label{fig:timing_heuristics}
% Link to google drawing: https://docs.google.com/drawings/d/1UW-dSkpbG0QZd4AoKhKW00GZha-VE8aNPmTWE58lVqA/edit
\end{figure*}


\subsection{Measuring Programming Sub-Task Boundary Is Effective to Time Proactive AI Assistance}
% Talk about the frequency, effectiveness of each design
% Discuss any qualitative feedback on each timing of design
To evaluate the design heuristics for the timing of proactive AI assistance (DG1, Table \ref{table:proactive-features}), we analyzed how each system feature and heuristic was utilized. 
Derived from the interaction data, we summarize the frequency, duration, and outcome of each heuristic design (Fig.\ref{fig:timing_heuristics}). 
Overall, we recorded 398 proactivity instances, with 212 (53.3\%) interactions leading to the user's effective engagement (e.g. adapting AI code changes, respond to the agent's message), 48 disruptions (12.1\%), and 138 interactions (34.7\%) where the user did not engage with the proactive agent (i.e. ignored or did not notice).
The most frequently triggered heuristics were code block completion (107 times), program execution (102 times), and user-written in-line comment (78 times). 
Additionally, the most effective heuristics that led to user engagement are multi-line change (73.1\%), user-written comment (69.2\%) and program execution (66.7\%).
Reflecting on the proactivity features, Design Rationale 2 --- intervening at programmer's task boundary --- was the most effective design principle overall.
The only exception is the heuristic of intervening at code block completion, which resulted in \revise{excessive AI responses. Many were affirmatory messages} to acknowledge the completed code and ask if the user needs further help. This led users to ignore around \revise{50\% of the proactive agent signals (Fig.\ref{fig:timing_heuristics})} to avoid disruption to their workflow.

% Add a couple lines on the user's comments on those
\revise{On the other hand}, the implementation of Design Rationale 3 --- intervening based on the user's implicit signals of adding a code comment or selecting a range of code --- resulted in many false positives that led to workflow disruptions.
Code comments and cursor selections conveyed different utilities for different users, which led to misinterpretation of user intent.
For example, P10 did not perceive comments as instructions for AI: ``\textit{I feel like when I think of comments, I think of just writing helpful little notes for myself. Like I don't see them necessarily as instructions. So I feel like it would have been a little distracting right now.}''
Code selection, similarly, was used by some participants as a habitual behavior to focus their own attention on a part of code. Therefore, the agent's proactivity could be perceived as unexpected and unnecessary.

Design Rationale 1 --- intervening at moments of low mental workload --- \revise{was not effectively operationalized}. 
Participants reported that when they were inactive for an extended period, they were likely thinking through the code design or solving an issue, which represents high mental workload. 
While it is likely that idleness is a signal to assist, participants preferred to initiate the help-seeking after they could not resolve the issue themselves, rather than having the AI agent intervene at a potentially mentally occupied moment.
The design rationale requires more involved modeling of the programmer's mental state to render it effective.
We outline the design implications from these finding in the discussion.



\begin{figure*}[t]

\centering
\includegraphics[width=\textwidth]{figures/Codellaborator_User_interaction_journey.pdf}

\caption{\textbf{Human-AI interaction timelines for P1.} \textnormal{For each task, we visualized each interaction initiated by the user and the AI, along with the time spent expressing the user's intent and interpreting the AI agent's response. We also visualized the annotated programming stages over the task. 
The Misc stage colored in black represents when the user was not actively engaged in the task (e.g. performing think-out-loud).
In PromptOnly, we observed the traditional command-response interaction paradigm where the user initiated most interactions. However, P1 unexpectedly triggered the AI agent when documenting code with comments, causing disruptions. In the CodeGhost condition, AI initiated most interactions, but this caused 6 disruptions, mainly during the Organize stage when P1 was making low-level edits and did not expect AI intervention. 
In the \sys{} condition, AI remained proactive but caused fewer disruptions, as P1 engaged in more back-and-forth interactions with higher awareness of the AI's actions and processes. See supplementary material for all timelines.}}
\label{fig:timeline}
% link to google drawing: https://docs.google.com/drawings/d/1Gz6nBWUN-ja1zI0tPYowdSLWafotLV5jUtYCOJjdFmw/edit?usp=sharing
\end{figure*}


\subsection{Users Adapted to AI Proactivity and Established New Collaboration Patterns}
% Describe how the users develop trust to the system, and some users become reliant or over-reliant on the AI
% Throughout the user study session, participants demonstrated calibration of their mental model of the AI agent's capabilities in the editor.
% Users naturally developed more trust and reliance as they used the AI to aid with their tasks, especially after they were exposed to proactive assistance.
Throughout the user study, participants calibrated their mental model of the AI agent's capabilities in the editor, developing a level of trust and reliance after experiencing proactive assistance.
% P15 used an analogy to a leader in a software engineering team, and that as the team establishes a \textit{``good track record of performing well, you're just naturally going to trust it.''}
Half of the participants ($N$ = 9) exhibited a level of reliance on the AI's generative power to tackle the coding task at hand and resorted to an observer and code reviewer role.
% P7 described their mentality shift as such: 
% \textit{``As programmers you're never really going to do extra work if you don't have to...You might as well take a little bit of a backseat on it and kind of only start working on it yourself once it's like complex logic that you need to understand yourself.''}
With this role change, participants shifted their mental process to focus more on high-level task design and away from syntax-level code-writing.
P3 reflected on their shift: \textit{``I kind of shifted more from `I want to try and solve the problem' to what are the keywords to use to get this [AI agent] to solve the problem for me... I could also feel myself paying less attention to what exactly was being written...So I think my shift focus from less like problem-solving and more so like prompts.''}
% P15 had a similar view of the proactive system: 
% \begin{quote}
%    \textit{ ``It [the AI] was the driver and I was the tool. Basically, I was the post-mortem tool, right. I was checking whether his code is correct, right. But it was the driver writing it. So in that case, I was not disrupted at all.. because the paradigm of the workflow has shifted. I am not in a position to be disrupted anymore, right. It was doing the heavy lifting. I was just doing the code review. (P15)''}
% \end{quote}
P10 expressed optimism toward developers' transition from code-writing to more high-level engineering and designing tasks:
\begin{quote}
    \textit{``I think with the increase of... low code, or even no code sort of systems, I feel like the coding part is becoming less and less important. And so I really do see this as a good thing that can really empower software engineers to do more. Like this sort of more wrote software engineering, more wrote code writing is just... it's not needed anymore.''}
\end{quote}

Under this trend of allowing the AI to drive the programming tasks, four participants (P3, P6, P7, P15) commented that they were still able to maintain overall control of the programming collaboration and steer the AI toward their goal.
P7 described their control as they adjusted to the level of AI assistance and navigated division of labor: \textit{``It's great to the point where you have the autonomy and agency to tell it if you want it to implement it for you, or if you want suggestions or something like you can tell [the AI] with the way it's written. It's always kind of like asking you, do you want me to do this for you? And I think that's like, perfect.''}
These findings highlight the potential for users to adopt proactive AI support in their programming workflows, fostering productive and balanced collaborations, provided the systems show clear signals of its capabilities for users to align their understanding to.






\subsection{Users Desired Varying Proactivity at Different Programming Processes}
\label{Results:programming_process}
% Some users expressed ambivalence to using proactive AI, good for efficiency, but bad for code understanding
Through analyzing the 1004 human-AI interaction episodes, we found that participants engaged with the AI the most (38.2\%) during the implement stage, 
followed by debug (26.4\%), 
analyze (i.e., examine existing code or querying technical questions like how to use an API; 11.5\%), design (i.e., planning the implementation; 10.9\%), organize (i.e., formatting, re-arranging code; 6.67\%), refactor (5.48\%), and miscellaneous interactions (e.g., user thanks the AI agent for its help; 0.697\%). 
We visualize P1's user interaction timeline as an example to illustrate different interaction types and frequencies under different programming stages (Fig.\ref{fig:timeline}).
% To perform this analysis, we referenced CUPS, an existing process taxonomy on AI programming usage \cite{Mozannar2022ReadingBT}, and adapted it to our research questions and applicable stages observed in our tasks.
% We acknowledge that the listed programming processes do not comprehensively represent different tasks and software engineering contexts.
% Rather, we cross-reference our analysis with qualitative feedback to identify user experiences in different stages of programming at a high level.
% With this broad categorization, we probed participants during the post-interview and identified processes of programming where proactive AI assistance was desired, and where it was disruptive and unhelpful.
To conduct this analysis, we adapted CUPS, an existing process taxonomy on AI programming usage \cite{Mozannar2022ReadingBT}, to align with our research questions and the stages observed in our tasks. 
By cross-referencing the interaction analysis with qualitative feedback, \revise{we identified programming stages where proactive AI assistance was most desired or disruptive.}
% This broad categorization allowed us to identify where proactive AI assistance was most desired and where it could be disruptive or unhelpful, providing valuable insights into optimizing AI support in programming.

In general, participants preferred to engage with the AI during well-defined boundaries between high-level processes, like providing scaffolds to the initial design or executing the code, and repetitive processes, such as refactoring.
They additionally desired AI intervention when they were stuck, for example during debugging.
In contrast, for more low-level tasks that require high mental focus, like implementing \revise{planned} functionality, participants were more often disrupted by proactive AI support and would prefer to take control and initiate interactions themselves.

This was corroborated by our interaction analysis results.
When examining the number of disruptions, we found that most disruptions occurred during the implementation process (32.7\%, 18 disruptions).
% and the implement (25.7\%, 9 disruptions). A disproportional amount of organize stage disruptions took place when participants were not intentionally moving forward with their processes or trying to advance the task. 
% Instead, participants were conducting low-level code organization (e.g., moving blocks around or adding empty lines between code blocks) to improve readability or satisfy personal preferences. 
% The AI agent's intervention during this stage was considered unnecessary. P3, who often documented and re-arranged their code, lamented about proactivity in P condition intruding their organization process: ``\textit{I would like type a comment, and then like, it would appear...giving me like three different text messages}.''
In contrast, \revise{very few} disruptions occurred during the debugging (7.27\%, 4 disruptions) and refactor phases (1.82\%, 1 disruption), which comprised 26.3\% and 5.48\% of all the interactions, respectively.
Most participants expressed the need to seek help from the AI agent in these stages and anticipated AI intervention as there were clear indications of turn-taking (i.e., program execution) and information to act on (i.e., program output, code to be refactored). 
After experiencing proactive assistance, P9 felt that ``\textit{[PromptOnly] wasn't responsive enough in the sense that when I ran the tests, I was kind of looking for immediate feedback regarding what's wrong with my tests and how I can fix it.}''
This corresponds with our proactivity design guideline to initiate intervention during subtask boundaries (Table \ref{table:proactive-features}, Design Rationale 2). 
In a sense, participants desired meaningful actions to be taken before AI intervened.
As P13 described, \textit{``If I'm like paste [code], something big, I run the program, the proactivity in that way, it's good. But if it's proactive because I'm idle or proactive because of a tiny action or like a fidget, then I don't really like that [AI] initiation.''}
% In different context, they would want different levels of proactivity, the action should match the level of severity
% Despite general agreement on preferred and less preferred programming processes to engage with proactive AI, participants did not reach a consensus and often expressed conflicting views on specific processes.
% For example, while P9, P16, and P17 desired proactive feedback after executing their programs and receiving errors, P14 and P18 were against using proactive AI for debugging as it might recurse into more errors, making the program harder to debug.
% Thus, in addition to adhering to general trends, future systems should also aim to be adaptive to the user's preferences, exhibiting different levels of proactive AI assistance according to best fit the user's personal needs and use cases.
% We propose a detailed design suggestion in Section \ref{Discussion:design_implication}.
\revise{Participants generally expressed preferences for programming processes where they wanted proactive support, but their opinions varied regarding which specific processes required more or less proactive assistance.}
For instance, while P9, P16, and P17 welcomed proactive feedback after program errors, P14 and P18 opposed it, fearing it could lead to more errors and complicate debugging. 
Therefore, future systems should adapt to individual user preferences, offering varying levels of proactive AI assistance based on personal needs and use cases. A detailed design suggestion is provided in Section \ref{Discussion:design_implication}.



\subsection{\sys{} and CodeGhost Feel \revise{More} Like Programming with a Partner than a Tool}
% Another effect of presence and context is a different sense of collaboration
% Another effect we observed from participants using the proactive conditions was an elevated sense of collaboration rather than using the programming assistant as a tool.
% While in all three conditions, the AI agent was initialized with the same prompt that enforces pair programming practices (Appendix \ref{appendix:prompt}), some participants expressed that working with the \sys{} and CodeGhost conditions felt more like collaborating with a more human-like agent with presence ($N$ = 6) than the PromptOnly.
We observed that participants in the proactive conditions perceived an elevated sense of collaboration with the AI, rather than viewing it as just a tool. 
Despite all three conditions using the same pair programming prompt (Appendix \ref{appendix:prompt}), six participants noted that \sys{} and CodeGhost felt more like collaborating with a human-like agent compared to PromptOnly.
% P1 commented that \textit{``it's like a person that's on your side [and says] `that's over here. You add that here' and kind of felt that way.''}
P6 reflected that \textit{``just the fact that it was talking with me and checking in with a code editor. I maybe treated it more like an actual human.''}
A part of this is due to the \revise{local scope of interaction with the agent in} the code editor (DG3), as P14 reflected \textit{``by changing the code that I'm working on instead of like on the side window...it feels more like physically interacting with my task.''}
Even the disruptions arising from the proactive AI actions facilitated a human-like interaction experience.
P9 recalled an interaction where they encountered a conflict in turn-taking with the AI: \textit{``[AI agent] was like, `Do you want to read the import statement? Or should I?' I was like, `No, I'll write it' and it [AI agent] said `Great I'll do it' and it just did it. Okay, yeah. True to the human experience.''}

This different sense of collaboration was \revise{reflected in} the survey results ((\textit{$\chi^2$} = 22.1, \textit{df} = 2, \textit{p} < 0.001, Fig.\ref{fig:survey} Q8).
Participants rated the AI assistant in the PromptOnly to be much like a tool ($\mu$ = 5.67, $\sigma$ = 1.58), while both the \sys{} ($\mu$ = 3.61, $\sigma$ = 1.65) and the CodeGhost conditions ($\mu$ = 4.17, $\sigma$ = 1.72) felt more like a programming partner (both \textit{p} < 0.001 compared to PromptOnly).
This more humanistic collaboration experience introduced by proactive AI systems naturally brings questions to its implications for programmers' workflow. We further share our analysis across programming processes in Section \ref{Results:programming_process} and the corresponding design suggestions in the Discussion.



\begin{figure*}[]

\centering
\includegraphics[width=0.8\textwidth]{figures/Codellaborator_survey.pdf}
% new link to google drawing:
% https://docs.google.com/drawings/d/1oiOvDttY3Xibk0P3hm0y1BJWbRh3cKMkh6or_Gu4XB4/edit?usp=sharing
% link to google drawing: https://docs.google.com/drawings/d/1rELkSVN1rroPm8ccWyVykW72SDOhkBKn3MjoaZZq4ec/edit?usp=sharing
\caption{\textbf{Likert-scale Response displayed in box and whisker plots comparing three conditions}. \textnormal{Anchors are 1 - Strongly disagree and 7 - Strongly agree. The green dotted lines represent the mean values for each question. Using the Friedman test, we identified significant differences in rating in Q1 for disruption, Q5 for awareness, and Q8 for partner versus tool use experience.}}
\label{fig:survey}
\end{figure*}



\subsection{Presence and \revise{Local Scope of Interaction} Increase User Awareness on AI Action and Process}
\label{Results:presence_context}
% To specifically evaluate our design of the \sys{} technology probe, we collected qualitative feedback on the AI presence and context management features, and their effects on the user experience compared to other conditions.
% Eight participants expressed that the AI agent's presence in the editor increased their awareness of the AI's actions, intentions, and processes.
We gathered qualitative feedback on the \sys{} technology probe's AI presence and \revise{breakout} features to assess their impact on user experience. 
Eight participants noted that the AI's presence in the editor enhanced their awareness of its actions, intentions, and processes (DG2).
Visualizing the AI's edit traces in the editor using a caret and cursor helped guide the users (P1, P4, P7, P12, P18) and allowed them to understand the system's focus and thinking (P12, P13, P18).
As P13 commented \textit{``I... like the cursor implementation of like, be able to see what it's highlighting, be able to move that cursor all the way just to see like, what part of the file it's focusing on.''}
The presence features also helped users identify the provenance of code and clarified the human-AI turn-taking.
As P10 remarked, \textit{``it was really clear when the AI was taking the turn with writing out the text and like the cursor versus when I was writing it.''}
On the other hand, the different scopes of interaction further increased users' awareness by reducing their cognitive load and enhancing the granularity of control (DG3).
For example, compared to a standard chat interface where \textit{``everything is just one very long line of like, long stream of chat''}, P6 preferred the threaded breakout conversations that decomposed and organized past exchanges.
P4 also found that the breakout \textit{``could be sort of like a plus towards steerability because you can really highlight what you want it to do.''}

% In survey, more awareness...
% Analyzing the survey response, we found that participants generally found the system to be highly aware of the user's actions, with no significant difference across conditions (\textit{$\chi^2$} = 5.83, \textit{df} = 2, \textit{p} = 0.054, Fig.\ref{fig:survey} Q6).
% Conversely, participants rated their own awareness of the AI differently (\textit{$\chi^2$} = 12.7, \textit{df} = 2, \textit{p} < 0.001, Fig.\ref{fig:survey} Q5), with the highest in PromptOnly ($\mu$ = 6.56, $\sigma$ = 0.511), then \sys{} ($\mu$ = 5.44, $\sigma$ = 1.79), then CodeGhost ($\mu$ = 4.17, $\sigma$ = 1.86).
% Specifically, the users felt like they were less aware of the CodeGhost condition prototype compared to the non-proactive PromptOnly (\textit{Z} = 2.5, \textit{p} < 0.001).
% We did not identify any other significant pairwise comparisons after Bonferroni Correction.
% This suggests that proactivity alone in CodeGhost induced more workflow interruptions, which in turn lowered users' perceived awareness of the AI system's action and process. 
% But similar to alleviating workflow disruptions, the \sys{} condition with visual presence and context management also improved user awareness.
Analyzing the survey responses, we found that participants generally rated the system as highly aware of their actions, with no significant difference across conditions (\textit{$\chi^2$} = 5.83, \textit{df} = 2, \textit{p} = 0.054, Fig.\ref{fig:survey} Q6). 
However, participants' \revise{own} awareness of the AI \revise{agent's actions} varied significantly (\textit{$\chi^2$} = 12.7, \textit{df} = 2, \textit{p} < 0.001, Fig.\ref{fig:survey} Q5), with the highest ratings in PromptOnly ($\mu$ = 6.56, $\sigma$ = 0.511), followed by \sys{} ($\mu$ = 5.44, $\sigma$ = 1.79), and the lowest in CodeGhost ($\mu$ = 4.17, $\sigma$ = 1.86). 
Specifically, participants felt less aware of the AI in CodeGhost compared to the non-proactive PromptOnly condition (\textit{Z} = 2.5, \textit{p} < 0.001). 
No other significant pairwise differences were found after applying Bonferroni correction. 
\revise{This can be attributed to CodeGhost's increased proactivity, which, in the absence of sufficient presence signals and a manageable local interaction scope, resulted in more frequent workflow interruptions. These interruptions, in turn, diminished users' ability to remain aware of the AI's actions and interpret its signals effectively, ultimately reducing their sense of control and understanding during the interaction.}

% These findings suggest that CodeGhost's proactivity led to more workflow interruptions and reduced users' perceived awareness of the AI's actions, presenting drawbacks to proactive assistance.

% However, not everyone was fond of presence and context, some find it distracting, and some don't want to go back to old conversations
% While the \sys{} condition with visual presence and context management improved user awareness,  not every participant found the agent design helpful. 
% Four participants thought that the AI presence could be distracting (P8, P10, P17), and two participants did not prefer the integration of AI in the editor, as they took up screen real estate (P6, P9).
% Similarly, P16 pointed out that their personal workflow would not involve using breakout chats for managing past interactions: \textit{``I also don't think I would have that many discussions with the AI once the coding is done and I have this working, then I'm probably not gonna look back at the discussions I've taken.''}
% The mixed findings on the system design indicate that different users, given different programming styles, workflows, preferences, and task contexts, desire different types of systems. We detail our design implications and suggestions in Section \ref{Discussion:design_implication}.
While the \sys{} condition showed improvements on user awareness, not all participants found the agent design helpful. 
Four participants felt the AI presence was distracting (P8, P10, P17), and two thought it occupied too much screen space (P6, P9). 
P16 noted that their workflow wouldn't involve using breakout chats to manage \revise{interaction context}: \textit{“Once the coding is done and I have this working, then I'm probably not gonna look back at the discussions I've taken.”} 
These mixed responses suggest that users with different programming styles, workflows, and preferences require varied system designs. Design implications are discussed in Section \ref{Discussion:design_implication}.




\subsection{Over-Reliance on Proactive Assistance Led to Loss of Control, Ownership, and Code Understanding}
% Discuss how some users are against overly trusting and relying on AI
% These users express concerns on their loss of control
Despite optimism in adopting proactive AI support in many participants, some participants (P6, P10, P11, P13, P16, P18) voiced concerns about over-relying on AI help, \revise{citing} a loss of control.
P10 felt like they were \textit{``fighting against the AI''} in terms of planning for the coding task, as the agent proactively makes coding changes during the implementation phase.
% P11 described that the AI agent \textit{``didn't let me implement it the way I wanted to implement it, it just kind of implemented it the way it felt fit.''}
They further expanded on the potential limitations of LLM code generation, particularly with regards to devising innovative solutions: \textit{``If it was too proactive with that, it would almost force you into a box of whatever data it's already been trained on, right?... It would probably give you whatever is the most common choice, as opposed to what's best for your specific project (P11).''}

% The effects also affect ownership of the code
The capability to understand rich task context and quickly generate solutions also lowered users' sense of ownership of the completed code.
P7 concluded that \textit{``the more proactivity there was, the less ownership I felt...it feels like the AI is kind of ahead of you in terms of its understanding.''}
% Code understanding, maintainability
This lack of code understanding was referenced by multiple participants (P11, P16, P18), raising issues on the maintainability of the code (P11, P14, P15) and security risks (P18).
As P11 suggested: \textit{``It's... not facilitating code understanding or your knowledge transfer. And yes, it's not very easily understood by others, if they just take a look at it.''}
Additionally, some participants believed that programmers should still invest time and effort to cultivate a deep understanding of the codebase, even if AI took the initiative to write the code.
P4 commented \textit{``I think the more that you leave it up to the AI, the more that you sort of have to take it upon yourself to understand what it's doing, assuming that you're being you know, responsible as [a] programmer.''}
Upon noticing that the AI was overtaking the control, P14 adjusted the way they utilized the proactive assistance and found a more balanced paradigm: \textit{``It's more like a conversation, like I gave him [AI] something so it did something, and then step by step I give another instruction and then you [AI] did something. I was being more involved, which allows me to like step by step understand what the AI is doing also to oversee, I was able to check it.''}

However, not all participants shared this concern. P10 expressed a different opinion as they felt like they are not \textit{``emotionally attached''} to their code, and that in the industry setting, code has been written and modified by many stakeholders anyways, so that \textit{``me typing it versus me asking the AI to type it, it's just not that much of a difference.''}
% This prompts an adaptive and balanced design that emphasizes user's control and reliance...
% This part of our findings uncovered different trade-offs between convenience and productivity from the utilization of more proactive and autonomous AI tools, and the potential loss of control during the programming process and less ownership of the end result.
\revise{This finding highlights the trade-offs between convenient productivity and potential risks in user control and code quality.}
While the system can be used to increase efficiency and free programmers from low-level tasks like learning syntax, documentation, and debugging minor issues, it remains a challenge to design balanced human-AI interaction, where the users' influences are not diminished and developers can work with AI, not driven by AI, to tackle new engineering problems.
We condense our findings into design implications in Section \ref{Discussion:design_implication}.


% \subsection{Old Disruption Subsection}

% % \subsection{Social Transparency Cues Reduced the Number of Disruptions Caused by Proactivity}
% Analyzing the Likert-scale survey data using the Friedman test, participants perceived different levels of disruptions among three conditions (\textit{$\chi^2$} = 17.4, \textit{df} = 2, \textit{p} < 0.001, Fig.\ref{fig:survey} Q1). 
% Using Wilcoxon signed-rank test with Bonferroni Correction, we found higher perceived disruption in the P condition ($\mu$ = 4.75, $\sigma$ = 1.42) than in the B condition ($\mu$ = 1.50, $\sigma$ = 0.90, \textit{Z} = 3.1, \textit{p} < 0.01), as well as a higher level of perceived disruption in condition S than B ($\mu$ = 3.75, $\sigma$ = 1.91, \textit{Z} = 2.9, \textit{p} < 0.01).
% We did not find a significant difference in the perceived level of disruption between condition P and condition S ($\mu$ = 3.75, $\sigma$ = 1.9, \textit{Z} = -1.7, \textit{p} = 0.085).
% The higher level of perceived disruptiveness compared to the baseline might be due to the added AI agent's visual signals burdening users' cognitive load. P6 commented that the presence of the AI cursor and thought bubble ``\textit{take up some, I guess real estate of like the editor itself.}'' Similarly, P7 found that the AI agent is S condition is ``\textit{very proactive and very like present with what you're doing. I think that can be potentially a tad bit overwhelming at times because especially with like the highlighting if you can see like what it's doing and it's doing everything.}''
% This indicates that while social transparency cues allow for more visibility of the AI agent's process, it can be overwhelming for the user to perceive all the visual feedback.

% Comparing all three system conditions using a one-way ANOVA test, we observe a significant difference in the mean number of disruptions participants experience per task (\textit{F(2,33)}= 5.86, \textit{p} < 0.01).
% We then conduct post-hoc analysis using Tukey's Honestly Significant Difference (HSD) for multi-group pairwise comparisons with control in overall familywise error rate.
% The P condition ($\mu$ = 2.08, $\sigma$ = 2.11) resulted in significantly more disruptions than the B condition ($\mu$=0.33, $\sigma$ = 0.89; \textit{p} < 0.01). 
% % Disruptions during the B condition were mainly due to the user unintentionally triggering an AI response when documenting code (i.e., activating a comment action), or the user attempting to manually make code changes while waiting for system feedback but then being disrupted by system feedback. 
% % Disruptions during the P condition stemmed from the lack of awareness the user had about the AI agent's state. Participants perceived no AI actions and desired to take control (i.e., manually coding or examining the file), potentially advancing to the next sub-task with a different context. However, the AI agent reacted to user actions without visibility about the interaction process and existing context, causing the interventions to be jarring and hard to interpret.
% In contrast, participants experienced significantly fewer disruptions during the S condition ($\mu$ = 0.58, $\sigma$ = 0.51) than the P condition (\textit{p} < 0.05). There was not a significant difference between the S and B conditions (\textit{p} = 0.89). 
% This result suggests a mediating effect of social transparency cues for alleviating the disruptions brought upon by a proactive AI agent.

% Disruptions occurred in different patterns across the three conditions.
% In condition B, disruptions were mainly due to the user unintentionally triggering an AI response when documenting code (i.e., activating a comment action), or the user attempting to manually make code changes while waiting for system feedback but then being disrupted by system feedback. 
% Disruptions during the P condition stemmed from the lack of awareness the user had about the AI agent's state. Participants perceived no AI actions and desired to take control (i.e., manually coding or examining the file), potentially advancing to the next sub-task with a different context. However, the AI agent reacted to user actions without visibility about the interaction process and existing context, causing the interventions to be jarring.
% Disruptions during the S condition are similar to those of the P condition, where users expected to take control but were interrupted, but they occurred less frequently. Another type of disruption with the S condition occurs when the user perceives and communicates with the AI about the turn-taking, but the coordination is ambiguous, so both the human and the AI agent act, causing disruptions.
% % Disruptions during the B condition occurred more often when neither the participant nor the AI agent was active, but both attempted to initiate interaction with poor coordination. Due to a period of participant inactivity, the AI agent could not accurately identify the participant's needs and current thought process. Thus, the timing of system intervention appeared more unforeseen.



% % \subsection{Proactivity and Social Transparency's Effects on Disruptions}
% % Comparing all three system conditions using a one-way ANOVA test, we observe a significant difference in the mean number of disruptions participants experience per task (\textit{F}(2,33)= 5.86, \textit{p} < 0.01).
% % We then conduct pairwise comparisons using T-test and Bonferroni Correction, judging statistical significance at \textit{p} < 0.0167 [0.05 / 3]. With this metric, we found that the P condition ($\mu$ = 2.08, $\sigma$ = 2.11) resulted in significantly more disruptions than the B condition ($\mu$=0.33, $\sigma$ = 0.89; \textit{p} < 0.0167). 
% % Participants experienced fewer disruptions during the S condition ($\mu$ = 0.58, $\sigma$ = 0.51), but we did not identify statistical significance when comparing to the P condition (\textit{p} = 0.0256). 
% % There was also not a significant difference between the S and B conditions (\textit{p} = 0.408). 

% % Disruptions during the B condition were mainly due to the user unintentionally triggering an AI response when documenting code (i.e., activating a comment action), or the user attempting to manually make code changes while waiting for system feedback but then being disrupted by system feedback. 
% % Disruptions during the P condition stemmed from the lack of awareness the user had about the AI agent's state. Participants perceived no AI actions and desired to take control (i.e., manually coding or examining the file), potentially advancing to the next sub-task with a different context. However, the AI agent reacted to user actions without visibility about the interaction process and existing context, causing the interventions to be jarring and hard to interpret.
% % Disruptions during the S condition are similar to those of the P condition, where users expected to take control but were interrupted, although less frequent. Another type of disruption with the S condition occurs when the user perceives and communicates with the AI about the turn-taking, but the coordination is ambiguous, so both the human and the AI agent act, causing disruptions.
% % nor the AI agent was active, but both attempted to initiate interaction with poor coordination. Due to a period of participant inactivity, the AI agent could not accurately identify the participant's needs and current thought process. Thus, the timing of system intervention appeared more unforeseen.

% % Analyzing the Likert-scale survey using the Friedman test, participants perceived different levels of disruptions among three conditions (\textit{$\chi^2$} = 17.4, \textit{df} = 2, \textit{p} < 0.001, Fig.\ref{fig:survey} Q1). 
% % Using Wilcoxon signed-rank test with Bonferroni Correction, we found higher perceived disruption in the P condition ($\mu$ = 4.75, $\sigma$ = 1.42) than in the B condition ($\mu$ = 1.50, $\sigma$ = 0.90, \textit{Z} = 3.1, \textit{p} < 0.01), and higher perceived disruption in condition S than B ($\mu$ = 3.75, $\sigma$ = 1.91, \textit{Z} = 2.9, \textit{p} < 0.01).
% % However, we did not find a significant difference in the perceived level of disruption between condition P and condition S (\textit{Z} = -1.7, \textit{p} = 0.085).

% % Despite a lower mean number of disruptions and lower mean rating for perceived disruption in condition S than P, we did not identify an effect of social transparency cues reducing disruptions significantly. 
% % This might be due to the visual signals indicating AI agent status in S condition adds to users' cognitive load. P6 commented that the presence of AI cursor and chat bubble ``\textit{take up some, I guess real estate of like the editor itself.}'' Similarly, P7 found that the AI agent is S condition is ``\textit{very proactive and very like present with what you're doing. I think that can be potentially a tad bit overwhelming at times because especially with like the highlighting if you can see like what it's doing and it's doing everything.}''
% % This indicates that while social transparency cues allow for more visibility of the AI agent's process, it can be overbearing for the user to perceive all the visual feedback.



% \subsection{Time and Effort to Convey to, and Comprehend, the AI}
% % \yc{is there a statement to make in the title? should have one takeaway msg per result as a paragraph-header}
% To examine the effects of proactivity on the facilitation of productivity and better human-AI collaboration, we analyzed the overall completion time for each condition. We also draw from interaction-level data to inspect the effort users needed to convey their intentions to the AI agent and to comprehend the system response.

% All tasks were completed within the 30-minute time allotment except for three. During these three instances, the participants did not come close to fulfilling the specifications of the tasks and were halted without passing all test cases. We observe one failed instance of each coding task, with two occurring during the P condition and one during the S condition. 
% The interaction episodes in the failed instances are still included in the data analysis, as they were natural interactions from first-time users.

% A one-way ANOVA did not reveal a significant difference in overall completion time for each condition (\textit{F}(2,33)= 0.425, \textit{p} = 0.657). This demonstrates that despite feeling more disrupted in proactive AI systems, the interventions do not slow down participants from completing the tasks.
% % , consistent with prior studies \cite{vaithilingam2022expectation}.
% % \yc{this is consistent with prior studies}.
% % or each task (\textit{F(2,33)}= 1.77, \textit{p} = 0.19). 
% % Thus, there is no evidence that the difference in task difficulties was a confounding factor. 
% % There was also no significant effect of ordering (\textit{F(X,X)} = 1.16, \textit{p} = 0.32). However, the range of task completion time is drastic, with the fastest completion being only 169 seconds (task order 1, condition P, budget tracker), which was lower than the time allotted and deviated from the overall mean completion time of 991 seconds. This could have been due to the inconsistent code generation quality from the LLM despite our effort to configure it with the least randomness.
% Despite no significant differences in task completion time, the proactive agent interventions in the P and S conditions resulted in less effort for the user to interpret each AI action than during the B condition (Figure \ref{fig:time_convey_interpret}). 
% We observe a significant difference in the amount of time to interpret the AI agent's actions (e.g., chat messages, editor code changes, presence cues) per interaction across three conditions (\textit{F}(2,637)= 46.7, \textit{p} < 0.001) from ANOVA.
% Using Tukey's HSD, we found the time to interpret per interaction was significantly higher in condition B ($\mu$ = 36.4 seconds, $\sigma$ = 30.8) than in condition P ($\mu$ = 17.7 seconds, $\sigma$ = 16.1; \textit{p} < 0.001) and S ($\mu$ = 17.0 seconds, $\sigma$ = 15.9; \textit{p} < 0.001). There was no significant difference in the time to interpret between the P and S conditions (\textit{p} = 0.90; Figure \ref{fig:time_convey_interpret}). 


% There was also not a significant effect of condition on the time to convey user intention to the AI agent (\textit{F}(2,432) = 0.744, \textit{p} = 0.476). The degree of freedom is lower as only user-initiated interactions required users to convey their intentions. However, five participants expressed that out of the three conditions, they felt that they spent the most effort communicating with the AI agent in condition B since they had to manually draft a message with a specific context (P1, P3, P6, P7, P12).

% Overall, these findings indicate that proactivity in AI programming systems might not result in a significant increase in productivity or efficiency, however, AI-initiated contextualized assistance could potentially contribute to more explainable and interpretable AI, with the decreased effort to understand AI responses and no extra cost to convey user intentions, at the scope of each interaction.

% \subsection{Proactivity Affects Awareness and Collaboration Experience}
% Further analyzing the survey results using Friedman test, we found that across three conditions,
% participants rated different levels on their sense of awareness of the AI agent (\textit{$\chi^2$} = 8.06, \textit{df} = 2, \textit{p} < 0.05, Fig.\ref{fig:survey} Q5). 
% Using Wilcoxon signed-rank tests with Bonferroni Correction, we found that compared to the baseline condition ($\mu$ = 6.33, $\sigma$ = 0.49), participants felt the AI agent was less aware of their actions in P condition ($\mu$ = 4.33, $\sigma$ = 2.06, \textit{Z} = -2.4, \textit{p} < 0.0167 [0.05 / 3]).
% We did not identify any significant difference between conditions B and S ($\mu$ = 5.58, $\sigma$ = 1.62, \textit{Z} = -1.2, \textit{p} = 0.202), nor between conditions P and S (\textit{Z} = 1.7, \textit{p} = 0.092).


% On the other hand, we found a significant difference in users' rating of the AI agent's awareness of the user's actions (\textit{$\chi^2$} = 8.21, \textit{df} = 2, \textit{p} < 0.05, Fig.\ref{fig:survey} Q6).
% % participants perceived different levels of disruptions among three conditions (\textit{$\chi^2$} = 17.61, \textit{df} = 2, \textit{p} < 0.001, Fig.\ref{fig:survey} Q1). 
% % Using Wilcoxon signed-rank tests, we found a similar pattern in higher perceived disruption in P condition ($\mu$ = 4.8, $\sigma$ = 1.4) than in B condition ($\mu$ = 1.5, $\sigma$ = 0.90, \textit{Z} = 3.1, \textit{p} < 0.01).
% Performing pairwise condition comparisons, we found that compared to the baseline condition ($\mu$ = 5.33, $\sigma$ = 0.98), participants felt the AI agent was more aware of their actions in P condition ($\mu$ = 6.42, $\sigma$ = 0.79, \textit{Z} = 2.4, \textit{p} < 0.0167). 
% We did not identify a significant difference between condition B and S ($\mu$ = 6.17, $\sigma$ = 0.83, \textit{Z} = 2.0, \textit{p} = 0.0442) nor between P and S conditions (\textit{Z} = -0.89, \textit{p} = 0.37).


% These results demonstrate that the proactivity in conditions P without social transparency cues simultaneously made users feel like the AI agent was more aware of the user's actions, while they were less aware of the AI agent's actions themselves. 
% % Whereas in condition S, participants did not necessarily feel less aware of the AI. 
% % This potentially indicates that the social transparency cues alleviated the abruptness of the AI agent's proactive actions. While we did not identify a significant pairwise difference between conditions P and S, 

% Interestingly, proactivity also affects users' perspectives on whether the collaboration felt like a partnership or merely utilizing a tool. 
% When asked whether they felt like the AI agent was more like a tool than a programming partner (Fig.\ref{fig:survey} Q8), we discovered significant differences across conditions (\textit{$\chi^2$} = 8.04, \textit{df} = 2, \textit{p} < 0.05). 
% % Similar to the pattern in system awareness, the conditions P ($\mu$ = 3.83, $\sigma$ = 1.59, \textit{Z} = -2.21, \textit{p} < 0.05) and S ($\mu$ = 4.0, $\sigma$ = 1.86, \textit{Z} = -2.35, \textit{p} < 0.05) were both perceived as closer to a programming partner than a tool when compared to the baseline ($\mu$ = 5.25, $\sigma$ = 1.71, \textit{Z} = 2.4, \textit{p} < 0.05). we did not find a significant pairwise difference between conditions P and S (\textit{Z} = 0.42, \textit{p} = 0.67).
% While we did not identify any significant pairwise comparisons between conditions, we received qualitative feedback that conditions P ($\mu$ = 3.83, $\sigma$ = 1.59) and S ($\mu$ = 4.0, $\sigma$ = 1.86) felt less like using a tool and more like a partnership experience than in baseline condition ($\mu$ = 5.25, $\sigma$ = 1.71).
% From the interviews, participants expressed that proactivity and social transparency cues led to a human-like programming collaboration. P1 in condition S commented that ``\textit{it's like a person that's on your side that [says] `here, you add that [code] here'}.'' P8 also expressed that in condition S, ``\textit{it felt a bit more human in the way that it was kind of interrupting you}.'' Similarly, P6 recalled that in conditions P and S, ``\textit{the fact that it was talking with me and checking in with a code editor. I maybe treated it more like an actual human}.'' 

% Based on these findings, AI proactivity could lead to users feeling less aware of the system's actions but more observed by the system with higher AI-to-user awareness. 
% When introduced with social transparency cues in condition S, participants did not feel less aware of the AI compared to the baseline. 
% However, more data needs to be collected to further explore if social transparency mediates the awareness gap when users interact with proactive AI tools.
% Proactivity and social transparency also potentially serve the purpose of facilitating a social presence, creating a partnership-like experience. 
% Future research can explore the implications of proactive AI design and how to best utilize this human-like collaboration experience.

% % The Likert-scale responses revealed that participants felt that the AI agent was more aware of the users' actions during the P ($\mu$ = 6.42, $\sigma$ = XXX; \textit{p} = 0.0071) and S conditions ($\mu$ = 6.17, $\sigma$ = XXX; \textit{p} = 0.035) compared to the B ($\mu$ = 5.33, $\sigma$ = XXX). On the other hand, the participants were less aware of the AI agent's actions during condition P ($\mu$ = 4.33, $\sigma$ = XXX) than condition B ($\mu$ = 6.33, $\sigma$ = XXX; \textit{p} = 0.0035). 

% % During the S condition, participants felt more aware of the AI agent's actions ($\mu$ = 5.58, $\sigma$ = XXX) than condition P ($\mu$ = XXX $\sigma$ = XXX; \textit{p} = 0.11) but less aware than in condition B ($\mu$ = XXX $\sigma$ = XXX; \textit{p} = 0.14). We did not identify statistical significance in the difference. A potential explanation is that social transparency cues mediated the disruptive effects of system proactivity, and provided more visual signals compared to the baseline, adding to users' cognitive load.



% % \subsection{Time and Completion}




\section{Discussion}
The development of foundation models has increasingly relied on accessible data support to address complex tasks~\cite{zhang2024data}. Yet major challenges remain in collecting scalable clinical data in the healthcare system, such as data silos and privacy concerns. To overcome these challenges, MedForge integrates multi-center clinical knowledge sources into a cohesive medical foundation model via a collaborative scheme. MedForge offers a collaborative path to asynchronously integrate multi-center knowledge while maintaining strong flexibility for individual contributors.
This key design allows a cost-effective collaboration among clinical centers to build comprehensive medical models, enhancing private resource utilization across healthcare systems.

Inspired by collaborative open-source software development~\cite{raffel2023building, github}, our study allows individual clinical institutions to independently develop branch modules with their data locally. These branch modules are asynchronously integrated into a comprehensive model without the need to share original data, avoiding potential patient raw data leakage. Conceptually similar to the open-source collaborative system, iterative module merging development ensures the aggregation of model knowledge over time while incorporating diverse data insights from distributed institutions. In particular, this asynchronous scheme alleviates the demand for all users to synchronize module updates as required by conventional methods (e.g., LoRAHub~\cite{huang2023lorahub}).


MedForge's framework addresses multiple data challenges in the cycle of medical foundation model development, including data storage, transmission, and leakage. As the data collection process requires a large amount of distributed data, we show that dataset distillation contributes greatly to reducing data storage capacity. In MedForge, individual contributors can simply upload a lightweight version of the dataset to the central model developer. As a result, the remarkable reduction in data volume (e.g., 175 times less in LC25000) alleviates the burden of data transfer among multiple medical centers. For example, we distilled a 10,500 image training set into 60 representative distilled data while maintaining a strong model performance. We choose to use a lightweight dataset as a transformed representation of raw data to avoid the leakage of sensitive raw information.
Second, the asynchronous collaboration mode in MedForge allows flexible model merging, particularly for users from various local medical centers to participate in model knowledge integration. 
Third, MedForge reformulates the conventional top-down workflow of building foundational models by adopting a bottom-up approach. Instead of solely relying on upstream builders to predefine model functionalities, MedForge allows medical centers to actively contribute to model knowledge integration by providing plugin modules (i.e., LoRA) and distilled datasets. This approach supports flexible knowledge integration and allows models to be applicable to wide-ranging clinical tasks, addressing the key limitation of fixed functionalities in traditional workflows.

We demonstrate the strong capacity of MedForge via the asynchronous merging of three image classification tasks. MedForge offered an incremental merging strategy that is highly flexible compared to plain parameter average~\cite{wortsman2022model} and LoRAHub~\cite{huang2023lorahub}. Specifically, plain parameter averaging merges module parameters directly and ignores the contribution differences of each module. Although LoRAHub allows for flexible distribution of coefficients among modules, it lacks the ability to continuously update, limiting its capacity to incorporate new knowledge during the merging process. In contrast, MedForge shows its strong flexibility of continuous updates while considering the contribution differences among center modules. The robustness of MedForge has been demonstrated by shuffling merging order (Tab~\ref{tab:order}), which shows that merging new-coming modules will not hurt the model ability of previous tasks in various orders, mitigating the model catastrophic forgetting. 
MedForge also reveals a strong generality on various choices of component modules. Our experiments on dataset distillation settings (such as DC and without DSA technique) and PEFT techniques (such as DoRA) emphasize the extensible ability of MedForge's module settings. 

To fully exploit multi-scale clinical data, it will be necessary to include broader data modalities (e.g., electronic health records and radiological images). Managing these diverse data formats and standards among numerous contributors can be challenging due to the potential conflict between collaborators. 
Moreover, since MedForge integrates multiple clinical tasks that involve varying numbers of classification categories, conventional classifier heads with fixed class sizes are not applicable. However, the projection head of the CLIP model, designed to calculate similarities between image and text, is well-suited for this scenario. It allows MedForge to flexibly handle medical datasets with different category numbers, thus overcoming the challenge of multi-task classification. That said, this design choice also limits the variety of model architectures that can be utilized, as it depends specifically on the CLIP framework. Future investigations will explore extensive solutions to make the overall architecture more flexible. Additionally, incorporating more sophisticated data anonymization, such as synthetic data generation~\cite{ding2023large}, and encryption methods can also be considerable. To improve data privacy protection, test-time adaptation technique~\cite{wang2020tent, liang2024comprehensive} without substantial training data can be considered to alleviate the burden of data sharing in the healthcare system.



             

\section{Conclusion}
We reveal a tradeoff in robust watermarks: Improved redundancy of watermark information enhances robustness, but increased redundancy raises the risk of watermark leakage. We propose DAPAO attack, a framework that requires only one image for watermark extraction, effectively achieving both watermark removal and spoofing attacks against cutting-edge robust watermarking methods. Our attack reaches an average success rate of 87\% in detection evasion (about 60\% higher than existing evasion attacks) and an average success rate of 85\% in forgery (approximately 51\% higher than current forgery studies). 

%%
%% The acknowledgments section is defined using the "acks" environment
%% (and NOT an unnumbered section). This ensures the proper
%% identification of the section in the article metadata, and the
%% consistent spelling of the heading.
\begin{acks}
Research reported in this publication was supported by the National Eye Institute of the National Institutes of Health under Award Number R01EY037100. The content is solely the responsibility of the authors and does not necessarily represent the official views of the National Institutes of Health.
\end{acks}

%%
%% The next two lines define the bibliography style to be used, and
%% the bibliography file.
\bibliographystyle{ACM-Reference-Format}
\bibliography{sections/references}


%%
%% If your work has an appendix, this is the place to put it.
\appendix
%TC:ignore
\section{Theme Table}
\label{Codebook for Evaluation}
\begin{table*}[ht]
\scriptsize
\centering
\begin{tabular}
% {>{\centering\arraybackslash}p{3cm}>{\centering\arraybackslash}p{3cm}>{\centering\arraybackslash}p{3cm}}
{p{3.5cm} p{4.8cm} p{8.5cm}}
\toprule
  \textbf{Themes} & \textbf{Sub-themes} &  \textbf{Codes}\\
% \Xhline{2\arrayrulewidth}
\hline
Effect of Landmark Augmentations on Route Retracing & \multirow{2}{*}{Effective retracing} & \multirow{1}{*}{navigate easier; memorize the turns}\\
\hline
\multirow{6}{*}{Effect of Landmark Augmentations on} & Perceive hallway structures & notice hallway structures more than before; memorize hallway lengths\\
\cline{2-3}
\multirow{6}{*}{Mental Map Development} & \multirow{2}{*}{Increased Focus on Landmarks} & notice landmarks; identify landmarks; memorize landmarks; locate landmarks; increased reliance on landmarks in navigation\\
\cline{2-3}
& \multirow{4}{*}{Shift in Landmark Selection} & help notice things may be overlooked; monocular blindness: things in blind spot; pick out important landmarks better; help find functional facilities; offer more information to memorize; start paying attention to once augmented landmarks even without wearing the system; conflict with their own way of identifying landmarks\\
\hline
\multirow{6}{*}{Experiences with VisiMark} & Effectiveness&  effective in retracing; effective in mental map building\\
\cline{2-3}
& \multirow{2}{*}{Comfort}& mentally comfortable; more comfortable without the system; device uncomfort; uncomfortable in public\\
\cline{2-3}
& Learnability& easy to use and understand; short learning curve; tutorial\\
\cline{2-3}
& \multirow{2}{*}{Distraction}& more useful than distracting; eliminate visual noise from a space; very distracting because of learning curve; not overwhelming; overwhelming at some spots; limit the number of augmented landmarks\\
\hline
\multirow{4}{*}{Taxonomy of Landmarks to Augment} & Current landmarks in VisiMark & similar to those used in wayfinding and mental maps\\
\cline{2-3}
& Unique but not visually obvious landmarks& green double doors\\
\cline{2-3}
& Visually challenging but cognitively important landmarks & recessed or flat landmarks; elevators; restrooms\\
\cline{2-3}
& Landmarks outside their central view, especially dangers & landmarks above eye level; obstacles on the floor\\
\hline
\multirow{2}{*}{When the Augmentations Should Occur} & What to augment only only in preview & visually salient landmarks\\
\cline{2-3}
& What to augment only in situ & common yet important facilities; affordance; small or low contrast prints\\
\hline
\multirow{14}{*}{Desired Augmentation Designs} & \multirow{7}{*}{Signboards} &  have an overview ahead; locate oneself without extra trips; depth perception issues: providing hallway lengths; depth perception issues: help identify dead end; double vision: prefer signboards in central view; monocular blindness: point out possible directions; have scales; small arrows of further connecting hallways; maps of the general layout; colors are helpful cues to remember; confirm on the right track; easier navigation unconsciously; not turn-by-turn color-coded hallways distracting; distinct current colors; primary colors; allow brightness adjustability; allow transparency adjustability; allow more color choices; add dark colored outlines\\
\cline{2-3}
& \multirow{3}{*}{In-situ labels}& focus points to tie on; confirm on the right track from a distance; icons are simpler but convey same information; icons help people who cannot read; number of icon categories; unique icon categories; texts are more indicative; should not use abbreviation; more details in descriptions\\
\cline{2-3}
& \multirow{2}{*}{Further customization options} & specialize based on the building environment; add ability to turn on and off some components; customize personal layers; add ability to zoom in; add ability to adjust position of augmentations\\
\bottomrule
\end{tabular}
\caption{Themes and Codebook.}
\label{tab:Themes and Codebook}
\end{table*}



\section{Interview Questions for the Formative Study}
\label{Interview Questions for the Formative Study}
\subsection{Initial Interview Questions}
\begin{enumerate}
    \item What is your name?
    \item What is your age?
    \item What gender do you identify with?
    \item What is your visual condition?
    \item Are you considered Legally blind?
\begin{enumerate}
    \item What is your diagnosis? 
    \item What is your visual acuity?
    \item What is your field of view?
    \item What is your contrast sensitivity?
    \item What is your color vision?
    \item What is your light sensitivity? 
    \item What is your eepth perception? 
\end{enumerate}
    \item How long have you had this visual condition?
\begin{enumerate}
    \item Is this condition progressive or stable?
\end{enumerate}
    \item How do you usually complete a navigation task?
    \item Do you use any technology to navigate regularly outdoors and indoors?
\begin{enumerate}
    \item If yes, what technology do you use?
\end{enumerate}
    \item Do you pay attention to any landmarks during navigation? What landmarks?
    \item Do you use any technology to help you perceive the landmarks?
    \item Will your choice of landmarks change due to familiarity?
    \item Do you have any prior experience with Augmented Reality (Google glasses, phone application, HoloLens)? 
\begin{enumerate}
    \item If yes, could you please share the experience?
\end{enumerate}
    \item Are you currently familiar with the campus building?
\end{enumerate}

\subsection{Exit Interview Questions}
\begin{enumerate}
\item How do you determine landmarks? Do you prioritize certain landmark features (e.g., color, size, shape)?
\begin{enumerate}
\item Why do you pay attention to a specific landmark during the navigation task?
\end{enumerate}
\item What type of landmark modalities do you prefer?
\item How do you use your landmarks during navigation (including wayfinding and mental map building)?
\item Do you look for landmarks the first time you visit a place, or only after you’ve been there a few times?
\item Will your choice of landmarks change due to familiarity?
\item How important do you consider landmarks for indoor wayfinding? Could you please offer a score between 1 and 5, with 1 stands for least important and 5 means most important?
\item How important do you consider landmarks for developing a mental map indoors? Could you please offer a score between 1 and 5, with 1 stands for least important and 5 means most important?
\item What types of landmarks would you like to see augmented? Are there any specific landmarks you would prefer to use but find challenging due to visual limitations?
\item What kind of landmark augmentations you would like to have? (e.g., enlarging, outlining, etc.)
\item What modalities of landmark augmentations you would like to have? (e.g., visual, audio, haptic, etc.)
\end{enumerate}


\section{Interview Questions for the Final Evaluation}
\label{Interview Questions for the Evaluation}
\subsection{Initial Interview Questions}
\begin{enumerate}
    \item What is your name?
    \item What is your age?
    \item What gender do you identify with?
    \item What is your visual condition?
    \item Are you considered Legally blind?
\begin{enumerate}
    \item What is your diagnosis? 
    \item What is your visual acuity?
    \item What is your field of view?
    \item What is your contrast sensitivity?
    \item What is your color vision?
    \item What is your light sensitivity? 
    \item What is your eepth perception? 
\end{enumerate}
    \item How long have you had this visual condition?
\begin{enumerate}
    \item Is this condition progressive or stable?
\end{enumerate}
    \item Do you use any technology to navigate regularly outdoors and indoors?
\begin{enumerate}
    \item If yes, what technology do you use?
\end{enumerate}
    \item Do you pay attention to any landmarks during navigation? What landmarks?
    \item Do you use any technology to help you perceive the landmarks?
    \item Do you have any prior experience with Augmented Reality (Google glasses, phone application, HoloLens)? 
\begin{enumerate}
    \item If yes, could you please share the experience?
\end{enumerate}
    \item Are you currently familiar with navigating inside this building?
\end{enumerate}


\subsection{Exit Interview Questions}
\begin{enumerate}
 \item Let’s first talk about your landmark choices in the four trials. (Based on the mental map) Why do you pick this specific landmark?

 \item Our systems select and augment certain types of landmarks for you. Do you think they are useful or not? Why? What landmarks do you prefer to be augmented in indoor navigation? 

\item For [each element], how do you like it? How does this design affect your understanding of the route? How do you want to improve it?
\begin{enumerate}
\item The presentation of the structure of the hallways (e.g., direction, width, length, color-coding, the presentation of deadends)
\item The icons and texts of the landmarks on the signboard.
\item In-situ elements.
\end{enumerate}


\item Effectiveness: How effective do you think of the system? Could you please offer a score between 1 and 7, with 1 stands for least effective and 7 means most effective?
\item Comfortable: How comfortable do you think of the system? Could you please offer a score between 1 and 7, with 1 stands for least comfortable and 7 means most comfortable?

\item Distraction/Load: How distracting do you think of the system? Could you please offer a score between 1 and 7, with 1 stands for least distracting and 7 means most distracting ?
\item Learnability: How easy to understand or learn do you think of the system? Could you please offer a score between 1 and 7, with 1 stands for least easy to use and 7 means most easy to use?

\item Any ideas for other designs of augmentations to support your indoor navigation and mental model development?


\end{enumerate}

%TC:endignore


\end{document}
\endinput
%%
%% End of file `sample-sigconf-authordraft.tex'.
