
\documentclass[11pt]{article}
\pdfoutput=1

\usepackage{mathrsfs}
\usepackage{amsmath,amssymb}
\usepackage{bm}
\usepackage{natbib}
\usepackage[usenames]{color}
\usepackage{amsthm}
\usepackage{algorithm}
\usepackage{algorithmic}

\usepackage{multirow} 
\usepackage{enumitem}

\usepackage{microtype}
\usepackage{graphicx}
\usepackage{subfigure}
\usepackage{booktabs,threeparttable,multirow}

\usepackage{amsmath}
\usepackage{amssymb,bm}
\usepackage{mathtools}
\usepackage{caption}
\usepackage{subcaption,color}
\usepackage{indentfirst}
\usepackage{amsfonts}
\usepackage{float,url}
\usepackage{bbm}
\usepackage{lipsum}
\usepackage{nicefrac}
\usepackage{pifont}


\usepackage[colorlinks,
linkcolor=red,
anchorcolor=blue,
citecolor=blue
]{hyperref}


\renewcommand{\baselinestretch}{1.05}


\usepackage{mylatexstyle}


\newcommand{\BILLE}{\makebox{\textsc{Bill-E}}\xspace}
\newcommand{\SOLLE}{\makebox{\textsc{Soll-E}}\xspace}
\newcommand{\ARMADAS}{\makebox{\textsc{Armadas}}\xspace}
\newcommand{\probName}{{\textsc{Single Robot Reconfiguration}}\xspace}

\newcommand{\configs}{\ensuremath{\mathcal{C}}}
\newcommand{\BigO}{\mathcal{O}}
\newcommand{\OPT}{\textsf{OPT}\xspace}
\newcommand{\size}[1]{|#1|}
\newcommand{\comp}{free component\xspace}
\newcommand{\comps}{free components\xspace}
\newcommand{\Comps}{Free components\xspace}
\newcommand{\bipGraph}[1]{G_{#1}}
\newcommand{\sumcomps}[1]{F_\Sigma}

\newcommand{\eDist}[1]{d_E(#1)}
\newcommand{\cDist}[1]{d_C(#1)}
\newcommand{\makesp}[1]{|#1|}
\newcommand{\weight}[1]{w(#1)}

\newcommand{\minMatching}{M_{\OPT}}

\newcommand{\lowerbound}{\ensuremath{D}}
\newcommand{\lowerboundOf}[1]{\ensuremath{\sigma(#1)}}

\newcommand{\disableaddcontentsline}{%
  \let\savedaddcontentsline\addcontentsline 
  \renewcommand{\addcontentsline}[3]{}
}
% recover \addcontentsline
\newcommand{\enableaddcontentsline}{%
  \let\addcontentsline\savedaddcontentsline
}

\usepackage{setspace}
%\setstretch{1.5}
\usepackage[left=1in, right=1in, top=1in, bottom=1in]{geometry}

\usepackage{xcolor}
\newcommand{\bin}[1]{\textcolor{blue}{[Bin: #1]}}

\ifdefined\final
\usepackage[disable]{todonotes}
\else
\usepackage[textsize=tiny]{todonotes}
\fi
\setlength{\marginparwidth}{0.8in}
\newcommand{\todoy}[2][]{\todo[size=\scriptsize,color=blue!20!white,#1]{Yuan: #2}}
\newcommand{\todoq}[2][]{\todo[size=\scriptsize,color=red!20!white,#1]{Quanquan: #2}}
\newcommand{\todoh}[2][]{\todo[size=\scriptsize,color=green!20!white,#1]{HZ: #2}}
\newcommand{\todot}[2][]{\todo[size=\scriptsize,color=green!20!white,#1]{YZ: #2}}

\newcommand{\yhzcomment}[1]{{\bf{{\color{orange}{{HZ {---} #1}}}}}}

% \def\CC{\textcolor{red}}
\def\CC{}


\title{\huge Re-examining Double Descent and Scaling Laws under Norm-based Capacity via Deterministic Equivalence}


\author
{
     Yichen Wang\thanks{Department of Computer Science, University of Warwick, UK. e-mail: {\tt yichen.wang7@warwick.ac.uk}} 
     ~~~
     Yudong Chen\thanks{Department of Computer Sciences, University of Wisconsin-Madison, USA. e-mail: {\tt yudong.chen@wisc.edu}}
      ~~~
     Lorenzo Rosasco\thanks{MaLGa Center – DIBRIS – Università di Genova, Genoa, Italy; also CBMM – Massachusets Institute of Technology, USA; also Istituto Italiano di Tecnologia, Genoa, Italy. e-mail: {\tt lrosasco@mit.edu}} 
~~~
    Fanghui Liu\thanks{Department of Computer Science, also Centre for Discrete Mathematics and its Applications (DIMAP), University of Warwick, UK. e-mail: {\tt fanghui.liu@warwick.ac.uk} (Corresponding author)} 
}



\date{}




% \def\CCC{\textcolor{red}}

% \def\CC{}




\begin{document}
\disableaddcontentsline




\maketitle


\begin{abstract}
We investigate double descent and scaling laws in terms of weights rather than the number of parameters. Specifically, we analyze linear and random features models using the deterministic equivalence approach from random matrix theory. We precisely characterize how the weights norm concentrate around deterministic quantities and elucidate the relationship between the expected test error and the norm-based capacity (complexity). Our results rigorously answer whether double descent exists under norm-based capacity and reshape the corresponding scaling laws. Moreover, they prompt a rethinking of the data–parameter paradigm—from under-parameterized to over-parameterized regimes—by shifting the focus to norms (weights) rather than parameter count.
\end{abstract}

%------------------------------------------------



% 
% 
The widespread integration of communication networks and smart devices in modern control systems has increased the vulnerability of industrial systems to online cyber-attacks, e.g., Industroyer, Blackenergy, etc \citep{osti_1505628}.
% Modern control systems have seen a large push to include communication networks and smart devices to increase performance, made possible by improvements in communication device cost and energy consumption. This trend has been coupled with the usage of open-standard communication protocols among industrial control systems, making them vulnerable to online cyber-attacks such as Industroyer, Blackenergy, etc \citep{osti_1505628}. 
To counter this, methods have been developed to improve security by achieving attack detection, mitigation, and monitoring, among others \citep{sandberg2022secure}. This paper focuses on active attack diagnosis to mitigate stealthy attacks. 
%
%\subsection{Literature review}

Active diagnosis techniques rely on the inclusion of additional moduli to control systems
% inclusion within the control system of additional moduli 
to alter the behavior of the system compared to information known by the attacker. 
For instance, the concept of additive watermarking was introduced in \cite{mo2015physical}, where noise signals of known mean and variance are added at the plant and compensated for it at the controller. 
This compensation, however, is not exact, causing some performance degradation. Thus, trade-offs between performance and detectability  are necessary \citep{zhu2023detection}.
% A later work \citep{zhu2023detection} designs the watermark signal by trading performance for detection. Thus, although additive watermarking serves as a good detection scheme, they endure performance losses even in the nominal case. 

In encrypted control \citep{darup2021encrypted}, the sensor data is encrypted, sent to the controller, and then operated on directly. Encrypted input signals are sent back to the plant for decryption. Although encryption is widespread in IT security, in control systems it presents some concerns, such as the introduction of time delays \citep{stabile2024verifiable}, while it may present inherent weaknesses \citep{alisic2023model}.
% they are not preferred as they introduce time delays \citep{stabile2024verifiable} which can cause instability, and some encryption schemes can be very weak  \citep{alisic2023model}. 

In moving target defense \citep{griffioen2020moving}, the plant is augmented with fictitious dynamics, known to the controller. The plant output is transmitted to the controller along with the fictitious states over a network under attack. 
The additional measurements then aide in the detection of attacks. 
This comes at the cost of higher communication bandwidth needs, which increases rapidly with the dimension of the augmented systems.
% Since the dynamics of the fictitious dynamics are exactly known to the controller, the attack is detected easily. However, when the scale of the system increases, the communication bandwidth used by moving the target defense approach increases rapidly. 

Other recently proposed works include two-way coding \citep{fang2019two}, a weak encryuption technique, and dynamic masking \citep{abdalmoaty2023privacy}, which enhances privacy as well as security, have been shown to be effective against zero-dynamics attacks.
% Two-way coding \citep{fang2019two} and dynamic masking \citep{abdalmoaty2023privacy} are other recently proposed approaches. Two-way coding is another form of weak encryption technique whilst dynamic masking proposes an architecture that enhances both privacy and security. These schemes are shown to be effective against zero dynamics attacks but remain to be studied for other classes of attacks. 
% Recent extensions include \citep{mukherjee2021secure,ramos2024privacy}.
% Some other works which are related are \citep{mukherjee2021secure}, an extension of \cite{fang2019two}. The work \citep{ramos2024privacy} is an extension of moving target defense for multi-agent systems. 
Furthermore, filtering techniques for attack detection are proposed by \cite{murguia2020security,hashemi2022codesign,escudero2023safety}, while not focusing on stealthy attacks.
% The works \citep{murguia2020security,hashemi2022codesign,escudero2023safety} develop filtering techniques to guarantee safety, without being focused on stealthy covert attacks.

Multiplicative watermarking (mWM) has been proposed by the authors as a diagnosis technique \citep{ferrari2020switching}. mWM consists of a pair of filters on each communication channel between the plant and its controller; the scheme is affine to weak encryption, whereby ``encoding'' and ``decoding'' are done by changing signals' dynamic characteristics through inverse pairs of filters. This enables original signals to be recovered exactly, and thus does not lead to performance degradation.
% A multiplicative watermark is an affine to a weak encryption technique, through which the signal is ``encoded'' by a filter, changing its dynamic behavior. The use of inverse pairs means that the original signal can be recovered, through ``decoding'' via an inverse filter. As such, differently to techniques based on additive watermarking, no performance is lost due to the injection of noise, and there are no bandwidth limitations.

%\subsection{Contributions}
One of the critical features of multiplicative watermarking is that to detect stealthy attacks, the mWM filter parameters must be switched over time. In this paper, an algorithm to optimally design the mWM parameters after a switching event is presented, enhancing detection performance, without changing the switching time.
% This is done without changing the switching time, which is taken as given.

\textcolor{black}{
To formalize the filter design problem, we suppose the defender is interested in optimal performance against adversaries injecting covert attacks with matched system parameters \citep{smith2015covert}, including the mWM parameters prior to the switch. This scenario represents a worst case where malicious agents can take full control of the system while remaining undetected.
Thus, the attack strategy is explicitly included within the formulation of the closed-loop system, and the mWM filters are chosen by solving an optimization problem minimizing the attack-energy-constrained output-to-output gain (AEC-OOG) \citep{anand2023risk}, a variation of the output-to-output gain proposed in  \cite{teixeira2015strategic}.
}
The main contributions of this paper are:
% We consider an adversary injecting a covert attack with matched system parameters \citep{smith2015covert}, i.e., an attacker with full knowledge of the control system parameters, including those of the mWM filters before the switch. This scenario is taken as a worst case, as it has been shown that this class of attacks can be made stealthy. To quantitatively define a cost, the output-to-output gain (OOG) \citep{teixeira2015strategic} is leveraged,
% a metric introduced to evaluate the impact of an additive attack in a control system. %Specifically, OOG evaluates the worst-case performance loss that an attacker injecting an undetectable attack can obtain. 
% Here, the maximum performance loss caused by a stealthy adversary with limited energy is taken, the attack-energy-constrained OOG (AEC-OOG) \citep{anand2023risk}. The main contributions of this paper are:
\begin{enumerate}
%[label=\alph*.]
\item The problem of optimally designing the switching mWM filters is formulated as an optimization problem, with the AEC-OOG is taken as the objective;%where the AEC-OOG is taken as the impact metric; 
\item The worst-case scenario of a covert attack with exact knowledge of plant and mWM filter parameters is embedded within the design problem;
% The optimization problem is defined to incorporate the worst-case scenario of a covert attack with exact knowledge of plant and mWM filter parameters;
\item The feasibility of the optimization problem is shown to be dependent only on stability conditions; 
\item A solution scheme is proposed to promote randomization of the mWM filter parameters such that an eavesdropping adversary cannot remain stealthy.
\end{enumerate} 

This builds on the results of \cite{ferrari2020switching}, where the focus was on the design of the switching protocols, rather than the parameters themselves.
Compared to previous work \citep{gallo2021design}, this paper introduces an optimization problem which is always feasible (thanks to the use of AEC-OOG in the objective), while also considering a more sophisticated class of covert attacks, where the presence of watermark is known to the adversary. 
Moreover, this paper poses a different objective than \citep{zhang2023hybrid}; indeed, while \citep{zhang2023hybrid} provided a design strategy to ensure certain privacy properties, in this paper we address the problem of optimal parameter design following a switching event.


%\subsection{Organization}
The rest of the paper is organized as follows. 
After formulating the problem in Section~\ref{sec:PF}, we propose our design algorithm in Section~\ref{sec:main}, and analyze its properties. It is then evaluated through a numerical example in Section~\ref{sec:NE}, and concluding remarks are given Section~\ref{sec:Con}.
% We provide the problem background in Section~\ref{sec:PF}. We formulate the design problem in Section~\ref{sec:main}, together with an analysis of its properties. The proposed algorithm is evaluated through a numerical example in Section \ref{sec:NE}. Concluding remarks are offered in Section \ref{sec:Con}.

\vspace{-0.cm}
In this section,  we introduce related concepts including GNN for recommendation, negative sampling, and Mixup \cite{mixup}.

\subsection{Graph Neural Networks for Recommendation}
 Recommendation is the most important technology in many e-commerce platforms, which has evolved from collaborative filtering to graph-based models. Graph-based recommendation represents all users and items by embedding and recommending items with maximum similarity score (by a inner product operation) for a given user. Here, we briefly describe the pipeline of GNN-based representation learning, including aggregation and optimization with negative sampling.

GNNs learn distributed vectors of nodes by leveraging node features and the graph structure. 
The neighborhood aggregation follows the ``message passing'' mechanism, which iteratively updates a node's embedding $h$ by aggregating the embeddings of its neighbors. Formally, the embedding $h_i^l$ of node $i$ in the $l$-th layer of GNN is defined as:
% GNNs use graph structures and node features to learn distributed vectors to represent graph information. Learning follows the "message passing" mechanism of neighborhood aggregation by iteratively updating a node's embedding $h$ by aggregating the embeddings of its neighbors. Formally, the representation $h_k^i$ of node $i$ in the $k$th layer of GNN is defined as:
 \begin{equation} \label{agg}
     \small
     h_i^l= \sigma\left(\text{AGG}\left(h_{i}^{l-1}, h_{j}^{l-1} \mid j \in N_{(i)},W_l\right)\right),
 \end{equation}
where the \(\sigma\) is activation function, $W_l$ denotes the trainable weights at layer l, $N_{(i)}$ denotes all nodes adjacent to $i$, $\text{AGG}$ is an aggregation function implemented by specific GNN model (\eg GraphSAGE, GCN, GAT, \etc), and $h_i^0$ is typically initialized as the input node feature $v_i$.

 
\subsection{Negative Sampling}

Negative sampling \cite{negsamp} is firstly proposed to serve as a simplified version of Noise Contrastive Estimation\cite{NCE}, which is an efficient way to compute the partition function of an unnormalized distribution to accelerate the training of Word2Vec\cite{word2vec}. The GNN has different non-Euclidean encoder layers with the following negative sampling objective:

\begin{equation}\label{negsampling}
    \mathcal{L} = \log(\sigma (e_{v_i}^Te_{v_p}))+\sum^{c}_{j=1}\mathbb{E}_{v_j\sim P_n(v)}  \log(1-\sigma (e_{v_i}^Te_{v_j})),
\end{equation}
where $v_i$ is a node in the graph, $v_p$ is sampled from the positive distribution of node $v_i$, $v_j$ is sampled from the negative distribution of node $v_i$, $e$ represents the embedding of the node, $\sigma$ represents the sigmoid function, $c$ represents the number of negative samples for each positive sample pair. 
%So Negative Sampling are free to simplify NCE as long as the vector representations retain their quality, it is an effective method to calculate the partition function of unnormalized distribution.


\subsection{Mixup}


\textbf{Mixup\cite{mixup}} is an simple yet effective data augmentation method that is originally proposed for image classification tasks. 
Mathematically, let $(x, y)$ denotes a sample of training data, where $x$ is the raw input samples and $y$ represents the one-hot label of $x$, the Mixup generates synthetic training samples $(\tilde{x}, \tilde{y})$  as follows:
\begin{equation}
% \vspace{-0.2cm}
\begin{split}
% \setlength\abovedisplayskip{0cm}
& \tilde{x}=\lambda x_{i}+(1-\lambda) x_{j}, \\
% \setlength\belowdisplayskip{0cm}
% \setlength\abovedisplayskip{0cm}
& \tilde{y}=\lambda y_{i}+(1-\lambda) y_{j}. \\
% \setlength\belowdisplayskip{0cm}
\end{split}
% \vspace{-0.2cm}
\end{equation}
It generates new samples by using linear interpolations to mix different images and their labels.

\section{Main results on linear regression}
\label{sec:linear}


\begin{figure*}[t]
    \centering
    \subfigure[Test Risk vs. $\gamma:=d/n$]{
        \includegraphics[width=0.22\textwidth]{arxiv_version/figures/Main_results_on_linear_regression/risk.pdf}
    }\label{fig:linear_regression_risk_vs_norm_1}
    \subfigure[$\ell_2$ norm vs. $\gamma$]{
        \includegraphics[width=0.22\textwidth]{arxiv_version/figures/Main_results_on_linear_regression/norm.pdf}
    }\label{fig:linear_regression_risk_vs_norm_2}
    \subfigure[Test Risk vs. $\ell_2$ norm]{
        \includegraphics[width=0.22\textwidth]{arxiv_version/figures/Main_results_on_linear_regression/risk_vs_norm.pdf}
    }\label{fig:linear_regression_risk_vs_norm_3}
    \subfigure[Risk vs. norm ($\lambda\!=\!0.05$)]{
        \includegraphics[width=0.22\textwidth]{arxiv_version/figures/Main_results_on_linear_regression/risk_vs_norm_single.pdf}
    }\label{fig:linear_regression_risk_vs_norm_4}
    \caption{Results for the ridge regression estimator. Points in these four figures are given by our experimental results, and the curves are given by our theoretical results via deterministic equivalents. Training data \(\{(\bx_i, y_i)\}_{i \in [n]}\), \(d = 1000\), sampled from a linear model \(y_i = \bx_i^\sT \bbeta_* + \varepsilon_i\), \(\sigma^2 = 0.0004\), \(\bx_i \sim \mathcal{N}(0, \bSigma)\), with \(\sigma_k(\bSigma)=k^{-1}\), \(\bbeta_{*,k}=k^{-\nicefrac{3}{2}}\).} 
    \label{fig:linear_regression_risk_vs_norm}
\end{figure*}

In this section, we study the non-asymptotic deterministic equivalent of the norm of the (ridge/ridgeless) estimator for linear regression. 
We also deliver the asymptotic results in \cref{app:asy_deter_equiv_lr}, which lays a foundation of asymptotic results for RFMs in \cref{sec:rff}.
Based on these results, we are able to mathematically characterize the test risk under norm-based capacity as shown in \cref{fig:linear_regression_risk_vs_norm}.

To deliver our results, we need the following lemma for the bias-variance decomposition of the estimator's norm.
\begin{lemma}[Bias-variance decomposition of $\mathcal{N}_{\lambda}^{\tt LS}$]
\label{lemma:biasvariance}
We have the bias-variance decomposition $\E_{\varepsilon}\|\hat{\bbeta}\|_2^2 =: \mathcal{N}_{\lambda}^{\tt LS} = \mathcal{B}^{\tt LS}_{\mathcal{N},\lambda} + \mathcal{V}^{\tt LS}_{\mathcal{N},\lambda}$, where $\mathcal{B}^{\tt LS}_{\mathcal{N},\lambda}$ and $\mathcal{V}^{\tt LS}_{\mathcal{N},\lambda}$ are defined as 
\[
\begin{aligned}
    \mathcal{B}^{\tt LS}_{\mathcal{N},\lambda} := \<\bbeta_*, (\bX^\sT\bX)^2(\bX^\sT\bX + \lambda\id)^{-2}\bbeta_*\>\,, \quad \mathcal{V}^{\tt LS}_{\mathcal{N},\lambda} := \sigma^2\Tr(\bX^\sT\bX(\bX^\sT\bX + \lambda\id)^{-2})\,.
\end{aligned}
\]
\end{lemma}

Our goal is to relate $\mathcal{B}^{\tt LS}_{\mathcal{N},\lambda}$ and $\mathcal{V}^{\tt LS}_{\mathcal{N},\lambda}$ to their respective deterministic equivalents defined below (proved in \cref{sec:linear_nonasym})
\begin{align}
    \sB_{\sN,\lambda}^{\tt LS} :=&~ \<\bbeta_*, \bSigma^2(\bSigma + \lambda_*\id)^{-2}\bbeta_*\> +{\color{red}\frac{\Tr(\bSigma(\bSigma + \lambda_*\id)^{-2})}{n}} \cdot \frac{\lambda_*^2 \langle \bbeta_*,\bSigma(\bSigma + \lambda_*\id)^{-2}\bbeta_*\rangle}{1-n^{-1}\Tr(\bSigma^2(\bSigma + \lambda_*\id)^{-2})} \,, \notag \\
    \sV_{\sN,\lambda}^{\tt LS} :=&~ \frac{\sigma^2\Tr({\color{blue}\bSigma}(\bSigma+\lambda_*\id)^{-2})}{n-\Tr(\bSigma^2(\bSigma+\lambda_*\id)^{-2})}\,. \label{eq:equiv-linear}
\end{align}
When checking \cref{eq:de_risk} and \cref{eq:equiv-linear}, we find that \textit{i)} the variance term is almost the same except that $\sV^{\tt LS}_{\sR,\lambda}$ has an additional $\bSigma$ ({\color{blue}in blue}). That means, under isotropic features $\bm \Sigma = \id_d$, they are the same.
\textit{ii)} For the bias term, we find that the second term of $\sB_{\sN,\lambda}^{\tt LS}$ ({\color{red}in red}) rescales $\sB_{\sR,\lambda}^{\tt LS}$ in \cref{eq:de_risk} by a factor $n^{-1}\Tr(\bSigma(\bSigma + \lambda_*\id)^{-2})$.

Accordingly, the norm-based capacity is able to characterize the bias and variance of the excess risk.
We will quantitatively characterize this relationship in \cref{sec:relationship_lrr}.


\subsection{Non-asymptotic analysis}
\label{sec:linear_nonasym}

To derive the non-asymptotic results, we make the following assumption on well-behaved data.
\begin{assumption}[Data concentration, \citealt{misiakiewicz2024non}]\label{ass:concentrated_LR} There exist $C_* > 0$ such that for any PSD matrix $\bA \in \mathbb{R}^{d \times d}$ with $\Tr(\bm{\Sigma A}) < \infty$ and $t\ge 0$, we have
    \[
    \begin{aligned}
         &~\mathbb{P}\left(\left| \bX^\sT \bA \bX - \Tr(\bm{\Sigma A}) \right| \geq t\|\bSigma^{1/2} \bA \bSigma^{1/2}\|_{\mathrm{F}} \right) \leq C_* e^{-\frac{t}{C_*}}\,.
    \end{aligned}
    \]
\end{assumption}


\begin{assumption}[Power-law assumption]\label{ass:powerlaw}
For the covariance matrix $\bm \Sigma$ and the target function $\bm \beta_*$, we assume that $ \sigma_k(\bm \Sigma) = k^{-\alpha}, \alpha >0$ and $ \bbeta_{*,k} =k^{-\nicefrac{\alpha\beta}{2}},  \beta \in \mathbb{R}$.
\end{assumption}

This assumption is close to classical source condition and capacity condition~\citep{caponnetto2007optimal} and is similarly used in \citet[Assumption 1]{paquette20244+}.

Based on the above two assumptions, we are ready to deliver the following result, see the proof in \cref{app:nonasy_deter_equiv_lr}.
\begin{theorem}[Deterministic equivalents of $\mathcal{N}_{\lambda}^{\tt LS}$, simplified version of \cref{prop:det_equiv_LR}, see \cref{fig:linear_regression_risk_vs_norm}]\label{prop:non-asy_equiv_norm_LR}
    Under \cref{ass:concentrated_LR} and \ref{ass:powerlaw}, for any $D,K >0$, if $\lambda > n^{-K}$, with probability at least $1-n^{-D}$, we have 
    \[
    \left|\mathcal{B}^{\tt LS}_{\mathcal{N},\lambda} - \sB_{\sN,\lambda}^{\tt LS}\right| \leq \widetilde{\mathcal{O}} (n^{-\frac{1}{2}}) \cdot \sB_{\sN,\lambda}^{\tt LS} \quad \text{and} \quad \left|\mathcal{V}^{\tt LS}_{\mathcal{N},\lambda} - \sV_{\sN,\lambda}^{\tt LS}\right| \leq \widetilde{\mathcal{O}} (n^{-\frac{1}{2}}) \cdot \sV_{\sN,\lambda}^{\tt LS}\,,
    \]
where these quantities are from \cref{lemma:biasvariance} and \cref{eq:equiv-linear}.
\end{theorem}
\noindent{\bf Remark:} Our results are numerically validated by \cref{fig:linear_regression_risk_vs_norm}. Besides, our theory is still valid under weaker assumptions related to \emph{effective dimension} used in \citet{misiakiewicz2024non} but the formulation will be more complex. We detail this in \cref{app:nonasy_deter_equiv_lr}.
The results for min-$\ell_2$-norm interpolator are given by \cref{prop:asy_equiv_norm_LR_minnorm}. We show that the solution $\lambda_n$ to the self-consistent equation $\Tr(\bSigma(\bSigma+\lambda_n\id)^{-1}) \sim n$ can be obtained from the variance $\sV_{\sN,0}^{\tt LS}=\sigma^2/\lambda_n$.


\subsection{Relationship between test risk and norm}
\label{sec:relationship_lrr}

Here we give some concrete examples on the relationship between $\sR$ and  $\sN$ in terms of isotropic features and power-law setting, see the proof in \cref{app:relationship}.

\begin{proposition}[Isotropic features for ridge regression, see \cref{fig:lampls}]\label{prop:relation_id}
    Consider covariance matrix $\bSigma = \id_d$, the deterministic equivalents $\sR^{\tt LS}_{\lambda}$ and $\sN^{\tt LS}_{\lambda}$ satisfy
    \[
    \begin{aligned}
        & \left(\|\bbeta_*\|_2^2 - \sR^{\tt LS}_{\lambda} - \sN^{\tt LS}_{\lambda}\right)\left(\|\bbeta_*\|_2^2 + \sR^{\tt LS}_{\lambda} - \sN^{\tt LS}_{\lambda}\right)^2d + 2\|\bbeta_*\|_2^2\left(\left(\|\bbeta_*\|_2^2 + \sR^{\tt LS}_{\lambda} - \sN^{\tt LS}_{\lambda}\right)^2-4\|\bbeta_*\|_2^2\sR^{\tt LS}_{\lambda} \right) \lambda\\
        &= 2\left( \left(\sR^{\tt LS}_{\lambda} - \sN^{\tt LS}_{\lambda}\right)^2 - \|\bbeta_*\|_2^4 \right) d \sigma^2\,.
    \end{aligned}
    \]
\end{proposition}
\vspace{-0.2cm}
\noindent{\bf Remark:} $\sR^{\tt LS}_\lambda$ and $\sN^{\tt LS}_\lambda$ formulates a third-order polynomial.
When $\lambda \to \infty$, it degenerates to $\sR^{\tt LS}_{\lambda} = (\|\bbeta_*\|_2 - \sqrt{\sN^{\tt LS}_{\lambda}})^2 $ when $ \sN^{\tt LS}_{\lambda} \leq \|\bbeta_*\|_2$.
Hence \( \sR^{\tt LS}_{\lambda} \) is monotonically decreasing with respect to \( \sN^{\tt LS}_{\lambda} \), empirically verified by \cref{fig:lampls}.
Besides, if we take $\lambda = \frac{d\sigma^2}{\|\bbeta_*\|_2^2}$, which is the {\bf optimal regularization parameter} discussed in \citet{wu2020optimal, nakkiran2020optimal}, the relationship in \cref{prop:relation_id} will become $\sR^{\tt LS}_{\lambda} = \|\bbeta_*\|_2^2 - \sN^{\tt LS}_{\lambda}$, which corresponds to a straight line. This is empirically shown in \cref{fig:lampls} with $\lambda=50$.
   

Apart from sufficiently large $\lambda$ and optimal $\lambda$ mentioned before, below we consider min-$\ell_2$-norm estimator. 
Note that when $\lambda \to 0$, the ridge regression estimator $\hat{\bbeta}$ converges to the min-$\ell_2$-norm estimator $\hat{\bbeta}_{\min}$.
However, the behavior of \(\lambda_*\) differs between the under-parameterized and over-parameterized regimes as \(\lambda \to 0\). In the under-parameterized regime, \(\lambda_*\) approaches 0, while in the over-parameterized regime, \(\lambda_*\) approaches a constant that satisfies \(\Tr(\bSigma(\bSigma + \lambda_n \id)^{-1}) = n\).
Thus, the min-\(\ell_2\)-norm estimator requires {\bf separate analysis of the two regimes}. 


\begin{figure}[t]
    \centering
    \subfigure[Test Risk vs.\ $\ell_2$-norm]{\label{fig:lampls}
        \includegraphics[width=0.3\textwidth]{arxiv_version/figures/Main_results_on_linear_regression/linear_regression_risk_vs_norm_id_multi_lambda.pdf}
    }
    \subfigure[Risk vs.\ norm ($\lambda \to 0$)]{\label{fig:lam0ls}
        \includegraphics[width=0.3\textwidth]{arxiv_version/figures/Main_results_on_linear_regression/risk_vs_norm_id_ridgeless.pdf}
    }
    \caption{Relationship between $\sR^{\tt LS}_\lambda$ and $\sN^{\tt LS}_\lambda$ in \cref{fig:lampls}; $\sR^{\tt LS}_0$ and $\sN^{\tt LS}_0$ in \cref{fig:lam0ls} under the linear model \(y_i = \bx_i^\sT \bbeta_* + \varepsilon_i\), with $d=500$, \(\bSigma = \id_d\), \(\|\bbeta_*\|_2^2=10\), and \(\sigma^2 = 1\).} %\fh{this is for isotropic data?} 
    \label{fig:linear_risk}\vspace{-0.2cm}
\end{figure}


\begin{proposition}[Relationship for min-$\ell_2$-norm interpolator in the {\bf under-parameterized} regime]\label{prop:relation_minnorm_underparam}
The deterministic equivalents $\sR^{\tt LS}_{0}$ and $\sN^{\tt LS}_{0}$, in under-parameterized regimes ($d < n$) admit the linear relationship
\[
    \begin{aligned}
        \sR^{\tt LS}_0 = {d}\left(\sN^{\tt LS}_0 - \|\bbeta_*\|_2^2\right)/{\Tr(\bSigma^{-1})}\,.
    \end{aligned}
\]
\end{proposition}

The relationship in the over-parameterized regime is more complicated. We present it in the special case of isotropic features in \cref{prop:relation_minnorm_id} of \cref{prop:relation_id}, and we also give an approximation in \cref{prop:relation_minnorm_pl} under the power-law assumption.


\begin{corollary}[Isotropic features for min-$\ell_2$-norm interpolator, see \cref{fig:lam0ls}]\label{prop:relation_minnorm_id}
    Consider covariance matrix $\bSigma = \id_d$, the relationship between $\sR^{\tt LS}_0$ and $\sN^{\tt LS}_0$ from under-parameterized to over-parameterized regimes admit
    \begin{equation*}
		\sR^{\tt LS}_0 = \left\{
		\begin{array}{rcl}
			\begin{aligned}
				&  \sN^{\tt LS}_0 - \|\bbeta_*\|_2^2\,,  ~~\text{if}~~ d<n ~\mbox{(under-parameterized)} ; \\
				& \sqrt{\left[\sN^{\tt LS}_0 - (\|\bbeta_*\|_2^2 - \sigma^2)\right]^2 + 4\|\bbeta_*\|_2^2 \sigma^2 } - \sigma^2 \,, \mbox{o/w}\,.
			\end{aligned}
		\end{array} \right.
    \end{equation*}
    For the variance part of $\sR^{\tt LS}_0$ and $\sN^{\tt LS}_0$, we have $\sV_{\sR,0}^{\tt LS} = \sV_{\sN,0}^{\tt LS}$; For the respective bias part, we have $\sB_{\sR,0}^{\tt LS} + \sB_{\sN,0}^{\tt LS} = \| \bm \beta_* \|_2^2$.
\end{corollary}
\noindent{\bf Remark:} 
In the under-parameterized regime, the test error $\sR^{\tt LS}_0$ is a linear function of the norm $\sN^{\tt LS}_0$. 
In the over-parameterized regime, $\sR^{\tt LS}_0$ and $\sN^{\tt LS}_0$ formulates a rectangular hyperbola: $\sR^{\tt LS}_0$ decreases with $\sN^{\tt LS}_0$ if $\sN^{\tt LS}_0 < \|\bbeta_*\|_2^2 - \sigma^2$ while $\sR^{\tt LS}_0$ increases with $\sN^{\tt LS}_0$ if $\sN^{\tt LS}_0 > \|\bbeta_*\|_2^2 - \sigma^2$.

Instead of assuming $\bSigma = \id_d$, we consider power-law features in \cref{ass:powerlaw} and characterize the relationship.
\begin{proposition}[Power-law features for min-$\ell_2$ norm estimator]\label{prop:relation_minnorm_pl}
    Under \cref{ass:powerlaw}, in the over-parameterized regime ($d>n$), we consider some special cases for analytic formulation: if $\alpha=1$, when $n \to d$, we have\footnote{The symbol $\approx$ here represents two types of approximations: i) approximation for self-consistent equations; ii) Taylor approximation of logarithmic function around zero (related to $n \to d$).}
    \[
    \begin{aligned}
        \sV_{\sR, 0}^{\tt LS} \approx \frac{2(\sV_{\sN, 0}^{\tt LS})^2}{d\sV_{\sN, 0}^{\tt LS}-d^2\sigma^2}\,, 
    \end{aligned}
    \]
    and further for different $\beta$, when $n \to d$, we have
    \begin{equation*}
		\sB_{\sR,0}^{\tt LS} \approx \left\{
		\begin{array}{rcl}
			\begin{aligned}
                    &  \frac{2\sB_{\sN, 0}^{\tt LS}(d-\sB_{\sN,0}^{\tt LS})}{d^2}\,,  ~~\text{$\beta=0$} ; \\
                    & \frac{2(\sB_{\sN, 0}^{\tt LS} - \Tr(\bSigma))}{d\sqrt{1+2\sB_{\sN, 0}^{\tt LS}-2\Tr(\bSigma)}} \,, ~~\text{$\beta=1$} ; \\
                    & \frac{216 (\sB_{\sN, 0}^{\tt LS})^4 \!-\! 324d^2 (\sB_{\sN, 0}^{\tt LS})^3 \!+\! 126d^4 (\sB_{\sN, 0}^{\tt LS})^2 \!+\! d^6 \sB_{\sN, 0}^{\tt LS} \!-\! 5d^8}{2d^5(6 \sB_{\sN, 0}^{\tt LS}-d^2)} \,, ~~\text{$\beta=-1$}.
			\end{aligned}
		\end{array} \right.
    \end{equation*}
\end{proposition}
\noindent{\bf Remark:}
The relationship between $\sR^{\tt LS}_0$ and $\sN^{\tt LS}_0$ is quite complex in the over-parameterized regime. We characterize some special cases here and find that they are still precise by our experiments in \cref{fig:linear_regression_power_law}.

\begin{figure*}[!ht]
    \centering
    \subfigure[\(\beta = 0\)]{\label{fig:lrpla}
        \includegraphics[width=0.22\textwidth]{arxiv_version/figures/Main_results_on_linear_regression/beta_0_B.pdf}
    }
    \subfigure[\(\beta = 1\)]{\label{fig:lrplb}
        \includegraphics[width=0.22\textwidth]{arxiv_version/figures/Main_results_on_linear_regression/beta_1_B.pdf}
    }
    \subfigure[\(\beta = -1\)]{\label{fig:lrplc}
        \includegraphics[width=0.22\textwidth]{arxiv_version/figures/Main_results_on_linear_regression/beta_-1_B.pdf}
    }
    \subfigure[$\sV_{\sR,0}^{\tt LS}$ vs. $\sV_{\sN,0}^{\tt LS}$]{\label{fig:lrpld}
        \includegraphics[width=0.22\textwidth]{arxiv_version/figures/Main_results_on_linear_regression/V.pdf}
    }
    \caption{\cref{fig:lrpla,fig:lrplb,fig:lrplc} show the relationship between $\sB_{\sR,0}^{\tt LS}$ and $\sB_{\sN,0}^{\tt LS}$ when $\alpha =1$ and $\beta$ takes on different values. \cref{fig:lrpld} shows the relationship between $\sV_{\sR,0}^{\tt LS}$ and $\sV_{\sN,0}^{\tt LS}$ when $\alpha = 1$. The {\color{blue}blue line} is the relationship obtained by deterministic equivalent experiments, and the {\color{red}red line} is the approximate relationship we give.}
    \label{fig:linear_regression_power_law}
\end{figure*}

\section{Main results on random feature regression}
\label{sec:rff}

In this section, we mathematically characterize \cref{fig:random_feature_risk_vs_norm} under norm-based capacity.
Similar to ridge regression, we have the following bias-variance decomposition for the norm.


\begin{lemma}[Bias-variance decomposition of $\mathcal{N}_{\lambda}^{\tt RFM}$]
\label{lemma:biasvariance_rf}
We have the bias-variance decomposition $\E_{\varepsilon}\|\hat{\ba}\|_2^2 =: \mathcal{N}_{\lambda}^{\tt RFM} = \mathcal{B}_{\mathcal{N},\lambda}^{\tt RFM} + \mathcal{V}_{\mathcal{N},\lambda}^{\tt RFM}$, where $\mathcal{B}_{\mathcal{N},\lambda}^{\tt RFM}$ and $\mathcal{V}_{\mathcal{N},\lambda}^{\tt RFM}$ are defined as 
\[
\begin{aligned}
    \mathcal{B}_{\mathcal{N},\lambda}^{\tt RFM} := \<\btheta_*, \bm{G}^\sT \bm{Z} (\bm{Z}^\sT \bm{Z} + \lambda\id)^{-2} \bm{Z}^\sT \bm{G}\btheta_* \>\,, \quad \mathcal{V}_{\mathcal{N},\lambda}^{\tt RFM} := \sigma^2\Tr\left(\bm{Z}^\sT \bm{Z}(\bm{Z}^\sT \bm{Z} + \lambda\id)^{-2}\right)\,.
\end{aligned}
\]
\end{lemma}

A main goal of this section is to prove that $\mathcal{B}^{\tt RFM}_{\mathcal{N},\lambda}$ and $\mathcal{V}^{\tt RFM}_{\mathcal{N},\lambda}$ admit the following deterministic equivalents, both  asymptotically (\cref{sec:linear_asym_rf}) and non-asymptotically (\cref{sec:linear_nonasym_rf}):
\begin{align}
    \sB_{\sN,\lambda}^{\tt RFM} :=&~ \frac{p\< \btheta_*, \bLambda ( \bLambda + \nu_2\id)^{-2} \btheta_* \>}{p - \Tr\left(\bLambda^2 (\bLambda + \nu_2\id)^{-2}\right)} + {\color{red}\frac{p\chi(\nu_2)}{n}} \notag \cdot \frac{\nu_2^2\left[ \< \btheta_*, (\bLambda + \nu_2\id)^{-2} \btheta_* \> + \chi(\nu_2) \< \btheta_*, \bLambda (\bLambda + \nu_2\id)^{-2} \btheta_* \> \right]}{1 - \Upsilon(\nu_1, \nu_2)}, \notag \\
    \sV_{\sN,\lambda}^{\tt RFM} :=&~ \sigma^2 \frac{{\color{blue}\frac{p}{n}\chi(\nu_2)}}{1-\Upsilon(\nu_1, \nu_2)}\,. \label{eq:equiv_random_feature}
\end{align}
When checking \cref{eq:de_risk_rf} and \cref{eq:equiv_random_feature}, we find that for variance, $\sB_{\sR,\lambda}^{\tt RFM}$ and $\sB_{\sN,\lambda}^{\tt RFM}$ only differ on the numerator, where $\Upsilon(\nu_1,\nu_2)$ is changed by $\frac{p}{n}\chi(\nu_2)$ ({\color{blue}in blue}).  For the bias term, we find that the second term of $\sB_{\sN,\lambda}^{\tt RFM}$ ({\color{red}in red}) rescales $\sB_{\sR,\lambda}^{\tt RFM}$ in \cref{eq:de_risk_rf} by a factor $\frac{p\chi(\nu_2)}{n}$.

Both of our asymptotic and non-asymptotic results are based on the following assumption on well-behaved data and features, but non-asymptotic results requires more technical assumptions we will deliver later.

\begin{assumption}[Concentration of the eigenfunctions \cite{defilippis2024dimension}]\label{ass:concentrated_RFRR} Recall the random vectors $\bpsi := (\xi_k \psi_k(\bx))_{k \geq 1}$ and $\bphi := (\xi_k \phi_k(\bw))_{k \geq 1}$. There exists $C_* > 0$ such that for any PSD matrix $\bA \in \mathbb{R}^{\infty \times \infty}$ with $\operatorname{Tr}(\bLambda \bA) < \infty$ and any $t\ge0$, we have
\[
\begin{aligned}
 & \mathbb{P} \left( \left| \bpsi^\sT \bA \bpsi - \Tr(\bLambda \bA) \right| \geq t \|\bLambda^{1/2} \bA \bLambda^{1/2}\|_F \right) 
\leq C_*  e^{-\frac{t}{C_*}}, \\
& \mathbb{P} \left( \left| \bphi^\sT \bA \bphi - \Tr(\bLambda \bA) \right| \geq t \|\bLambda^{1/2} \bA \bLambda^{1/2}\|_F \right) 
\leq C_*  e^{-\frac{t}{C_*}}.  
\end{aligned}
\]
\end{assumption}

This assumption requires well-behaved data---similarly to \cref{ass:concentrated_LR}---and additionally well-behaved random features. It holds for the classical sub-Gaussian case and log-Sobolev inequality or convex Lipschitz concentration \citep{cheng2022dimension}.

\vspace{-0.cm}
\subsection{Asymptotic deterministic equivalence}
\label{sec:linear_asym_rf}
\vspace{-0.cm}


Here we present the asymptotic results of $\E_{\varepsilon}\|\hat{\ba}\|_2^2$, see the proof in \cref{app:asy_deter_equiv_rf}.
\begin{proposition}[Asymptotic deterministic equivalence]\label{prop:asy_equiv_norm_RFRR}
    Given the bias-variance decomposition of $\E_{\varepsilon}\|\hat{\ba}\|_2^2$ in \cref{lemma:biasvariance_rf}, 
    under \cref{ass:concentrated_RFRR}, we have the following asymptotic deterministic equivalents $\mathcal{N}^{\tt RFM}_\lambda \sim \sN^{\tt RFM}_\lambda := \sB_{\sN,\lambda}^{\tt RFM} + \sV_{\sN,\lambda}^{\tt RFM}$ such that $\mathcal{B}^{\tt RFM}_{\mathcal{N},\lambda} \sim \sB_{\sN,\lambda}^{\tt RFM}$, $\mathcal{V}^{\tt RFM}_{\mathcal{N},\lambda} \sim \sV_{\sN,\lambda}^{\tt RFM}$, where $\sB_{\sN,\lambda}^{\tt RFM}$ and $\sV_{\sN,\lambda}^{\tt RFM}$ are defined by \cref{eq:equiv_random_feature}.
\end{proposition}
\cref{prop:asy_equiv_norm_RFRR} is numerically validated by \cref{fig:random_feature_risk_vs_norm}, supporting that $\E_{\varepsilon}\|\hat{\ba}\|_2^2$ is able to characterize the bias and variance of the excess risk.
We will quantitatively characterize this relationship in \cref{sec:relationship_rf}.

Similar to linear regression, we also need to analyze the under-/over-parameterized regimes separately for RFMs when $\lambda \rightarrow 0$.
In the under-parameterized regime, $\nu_1$ converges to $0$, and $\nu_2$ converges to a value $\lambda_p$ satisfying $\Tr(\bLambda(\bLambda + \lambda_p\id)^{-1}) = p$; while in the over-parameterized regime, $\nu_2$ converges to $\lambda_n$ satisfying $\Tr(\bLambda(\bLambda + \lambda_n\id)^{-1}) = n$, and $\nu_1$ converges to $\nu_2(1-\nicefrac{n}{p})$. 
We have the following result, see the proof in \cref{app:asy_deter_equiv_rf}. 


\begin{corollary}[Asymptotic deterministic equivalence of $\sN_{0}^{\tt RFM}$]\label{prop:asy_equiv_norm_RFRR_minnorm}
    Under \cref{ass:concentrated_RFRR}, for the min-$\ell_2$-norm estimator $\hat{\ba}_{\min}$, in the under-parameterized regime ($p<n$), we have
    \[
    \begin{aligned}
        \mathcal{B}^{\tt RFM}_{\mathcal{N},0} \sim \frac{p\<\btheta_*, \bLambda (\bLambda +\lambda_p\id)^{-2} \btheta_*\>}{n-\Tr(\bLambda^2(\bLambda +\lambda_p\id)^{-2})} + \frac{p\<\btheta_*, (\bLambda +\lambda_p\id)^{-1} \btheta_*\>}{n-p}\,, \quad 
        \mathcal{V}^{\tt RFM}_{\mathcal{N},0} \sim \frac{\sigma^2p}{\lambda_p(n-p)},
    \end{aligned}
    \]
    where $\lambda_p$ is from $\Tr(\bLambda(\bLambda+\lambda_p\id)^{-1}) \sim p$. In the over-parameterized regime ($p>n$), we have
    \[
    \begin{aligned}
        \mathcal{B}^{\tt RFM}_{\mathcal{N},0} \sim \frac{p\<\btheta_*, ( \bLambda + \lambda_n\id)^{-1} \btheta_*\>}{p-n}\,,
        \quad
        \mathcal{V}^{\tt RFM}_{\mathcal{N},0} \sim \frac{\sigma^2p}{\lambda_n(p-n)}\,,
    \end{aligned}
    \]
    where $\lambda_n$ is defined by $\Tr(\bLambda(\bLambda+\lambda_n\id)^{-1}) \sim n$.
\end{corollary}
\noindent{\bf Remark:} Notice that $\mathcal{V}^{\tt RFM}_{\mathcal{N},0}$ admits the similar formulation in under-/over-parameterized regimes but differs in $\lambda_n$ and $\lambda_p$. An interesting point to note is that, in the over-parameterized regime, $\lambda_n$ is a constant when $n$ constant. Therefore, $\mathcal{B}^{\tt RFM}_{\mathcal{N},0}$ and $\mathcal{V}^{\tt RFM}_{\mathcal{N},0}$ are proportional to each other.

\vspace{-0.cm}
\subsection{Non-asymptotic deterministic equivalence}
\label{sec:linear_nonasym_rf}
\vspace{-0.cm}

To present our non-asymptotic results, we additionally consider the following classical power-law assumption.

\begin{assumption}[Power-law, \citealt{defilippis2024dimension}]
\label{ass:powerlaw_rf}
    We assume that $\{ \xi_k^2\}_{k=1}^{\infty}$ in $\bLambda$ and $\btheta_*$ satisfy
    \[
    \xi_k^2 = k^{-\alpha}, \quad \theta_{\ast, k} = k^{-\frac{1 + 2\alpha\tau}{2}}\,, \mbox{with}~\alpha > 1,~ r>0\,.
    \]
\end{assumption}

The assumption coincides with the source condition $\|\bLambda^{-r} \btheta_*\|_2 < \infty$ ($r>0$) and capacity condition $\Tr(\bLambda^{1/\alpha}) < \infty$ ($\alpha > 1$) \citep{caponnetto2007optimal}.

We have the following non-asymptotic result on variance. 
\begin{theorem}[Non-asymptotic deterministic equivalents for variance, simplified version of \cref{prop:det_equiv_RFRR_V}]\label{prop:non-asy_equiv_norm_RFRR_V}
    Under \cref{ass:concentrated_RFRR} and \ref{ass:powerlaw_rf}, for any $D,K >0$, if $\lambda > n^{-K}$, then with probability at least $1-n^{-D}-p^{-D}$, we have
    \[
    \begin{aligned}
          \left|\mathcal{V}^{\tt RFM}_{\mathcal{N},\lambda} - \sV_{\sN,\lambda}^{\tt RFM}\right| \leq \widetilde{\mathcal{O}}(n^{-\nicefrac{1}{2}}+p^{-\nicefrac{1}{2}}) \cdot \sV_{\sN,\lambda}^{\tt RFM}\,.
    \end{aligned}
    \]
\end{theorem}
\noindent{\bf Remark:} Our results remain valid under weaker assumptions related to \emph{effective dimension} used in \citet{defilippis2024dimension}; see  \cref{app:nonasy_deter_equiv_rf}. 
We expect similar results to hold for bias as well, but additional technical assumptions and calculations may be needed; see more discussion in \cref{app:discuss_bias}.
We leave this to future work.  

\subsection{Relationship and scaling law}\label{sec:relationship_rf}

Here we discuss the risk-norm relationship for min-$\ell_2$-norm interpolator, and then build the scaling law under certain settings; see the proof in \cref{app:relationship_rf}.
\begin{proposition}[Relationship for min-$\ell_2$-norm interpolator in the {\bf over-parameterized} regime]\label{prop:relation_minnorm_overparam}
The deterministic equivalents $\sR^{\tt RFM}_{0}$ and $\sN^{\tt RFM}_{0}$, in over-parameterized regimes ($p>n$) admit the linear relationship due to $\lambda_n$ as a constant
% \begin{equation}\label{eq:rfflam0}
%   \sR_{0}^{\tt RFM} = \lambda_n\sN_{0}^{\tt RFM} + C_{n,\bLambda,\btheta_*,\sigma}\,,  
% \end{equation}
\begin{equation}\label{eq:rfflam0}
    \sR_{0}^{\tt RFM} 
    = 
    \lambda_n\sN_{0}^{\tt RFM} 
    - 
    \left[\lambda_n\<\btheta_*, ( \bLambda + \lambda_n\id)^{-1} \btheta_*\> + \sigma^2\right] 
    + 
    \frac{n\lambda_n^2 \<\btheta_*, ( \bLambda + \lambda_n \id)^{-2} \btheta_*\> + \sigma^2\Tr(\bLambda^2(\bLambda+\lambda_n\id)^{-2})}{ n - \Tr(\bLambda^2(\bLambda+\lambda_n\id)^{-2})}\,. 
\end{equation}
%where $C_{n,\bLambda,\btheta_*,\sigma}$ is a constant independent of $p$ but depending on $n$, $\bLambda$, $\btheta_*$ and $\sigma$, given in \cref{app:relationship_rf}.
\end{proposition}

The relationship in the under-parameterized regime is more complicated. We present it in the special case of isotropic features in \cref{prop:relation_minnorm_id_rf} and give an approximation in \cref{prop:relation_minnorm_powerlaw_rf} under the power-law assumption.

\begin{corollary}[Isotropic features for min-$\ell_2$-norm interpolator]\label{prop:relation_minnorm_id_rf}
    Consider covariance matrix $\bLambda = \id_m$ ($n<m<\infty$), in the over-parameterized regime ($p>n$), the deterministic equivalents $\sR^{\tt RFM}_0$ and $\sN^{\tt RFM}_0$ specifies the linear relationship in \cref{eq:rfflam0} as $\sR_{0}^{\tt RFM} = \frac{m-n}{n} \sN_{0}^{\tt RFM} +\frac{2n-m}{m-n} \sigma^2$.\\
While in the under-parameterized regime ($p<n$), we focus on bias and variance separately
    \[
    \begin{aligned}
     \mbox{Variance:}~ \left(\sV^{\tt RFM}_{\sR,0} \right)^2 = \frac{m-n}{n} \sV^{\tt RFM}_{\sR,0} \sV^{\tt RFM}_{\sN,0} + \frac{m \sigma^2}{n} \sV^{\tt RFM}_{\sN,0}\,,
    \end{aligned}
    \]
    \[
    \begin{aligned}
      \mbox{Bias:}~  &~(m-n)\sB^{\tt RFM}_{\sN,0}(m\sB^{\tt RFM}_{\sR,0}-n\|\btheta_*\|_2^2)(m(\sB^{\tt RFM}_{\sR,0})^2-n\|\btheta_*\|_2^4)\\
        &= nm(\sB^{\tt RFM}_{\sR,0} -\|\btheta_*\|_2^2)^2[m(\sB^{\tt RFM}_{\sR,0})^2 + n\|\btheta_*\|_2^2\sB^{\tt RFM}_{\sR,0} - 2n\|\btheta_*\|_2^4].
    \end{aligned}
    \]
\end{corollary}

\noindent{\bf Remark:} 
In the under-parameterized regime, $\sV^{\tt RFM}_{\sR,0}$ and $\sV^{\tt RFM}_{\sN,0}$ are related by a hyperbola, the asymptote of which is $\sV^{\tt RFM}_{\sR,0} = \frac{m-n}{n}\sV^{\tt RFM}_{\sN,0} + \frac{m}{m-n} \sigma^2$. Further, for $p \to n$, we have $\sB^{\tt RFM}_{\sR,0} \approx \frac{m-n}{n}\sB^{\tt RFM}_{\sR,0} + \frac{2(m-n)}{m}\|\btheta_*\|_2^2$, see discussion in \cref{app:relationship_rf}.


\begin{figure}[t]
    \centering
    \subfigure[$\alpha = 2.5$, $r=0.2$]{
        \includegraphics[width=0.3\textwidth]{arxiv_version/figures/Main_results_on_random_feature_regression/risk_vs_norm_ridgeless_alpha2.5_r0.2_n500.pdf}
    }
    \subfigure[$\alpha = 1.5$, $r=0.8$]{
        \includegraphics[width=0.3\textwidth]{arxiv_version/figures/Main_results_on_random_feature_regression/risk_vs_norm_ridgeless_alpha1.5_r0.8_n500.pdf}
    }
    \caption{Validation of \cref{prop:relation_minnorm_powerlaw_rf}. 
    The solid line represents the result of the deterministic equivalents, well approximated by the {\color{red}red dashed line} of \cref{eq:RORFMover} in the over-parameterized regime, and the {\color{blue}blue dashed line} of \cref{eq:RORFMunder} when $p \to n$ in the under-parameterized regime.}
    \label{fig:random_feature_risk_vs_norm_approx}\vspace{-0.05cm}
\end{figure}



Under power-law, we need to handle the self-consistent equations to approximate the infinite summation. We have the following approximation.
\begin{corollary}[Relationship for min-$\ell_2$ norm interpolator under power law assumption]\label{prop:relation_minnorm_powerlaw_rf}
    Under \cref{ass:powerlaw_rf}, The deterministic equivalents $\sR^{\tt RFM}_{0}$ and $\sN^{\tt RFM}_{0}$, in over-parameterized regimes ($p>n$) admit \footnote{The symbol $\approx$ here denotes using an integral to approximate an infinite sum when calculating $\Tr(\cdot)$.}
    \begin{equation}\label{eq:RORFMover}
            \sR_0^{\tt RFM} \approx \left(\nicefrac{n}{C_\alpha}\right)^{-\alpha} \sN_0^{\tt RFM} + C_{n,\alpha,r,1}\,,  
    \end{equation}
    while in the under-parameterized regime ($p<n$), we have
    \begin{align}\label{eq:RORFMunder}
        \sR_0^{\tt RFM} \approx \left(\nicefrac{n}{C_{\alpha}}\right)^{-\alpha}\sN_0^{\tt RFM} + C_{n,\alpha,r,2}\,, \quad \mbox{when}~p \to n \,,
    \end{align}
where \( C_{n,\alpha,r,1 (2)} \) are constants (see \cref{app:relationship_rf} for details) that only depend on $n$, $\alpha$ and $r$, and it admits that $C_{n,\alpha,r,1} <  C_{n,\alpha,r,2}$.
\end{corollary}


\noindent{\bf Remark:}
In the over-parameterized regime, the relationship between \(\sR_0^{\tt RFM}\) and \(\sN_0^{\tt RFM}\) is a monotonically increasing linear function, with a growth rate controlled by the factor decaying with $n$.
In the under-parameterized regime, as \(p \to n\) (which also leads to \(\sR_0^{\tt RFM}\) and \(\sN_0^{\tt RFM} \to \infty\)), \(\sR_0^{\tt RFM}\) still grows linearly w.r.t \(\sN_0^{\tt RFM}\), with the same growth rate factor decaying with $n$. Furthermore, since \(C_{n,\alpha,r,1} < C_{n,\alpha,r,2}\), the test risk curve shows that over-parameterization is better than under-parameterization.
This approximation is also empirically verified to be precise in \cref{fig:random_feature_risk_vs_norm_approx}.


\begin{figure}[t]
    \centering
    \includegraphics[width=0.5\textwidth]{arxiv_version/figures/Scaling_Law/scaling_law_norm_based_capacity.pdf} 
    \caption{The value of exponents $\gamma_n$ and $\gamma_\sN$ in different regions (divided by $q$ and $\ell$) for $r \in (0, \frac{1}{2})$. Variance dominated region is colored by {\color{regionorange}orange}, {\color{regionyellow}yellow} and {\color{regionbrown}brown}, bias dominated region is colored by {\color{regionblue}blue} and {\color{regiongreen}green}.} 
    \label{fig:scaling_law_norm_based_capacity} \vspace{-0.35cm}
\end{figure}


To study scaling law, we follow the same setting of \citet{defilippis2024dimension} by choosing $p = n^q$ and $\lambda = n^{-(\ell-1)}$ with $q,l \geq 0$. We have the scaling law as below; see the proof in \cref{app:scaling_law}.
\begin{proposition}\label{prop:scaling_law_norm_based_capacity}
Under \cref{ass:powerlaw_rf}, for $r \in (0, \frac{1}{2})$, taking $p = n^q$ and $\lambda = n^{-(\ell-1)}$ with $q,l \geq 0$, we formulate the scaling law under norm-based capacity in different areas is 
\begin{equation*}
    \sR_\lambda^{\tt RFM} = \Theta\left(n^{\gamma_n} \cdot \left(\sN_\lambda^{\tt RFM}\right)^{\gamma_{\sN}}\right)\,, 
\end{equation*}    
where the rate $\{ \gamma_n, \gamma_{\sN} \}$ in different areas is given in \cref{fig:scaling_law_norm_based_capacity}.
\end{proposition}

\noindent{\bf Remark:}
In regions \ding{172}, \ding{173}, \ding{174}, and \ding{175} of \cref{fig:scaling_law_norm_based_capacity}, the exponent of \(\sN_\lambda^{\tt RFM}\) is positive, i.e., $\gamma_{\sN} >0$, and  \(\sR_\lambda^{\tt RFM}\) increases monotonically with  \(\sN_\lambda^{\tt RFM}\). However, in region \ding{176}, we have $\gamma_{\sN} <0$, and \(\sR_\lambda^{\tt RFM}\) decreases monotonically with \(\sN_\lambda^{\tt RFM}\).\\

\section{Discussion and Future Work}\label{sec:discussion}
This paper pioneers the novel approach of selective response, showing that withholding responses can be a powerful tool for GenAI systems. By opting not to answer every query as accurately as it can---particularly when new or complex topics emerge---GenAI can encourage user participation on community-driven platforms and thereby generate more high-quality data for future training. This mechanism ultimately enhances GenAI's long-term performance and revenue. From a welfare perspective, our results indicate that such selective engagement can also benefit users, leading to better solutions and increased overall satisfaction. Since this work is the first to address selective response strategies for GenAI, numerous promising directions remain for future research; we highlight some of them below. 

First, from a technical standpoint, all of the results in this paper rely on Assumption~\ref{assumption: data lip}, involving the lipshitz condition of the accuracy function and the sensitivity parameter $\beta$. Future work could seek to relax this assumption. Furthermore, our constrained optimization approach in Subsection~\ref{sec: welfare constrained revenue maximization} could be extended to approximate the optimal (continuous) strategy instead of the optimal discrete strategy.

Second, our stylized model adopts the simplifying---though unrealistic---assumption that only a single GenAI platform exists. Admittedly, this makes it easier to focus on the idea of selective responses, and indeed, this assumption is pivotal in keeping our analysis tractable. Future research could explore scenarios with multiple GenAI platforms and human-centered forums. In such settings, one platform's selective response might redirect users not only to forums but also to competing GenAI platforms, leading to the tragedy of the commons \cite{hardin1968tragedy}: Although all GenAI platforms benefit from fresh data generation, none may choose to respond selectively if it means losing users to competitors. 

Third, we assumed Forum behaves non-strategically. In reality, human-centered platforms often monetize their data by selling it to GenAI platforms, adding a further layer of strategic interaction for GenAI. Moreover, data transfer between the platforms can form the basis for collaboration: GenAI could employ selective response to bolster Forum content creation, and Forum could, in turn, attribute that content to GenAI for subsequent use in retraining.


%Third, we make the (again) simplifying assumption that Forum is non-strategic. However, in practice, human-centered platforms can sell their data to GenAI platforms. This adds additional considerations for GenAI. Furthermore, data transmission between the platforms can also become the basis for collaboration: GenAI can use selective response to ensure enough content is generated in Forum, and Forum could provide the data attributed to this mechanism back to GenAI. 


%Second, this paper makes the simplifying yet unrealistic assumption of the existence of one GenAI platform. Indeed, this simplifies many aspects and allows us to analyze selective responses. Future work could address the data generation process with more than one GenAI platform and possibly several human-centered forums. In such a case, selective response of one GenAI platform can either drive users to forums or to other GenAI platforms; thus, we might face a tragedy of the commons situation~\ref{hardin1968tragedy}, where all GenAI platforms are interested in fresh data generation but none volunteer to selectively respond and lose users. 

%This paper examines the competition between a generative AI platform and human-based platforms, challenging the assumption that always providing answers is optimal. We analyzed the impact of withholding answers on GenAI's revenue and developed an efficient approximately optimal algorithm for this purpose. We further explored how withholding affects users, showing that it can lead to better outcomes compared to always answering. Specifically, we demonstrated that withholding can Pareto-dominate this strategy and derived the necessary and sufficient conditions for that. Finally, we proposed a second approximately optimal algorithm that maximizes GenAI's revenue while ensuring users are better off than when GenAI answers all queries.

%On a more conceptual level, our model assumes that GenAI’s data comes solely from the competing platform (Forum). Future research could explore a scenario where GenAI can purchase additional data from a third party. This extension could provide valuable insights into the interplay between withholding answers and data purchasing, and whether these two strategies can complement each other or must be traded off.

\vspace{-0.cm}
\section{Conclusion}
Leveraging deterministic equivalence, we derive a sharp characterization of the relationship between the expected test error and the $\ell_2$-norm based capacity for both linear models and RFMs. 
These results suggest a reshaping of double descent and scaling law when considering the weights norm rather than their number,implying a number of new insights from phase transition to scaling law.

\section*{Acknowledgment}
We thank for Denny Wu's discussion on asymptotic deterministic equivalence and optimal regularization.
Y. Chen was supported in part by National Science Foundation grants CCF-2233152.
L. Rosasco acknowledge the Ministry of Education, University and Research (Grant ML4IP R205T7J2KP). L. Rosasco acknowledges the European Research Council (Grant SLING 819789), the US Air Force Office of Scientific Research (FA8655-22-1-7034). The research by L. Rosasco has been supported by the MIUR Grant PRIN 202244A7YL and the MUR PNRR project PE0000013 CUP J53C22003010006 ’Future Artificial Intelligence Research (FAIR)’.
F. Liu was supported by Royal Society KTP R1 241011 Kan Tong Po Visiting Fellowships. 



\bibliography{references}
\bibliographystyle{ims}


\newpage
\appendix
\enableaddcontentsline
\tableofcontents
\newpage
%\section{Appendix}
% You may include other additional sections here.


% \newtheorem{theorem}{Theorem}[section]
% \newtheorem{lemma}{Lemma}[section] 
% \newtheorem{assumption}{Assumption}[section] 

% \newtheorem{proposition}{Proposition}[section]
% \newtheorem{definition}{Definition}[section]

% \newtheorem{cor}{Corollary}[section]
 % \newcommand{\draftcolor}{purple}
 \newcommand{\draftcolor}{black}

\newcommand{\Dim}{\textcolor{\draftcolor}{\ensuremath{d}}}
\newcommand{\EfDim}{\textcolor{\draftcolor}{\ensuremath{s}}}
\newcommand{\error}{\textcolor{\draftcolor}{\ensuremath{\tau}}}
\newcommand{\R}{\textcolor{\draftcolor}{\ensuremath{\mathbb{R}}}}
\newcommand{\Fx}{\textcolor{\draftcolor}{\x}}
\newcommand{\T}{\textcolor{\draftcolor}{\ensuremath{T}}}
\newcommand{\BY}{\textcolor{\draftcolor}{\y}}
\newcommand{\p}{\textcolor{\draftcolor}{\ensuremath{p}}}
\newcommand{\q}{\textcolor{\draftcolor}{\ensuremath{q}}}
\newcommand{\dn}{\textcolor{\draftcolor}{\ensuremath{n}}}
\newcommand{\dataset}{\textcolor{\draftcolor}{\ensuremath{\mathcal{D}_n}}}
\newcommand{\x}{\textcolor{\draftcolor}{\ensuremath{\boldsymbol{x}}}}
\newcommand{\y}{\textcolor{\draftcolor}{\ensuremath{\boldsymbol{y}}}}
\newcommand{\score}{\textcolor{\draftcolor}{\ensuremath{\boldsymbol{s}}}}
\newcommand{\scoref}{\textcolor{\draftcolor}{\ensuremath{\score_{\NetP}}}}
\newcommand{\scorefh}{\textcolor{\draftcolor}{\ensuremath{\score_{\DnSMR}}}}
\newcommand{\scorefN}{\textcolor{\draftcolor}{\ensuremath{\score_{\SMRN}}}}
\newcommand{\scorefds}{\textcolor{\draftcolor}{\ensuremath{\score_{\DnSM}}}}
\newcommand{\scorefdssigma}{\textcolor{\draftcolor}{\ensuremath{\score_{\DnSMsigma}}}}
\newcommand{\scorefdsp}{\textcolor{\draftcolor}{\ensuremath{\score_{\DnSMR}}}}
\newcommand{\scorefS}{\textcolor{\draftcolor}{\ensuremath{\score_{\NetPS}}}}
\newcommand{\NetP}{\textcolor{\draftcolor}{\ensuremath{\Theta}}}
\newcommand{\NetPS}{\textcolor{\draftcolor}{\ensuremath{\NetP^*}}}
\newcommand{\E}{\textcolor{\draftcolor}{\ensuremath{\mathbb{E}}}}
\newcommand{\norm}[1]{|\!|#1|\!|}
\newcommand{\DnSM}{\textcolor{\draftcolor}{\ensuremath{\NetP^*_{\operatorname{ds}}}}}
\newcommand{\DnSMsigma}{\textcolor{\draftcolor}{\ensuremath{\NetP^*_{\operatorname{ds},\varnoise}}}}
\newcommand{\Perx}{\textcolor{\draftcolor}{\ensuremath{\Tilde{\x}}}}
\newcommand{\disPxC}{\textcolor{\draftcolor}{\ensuremath{q_{\varnoise}}}}
\newcommand{\disPxCi}{\textcolor{\draftcolor}{\ensuremath{q_{\varnoise_{i}}}}}
\newcommand{\tv}{\textcolor{\draftcolor}{\ensuremath{\operatorname{tv}}}}
\newcommand{\sparsity}{\textcolor{\draftcolor}{\ensuremath{s}}}
\newcommand{\sm}{\textcolor{\draftcolor}{\ensuremath{\sparsity'}}}
\newcommand{\varnoise}{\textcolor{\draftcolor}{\ensuremath{\sigma}}}
\newcommand{\tuning}{\textcolor{\draftcolor}{\ensuremath{r}}}
\newcommand{\tuningorc}{\textcolor{\draftcolor}{\ensuremath{r^*}}}
\newcommand{\penaltyF}{\textcolor{\draftcolor}{\ensuremath{\mathcal{P}}}}
\newcommand{\DnSMR}{\textcolor{\draftcolor}{\ensuremath{\widehat{\NetP}_{\ell_1}}}}
\newcommand{\SMR}{\textcolor{\draftcolor}{\ensuremath{{\NetP}^{*}_{\penaltyF}}}}
\newcommand{\SMRN}{\textcolor{\draftcolor}{\ensuremath{\widehat{\NetP}_{\penaltyF,n}}}}
\newcommand{\ADnSM}{\textcolor{\draftcolor}{\ensuremath{\widetilde{\NetP}_{\operatorname{ds}}}}}
\newcommand{\ADnSMR}{\textcolor{\draftcolor}{\ensuremath{\widetilde{\NetP}_{\operatorname{ds},\penaltyF}}}}
\newcommand{\ASM}{\textcolor{\draftcolor}{\ensuremath{\widetilde{\Theta}^*}}}
\newcommand{\ASMP}{\textcolor{\draftcolor}{\ensuremath{\widetilde{\Theta}_{\penaltyF}^*}}}
\newcommand{\noise}{\textcolor{\draftcolor}{\ensuremath{\boldsymbol{z}}}}
\newcommand{\StandNormal}{\textcolor{\draftcolor}{\ensuremath{\mathcal{N}(\zerov,\mathbb{I}_{\Dim})}}}
\newcommand{\paramspace}{\textcolor{\draftcolor}{\ensuremath{\mathcal{B}}}}
\newcommand{\paramspaceone}{\textcolor{\draftcolor}{\ensuremath{\mathcal{B}_1}}}
\newcommand{\ddpm}{\textcolor{\draftcolor}{DDPMs}}
\newcommand{\ddim}{\textcolor{\draftcolor}{DDIMs}}
\newcommand{\sgm}{\textcolor{\draftcolor}{SGMs}}
\newcommand{\sde}{\textcolor{\draftcolor}{Score SDEs}}
\newcommand{\sdesam}{\textcolor{\draftcolor}{SDEs}}
\newcommand{\ode}{\textcolor{\draftcolor}{ODEs}}
\newcommand{\Ascorefds}{\textcolor{\draftcolor}{\ensuremath{s_{\ADnSM}}}}
\newcommand{\Identity}{\textcolor{\draftcolor}{\ensuremath{\mathbb{I}_{\Dim}}}}
\newcommand{\AscorefdsR}{\textcolor{\draftcolor}{\ensuremath{s_{\ADnSMR}}}}
\newcommand{\varnoiseC}{\textcolor{\draftcolor}{\ensuremath{N}}}
\newcommand{\varnoisemin}{\textcolor{\draftcolor}{\ensuremath{\varnoise_{\operatorname{min}}}}}
\newcommand{\varnoisemax}{\textcolor{\draftcolor}{\ensuremath{\varnoise_{\operatorname{max}}}}}
\newcommand{\here}{\cellcolor{green!40}\textcolor{green!40}{**}}
\newcommand{\herel}{\cellcolor{green!20}\textcolor{green!20}{**}}

\newcommand{\normMone}[1]{|\!|\!|#1|\!|\!|_1}

\newcommand{\scaleM}{\ensuremath{\hat{\kappa}}}
\newcommand{\scaleA}{\ensuremath{\Tilde{\kappa}}}
\newcommand{\SMRNA}{\textcolor{\draftcolor}{\ensuremath{\widetilde{\NetP}_{\penaltyF,n}}}}
\makeatletter
\newcommand{\deq}{\mathrel{\rlap{%
\raisebox{0.3ex}{$\m@th\cdot$}}%
\raisebox{-0.3ex}{$\m@th\cdot$}}=}

\newcommand{\eqd}{=\mathrel{\rlap{%
\raisebox{0.3ex}{$\m@th\cdot$}}%
\raisebox{-0.3ex}{$\m@th\cdot$}}}
\makeatother

\usepackage{tikz}
  \usetikzlibrary{arrows.meta}
  \usetikzlibrary{positioning}
  \tikzset{
    included node/.style={circle, draw=black!100, thick, on grid, minimum width=0.5cm}, % on grid added
    hidden node/.style={circle, draw=black!30, thick, on grid, minimum width=0.5cm},
    included connection/.style={->, thick, draw=black!100},
    hidden connection/.style={->, thick, draw=black!30, draw opacity=0},
    fused connection/.style={->, thick, draw=black!100},
    fidden connection/.style={->, thick, draw=black!30, draw opacity=0},
    node title/.style={above=0.8cm of five-one, font=\bfseries},
    general/.style={node distance=1cm and 0.7cm}
}
\newcommand{\xz}{\boldsymbol{x}}
\newcommand{\xt}{\boldsymbol{x}_t}
\newcommand{\Fd}{\mathbb{Q}}
\newcommand{\Fds}{\Fd^s}
\newcommand{\Bd}{P}
\newcommand{\Fdd}{q}
\newcommand{\Fdds}{\Fdd^s}
\newcommand{\Bdd}{p}
\newcommand{\Bdds}{p^s}
\newcommand{\Bds}{\Bd^s}
\newcommand{\BdA}{\widehat{P}}
\newcommand{\BddA}{\hat{p}}
\newcommand{\KL}{\operatorname{KL}}
\newcommand{\TV}{\operatorname{TV}}
\newcommand{\st}{\boldsymbol{s}_t}
\newcommand{\sth}{\hat{\boldsymbol{s}}}
\newcommand{\ut}{\boldsymbol{u}_t}
\newcommand{\uth}{\hat{\boldsymbol{u}}_t}
\newcommand{\us}{\boldsymbol{u}^s}
\newcommand{\scale}{\kappa}
\newcommand{\scaleS}{\kappa^*}
\newcommand{\scaleh}{\hat{\scale}}
\newcommand{\Rs}{\mathcal{R}}
\newcommand{\Lc}{L_{c}}
\newcommand{\FSM}{M}
\newcommand{\serror}{\epsilon}
% \newcommand{\argmin}{\ensuremath{\arg\,\min}}
\newcommand{\zerov}{\boldsymbol{0}_{\Dim}}
\newcommand{\DerBound}{B}
\newcommand{\COr}{c}
\newcommand{\batchS}{b_{\operatorname{s}}}
\newcommand{\Esparsity}{\epsilon}
\newcommand{\mnist}{\texttt{MNIST}}
\newcommand{\fmnist}{\texttt{FashionMNIST}}
\newcommand{\cifar}{\texttt{CIFAR10}}
\newcommand{\ButF}{\texttt{Butterflies}}
\newcommand{\deltaT}{\ensuremath{\Delta_{T} (\log \Fdd,\log \Fdds)}}
% \DeclareMathOperator*{\argmin}{\arg\,\min}














\section{Preliminary and background}
\label{app:pre_result}

We provide an overview of the preliminary results used in this work. For self-contained completeness, we include results on asymptotic deterministic equivalence in \cref{app:pre_asy_deter_equiv}, results on ridge regression in \cref{app:pre_lr}, and results on random feature ridge regression in \cref{app:pre_rfrr}. Additionally, \cref{app:pre_non-asy_deter_equiv} presents results on non-asymptotic deterministic equivalence, along with definitions of quantities required for these results. Finally, \cref{app:pre_scaling_law} introduces key results for deriving the scaling law.



\subsection{Preliminary results on asymptotic deterministic equivalence}
\label{app:pre_asy_deter_equiv}

For the ease of description, we include preliminary results on asymptotic deterministic equivalence here. In fact, these assumptions and results can be recovered from non-asymptotic results, e.g., \cite{misiakiewicz2024non}.

For linear regression, the asymptotic deterministic equivalence aim to find $\mathcal{B}_{\mathcal{R},\lambda}^{\tt LS} \sim \sB_{\sR, \lambda}^{\tt LS}$, $\mathcal{V}_{\mathcal{R},\lambda}^{\tt LS} \sim \sV_{\sR, \lambda}^{\tt LS}$, where $\sB_{\sR, \lambda}^{\tt LS}$ and $\sV_{\sR, \lambda}^{\tt LS}$ are some deterministic quantities.
For asymptotic results, a series of assumptions in high-dimensional statistics via random matrix theory are required, on well-behaved data, spectral properties of $\bSigma$ under nonlinear transformation in high-dimensional regime.
We put the assumption from \citet{bach2024high} here that are also widely used in previous literature \cite{dobriban2018high, richards2021asymptotics}. 

\begin{assumption}\citep{bach2024high}\label{ass:asym}
    We assume that:
    \begin{itemize}
    \item[\textbf{(A4)}] The sample size $n$ and dimension $d$ grow to infinity with $\frac{d}{n} \to \gamma > 0$.
    \item[\textbf{(A5)}] $\bX = \bT \bSigma^{1/2}$, where $\bT \in \mathbb{R}^{n \times d}$ has i.i.d.\ sub-Gaussian entries with zero mean and unit variance.
    \item[\textbf{(A6)}] $\bSigma$ is invertible with $\| \bSigma \|_{\text{op}}< \infty$ and its spectral measure $ \frac{1}{d} \sum_{i=1}^d \delta_{\sigma_i} $ converges to a compactly supported probability distribution $\mu$ on $\mathbb{R}^+$.
    \item[\textbf{(A7)}] $\|\bbeta_\ast\|_2 < \infty$ and the measure $ \sum_{i=1}^d (\bv_i^\sT \bbeta_\ast)^2 \delta_{\sigma_i} $ converges to a measure $\nu$ with bounded mass, where $\bv_i$ is the unit-norm eigenvector of $\bSigma$ related to its respective eigenvalue $\sigma_i$.
    \end{itemize}
\end{assumption}

\begin{definition}[Effective regularization]
    For $n$, $\bSigma$, and $\lambda \geq 0$, we define the \emph{effective regularization} $\lambda_*$ to be the unique non-negative solution to the self-consistent equation
\begin{equation}\label{eq:def_lambda_star_asy}
    n - \frac{\lambda}{\lambda_*} \sim \Tr ( \bSigma ( \bSigma + \lambda_* )^{-1} ).
\end{equation}
\end{definition}

\begin{definition}[Degrees of freedom]\label{def:df}
\[
{\rm df}_1(\lambda_*) := \Tr ( \bSigma ( \bSigma + \lambda_*)^{-1}), \quad {\rm df}_2(\lambda_*) := \Tr ( \bSigma^2 ( \bSigma + \lambda_*)^{-2}).
\]
\end{definition}

\begin{proposition}\citep[Restatement of Proposition 1]{bach2024high}\label{prop:spectral}
    Assume \textbf{(A4)}, \textbf{(A5)}, \textbf{(A6)}, we consider $\bA$ and $\bB$ with bounded operator norm, admitting the convergence of the empirical measures, i.e., $ \sum_{i=1}^d   \bv_i^\sT \bA \bv_i  \cdot\delta_{\sigma_i} \rightarrow \nu_A$
    and $ \sum_{i=1}^d   \bv_i^\sT \bB \bv_i  \cdot\delta_{\sigma_i} \rightarrow \nu_B$ with bounded total variation, respectively. Then, for $\lambda \geq 0$, with $\lambda_*$ satisfying Eq.~\eqref{eq:def_lambda_star_asy},
    we have the following {\bf asymptotic deterministic equivalence}
    \begin{align}
        \label{eq:trA1}
        \Tr ( \bA \bX^\sT \bX ( \bX^\sT \bX +\lambda )^{-1} ) \sim&~ \Tr ( \bA \bSigma ( \bSigma + \lambda_* )^{-1} )\,,
        \\
        \label{eq:trAB1}
        \Tr ( \bA \bX^\sT \bX ( \bX^\sT \bX + \lambda )^{-1} \bB \bX^\sT \bX ( \bX^\sT \bX + \lambda )^{-1}) \sim&~ \Tr ( \bA \bSigma ( \bSigma + \lambda_* )^{-1} \bB \bSigma ( \bSigma + \lambda_* )^{-1} ) \nonumber \\
        + \lambda_*^2 \Tr ( \bA ( \bSigma + \lambda_* )^{-2}  \bSigma ) &\cdot \Tr ( \bB ( \bSigma + \lambda_* )^{-2} \bSigma ) \cdot \frac{1}{ n -  {\rm df}_2(\lambda_*) }\,,\\
        \label{eq:trA2}
        \Tr ( \bA ( \bX^\sT \bX +\lambda )^{-1} ) \sim&~ \frac{\lambda_*}{\lambda} \Tr ( \bA ( \bSigma + \lambda_* )^{-1} )\,,
        \\
        \label{eq:trAB2}
        \Tr ( \bA ( \bX^\sT \bX + \lambda )^{-1} \bB ( \bX^\sT \bX + \lambda )^{-1}) \sim&~ \frac{\lambda_*^2}{\lambda^2} \Tr ( \bA ( \bSigma + \lambda_* )^{-1} \bB ( \bSigma + \lambda_* )^{-1} ) \nonumber \\
        + \frac{\lambda_*^2}{\lambda^2} \Tr ( \bA ( \bSigma + \lambda_* )^{-2}  \bSigma ) &\cdot \Tr ( \bB ( \bSigma + \lambda_* )^{-2} \bSigma ) \cdot \frac{1}{ n -  {\rm df}_2(\lambda_*) }\,.
    \end{align}
\end{proposition}

\begin{proposition}\citep[Restatement of Proposition 2]{bach2024high}
\label{prop:spectralK}
Assume \textbf{(A4)}, \textbf{(A5)}, \textbf{(A6)}, we consider $\bA$ and $\bB$ with bounded operator norm, admitting the convergence of the empirical measures, i.e., $ \sum_{i=1}^d   \bv_i^\sT \bA \bv_i  \cdot\delta_{\sigma_i} \rightarrow \nu_A$ and $ \sum_{i=1}^d   \bv_i^\sT \bB \bv_i  \cdot\delta_{\sigma_i} \rightarrow \nu_B$ with bounded total variation, respectively. Then, for $\lambda \in \mathbb{C} \backslash \mathbb{R}_+$, with $\lambda_*$ satisfying Eq.~\eqref{eq:def_lambda_star_asy}, we have the following {\bf asymptotic deterministic equivalence}
\begin{align}
\label{eq:trA1K}
\Tr ( \bA \bT^\sT ( \bT \bSigma \bT^\sT + \lambda )^{-1} \bT) \sim&~ \Tr ( \bA ( \bSigma + \lambda_* )^{-1} ),
\\
\label{eq:trAB1K}
\Tr ( \bA \bT^\sT ( \bT \bSigma \bT^\sT + \lambda )^{-1} \bT \bB \bT^\sT ( \bT \bSigma \bT^\sT + \lambda )^{-1} \bT) \nonumber \sim&~ \Tr ( \bA ( \bSigma + \lambda_* )^{-1} \bB ( \bSigma + \lambda_* )^{-1} )\\
+ \lambda_*^2 \Tr ( \bA ( \bSigma + \lambda_* )^{-2} )&~ \cdot \Tr ( \bB ( \bSigma + \lambda_* )^{-2} ) \cdot \frac{1}{ n -  {\rm df}_2(\lambda_*) }\,.
\end{align}
\end{proposition}

Note that the results in \cref{prop:spectral}, \ref{prop:spectralK} still hold even for the random features model.
We will explain this in details in \cref{app:proof_rf}.


\subsection{Preliminary results on ridge regression via deterministic equivalence}
\label{app:pre_lr}

For well-specified ridge regression, we use $n$ i.i.d. samples $\{ (\bm x_i, y_i) \}_{i=1}^n$ to learn a linear target function $\bbeta_*$
\[
    y_i = \bx_i^\sT \bbeta_* + \varepsilon_i\,.
\]
The estimator is given by solving the following empirical risk minimization with an $\ell_2$ regularization term
\[
    \hat{\bm \beta} := \argmin_{\bbeta \in \mathbb{R}^d} \left\{ \sum_{i =1}^n \left(y_i - \bm x_i^\sT\bm \beta\right)^2 + \lambda \|\bbeta\|_2^2 \right\} = (\bX^\sT \bX + \lambda \id)^{-1} \bX^\sT \bm{y}\,.
\]
Accordingly, the bias-variance decomposition is given by
\begin{align}
    \mathcal{B}_{\mathcal{R},\lambda}^{\tt LS} :=&~ \|\bbeta_* - \mathbb{E}_{\varepsilon}[\hat{\bbeta}]\|_{\bSigma}^2 = \lambda^2 \langle \bbeta_*,(\bX^\sT \bX + \lambda\id)^{-1} \bSigma (\bX^\sT \bX + \lambda\id)^{-1} \bbeta_* \rangle\,,\label{eq:lr_risk_bias}\\
    \mathcal{V}_{\mathcal{R},\lambda}^{\tt LS} :=&~ \Tr\left(\bSigma \mathrm{Cov}_{\varepsilon}(\hat{\bbeta})\right) = \sigma^2\Tr(\bSigma \bX^\sT \bX (\bX^\sT \bX + \lambda\id)^{-2})\,.\label{eq:lr_risk_variance}
\end{align}
Accordingly, the risk admits the following deterministic equivalents via bias-variance decomposition.
\begin{proposition}\citep[Restatement of Proposition 3]{bach2024high}\label{prop:asy_equiv_risk_LR}
    Given the bias variance decomposition in \cref{eq:lr_risk_bias} and \cref{eq:lr_risk_variance}, \(\bX\), \(\bSigma\) and \(\bbeta_*\) satisfy \cref{ass:asym}, we have the following asymptotic deterministic equivalents $\mathcal{R}_{\lambda}^{\tt LS}  \sim \sR_{\lambda}^{\tt LS} := \sB_{\sR,\lambda}^{\tt LS} + \sV_{\sR,\lambda}^{\tt LS}$ such that $\mathcal{B}^{\tt LS}_{\mathcal{R},\lambda} \sim \sB_{\sR,\lambda}^{\tt LS}$, $\mathcal{V}^{\tt LS}_{\mathcal{R},\lambda} \sim \sV_{\sR,\lambda}^{\tt LS}$, where $\sB_{\sR,\lambda}^{\tt LS}$ and $\sV_{\sR,\lambda}^{\tt LS}$ are defined by \cref{eq:de_risk}.
\end{proposition}


\begin{proposition}\citep[Restatement of results in Sec 5]{bach2024high} \label{prop:asy_equiv_error_LR_minnorm} 
Under the same assumption as \cref{prop:asy_equiv_risk_LR}, for the minimum $\ell_2$-norm estimator $\hat{\bbeta}_{\min}$, we have for the under-parameterized regime ($d<n$):
    \[
        \mathcal{B}_{\mathcal{R},0}^{\tt LS} = 0,\quad \mathcal{V}_{\mathcal{R},0}^{\tt LS} \sim \sigma^2\frac{d}{n-d}\,.
    \]
    In the over-parameterized regime ($d>n$), we have
    \[
        \mathcal{B}_{\mathcal{R},0}^{\tt LS} \sim \frac{\lambda_n^2\<\bbeta_*,\bSigma(\bSigma+\lambda_n\id)^{-2}\bbeta_*\>}{1-n^{-1}\Tr(\bSigma^2(\bSigma+\lambda_n\id)^{-2})}\,,\qquad
        \mathcal{V}_{\mathcal{R},0}^{\tt LS} \sim \frac{\sigma^2\Tr(\bSigma^2(\bSigma+\lambda_n\id)^{-2})}{n-\Tr(\bSigma^2(\bSigma+\lambda_n\id)^{-2})}\,,
    \]
    where $\lambda_n$ defined by $\Tr(\bSigma(\bSigma+\lambda_n\id)^{-1}) \sim n$.
\end{proposition}


\subsection{Preliminary results on random feature ridge regression via deterministic equivalence}
\label{app:pre_rfrr}

Recall \cref{eq:rffa}, the parameter $\bm a$ can be learned by the following empirical risk minimization with an $\ell_2$ regularization 
\[
    \hat{\ba} := \argmin_{\ba \in \mathbb{R}^p} \left\{ \sum_{i =1}^n \left(y_i - \frac{1}{\sqrt{p}} \sum_{j=1}^p \bm a_j \varphi(\bm x, \bm w_j) \right)^2 + \lambda \|\ba\|_2^2 \right\} = (\bZ^\sT \bZ + \lambda \id)^{-1} \bZ^\sT \by\,.
\]
Assuming that the target function $f_* \in L^2(\mu_\bx)$ admits $f_*(\bx)=\sum_{k\geq1}\btheta_{*,k}\psi_k(\bx)$, the excess risk $\mathcal{R}^{\tt RFM} := \E_{\varepsilon} \left\|\btheta_* - \frac{\bF^\sT\hat{\ba}}{\sqrt{p}}\right\|_2^2$ admits the following bias-variance decomposition
\begin{align}
    \mathcal{B}_{\mathcal{R},\lambda}^{\tt RFM} :=&~ \left\|\btheta_* - \frac{\bF^\sT \mathbb{E}_{\varepsilon}[\hat{\ba}]}{\sqrt{p}}\right\|_2^2 = \left\|\btheta_* - p^{-1/2} \bF^\sT (\bZ^\sT \bZ + \lambda\id)^{-1} \bZ^\sT \bG \bm \theta_* \right\|_2^2\,,\label{eq:rf_risk_bias}\\
    \mathcal{V}_{\mathcal{R},\lambda}^{\tt RFM} :=&~ \Tr\left(\hbLambda_\bF \mathrm{Cov}_{\varepsilon}(\hat{\ba})\right) = \sigma^2\Tr(\hbLambda_\bF \bZ^\sT\bZ(\bZ^\sT\bZ+\lambda\id)^{-2}) \,.\label{eq:rf_risk_variance}
\end{align}
Accordingly, the risk admits the following deterministic equivalents via bias-variance decomposition.
\begin{proposition}\citep[Asymptotic version of Theorem 3.3]{defilippis2024dimension}\label{prop:asy_equiv_risk_RFRR}
    Given the bias variance decomposition in \cref{eq:rf_risk_bias} and \cref{eq:rf_risk_variance}, 
    under \cref{ass:concentrated_RFRR}, we have the following asymptotic deterministic equivalents $\mathcal{R}_{\lambda}^{\tt RFM}  \sim \sR_{\lambda}^{\tt RFM} := \sB_{\sR,\lambda}^{\tt RFM} + \sV_{\sR,\lambda}^{\tt RFM}$ such that $\mathcal{B}^{\tt RFM}_{\mathcal{R},\lambda} \sim \sB_{\sR,\lambda}^{\tt RFM}$, $\mathcal{V}^{\tt RFM}_{\mathcal{R},\lambda} \sim \sV_{\sR,\lambda}^{\tt RFM}$, where $\sB_{\sR,\lambda}^{\tt RFM}$ and $\sV_{\sR,\lambda}^{\tt RFM}$ are defined by \cref{eq:de_risk_rf}.
\end{proposition}
Note that the above results are delivered in a non-asymptotic way \citep{defilippis2024dimension}, but more notations and technical assumptions are required. We give an overview of non-asymptotic deterministic equivalence as below.


\subsection{Preliminary results on non-asymptotic deterministic equivalence}\label{app:pre_non-asy_deter_equiv}

Regarding non-asymptotic results, we require a series of notations and assumptions. We give a brief introduction here for self-completeness. More details can be found in \citet{cheng2022dimension,misiakiewicz2024non,defilippis2024dimension}.

Given $\bx \in \R^d$ with $d \in \naturals$, the associated covariance matrix is given by $\bSigma = \E[\bx \bx^\sT]$. We denote the eigenvalue of $\bSigma$ in non-increasing order as $\sigma_1 \geq \sigma_2 \geq \sigma_3 \geq \cdots \geq \sigma_d$. 

We introduce the non-asymptotic version of \cref{eq:def_lambda_star_asy} as below.
\begin{definition}[Effective regularization]\label{def:effective_regularization}
    Given $n$, $\bSigma$, and $\lambda \geq 0$, the \emph{effective regularization} $\lambda_*$ is defined as the unique non-negative solution of the following self-consistent equation
    \begin{equation*}
        n - \frac{\lambda}{\lambda_*} = \Tr \big( \bSigma ( \bSigma + \lambda_* )^{-1} \big).
    \end{equation*}
\end{definition}
\noindent{\bf Remark:} 
Existence and uniqueness of $\lambda_*$ are guaranteed since the left-hand side of the equation is monotonically increasing in $\lambda_*$, while the right-hand side is monotonically decreasing. 

In the next, we introduce the following definitions on ``effective dimension'', a metric to describe the model capacity, widely used in statistical learning theory.

Define \(r_{\bSigma}(k) := \frac{\Tr(\bSigma_{\geq k})}{\| \bSigma_{\geq k} \|_{\rm op}} = \frac{\sum_{j=k}^d \sigma_j}{\sigma_k}\) as the intrinsic dimension, we require the following definition

\begin{equation}\label{eq:rho_lambda}
    \rho_{\lambda} (n) := 1 +  \frac{n \sigma_{\lfloor \eta_* \cdot n \rfloor}}{\lambda}\left\{ 1 + \frac{r_{\bSigma} (\lfloor \eta_* \cdot n \rfloor) \vee n}{n} \log \big(r_{\bSigma} (\lfloor \eta_* \cdot n \rfloor) \vee n \big) \right\},
\end{equation}
where $\eta_* \in (0,1/2)$ is a constant that will only depend on $C_*$ defined in \cref{ass:concentrated_LR}. And we used the convention that $\sigma_{\lfloor \eta_* \cdot n \rfloor} = 0$ if $\lfloor \eta_* \cdot n \rfloor > d$.


In this section we consider functionals that depend on $\bX$ and deterministic matrices. For a general PSD~matrix $\bA \in \R^{d\times d}$, define the functionals
\begin{align}
    \Phi_1(\bX; \bA, \lambda) :=&~ \Tr \left(\bA \bSigma^{1/2} (\bX^\sT \bX + \lambda)^{-1} \bSigma^{1/2}\right),\label{eq:Phi_1}\\
    \Phi_2(\bX; \bA, \lambda) :=&~ \Tr \left(\bA\bX^\sT \bX (\bX^\sT \bX + \lambda)^{-1}\right),\label{eq:Phi_2}\\
    \Phi_3(\bX; \bA, \lambda) :=&~ \Tr \left(\bA \bSigma^{1/2} (\bX^\sT \bX + \lambda)^{-1} \bSigma (\bX^\sT \bX + \lambda)^{-1} \bSigma^{1/2}\right),\label{eq:Phi_3}\\
    \Phi_4(\bX; \bA, \lambda) :=&~ \Tr \left(\bA \bSigma^{1/2} (\bX^\sT \bX + \lambda)^{-1} \frac{\bX^\sT \bX}{n} (\bX^\sT \bX + \lambda)^{-1} \bSigma^{1/2}\right).\label{eq:Phi_4}
\end{align}
These functionals can be approximated through quantities that scale proportionally to
\begin{align}
    \Psi_1(\lambda_*; \bA) :=&~ \Tr\left(\bA \bSigma (\bSigma + \lambda_*\id)^{-1}\right),\label{eq:Psi_1}\\
    \Psi_2(\lambda_*; \bA) :=&~ \frac{1}{n} \cdot \frac{\Tr\left(\bA \bSigma^2 (\bSigma + \lambda_*\id)^{-2}\right)}{n - \Tr\left(\bSigma^2 (\bSigma + \lambda_*\id)^{-2}\right)}.\label{eq:Psi_2}
\end{align}



The following theorem gathers the approximation guarantees for the different functionals stated above, and is obtained by modifying \citet[Theorem A.2]{defilippis2024dimension}. 
We generalize \cref{eq:det_equiv_phi2_main} for any PSD matrix $\bm A$, which will be required for our results on the deterministic equivalence of $\ell_2$ norm. The proof can be found in \cref{app:proof_non-asy_results}.

\begin{theorem}[Dimension-free deterministic equivalents, Theorem A.2 of \cite{defilippis2024dimension}]\label{thm:main_det_equiv_summary}
    Assume the features $\{\bx_i\}_{i \in [n]}$ satisfy \cref{ass:concentrated_LR} with a constant $C_* > 0$. Then for any $D, K > 0$, there exist constants $\eta_* \in (0, 1/2)$, $C_{D, K} > 0$ and $C_{*, D, K} > 0$ ensuring the following property holds. For any $n \geq C_{D, K}$ and $\lambda > 0$, if the following condition is satisfied:
    \begin{equation}\label{eq:conditions_det_equiv_main}
        \lambda \cdot \rho_{\lambda}(n) \geq \|\bm{\Sigma}\|_{\mathrm{op}} \cdot n^{-K}, \quad \rho_{\lambda}(n)^{\nicefrac{5}{2}} \log^{\nicefrac{3}{2}}(n) \leq K \sqrt{n},
    \end{equation}
    then for any PSD matrix $\bA$, with probability at least $1 - n^{-D}$, we have that
    \begin{align}
        |\Phi_1(\bX; \bA, \lambda) - \frac{\lambda_*}{\lambda} \Psi_1(\lambda_*; \bA)| &\leq C_{*, D, K} \frac{\rho_{\lambda}(n)^{\nicefrac{5}{2}} \log^{\nicefrac{3}{2}}(n)}{\sqrt{n}} \cdot \frac{\lambda_*}{\lambda} \Psi_1(\lambda_*; \bA),\label{eq:det_equiv_phi1_main}\\
        |\Phi_2(\bX; \id, \lambda) - \Psi_1(\lambda_*; \id)| &\leq C_{*, D, K} \frac{\rho_{\lambda}(n)^4 \log^{\nicefrac{3}{2}}(n)}{\sqrt{n}} \Psi_1(\lambda_*; \id),\label{eq:det_equiv_phi2_main}\\
        |\Phi_3(\bX; \bA, \lambda) - \left(\frac{n \lambda_*}{\lambda}\right)^2 \Psi_2(\lambda_*; \bA)| &\leq C_{*, D, K} \frac{\rho_{\lambda}(n)^6 \log^{\nicefrac{5}{2}}(n)}{\sqrt{n}} \cdot \left(\frac{n \lambda_*}{\lambda}\right)^2 \Psi_2(\lambda_*; \bA),\label{eq:det_equiv_phi3_main}\\
        |\Phi_4(\bX; \bA, \lambda) - \Psi_2(\lambda_*; \bA)| &\leq C_{*, D, K} \frac{\rho_{\lambda}(n)^6 \log^{\nicefrac{3}{2}}(n)}{\sqrt{n}} \Psi_2(\lambda_*; \bA).\label{eq:det_equiv_phi4_main}
    \end{align}
\end{theorem}


Next, we present some of the concepts to be used in deriving random feature ridge regression. Similar to how ridge regression depends on \(\lambda_*\), as defined in \cref{def:effective_regularization}, the deterministic equivalence of random feature ridge regression relies on \(\nu_1\) and \(\nu_2\), which are the solutions to the coupled equations
\begin{equation}\label{eq:fixed_points_appendix}
    n - \frac{\lambda}{\nu_1} = \Tr\left( \bLambda ( \bLambda + \nu_2)^{-1} \right)\,, \quad p - \frac{p\nu_1}{\nu_2} = \Tr \left( \bLambda ( \bLambda + \nu_2 )^{-1} \right).
\end{equation}
Writing $\nu_1$ as a function of $\nu_2$ produces the equations as below
\begin{equation}\label{eq:def:nu}
    1 + \frac{n}{p} - \sqrt{\left(1 - \frac{n}{p}\right)^2 + 4\frac{\lambda}{p\nu_2}}  = \frac{2}{p} \Tr \left( \bLambda ( \bLambda + \nu_2 )^{-1} \right)\,, \quad \nu_1 := \frac{\nu_2}{2} \left[ 1 - \frac{n}{p} + \sqrt{\left(1 - \frac{n}{p}\right)^2 + 4\frac{\lambda}{p\nu_2}} \right].
\end{equation} 



For random features, our results also depend on the capacity of $\bLambda$. Recall the definition of \(r_\bLambda(k) := \frac{\Tr(\bLambda{\geq k})}{\| \bLambda{\geq k} \|_{\rm op}}\) as the intrinsic dimension of \(\bLambda\) at level \(k\), we sequentially define the following quantities that can be found in \citet{misiakiewicz2024non,defilippis2024dimension}.

\begin{align}
    M_\bLambda (k) =&~ 1 + \frac{r_{\bLambda} (\lfloor \eta_* \cdot k \rfloor) \vee k}{k} \log \left( r_{\bLambda} (\lfloor \eta_* \cdot k \rfloor) \vee k \right)\,,\\
    \rho_\kappa (p) =&~ 1 + \frac{p \cdot \xi^2_{\lfloor \eta_* \cdot p \rfloor}}{\kappa}  M_\bLambda (p)\,, \label{eq:def_rho_p}
    \\
    \trho_\kappa (n,p) =&~ 1 + \ind \{ n \leq p/\eta_*\} \cdot \left\{ \frac{n \xi_{\lfloor \eta_* \cdot n \rfloor}^2}{\kappa} + \frac{n}{p} \cdot \rho_\kappa (p)\right\} M_\bLambda (n)\,, \label{eq:def_trho_n_p}
\end{align}
where the constant \(\eta_* \in (0,1/2)\) only depends on \(C_*\) introduced in Assumption \ref{ass:concentrated_RFRR}. 


For an integer $\evn \in \naturals$, we split the covariance matrix $\bLambda$ into low degree part and high degree part as
\[
\bLambda_0 := \diag (\xi_1^2, \xi_2^2 , \ldots , \xi_{\evn}^2)\,, \quad \bLambda_+ := \diag (\xi_{\evn+1}^2, \xi_{\evn+2}^2 , \ldots )\,.
\]

After we define the high degree feature covariance \(\bLambda_+\), we can define the function \(\gamma (\kappa) := \kappa + \Tr(\bLambda_{+})\). To simplify the statement, we assume that we can choose $\evn$ such that $p^2 \xi_{\evn +1}^2 \leq \gamma (p\lambda/n)$, which is always satisfied under \cref{ass:concentrated_RFRR}. For convenience, we will further denote
\begin{equation}\label{eq:def_gamma_lamb_plus}
\gamma_+ := \gamma (p\nu_1) , \quad \quad \gamma_\lambda := \gamma (p\lambda / n).
\end{equation}

For random feature ridge regression, we will first demonstrate that the \(\ell_2\) norm concentrates around a quantity that depends only on \(\hbLambda_\bF\). To this end, we define the following functionals with respect to \(\bZ\).
\begin{equation}\label{eq:functionals_Z}
\begin{aligned}
\Phi_3(\bZ; \bA, \kappa) &:= \Tr \left( \bA \hbLambda_\bF^{1/2} (\bZ^\sT \bZ + \kappa)^{-1} \hbLambda_\bF (\bZ^\sT \bZ + \kappa)^{-1} \hbLambda_\bF^{1/2} \right),\\
\Phi_4(\bZ; \bA, \kappa) &:= \Tr \left( \bA \hbLambda_\bF^{1/2} (\bZ^\sT \bZ + \kappa)^{-1} \frac{\bZ^\sT \bZ}{n} (\bZ^\sT \bZ + \kappa)^{-1} \hbLambda_\bF^{1/2} \right).
\end{aligned}
\end{equation}
Given that \(\bZ\) consists of i.i.d. rows with covariance \(\hbLambda_\bF = \bF \bF^\sT / p\), we will demonstrate that the aforementioned functionals can be approximated by those of \(\bF\), which, in turn, can be represented using the following functionals:

\begin{equation}\label{eq:det_equiv_Z_F}
\begin{aligned}
\widetilde{\Phi}_5(\bF; \bA, \kappa) &:= \frac{1}{n} \cdot \frac{\widetilde{\Phi}_6(\bF; \bA, \kappa)}{n - \widetilde{\Phi}_6(\bF; \bm{I}, \kappa)},\\
\widetilde{\Phi}_6(\bF; \bA, \kappa) &:= \Tr \left( \bA (\bF \bF^\sT)^2 (\bF \bF^\sT + \kappa)^{-2} \right).\\
\end{aligned}
\end{equation}

\begin{proposition}[Deterministic equivalents for $\Phi(\bZ)$ conditional on $\bF$, Proposition B.6 of \cite{defilippis2024dimension}]\label{prop:det_Z} Assume \(\{\bz_i\}_{i \in [n]}\) and \(\{\boldf\}_{i \in [p]}\) satisfy \cref{ass:concentrated_RFRR} with a constant \(C_* > 0\), and $\bF \in \mathcal{A}_{\bF}$ defined in \citet[Eq. (79)]{defilippis2024dimension}. Then for any $D, K > 0$, there exist constants $\eta_* \in (0, 1/2)$, $C_{D, K} > 0$ and $C_{*, D, K} > 0$ ensuring the following property holds. Let $\rho_{\kappa}(p)$ and $\tilde{\rho}_{\kappa}(n, p)$ be defined as per \cref{eq:def_rho_p} and \cref{eq:def_trho_n_p}, $\gamma_+$ be defined as \cref{eq:def_gamma_lamb_plus}. For any $n \geq C_{D, K}$ and $\lambda > 0$, if the following
condition is satisfied:

\begin{equation*}
\begin{aligned}
    \lambda  \geq n^{-K}\,, \quad \quad \trho_{\lambda} (n,p)^{5/2} \log^{3/2} (n) \leq K \sqrt{n}\,, \quad \quad \trho_\lambda (n,p)^2 \cdot \rho_{\gamma_+} (p)^{5/2} \log^3 (p) \leq K \sqrt{p}\,,
    \end{aligned}
\end{equation*}
then for any PSD matrix $\bA \in \mathbb{R}^{p \times p}$ (independent of \(\bZ\) conditional on \(\bF\)), we have with probability at least $1 - n^{-D}$ that

\begin{align}
\left| \Phi_3(\bZ; \bA, \lambda) - \left( \frac{n \nu_1}{\lambda} \right)^2 \widetilde{\Phi}_5(\bF; \bA, p \nu_1) \right| &\leq C_{*, D, K} \cdot \mathcal{E}_1(n, p) \cdot \left( \frac{n \nu_1}{\lambda} \right)^2 \widetilde{\Phi}_5(\bF; \bA, p \nu_1), \\
\left| \Phi_4(\bZ; \bA, \lambda) - \widetilde{\Phi}_5(\bF; \bA, p \nu_1) \right| &\leq C_{*, D, K} \cdot \mathcal{E}_1(n, p) \cdot \widetilde{\Phi}_5(\bF; \bA, p \nu_1),
\end{align}
where the rate \( \mathcal{E}_1(n, p)\) is given by \( \cE_1 (n,p) := \frac{\trho_\lambda (n,p)^6 \log^{5/2} (n)}{\sqrt{n}} + \frac{\trho_\lambda (n,p)^2 \cdot \rho_{\gamma_+} (p)^{5/2}  \log^3 (p)}{\sqrt{p}}\).

\end{proposition}



\subsection{Preliminary results on scaling law}\label{app:pre_scaling_law}

For the derivation of the scaling law, we use the results in \citet[Appendix D]{defilippis2024dimension}. We define $T^s_{\delta,\gamma}(\nu)$ as
\begin{equation*}
    T^s_{\delta,\gamma}(\nu) := \sum_{k = 1}^\infty \frac{k^{-s-\delta\alpha}}{(k^{-\alpha}+\nu)^{\gamma}}\,, \quad s \in {0,1},\;0\leq\delta\leq\gamma.
\end{equation*}
Under \cref{ass:powerlaw_rf}, according to \citet[Appendix D]{defilippis2024dimension}, we have the following results
\begin{equation}\label{eq:rate_T}
T_{\delta\gamma}^{s}(\nu) = O\left(\nu^{\nicefrac{1}{\alpha}\left[s-1 + \alpha(\delta-\gamma)\right]\wedge0}\right).
\end{equation}

Next, we present some rates of the quantities used in the deterministic equivalence of random feature ridge regression. The rate of $\nu_2$ is given by
\begin{equation}\label{eq:rate_nu2}
    \nu_2 \approx O\left(n^{-\alpha\left(1 \wedge q \wedge \nicefrac{\ell}{\alpha}\right)}\right),
\end{equation}
and in particular, for \(\Upsilon(\nu_1, \nu_2)\) and \(\chi (\nu_2)\), we have
\begin{equation}\label{eq:rates:Upsilon2}
    1 - \Upsilon(\nu_1, \nu_2) = O(1)\,,
\end{equation}

\begin{equation}\label{eq:rates:chi}
    \chi (\nu_2) = n^{-q}O\left(\nu_2^{-1-\nicefrac{1}{\alpha}}\right)\,.
\end{equation}



\section{Proofs on additional non-asymptotic deterministic equivalents}
\label{app:proof_non-asy_results}

In this section, we aim to generalize \cref{eq:det_equiv_phi2_main} for any PSD matrix $\bm A$, i.e.
\begin{equation*}
         \big\vert \Phi_2(\bX;\bA) - \Psi_2 (\mu_* ; \bA) \big\vert \leq \widetilde{\mathcal{O}}(n^{-\frac{1}{2}}) \cdot \Psi_2 (\mu_* ; \bA) \,,
\end{equation*}
that is required to derive our non-asymptotic deterministic equivalence for the bias term of the $\ell_2$ norm.


By introducing a change of variable $\mu_* := \mu_* (\lambda) = \lambda / \lambda_*$, we find that $\mu_*$ satisfies the following fixed-point equation:
\begin{align}\label{eq:det_equiv_fixed_point_mu_star}
    \mu_* = \frac{n}{1 + \Tr(\bSigma (\mu_* \bSigma + \lambda)^{-1} )}.
\end{align}
We define \(\bt\) and \(\bT\) as follows
\[
    \bt = \bSigma^{-\nicefrac{1}{2}}\bx\,, \quad \bT = \bX\bSigma^{-\nicefrac{1}{2}}\,. 
\]
And the following resolvents are also defined 
\[
    \bR := (\bX^\sT \bX + \lambda)^{-1}\,, \quad \overline{\bR} := (\mu_*\bSigma + \lambda)^{-1}\,, \quad 
    \bM := \bSigma^{\nicefrac{1}{2}} \bR \bSigma^{\nicefrac{1}{2}}\,, \quad \overline{\bM} := \bSigma^{\nicefrac{1}{2}} \overline{\bR} \bSigma^{\nicefrac{1}{2}}.
\]

Since the proof relies on a leave-one-out argument, we define \(\bX_- \in \mathbb{R}^{(n-1) \times d}\) as the data matrix obtained by removing one data. We also introduce the associated resolvent and rescaled resolvent:
\[
    \bR_- := (\bX_-^\sT \bX_- + \lambda)^{-1}\,, \quad
    \overline{\bR}_- :=\left(\frac{n}{1+\kappa}\bSigma + \lambda\right)^{-1}\,, \quad \bM_- := \bSigma^{\nicefrac{1}{2}} \bR_- \bSigma^{\nicefrac{1}{2}}\,, \quad \overline{\bM}_- := \bSigma^{\nicefrac{1}{2}} \overline{\bR}_- \bSigma^{\nicefrac{1}{2}}\,,
\]
where \(\kappa = \E[\Tr(\bM_-)]\).

For the sake of narrative convenience, we introduce a functional used in \cite{misiakiewicz2024non}
\[
\Psi_1(\mu_*; \bA) := \Tr(\bA\bSigma(\mu_*\bSigma + \lambda)^{-1})\,.
\]

Next, we give the proof of \cref{eq:det_equiv_phi2_main}. We consider the functional
\[
\Phi_2 (\bX ; \bA) = \Tr(\bA \bSigma^{-\nicefrac{1}{2}} \bX^\sT \bX ( \bX^\sT \bX + \lambda)^{-1}\bSigma^{\nicefrac{1}{2}} ) = \Tr( \bA \bT^\sT \bT \bM)  .
\]
{\bf Remark:} Note that, to align more closely with the proof in \cite{misiakiewicz2024non}, the \(\Phi_2 (\bX; \bA)\) defined here differs slightly from the \(\Phi_2 (\bX; \bA, \lambda)\) in \cref{eq:det_equiv_phi2_main}. However, the two definitions are equivalent if we take \(\bA\) here as \(\bA = \bSigma^{-\nicefrac{1}{2}} \bB \bSigma^{\nicefrac{1}{2}}\), which recovers the formulation in \cref{eq:det_equiv_phi2_main}.


We show that $\Phi_2 (\bX;\bA)$ is well approximated by the following deterministic equivalent:
\[
\Psi_2(\mu_*; \bA) = \Tr(\bA \mu_* \bSigma ( \mu_* \bSigma + \lambda)^{-1} ) = \Tr(\bA \bSigma ( \bSigma + \lambda_* )^{-1} ).
\]

\begin{theorem}[Deterministic equivalent for $\Tr(\bA \bT^\sT \bT \bM)$]\label{thm_app:det_equiv_TrAZZM}
    Assume the features $\{\bx_i\}_{i\in[n]}$ satisfy \cref{ass:concentrated_LR} with a constant $C_* > 0$. Then for any $D,K>0$, there exist constants $\eta \in (0,1/2)$, $C_{D,K} >0$, and $C_{*,D,K}>0$ ensuring the following property holds. For any $n \geq C_{D,K}$ and  $\lambda >0$, if the following condition is satisfied:
    \begin{equation}\label{eq:conditions_thm4}
\lambda \cdot \rho_\lambda (n) \geq n^{-K}, \qquad  \rho_\lambda (n)^{2} \log^{\frac{3}{2}} (n) \leq K \sqrt{n} ,
    \end{equation}
    then for any PSD matrix $\bA$, with probability at least $1 - n^{-D}$, we have that
    \begin{equation}
         \big\vert \Phi_2(\bX;\bA) - \Psi_2 (\mu_* ; \bA) \big\vert \leq C_{*,D,K} \frac{ \rho_\lambda (n)^{4} \log^{\frac32} (n ) }{\sqrt{n}}   \Psi_2 (\mu_* ; \bA) .
    \end{equation}
\end{theorem}
\noindent{\bf Remark:} \cref{thm_app:det_equiv_TrAZZM} generalizes \cref{eq:det_equiv_phi2_main}. Note that there are some differences between \(\rho_\lambda\) as defined in \cref{eq:rho_lambda} and \(\nu_\lambda\) as defined in \cite{misiakiewicz2024non}. However, based on the discussion in \citet[Appendix A]{defilippis2024dimension}, \(\nu_\lambda\) can be easily adjusted to match \(\rho_\lambda\). Therefore, while we follow the argument in \cite{misiakiewicz2024non}, we use \(\rho_\lambda\) directly in this work to minimize additional notation.


Following the approach outlined in \cite{misiakiewicz2024non}, our proof involves separately bounding the deterministic and martingale components. This is accomplished in the following two propositions.

\begin{proposition}[Deterministic part of $\Tr(\bA \bT^\sT \bT \bM)$]\label{prop:TrAZZM_LOO}
    Under the same assumption as \cref{thm_app:det_equiv_TrAZZM}, there exist constants $C_K$ and $C_{*,K}$, such that for all $n \geq C_K$ and $\lambda >0$ satisfying \cref{eq:conditions_thm4}, and for any PSD matrix $\bA$, we have
    \begin{equation}\label{eq:det_part_TrAZZM}
         \big\vert\E [ \Phi_2(\bX;\bA) ] - \Psi_2 (\mu_* ; \bA) \big\vert \leq C_{*,K} \frac{ \rho_\lambda (n)^{4} }{\sqrt{n}}   \Psi_2 (\mu_* ; \bA) .
    \end{equation}
\end{proposition}

\begin{proposition}[Martingale part of $\Tr(\bA \bT^\sT \bT \bM)$] \label{prop:TrAZZM_martingale}
    Under the same assumption as \cref{thm_app:det_equiv_TrAZZM}, there exist constants $C_{K,D}$ and $C_{*,D,K}$, such that for all $n \geq C_{K,D}$ and  $\lambda >0$ satisfying \cref{eq:conditions_thm4}, and for any PSD matrix $\bA$, we have with probability at least $1 -n^{-D}$ that
    \begin{equation}\label{eq:mart_part_TrAZZM}
         \big\vert \Phi_2(\bX;\bA)  - \E [ \Phi_2(\bX;\bA) ]  \big\vert \leq C_{*,D,K} \frac{ \rho_\lambda (n)^{3} \log^{\frac32} (n)}{\sqrt{n}} \Psi_2 (\mu_* ; \bA) .
    \end{equation}
\end{proposition}

Theorem \ref{thm_app:det_equiv_TrAZZM} is obtained by combining the bounds \eqref{eq:det_part_TrAZZM} and \eqref{eq:mart_part_TrAZZM}. Next, we prove the two propositions above separately.


\begin{proof}[Proof of Proposition \ref{prop:TrAZZM_LOO}]
First, by Sherman-Morrison identity 
\[
    \bM = \bM_- - \frac{\bM_-\bt\bt^\sT\bM_-}{1+\bt^\sT\bM_-\bt}\,,\quad \text{and} \quad \bM\bt = \frac{\bM_-\bt}{1+\bt^\sT\bM_-\bt}\,,
\]
we decompose $\E[\Phi_2(\bX;\bA)]$ as 

\[
\begin{aligned}
\E \left[ \Tr( \bA \bT^\sT \bT \bM) \right] =&~ n\E \left[ \frac{\bt^\sT \bM_- \bA \bt}{1 + S}\right] \\
= &~ n\frac{\E[\Tr ( \bA \bM_- )]}{1 + \kappa} + n\E \left[ \frac{\kappa - S}{(1+\kappa)(1 + S)} \bt^\sT \bM_- \bA \bt\right],
\end{aligned}
\]
where we denoted $S = \bt^\sT \bM_- \bt$. Therefore, bounding the following two terms is sufficient
\begin{equation}\label{eq:decompo_deterministic_2}
\begin{aligned}
    \left|\E[\Phi_2(\bX;\bA)] -\Psi_2(\mu_*;\bA)\right| \le&~
    \left|\frac{n\E[\Tr (  \bA \bM_- )]}{1 + \kappa} - \Psi_2( \mu_*;\bA)\right|+ 
    \left| n\E \left[ \frac{\kappa - S}{(1+\kappa)(1 + S)} \bt^\sT \bM_- \bA \bt\right] \right|.
\end{aligned}
\end{equation}


For the first term,  recall that $\tmu_*$ is the solution of the equation \eqref{eq:det_equiv_fixed_point_mu_star} where we replaced $n$ by $n-1$, and $\tmu_- := n/(1+ \kappa)$. By \citet[Proposition 2]{misiakiewicz2024non}, we have 
\[
\left| \E[\Tr (  \bA \bM_- )] - \Psi_1 (\tmu_* ; \bA ) \right| \leq \cE^{(\sfD)}_{1,n-1}  \cdot \Psi_1 (\tmu_* ; \bA )\,,
\]
where $\cE^{(\sfD)}_{1,n-1} = C_{*,K} \frac{ \rho_\lambda (n)^{5/2} }{\sqrt{n-1}}$. For $n \geq C$, we have $\cE_{1,n-1}^{(\sfD)} \leq C \cE_{1,n}^{(\sfD)}$ and by \citet[Lemma 3]{misiakiewicz2024non}, we have
\[
|\Psi_1 (\tmu_* ; \bA ) - \Psi_1 (\mu_* ; \bA )| \leq C \frac{\rho_\lambda(n)}{n} \Psi_1 (\mu_* ; \bA )\,.
\]
Combining the above bounds, we obtain
\[
\left| \E[\Tr (  \bA \bM_- )] - \Psi_1 (\mu_* ; \bA ) \right| \leq \cE^{(\sfD)}_{1,n}  \cdot \Psi_1 (\mu_* ; \bA )\,.
\]
Furthermore, from the proof of \citet[Proposition 4, Claim 3]{misiakiewicz2024non}, we have
\[
\frac{| \mu_* - \tmu_- |}{\tmu_-} \leq C_{*,K} \frac{ \rho_\lambda (n)^{5/2}}{\sqrt{n}}.
\]
Then we conclude that
\begin{align*}
| \mu_* - \tmu_- | &\leq C_{*,K} \frac{ \rho_\lambda (n)^{5/2}}{\sqrt{n}} \cdot \tmu_- \\
&\leq C_{*,K} \frac{ \rho_\lambda (n)^{5/2}}{\sqrt{n}} \cdot \left(1 + C_{*,K} \frac{ \rho_\lambda (n)^{5/2}}{\sqrt{n}}\right)\mu_* \\
&\leq C_{*,K} \frac{ \rho_\lambda (n)^{5/2}}{\sqrt{n}} \cdot \mu_*,    
\end{align*}
where we use condition \eqref{eq:conditions_thm4} in the last inequality.


Combining this inequality with the previous bounds, we obtain
\[
\begin{aligned}
 \left|\frac{n\E[\Tr (\bA \bM_-)]}{1 + \kappa} - \Psi_2( \mu_* ; \bA)\right| =&~ \left|\tmu_-\E[\Tr (  \bA \bM_- )] - \mu_*\Psi_1(\mu_* ;\bA)\right| \\
 \leq&~ \tmu_- \left| \E[\Tr (\bA \bM_-)] - \Psi_1 ( \mu_* ; \bA) \right| + \frac{| \tmu_- - \mu_*|}{\mu_*} \cdot \mu_* \Psi_1 ( \mu_* ; \bA)\\ 
 \leq&~ C \cE_{1,n}^{(\sfD)} \cdot \mu_*\Psi_1 (\mu_* ; \bA)\\
 =&~ C \cE_{1,n}^{(\sfD)} \cdot \Psi_2 (\mu_* ; \bA)\,.\\
\end{aligned}
\]

In the next, we aim to estimate the second term in Eq.~\eqref{eq:decompo_deterministic_2}. Here we can reduce $\bA$ to be a rank-one matrix $\bA:= \bm v \bm v^{\sT}$ following \citet[Eq. (77)]{misiakiewicz2024non}. We simply apply H\"older's inequality and obtain
\begin{align*}
n\left| \E\left[\frac{\kappa- S}{(1+\kappa)(1+S)}\bt^\sT \bM_- \bA \bt\right]\right| =&~ n \E\left[\left|\frac{\kappa- S}{(1+\kappa)(1+S)}\bt^\sT \bM_- \bv \bv^\sT \bt\right|\right]\\
\leq&~ n\mathbb{E}_{\bm M_-} \left[ \E_{\bt}\left[(\kappa - S)^2 \right]^{\nicefrac{1}{2}} \E_{\bt}\left[(\bt^\sT \bM_- \bv \bv^\sT \bt)^2\right]^{\nicefrac{1}{2}} \right]\\
\leq&~ n\mathbb{E}_{\bm M_-} \left[ \E_{\bt}\left[(\kappa - S)^2 \right]\right]^{\nicefrac{1}{2}} \E_{\bM_-}\left[\E_{\bt}\left[(\bt^\sT \bM_- \bv \bv^\sT \bt)^2\right] \right]^{\nicefrac{1}{2}}\\
\leq&~ n\mathbb{E}_{\bm M_-} \left[ \E_{\bt}\left[(\kappa - S)^2 \right]\right]^{\nicefrac{1}{2}} \E_{\bM_-}\left[\E_{\bt}\left[(\bt^\sT \bM_- \bv)^{4}\right]^{\nicefrac{1}{2}} \E_\bt\left[(\bv^\sT \bt)^4\right]^{\nicefrac{1}{2}} \right]^{\nicefrac{1}{2}}.
\end{align*}
Each of these terms can be bounded, according to the proof of \citet[Proposition 2]{misiakiewicz2024non}, for the first term, we get
\[
 \E_{\bM_-} \left[ \E_{\bt} \left[ ( \bt^\sT \bM_- \bt - \kappa)^2\right] \right]^{1/2 } \leq  C_{*,K} \frac{ \rho_\lambda (n)}{\sqrt{n}} .
\]

For the second term, first according to \citet[Lemma 2]{misiakiewicz2024non}, we have
\[
\E_\bt \left[(\bt^\sT \bM_- \bv)^4\right]^{\nicefrac{1}{2}} \leq C_{*,K} \bv^\sT \bM_-^2 \bv \,,
\]
\[
\E_\bt \left[(\bv^\sT \bt)^4\right]^{\nicefrac{1}{2}} \leq C_{*,K} \bv^\sT \bv \,.
\]
Thus we have
\[
\begin{aligned}
\E_{\bM_-}\left[\E_{\bt}\left[(\bt^\sT \bM_- \bv)^{4}\right]^{\nicefrac{1}{2}} \E_\bt\left[(\bv^\sT \bt)^4\right]^{\nicefrac{1}{2}} \right]^{\nicefrac{1}{2}}
&\leq \E_{\bM_-}\left[C_{*,K} \bv^\sT \bM_-^2 \bv \bv^\sT \bv \right]^{\nicefrac{1}{2}}\\
&= C_{*,K} \E_{\bM_-}\left[\Tr(\bA\bM_-^2 \bA) \right]^{\nicefrac{1}{2}}.
\end{aligned}
\]
Then according to \citet[Lemma 4.(b)]{misiakiewicz2024non}, we have
\[
\E_{\bM_-}\left[\Tr(\bA\bM_-^2 \bA) \right] \leq C_{*,K}\rho^2_{\lambda}(n)\Tr(\bA \obM_-^2 \bA) = C_{*,K}\rho^2_{\lambda}(n)\Tr(\bA \obM_-)^2,
\]
where the last inequality holds due to $\bA\obM_-$ being a rank-1 matrix. Combining the bounds for the second term, we have
\[
\begin{aligned}
\E_{\bM_-}\left[\E_{\bt}\left[(\bt^\sT \bM_- \bv)^{4}\right]^{\nicefrac{1}{2}} \E_\bt\left[(\bv^\sT \bt)^4\right]^{\nicefrac{1}{2}} \right]^{\nicefrac{1}{2}} \leq  C_{*,K}\rho_{\lambda}(n)\Tr(\bA\obM_-) \leq C_{*,K}\rho_{\lambda}^2(n)\Tr(\bA\obM).
\end{aligned}
\]
By combining the above bounds for the first and second term, we have
\[
\begin{aligned}
n\left| \E\left[\frac{\kappa- S}{(1+\kappa)(1+S)}\bT^\sT \bM_- \bA \bT\right]\right| &\leq C_{*,K} \frac{ \rho_\lambda^3 (n) }{\sqrt{n}} n\Tr(\bA\obM)\\
&\leq C_{*,K} \frac{ \rho_\lambda^4 (n) }{\sqrt{n}} \mu_*\Tr(\bA\obM),
\end{aligned}
\]
where we use $\mu_* = \frac{n}{1+\Tr(\obM)} \geq \frac{n}{2\rho_\lambda(n)}$ according to \citet[Lemma 3]{misiakiewicz2024non} in the last inequality.

Combining the above bounds concludes the proof.
\end{proof}


\begin{proof}[Proof of Proposition \cref{prop:TrAZZM_martingale}]
The martingale argument follows a similar approach to the proofs of \citet[Propositions 3 and 5]{misiakiewicz2024non}. The key remaining steps are to adjust Step 2 in \citet[Proposition 3]{misiakiewicz2024non} and establish high-probability bounds for each term in the martingale difference sequence.

We rewrite this term as a martingale difference sequence
\[
\begin{aligned}
 S_n := \Tr(\bA \bT^\sT \bT \bM) - \E[\Tr(\bA \bT^\sT \bT \bM)] = \sum_{i =1}^n  \left( \E_i - \E_{i-1} \right) \Tr( \bA \bT^\sT \bT \bM) =:\sum_{i = 1}^n \Delta_i\,,
\end{aligned}
\]
where \(\E_i\) is denoted as the expectation over \(\{\bx_{i+1},\cdots,\bx_n\}\).

We show below that $|\Delta_i| \leq R$ with probability at least $1 - n^{-D}$ with
\begin{equation}\label{eq:Rchoice_AMZZM}
R =  C_{*,D,K} \frac{ \rho_\lambda(n)^2 \log(n)}{n} \Psi_2( \mu_* ; \bA).
\end{equation}

For Step 3 and bounding $\E_{i-1}[\Delta_i \ind_{\Delta_i \not\in[-R,R]}] $, observe that with probability at least $1-n^{-D}$, by \citet[Lemma 4.(b)]{misiakiewicz2024non}
\[
\begin{aligned}
\E_{i-1} [ \Delta_i^2]^{\nicefrac{1}{2}} 
&~\leq 
2 \E_{i-1} \left[ \frac{(\bt^\sT \bM_- \bA \bt)^2}{(1+S)^2}\right]^{\nicefrac{1}{2}} \\
&~\leq 
C_{*,D,K} \frac{\rho_\lambda (n)^3\log^{1/2}(n)}{n} \mu_* \Tr(\bA \obM) \\
&~\leq 
C_{*,D,K} \frac{\rho_\lambda (n)^3\log^{1/2}(n)}{n} \Psi_2 (\mu_* ; \bA).
\end{aligned}
\]

We establish a high-probability bound for \(\Delta_i\) by first decomposing it and strategically adding and subtracting carefully chosen terms. Observing that
\[
\Delta_i = \left( \E_i - \E_{i-1} \right) \Tr( \bA \bT^\sT \bT \bM) = \left( \E_i - \E_{i-1} \right) \left( \Tr( \bA \bT^\sT \bT \bM) - \Tr( \bA \bT_i^\sT \bT_i \bM_i)\right)  ,
\]
where $\bM_i$ is the rescaled resolvent removes $\bx_i$, and we used that $\E_i\left[\bA\bT_i^\sT \bT_i \bM_i\right] = \E_{i-1}\left[\bA\bT_i^\sT \bT_i \bM_i\right]$, and we'll write (recall that $S_i = \bt_i^\sT \bM_i \bt_i$)
\begin{align*}
    \Tr(\bA \bT^\sT \bT \bM ) -\Tr(\bA \bT_i^\sT \bT_i \bM_i )
    =&~
    \Tr(\bA (\bt_i \bt_i^\sT + \bT_i^\sT \bT_i) \bM ) -\Tr(\bA \bT_i^\sT \bT_i \bM_i )\\
    =&~ 
    \bt_i^\sT \bM \bA\bt_i +   \Tr( \bA \bT_i^\sT \bT_i \bM)
    -\Tr( \bA \bT_i^\sT\bT_i \bM_i)\\
    =&~ 
    \frac1{(1+S_i)} \left\{ \bt_i^\sT \bM_i \bA \bt_i -   \Tr(\bA \bT_i^\sT \bT_i \bM_i \bt_i\bt_i^\sT \bM_i) \right\}\\
    =&~ 
    \frac1{(1+S_i)} \Tr(\bt_i \bt_i^\sT \bM_i \bA (\id - \bT_i^\sT \bT_i \bM_i)).
\end{align*}
Observing that 
\begin{equation*}
  \id - \bT_i^\sT\bT_i \bM_i = \lambda \bSigma^{-1} \bM_i,
\end{equation*}
we can write for $j\in\{i-1,i\}$, with probability at least $1 - n^{-D}$,
\begin{align*}
\left|\E_{j} \left[
\frac1{(1+S_i)} \Tr( \bt_i\bt_i^\sT \bM_i \bA (\id -\bT_i^\sT\bT_i \bM_i ))
\right]\right| 
\le &
\lambda \E_{j} \left[ | \bt_i^\sT \bM_i \bA \bSigma^{-1} \bM_i \bt_i | \right]\\ 
\le&~
\E_{j} \left[ | \bt_i^\sT \bM_i \bA \bt_i | \right]\\
\le&~
C_{*,D}  \log(n) \E_{j} \left[ \Tr(\bA\bM_i) \right]\\
\leq&~ C_{*,D}  \rho_\lambda(n)  \log(n) \Tr(\bA \obM)\\
\leq&~ C_{*,D} \frac{ \rho_\lambda(n)^2  \log(n)}{n} \mu_*\Tr(\bA\obM)\\
=&~ C_{*,D} \frac{ \rho_\lambda(n)^2  \log(n)}{n} \Psi_2 (\mu_* ; \bA),
\end{align*}
where we used that $\bM_i \preceq \bSigma / \lambda$ by definition in the second inequality, \citet[Lemma 4.(b)]{misiakiewicz2024non} in the fourth inequality, and $\mu_* = \frac{n}{1+\Tr(\obM)} \geq \frac{n}{2\rho_\lambda(n)}$ in the last inequality.

Applying a union bound and adjusting the choice of \(D\), we conclude that with probability at least \(1 - n^{-D}\), the following holds for all \(i \in [n]\):
\[
| \Delta_i | \leq C_{*,D,K} \frac{ \rho_\lambda (n)^2 \log (n) }{n} \Psi_2 (\mu_* ; \bA)\,.
\]
\end{proof}




\section{Proofs for ridge regression}

In this section, we provide the proof of deterministic equivalence for ridge regression in both the asymptotic (\cref{app:asy_deter_equiv_lr}) and non-asymptotic (\cref{app:nonasy_deter_equiv_lr}) settings. Additionally, we provide the proof of the relationship between test risk and the $\ell_2$ norm given in the main text, as detailed in \cref{app:relationship}.


\subsection{Asymptotic deterministic equivalence for ridge regression}
\label{app:asy_deter_equiv_lr}

In this section, we establish the asymptotic approximation guarantees for linear regression, focusing on the relationships between the $\ell_2$ norm of the estimator and its deterministic equivalent.
These results can be recovered by our non-asymptotic results, but we put them here just for completeness.

Before presenting the results on deterministic equivalence for ridge regression and their proofs, we begin by introducing a couple of useful corollaries from \cref{prop:spectral,prop:spectralK}.

\begin{corollary}
\label{prop:spectral2}
    Under the same condition of \cref{prop:spectral}, we have
    \begin{align}\label{eq:trA3}
        \Tr ( \bA \bX^\sT \bX ( \bX^\sT \bX + \lambda )^{-2}) \sim&~ \frac{\Tr(\bA\bSigma(\bSigma + \lambda_*\id)^{-2})}{n - {\rm df}_2(\lambda_*)}\,.
        \end{align}
        Specifically, if $\bA = \bSigma$, we have
        \begin{align}\label{eq:trS3}
        \Tr ( \bSigma \bX^\sT \bX ( \bX^\sT \bX + \lambda )^{-2}) \sim&~ \frac{{\rm df}_2(\lambda_*)}{n - {\rm df}_2(\lambda_*)}\,.
    \end{align}
\end{corollary}

\begin{corollary}
\label{prop:spectralK2}
Under the same condition of \cref{prop:spectralK}, we have
\begin{align}
\label{eq:trA3K}
\Tr ( \bA \bT^\sT ( \bT \bSigma \bT^\sT + \lambda )^{-2} \bT) \sim&~ \frac{\Tr ( \bA ( \bSigma + \lambda_* )^{-2})}{ n -  {\rm df}_2(\lambda_*) }\,.
\end{align}
\end{corollary}

Using the equation 
\[
\Tr ( \bA \bX^\sT \bX ( \bX^\sT \bX + \lambda )^{-2}) = \frac{1}{\lambda} \left( \Tr ( \bA \bX^\sT \bX ( \bX^\sT \bX + \lambda )^{-1}) - \Tr ( \bA (\bX^\sT \bX)^2 ( \bX^\sT \bX + \lambda )^{-2}) \right)\,,
\] 
we can directly obtain \cref{prop:spectral2,prop:spectralK2} from \cref{prop:spectral,prop:spectralK}

After introduce the two corollaries above, we first give the proof of the bias-variance decomposition in \cref{lemma:biasvariance}.

\begin{proof}[Proof of \cref{lemma:biasvariance}]
Here we give the bias-variance decomposition of $\E_{\varepsilon}\|\hat{\bbeta}\|_2^2$. The formulation of $\E_{\varepsilon}\|\hat{\bbeta}\|_2^2$ is given by
\[
\begin{aligned}
     \E_{\varepsilon}\|\hat{\bbeta}\|_2^2 =\|\left( \bX^\sT \bX + \lambda \id \right)^{-1} \bX^\sT \by \|_2^2\,,
\end{aligned}
\]
which can be decomposed as
\[
\begin{aligned}
    \E_{\varepsilon}\|\hat{\bbeta}\|_2^2 =&~ \E_{\varepsilon}\|\left( \bX^\sT \bX + \lambda \id \right)^{-1} \bX^\sT (\bX\bbeta_* + \bm\varepsilon) \|_2^2\\
    =&~ \|\left( \bX^\sT \bX + \lambda \id \right)^{-1} \bX^\sT \bX\bbeta_* \|_2^2 + \E_{\varepsilon}\|\left( \bX^\sT \bX + \lambda \id \right)^{-1} \bX^\sT \bm\varepsilon \|_2^2\\
    =&~\<\bbeta_*, (\bX^\sT\bX)^2(\bX^\sT\bX + \lambda\id)^{-2}\bbeta_*\> + \sigma^2\Tr(\bX^\sT\bX(\bX^\sT\bX + \lambda\id)^{-2})\\
    =:&~ \mathcal{B}_{\mathcal{N},\lambda}^{\tt LS} + \mathcal{V}_{\mathcal{N},\lambda}^{\tt LS}\,.
\end{aligned}
\]
Accordingly, we can see that it shares the similar spirit with the bias-variance decomposition.
\end{proof}

Now we are ready to derive the deterministic equivalence, i.e., $\E_{\varepsilon}\|\hat{\bbeta}\|_2^2$, under the bias-variance decomposition. Our results can handle ridge estimator $\hat{\bbeta}$ in \cref{prop:asy_equiv_norm_LR} and interpolator $\hat{\bbeta}_{\min}$ in \cref{prop:asy_equiv_norm_LR_minnorm}, respectively.
\begin{proposition}[Asymptotic deterministic equivalence of the norm of ridge regression estimator]\label{prop:asy_equiv_norm_LR}
    Given the bias variance decomposition of $\E_{\varepsilon}\|\hat{\bbeta}\|_2^2$ in \cref{lemma:biasvariance}, 
    under \cref{ass:asym}, we have the following asymptotic deterministic equivalents $\mathcal{N}^{\tt LS}_{\lambda}  \sim \sN^{\tt LS}_{\lambda} := \sB_{\sN,\lambda}^{\tt LS} + \sV_{\sN,\lambda}^{\tt LS}$ such that $\mathcal{B}^{\tt LS}_{\mathcal{N},\lambda} \sim \sB_{\sN,\lambda}^{\tt LS}$, $\mathcal{V}^{\tt LS}_{\mathcal{N},\lambda} \sim \sV_{\sN,\lambda}^{\tt LS}$, where $\sB_{\sN,\lambda}^{\tt LS}$ and $\sV_{\sN,\lambda}^{\tt LS}$ are defined by \cref{eq:equiv-linear}.
\end{proposition}


And we present the proof of \cref{prop:asy_equiv_norm_LR} as below.
\begin{proof}[Proof of \cref{prop:asy_equiv_norm_LR}]
We give the asymptotic deterministic equivalents for $\mathcal{B}_{\mathcal{N},\lambda}^{\tt LS}$ and $\mathcal{V}_{\mathcal{N},\lambda}^{\tt LS}$, respectively. For the bias term $\mathcal{B}_{\mathcal{N},\lambda}^{\tt LS}$, we use \cref{eq:trAB1} by taking $\bA = \bbeta_*\bbeta_*^\sT$ and $\bB = \id$ and thus obtain
\[
\begin{aligned}
    \mathcal{B}_{\mathcal{N},\lambda}^{\tt LS} = &~ \<\bbeta_*, (\bX^\sT\bX)^2(\bX^\sT\bX + \lambda\id)^{-2}\bbeta_*\>\\
    = &~ \Tr(\bbeta_*\bbeta_*^\sT(\bX^\sT\bX)^2(\bX^\sT\bX + \lambda\id)^{-2})\\
    \sim &~ \Tr(\bbeta_*\bbeta_*^\sT\bSigma^2(\bSigma + \lambda_*\id)^{-2}) + \lambda_*^2 \Tr(\bbeta_*\bbeta_*^\sT\bSigma(\bSigma + \lambda_*\id)^{-2}) \cdot \Tr(\bSigma(\bSigma + \lambda_*\id)^{-2}) \cdot \frac{1}{n-\Tr(\bSigma^2(\bSigma + \lambda_*\id)^{-2})}\\
    = &~ \<\bbeta_*, \bSigma^2(\bSigma + \lambda_*\id)^{-2}\bbeta_*\> + \frac{\Tr(\bSigma(\bSigma + \lambda_*\id)^{-2})}{n} \cdot \frac{\lambda_*^2 \<\bbeta_*,\bSigma(\bSigma + \lambda_*\id)^{-2}\bbeta_*\>}{1-n^{-1}\Tr(\bSigma^2(\bSigma + \lambda_*\id)^{-2})}\\
    =: &~ \sB_{\sN, \lambda}^{\tt LS}\,.
\end{aligned}
\]
For the variance term $\mathcal{V}_{\mathcal{N}}^{\tt LS}$, we use \cref{eq:trA3} by taking $\bA = \id$ and obtain
\[
\begin{aligned}
    \mathcal{V}_{\mathcal{N}}^{\tt LS} = &~ \sigma^2\Tr(\bX^\sT\bX(\bX^\sT\bX + \lambda\id)^{-2}) \sim \frac{\sigma^2\Tr(\bSigma(\bSigma + \lambda_*\id)^{-2})}{n - \Tr(\bSigma^2(\bSigma + \lambda_*\id)^{-2})} =: \sV_{\sN, \lambda}^{\tt LS}\,.
\end{aligned}
\]
\end{proof}

As discussed in the main text, the asymptotic behavior of $\lambda_*$ differs between the under-parameterized and over-parameterized regimes as $\lambda \to 0$, though the ridge regression estimator $\hat{\bbeta}$ converges to the min-$\ell_2$-norm estimator $\hat{\bbeta}_{\min}$.
To be specific, in the under-parameterized regime, $\lambda_*$ converges to $0$ as $\lambda \to 0$; while in the over-parameterized regime, $\lambda_*$ converges to a constant that admits $\Tr(\bSigma(\bSigma + \lambda_n \id)^{-1}) \sim n$ when $\lambda \to 0$. 
Accordingly, for the minimum $\ell_2$-norm estimator, it is necessary to analyze the two regimes separately.
We have the following results on the characterization of the deterministic equivalence of $\| \hat{\bbeta}_{\min} \|_2$. 

\begin{corollary}[Asymptotic deterministic equivalence of the norm of interpolator]\label{prop:asy_equiv_norm_LR_minnorm}
    Under \cref{ass:asym}, for the minimum $\ell_2$-norm estimator $\hat{\bbeta}_{\min}$, we have the following deterministic equivalence: for the under-parameterized regime ($d<n$), we have
    \[
    \begin{aligned}
        \mathcal{B}^{\tt LS}_{\mathcal{N},0} = \|\bbeta_*\|_2^2\,,\quad \mathcal{V}^{\tt LS}_{\mathcal{N},0} \sim&~ \frac{\sigma^2}{n-d}\Tr(\bSigma^{-1})\,.
    \end{aligned}
    \]
    In the over-parameterized regime ($d>n$), we have
    \[
    \begin{aligned}
        \mathcal{B}^{\tt LS}_{\mathcal{N},0} \sim&~ \<\bbeta_*,\bSigma(\bSigma+\lambda_n\id)^{-1}\bbeta_*\>\,,\\
        \mathcal{V}^{\tt LS}_{\mathcal{N},0} \sim&~ \frac{\sigma^2\Tr(\bSigma(\bSigma+\lambda_n\id)^{-2})}{n-\Tr(\bSigma^2(\bSigma+\lambda_n\id)^{-2})} = \frac{\sigma^2}{\lambda_n}\,,
    \end{aligned}
    \]
    where $\lambda_n$ is defined by $\Tr(\bSigma(\bSigma+\lambda_n\id)^{-1}) \sim n$.
\end{corollary}
\begin{proof}[Proof of \cref{prop:asy_equiv_norm_LR_minnorm}]
We separate the results in the under-parameterized and over-parameterized regimes.

In the under-parameterized regime ($d<n$), for minimum norm estimator $\hat{\bbeta}_{\min}$, we have (for $\bX^\sT\bX$ is invertible)
\[
\begin{aligned}
    \hat{\bbeta}_{\min} = \left(\bX^\sT\bX\right)^{-1}\bX^\sT\by = \left(\bX^\sT\bX\right)^{-1}\bX^\sT(\bX\bbeta_*+\bm\varepsilon) = \bbeta_* + \left(\bX^\sT\bX\right)^{-1}\bX^\sT\bm\varepsilon\,.
\end{aligned}
\]
Accordingly, we can directly obtain the bias-variance decomposition as well as their deterministic equivalents
\[
\begin{aligned}
    \mathcal{B}_{\mathcal{N},0}^{\tt LS} = \|\bbeta_*\|_2^2\,, \quad \mathcal{V}_{\mathcal{N},0}^{\tt LS} = \sigma^2\Tr(\bX^\sT\bX(\bX^\sT\bX)^{-2}) \sim \sigma^2\frac{\Tr(\bSigma^{-1})}{n-d}\,,
\end{aligned}
\]
where we use \cref{eq:trA3} and take $\lambda \to 0$ for the variance term.

In the over-parameterized regime ($d>n$), we take the limit $\lambda \to 0$ within ridge regression and use \cref{prop:asy_equiv_norm_LR}.
Define $\lambda_n$ as $\Tr(\bSigma(\bSigma+\lambda_n\id)^{-1}) \sim n$, we have for the bias term
\[
\begin{aligned}
    \mathcal{B}_{\mathcal{N},0}^{\tt LS} \sim &~ \<\bbeta_*, \bSigma^2(\bSigma + \lambda_n\id)^{-2}\bbeta_*\> + \frac{\Tr(\bSigma(\bSigma + \lambda_n\id)^{-2})}{n} \cdot \frac{\lambda_n^2 \<\bbeta_*,\bSigma(\bSigma + \lambda_n\id)^{-2}\bbeta_*\>}{1-n^{-1}\Tr(\bSigma^2(\bSigma + \lambda_n\id)^{-2})}\\
    =&~ \<\bbeta_*, \bSigma(\bSigma + \lambda_n\id)^{-1}\bbeta_*\> - \lambda_n \<\bbeta_*, \bSigma(\bSigma + \lambda_n\id)^{-2}\bbeta_*\> + \frac{\Tr(\bSigma(\bSigma + \lambda_n\id)^{-2})}{n} \cdot \frac{\lambda_n^2 \<\bbeta_*,\bSigma(\bSigma + \lambda_n\id)^{-2}\bbeta_*\>}{1-n^{-1}\Tr(\bSigma^2(\bSigma + \lambda_n\id)^{-2})}\\
    =&~ \<\bbeta_*, \bSigma(\bSigma + \lambda_n\id)^{-1}\bbeta_*\>\,.
\end{aligned}
\]
For the variance term, we have
\[
\begin{aligned}
    \mathcal{V}_{\mathcal{N},0}^{\tt LS} \sim&~ \frac{\sigma^2\Tr(\bSigma(\bSigma+\lambda_n\id)^{-2})}{n-\Tr(\bSigma^2(\bSigma+\lambda_n\id)^{-2})}\,.
\end{aligned}
\]
Finally we conclude the proof.
\end{proof}


\subsection{Non-asymptotic deterministic equivalence for ridge regression}
\label{app:nonasy_deter_equiv_lr}

The deterministic equivalents for the bias and variance terms of the test risk are given by \citet{misiakiewicz2024non} as 
\[
\begin{aligned}
\sB_{\sR,\lambda}^{\tt LS} :=&~ \frac{\lambda_*^2 \langle \bbeta_*, \bSigma(\bSigma + \lambda_*\id)^{-2} \bbeta_* \rangle}{1 - n^{-1} \Tr\left(\bSigma^2 (\bSigma + \lambda_*\id)^{-2}\right)}\,, \qquad \sV_{\sR,\lambda}^{\tt LS} :=&~ \frac{\sigma_{\varepsilon}^2 \Tr\left(\bSigma^2 (\bSigma + \lambda_*\id)^{-2}\right)}{n - \Tr\left(\bSigma^2 (\bSigma + \lambda_*\id)^{-2}\right)}\,.
\end{aligned}
\]
In this section, we establish the approximation guarantees for linear ridge regression.
Instead of using the power-law assumption \cref{ass:powerlaw} in the main text, we adopt the following weaker assumption.
\begin{assumption}[\citet{defilippis2024dimension}]\label{ass:technical_LR} There exists $C>1$
\[
    \frac{\< \bbeta_*, \bSigma(\bSigma+\lambda_*)^{-1}\bbeta_* \>}{\< \bbeta_*, \bSigma^2(\bSigma+\lambda_*)^{-2}\bbeta_* \>} \leq C\,.
\]
\end{assumption}
\noindent{\bf Remark:}
This assumption holds in many settings of interest, such as power law assumptions like those in \cref{ass:powerlaw}, since under this assumption the numerator and denominator are bounded sums of finite terms. It is a technical assumption used to address the difference between two deterministic equivalents that are needed in our work for norm-based capacity.
In fact, this assumption is used for RFMs in \cite{defilippis2024dimension} as the authors also face with the issue on the difference between two deterministic equivalents.


We generalize \cref{prop:non-asy_equiv_norm_LR} as below.
\begin{theorem}[Deterministic equivalents of the $\ell_2$-norm of the estimator. Full version of \cref{prop:non-asy_equiv_norm_LR}]\label{prop:det_equiv_LR}
    Assume well-behaved data $\{ \bm x_i \}_{i=1}^n$ satisfy \cref{ass:concentrated_LR} and \cref{ass:technical_LR}. Then for any $D,K > 0$, there exist constants $\eta_* \in (0, 1/2)$ and $C_{*,D,K} > 0$ ensuring the following property holds. For any $n \geq C_{*,D,K}$, $\lambda > 0$, if the following condition is satisfied:
    \begin{equation*}
        \lambda \geq n^{-K}\,, \quad \rho_{\lambda}(n)^{5/2} \log^{3/2}(n) \leq K \sqrt{n}\,,
    \end{equation*} 
    then with probability at least $1-n^{-D}$, we have that
    \[
    \begin{aligned}
         \left|\mathcal{B}_{\mathcal{N},\lambda}^{\tt LS} - \sB_{\sN,\lambda}^{\tt LS}\right| \leq&~ C_{x, D, K} \frac{\rho_{\lambda}(n)^6 \log^{3/2}(n)}{\sqrt{n}}\sB_{\sN,\lambda}^{\tt LS}\,,\\
        \left|\mathcal{V}_{\mathcal{N},\lambda}^{\tt LS} - \sV_{\sN,\lambda}^{\tt LS}\right| \leq&~ C_{x, D, K} \frac{\rho_{\lambda}(n)^6 \log^{3/2}(n)}{\sqrt{n}} \sV_{\sN,\lambda}^{\tt LS}\,.
    \end{aligned}
    \]
\end{theorem}

\begin{proof}[Proof of \cref{prop:det_equiv_LR}] 
{\bf Part 1: Deterministic equivalents for the bias term.}

Here we prove the deterministic equivalents of $\mathcal{B}_{\mathcal{N},\lambda}^{\tt LS}$ and $\mathcal{V}_{\mathcal{N},\lambda}^{\tt LS}$. First, we decompose $\mathcal{B}_{\mathcal{N},\lambda}^{\tt LS}$ into
\[
\begin{aligned}
    \mathcal{B}_{\mathcal{N},\lambda}^{\tt LS} &= \Tr\left(\bbeta_*\bbeta_*^\sT\bX^\sT \bX (\bX^\sT \bX + \lambda)^{-1}\right) - \lambda\Tr\left(\bbeta_*\bbeta_*^\sT\bX^\sT \bX (\bX^\sT \bX + \lambda)^{-2}\right),\\
    &= \Phi_2(\bX; \tilde{\bA}_1, \lambda) - n\lambda \Phi_4(\bX; \tilde{\bA}_2, \lambda)\,,
\end{aligned}
\]
where $\tilde{\bA}_1 := \bbeta_*\bbeta_*^\sT$, $\tilde{\bA}_2 := \bSigma^{-1/2}\bbeta_*\bbeta_*^\sT\bSigma^{-1/2}$. 
Therefore, using \cref{thm:main_det_equiv_summary}, with probability at least $1-n^{-D}$, we have
\[
\begin{aligned}
    \left|\Phi_2(\bX; \tilde{\bA}_1, \lambda) - \Psi_1(\lambda_*; \tilde{\bA}_1)\right| &\leq C_{x, D, K} \frac{\rho_{\lambda}(n)^{5/2} \log^{3/2}(n)}{\sqrt{n}} \Psi_1(\lambda_*; \tilde{\bA}_1)\,,\\
    \left|n\lambda\Phi_4(\bX; \tilde{\bA}_2, \lambda) - n\lambda\Psi_2(\lambda_*; \tilde{\bA}_2)\right| &\leq C_{x, D, K} \frac{\rho_{\lambda}(n)^6 \log^{3/2}(n)}{\sqrt{n}} n\lambda\Psi_2(\lambda_*; \tilde{\bA}_2)\,.
\end{aligned}
\]
Combining the above bounds, we deduce that
\[
\begin{aligned}
    \left|\mathcal{B}_{\mathcal{N},\lambda}^{\tt LS} - \left(\Psi_1(\lambda_*; \tilde{\bA}_1) - n\lambda\Psi_2(\lambda_*; \tilde{\bA}_2)\right)\right| \leq C_{x, D, K} \frac{\rho_{\lambda}(n)^6 \log^{3/2}(n)}{\sqrt{n}}\left(\Psi_1(\lambda_*; \tilde{\bA}_1)+n\lambda\Psi_2(\lambda_*; \tilde{\bA}_2)\right).
\end{aligned}
\]
Note that
\[
\begin{aligned}
    \Psi_1(\lambda_*; \tilde{\bA}_1) - n\lambda\Psi_2(\lambda_*; \tilde{\bA}_2) = \sB_{\sN,\lambda}^{\tt LS}\,.    
\end{aligned}
\]
For $n\lambda\Psi_2(\lambda_*; \tilde{\bA}_2)$, recall that $\Psi_2(\lambda_*; \bA) := \frac{1}{n} \frac{\Tr(\bA \bSigma^2 (\bSigma + \lambda_*\id)^{-2})}{n - \Tr(\bSigma^2 (\bSigma + \lambda_*\id)^{-2})}$, and according to \cref{def:effective_regularization} 
and \cref{ass:technical_LR}, we have
\[
\begin{aligned}
    n\lambda\Psi_2(\lambda_*; \tilde{\bA}_2) =&~ \lambda\frac{\Tr(\bbeta_*\bbeta_*^\sT\bSigma (\bSigma + \lambda_*\id)^{-2})}{n - \Tr(\bSigma^2 (\bSigma + \lambda_*\id)^{-2})}\\
    \leq&~ \lambda_*\Tr(\bbeta_*\bbeta_*^\sT\bSigma (\bSigma + \lambda_*\id)^{-2})\\
    =&~ \Tr(\bbeta_*\bbeta_*^\sT \bSigma (\bSigma + \lambda_*\id)^{-1}) - \Tr(\bbeta_*\bbeta_*^\sT\bSigma^2 (\bSigma + \lambda_*\id)^{-2})\\
    \leq&~ \left(1-\frac{1}{C}\right) \Tr(\bbeta_*\bbeta_*^\sT\bSigma(\bSigma+\lambda_*)^{-1})\,,
\end{aligned}
\]
and therefore
\[
\begin{aligned}
    \Psi_1(\lambda_*; \tilde{\bA}_1)+n\lambda\Psi_2(\lambda_*; \tilde{\bA}_2) \leq&~ \left(2-\frac{1}{C}\right) \Tr(\bbeta_*\bbeta_*^\sT\bSigma(\bSigma+\lambda_*)^{-1})\\
    \leq&~ \left(2C-1\right)\frac{1}{C}\Tr(\bbeta_*\bbeta_*^\sT\bSigma(\bSigma+\lambda_*)^{-1})\\
    \leq&~ \left(2C-1\right)\left(\Psi_1(\lambda_*; \tilde{\bA}_1)-n\lambda\Psi_2(\lambda_*; \tilde{\bA}_2)\right).
\end{aligned}
\]
Then we conclude that
\[
    \left|\mathcal{B}_{\mathcal{N},\lambda}^{\tt LS} - \sB_{\sN,\lambda}^{\tt LS}\right| \leq C_{x, D, K} \frac{\rho_{\lambda}(n)^6 \log^{3/2}(n)}{\sqrt{n}}\sB_{\sN,\lambda}^{\tt LS},
\]
with probability at least $1-n^{-D}$.

{\bf Part 2: Deterministic equivalents for the variance term.} Next, we prove the deterministic equivalent of $\mathcal{V}_{\mathcal{N},\lambda}^{\tt LS}$. First, note that $\mathcal{V}_{\mathcal{N},\lambda}^{\tt LS}$ can be written in terms of the functional $\Phi_4(\bX; \bA, \lambda)$ defined in \cref{eq:Phi_4}
\[
    \mathcal{V}_{\mathcal{N},\lambda}^{\tt LS} = n\sigma_{\varepsilon}^2\Phi_4(\bX; \bSigma^{-1}, \lambda)\,.
\]
Thus, under the assumptions, we can apply \cref{thm:main_det_equiv_summary} to obtain that with probability at least $1-n^{-D}$
\[
\left|n\sigma_{\varepsilon}^2\Phi_4(\bX; \bSigma^{-1}, \lambda) - n\sigma_{\varepsilon}^2\Psi_2(\lambda_*; \bSigma^{-1})\right| \leq C_{x, D, K} \frac{\rho_{\lambda}(n)^6 \log^{3/2}(n)}{\sqrt{n}} n\sigma_{\varepsilon}^2\Psi_2(\lambda_*; \bSigma^{-1})\,.
\]
Recall that $\Psi_2(\lambda_*; \bA) := \frac{1}{n} \frac{\Tr(\bA \bSigma^2 (\bSigma + \lambda_*\id)^{-2})}{n - \Tr(\bSigma^2 (\bSigma + \lambda_*\id)^{-2})}$, then we have
\[
\left|\mathcal{V}_{\mathcal{N},\lambda}^{\tt LS} - \sV_{\sN,\lambda}^{\tt LS}\right| \leq C_{x, D, K} \frac{\rho_{\lambda}(n)^6 \log^{3/2}(n)}{\sqrt{n}} \sV_{\sN,\lambda}^{\tt LS}\,,
\]
with probability at least $1-n^{-D}$.
\end{proof}


\subsection{Proofs on relationship between the test risk and \texorpdfstring{$\ell_2$}{L2} norm of ridge regression estimator}
\label{app:relationship}

In this section, we will prove the relationship between the test risk and $\ell_2$ norm of the ridge regression estimator that we give in the main text.



\label{proof:linear:relation}
\begin{proof}[Proof of \cref{prop:relation_id}]
According to the formulation of $\sB_{\sN,\lambda}^{\tt LS}$ and $\sV_{\sN,\lambda}^{\tt LS}$ in \cref{eq:equiv-linear}, for $\bSigma=\id_d$, we have
\[
    \sB_{\sN,\lambda}^{\tt LS} = \frac{1}{(1+\lambda_*)^2}\|\bbeta_*\|_2^2 + \frac{d}{n(1+\lambda_*)^2} \cdot \frac{\lambda_*^2 \frac{1}{(1+\lambda_*)^2}\|\bbeta_*\|_2^2}{1-\frac{d}{n(1+\lambda_*)^2}} \,, \quad \sV_{\sN,\lambda}^{\tt LS} = \frac{\sigma^2\frac{d}{(1+\lambda_*)^2}}{n-\frac{d}{(1+\lambda_*)^2}}\,,
\]
\[
    \sN_{\lambda}^{\tt LS} = \frac{d}{(1+\lambda_*)^2}\|\bbeta_*\|_2^2 + \frac{d}{n(1+\lambda_*)^2} \cdot \frac{\lambda_*^2 \frac{d}{(1+\lambda_*)^2}\|\bbeta_*\|_2^2}{1-\frac{d}{n(1+\lambda_*)^2}} + \frac{\sigma^2\frac{d}{(1+\lambda_*)^2}}{n-\frac{d}{(1+\lambda_*)^2}}\,,
\]
where $\lambda_*$ admits a closed-form solution
\[
\lambda_*=\frac{d+\lambda-n+\sqrt{4\lambda n + (n-d-\lambda)^2}}{2n} \,.
\]
Recall the formulation $\sB_{\sR,\lambda}^{\tt LS}$ and $\sV_{\sR,\lambda}^{\tt LS}$ (for test risk) in \cref{eq:de_risk}, for $\bSigma=\id_d$, we have
\[
\begin{aligned}
    \sB_{\sR,\lambda}^{\tt LS} = \frac{\lambda_*^2 \frac{1}{(1+\lambda_*)^2}\|\bbeta_*\|_2^2}{1-\frac{d}{n(1+\lambda_*)^2}} \,, \quad \sV_{\sR,\lambda}^{\tt LS} = \frac{\sigma^2\frac{d}{(1+\lambda_*)^2}}{n-\frac{d}{(1+\lambda_*)^2}}\,, \quad
    \sR_{\lambda}^{\tt LS} = \frac{\lambda_*^2 \frac{d}{(1+\lambda_*)^2}\|\bbeta_*\|_2^2}{1-\frac{d}{n(1+\lambda_*)^2}} + \frac{\sigma^2\frac{d}{(1+\lambda_*)^2}}{n-\frac{d}{(1+\lambda_*)^2}}\,.
\end{aligned}
\]
Accordingly, to establish the relationship between $\sR_{\lambda}^{\tt LS}$ and $\sN_{\lambda}^{\tt LS}$, we combine their formulation and eliminate $n$ to obtain\footnote{Due to the complexity of the calculations, we use Mathematica Wolfram to eliminate $n$. The same approach is applied later whenever $n$ or $p$ elimination is required.}
\[
\begin{aligned}
    2( (\sR^{\tt LS}_{\lambda} - \sN^{\tt LS}_{\lambda})^2 - \|\bbeta_*\|_2^4 ) d \sigma^2 =&~ (\|\bbeta_*\|_2^2 - \sR^{\tt LS}_{\lambda} - \sN^{\tt LS}_{\lambda})(\|\bbeta_*\|_2^2 + \sR^{\tt LS}_{\lambda} - \sN^{\tt LS}_{\lambda})^2d\\
    &~+ 2\|\bbeta_*\|_2^2((\|\bbeta_*\|_2^2 + \sR^{\tt LS}_{\lambda} - \sN^{\tt LS}_{\lambda})^2-4\|\bbeta_*\|_2^2\sR^{\tt LS}_{\lambda} ) \lambda\,.
\end{aligned}
\]
\end{proof}

\begin{proof}[Proof of \cref{prop:relation_minnorm_id}]
According to \cref{prop:asy_equiv_error_LR_minnorm} and \cref{prop:asy_equiv_norm_LR_minnorm}, for minimum $\ell_2$-norm estimator and $\bSigma = \id_d$, for the under-parameterized regime ($d<n$), we have
\[
\begin{aligned}
\sB_{\sR,0}^{\tt LS} = 0\,, \quad \sV_{\sR,0}^{\tt LS} = \frac{\sigma^2d}{n-d}\,; \quad \quad \sB_{\sN,0}^{\tt LS} = \|\bbeta_*\|_2^2\,, \quad \sV_{\sN,0}^{\tt LS} = \frac{\sigma^2d}{n-d}\,. 
\end{aligned}
\]
From these expressions, we can conclude that
\[
\begin{aligned}
    \sR_{0}^{\tt LS} = \sB_{\sR,0}^{\tt LS} + \sV_{\sR,0}^{\tt LS} = \frac{\sigma^2d}{n-d}\,; \quad \quad \sN_{0}^{\tt LS} = \sB_{\sN,0}^{\tt LS} + \sV_{\sN,0}^{\tt LS} = \|\bbeta_*\|_2^2 + \frac{\sigma^2d}{n-d}\,. 
\end{aligned}
\]
Finally, in the under-parameterized regime, it follows that
\begin{equation*}
 \sR_{0}^{\tt LS} = \sN_{0}^{\tt LS} - \|\bbeta_*\|_2^2\,.   
\end{equation*}

In the over-parameterized regime ($d>n$), the effective regularization $\lambda_*$ will have an explicit formulation as $\lambda_* = \frac{d-n}{n}$, thus for the bias and variance of the test error, we have
\[
\begin{aligned}
\sB_{\sR,0}^{\tt LS} = \frac{\lambda_n^2\<\bbeta_*,\bSigma(\bSigma+\lambda_n\id)^{-2}\bbeta_*\>}{1-n^{-1}\Tr(\bSigma^2(\bSigma+\lambda_n)^{-2})} = \frac{\lambda_n^2 \frac{1}{(1 + \lambda_n)^2}\|\bbeta_*\|_2^2}{1 - \frac{1}{n}\frac{d}{(1+\lambda_n)^2}} = \|\bbeta_*\|_2^2\frac{d-n}{d}\,,
\end{aligned}
\]
\[
\begin{aligned}
\sV_{\sR,0}^{\tt LS} = \frac{\sigma^2\Tr(\bSigma^2(\bSigma+\lambda_n\id)^{-2})}{n-\Tr(\bSigma^2(\bSigma+\lambda_n\id)^{-2})} = \frac{\sigma^2\frac{d}{(1+\lambda_n)^2}}{n-\frac{d}{(1+\lambda_n)^2}} = \sigma^2\frac{n}{d-n}\,,
\end{aligned}
\]
and combining the bias and variance, we have
\begin{align}\label{eq:r_under}
\sR_{0}^{\tt LS} = \sB_{\sR,0}^{\tt LS} + \sV_{\sR,0}^{\tt LS} = \|\bbeta_*\|_2^2\frac{d-n}{d} + \sigma^2\frac{n}{d-n}\,.
\end{align}
For the bias and variance of the norm, we have
\[
\begin{aligned}
\sB_{\sN,0}^{\tt LS} = \<\bbeta_*,\bSigma(\bSigma+\lambda_n\id)^{-1}\bbeta_*\> = \frac{1}{1+\lambda_n}\|\bbeta_*\|_2^2 = \|\bbeta_*\|_2^2\frac{n}{d}\,,
\end{aligned}
\]
\[
\begin{aligned}
\sV_{\sN,0}^{\tt LS} = \frac{\sigma\Tr(\bSigma(\bSigma+\lambda_n\id)^{-2})}{n-\Tr(\bSigma^2(\bSigma+\lambda_n\id)^{-2})} = \frac{\sigma^2\frac{d}{(1+\lambda_n)^2}}{n-\frac{d}{(1+\lambda_n)^2}} = \sigma^2\frac{n}{d-n}\,,
\end{aligned}
\]
and combining the bias and variance, we have

\begin{align}\label{eq:n_under}
\sN_{0}^{\tt LS} = \sB_{\sN,0}^{\tt LS} + \sV_{\sN,0}^{\tt LS} = \|\bbeta_*\|_2^2\frac{n}{d} + \sigma^2\frac{n}{d-n}\,.
\end{align}
Finally, combining \cref{eq:r_under} and \cref{eq:n_under}, we eliminate $n$ and thus obtain
\[
\begin{aligned}
    \sR_{0}^{\tt LS} = \sqrt{(\sN^{\tt LS}_0)^2 \!-\! 2(\|\bbeta_*\|_2^2 \!-\! \sigma^2)\sN^{\tt LS}_0 + (\|\bbeta_*\|_2^2 \!+\! \sigma^2)^2} \!-\!\sigma^2\,.
\end{aligned}
\]
By taking the derivative of $\sR_{0}^{\tt LS}$ with respect to $\sN_{0}^{\tt LS}$, we get
\[
\frac{\partial \sR_{0}^{\tt LS}}{\partial \sN_{0}^{\tt LS}} = \frac{\sN_{0}^{\tt LS} - (\|\bbeta_*\|_2^2 - \sigma^2)}{\sqrt{(\sN_{0}^{\tt LS})^2 - 2(\|\bbeta_*\|_2^2 - \sigma^2)\sN_{0}^{\tt LS} + (\|\bbeta_*\|_2^2 + \sigma^2)^2}}\,.
\]
From the derivative function, we observe that $\sR_{0}^{\tt LS}$ decreases monotonically with $\sN_{0}^{\tt LS}$ when $\sN_{0}^{\tt LS} < \|\boldsymbol{\beta}_*\|_2^2 - \sigma^2$, and increases monotonically with $\sN_{0}^{\tt LS}$ when $\sN_{0}^{\tt LS} > \|\boldsymbol{\beta}_*\|_2^2 - \sigma^2$.
\end{proof}

\begin{proof}[Proof of \cref{prop:relation_minnorm_underparam}]
According to \cref{prop:asy_equiv_error_LR_minnorm} and \cref{prop:asy_equiv_norm_LR_minnorm}, for minimum $\ell_2$-norm estimator, in the under-parameterized regime ($d<n$), we have
\[
\begin{aligned}
\sB_{\sR,0}^{\tt LS} = 0\,, \quad \sV_{\sR,0}^{\tt LS} = \frac{\sigma^2d}{n-d}\,; \quad \quad \sB_{\sN,0}^{\tt LS} = \|\bbeta_*\|_2^2\,, \quad \sV_{\sN,0}^{\tt LS} = \frac{\sigma^2\Tr(\bSigma^{-1})}{n-d}\,. 
\end{aligned}
\]
From these expressions, we can conclude that
\[
\begin{aligned}
    \sR_{0}^{\tt LS} = \sB_{\sR,0}^{\tt LS} + \sV_{\sR,0}^{\tt LS} = \frac{\sigma^2d}{n-d}\,; \quad \quad \sN_{0}^{\tt LS} = \sB_{\sN,0}^{\tt LS} + \sV_{\sN,0}^{\tt LS} = \|\bbeta_*\|_2^2 + \frac{\sigma^2\Tr(\bSigma^{-1})}{n-d}\,. 
\end{aligned}
\]
Finally, combing the above equation and eliminate \(n\), in the under-parameterized regime, it follows that
\begin{equation}\label{eq:rn_under}
 \sR_{0}^{\tt LS} = \frac{d}{\Tr(\bSigma^{-1})}\left(\sN_{0}^{\tt LS} - \|\bbeta_*\|_2^2\right)\,.   
\end{equation}
\end{proof}


\begin{proof}[Proof of \cref{prop:relation_minnorm_pl}]
In the over-parameterized regime ($d > n$), according to \cref{prop:asy_equiv_error_LR_minnorm} and \cref{prop:asy_equiv_norm_LR_minnorm}, under \cref{ass:powerlaw}, we have
\[
\begin{aligned}
    \sB_{\sR,0}^{\tt LS} =&~ \frac{\lambda_n^2\<\bbeta_*,\bSigma(\bSigma+\lambda_n\id)^{-2}\bbeta_*\>}{1-n^{-1}\Tr(\bSigma^2(\bSigma+\lambda_n\id)^{-2})} = \frac{\lambda_n^2\Tr(\bSigma^{1+\beta}(\bSigma+\lambda_n\id)^{-2})}{1-n^{-1}\Tr(\bSigma^2(\bSigma+\lambda_n\id)^{-2})}\,,\\
    \sV_{\sR,0}^{\tt LS} =&~ \frac{\sigma^2 \Tr\left(\bSigma^2 (\bSigma + \lambda_n\id)^{-2}\right)}{n - \Tr\left(\bSigma^2 (\bSigma + \lambda_n\id)^{-2}\right)}\,,\\
    \sB_{\sN,0}^{\tt LS} =&~ \<\bbeta_*,\bSigma(\bSigma+\lambda_n\id)^{-1}\bbeta_*\> = \Tr(\bSigma^{1+\beta}(\bSigma+\lambda_n\id)^{-1})\,,\\
    \sV_{\sN,0}^{\tt LS} =&~ \frac{\sigma^2 \Tr\left(\bSigma (\bSigma + \lambda_n\id)^{-2}\right)}{n - \Tr\left(\bSigma^2 (\bSigma + \lambda_n\id)^{-2}\right)}\,.
\end{aligned}
\]
To compute these quantities, here we introduce the following continuum approximations to eigensums.
\begin{equation}\label{eq:df1_inter_approx} 
\int_{1}^{d+1} \frac{k^{-\alpha}}{k^{-\alpha} + \lambda_n}\, \mathrm{d}k \leq \Tr(\bSigma(\bSigma+\lambda_n)^{-1}) = \sum_{i=1}^{d} \frac{\sigma_i}{\sigma_i + \lambda_n} \leq 
   \int_{0}^{d} \frac{k^{-\alpha}}{k^{-\alpha} + \lambda_n}\, \mathrm{d}k \,,
\end{equation}
due to the fact that the integrand is non-increasing function of $k$.
Similarly, we also have
\begin{equation}\label{eq:df2_inter_approx}\int_{1}^{d+1} \frac{k^{-2\alpha}}{(k^{-\alpha} + \lambda_n)^2}\, \mathrm{d}k \leq \Tr(\bSigma^2(\bSigma+\lambda_n)^{-2}) = \sum_{i=1}^{d} \frac{\sigma_i^2}{(\sigma_i + \lambda_n)^2} \leq  \int_{0}^{d} \frac{k^{-2\alpha}}{(k^{-\alpha} + \lambda_n)^2}\, \mathrm{d}k \,.
\end{equation}


We consider some special cases that are useful for discussion.
When $\alpha=1$, we have

\begin{equation}\label{eq:df1_inter_approx_alpha1}
   \frac{\log(1+d\lambda_n + \lambda_n) - \log (1+\lambda_n)}{\lambda_n}  \leq \Tr(\bSigma(\bSigma+\lambda_n)^{-1}) \leq  \frac{\log(1+d\lambda_n)}{\lambda_n} \,,
\end{equation}

\begin{equation}\label{eq:df2_inter_approx_alpha1}
\frac{d+1}{\lambda_n d +\lambda_n +1} - \frac{1}{\lambda_n+1} \leq    \Tr(\bSigma^2(\bSigma+\lambda_n)^{-2}) = \sum_{i=1}^{d} \frac{\sigma_i^2}{(\sigma_i + \lambda_n)^2} \leq \frac{d}{1+d\lambda_n}\,.
\end{equation}
Recall that \(\lambda_n\) is defined by \(\Tr(\bSigma(\bSigma + \lambda_n \id)^{-1}) = n\). Using \cref{eq:df1_inter_approx}, we have 
\[
\frac{\log(1 + d\lambda_n)}{\lambda_n} \approx n.
\]
Observe that as \(n \to d\), \(\lambda_n \to 0\), allowing us to apply a Taylor expansion:
\[
\frac{\log(1 + d\lambda_n)}{\lambda_n} \approx \frac{d\lambda_n - \frac{1}{2}(d\lambda_n)^2}{\lambda_n} = d - \frac{1}{2}d^2\lambda_n.
\]
Based on this approximation, \(\lambda_n\) can be expressed as
\[
\lambda_n \approx \frac{2(d - n)}{d^2}.
\]
In the following discussion, we consider the case $n \to d$. Thus, we have the approximation
\[
\Tr(\bSigma(\bSigma+\lambda_n)^{-1}) \approx n\,, \quad \Tr(\bSigma^2(\bSigma+\lambda_n)^{-2}) \approx \frac{d}{1+d\lambda_n}\,.
\]
Then we have
\[
\begin{aligned}
    \sV_{\sR,0}^{\tt LS} =&~ \frac{\sigma^2 \Tr\left(\bSigma^2 (\bSigma + \lambda_n\id)^{-2}\right)}{n - \Tr\left(\bSigma^2 (\bSigma + \lambda_n\id)^{-2}\right)} \approx \frac{\sigma^2\frac{d}{1+d\lambda_n}}{n-\frac{d}{1+d\lambda_n}} = \frac{\sigma^2 d}{n+d(n\lambda_n-1)}\,,\\
    \sV_{\sN,0}^{\tt LS} =&~ \frac{\sigma^2 \Tr\left(\bSigma (\bSigma + \lambda_n\id)^{-2}\right)}{n - \Tr\left(\bSigma^2 (\bSigma + \lambda_n\id)^{-2}\right)} \approx \frac{\sigma^2\frac{1}{\lambda_n}(d - \frac{1}{2}d^2\lambda_n - \frac{d}{1+d\lambda_n})}{n-\frac{d}{1+d\lambda_n}} = \frac{\sigma^2d^2(d\lambda_n-1)}{2(n+d(n\lambda_n-1))}\,.
\end{aligned}
\]
Use these two formulation to eliminate $n$, we obtain
\[
\sV_{\sR, 0}^{\tt LS} \approx \frac{2(\sV_{\sN, 0}^{\tt LS})^2}{d\sV_{\sN, 0}^{\tt LS}-d^2\sigma^2}\,.
\]

Next we discuss the situation under different $\beta$.

For $\beta=0$, we have
\[
\begin{aligned}
    \sB_{\sR,0}^{\tt LS} =&~ \frac{\lambda_n^2\Tr(\bSigma(\bSigma+\lambda_n\id)^{-2})}{1-n^{-1}\Tr(\bSigma^2(\bSigma+\lambda_n\id)^{-2})} \approx \frac{\lambda_n(d - \frac{1}{2}d^2\lambda_n - \frac{d}{1+d\lambda_n})}{1-\frac{d}{n(1+d\lambda_n)}} = n\lambda_n\,,\\
    \sB_{\sN,0}^{\tt LS} =&~ \Tr(\bSigma(\bSigma+\lambda_n\id)^{-1}) \approx d - \frac{1}{2}d^2\lambda_n\,,\\
\end{aligned}
\]
Use these two formulation to eliminate $n$, we obtain
\[
\sB_{\sR,0}^{\tt LS} \approx \frac{2\sB_{\sN, 0}^{\tt LS}(d-\sB_{\sN, 0}^{\tt LS})}{d^2}\,.
\]

For $\beta=1$, we have
\[
\begin{aligned}
    \sB_{\sR,0}^{\tt LS} =&~ \frac{\lambda_n^2\Tr(\bSigma^2(\bSigma+\lambda_n\id)^{-2})}{1-n^{-1}\Tr(\bSigma^2(\bSigma+\lambda_n\id)^{-2})} \approx \frac{\lambda_n^2\frac{d}{1+d\lambda_n}}{1-\frac{d}{n(1+d\lambda_n)}} = \frac{nd\lambda_n^2}{n(1+d\lambda_n)-d}\,,\\
    \sB_{\sN,0}^{\tt LS} =&~ \Tr(\bSigma^2(\bSigma+\lambda_n\id)^{-1}) = \Tr(\bSigma) - \lambda_n\Tr(\bSigma(\bSigma+\lambda_n\id)^{-1}) \approx \Tr(\bSigma) - n\lambda_n\,.\\
\end{aligned}
\]
Use these two formulation to eliminate $n$, we obtain
\[
\sB_{\sR,0}^{\tt LS} \approx \frac{2\sqrt{(\sB_{\sN, 0}^{\tt LS})^2-2\Tr(\bSigma)\sB_{\sN, 0}^{\tt LS}+\Tr(\bSigma)^2}}{\sqrt{d^2+2d^2\sB_{\sN, 0}^{\tt LS}-2d^2\Tr(\bSigma)}} = \frac{2(\sB_{\sN, 0}^{\tt LS} - \Tr(\bSigma))}{d\sqrt{1+2\sB_{\sN, 0}^{\tt LS}-2\Tr(\bSigma)}}\,.
\]

For $\beta=-1$, we need to use another two continuum approximations to eigensums
\begin{equation*}
    \Tr((\bSigma+\lambda_n)^{-1}) = \sum_{i=1}^{d} \frac{1}{\sigma_i + \lambda_n} \approx \int_{0}^{d} \frac{1}{k^{-\alpha} + \lambda_n}\, \mathrm{d}k = \frac{d\lambda_n - \log(1+d\lambda_n)}{\lambda_n^2}\,,
\end{equation*}
\begin{equation*}
    \Tr((\bSigma+\lambda_n)^{-2}) = \sum_{i=1}^{d} \frac{1}{(\sigma_i + \lambda_n)^2} \approx \int_{0}^{d} \frac{1}{(k^{-\alpha} + \lambda_n)^2}\, \mathrm{d}k = \frac{\frac{d\lambda_n(2+d\lambda_n)}{1+d\lambda_n}-2\log(1+d\lambda_n)}{\lambda_n^3}\,.
\end{equation*}
Once again, we apply the Taylor expansion, but this time expanding to the third order
\[
\log(1 + d\lambda_n) \approx d\lambda_n - \frac{1}{2}(d\lambda_n)^2 + \frac{1}{3}(d\lambda_n)^3\,.
\]
Then we have
\begin{equation*}
    \Tr((\bSigma+\lambda_n)^{-1}) \approx \frac{d\lambda_n - \log(1+d\lambda_n)}{\lambda_n^2} \approx \frac{1}{2}d^2-\frac{1}{3}d^3\lambda_n\,,
\end{equation*}
\begin{equation*}
    \Tr((\bSigma+\lambda_n)^{-2}) \approx \frac{\frac{d\lambda_n(2+d\lambda_n)}{1+d\lambda_n}-2\log(1+d\lambda_n)}{\lambda_n^3} = \frac{\frac{1}{3}d^3-\frac{2}{3}d^4\lambda_n}{1+d\lambda_n}\,.
\end{equation*}
Using the approximation sated above, we have
\[
\begin{aligned}
    \sB_{\sR,0}^{\tt LS} =&~ \frac{\lambda_n^2\Tr((\bSigma+\lambda_n\id)^{-2})}{1-n^{-1}\Tr(\bSigma^2(\bSigma+\lambda_n\id)^{-2})} \approx \frac{\lambda_n^2(\nicefrac{(\frac{1}{3}d^3-\frac{2}{3}d^4\lambda_n)}{(1+d\lambda_n)})}{1-\frac{d}{n(1+d\lambda_n)}} \,,\\
    \sB_{\sN,0}^{\tt LS} =&~ \Tr((\bSigma+\lambda_n\id)^{-1}) = \frac{1}{2}d^2-\frac{1}{3}d^3\lambda_n \,.\\
\end{aligned}
\]
Use these two formulation to eliminate $n$, we obtain
\[
\sB_{\sR,0}^{\tt LS} \approx \frac{216 (\sB_{\sN, 0}^{\tt LS})^4 \!-\! 324d^2 (\sB_{\sN, 0}^{\tt LS})^3 \!+\! 126d^4 (\sB_{\sN, 0}^{\tt LS})^2 \!+\! d^6 \sB_{\sN, 0}^{\tt LS} \!-\! 5d^8}{2d^5(6 \sB_{\sN, 0}^{\tt LS}-d^2)}\,.
\]
\end{proof}

Here we present some experimental results to check the relationship between $\sB_{\sR,0}^{\tt LS}$ and $\sB_{\sN,0}^{\tt LS}$, as well as $\sV_{\sR,0}^{\tt LS}$ and $\sV_{\sN,0}^{\tt LS}$, see \cref{fig:linear_regression_power_law}.
We can see that our approximate relationship on variance (see the {\color{red}red} line in \cref{fig:lrpld}) provides the precise estimation.
For the bias (see the left three figures of \cref{fig:linear_regression_power_law}), our approximate relationship is accurate if $\sB_{\sN,0}^{\tt LS}$ is large.






\section{Proofs for random feature ridge regression}\label{app:proof_rf}

In this section, we provide the proof of deterministic equivalence for random feature ridge regression in both the asymptotic (\cref{app:asy_deter_equiv_rf}) and non-asymptotic (\cref{app:nonasy_deter_equiv_rf}) settings. Additionally, we provide the proof of the relationship between test risk and the $\ell_2$ norm given in the main text, as detailed in \cref{app:relationship_rf}.

Though \citet{bach2024high}'s results are for linear regression, we can still deliver the asymptotic results for RFMs, which requires some knowledge from \cref{eq:det_equiv_phi2_main,eq:det_equiv_phi1_main}.

We firstly confirm that \cref{ass:asym} in \cref{app:pre_asy_deter_equiv}, used to derive all asymptotic results, can be replaced by the Hanson-Wright assumption employed in the non-asymptotic analysis.
It is evident that \cref{eq:trA1,eq:trA2} are obtained directly by taking the limits of \cref{eq:det_equiv_phi2_main,eq:det_equiv_phi1_main} as \(n \to \infty\).

Additionally, a key step in the proof of \cref{eq:trAB1,eq:trAB2} in \citet{bach2024high} involves showing that \(\Delta\) is almost surely negligible, where \(\Delta\) is defined as
\[
\Delta = \frac{1}{n} \sum_{i=1}^{n} \frac{\bx_i\bx_i^\sT(\hbSigma_{-i}-z\id)^{-1}-\bSigma(\hbSigma-z\id)^{-1}}{1 + \bx_i^\sT(n\widehat\bSigma_{-i}-nz\id)^{-1}\bx_i}\,,
\]
with \(\hbSigma = \frac{1}{n}\sum_{i=1}^{n}\bx_i\bx_i^\sT\), \(\hbSigma_{-i} = \frac{1}{n}\sum_{j\neq i}\bx_j\bx_j^\sT\), and \(z \in \R\).

In \citet{bach2024high}'s analysis, the negligibility of \(\Delta\) arises from the assumption that the components of \(\bx_i\) follow a sub-Gaussian distribution, which leads to the Hanson-Wright inequality
\[
\mathbb{P} \left[ \left| \bx_i^\sT \bx_i - \mathrm{tr}(\bSigma) \right| \leq c \left( t \|\bSigma\|_{\mathrm{op}} + \sqrt{t} \|\bSigma\|_F \right) \right] \geq 1 - 2e^{-t}.
\]

In this way, \cref{ass:concentrated_LR} is also sufficient to establish the negligibility of \(\Delta\).

After obtain \cref{eq:trA1,eq:trA2} and the negligibility of \(\Delta\), we can follow \citet{bach2024high}'s argument and derive the rest asymptotic deterministic equivalence.

Finally, with these observations, we can eliminate the reliance on \cref{ass:asym} and instead rely solely on \cref{ass:concentrated_LR} to derive all the asymptotic results.


\subsection{Asymptotic deterministic equivalence for random features ridge regression}
\label{app:asy_deter_equiv_rf}

In this section, we establish the asymptotic approximation guarantees for random feature regression in terms of its $\ell_2$-norm based capacity. Before presenting the proof of \cref{prop:asy_equiv_norm_RFRR}, we firstly give the proof of the bias-variance decomposition in \cref{lemma:biasvariance_rf}.

\begin{proof}[Proof of \cref{lemma:biasvariance_rf}]
Here we give the bias-variance decomposition of $\E_{\varepsilon}\|\hat{\ba}\|_2^2$. The formulation of $\E_{\varepsilon}\|\hat{\ba}\|_2^2$ is given by
\[
\E_{\varepsilon}\|\hat{\ba}\|_2^2 = \E_{\varepsilon} \|(\bZ^\sT \bZ + \lambda \id)^{-1} \bZ^\sT \by\|_2^2\,,
\]
which admits a similar bias-variance decomposition
\[
\begin{aligned}
    \E_{\varepsilon}\|\hat{\ba}\|_2^2 =&~ \E_{\varepsilon}\|(\bZ^\sT \bZ + \lambda \id)^{-1} \bZ^\sT (\bG \btheta_*+\bm\varepsilon)\|_2^2\\
    =&~ \|(\bZ^\sT \bZ + \lambda \id)^{-1} \bZ^\sT \bG \btheta_*\|_2^2 + \E_{\varepsilon}\|(\bZ^\sT \bZ + \lambda \id)^{-1} \bZ^\sT \bm\varepsilon\|_2^2\\
    =&~ \<\btheta_*, \bG^\sT \bZ (\bZ^\sT \bZ + \lambda\id)^{-2} \bZ^\sT \bG\btheta_* \> + \sigma^2\Tr\left(\bZ^\sT \bZ(\bZ^\sT \bZ + \lambda\id)^{-2}\right)\\
    =:&~ \mathcal{B}_{\mathcal{N},\lambda}^{\tt RFM} + \mathcal{V}_{\mathcal{N},\lambda}^{\tt RFM}\,.
\end{aligned}
\]
Accordingly, we conclude the proof.
\end{proof}

Now we are ready to present the proof of \cref{prop:asy_equiv_norm_RFRR} as below.

\begin{proof}[Proof of \cref{prop:asy_equiv_norm_RFRR}]
We give the asymptotic deterministic equivalents for the norm from the bias $\mathcal{B}_{\mathcal{N},\lambda}^{\tt RFM}$ and variance $\mathcal{V}_{\mathcal{N},\lambda}^{\tt RFM}$, respectively. We provide asymptotic expansions in two steps, by first considering the deterministic equivalent over $\bG$, and then over $\bF$.

Under \cref{ass:concentrated_RFRR}, we can apply \cref{prop:spectral,prop:spectral2,prop:spectralK,prop:spectralK2} directly in the proof below.

\paragraph{Deterministic equivalent over $\bG$:}
For the bias term, we use \cref{eq:trAB1K} in \cref{prop:spectralK} with $\bT=\bG$, $\bSigma=\bF^\sT\bF$, $\bA=\btheta_*\btheta_*^\sT$ and $\bB=\bF^\sT\bF$ and obtain
\begin{equation}\label{eq:bnrfm}
   \begin{split}
          \mathcal{B}_{\mathcal{N},\lambda}^{\tt RFM} =&~ \<\btheta_*, \bG^\sT \bZ (\bZ^\sT \bZ + \lambda\id)^{-2} \bZ^\sT \bG\btheta_* \>\\
    =&~ \Tr(\btheta_*^\sT \bG^\sT \bZ (\bZ^\sT \bZ + \lambda\id)^{-2} \bZ^\sT \bG\btheta_* )\\
    =&~ p\Tr(\btheta_* \btheta_*^\sT \bG^\sT ( \bG \bF^\sT \bF \bG^\sT + p\lambda\id)^{-1} \bG \bF^\sT \bF \bG^\sT ( \bG \bF^\sT \bF \bG^\sT + p\lambda\id)^{-1} \bG )\\
    \sim&~ p \underbrace{\Tr(\btheta_* \btheta_*^\sT ( \bF^\sT \bF + \nu_1\id)^{-1} \bF^\sT \bF ( \bF^\sT \bF + \nu_1\id)^{-1} )}_{\tt I_1} \\
    &~ + p\nu_1^2 \underbrace{\Tr(\btheta_* \btheta_*^\sT ( \bF^\sT \bF + \nu_1\id)^{-2})}_{:=I_2} \cdot \underbrace{\Tr(\bF^\sT \bF ( \bF^\sT \bF + \nu_1\id)^{-2})}_{:=I_3} \cdot \frac{1}{n-\widehat{\rm df}_2(\nu_1)} \,,
   \end{split} 
\end{equation}
where $\nu_1$ defined by $\nu_1(1-\frac{1}{n}\widehat{\rm df}_1(\nu_1)) \sim \frac{p\lambda}{n}$, $\widehat{\rm df}_1(\nu_1)$ and $\widehat{\rm df}_2(\nu_1)$ are degrees of freedom associated to $\bF^\sT \bF$ in \cref{def:df}.

For the variance term, we use \cref{eq:trA3K} with $\bT=\bG$ in \cref{prop:spectralK}, $\bA=\bF^\sT\bF$, $\bSigma=\bF^\sT\bF$ and obtain
\[
\begin{aligned}
    \mathcal{V}_{\mathcal{N},\lambda}^{\tt RFM} =&~ \sigma^2\Tr\left(\bZ^\sT \bZ(\bZ^\sT \bZ + \lambda\id)^{-2}\right) = \sigma^2\Tr\left(\bZ \bZ^\sT(\bZ \bZ^\sT + \lambda\id)^{-2}\right)\\    =&~\sigma^2p\Tr\left(\bG\bF^\sT\bF\bG^\sT(\bG\bF^\sT\bF\bG^\sT + p\lambda\id)^{-2}\right)\\
    \sim&~\sigma^2p\frac{\Tr(\bF^\sT\bF(\bF^\sT\bF+\nu_1\id)^{-2})}{n-\widehat{\rm df}_2(\nu_1)}\,.
\end{aligned}
\]

\paragraph{Deterministic equivalent over $\bF$:}

In the next, we aim to eliminate the randomness over $\bF$ in \cref{eq:bnrfm} from the bias part.
First our result depends on the asymptotic equivalents for $\widehat{\rm df}_1(\nu_1)$ and $\widehat{\rm df}_2(\nu_1)$. For $\widehat{\rm df}_1(\nu_1)$, we use \cref{eq:trA1} in \cref{prop:spectral} with $\bX=\bF$ and obtain
\[
\begin{aligned}
    \widehat{\rm df}_1(\nu_1) = \Tr(\bF^\sT \bF (\bF^\sT \bF + \nu_1\id)^{-1}) \sim \Tr(\bLambda(\bLambda + \nu_2\id)^{-1})={\rm df}_1(\nu_2)\,,
\end{aligned}
\]
where $\nu_2$ defined by $\nu_2(1-\frac{1}{p}{\rm df}_1(\nu_2)) \sim \frac{\nu_1}{p}$. Hence $\nu_1$ can be defined by $\nu_1(1-\frac{1}{n}{\rm df}_1(\nu_2))\sim\frac{p\lambda}{n}$ from \cref{eq:def_nu}.

For $\widehat{\rm df}_2(\nu_1)$, we use \cref{eq:trAB1} in \cref{prop:spectral} with $\bX=\bF$, $\bA=\bB=\id$ and obtain
\begin{equation}\label{eq:df2v1}
    \begin{split}
    \widehat{\rm df}_2(\nu_1) &=~ \Tr(\bF^\sT \bF (\bF^\sT \bF + \nu_1\id)^{-1} \bF^\sT \bF (\bF^\sT \bF + \nu_1\id)^{-1})\\
    &\sim~ \Tr(\bLambda^2(\bLambda + \nu_2\id)^{-2}) + \nu_2^2 \Tr(\bLambda(\bLambda + \nu_2\id)^{-2}) \cdot \Tr(\bLambda^2(\bLambda + \nu_2\id)^{-2}) \cdot \frac{1}{p - {\rm df}_2(\nu_2)}\\
    &=:~ n\Upsilon(\nu_1, \nu_2)\,. 
    \end{split}
\end{equation}

For $I_3:= \Tr(\bF^\sT \bF ( \bF^\sT \bF + \nu_1\id)^{-2})$, we use \cref{eq:trA3} with $\bX=\bF$ and obtain
\begin{align}\label{eq:I3}
\Tr(\bF^\sT \bF ( \bF^\sT \bF + \nu_1\id)^{-2}) \sim&~ \Tr(\bLambda(\bLambda + \nu_2\id)^{-2}) \cdot \frac{1}{p - {\rm df}_2(\nu_2)}\,.
\end{align}
Then we use \cref{eq:trA3} again with $\bX=\bF$, $\bA = \btheta_*\btheta_*^\sT$ to obtain the deterministic equivalent of $I_1$
\[
\begin{aligned}
\Tr(\btheta_* \btheta_*^\sT ( \bF^\sT \bF + \nu_1\id)^{-1} \bF^\sT \bF ( \bF^\sT \bF + \nu_1\id)^{-1}) =&~ \Tr(\btheta_* \btheta_*^\sT \bF^\sT \bF ( \bF^\sT \bF + \nu_1\id)^{-2})\\
\sim&~ \Tr(\btheta_* \btheta_*^\sT \bLambda ( \bLambda + \nu_2\id)^{-2}) \cdot \frac{1}{p - {\rm df}_2(\nu_2)}\\
=&~ \btheta_*^\sT \bLambda ( \bLambda + \nu_2\id)^{-2} \btheta_* \cdot \frac{1}{p - {\rm df}_2(\nu_2)}.
\end{aligned}
\]
Further, for $I_2$, use \cref{eq:trAB2} with $\bA=\btheta_*\btheta_*^\sT$ and $\bB=\id$, we obtain
\[
\begin{aligned}
\Tr(\btheta_* \btheta_*^\sT ( \bF^\sT \bF + \nu_1\id)^{-2}) \sim&~ \frac{\nu_2^2}{\nu_1^2}\Tr(\btheta_* \btheta_*^\sT (\bLambda + \nu_2\id)^{-2})\\
&~+ \frac{\nu_2^2}{\nu_1^2}\Tr(\btheta_* \btheta_*^\sT (\bLambda + \nu_2\id)^{-2} \bLambda) \cdot \Tr( (\bLambda + \nu_2\id)^{-2} \bLambda) \cdot \frac{1}{p - {\rm df}_2(\nu_2)}.
\end{aligned}
\]
Finally, combine the above equivalents, for the bias, we obtain
\[
\begin{aligned}
    \mathcal{B}_{\mathcal{N},\lambda}^{\tt RFM} \sim&~ p \btheta_*^\sT \bLambda ( \bLambda + \nu_2\id)^{-2} \btheta_* \cdot \frac{1}{p - {\rm df}_2(\nu_2)}\\
    &~+ p \nu_1^2 \left(\frac{\nu_2^2}{\nu_1^2}\Tr(\btheta_* \btheta_*^\sT (\bLambda + \nu_2\id)^{-2}) + \frac{\nu_2^2}{\nu_1^2}\Tr(\btheta_* \btheta_*^\sT (\bLambda + \nu_2\id)^{-2} \bLambda) \cdot \Tr( (\bLambda + \nu_2\id)^{-2} \bLambda) \cdot \frac{1}{p - {\rm df}_2(\nu_2)} \right)\\
    &~\cdot \Tr(\bLambda(\bLambda + \nu_2\id)^{-2}) \cdot \frac{1}{p - {\rm df}_2(\nu_2)} \cdot \frac{1}{n - n\Upsilon(\nu_1, \nu_2)}\\
    =&~ p\btheta_*^\sT \bLambda ( \bLambda + \nu_2\id)^{-2} \btheta_* \cdot \frac{1}{p - {\rm df}_2(\nu_2)}\\
    &~+ \frac{p}{n} \left(\nu_2^2 \btheta_*^\sT (\bLambda + \nu_2\id)^{-2} \btheta_* + \nu_2^2 \btheta_*^\sT \bLambda (\bLambda + \nu_2\id)^{-2} \btheta_* \cdot \Tr( \bLambda (\bLambda + \nu_2\id)^{-2} ) \cdot \frac{1}{p - {\rm df}_2(\nu_2)} \right)\\
    &~\cdot \Tr(\bLambda(\bLambda + \nu_2\id)^{-2}) \cdot \frac{1}{p - {\rm df}_2(\nu_2)} \cdot \frac{1}{1 - \Upsilon(\nu_1, \nu_2)}\\
    =&~\frac{p\< \btheta_*, \bLambda ( \bLambda + \nu_2\id)^{-2} \btheta_* \>}{p - \Tr\left(\bLambda^2 (\bLambda + \nu_2\id)^{-2}\right)} + \frac{p\chi(\nu_2)}{n} \cdot \frac{\nu_2^2\left[ \< \btheta_*, (\bLambda + \nu_2\id)^{-2} \btheta_* \> \!+\! \chi(\nu_2) \< \btheta_*, \bLambda (\bLambda + \nu_2\id)^{-2} \btheta_* \> \right]}{1 - \Upsilon(\nu_1, \nu_2)}\,.
\end{aligned}
\]
Similarly, for the variance, using \cref{eq:df2v1} and \cref{eq:I3} for $I_3$, we have
\[
\begin{aligned}
    \mathcal{V}_{\mathcal{N},\lambda}^{\tt RFM} \sim&~ \sigma^2 p \Tr(\bLambda(\bLambda + \nu_2\id)^{-2}) \cdot \frac{1}{p - {\rm df}_2(\nu_2)}\cdot \frac{1}{n-n\Upsilon(\nu_1,\nu_2)}\\
    \sim&~ \sigma^2 \frac{\frac{p}{n}\chi(\nu_2)}{1-\Upsilon(\nu_1, \nu_2)}\,.
\end{aligned}
\]
Accordingly, we finish the proof.
\end{proof}

In the next, we present the proof for min-$\ell_2$-norm interpolator under RFMs.

\begin{proof}[Proof of \cref{prop:asy_equiv_norm_RFRR_minnorm}]
Similar to linear regression, we separate the two regimes $p<n$ and $p>n$ as well. For both of them, we provide asymptotic expansions in two steps, first with respect to $\bG$ and then $\bF$ in the under-parameterized regime and vice-versa for the over-parameterized regime.
\paragraph{Under-parameterized regime: Deterministic equivalent over $\bG$} For the variance term, we can use \cref{eq:trA3K} with $\bT=\bG$, $\bSigma=\bF^\sT\bF$, $\bA=\bF^\sT\bF$ and obtain
\[
\begin{aligned}
\mathcal{V}_{\mathcal{N},0}^{\tt RFM} =&~ \sigma^2 \cdot \Tr(\bZ^\sT \bZ(\bZ^\sT \bZ + \lambda\id)^{-2})\\
=&~ \sigma^2 \cdot p\Tr(\bF\bG^\sT \bG \bF^\sT(\bF\bG^\sT \bG \bF^\sT + p\lambda\id)^{-2})\\
=&~ \sigma^2 \cdot p\Tr(\bF^\sT \bF\bG^\sT ( \bG \bF^\sT \bF \bG^\sT + p\lambda\id)^{-2}\bG )\\
\sim&~ \sigma^2 \cdot p\Tr(\bF^\sT \bF ( \bF^\sT \bF + \tilde\lambda\id)^{-2} ) \cdot \frac{1}{n-p}\\
\sim&~ \sigma^2 \cdot \Tr(( \bF \bF^\sT )^{-1}) \cdot \frac{p}{n-p}\,,\\
\end{aligned}
\]
where $\tilde\lambda$ is defined by
\begin{equation}\label{eq:tilde_lambda}
    \tilde\lambda(1-\frac{1}{n}\widetilde{\rm df}_1(\tilde\lambda)) \sim \frac{p\lambda}{n}\,,
\end{equation}
where $\widetilde{\rm df}_1(\tilde\lambda)$ and $\widetilde{\rm df}_2(\tilde\lambda)$ are degrees of freedom associated to $\bF^\sT \bF$. In the under-parameterized regime ($p<n$), when $\lambda$ goes to zero, we have $\tilde\lambda \to 0$ and  $\widetilde{\rm df}_2(\tilde\lambda) \to p$ \citep{bach2024high}.

For the bias term, we use \cref{eq:trAB1K} with $\bT=\bG$, $\bSigma=\bF^\sT\bF$, $\bA=\btheta_* \btheta_*^\sT$, $\bB=\bF^\sT\bF$ and then obtain
\[
\begin{aligned}
\mathcal{B}_{\mathcal{N},0}^{\tt RFM} =&~ \Tr(\btheta_*^\sT \bG^\sT \bZ (\bZ^\sT \bZ + \lambda\id)^{-2} \bZ^\sT \bG\btheta_* )\\
=&~ p\Tr(\btheta_*^\sT \bG^\sT \bG \bF^\sT (\bF \bG^\sT \bG \bF^\sT + p\lambda\id)^{-2} \bF \bG^\sT \bG \btheta_* )\\
=&~ p\Tr(\btheta_* \btheta_*^\sT \bG^\sT ( \bG \bF^\sT \bF \bG^\sT + p\lambda\id)^{-1} \bG \bF^\sT \bF \bG^\sT ( \bG \bF^\sT \bF \bG^\sT + p\lambda\id)^{-1} \bG )\\
\sim&~ p\Tr(\btheta_* \btheta_*^\sT ( \bF^\sT \bF + \tilde\lambda\id)^{-1} \bF^\sT \bF ( \bF^\sT \bF + \tilde\lambda\id)^{-1} )\\ 
&~+ p \tilde\lambda^2 \Tr(\btheta_* \btheta_*^\sT ( \bF^\sT \bF + \tilde\lambda\id)^{-2}) \cdot \Tr(\bF^\sT \bF  ( \bF^\sT \bF + \tilde\lambda\id)^{-2}) \cdot \frac{1}{n-p}\\
\sim&~ p\Tr(\btheta_* \btheta_*^\sT \bF^\sT ( \bF \bF^\sT )^{-2} \bF) + p \Tr(\btheta_* \btheta_*^\sT ( \id - \bF^\sT (\bF\bF^\sT)^{-1} \bF )) \cdot \Tr(( \bF \bF^\sT)^{-1}) \cdot \frac{1}{n-p}\,.
\end{aligned}
\]

In the next, we are ready to eliminate the randomness over $\bF$.
\paragraph{Under-parameterized regime: deterministic equivalent over $\bF$}
For the variance term, from \citet[Sec 3.2]{bach2024high} we know that $\nicefrac{1}{\lambda_p}$ is almost surely the limit of $\Tr((\bF\bF^\sT)^{-1})$, thus we have
\[
\begin{aligned}
\Tr((\bF\bF^\sT)^{-1}) \sim \frac{1}{\lambda_p}\,,
\end{aligned}
\]
where $\lambda_p$ defined by ${\rm df_1}(\lambda_p) = p$, where ${\rm df_1}(\lambda_p)$ and ${\rm df_2}(\lambda_p)$ are degrees of freedom associated to $\bLambda$. Hence we can obtain
\[
\begin{aligned}
\mathcal{V}_{\mathcal{N},0}^{\tt RFM} \sim \sigma^2 \cdot \frac{1}{\lambda_p} \cdot \frac{p}{n-p} = \frac{\sigma^2p}{\lambda_p(n-p)}\,.
\end{aligned}
\]

For the bias term, denote $\bD:=\bF\bLambda^{-1/2}$, we first use \cref{eq:trA3K} with $\bT=\bD$, $\bSigma=\bLambda$, $\bA=\bLambda^{1/2} \btheta_* \btheta_*^\sT \bLambda^{1/2}$ and obtain the deterministic equivalent of the first term in $
\mathcal{B}_{\mathcal{N},0}^{\tt RFM}$
\[
\begin{aligned}
\Tr(\btheta_* \btheta_*^\sT \bF^\sT ( \bF \bF^\sT )^{-2} \bF) = \Tr( \bLambda^{1/2} \btheta_* \btheta_*^\sT \bLambda^{1/2} \bD^\sT ( \bD \bLambda \bD^\sT )^{-2} \bD) \sim \Tr( \btheta_* \btheta_*^\sT \bLambda ( \bLambda + \lambda_p )^{-2} ) \cdot \frac{1}{n-{\rm df}_2(\lambda_p)}\,.
\end{aligned}
\]
Then we use \cref{eq:trAB1K} with $\bT=\bD$, $\bSigma=\bLambda$, $\bA=\bLambda^{1/2} \btheta_* \btheta_*^\sT \bLambda^{1/2}$ and obtain 
\[
\begin{aligned}
\Tr(\btheta_* \btheta_*^\sT \bF^\sT (\bF \bF^\sT)^{-1} \bF) = \Tr( \bLambda^{1/2} \btheta_* \btheta_*^\sT \bLambda^{1/2} \bD^\sT ( \bD \bLambda \bD^\sT )^{-1} \bD) \sim \Tr(\btheta_* \btheta_*^\sT \bLambda (\bLambda +\lambda_p)^{-1})\,,
\end{aligned}
\]
Then the deterministic equivalent of the second term in $\mathcal{B}_{\mathcal{N},0}^{\tt RFM} $ is given by
\[
\begin{aligned}
\Tr(\btheta_* \btheta_*^\sT ( \id - \bF^\sT (\bF\bF^\sT)^{-1} \bF )) \sim \lambda_p \btheta_*^\sT (\bLambda +\lambda_p)^{-1} \btheta_*.
\end{aligned}
\]
Finally, combine the above equivalents and we have
\[
\begin{aligned}
\mathcal{B}_{\mathcal{N},0}^{\tt RFM} \sim&~ \btheta_*^\sT \bLambda (\bLambda +\lambda_p)^{-2} \btheta_* \cdot \frac{p}{n-{\rm df}_2(\lambda_p)} + \btheta_*^\sT (\bLambda +\lambda_p)^{-1} \btheta_* \cdot \frac{p}{n-p}\\
=&~ \frac{p\<\btheta_*, \bLambda (\bLambda +\lambda_p)^{-2} \btheta_*\>}{n-\Tr(\bLambda^2(\bLambda+\lambda_n\id)^{-2})} + \frac{p\<\btheta_*, (\bLambda +\lambda_p)^{-1} \btheta_*\>}{n-p}\,.
\end{aligned}
\]
\paragraph{Over-parameterized regime: deterministic equivalent over $\bF$}

Denote $ \bK:=\bLambda^{1/2}\bG^\sT\bG\bLambda^{1/2}$, for the variance term, we use \cref{eq:trA3K} with $\bT=\bD$, $\bSigma=\bA=\bK$ and obtain 
\[
\begin{aligned}
\mathcal{V}_{\mathcal{N},0}^{\tt RFM} =&~ \sigma^2 \cdot p\Tr(\bF\bG^\sT \bG \bF^\sT(\bF\bG^\sT \bG \bF^\sT + p\lambda\id)^{-2})\\
=&~ \sigma^2 \cdot p\Tr(\bK \bD^\sT (\bD \bK \bD^\sT + p\lambda\id)^{-2} \bD)\\
\sim&~ \sigma^2 \cdot p\Tr(\bK (\bK + \hat\lambda\id)^{-2}) \cdot \frac{1}{p-n}\\
\sim&~ \sigma^2 \cdot \Tr( (\bG \bLambda \bG^\sT )^{-1}) \cdot \frac{p}{p-n}\,,
\end{aligned}
\]
where $\hat\lambda$ is defined by
\begin{equation}\label{eq:hat_lambda}
    \hat\lambda(1-\frac{1}{n}\widehat{\rm df}_1(\hat\lambda)) \sim \frac{p\lambda}{n}\,,
\end{equation}
where $\widehat{\rm df}_1(\hat\lambda)$ and $\widehat{\rm df}_2(\hat\lambda)$ are degrees of freedom associated to $\bK$. In the over-parameterized regime ($p>n$), when $\lambda$ goes to zero, we have $\hat\lambda \to 0$ and  $\widehat{\rm df}_2(\hat\lambda) \to n$ \citep{bach2024high}.

For the bias term, we use \cref{eq:trA3K} with $\bT=\bD$, $\bSigma=\bK$, $\bA=\bLambda^{1/2} \bG^\sT \bG \btheta_* \btheta_*^\sT \bG^\sT \bG \bLambda^{1/2}$ and obtain 
\[
\begin{aligned}
\mathcal{B}_{\mathcal{N},0}^{\tt RFM} =&~ p\Tr(\btheta_*^\sT \bG^\sT \bG \bF^\sT (\bF \bG^\sT \bG \bF^\sT + p\lambda\id)^{-2} \bF \bG^\sT \bG \btheta_* )\\
=&~ p\Tr(\bLambda^{1/2} \bG^\sT \bG \btheta_* \btheta_*^\sT \bG^\sT \bG \bLambda^{1/2} \bD (\bD \bK \bD^\sT + p\lambda\id)^{-2} \bD )\\
\sim&~ p\Tr(\bLambda^{1/2} \bG^\sT \bG \btheta_* \btheta_*^\sT \bG^\sT \bG \bLambda^{1/2} (\bK + \hat\lambda\id)^{-2} ) \cdot \frac{1}{p-n}\\
\sim&~ \Tr( \btheta_* \btheta_*^\sT \bG^\sT (\bG \bLambda \bG^\sT)^{-1} \bG ) \cdot \frac{p}{p-n}\,.
\end{aligned}
\]

\paragraph{Over-parameterized regime: deterministic equivalent over $\bG$}

For the variance term, we have
\[
\begin{aligned}
\mathcal{V}_{\mathcal{N},0}^{\tt RFM} \sim \sigma^2 \cdot \frac{1}{\lambda_n} \cdot \frac{p}{p-n} = \frac{\sigma^2p}{\lambda_n(p-n)}.
\end{aligned}
\]

For the bias term, we have
\[
\begin{aligned}
\mathcal{B}_{\mathcal{N},0}^{\tt RFM} \sim&~ \Tr( \btheta_* \btheta_*^\sT ( \bLambda + \lambda_n)^{-1} ) \cdot \frac{p}{p-n}\\
=&~ \btheta_*^\sT ( \bLambda + \lambda_n)^{-1} \btheta_* \cdot \frac{p}{p-n}\\
=&~ \frac{p\<\btheta_*, ( \bLambda + \lambda_n)^{-1} \btheta_*\>}{p-n}\,.
\end{aligned}
\]
Finally, we conclude the proof.
\end{proof}

To build the connection between the test risk and norm for the min-$\ell_2$-norm estimator for random features regression, we also need the deterministic equivalent of the test risk as below.

\begin{proposition}[Asymptotic deterministic equivalence of the test risk of the min-$\ell_2$-norm interpolator]\label{prop:asy_equiv_error_RFRR_minnorm}
    Under \cref{ass:concentrated_RFRR}, for the minimum $\ell_2$-norm estimator $\hat{\ba}_{\min}$, we have the following deterministic equivalence: for the under-parameterized regime ($p<n$), we have
    \[
    \begin{aligned}
        \mathcal{B}^{\tt RFM}_{\mathcal{R},0} \sim \frac{n\lambda_p \<\btheta_*, (\bLambda +\lambda_p\id)^{-1} \btheta_*\>}{n-p}\,,\quad \mathcal{V}^{\tt RFM}_{\mathcal{R},0} \sim&~ \frac{\sigma^2p}{n-p}\,,
    \end{aligned}
    \]
    where $\lambda_p$ is defined by $\Tr(\bLambda(\bLambda+\lambda_p\id)^{-1}) \sim p$. In the over-parameterized regime ($p>n$), we have
    \[
    \begin{aligned}
        \mathcal{B}^{\tt RFM}_{\mathcal{R},0} \sim&~ \frac{n\lambda_n^2 \<\btheta_*, ( \bLambda + \lambda_n \id)^{-2} \btheta_*\>}{ n - \Tr(\bLambda^2(\bLambda+\lambda_n\id)^{-2})} + \frac{n\lambda_n \<\btheta_*, ( \bLambda + \lambda_n\id)^{-1} \btheta_*\>}{p-n}\,,\\
        \mathcal{V}^{\tt RFM}_{\mathcal{R},0} \sim&~  \frac{\sigma^2\Tr(\bLambda^2(\bLambda+\lambda_n\id)^{-2})}{n - \Tr(\bLambda^2(\bLambda+\lambda_n\id)^{-2})} + \frac{\sigma^2n}{p-n}\,,
    \end{aligned}
    \]
    where $\lambda_n$ is defined by $\Tr(\bLambda(\bLambda+\lambda_n\id)^{-1}) \sim n$.
\end{proposition}

\begin{proof}[Proof of \cref{prop:asy_equiv_error_RFRR_minnorm}]
For the proof, we separate the two regimes $p<n$ and $p>n$. For both of them, we provide asymptotic expansions in two steps, first with respect to $\bG$ and then $\bF$ in the under-parameterized regime and vice-versa for the over-parameterized regime.


\paragraph{Under-parameterized regime: deterministic equivalent over $\bG$}

For the variance term, in the under-parameterized regime, when $\lambda \to 0$, the variance term will become $\mathcal{V}^{\tt RFM}_{\mathcal{R},0} = \sigma^2 \cdot \Tr(\widehat{\bLambda}_{\bF} (\bZ^\sT \bZ)^{-1})$. Accordingly, using \citet[Eq. (12)]{bach2024high}, we have 
\[
\begin{aligned}
\mathcal{V}^{\tt RFM}_{\mathcal{R},0} =&~ \sigma^2 \cdot \Tr(\widehat{\bLambda}_{\bF} (\bZ^\sT \bZ)^{-1})\\
=&~ \sigma^2 \cdot \Tr(\bF\bF^\sT(\bF\bG^\sT\bG\bF^\sT)^{-1})\\
\sim&~ \frac{\sigma^2}{n-p} \cdot \Tr(\bF\bF^\sT(\bF\bF^\sT)^{-1})\\
=&~\frac{\sigma^2p}{n-p}\,.
\end{aligned}
\]

For the bias term, it can be decomposed into
\[
\begin{aligned}
\mathcal{B}^{\tt RFM}_{\mathcal{R},0} =&~ \|\btheta_* - p^{-1/2} \bF^\sT (\bZ^\sT \bZ + \lambda\id)^{-1} \bZ^\sT \bm{G} \btheta_*\|_2^2\\
=&~ \btheta_*^\sT \btheta_* -2 p^{-1/2}\btheta_*^\sT \bF^\sT (\bZ^\sT \bZ + \lambda\id)^{-1} \bZ^\sT \bm{G} \btheta_* + \btheta_*^\sT \bG^\sT \bZ (\bZ^\sT \bZ + \lambda\id)^{-1} \widehat{\bLambda}_{\bF} (\bZ^\sT \bZ + \lambda\id)^{-1} \bZ^\sT \bm{G} \btheta_*.
\end{aligned}
\]
For the second term: $p^{-1/2}\btheta_*^\sT \bF^\sT (\bZ^\sT \bZ + \lambda\id)^{-1} \bZ^\sT \bm{G} \btheta_*$, we can use \cref{eq:trA1K} with $\bT=\bG$, $\bSigma=\bF^\sT\bF$, $\bA=\btheta_*\btheta_*^\sT \bF^\sT \bF$ and obtain
\[
\begin{aligned}
p^{-1/2}\btheta_*^\sT \bF^\sT (\bZ^\sT \bZ + \lambda\id)^{-1} \bZ^\sT \bm{G} \btheta_* =&~ \Tr( \btheta_*\btheta_*^\sT \bF^\sT \bF \bG^\sT (\bG\bF^\sT\bF\bG^\sT + p\lambda\id)^{-1} \bG)\\
\sim&~ \Tr( \btheta_*\btheta_*^\sT \bF^\sT \bF (\bF^\sT\bF + \tilde\lambda\id)^{-1})\\
\sim&~ \Tr( \btheta_*\btheta_*^\sT \bF^\sT (\bF \bF^\sT)^{-1}\bF)\,,
\end{aligned}
\]
where the implicit regularization parameter $\tilde\lambda$ is defined by \cref{eq:tilde_lambda}.

For the third term: $\btheta_*^\sT \bG^\sT \bZ (\bZ^\sT \bZ + \lambda\id)^{-1} \widehat{\bLambda}_{\bF} (\bZ^\sT \bZ + \lambda\id)^{-1} \bZ^\sT \bm{G} \btheta_*$, we can use \cref{eq:trAB1K} with $\bT=\bG$, $\bSigma=\bF^\sT\bF$, $\bA=\btheta_*\btheta_*^\sT$, $\bB=\bF^\sT\bF\bF^\sT\bF$ and obtain
\[
\begin{aligned}
&~\btheta_*^\sT \bG^\sT \bZ (\bZ^\sT \bZ + \lambda\id)^{-1} \widehat{\bLambda}_{\bF} (\bZ^\sT \bZ + \lambda\id)^{-1} \bZ^\sT \bm{G} \btheta_*\\
=&~\Tr(\btheta_* \btheta_*^\sT \bG^\sT \bG \bF^\sT(\bF \bG^\sT \bG \bF^\sT + p\lambda\id)^{-1} \bF\bF^\sT (\bF \bG^\sT \bG \bF^\sT + p\lambda\id)^{-1} \bF \bG^\sT \bG )\\
=&~\Tr(\btheta_* \btheta_*^\sT \bG^\sT ( \bG \bF^\sT \bF \bG^\sT + p\lambda\id)^{-1} \bG \bF^\sT \bF\bF^\sT \bF \bG^\sT ( \bG \bF^\sT \bF \bG^\sT + p\lambda\id)^{-1} \bG )\\
\sim&~ \Tr(\btheta_* \btheta_*^\sT (\bF^\sT \bF + \tilde\lambda\id)^{-1} \bF^\sT \bF\bF^\sT \bF (\bF^\sT \bF + \tilde\lambda\id)^{-1})\\
&~+ \tilde\lambda^2 \Tr(\btheta_* \btheta_*^\sT (\bF^\sT \bF + \tilde\lambda\id)^{-2}) \cdot \Tr(\bF^\sT \bF\bF^\sT \bF (\bF^\sT \bF + \tilde\lambda\id)^{-2}) \cdot \frac{1}{n-p}\\
\sim&~ \Tr(\btheta_* \btheta_*^\sT \bF^\sT (\bF \bF^\sT)^{-1} \bF) + \Tr(\btheta_* \btheta_*^\sT (\id -\bF^\sT (\bF \bF^\sT)^{-1} \bF )) \cdot \frac{p}{n-p}\,.
\end{aligned}
\]
Combining the above equivalents, we have
\[
\begin{aligned}
\mathcal{B}^{\tt RFM}_{\mathcal{R},0} =&~ \btheta_*^\sT \btheta_* -\Tr(\btheta_* \btheta_*^\sT \bF^\sT (\bF \bF^\sT)^{-1} \bF) + \Tr(\btheta_* \btheta_*^\sT (\id -\bF^\sT (\bF \bF^\sT)^{-1} \bF )) \cdot \frac{p}{n-p}\\
=&~ \btheta_*^\sT \btheta_* \cdot \frac{n}{n-p} -\Tr(\btheta_* \btheta_*^\sT \bF^\sT (\bF \bF^\sT)^{-1} \bF) \cdot \frac{n}{n-p}\,.
\end{aligned}
\]
\paragraph{Under-parameterized regime: deterministic equivalent over $\bF$}
For the bias term, we can use \cref{eq:trA1K} with $\bT = \bD := \bF \bLambda^{-1/2}$, $\bA=\bLambda^{1/2} \btheta_* \btheta_*^\sT \bLambda^{1/2}$ and obtain
\[
\begin{aligned}
\Tr(\btheta_* \btheta_*^\sT \bF^\sT (\bF \bF^\sT)^{-1} \bF) =&~ \Tr(\bLambda^{1/2} \btheta_* \btheta_*^\sT \bLambda^{1/2} \bD^\sT (\bD \bLambda \bD^\sT)^{-1} \bD )\\
\sim&~ \Tr(\bLambda^{1/2} \btheta_* \btheta_*^\sT \bLambda^{1/2} (\bLambda +\lambda_p)^{-1})\\
=&~ \btheta_*^\sT \bLambda (\bLambda +\lambda_p)^{-1} \btheta_*\,.
\end{aligned}
\]
Thus, we finally obtain
\[
\begin{aligned}
\mathcal{B}^{\tt RFM}_{\mathcal{R},0} \sim&~ \btheta_*^\sT \btheta_* \cdot \frac{n}{n-p} - \btheta_*^\sT \bLambda (\bLambda +\lambda_p)^{-1} \btheta_* \cdot \frac{n}{n-p}\\
=&~ \lambda_p \btheta_*^\sT (\bLambda +\lambda_p)^{-1} \btheta_* \cdot \frac{n}{n-p}\\
=&~ \frac{n\lambda_p \<\btheta_*, (\bLambda +\lambda_p\id)^{-1} \btheta_*\>}{n-p}\,.
\end{aligned}
\]
\paragraph{Over-parameterized regime: deterministic equivalent over $\bF$}
For the variance term, with $\bD := \bF \bLambda^{-1/2}$ and $\bK := \bLambda^{1/2} \bG^\sT \bG \bLambda^{1/2}$ we can obtain
\[
\begin{aligned}
\mathcal{V}^{\tt RFM}_{\mathcal{R},0} &= \sigma^2 \cdot \mathrm{Tr}(\widehat{\bLambda}_{\bF} \bZ^\sT \bZ (\bZ^\sT \bZ + \lambda\id)^{-2})\\
&= \sigma^2 \cdot \mathrm{Tr}(\bF \bF^\sT \bF \bG^\sT \bG \bF^\sT (\bF \bG^\sT \bG \bF^\sT + p\lambda\id)^{-2})\\
&= \sigma^2 \cdot \mathrm{Tr}(\bD \bLambda \bD^\sT \bD \bLambda^{1/2} \bG^\sT \bG \bLambda^{1/2} \bD^\sT (\bD \bLambda^{1/2} \bG^\sT \bG \bLambda^{1/2} \bD^\sT + p\lambda\id)^{-2})\\
&= \sigma^2 \cdot \mathrm{Tr}(\bLambda \bD^\sT (\bD \bK \bD^\sT + p\lambda\id)^{-1} \bD \bK \bD^\sT (\bD \bK \bD^\sT + p\lambda\id)^{-1} \bD )\,,
\end{aligned}
\]
then we directly use \cref{eq:trAB1K} with $\bT=\bD$, $\bSigma=\bK$, $\bA=\bLambda$, $\bB=\bK$ and obtain
\[
\begin{aligned}
&~\mathrm{Tr}(\bLambda \bD^\sT (\bD \bK \bD^\sT + p\lambda\id)^{-1} \bD \bK \bD^\sT (\bD \bK \bD^\sT + p\lambda\id)^{-1} \bD )\\
\sim&~ \mathrm{Tr}(\bLambda ( \bK + \hat\lambda\id)^{-1} \bK ( \bK + \hat\lambda\id)^{-1} ) + \hat\lambda^2 \mathrm{Tr}(\bLambda ( \bK + \hat\lambda\id)^{-2} ) \cdot \mathrm{Tr}( \bK ( \bK + \hat\lambda\id)^{-2} ) \cdot \frac{1}{p-n}\\
\sim&~ \Tr(\bLambda^2 \bG^\sT (\bG \bLambda \bG^\sT)^{-2} \bG ) + \mathrm{Tr}(\bLambda ( \id - \bLambda^{1/2}\bG^\sT (\bG \bLambda \bG^\sT)^{-1} \bG \bLambda^{1/2} ) ) \cdot \mathrm{Tr}( (\bG \bLambda \bG^\sT)^{-1} ) \cdot \frac{1}{p-n}\,,
\end{aligned}
\]
where the implicit regularization parameter $\hat\lambda$ is defined by \cref{eq:hat_lambda}.

For the bias term, first we have
\[
\begin{aligned}
p^{-1/2}\btheta_*^\sT \bF^\sT (\bZ^\sT \bZ + \lambda\id)^{-1} \bZ^\sT \bm{G} \btheta_* =&~ \Tr( \btheta_*\btheta_*^\sT \bF^\sT (\bF\bG^\sT \bG\bF^\sT+ p\lambda\id)^{-1} \bF \bG^\sT \bG)\\
=&~ \Tr(\bLambda^{1/2} \bG^\sT \bG \btheta_*\btheta_*^\sT \bLambda^{1/2} \bD^\sT (\bD \bK \bD^\sT+ p\lambda\id)^{-1} \bD )\,,
\end{aligned}
\]
then we use \cref{eq:trA1K} with $\bT=\bD$, $\bSigma=\bK$, $\bA=\bLambda^{1/2} \bG^\sT \bG \btheta_*\btheta_*^\sT \bLambda^{1/2}$ and obtain
\[
\begin{aligned}
\Tr(\bLambda^{1/2} \bG^\sT \bG \btheta_*\btheta_*^\sT \bLambda^{1/2} \bD^\sT (\bD \bK \bD^\sT+ p\lambda\id)^{-1} \bD ) \sim&~ \Tr( \btheta_*\btheta_*^\sT \bLambda \bG^\sT ( \bG \bLambda \bG^\sT)^{-1} \bG)\,.
\end{aligned}
\]
Furthermore, we use \cref{eq:trAB1K} with $\bT=\bD$, $\bSigma=\bK$, $\bA=\bLambda^{1/2} \bG^\sT \bG \btheta_* \btheta_*^\sT \bG^\sT \bG \bLambda^{1/2}$, $\bB=\bLambda$ and obtain
\[
\begin{aligned}
&~\btheta_*^\sT \bG^\sT \bZ (\bZ^\sT \bZ + \lambda\id)^{-1} \widehat{\bLambda}_{\bF} (\bZ^\sT \bZ + \lambda\id)^{-1} \bZ^\sT \bm{G} \btheta_*\\
=&~\Tr(\bLambda^{1/2} \bG^\sT \bG \btheta_* \btheta_*^\sT \bG^\sT \bG \bLambda^{1/2} \bD^\sT(\bD \bK \bD^\sT + p\lambda\id)^{-1} \bD \bLambda \bD^\sT (\bD \bK \bD^\sT + p\lambda\id)^{-1} \bD )\\
\sim&~ \Tr(\bLambda^{1/2} \bG^\sT \bG \btheta_* \btheta_*^\sT \bG^\sT \bG \bLambda^{1/2} ( \bK + \hat\lambda\id)^{-1} \bLambda ( \bK + \hat\lambda\id)^{-1} )\\
&~+ \hat\lambda^2 \Tr(\bLambda^{1/2} \bG^\sT \bG \btheta_* \btheta_*^\sT \bG^\sT \bG \bLambda^{1/2} ( \bK + \hat\lambda\id)^{-2} ) \cdot \Tr( \bLambda ( \bK + \hat\lambda\id)^{-2} ) \cdot \frac{1}{p-n}\\
\sim&~ \Tr( \btheta_* \btheta_*^\sT \bG^\sT ( \bG \bLambda \bG^\sT )^{-1} \bG \bLambda^2 \bG^\sT ( \bG \bLambda \bG^\sT )^{-1} \bG)\\
&~+ \Tr( \btheta_* \btheta_*^\sT \bG^\sT ( \bG \bLambda \bG^\sT)^{-1} \bG) \cdot \Tr(\bLambda ( \id - \bLambda^{1/2}\bG^\sT (\bG \bLambda \bG^\sT)^{-1} \bG \bLambda^{1/2} ) ) \cdot \frac{1}{p-n}\,.
\end{aligned}
\]
In the next, we are ready to eliminate the randomness over $\bG$.
\paragraph{Over-parameterized regime: deterministic equivalent over $\bG$}
For the variance term, we use \cref{eq:trA3K} to obtain
\[
\begin{aligned}
\Tr(\bLambda^2 \bG^\sT (\bG \bLambda \bG^\sT)^{-2} \bG ) \sim \frac{{\rm df}_2(\lambda_n)}{n - {\rm df}_2(\lambda_n)}\,.
\end{aligned}
\]
Then we use \cref{eq:trA1K} to obtain
\[
\begin{aligned}
\Tr(\bLambda^2 \bG^\sT (\bG \bLambda \bG^\sT)^{-1} \bG ) \sim \Tr(\bLambda^2(\bLambda + \lambda_n)^{-1}),
\end{aligned}
\]
where $\lambda_n$ is defined by ${\rm df_1}(\lambda_n) = n$. Hence we have
\[
\begin{aligned}
\Tr(\bLambda ( \id - \bLambda^{1/2}\bG^\sT (\bG \bLambda \bG^\sT)^{-1} \bG \bLambda^{1/2} ) ) \sim n\lambda_n.
\end{aligned}
\]
Combine the above equivalents, we have
\[
\begin{aligned}
\mathcal{V}^{\tt RFM}_{\mathcal{R},0} \sim&~  \sigma^2 \cdot  \frac{{\rm df}_2(\lambda_n)}{n - {\rm df}_2(\lambda_n)} + \sigma^2 \cdot \frac{n}{p-n}\\
=&~ \frac{\sigma^2\Tr(\bLambda^2(\bLambda+\lambda_n\id)^{-2})}{n - \Tr(\bLambda^2(\bLambda+\lambda_n\id)^{-2})} + \frac{\sigma^2n}{p-n}\,.
\end{aligned}
\]

For the bias term, we first use \cref{eq:trA1K} to obtain
\[
\begin{aligned}
\Tr( \btheta_*\btheta_*^\sT \bLambda \bG^\sT ( \bG \bLambda \bG^\sT)^{-1} \bG) \sim \Tr(\btheta_*\btheta_*^\sT \bLambda (\bLambda + \lambda_n)^{-1})\,.
\end{aligned}
\]
Moreover, we use \cref{eq:trAB1K} to obtain
\[
\begin{aligned}
\Tr( \btheta_* \btheta_*^\sT \bG^\sT ( \bG \bLambda \bG^\sT )^{-1} \bG \bLambda^2 \bG^\sT ( \bG \bLambda \bG^\sT )^{-1} \bG) \sim&~ \Tr(\btheta_* \btheta_*^\sT \bLambda^2 ( \bLambda + \lambda_n )^{-2})\\ 
&~+ \lambda_n^2 \cdot \Tr(\btheta_* \btheta_*^\sT ( \bLambda + \lambda_n )^{-2}) \cdot \frac{{\rm df}_2(\lambda_n)}{n - {\rm df}_2(\lambda_n)}.
\end{aligned}
\]
Accordingly, we finally conclude that
\[
\begin{aligned}
\mathcal{B}^{\tt RFM}_{\mathcal{R},0} \sim&~ \lambda_n^2 \btheta_*^\sT ( \bLambda + \lambda_n \id)^{-2} \btheta_* \cdot \frac{n}{ n - {\rm df}_2(\lambda_n)} + \lambda_n \btheta_*^\sT ( \bLambda+ \lambda_n\id)^{-1} \btheta_* \cdot \frac{n}{p-n}\\
=&~ \frac{n\lambda_n^2 \<\btheta_*, ( \bLambda + \lambda_n \id)^{-2} \btheta_*\>}{ n - \Tr(\bLambda^2(\bLambda+\lambda_n\id)^{-2})} + \frac{n\lambda_n \<\btheta_*, ( \bLambda + \lambda_n\id)^{-1} \btheta_*\>}{p-n}\,.
\end{aligned}
\]
\end{proof}




\subsection{Non-asymptotic deterministic equivalence for random features ridge regression}
\label{app:nonasy_deter_equiv_rf}

Here we present the proof for the non-asymptotic results on the variance and then discuss the related results on bias due to the insufficient deterministic equivalence.

\subsubsection{Proof on the variance term}

\begin{theorem}[Deterministic equivalence of variance part of the $\ell_2$ norm]\label{prop:det_equiv_RFRR_V}
    Assume the features $\{\bz_i\}_{i \in [n]}$ and $\{\boldf_j\}_{j \in [p]}$ satisfy \cref{ass:concentrated_RFRR} with a constant $C_* > 0$. Then for any $D,K > 0$, there exist constant $\eta_* \in (0, 1/2)$ and $C_{*,D,K} > 0$ ensuring the following property holds. For any $n,p \geq C_{*,D,K}$, $\lambda > 0$, if the following condition is satisfied:
    \begin{equation*}
        \lambda  \geq n^{-K}, \quad \gamma_\lambda \geq p^{-K}, \quad \trho_{\lambda} (n,p)^{5/2} \log^{3/2} (n) \leq K \sqrt{n}\,, \quad \trho_\lambda (n,p)^2 \cdot \rho_{\gamma_+} (p)^{7} \log^4 (p) \leq K \sqrt{p}\,,
    \end{equation*}
    then with probability at least $1-n^{-D}-p^{-D}$, we have that
    \[
    \begin{aligned}
        \left|\mathcal{V}_{\mathcal{N},\lambda}^{\tt RFM} - \sV_{\sN,\lambda}^{\tt RFM}\right| \leq&~ C_{x, D, K} \cdot \mathcal{E}_V(n, p) \cdot \sV_{\sN,\lambda}^{\tt RFM}\,.
    \end{aligned}
    \]
    where the approximation rate is given by
    \[
    \mathcal{E}_V(n, p) := \frac{\widetilde{\rho}_\lambda(n, p)^6 \log^{5/2}(n)}{\sqrt{n}} + \frac{\widetilde{\rho}_\lambda(n, p)^2 \cdot \rho_{\gamma_+}(p)^7 \log^3(p)}{\sqrt{p}}.
    \]
\end{theorem}

\begin{proof}[Proof of \cref{prop:det_equiv_RFRR_V}]
First, note that $\mathcal{V}_{\mathcal{N},\lambda}^{\tt RFM}$ can be written in terms of the functional $\Phi_4$ defined in \cref{eq:functionals_Z}:
\[
\mathcal{V}_{\mathcal{N},\lambda}^{\tt RFM} = \sigma^2 \cdot n \Phi_4 ( \bZ ; \hbLambda^{-1}_\bF,\lambda).
\]
Recall that $\cA_\cF$ is the event defined in \citet[Eq. (79)]{defilippis2024dimension}. Under the assumptions, we have
\[
\P (\cA_\cF) \geq 1 - p^{-D}.
\]
Hence, applying \cref{prop:det_Z} for $\bF \in \cA_\cF$ and via union bound, we obtain that with probability at least $1 - p^{-D} - n^{-D}$,
\begin{equation}\label{eq:var_remove_Z}
\left| n\Phi_4 ( \bZ ; \hbLambda^{-1}_\bF,\lambda) - n\tPhi_5 ( \bF ; \hbLambda^{-1}_\bF, p\nu_1 )\right|  \leq C_{*,D,K} \cdot \cE_1 (p,n) \cdot n\tPhi_5 ( \bF ; \hbLambda^{-1}_\bF, p\nu_1 ),
\end{equation}
and we recall the expressions
\[
n\tPhi_5 ( \bF ; \hbLambda^{-1}_\bF, p\nu_1 ) = \frac{\tPhi_6 ( \bF ; \hbLambda^{-1}_\bF , p\nu_1) }{n - \tPhi_6 ( \bF ; \id , p\nu_1)}, \quad \quad \tPhi_6 ( \bF ; \hbLambda^{-1}_\bF , p\nu_1) = p\Tr \big( \bF \bF^\sT ( \bF \bF^\sT + p \nu_1)^{-2} \big).
\]
From \citet[Lemma B.11]{defilippis2024dimension}, we have with probability at least $1 - p^{-D}$
\[
\begin{aligned}
\left| p\Tr (\bF \bF^\sT ( \bF \bF^\sT + p\nu_1)^{-2} ) - p^2\Psi_3 ( \nu_2 ; \bLambda^{-1} )\right| \leq&~ C_{*,D,K} \cdot \rho_{\gamma_+} (p) \cdot \cE_3 (p) \cdot p^2\Psi_3 ( \nu_2 ; \bLambda^{-1} )\,,
\end{aligned}
\]
where the approximation rate \(\cE_3 (p)\) is given by
\[
    \mathcal{E}_3(p) := \frac{\rho_{\gamma_+}(p)^6 \log^3(p)}{\sqrt{p}}.
\]


Furthermore, from the proof of \citet[Theorem B.12]{defilippis2024dimension}, we have with probability at least $1 - p^{-D}$,
\[
\left| (1 - n^{-1} \tPhi_6 (\bF; \id, p\nu_1))^{-1} - (1 -  \Upsilon (\nu_1,\nu_2) )^{-1} \right| \leq C_{*,D,K} \cdot \trho_\lambda (n,p) \rho_{\gamma_+} (p) \cE_3 (p) \cdot  (1 -   \Upsilon (\nu_1,\nu_2))^{-1}.
\]
Combining those two bounds, we obtain
\[
\left| \frac{\tPhi_6 ( \bF ; \hbLambda^{-1}_\bF , p\nu_1) }{n - \tPhi_6 ( \bF ; \id , p\nu_1)} - \frac{p^2\Psi_3 ( \nu_2 ; \bLambda^{-1} )}{n - n\Upsilon (\nu_1,\nu_2)} \right| \leq C_{*,D,K} \cdot \trho_\lambda (n,p) \rho_{\gamma_+} (p) \cE_3 (p) \cdot \frac{p^2\Psi_3 ( \nu_2 ; \bLambda^{-1} )}{n - n\Upsilon (\nu_1,\nu_2)}.
\]
Finally, we can combine this bound with \cref{eq:var_remove_Z} to obtain via union bound that with probability at least $1 - n^{-D} - p^{-D}$, 
\[
\left| n\Phi_4 ( \bZ ; \hbLambda^{-1}_\bF,\lambda) - \frac{p^2\Psi_3 ( \nu_2 ; \bLambda^{-1} )}{n - n\Upsilon (\nu_1,\nu_2)} \right| \leq C_{*,D,K} \left\{ \cE_1 (p,n) + \trho_\lambda (n,p) \rho_{\gamma_+} (p) \cE_3 (p) \right\} \frac{p^2\Psi_3 ( \nu_2 ; \bLambda^{-1} )}{n - n\Upsilon (\nu_1,\nu_2)}\,.
\]
Replacing the rate $\cE_j$ by their expressions conclude the proof of this theorem.
\end{proof}

\subsubsection{Discussion on the bias term}\label{app:discuss_bias}

We present the deterministic equivalence of the bias term as an informal result, without a Existing deterministic equivalence results appear insufficient to directly establish this desired bias result. While we believe this is doable under  additional assumptions, a complete proof is beyond the scope of this paper..

In the proof of the bias term, deterministic equivalences for functionals of the form 
\[
\Tr \left( \bA \left( \bX^\sT \bX \right)^2 (\bX^\sT \bX + \lambda)^{-2} \right)
\]
are required. However, such equivalences are currently unavailable, necessitating the introduction of technical assumptions to leverage the deterministic equivalences of \(\Phi_2(\bX; \bA, \lambda)\) and \(\Phi_4(\bX; \bA, \lambda)\).

Furthermore, the proof of the bias term in \cite{defilippis2024dimension} suggests that deriving deterministic equivalences for the bias of the \(\ell_2\) norm, analogous to \citet[Proposition B.7]{defilippis2024dimension}, is also required but remains unresolved.

Addressing these gaps in deterministic equivalence is an important direction for future work, particularly to establish rigorous proofs for the currently missing results.



\subsection{Proofs on relationship between test risk and \texorpdfstring{$\ell_2$}{L2} norm of random feature ridge regression estimator}
\label{app:relationship_rf}

To derive the relationship between test risk and norm for the random feature model, we first examine the linear relationship in the over-parameterized regime. Next, we analyze the case where \(\bLambda = \id_m\) with \(n < m < \infty\) (finite rank), followed by the relationship under the power-law assumption.



\subsubsection{Proof for min-norm interpolator in the over-parameterized regime}
According to the formulation in \cref{prop:asy_equiv_norm_RFRR_minnorm} and \cref{prop:asy_equiv_error_RFRR_minnorm}, we have for the under-parameterized regime ($p<n$), we have
\[
\begin{aligned}
    \mathcal{B}^{\tt RFM}_{\mathcal{N},0} \sim& \sB^{\tt RFM}_{\sN,0} = \frac{p\<\btheta_*, \bLambda (\bLambda +\lambda_p\id)^{-2} \btheta_*\>}{n - \Tr(\bLambda^2(\bLambda+\lambda_p\id)^{-2})} + \frac{p\<\btheta_*, (\bLambda +\lambda_p\id)^{-1} \btheta_*\>}{n-p}\,,\\
    \mathcal{V}^{\tt RFM}_{\mathcal{N},0} \sim& \sV^{\tt RFM}_{\sN,0} = \frac{\sigma^2p}{\lambda_p(n-p)}\,,
\end{aligned}
\]
\[
\begin{aligned}
    \mathcal{B}^{\tt RFM}_{\mathcal{R},0} \sim \sB^{\tt RFM}_{\sR,0} = \frac{n\lambda_p \<\btheta_*, (\bLambda +\lambda_p\id)^{-1} \btheta_*\>}{n-p}\,,\quad \mathcal{V}^{\tt RFM}_{\mathcal{R},0} \sim \sV^{\tt RFM}_{\sR,0} = \frac{\sigma^2p}{n-p}\,.
\end{aligned}
\]
In the over-parameterized regime ($p>n$), we have
\[
\begin{aligned}
    \mathcal{B}^{\tt RFM}_{\mathcal{N},0} \sim \sB^{\tt RFM}_{\sN,0} = \frac{p\<\btheta_*, ( \bLambda + \lambda_n\id)^{-1} \btheta_*\>}{p-n}\,,\quad
    \mathcal{V}^{\tt RFM}_{\mathcal{N},0} \sim \sV^{\tt RFM}_{\sN,0} = \frac{\sigma^2p}{\lambda_n(p-n)}\,,
\end{aligned}
\]
\[
\begin{aligned}
    \mathcal{B}^{\tt RFM}_{\mathcal{R},0} \sim&~ \sB^{\tt RFM}_{\sR,0} = \frac{n\lambda_n^2 \<\btheta_*, ( \bLambda + \lambda_n \id)^{-2} \btheta_*\>}{ n - \Tr(\bLambda^2(\bLambda+\lambda_n\id)^{-2})} + \frac{n\lambda_n \<\btheta_*, ( \bLambda + \lambda_n\id)^{-1} \btheta_*\>}{p-n}\,,\\
    \mathcal{V}^{\tt RFM}_{\mathcal{R},0} \sim&~  \sV^{\tt RFM}_{\sR,0} = \frac{\sigma^2\Tr(\bLambda^2(\bLambda+\lambda_n\id)^{-2})}{n - \Tr(\bLambda^2(\bLambda+\lambda_n\id)^{-2})} + \frac{\sigma^2n}{p-n}\,.
\end{aligned}
\]
With these formulations we can introduce the relationship between test risk and norm in the over-parameterized regime as follows. 

\begin{proof}[Proof of \cref{prop:relation_minnorm_overparam}]
In the over-parameterized regime ($p > n$), we have
\[
    \sN_{0}^{\tt RFM} 
    = 
    \sB^{\tt RFM}_{\sN,0} + \sV^{\tt RFM}_{\sN,0} 
    = 
    \frac{p\<\btheta_*, ( \bLambda + \lambda_n\id)^{-1} \btheta_*\>}{p-n} + \frac{\sigma^2p}{\lambda_n(p-n)} 
    = 
    \left[\<\btheta_*, ( \bLambda + \lambda_n\id)^{-1} \btheta_*\> + \frac{\sigma^2}{\lambda_n}\right] \frac{p}{p-n}\,.
\]
\[
\begin{aligned}
    \sR_{0}^{\tt RFM} 
    =&
    \frac{n\lambda_n^2 \<\btheta_*, ( \bLambda + \lambda_n \id)^{-2} \btheta_*\>}{ n - \Tr(\bLambda^2(\bLambda+\lambda_n\id)^{-2})} 
    + 
    \frac{n\lambda_n \<\btheta_*, ( \bLambda + \lambda_n\id)^{-1} \btheta_*\>}{p-n}
    + 
    \frac{\sigma^2\Tr(\bLambda^2(\bLambda+\lambda_n\id)^{-2})}{n - \Tr(\bLambda^2(\bLambda+\lambda_n\id)^{-2})} 
    + 
    \frac{\sigma^2n}{p-n}\\
    =& 
    \frac{n\lambda_n^2 \<\btheta_*, ( \bLambda + \lambda_n \id)^{-2} \btheta_*\> + \sigma^2\Tr(\bLambda^2(\bLambda+\lambda_n\id)^{-2})}{ n - \Tr(\bLambda^2(\bLambda+\lambda_n\id)^{-2})} 
    + 
    \left[n\lambda_n \<\btheta_*, ( \bLambda + \lambda_n\id)^{-1} \btheta_*\> + \sigma^2n\right]\frac{1}{p-n}\,.
\end{aligned}
\]
Then we eliminate $p$ and obtain that the deterministic equivalents of the estimator's test risk and norm, $\sR^{\tt RFM}_{0}$ and $\sN^{\tt RFM}_{0}$, in over-parameterized regimes ($p>n$) admit
\[
\sR_{0}^{\tt RFM} 
= 
\lambda_n\sN_{0}^{\tt RFM} 
- 
\left[\lambda_n\<\btheta_*, ( \bLambda + \lambda_n\id)^{-1} \btheta_*\> + \sigma^2\right] 
+ 
\frac{n\lambda_n^2 \<\btheta_*, ( \bLambda + \lambda_n \id)^{-2} \btheta_*\> + \sigma^2\Tr(\bLambda^2(\bLambda+\lambda_n\id)^{-2})}{ n - \Tr(\bLambda^2(\bLambda+\lambda_n\id)^{-2})}\,. 
\]
\end{proof}


\subsubsection{Proof on isotropic features with finite rank}
Here we present the proof of \cref{prop:relation_minnorm_id_rf} with \( \bLambda = \id_m \).

\begin{proof}[Proof of \cref{prop:relation_minnorm_id_rf}]
Here we consider the case where \( \bLambda = \id_m \). Under this condition, the definitions of \( \lambda_p \) and \( \lambda_n \) above are simplified to \( \frac{m}{1+\lambda_p} = p \) and \( \frac{m}{1+\lambda_n} = n \), respectively. Consequently, \( \lambda_p \) and \( \lambda_n \) have explicit expressions given by \( \lambda_p = \frac{m-p}{p} \) and \( \lambda_n = \frac{m-n}{n} \), respectively.

First, in the over-parameterized regime ($p>n$), we have
\[
\begin{aligned}
     \sB^{\tt RFM}_{\sN,0} = \frac{p\frac{1}{1+\lambda_n}\|\btheta_*\|_2^2}{p - n} = \frac{np}{m(p-n)} \|\btheta_*\|_2^2\,,\quad
    \sV^{\tt RFM}_{\sN,0} = \frac{\sigma^2p}{\lambda_n(p-n)} = \frac{\sigma^2np}{(m-n)(p-n)}\,.
\end{aligned}
\]
\[
\begin{aligned}
     \sB^{\tt RFM}_{\sR,0} =& \frac{n\lambda_n^2 \frac{1}{(1+\lambda_n)^2}\|\btheta_*\|_2^2}{n-\frac{m}{(1+\lambda_n)^2}} + \frac{n\lambda_n\frac{1}{1+\lambda_n}\|\btheta_*\|_2^2}{p-n} = \frac{p(m-n)}{m(p-n)} \|\btheta_*\|_2^2\,,\\ 
     \sV^{\tt RFM}_{\sR,0} =& \frac{\sigma^2\frac{m}{(1+\lambda_n)^2}}{n-\frac{m}{(1+\lambda_n)^2}}+\frac{\sigma^2n}{p-n}=\frac{\sigma^2n}{m-n} + \frac{\sigma^2n}{p-n}\,.
\end{aligned}
\]
We eliminate $p$ and obtain that the relationship between $\sV^{\tt RFM}_{\sR,0}$ and $\sV^{\tt RFM}_{\sN,0}$ is
\[
\begin{aligned}
\sV^{\tt RFM}_{\sR,0} = \frac{m-n}{n}\sV^{\tt RFM}_{\sN,0} + \frac{2n -m}{m-n} \sigma^2\,.
\end{aligned}
\]
similarly, the relationship between $\sB^{\tt RFM}_{\sR,0}$ and $\sB^{\tt RFM}_{\sN,0}$ is
\[
\begin{aligned}
\sB^{\tt RFM}_{\sR,0} = \frac{m-n}{n}\sB^{\tt RFM}_{\sN,0}\,.
\end{aligned}
\]
Combining the above two relationship, we obtain the relationship between test risk $\sR_{0}^{\tt RFM}$ and norm $\sN_{0}^{\tt RFM}$ as

\[
\begin{aligned}
\sR_{0}^{\tt RFM} = \frac{m-n}{n} \sN_{0}^{\tt RFM} +\frac{2n-m}{m-n} \sigma^2.
\end{aligned}
\]
Accordingly, in the under-parameterized regime ($p<n$), we have
\[
\begin{aligned}
     \sB^{\tt RFM}_{\sN,0} =& \frac{p\frac{1}{(1+\lambda_p)^2}\|\btheta_*\|_2^2}{n - \frac{m}{(1+\lambda_p)^2}} + \frac{p\frac{1}{1+\lambda_p}\|\btheta_*\|_2^2}{n-p} = \frac{p}{m} \left( \frac{p^2}{nm - p^2} + \frac{p}{n-p} \right) \|\btheta_*\|_2^2\,,\\
    \sV^{\tt RFM}_{\sN,0} =& \frac{\sigma^2p}{\lambda_p(p-n)} = \frac{\sigma^2p^2}{(m-p)(n-p)}\,.
\end{aligned}
\]
\[
\begin{aligned}
     \sB^{\tt RFM}_{\sR,0} = \frac{n\lambda_p \frac{1}{1+\lambda_p}\|\btheta_*\|_2^2}{n-p} = \frac{n(m-p)}{m(n-p)} \|\btheta_*\|_2^2\,,\quad \sV^{\tt RFM}_{\sR,0} = \frac{\sigma^2p}{n-p}\,.
\end{aligned}
\]
Then we eliminate $p$ and obtain that, in the under-parameterized regime ($p<n$), the relationship between $\sV^{\tt RFM}_{\sR,0}$ and $\sV^{\tt RFM}_{\sN,0}$ is
\[
\begin{aligned}
\sV^{\tt RFM}_{\sR,0} = \frac{(m-n) \sV^{\tt RFM}_{\sN,0} + \sqrt{(m-n)^2(\sV^{\tt RFM}_{\sN,0})^2 + 4nm\sigma^2\sV^{\tt RFM}_{\sN,0}}}{2n}\,,
\end{aligned}
\]
which can be further simplified as a hyperbolic function
\begin{equation*}
     \left(\sV^{\tt RFM}_{\sR,0} \right)^2 = \frac{m-n}{n} \sV^{\tt RFM}_{\sR,0} \sV^{\tt RFM}_{\sN,0} + \frac{m \sigma^2}{n} \sV^{\tt RFM}_{\sN,0}\,,
\end{equation*}
and the asymptote of this hyperbola is $\sV^{\tt RFM}_{\sR,0} = \frac{m-n}{n}\sV^{\tt RFM}_{\sN,0} + \frac{m}{m-n} \sigma^2$.

Besides, we eliminate $p$ and obtain the relationship between $\sB^{\tt RFM}_{\sR,0}$ and $\sB^{\tt RFM}_{\sN,0}$ as
\[
\begin{aligned}
&\frac{\|\btheta_*\|_2^6 n^2 \left( 2 \|\btheta_*\|_2^2 + \sB^{\tt RFM}_{\sN,0} - \frac{\sB^{\tt RFM}_{\sN,0} n}{m} \right)}{m} 
=
(\sB^{\tt RFM}_{\sR,0})^4 n 
+ 
(\sB^{\tt RFM}_{\sR,0})^2 \|\btheta_*\|_2^2 n \left( \|\btheta_*\|_2^2 + \sB^{\tt RFM}_{\sN,0} - \frac{4 \|\btheta_*\|_2^2 n}{m} - \frac{\sB^{\tt RFM}_{\sN,0} n}{m} \right)\\
& + \sB^{\tt RFM}_{\sR,0} \|\btheta_*\|_2^4 n \left( \sB^{\tt RFM}_{\sN,0} + \frac{5 \|\btheta_*\|_2^2 n}{m} - \frac{\sB^{\tt RFM}_{\sN,0} n}{m} \right)
+ 
(\sB^{\tt RFM}_{\sR,0})^3 \left( -\sB^{\tt RFM}_{\sN,0} m - 2 \|\btheta_*\|_2^2 n + \sB^{\tt RFM}_{\sN,0} n + \frac{\|\btheta_*\|_2^2 n^2}{m} \right),
\end{aligned}
\]
which can be simplified to
\[
\begin{aligned}
    \sB^{\tt RFM}_{\sN,0}(m-n)(m\sB^{\tt RFM}_{\sR,0}-n\|\btheta_*\|_2^2)(m(\sB^{\tt RFM}_{\sR,0})^2-n\|\btheta_*\|_2^4) 
    = 
    nm(\sB^{\tt RFM}_{\sR,0}-\|\btheta_*\|_2^2)^2(m(\sB^{\tt RFM}_{\sR,0})^2-2n\|\btheta_*\|_2^4+n\|\btheta_*\|_2^2\sB^{\tt RFM}_{\sR,0})\,.
\end{aligned}
\]
We can find that in this case, the relationship can be easily written as
\[
\begin{aligned}
    \sB^{\tt RFM}_{\sN,0} =& \frac{nm(\sB^{\tt RFM}_{\sR,0}-\|\btheta_*\|_2^2)^2(m(\sB^{\tt RFM}_{\sR,0})^2-2n\|\btheta_*\|_2^4+n\|\btheta_*\|_2^2\sB^{\tt RFM}_{\sR,0})}{(m-n)(m\sB^{\tt RFM}_{\sR,0}-n\|\btheta_*\|_2^2)(m(\sB^{\tt RFM}_{\sR,0})^2-n\|\btheta_*\|_2^4)}\,.
\end{aligned}
\]
Next we will show that when $p \to n$, which also implies that $\sB^{\tt RFM}_{\sN,0} \to \infty$ and $\sB^{\tt RFM}_{\sR,0} \to \infty$, this relationship is approximately linear.

Recall that the relationship between $\sB^{\tt RFM}_{\sR,0}$ and $\sB^{\tt RFM}_{\sN,0}$ is given by \(\sB^{\tt RFM}_{\sR,0} = \frac{(m-n)}{n}\sB^{\tt RFM}_{\sN,0}\), and is equivalent to \(\sB^{\tt RFM}_{\sN,0} = \frac{n}{(m-n)}\sB^{\tt RFM}_{\sR,0} := f(\sB^{\tt RFM}_{\sR,0})\). We then do a difference and get
\[
\sB^{\tt RFM}_{\sN,0} - f(\sB^{\tt RFM}_{\sR,0}) = \frac{nm(\sB^{\tt RFM}_{\sR,0}-\|\btheta_*\|_2^2)^2(m(\sB^{\tt RFM}_{\sR,0})^2-2n\|\btheta_*\|_2^4+n\|\btheta_*\|_2^2\sB^{\tt RFM}_{\sR,0})}{(m-n)(m\sB^{\tt RFM}_{\sR,0}-n\|\btheta_*\|_2^2)(m(\sB^{\tt RFM}_{\sR,0})^2-n\|\btheta_*\|_2^4)} - \frac{n}{m-n}\sB^{\tt RFM}_{\sR,0}\,,
\]
then take \(\sB^{\tt RFM}_{\sR,0} \to \infty\) and we get
\[
\lim_{\sB^{\tt RFM}_{\sR,0} \to \infty}\sB^{\tt RFM}_{\sN,0} - f(\sB^{\tt RFM}_{\sR,0}) = -\frac{2n}{m}\|\btheta_*\|_2^2\,.
\]
Finally, organizing this equation and we get
\[
    \sB^{\tt RFM}_{\sR,0} \approx \frac{m-n}{n}\sB^{\tt RFM}_{\sN,0} + \frac{2(m-n)}{m}\|\btheta_*\|_2^2\,.
\]
\end{proof}

\subsubsection{Proof on features under power law assumption}

\begin{proof}[Proof of \cref{prop:relation_minnorm_powerlaw_rf}]
First, we use integral approximation to give approximations to some quantities commonly used in deterministic equivalence to prepare for the subsequent derivations.

According to the integral approximation in \citet[Lemma 1]{simonmore}, we have
\begin{equation}\label{eq:integrate_approx1}
    \Tr(\bLambda (\bLambda + \nu_2)^{-1}) \approx C_1 \nu_2^{-\frac{1}{\alpha}},\quad \Tr(\bLambda^2 (\bLambda + \nu_2)^{-2}) \approx C_2 \nu_2^{-\frac{1}{\alpha}},\quad \Tr(\bLambda (\bLambda + \nu_2)^{-2}) \approx (C_1-C_2) \nu_2^{-\frac{1}{\alpha}-1}\,,
\end{equation}
where $C_1$ and $C_2$ are
\begin{equation}\label{eq:C1C2}
    C_1 = \frac{\pi}{\alpha \sin\left(\nicefrac{\pi}{\alpha}\right)}\,,\quad C_2 = \frac{\pi(\alpha-1)}{\alpha^2 \sin\left(\nicefrac{\pi}{\alpha}\right)}\,,\quad \text{with $C_1 > C_2$}\,.
\end{equation}
Besides, according to definition of \(T(\nu)\) \cref{app:pre_scaling_law}, we have 
\[
\begin{aligned}
\< \btheta_*, (\bLambda + \nu_2)^{-1} \btheta_* \> =&~ T^1_{2r,1}(\nu_2) \approx C_3 \nu_2^{(2r-1)\wedge0},\\
\< \btheta_*, \bLambda(\bLambda + \nu_2)^{-2} \btheta_* \> =&~ T^1_{2r+1,2}(\nu_2) \approx C_4 \nu_2^{(2r-1)\wedge0}.\\
\end{aligned}
\]
When $r \in (0, \frac{1}{2})$, according to the integral approximation, we have
\begin{equation}
    C_3 = \frac{\pi}{\alpha \sin(2\pi r)}\,,\quad C_4 = \frac{2\pi r}{\alpha \sin(2\pi r)}\,,\quad \text{with $C_3 > C_4$}.
\end{equation}
Otherwise, if $r \in [\frac{1}{2}, \infty)$, we have
\[
\frac{1}{\alpha(2r-1)}< C_3 < \frac{1}{\alpha(2r-1)}+1\,,\quad \frac{1}{\alpha(2r-1)}< C_4 < \frac{1}{\alpha(2r-1)}+1\,,\quad \text{with $C_3 > C_4$}.
\]
For $\< \btheta_*, (\bLambda + \nu_2)^{-2} \btheta_* \>$, we have to discuss its approximation in the case $r \in (0, \frac{1}{2})$, $r \in [\frac{1}{2}, 1)$ and $r \in [\frac{1}{2}, \infty)$ separately.
\[
\langle \bm{\theta}_*, (\bm{\Lambda} + \nu_2)^{-2} \bm{\theta}_* \rangle \approx
\begin{cases} 
    (C_3 - C_4) \nu_2^{2r - 2}, & \text{if } r \in (0, \frac{1}{2})\,; \\
    C_5 \nu_2^{2r-2}, & \text{if } r \in [\frac{1}{2}, 1)\,; \\
    C_6, & \text{if } r \in [1, \infty)\,,
\end{cases}
\]
where $\frac{1}{2\alpha(r-1)} < C_6 < \frac{1}{2\alpha(r-1)} + 1$.


With the results of the integral approximation above, we next derive the relationship between $\sR_0^{\tt RFM}$ and $\sN_0^{\tt RFM}$ {\bf separately in over-parameterized regime ($p > n$) and under-parameterized regime ($p < n$).}

\paragraph{The relationship in over-parameterized regime ($p > n$)}

According to the self-consistent equation
\[
\begin{aligned}
1 + \frac{n}{p} - \sqrt{\left( 1 - \frac{n}{p} \right)^2 + \frac{4\lambda}{p\nu_2}} = \frac{2}{p} \operatorname{Tr} \left( \bLambda \left( \bLambda + \nu_2 \right)^{-1} \right),
\end{aligned}
\]
\[
\begin{aligned}
\nu_1 = \frac{\nu_2}{2} \left[ 1 - \frac{n}{p} + \sqrt{\left( 1 - \frac{n}{p} \right)^2 + \frac{4\lambda}{p\nu_2}} \right],
\end{aligned}
\]
In the over-parameterized regime (\(p > n\)), as \(\lambda \to 0\), for the first equation, \(\frac{4\lambda}{p\nu_2}\) will approach \(0\), and \(\Tr(\bLambda(\bLambda + \nu_2)^{-1})\) will converge to \(n\). Consequently, by \cref{eq:integrate_approx1}, \(\nu_2\) will converge to the constant \((\frac{n}{C_1})^{-\alpha}\). Furthermore, from the second equation, \(\nu_1\) will converge to \(\nu_2(1 - \frac{n}{p})\). Thus, according to \cref{eq:integrate_approx1}, we have
\[
\begin{aligned}
\Tr(\bLambda (\bLambda + \nu_2)^{-1}) \approx n,\quad \Tr(\bLambda^2 (\bLambda + \nu_2)^{-2}) \approx \frac{C_2}{C_1}n,\quad \Tr(\bLambda (\bLambda + \nu_2)^{-2}) \approx (C_1-C_2) (\frac{n}{C_1})^{\alpha+1}.
\end{aligned}
\]
Thus, in the over-parameterized regime
\[
\begin{aligned}
\Upsilon(\nu_1, \nu_2) =&~ \frac{p}{n} \left[ \left( 1 - \frac{\nu_1}{\nu_2} \right)^2 + \left( \frac{\nu_1}{\nu_2} \right)^2 \frac{\operatorname{Tr}\left(\bLambda^2 (\bLambda + \nu_2)^{-2}\right)}{p - \operatorname{Tr}\left(\bLambda^2 (\bLambda + \nu_2)^{-2}\right)} \right]\\
\approx&~ \frac{p}{n} \left[ \left( \frac{n}{p} \right)^2 + \left( 1 - \frac{n}{p} \right)^2 \frac{\operatorname{Tr}\left(\bLambda^2 (\bLambda + \nu_2)^{-2}\right)}{p - \operatorname{Tr}\left(\bLambda^2 (\bLambda + \nu_2)^{-2}\right)} \right]\\
\approx&~ \frac{\frac{C_2}{C_1}p -2 \frac{C_2}{C_1} n + n}{p - \frac{C_2}{C_1}n}\,,
\end{aligned}
\]
\[
\begin{aligned}
\chi(\nu_2) =~ \frac{\Tr(\bLambda (\bLambda + \nu_2)^{-2})}{p - \Tr(\bLambda^2 (\bLambda + \nu_2)^{-2})} \approx~ \frac{(C_1-C_2)(\frac{n}{C_1})^{\alpha+1}}{p-\frac{C_2}{C_1}n}\,.
\end{aligned}
\]
According to the approximation, we have the deterministic equivalents of variance terms
\[
\begin{aligned}
\sV_{\sR,0}^{\tt RFM} =&~ \sigma^2 \frac{\Upsilon(\nu_1, \nu_2)}{1 - \Upsilon(\nu_1, \nu_2)} \approx \sigma^2 \frac{(C_1-2C_2)n + C_2p}{(C_1-C_2)(p-n)}\,,\\
\sV_{\sN,0}^{\tt RFM} =&~ \sigma^2 \frac{p}{n} \frac{\chi(\nu_2)}{1 - \Upsilon(\nu_1, \nu_2)} \approx \sigma^2 \frac{(\frac{n}{C_1})^\alpha p}{p-n}\,.
\end{aligned}
\]
Then recall \cref{eq:C1C2}, we eliminate $p$ and obtain
\begin{equation}\label{eq:V_over_power_law}
    \sV_{\sR,0}^{\tt RFM} \approx \left(\frac{n}{C_1}\right)^{-\alpha}\sV_{\sN,0}^{\tt RFM} + \sigma^2\frac{2C_2-C_1}{C_1-C_2} = \left(\frac{n}{C_1}\right)^{-\alpha}\sV_{\sN,0}^{\tt RFM} + \sigma^2(\alpha - 2)\,.
\end{equation}

For the bias terms, due to the varying approximation behaviors of the quantities containing $\btheta_*$ for different values of $r$, we have to discuss their approximations in the conditions $r \in (0, \frac{1}{2})$, $r \in [\frac{1}{2}, 1)$ and $r \in [\frac{1}{2}, \infty)$ separately.

\paragraph{Condition 1: $r \in (0, \frac{1}{2})$}
\[
\begin{aligned}
\sB_{\sR,0}^{\tt RFM} =&~ \frac{\nu_2^2}{1 - \Upsilon(\nu_1, \nu_2)} \left[ \< \btheta_*, (\bLambda + \nu_2)^{-2} \btheta_* \> + \chi(\nu_2) \< \btheta_*, \bLambda (\bLambda + \nu_2)^{-2} \btheta_* \> \right]\\
\approx&~\frac{\left(\frac{n}{C_1}\right)^{-2\alpha r}\left((C_1C_4 -C_2C_3)n+C_1(C_3-C_4)p\right)}{(C1-C2)(p-n)},\\
\sB_{\sN,0}^{\tt RFM} =&~ \frac{\nu_2}{\nu_1} \< \btheta_*, (\bLambda + \nu_2)^{-1} \btheta_* \> - \frac{\lambda}{n} \frac{\nu_2^2}{\nu_1^2} \frac{\< \btheta_*, (\bLambda + \nu_2)^{-2} \btheta_* \> + \chi(\nu_2) \< \btheta_*, \bLambda (\bLambda + \nu_2)^{-2} \btheta_* \>}{1 - \Upsilon(\nu_1, \nu_2)}\\
\approx&~ \frac{\nu_2}{\nu_1} \< \btheta_*, (\bLambda + \nu_2)^{-1} \btheta_* \>\\
\approx&~ \frac{\left(\frac{n}{C_1}\right)^{-\alpha(2r-1)}C_3 p}{p-n}.
\end{aligned}
\]
Then we eliminate $p$ and obtain
\begin{equation}\label{eq:B_over_power_law_1}
    \sB_{\sR,0}^{\tt RFM} \approx \left(\frac{n}{C_1}\right)^{-\alpha} \sB_{\sN,0}^{\tt RFM} + \left(\frac{n}{C_1}\right)^{-2\alpha r}\frac{C_2C_3-C_1C_4}{C_1-C_2}\,.
\end{equation}

\paragraph{Condition 2: $r \in [\frac{1}{2}, 1)$}
\[
\begin{aligned}
\sB_{\sR,0}^{\tt RFM} =&~ \frac{\nu_2^2}{1 - \Upsilon(\nu_1, \nu_2)} \left[ \< \btheta_*, (\bLambda + \nu_2)^{-2} \btheta_* \> + \chi(\nu_2) \< \btheta_*, \bLambda (\bLambda + \nu_2)^{-2} \btheta_* \> \right]\\
\approx&~\frac{\left(\frac{n}{C_1}\right)^{-\alpha}\left(C_1\left(C_4 n + C_5 \left(\frac{n}{C_1}\right)^{-\alpha (2r-1)} p\right)-C_2 n \left(C_4 + C_5 \left(\frac{n}{C_1}\right)^{-\alpha (2r-1)}\right)\right)}{(C1-C2)(p-n)},\\
\sB_{\sN,0}^{\tt RFM} =&~ \< \btheta_*, \bLambda ( \bLambda + \nu_2)^{-2} \btheta_* \> \cdot \frac{p}{p - {\rm df}_2(\nu_2)}\\
&~+ \frac{p}{n} \nu_2^2 \left( \< \btheta_*, (\bLambda + \nu_2)^{-2} \btheta_* \> + \chi(\nu_2) \< \btheta_*, \bLambda (\bLambda + \nu_2)^{-2} \btheta_* \> \right) \cdot \frac{\chi(\nu_2)}{1 - \Upsilon(\nu_1, \nu_2)}\\
\approx&~ \frac{\left(C_4+C_5\left(\frac{n}{C_1}\right)^{-\alpha(2r-1)}\right)p}{p-n}\,.
\end{aligned}
\]
Then we eliminate $p$ and obtain
\[
\begin{aligned}
\sB_{\sR,0}^{\tt RFM} \approx&~ \left(\frac{n}{C_1}\right)^{-\alpha} \sB_{\sN,0}^{\tt RFM} + \left(\frac{n}{C_1}\right)^{-\alpha}\frac{-C_1C_4+C_2C_4+C_2C_5\left(\frac{n}{C_1}\right)^{-\alpha(2r-1)}}{C_1-C_2}\\
\approx&~ \left(\frac{n}{C_1}\right)^{-\alpha} \sB_{\sN,0}^{\tt RFM} - \left(\frac{n}{C_1}\right)^{-\alpha}C_4\,.
\end{aligned}
\]
The last ``$\approx$'' holds because $\left(\frac{n}{C_1}\right)^{-\alpha(2r-1)} = o(1)$.

\paragraph{Condition 3: $r \in [1, \infty)$}
\[
\begin{aligned}
\sB_{\sR,0}^{\tt RFM} =&~ \frac{\nu_2^2}{1 - \Upsilon(\nu_1, \nu_2)} \left[ \< \btheta_*, (\bLambda + \nu_2)^{-2} \btheta_* \> + \chi(\nu_2) \< \btheta_*, \bLambda (\bLambda + \nu_2)^{-2} \btheta_* \> \right]\\
\approx&~\frac{\left(\frac{n}{C_1}\right)^{-2\alpha}\left(C_1\left(C_4 n \left(\frac{n}{C_1}\right)^{\alpha} + C_6 p\right) - C_2 n \left(C_6 + C_4 \left(\frac{n}{C_1}\right)^{\alpha}\right)\right)}{(C1-C2)(p-n)},\\
\sB_{\sN,0}^{\tt RFM} =&~ \< \btheta_*, \bLambda ( \bLambda + \nu_2)^{-2} \btheta_* \> \cdot \frac{p}{p - {\rm df}_2(\nu_2)}\\
&~+ \frac{p}{n} \nu_2^2 \left( \< \btheta_*, (\bLambda + \nu_2)^{-2} \btheta_* \> + \chi(\nu_2) \< \btheta_*, \bLambda (\bLambda + \nu_2)^{-2} \btheta_* \> \right) \cdot \frac{\chi(\nu_2)}{1 - \Upsilon(\nu_1, \nu_2)}\\
\approx&~ \frac{\left(C_4+C_6\left(\frac{n}{C_1}\right)^{-\alpha}\right)p}{p-n}\,.
\end{aligned}
\]
Then we eliminate $p$ and obtain
\[
\begin{aligned}
\sB_{\sR,0}^{\tt RFM} \approx&~ \left(\frac{n}{C_1}\right)^{-\alpha} \sB_{\sN,0}^{\tt RFM} + \left(\frac{n}{C_1}\right)^{-\alpha}\frac{-C_1C_4+C_2C_4+C_2C_6\left(\frac{n}{C_1}\right)^{-\alpha}}{C_1-C_2}\\
\approx&~ \left(\frac{n}{C_1}\right)^{-\alpha} \sB_{\sN,0}^{\tt RFM} - \left(\frac{n}{C_1}\right)^{-\alpha}C_4\,.
\end{aligned}
\]
The last ``$\approx$'' holds because $\left(\frac{n}{C_1}\right)^{-\alpha(2r-1)} = o(1)$.

Combining the above condition \(r \in [\frac{1}{2}, 1)\) and \(r \in [1, \infty)\), we have for \(r \in [\frac{1}{2}, \infty)\)

\begin{equation}\label{eq:B_over_power_law_2}
\sB_{\sR,0}^{\tt RFM} \approx \left(\frac{n}{C_1}\right)^{-\alpha} \sB_{\sN,0}^{\tt RFM} - \left(\frac{n}{C_1}\right)^{-\alpha}C_4\,.    
\end{equation}

From \cref{eq:V_over_power_law,eq:B_over_power_law_1,eq:B_over_power_law_2}, we know that the relationship between \(\sR_0^{\tt RFM}\) and \(\sN_0^{\tt RFM}\) in the over-parameterized regime can be written as
\[
\sR_0^{\tt RFM} \approx \left(\nicefrac{n}{C_\alpha}\right)^{-\alpha} \sN_0^{\tt RFM} + C_{n,\alpha,r,1}\,.
\]


\paragraph{The relationship in under-parameterized regime ($p < n$)} While in the under-parameterized regime ($p < n$), When $\lambda \to 0$, $\Tr(\bLambda(\bLambda + \nu_2)^{-1})$ will converge to $p$, which means $\nu_2$ will converge to $(\frac{p}{C_1})^{-\alpha}$ and $\nu_1$ will converge to 0, with $\frac{\lambda}{\nu_1}\to n-p$. 

Accordingly, in the under-parameterized regime
\[
\begin{aligned}
\Upsilon(\nu_1, \nu_2) = \frac{p}{n} \left[ \left( 1 - \frac{\nu_1}{\nu_2} \right)^2 + \left( \frac{\nu_1}{\nu_2} \right)^2 \frac{\operatorname{Tr}\left(\bLambda^2 (\bLambda + \nu_2)^{-2}\right)}{p - \operatorname{Tr}\left(\bLambda^2 (\bLambda + \nu_2)^{-2}\right)} \right] \to \frac{p}{n}\,,
\end{aligned}
\]
\[
\begin{aligned}
\chi(\nu_2) = \frac{\operatorname{Tr}\left(\bLambda (\bLambda + \nu_2)^{-2}\right)}{p - \operatorname{Tr}\left(\bLambda^2 (\bLambda + \nu_2)^{-2}\right)} \to \frac{1}{\nu_2} \approx (\frac{p}{C_1})^{\alpha}\,.
\end{aligned}
\]
Then we can further obtain that, for the variance
\[
\begin{aligned}
\sV_{\sR,0}^{\tt RFM} =&~ \sigma^2 \frac{\Upsilon(\nu_1, \nu_2)}{1 - \Upsilon(\nu_1, \nu_2)} \approx \sigma^2 \frac{p}{n-p},\\
\sV_{\sN,0}^{\tt RFM} =&~ \sigma^2 \frac{p}{n} \frac{\chi(\nu_2)}{1 - \Upsilon(\nu_1, \nu_2)} \approx \sigma^2 C_1^{-\alpha} \frac{p^{\alpha+1}}{n-p}.
\end{aligned}
\]
For the relationship in the under-parameterized regime, we separately consider two cases, i.e. $p \ll n$ and $p\to n$.

First, we derive the relationship in the under-parameterized regime ($p < n$) as $p \to n$, based on the relationship in the over-parameterized regime.
Recall the relationship between $\sV_{\sR,0}^{\tt RFM}$ and $\sV_{\sN,0}^{\tt RFM}$ in the over-parameterized regime, as presented in \cref{eq:V_over_power_law}, given by 
\[
\sV_{\sR,0}^{\tt RFM} \approx \left(\frac{n}{C_1}\right)^{-\alpha} \sV_{\sN,0}^{\tt RFM} + \sigma^2 (\alpha-2) =:h(\sV_{\sN,0}^{\tt RFM})\,.
\]
Substituting the expression for $\sV_{\sN,0}^{\tt RFM}$ in the under-parameterized regime into this relationship, we obtain
\[
\sV_{\sR,0}^{\tt RFM} \approx \left(\frac{n}{C_1}\right)^{-\alpha} \sigma^2 C_1^{-\alpha} \frac{p^{\alpha+1}}{n-p} + \sigma^2 (\alpha-2)\,,
\]
then we compute $\sV_{\sR,0}^{\tt RFM} - h(\sV_{\sN,0}^{\tt RFM})$ and obtain
\[
\begin{aligned}
    \sV_{\sR,0}^{\tt RFM} - h(\sV_{\sN,0}^{\tt RFM}) =&~ \sigma^2 \frac{p}{n-p} - \left(\frac{n}{C_1}\right)^{-\alpha} \sigma^2 C_1^{-\alpha} \frac{p^{\alpha+1}}{n-p} - \sigma^2 (\alpha-2)\\
    =&~ \sigma^2\left(\frac{p-p^{\alpha+1}n^{-\alpha}}{n-p}\right) - \sigma^2 (\alpha-2)\,.
\end{aligned}
\]
Taking limits on the left and right sides of the equation, we get
\[
\lim_{p \to n} \left(\sV_{\sR,0}^{\tt RFM} - h(\sV_{\sN,0}^{\tt RFM})\right) = 2\sigma^2\,.
\]
Then when $p \to n$, we have
\begin{equation}\label{eq:V_under_power_law}
    \sV_{\sR,0}^{\tt RFM} \approx \left(\frac{n}{C_1}\right)^{-\alpha} \sV_{\sN,0}^{\tt RFM} + \sigma^2 \alpha\,.
\end{equation}
For $p \ll n$, we have $\frac{1}{n-p} \approx \frac{1}{n}$, then 
\[
\begin{aligned}
\sV_{\sR,0}^{\tt RFM} =&~ \sigma^2 \frac{\Upsilon(\nu_1, \nu_2)}{1 - \Upsilon(\nu_1, \nu_2)} \approx \sigma^2 \frac{p}{n},\\
\sV_{\sN,0}^{\tt RFM} =&~ \sigma^2 \frac{p}{n} \frac{\chi(\nu_2)}{1 - \Upsilon(\nu_1, \nu_2)} \approx \sigma^2 C_1^{-\alpha} \frac{p^{\alpha+1}}{n}.
\end{aligned}
\]
Eliminate $p$ and we have
\[
\sV_{\sR,0}^{\tt RFM} \approx \left(\sigma^2\right)^{\frac{\alpha}{\alpha+1}} C_1^{\frac{\alpha}{\alpha+1}} \left(\sV_{\sR,0}^{\tt RFM}\right)^\frac{1}{\alpha+1}\,.
\]
Next, for the bias term we have
\[
\begin{aligned}
\sB_{\sR,0}^{\tt RFM} =&~ \frac{\nu_2^2}{1 - \Upsilon(\nu_1, \nu_2)} \left[ \< \btheta_*, (\bLambda + \nu_2)^{-2} \btheta_* \> + \chi(\nu_2) \< \btheta_*, \bLambda (\bLambda + \nu_2)^{-2} \btheta_* \> \right]\\
\approx&~ \frac{\nu_2}{1 - \Upsilon(\nu_1, \nu_2)} \< \btheta_*, (\bLambda + \nu_2)^{-1} \btheta_* \>\\
\approx&~ \frac{n}{n-p} C_3 \nu_2^{2r\wedge1}.\\
\sB_{\sN,0}^{\tt RFM} =&~ p \< \btheta_*, \bLambda ( \bLambda + \nu_2)^{-2} \btheta_* \> \cdot \frac{1}{p - {\rm df}_2(\nu_2)}\\
&~+ \frac{p}{n} \chi(\nu_2) \frac{\nu_2^2}{1 - \Upsilon(\nu_1, \nu_2)} \left[ \< \btheta_*, (\bLambda + \nu_2)^{-2} \btheta_* \> + \chi(\nu_2) \< \btheta_*, \bLambda (\bLambda + \nu_2)^{-2} \btheta_* \> \right] \\
\approx&~ p \< \btheta_*, \bLambda ( \bLambda + \nu_2)^{-2} \btheta_* \> \cdot \frac{1}{p - {\rm df}_2(\nu_2)} + \frac{p}{n} \chi(\nu_2) \frac{\nu_2}{1 - \Upsilon(\nu_1, \nu_2)} \< \btheta_*, (\bLambda + \nu_2)^{-1} \btheta_* \>\\
\approx&~ \frac{p}{p-\frac{C_2}{C_1}p} C_4 \nu_2^{(2r-1)\wedge0} + \frac{p}{n-p} C_3 \nu_2^{(2r-1)\wedge0}\\
\approx&~ \left(\frac{C_1C_4}{C_1-C_2} + \frac{p}{n-p}C_3\right) \nu_2^{(2r-1)\wedge0}.
\end{aligned}
\]
Then we use the approximation $\nu_2 \approx (\frac{p}{C_1})^{-\alpha}$ and obtain
\[
\begin{aligned}
\sB_{\sR,0}^{\tt RFM} 
\approx \frac{n}{n-p} C_3 \nu_2^{2r\wedge1} \approx \frac{n}{n-p} C_3 \left( \frac{p}{C_1} \right)^{-\alpha\left(2r\wedge1\right)},
\end{aligned}
\]
\[
\begin{aligned}
\sB_{\sN,0}^{\tt RFM} \approx \left(\frac{C_1C_4}{C_1-C_2}+\frac{p}{n-p}C_3\right) \nu_2^{(2r-1)\wedge0} \approx \left(\frac{C_1C_4}{C_1-C_2}+\frac{p}{n-p}C_3\right) \left(\frac{p}{C_1}\right)^{-\alpha\left[(2r-1)\wedge0\right]}.
\end{aligned}
\]
Similarly to the bias term, we derive the relationship in the under-parameterized regime ($p < n$) as $p \to n$, based on the relationship in the over-parameterized regime. And we discuss the relationship when $r \in (0, \frac{1}{2})$ and $r \in [\frac{1}{2}, \infty)$ separately.

\paragraph{Condition 1: $r \in (0, \frac{1}{2})$.}
Recall the relationship between $\sB_{\sR,0}^{\tt RFM}$ and $\sB_{\sN,0}^{\tt RFM}$ in the over-parameterized regime, as presented in \cref{eq:B_over_power_law_1}, given by: 
\[
\begin{aligned}
\sB_{\sR,0}^{\tt RFM} = \left(\frac{n}{C_1}\right)^{-\alpha} \sB_{\sN,0}^{\tt RFM} + \left(\frac{n}{C_1}\right)^{-2\alpha r}\frac{C_2C_3-C_1C_4}{C_1-C_2} =: f(\sB_{\sN,0}^{\tt RFM}).
\end{aligned}
\]
Substituting the expression for $\sB_{\sN,0}^{\tt RFM}$ in the under-parameterized regime into this relationship, we obtain:
\[
\begin{aligned}
f(\sB_{\sN,0}^{\tt RFM}) =&~ \left(\frac{n}{C_1}\right)^{-\alpha} \left(\frac{C_1C_4}{C_1-C_2}+\frac{p}{n-p}C_3\right) \left(\frac{p}{C_1}\right)^{-\alpha(2r-1)} + \left(\frac{n}{C_1}\right)^{-2\alpha r}\frac{C_2C_3-C_1C_4}{C_1-C_2},
\end{aligned}
\]
then we compute $\sB_{\sR,0}^{\tt RFM} - f(\sB_{\sN,0}^{\tt RFM})$ and obtain
\[
\begin{aligned}
\sB_{\sR,0}^{\tt RFM} - f(\sB_{\sN,0}^{\tt RFM}) = C_1^{2\alpha r}\Big(\frac{n}{n-p}C_3p^{-2\alpha r} - \frac{C_1C_4}{C_1-C_2}p^{-\alpha(2r-1)}n^{-\alpha} -\frac{p}{n-p}C_3p^{-\alpha(2r-1)}n^{-\alpha} - \frac{C_2C_3-C_1C_4}{C_1-C_2}n^{-2\alpha r}\Big).\\
\end{aligned}
\]
To simplify this equation, we begin by computing $\frac{n}{n-p}C_3p^{-2\alpha r} - \frac{p}{n-p}C_3p^{-\alpha(2r-1)}n^{-\alpha}$ and obtain
\[
\begin{aligned}
\frac{n}{n-p}C_3p^{-2\alpha r} - \frac{p}{n-p}C_3p^{-\alpha(2r-1)}n^{-\alpha} =&~ C_3p^{-\alpha(2r-1)} \left(\frac{n}{n-p}p^{-\alpha} - \frac{p}{n-p}n^{-\alpha} \right)\\
=&~ C_3p^{-\alpha(2r-1)}\frac{np^{-\alpha} - pn^{-\alpha}}{n-p},
\end{aligned}
\]
where $\frac{np^{-\alpha} - pn^{-\alpha}}{n-p}$ is monotonically decreasing in $p$ (monotonicity can be obtained by simple derivatives), and by applying L'Hôpital's rule, we have:
\[
\lim_{p \to n} \frac{np^{-\alpha} - pn^{-\alpha}}{n-p} = \lim_{p \to n} \frac{-\alpha n p^{-\alpha-1}-n^{-\alpha}}{-1} = (\alpha+1)n^{-\alpha}.
\]
Thus we have
\[
\begin{aligned}
\lim_{p \to n} C_3p^{-\alpha(2r-1)}\frac{np^{-\alpha} - pn^{-\alpha}}{n-p} = (\alpha+1)C_3n^{-2\alpha r}.
\end{aligned}
\]
Thus we have
\[
\begin{aligned}
&~\lim_{p \to n} C_1^{2\alpha r}\Big(C_3p^{-\alpha(2r-1)}\frac{np^{-\alpha} - pn^{-\alpha}}{n-p} - \frac{C_1C_4}{C_1-C_2}p^{-\alpha(2r-1)}n^{-\alpha} - \frac{C_2C_3-C_1C_4}{C_1-C_2}n^{-2\alpha r}\Big)\\
=&~ C_1^{2\alpha r}\Big((\alpha+1)C_3n^{-2\alpha r} - \frac{C_1C_4}{C_1-C_2}n^{-2\alpha r} - \frac{C_2C_3-C_1C_4}{C_1-C_2}n^{-2\alpha r}\Big)\\
=&~ C_1^{2\alpha r}C_3n^{-2\alpha r}\Big((\alpha+1) - \frac{C_2}{C_1-C_2}\Big).
\end{aligned}
\]
Recall that from \cref{eq:C1C2} we have
\[
\begin{aligned}
C_1 = \frac{\pi}{\alpha \sin\left(\nicefrac{\pi}{\alpha}\right)}\,, \quad C_2 = \frac{\pi(\alpha-1)}{\alpha^2 \sin\left(\nicefrac{\pi}{\alpha}\right)},
\end{aligned}
\]
thus 
\[
\begin{aligned}
(\alpha+1) - \frac{C_2}{C_1-C_2} = (\alpha+1) - \frac{ \frac{\pi(\alpha-1)}{\alpha^2 \sin\left(\nicefrac{\pi}{\alpha}\right)}}{\frac{\pi}{\alpha \sin\left(\nicefrac{\pi}{\alpha}\right)} - \frac{\pi(\alpha-1)}{\alpha^2 \sin\left(\nicefrac{\pi}{\alpha}\right)}} = 2.
\end{aligned}
\]
Finally, we have
\[
\begin{aligned}
\lim_{p \to n}\left( \sB_{\sR,0}^{\tt RFM} - f(\sB_{\sN,0}^{\tt RFM}) \right) = 2C_1^{2\alpha r}C_3n^{-2\alpha r} = 2C_3\left(\frac{n}{C_1}\right)^{-2\alpha r},
\end{aligned}
\]
and then the relationship between $\sB_{\sR,0}^{\tt RFM}$ and $\sB_{\sN,0}^{\tt RFM}$ is 
\begin{equation}\label{eq:B_under_power_law_1}
    \begin{split}
        \sB_{\sR,0}^{\tt RFM} \approx&~ \left(\frac{n}{C_1}\right)^{-\alpha} \sB_{\sN,0}^{\tt RFM} + \left(\frac{n}{C_1}\right)^{-2\alpha r}\frac{C_2C_3-C_1C_4}{C_1-C_2} + 2C_3\left(\frac{n}{C_1}\right)^{-2\alpha r}\\
        \approx&~ \left(\frac{n}{C_1}\right)^{-\alpha} \sB_{\sN,0}^{\tt RFM} + \left(\frac{n}{C_1}\right)^{-2\alpha r}\frac{2C_1C_3-C_2C_3-C_1C_4}{C_1-C_2}.
    \end{split}
\end{equation}


\paragraph{Condition 2: $r \in [\frac{1}{2}, \infty)$.}

In this condition, the approximation of $\sB_{\sR,0}^{\tt RFM}$ and $\sB_{\sN,0}^{\tt RFM}$ can be simplified to 
\[
\begin{aligned}
\sB_{\sR,0}^{\tt RFM} 
\approx \frac{n}{n-p} C_3 \nu_2^{2r\wedge1} \approx \frac{n}{n-p} C_3 \left( \frac{p}{C_1} \right)^{-\alpha\left(2r\wedge1\right)} = \frac{n}{n-p} C_3 \left( \frac{p}{C_1} \right)^{-\alpha}\,,
\end{aligned}
\]
\[
\begin{aligned}
\sB_{\sN,0}^{\tt RFM} \approx \left(\frac{C_1C_4}{C_1-C_2}+\frac{p}{n-p}C_3\right) \nu_2^{(2r-1)\wedge0} \approx \left(\frac{C_1C_4}{C_1-C_2}+\frac{p}{n-p}C_3\right) \left(\frac{p}{C_1}\right)^{-\alpha\left[(2r-1)\wedge0\right]} = \frac{C_1C_4}{C_1-C_2}+\frac{p}{n-p}C_3\,.
\end{aligned}
\]
Recall the relationship between $\sB_{\sR,0}^{\tt RFM}$ and $\sB_{\sN,0}^{\tt RFM}$ in the over-parameterized regime is presented in \cref{eq:B_over_power_law_2}, given by: 
\[
\begin{aligned}
\sB_{\sR,0}^{\tt RFM} \approx&~ \left(\frac{n}{C_1}\right)^{-\alpha} \sB_{\sN,0}^{\tt RFM} - \left(\frac{n}{C_1}\right)^{-\alpha}C_4 =: g(\sB_{\sN,0}^{\tt RFM})\,.
\end{aligned}
\]
Substituting the expression for $\sB_{\sN,0}^{\tt RFM}$ in the under-parameterized regime into this relationship, we obtain:
\[
\begin{aligned}
g(\sB_{\sN,0}^{\tt RFM}) =&~ \left(\frac{n}{C_1}\right)^{-\alpha} \left(\frac{C_1C_4}{C_1-C_2}+\frac{p}{n-p}C_3\right) - \left(\frac{n}{C_1}\right)^{-\alpha}C_4\,,
\end{aligned}
\]
then we compute $\sB_{\sR,0}^{\tt RFM} - g(\sB_{\sN,0}^{\tt RFM})$ and obtain
\[
\begin{aligned}
    \sB_{\sR,0}^{\tt RFM} - g(\sB_{\sN,0}^{\tt RFM}) = C_3 C_1^{\alpha} \frac{np^{-\alpha} - pn^{-\alpha}}{n-p} - \left(\frac{n}{C_1}\right)^{-\alpha} \left( \frac{C_2C_4}{C_1-C_2} \right)\,.
\end{aligned}
\]
Thus we have
\[
\begin{aligned}
    \lim_{p \to n}\left( \sB_{\sR,0}^{\tt RFM} - f(\sB_{\sN,0}^{\tt RFM}) \right) =&~ \left(\frac{n}{C_1}\right)^{-\alpha} \left((\alpha+1)C_3 - \frac{C_2C_4}{C_1-C_2}\right)\\
    \approx&~ \left(\frac{n}{C_1}\right)^{-\alpha} \left((\alpha+1)C_4 - \frac{C_2}{C_1-C_2}C_4\right)\\
    =&~ \left(\frac{n}{C_1}\right)^{-\alpha} 2 C_4\,,
\end{aligned}
\]
and the relationship between $\sB_{\sR,0}^{\tt RFM}$ and $\sB_{\sN,0}^{\tt RFM}$ is 
\begin{equation}\label{eq:B_under_power_law_2}
    \begin{split}
        \sB_{\sR,0}^{\tt RFM} \approx&~ \left(\frac{n}{C_1}\right)^{-\alpha} \sB_{\sN,0}^{\tt RFM} - \left(\frac{n}{C_1}\right)^{-\alpha}C_4 + \left(\frac{n}{C_1}\right)^{-\alpha}2C_4\\
        \approx&~ \left(\frac{n}{C_1}\right)^{-\alpha} \sB_{\sN,0}^{\tt RFM} + \left(\frac{n}{C_1}\right)^{-\alpha}C_4\,.
    \end{split}
\end{equation}


When $p \ll n$, we discuss cases $r \in (0, \frac{1}{2})$ and $r \in (\frac{1}{2}, \infty)$ separately. 

If $r \in (0, \frac{1}{2})$, we have $\frac{n}{n-p} \approx 1$ and $\frac{p}{n-p} \approx 0$, then
\[
\begin{aligned}
\sB_{\sR,0}^{\tt RFM} 
\approx C_3 \nu_2^{2r\wedge1} \approx C_3 \left( \frac{p}{C_1} \right)^{-\alpha2r},
\end{aligned}
\]
\[
\begin{aligned}
\sB_{\sN,0}^{\tt RFM} \approx \frac{C_1C_4}{C_1-C_2} \nu_2^{(2r-1)\wedge0} \approx \frac{C_1C_4}{C_1-C_2} \left(\frac{p}{C_1}\right)^{-\alpha\left(2r-1\right)}.
\end{aligned}
\]
Then we eliminate $p$ and obtain
\[
\begin{aligned}
\sB_{\sR,0}^{\tt RFM} \approx C_3 \left(\frac{C_1-C_2}{C_1C_4}\right)^{\nicefrac{2r}{(2r-1)}} \left(\sB_{\sN,0}^{\tt RFM}\right)^{\nicefrac{2r}{(2r-1)}}.
\end{aligned}
\]

If $2r \ge 1$, we have
\[
\begin{aligned}
\sB_{\sR,0}^{\tt RFM} \approx&~ \frac{n}{n-p} C_3 \nu_2 \approx \frac{n}{n-p} C_3\left(\frac{p}{C_1}\right)^{-\alpha},
\end{aligned}
\]
\[
\begin{aligned}
\sB_{\sN,0}^{\tt RFM} \approx&~ \frac{C_1C_4}{C_1-C_2} + \frac{p}{n-p}C_3 .
\end{aligned}
\]
Then we eliminate $p$ and obtain
\[
\begin{aligned}
\sB_{\sR,0}^{\tt RFM} \approx \left(\frac{C_1C_3-C_2C_3-C_1C_4}{C_1-C_2}+\sB_{\sN,0}^{\tt RFM}\right)\left(\frac{n\left(\sB_{\sN,0}^{\tt RFM}-\frac{C_1C_4}{C_1-C_2}\right)}{C_1\left(C_3+\sB_{\sN,0}^{\tt RFM}-\frac{C_1C_4}{C_1-C_2}\right)}\right)^{-\alpha}.
\end{aligned}
\]

From \cref{eq:V_under_power_law,eq:B_under_power_law_1,eq:B_under_power_law_2}, we know that the relationship between \(\sR_0^{\tt RFM}\) and \(\sN_0^{\tt RFM}\) in the under-parameterized regime when \(p \to n\) can be written as
\[
\sR_0^{\tt RFM} \approx \left(\nicefrac{n}{C_\alpha}\right)^{-\alpha} \sN_0^{\tt RFM} + C_{n,\alpha,r,2}\,.
\]

\end{proof}



    



\section{Scaling laws}
\label{app:scaling_law}
To derive the scaling laws based on norm-based capacity, we first give the decay rate of the $\ell_2$ norm w.r.t.\ \(n\).

The rate of the deterministic equivalent of the random feature ridge regression estimator's $\ell_2$ norm  is given by
\[
\sN_\lambda^{\tt RFM} = \Theta\left(n^{-\gamma_{\sB_{\sN,\lambda}^{\tt RFM}}}+\sigma^2n^{-\gamma_{\sV_{\sN,\lambda}^{\tt RFM}}}\right) = \Theta\left(n^{-\gamma_{\sN_\lambda^{\tt RFM}}}\right)\,,
\]
where $\gamma_{\sN_\lambda^{\tt RFM}} := \gamma_{\sB_{\sN,\lambda}^{\tt RFM}} \wedge \gamma_{\sV_{\sN,\lambda}^{\tt RFM}}$ for $\sigma^2 \neq 0$.
    
\subsection{Variance term}
Using \cref{eq:rate_nu2,eq:rates:Upsilon2,eq:rates:chi}, we have
\[
\begin{aligned}
\sV_{\sN,\lambda}^{\tt RFM} =&~ \sigma^2 \frac{p}{n} \frac{\chi(\nu_2)}{1 - \Upsilon(\nu_1, \nu_2)} =~ n^{q-1}n^{-q}O\left(\nu_2^{-1-\nicefrac{1}{\alpha}}\right)\\
=&~O\left(n^{-\left(1- \left(\alpha+1\right)\left(1 \wedge q \wedge \nicefrac{\ell}{\alpha}\right)\right)}\right).
\end{aligned}
\]
Hence, the variance term of the norm decays with $n$ with rate
\[
\gamma_{\sV_{\sN,\lambda}^{\tt RFM}}(\ell, q) = 1 - \left(\alpha+1\right)\left(\frac{\ell}{\alpha}\wedge q\wedge 1\right). 
\]

\subsection{Bias term}
First, one could notice, using the integral approximation and \cref{eq:rate_T,eq:rate_nu2}, that
\[
\begin{aligned}
\frac{p}{p - {\rm df}_2(\nu_2)} =&~ \left(1 + n^{-q} O\left(\nu_2^{-\nicefrac{1}{\alpha}}\right) \right) = \left(1 + O\left(n^{-q}n^{\left(1 \wedge q \wedge \nicefrac{\ell}{\alpha}\right)}\right) \right) = O\left(1\right).
\end{aligned}
\]
Thus for the bias term, using \cref{eq:rate_T,eq:rate_nu2,eq:rates:Upsilon2,eq:rates:chi} we have
\[
\begin{aligned}
\sB_{\sN,\lambda}^{\tt RFM} =&~ \< \btheta_*, \bLambda ( \bLambda + \nu_2)^{-2} \btheta_* \> \cdot \frac{p}{p - {\rm df}_2(\nu_2)}\\
&~+ \frac{p}{n} \nu_2^2 \left( \< \btheta_*, (\bLambda + \nu_2)^{-2} \btheta_* \> + \chi(\nu_2) \< \btheta_*, \bLambda (\bLambda + \nu_2)^{-2} \btheta_* \> \right) \cdot \frac{\chi(\nu_2)}{1 - \Upsilon(\nu_1, \nu_2)}\\
=&~ T_{2r+1, 2}^1(\nu_2) + n^{q-1}\nu_2^2\left( T_{2r,2}^1(\nu_2) + \chi(\nu_2) T_{2r+1,2}^1(\nu_2)\right)\chi(\nu_2)\\
=&~ \nu_2^{(2r-1)\wedge 0} + n^{q-1} \nu_2^2 O\left(\nu_2^{(2r-2)\wedge 0} +  n^{-q}\nu_2^{-1-\nicefrac{1}{\alpha}+(2r-1)\wedge 0}\right) n^{-q}O\left(\nu_2^{-1-\nicefrac{1}{\alpha}}\right)\\
=&~ \nu_2^{(2r-1)\wedge 0} + n^{-1} O\left(\nu_2^{2r\wedge 2} +  n^{-q}\nu_2^{-\nicefrac{1}{\alpha}+2r\wedge 1}\right) O\left(\nu_2^{-1-\nicefrac{1}{\alpha}}\right)\\
=&~ O\left( n^{-\alpha \left(1 \wedge q \wedge \nicefrac{\ell}{\alpha}\right) \left[(2r-1)\wedge 0\right]} \right)\\
&~+ O\left(n^{-\alpha \left(1 \wedge q \wedge \nicefrac{\ell}{\alpha}\right) \left[(2r-1)\wedge 1\right] + \left(1 \wedge q \wedge \nicefrac{\ell}{\alpha}\right) - 1}
+ 
n^{-\alpha \left(1 \wedge q \wedge \nicefrac{\ell}{\alpha}\right) \left[(2r-1)\wedge 0\right] + 2\left(1 \wedge q \wedge \nicefrac{\ell}{\alpha}\right) - 1 - q}\right)\\
=&~ O\left( n^{-\alpha \left(1 \wedge q \wedge \nicefrac{\ell}{\alpha}\right) \left[(2r-1)\wedge 0\right]}
+ 
n^{-\alpha \left(1 \wedge q \wedge \nicefrac{\ell}{\alpha}\right) \left[(2r-1)\wedge 1\right] + \left(1 \wedge q \wedge \nicefrac{\ell}{\alpha}\right) - 1}
\right)\\
=&~ O\left( n^{-\alpha \left(1 \wedge q \wedge \nicefrac{\ell}{\alpha}\right) \left[(2r-1)\wedge 0\right]}\right).
\end{aligned}
\]
Hence, the bias term of the norm decays with $n$ with rate
\[
\begin{aligned}
\gamma_{\sB_{\sN,\lambda}^{\tt RFM}}(\ell, q) =&~ \alpha \left(1 \wedge q \wedge \nicefrac{\ell}{\alpha}\right) \left[(2r-1)\wedge 0\right].
\end{aligned}
\]
Recalling that we have
\[
\gamma_{\sN_\lambda^{\tt RFM}} := \gamma_{\sB_{\sN,\lambda}^{\tt RFM}} \wedge \gamma_{\sV_{\sN,\lambda}^{\tt RFM}}\,,
\]
according to which, we obtain the norm exponent $\gamma_{\sN_\lambda^{\tt RFM}}$ as a function of $\ell$ and $q$, showing in \cref{fig:scaling_law}. As observed in \cref{fig:scaling_law}, $\gamma_{\sN_\lambda^{\tt RFM}}$ is non-positive across all regions, indicating that the norm either increases or remains constant with \(n\) in every case.

\begin{figure}[H]
    \centering
    \includegraphics[width=0.6\textwidth]{arxiv_version/figures/Scaling_Law/scaling_law_norm.pdf} 
    \caption{The norm rate $\gamma_{\sN_\lambda^{\tt RFM}}$ as a function of $(\ell,q)$. Variance dominated region is colored by {\color{regionorange}orange}, {\color{regionyellow}yellow} and {\color{regionbrown}brown}, bias dominated region is colored by {\color{regionblue}blue} and {\color{regiongreen}green}.} 
    \label{fig:scaling_law} 
\end{figure}

Next for the condition $r \in (0, \frac{1}{2})$, we derive the scaling law under norm-based capacity.

\paragraph{Region 1: $\ell > \alpha$ and $q > 1$}

In this region, according to \citet[Corollary 4.1]{defilippis2024dimension}, we have
\[
\sR_\lambda^{\tt RFM} = \Theta\left( n^{-0} \right) = \Theta\left( 1 \right)\,,
\]
and according to \cref{fig:scaling_law}, we have
\[
\sN_\lambda^{\tt RFM} = \Theta\left( n^{\alpha} \right)\,,
\]
combing the above rate, we can obtain that
\[
\sR_\lambda^{\tt RFM} = \Theta\left( n^{-\alpha} \cdot \sN_\lambda^{\tt RFM} \right)\,.
\]

\paragraph{Region 2: $\frac{\alpha}{2\alpha r+1} < \ell < \alpha$ and $q > \frac{\ell}{\alpha}$}

In this region, according to \citet[Corollary 4.1]{defilippis2024dimension}, we have
\[
\sR_\lambda^{\tt RFM} = \Theta\left( n^{-\left(1-\frac{\ell}{\alpha}\right)} \right)\,,
\]
and according to \cref{fig:scaling_law}, we have
\[
\sN_\lambda^{\tt RFM} = \Theta\left( n^{-\left(1-\frac{(\alpha+1)\ell}{\alpha}\right)} \right)\,,
\]
combing the above rate, we can obtain that
\[
\sR_\lambda^{\tt RFM} = \Theta\left( n^{-\ell} \cdot \sN_\lambda^{\tt RFM} \right)\,.
\]

\paragraph{Region 3: $\frac{1}{2\alpha r+1} < q < 1$ and $q < \frac{\ell}{\alpha}$}

In this region, according to \citet[Corollary 4.1]{defilippis2024dimension}, we have
\[
\sR_\lambda^{\tt RFM} = \Theta\left( n^{-\left(1-q\right)} \right)\,,
\]
and according to \cref{fig:scaling_law}, we have
\[
\sN_\lambda^{\tt RFM} = \Theta\left( n^{-\left(1-(\alpha+1)q\right)} \right)\,,
\]
combing the above rate and eliminate $q$, we can obtain that
\[
\sR_\lambda^{\tt RFM} = \Theta\left( n^{-\frac{\alpha}{\alpha+1}} \cdot \left(\sN_\lambda^{\tt RFM}\right)^{\frac{1}{\alpha+1}} \right)\,.
\]

\paragraph{Region 4: $\ell < \frac{\alpha}{2\alpha r+1}$ and $q > \frac{\ell}{\alpha}$}

In this region, according to \citet[Corollary 4.1]{defilippis2024dimension}, we have
\[
\sR_\lambda^{\tt RFM} = \Theta\left( n^{-2\ell r} \right)\,,
\]
and according to \cref{fig:scaling_law}, we have
\[
\sN_\lambda^{\tt RFM} = \Theta\left( n^{-\ell(2r-1)} \right)\,,
\]
combing the above rate, we can obtain that
\[
\sR_\lambda^{\tt RFM} = \Theta\left( n^{-1} \cdot \sN_\lambda^{\tt RFM} \right)\,.
\]

\paragraph{Region 5: $q < \frac{1}{2\alpha r+1}$ and $q < \frac{\ell}{\alpha}$}

In this region, according to \citet[Corollary 4.1]{defilippis2024dimension}, we have
\[
\sR_\lambda^{\tt RFM} = \Theta\left( n^{-2\alpha q r} \right)\,,
\]
and according to \cref{fig:scaling_law}, we have
\[
\sN_\lambda^{\tt RFM} = \Theta\left( n^{-\alpha q(2r-1)} \right)\,,
\]
combing the above rate, we can obtain that
\[
\sR_\lambda^{\tt RFM} = \Theta\left( n^0 \cdot \left(\sN_\lambda^{\tt RFM}\right)^{-\frac{2r}{1-2r}} \right)\,.
\]







\newpage
\section{Experiment}\label{sec-experiment}
\subsection{Experimental Setup}
We briefly introduce experimental settings to verify our proposed MoR, including Datasets \& Baselines, Implementation Details, and Evaluation Metrics. More details are in Appendix~\ref{app-expr-setting}.

\textbf{Datasets \& Baselines:} We use three TG-KBs from STaRK~\cite{wu2024stark} covering three knowledge domains, including E-commerce Products (Amazon), Academic Papers (MAG), and Biomedicine (Prime). We compare our MoR with baselines established by~\citet{wu2024stark} and categorize them into textual/structural/hybrid-based ones. More recent state-of-the-art hybird retrieval approaches fro TG-KBs such as KAR~\cite{xia2024knowledge} and MFAR$^{*}$~\cite{li2024multi} are also compared.


\textbf{Implementation Details:} 
To enhance the planning capability of our planning module, we fine-tune the Llama 3.2 (3B) on 1000 sampled queries with their corresponding ground-truth planning graphs, serving as the textual graph generator. In the absence of ground-truths, we synthesize them using LLMs. For the Prime dataset, we empirically find that directly prompting LLMs can hardly generate accurate planning graphs due to the lack of biomedical domain knowledge~\cite{Shen2024TagLLMRG}. Therefore, we adopt an alternative approach. First, we instruct LLMs to extract triplets from each query and then construct the planning graphs by merging triplets with shared entities. 
During mixed traversal, textual matching can be implemented using any lexical or semantic methods. For this study, we employ BM25 for Amazon and MAG and fine-tune a contriever to complement the biomedical knowledge for Prime.
To initialize the structural traversal, we employ textual matching to locate the top 5 nodes that are most relevant to the query as seeds. Additionally, at each layer, we incorporate the top 10 nodes retrieved via textual matching and append them to the current candidate set for the next round of traversal. Notably, due to the uncertainty of LLMs, the generated planning graphs can be invalid. In this case, we will directly conduct textual matching to retrieve candidates. For our ablations without reranker, we employ Ada-002~\cite{wu2024stark} with cosine similarity as the scorer to rank candidates for evaluating performance.

\textbf{Evaluation Metrics:}
We follow~\citet{wu2024stark} for evaluation by reporting Hit@1 (H@1), Hit@5 (H@5), Recall@20 (R@20), and mean reciprocal rank MRR to evaluate in the full spectrum. 


 

\newpage
\subsection{Overall Retrieval Performance}
We compare MoR with other baselines on three TG-KBs in Table~\ref{tab-merged}. Generally, hybrid methods, AvaTAR, KAR, MFAR$^{*}$, and our MoR, achieve better performance than purely textual or structural methods owing to their ability to integrate both structural and textual knowledge. 
Among all baselines, our proposed MoR achieves the overall best performance with a substantial margin on average, with the first ranking on MAG and the second ranking on Amazon/Prime datasets. This demonstrates the effectiveness of our proposed mixture of structural and textual knowledge retrieval. 
Textual retrieval performs better on Amazon than on MAG, suggesting that Amazon queries rely more on textual knowledge. In contrast, its weaker performance on MAG is due to MAG's lower textual richness and stronger structural signals. This disparity aligns with the distribution analysis presented by~\citet{wu2024stark} and supports our hypothesis that queries in different TG-KB datasets require varying desires for textual and structural knowledge. Meanwhile, structural retrieval methods such as conventional knowledge graph-based ones perform poorly because they are designed for graphs with minimal textual information compared to TG-KBs.
Different from Amazon and MAG, all existing methods without supervised tuning (e.g., Ada-002) exhibit significantly lower performance on Prime. This is due to the extreme domain expertise required in biology, where word-count-based, pre-trained textual similarity-based, and even more powerful LLMs are all poorly applicable here. Through fine-tuning, MFAR$^{*}$ and our proposed MoR generally achieve better performance, demonstrating the necessity of domain-specific knowledge for answering queries in knowledge-intensive domains. 




\newpage
\subsection{Ablation Study}
After verifying the superiority of MoR, we conduct ablation studies to assess its different components, including module and feature ablation.

\subsubsection{Module Ablation}


To assess the contribution of each module in MoR, namely, Text Matching-based Retrieval, Neighborhood-Fetching-based Structural Retrieval, and Reranker, we conduct a series of ablation experiments. First, we remove the Reranker, resulting in the variant MoR$_{\text{w/o R}}$. On top of that, we further separately eliminate Text Retrieval and Structural Retrieval, yielding MoR$_{\text{w/o RT}}$ and MoR$_{\text{w/o RS}}$, respectively.
As shown in Table~\ref{tab-merged}, the complete MoR framework consistently achieves the highest performance across all datasets, demonstrating the synergistic effect of the Textual Retriever, Structural Retriever, and Reranker.
After removing Reranker, MoR$_{\text{w/o R}}$ exhibits a consistent performance drop across all datasets and evaluation metrics. This underscores the importance of the Reranker in refining retrieval by filtering noisy candidates from the intermediate reasoning stage. 
Eliminating Text Retrieval, i.e., MoR$_{\text{w/o RT}}$, leads to a notable performance drop on Amazon but an unexpected improvement on MAG. This suggests that while textual knowledge benefits Amazon, it introduces misleading hard negatives that compromise the ranking method (e.g., Ada-002) for MAG. Conversely, removing Structural Retrieval, MoR$_{\text{w/o RS}}$, results in a slight performance decrease further on MAG, reinforcing the importance of structural knowledge in MAG-related queries.
%
These results underscore the Reranker's crucial role in adaptively harmonizing, balancing, and selecting knowledge from both structural and textual retrieval experts.






\begin{table}[t!]
\small
\setlength\tabcolsep{4.5pt}
\centering
\begin{tabular}{l|ccc|cccc}
\toprule
\textbf{Dataset} &\textbf{TF} & \textbf{SF} & \textbf{TI} & \textbf{H@1} & \textbf{H@5} & \textbf{R@20} & \textbf{MRR} \\ \midrule
\multirow{7}{*}{\textbf{MAG}} 
& \cmark & \xmark & \xmark & 48.96 & 73.02 & 72.44 & 59.79 \\
&      \xmark            & \cmark       &         \xmark         & 18.79 & 41.91 & 52.85 & 29.84 \\
&        \xmark          &         \xmark         & \cmark       & 18.16 & 41.53 & 52.78 & 29.31 \\
\cline{2-8}
& \cmark       & \cmark       &    \xmark              & 58.04 & 77.14 & 74.42 & 66.75 \\
& \cmark       &        \xmark          & \cmark       & \underline{58.16} & \underline{77.59} & \underline{74.96} & \underline{66.85} \\
&          \xmark        & \cmark       & \cmark       & 17.93 & 38.01 & 46.79 & 27.48 \\
\cline{2-8}
& \cmark       & \cmark       & \cmark       & \textbf{58.19} & \textbf{78.34} & \textbf{75.01} & \textbf{67.14} \\ \midrule
\multirow{7}{*}{\textbf{Amazon}}    
& \cmark       &      \xmark            &       \xmark           & \underline{51.21} & \underline{74.05} & \underline{59.79} & \underline{61.27} \\
&        \xmark          & \cmark       &      \xmark            & 8.09  & 24.48 & 25.62 & 16.94 \\
&         \xmark         &      \xmark            & \cmark       & 5.84  & 16.62 & 12.94 & 11.57 \\
\cline{2-8}
& \cmark       & \cmark       &      \xmark            & 50.91 & 73.38 & 59.58 & 61.15 \\
& \cmark       &         \xmark         & \cmark       & 51.09 & 73.56 & 59.61 & 61.14 \\
&            \xmark      & \cmark       & \cmark       & 8.09  & 24.48 & 25.62 & 16.94 \\
\cline{2-8}
& \cmark       & \cmark       & \cmark       & \textbf{52.19} & \textbf{74.65} & \textbf{59.92} & \textbf{62.24} \\ \bottomrule
\end{tabular}
\caption{Ablation study investigating the importance of three features, Textual Fingerprint (\textbf{TF}), Structural Fingerprint (\textbf{SF}), and Traversal Identifier (\textbf{TI}), of the traversal trajectories used in our Structure-aware Reranker.}
\label{tab-feature-ablation}
\vspace{-2ex}
\end{table}



\subsubsection{Feature Ablation}
The above ablation study highlights the crucial role of Structure-aware Reranker in adaptively integrating structural and textual knowledge. To further analyze the contributions of its three key features, \textbf{Textual Fingerprint (TF)}, \textbf{Structural Fingerprint (SF)}, and \textbf{Traversal Identifier (TI)} defined in Section~\ref{sec-organizing}, we conduct a feature ablation analysis and report retrieval performance across different feature configurations in Table~\ref{tab-feature-ablation}.
%Overall and individual performance
Overall, using three features together yields the best performance on both MAG and Amazon, highlighting their synergistic effect. Individually, TF contributes the most and outperforms SF and TI on both datasets. 
The reason is that based on the definition in Section~\ref{sec-organizing}, TF directly captures the relevance between the query and the retrieved nodes along the trajectory, whereas SF and TI primarily characterize the structural patterns and retrieval types, serving more as complementary factors. Therefore, equipping TF with these complementary factors (i.e., SF or TI) yields around 10\% additional gains on MAG. This is because SF and TI help the reranker selectively emphasize the relevance scores given by TF for certain nodes along the path. However, this boost is not observed on Amazon. We hypothesize that the textual knowledge needed there is predominantly derived from the final node on each path, making the structural cues provided by SF and TI less beneficial and even prone to overfitting. A deeper analysis to further justify this hypothesis is in Section~\ref{sec-further}. Overall, these findings underscore the varying importance of structural features in ranking across datasets.



\begin{table}[t!]
\small
\setlength\tabcolsep{4pt}
\centering
\begin{tabular}{l|ccc|ccc}
\toprule
\multirow{2}{*}{\textbf{Feature}} & \multicolumn{3}{c|}{\textbf{MAG}} & \multicolumn{3}{c}{\textbf{Amazon}} \\

 & H@1 & R@20 & MRR & H@1 & R@20 & MRR \\
\midrule
Last Node & 49.91 & 73.49 & 59.92 & 50.36 & 59.62 & 61.05   \\
Full Path & \textbf{58.19} & \textbf{75.01} & \textbf{67.14} & \textbf{52.19} & \textbf{59.92} & \textbf{62.24}   \\
\bottomrule
\end{tabular}
\caption{Comparing reranking performance using last node in the retrieved trajectory and the whole trajectory.}
\label{tab-Reranker-ablation}
\vspace{-2ex}
\end{table}

\begin{figure}[t!]
    \centering
    \includegraphics[width=0.49\textwidth, height = 0.22\textwidth]{figures/query-pattern-20250215.png}
    \vspace{-4.5ex}
    \caption{Imbalance number of queries and performance of different retrievers across different logical structures.}
    \label{fig-analysis}
    \vspace{-3ex}
\end{figure}





\subsection{Further Analysis}\label{sec-further}
This section understands MoR’s behavior by examining three questions, each of which enriches our insight into MoR’s functionality and offers novel perspectives inspiring future query retrieval research.

\textbf{Do structure signals affect reranking?}
To assess the impact of trajectory information on the Reranker's decision-making, we introduce a node-based Reranker that constructs trajectory features using only TF/SF/TI of the last node. In Table~\ref{tab-Reranker-ablation}, the path-based Reranker outperforms the node-based variant, especially on MAG. This highlights the critical role of trajectory features/structural knowledge in reranking. The minor performance boost on Amazon after switching to the full path trajectory indicates its textual knowledge preference over the last node rather than the whole trajectory.


\textbf{How does MoR perform on different logical structures?}
Figure~\ref{fig-analysis} shows the average performance of MoR on each query group categorized by their logical structures, where "Others" refer to queries with undefined logical structures in~\citet{wu2024stark} MoR consistently outperforms structural and textual retrievers across different logical structures. Among all queries, MoR performs the worst on "P → P" queries due to the ambiguity, although well-known, uniquely caused by repeated product entities from multi-step traversal.
The average-performing ``Others" group underscores the utility of diverse planning strategies for the same query.
Lastly, the skewed query distribution and retrieval performance across planning patterns reflect the varying nature of real-world planning needs. We hope these insights inspire research on data-centric reasoning designs and error control of planning.


\begin{figure}[t!]
    \centering
    \includegraphics[width=0.5\textwidth]{figures/heatmap-20250215.pdf}
    \vspace{-3ex}
    \caption{Saliency map visualization of query attention over three entities along the retrieved paths}
    \label{fig-map}
    \vspace{-2ex}
\end{figure}

\textbf{Does MoR indeed adaptively leverage the trajectory knowledge?} To understand how our proposed reranker prioritizes candidates in the Top-K results, we visualize the saliency map by computing the gradient of ranking scores with respect to the textual fingerprint (TF) of three nodes along the traversed path, which quantifies their importance for answering a given query. Figure~\ref{fig-map} illustrates this by analyzing trajectories for 100 ground-truth candidates across 100 queries on the Amazon and MAG datasets. Each dimension corresponds to a traversed node, with the final one representing the candidate itself. 
While the saliency score is concentrated in the last dimension for Amazon, 
MAG exhibits a more evenly distributed saliency pattern, where multiple nodes along the path contribute significantly to ranking score computation. This suggests that structural knowledge is more critical for answering queries in MAG, aligning with the previously observed lower performance of purely textual retrieval on MAG in Table~\ref{tab-merged}. Further case studies explain why the reranker attends different nodes for different queries. In Figure~\ref{fig-map}(a), the reranker favors the last two dimensions as the rich textual restriction (i.e., "Northwest Company..." and "NFL Seattle...") aids in identifying the correct node at the corresponding reasoning step, as discussed in Section~\ref{sec-reasoning}. The correct nodes, having higher similarity scores with the query, help guide the retrieval process toward the ground truth.
Conversely, in Figure~\ref{fig-map}(b),
since the first node ("University of Lausanne") helps narrow the search space and the last node ("frameless...") further filter candidates, both nodes have high saliency scores. Overall, our findings demonstrate that the reranker dynamically adapts its reliance on structural and textual knowledge depending on the dataset and query. 
















\end{document}