
\documentclass[11pt]{article}
\pdfoutput=1

\usepackage{mathrsfs}
\usepackage{amsmath,amssymb}
\usepackage{bm}
\usepackage{natbib}
\usepackage[usenames]{color}
\usepackage{amsthm}
\usepackage{algorithm}
\usepackage{algorithmic}

\usepackage{multirow} 
\usepackage{enumitem}

\usepackage{microtype}
\usepackage{graphicx}
\usepackage{subfigure}
\usepackage{booktabs,threeparttable,multirow}

\usepackage{amsmath}
\usepackage{amssymb,bm}
\usepackage{mathtools}
\usepackage{caption}
\usepackage{subcaption,color}
\usepackage{indentfirst}
\usepackage{amsfonts}
\usepackage{float,url}
\usepackage{bbm}
\usepackage{lipsum}
\usepackage{nicefrac}
\usepackage{pifont}


\usepackage[colorlinks,
linkcolor=red,
anchorcolor=blue,
citecolor=blue
]{hyperref}


\renewcommand{\baselinestretch}{1.05}


\usepackage{mylatexstyle}


\newcommand{\R}{\mathbb{R}}
\newcommand{\Sym}{\mathbb{S}}
\newcommand{\Lam}{\bm{\Lambda}}
\newcommand{\Rext}{\mathbf{R} \cup \{ \infty \}}
\newcommand{\dd}{\,\mathrm{d}}
\newcommand{\ip}[2]{\langle #1, #2 \rangle}
\newcommand{\norm}[1]{\| #1 \|}
\newcommand{\logdet}[1]{\log \det( #1 )}

% bb
\newcommand{\bbE}{\mathbb{E}}

%bold
\newcommand{\bxi}{\mathbf{\xi}}
\newcommand{\btau}{\mathbf{\tau}}
\newcommand{\bs}{\mathbf{s}}
\newcommand{\bz}{\mathbf{z}}
\newcommand{\bX}{\mathbf{X}}
\newcommand{\bx}{\mathbf{x}}
\newcommand{\ba}{\mathbf{a}}
\newcommand{\bc}{\mathbf{c}}
\newcommand{\bh}{\mathbf{h}}
\newcommand{\bw}{\mathbf{w}}
\newcommand{\bg}{\mathbf{g}}
\newcommand{\bp}{\mathbf{p}}
\newcommand{\bq}{\mathbf{q}}
\newcommand{\by}{\mathbf{y}}
\newcommand{\bl}{\bm{\lambda}}
\newcommand{\be}{\bm{\varepsilon}}
\newcommand{\bt}{\bm{\theta}}
\newcommand{\bmu}{\bm{\mu}}
\newcommand{\bsigma}{\bm{\sigma}}
\newcommand{\bnu}{\bm{\nu}}
\newcommand{\bphi}{\bm{\phi}}
\newcommand{\T}{\mathrm{T}}

\newcommand{\bmm}{\mathbf{m}}
\newcommand{\bS}{\mathbf{S}}
\newcommand{\bH}{\mathbf{H}}
\newcommand{\bV}{\mathbf{V}}
\newcommand{\bA}{\mathbf{A}}
\newcommand{\bD}{\mathbf{D}}


\newcommand{\ind}{\mathbf{i}}
\newcommand{\bE}{\mathbf{E}}
\newcommand{\bu}{\mathbf{u}}
\newcommand{\bG}{\mathbf{\Gamma}}
\newcommand{\bW}{\mathbf{W}}

%cal
\newcommand{\cL}{\mathcal{L}}
\newcommand{\cJ}{\mathcal{J}}
\newcommand{\cC}{\mathcal{C}}
\newcommand{\cD}{\mathcal{D}}
\newcommand{\cH}{\mathcal{H}}
\newcommand{\cX}{\mathcal{X}}
\newcommand{\cY}{\mathcal{Y}}
\newcommand{\cK}{\mathcal{K}}
\newcommand{\cP}{\mathcal{P}}
\newcommand{\PM}{\mathcal{P}}
\newcommand{\cM}{\mathcal{M}}
\newcommand{\cN}{\mathcal{N}}
\newcommand{\cQ}{\mathcal{Q}}
\newcommand{\cF}{\mathcal{F}}
\newcommand{\bbS}{\mathbb{S}}

% names
\newcommand{\dist}{\mathrm{dist}}
\newcommand{\epi}{\mathrm{epi}}

% other
\newcommand{\W}{\mathrm{W}}
\newcommand{\V}{\mathrm{V}}

% vector/matrix notation
\newcommand{\vc}[1]{\bm{#1}}
\newcommand{\matr}[1]{\mathbf{#1}}

\newcommand{\Ent}{\mathrm{H}}

%operators
%operators
\DeclareMathOperator{\Div}{Div}
\DeclareMathOperator{\dom}{dom}
\DeclareMathOperator{\ran}{ran}
\DeclareMathOperator{\conv}{conv}
\DeclareMathOperator{\relint}{relint}
\DeclareMathOperator{\inter}{int}
\DeclareMathOperator{\boundary}{bdry}
\DeclareMathOperator{\moreau}{M}
\DeclareMathOperator{\hull}{H}
\DeclareMathOperator{\prox}{P}
\DeclareMathOperator{\proj}{proj}
\DeclareMathOperator{\tr}{tr}
\DeclareMathOperator{\diag}{diag}
\DeclareMathOperator{\Id}{I}
\DeclareMathOperator{\cl}{cl}
\DeclareMathOperator*{\argmin}{arg\,min}
\DeclareMathOperator*{\argmax}{arg\,max}
\DeclareMathOperator*{\limin}{lim~inf}
\DeclareMathOperator*{\mean}{mean}

\DeclareMathOperator{\KLop}{KL}
\newcommand{\myKL}{\mathbb{D}_{\KLop}\infdivx}
\newcommand{\myKLs}{{\mathbb{D}_{\KLop}}^*\infdivx}
\newcommand{\Breg}[1]{\mathbb{B}_{#1}\infdivx}

%\renewcommand{\algorithmicrequire}{\textbf{Input:}}
%\renewcommand{\algorithmicensure}{\textbf{Iterate:}}



\newcommand{\disableaddcontentsline}{%
  \let\savedaddcontentsline\addcontentsline 
  \renewcommand{\addcontentsline}[3]{}
}
% recover \addcontentsline
\newcommand{\enableaddcontentsline}{%
  \let\addcontentsline\savedaddcontentsline
}

\usepackage{setspace}
%\setstretch{1.5}
\usepackage[left=1in, right=1in, top=1in, bottom=1in]{geometry}

\usepackage{xcolor}
\newcommand{\bin}[1]{\textcolor{blue}{[Bin: #1]}}

\ifdefined\final
\usepackage[disable]{todonotes}
\else
\usepackage[textsize=tiny]{todonotes}
\fi
\setlength{\marginparwidth}{0.8in}
\newcommand{\todoy}[2][]{\todo[size=\scriptsize,color=blue!20!white,#1]{Yuan: #2}}
\newcommand{\todoq}[2][]{\todo[size=\scriptsize,color=red!20!white,#1]{Quanquan: #2}}
\newcommand{\todoh}[2][]{\todo[size=\scriptsize,color=green!20!white,#1]{HZ: #2}}
\newcommand{\todot}[2][]{\todo[size=\scriptsize,color=green!20!white,#1]{YZ: #2}}

\newcommand{\yhzcomment}[1]{{\bf{{\color{orange}{{HZ {---} #1}}}}}}

% \def\CC{\textcolor{red}}
\def\CC{}


\title{\huge Re-examining Double Descent and Scaling Laws under Norm-based Capacity via Deterministic Equivalence}


\author
{
     Yichen Wang\thanks{Department of Computer Science, University of Warwick, UK. e-mail: {\tt yichen.wang7@warwick.ac.uk}} 
     ~~~
     Yudong Chen\thanks{Department of Computer Sciences, University of Wisconsin-Madison, USA. e-mail: {\tt yudong.chen@wisc.edu}}
      ~~~
     Lorenzo Rosasco\thanks{MaLGa Center – DIBRIS – Università di Genova, Genoa, Italy; also CBMM – Massachusets Institute of Technology, USA; also Istituto Italiano di Tecnologia, Genoa, Italy. e-mail: {\tt lrosasco@mit.edu}} 
~~~
    Fanghui Liu\thanks{Department of Computer Science, also Centre for Discrete Mathematics and its Applications (DIMAP), University of Warwick, UK. e-mail: {\tt fanghui.liu@warwick.ac.uk} (Corresponding author)} 
}



\date{}




% \def\CCC{\textcolor{red}}

% \def\CC{}




\begin{document}
\disableaddcontentsline




\maketitle


\begin{abstract}
We investigate double descent and scaling laws in terms of weights rather than the number of parameters. Specifically, we analyze linear and random features models using the deterministic equivalence approach from random matrix theory. We precisely characterize how the weights norm concentrate around deterministic quantities and elucidate the relationship between the expected test error and the norm-based capacity (complexity). Our results rigorously answer whether double descent exists under norm-based capacity and reshape the corresponding scaling laws. Moreover, they prompt a rethinking of the data–parameter paradigm—from under-parameterized to over-parameterized regimes—by shifting the focus to norms (weights) rather than parameter count.
\end{abstract}

%------------------------------------------------



\section{Introduction}

Large language models (LLMs) have achieved remarkable success in automated math problem solving, particularly through code-generation capabilities integrated with proof assistants~\citep{lean,isabelle,POT,autoformalization,MATH}. Although LLMs excel at generating solution steps and correct answers in algebra and calculus~\citep{math_solving}, their unimodal nature limits performance in plane geometry, where solution depends on both diagram and text~\citep{math_solving}. 

Specialized vision-language models (VLMs) have accordingly been developed for plane geometry problem solving (PGPS)~\citep{geoqa,unigeo,intergps,pgps,GOLD,LANS,geox}. Yet, it remains unclear whether these models genuinely leverage diagrams or rely almost exclusively on textual features. This ambiguity arises because existing PGPS datasets typically embed sufficient geometric details within problem statements, potentially making the vision encoder unnecessary~\citep{GOLD}. \cref{fig:pgps_examples} illustrates example questions from GeoQA and PGPS9K, where solutions can be derived without referencing the diagrams.

\begin{figure}
    \centering
    \begin{subfigure}[t]{.49\linewidth}
        \centering
        \includegraphics[width=\linewidth]{latex/figures/images/geoqa_example.pdf}
        \caption{GeoQA}
        \label{fig:geoqa_example}
    \end{subfigure}
    \begin{subfigure}[t]{.48\linewidth}
        \centering
        \includegraphics[width=\linewidth]{latex/figures/images/pgps_example.pdf}
        \caption{PGPS9K}
        \label{fig:pgps9k_example}
    \end{subfigure}
    \caption{
    Examples of diagram-caption pairs and their solution steps written in formal languages from GeoQA and PGPS9k datasets. In the problem description, the visual geometric premises and numerical variables are highlighted in green and red, respectively. A significant difference in the style of the diagram and formal language can be observable. %, along with the differences in formal languages supported by the corresponding datasets.
    \label{fig:pgps_examples}
    }
\end{figure}



We propose a new benchmark created via a synthetic data engine, which systematically evaluates the ability of VLM vision encoders to recognize geometric premises. Our empirical findings reveal that previously suggested self-supervised learning (SSL) approaches, e.g., vector quantized variataional auto-encoder (VQ-VAE)~\citep{unimath} and masked auto-encoder (MAE)~\citep{scagps,geox}, and widely adopted encoders, e.g., OpenCLIP~\citep{clip} and DinoV2~\citep{dinov2}, struggle to detect geometric features such as perpendicularity and degrees. 

To this end, we propose \geoclip{}, a model pre-trained on a large corpus of synthetic diagram–caption pairs. By varying diagram styles (e.g., color, font size, resolution, line width), \geoclip{} learns robust geometric representations and outperforms prior SSL-based methods on our benchmark. Building on \geoclip{}, we introduce a few-shot domain adaptation technique that efficiently transfers the recognition ability to real-world diagrams. We further combine this domain-adapted GeoCLIP with an LLM, forming a domain-agnostic VLM for solving PGPS tasks in MathVerse~\citep{mathverse}. 
%To accommodate diverse diagram styles and solution formats, we unify the solution program languages across multiple PGPS datasets, ensuring comprehensive evaluation. 

In our experiments on MathVerse~\citep{mathverse}, which encompasses diverse plane geometry tasks and diagram styles, our VLM with a domain-adapted \geoclip{} consistently outperforms both task-specific PGPS models and generalist VLMs. 
% In particular, it achieves higher accuracy on tasks requiring geometric-feature recognition, even when critical numerical measurements are moved from text to diagrams. 
Ablation studies confirm the effectiveness of our domain adaptation strategy, showing improvements in optical character recognition (OCR)-based tasks and robust diagram embeddings across different styles. 
% By unifying the solution program languages of existing datasets and incorporating OCR capability, we enable a single VLM, named \geovlm{}, to handle a broad class of plane geometry problems.

% Contributions
We summarize the contributions as follows:
We propose a novel benchmark for systematically assessing how well vision encoders recognize geometric premises in plane geometry diagrams~(\cref{sec:visual_feature}); We introduce \geoclip{}, a vision encoder capable of accurately detecting visual geometric premises~(\cref{sec:geoclip}), and a few-shot domain adaptation technique that efficiently transfers this capability across different diagram styles (\cref{sec:domain_adaptation});
We show that our VLM, incorporating domain-adapted GeoCLIP, surpasses existing specialized PGPS VLMs and generalist VLMs on the MathVerse benchmark~(\cref{sec:experiments}) and effectively interprets diverse diagram styles~(\cref{sec:abl}).

\iffalse
\begin{itemize}
    \item We propose a novel benchmark for systematically assessing how well vision encoders recognize geometric premises, e.g., perpendicularity and angle measures, in plane geometry diagrams.
	\item We introduce \geoclip{}, a vision encoder capable of accurately detecting visual geometric premises, and a few-shot domain adaptation technique that efficiently transfers this capability across different diagram styles.
	\item We show that our final VLM, incorporating GeoCLIP-DA, effectively interprets diverse diagram styles and achieves state-of-the-art performance on the MathVerse benchmark, surpassing existing specialized PGPS models and generalist VLM models.
\end{itemize}
\fi

\iffalse

Large language models (LLMs) have made significant strides in automated math word problem solving. In particular, their code-generation capabilities combined with proof assistants~\citep{lean,isabelle} help minimize computational errors~\citep{POT}, improve solution precision~\citep{autoformalization}, and offer rigorous feedback and evaluation~\citep{MATH}. Although LLMs excel in generating solution steps and correct answers for algebra and calculus~\citep{math_solving}, their uni-modal nature limits performance in domains like plane geometry, where both diagrams and text are vital.

Plane geometry problem solving (PGPS) tasks typically include diagrams and textual descriptions, requiring solvers to interpret premises from both sources. To facilitate automated solutions for these problems, several studies have introduced formal languages tailored for plane geometry to represent solution steps as a program with training datasets composed of diagrams, textual descriptions, and solution programs~\citep{geoqa,unigeo,intergps,pgps}. Building on these datasets, a number of PGPS specialized vision-language models (VLMs) have been developed so far~\citep{GOLD, LANS, geox}.

Most existing VLMs, however, fail to use diagrams when solving geometry problems. Well-known PGPS datasets such as GeoQA~\citep{geoqa}, UniGeo~\citep{unigeo}, and PGPS9K~\citep{pgps}, can be solved without accessing diagrams, as their problem descriptions often contain all geometric information. \cref{fig:pgps_examples} shows an example from GeoQA and PGPS9K datasets, where one can deduce the solution steps without knowing the diagrams. 
As a result, models trained on these datasets rely almost exclusively on textual information, leaving the vision encoder under-utilized~\citep{GOLD}. 
Consequently, the VLMs trained on these datasets cannot solve the plane geometry problem when necessary geometric properties or relations are excluded from the problem statement.

Some studies seek to enhance the recognition of geometric premises from a diagram by directly predicting the premises from the diagram~\citep{GOLD, intergps} or as an auxiliary task for vision encoders~\citep{geoqa,geoqa-plus}. However, these approaches remain highly domain-specific because the labels for training are difficult to obtain, thus limiting generalization across different domains. While self-supervised learning (SSL) methods that depend exclusively on geometric diagrams, e.g., vector quantized variational auto-encoder (VQ-VAE)~\citep{unimath} and masked auto-encoder (MAE)~\citep{scagps,geox}, have also been explored, the effectiveness of the SSL approaches on recognizing geometric features has not been thoroughly investigated.

We introduce a benchmark constructed with a synthetic data engine to evaluate the effectiveness of SSL approaches in recognizing geometric premises from diagrams. Our empirical results with the proposed benchmark show that the vision encoders trained with SSL methods fail to capture visual \geofeat{}s such as perpendicularity between two lines and angle measure.
Furthermore, we find that the pre-trained vision encoders often used in general-purpose VLMs, e.g., OpenCLIP~\citep{clip} and DinoV2~\citep{dinov2}, fail to recognize geometric premises from diagrams.

To improve the vision encoder for PGPS, we propose \geoclip{}, a model trained with a massive amount of diagram-caption pairs.
Since the amount of diagram-caption pairs in existing benchmarks is often limited, we develop a plane diagram generator that can randomly sample plane geometry problems with the help of existing proof assistant~\citep{alphageometry}.
To make \geoclip{} robust against different styles, we vary the visual properties of diagrams, such as color, font size, resolution, and line width.
We show that \geoclip{} performs better than the other SSL approaches and commonly used vision encoders on the newly proposed benchmark.

Another major challenge in PGPS is developing a domain-agnostic VLM capable of handling multiple PGPS benchmarks. As shown in \cref{fig:pgps_examples}, the main difficulties arise from variations in diagram styles. 
To address the issue, we propose a few-shot domain adaptation technique for \geoclip{} which transfers its visual \geofeat{} perception from the synthetic diagrams to the real-world diagrams efficiently. 

We study the efficacy of the domain adapted \geoclip{} on PGPS when equipped with the language model. To be specific, we compare the VLM with the previous PGPS models on MathVerse~\citep{mathverse}, which is designed to evaluate both the PGPS and visual \geofeat{} perception performance on various domains.
While previous PGPS models are inapplicable to certain types of MathVerse problems, we modify the prediction target and unify the solution program languages of the existing PGPS training data to make our VLM applicable to all types of MathVerse problems.
Results on MathVerse demonstrate that our VLM more effectively integrates diagrammatic information and remains robust under conditions of various diagram styles.

\begin{itemize}
    \item We propose a benchmark to measure the visual \geofeat{} recognition performance of different vision encoders.
    % \item \sh{We introduce geometric CLIP (\geoclip{} and train the VLM equipped with \geoclip{} to predict both solution steps and the numerical measurements of the problem.}
    \item We introduce \geoclip{}, a vision encoder which can accurately recognize visual \geofeat{}s and a few-shot domain adaptation technique which can transfer such ability to different domains efficiently. 
    % \item \sh{We develop our final PGPS model, \geovlm{}, by adapting \geoclip{} to different domains and training with unified languages of solution program data.}
    % We develop a domain-agnostic VLM, namely \geovlm{}, by applying a simple yet effective domain adaptation method to \geoclip{} and training on the refined training data.
    \item We demonstrate our VLM equipped with GeoCLIP-DA effectively interprets diverse diagram styles, achieving superior performance on MathVerse compared to the existing PGPS models.
\end{itemize}

\fi 


\vspace{-0.cm}
\section{Preliminaries}\label{subs:preliminaries}

For a comprehensive introduction to feedback linearization, the interested reader is referred to \cite{Isidori1995}.

Consider a multivariable nonlinear system
    \begin{equation}
    \left\{
\begin{split}
     \dot{\xv}&= \fv(\xv)+\Gm(\xv)\uv,\\
    \yv &= \hv(\xv),
    \label{eq:sys}
    \end{split}
    \right.
    \end{equation}
where \mbox{$\xv\in \mathbb{R}^n$} is the state, the input matrix is \mbox{$\Gm(\xv)=\begin{bmatrix}
    \gv_1(\xv)  \;\cdots \; \gv_{p}(\xv)
\end{bmatrix}\in \mathbb{R}^{n \times p}$}, $\fv(\xv)$,  $\gv_1(\xv), \ldots, \gv_{p}(\xv)$ are smooth vector fields, and $\hv(\xv)=\begin{bmatrix} h_1(\xv)  \cdots h_{p}(\xv)\end{bmatrix}^\top$ is a smooth function defined on an open set of $\mathbb{R}^n$.
The  system (\ref{eq:sys}) is said to have \emph{(vector) relative degree} $\rv = \{
    r_1, \ldots, r_{p}
\}$ at a point $\xv^\circ$ w.r.t. the input-output pair  $(\uv,\yv)$ if  
\begin{flalign}
&\text{\textrm{(i)}}    & L_{\gv_j}L^{k}_{\fv} h_i(\xv) &=0,&
\end{flalign}
for all $1\leq j \leq p$, for all $k\leq r_i-1$, for all $1\leq i \leq p$ and for all $\xv$ in a neighborhood of $\xv^\circ$, and\\
\textrm{(ii)}\;\;  the $p\times p$ matrix 
   \begin{align}
    \Am(\xv) &:= 
         \begin{pmatrix}
            L_{\gv_1}L^{r_1-1}_{\fv}h_1(\xv) & \cdots&  L_{\gv_{p}}L^{r_1-1}_{\fv}h_1(\xv) \\ 
            L_{\gv_1}L^{r_2-1}_{\fv}h_2(\xv) & \cdots&  L_{\gv_{p}}L^{r_2-1}_{\fv}h_2(\xv) \\  
            \vdots & & \vdots \\
            L_{\gv_1}L^{r_{p}-1}_{\fv}h_{p}(\xv) & \cdots&  L_{\gv_{p}}L^{r_{p}-1}_{\fv}h_{p}(\xv) 
        \end{pmatrix}
    \label{eq:intbmatrix}
\end{align}  
is nonsingular at $\xv = \xv^\circ$. 
The output array at the $\rv$-th derivative may then be written as an affine system of the form
\begin{equation}
{\yv}^{(\rv)} :=\begin{bmatrix}
    y_1^{(r_1)} \;\cdots\;  y_{p}^{(r_{p})}
\end{bmatrix}^\top =\bv(\xv)+\Am(\xv)\uv,
\label{eq:y_r_now}
\end{equation}
with 
\begin{equation}
 \bv(\xv):=\begin{bmatrix}
L_{\boldsymbol{f}}^{(r_1)}{h_1(\xv)} \; \cdots \; 
L_{\boldsymbol{f}}^{(r_{p})}{h_{p}(\xv)}
\end{bmatrix}^\top.
\label{eq:b}
\end{equation}


Suppose the system \eqref{eq:sys} has some \emph{(vector) relative degree} $\rv:=\{r_1,\ldots,r_p\}$ at $\xv^\circ$ and that the matrix $\Gm(\xv^\circ)$ has rank $p$ in a  neighborhood $\mathcal{U}$ of $\xv^\circ$. Suppose also that  \mbox{$r_1+r_2+\ldots+r_p=n$}, and choose the  control input to be $$\uv = \Am^{-1}(\xv)[-\bv(\xv)+\vv],
$$
where $\vv \in \mathbb{R}^{p}$ can be assigned freely and $\Am(\xv), \bv(\xv)$ are defined as in~(\ref{eq:intbmatrix}) and~(\ref{eq:b}).
Then the output dynamics \eqref{eq:y_r_now} become
$$
\yv^{(\rv)} = \vv.
$$
We refer to \(\yv\) as a \textit{linearizing output array}, which possesses the property that the entire state and input of the system can be expressed in terms of \(\yv\) and its time derivatives. 



\section{Main results on linear regression}
\label{sec:linear}


\begin{figure*}[t]
    \centering
    \subfigure[Test Risk vs. $\gamma:=d/n$]{
        \includegraphics[width=0.22\textwidth]{arxiv_version/figures/Main_results_on_linear_regression/risk.pdf}
    }\label{fig:linear_regression_risk_vs_norm_1}
    \subfigure[$\ell_2$ norm vs. $\gamma$]{
        \includegraphics[width=0.22\textwidth]{arxiv_version/figures/Main_results_on_linear_regression/norm.pdf}
    }\label{fig:linear_regression_risk_vs_norm_2}
    \subfigure[Test Risk vs. $\ell_2$ norm]{
        \includegraphics[width=0.22\textwidth]{arxiv_version/figures/Main_results_on_linear_regression/risk_vs_norm.pdf}
    }\label{fig:linear_regression_risk_vs_norm_3}
    \subfigure[Risk vs. norm ($\lambda\!=\!0.05$)]{
        \includegraphics[width=0.22\textwidth]{arxiv_version/figures/Main_results_on_linear_regression/risk_vs_norm_single.pdf}
    }\label{fig:linear_regression_risk_vs_norm_4}
    \caption{Results for the ridge regression estimator. Points in these four figures are given by our experimental results, and the curves are given by our theoretical results via deterministic equivalents. Training data \(\{(\bx_i, y_i)\}_{i \in [n]}\), \(d = 1000\), sampled from a linear model \(y_i = \bx_i^\sT \bbeta_* + \varepsilon_i\), \(\sigma^2 = 0.0004\), \(\bx_i \sim \mathcal{N}(0, \bSigma)\), with \(\sigma_k(\bSigma)=k^{-1}\), \(\bbeta_{*,k}=k^{-\nicefrac{3}{2}}\).} 
    \label{fig:linear_regression_risk_vs_norm}
\end{figure*}

In this section, we study the non-asymptotic deterministic equivalent of the norm of the (ridge/ridgeless) estimator for linear regression. 
We also deliver the asymptotic results in \cref{app:asy_deter_equiv_lr}, which lays a foundation of asymptotic results for RFMs in \cref{sec:rff}.
Based on these results, we are able to mathematically characterize the test risk under norm-based capacity as shown in \cref{fig:linear_regression_risk_vs_norm}.

To deliver our results, we need the following lemma for the bias-variance decomposition of the estimator's norm.
\begin{lemma}[Bias-variance decomposition of $\mathcal{N}_{\lambda}^{\tt LS}$]
\label{lemma:biasvariance}
We have the bias-variance decomposition $\E_{\varepsilon}\|\hat{\bbeta}\|_2^2 =: \mathcal{N}_{\lambda}^{\tt LS} = \mathcal{B}^{\tt LS}_{\mathcal{N},\lambda} + \mathcal{V}^{\tt LS}_{\mathcal{N},\lambda}$, where $\mathcal{B}^{\tt LS}_{\mathcal{N},\lambda}$ and $\mathcal{V}^{\tt LS}_{\mathcal{N},\lambda}$ are defined as 
\[
\begin{aligned}
    \mathcal{B}^{\tt LS}_{\mathcal{N},\lambda} := \<\bbeta_*, (\bX^\sT\bX)^2(\bX^\sT\bX + \lambda\id)^{-2}\bbeta_*\>\,, \quad \mathcal{V}^{\tt LS}_{\mathcal{N},\lambda} := \sigma^2\Tr(\bX^\sT\bX(\bX^\sT\bX + \lambda\id)^{-2})\,.
\end{aligned}
\]
\end{lemma}

Our goal is to relate $\mathcal{B}^{\tt LS}_{\mathcal{N},\lambda}$ and $\mathcal{V}^{\tt LS}_{\mathcal{N},\lambda}$ to their respective deterministic equivalents defined below (proved in \cref{sec:linear_nonasym})
\begin{align}
    \sB_{\sN,\lambda}^{\tt LS} :=&~ \<\bbeta_*, \bSigma^2(\bSigma + \lambda_*\id)^{-2}\bbeta_*\> +{\color{red}\frac{\Tr(\bSigma(\bSigma + \lambda_*\id)^{-2})}{n}} \cdot \frac{\lambda_*^2 \langle \bbeta_*,\bSigma(\bSigma + \lambda_*\id)^{-2}\bbeta_*\rangle}{1-n^{-1}\Tr(\bSigma^2(\bSigma + \lambda_*\id)^{-2})} \,, \notag \\
    \sV_{\sN,\lambda}^{\tt LS} :=&~ \frac{\sigma^2\Tr({\color{blue}\bSigma}(\bSigma+\lambda_*\id)^{-2})}{n-\Tr(\bSigma^2(\bSigma+\lambda_*\id)^{-2})}\,. \label{eq:equiv-linear}
\end{align}
When checking \cref{eq:de_risk} and \cref{eq:equiv-linear}, we find that \textit{i)} the variance term is almost the same except that $\sV^{\tt LS}_{\sR,\lambda}$ has an additional $\bSigma$ ({\color{blue}in blue}). That means, under isotropic features $\bm \Sigma = \id_d$, they are the same.
\textit{ii)} For the bias term, we find that the second term of $\sB_{\sN,\lambda}^{\tt LS}$ ({\color{red}in red}) rescales $\sB_{\sR,\lambda}^{\tt LS}$ in \cref{eq:de_risk} by a factor $n^{-1}\Tr(\bSigma(\bSigma + \lambda_*\id)^{-2})$.

Accordingly, the norm-based capacity is able to characterize the bias and variance of the excess risk.
We will quantitatively characterize this relationship in \cref{sec:relationship_lrr}.


\subsection{Non-asymptotic analysis}
\label{sec:linear_nonasym}

To derive the non-asymptotic results, we make the following assumption on well-behaved data.
\begin{assumption}[Data concentration, \citealt{misiakiewicz2024non}]\label{ass:concentrated_LR} There exist $C_* > 0$ such that for any PSD matrix $\bA \in \mathbb{R}^{d \times d}$ with $\Tr(\bm{\Sigma A}) < \infty$ and $t\ge 0$, we have
    \[
    \begin{aligned}
         &~\mathbb{P}\left(\left| \bX^\sT \bA \bX - \Tr(\bm{\Sigma A}) \right| \geq t\|\bSigma^{1/2} \bA \bSigma^{1/2}\|_{\mathrm{F}} \right) \leq C_* e^{-\frac{t}{C_*}}\,.
    \end{aligned}
    \]
\end{assumption}


\begin{assumption}[Power-law assumption]\label{ass:powerlaw}
For the covariance matrix $\bm \Sigma$ and the target function $\bm \beta_*$, we assume that $ \sigma_k(\bm \Sigma) = k^{-\alpha}, \alpha >0$ and $ \bbeta_{*,k} =k^{-\nicefrac{\alpha\beta}{2}},  \beta \in \mathbb{R}$.
\end{assumption}

This assumption is close to classical source condition and capacity condition~\citep{caponnetto2007optimal} and is similarly used in \citet[Assumption 1]{paquette20244+}.

Based on the above two assumptions, we are ready to deliver the following result, see the proof in \cref{app:nonasy_deter_equiv_lr}.
\begin{theorem}[Deterministic equivalents of $\mathcal{N}_{\lambda}^{\tt LS}$, simplified version of \cref{prop:det_equiv_LR}, see \cref{fig:linear_regression_risk_vs_norm}]\label{prop:non-asy_equiv_norm_LR}
    Under \cref{ass:concentrated_LR} and \ref{ass:powerlaw}, for any $D,K >0$, if $\lambda > n^{-K}$, with probability at least $1-n^{-D}$, we have 
    \[
    \left|\mathcal{B}^{\tt LS}_{\mathcal{N},\lambda} - \sB_{\sN,\lambda}^{\tt LS}\right| \leq \widetilde{\mathcal{O}} (n^{-\frac{1}{2}}) \cdot \sB_{\sN,\lambda}^{\tt LS} \quad \text{and} \quad \left|\mathcal{V}^{\tt LS}_{\mathcal{N},\lambda} - \sV_{\sN,\lambda}^{\tt LS}\right| \leq \widetilde{\mathcal{O}} (n^{-\frac{1}{2}}) \cdot \sV_{\sN,\lambda}^{\tt LS}\,,
    \]
where these quantities are from \cref{lemma:biasvariance} and \cref{eq:equiv-linear}.
\end{theorem}
\noindent{\bf Remark:} Our results are numerically validated by \cref{fig:linear_regression_risk_vs_norm}. Besides, our theory is still valid under weaker assumptions related to \emph{effective dimension} used in \citet{misiakiewicz2024non} but the formulation will be more complex. We detail this in \cref{app:nonasy_deter_equiv_lr}.
The results for min-$\ell_2$-norm interpolator are given by \cref{prop:asy_equiv_norm_LR_minnorm}. We show that the solution $\lambda_n$ to the self-consistent equation $\Tr(\bSigma(\bSigma+\lambda_n\id)^{-1}) \sim n$ can be obtained from the variance $\sV_{\sN,0}^{\tt LS}=\sigma^2/\lambda_n$.


\subsection{Relationship between test risk and norm}
\label{sec:relationship_lrr}

Here we give some concrete examples on the relationship between $\sR$ and  $\sN$ in terms of isotropic features and power-law setting, see the proof in \cref{app:relationship}.

\begin{proposition}[Isotropic features for ridge regression, see \cref{fig:lampls}]\label{prop:relation_id}
    Consider covariance matrix $\bSigma = \id_d$, the deterministic equivalents $\sR^{\tt LS}_{\lambda}$ and $\sN^{\tt LS}_{\lambda}$ satisfy
    \[
    \begin{aligned}
        & \left(\|\bbeta_*\|_2^2 - \sR^{\tt LS}_{\lambda} - \sN^{\tt LS}_{\lambda}\right)\left(\|\bbeta_*\|_2^2 + \sR^{\tt LS}_{\lambda} - \sN^{\tt LS}_{\lambda}\right)^2d + 2\|\bbeta_*\|_2^2\left(\left(\|\bbeta_*\|_2^2 + \sR^{\tt LS}_{\lambda} - \sN^{\tt LS}_{\lambda}\right)^2-4\|\bbeta_*\|_2^2\sR^{\tt LS}_{\lambda} \right) \lambda\\
        &= 2\left( \left(\sR^{\tt LS}_{\lambda} - \sN^{\tt LS}_{\lambda}\right)^2 - \|\bbeta_*\|_2^4 \right) d \sigma^2\,.
    \end{aligned}
    \]
\end{proposition}
\vspace{-0.2cm}
\noindent{\bf Remark:} $\sR^{\tt LS}_\lambda$ and $\sN^{\tt LS}_\lambda$ formulates a third-order polynomial.
When $\lambda \to \infty$, it degenerates to $\sR^{\tt LS}_{\lambda} = (\|\bbeta_*\|_2 - \sqrt{\sN^{\tt LS}_{\lambda}})^2 $ when $ \sN^{\tt LS}_{\lambda} \leq \|\bbeta_*\|_2$.
Hence \( \sR^{\tt LS}_{\lambda} \) is monotonically decreasing with respect to \( \sN^{\tt LS}_{\lambda} \), empirically verified by \cref{fig:lampls}.
Besides, if we take $\lambda = \frac{d\sigma^2}{\|\bbeta_*\|_2^2}$, which is the {\bf optimal regularization parameter} discussed in \citet{wu2020optimal, nakkiran2020optimal}, the relationship in \cref{prop:relation_id} will become $\sR^{\tt LS}_{\lambda} = \|\bbeta_*\|_2^2 - \sN^{\tt LS}_{\lambda}$, which corresponds to a straight line. This is empirically shown in \cref{fig:lampls} with $\lambda=50$.
   

Apart from sufficiently large $\lambda$ and optimal $\lambda$ mentioned before, below we consider min-$\ell_2$-norm estimator. 
Note that when $\lambda \to 0$, the ridge regression estimator $\hat{\bbeta}$ converges to the min-$\ell_2$-norm estimator $\hat{\bbeta}_{\min}$.
However, the behavior of \(\lambda_*\) differs between the under-parameterized and over-parameterized regimes as \(\lambda \to 0\). In the under-parameterized regime, \(\lambda_*\) approaches 0, while in the over-parameterized regime, \(\lambda_*\) approaches a constant that satisfies \(\Tr(\bSigma(\bSigma + \lambda_n \id)^{-1}) = n\).
Thus, the min-\(\ell_2\)-norm estimator requires {\bf separate analysis of the two regimes}. 


\begin{figure}[t]
    \centering
    \subfigure[Test Risk vs.\ $\ell_2$-norm]{\label{fig:lampls}
        \includegraphics[width=0.3\textwidth]{arxiv_version/figures/Main_results_on_linear_regression/linear_regression_risk_vs_norm_id_multi_lambda.pdf}
    }
    \subfigure[Risk vs.\ norm ($\lambda \to 0$)]{\label{fig:lam0ls}
        \includegraphics[width=0.3\textwidth]{arxiv_version/figures/Main_results_on_linear_regression/risk_vs_norm_id_ridgeless.pdf}
    }
    \caption{Relationship between $\sR^{\tt LS}_\lambda$ and $\sN^{\tt LS}_\lambda$ in \cref{fig:lampls}; $\sR^{\tt LS}_0$ and $\sN^{\tt LS}_0$ in \cref{fig:lam0ls} under the linear model \(y_i = \bx_i^\sT \bbeta_* + \varepsilon_i\), with $d=500$, \(\bSigma = \id_d\), \(\|\bbeta_*\|_2^2=10\), and \(\sigma^2 = 1\).} %\fh{this is for isotropic data?} 
    \label{fig:linear_risk}\vspace{-0.2cm}
\end{figure}


\begin{proposition}[Relationship for min-$\ell_2$-norm interpolator in the {\bf under-parameterized} regime]\label{prop:relation_minnorm_underparam}
The deterministic equivalents $\sR^{\tt LS}_{0}$ and $\sN^{\tt LS}_{0}$, in under-parameterized regimes ($d < n$) admit the linear relationship
\[
    \begin{aligned}
        \sR^{\tt LS}_0 = {d}\left(\sN^{\tt LS}_0 - \|\bbeta_*\|_2^2\right)/{\Tr(\bSigma^{-1})}\,.
    \end{aligned}
\]
\end{proposition}

The relationship in the over-parameterized regime is more complicated. We present it in the special case of isotropic features in \cref{prop:relation_minnorm_id} of \cref{prop:relation_id}, and we also give an approximation in \cref{prop:relation_minnorm_pl} under the power-law assumption.


\begin{corollary}[Isotropic features for min-$\ell_2$-norm interpolator, see \cref{fig:lam0ls}]\label{prop:relation_minnorm_id}
    Consider covariance matrix $\bSigma = \id_d$, the relationship between $\sR^{\tt LS}_0$ and $\sN^{\tt LS}_0$ from under-parameterized to over-parameterized regimes admit
    \begin{equation*}
		\sR^{\tt LS}_0 = \left\{
		\begin{array}{rcl}
			\begin{aligned}
				&  \sN^{\tt LS}_0 - \|\bbeta_*\|_2^2\,,  ~~\text{if}~~ d<n ~\mbox{(under-parameterized)} ; \\
				& \sqrt{\left[\sN^{\tt LS}_0 - (\|\bbeta_*\|_2^2 - \sigma^2)\right]^2 + 4\|\bbeta_*\|_2^2 \sigma^2 } - \sigma^2 \,, \mbox{o/w}\,.
			\end{aligned}
		\end{array} \right.
    \end{equation*}
    For the variance part of $\sR^{\tt LS}_0$ and $\sN^{\tt LS}_0$, we have $\sV_{\sR,0}^{\tt LS} = \sV_{\sN,0}^{\tt LS}$; For the respective bias part, we have $\sB_{\sR,0}^{\tt LS} + \sB_{\sN,0}^{\tt LS} = \| \bm \beta_* \|_2^2$.
\end{corollary}
\noindent{\bf Remark:} 
In the under-parameterized regime, the test error $\sR^{\tt LS}_0$ is a linear function of the norm $\sN^{\tt LS}_0$. 
In the over-parameterized regime, $\sR^{\tt LS}_0$ and $\sN^{\tt LS}_0$ formulates a rectangular hyperbola: $\sR^{\tt LS}_0$ decreases with $\sN^{\tt LS}_0$ if $\sN^{\tt LS}_0 < \|\bbeta_*\|_2^2 - \sigma^2$ while $\sR^{\tt LS}_0$ increases with $\sN^{\tt LS}_0$ if $\sN^{\tt LS}_0 > \|\bbeta_*\|_2^2 - \sigma^2$.

Instead of assuming $\bSigma = \id_d$, we consider power-law features in \cref{ass:powerlaw} and characterize the relationship.
\begin{proposition}[Power-law features for min-$\ell_2$ norm estimator]\label{prop:relation_minnorm_pl}
    Under \cref{ass:powerlaw}, in the over-parameterized regime ($d>n$), we consider some special cases for analytic formulation: if $\alpha=1$, when $n \to d$, we have\footnote{The symbol $\approx$ here represents two types of approximations: i) approximation for self-consistent equations; ii) Taylor approximation of logarithmic function around zero (related to $n \to d$).}
    \[
    \begin{aligned}
        \sV_{\sR, 0}^{\tt LS} \approx \frac{2(\sV_{\sN, 0}^{\tt LS})^2}{d\sV_{\sN, 0}^{\tt LS}-d^2\sigma^2}\,, 
    \end{aligned}
    \]
    and further for different $\beta$, when $n \to d$, we have
    \begin{equation*}
		\sB_{\sR,0}^{\tt LS} \approx \left\{
		\begin{array}{rcl}
			\begin{aligned}
                    &  \frac{2\sB_{\sN, 0}^{\tt LS}(d-\sB_{\sN,0}^{\tt LS})}{d^2}\,,  ~~\text{$\beta=0$} ; \\
                    & \frac{2(\sB_{\sN, 0}^{\tt LS} - \Tr(\bSigma))}{d\sqrt{1+2\sB_{\sN, 0}^{\tt LS}-2\Tr(\bSigma)}} \,, ~~\text{$\beta=1$} ; \\
                    & \frac{216 (\sB_{\sN, 0}^{\tt LS})^4 \!-\! 324d^2 (\sB_{\sN, 0}^{\tt LS})^3 \!+\! 126d^4 (\sB_{\sN, 0}^{\tt LS})^2 \!+\! d^6 \sB_{\sN, 0}^{\tt LS} \!-\! 5d^8}{2d^5(6 \sB_{\sN, 0}^{\tt LS}-d^2)} \,, ~~\text{$\beta=-1$}.
			\end{aligned}
		\end{array} \right.
    \end{equation*}
\end{proposition}
\noindent{\bf Remark:}
The relationship between $\sR^{\tt LS}_0$ and $\sN^{\tt LS}_0$ is quite complex in the over-parameterized regime. We characterize some special cases here and find that they are still precise by our experiments in \cref{fig:linear_regression_power_law}.

\begin{figure*}[!ht]
    \centering
    \subfigure[\(\beta = 0\)]{\label{fig:lrpla}
        \includegraphics[width=0.22\textwidth]{arxiv_version/figures/Main_results_on_linear_regression/beta_0_B.pdf}
    }
    \subfigure[\(\beta = 1\)]{\label{fig:lrplb}
        \includegraphics[width=0.22\textwidth]{arxiv_version/figures/Main_results_on_linear_regression/beta_1_B.pdf}
    }
    \subfigure[\(\beta = -1\)]{\label{fig:lrplc}
        \includegraphics[width=0.22\textwidth]{arxiv_version/figures/Main_results_on_linear_regression/beta_-1_B.pdf}
    }
    \subfigure[$\sV_{\sR,0}^{\tt LS}$ vs. $\sV_{\sN,0}^{\tt LS}$]{\label{fig:lrpld}
        \includegraphics[width=0.22\textwidth]{arxiv_version/figures/Main_results_on_linear_regression/V.pdf}
    }
    \caption{\cref{fig:lrpla,fig:lrplb,fig:lrplc} show the relationship between $\sB_{\sR,0}^{\tt LS}$ and $\sB_{\sN,0}^{\tt LS}$ when $\alpha =1$ and $\beta$ takes on different values. \cref{fig:lrpld} shows the relationship between $\sV_{\sR,0}^{\tt LS}$ and $\sV_{\sN,0}^{\tt LS}$ when $\alpha = 1$. The {\color{blue}blue line} is the relationship obtained by deterministic equivalent experiments, and the {\color{red}red line} is the approximate relationship we give.}
    \label{fig:linear_regression_power_law}
\end{figure*}

\section{Main results on random feature regression}
\label{sec:rff}

In this section, we mathematically characterize \cref{fig:random_feature_risk_vs_norm} under norm-based capacity.
Similar to ridge regression, we have the following bias-variance decomposition for the norm.


\begin{lemma}[Bias-variance decomposition of $\mathcal{N}_{\lambda}^{\tt RFM}$]
\label{lemma:biasvariance_rf}
We have the bias-variance decomposition $\E_{\varepsilon}\|\hat{\ba}\|_2^2 =: \mathcal{N}_{\lambda}^{\tt RFM} = \mathcal{B}_{\mathcal{N},\lambda}^{\tt RFM} + \mathcal{V}_{\mathcal{N},\lambda}^{\tt RFM}$, where $\mathcal{B}_{\mathcal{N},\lambda}^{\tt RFM}$ and $\mathcal{V}_{\mathcal{N},\lambda}^{\tt RFM}$ are defined as 
\[
\begin{aligned}
    \mathcal{B}_{\mathcal{N},\lambda}^{\tt RFM} := \<\btheta_*, \bm{G}^\sT \bm{Z} (\bm{Z}^\sT \bm{Z} + \lambda\id)^{-2} \bm{Z}^\sT \bm{G}\btheta_* \>\,, \quad \mathcal{V}_{\mathcal{N},\lambda}^{\tt RFM} := \sigma^2\Tr\left(\bm{Z}^\sT \bm{Z}(\bm{Z}^\sT \bm{Z} + \lambda\id)^{-2}\right)\,.
\end{aligned}
\]
\end{lemma}

A main goal of this section is to prove that $\mathcal{B}^{\tt RFM}_{\mathcal{N},\lambda}$ and $\mathcal{V}^{\tt RFM}_{\mathcal{N},\lambda}$ admit the following deterministic equivalents, both  asymptotically (\cref{sec:linear_asym_rf}) and non-asymptotically (\cref{sec:linear_nonasym_rf}):
\begin{align}
    \sB_{\sN,\lambda}^{\tt RFM} :=&~ \frac{p\< \btheta_*, \bLambda ( \bLambda + \nu_2\id)^{-2} \btheta_* \>}{p - \Tr\left(\bLambda^2 (\bLambda + \nu_2\id)^{-2}\right)} + {\color{red}\frac{p\chi(\nu_2)}{n}} \notag \cdot \frac{\nu_2^2\left[ \< \btheta_*, (\bLambda + \nu_2\id)^{-2} \btheta_* \> + \chi(\nu_2) \< \btheta_*, \bLambda (\bLambda + \nu_2\id)^{-2} \btheta_* \> \right]}{1 - \Upsilon(\nu_1, \nu_2)}, \notag \\
    \sV_{\sN,\lambda}^{\tt RFM} :=&~ \sigma^2 \frac{{\color{blue}\frac{p}{n}\chi(\nu_2)}}{1-\Upsilon(\nu_1, \nu_2)}\,. \label{eq:equiv_random_feature}
\end{align}
When checking \cref{eq:de_risk_rf} and \cref{eq:equiv_random_feature}, we find that for variance, $\sB_{\sR,\lambda}^{\tt RFM}$ and $\sB_{\sN,\lambda}^{\tt RFM}$ only differ on the numerator, where $\Upsilon(\nu_1,\nu_2)$ is changed by $\frac{p}{n}\chi(\nu_2)$ ({\color{blue}in blue}).  For the bias term, we find that the second term of $\sB_{\sN,\lambda}^{\tt RFM}$ ({\color{red}in red}) rescales $\sB_{\sR,\lambda}^{\tt RFM}$ in \cref{eq:de_risk_rf} by a factor $\frac{p\chi(\nu_2)}{n}$.

Both of our asymptotic and non-asymptotic results are based on the following assumption on well-behaved data and features, but non-asymptotic results requires more technical assumptions we will deliver later.

\begin{assumption}[Concentration of the eigenfunctions \cite{defilippis2024dimension}]\label{ass:concentrated_RFRR} Recall the random vectors $\bpsi := (\xi_k \psi_k(\bx))_{k \geq 1}$ and $\bphi := (\xi_k \phi_k(\bw))_{k \geq 1}$. There exists $C_* > 0$ such that for any PSD matrix $\bA \in \mathbb{R}^{\infty \times \infty}$ with $\operatorname{Tr}(\bLambda \bA) < \infty$ and any $t\ge0$, we have
\[
\begin{aligned}
 & \mathbb{P} \left( \left| \bpsi^\sT \bA \bpsi - \Tr(\bLambda \bA) \right| \geq t \|\bLambda^{1/2} \bA \bLambda^{1/2}\|_F \right) 
\leq C_*  e^{-\frac{t}{C_*}}, \\
& \mathbb{P} \left( \left| \bphi^\sT \bA \bphi - \Tr(\bLambda \bA) \right| \geq t \|\bLambda^{1/2} \bA \bLambda^{1/2}\|_F \right) 
\leq C_*  e^{-\frac{t}{C_*}}.  
\end{aligned}
\]
\end{assumption}

This assumption requires well-behaved data---similarly to \cref{ass:concentrated_LR}---and additionally well-behaved random features. It holds for the classical sub-Gaussian case and log-Sobolev inequality or convex Lipschitz concentration \citep{cheng2022dimension}.

\vspace{-0.cm}
\subsection{Asymptotic deterministic equivalence}
\label{sec:linear_asym_rf}
\vspace{-0.cm}


Here we present the asymptotic results of $\E_{\varepsilon}\|\hat{\ba}\|_2^2$, see the proof in \cref{app:asy_deter_equiv_rf}.
\begin{proposition}[Asymptotic deterministic equivalence]\label{prop:asy_equiv_norm_RFRR}
    Given the bias-variance decomposition of $\E_{\varepsilon}\|\hat{\ba}\|_2^2$ in \cref{lemma:biasvariance_rf}, 
    under \cref{ass:concentrated_RFRR}, we have the following asymptotic deterministic equivalents $\mathcal{N}^{\tt RFM}_\lambda \sim \sN^{\tt RFM}_\lambda := \sB_{\sN,\lambda}^{\tt RFM} + \sV_{\sN,\lambda}^{\tt RFM}$ such that $\mathcal{B}^{\tt RFM}_{\mathcal{N},\lambda} \sim \sB_{\sN,\lambda}^{\tt RFM}$, $\mathcal{V}^{\tt RFM}_{\mathcal{N},\lambda} \sim \sV_{\sN,\lambda}^{\tt RFM}$, where $\sB_{\sN,\lambda}^{\tt RFM}$ and $\sV_{\sN,\lambda}^{\tt RFM}$ are defined by \cref{eq:equiv_random_feature}.
\end{proposition}
\cref{prop:asy_equiv_norm_RFRR} is numerically validated by \cref{fig:random_feature_risk_vs_norm}, supporting that $\E_{\varepsilon}\|\hat{\ba}\|_2^2$ is able to characterize the bias and variance of the excess risk.
We will quantitatively characterize this relationship in \cref{sec:relationship_rf}.

Similar to linear regression, we also need to analyze the under-/over-parameterized regimes separately for RFMs when $\lambda \rightarrow 0$.
In the under-parameterized regime, $\nu_1$ converges to $0$, and $\nu_2$ converges to a value $\lambda_p$ satisfying $\Tr(\bLambda(\bLambda + \lambda_p\id)^{-1}) = p$; while in the over-parameterized regime, $\nu_2$ converges to $\lambda_n$ satisfying $\Tr(\bLambda(\bLambda + \lambda_n\id)^{-1}) = n$, and $\nu_1$ converges to $\nu_2(1-\nicefrac{n}{p})$. 
We have the following result, see the proof in \cref{app:asy_deter_equiv_rf}. 


\begin{corollary}[Asymptotic deterministic equivalence of $\sN_{0}^{\tt RFM}$]\label{prop:asy_equiv_norm_RFRR_minnorm}
    Under \cref{ass:concentrated_RFRR}, for the min-$\ell_2$-norm estimator $\hat{\ba}_{\min}$, in the under-parameterized regime ($p<n$), we have
    \[
    \begin{aligned}
        \mathcal{B}^{\tt RFM}_{\mathcal{N},0} \sim \frac{p\<\btheta_*, \bLambda (\bLambda +\lambda_p\id)^{-2} \btheta_*\>}{n-\Tr(\bLambda^2(\bLambda +\lambda_p\id)^{-2})} + \frac{p\<\btheta_*, (\bLambda +\lambda_p\id)^{-1} \btheta_*\>}{n-p}\,, \quad 
        \mathcal{V}^{\tt RFM}_{\mathcal{N},0} \sim \frac{\sigma^2p}{\lambda_p(n-p)},
    \end{aligned}
    \]
    where $\lambda_p$ is from $\Tr(\bLambda(\bLambda+\lambda_p\id)^{-1}) \sim p$. In the over-parameterized regime ($p>n$), we have
    \[
    \begin{aligned}
        \mathcal{B}^{\tt RFM}_{\mathcal{N},0} \sim \frac{p\<\btheta_*, ( \bLambda + \lambda_n\id)^{-1} \btheta_*\>}{p-n}\,,
        \quad
        \mathcal{V}^{\tt RFM}_{\mathcal{N},0} \sim \frac{\sigma^2p}{\lambda_n(p-n)}\,,
    \end{aligned}
    \]
    where $\lambda_n$ is defined by $\Tr(\bLambda(\bLambda+\lambda_n\id)^{-1}) \sim n$.
\end{corollary}
\noindent{\bf Remark:} Notice that $\mathcal{V}^{\tt RFM}_{\mathcal{N},0}$ admits the similar formulation in under-/over-parameterized regimes but differs in $\lambda_n$ and $\lambda_p$. An interesting point to note is that, in the over-parameterized regime, $\lambda_n$ is a constant when $n$ constant. Therefore, $\mathcal{B}^{\tt RFM}_{\mathcal{N},0}$ and $\mathcal{V}^{\tt RFM}_{\mathcal{N},0}$ are proportional to each other.

\vspace{-0.cm}
\subsection{Non-asymptotic deterministic equivalence}
\label{sec:linear_nonasym_rf}
\vspace{-0.cm}

To present our non-asymptotic results, we additionally consider the following classical power-law assumption.

\begin{assumption}[Power-law, \citealt{defilippis2024dimension}]
\label{ass:powerlaw_rf}
    We assume that $\{ \xi_k^2\}_{k=1}^{\infty}$ in $\bLambda$ and $\btheta_*$ satisfy
    \[
    \xi_k^2 = k^{-\alpha}, \quad \theta_{\ast, k} = k^{-\frac{1 + 2\alpha\tau}{2}}\,, \mbox{with}~\alpha > 1,~ r>0\,.
    \]
\end{assumption}

The assumption coincides with the source condition $\|\bLambda^{-r} \btheta_*\|_2 < \infty$ ($r>0$) and capacity condition $\Tr(\bLambda^{1/\alpha}) < \infty$ ($\alpha > 1$) \citep{caponnetto2007optimal}.

We have the following non-asymptotic result on variance. 
\begin{theorem}[Non-asymptotic deterministic equivalents for variance, simplified version of \cref{prop:det_equiv_RFRR_V}]\label{prop:non-asy_equiv_norm_RFRR_V}
    Under \cref{ass:concentrated_RFRR} and \ref{ass:powerlaw_rf}, for any $D,K >0$, if $\lambda > n^{-K}$, then with probability at least $1-n^{-D}-p^{-D}$, we have
    \[
    \begin{aligned}
          \left|\mathcal{V}^{\tt RFM}_{\mathcal{N},\lambda} - \sV_{\sN,\lambda}^{\tt RFM}\right| \leq \widetilde{\mathcal{O}}(n^{-\nicefrac{1}{2}}+p^{-\nicefrac{1}{2}}) \cdot \sV_{\sN,\lambda}^{\tt RFM}\,.
    \end{aligned}
    \]
\end{theorem}
\noindent{\bf Remark:} Our results remain valid under weaker assumptions related to \emph{effective dimension} used in \citet{defilippis2024dimension}; see  \cref{app:nonasy_deter_equiv_rf}. 
We expect similar results to hold for bias as well, but additional technical assumptions and calculations may be needed; see more discussion in \cref{app:discuss_bias}.
We leave this to future work.  

\subsection{Relationship and scaling law}\label{sec:relationship_rf}

Here we discuss the risk-norm relationship for min-$\ell_2$-norm interpolator, and then build the scaling law under certain settings; see the proof in \cref{app:relationship_rf}.
\begin{proposition}[Relationship for min-$\ell_2$-norm interpolator in the {\bf over-parameterized} regime]\label{prop:relation_minnorm_overparam}
The deterministic equivalents $\sR^{\tt RFM}_{0}$ and $\sN^{\tt RFM}_{0}$, in over-parameterized regimes ($p>n$) admit the linear relationship due to $\lambda_n$ as a constant
% \begin{equation}\label{eq:rfflam0}
%   \sR_{0}^{\tt RFM} = \lambda_n\sN_{0}^{\tt RFM} + C_{n,\bLambda,\btheta_*,\sigma}\,,  
% \end{equation}
\begin{equation}\label{eq:rfflam0}
    \sR_{0}^{\tt RFM} 
    = 
    \lambda_n\sN_{0}^{\tt RFM} 
    - 
    \left[\lambda_n\<\btheta_*, ( \bLambda + \lambda_n\id)^{-1} \btheta_*\> + \sigma^2\right] 
    + 
    \frac{n\lambda_n^2 \<\btheta_*, ( \bLambda + \lambda_n \id)^{-2} \btheta_*\> + \sigma^2\Tr(\bLambda^2(\bLambda+\lambda_n\id)^{-2})}{ n - \Tr(\bLambda^2(\bLambda+\lambda_n\id)^{-2})}\,. 
\end{equation}
%where $C_{n,\bLambda,\btheta_*,\sigma}$ is a constant independent of $p$ but depending on $n$, $\bLambda$, $\btheta_*$ and $\sigma$, given in \cref{app:relationship_rf}.
\end{proposition}

The relationship in the under-parameterized regime is more complicated. We present it in the special case of isotropic features in \cref{prop:relation_minnorm_id_rf} and give an approximation in \cref{prop:relation_minnorm_powerlaw_rf} under the power-law assumption.

\begin{corollary}[Isotropic features for min-$\ell_2$-norm interpolator]\label{prop:relation_minnorm_id_rf}
    Consider covariance matrix $\bLambda = \id_m$ ($n<m<\infty$), in the over-parameterized regime ($p>n$), the deterministic equivalents $\sR^{\tt RFM}_0$ and $\sN^{\tt RFM}_0$ specifies the linear relationship in \cref{eq:rfflam0} as $\sR_{0}^{\tt RFM} = \frac{m-n}{n} \sN_{0}^{\tt RFM} +\frac{2n-m}{m-n} \sigma^2$.\\
While in the under-parameterized regime ($p<n$), we focus on bias and variance separately
    \[
    \begin{aligned}
     \mbox{Variance:}~ \left(\sV^{\tt RFM}_{\sR,0} \right)^2 = \frac{m-n}{n} \sV^{\tt RFM}_{\sR,0} \sV^{\tt RFM}_{\sN,0} + \frac{m \sigma^2}{n} \sV^{\tt RFM}_{\sN,0}\,,
    \end{aligned}
    \]
    \[
    \begin{aligned}
      \mbox{Bias:}~  &~(m-n)\sB^{\tt RFM}_{\sN,0}(m\sB^{\tt RFM}_{\sR,0}-n\|\btheta_*\|_2^2)(m(\sB^{\tt RFM}_{\sR,0})^2-n\|\btheta_*\|_2^4)\\
        &= nm(\sB^{\tt RFM}_{\sR,0} -\|\btheta_*\|_2^2)^2[m(\sB^{\tt RFM}_{\sR,0})^2 + n\|\btheta_*\|_2^2\sB^{\tt RFM}_{\sR,0} - 2n\|\btheta_*\|_2^4].
    \end{aligned}
    \]
\end{corollary}

\noindent{\bf Remark:} 
In the under-parameterized regime, $\sV^{\tt RFM}_{\sR,0}$ and $\sV^{\tt RFM}_{\sN,0}$ are related by a hyperbola, the asymptote of which is $\sV^{\tt RFM}_{\sR,0} = \frac{m-n}{n}\sV^{\tt RFM}_{\sN,0} + \frac{m}{m-n} \sigma^2$. Further, for $p \to n$, we have $\sB^{\tt RFM}_{\sR,0} \approx \frac{m-n}{n}\sB^{\tt RFM}_{\sR,0} + \frac{2(m-n)}{m}\|\btheta_*\|_2^2$, see discussion in \cref{app:relationship_rf}.


\begin{figure}[t]
    \centering
    \subfigure[$\alpha = 2.5$, $r=0.2$]{
        \includegraphics[width=0.3\textwidth]{arxiv_version/figures/Main_results_on_random_feature_regression/risk_vs_norm_ridgeless_alpha2.5_r0.2_n500.pdf}
    }
    \subfigure[$\alpha = 1.5$, $r=0.8$]{
        \includegraphics[width=0.3\textwidth]{arxiv_version/figures/Main_results_on_random_feature_regression/risk_vs_norm_ridgeless_alpha1.5_r0.8_n500.pdf}
    }
    \caption{Validation of \cref{prop:relation_minnorm_powerlaw_rf}. 
    The solid line represents the result of the deterministic equivalents, well approximated by the {\color{red}red dashed line} of \cref{eq:RORFMover} in the over-parameterized regime, and the {\color{blue}blue dashed line} of \cref{eq:RORFMunder} when $p \to n$ in the under-parameterized regime.}
    \label{fig:random_feature_risk_vs_norm_approx}\vspace{-0.05cm}
\end{figure}



Under power-law, we need to handle the self-consistent equations to approximate the infinite summation. We have the following approximation.
\begin{corollary}[Relationship for min-$\ell_2$ norm interpolator under power law assumption]\label{prop:relation_minnorm_powerlaw_rf}
    Under \cref{ass:powerlaw_rf}, The deterministic equivalents $\sR^{\tt RFM}_{0}$ and $\sN^{\tt RFM}_{0}$, in over-parameterized regimes ($p>n$) admit \footnote{The symbol $\approx$ here denotes using an integral to approximate an infinite sum when calculating $\Tr(\cdot)$.}
    \begin{equation}\label{eq:RORFMover}
            \sR_0^{\tt RFM} \approx \left(\nicefrac{n}{C_\alpha}\right)^{-\alpha} \sN_0^{\tt RFM} + C_{n,\alpha,r,1}\,,  
    \end{equation}
    while in the under-parameterized regime ($p<n$), we have
    \begin{align}\label{eq:RORFMunder}
        \sR_0^{\tt RFM} \approx \left(\nicefrac{n}{C_{\alpha}}\right)^{-\alpha}\sN_0^{\tt RFM} + C_{n,\alpha,r,2}\,, \quad \mbox{when}~p \to n \,,
    \end{align}
where \( C_{n,\alpha,r,1 (2)} \) are constants (see \cref{app:relationship_rf} for details) that only depend on $n$, $\alpha$ and $r$, and it admits that $C_{n,\alpha,r,1} <  C_{n,\alpha,r,2}$.
\end{corollary}


\noindent{\bf Remark:}
In the over-parameterized regime, the relationship between \(\sR_0^{\tt RFM}\) and \(\sN_0^{\tt RFM}\) is a monotonically increasing linear function, with a growth rate controlled by the factor decaying with $n$.
In the under-parameterized regime, as \(p \to n\) (which also leads to \(\sR_0^{\tt RFM}\) and \(\sN_0^{\tt RFM} \to \infty\)), \(\sR_0^{\tt RFM}\) still grows linearly w.r.t \(\sN_0^{\tt RFM}\), with the same growth rate factor decaying with $n$. Furthermore, since \(C_{n,\alpha,r,1} < C_{n,\alpha,r,2}\), the test risk curve shows that over-parameterization is better than under-parameterization.
This approximation is also empirically verified to be precise in \cref{fig:random_feature_risk_vs_norm_approx}.


\begin{figure}[t]
    \centering
    \includegraphics[width=0.5\textwidth]{arxiv_version/figures/Scaling_Law/scaling_law_norm_based_capacity.pdf} 
    \caption{The value of exponents $\gamma_n$ and $\gamma_\sN$ in different regions (divided by $q$ and $\ell$) for $r \in (0, \frac{1}{2})$. Variance dominated region is colored by {\color{regionorange}orange}, {\color{regionyellow}yellow} and {\color{regionbrown}brown}, bias dominated region is colored by {\color{regionblue}blue} and {\color{regiongreen}green}.} 
    \label{fig:scaling_law_norm_based_capacity} \vspace{-0.35cm}
\end{figure}


To study scaling law, we follow the same setting of \citet{defilippis2024dimension} by choosing $p = n^q$ and $\lambda = n^{-(\ell-1)}$ with $q,l \geq 0$. We have the scaling law as below; see the proof in \cref{app:scaling_law}.
\begin{proposition}\label{prop:scaling_law_norm_based_capacity}
Under \cref{ass:powerlaw_rf}, for $r \in (0, \frac{1}{2})$, taking $p = n^q$ and $\lambda = n^{-(\ell-1)}$ with $q,l \geq 0$, we formulate the scaling law under norm-based capacity in different areas is 
\begin{equation*}
    \sR_\lambda^{\tt RFM} = \Theta\left(n^{\gamma_n} \cdot \left(\sN_\lambda^{\tt RFM}\right)^{\gamma_{\sN}}\right)\,, 
\end{equation*}    
where the rate $\{ \gamma_n, \gamma_{\sN} \}$ in different areas is given in \cref{fig:scaling_law_norm_based_capacity}.
\end{proposition}

\noindent{\bf Remark:}
In regions \ding{172}, \ding{173}, \ding{174}, and \ding{175} of \cref{fig:scaling_law_norm_based_capacity}, the exponent of \(\sN_\lambda^{\tt RFM}\) is positive, i.e., $\gamma_{\sN} >0$, and  \(\sR_\lambda^{\tt RFM}\) increases monotonically with  \(\sN_\lambda^{\tt RFM}\). However, in region \ding{176}, we have $\gamma_{\sN} <0$, and \(\sR_\lambda^{\tt RFM}\) decreases monotonically with \(\sN_\lambda^{\tt RFM}\).\\

\section{ Task Generalization Beyond i.i.d. Sampling and Parity Functions
}\label{sec:Discussion}
% Discussion: From Theory to Beyond
% \misha{what is beyond?}
% \amir{we mean two things: in the first subsection beyond i.i.d subsampling of parity tasks and in the second subsection beyond parity task}
% \misha{it has to be beyond something, otherwise it is not clear what it is about} \hz{this is suggested by GPT..., maybe can be interpreted as from theory to beyond theory. We can do explicit like Discussion: Beyond i.i.d. task sampling and the Parity Task}
% \misha{ why is "discussion" in the title?}\amir{Because it is a discussion, it is not like separate concrete explnation about why these thing happens or when they happen, they just discuss some interesting scenraios how it relates to our theory.   } \misha{it is not really a discussion -- there is a bunch of experiments}

In this section, we extend our experiments beyond i.i.d. task sampling and parity functions. We show an adversarial example where biased task selection substantially hinders task generalization for sparse parity problem. In addition, we demonstrate that exponential task scaling extends to a non-parity tasks including arithmetic and multi-step language translation.

% In this section, we extend our experiments beyond i.i.d. task sampling and parity functions. On the one hand, we find that biased task selection can significantly degrade task generalization; on the other hand, we show that exponential task scaling generalizes to broader scenarios.
% \misha{we should add a sentence or two giving more detail}


% 1. beyond i.i.d tasks sampling
% 2. beyond parity -> language, arithmetic -> task dependency + implicit bias of transformer (cannot implement this algorithm for arithmatic)



% In this section, we emphasize the challenge of quantifying the level of out-of-distribution (OOD) differences between training tasks and testing tasks, even for a simple parity task. To illustrate this, we present two scenarios where tasks differ between training and testing. For each scenario, we invite the reader to assess, before examining the experimental results, which cases might appear “more” OOD. All scenarios consider \( d = 10 \). \kaiyue{this sentence should be put into 5.1}






% for parity problem




% \begin{table*}[th!]
%     \centering
%     \caption{Generalization Results for Scenarios 1 and 2 for $d=10$.}
%     \begin{tabular}{|c|c|c|c|}
%         \hline
%         \textbf{Scenario} & \textbf{Type/Variation} & \textbf{Coordinates} & \textbf{Generalization accuracy} \\
%         \hline
%         \multirow{3}{*}{Generalization with Missing Pair} & Type 1 & \( c_1 = 4, c_2 = 6 \) & 47.8\%\\ 
%         & Type 2 & \( c_1 = 4, c_2 = 6 \) & 96.1\%\\ 
%         & Type 3 & \( c_1 = 4, c_2 = 6 \) & 99.5\%\\ 
%         \hline
%         \multirow{3}{*}{Generalization with Missing Pair} & Type 1 &  \( c_1 = 8, c_2 = 9 \) & 40.4\%\\ 
%         & Type 2 & \( c_1 = 8, c_2 = 9 \) & 84.6\% \\ 
%         & Type 3 & \( c_1 = 8, c_2 = 9 \) & 99.1\%\\ 
%         \hline
%         \multirow{1}{*}{Generalization with Missing Coordinate} & --- & \( c_1 = 5 \) & 45.6\% \\ 
%         \hline
%     \end{tabular}
%     \label{tab:generalization_results}
% \end{table*}

\subsection{Task Generalization Beyond i.i.d. Task Sampling }\label{sec: Experiment beyond iid sampling}

% \begin{table*}[ht!]
%     \centering
%     \caption{Generalization Results for Scenarios 1 and 2 for $d=10, k=3$.}
%     \begin{tabular}{|c|c|c|}
%         \hline
%         \textbf{Scenario}  & \textbf{Tasks excluded from training} & \textbf{Generalization accuracy} \\
%         \hline
%         \multirow{1}{*}{Generalization with Missing Pair}
%         & $\{4,6\} \subseteq \{s_1, s_2, s_3\}$ & 96.2\%\\ 
%         \hline
%         \multirow{1}{*}{Generalization with Missing Coordinate}
%         & \( s_2 = 5 \) & 45.6\% \\ 
%         \hline
%     \end{tabular}
%     \label{tab:generalization_results}
% \end{table*}




In previous sections, we focused on \textit{i.i.d. settings}, where the set of training tasks $\mathcal{F}_{train}$ were sampled uniformly at random from the entire class $\mathcal{F}$. Here, we explore scenarios that deliberately break this uniformity to examine the effect of task selection on out-of-distribution (OOD) generalization.\\

\textit{How does the selection of training tasks influence a model’s ability to generalize to unseen tasks? Can we predict which setups are more prone to failure?}\\

\noindent To investigate this, we consider two cases parity problems with \( d = 10 \) and \( k = 3 \), where each task is represented by its tuple of secret indices \( (s_1, s_2, s_3) \):

\begin{enumerate}[leftmargin=0.4 cm]
    \item \textbf{Generalization with a Missing Coordinate.} In this setup, we exclude all training tasks where the second coordinate takes the value \( s_2 = 5 \), such as \( (1,5,7) \). At test time, we evaluate whether the model can generalize to unseen tasks where \( s_2 = 5 \) appears.
    \item \textbf{Generalization with Missing Pair.} Here, we remove all training tasks that contain both \( 4 \) \textit{and} \( 6 \) in the tuple \( (s_1, s_2, s_3) \), such as \( (2,4,6) \) and \( (4,5,6) \). At test time, we assess whether the model can generalize to tasks where both \( 4 \) and \( 6 \) appear together.
\end{enumerate}

% \textbf{Before proceeding, consider the following question:} 
\noindent \textbf{If you had to guess.} Which scenario is more challenging for generalization to unseen tasks? We provide the experimental result in Table~\ref{tab:generalization_results}.

 % while the model struggles for one of them while as it generalizes almost perfectly in the other one. 

% in the first scenario, it generalizes almost perfectly in the second. This highlights how exposure to partial task structures can enhance generalization, even when certain combinations are entirely absent from the training set. 

In the first scenario, despite being trained on all tasks except those where \( s_2 = 5 \), which is of size $O(\d^T)$, the model struggles to generalize to these excluded cases, with prediction at chance level. This is intriguing as one may expect model to generalize across position. The failure  suggests that positional diversity plays a crucial role in the task generalization of Transformers. 

In contrast, in the second scenario, though the model has never seen tasks with both \( 4 \) \textit{and} \( 6 \) together, it has encountered individual instances where \( 4 \) appears in the second position (e.g., \( (1,4,5) \)) or where \( 6 \) appears in the third position (e.g., \( (2,3,6) \)). This exposure appears to facilitate generalization to test cases where both \( 4 \) \textit{and} \( 6 \) are present. 



\begin{table*}[t!]
    \centering
    \caption{Generalization Results for Scenarios 1 and 2 for $d=10, k=3$.}
    \resizebox{\textwidth}{!}{  % Scale to full width
        \begin{tabular}{|c|c|c|}
            \hline
            \textbf{Scenario}  & \textbf{Tasks excluded from training} & \textbf{Generalization accuracy} \\
            \hline
            Generalization with Missing Pair & $\{4,6\} \subseteq \{s_1, s_2, s_3\}$ & 96.2\%\\ 
            \hline
            Generalization with Missing Coordinate & \( s_2 = 5 \) & 45.6\% \\ 
            \hline
        \end{tabular}
    }
    \label{tab:generalization_results}
\end{table*}

As a result, when the training tasks are not i.i.d, an adversarial selection such as exclusion of specific positional configurations may lead to failure to unseen task generalization even though the size of $\mathcal{F}_{train}$ is exponentially large. 


% \paragraph{\textbf{Key Takeaways}}
% \begin{itemize}
%     \item Out-of-distribution generalization in the parity problem is highly sensitive to the diversity and positional coverage of training tasks.
%     \item Adversarial exclusion of specific pairs or positional configurations can lead to systematic failures, even when most tasks are observed during training.
% \end{itemize}




%################ previous veriosn down
% \textit{How does the choice of training tasks affect the ability of a model to generalize to unseen tasks? Can we predict which setups are likely to lead to failure?}

% To explore these questions, we crafted specific training and test task splits to investigate what makes one setup appear “more” OOD than another.

% \paragraph{Generalization with Missing Pair.}

% Imagine we have tasks constructed from subsets of \(k=3\) elements out of a larger set of \(d\) coordinates. What happens if certain pairs of coordinates are adversarially excluded during training? For example, suppose \(d=5\) and two specific coordinates, \(c_1 = 1\) and \(c_2 = 2\), are excluded. The remaining tasks are formed from subsets of the other coordinates. How would a model perform when tested on tasks involving the excluded pair \( (c_1, c_2) \)? 

% To probe this, we devised three variations in how training tasks are constructed:
%     \begin{enumerate}
%         \item \textbf{Type 1:} The training set includes all tasks except those containing both \( c_1 = 1 \) and \( c_2 = 2 \). 
%         For this example, the training set includes only $\{(3,4,5)\}$. The test set consists of all tasks containing the rest of tuples.

%         \item \textbf{Type 2:} Similar to Type 1, but the training set additionally includes half of the tasks containing either \( c_1 = 1 \) \textit{or} \( c_2 = 2 \) (but not both). 
%         For the example, the training set includes all tasks from Type 1 and adds tasks like \(\{(1, 3, 4), (2, 3, 5)\}\) (half of those containing \( c_1 = 1 \) or \( c_2 = 2 \)).

%         \item \textbf{Type 3:} Similar to Type 2, but the training set also includes half of the tasks containing both \( c_1 = 1 \) \textit{and} \( c_2 = 2 \). 
%         For the example, the training set includes all tasks from Type 2 and adds, for instance, \(\{(1, 2, 5)\}\) (half of the tasks containing both \( c_1 \) and \( c_2 \)).
%     \end{enumerate}

% By systematically increasing the diversity of training tasks in a controlled way, while ensuring no overlap between training and test configurations, we observe an improvement in OOD generalization. 

% % \textit{However, the question is this improvement similar across all coordinate pairs, or does it depend on the specific choices of \(c_1\) and \(c_2\) in the tasks?} 

% \textbf{Before proceeding, consider the following question:} Is the observed improvement consistent across all coordinate pairs, or does it depend on the specific choices of \(c_1\) and \(c_2\) in the tasks? 

% For instance, consider two cases for \(d = 10, k = 3\): (i) \(c_1 = 4, c_2 = 6\) and (ii) \(c_1 = 8, c_2 = 9\). Would you expect similar OOD generalization behavior for these two cases across the three training setups we discussed?



% \paragraph{Answer to the Question.} for both cases of \( c_1, c_2 \), we observe that generalization fails in Type 1, suggesting that the position of the tasks the model has been trained on significantly impacts its generalization capability. For Type 2, we find that \( c_1 = 4, c_2 = 6 \) performs significantly better than \( c_1 = 8, c_2 = 9 \). 

% Upon examining the tasks where the transformer fails for \( c_1 = 8, c_2 = 9 \), we see that the model only fails at tasks of the form \((*, 8, 9)\) while perfectly generalizing to the rest. This indicates that the model has never encountered the value \( 8 \) in the second position during training, which likely explains its failure to generalize. In contrast, for \( c_1 = 4, c_2 = 6 \), while the model has not seen tasks of the form \((*, 4, 6)\), it has encountered tasks where \( 4 \) appears in the second position, such as \((1, 4, 5)\), and tasks where \( 6 \) appears in the third position, such as \((2, 3, 6)\). This difference may explain why the model generalizes almost perfectly in Type 2 for \( c_1 = 4, c_2 = 6 \), but not for \( c_1 = 8, c_2 = 9 \).



% \paragraph{Generalization with Missing Coordinates.}
% Next, we investigate whether a model can generalize to tasks where a specific coordinate appears in an unseen position during training. For instance, consider \( c_1 = 5 \), and exclude all tasks where \( c_1 \) appears in the second position. Despite being trained on all other tasks, the model fails to generalize to these excluded cases, highlighting the importance of positional diversity in training tasks.



% \paragraph{Key Takeaways.}
% \begin{itemize}
%     \item OOD generalization depends heavily on the diversity and positional coverage of training tasks for the parity problem.
%     \item adversarial exclusion of specific pairs or positional configurations in the parity problem can lead to failure, even when the majority of tasks are observed during training.
% \end{itemize}


%################ previous veriosn up

% \paragraph{Key Takeaways} These findings highlight the complexity of OOD generalization, even in seemingly simple tasks like parity. They also underscore the importance of task design: the diversity of training tasks can significantly influence a model’s ability to generalize to unseen tasks. By better understanding these dynamics, we can design more robust training regimes that foster generalization across a wider range of scenarios.


% #############


% Upon examining the tasks where the transformer fails for \( c_1 = 8, c_2 = 9 \), we see that the model only fails at tasks of the form \((*, 8, 9)\) while perfectly generalizing to the rest. This indicates that the model has never encountered the value \( 8 \) in the second position during training, which likely explains its failure to generalize. In contrast, for \( c_1 = 4, c_2 = 6 \), while the model has not seen tasks of the form \((*, 4, 6)\), it has encountered tasks where \( 4 \) appears in the second position, such as \((1, 4, 5)\), and tasks where \( 6 \) appears in the third position, such as \((2, 3, 6)\). This difference may explain why the model generalizes almost perfectly in Type 2 for \( c_1 = 4, c_2 = 6 \), but not for \( c_1 = 8, c_2 = 9 \).

% we observe a striking pattern: generalization fails entirely in Type 1, regardless of the coordinate pair (\(c_1, c_2\)). However, in Type 2, generalization varies: \(c_1 = 4, c_2 = 6\) achieves 96\% accuracy, while \(c_1 = 8, c_2 = 9\) lags behind at 70\%. Why? Upon closer inspection, the model struggles specifically with tasks like \((*, 8, 9)\), where the combination \(c_1 = 8\) and \(c_2 = 9\) is entirely novel. In contrast, for \(c_1 = 4, c_2 = 6\), the model benefits from having seen tasks where \(4\) appears in the second position or \(6\) in the third. This suggests that positional exposure during training plays a key role in generalization.

% To test whether task structure influences generalization, we consider two variations:
% \begin{enumerate}
%     \item \textbf{Sorted Tuples:} Tasks are always sorted in ascending order.
%     \item \textbf{Unsorted Tuples:} Tasks can appear in any order.
% \end{enumerate}

% If the model struggles with generalizing to the excluded position, does introducing variability through unsorted tuples help mitigate this limitation?

% \paragraph{Discussion of Results}

% In \textbf{Generalization with Missing Pairs}, we observe a striking pattern: generalization fails entirely in Type 1, regardless of the coordinate pair (\(c_1, c_2\)). However, in Type 2, generalization varies: \(c_1 = 4, c_2 = 6\) achieves 96\% accuracy, while \(c_1 = 8, c_2 = 9\) lags behind at 70\%. Why? Upon closer inspection, the model struggles specifically with tasks like \((*, 8, 9)\), where the combination \(c_1 = 8\) and \(c_2 = 9\) is entirely novel. In contrast, for \(c_1 = 4, c_2 = 6\), the model benefits from having seen tasks where \(4\) appears in the second position or \(6\) in the third. This suggests that positional exposure during training plays a key role in generalization.

% In \textbf{Generalization with Missing Coordinates}, the results confirm this hypothesis. When \(c_1 = 5\) is excluded from the second position during training, the model fails to generalize to such tasks in the sorted case. However, allowing unsorted tuples introduces positional diversity, leading to near-perfect generalization. This raises an intriguing question: does the model inherently overfit to positional patterns, and can task variability help break this tendency?




% In this subsection, we show that the selection of training tasks can affect the quality of the unseen task generalization significantly in practice. To illustrate this, we present two scenarios where tasks differ between training and testing. For each scenario, we invite the reader to assess, before examining the experimental results, which cases might appear “more” OOD. 

% % \amir{add examples, }

% \kaiyue{I think the name of scenarios here are not very clear}
% \begin{itemize}
%     \item \textbf{Scenario 1:  Generalization Across Excluded Coordinate Pairs (\( k = 3 \))} \\
%     In this scenario, we select two coordinates \( c_1 \) and \( c_2 \) out of \( d \) and construct three types of training sets. 

%     Suppose \( d = 5 \), \( c_1 = 1 \), and \( c_2 = 2 \). The tuples are all possible subsets of \( \{1, 2, 3, 4, 5\} \) with \( k = 3 \):
%     \[
%     \begin{aligned}
%     \big\{ & (1, 2, 3), (1, 2, 4), (1, 2, 5), (1, 3, 4), (1, 3, 5), \\
%            & (1, 4, 5), (2, 3, 4), (2, 3, 5), (2, 4, 5), (3, 4, 5) \big\}.
%     \end{aligned}
%     \]

%     \begin{enumerate}
%         \item \textbf{Type 1:} The training set includes all tuples except those containing both \( c_1 = 1 \) and \( c_2 = 2 \). 
%         For this example, the training set includes only $\{(3,4,5)\}$ tuple. The test set consists of tuples containing the rest of tuples.

%         \item \textbf{Type 2:} Similar to Type 1, but the training set additionally includes half of the tuples containing either \( c_1 = 1 \) \textit{or} \( c_2 = 2 \) (but not both). 
%         For the example, the training set includes all tuples from Type 1 and adds tuples like \(\{(1, 3, 4), (2, 3, 5)\}\) (half of those containing \( c_1 = 1 \) or \( c_2 = 2 \)).

%         \item \textbf{Type 3:} Similar to Type 2, but the training set also includes half of the tuples containing both \( c_1 = 1 \) \textit{and} \( c_2 = 2 \). 
%         For the example, the training set includes all tuples from Type 2 and adds, for instance, \(\{(1, 2, 5)\}\) (half of the tuples containing both \( c_1 \) and \( c_2 \)).
%     \end{enumerate}

% % \begin{itemize}
% %     \item \textbf{Type 1:} The training set includes tuples \(\{1, 3, 4\}, \{2, 3, 4\}\) (excluding tuples with both \( c_1 \) and \( c_2 \): \(\{1, 2, 3\}, \{1, 2, 4\}\)). The test set contains the excluded tuples.
% %     \item \textbf{Type 2:} The training set includes all tuples in Type 1 plus half of the tuples containing either \( c_1 = 1 \) or \( c_2 = 2 \) (e.g., \(\{1, 2, 3\}\)).
% %     \item \textbf{Type 3:} The training set includes all tuples in Type 2 plus half of the tuples containing both \( c_1 = 1 \) and \( c_2 = 2 \) (e.g., \(\{1, 2, 4\}\)).
% % \end{itemize}
    
%     \item \textbf{Scenario 2: Scenario 2: Generalization Across a Fixed Coordinate (\( k = 3 \))} \\
%     In this scenario, we select one coordinate \( c_1 \) out of \( d \) (\( c_1 = 5 \)). The training set includes all task tuples except those where \( c_1 \) is the second coordinate of the tuple. For this scenario, we examine two variations:
%     \begin{enumerate}
%         \item \textbf{Sorted Tuples:} Task tuples are always sorted (e.g., \( (x_1, x_2, x_3) \) with \( x_1 \leq x_2 \leq x_3 \)).
%         \item \textbf{Unsorted Tuples:} Task tuples can appear in any order.
%     \end{enumerate}
% \end{itemize}




% \paragraph{Discussion of Results.} In the first scenario, for both cases of \( c_1, c_2 \), we observe that generalization fails in Type 1, suggesting that the position of the tasks the model has been trained on significantly impacts its generalization capability. For Type 2, we find that \( c_1 = 4, c_2 = 6 \) performs significantly better than \( c_1 = 8, c_2 = 9 \). 

% Upon examining the tasks where the transformer fails for \( c_1 = 8, c_2 = 9 \), we see that the model only fails at tasks of the form \((*, 8, 9)\) while perfectly generalizing to the rest. This indicates that the model has never encountered the value \( 8 \) in the second position during training, which likely explains its failure to generalize. In contrast, for \( c_1 = 4, c_2 = 6 \), while the model has not seen tasks of the form \((*, 4, 6)\), it has encountered tasks where \( 4 \) appears in the second position, such as \((1, 4, 5)\), and tasks where \( 6 \) appears in the third position, such as \((2, 3, 6)\). This difference may explain why the model generalizes almost perfectly in Type 2 for \( c_1 = 4, c_2 = 6 \), but not for \( c_1 = 8, c_2 = 9 \).

% This position-based explanation appears compelling, so in the second scenario, we focus on a single position to investigate further. Here, we find that the transformer fails to generalize to tasks where \( 5 \) appears in the second position, provided it has never seen any such tasks during training. However, when we allow for more task diversity in the unsorted case, the model achieves near-perfect generalization. 

% This raises an important question: does the transformer have a tendency to overfit to positional patterns, and does introducing more task variability, as in the unsorted case, prevent this overfitting and enable generalization to unseen positional configurations?

% These findings highlight that even in a simple task like parity, it is remarkably challenging to understand and quantify the sources and levels of OOD behavior. This motivates further investigation into the nuances of task design and its impact on model generalization.


\subsection{Task Generalization Beyond Parity Problems}

% \begin{figure}[t!]
%     \centering
%     \includegraphics[width=0.45\textwidth]{Figures/arithmetic_v1.png}
%     \vspace{-0.3cm}
%     \caption{Task generalization for arithmetic task with CoT, it has $\d =2$ and $T = d-1$ as the ambient dimension, hence $D\ln(DT) = 2\ln(2T)$. We show that the empirical scaling closely follows the theoretical scaling.}
%     \label{fig:arithmetic}
% \end{figure}



% \begin{wrapfigure}{r}{0.4\textwidth}  % 'r' for right, 'l' for left
%     \centering
%     \includegraphics[width=0.4\textwidth]{Figures/arithmetic_v1.png}
%     \vspace{-0.3cm}
%     \caption{Task generalization for the arithmetic task with CoT. It has $d =2$ and $T = d-1$ as the ambient dimension, hence $D\ln(DT) = 2\ln(2T)$. We show that the empirical scaling closely follows the theoretical scaling.}
%     \label{fig:arithmetic}
% \end{wrapfigure}

\subsubsection{Arithmetic Task}\label{subsec:arithmetic}











We introduce the family of \textit{Arithmetic} task that, like the sparse parity problem, operates on 
\( d \) binary inputs \( b_1, b_2, \dots, b_d \). The task involves computing a structured arithmetic expression over these inputs using a sequence of addition and multiplication operations.
\newcommand{\op}{\textrm{op}}

Formally, we define the function:
\[
\text{Arithmetic}_{S} \colon \{0,1\}^d \to \{0,1,\dots,d\},
\]
where \( S = (\op_1, \op_2, \dots, \op_{d-1}) \) is a sequence of \( d-1 \) operations, each \( \op_k \) chosen from \( \{+, \times\} \). The function evaluates the expression by applying the operations sequentially from left-to-right order: for example, if \( S = (+, \times, +) \), then the arithmetic function would compute:
\[
\text{Arithmetic}_{S}(b_1, b_2, b_3, b_4) = ((b_1 + b_2) \times b_3) + b_4.
\]
% Thus, the sequence of operations \( S \) defines how the binary inputs are combined to produce an integer output between \( 0 \) and \( d \).
% \[
% \text{Arithmetic}_{S} 
% (b_1,\,b_2,\,\dots,b_d)
% =
% \Bigl(\dots\bigl(\,(b_1 \;\op_1\; b_2)\;\op_2\; b_3\bigr)\,\dots\Bigr) 
% \;\op_{d-1}\; b_d.
% \]
% We now introduce an \emph{Arithmetic} task that, like the sparse parity problem, operates on $d$ binary inputs $b_1, b_2, \dots, b_d$. Specifically, we define an arithmetic function
% \[
% \text{Arithmetic}_{S}\colon \{0,1\}^d \;\to\; \{0,1,\dots,d\},
% \]
% where $S = (i_1, i_2, \dots, i_{d-1})$ is a sequence of $d-1$ operations, each $i_k \in \{+,\,\times\}$. The value of $\text{Arithmetic}_{S}$ is obtained by applying the prescribed addition and multiplication operations in order, namely:
% \[
% \text{Arithmetic}_{S}(b_1,\,b_2,\,\dots,b_d)
% \;=\;
% \Bigl(\dots\bigl(\,(b_1 \;i_1\; b_2)\;i_2\; b_3\bigr)\,\dots\Bigr) 
% \;i_{d-1}\; b_d.
% \]

% This is an example of our framework where $T = d-1$ and $|\Theta_t| = 2$ with total $2^d$ possible tasks. 




By introducing a step-by-step CoT, arithmetic class belongs to $ARC(2, d-1)$: this is because at every step, there is only $\d = |\Theta_t| = 2$ choices (either $+$ or $\times$) while the length is  $T = d-1$, resulting a total number of $2^{d-1}$ tasks. 


\begin{minipage}{0.5\textwidth}  % Left: Text
    Task generalization for the arithmetic task with CoT. It has $d =2$ and $T = d-1$ as the ambient dimension, hence $D\ln(DT) = 2\ln(2T)$. We show that the empirical scaling closely follows the theoretical scaling.
\end{minipage}
\hfill
\begin{minipage}{0.4\textwidth}  % Right: Image
    \centering
    \includegraphics[width=\textwidth]{Figures/arithmetic_v1.png}
    \refstepcounter{figure}  % Manually advances the figure counter
    \label{fig:arithmetic}  % Now this label correctly refers to the figure
\end{minipage}

Notably, when scaling with \( T \), we observe in the figure above that the task scaling closely follow the theoretical $O(D\log(DT))$ dependency. Given that the function class grows exponentially as \( 2^T \), it is truly remarkable that training on only a few hundred tasks enables generalization to an exponentially larger space—on the order of \( 2^{25} > 33 \) Million tasks. This exponential scaling highlights the efficiency of structured learning, where a modest number of training examples can yield vast generalization capability.





% Our theory suggests that only $\Tilde{O}(\ln(T))$ i.i.d training tasks is enough to generalize to the rest of unseen tasks. However, we show in Figure \ref{fig:arithmetic} that transformer is not able to match  that. The transformer out-of distribution generalization behavior is not consistent across different dimensions when we scale the number of training tasks with $\ln(T)$. \hongzhou{implicit bias, optimization, etc}
 






% \subsection{Task generalization Beyond parity problem}

% \subsection{Arithmetic} In this setting, we still use the set-up we introduced in \ref{subsec:parity_exmaple}, the input is still a set of $d$ binary variable, $b_1, b_2,\dots,b_d$ and ${Arithmatic_{S}}:\{0,1\}\rightarrow \{0, 1, \dots, d\}$, where $S = (i_1,i_2,\dots,i_{d-1})$ is a tuple of size $d-1$ where each coordinate is either add($+
% $) or multiplication ($\times$). The function is as following,

% \begin{align*}
%     Arithmatic_{S}(b_1, b_2,\dots,b_d) = (\dots(b1(i1)b2)(i3)b3\dots)(i{d-1})
% \end{align*}
    


\subsubsection{Multi-Step Language Translation Task}

 \begin{figure*}[h!]
    \centering
    \includegraphics[width=0.9\textwidth]{Figures/combined_plot_horiz.png}
    \vspace{-0.2cm}
    \caption{Task generalization for language translation task: $\d$ is the number of languages and $T$ is the length of steps.}
    \vspace{-2mm}
    \label{fig:language}
\end{figure*}
% \vspace{-2mm}

In this task, we study a sequential translation process across multiple languages~\cite{garg2022can}. Given a set of \( D \) languages, we construct a translation chain by randomly sampling a sequence of \( T \) languages \textbf{with replacement}:  \(L_1, L_2, \dots, L_T,\)
where each \( L_t \) is a sampled language. Starting with a word, we iteratively translate it through the sequence:
\vspace{-2mm}
\[
L_1 \to L_2 \to L_3 \to \dots \to L_T.
\]
For example, if the sampled sequence is EN → FR → DE → FR, translating the word "butterfly" follows:
\vspace{-1mm}
\[
\text{butterfly} \to \text{papillon} \to \text{schmetterling} \to \text{papillon}.
\]
This task follows an \textit{AutoRegressive Compositional} structure by itself, specifically \( ARC(D, T-1) \), where at each step, the conditional generation only depends on the target language, making \( D \) as the number of languages and the total number of possible tasks is \( D^{T-1} \). This example illustrates that autoregressive compositional structures naturally arise in real-world languages, even without explicit CoT. 

We examine task scaling along \( D \) (number of languages) and \( T \) (sequence length). As shown in Figure~\ref{fig:language}, empirical  \( D \)-scaling closely follows the theoretical \( O(D \ln D T) \). However, in the \( T \)-scaling case, we observe a linear dependency on \( T \) rather than the logarithmic dependency \(O(\ln T) \). A possible explanation is error accumulation across sequential steps—longer sequences require higher precision in intermediate steps to maintain accuracy. This contrasts with our theoretical analysis, which focuses on asymptotic scaling and does not explicitly account for compounding errors in finite-sample settings.

% We examine task scaling along \( D \) (number of languages) and \( T \) (sequence length). As shown in Figure~\ref{fig:language}, empirical scaling closely follows the theoretical \( O(D \ln D T) \) trend, with slight exceptions at $ T=10 \text{ and } 3$ in Panel B. One possible explanation for this deviation could be error accumulation across sequential steps—longer sequences require each intermediate translation to be approximated with higher precision to maintain test accuracy. This contrasts with our theoretical analysis, which primarily focuses on asymptotic scaling and does not explicitly account for compounding errors in finite-sample settings.

Despite this, the task scaling is still remarkable — training on a few hundred tasks enables generalization to   $4^{10} \approx 10^6$ tasks!






% , this case, we are in a regime where \( D \ll T \), we observe  that the task complexity empirically scales as \( T \log T \) rather than \( D \log T \). 


% the model generalizes to an exponentially larger space of \( 2^T \) unseen tasks. In case $T=25$, this is $2^{25} > 33$ Million tasks. This remarkable exponential generalization demonstrates the power of structured task composition in enabling efficient generalization.


% In the case of parity tasks, introducing CoT effectively decomposes the problem from \( ARC(D^T, 1) \) to \( ARC(D, T) \), significantly improving task generalization.

% Again, in the regime scaling $T$, we again observe a $T\log T$ dependency. Knowing that the function class is scaling as $D^T$, it is remarkable that training on a few hundreds tasks can generalize to $4^{10} \approx 1M$ tasks. 





% We further performed a preliminary investigation on a semi-synthetic word-level translation task to show that (1) task generalization via composition structure is feasible beyond parity and (2) understanding the fine-grained mechanism leading to this generalization is still challenging. 
% \noindent
% \noindent
% \begin{minipage}[t]{\columnwidth}
%     \centering
%     \textbf{\scriptsize In-context examples:}
%     \[
%     \begin{array}{rl}
%         \textbf{Input} & \hspace{1.5em} \textbf{Output} \\
%         \hline
%         \textcolor{blue}{car}   & \hspace{1.5em} \textcolor{red}{voiture \;,\; coche} \\
%         \textcolor{blue}{house} & \hspace{1.5em} \textcolor{red}{maison \;,\; casa} \\
%         \textcolor{blue}{dog}   & \hspace{1.5em} \textcolor{red}{chien \;,\; perro} 
%     \end{array}
%     \]
%     \textbf{\scriptsize Query:}
%     \[
%     \begin{array}{rl}
%         \textbf{Input} & \textbf{Output} \\
%         \hline
%         \textcolor{blue}{cat} & \hspace{1.5em} \textcolor{red}{?} \\
%     \end{array}
%     \]
% \end{minipage}



% \begin{figure}[h!]
%     \centering
%     \includegraphics[width=0.45\textwidth]{Figures/translation_scale_d.png}
%     \vspace{-0.2cm}
%     \caption{Task generalization behavior for word translation task.}
%     \label{fig:arithmetic}
% \end{figure}


\vspace{-1mm}
\section{Conclusions}
% \misha{is it conclusion of the section or of the whole paper?}    
% \amir{The whole paper. It is very short, do we need a separate section?}
% \misha{it should not be a subsection if it is the conclusion the whole thing. We can just remove it , it does not look informative} \hz{let's do it in a whole section, just to conclude and end the paper, even though it is not informative}
%     \kaiyue{Proposal: Talk about the implication of this result on theory development. For example, it calls for more fine-grained theoretical study in this space.  }

% \huaqing{Please feel free to edit it if you have better wording or suggestions.}

% In this work, we propose a theoretical framework to quantitatively investigate task generalization with compositional autoregressive tasks. We show that task to $D^T$ task is theoretically achievable by training on only $O (D\log DT)$ tasks, and empirically observe that transformers trained on parity problem indeed achieves such task generalization. However, for other tasks beyond parity, transformers seem to fail to achieve this bond. This calls for more fine-grained theoretical study the phenomenon of task generalization specific to transformer model. It may also be interesting to study task generalization beyond the setting of in-context learning. 
% \misha{what does this add?} \amir{It does not, i dont have any particular opinion to keep it. @Hongzhou if you want to add here?}\hz{While it may not introduce anything new, we are following a good practice to have a short conclusion. It provides a clear closing statement, reinforces key takeaways, and helps the reader leave with a well-framed understanding of our contributions. }
% In this work, we quantitatively investigate task generalization under autoregressive compositional structure. We demonstrate that task generalization to $D^T$ tasks is theoretically achievable by training on only $\tilde O(D)$ tasks. Empirically, we observe that transformers trained indeed achieve such exponential task generalization on problems such as parity, arithmetic and multi-step language translation. We believe our analysis opens up a new angle to understand the remarkable generalization ability of Transformer in practice. 

% However, for tasks beyond the parity problem, transformers appear to fail to reach this bound. This highlights the need for a more fine-grained theoretical exploration of task generalization, especially for transformer models. Additionally, it may be valuable to investigate task generalization beyond the scope of in-context learning.



In this work, we quantitatively investigated task generalization under the autoregressive compositional structure, demonstrating both theoretically and empirically that exponential task generalization to $D^T$ tasks can be achieved with training on only $\tilde{O}(D)$ tasks. %Our theoretical results establish a fundamental scaling law for task generalization, while our experiments validate these insights across problems such as parity, arithmetic, and multi-step language translation. The remarkable ability of transformers to generalize exponentially highlights the power of structured learning and provides a new perspective on how large language models extend their capabilities beyond seen tasks. 
We recap our key contributions  as follows:
\begin{itemize}
    \item \textbf{Theoretical Framework for Task Generalization.} We introduced the \emph{AutoRegressive Compositional} (ARC) framework to model structured task learning, demonstrating that a model trained on only $\tilde{O}(D)$ tasks can generalize to an exponentially large space of $D^T$ tasks.
    
    \item \textbf{Formal Sample Complexity Bound.} We established a fundamental scaling law that quantifies the number of tasks required for generalization, proving that exponential generalization is theoretically achievable with only a logarithmic increase in training samples.
    
    \item \textbf{Empirical Validation on Parity Functions.} We showed that Transformers struggle with standard in-context learning (ICL) on parity tasks but achieve exponential generalization when Chain-of-Thought (CoT) reasoning is introduced. Our results provide the first empirical demonstration of structured learning enabling efficient generalization in this setting.
    
    \item \textbf{Scaling Laws in Arithmetic and Language Translation.} Extending beyond parity functions, we demonstrated that the same compositional principles hold for arithmetic operations and multi-step language translation, confirming that structured learning significantly reduces the task complexity required for generalization.
    
    \item \textbf{Impact of Training Task Selection.} We analyzed how different task selection strategies affect generalization, showing that adversarially chosen training tasks can hinder generalization, while diverse training distributions promote robust learning across unseen tasks.
\end{itemize}



We introduce a framework for studying the role of compositionality in learning tasks and how this structure can significantly enhance generalization to unseen tasks. Additionally, we provide empirical evidence on learning tasks, such as the parity problem, demonstrating that transformers follow the scaling behavior predicted by our compositionality-based theory. Future research will  explore how these principles extend to real-world applications such as program synthesis, mathematical reasoning, and decision-making tasks. 


By establishing a principled framework for task generalization, our work advances the understanding of how models can learn efficiently beyond supervised training and adapt to new task distributions. We hope these insights will inspire further research into the mechanisms underlying task generalization and compositional generalization.

\section*{Acknowledgements}
We acknowledge support from the National Science Foundation (NSF) and the Simons Foundation for the Collaboration on the Theoretical Foundations of Deep Learning through awards DMS-2031883 and \#814639 as well as the  TILOS institute (NSF CCF-2112665) and the Office of Naval Research (ONR N000142412631). 
This work used the programs (1) XSEDE (Extreme science and engineering discovery environment)  which is supported by NSF grant numbers ACI-1548562, and (2) ACCESS (Advanced cyberinfrastructure coordination ecosystem: services \& support) which is supported by NSF grants numbers \#2138259, \#2138286, \#2138307, \#2137603, and \#2138296. Specifically, we used the resources from SDSC Expanse GPU compute nodes, and NCSA Delta system, via allocations TG-CIS220009. 


\vspace{-0.cm}
\section{Conclusion}
Leveraging deterministic equivalence, we derive a sharp characterization of the relationship between the expected test error and the $\ell_2$-norm based capacity for both linear models and RFMs. 
These results suggest a reshaping of double descent and scaling law when considering the weights norm rather than their number,implying a number of new insights from phase transition to scaling law.

\section*{Acknowledgment}
We thank for Denny Wu's discussion on asymptotic deterministic equivalence and optimal regularization.
Y. Chen was supported in part by National Science Foundation grants CCF-2233152.
L. Rosasco acknowledge the Ministry of Education, University and Research (Grant ML4IP R205T7J2KP). L. Rosasco acknowledges the European Research Council (Grant SLING 819789), the US Air Force Office of Scientific Research (FA8655-22-1-7034). The research by L. Rosasco has been supported by the MIUR Grant PRIN 202244A7YL and the MUR PNRR project PE0000013 CUP J53C22003010006 ’Future Artificial Intelligence Research (FAIR)’.
F. Liu was supported by Royal Society KTP R1 241011 Kan Tong Po Visiting Fellowships. 



\bibliography{references}
\bibliographystyle{ims}


\newpage
\appendix
\enableaddcontentsline
\tableofcontents
\newpage
%\section{Appendix}
% You may include other additional sections here.


\section{Notations}
\label{app:notation}

\begin{table}[H]
\begin{threeparttable}
\caption{Core notations used the main text and appendix.}
\label{table:symbols_and_notations}
\small
\centering
\fontsize{7}{7}\selectfont
\begin{tabular}{c | c | c}
\toprule
Notation & Dimension(s) & Definition \\
\midrule
\(\mathcal{N}_{\lambda}^{\tt LS}\) & - & The \(\ell_2\) norm of the linear regression estimator under regularization \(\lambda\) for linear regression \\
\(\mathcal{B}_{\mathcal{N},\lambda}^{\tt LS}\) & - & The bias of \(\mathcal{N}_{\lambda}^{\tt LS}\) \\
\(\mathcal{V}_{\mathcal{N},\lambda}^{\tt LS}\) & - & The variance of \(\mathcal{N}_{\lambda}^{\tt LS}\) \\ \midrule
\(\sN_{\lambda}^{\tt LS}\) & - & The deterministic equivalent of \(\mathcal{N}_{\lambda}^{\tt LS}\) \\
\(\sB_{\sN, \lambda}^{\tt LS}\) & - & The deterministic equivalent of \(\mathcal{B}_{\mathcal{N},\lambda}^{\tt LS}\) \\
\(\sV_{\sN, \lambda}^{\tt LS}\) & - & The deterministic equivalent of \(\mathcal{V}_{\mathcal{N},\lambda}^{\tt LS}\) \\
\midrule
$\left \| \bm{v} \right \|_2$ & - & Euclidean norms of vectors $\bm{v}$ \\
$\left \| \bm{v} \right \|_\bSigma$ & - & $\sqrt{\bv^\sT \bSigma \bv}$ \\
\midrule
$n$ & - & Number of training samples \\
$d$ & - & Dimension of the data for linear regression \\
$p$ & - & Number of features for random feature model \\
$\lambda$ & - & Regularization parameter \\
$\lambda_*$ & - & Effective regularization parameter for linear ridge regression \\
$\nu_1\,,\nu_2$ & - & Effective regularization parameters for random feature ridge regression \\
$\sigma_k(\bM)$ & - & The $k$-th eigenvalue of $\bM$ \\
\midrule
$\bm{x}$ & $\mathbb{R}^{d}$ & The data vector \\
$\bX$ & $\mathbb{R}^{n \times d}$ & The data matrix\\
$\bSigma$ & $\mathbb{R}^{d \times d}$ & The covariance matrix of $\bx$\\
$y$ & $\mathbb{R}$ & The label \\
$\by$ & $\mathbb{R}^{n}$ & The label vector \\
$\bbeta_*$ & $\mathbb{R}^{d}$ & The target function for linear regression \\
$\hat{\bbeta}$ & $\mathbb{R}^{d}$ & The estimator of ridge regression model \\
$\hat{\bbeta}_{\min}$ & $\mathbb{R}^{d}$ & The min-$\ell_2$-norm estimator of ridge regression model \\
$\varepsilon$ & $\mathbb{R}$ & The noise \\
$\varepsilon_i$ & $\mathbb{R}$ & The $i$-th noise \\
$\bm\varepsilon$ & $\mathbb{R}^{n}$ & The noise vector \\
$\sigma^2$ & $\mathbb{R}$ & The variance of the noise\\ \midrule
$\bw_i$ & $\mathbb{R}^{d}$ & The $i$-th weight vector for random feature model \\ 
$\varphi(\cdot;\cdot)$ & - & Nonlinear activation function for random feature model \\
$\bz_i$ & $\mathbb{R}^{p}$ & The $i$-th feature for random feature model \\
$\bZ$ & $\mathbb{R}^{n \times p}$ & Feature matrix for random feature model \\
$\hat{\ba}$ & $\mathbb{R}^{p}$ & The estimator of random feature ridge regression model\\
$\hat{\ba}_{\min}$ & $\mathbb{R}^{p}$ & The min-$\ell_2$-norm estimator of random feature ridge regression model\\ \midrule
$f_*(\cdot)$ & - & The target function \\
$\mu_\bx$ & - & The distribution of $\bx$ \\
$\mu_\bw$ & - & The distribution of $\bw$ \\
$\mathbb{T}$ & - & An integral
operator defined by $(\mathbb{T}f)(\bw) := \int_{\mathbb{R}^d} \varphi(\bx; \bw) f(\bx) \mathrm{d}\mu_\bx \,,\quad \forall f \in L_2(\mu_\bx)$ \\
$\mathcal{V}$ & - & The image of $\mathbb{T}$\\
$\xi_k$ & $\mathbb{R}$ & The $k$-th eigenvalue of $\mathbb{T}$, defined by
$\mathbb{T} = \sum_{k=1}^\infty \xi_k \psi_k \phi_k^*$ \\
$\psi_k$ & - & The $k$-th eigenfunction of $\mathbb{T}$ in the space $L_2(\mu_\bx)$, defined by the decomposition
$\mathbb{T} = \sum_{k=1}^\infty \xi_k \psi_k \phi_k^*$ \\
$\phi_k$ & - & The $k$-th eigenfunction of $\mathbb{T}$ in the space $\mathcal{V}$, defined by the decomposition
$\mathbb{T} = \sum_{k=1}^\infty \xi_k \psi_k \phi_k^*$ \\
$\bLambda$ & $\mathbb{R}^{\infty \times \infty}$ & The spectral matrix of $\mathbb{T}$, $\bLambda = \operatorname{diag}(\xi_1^2, \xi_2^2, \ldots) \in \mathbb{R}^{\infty \times \infty}$ \\
$\bg_i$ & $\mathbb{R}^{\infty}$ & $\bg_i := (\psi_k(\bx_i))_{k \geq 1}$\\
$\boldf_i$ & $\mathbb{R}^{\infty}$ & $\boldf_i := (\xi_k\phi_k(\bw_i))_{k \geq 1}$\\
$\bG$ & $\mathbb{R}^{n \times \infty}$ & 
$\bG \!:=\! [\bg_1, \ldots, \bg_n]^\sT \!\in\! \mathbb{R}^{n \times \infty}$ with $\bg_i := (\psi_k(\bx_i))_{k \geq 1}$\\
$\bF$ & $\mathbb{R}^{p \times \infty}$ & $\bF \!:=\! [\boldf_1, \ldots, \boldf_p]^\sT \!\in\! \mathbb{R}^{p \times \infty}$\\
$\hbLambda_\bF$ & $\mathbb{R}^{p \times p}$ & $\hbLambda_\bF := \E_\bz[\bz\bz^\sT|\bF] = \frac{1}{p}\bF\bF^\sT \in \R^{p \times p}$ \\
$\btheta_{*,k}$ & $\mathbb{R}$ & The coefficients associated with the eigenfunction $\psi_k$ in the expansion of $f_*(\bx)=\sum_{k\geq1}\btheta_{*,k}\psi_k(\bx)$ \\
$\btheta_*$ & $\mathbb{R}^{\infty}$ & $\btheta_* = (\btheta_{*,k})_{k \geq 1}$ \\
\midrule
\end{tabular}
\begin{tablenotes}
    \footnotesize
    \item[1] Replacing $\mathcal{N}$ with $\mathcal{R}$ ($\sN$ with $\sR$), we get the notations associated to the test risk.
    \item[2] Replacing $\lambda$ with $0$, we get the notations associated to the min-$\ell_2$-norm solution.
    \item[3] Replacing ${\tt LS}$ with ${\tt RFM}$, we get the notations associated to random feature regression.
\end{tablenotes}
\end{threeparttable}
\end{table}


\section{Preliminary and background}
\label{app:pre_result}

We provide an overview of the preliminary results used in this work. For self-contained completeness, we include results on asymptotic deterministic equivalence in \cref{app:pre_asy_deter_equiv}, results on ridge regression in \cref{app:pre_lr}, and results on random feature ridge regression in \cref{app:pre_rfrr}. Additionally, \cref{app:pre_non-asy_deter_equiv} presents results on non-asymptotic deterministic equivalence, along with definitions of quantities required for these results. Finally, \cref{app:pre_scaling_law} introduces key results for deriving the scaling law.



\subsection{Preliminary results on asymptotic deterministic equivalence}
\label{app:pre_asy_deter_equiv}

For the ease of description, we include preliminary results on asymptotic deterministic equivalence here. In fact, these assumptions and results can be recovered from non-asymptotic results, e.g., \cite{misiakiewicz2024non}.

For linear regression, the asymptotic deterministic equivalence aim to find $\mathcal{B}_{\mathcal{R},\lambda}^{\tt LS} \sim \sB_{\sR, \lambda}^{\tt LS}$, $\mathcal{V}_{\mathcal{R},\lambda}^{\tt LS} \sim \sV_{\sR, \lambda}^{\tt LS}$, where $\sB_{\sR, \lambda}^{\tt LS}$ and $\sV_{\sR, \lambda}^{\tt LS}$ are some deterministic quantities.
For asymptotic results, a series of assumptions in high-dimensional statistics via random matrix theory are required, on well-behaved data, spectral properties of $\bSigma$ under nonlinear transformation in high-dimensional regime.
We put the assumption from \citet{bach2024high} here that are also widely used in previous literature \cite{dobriban2018high, richards2021asymptotics}. 

\begin{assumption}\citep{bach2024high}\label{ass:asym}
    We assume that:
    \begin{itemize}
    \item[\textbf{(A4)}] The sample size $n$ and dimension $d$ grow to infinity with $\frac{d}{n} \to \gamma > 0$.
    \item[\textbf{(A5)}] $\bX = \bT \bSigma^{1/2}$, where $\bT \in \mathbb{R}^{n \times d}$ has i.i.d.\ sub-Gaussian entries with zero mean and unit variance.
    \item[\textbf{(A6)}] $\bSigma$ is invertible with $\| \bSigma \|_{\text{op}}< \infty$ and its spectral measure $ \frac{1}{d} \sum_{i=1}^d \delta_{\sigma_i} $ converges to a compactly supported probability distribution $\mu$ on $\mathbb{R}^+$.
    \item[\textbf{(A7)}] $\|\bbeta_\ast\|_2 < \infty$ and the measure $ \sum_{i=1}^d (\bv_i^\sT \bbeta_\ast)^2 \delta_{\sigma_i} $ converges to a measure $\nu$ with bounded mass, where $\bv_i$ is the unit-norm eigenvector of $\bSigma$ related to its respective eigenvalue $\sigma_i$.
    \end{itemize}
\end{assumption}

\begin{definition}[Effective regularization]
    For $n$, $\bSigma$, and $\lambda \geq 0$, we define the \emph{effective regularization} $\lambda_*$ to be the unique non-negative solution to the self-consistent equation
\begin{equation}\label{eq:def_lambda_star_asy}
    n - \frac{\lambda}{\lambda_*} \sim \Tr ( \bSigma ( \bSigma + \lambda_* )^{-1} ).
\end{equation}
\end{definition}

\begin{definition}[Degrees of freedom]\label{def:df}
\[
{\rm df}_1(\lambda_*) := \Tr ( \bSigma ( \bSigma + \lambda_*)^{-1}), \quad {\rm df}_2(\lambda_*) := \Tr ( \bSigma^2 ( \bSigma + \lambda_*)^{-2}).
\]
\end{definition}

\begin{proposition}\citep[Restatement of Proposition 1]{bach2024high}\label{prop:spectral}
    Assume \textbf{(A4)}, \textbf{(A5)}, \textbf{(A6)}, we consider $\bA$ and $\bB$ with bounded operator norm, admitting the convergence of the empirical measures, i.e., $ \sum_{i=1}^d   \bv_i^\sT \bA \bv_i  \cdot\delta_{\sigma_i} \rightarrow \nu_A$
    and $ \sum_{i=1}^d   \bv_i^\sT \bB \bv_i  \cdot\delta_{\sigma_i} \rightarrow \nu_B$ with bounded total variation, respectively. Then, for $\lambda \geq 0$, with $\lambda_*$ satisfying Eq.~\eqref{eq:def_lambda_star_asy},
    we have the following {\bf asymptotic deterministic equivalence}
    \begin{align}
        \label{eq:trA1}
        \Tr ( \bA \bX^\sT \bX ( \bX^\sT \bX +\lambda )^{-1} ) \sim&~ \Tr ( \bA \bSigma ( \bSigma + \lambda_* )^{-1} )\,,
        \\
        \label{eq:trAB1}
        \Tr ( \bA \bX^\sT \bX ( \bX^\sT \bX + \lambda )^{-1} \bB \bX^\sT \bX ( \bX^\sT \bX + \lambda )^{-1}) \sim&~ \Tr ( \bA \bSigma ( \bSigma + \lambda_* )^{-1} \bB \bSigma ( \bSigma + \lambda_* )^{-1} ) \nonumber \\
        + \lambda_*^2 \Tr ( \bA ( \bSigma + \lambda_* )^{-2}  \bSigma ) &\cdot \Tr ( \bB ( \bSigma + \lambda_* )^{-2} \bSigma ) \cdot \frac{1}{ n -  {\rm df}_2(\lambda_*) }\,,\\
        \label{eq:trA2}
        \Tr ( \bA ( \bX^\sT \bX +\lambda )^{-1} ) \sim&~ \frac{\lambda_*}{\lambda} \Tr ( \bA ( \bSigma + \lambda_* )^{-1} )\,,
        \\
        \label{eq:trAB2}
        \Tr ( \bA ( \bX^\sT \bX + \lambda )^{-1} \bB ( \bX^\sT \bX + \lambda )^{-1}) \sim&~ \frac{\lambda_*^2}{\lambda^2} \Tr ( \bA ( \bSigma + \lambda_* )^{-1} \bB ( \bSigma + \lambda_* )^{-1} ) \nonumber \\
        + \frac{\lambda_*^2}{\lambda^2} \Tr ( \bA ( \bSigma + \lambda_* )^{-2}  \bSigma ) &\cdot \Tr ( \bB ( \bSigma + \lambda_* )^{-2} \bSigma ) \cdot \frac{1}{ n -  {\rm df}_2(\lambda_*) }\,.
    \end{align}
\end{proposition}

\begin{proposition}\citep[Restatement of Proposition 2]{bach2024high}
\label{prop:spectralK}
Assume \textbf{(A4)}, \textbf{(A5)}, \textbf{(A6)}, we consider $\bA$ and $\bB$ with bounded operator norm, admitting the convergence of the empirical measures, i.e., $ \sum_{i=1}^d   \bv_i^\sT \bA \bv_i  \cdot\delta_{\sigma_i} \rightarrow \nu_A$ and $ \sum_{i=1}^d   \bv_i^\sT \bB \bv_i  \cdot\delta_{\sigma_i} \rightarrow \nu_B$ with bounded total variation, respectively. Then, for $\lambda \in \mathbb{C} \backslash \mathbb{R}_+$, with $\lambda_*$ satisfying Eq.~\eqref{eq:def_lambda_star_asy}, we have the following {\bf asymptotic deterministic equivalence}
\begin{align}
\label{eq:trA1K}
\Tr ( \bA \bT^\sT ( \bT \bSigma \bT^\sT + \lambda )^{-1} \bT) \sim&~ \Tr ( \bA ( \bSigma + \lambda_* )^{-1} ),
\\
\label{eq:trAB1K}
\Tr ( \bA \bT^\sT ( \bT \bSigma \bT^\sT + \lambda )^{-1} \bT \bB \bT^\sT ( \bT \bSigma \bT^\sT + \lambda )^{-1} \bT) \nonumber \sim&~ \Tr ( \bA ( \bSigma + \lambda_* )^{-1} \bB ( \bSigma + \lambda_* )^{-1} )\\
+ \lambda_*^2 \Tr ( \bA ( \bSigma + \lambda_* )^{-2} )&~ \cdot \Tr ( \bB ( \bSigma + \lambda_* )^{-2} ) \cdot \frac{1}{ n -  {\rm df}_2(\lambda_*) }\,.
\end{align}
\end{proposition}

Note that the results in \cref{prop:spectral}, \ref{prop:spectralK} still hold even for the random features model.
We will explain this in details in \cref{app:proof_rf}.


\subsection{Preliminary results on ridge regression via deterministic equivalence}
\label{app:pre_lr}

For well-specified ridge regression, we use $n$ i.i.d. samples $\{ (\bm x_i, y_i) \}_{i=1}^n$ to learn a linear target function $\bbeta_*$
\[
    y_i = \bx_i^\sT \bbeta_* + \varepsilon_i\,.
\]
The estimator is given by solving the following empirical risk minimization with an $\ell_2$ regularization term
\[
    \hat{\bm \beta} := \argmin_{\bbeta \in \mathbb{R}^d} \left\{ \sum_{i =1}^n \left(y_i - \bm x_i^\sT\bm \beta\right)^2 + \lambda \|\bbeta\|_2^2 \right\} = (\bX^\sT \bX + \lambda \id)^{-1} \bX^\sT \bm{y}\,.
\]
Accordingly, the bias-variance decomposition is given by
\begin{align}
    \mathcal{B}_{\mathcal{R},\lambda}^{\tt LS} :=&~ \|\bbeta_* - \mathbb{E}_{\varepsilon}[\hat{\bbeta}]\|_{\bSigma}^2 = \lambda^2 \langle \bbeta_*,(\bX^\sT \bX + \lambda\id)^{-1} \bSigma (\bX^\sT \bX + \lambda\id)^{-1} \bbeta_* \rangle\,,\label{eq:lr_risk_bias}\\
    \mathcal{V}_{\mathcal{R},\lambda}^{\tt LS} :=&~ \Tr\left(\bSigma \mathrm{Cov}_{\varepsilon}(\hat{\bbeta})\right) = \sigma^2\Tr(\bSigma \bX^\sT \bX (\bX^\sT \bX + \lambda\id)^{-2})\,.\label{eq:lr_risk_variance}
\end{align}
Accordingly, the risk admits the following deterministic equivalents via bias-variance decomposition.
\begin{proposition}\citep[Restatement of Proposition 3]{bach2024high}\label{prop:asy_equiv_risk_LR}
    Given the bias variance decomposition in \cref{eq:lr_risk_bias} and \cref{eq:lr_risk_variance}, \(\bX\), \(\bSigma\) and \(\bbeta_*\) satisfy \cref{ass:asym}, we have the following asymptotic deterministic equivalents $\mathcal{R}_{\lambda}^{\tt LS}  \sim \sR_{\lambda}^{\tt LS} := \sB_{\sR,\lambda}^{\tt LS} + \sV_{\sR,\lambda}^{\tt LS}$ such that $\mathcal{B}^{\tt LS}_{\mathcal{R},\lambda} \sim \sB_{\sR,\lambda}^{\tt LS}$, $\mathcal{V}^{\tt LS}_{\mathcal{R},\lambda} \sim \sV_{\sR,\lambda}^{\tt LS}$, where $\sB_{\sR,\lambda}^{\tt LS}$ and $\sV_{\sR,\lambda}^{\tt LS}$ are defined by \cref{eq:de_risk}.
\end{proposition}


\begin{proposition}\citep[Restatement of results in Sec 5]{bach2024high} \label{prop:asy_equiv_error_LR_minnorm} 
Under the same assumption as \cref{prop:asy_equiv_risk_LR}, for the minimum $\ell_2$-norm estimator $\hat{\bbeta}_{\min}$, we have for the under-parameterized regime ($d<n$):
    \[
        \mathcal{B}_{\mathcal{R},0}^{\tt LS} = 0,\quad \mathcal{V}_{\mathcal{R},0}^{\tt LS} \sim \sigma^2\frac{d}{n-d}\,.
    \]
    In the over-parameterized regime ($d>n$), we have
    \[
        \mathcal{B}_{\mathcal{R},0}^{\tt LS} \sim \frac{\lambda_n^2\<\bbeta_*,\bSigma(\bSigma+\lambda_n\id)^{-2}\bbeta_*\>}{1-n^{-1}\Tr(\bSigma^2(\bSigma+\lambda_n\id)^{-2})}\,,\qquad
        \mathcal{V}_{\mathcal{R},0}^{\tt LS} \sim \frac{\sigma^2\Tr(\bSigma^2(\bSigma+\lambda_n\id)^{-2})}{n-\Tr(\bSigma^2(\bSigma+\lambda_n\id)^{-2})}\,,
    \]
    where $\lambda_n$ defined by $\Tr(\bSigma(\bSigma+\lambda_n\id)^{-1}) \sim n$.
\end{proposition}


\subsection{Preliminary results on random feature ridge regression via deterministic equivalence}
\label{app:pre_rfrr}

Recall \cref{eq:rffa}, the parameter $\bm a$ can be learned by the following empirical risk minimization with an $\ell_2$ regularization 
\[
    \hat{\ba} := \argmin_{\ba \in \mathbb{R}^p} \left\{ \sum_{i =1}^n \left(y_i - \frac{1}{\sqrt{p}} \sum_{j=1}^p \bm a_j \varphi(\bm x, \bm w_j) \right)^2 + \lambda \|\ba\|_2^2 \right\} = (\bZ^\sT \bZ + \lambda \id)^{-1} \bZ^\sT \by\,.
\]
Assuming that the target function $f_* \in L^2(\mu_\bx)$ admits $f_*(\bx)=\sum_{k\geq1}\btheta_{*,k}\psi_k(\bx)$, the excess risk $\mathcal{R}^{\tt RFM} := \E_{\varepsilon} \left\|\btheta_* - \frac{\bF^\sT\hat{\ba}}{\sqrt{p}}\right\|_2^2$ admits the following bias-variance decomposition
\begin{align}
    \mathcal{B}_{\mathcal{R},\lambda}^{\tt RFM} :=&~ \left\|\btheta_* - \frac{\bF^\sT \mathbb{E}_{\varepsilon}[\hat{\ba}]}{\sqrt{p}}\right\|_2^2 = \left\|\btheta_* - p^{-1/2} \bF^\sT (\bZ^\sT \bZ + \lambda\id)^{-1} \bZ^\sT \bG \bm \theta_* \right\|_2^2\,,\label{eq:rf_risk_bias}\\
    \mathcal{V}_{\mathcal{R},\lambda}^{\tt RFM} :=&~ \Tr\left(\hbLambda_\bF \mathrm{Cov}_{\varepsilon}(\hat{\ba})\right) = \sigma^2\Tr(\hbLambda_\bF \bZ^\sT\bZ(\bZ^\sT\bZ+\lambda\id)^{-2}) \,.\label{eq:rf_risk_variance}
\end{align}
Accordingly, the risk admits the following deterministic equivalents via bias-variance decomposition.
\begin{proposition}\citep[Asymptotic version of Theorem 3.3]{defilippis2024dimension}\label{prop:asy_equiv_risk_RFRR}
    Given the bias variance decomposition in \cref{eq:rf_risk_bias} and \cref{eq:rf_risk_variance}, 
    under \cref{ass:concentrated_RFRR}, we have the following asymptotic deterministic equivalents $\mathcal{R}_{\lambda}^{\tt RFM}  \sim \sR_{\lambda}^{\tt RFM} := \sB_{\sR,\lambda}^{\tt RFM} + \sV_{\sR,\lambda}^{\tt RFM}$ such that $\mathcal{B}^{\tt RFM}_{\mathcal{R},\lambda} \sim \sB_{\sR,\lambda}^{\tt RFM}$, $\mathcal{V}^{\tt RFM}_{\mathcal{R},\lambda} \sim \sV_{\sR,\lambda}^{\tt RFM}$, where $\sB_{\sR,\lambda}^{\tt RFM}$ and $\sV_{\sR,\lambda}^{\tt RFM}$ are defined by \cref{eq:de_risk_rf}.
\end{proposition}
Note that the above results are delivered in a non-asymptotic way \citep{defilippis2024dimension}, but more notations and technical assumptions are required. We give an overview of non-asymptotic deterministic equivalence as below.


\subsection{Preliminary results on non-asymptotic deterministic equivalence}\label{app:pre_non-asy_deter_equiv}

Regarding non-asymptotic results, we require a series of notations and assumptions. We give a brief introduction here for self-completeness. More details can be found in \citet{cheng2022dimension,misiakiewicz2024non,defilippis2024dimension}.

Given $\bx \in \R^d$ with $d \in \naturals$, the associated covariance matrix is given by $\bSigma = \E[\bx \bx^\sT]$. We denote the eigenvalue of $\bSigma$ in non-increasing order as $\sigma_1 \geq \sigma_2 \geq \sigma_3 \geq \cdots \geq \sigma_d$. 

We introduce the non-asymptotic version of \cref{eq:def_lambda_star_asy} as below.
\begin{definition}[Effective regularization]\label{def:effective_regularization}
    Given $n$, $\bSigma$, and $\lambda \geq 0$, the \emph{effective regularization} $\lambda_*$ is defined as the unique non-negative solution of the following self-consistent equation
    \begin{equation*}
        n - \frac{\lambda}{\lambda_*} = \Tr \big( \bSigma ( \bSigma + \lambda_* )^{-1} \big).
    \end{equation*}
\end{definition}
\noindent{\bf Remark:} 
Existence and uniqueness of $\lambda_*$ are guaranteed since the left-hand side of the equation is monotonically increasing in $\lambda_*$, while the right-hand side is monotonically decreasing. 

In the next, we introduce the following definitions on ``effective dimension'', a metric to describe the model capacity, widely used in statistical learning theory.

Define \(r_{\bSigma}(k) := \frac{\Tr(\bSigma_{\geq k})}{\| \bSigma_{\geq k} \|_{\rm op}} = \frac{\sum_{j=k}^d \sigma_j}{\sigma_k}\) as the intrinsic dimension, we require the following definition

\begin{equation}\label{eq:rho_lambda}
    \rho_{\lambda} (n) := 1 +  \frac{n \sigma_{\lfloor \eta_* \cdot n \rfloor}}{\lambda}\left\{ 1 + \frac{r_{\bSigma} (\lfloor \eta_* \cdot n \rfloor) \vee n}{n} \log \big(r_{\bSigma} (\lfloor \eta_* \cdot n \rfloor) \vee n \big) \right\},
\end{equation}
where $\eta_* \in (0,1/2)$ is a constant that will only depend on $C_*$ defined in \cref{ass:concentrated_LR}. And we used the convention that $\sigma_{\lfloor \eta_* \cdot n \rfloor} = 0$ if $\lfloor \eta_* \cdot n \rfloor > d$.


In this section we consider functionals that depend on $\bX$ and deterministic matrices. For a general PSD~matrix $\bA \in \R^{d\times d}$, define the functionals
\begin{align}
    \Phi_1(\bX; \bA, \lambda) :=&~ \Tr \left(\bA \bSigma^{1/2} (\bX^\sT \bX + \lambda)^{-1} \bSigma^{1/2}\right),\label{eq:Phi_1}\\
    \Phi_2(\bX; \bA, \lambda) :=&~ \Tr \left(\bA\bX^\sT \bX (\bX^\sT \bX + \lambda)^{-1}\right),\label{eq:Phi_2}\\
    \Phi_3(\bX; \bA, \lambda) :=&~ \Tr \left(\bA \bSigma^{1/2} (\bX^\sT \bX + \lambda)^{-1} \bSigma (\bX^\sT \bX + \lambda)^{-1} \bSigma^{1/2}\right),\label{eq:Phi_3}\\
    \Phi_4(\bX; \bA, \lambda) :=&~ \Tr \left(\bA \bSigma^{1/2} (\bX^\sT \bX + \lambda)^{-1} \frac{\bX^\sT \bX}{n} (\bX^\sT \bX + \lambda)^{-1} \bSigma^{1/2}\right).\label{eq:Phi_4}
\end{align}
These functionals can be approximated through quantities that scale proportionally to
\begin{align}
    \Psi_1(\lambda_*; \bA) :=&~ \Tr\left(\bA \bSigma (\bSigma + \lambda_*\id)^{-1}\right),\label{eq:Psi_1}\\
    \Psi_2(\lambda_*; \bA) :=&~ \frac{1}{n} \cdot \frac{\Tr\left(\bA \bSigma^2 (\bSigma + \lambda_*\id)^{-2}\right)}{n - \Tr\left(\bSigma^2 (\bSigma + \lambda_*\id)^{-2}\right)}.\label{eq:Psi_2}
\end{align}



The following theorem gathers the approximation guarantees for the different functionals stated above, and is obtained by modifying \citet[Theorem A.2]{defilippis2024dimension}. 
We generalize \cref{eq:det_equiv_phi2_main} for any PSD matrix $\bm A$, which will be required for our results on the deterministic equivalence of $\ell_2$ norm. The proof can be found in \cref{app:proof_non-asy_results}.

\begin{theorem}[Dimension-free deterministic equivalents, Theorem A.2 of \cite{defilippis2024dimension}]\label{thm:main_det_equiv_summary}
    Assume the features $\{\bx_i\}_{i \in [n]}$ satisfy \cref{ass:concentrated_LR} with a constant $C_* > 0$. Then for any $D, K > 0$, there exist constants $\eta_* \in (0, 1/2)$, $C_{D, K} > 0$ and $C_{*, D, K} > 0$ ensuring the following property holds. For any $n \geq C_{D, K}$ and $\lambda > 0$, if the following condition is satisfied:
    \begin{equation}\label{eq:conditions_det_equiv_main}
        \lambda \cdot \rho_{\lambda}(n) \geq \|\bm{\Sigma}\|_{\mathrm{op}} \cdot n^{-K}, \quad \rho_{\lambda}(n)^{\nicefrac{5}{2}} \log^{\nicefrac{3}{2}}(n) \leq K \sqrt{n},
    \end{equation}
    then for any PSD matrix $\bA$, with probability at least $1 - n^{-D}$, we have that
    \begin{align}
        |\Phi_1(\bX; \bA, \lambda) - \frac{\lambda_*}{\lambda} \Psi_1(\lambda_*; \bA)| &\leq C_{*, D, K} \frac{\rho_{\lambda}(n)^{\nicefrac{5}{2}} \log^{\nicefrac{3}{2}}(n)}{\sqrt{n}} \cdot \frac{\lambda_*}{\lambda} \Psi_1(\lambda_*; \bA),\label{eq:det_equiv_phi1_main}\\
        |\Phi_2(\bX; \id, \lambda) - \Psi_1(\lambda_*; \id)| &\leq C_{*, D, K} \frac{\rho_{\lambda}(n)^4 \log^{\nicefrac{3}{2}}(n)}{\sqrt{n}} \Psi_1(\lambda_*; \id),\label{eq:det_equiv_phi2_main}\\
        |\Phi_3(\bX; \bA, \lambda) - \left(\frac{n \lambda_*}{\lambda}\right)^2 \Psi_2(\lambda_*; \bA)| &\leq C_{*, D, K} \frac{\rho_{\lambda}(n)^6 \log^{\nicefrac{5}{2}}(n)}{\sqrt{n}} \cdot \left(\frac{n \lambda_*}{\lambda}\right)^2 \Psi_2(\lambda_*; \bA),\label{eq:det_equiv_phi3_main}\\
        |\Phi_4(\bX; \bA, \lambda) - \Psi_2(\lambda_*; \bA)| &\leq C_{*, D, K} \frac{\rho_{\lambda}(n)^6 \log^{\nicefrac{3}{2}}(n)}{\sqrt{n}} \Psi_2(\lambda_*; \bA).\label{eq:det_equiv_phi4_main}
    \end{align}
\end{theorem}


Next, we present some of the concepts to be used in deriving random feature ridge regression. Similar to how ridge regression depends on \(\lambda_*\), as defined in \cref{def:effective_regularization}, the deterministic equivalence of random feature ridge regression relies on \(\nu_1\) and \(\nu_2\), which are the solutions to the coupled equations
\begin{equation}\label{eq:fixed_points_appendix}
    n - \frac{\lambda}{\nu_1} = \Tr\left( \bLambda ( \bLambda + \nu_2)^{-1} \right)\,, \quad p - \frac{p\nu_1}{\nu_2} = \Tr \left( \bLambda ( \bLambda + \nu_2 )^{-1} \right).
\end{equation}
Writing $\nu_1$ as a function of $\nu_2$ produces the equations as below
\begin{equation}\label{eq:def:nu}
    1 + \frac{n}{p} - \sqrt{\left(1 - \frac{n}{p}\right)^2 + 4\frac{\lambda}{p\nu_2}}  = \frac{2}{p} \Tr \left( \bLambda ( \bLambda + \nu_2 )^{-1} \right)\,, \quad \nu_1 := \frac{\nu_2}{2} \left[ 1 - \frac{n}{p} + \sqrt{\left(1 - \frac{n}{p}\right)^2 + 4\frac{\lambda}{p\nu_2}} \right].
\end{equation} 



For random features, our results also depend on the capacity of $\bLambda$. Recall the definition of \(r_\bLambda(k) := \frac{\Tr(\bLambda{\geq k})}{\| \bLambda{\geq k} \|_{\rm op}}\) as the intrinsic dimension of \(\bLambda\) at level \(k\), we sequentially define the following quantities that can be found in \citet{misiakiewicz2024non,defilippis2024dimension}.

\begin{align}
    M_\bLambda (k) =&~ 1 + \frac{r_{\bLambda} (\lfloor \eta_* \cdot k \rfloor) \vee k}{k} \log \left( r_{\bLambda} (\lfloor \eta_* \cdot k \rfloor) \vee k \right)\,,\\
    \rho_\kappa (p) =&~ 1 + \frac{p \cdot \xi^2_{\lfloor \eta_* \cdot p \rfloor}}{\kappa}  M_\bLambda (p)\,, \label{eq:def_rho_p}
    \\
    \trho_\kappa (n,p) =&~ 1 + \ind \{ n \leq p/\eta_*\} \cdot \left\{ \frac{n \xi_{\lfloor \eta_* \cdot n \rfloor}^2}{\kappa} + \frac{n}{p} \cdot \rho_\kappa (p)\right\} M_\bLambda (n)\,, \label{eq:def_trho_n_p}
\end{align}
where the constant \(\eta_* \in (0,1/2)\) only depends on \(C_*\) introduced in Assumption \ref{ass:concentrated_RFRR}. 


For an integer $\evn \in \naturals$, we split the covariance matrix $\bLambda$ into low degree part and high degree part as
\[
\bLambda_0 := \diag (\xi_1^2, \xi_2^2 , \ldots , \xi_{\evn}^2)\,, \quad \bLambda_+ := \diag (\xi_{\evn+1}^2, \xi_{\evn+2}^2 , \ldots )\,.
\]

After we define the high degree feature covariance \(\bLambda_+\), we can define the function \(\gamma (\kappa) := \kappa + \Tr(\bLambda_{+})\). To simplify the statement, we assume that we can choose $\evn$ such that $p^2 \xi_{\evn +1}^2 \leq \gamma (p\lambda/n)$, which is always satisfied under \cref{ass:concentrated_RFRR}. For convenience, we will further denote
\begin{equation}\label{eq:def_gamma_lamb_plus}
\gamma_+ := \gamma (p\nu_1) , \quad \quad \gamma_\lambda := \gamma (p\lambda / n).
\end{equation}

For random feature ridge regression, we will first demonstrate that the \(\ell_2\) norm concentrates around a quantity that depends only on \(\hbLambda_\bF\). To this end, we define the following functionals with respect to \(\bZ\).
\begin{equation}\label{eq:functionals_Z}
\begin{aligned}
\Phi_3(\bZ; \bA, \kappa) &:= \Tr \left( \bA \hbLambda_\bF^{1/2} (\bZ^\sT \bZ + \kappa)^{-1} \hbLambda_\bF (\bZ^\sT \bZ + \kappa)^{-1} \hbLambda_\bF^{1/2} \right),\\
\Phi_4(\bZ; \bA, \kappa) &:= \Tr \left( \bA \hbLambda_\bF^{1/2} (\bZ^\sT \bZ + \kappa)^{-1} \frac{\bZ^\sT \bZ}{n} (\bZ^\sT \bZ + \kappa)^{-1} \hbLambda_\bF^{1/2} \right).
\end{aligned}
\end{equation}
Given that \(\bZ\) consists of i.i.d. rows with covariance \(\hbLambda_\bF = \bF \bF^\sT / p\), we will demonstrate that the aforementioned functionals can be approximated by those of \(\bF\), which, in turn, can be represented using the following functionals:

\begin{equation}\label{eq:det_equiv_Z_F}
\begin{aligned}
\widetilde{\Phi}_5(\bF; \bA, \kappa) &:= \frac{1}{n} \cdot \frac{\widetilde{\Phi}_6(\bF; \bA, \kappa)}{n - \widetilde{\Phi}_6(\bF; \bm{I}, \kappa)},\\
\widetilde{\Phi}_6(\bF; \bA, \kappa) &:= \Tr \left( \bA (\bF \bF^\sT)^2 (\bF \bF^\sT + \kappa)^{-2} \right).\\
\end{aligned}
\end{equation}

\begin{proposition}[Deterministic equivalents for $\Phi(\bZ)$ conditional on $\bF$, Proposition B.6 of \cite{defilippis2024dimension}]\label{prop:det_Z} Assume \(\{\bz_i\}_{i \in [n]}\) and \(\{\boldf\}_{i \in [p]}\) satisfy \cref{ass:concentrated_RFRR} with a constant \(C_* > 0\), and $\bF \in \mathcal{A}_{\bF}$ defined in \citet[Eq. (79)]{defilippis2024dimension}. Then for any $D, K > 0$, there exist constants $\eta_* \in (0, 1/2)$, $C_{D, K} > 0$ and $C_{*, D, K} > 0$ ensuring the following property holds. Let $\rho_{\kappa}(p)$ and $\tilde{\rho}_{\kappa}(n, p)$ be defined as per \cref{eq:def_rho_p} and \cref{eq:def_trho_n_p}, $\gamma_+$ be defined as \cref{eq:def_gamma_lamb_plus}. For any $n \geq C_{D, K}$ and $\lambda > 0$, if the following
condition is satisfied:

\begin{equation*}
\begin{aligned}
    \lambda  \geq n^{-K}\,, \quad \quad \trho_{\lambda} (n,p)^{5/2} \log^{3/2} (n) \leq K \sqrt{n}\,, \quad \quad \trho_\lambda (n,p)^2 \cdot \rho_{\gamma_+} (p)^{5/2} \log^3 (p) \leq K \sqrt{p}\,,
    \end{aligned}
\end{equation*}
then for any PSD matrix $\bA \in \mathbb{R}^{p \times p}$ (independent of \(\bZ\) conditional on \(\bF\)), we have with probability at least $1 - n^{-D}$ that

\begin{align}
\left| \Phi_3(\bZ; \bA, \lambda) - \left( \frac{n \nu_1}{\lambda} \right)^2 \widetilde{\Phi}_5(\bF; \bA, p \nu_1) \right| &\leq C_{*, D, K} \cdot \mathcal{E}_1(n, p) \cdot \left( \frac{n \nu_1}{\lambda} \right)^2 \widetilde{\Phi}_5(\bF; \bA, p \nu_1), \\
\left| \Phi_4(\bZ; \bA, \lambda) - \widetilde{\Phi}_5(\bF; \bA, p \nu_1) \right| &\leq C_{*, D, K} \cdot \mathcal{E}_1(n, p) \cdot \widetilde{\Phi}_5(\bF; \bA, p \nu_1),
\end{align}
where the rate \( \mathcal{E}_1(n, p)\) is given by \( \cE_1 (n,p) := \frac{\trho_\lambda (n,p)^6 \log^{5/2} (n)}{\sqrt{n}} + \frac{\trho_\lambda (n,p)^2 \cdot \rho_{\gamma_+} (p)^{5/2}  \log^3 (p)}{\sqrt{p}}\).

\end{proposition}



\subsection{Preliminary results on scaling law}\label{app:pre_scaling_law}

For the derivation of the scaling law, we use the results in \citet[Appendix D]{defilippis2024dimension}. We define $T^s_{\delta,\gamma}(\nu)$ as
\begin{equation*}
    T^s_{\delta,\gamma}(\nu) := \sum_{k = 1}^\infty \frac{k^{-s-\delta\alpha}}{(k^{-\alpha}+\nu)^{\gamma}}\,, \quad s \in {0,1},\;0\leq\delta\leq\gamma.
\end{equation*}
Under \cref{ass:powerlaw_rf}, according to \citet[Appendix D]{defilippis2024dimension}, we have the following results
\begin{equation}\label{eq:rate_T}
T_{\delta\gamma}^{s}(\nu) = O\left(\nu^{\nicefrac{1}{\alpha}\left[s-1 + \alpha(\delta-\gamma)\right]\wedge0}\right).
\end{equation}

Next, we present some rates of the quantities used in the deterministic equivalence of random feature ridge regression. The rate of $\nu_2$ is given by
\begin{equation}\label{eq:rate_nu2}
    \nu_2 \approx O\left(n^{-\alpha\left(1 \wedge q \wedge \nicefrac{\ell}{\alpha}\right)}\right),
\end{equation}
and in particular, for \(\Upsilon(\nu_1, \nu_2)\) and \(\chi (\nu_2)\), we have
\begin{equation}\label{eq:rates:Upsilon2}
    1 - \Upsilon(\nu_1, \nu_2) = O(1)\,,
\end{equation}

\begin{equation}\label{eq:rates:chi}
    \chi (\nu_2) = n^{-q}O\left(\nu_2^{-1-\nicefrac{1}{\alpha}}\right)\,.
\end{equation}



\section{Proofs on additional non-asymptotic deterministic equivalents}
\label{app:proof_non-asy_results}

In this section, we aim to generalize \cref{eq:det_equiv_phi2_main} for any PSD matrix $\bm A$, i.e.
\begin{equation*}
         \big\vert \Phi_2(\bX;\bA) - \Psi_2 (\mu_* ; \bA) \big\vert \leq \widetilde{\mathcal{O}}(n^{-\frac{1}{2}}) \cdot \Psi_2 (\mu_* ; \bA) \,,
\end{equation*}
that is required to derive our non-asymptotic deterministic equivalence for the bias term of the $\ell_2$ norm.


By introducing a change of variable $\mu_* := \mu_* (\lambda) = \lambda / \lambda_*$, we find that $\mu_*$ satisfies the following fixed-point equation:
\begin{align}\label{eq:det_equiv_fixed_point_mu_star}
    \mu_* = \frac{n}{1 + \Tr(\bSigma (\mu_* \bSigma + \lambda)^{-1} )}.
\end{align}
We define \(\bt\) and \(\bT\) as follows
\[
    \bt = \bSigma^{-\nicefrac{1}{2}}\bx\,, \quad \bT = \bX\bSigma^{-\nicefrac{1}{2}}\,. 
\]
And the following resolvents are also defined 
\[
    \bR := (\bX^\sT \bX + \lambda)^{-1}\,, \quad \overline{\bR} := (\mu_*\bSigma + \lambda)^{-1}\,, \quad 
    \bM := \bSigma^{\nicefrac{1}{2}} \bR \bSigma^{\nicefrac{1}{2}}\,, \quad \overline{\bM} := \bSigma^{\nicefrac{1}{2}} \overline{\bR} \bSigma^{\nicefrac{1}{2}}.
\]

Since the proof relies on a leave-one-out argument, we define \(\bX_- \in \mathbb{R}^{(n-1) \times d}\) as the data matrix obtained by removing one data. We also introduce the associated resolvent and rescaled resolvent:
\[
    \bR_- := (\bX_-^\sT \bX_- + \lambda)^{-1}\,, \quad
    \overline{\bR}_- :=\left(\frac{n}{1+\kappa}\bSigma + \lambda\right)^{-1}\,, \quad \bM_- := \bSigma^{\nicefrac{1}{2}} \bR_- \bSigma^{\nicefrac{1}{2}}\,, \quad \overline{\bM}_- := \bSigma^{\nicefrac{1}{2}} \overline{\bR}_- \bSigma^{\nicefrac{1}{2}}\,,
\]
where \(\kappa = \E[\Tr(\bM_-)]\).

For the sake of narrative convenience, we introduce a functional used in \cite{misiakiewicz2024non}
\[
\Psi_1(\mu_*; \bA) := \Tr(\bA\bSigma(\mu_*\bSigma + \lambda)^{-1})\,.
\]

Next, we give the proof of \cref{eq:det_equiv_phi2_main}. We consider the functional
\[
\Phi_2 (\bX ; \bA) = \Tr(\bA \bSigma^{-\nicefrac{1}{2}} \bX^\sT \bX ( \bX^\sT \bX + \lambda)^{-1}\bSigma^{\nicefrac{1}{2}} ) = \Tr( \bA \bT^\sT \bT \bM)  .
\]
{\bf Remark:} Note that, to align more closely with the proof in \cite{misiakiewicz2024non}, the \(\Phi_2 (\bX; \bA)\) defined here differs slightly from the \(\Phi_2 (\bX; \bA, \lambda)\) in \cref{eq:det_equiv_phi2_main}. However, the two definitions are equivalent if we take \(\bA\) here as \(\bA = \bSigma^{-\nicefrac{1}{2}} \bB \bSigma^{\nicefrac{1}{2}}\), which recovers the formulation in \cref{eq:det_equiv_phi2_main}.


We show that $\Phi_2 (\bX;\bA)$ is well approximated by the following deterministic equivalent:
\[
\Psi_2(\mu_*; \bA) = \Tr(\bA \mu_* \bSigma ( \mu_* \bSigma + \lambda)^{-1} ) = \Tr(\bA \bSigma ( \bSigma + \lambda_* )^{-1} ).
\]

\begin{theorem}[Deterministic equivalent for $\Tr(\bA \bT^\sT \bT \bM)$]\label{thm_app:det_equiv_TrAZZM}
    Assume the features $\{\bx_i\}_{i\in[n]}$ satisfy \cref{ass:concentrated_LR} with a constant $C_* > 0$. Then for any $D,K>0$, there exist constants $\eta \in (0,1/2)$, $C_{D,K} >0$, and $C_{*,D,K}>0$ ensuring the following property holds. For any $n \geq C_{D,K}$ and  $\lambda >0$, if the following condition is satisfied:
    \begin{equation}\label{eq:conditions_thm4}
\lambda \cdot \rho_\lambda (n) \geq n^{-K}, \qquad  \rho_\lambda (n)^{2} \log^{\frac{3}{2}} (n) \leq K \sqrt{n} ,
    \end{equation}
    then for any PSD matrix $\bA$, with probability at least $1 - n^{-D}$, we have that
    \begin{equation}
         \big\vert \Phi_2(\bX;\bA) - \Psi_2 (\mu_* ; \bA) \big\vert \leq C_{*,D,K} \frac{ \rho_\lambda (n)^{4} \log^{\frac32} (n ) }{\sqrt{n}}   \Psi_2 (\mu_* ; \bA) .
    \end{equation}
\end{theorem}
\noindent{\bf Remark:} \cref{thm_app:det_equiv_TrAZZM} generalizes \cref{eq:det_equiv_phi2_main}. Note that there are some differences between \(\rho_\lambda\) as defined in \cref{eq:rho_lambda} and \(\nu_\lambda\) as defined in \cite{misiakiewicz2024non}. However, based on the discussion in \citet[Appendix A]{defilippis2024dimension}, \(\nu_\lambda\) can be easily adjusted to match \(\rho_\lambda\). Therefore, while we follow the argument in \cite{misiakiewicz2024non}, we use \(\rho_\lambda\) directly in this work to minimize additional notation.


Following the approach outlined in \cite{misiakiewicz2024non}, our proof involves separately bounding the deterministic and martingale components. This is accomplished in the following two propositions.

\begin{proposition}[Deterministic part of $\Tr(\bA \bT^\sT \bT \bM)$]\label{prop:TrAZZM_LOO}
    Under the same assumption as \cref{thm_app:det_equiv_TrAZZM}, there exist constants $C_K$ and $C_{*,K}$, such that for all $n \geq C_K$ and $\lambda >0$ satisfying \cref{eq:conditions_thm4}, and for any PSD matrix $\bA$, we have
    \begin{equation}\label{eq:det_part_TrAZZM}
         \big\vert\E [ \Phi_2(\bX;\bA) ] - \Psi_2 (\mu_* ; \bA) \big\vert \leq C_{*,K} \frac{ \rho_\lambda (n)^{4} }{\sqrt{n}}   \Psi_2 (\mu_* ; \bA) .
    \end{equation}
\end{proposition}

\begin{proposition}[Martingale part of $\Tr(\bA \bT^\sT \bT \bM)$] \label{prop:TrAZZM_martingale}
    Under the same assumption as \cref{thm_app:det_equiv_TrAZZM}, there exist constants $C_{K,D}$ and $C_{*,D,K}$, such that for all $n \geq C_{K,D}$ and  $\lambda >0$ satisfying \cref{eq:conditions_thm4}, and for any PSD matrix $\bA$, we have with probability at least $1 -n^{-D}$ that
    \begin{equation}\label{eq:mart_part_TrAZZM}
         \big\vert \Phi_2(\bX;\bA)  - \E [ \Phi_2(\bX;\bA) ]  \big\vert \leq C_{*,D,K} \frac{ \rho_\lambda (n)^{3} \log^{\frac32} (n)}{\sqrt{n}} \Psi_2 (\mu_* ; \bA) .
    \end{equation}
\end{proposition}

Theorem \ref{thm_app:det_equiv_TrAZZM} is obtained by combining the bounds \eqref{eq:det_part_TrAZZM} and \eqref{eq:mart_part_TrAZZM}. Next, we prove the two propositions above separately.


\begin{proof}[Proof of Proposition \ref{prop:TrAZZM_LOO}]
First, by Sherman-Morrison identity 
\[
    \bM = \bM_- - \frac{\bM_-\bt\bt^\sT\bM_-}{1+\bt^\sT\bM_-\bt}\,,\quad \text{and} \quad \bM\bt = \frac{\bM_-\bt}{1+\bt^\sT\bM_-\bt}\,,
\]
we decompose $\E[\Phi_2(\bX;\bA)]$ as 

\[
\begin{aligned}
\E \left[ \Tr( \bA \bT^\sT \bT \bM) \right] =&~ n\E \left[ \frac{\bt^\sT \bM_- \bA \bt}{1 + S}\right] \\
= &~ n\frac{\E[\Tr ( \bA \bM_- )]}{1 + \kappa} + n\E \left[ \frac{\kappa - S}{(1+\kappa)(1 + S)} \bt^\sT \bM_- \bA \bt\right],
\end{aligned}
\]
where we denoted $S = \bt^\sT \bM_- \bt$. Therefore, bounding the following two terms is sufficient
\begin{equation}\label{eq:decompo_deterministic_2}
\begin{aligned}
    \left|\E[\Phi_2(\bX;\bA)] -\Psi_2(\mu_*;\bA)\right| \le&~
    \left|\frac{n\E[\Tr (  \bA \bM_- )]}{1 + \kappa} - \Psi_2( \mu_*;\bA)\right|+ 
    \left| n\E \left[ \frac{\kappa - S}{(1+\kappa)(1 + S)} \bt^\sT \bM_- \bA \bt\right] \right|.
\end{aligned}
\end{equation}


For the first term,  recall that $\tmu_*$ is the solution of the equation \eqref{eq:det_equiv_fixed_point_mu_star} where we replaced $n$ by $n-1$, and $\tmu_- := n/(1+ \kappa)$. By \citet[Proposition 2]{misiakiewicz2024non}, we have 
\[
\left| \E[\Tr (  \bA \bM_- )] - \Psi_1 (\tmu_* ; \bA ) \right| \leq \cE^{(\sfD)}_{1,n-1}  \cdot \Psi_1 (\tmu_* ; \bA )\,,
\]
where $\cE^{(\sfD)}_{1,n-1} = C_{*,K} \frac{ \rho_\lambda (n)^{5/2} }{\sqrt{n-1}}$. For $n \geq C$, we have $\cE_{1,n-1}^{(\sfD)} \leq C \cE_{1,n}^{(\sfD)}$ and by \citet[Lemma 3]{misiakiewicz2024non}, we have
\[
|\Psi_1 (\tmu_* ; \bA ) - \Psi_1 (\mu_* ; \bA )| \leq C \frac{\rho_\lambda(n)}{n} \Psi_1 (\mu_* ; \bA )\,.
\]
Combining the above bounds, we obtain
\[
\left| \E[\Tr (  \bA \bM_- )] - \Psi_1 (\mu_* ; \bA ) \right| \leq \cE^{(\sfD)}_{1,n}  \cdot \Psi_1 (\mu_* ; \bA )\,.
\]
Furthermore, from the proof of \citet[Proposition 4, Claim 3]{misiakiewicz2024non}, we have
\[
\frac{| \mu_* - \tmu_- |}{\tmu_-} \leq C_{*,K} \frac{ \rho_\lambda (n)^{5/2}}{\sqrt{n}}.
\]
Then we conclude that
\begin{align*}
| \mu_* - \tmu_- | &\leq C_{*,K} \frac{ \rho_\lambda (n)^{5/2}}{\sqrt{n}} \cdot \tmu_- \\
&\leq C_{*,K} \frac{ \rho_\lambda (n)^{5/2}}{\sqrt{n}} \cdot \left(1 + C_{*,K} \frac{ \rho_\lambda (n)^{5/2}}{\sqrt{n}}\right)\mu_* \\
&\leq C_{*,K} \frac{ \rho_\lambda (n)^{5/2}}{\sqrt{n}} \cdot \mu_*,    
\end{align*}
where we use condition \eqref{eq:conditions_thm4} in the last inequality.


Combining this inequality with the previous bounds, we obtain
\[
\begin{aligned}
 \left|\frac{n\E[\Tr (\bA \bM_-)]}{1 + \kappa} - \Psi_2( \mu_* ; \bA)\right| =&~ \left|\tmu_-\E[\Tr (  \bA \bM_- )] - \mu_*\Psi_1(\mu_* ;\bA)\right| \\
 \leq&~ \tmu_- \left| \E[\Tr (\bA \bM_-)] - \Psi_1 ( \mu_* ; \bA) \right| + \frac{| \tmu_- - \mu_*|}{\mu_*} \cdot \mu_* \Psi_1 ( \mu_* ; \bA)\\ 
 \leq&~ C \cE_{1,n}^{(\sfD)} \cdot \mu_*\Psi_1 (\mu_* ; \bA)\\
 =&~ C \cE_{1,n}^{(\sfD)} \cdot \Psi_2 (\mu_* ; \bA)\,.\\
\end{aligned}
\]

In the next, we aim to estimate the second term in Eq.~\eqref{eq:decompo_deterministic_2}. Here we can reduce $\bA$ to be a rank-one matrix $\bA:= \bm v \bm v^{\sT}$ following \citet[Eq. (77)]{misiakiewicz2024non}. We simply apply H\"older's inequality and obtain
\begin{align*}
n\left| \E\left[\frac{\kappa- S}{(1+\kappa)(1+S)}\bt^\sT \bM_- \bA \bt\right]\right| =&~ n \E\left[\left|\frac{\kappa- S}{(1+\kappa)(1+S)}\bt^\sT \bM_- \bv \bv^\sT \bt\right|\right]\\
\leq&~ n\mathbb{E}_{\bm M_-} \left[ \E_{\bt}\left[(\kappa - S)^2 \right]^{\nicefrac{1}{2}} \E_{\bt}\left[(\bt^\sT \bM_- \bv \bv^\sT \bt)^2\right]^{\nicefrac{1}{2}} \right]\\
\leq&~ n\mathbb{E}_{\bm M_-} \left[ \E_{\bt}\left[(\kappa - S)^2 \right]\right]^{\nicefrac{1}{2}} \E_{\bM_-}\left[\E_{\bt}\left[(\bt^\sT \bM_- \bv \bv^\sT \bt)^2\right] \right]^{\nicefrac{1}{2}}\\
\leq&~ n\mathbb{E}_{\bm M_-} \left[ \E_{\bt}\left[(\kappa - S)^2 \right]\right]^{\nicefrac{1}{2}} \E_{\bM_-}\left[\E_{\bt}\left[(\bt^\sT \bM_- \bv)^{4}\right]^{\nicefrac{1}{2}} \E_\bt\left[(\bv^\sT \bt)^4\right]^{\nicefrac{1}{2}} \right]^{\nicefrac{1}{2}}.
\end{align*}
Each of these terms can be bounded, according to the proof of \citet[Proposition 2]{misiakiewicz2024non}, for the first term, we get
\[
 \E_{\bM_-} \left[ \E_{\bt} \left[ ( \bt^\sT \bM_- \bt - \kappa)^2\right] \right]^{1/2 } \leq  C_{*,K} \frac{ \rho_\lambda (n)}{\sqrt{n}} .
\]

For the second term, first according to \citet[Lemma 2]{misiakiewicz2024non}, we have
\[
\E_\bt \left[(\bt^\sT \bM_- \bv)^4\right]^{\nicefrac{1}{2}} \leq C_{*,K} \bv^\sT \bM_-^2 \bv \,,
\]
\[
\E_\bt \left[(\bv^\sT \bt)^4\right]^{\nicefrac{1}{2}} \leq C_{*,K} \bv^\sT \bv \,.
\]
Thus we have
\[
\begin{aligned}
\E_{\bM_-}\left[\E_{\bt}\left[(\bt^\sT \bM_- \bv)^{4}\right]^{\nicefrac{1}{2}} \E_\bt\left[(\bv^\sT \bt)^4\right]^{\nicefrac{1}{2}} \right]^{\nicefrac{1}{2}}
&\leq \E_{\bM_-}\left[C_{*,K} \bv^\sT \bM_-^2 \bv \bv^\sT \bv \right]^{\nicefrac{1}{2}}\\
&= C_{*,K} \E_{\bM_-}\left[\Tr(\bA\bM_-^2 \bA) \right]^{\nicefrac{1}{2}}.
\end{aligned}
\]
Then according to \citet[Lemma 4.(b)]{misiakiewicz2024non}, we have
\[
\E_{\bM_-}\left[\Tr(\bA\bM_-^2 \bA) \right] \leq C_{*,K}\rho^2_{\lambda}(n)\Tr(\bA \obM_-^2 \bA) = C_{*,K}\rho^2_{\lambda}(n)\Tr(\bA \obM_-)^2,
\]
where the last inequality holds due to $\bA\obM_-$ being a rank-1 matrix. Combining the bounds for the second term, we have
\[
\begin{aligned}
\E_{\bM_-}\left[\E_{\bt}\left[(\bt^\sT \bM_- \bv)^{4}\right]^{\nicefrac{1}{2}} \E_\bt\left[(\bv^\sT \bt)^4\right]^{\nicefrac{1}{2}} \right]^{\nicefrac{1}{2}} \leq  C_{*,K}\rho_{\lambda}(n)\Tr(\bA\obM_-) \leq C_{*,K}\rho_{\lambda}^2(n)\Tr(\bA\obM).
\end{aligned}
\]
By combining the above bounds for the first and second term, we have
\[
\begin{aligned}
n\left| \E\left[\frac{\kappa- S}{(1+\kappa)(1+S)}\bT^\sT \bM_- \bA \bT\right]\right| &\leq C_{*,K} \frac{ \rho_\lambda^3 (n) }{\sqrt{n}} n\Tr(\bA\obM)\\
&\leq C_{*,K} \frac{ \rho_\lambda^4 (n) }{\sqrt{n}} \mu_*\Tr(\bA\obM),
\end{aligned}
\]
where we use $\mu_* = \frac{n}{1+\Tr(\obM)} \geq \frac{n}{2\rho_\lambda(n)}$ according to \citet[Lemma 3]{misiakiewicz2024non} in the last inequality.

Combining the above bounds concludes the proof.
\end{proof}


\begin{proof}[Proof of Proposition \cref{prop:TrAZZM_martingale}]
The martingale argument follows a similar approach to the proofs of \citet[Propositions 3 and 5]{misiakiewicz2024non}. The key remaining steps are to adjust Step 2 in \citet[Proposition 3]{misiakiewicz2024non} and establish high-probability bounds for each term in the martingale difference sequence.

We rewrite this term as a martingale difference sequence
\[
\begin{aligned}
 S_n := \Tr(\bA \bT^\sT \bT \bM) - \E[\Tr(\bA \bT^\sT \bT \bM)] = \sum_{i =1}^n  \left( \E_i - \E_{i-1} \right) \Tr( \bA \bT^\sT \bT \bM) =:\sum_{i = 1}^n \Delta_i\,,
\end{aligned}
\]
where \(\E_i\) is denoted as the expectation over \(\{\bx_{i+1},\cdots,\bx_n\}\).

We show below that $|\Delta_i| \leq R$ with probability at least $1 - n^{-D}$ with
\begin{equation}\label{eq:Rchoice_AMZZM}
R =  C_{*,D,K} \frac{ \rho_\lambda(n)^2 \log(n)}{n} \Psi_2( \mu_* ; \bA).
\end{equation}

For Step 3 and bounding $\E_{i-1}[\Delta_i \ind_{\Delta_i \not\in[-R,R]}] $, observe that with probability at least $1-n^{-D}$, by \citet[Lemma 4.(b)]{misiakiewicz2024non}
\[
\begin{aligned}
\E_{i-1} [ \Delta_i^2]^{\nicefrac{1}{2}} 
&~\leq 
2 \E_{i-1} \left[ \frac{(\bt^\sT \bM_- \bA \bt)^2}{(1+S)^2}\right]^{\nicefrac{1}{2}} \\
&~\leq 
C_{*,D,K} \frac{\rho_\lambda (n)^3\log^{1/2}(n)}{n} \mu_* \Tr(\bA \obM) \\
&~\leq 
C_{*,D,K} \frac{\rho_\lambda (n)^3\log^{1/2}(n)}{n} \Psi_2 (\mu_* ; \bA).
\end{aligned}
\]

We establish a high-probability bound for \(\Delta_i\) by first decomposing it and strategically adding and subtracting carefully chosen terms. Observing that
\[
\Delta_i = \left( \E_i - \E_{i-1} \right) \Tr( \bA \bT^\sT \bT \bM) = \left( \E_i - \E_{i-1} \right) \left( \Tr( \bA \bT^\sT \bT \bM) - \Tr( \bA \bT_i^\sT \bT_i \bM_i)\right)  ,
\]
where $\bM_i$ is the rescaled resolvent removes $\bx_i$, and we used that $\E_i\left[\bA\bT_i^\sT \bT_i \bM_i\right] = \E_{i-1}\left[\bA\bT_i^\sT \bT_i \bM_i\right]$, and we'll write (recall that $S_i = \bt_i^\sT \bM_i \bt_i$)
\begin{align*}
    \Tr(\bA \bT^\sT \bT \bM ) -\Tr(\bA \bT_i^\sT \bT_i \bM_i )
    =&~
    \Tr(\bA (\bt_i \bt_i^\sT + \bT_i^\sT \bT_i) \bM ) -\Tr(\bA \bT_i^\sT \bT_i \bM_i )\\
    =&~ 
    \bt_i^\sT \bM \bA\bt_i +   \Tr( \bA \bT_i^\sT \bT_i \bM)
    -\Tr( \bA \bT_i^\sT\bT_i \bM_i)\\
    =&~ 
    \frac1{(1+S_i)} \left\{ \bt_i^\sT \bM_i \bA \bt_i -   \Tr(\bA \bT_i^\sT \bT_i \bM_i \bt_i\bt_i^\sT \bM_i) \right\}\\
    =&~ 
    \frac1{(1+S_i)} \Tr(\bt_i \bt_i^\sT \bM_i \bA (\id - \bT_i^\sT \bT_i \bM_i)).
\end{align*}
Observing that 
\begin{equation*}
  \id - \bT_i^\sT\bT_i \bM_i = \lambda \bSigma^{-1} \bM_i,
\end{equation*}
we can write for $j\in\{i-1,i\}$, with probability at least $1 - n^{-D}$,
\begin{align*}
\left|\E_{j} \left[
\frac1{(1+S_i)} \Tr( \bt_i\bt_i^\sT \bM_i \bA (\id -\bT_i^\sT\bT_i \bM_i ))
\right]\right| 
\le &
\lambda \E_{j} \left[ | \bt_i^\sT \bM_i \bA \bSigma^{-1} \bM_i \bt_i | \right]\\ 
\le&~
\E_{j} \left[ | \bt_i^\sT \bM_i \bA \bt_i | \right]\\
\le&~
C_{*,D}  \log(n) \E_{j} \left[ \Tr(\bA\bM_i) \right]\\
\leq&~ C_{*,D}  \rho_\lambda(n)  \log(n) \Tr(\bA \obM)\\
\leq&~ C_{*,D} \frac{ \rho_\lambda(n)^2  \log(n)}{n} \mu_*\Tr(\bA\obM)\\
=&~ C_{*,D} \frac{ \rho_\lambda(n)^2  \log(n)}{n} \Psi_2 (\mu_* ; \bA),
\end{align*}
where we used that $\bM_i \preceq \bSigma / \lambda$ by definition in the second inequality, \citet[Lemma 4.(b)]{misiakiewicz2024non} in the fourth inequality, and $\mu_* = \frac{n}{1+\Tr(\obM)} \geq \frac{n}{2\rho_\lambda(n)}$ in the last inequality.

Applying a union bound and adjusting the choice of \(D\), we conclude that with probability at least \(1 - n^{-D}\), the following holds for all \(i \in [n]\):
\[
| \Delta_i | \leq C_{*,D,K} \frac{ \rho_\lambda (n)^2 \log (n) }{n} \Psi_2 (\mu_* ; \bA)\,.
\]
\end{proof}




\section{Proofs for ridge regression}

In this section, we provide the proof of deterministic equivalence for ridge regression in both the asymptotic (\cref{app:asy_deter_equiv_lr}) and non-asymptotic (\cref{app:nonasy_deter_equiv_lr}) settings. Additionally, we provide the proof of the relationship between test risk and the $\ell_2$ norm given in the main text, as detailed in \cref{app:relationship}.


\subsection{Asymptotic deterministic equivalence for ridge regression}
\label{app:asy_deter_equiv_lr}

In this section, we establish the asymptotic approximation guarantees for linear regression, focusing on the relationships between the $\ell_2$ norm of the estimator and its deterministic equivalent.
These results can be recovered by our non-asymptotic results, but we put them here just for completeness.

Before presenting the results on deterministic equivalence for ridge regression and their proofs, we begin by introducing a couple of useful corollaries from \cref{prop:spectral,prop:spectralK}.

\begin{corollary}
\label{prop:spectral2}
    Under the same condition of \cref{prop:spectral}, we have
    \begin{align}\label{eq:trA3}
        \Tr ( \bA \bX^\sT \bX ( \bX^\sT \bX + \lambda )^{-2}) \sim&~ \frac{\Tr(\bA\bSigma(\bSigma + \lambda_*\id)^{-2})}{n - {\rm df}_2(\lambda_*)}\,.
        \end{align}
        Specifically, if $\bA = \bSigma$, we have
        \begin{align}\label{eq:trS3}
        \Tr ( \bSigma \bX^\sT \bX ( \bX^\sT \bX + \lambda )^{-2}) \sim&~ \frac{{\rm df}_2(\lambda_*)}{n - {\rm df}_2(\lambda_*)}\,.
    \end{align}
\end{corollary}

\begin{corollary}
\label{prop:spectralK2}
Under the same condition of \cref{prop:spectralK}, we have
\begin{align}
\label{eq:trA3K}
\Tr ( \bA \bT^\sT ( \bT \bSigma \bT^\sT + \lambda )^{-2} \bT) \sim&~ \frac{\Tr ( \bA ( \bSigma + \lambda_* )^{-2})}{ n -  {\rm df}_2(\lambda_*) }\,.
\end{align}
\end{corollary}

Using the equation 
\[
\Tr ( \bA \bX^\sT \bX ( \bX^\sT \bX + \lambda )^{-2}) = \frac{1}{\lambda} \left( \Tr ( \bA \bX^\sT \bX ( \bX^\sT \bX + \lambda )^{-1}) - \Tr ( \bA (\bX^\sT \bX)^2 ( \bX^\sT \bX + \lambda )^{-2}) \right)\,,
\] 
we can directly obtain \cref{prop:spectral2,prop:spectralK2} from \cref{prop:spectral,prop:spectralK}

After introduce the two corollaries above, we first give the proof of the bias-variance decomposition in \cref{lemma:biasvariance}.

\begin{proof}[Proof of \cref{lemma:biasvariance}]
Here we give the bias-variance decomposition of $\E_{\varepsilon}\|\hat{\bbeta}\|_2^2$. The formulation of $\E_{\varepsilon}\|\hat{\bbeta}\|_2^2$ is given by
\[
\begin{aligned}
     \E_{\varepsilon}\|\hat{\bbeta}\|_2^2 =\|\left( \bX^\sT \bX + \lambda \id \right)^{-1} \bX^\sT \by \|_2^2\,,
\end{aligned}
\]
which can be decomposed as
\[
\begin{aligned}
    \E_{\varepsilon}\|\hat{\bbeta}\|_2^2 =&~ \E_{\varepsilon}\|\left( \bX^\sT \bX + \lambda \id \right)^{-1} \bX^\sT (\bX\bbeta_* + \bm\varepsilon) \|_2^2\\
    =&~ \|\left( \bX^\sT \bX + \lambda \id \right)^{-1} \bX^\sT \bX\bbeta_* \|_2^2 + \E_{\varepsilon}\|\left( \bX^\sT \bX + \lambda \id \right)^{-1} \bX^\sT \bm\varepsilon \|_2^2\\
    =&~\<\bbeta_*, (\bX^\sT\bX)^2(\bX^\sT\bX + \lambda\id)^{-2}\bbeta_*\> + \sigma^2\Tr(\bX^\sT\bX(\bX^\sT\bX + \lambda\id)^{-2})\\
    =:&~ \mathcal{B}_{\mathcal{N},\lambda}^{\tt LS} + \mathcal{V}_{\mathcal{N},\lambda}^{\tt LS}\,.
\end{aligned}
\]
Accordingly, we can see that it shares the similar spirit with the bias-variance decomposition.
\end{proof}

Now we are ready to derive the deterministic equivalence, i.e., $\E_{\varepsilon}\|\hat{\bbeta}\|_2^2$, under the bias-variance decomposition. Our results can handle ridge estimator $\hat{\bbeta}$ in \cref{prop:asy_equiv_norm_LR} and interpolator $\hat{\bbeta}_{\min}$ in \cref{prop:asy_equiv_norm_LR_minnorm}, respectively.
\begin{proposition}[Asymptotic deterministic equivalence of the norm of ridge regression estimator]\label{prop:asy_equiv_norm_LR}
    Given the bias variance decomposition of $\E_{\varepsilon}\|\hat{\bbeta}\|_2^2$ in \cref{lemma:biasvariance}, 
    under \cref{ass:asym}, we have the following asymptotic deterministic equivalents $\mathcal{N}^{\tt LS}_{\lambda}  \sim \sN^{\tt LS}_{\lambda} := \sB_{\sN,\lambda}^{\tt LS} + \sV_{\sN,\lambda}^{\tt LS}$ such that $\mathcal{B}^{\tt LS}_{\mathcal{N},\lambda} \sim \sB_{\sN,\lambda}^{\tt LS}$, $\mathcal{V}^{\tt LS}_{\mathcal{N},\lambda} \sim \sV_{\sN,\lambda}^{\tt LS}$, where $\sB_{\sN,\lambda}^{\tt LS}$ and $\sV_{\sN,\lambda}^{\tt LS}$ are defined by \cref{eq:equiv-linear}.
\end{proposition}


And we present the proof of \cref{prop:asy_equiv_norm_LR} as below.
\begin{proof}[Proof of \cref{prop:asy_equiv_norm_LR}]
We give the asymptotic deterministic equivalents for $\mathcal{B}_{\mathcal{N},\lambda}^{\tt LS}$ and $\mathcal{V}_{\mathcal{N},\lambda}^{\tt LS}$, respectively. For the bias term $\mathcal{B}_{\mathcal{N},\lambda}^{\tt LS}$, we use \cref{eq:trAB1} by taking $\bA = \bbeta_*\bbeta_*^\sT$ and $\bB = \id$ and thus obtain
\[
\begin{aligned}
    \mathcal{B}_{\mathcal{N},\lambda}^{\tt LS} = &~ \<\bbeta_*, (\bX^\sT\bX)^2(\bX^\sT\bX + \lambda\id)^{-2}\bbeta_*\>\\
    = &~ \Tr(\bbeta_*\bbeta_*^\sT(\bX^\sT\bX)^2(\bX^\sT\bX + \lambda\id)^{-2})\\
    \sim &~ \Tr(\bbeta_*\bbeta_*^\sT\bSigma^2(\bSigma + \lambda_*\id)^{-2}) + \lambda_*^2 \Tr(\bbeta_*\bbeta_*^\sT\bSigma(\bSigma + \lambda_*\id)^{-2}) \cdot \Tr(\bSigma(\bSigma + \lambda_*\id)^{-2}) \cdot \frac{1}{n-\Tr(\bSigma^2(\bSigma + \lambda_*\id)^{-2})}\\
    = &~ \<\bbeta_*, \bSigma^2(\bSigma + \lambda_*\id)^{-2}\bbeta_*\> + \frac{\Tr(\bSigma(\bSigma + \lambda_*\id)^{-2})}{n} \cdot \frac{\lambda_*^2 \<\bbeta_*,\bSigma(\bSigma + \lambda_*\id)^{-2}\bbeta_*\>}{1-n^{-1}\Tr(\bSigma^2(\bSigma + \lambda_*\id)^{-2})}\\
    =: &~ \sB_{\sN, \lambda}^{\tt LS}\,.
\end{aligned}
\]
For the variance term $\mathcal{V}_{\mathcal{N}}^{\tt LS}$, we use \cref{eq:trA3} by taking $\bA = \id$ and obtain
\[
\begin{aligned}
    \mathcal{V}_{\mathcal{N}}^{\tt LS} = &~ \sigma^2\Tr(\bX^\sT\bX(\bX^\sT\bX + \lambda\id)^{-2}) \sim \frac{\sigma^2\Tr(\bSigma(\bSigma + \lambda_*\id)^{-2})}{n - \Tr(\bSigma^2(\bSigma + \lambda_*\id)^{-2})} =: \sV_{\sN, \lambda}^{\tt LS}\,.
\end{aligned}
\]
\end{proof}

As discussed in the main text, the asymptotic behavior of $\lambda_*$ differs between the under-parameterized and over-parameterized regimes as $\lambda \to 0$, though the ridge regression estimator $\hat{\bbeta}$ converges to the min-$\ell_2$-norm estimator $\hat{\bbeta}_{\min}$.
To be specific, in the under-parameterized regime, $\lambda_*$ converges to $0$ as $\lambda \to 0$; while in the over-parameterized regime, $\lambda_*$ converges to a constant that admits $\Tr(\bSigma(\bSigma + \lambda_n \id)^{-1}) \sim n$ when $\lambda \to 0$. 
Accordingly, for the minimum $\ell_2$-norm estimator, it is necessary to analyze the two regimes separately.
We have the following results on the characterization of the deterministic equivalence of $\| \hat{\bbeta}_{\min} \|_2$. 

\begin{corollary}[Asymptotic deterministic equivalence of the norm of interpolator]\label{prop:asy_equiv_norm_LR_minnorm}
    Under \cref{ass:asym}, for the minimum $\ell_2$-norm estimator $\hat{\bbeta}_{\min}$, we have the following deterministic equivalence: for the under-parameterized regime ($d<n$), we have
    \[
    \begin{aligned}
        \mathcal{B}^{\tt LS}_{\mathcal{N},0} = \|\bbeta_*\|_2^2\,,\quad \mathcal{V}^{\tt LS}_{\mathcal{N},0} \sim&~ \frac{\sigma^2}{n-d}\Tr(\bSigma^{-1})\,.
    \end{aligned}
    \]
    In the over-parameterized regime ($d>n$), we have
    \[
    \begin{aligned}
        \mathcal{B}^{\tt LS}_{\mathcal{N},0} \sim&~ \<\bbeta_*,\bSigma(\bSigma+\lambda_n\id)^{-1}\bbeta_*\>\,,\\
        \mathcal{V}^{\tt LS}_{\mathcal{N},0} \sim&~ \frac{\sigma^2\Tr(\bSigma(\bSigma+\lambda_n\id)^{-2})}{n-\Tr(\bSigma^2(\bSigma+\lambda_n\id)^{-2})} = \frac{\sigma^2}{\lambda_n}\,,
    \end{aligned}
    \]
    where $\lambda_n$ is defined by $\Tr(\bSigma(\bSigma+\lambda_n\id)^{-1}) \sim n$.
\end{corollary}
\begin{proof}[Proof of \cref{prop:asy_equiv_norm_LR_minnorm}]
We separate the results in the under-parameterized and over-parameterized regimes.

In the under-parameterized regime ($d<n$), for minimum norm estimator $\hat{\bbeta}_{\min}$, we have (for $\bX^\sT\bX$ is invertible)
\[
\begin{aligned}
    \hat{\bbeta}_{\min} = \left(\bX^\sT\bX\right)^{-1}\bX^\sT\by = \left(\bX^\sT\bX\right)^{-1}\bX^\sT(\bX\bbeta_*+\bm\varepsilon) = \bbeta_* + \left(\bX^\sT\bX\right)^{-1}\bX^\sT\bm\varepsilon\,.
\end{aligned}
\]
Accordingly, we can directly obtain the bias-variance decomposition as well as their deterministic equivalents
\[
\begin{aligned}
    \mathcal{B}_{\mathcal{N},0}^{\tt LS} = \|\bbeta_*\|_2^2\,, \quad \mathcal{V}_{\mathcal{N},0}^{\tt LS} = \sigma^2\Tr(\bX^\sT\bX(\bX^\sT\bX)^{-2}) \sim \sigma^2\frac{\Tr(\bSigma^{-1})}{n-d}\,,
\end{aligned}
\]
where we use \cref{eq:trA3} and take $\lambda \to 0$ for the variance term.

In the over-parameterized regime ($d>n$), we take the limit $\lambda \to 0$ within ridge regression and use \cref{prop:asy_equiv_norm_LR}.
Define $\lambda_n$ as $\Tr(\bSigma(\bSigma+\lambda_n\id)^{-1}) \sim n$, we have for the bias term
\[
\begin{aligned}
    \mathcal{B}_{\mathcal{N},0}^{\tt LS} \sim &~ \<\bbeta_*, \bSigma^2(\bSigma + \lambda_n\id)^{-2}\bbeta_*\> + \frac{\Tr(\bSigma(\bSigma + \lambda_n\id)^{-2})}{n} \cdot \frac{\lambda_n^2 \<\bbeta_*,\bSigma(\bSigma + \lambda_n\id)^{-2}\bbeta_*\>}{1-n^{-1}\Tr(\bSigma^2(\bSigma + \lambda_n\id)^{-2})}\\
    =&~ \<\bbeta_*, \bSigma(\bSigma + \lambda_n\id)^{-1}\bbeta_*\> - \lambda_n \<\bbeta_*, \bSigma(\bSigma + \lambda_n\id)^{-2}\bbeta_*\> + \frac{\Tr(\bSigma(\bSigma + \lambda_n\id)^{-2})}{n} \cdot \frac{\lambda_n^2 \<\bbeta_*,\bSigma(\bSigma + \lambda_n\id)^{-2}\bbeta_*\>}{1-n^{-1}\Tr(\bSigma^2(\bSigma + \lambda_n\id)^{-2})}\\
    =&~ \<\bbeta_*, \bSigma(\bSigma + \lambda_n\id)^{-1}\bbeta_*\>\,.
\end{aligned}
\]
For the variance term, we have
\[
\begin{aligned}
    \mathcal{V}_{\mathcal{N},0}^{\tt LS} \sim&~ \frac{\sigma^2\Tr(\bSigma(\bSigma+\lambda_n\id)^{-2})}{n-\Tr(\bSigma^2(\bSigma+\lambda_n\id)^{-2})}\,.
\end{aligned}
\]
Finally we conclude the proof.
\end{proof}


\subsection{Non-asymptotic deterministic equivalence for ridge regression}
\label{app:nonasy_deter_equiv_lr}

The deterministic equivalents for the bias and variance terms of the test risk are given by \citet{misiakiewicz2024non} as 
\[
\begin{aligned}
\sB_{\sR,\lambda}^{\tt LS} :=&~ \frac{\lambda_*^2 \langle \bbeta_*, \bSigma(\bSigma + \lambda_*\id)^{-2} \bbeta_* \rangle}{1 - n^{-1} \Tr\left(\bSigma^2 (\bSigma + \lambda_*\id)^{-2}\right)}\,, \qquad \sV_{\sR,\lambda}^{\tt LS} :=&~ \frac{\sigma_{\varepsilon}^2 \Tr\left(\bSigma^2 (\bSigma + \lambda_*\id)^{-2}\right)}{n - \Tr\left(\bSigma^2 (\bSigma + \lambda_*\id)^{-2}\right)}\,.
\end{aligned}
\]
In this section, we establish the approximation guarantees for linear ridge regression.
Instead of using the power-law assumption \cref{ass:powerlaw} in the main text, we adopt the following weaker assumption.
\begin{assumption}[\citet{defilippis2024dimension}]\label{ass:technical_LR} There exists $C>1$
\[
    \frac{\< \bbeta_*, \bSigma(\bSigma+\lambda_*)^{-1}\bbeta_* \>}{\< \bbeta_*, \bSigma^2(\bSigma+\lambda_*)^{-2}\bbeta_* \>} \leq C\,.
\]
\end{assumption}
\noindent{\bf Remark:}
This assumption holds in many settings of interest, such as power law assumptions like those in \cref{ass:powerlaw}, since under this assumption the numerator and denominator are bounded sums of finite terms. It is a technical assumption used to address the difference between two deterministic equivalents that are needed in our work for norm-based capacity.
In fact, this assumption is used for RFMs in \cite{defilippis2024dimension} as the authors also face with the issue on the difference between two deterministic equivalents.


We generalize \cref{prop:non-asy_equiv_norm_LR} as below.
\begin{theorem}[Deterministic equivalents of the $\ell_2$-norm of the estimator. Full version of \cref{prop:non-asy_equiv_norm_LR}]\label{prop:det_equiv_LR}
    Assume well-behaved data $\{ \bm x_i \}_{i=1}^n$ satisfy \cref{ass:concentrated_LR} and \cref{ass:technical_LR}. Then for any $D,K > 0$, there exist constants $\eta_* \in (0, 1/2)$ and $C_{*,D,K} > 0$ ensuring the following property holds. For any $n \geq C_{*,D,K}$, $\lambda > 0$, if the following condition is satisfied:
    \begin{equation*}
        \lambda \geq n^{-K}\,, \quad \rho_{\lambda}(n)^{5/2} \log^{3/2}(n) \leq K \sqrt{n}\,,
    \end{equation*} 
    then with probability at least $1-n^{-D}$, we have that
    \[
    \begin{aligned}
         \left|\mathcal{B}_{\mathcal{N},\lambda}^{\tt LS} - \sB_{\sN,\lambda}^{\tt LS}\right| \leq&~ C_{x, D, K} \frac{\rho_{\lambda}(n)^6 \log^{3/2}(n)}{\sqrt{n}}\sB_{\sN,\lambda}^{\tt LS}\,,\\
        \left|\mathcal{V}_{\mathcal{N},\lambda}^{\tt LS} - \sV_{\sN,\lambda}^{\tt LS}\right| \leq&~ C_{x, D, K} \frac{\rho_{\lambda}(n)^6 \log^{3/2}(n)}{\sqrt{n}} \sV_{\sN,\lambda}^{\tt LS}\,.
    \end{aligned}
    \]
\end{theorem}

\begin{proof}[Proof of \cref{prop:det_equiv_LR}] 
{\bf Part 1: Deterministic equivalents for the bias term.}

Here we prove the deterministic equivalents of $\mathcal{B}_{\mathcal{N},\lambda}^{\tt LS}$ and $\mathcal{V}_{\mathcal{N},\lambda}^{\tt LS}$. First, we decompose $\mathcal{B}_{\mathcal{N},\lambda}^{\tt LS}$ into
\[
\begin{aligned}
    \mathcal{B}_{\mathcal{N},\lambda}^{\tt LS} &= \Tr\left(\bbeta_*\bbeta_*^\sT\bX^\sT \bX (\bX^\sT \bX + \lambda)^{-1}\right) - \lambda\Tr\left(\bbeta_*\bbeta_*^\sT\bX^\sT \bX (\bX^\sT \bX + \lambda)^{-2}\right),\\
    &= \Phi_2(\bX; \tilde{\bA}_1, \lambda) - n\lambda \Phi_4(\bX; \tilde{\bA}_2, \lambda)\,,
\end{aligned}
\]
where $\tilde{\bA}_1 := \bbeta_*\bbeta_*^\sT$, $\tilde{\bA}_2 := \bSigma^{-1/2}\bbeta_*\bbeta_*^\sT\bSigma^{-1/2}$. 
Therefore, using \cref{thm:main_det_equiv_summary}, with probability at least $1-n^{-D}$, we have
\[
\begin{aligned}
    \left|\Phi_2(\bX; \tilde{\bA}_1, \lambda) - \Psi_1(\lambda_*; \tilde{\bA}_1)\right| &\leq C_{x, D, K} \frac{\rho_{\lambda}(n)^{5/2} \log^{3/2}(n)}{\sqrt{n}} \Psi_1(\lambda_*; \tilde{\bA}_1)\,,\\
    \left|n\lambda\Phi_4(\bX; \tilde{\bA}_2, \lambda) - n\lambda\Psi_2(\lambda_*; \tilde{\bA}_2)\right| &\leq C_{x, D, K} \frac{\rho_{\lambda}(n)^6 \log^{3/2}(n)}{\sqrt{n}} n\lambda\Psi_2(\lambda_*; \tilde{\bA}_2)\,.
\end{aligned}
\]
Combining the above bounds, we deduce that
\[
\begin{aligned}
    \left|\mathcal{B}_{\mathcal{N},\lambda}^{\tt LS} - \left(\Psi_1(\lambda_*; \tilde{\bA}_1) - n\lambda\Psi_2(\lambda_*; \tilde{\bA}_2)\right)\right| \leq C_{x, D, K} \frac{\rho_{\lambda}(n)^6 \log^{3/2}(n)}{\sqrt{n}}\left(\Psi_1(\lambda_*; \tilde{\bA}_1)+n\lambda\Psi_2(\lambda_*; \tilde{\bA}_2)\right).
\end{aligned}
\]
Note that
\[
\begin{aligned}
    \Psi_1(\lambda_*; \tilde{\bA}_1) - n\lambda\Psi_2(\lambda_*; \tilde{\bA}_2) = \sB_{\sN,\lambda}^{\tt LS}\,.    
\end{aligned}
\]
For $n\lambda\Psi_2(\lambda_*; \tilde{\bA}_2)$, recall that $\Psi_2(\lambda_*; \bA) := \frac{1}{n} \frac{\Tr(\bA \bSigma^2 (\bSigma + \lambda_*\id)^{-2})}{n - \Tr(\bSigma^2 (\bSigma + \lambda_*\id)^{-2})}$, and according to \cref{def:effective_regularization} 
and \cref{ass:technical_LR}, we have
\[
\begin{aligned}
    n\lambda\Psi_2(\lambda_*; \tilde{\bA}_2) =&~ \lambda\frac{\Tr(\bbeta_*\bbeta_*^\sT\bSigma (\bSigma + \lambda_*\id)^{-2})}{n - \Tr(\bSigma^2 (\bSigma + \lambda_*\id)^{-2})}\\
    \leq&~ \lambda_*\Tr(\bbeta_*\bbeta_*^\sT\bSigma (\bSigma + \lambda_*\id)^{-2})\\
    =&~ \Tr(\bbeta_*\bbeta_*^\sT \bSigma (\bSigma + \lambda_*\id)^{-1}) - \Tr(\bbeta_*\bbeta_*^\sT\bSigma^2 (\bSigma + \lambda_*\id)^{-2})\\
    \leq&~ \left(1-\frac{1}{C}\right) \Tr(\bbeta_*\bbeta_*^\sT\bSigma(\bSigma+\lambda_*)^{-1})\,,
\end{aligned}
\]
and therefore
\[
\begin{aligned}
    \Psi_1(\lambda_*; \tilde{\bA}_1)+n\lambda\Psi_2(\lambda_*; \tilde{\bA}_2) \leq&~ \left(2-\frac{1}{C}\right) \Tr(\bbeta_*\bbeta_*^\sT\bSigma(\bSigma+\lambda_*)^{-1})\\
    \leq&~ \left(2C-1\right)\frac{1}{C}\Tr(\bbeta_*\bbeta_*^\sT\bSigma(\bSigma+\lambda_*)^{-1})\\
    \leq&~ \left(2C-1\right)\left(\Psi_1(\lambda_*; \tilde{\bA}_1)-n\lambda\Psi_2(\lambda_*; \tilde{\bA}_2)\right).
\end{aligned}
\]
Then we conclude that
\[
    \left|\mathcal{B}_{\mathcal{N},\lambda}^{\tt LS} - \sB_{\sN,\lambda}^{\tt LS}\right| \leq C_{x, D, K} \frac{\rho_{\lambda}(n)^6 \log^{3/2}(n)}{\sqrt{n}}\sB_{\sN,\lambda}^{\tt LS},
\]
with probability at least $1-n^{-D}$.

{\bf Part 2: Deterministic equivalents for the variance term.} Next, we prove the deterministic equivalent of $\mathcal{V}_{\mathcal{N},\lambda}^{\tt LS}$. First, note that $\mathcal{V}_{\mathcal{N},\lambda}^{\tt LS}$ can be written in terms of the functional $\Phi_4(\bX; \bA, \lambda)$ defined in \cref{eq:Phi_4}
\[
    \mathcal{V}_{\mathcal{N},\lambda}^{\tt LS} = n\sigma_{\varepsilon}^2\Phi_4(\bX; \bSigma^{-1}, \lambda)\,.
\]
Thus, under the assumptions, we can apply \cref{thm:main_det_equiv_summary} to obtain that with probability at least $1-n^{-D}$
\[
\left|n\sigma_{\varepsilon}^2\Phi_4(\bX; \bSigma^{-1}, \lambda) - n\sigma_{\varepsilon}^2\Psi_2(\lambda_*; \bSigma^{-1})\right| \leq C_{x, D, K} \frac{\rho_{\lambda}(n)^6 \log^{3/2}(n)}{\sqrt{n}} n\sigma_{\varepsilon}^2\Psi_2(\lambda_*; \bSigma^{-1})\,.
\]
Recall that $\Psi_2(\lambda_*; \bA) := \frac{1}{n} \frac{\Tr(\bA \bSigma^2 (\bSigma + \lambda_*\id)^{-2})}{n - \Tr(\bSigma^2 (\bSigma + \lambda_*\id)^{-2})}$, then we have
\[
\left|\mathcal{V}_{\mathcal{N},\lambda}^{\tt LS} - \sV_{\sN,\lambda}^{\tt LS}\right| \leq C_{x, D, K} \frac{\rho_{\lambda}(n)^6 \log^{3/2}(n)}{\sqrt{n}} \sV_{\sN,\lambda}^{\tt LS}\,,
\]
with probability at least $1-n^{-D}$.
\end{proof}


\subsection{Proofs on relationship between the test risk and \texorpdfstring{$\ell_2$}{L2} norm of ridge regression estimator}
\label{app:relationship}

In this section, we will prove the relationship between the test risk and $\ell_2$ norm of the ridge regression estimator that we give in the main text.



\label{proof:linear:relation}
\begin{proof}[Proof of \cref{prop:relation_id}]
According to the formulation of $\sB_{\sN,\lambda}^{\tt LS}$ and $\sV_{\sN,\lambda}^{\tt LS}$ in \cref{eq:equiv-linear}, for $\bSigma=\id_d$, we have
\[
    \sB_{\sN,\lambda}^{\tt LS} = \frac{1}{(1+\lambda_*)^2}\|\bbeta_*\|_2^2 + \frac{d}{n(1+\lambda_*)^2} \cdot \frac{\lambda_*^2 \frac{1}{(1+\lambda_*)^2}\|\bbeta_*\|_2^2}{1-\frac{d}{n(1+\lambda_*)^2}} \,, \quad \sV_{\sN,\lambda}^{\tt LS} = \frac{\sigma^2\frac{d}{(1+\lambda_*)^2}}{n-\frac{d}{(1+\lambda_*)^2}}\,,
\]
\[
    \sN_{\lambda}^{\tt LS} = \frac{d}{(1+\lambda_*)^2}\|\bbeta_*\|_2^2 + \frac{d}{n(1+\lambda_*)^2} \cdot \frac{\lambda_*^2 \frac{d}{(1+\lambda_*)^2}\|\bbeta_*\|_2^2}{1-\frac{d}{n(1+\lambda_*)^2}} + \frac{\sigma^2\frac{d}{(1+\lambda_*)^2}}{n-\frac{d}{(1+\lambda_*)^2}}\,,
\]
where $\lambda_*$ admits a closed-form solution
\[
\lambda_*=\frac{d+\lambda-n+\sqrt{4\lambda n + (n-d-\lambda)^2}}{2n} \,.
\]
Recall the formulation $\sB_{\sR,\lambda}^{\tt LS}$ and $\sV_{\sR,\lambda}^{\tt LS}$ (for test risk) in \cref{eq:de_risk}, for $\bSigma=\id_d$, we have
\[
\begin{aligned}
    \sB_{\sR,\lambda}^{\tt LS} = \frac{\lambda_*^2 \frac{1}{(1+\lambda_*)^2}\|\bbeta_*\|_2^2}{1-\frac{d}{n(1+\lambda_*)^2}} \,, \quad \sV_{\sR,\lambda}^{\tt LS} = \frac{\sigma^2\frac{d}{(1+\lambda_*)^2}}{n-\frac{d}{(1+\lambda_*)^2}}\,, \quad
    \sR_{\lambda}^{\tt LS} = \frac{\lambda_*^2 \frac{d}{(1+\lambda_*)^2}\|\bbeta_*\|_2^2}{1-\frac{d}{n(1+\lambda_*)^2}} + \frac{\sigma^2\frac{d}{(1+\lambda_*)^2}}{n-\frac{d}{(1+\lambda_*)^2}}\,.
\end{aligned}
\]
Accordingly, to establish the relationship between $\sR_{\lambda}^{\tt LS}$ and $\sN_{\lambda}^{\tt LS}$, we combine their formulation and eliminate $n$ to obtain\footnote{Due to the complexity of the calculations, we use Mathematica Wolfram to eliminate $n$. The same approach is applied later whenever $n$ or $p$ elimination is required.}
\[
\begin{aligned}
    2( (\sR^{\tt LS}_{\lambda} - \sN^{\tt LS}_{\lambda})^2 - \|\bbeta_*\|_2^4 ) d \sigma^2 =&~ (\|\bbeta_*\|_2^2 - \sR^{\tt LS}_{\lambda} - \sN^{\tt LS}_{\lambda})(\|\bbeta_*\|_2^2 + \sR^{\tt LS}_{\lambda} - \sN^{\tt LS}_{\lambda})^2d\\
    &~+ 2\|\bbeta_*\|_2^2((\|\bbeta_*\|_2^2 + \sR^{\tt LS}_{\lambda} - \sN^{\tt LS}_{\lambda})^2-4\|\bbeta_*\|_2^2\sR^{\tt LS}_{\lambda} ) \lambda\,.
\end{aligned}
\]
\end{proof}

\begin{proof}[Proof of \cref{prop:relation_minnorm_id}]
According to \cref{prop:asy_equiv_error_LR_minnorm} and \cref{prop:asy_equiv_norm_LR_minnorm}, for minimum $\ell_2$-norm estimator and $\bSigma = \id_d$, for the under-parameterized regime ($d<n$), we have
\[
\begin{aligned}
\sB_{\sR,0}^{\tt LS} = 0\,, \quad \sV_{\sR,0}^{\tt LS} = \frac{\sigma^2d}{n-d}\,; \quad \quad \sB_{\sN,0}^{\tt LS} = \|\bbeta_*\|_2^2\,, \quad \sV_{\sN,0}^{\tt LS} = \frac{\sigma^2d}{n-d}\,. 
\end{aligned}
\]
From these expressions, we can conclude that
\[
\begin{aligned}
    \sR_{0}^{\tt LS} = \sB_{\sR,0}^{\tt LS} + \sV_{\sR,0}^{\tt LS} = \frac{\sigma^2d}{n-d}\,; \quad \quad \sN_{0}^{\tt LS} = \sB_{\sN,0}^{\tt LS} + \sV_{\sN,0}^{\tt LS} = \|\bbeta_*\|_2^2 + \frac{\sigma^2d}{n-d}\,. 
\end{aligned}
\]
Finally, in the under-parameterized regime, it follows that
\begin{equation*}
 \sR_{0}^{\tt LS} = \sN_{0}^{\tt LS} - \|\bbeta_*\|_2^2\,.   
\end{equation*}

In the over-parameterized regime ($d>n$), the effective regularization $\lambda_*$ will have an explicit formulation as $\lambda_* = \frac{d-n}{n}$, thus for the bias and variance of the test error, we have
\[
\begin{aligned}
\sB_{\sR,0}^{\tt LS} = \frac{\lambda_n^2\<\bbeta_*,\bSigma(\bSigma+\lambda_n\id)^{-2}\bbeta_*\>}{1-n^{-1}\Tr(\bSigma^2(\bSigma+\lambda_n)^{-2})} = \frac{\lambda_n^2 \frac{1}{(1 + \lambda_n)^2}\|\bbeta_*\|_2^2}{1 - \frac{1}{n}\frac{d}{(1+\lambda_n)^2}} = \|\bbeta_*\|_2^2\frac{d-n}{d}\,,
\end{aligned}
\]
\[
\begin{aligned}
\sV_{\sR,0}^{\tt LS} = \frac{\sigma^2\Tr(\bSigma^2(\bSigma+\lambda_n\id)^{-2})}{n-\Tr(\bSigma^2(\bSigma+\lambda_n\id)^{-2})} = \frac{\sigma^2\frac{d}{(1+\lambda_n)^2}}{n-\frac{d}{(1+\lambda_n)^2}} = \sigma^2\frac{n}{d-n}\,,
\end{aligned}
\]
and combining the bias and variance, we have
\begin{align}\label{eq:r_under}
\sR_{0}^{\tt LS} = \sB_{\sR,0}^{\tt LS} + \sV_{\sR,0}^{\tt LS} = \|\bbeta_*\|_2^2\frac{d-n}{d} + \sigma^2\frac{n}{d-n}\,.
\end{align}
For the bias and variance of the norm, we have
\[
\begin{aligned}
\sB_{\sN,0}^{\tt LS} = \<\bbeta_*,\bSigma(\bSigma+\lambda_n\id)^{-1}\bbeta_*\> = \frac{1}{1+\lambda_n}\|\bbeta_*\|_2^2 = \|\bbeta_*\|_2^2\frac{n}{d}\,,
\end{aligned}
\]
\[
\begin{aligned}
\sV_{\sN,0}^{\tt LS} = \frac{\sigma\Tr(\bSigma(\bSigma+\lambda_n\id)^{-2})}{n-\Tr(\bSigma^2(\bSigma+\lambda_n\id)^{-2})} = \frac{\sigma^2\frac{d}{(1+\lambda_n)^2}}{n-\frac{d}{(1+\lambda_n)^2}} = \sigma^2\frac{n}{d-n}\,,
\end{aligned}
\]
and combining the bias and variance, we have

\begin{align}\label{eq:n_under}
\sN_{0}^{\tt LS} = \sB_{\sN,0}^{\tt LS} + \sV_{\sN,0}^{\tt LS} = \|\bbeta_*\|_2^2\frac{n}{d} + \sigma^2\frac{n}{d-n}\,.
\end{align}
Finally, combining \cref{eq:r_under} and \cref{eq:n_under}, we eliminate $n$ and thus obtain
\[
\begin{aligned}
    \sR_{0}^{\tt LS} = \sqrt{(\sN^{\tt LS}_0)^2 \!-\! 2(\|\bbeta_*\|_2^2 \!-\! \sigma^2)\sN^{\tt LS}_0 + (\|\bbeta_*\|_2^2 \!+\! \sigma^2)^2} \!-\!\sigma^2\,.
\end{aligned}
\]
By taking the derivative of $\sR_{0}^{\tt LS}$ with respect to $\sN_{0}^{\tt LS}$, we get
\[
\frac{\partial \sR_{0}^{\tt LS}}{\partial \sN_{0}^{\tt LS}} = \frac{\sN_{0}^{\tt LS} - (\|\bbeta_*\|_2^2 - \sigma^2)}{\sqrt{(\sN_{0}^{\tt LS})^2 - 2(\|\bbeta_*\|_2^2 - \sigma^2)\sN_{0}^{\tt LS} + (\|\bbeta_*\|_2^2 + \sigma^2)^2}}\,.
\]
From the derivative function, we observe that $\sR_{0}^{\tt LS}$ decreases monotonically with $\sN_{0}^{\tt LS}$ when $\sN_{0}^{\tt LS} < \|\boldsymbol{\beta}_*\|_2^2 - \sigma^2$, and increases monotonically with $\sN_{0}^{\tt LS}$ when $\sN_{0}^{\tt LS} > \|\boldsymbol{\beta}_*\|_2^2 - \sigma^2$.
\end{proof}

\begin{proof}[Proof of \cref{prop:relation_minnorm_underparam}]
According to \cref{prop:asy_equiv_error_LR_minnorm} and \cref{prop:asy_equiv_norm_LR_minnorm}, for minimum $\ell_2$-norm estimator, in the under-parameterized regime ($d<n$), we have
\[
\begin{aligned}
\sB_{\sR,0}^{\tt LS} = 0\,, \quad \sV_{\sR,0}^{\tt LS} = \frac{\sigma^2d}{n-d}\,; \quad \quad \sB_{\sN,0}^{\tt LS} = \|\bbeta_*\|_2^2\,, \quad \sV_{\sN,0}^{\tt LS} = \frac{\sigma^2\Tr(\bSigma^{-1})}{n-d}\,. 
\end{aligned}
\]
From these expressions, we can conclude that
\[
\begin{aligned}
    \sR_{0}^{\tt LS} = \sB_{\sR,0}^{\tt LS} + \sV_{\sR,0}^{\tt LS} = \frac{\sigma^2d}{n-d}\,; \quad \quad \sN_{0}^{\tt LS} = \sB_{\sN,0}^{\tt LS} + \sV_{\sN,0}^{\tt LS} = \|\bbeta_*\|_2^2 + \frac{\sigma^2\Tr(\bSigma^{-1})}{n-d}\,. 
\end{aligned}
\]
Finally, combing the above equation and eliminate \(n\), in the under-parameterized regime, it follows that
\begin{equation}\label{eq:rn_under}
 \sR_{0}^{\tt LS} = \frac{d}{\Tr(\bSigma^{-1})}\left(\sN_{0}^{\tt LS} - \|\bbeta_*\|_2^2\right)\,.   
\end{equation}
\end{proof}


\begin{proof}[Proof of \cref{prop:relation_minnorm_pl}]
In the over-parameterized regime ($d > n$), according to \cref{prop:asy_equiv_error_LR_minnorm} and \cref{prop:asy_equiv_norm_LR_minnorm}, under \cref{ass:powerlaw}, we have
\[
\begin{aligned}
    \sB_{\sR,0}^{\tt LS} =&~ \frac{\lambda_n^2\<\bbeta_*,\bSigma(\bSigma+\lambda_n\id)^{-2}\bbeta_*\>}{1-n^{-1}\Tr(\bSigma^2(\bSigma+\lambda_n\id)^{-2})} = \frac{\lambda_n^2\Tr(\bSigma^{1+\beta}(\bSigma+\lambda_n\id)^{-2})}{1-n^{-1}\Tr(\bSigma^2(\bSigma+\lambda_n\id)^{-2})}\,,\\
    \sV_{\sR,0}^{\tt LS} =&~ \frac{\sigma^2 \Tr\left(\bSigma^2 (\bSigma + \lambda_n\id)^{-2}\right)}{n - \Tr\left(\bSigma^2 (\bSigma + \lambda_n\id)^{-2}\right)}\,,\\
    \sB_{\sN,0}^{\tt LS} =&~ \<\bbeta_*,\bSigma(\bSigma+\lambda_n\id)^{-1}\bbeta_*\> = \Tr(\bSigma^{1+\beta}(\bSigma+\lambda_n\id)^{-1})\,,\\
    \sV_{\sN,0}^{\tt LS} =&~ \frac{\sigma^2 \Tr\left(\bSigma (\bSigma + \lambda_n\id)^{-2}\right)}{n - \Tr\left(\bSigma^2 (\bSigma + \lambda_n\id)^{-2}\right)}\,.
\end{aligned}
\]
To compute these quantities, here we introduce the following continuum approximations to eigensums.
\begin{equation}\label{eq:df1_inter_approx} 
\int_{1}^{d+1} \frac{k^{-\alpha}}{k^{-\alpha} + \lambda_n}\, \mathrm{d}k \leq \Tr(\bSigma(\bSigma+\lambda_n)^{-1}) = \sum_{i=1}^{d} \frac{\sigma_i}{\sigma_i + \lambda_n} \leq 
   \int_{0}^{d} \frac{k^{-\alpha}}{k^{-\alpha} + \lambda_n}\, \mathrm{d}k \,,
\end{equation}
due to the fact that the integrand is non-increasing function of $k$.
Similarly, we also have
\begin{equation}\label{eq:df2_inter_approx}\int_{1}^{d+1} \frac{k^{-2\alpha}}{(k^{-\alpha} + \lambda_n)^2}\, \mathrm{d}k \leq \Tr(\bSigma^2(\bSigma+\lambda_n)^{-2}) = \sum_{i=1}^{d} \frac{\sigma_i^2}{(\sigma_i + \lambda_n)^2} \leq  \int_{0}^{d} \frac{k^{-2\alpha}}{(k^{-\alpha} + \lambda_n)^2}\, \mathrm{d}k \,.
\end{equation}


We consider some special cases that are useful for discussion.
When $\alpha=1$, we have

\begin{equation}\label{eq:df1_inter_approx_alpha1}
   \frac{\log(1+d\lambda_n + \lambda_n) - \log (1+\lambda_n)}{\lambda_n}  \leq \Tr(\bSigma(\bSigma+\lambda_n)^{-1}) \leq  \frac{\log(1+d\lambda_n)}{\lambda_n} \,,
\end{equation}

\begin{equation}\label{eq:df2_inter_approx_alpha1}
\frac{d+1}{\lambda_n d +\lambda_n +1} - \frac{1}{\lambda_n+1} \leq    \Tr(\bSigma^2(\bSigma+\lambda_n)^{-2}) = \sum_{i=1}^{d} \frac{\sigma_i^2}{(\sigma_i + \lambda_n)^2} \leq \frac{d}{1+d\lambda_n}\,.
\end{equation}
Recall that \(\lambda_n\) is defined by \(\Tr(\bSigma(\bSigma + \lambda_n \id)^{-1}) = n\). Using \cref{eq:df1_inter_approx}, we have 
\[
\frac{\log(1 + d\lambda_n)}{\lambda_n} \approx n.
\]
Observe that as \(n \to d\), \(\lambda_n \to 0\), allowing us to apply a Taylor expansion:
\[
\frac{\log(1 + d\lambda_n)}{\lambda_n} \approx \frac{d\lambda_n - \frac{1}{2}(d\lambda_n)^2}{\lambda_n} = d - \frac{1}{2}d^2\lambda_n.
\]
Based on this approximation, \(\lambda_n\) can be expressed as
\[
\lambda_n \approx \frac{2(d - n)}{d^2}.
\]
In the following discussion, we consider the case $n \to d$. Thus, we have the approximation
\[
\Tr(\bSigma(\bSigma+\lambda_n)^{-1}) \approx n\,, \quad \Tr(\bSigma^2(\bSigma+\lambda_n)^{-2}) \approx \frac{d}{1+d\lambda_n}\,.
\]
Then we have
\[
\begin{aligned}
    \sV_{\sR,0}^{\tt LS} =&~ \frac{\sigma^2 \Tr\left(\bSigma^2 (\bSigma + \lambda_n\id)^{-2}\right)}{n - \Tr\left(\bSigma^2 (\bSigma + \lambda_n\id)^{-2}\right)} \approx \frac{\sigma^2\frac{d}{1+d\lambda_n}}{n-\frac{d}{1+d\lambda_n}} = \frac{\sigma^2 d}{n+d(n\lambda_n-1)}\,,\\
    \sV_{\sN,0}^{\tt LS} =&~ \frac{\sigma^2 \Tr\left(\bSigma (\bSigma + \lambda_n\id)^{-2}\right)}{n - \Tr\left(\bSigma^2 (\bSigma + \lambda_n\id)^{-2}\right)} \approx \frac{\sigma^2\frac{1}{\lambda_n}(d - \frac{1}{2}d^2\lambda_n - \frac{d}{1+d\lambda_n})}{n-\frac{d}{1+d\lambda_n}} = \frac{\sigma^2d^2(d\lambda_n-1)}{2(n+d(n\lambda_n-1))}\,.
\end{aligned}
\]
Use these two formulation to eliminate $n$, we obtain
\[
\sV_{\sR, 0}^{\tt LS} \approx \frac{2(\sV_{\sN, 0}^{\tt LS})^2}{d\sV_{\sN, 0}^{\tt LS}-d^2\sigma^2}\,.
\]

Next we discuss the situation under different $\beta$.

For $\beta=0$, we have
\[
\begin{aligned}
    \sB_{\sR,0}^{\tt LS} =&~ \frac{\lambda_n^2\Tr(\bSigma(\bSigma+\lambda_n\id)^{-2})}{1-n^{-1}\Tr(\bSigma^2(\bSigma+\lambda_n\id)^{-2})} \approx \frac{\lambda_n(d - \frac{1}{2}d^2\lambda_n - \frac{d}{1+d\lambda_n})}{1-\frac{d}{n(1+d\lambda_n)}} = n\lambda_n\,,\\
    \sB_{\sN,0}^{\tt LS} =&~ \Tr(\bSigma(\bSigma+\lambda_n\id)^{-1}) \approx d - \frac{1}{2}d^2\lambda_n\,,\\
\end{aligned}
\]
Use these two formulation to eliminate $n$, we obtain
\[
\sB_{\sR,0}^{\tt LS} \approx \frac{2\sB_{\sN, 0}^{\tt LS}(d-\sB_{\sN, 0}^{\tt LS})}{d^2}\,.
\]

For $\beta=1$, we have
\[
\begin{aligned}
    \sB_{\sR,0}^{\tt LS} =&~ \frac{\lambda_n^2\Tr(\bSigma^2(\bSigma+\lambda_n\id)^{-2})}{1-n^{-1}\Tr(\bSigma^2(\bSigma+\lambda_n\id)^{-2})} \approx \frac{\lambda_n^2\frac{d}{1+d\lambda_n}}{1-\frac{d}{n(1+d\lambda_n)}} = \frac{nd\lambda_n^2}{n(1+d\lambda_n)-d}\,,\\
    \sB_{\sN,0}^{\tt LS} =&~ \Tr(\bSigma^2(\bSigma+\lambda_n\id)^{-1}) = \Tr(\bSigma) - \lambda_n\Tr(\bSigma(\bSigma+\lambda_n\id)^{-1}) \approx \Tr(\bSigma) - n\lambda_n\,.\\
\end{aligned}
\]
Use these two formulation to eliminate $n$, we obtain
\[
\sB_{\sR,0}^{\tt LS} \approx \frac{2\sqrt{(\sB_{\sN, 0}^{\tt LS})^2-2\Tr(\bSigma)\sB_{\sN, 0}^{\tt LS}+\Tr(\bSigma)^2}}{\sqrt{d^2+2d^2\sB_{\sN, 0}^{\tt LS}-2d^2\Tr(\bSigma)}} = \frac{2(\sB_{\sN, 0}^{\tt LS} - \Tr(\bSigma))}{d\sqrt{1+2\sB_{\sN, 0}^{\tt LS}-2\Tr(\bSigma)}}\,.
\]

For $\beta=-1$, we need to use another two continuum approximations to eigensums
\begin{equation*}
    \Tr((\bSigma+\lambda_n)^{-1}) = \sum_{i=1}^{d} \frac{1}{\sigma_i + \lambda_n} \approx \int_{0}^{d} \frac{1}{k^{-\alpha} + \lambda_n}\, \mathrm{d}k = \frac{d\lambda_n - \log(1+d\lambda_n)}{\lambda_n^2}\,,
\end{equation*}
\begin{equation*}
    \Tr((\bSigma+\lambda_n)^{-2}) = \sum_{i=1}^{d} \frac{1}{(\sigma_i + \lambda_n)^2} \approx \int_{0}^{d} \frac{1}{(k^{-\alpha} + \lambda_n)^2}\, \mathrm{d}k = \frac{\frac{d\lambda_n(2+d\lambda_n)}{1+d\lambda_n}-2\log(1+d\lambda_n)}{\lambda_n^3}\,.
\end{equation*}
Once again, we apply the Taylor expansion, but this time expanding to the third order
\[
\log(1 + d\lambda_n) \approx d\lambda_n - \frac{1}{2}(d\lambda_n)^2 + \frac{1}{3}(d\lambda_n)^3\,.
\]
Then we have
\begin{equation*}
    \Tr((\bSigma+\lambda_n)^{-1}) \approx \frac{d\lambda_n - \log(1+d\lambda_n)}{\lambda_n^2} \approx \frac{1}{2}d^2-\frac{1}{3}d^3\lambda_n\,,
\end{equation*}
\begin{equation*}
    \Tr((\bSigma+\lambda_n)^{-2}) \approx \frac{\frac{d\lambda_n(2+d\lambda_n)}{1+d\lambda_n}-2\log(1+d\lambda_n)}{\lambda_n^3} = \frac{\frac{1}{3}d^3-\frac{2}{3}d^4\lambda_n}{1+d\lambda_n}\,.
\end{equation*}
Using the approximation sated above, we have
\[
\begin{aligned}
    \sB_{\sR,0}^{\tt LS} =&~ \frac{\lambda_n^2\Tr((\bSigma+\lambda_n\id)^{-2})}{1-n^{-1}\Tr(\bSigma^2(\bSigma+\lambda_n\id)^{-2})} \approx \frac{\lambda_n^2(\nicefrac{(\frac{1}{3}d^3-\frac{2}{3}d^4\lambda_n)}{(1+d\lambda_n)})}{1-\frac{d}{n(1+d\lambda_n)}} \,,\\
    \sB_{\sN,0}^{\tt LS} =&~ \Tr((\bSigma+\lambda_n\id)^{-1}) = \frac{1}{2}d^2-\frac{1}{3}d^3\lambda_n \,.\\
\end{aligned}
\]
Use these two formulation to eliminate $n$, we obtain
\[
\sB_{\sR,0}^{\tt LS} \approx \frac{216 (\sB_{\sN, 0}^{\tt LS})^4 \!-\! 324d^2 (\sB_{\sN, 0}^{\tt LS})^3 \!+\! 126d^4 (\sB_{\sN, 0}^{\tt LS})^2 \!+\! d^6 \sB_{\sN, 0}^{\tt LS} \!-\! 5d^8}{2d^5(6 \sB_{\sN, 0}^{\tt LS}-d^2)}\,.
\]
\end{proof}

Here we present some experimental results to check the relationship between $\sB_{\sR,0}^{\tt LS}$ and $\sB_{\sN,0}^{\tt LS}$, as well as $\sV_{\sR,0}^{\tt LS}$ and $\sV_{\sN,0}^{\tt LS}$, see \cref{fig:linear_regression_power_law}.
We can see that our approximate relationship on variance (see the {\color{red}red} line in \cref{fig:lrpld}) provides the precise estimation.
For the bias (see the left three figures of \cref{fig:linear_regression_power_law}), our approximate relationship is accurate if $\sB_{\sN,0}^{\tt LS}$ is large.






\section{Proofs for random feature ridge regression}\label{app:proof_rf}

In this section, we provide the proof of deterministic equivalence for random feature ridge regression in both the asymptotic (\cref{app:asy_deter_equiv_rf}) and non-asymptotic (\cref{app:nonasy_deter_equiv_rf}) settings. Additionally, we provide the proof of the relationship between test risk and the $\ell_2$ norm given in the main text, as detailed in \cref{app:relationship_rf}.

Though \citet{bach2024high}'s results are for linear regression, we can still deliver the asymptotic results for RFMs, which requires some knowledge from \cref{eq:det_equiv_phi2_main,eq:det_equiv_phi1_main}.

We firstly confirm that \cref{ass:asym} in \cref{app:pre_asy_deter_equiv}, used to derive all asymptotic results, can be replaced by the Hanson-Wright assumption employed in the non-asymptotic analysis.
It is evident that \cref{eq:trA1,eq:trA2} are obtained directly by taking the limits of \cref{eq:det_equiv_phi2_main,eq:det_equiv_phi1_main} as \(n \to \infty\).

Additionally, a key step in the proof of \cref{eq:trAB1,eq:trAB2} in \citet{bach2024high} involves showing that \(\Delta\) is almost surely negligible, where \(\Delta\) is defined as
\[
\Delta = \frac{1}{n} \sum_{i=1}^{n} \frac{\bx_i\bx_i^\sT(\hbSigma_{-i}-z\id)^{-1}-\bSigma(\hbSigma-z\id)^{-1}}{1 + \bx_i^\sT(n\widehat\bSigma_{-i}-nz\id)^{-1}\bx_i}\,,
\]
with \(\hbSigma = \frac{1}{n}\sum_{i=1}^{n}\bx_i\bx_i^\sT\), \(\hbSigma_{-i} = \frac{1}{n}\sum_{j\neq i}\bx_j\bx_j^\sT\), and \(z \in \R\).

In \citet{bach2024high}'s analysis, the negligibility of \(\Delta\) arises from the assumption that the components of \(\bx_i\) follow a sub-Gaussian distribution, which leads to the Hanson-Wright inequality
\[
\mathbb{P} \left[ \left| \bx_i^\sT \bx_i - \mathrm{tr}(\bSigma) \right| \leq c \left( t \|\bSigma\|_{\mathrm{op}} + \sqrt{t} \|\bSigma\|_F \right) \right] \geq 1 - 2e^{-t}.
\]

In this way, \cref{ass:concentrated_LR} is also sufficient to establish the negligibility of \(\Delta\).

After obtain \cref{eq:trA1,eq:trA2} and the negligibility of \(\Delta\), we can follow \citet{bach2024high}'s argument and derive the rest asymptotic deterministic equivalence.

Finally, with these observations, we can eliminate the reliance on \cref{ass:asym} and instead rely solely on \cref{ass:concentrated_LR} to derive all the asymptotic results.


\subsection{Asymptotic deterministic equivalence for random features ridge regression}
\label{app:asy_deter_equiv_rf}

In this section, we establish the asymptotic approximation guarantees for random feature regression in terms of its $\ell_2$-norm based capacity. Before presenting the proof of \cref{prop:asy_equiv_norm_RFRR}, we firstly give the proof of the bias-variance decomposition in \cref{lemma:biasvariance_rf}.

\begin{proof}[Proof of \cref{lemma:biasvariance_rf}]
Here we give the bias-variance decomposition of $\E_{\varepsilon}\|\hat{\ba}\|_2^2$. The formulation of $\E_{\varepsilon}\|\hat{\ba}\|_2^2$ is given by
\[
\E_{\varepsilon}\|\hat{\ba}\|_2^2 = \E_{\varepsilon} \|(\bZ^\sT \bZ + \lambda \id)^{-1} \bZ^\sT \by\|_2^2\,,
\]
which admits a similar bias-variance decomposition
\[
\begin{aligned}
    \E_{\varepsilon}\|\hat{\ba}\|_2^2 =&~ \E_{\varepsilon}\|(\bZ^\sT \bZ + \lambda \id)^{-1} \bZ^\sT (\bG \btheta_*+\bm\varepsilon)\|_2^2\\
    =&~ \|(\bZ^\sT \bZ + \lambda \id)^{-1} \bZ^\sT \bG \btheta_*\|_2^2 + \E_{\varepsilon}\|(\bZ^\sT \bZ + \lambda \id)^{-1} \bZ^\sT \bm\varepsilon\|_2^2\\
    =&~ \<\btheta_*, \bG^\sT \bZ (\bZ^\sT \bZ + \lambda\id)^{-2} \bZ^\sT \bG\btheta_* \> + \sigma^2\Tr\left(\bZ^\sT \bZ(\bZ^\sT \bZ + \lambda\id)^{-2}\right)\\
    =:&~ \mathcal{B}_{\mathcal{N},\lambda}^{\tt RFM} + \mathcal{V}_{\mathcal{N},\lambda}^{\tt RFM}\,.
\end{aligned}
\]
Accordingly, we conclude the proof.
\end{proof}

Now we are ready to present the proof of \cref{prop:asy_equiv_norm_RFRR} as below.

\begin{proof}[Proof of \cref{prop:asy_equiv_norm_RFRR}]
We give the asymptotic deterministic equivalents for the norm from the bias $\mathcal{B}_{\mathcal{N},\lambda}^{\tt RFM}$ and variance $\mathcal{V}_{\mathcal{N},\lambda}^{\tt RFM}$, respectively. We provide asymptotic expansions in two steps, by first considering the deterministic equivalent over $\bG$, and then over $\bF$.

Under \cref{ass:concentrated_RFRR}, we can apply \cref{prop:spectral,prop:spectral2,prop:spectralK,prop:spectralK2} directly in the proof below.

\paragraph{Deterministic equivalent over $\bG$:}
For the bias term, we use \cref{eq:trAB1K} in \cref{prop:spectralK} with $\bT=\bG$, $\bSigma=\bF^\sT\bF$, $\bA=\btheta_*\btheta_*^\sT$ and $\bB=\bF^\sT\bF$ and obtain
\begin{equation}\label{eq:bnrfm}
   \begin{split}
          \mathcal{B}_{\mathcal{N},\lambda}^{\tt RFM} =&~ \<\btheta_*, \bG^\sT \bZ (\bZ^\sT \bZ + \lambda\id)^{-2} \bZ^\sT \bG\btheta_* \>\\
    =&~ \Tr(\btheta_*^\sT \bG^\sT \bZ (\bZ^\sT \bZ + \lambda\id)^{-2} \bZ^\sT \bG\btheta_* )\\
    =&~ p\Tr(\btheta_* \btheta_*^\sT \bG^\sT ( \bG \bF^\sT \bF \bG^\sT + p\lambda\id)^{-1} \bG \bF^\sT \bF \bG^\sT ( \bG \bF^\sT \bF \bG^\sT + p\lambda\id)^{-1} \bG )\\
    \sim&~ p \underbrace{\Tr(\btheta_* \btheta_*^\sT ( \bF^\sT \bF + \nu_1\id)^{-1} \bF^\sT \bF ( \bF^\sT \bF + \nu_1\id)^{-1} )}_{\tt I_1} \\
    &~ + p\nu_1^2 \underbrace{\Tr(\btheta_* \btheta_*^\sT ( \bF^\sT \bF + \nu_1\id)^{-2})}_{:=I_2} \cdot \underbrace{\Tr(\bF^\sT \bF ( \bF^\sT \bF + \nu_1\id)^{-2})}_{:=I_3} \cdot \frac{1}{n-\widehat{\rm df}_2(\nu_1)} \,,
   \end{split} 
\end{equation}
where $\nu_1$ defined by $\nu_1(1-\frac{1}{n}\widehat{\rm df}_1(\nu_1)) \sim \frac{p\lambda}{n}$, $\widehat{\rm df}_1(\nu_1)$ and $\widehat{\rm df}_2(\nu_1)$ are degrees of freedom associated to $\bF^\sT \bF$ in \cref{def:df}.

For the variance term, we use \cref{eq:trA3K} with $\bT=\bG$ in \cref{prop:spectralK}, $\bA=\bF^\sT\bF$, $\bSigma=\bF^\sT\bF$ and obtain
\[
\begin{aligned}
    \mathcal{V}_{\mathcal{N},\lambda}^{\tt RFM} =&~ \sigma^2\Tr\left(\bZ^\sT \bZ(\bZ^\sT \bZ + \lambda\id)^{-2}\right) = \sigma^2\Tr\left(\bZ \bZ^\sT(\bZ \bZ^\sT + \lambda\id)^{-2}\right)\\    =&~\sigma^2p\Tr\left(\bG\bF^\sT\bF\bG^\sT(\bG\bF^\sT\bF\bG^\sT + p\lambda\id)^{-2}\right)\\
    \sim&~\sigma^2p\frac{\Tr(\bF^\sT\bF(\bF^\sT\bF+\nu_1\id)^{-2})}{n-\widehat{\rm df}_2(\nu_1)}\,.
\end{aligned}
\]

\paragraph{Deterministic equivalent over $\bF$:}

In the next, we aim to eliminate the randomness over $\bF$ in \cref{eq:bnrfm} from the bias part.
First our result depends on the asymptotic equivalents for $\widehat{\rm df}_1(\nu_1)$ and $\widehat{\rm df}_2(\nu_1)$. For $\widehat{\rm df}_1(\nu_1)$, we use \cref{eq:trA1} in \cref{prop:spectral} with $\bX=\bF$ and obtain
\[
\begin{aligned}
    \widehat{\rm df}_1(\nu_1) = \Tr(\bF^\sT \bF (\bF^\sT \bF + \nu_1\id)^{-1}) \sim \Tr(\bLambda(\bLambda + \nu_2\id)^{-1})={\rm df}_1(\nu_2)\,,
\end{aligned}
\]
where $\nu_2$ defined by $\nu_2(1-\frac{1}{p}{\rm df}_1(\nu_2)) \sim \frac{\nu_1}{p}$. Hence $\nu_1$ can be defined by $\nu_1(1-\frac{1}{n}{\rm df}_1(\nu_2))\sim\frac{p\lambda}{n}$ from \cref{eq:def_nu}.

For $\widehat{\rm df}_2(\nu_1)$, we use \cref{eq:trAB1} in \cref{prop:spectral} with $\bX=\bF$, $\bA=\bB=\id$ and obtain
\begin{equation}\label{eq:df2v1}
    \begin{split}
    \widehat{\rm df}_2(\nu_1) &=~ \Tr(\bF^\sT \bF (\bF^\sT \bF + \nu_1\id)^{-1} \bF^\sT \bF (\bF^\sT \bF + \nu_1\id)^{-1})\\
    &\sim~ \Tr(\bLambda^2(\bLambda + \nu_2\id)^{-2}) + \nu_2^2 \Tr(\bLambda(\bLambda + \nu_2\id)^{-2}) \cdot \Tr(\bLambda^2(\bLambda + \nu_2\id)^{-2}) \cdot \frac{1}{p - {\rm df}_2(\nu_2)}\\
    &=:~ n\Upsilon(\nu_1, \nu_2)\,. 
    \end{split}
\end{equation}

For $I_3:= \Tr(\bF^\sT \bF ( \bF^\sT \bF + \nu_1\id)^{-2})$, we use \cref{eq:trA3} with $\bX=\bF$ and obtain
\begin{align}\label{eq:I3}
\Tr(\bF^\sT \bF ( \bF^\sT \bF + \nu_1\id)^{-2}) \sim&~ \Tr(\bLambda(\bLambda + \nu_2\id)^{-2}) \cdot \frac{1}{p - {\rm df}_2(\nu_2)}\,.
\end{align}
Then we use \cref{eq:trA3} again with $\bX=\bF$, $\bA = \btheta_*\btheta_*^\sT$ to obtain the deterministic equivalent of $I_1$
\[
\begin{aligned}
\Tr(\btheta_* \btheta_*^\sT ( \bF^\sT \bF + \nu_1\id)^{-1} \bF^\sT \bF ( \bF^\sT \bF + \nu_1\id)^{-1}) =&~ \Tr(\btheta_* \btheta_*^\sT \bF^\sT \bF ( \bF^\sT \bF + \nu_1\id)^{-2})\\
\sim&~ \Tr(\btheta_* \btheta_*^\sT \bLambda ( \bLambda + \nu_2\id)^{-2}) \cdot \frac{1}{p - {\rm df}_2(\nu_2)}\\
=&~ \btheta_*^\sT \bLambda ( \bLambda + \nu_2\id)^{-2} \btheta_* \cdot \frac{1}{p - {\rm df}_2(\nu_2)}.
\end{aligned}
\]
Further, for $I_2$, use \cref{eq:trAB2} with $\bA=\btheta_*\btheta_*^\sT$ and $\bB=\id$, we obtain
\[
\begin{aligned}
\Tr(\btheta_* \btheta_*^\sT ( \bF^\sT \bF + \nu_1\id)^{-2}) \sim&~ \frac{\nu_2^2}{\nu_1^2}\Tr(\btheta_* \btheta_*^\sT (\bLambda + \nu_2\id)^{-2})\\
&~+ \frac{\nu_2^2}{\nu_1^2}\Tr(\btheta_* \btheta_*^\sT (\bLambda + \nu_2\id)^{-2} \bLambda) \cdot \Tr( (\bLambda + \nu_2\id)^{-2} \bLambda) \cdot \frac{1}{p - {\rm df}_2(\nu_2)}.
\end{aligned}
\]
Finally, combine the above equivalents, for the bias, we obtain
\[
\begin{aligned}
    \mathcal{B}_{\mathcal{N},\lambda}^{\tt RFM} \sim&~ p \btheta_*^\sT \bLambda ( \bLambda + \nu_2\id)^{-2} \btheta_* \cdot \frac{1}{p - {\rm df}_2(\nu_2)}\\
    &~+ p \nu_1^2 \left(\frac{\nu_2^2}{\nu_1^2}\Tr(\btheta_* \btheta_*^\sT (\bLambda + \nu_2\id)^{-2}) + \frac{\nu_2^2}{\nu_1^2}\Tr(\btheta_* \btheta_*^\sT (\bLambda + \nu_2\id)^{-2} \bLambda) \cdot \Tr( (\bLambda + \nu_2\id)^{-2} \bLambda) \cdot \frac{1}{p - {\rm df}_2(\nu_2)} \right)\\
    &~\cdot \Tr(\bLambda(\bLambda + \nu_2\id)^{-2}) \cdot \frac{1}{p - {\rm df}_2(\nu_2)} \cdot \frac{1}{n - n\Upsilon(\nu_1, \nu_2)}\\
    =&~ p\btheta_*^\sT \bLambda ( \bLambda + \nu_2\id)^{-2} \btheta_* \cdot \frac{1}{p - {\rm df}_2(\nu_2)}\\
    &~+ \frac{p}{n} \left(\nu_2^2 \btheta_*^\sT (\bLambda + \nu_2\id)^{-2} \btheta_* + \nu_2^2 \btheta_*^\sT \bLambda (\bLambda + \nu_2\id)^{-2} \btheta_* \cdot \Tr( \bLambda (\bLambda + \nu_2\id)^{-2} ) \cdot \frac{1}{p - {\rm df}_2(\nu_2)} \right)\\
    &~\cdot \Tr(\bLambda(\bLambda + \nu_2\id)^{-2}) \cdot \frac{1}{p - {\rm df}_2(\nu_2)} \cdot \frac{1}{1 - \Upsilon(\nu_1, \nu_2)}\\
    =&~\frac{p\< \btheta_*, \bLambda ( \bLambda + \nu_2\id)^{-2} \btheta_* \>}{p - \Tr\left(\bLambda^2 (\bLambda + \nu_2\id)^{-2}\right)} + \frac{p\chi(\nu_2)}{n} \cdot \frac{\nu_2^2\left[ \< \btheta_*, (\bLambda + \nu_2\id)^{-2} \btheta_* \> \!+\! \chi(\nu_2) \< \btheta_*, \bLambda (\bLambda + \nu_2\id)^{-2} \btheta_* \> \right]}{1 - \Upsilon(\nu_1, \nu_2)}\,.
\end{aligned}
\]
Similarly, for the variance, using \cref{eq:df2v1} and \cref{eq:I3} for $I_3$, we have
\[
\begin{aligned}
    \mathcal{V}_{\mathcal{N},\lambda}^{\tt RFM} \sim&~ \sigma^2 p \Tr(\bLambda(\bLambda + \nu_2\id)^{-2}) \cdot \frac{1}{p - {\rm df}_2(\nu_2)}\cdot \frac{1}{n-n\Upsilon(\nu_1,\nu_2)}\\
    \sim&~ \sigma^2 \frac{\frac{p}{n}\chi(\nu_2)}{1-\Upsilon(\nu_1, \nu_2)}\,.
\end{aligned}
\]
Accordingly, we finish the proof.
\end{proof}

In the next, we present the proof for min-$\ell_2$-norm interpolator under RFMs.

\begin{proof}[Proof of \cref{prop:asy_equiv_norm_RFRR_minnorm}]
Similar to linear regression, we separate the two regimes $p<n$ and $p>n$ as well. For both of them, we provide asymptotic expansions in two steps, first with respect to $\bG$ and then $\bF$ in the under-parameterized regime and vice-versa for the over-parameterized regime.
\paragraph{Under-parameterized regime: Deterministic equivalent over $\bG$} For the variance term, we can use \cref{eq:trA3K} with $\bT=\bG$, $\bSigma=\bF^\sT\bF$, $\bA=\bF^\sT\bF$ and obtain
\[
\begin{aligned}
\mathcal{V}_{\mathcal{N},0}^{\tt RFM} =&~ \sigma^2 \cdot \Tr(\bZ^\sT \bZ(\bZ^\sT \bZ + \lambda\id)^{-2})\\
=&~ \sigma^2 \cdot p\Tr(\bF\bG^\sT \bG \bF^\sT(\bF\bG^\sT \bG \bF^\sT + p\lambda\id)^{-2})\\
=&~ \sigma^2 \cdot p\Tr(\bF^\sT \bF\bG^\sT ( \bG \bF^\sT \bF \bG^\sT + p\lambda\id)^{-2}\bG )\\
\sim&~ \sigma^2 \cdot p\Tr(\bF^\sT \bF ( \bF^\sT \bF + \tilde\lambda\id)^{-2} ) \cdot \frac{1}{n-p}\\
\sim&~ \sigma^2 \cdot \Tr(( \bF \bF^\sT )^{-1}) \cdot \frac{p}{n-p}\,,\\
\end{aligned}
\]
where $\tilde\lambda$ is defined by
\begin{equation}\label{eq:tilde_lambda}
    \tilde\lambda(1-\frac{1}{n}\widetilde{\rm df}_1(\tilde\lambda)) \sim \frac{p\lambda}{n}\,,
\end{equation}
where $\widetilde{\rm df}_1(\tilde\lambda)$ and $\widetilde{\rm df}_2(\tilde\lambda)$ are degrees of freedom associated to $\bF^\sT \bF$. In the under-parameterized regime ($p<n$), when $\lambda$ goes to zero, we have $\tilde\lambda \to 0$ and  $\widetilde{\rm df}_2(\tilde\lambda) \to p$ \citep{bach2024high}.

For the bias term, we use \cref{eq:trAB1K} with $\bT=\bG$, $\bSigma=\bF^\sT\bF$, $\bA=\btheta_* \btheta_*^\sT$, $\bB=\bF^\sT\bF$ and then obtain
\[
\begin{aligned}
\mathcal{B}_{\mathcal{N},0}^{\tt RFM} =&~ \Tr(\btheta_*^\sT \bG^\sT \bZ (\bZ^\sT \bZ + \lambda\id)^{-2} \bZ^\sT \bG\btheta_* )\\
=&~ p\Tr(\btheta_*^\sT \bG^\sT \bG \bF^\sT (\bF \bG^\sT \bG \bF^\sT + p\lambda\id)^{-2} \bF \bG^\sT \bG \btheta_* )\\
=&~ p\Tr(\btheta_* \btheta_*^\sT \bG^\sT ( \bG \bF^\sT \bF \bG^\sT + p\lambda\id)^{-1} \bG \bF^\sT \bF \bG^\sT ( \bG \bF^\sT \bF \bG^\sT + p\lambda\id)^{-1} \bG )\\
\sim&~ p\Tr(\btheta_* \btheta_*^\sT ( \bF^\sT \bF + \tilde\lambda\id)^{-1} \bF^\sT \bF ( \bF^\sT \bF + \tilde\lambda\id)^{-1} )\\ 
&~+ p \tilde\lambda^2 \Tr(\btheta_* \btheta_*^\sT ( \bF^\sT \bF + \tilde\lambda\id)^{-2}) \cdot \Tr(\bF^\sT \bF  ( \bF^\sT \bF + \tilde\lambda\id)^{-2}) \cdot \frac{1}{n-p}\\
\sim&~ p\Tr(\btheta_* \btheta_*^\sT \bF^\sT ( \bF \bF^\sT )^{-2} \bF) + p \Tr(\btheta_* \btheta_*^\sT ( \id - \bF^\sT (\bF\bF^\sT)^{-1} \bF )) \cdot \Tr(( \bF \bF^\sT)^{-1}) \cdot \frac{1}{n-p}\,.
\end{aligned}
\]

In the next, we are ready to eliminate the randomness over $\bF$.
\paragraph{Under-parameterized regime: deterministic equivalent over $\bF$}
For the variance term, from \citet[Sec 3.2]{bach2024high} we know that $\nicefrac{1}{\lambda_p}$ is almost surely the limit of $\Tr((\bF\bF^\sT)^{-1})$, thus we have
\[
\begin{aligned}
\Tr((\bF\bF^\sT)^{-1}) \sim \frac{1}{\lambda_p}\,,
\end{aligned}
\]
where $\lambda_p$ defined by ${\rm df_1}(\lambda_p) = p$, where ${\rm df_1}(\lambda_p)$ and ${\rm df_2}(\lambda_p)$ are degrees of freedom associated to $\bLambda$. Hence we can obtain
\[
\begin{aligned}
\mathcal{V}_{\mathcal{N},0}^{\tt RFM} \sim \sigma^2 \cdot \frac{1}{\lambda_p} \cdot \frac{p}{n-p} = \frac{\sigma^2p}{\lambda_p(n-p)}\,.
\end{aligned}
\]

For the bias term, denote $\bD:=\bF\bLambda^{-1/2}$, we first use \cref{eq:trA3K} with $\bT=\bD$, $\bSigma=\bLambda$, $\bA=\bLambda^{1/2} \btheta_* \btheta_*^\sT \bLambda^{1/2}$ and obtain the deterministic equivalent of the first term in $
\mathcal{B}_{\mathcal{N},0}^{\tt RFM}$
\[
\begin{aligned}
\Tr(\btheta_* \btheta_*^\sT \bF^\sT ( \bF \bF^\sT )^{-2} \bF) = \Tr( \bLambda^{1/2} \btheta_* \btheta_*^\sT \bLambda^{1/2} \bD^\sT ( \bD \bLambda \bD^\sT )^{-2} \bD) \sim \Tr( \btheta_* \btheta_*^\sT \bLambda ( \bLambda + \lambda_p )^{-2} ) \cdot \frac{1}{n-{\rm df}_2(\lambda_p)}\,.
\end{aligned}
\]
Then we use \cref{eq:trAB1K} with $\bT=\bD$, $\bSigma=\bLambda$, $\bA=\bLambda^{1/2} \btheta_* \btheta_*^\sT \bLambda^{1/2}$ and obtain 
\[
\begin{aligned}
\Tr(\btheta_* \btheta_*^\sT \bF^\sT (\bF \bF^\sT)^{-1} \bF) = \Tr( \bLambda^{1/2} \btheta_* \btheta_*^\sT \bLambda^{1/2} \bD^\sT ( \bD \bLambda \bD^\sT )^{-1} \bD) \sim \Tr(\btheta_* \btheta_*^\sT \bLambda (\bLambda +\lambda_p)^{-1})\,,
\end{aligned}
\]
Then the deterministic equivalent of the second term in $\mathcal{B}_{\mathcal{N},0}^{\tt RFM} $ is given by
\[
\begin{aligned}
\Tr(\btheta_* \btheta_*^\sT ( \id - \bF^\sT (\bF\bF^\sT)^{-1} \bF )) \sim \lambda_p \btheta_*^\sT (\bLambda +\lambda_p)^{-1} \btheta_*.
\end{aligned}
\]
Finally, combine the above equivalents and we have
\[
\begin{aligned}
\mathcal{B}_{\mathcal{N},0}^{\tt RFM} \sim&~ \btheta_*^\sT \bLambda (\bLambda +\lambda_p)^{-2} \btheta_* \cdot \frac{p}{n-{\rm df}_2(\lambda_p)} + \btheta_*^\sT (\bLambda +\lambda_p)^{-1} \btheta_* \cdot \frac{p}{n-p}\\
=&~ \frac{p\<\btheta_*, \bLambda (\bLambda +\lambda_p)^{-2} \btheta_*\>}{n-\Tr(\bLambda^2(\bLambda+\lambda_n\id)^{-2})} + \frac{p\<\btheta_*, (\bLambda +\lambda_p)^{-1} \btheta_*\>}{n-p}\,.
\end{aligned}
\]
\paragraph{Over-parameterized regime: deterministic equivalent over $\bF$}

Denote $ \bK:=\bLambda^{1/2}\bG^\sT\bG\bLambda^{1/2}$, for the variance term, we use \cref{eq:trA3K} with $\bT=\bD$, $\bSigma=\bA=\bK$ and obtain 
\[
\begin{aligned}
\mathcal{V}_{\mathcal{N},0}^{\tt RFM} =&~ \sigma^2 \cdot p\Tr(\bF\bG^\sT \bG \bF^\sT(\bF\bG^\sT \bG \bF^\sT + p\lambda\id)^{-2})\\
=&~ \sigma^2 \cdot p\Tr(\bK \bD^\sT (\bD \bK \bD^\sT + p\lambda\id)^{-2} \bD)\\
\sim&~ \sigma^2 \cdot p\Tr(\bK (\bK + \hat\lambda\id)^{-2}) \cdot \frac{1}{p-n}\\
\sim&~ \sigma^2 \cdot \Tr( (\bG \bLambda \bG^\sT )^{-1}) \cdot \frac{p}{p-n}\,,
\end{aligned}
\]
where $\hat\lambda$ is defined by
\begin{equation}\label{eq:hat_lambda}
    \hat\lambda(1-\frac{1}{n}\widehat{\rm df}_1(\hat\lambda)) \sim \frac{p\lambda}{n}\,,
\end{equation}
where $\widehat{\rm df}_1(\hat\lambda)$ and $\widehat{\rm df}_2(\hat\lambda)$ are degrees of freedom associated to $\bK$. In the over-parameterized regime ($p>n$), when $\lambda$ goes to zero, we have $\hat\lambda \to 0$ and  $\widehat{\rm df}_2(\hat\lambda) \to n$ \citep{bach2024high}.

For the bias term, we use \cref{eq:trA3K} with $\bT=\bD$, $\bSigma=\bK$, $\bA=\bLambda^{1/2} \bG^\sT \bG \btheta_* \btheta_*^\sT \bG^\sT \bG \bLambda^{1/2}$ and obtain 
\[
\begin{aligned}
\mathcal{B}_{\mathcal{N},0}^{\tt RFM} =&~ p\Tr(\btheta_*^\sT \bG^\sT \bG \bF^\sT (\bF \bG^\sT \bG \bF^\sT + p\lambda\id)^{-2} \bF \bG^\sT \bG \btheta_* )\\
=&~ p\Tr(\bLambda^{1/2} \bG^\sT \bG \btheta_* \btheta_*^\sT \bG^\sT \bG \bLambda^{1/2} \bD (\bD \bK \bD^\sT + p\lambda\id)^{-2} \bD )\\
\sim&~ p\Tr(\bLambda^{1/2} \bG^\sT \bG \btheta_* \btheta_*^\sT \bG^\sT \bG \bLambda^{1/2} (\bK + \hat\lambda\id)^{-2} ) \cdot \frac{1}{p-n}\\
\sim&~ \Tr( \btheta_* \btheta_*^\sT \bG^\sT (\bG \bLambda \bG^\sT)^{-1} \bG ) \cdot \frac{p}{p-n}\,.
\end{aligned}
\]

\paragraph{Over-parameterized regime: deterministic equivalent over $\bG$}

For the variance term, we have
\[
\begin{aligned}
\mathcal{V}_{\mathcal{N},0}^{\tt RFM} \sim \sigma^2 \cdot \frac{1}{\lambda_n} \cdot \frac{p}{p-n} = \frac{\sigma^2p}{\lambda_n(p-n)}.
\end{aligned}
\]

For the bias term, we have
\[
\begin{aligned}
\mathcal{B}_{\mathcal{N},0}^{\tt RFM} \sim&~ \Tr( \btheta_* \btheta_*^\sT ( \bLambda + \lambda_n)^{-1} ) \cdot \frac{p}{p-n}\\
=&~ \btheta_*^\sT ( \bLambda + \lambda_n)^{-1} \btheta_* \cdot \frac{p}{p-n}\\
=&~ \frac{p\<\btheta_*, ( \bLambda + \lambda_n)^{-1} \btheta_*\>}{p-n}\,.
\end{aligned}
\]
Finally, we conclude the proof.
\end{proof}

To build the connection between the test risk and norm for the min-$\ell_2$-norm estimator for random features regression, we also need the deterministic equivalent of the test risk as below.

\begin{proposition}[Asymptotic deterministic equivalence of the test risk of the min-$\ell_2$-norm interpolator]\label{prop:asy_equiv_error_RFRR_minnorm}
    Under \cref{ass:concentrated_RFRR}, for the minimum $\ell_2$-norm estimator $\hat{\ba}_{\min}$, we have the following deterministic equivalence: for the under-parameterized regime ($p<n$), we have
    \[
    \begin{aligned}
        \mathcal{B}^{\tt RFM}_{\mathcal{R},0} \sim \frac{n\lambda_p \<\btheta_*, (\bLambda +\lambda_p\id)^{-1} \btheta_*\>}{n-p}\,,\quad \mathcal{V}^{\tt RFM}_{\mathcal{R},0} \sim&~ \frac{\sigma^2p}{n-p}\,,
    \end{aligned}
    \]
    where $\lambda_p$ is defined by $\Tr(\bLambda(\bLambda+\lambda_p\id)^{-1}) \sim p$. In the over-parameterized regime ($p>n$), we have
    \[
    \begin{aligned}
        \mathcal{B}^{\tt RFM}_{\mathcal{R},0} \sim&~ \frac{n\lambda_n^2 \<\btheta_*, ( \bLambda + \lambda_n \id)^{-2} \btheta_*\>}{ n - \Tr(\bLambda^2(\bLambda+\lambda_n\id)^{-2})} + \frac{n\lambda_n \<\btheta_*, ( \bLambda + \lambda_n\id)^{-1} \btheta_*\>}{p-n}\,,\\
        \mathcal{V}^{\tt RFM}_{\mathcal{R},0} \sim&~  \frac{\sigma^2\Tr(\bLambda^2(\bLambda+\lambda_n\id)^{-2})}{n - \Tr(\bLambda^2(\bLambda+\lambda_n\id)^{-2})} + \frac{\sigma^2n}{p-n}\,,
    \end{aligned}
    \]
    where $\lambda_n$ is defined by $\Tr(\bLambda(\bLambda+\lambda_n\id)^{-1}) \sim n$.
\end{proposition}

\begin{proof}[Proof of \cref{prop:asy_equiv_error_RFRR_minnorm}]
For the proof, we separate the two regimes $p<n$ and $p>n$. For both of them, we provide asymptotic expansions in two steps, first with respect to $\bG$ and then $\bF$ in the under-parameterized regime and vice-versa for the over-parameterized regime.


\paragraph{Under-parameterized regime: deterministic equivalent over $\bG$}

For the variance term, in the under-parameterized regime, when $\lambda \to 0$, the variance term will become $\mathcal{V}^{\tt RFM}_{\mathcal{R},0} = \sigma^2 \cdot \Tr(\widehat{\bLambda}_{\bF} (\bZ^\sT \bZ)^{-1})$. Accordingly, using \citet[Eq. (12)]{bach2024high}, we have 
\[
\begin{aligned}
\mathcal{V}^{\tt RFM}_{\mathcal{R},0} =&~ \sigma^2 \cdot \Tr(\widehat{\bLambda}_{\bF} (\bZ^\sT \bZ)^{-1})\\
=&~ \sigma^2 \cdot \Tr(\bF\bF^\sT(\bF\bG^\sT\bG\bF^\sT)^{-1})\\
\sim&~ \frac{\sigma^2}{n-p} \cdot \Tr(\bF\bF^\sT(\bF\bF^\sT)^{-1})\\
=&~\frac{\sigma^2p}{n-p}\,.
\end{aligned}
\]

For the bias term, it can be decomposed into
\[
\begin{aligned}
\mathcal{B}^{\tt RFM}_{\mathcal{R},0} =&~ \|\btheta_* - p^{-1/2} \bF^\sT (\bZ^\sT \bZ + \lambda\id)^{-1} \bZ^\sT \bm{G} \btheta_*\|_2^2\\
=&~ \btheta_*^\sT \btheta_* -2 p^{-1/2}\btheta_*^\sT \bF^\sT (\bZ^\sT \bZ + \lambda\id)^{-1} \bZ^\sT \bm{G} \btheta_* + \btheta_*^\sT \bG^\sT \bZ (\bZ^\sT \bZ + \lambda\id)^{-1} \widehat{\bLambda}_{\bF} (\bZ^\sT \bZ + \lambda\id)^{-1} \bZ^\sT \bm{G} \btheta_*.
\end{aligned}
\]
For the second term: $p^{-1/2}\btheta_*^\sT \bF^\sT (\bZ^\sT \bZ + \lambda\id)^{-1} \bZ^\sT \bm{G} \btheta_*$, we can use \cref{eq:trA1K} with $\bT=\bG$, $\bSigma=\bF^\sT\bF$, $\bA=\btheta_*\btheta_*^\sT \bF^\sT \bF$ and obtain
\[
\begin{aligned}
p^{-1/2}\btheta_*^\sT \bF^\sT (\bZ^\sT \bZ + \lambda\id)^{-1} \bZ^\sT \bm{G} \btheta_* =&~ \Tr( \btheta_*\btheta_*^\sT \bF^\sT \bF \bG^\sT (\bG\bF^\sT\bF\bG^\sT + p\lambda\id)^{-1} \bG)\\
\sim&~ \Tr( \btheta_*\btheta_*^\sT \bF^\sT \bF (\bF^\sT\bF + \tilde\lambda\id)^{-1})\\
\sim&~ \Tr( \btheta_*\btheta_*^\sT \bF^\sT (\bF \bF^\sT)^{-1}\bF)\,,
\end{aligned}
\]
where the implicit regularization parameter $\tilde\lambda$ is defined by \cref{eq:tilde_lambda}.

For the third term: $\btheta_*^\sT \bG^\sT \bZ (\bZ^\sT \bZ + \lambda\id)^{-1} \widehat{\bLambda}_{\bF} (\bZ^\sT \bZ + \lambda\id)^{-1} \bZ^\sT \bm{G} \btheta_*$, we can use \cref{eq:trAB1K} with $\bT=\bG$, $\bSigma=\bF^\sT\bF$, $\bA=\btheta_*\btheta_*^\sT$, $\bB=\bF^\sT\bF\bF^\sT\bF$ and obtain
\[
\begin{aligned}
&~\btheta_*^\sT \bG^\sT \bZ (\bZ^\sT \bZ + \lambda\id)^{-1} \widehat{\bLambda}_{\bF} (\bZ^\sT \bZ + \lambda\id)^{-1} \bZ^\sT \bm{G} \btheta_*\\
=&~\Tr(\btheta_* \btheta_*^\sT \bG^\sT \bG \bF^\sT(\bF \bG^\sT \bG \bF^\sT + p\lambda\id)^{-1} \bF\bF^\sT (\bF \bG^\sT \bG \bF^\sT + p\lambda\id)^{-1} \bF \bG^\sT \bG )\\
=&~\Tr(\btheta_* \btheta_*^\sT \bG^\sT ( \bG \bF^\sT \bF \bG^\sT + p\lambda\id)^{-1} \bG \bF^\sT \bF\bF^\sT \bF \bG^\sT ( \bG \bF^\sT \bF \bG^\sT + p\lambda\id)^{-1} \bG )\\
\sim&~ \Tr(\btheta_* \btheta_*^\sT (\bF^\sT \bF + \tilde\lambda\id)^{-1} \bF^\sT \bF\bF^\sT \bF (\bF^\sT \bF + \tilde\lambda\id)^{-1})\\
&~+ \tilde\lambda^2 \Tr(\btheta_* \btheta_*^\sT (\bF^\sT \bF + \tilde\lambda\id)^{-2}) \cdot \Tr(\bF^\sT \bF\bF^\sT \bF (\bF^\sT \bF + \tilde\lambda\id)^{-2}) \cdot \frac{1}{n-p}\\
\sim&~ \Tr(\btheta_* \btheta_*^\sT \bF^\sT (\bF \bF^\sT)^{-1} \bF) + \Tr(\btheta_* \btheta_*^\sT (\id -\bF^\sT (\bF \bF^\sT)^{-1} \bF )) \cdot \frac{p}{n-p}\,.
\end{aligned}
\]
Combining the above equivalents, we have
\[
\begin{aligned}
\mathcal{B}^{\tt RFM}_{\mathcal{R},0} =&~ \btheta_*^\sT \btheta_* -\Tr(\btheta_* \btheta_*^\sT \bF^\sT (\bF \bF^\sT)^{-1} \bF) + \Tr(\btheta_* \btheta_*^\sT (\id -\bF^\sT (\bF \bF^\sT)^{-1} \bF )) \cdot \frac{p}{n-p}\\
=&~ \btheta_*^\sT \btheta_* \cdot \frac{n}{n-p} -\Tr(\btheta_* \btheta_*^\sT \bF^\sT (\bF \bF^\sT)^{-1} \bF) \cdot \frac{n}{n-p}\,.
\end{aligned}
\]
\paragraph{Under-parameterized regime: deterministic equivalent over $\bF$}
For the bias term, we can use \cref{eq:trA1K} with $\bT = \bD := \bF \bLambda^{-1/2}$, $\bA=\bLambda^{1/2} \btheta_* \btheta_*^\sT \bLambda^{1/2}$ and obtain
\[
\begin{aligned}
\Tr(\btheta_* \btheta_*^\sT \bF^\sT (\bF \bF^\sT)^{-1} \bF) =&~ \Tr(\bLambda^{1/2} \btheta_* \btheta_*^\sT \bLambda^{1/2} \bD^\sT (\bD \bLambda \bD^\sT)^{-1} \bD )\\
\sim&~ \Tr(\bLambda^{1/2} \btheta_* \btheta_*^\sT \bLambda^{1/2} (\bLambda +\lambda_p)^{-1})\\
=&~ \btheta_*^\sT \bLambda (\bLambda +\lambda_p)^{-1} \btheta_*\,.
\end{aligned}
\]
Thus, we finally obtain
\[
\begin{aligned}
\mathcal{B}^{\tt RFM}_{\mathcal{R},0} \sim&~ \btheta_*^\sT \btheta_* \cdot \frac{n}{n-p} - \btheta_*^\sT \bLambda (\bLambda +\lambda_p)^{-1} \btheta_* \cdot \frac{n}{n-p}\\
=&~ \lambda_p \btheta_*^\sT (\bLambda +\lambda_p)^{-1} \btheta_* \cdot \frac{n}{n-p}\\
=&~ \frac{n\lambda_p \<\btheta_*, (\bLambda +\lambda_p\id)^{-1} \btheta_*\>}{n-p}\,.
\end{aligned}
\]
\paragraph{Over-parameterized regime: deterministic equivalent over $\bF$}
For the variance term, with $\bD := \bF \bLambda^{-1/2}$ and $\bK := \bLambda^{1/2} \bG^\sT \bG \bLambda^{1/2}$ we can obtain
\[
\begin{aligned}
\mathcal{V}^{\tt RFM}_{\mathcal{R},0} &= \sigma^2 \cdot \mathrm{Tr}(\widehat{\bLambda}_{\bF} \bZ^\sT \bZ (\bZ^\sT \bZ + \lambda\id)^{-2})\\
&= \sigma^2 \cdot \mathrm{Tr}(\bF \bF^\sT \bF \bG^\sT \bG \bF^\sT (\bF \bG^\sT \bG \bF^\sT + p\lambda\id)^{-2})\\
&= \sigma^2 \cdot \mathrm{Tr}(\bD \bLambda \bD^\sT \bD \bLambda^{1/2} \bG^\sT \bG \bLambda^{1/2} \bD^\sT (\bD \bLambda^{1/2} \bG^\sT \bG \bLambda^{1/2} \bD^\sT + p\lambda\id)^{-2})\\
&= \sigma^2 \cdot \mathrm{Tr}(\bLambda \bD^\sT (\bD \bK \bD^\sT + p\lambda\id)^{-1} \bD \bK \bD^\sT (\bD \bK \bD^\sT + p\lambda\id)^{-1} \bD )\,,
\end{aligned}
\]
then we directly use \cref{eq:trAB1K} with $\bT=\bD$, $\bSigma=\bK$, $\bA=\bLambda$, $\bB=\bK$ and obtain
\[
\begin{aligned}
&~\mathrm{Tr}(\bLambda \bD^\sT (\bD \bK \bD^\sT + p\lambda\id)^{-1} \bD \bK \bD^\sT (\bD \bK \bD^\sT + p\lambda\id)^{-1} \bD )\\
\sim&~ \mathrm{Tr}(\bLambda ( \bK + \hat\lambda\id)^{-1} \bK ( \bK + \hat\lambda\id)^{-1} ) + \hat\lambda^2 \mathrm{Tr}(\bLambda ( \bK + \hat\lambda\id)^{-2} ) \cdot \mathrm{Tr}( \bK ( \bK + \hat\lambda\id)^{-2} ) \cdot \frac{1}{p-n}\\
\sim&~ \Tr(\bLambda^2 \bG^\sT (\bG \bLambda \bG^\sT)^{-2} \bG ) + \mathrm{Tr}(\bLambda ( \id - \bLambda^{1/2}\bG^\sT (\bG \bLambda \bG^\sT)^{-1} \bG \bLambda^{1/2} ) ) \cdot \mathrm{Tr}( (\bG \bLambda \bG^\sT)^{-1} ) \cdot \frac{1}{p-n}\,,
\end{aligned}
\]
where the implicit regularization parameter $\hat\lambda$ is defined by \cref{eq:hat_lambda}.

For the bias term, first we have
\[
\begin{aligned}
p^{-1/2}\btheta_*^\sT \bF^\sT (\bZ^\sT \bZ + \lambda\id)^{-1} \bZ^\sT \bm{G} \btheta_* =&~ \Tr( \btheta_*\btheta_*^\sT \bF^\sT (\bF\bG^\sT \bG\bF^\sT+ p\lambda\id)^{-1} \bF \bG^\sT \bG)\\
=&~ \Tr(\bLambda^{1/2} \bG^\sT \bG \btheta_*\btheta_*^\sT \bLambda^{1/2} \bD^\sT (\bD \bK \bD^\sT+ p\lambda\id)^{-1} \bD )\,,
\end{aligned}
\]
then we use \cref{eq:trA1K} with $\bT=\bD$, $\bSigma=\bK$, $\bA=\bLambda^{1/2} \bG^\sT \bG \btheta_*\btheta_*^\sT \bLambda^{1/2}$ and obtain
\[
\begin{aligned}
\Tr(\bLambda^{1/2} \bG^\sT \bG \btheta_*\btheta_*^\sT \bLambda^{1/2} \bD^\sT (\bD \bK \bD^\sT+ p\lambda\id)^{-1} \bD ) \sim&~ \Tr( \btheta_*\btheta_*^\sT \bLambda \bG^\sT ( \bG \bLambda \bG^\sT)^{-1} \bG)\,.
\end{aligned}
\]
Furthermore, we use \cref{eq:trAB1K} with $\bT=\bD$, $\bSigma=\bK$, $\bA=\bLambda^{1/2} \bG^\sT \bG \btheta_* \btheta_*^\sT \bG^\sT \bG \bLambda^{1/2}$, $\bB=\bLambda$ and obtain
\[
\begin{aligned}
&~\btheta_*^\sT \bG^\sT \bZ (\bZ^\sT \bZ + \lambda\id)^{-1} \widehat{\bLambda}_{\bF} (\bZ^\sT \bZ + \lambda\id)^{-1} \bZ^\sT \bm{G} \btheta_*\\
=&~\Tr(\bLambda^{1/2} \bG^\sT \bG \btheta_* \btheta_*^\sT \bG^\sT \bG \bLambda^{1/2} \bD^\sT(\bD \bK \bD^\sT + p\lambda\id)^{-1} \bD \bLambda \bD^\sT (\bD \bK \bD^\sT + p\lambda\id)^{-1} \bD )\\
\sim&~ \Tr(\bLambda^{1/2} \bG^\sT \bG \btheta_* \btheta_*^\sT \bG^\sT \bG \bLambda^{1/2} ( \bK + \hat\lambda\id)^{-1} \bLambda ( \bK + \hat\lambda\id)^{-1} )\\
&~+ \hat\lambda^2 \Tr(\bLambda^{1/2} \bG^\sT \bG \btheta_* \btheta_*^\sT \bG^\sT \bG \bLambda^{1/2} ( \bK + \hat\lambda\id)^{-2} ) \cdot \Tr( \bLambda ( \bK + \hat\lambda\id)^{-2} ) \cdot \frac{1}{p-n}\\
\sim&~ \Tr( \btheta_* \btheta_*^\sT \bG^\sT ( \bG \bLambda \bG^\sT )^{-1} \bG \bLambda^2 \bG^\sT ( \bG \bLambda \bG^\sT )^{-1} \bG)\\
&~+ \Tr( \btheta_* \btheta_*^\sT \bG^\sT ( \bG \bLambda \bG^\sT)^{-1} \bG) \cdot \Tr(\bLambda ( \id - \bLambda^{1/2}\bG^\sT (\bG \bLambda \bG^\sT)^{-1} \bG \bLambda^{1/2} ) ) \cdot \frac{1}{p-n}\,.
\end{aligned}
\]
In the next, we are ready to eliminate the randomness over $\bG$.
\paragraph{Over-parameterized regime: deterministic equivalent over $\bG$}
For the variance term, we use \cref{eq:trA3K} to obtain
\[
\begin{aligned}
\Tr(\bLambda^2 \bG^\sT (\bG \bLambda \bG^\sT)^{-2} \bG ) \sim \frac{{\rm df}_2(\lambda_n)}{n - {\rm df}_2(\lambda_n)}\,.
\end{aligned}
\]
Then we use \cref{eq:trA1K} to obtain
\[
\begin{aligned}
\Tr(\bLambda^2 \bG^\sT (\bG \bLambda \bG^\sT)^{-1} \bG ) \sim \Tr(\bLambda^2(\bLambda + \lambda_n)^{-1}),
\end{aligned}
\]
where $\lambda_n$ is defined by ${\rm df_1}(\lambda_n) = n$. Hence we have
\[
\begin{aligned}
\Tr(\bLambda ( \id - \bLambda^{1/2}\bG^\sT (\bG \bLambda \bG^\sT)^{-1} \bG \bLambda^{1/2} ) ) \sim n\lambda_n.
\end{aligned}
\]
Combine the above equivalents, we have
\[
\begin{aligned}
\mathcal{V}^{\tt RFM}_{\mathcal{R},0} \sim&~  \sigma^2 \cdot  \frac{{\rm df}_2(\lambda_n)}{n - {\rm df}_2(\lambda_n)} + \sigma^2 \cdot \frac{n}{p-n}\\
=&~ \frac{\sigma^2\Tr(\bLambda^2(\bLambda+\lambda_n\id)^{-2})}{n - \Tr(\bLambda^2(\bLambda+\lambda_n\id)^{-2})} + \frac{\sigma^2n}{p-n}\,.
\end{aligned}
\]

For the bias term, we first use \cref{eq:trA1K} to obtain
\[
\begin{aligned}
\Tr( \btheta_*\btheta_*^\sT \bLambda \bG^\sT ( \bG \bLambda \bG^\sT)^{-1} \bG) \sim \Tr(\btheta_*\btheta_*^\sT \bLambda (\bLambda + \lambda_n)^{-1})\,.
\end{aligned}
\]
Moreover, we use \cref{eq:trAB1K} to obtain
\[
\begin{aligned}
\Tr( \btheta_* \btheta_*^\sT \bG^\sT ( \bG \bLambda \bG^\sT )^{-1} \bG \bLambda^2 \bG^\sT ( \bG \bLambda \bG^\sT )^{-1} \bG) \sim&~ \Tr(\btheta_* \btheta_*^\sT \bLambda^2 ( \bLambda + \lambda_n )^{-2})\\ 
&~+ \lambda_n^2 \cdot \Tr(\btheta_* \btheta_*^\sT ( \bLambda + \lambda_n )^{-2}) \cdot \frac{{\rm df}_2(\lambda_n)}{n - {\rm df}_2(\lambda_n)}.
\end{aligned}
\]
Accordingly, we finally conclude that
\[
\begin{aligned}
\mathcal{B}^{\tt RFM}_{\mathcal{R},0} \sim&~ \lambda_n^2 \btheta_*^\sT ( \bLambda + \lambda_n \id)^{-2} \btheta_* \cdot \frac{n}{ n - {\rm df}_2(\lambda_n)} + \lambda_n \btheta_*^\sT ( \bLambda+ \lambda_n\id)^{-1} \btheta_* \cdot \frac{n}{p-n}\\
=&~ \frac{n\lambda_n^2 \<\btheta_*, ( \bLambda + \lambda_n \id)^{-2} \btheta_*\>}{ n - \Tr(\bLambda^2(\bLambda+\lambda_n\id)^{-2})} + \frac{n\lambda_n \<\btheta_*, ( \bLambda + \lambda_n\id)^{-1} \btheta_*\>}{p-n}\,.
\end{aligned}
\]
\end{proof}




\subsection{Non-asymptotic deterministic equivalence for random features ridge regression}
\label{app:nonasy_deter_equiv_rf}

Here we present the proof for the non-asymptotic results on the variance and then discuss the related results on bias due to the insufficient deterministic equivalence.

\subsubsection{Proof on the variance term}

\begin{theorem}[Deterministic equivalence of variance part of the $\ell_2$ norm]\label{prop:det_equiv_RFRR_V}
    Assume the features $\{\bz_i\}_{i \in [n]}$ and $\{\boldf_j\}_{j \in [p]}$ satisfy \cref{ass:concentrated_RFRR} with a constant $C_* > 0$. Then for any $D,K > 0$, there exist constant $\eta_* \in (0, 1/2)$ and $C_{*,D,K} > 0$ ensuring the following property holds. For any $n,p \geq C_{*,D,K}$, $\lambda > 0$, if the following condition is satisfied:
    \begin{equation*}
        \lambda  \geq n^{-K}, \quad \gamma_\lambda \geq p^{-K}, \quad \trho_{\lambda} (n,p)^{5/2} \log^{3/2} (n) \leq K \sqrt{n}\,, \quad \trho_\lambda (n,p)^2 \cdot \rho_{\gamma_+} (p)^{7} \log^4 (p) \leq K \sqrt{p}\,,
    \end{equation*}
    then with probability at least $1-n^{-D}-p^{-D}$, we have that
    \[
    \begin{aligned}
        \left|\mathcal{V}_{\mathcal{N},\lambda}^{\tt RFM} - \sV_{\sN,\lambda}^{\tt RFM}\right| \leq&~ C_{x, D, K} \cdot \mathcal{E}_V(n, p) \cdot \sV_{\sN,\lambda}^{\tt RFM}\,.
    \end{aligned}
    \]
    where the approximation rate is given by
    \[
    \mathcal{E}_V(n, p) := \frac{\widetilde{\rho}_\lambda(n, p)^6 \log^{5/2}(n)}{\sqrt{n}} + \frac{\widetilde{\rho}_\lambda(n, p)^2 \cdot \rho_{\gamma_+}(p)^7 \log^3(p)}{\sqrt{p}}.
    \]
\end{theorem}

\begin{proof}[Proof of \cref{prop:det_equiv_RFRR_V}]
First, note that $\mathcal{V}_{\mathcal{N},\lambda}^{\tt RFM}$ can be written in terms of the functional $\Phi_4$ defined in \cref{eq:functionals_Z}:
\[
\mathcal{V}_{\mathcal{N},\lambda}^{\tt RFM} = \sigma^2 \cdot n \Phi_4 ( \bZ ; \hbLambda^{-1}_\bF,\lambda).
\]
Recall that $\cA_\cF$ is the event defined in \citet[Eq. (79)]{defilippis2024dimension}. Under the assumptions, we have
\[
\P (\cA_\cF) \geq 1 - p^{-D}.
\]
Hence, applying \cref{prop:det_Z} for $\bF \in \cA_\cF$ and via union bound, we obtain that with probability at least $1 - p^{-D} - n^{-D}$,
\begin{equation}\label{eq:var_remove_Z}
\left| n\Phi_4 ( \bZ ; \hbLambda^{-1}_\bF,\lambda) - n\tPhi_5 ( \bF ; \hbLambda^{-1}_\bF, p\nu_1 )\right|  \leq C_{*,D,K} \cdot \cE_1 (p,n) \cdot n\tPhi_5 ( \bF ; \hbLambda^{-1}_\bF, p\nu_1 ),
\end{equation}
and we recall the expressions
\[
n\tPhi_5 ( \bF ; \hbLambda^{-1}_\bF, p\nu_1 ) = \frac{\tPhi_6 ( \bF ; \hbLambda^{-1}_\bF , p\nu_1) }{n - \tPhi_6 ( \bF ; \id , p\nu_1)}, \quad \quad \tPhi_6 ( \bF ; \hbLambda^{-1}_\bF , p\nu_1) = p\Tr \big( \bF \bF^\sT ( \bF \bF^\sT + p \nu_1)^{-2} \big).
\]
From \citet[Lemma B.11]{defilippis2024dimension}, we have with probability at least $1 - p^{-D}$
\[
\begin{aligned}
\left| p\Tr (\bF \bF^\sT ( \bF \bF^\sT + p\nu_1)^{-2} ) - p^2\Psi_3 ( \nu_2 ; \bLambda^{-1} )\right| \leq&~ C_{*,D,K} \cdot \rho_{\gamma_+} (p) \cdot \cE_3 (p) \cdot p^2\Psi_3 ( \nu_2 ; \bLambda^{-1} )\,,
\end{aligned}
\]
where the approximation rate \(\cE_3 (p)\) is given by
\[
    \mathcal{E}_3(p) := \frac{\rho_{\gamma_+}(p)^6 \log^3(p)}{\sqrt{p}}.
\]


Furthermore, from the proof of \citet[Theorem B.12]{defilippis2024dimension}, we have with probability at least $1 - p^{-D}$,
\[
\left| (1 - n^{-1} \tPhi_6 (\bF; \id, p\nu_1))^{-1} - (1 -  \Upsilon (\nu_1,\nu_2) )^{-1} \right| \leq C_{*,D,K} \cdot \trho_\lambda (n,p) \rho_{\gamma_+} (p) \cE_3 (p) \cdot  (1 -   \Upsilon (\nu_1,\nu_2))^{-1}.
\]
Combining those two bounds, we obtain
\[
\left| \frac{\tPhi_6 ( \bF ; \hbLambda^{-1}_\bF , p\nu_1) }{n - \tPhi_6 ( \bF ; \id , p\nu_1)} - \frac{p^2\Psi_3 ( \nu_2 ; \bLambda^{-1} )}{n - n\Upsilon (\nu_1,\nu_2)} \right| \leq C_{*,D,K} \cdot \trho_\lambda (n,p) \rho_{\gamma_+} (p) \cE_3 (p) \cdot \frac{p^2\Psi_3 ( \nu_2 ; \bLambda^{-1} )}{n - n\Upsilon (\nu_1,\nu_2)}.
\]
Finally, we can combine this bound with \cref{eq:var_remove_Z} to obtain via union bound that with probability at least $1 - n^{-D} - p^{-D}$, 
\[
\left| n\Phi_4 ( \bZ ; \hbLambda^{-1}_\bF,\lambda) - \frac{p^2\Psi_3 ( \nu_2 ; \bLambda^{-1} )}{n - n\Upsilon (\nu_1,\nu_2)} \right| \leq C_{*,D,K} \left\{ \cE_1 (p,n) + \trho_\lambda (n,p) \rho_{\gamma_+} (p) \cE_3 (p) \right\} \frac{p^2\Psi_3 ( \nu_2 ; \bLambda^{-1} )}{n - n\Upsilon (\nu_1,\nu_2)}\,.
\]
Replacing the rate $\cE_j$ by their expressions conclude the proof of this theorem.
\end{proof}

\subsubsection{Discussion on the bias term}\label{app:discuss_bias}

We present the deterministic equivalence of the bias term as an informal result, without a Existing deterministic equivalence results appear insufficient to directly establish this desired bias result. While we believe this is doable under  additional assumptions, a complete proof is beyond the scope of this paper..

In the proof of the bias term, deterministic equivalences for functionals of the form 
\[
\Tr \left( \bA \left( \bX^\sT \bX \right)^2 (\bX^\sT \bX + \lambda)^{-2} \right)
\]
are required. However, such equivalences are currently unavailable, necessitating the introduction of technical assumptions to leverage the deterministic equivalences of \(\Phi_2(\bX; \bA, \lambda)\) and \(\Phi_4(\bX; \bA, \lambda)\).

Furthermore, the proof of the bias term in \cite{defilippis2024dimension} suggests that deriving deterministic equivalences for the bias of the \(\ell_2\) norm, analogous to \citet[Proposition B.7]{defilippis2024dimension}, is also required but remains unresolved.

Addressing these gaps in deterministic equivalence is an important direction for future work, particularly to establish rigorous proofs for the currently missing results.



\subsection{Proofs on relationship between test risk and \texorpdfstring{$\ell_2$}{L2} norm of random feature ridge regression estimator}
\label{app:relationship_rf}

To derive the relationship between test risk and norm for the random feature model, we first examine the linear relationship in the over-parameterized regime. Next, we analyze the case where \(\bLambda = \id_m\) with \(n < m < \infty\) (finite rank), followed by the relationship under the power-law assumption.



\subsubsection{Proof for min-norm interpolator in the over-parameterized regime}
According to the formulation in \cref{prop:asy_equiv_norm_RFRR_minnorm} and \cref{prop:asy_equiv_error_RFRR_minnorm}, we have for the under-parameterized regime ($p<n$), we have
\[
\begin{aligned}
    \mathcal{B}^{\tt RFM}_{\mathcal{N},0} \sim& \sB^{\tt RFM}_{\sN,0} = \frac{p\<\btheta_*, \bLambda (\bLambda +\lambda_p\id)^{-2} \btheta_*\>}{n - \Tr(\bLambda^2(\bLambda+\lambda_p\id)^{-2})} + \frac{p\<\btheta_*, (\bLambda +\lambda_p\id)^{-1} \btheta_*\>}{n-p}\,,\\
    \mathcal{V}^{\tt RFM}_{\mathcal{N},0} \sim& \sV^{\tt RFM}_{\sN,0} = \frac{\sigma^2p}{\lambda_p(n-p)}\,,
\end{aligned}
\]
\[
\begin{aligned}
    \mathcal{B}^{\tt RFM}_{\mathcal{R},0} \sim \sB^{\tt RFM}_{\sR,0} = \frac{n\lambda_p \<\btheta_*, (\bLambda +\lambda_p\id)^{-1} \btheta_*\>}{n-p}\,,\quad \mathcal{V}^{\tt RFM}_{\mathcal{R},0} \sim \sV^{\tt RFM}_{\sR,0} = \frac{\sigma^2p}{n-p}\,.
\end{aligned}
\]
In the over-parameterized regime ($p>n$), we have
\[
\begin{aligned}
    \mathcal{B}^{\tt RFM}_{\mathcal{N},0} \sim \sB^{\tt RFM}_{\sN,0} = \frac{p\<\btheta_*, ( \bLambda + \lambda_n\id)^{-1} \btheta_*\>}{p-n}\,,\quad
    \mathcal{V}^{\tt RFM}_{\mathcal{N},0} \sim \sV^{\tt RFM}_{\sN,0} = \frac{\sigma^2p}{\lambda_n(p-n)}\,,
\end{aligned}
\]
\[
\begin{aligned}
    \mathcal{B}^{\tt RFM}_{\mathcal{R},0} \sim&~ \sB^{\tt RFM}_{\sR,0} = \frac{n\lambda_n^2 \<\btheta_*, ( \bLambda + \lambda_n \id)^{-2} \btheta_*\>}{ n - \Tr(\bLambda^2(\bLambda+\lambda_n\id)^{-2})} + \frac{n\lambda_n \<\btheta_*, ( \bLambda + \lambda_n\id)^{-1} \btheta_*\>}{p-n}\,,\\
    \mathcal{V}^{\tt RFM}_{\mathcal{R},0} \sim&~  \sV^{\tt RFM}_{\sR,0} = \frac{\sigma^2\Tr(\bLambda^2(\bLambda+\lambda_n\id)^{-2})}{n - \Tr(\bLambda^2(\bLambda+\lambda_n\id)^{-2})} + \frac{\sigma^2n}{p-n}\,.
\end{aligned}
\]
With these formulations we can introduce the relationship between test risk and norm in the over-parameterized regime as follows. 

\begin{proof}[Proof of \cref{prop:relation_minnorm_overparam}]
In the over-parameterized regime ($p > n$), we have
\[
    \sN_{0}^{\tt RFM} 
    = 
    \sB^{\tt RFM}_{\sN,0} + \sV^{\tt RFM}_{\sN,0} 
    = 
    \frac{p\<\btheta_*, ( \bLambda + \lambda_n\id)^{-1} \btheta_*\>}{p-n} + \frac{\sigma^2p}{\lambda_n(p-n)} 
    = 
    \left[\<\btheta_*, ( \bLambda + \lambda_n\id)^{-1} \btheta_*\> + \frac{\sigma^2}{\lambda_n}\right] \frac{p}{p-n}\,.
\]
\[
\begin{aligned}
    \sR_{0}^{\tt RFM} 
    =&
    \frac{n\lambda_n^2 \<\btheta_*, ( \bLambda + \lambda_n \id)^{-2} \btheta_*\>}{ n - \Tr(\bLambda^2(\bLambda+\lambda_n\id)^{-2})} 
    + 
    \frac{n\lambda_n \<\btheta_*, ( \bLambda + \lambda_n\id)^{-1} \btheta_*\>}{p-n}
    + 
    \frac{\sigma^2\Tr(\bLambda^2(\bLambda+\lambda_n\id)^{-2})}{n - \Tr(\bLambda^2(\bLambda+\lambda_n\id)^{-2})} 
    + 
    \frac{\sigma^2n}{p-n}\\
    =& 
    \frac{n\lambda_n^2 \<\btheta_*, ( \bLambda + \lambda_n \id)^{-2} \btheta_*\> + \sigma^2\Tr(\bLambda^2(\bLambda+\lambda_n\id)^{-2})}{ n - \Tr(\bLambda^2(\bLambda+\lambda_n\id)^{-2})} 
    + 
    \left[n\lambda_n \<\btheta_*, ( \bLambda + \lambda_n\id)^{-1} \btheta_*\> + \sigma^2n\right]\frac{1}{p-n}\,.
\end{aligned}
\]
Then we eliminate $p$ and obtain that the deterministic equivalents of the estimator's test risk and norm, $\sR^{\tt RFM}_{0}$ and $\sN^{\tt RFM}_{0}$, in over-parameterized regimes ($p>n$) admit
\[
\sR_{0}^{\tt RFM} 
= 
\lambda_n\sN_{0}^{\tt RFM} 
- 
\left[\lambda_n\<\btheta_*, ( \bLambda + \lambda_n\id)^{-1} \btheta_*\> + \sigma^2\right] 
+ 
\frac{n\lambda_n^2 \<\btheta_*, ( \bLambda + \lambda_n \id)^{-2} \btheta_*\> + \sigma^2\Tr(\bLambda^2(\bLambda+\lambda_n\id)^{-2})}{ n - \Tr(\bLambda^2(\bLambda+\lambda_n\id)^{-2})}\,. 
\]
\end{proof}


\subsubsection{Proof on isotropic features with finite rank}
Here we present the proof of \cref{prop:relation_minnorm_id_rf} with \( \bLambda = \id_m \).

\begin{proof}[Proof of \cref{prop:relation_minnorm_id_rf}]
Here we consider the case where \( \bLambda = \id_m \). Under this condition, the definitions of \( \lambda_p \) and \( \lambda_n \) above are simplified to \( \frac{m}{1+\lambda_p} = p \) and \( \frac{m}{1+\lambda_n} = n \), respectively. Consequently, \( \lambda_p \) and \( \lambda_n \) have explicit expressions given by \( \lambda_p = \frac{m-p}{p} \) and \( \lambda_n = \frac{m-n}{n} \), respectively.

First, in the over-parameterized regime ($p>n$), we have
\[
\begin{aligned}
     \sB^{\tt RFM}_{\sN,0} = \frac{p\frac{1}{1+\lambda_n}\|\btheta_*\|_2^2}{p - n} = \frac{np}{m(p-n)} \|\btheta_*\|_2^2\,,\quad
    \sV^{\tt RFM}_{\sN,0} = \frac{\sigma^2p}{\lambda_n(p-n)} = \frac{\sigma^2np}{(m-n)(p-n)}\,.
\end{aligned}
\]
\[
\begin{aligned}
     \sB^{\tt RFM}_{\sR,0} =& \frac{n\lambda_n^2 \frac{1}{(1+\lambda_n)^2}\|\btheta_*\|_2^2}{n-\frac{m}{(1+\lambda_n)^2}} + \frac{n\lambda_n\frac{1}{1+\lambda_n}\|\btheta_*\|_2^2}{p-n} = \frac{p(m-n)}{m(p-n)} \|\btheta_*\|_2^2\,,\\ 
     \sV^{\tt RFM}_{\sR,0} =& \frac{\sigma^2\frac{m}{(1+\lambda_n)^2}}{n-\frac{m}{(1+\lambda_n)^2}}+\frac{\sigma^2n}{p-n}=\frac{\sigma^2n}{m-n} + \frac{\sigma^2n}{p-n}\,.
\end{aligned}
\]
We eliminate $p$ and obtain that the relationship between $\sV^{\tt RFM}_{\sR,0}$ and $\sV^{\tt RFM}_{\sN,0}$ is
\[
\begin{aligned}
\sV^{\tt RFM}_{\sR,0} = \frac{m-n}{n}\sV^{\tt RFM}_{\sN,0} + \frac{2n -m}{m-n} \sigma^2\,.
\end{aligned}
\]
similarly, the relationship between $\sB^{\tt RFM}_{\sR,0}$ and $\sB^{\tt RFM}_{\sN,0}$ is
\[
\begin{aligned}
\sB^{\tt RFM}_{\sR,0} = \frac{m-n}{n}\sB^{\tt RFM}_{\sN,0}\,.
\end{aligned}
\]
Combining the above two relationship, we obtain the relationship between test risk $\sR_{0}^{\tt RFM}$ and norm $\sN_{0}^{\tt RFM}$ as

\[
\begin{aligned}
\sR_{0}^{\tt RFM} = \frac{m-n}{n} \sN_{0}^{\tt RFM} +\frac{2n-m}{m-n} \sigma^2.
\end{aligned}
\]
Accordingly, in the under-parameterized regime ($p<n$), we have
\[
\begin{aligned}
     \sB^{\tt RFM}_{\sN,0} =& \frac{p\frac{1}{(1+\lambda_p)^2}\|\btheta_*\|_2^2}{n - \frac{m}{(1+\lambda_p)^2}} + \frac{p\frac{1}{1+\lambda_p}\|\btheta_*\|_2^2}{n-p} = \frac{p}{m} \left( \frac{p^2}{nm - p^2} + \frac{p}{n-p} \right) \|\btheta_*\|_2^2\,,\\
    \sV^{\tt RFM}_{\sN,0} =& \frac{\sigma^2p}{\lambda_p(p-n)} = \frac{\sigma^2p^2}{(m-p)(n-p)}\,.
\end{aligned}
\]
\[
\begin{aligned}
     \sB^{\tt RFM}_{\sR,0} = \frac{n\lambda_p \frac{1}{1+\lambda_p}\|\btheta_*\|_2^2}{n-p} = \frac{n(m-p)}{m(n-p)} \|\btheta_*\|_2^2\,,\quad \sV^{\tt RFM}_{\sR,0} = \frac{\sigma^2p}{n-p}\,.
\end{aligned}
\]
Then we eliminate $p$ and obtain that, in the under-parameterized regime ($p<n$), the relationship between $\sV^{\tt RFM}_{\sR,0}$ and $\sV^{\tt RFM}_{\sN,0}$ is
\[
\begin{aligned}
\sV^{\tt RFM}_{\sR,0} = \frac{(m-n) \sV^{\tt RFM}_{\sN,0} + \sqrt{(m-n)^2(\sV^{\tt RFM}_{\sN,0})^2 + 4nm\sigma^2\sV^{\tt RFM}_{\sN,0}}}{2n}\,,
\end{aligned}
\]
which can be further simplified as a hyperbolic function
\begin{equation*}
     \left(\sV^{\tt RFM}_{\sR,0} \right)^2 = \frac{m-n}{n} \sV^{\tt RFM}_{\sR,0} \sV^{\tt RFM}_{\sN,0} + \frac{m \sigma^2}{n} \sV^{\tt RFM}_{\sN,0}\,,
\end{equation*}
and the asymptote of this hyperbola is $\sV^{\tt RFM}_{\sR,0} = \frac{m-n}{n}\sV^{\tt RFM}_{\sN,0} + \frac{m}{m-n} \sigma^2$.

Besides, we eliminate $p$ and obtain the relationship between $\sB^{\tt RFM}_{\sR,0}$ and $\sB^{\tt RFM}_{\sN,0}$ as
\[
\begin{aligned}
&\frac{\|\btheta_*\|_2^6 n^2 \left( 2 \|\btheta_*\|_2^2 + \sB^{\tt RFM}_{\sN,0} - \frac{\sB^{\tt RFM}_{\sN,0} n}{m} \right)}{m} 
=
(\sB^{\tt RFM}_{\sR,0})^4 n 
+ 
(\sB^{\tt RFM}_{\sR,0})^2 \|\btheta_*\|_2^2 n \left( \|\btheta_*\|_2^2 + \sB^{\tt RFM}_{\sN,0} - \frac{4 \|\btheta_*\|_2^2 n}{m} - \frac{\sB^{\tt RFM}_{\sN,0} n}{m} \right)\\
& + \sB^{\tt RFM}_{\sR,0} \|\btheta_*\|_2^4 n \left( \sB^{\tt RFM}_{\sN,0} + \frac{5 \|\btheta_*\|_2^2 n}{m} - \frac{\sB^{\tt RFM}_{\sN,0} n}{m} \right)
+ 
(\sB^{\tt RFM}_{\sR,0})^3 \left( -\sB^{\tt RFM}_{\sN,0} m - 2 \|\btheta_*\|_2^2 n + \sB^{\tt RFM}_{\sN,0} n + \frac{\|\btheta_*\|_2^2 n^2}{m} \right),
\end{aligned}
\]
which can be simplified to
\[
\begin{aligned}
    \sB^{\tt RFM}_{\sN,0}(m-n)(m\sB^{\tt RFM}_{\sR,0}-n\|\btheta_*\|_2^2)(m(\sB^{\tt RFM}_{\sR,0})^2-n\|\btheta_*\|_2^4) 
    = 
    nm(\sB^{\tt RFM}_{\sR,0}-\|\btheta_*\|_2^2)^2(m(\sB^{\tt RFM}_{\sR,0})^2-2n\|\btheta_*\|_2^4+n\|\btheta_*\|_2^2\sB^{\tt RFM}_{\sR,0})\,.
\end{aligned}
\]
We can find that in this case, the relationship can be easily written as
\[
\begin{aligned}
    \sB^{\tt RFM}_{\sN,0} =& \frac{nm(\sB^{\tt RFM}_{\sR,0}-\|\btheta_*\|_2^2)^2(m(\sB^{\tt RFM}_{\sR,0})^2-2n\|\btheta_*\|_2^4+n\|\btheta_*\|_2^2\sB^{\tt RFM}_{\sR,0})}{(m-n)(m\sB^{\tt RFM}_{\sR,0}-n\|\btheta_*\|_2^2)(m(\sB^{\tt RFM}_{\sR,0})^2-n\|\btheta_*\|_2^4)}\,.
\end{aligned}
\]
Next we will show that when $p \to n$, which also implies that $\sB^{\tt RFM}_{\sN,0} \to \infty$ and $\sB^{\tt RFM}_{\sR,0} \to \infty$, this relationship is approximately linear.

Recall that the relationship between $\sB^{\tt RFM}_{\sR,0}$ and $\sB^{\tt RFM}_{\sN,0}$ is given by \(\sB^{\tt RFM}_{\sR,0} = \frac{(m-n)}{n}\sB^{\tt RFM}_{\sN,0}\), and is equivalent to \(\sB^{\tt RFM}_{\sN,0} = \frac{n}{(m-n)}\sB^{\tt RFM}_{\sR,0} := f(\sB^{\tt RFM}_{\sR,0})\). We then do a difference and get
\[
\sB^{\tt RFM}_{\sN,0} - f(\sB^{\tt RFM}_{\sR,0}) = \frac{nm(\sB^{\tt RFM}_{\sR,0}-\|\btheta_*\|_2^2)^2(m(\sB^{\tt RFM}_{\sR,0})^2-2n\|\btheta_*\|_2^4+n\|\btheta_*\|_2^2\sB^{\tt RFM}_{\sR,0})}{(m-n)(m\sB^{\tt RFM}_{\sR,0}-n\|\btheta_*\|_2^2)(m(\sB^{\tt RFM}_{\sR,0})^2-n\|\btheta_*\|_2^4)} - \frac{n}{m-n}\sB^{\tt RFM}_{\sR,0}\,,
\]
then take \(\sB^{\tt RFM}_{\sR,0} \to \infty\) and we get
\[
\lim_{\sB^{\tt RFM}_{\sR,0} \to \infty}\sB^{\tt RFM}_{\sN,0} - f(\sB^{\tt RFM}_{\sR,0}) = -\frac{2n}{m}\|\btheta_*\|_2^2\,.
\]
Finally, organizing this equation and we get
\[
    \sB^{\tt RFM}_{\sR,0} \approx \frac{m-n}{n}\sB^{\tt RFM}_{\sN,0} + \frac{2(m-n)}{m}\|\btheta_*\|_2^2\,.
\]
\end{proof}

\subsubsection{Proof on features under power law assumption}

\begin{proof}[Proof of \cref{prop:relation_minnorm_powerlaw_rf}]
First, we use integral approximation to give approximations to some quantities commonly used in deterministic equivalence to prepare for the subsequent derivations.

According to the integral approximation in \citet[Lemma 1]{simonmore}, we have
\begin{equation}\label{eq:integrate_approx1}
    \Tr(\bLambda (\bLambda + \nu_2)^{-1}) \approx C_1 \nu_2^{-\frac{1}{\alpha}},\quad \Tr(\bLambda^2 (\bLambda + \nu_2)^{-2}) \approx C_2 \nu_2^{-\frac{1}{\alpha}},\quad \Tr(\bLambda (\bLambda + \nu_2)^{-2}) \approx (C_1-C_2) \nu_2^{-\frac{1}{\alpha}-1}\,,
\end{equation}
where $C_1$ and $C_2$ are
\begin{equation}\label{eq:C1C2}
    C_1 = \frac{\pi}{\alpha \sin\left(\nicefrac{\pi}{\alpha}\right)}\,,\quad C_2 = \frac{\pi(\alpha-1)}{\alpha^2 \sin\left(\nicefrac{\pi}{\alpha}\right)}\,,\quad \text{with $C_1 > C_2$}\,.
\end{equation}
Besides, according to definition of \(T(\nu)\) \cref{app:pre_scaling_law}, we have 
\[
\begin{aligned}
\< \btheta_*, (\bLambda + \nu_2)^{-1} \btheta_* \> =&~ T^1_{2r,1}(\nu_2) \approx C_3 \nu_2^{(2r-1)\wedge0},\\
\< \btheta_*, \bLambda(\bLambda + \nu_2)^{-2} \btheta_* \> =&~ T^1_{2r+1,2}(\nu_2) \approx C_4 \nu_2^{(2r-1)\wedge0}.\\
\end{aligned}
\]
When $r \in (0, \frac{1}{2})$, according to the integral approximation, we have
\begin{equation}
    C_3 = \frac{\pi}{\alpha \sin(2\pi r)}\,,\quad C_4 = \frac{2\pi r}{\alpha \sin(2\pi r)}\,,\quad \text{with $C_3 > C_4$}.
\end{equation}
Otherwise, if $r \in [\frac{1}{2}, \infty)$, we have
\[
\frac{1}{\alpha(2r-1)}< C_3 < \frac{1}{\alpha(2r-1)}+1\,,\quad \frac{1}{\alpha(2r-1)}< C_4 < \frac{1}{\alpha(2r-1)}+1\,,\quad \text{with $C_3 > C_4$}.
\]
For $\< \btheta_*, (\bLambda + \nu_2)^{-2} \btheta_* \>$, we have to discuss its approximation in the case $r \in (0, \frac{1}{2})$, $r \in [\frac{1}{2}, 1)$ and $r \in [\frac{1}{2}, \infty)$ separately.
\[
\langle \bm{\theta}_*, (\bm{\Lambda} + \nu_2)^{-2} \bm{\theta}_* \rangle \approx
\begin{cases} 
    (C_3 - C_4) \nu_2^{2r - 2}, & \text{if } r \in (0, \frac{1}{2})\,; \\
    C_5 \nu_2^{2r-2}, & \text{if } r \in [\frac{1}{2}, 1)\,; \\
    C_6, & \text{if } r \in [1, \infty)\,,
\end{cases}
\]
where $\frac{1}{2\alpha(r-1)} < C_6 < \frac{1}{2\alpha(r-1)} + 1$.


With the results of the integral approximation above, we next derive the relationship between $\sR_0^{\tt RFM}$ and $\sN_0^{\tt RFM}$ {\bf separately in over-parameterized regime ($p > n$) and under-parameterized regime ($p < n$).}

\paragraph{The relationship in over-parameterized regime ($p > n$)}

According to the self-consistent equation
\[
\begin{aligned}
1 + \frac{n}{p} - \sqrt{\left( 1 - \frac{n}{p} \right)^2 + \frac{4\lambda}{p\nu_2}} = \frac{2}{p} \operatorname{Tr} \left( \bLambda \left( \bLambda + \nu_2 \right)^{-1} \right),
\end{aligned}
\]
\[
\begin{aligned}
\nu_1 = \frac{\nu_2}{2} \left[ 1 - \frac{n}{p} + \sqrt{\left( 1 - \frac{n}{p} \right)^2 + \frac{4\lambda}{p\nu_2}} \right],
\end{aligned}
\]
In the over-parameterized regime (\(p > n\)), as \(\lambda \to 0\), for the first equation, \(\frac{4\lambda}{p\nu_2}\) will approach \(0\), and \(\Tr(\bLambda(\bLambda + \nu_2)^{-1})\) will converge to \(n\). Consequently, by \cref{eq:integrate_approx1}, \(\nu_2\) will converge to the constant \((\frac{n}{C_1})^{-\alpha}\). Furthermore, from the second equation, \(\nu_1\) will converge to \(\nu_2(1 - \frac{n}{p})\). Thus, according to \cref{eq:integrate_approx1}, we have
\[
\begin{aligned}
\Tr(\bLambda (\bLambda + \nu_2)^{-1}) \approx n,\quad \Tr(\bLambda^2 (\bLambda + \nu_2)^{-2}) \approx \frac{C_2}{C_1}n,\quad \Tr(\bLambda (\bLambda + \nu_2)^{-2}) \approx (C_1-C_2) (\frac{n}{C_1})^{\alpha+1}.
\end{aligned}
\]
Thus, in the over-parameterized regime
\[
\begin{aligned}
\Upsilon(\nu_1, \nu_2) =&~ \frac{p}{n} \left[ \left( 1 - \frac{\nu_1}{\nu_2} \right)^2 + \left( \frac{\nu_1}{\nu_2} \right)^2 \frac{\operatorname{Tr}\left(\bLambda^2 (\bLambda + \nu_2)^{-2}\right)}{p - \operatorname{Tr}\left(\bLambda^2 (\bLambda + \nu_2)^{-2}\right)} \right]\\
\approx&~ \frac{p}{n} \left[ \left( \frac{n}{p} \right)^2 + \left( 1 - \frac{n}{p} \right)^2 \frac{\operatorname{Tr}\left(\bLambda^2 (\bLambda + \nu_2)^{-2}\right)}{p - \operatorname{Tr}\left(\bLambda^2 (\bLambda + \nu_2)^{-2}\right)} \right]\\
\approx&~ \frac{\frac{C_2}{C_1}p -2 \frac{C_2}{C_1} n + n}{p - \frac{C_2}{C_1}n}\,,
\end{aligned}
\]
\[
\begin{aligned}
\chi(\nu_2) =~ \frac{\Tr(\bLambda (\bLambda + \nu_2)^{-2})}{p - \Tr(\bLambda^2 (\bLambda + \nu_2)^{-2})} \approx~ \frac{(C_1-C_2)(\frac{n}{C_1})^{\alpha+1}}{p-\frac{C_2}{C_1}n}\,.
\end{aligned}
\]
According to the approximation, we have the deterministic equivalents of variance terms
\[
\begin{aligned}
\sV_{\sR,0}^{\tt RFM} =&~ \sigma^2 \frac{\Upsilon(\nu_1, \nu_2)}{1 - \Upsilon(\nu_1, \nu_2)} \approx \sigma^2 \frac{(C_1-2C_2)n + C_2p}{(C_1-C_2)(p-n)}\,,\\
\sV_{\sN,0}^{\tt RFM} =&~ \sigma^2 \frac{p}{n} \frac{\chi(\nu_2)}{1 - \Upsilon(\nu_1, \nu_2)} \approx \sigma^2 \frac{(\frac{n}{C_1})^\alpha p}{p-n}\,.
\end{aligned}
\]
Then recall \cref{eq:C1C2}, we eliminate $p$ and obtain
\begin{equation}\label{eq:V_over_power_law}
    \sV_{\sR,0}^{\tt RFM} \approx \left(\frac{n}{C_1}\right)^{-\alpha}\sV_{\sN,0}^{\tt RFM} + \sigma^2\frac{2C_2-C_1}{C_1-C_2} = \left(\frac{n}{C_1}\right)^{-\alpha}\sV_{\sN,0}^{\tt RFM} + \sigma^2(\alpha - 2)\,.
\end{equation}

For the bias terms, due to the varying approximation behaviors of the quantities containing $\btheta_*$ for different values of $r$, we have to discuss their approximations in the conditions $r \in (0, \frac{1}{2})$, $r \in [\frac{1}{2}, 1)$ and $r \in [\frac{1}{2}, \infty)$ separately.

\paragraph{Condition 1: $r \in (0, \frac{1}{2})$}
\[
\begin{aligned}
\sB_{\sR,0}^{\tt RFM} =&~ \frac{\nu_2^2}{1 - \Upsilon(\nu_1, \nu_2)} \left[ \< \btheta_*, (\bLambda + \nu_2)^{-2} \btheta_* \> + \chi(\nu_2) \< \btheta_*, \bLambda (\bLambda + \nu_2)^{-2} \btheta_* \> \right]\\
\approx&~\frac{\left(\frac{n}{C_1}\right)^{-2\alpha r}\left((C_1C_4 -C_2C_3)n+C_1(C_3-C_4)p\right)}{(C1-C2)(p-n)},\\
\sB_{\sN,0}^{\tt RFM} =&~ \frac{\nu_2}{\nu_1} \< \btheta_*, (\bLambda + \nu_2)^{-1} \btheta_* \> - \frac{\lambda}{n} \frac{\nu_2^2}{\nu_1^2} \frac{\< \btheta_*, (\bLambda + \nu_2)^{-2} \btheta_* \> + \chi(\nu_2) \< \btheta_*, \bLambda (\bLambda + \nu_2)^{-2} \btheta_* \>}{1 - \Upsilon(\nu_1, \nu_2)}\\
\approx&~ \frac{\nu_2}{\nu_1} \< \btheta_*, (\bLambda + \nu_2)^{-1} \btheta_* \>\\
\approx&~ \frac{\left(\frac{n}{C_1}\right)^{-\alpha(2r-1)}C_3 p}{p-n}.
\end{aligned}
\]
Then we eliminate $p$ and obtain
\begin{equation}\label{eq:B_over_power_law_1}
    \sB_{\sR,0}^{\tt RFM} \approx \left(\frac{n}{C_1}\right)^{-\alpha} \sB_{\sN,0}^{\tt RFM} + \left(\frac{n}{C_1}\right)^{-2\alpha r}\frac{C_2C_3-C_1C_4}{C_1-C_2}\,.
\end{equation}

\paragraph{Condition 2: $r \in [\frac{1}{2}, 1)$}
\[
\begin{aligned}
\sB_{\sR,0}^{\tt RFM} =&~ \frac{\nu_2^2}{1 - \Upsilon(\nu_1, \nu_2)} \left[ \< \btheta_*, (\bLambda + \nu_2)^{-2} \btheta_* \> + \chi(\nu_2) \< \btheta_*, \bLambda (\bLambda + \nu_2)^{-2} \btheta_* \> \right]\\
\approx&~\frac{\left(\frac{n}{C_1}\right)^{-\alpha}\left(C_1\left(C_4 n + C_5 \left(\frac{n}{C_1}\right)^{-\alpha (2r-1)} p\right)-C_2 n \left(C_4 + C_5 \left(\frac{n}{C_1}\right)^{-\alpha (2r-1)}\right)\right)}{(C1-C2)(p-n)},\\
\sB_{\sN,0}^{\tt RFM} =&~ \< \btheta_*, \bLambda ( \bLambda + \nu_2)^{-2} \btheta_* \> \cdot \frac{p}{p - {\rm df}_2(\nu_2)}\\
&~+ \frac{p}{n} \nu_2^2 \left( \< \btheta_*, (\bLambda + \nu_2)^{-2} \btheta_* \> + \chi(\nu_2) \< \btheta_*, \bLambda (\bLambda + \nu_2)^{-2} \btheta_* \> \right) \cdot \frac{\chi(\nu_2)}{1 - \Upsilon(\nu_1, \nu_2)}\\
\approx&~ \frac{\left(C_4+C_5\left(\frac{n}{C_1}\right)^{-\alpha(2r-1)}\right)p}{p-n}\,.
\end{aligned}
\]
Then we eliminate $p$ and obtain
\[
\begin{aligned}
\sB_{\sR,0}^{\tt RFM} \approx&~ \left(\frac{n}{C_1}\right)^{-\alpha} \sB_{\sN,0}^{\tt RFM} + \left(\frac{n}{C_1}\right)^{-\alpha}\frac{-C_1C_4+C_2C_4+C_2C_5\left(\frac{n}{C_1}\right)^{-\alpha(2r-1)}}{C_1-C_2}\\
\approx&~ \left(\frac{n}{C_1}\right)^{-\alpha} \sB_{\sN,0}^{\tt RFM} - \left(\frac{n}{C_1}\right)^{-\alpha}C_4\,.
\end{aligned}
\]
The last ``$\approx$'' holds because $\left(\frac{n}{C_1}\right)^{-\alpha(2r-1)} = o(1)$.

\paragraph{Condition 3: $r \in [1, \infty)$}
\[
\begin{aligned}
\sB_{\sR,0}^{\tt RFM} =&~ \frac{\nu_2^2}{1 - \Upsilon(\nu_1, \nu_2)} \left[ \< \btheta_*, (\bLambda + \nu_2)^{-2} \btheta_* \> + \chi(\nu_2) \< \btheta_*, \bLambda (\bLambda + \nu_2)^{-2} \btheta_* \> \right]\\
\approx&~\frac{\left(\frac{n}{C_1}\right)^{-2\alpha}\left(C_1\left(C_4 n \left(\frac{n}{C_1}\right)^{\alpha} + C_6 p\right) - C_2 n \left(C_6 + C_4 \left(\frac{n}{C_1}\right)^{\alpha}\right)\right)}{(C1-C2)(p-n)},\\
\sB_{\sN,0}^{\tt RFM} =&~ \< \btheta_*, \bLambda ( \bLambda + \nu_2)^{-2} \btheta_* \> \cdot \frac{p}{p - {\rm df}_2(\nu_2)}\\
&~+ \frac{p}{n} \nu_2^2 \left( \< \btheta_*, (\bLambda + \nu_2)^{-2} \btheta_* \> + \chi(\nu_2) \< \btheta_*, \bLambda (\bLambda + \nu_2)^{-2} \btheta_* \> \right) \cdot \frac{\chi(\nu_2)}{1 - \Upsilon(\nu_1, \nu_2)}\\
\approx&~ \frac{\left(C_4+C_6\left(\frac{n}{C_1}\right)^{-\alpha}\right)p}{p-n}\,.
\end{aligned}
\]
Then we eliminate $p$ and obtain
\[
\begin{aligned}
\sB_{\sR,0}^{\tt RFM} \approx&~ \left(\frac{n}{C_1}\right)^{-\alpha} \sB_{\sN,0}^{\tt RFM} + \left(\frac{n}{C_1}\right)^{-\alpha}\frac{-C_1C_4+C_2C_4+C_2C_6\left(\frac{n}{C_1}\right)^{-\alpha}}{C_1-C_2}\\
\approx&~ \left(\frac{n}{C_1}\right)^{-\alpha} \sB_{\sN,0}^{\tt RFM} - \left(\frac{n}{C_1}\right)^{-\alpha}C_4\,.
\end{aligned}
\]
The last ``$\approx$'' holds because $\left(\frac{n}{C_1}\right)^{-\alpha(2r-1)} = o(1)$.

Combining the above condition \(r \in [\frac{1}{2}, 1)\) and \(r \in [1, \infty)\), we have for \(r \in [\frac{1}{2}, \infty)\)

\begin{equation}\label{eq:B_over_power_law_2}
\sB_{\sR,0}^{\tt RFM} \approx \left(\frac{n}{C_1}\right)^{-\alpha} \sB_{\sN,0}^{\tt RFM} - \left(\frac{n}{C_1}\right)^{-\alpha}C_4\,.    
\end{equation}

From \cref{eq:V_over_power_law,eq:B_over_power_law_1,eq:B_over_power_law_2}, we know that the relationship between \(\sR_0^{\tt RFM}\) and \(\sN_0^{\tt RFM}\) in the over-parameterized regime can be written as
\[
\sR_0^{\tt RFM} \approx \left(\nicefrac{n}{C_\alpha}\right)^{-\alpha} \sN_0^{\tt RFM} + C_{n,\alpha,r,1}\,.
\]


\paragraph{The relationship in under-parameterized regime ($p < n$)} While in the under-parameterized regime ($p < n$), When $\lambda \to 0$, $\Tr(\bLambda(\bLambda + \nu_2)^{-1})$ will converge to $p$, which means $\nu_2$ will converge to $(\frac{p}{C_1})^{-\alpha}$ and $\nu_1$ will converge to 0, with $\frac{\lambda}{\nu_1}\to n-p$. 

Accordingly, in the under-parameterized regime
\[
\begin{aligned}
\Upsilon(\nu_1, \nu_2) = \frac{p}{n} \left[ \left( 1 - \frac{\nu_1}{\nu_2} \right)^2 + \left( \frac{\nu_1}{\nu_2} \right)^2 \frac{\operatorname{Tr}\left(\bLambda^2 (\bLambda + \nu_2)^{-2}\right)}{p - \operatorname{Tr}\left(\bLambda^2 (\bLambda + \nu_2)^{-2}\right)} \right] \to \frac{p}{n}\,,
\end{aligned}
\]
\[
\begin{aligned}
\chi(\nu_2) = \frac{\operatorname{Tr}\left(\bLambda (\bLambda + \nu_2)^{-2}\right)}{p - \operatorname{Tr}\left(\bLambda^2 (\bLambda + \nu_2)^{-2}\right)} \to \frac{1}{\nu_2} \approx (\frac{p}{C_1})^{\alpha}\,.
\end{aligned}
\]
Then we can further obtain that, for the variance
\[
\begin{aligned}
\sV_{\sR,0}^{\tt RFM} =&~ \sigma^2 \frac{\Upsilon(\nu_1, \nu_2)}{1 - \Upsilon(\nu_1, \nu_2)} \approx \sigma^2 \frac{p}{n-p},\\
\sV_{\sN,0}^{\tt RFM} =&~ \sigma^2 \frac{p}{n} \frac{\chi(\nu_2)}{1 - \Upsilon(\nu_1, \nu_2)} \approx \sigma^2 C_1^{-\alpha} \frac{p^{\alpha+1}}{n-p}.
\end{aligned}
\]
For the relationship in the under-parameterized regime, we separately consider two cases, i.e. $p \ll n$ and $p\to n$.

First, we derive the relationship in the under-parameterized regime ($p < n$) as $p \to n$, based on the relationship in the over-parameterized regime.
Recall the relationship between $\sV_{\sR,0}^{\tt RFM}$ and $\sV_{\sN,0}^{\tt RFM}$ in the over-parameterized regime, as presented in \cref{eq:V_over_power_law}, given by 
\[
\sV_{\sR,0}^{\tt RFM} \approx \left(\frac{n}{C_1}\right)^{-\alpha} \sV_{\sN,0}^{\tt RFM} + \sigma^2 (\alpha-2) =:h(\sV_{\sN,0}^{\tt RFM})\,.
\]
Substituting the expression for $\sV_{\sN,0}^{\tt RFM}$ in the under-parameterized regime into this relationship, we obtain
\[
\sV_{\sR,0}^{\tt RFM} \approx \left(\frac{n}{C_1}\right)^{-\alpha} \sigma^2 C_1^{-\alpha} \frac{p^{\alpha+1}}{n-p} + \sigma^2 (\alpha-2)\,,
\]
then we compute $\sV_{\sR,0}^{\tt RFM} - h(\sV_{\sN,0}^{\tt RFM})$ and obtain
\[
\begin{aligned}
    \sV_{\sR,0}^{\tt RFM} - h(\sV_{\sN,0}^{\tt RFM}) =&~ \sigma^2 \frac{p}{n-p} - \left(\frac{n}{C_1}\right)^{-\alpha} \sigma^2 C_1^{-\alpha} \frac{p^{\alpha+1}}{n-p} - \sigma^2 (\alpha-2)\\
    =&~ \sigma^2\left(\frac{p-p^{\alpha+1}n^{-\alpha}}{n-p}\right) - \sigma^2 (\alpha-2)\,.
\end{aligned}
\]
Taking limits on the left and right sides of the equation, we get
\[
\lim_{p \to n} \left(\sV_{\sR,0}^{\tt RFM} - h(\sV_{\sN,0}^{\tt RFM})\right) = 2\sigma^2\,.
\]
Then when $p \to n$, we have
\begin{equation}\label{eq:V_under_power_law}
    \sV_{\sR,0}^{\tt RFM} \approx \left(\frac{n}{C_1}\right)^{-\alpha} \sV_{\sN,0}^{\tt RFM} + \sigma^2 \alpha\,.
\end{equation}
For $p \ll n$, we have $\frac{1}{n-p} \approx \frac{1}{n}$, then 
\[
\begin{aligned}
\sV_{\sR,0}^{\tt RFM} =&~ \sigma^2 \frac{\Upsilon(\nu_1, \nu_2)}{1 - \Upsilon(\nu_1, \nu_2)} \approx \sigma^2 \frac{p}{n},\\
\sV_{\sN,0}^{\tt RFM} =&~ \sigma^2 \frac{p}{n} \frac{\chi(\nu_2)}{1 - \Upsilon(\nu_1, \nu_2)} \approx \sigma^2 C_1^{-\alpha} \frac{p^{\alpha+1}}{n}.
\end{aligned}
\]
Eliminate $p$ and we have
\[
\sV_{\sR,0}^{\tt RFM} \approx \left(\sigma^2\right)^{\frac{\alpha}{\alpha+1}} C_1^{\frac{\alpha}{\alpha+1}} \left(\sV_{\sR,0}^{\tt RFM}\right)^\frac{1}{\alpha+1}\,.
\]
Next, for the bias term we have
\[
\begin{aligned}
\sB_{\sR,0}^{\tt RFM} =&~ \frac{\nu_2^2}{1 - \Upsilon(\nu_1, \nu_2)} \left[ \< \btheta_*, (\bLambda + \nu_2)^{-2} \btheta_* \> + \chi(\nu_2) \< \btheta_*, \bLambda (\bLambda + \nu_2)^{-2} \btheta_* \> \right]\\
\approx&~ \frac{\nu_2}{1 - \Upsilon(\nu_1, \nu_2)} \< \btheta_*, (\bLambda + \nu_2)^{-1} \btheta_* \>\\
\approx&~ \frac{n}{n-p} C_3 \nu_2^{2r\wedge1}.\\
\sB_{\sN,0}^{\tt RFM} =&~ p \< \btheta_*, \bLambda ( \bLambda + \nu_2)^{-2} \btheta_* \> \cdot \frac{1}{p - {\rm df}_2(\nu_2)}\\
&~+ \frac{p}{n} \chi(\nu_2) \frac{\nu_2^2}{1 - \Upsilon(\nu_1, \nu_2)} \left[ \< \btheta_*, (\bLambda + \nu_2)^{-2} \btheta_* \> + \chi(\nu_2) \< \btheta_*, \bLambda (\bLambda + \nu_2)^{-2} \btheta_* \> \right] \\
\approx&~ p \< \btheta_*, \bLambda ( \bLambda + \nu_2)^{-2} \btheta_* \> \cdot \frac{1}{p - {\rm df}_2(\nu_2)} + \frac{p}{n} \chi(\nu_2) \frac{\nu_2}{1 - \Upsilon(\nu_1, \nu_2)} \< \btheta_*, (\bLambda + \nu_2)^{-1} \btheta_* \>\\
\approx&~ \frac{p}{p-\frac{C_2}{C_1}p} C_4 \nu_2^{(2r-1)\wedge0} + \frac{p}{n-p} C_3 \nu_2^{(2r-1)\wedge0}\\
\approx&~ \left(\frac{C_1C_4}{C_1-C_2} + \frac{p}{n-p}C_3\right) \nu_2^{(2r-1)\wedge0}.
\end{aligned}
\]
Then we use the approximation $\nu_2 \approx (\frac{p}{C_1})^{-\alpha}$ and obtain
\[
\begin{aligned}
\sB_{\sR,0}^{\tt RFM} 
\approx \frac{n}{n-p} C_3 \nu_2^{2r\wedge1} \approx \frac{n}{n-p} C_3 \left( \frac{p}{C_1} \right)^{-\alpha\left(2r\wedge1\right)},
\end{aligned}
\]
\[
\begin{aligned}
\sB_{\sN,0}^{\tt RFM} \approx \left(\frac{C_1C_4}{C_1-C_2}+\frac{p}{n-p}C_3\right) \nu_2^{(2r-1)\wedge0} \approx \left(\frac{C_1C_4}{C_1-C_2}+\frac{p}{n-p}C_3\right) \left(\frac{p}{C_1}\right)^{-\alpha\left[(2r-1)\wedge0\right]}.
\end{aligned}
\]
Similarly to the bias term, we derive the relationship in the under-parameterized regime ($p < n$) as $p \to n$, based on the relationship in the over-parameterized regime. And we discuss the relationship when $r \in (0, \frac{1}{2})$ and $r \in [\frac{1}{2}, \infty)$ separately.

\paragraph{Condition 1: $r \in (0, \frac{1}{2})$.}
Recall the relationship between $\sB_{\sR,0}^{\tt RFM}$ and $\sB_{\sN,0}^{\tt RFM}$ in the over-parameterized regime, as presented in \cref{eq:B_over_power_law_1}, given by: 
\[
\begin{aligned}
\sB_{\sR,0}^{\tt RFM} = \left(\frac{n}{C_1}\right)^{-\alpha} \sB_{\sN,0}^{\tt RFM} + \left(\frac{n}{C_1}\right)^{-2\alpha r}\frac{C_2C_3-C_1C_4}{C_1-C_2} =: f(\sB_{\sN,0}^{\tt RFM}).
\end{aligned}
\]
Substituting the expression for $\sB_{\sN,0}^{\tt RFM}$ in the under-parameterized regime into this relationship, we obtain:
\[
\begin{aligned}
f(\sB_{\sN,0}^{\tt RFM}) =&~ \left(\frac{n}{C_1}\right)^{-\alpha} \left(\frac{C_1C_4}{C_1-C_2}+\frac{p}{n-p}C_3\right) \left(\frac{p}{C_1}\right)^{-\alpha(2r-1)} + \left(\frac{n}{C_1}\right)^{-2\alpha r}\frac{C_2C_3-C_1C_4}{C_1-C_2},
\end{aligned}
\]
then we compute $\sB_{\sR,0}^{\tt RFM} - f(\sB_{\sN,0}^{\tt RFM})$ and obtain
\[
\begin{aligned}
\sB_{\sR,0}^{\tt RFM} - f(\sB_{\sN,0}^{\tt RFM}) = C_1^{2\alpha r}\Big(\frac{n}{n-p}C_3p^{-2\alpha r} - \frac{C_1C_4}{C_1-C_2}p^{-\alpha(2r-1)}n^{-\alpha} -\frac{p}{n-p}C_3p^{-\alpha(2r-1)}n^{-\alpha} - \frac{C_2C_3-C_1C_4}{C_1-C_2}n^{-2\alpha r}\Big).\\
\end{aligned}
\]
To simplify this equation, we begin by computing $\frac{n}{n-p}C_3p^{-2\alpha r} - \frac{p}{n-p}C_3p^{-\alpha(2r-1)}n^{-\alpha}$ and obtain
\[
\begin{aligned}
\frac{n}{n-p}C_3p^{-2\alpha r} - \frac{p}{n-p}C_3p^{-\alpha(2r-1)}n^{-\alpha} =&~ C_3p^{-\alpha(2r-1)} \left(\frac{n}{n-p}p^{-\alpha} - \frac{p}{n-p}n^{-\alpha} \right)\\
=&~ C_3p^{-\alpha(2r-1)}\frac{np^{-\alpha} - pn^{-\alpha}}{n-p},
\end{aligned}
\]
where $\frac{np^{-\alpha} - pn^{-\alpha}}{n-p}$ is monotonically decreasing in $p$ (monotonicity can be obtained by simple derivatives), and by applying L'Hôpital's rule, we have:
\[
\lim_{p \to n} \frac{np^{-\alpha} - pn^{-\alpha}}{n-p} = \lim_{p \to n} \frac{-\alpha n p^{-\alpha-1}-n^{-\alpha}}{-1} = (\alpha+1)n^{-\alpha}.
\]
Thus we have
\[
\begin{aligned}
\lim_{p \to n} C_3p^{-\alpha(2r-1)}\frac{np^{-\alpha} - pn^{-\alpha}}{n-p} = (\alpha+1)C_3n^{-2\alpha r}.
\end{aligned}
\]
Thus we have
\[
\begin{aligned}
&~\lim_{p \to n} C_1^{2\alpha r}\Big(C_3p^{-\alpha(2r-1)}\frac{np^{-\alpha} - pn^{-\alpha}}{n-p} - \frac{C_1C_4}{C_1-C_2}p^{-\alpha(2r-1)}n^{-\alpha} - \frac{C_2C_3-C_1C_4}{C_1-C_2}n^{-2\alpha r}\Big)\\
=&~ C_1^{2\alpha r}\Big((\alpha+1)C_3n^{-2\alpha r} - \frac{C_1C_4}{C_1-C_2}n^{-2\alpha r} - \frac{C_2C_3-C_1C_4}{C_1-C_2}n^{-2\alpha r}\Big)\\
=&~ C_1^{2\alpha r}C_3n^{-2\alpha r}\Big((\alpha+1) - \frac{C_2}{C_1-C_2}\Big).
\end{aligned}
\]
Recall that from \cref{eq:C1C2} we have
\[
\begin{aligned}
C_1 = \frac{\pi}{\alpha \sin\left(\nicefrac{\pi}{\alpha}\right)}\,, \quad C_2 = \frac{\pi(\alpha-1)}{\alpha^2 \sin\left(\nicefrac{\pi}{\alpha}\right)},
\end{aligned}
\]
thus 
\[
\begin{aligned}
(\alpha+1) - \frac{C_2}{C_1-C_2} = (\alpha+1) - \frac{ \frac{\pi(\alpha-1)}{\alpha^2 \sin\left(\nicefrac{\pi}{\alpha}\right)}}{\frac{\pi}{\alpha \sin\left(\nicefrac{\pi}{\alpha}\right)} - \frac{\pi(\alpha-1)}{\alpha^2 \sin\left(\nicefrac{\pi}{\alpha}\right)}} = 2.
\end{aligned}
\]
Finally, we have
\[
\begin{aligned}
\lim_{p \to n}\left( \sB_{\sR,0}^{\tt RFM} - f(\sB_{\sN,0}^{\tt RFM}) \right) = 2C_1^{2\alpha r}C_3n^{-2\alpha r} = 2C_3\left(\frac{n}{C_1}\right)^{-2\alpha r},
\end{aligned}
\]
and then the relationship between $\sB_{\sR,0}^{\tt RFM}$ and $\sB_{\sN,0}^{\tt RFM}$ is 
\begin{equation}\label{eq:B_under_power_law_1}
    \begin{split}
        \sB_{\sR,0}^{\tt RFM} \approx&~ \left(\frac{n}{C_1}\right)^{-\alpha} \sB_{\sN,0}^{\tt RFM} + \left(\frac{n}{C_1}\right)^{-2\alpha r}\frac{C_2C_3-C_1C_4}{C_1-C_2} + 2C_3\left(\frac{n}{C_1}\right)^{-2\alpha r}\\
        \approx&~ \left(\frac{n}{C_1}\right)^{-\alpha} \sB_{\sN,0}^{\tt RFM} + \left(\frac{n}{C_1}\right)^{-2\alpha r}\frac{2C_1C_3-C_2C_3-C_1C_4}{C_1-C_2}.
    \end{split}
\end{equation}


\paragraph{Condition 2: $r \in [\frac{1}{2}, \infty)$.}

In this condition, the approximation of $\sB_{\sR,0}^{\tt RFM}$ and $\sB_{\sN,0}^{\tt RFM}$ can be simplified to 
\[
\begin{aligned}
\sB_{\sR,0}^{\tt RFM} 
\approx \frac{n}{n-p} C_3 \nu_2^{2r\wedge1} \approx \frac{n}{n-p} C_3 \left( \frac{p}{C_1} \right)^{-\alpha\left(2r\wedge1\right)} = \frac{n}{n-p} C_3 \left( \frac{p}{C_1} \right)^{-\alpha}\,,
\end{aligned}
\]
\[
\begin{aligned}
\sB_{\sN,0}^{\tt RFM} \approx \left(\frac{C_1C_4}{C_1-C_2}+\frac{p}{n-p}C_3\right) \nu_2^{(2r-1)\wedge0} \approx \left(\frac{C_1C_4}{C_1-C_2}+\frac{p}{n-p}C_3\right) \left(\frac{p}{C_1}\right)^{-\alpha\left[(2r-1)\wedge0\right]} = \frac{C_1C_4}{C_1-C_2}+\frac{p}{n-p}C_3\,.
\end{aligned}
\]
Recall the relationship between $\sB_{\sR,0}^{\tt RFM}$ and $\sB_{\sN,0}^{\tt RFM}$ in the over-parameterized regime is presented in \cref{eq:B_over_power_law_2}, given by: 
\[
\begin{aligned}
\sB_{\sR,0}^{\tt RFM} \approx&~ \left(\frac{n}{C_1}\right)^{-\alpha} \sB_{\sN,0}^{\tt RFM} - \left(\frac{n}{C_1}\right)^{-\alpha}C_4 =: g(\sB_{\sN,0}^{\tt RFM})\,.
\end{aligned}
\]
Substituting the expression for $\sB_{\sN,0}^{\tt RFM}$ in the under-parameterized regime into this relationship, we obtain:
\[
\begin{aligned}
g(\sB_{\sN,0}^{\tt RFM}) =&~ \left(\frac{n}{C_1}\right)^{-\alpha} \left(\frac{C_1C_4}{C_1-C_2}+\frac{p}{n-p}C_3\right) - \left(\frac{n}{C_1}\right)^{-\alpha}C_4\,,
\end{aligned}
\]
then we compute $\sB_{\sR,0}^{\tt RFM} - g(\sB_{\sN,0}^{\tt RFM})$ and obtain
\[
\begin{aligned}
    \sB_{\sR,0}^{\tt RFM} - g(\sB_{\sN,0}^{\tt RFM}) = C_3 C_1^{\alpha} \frac{np^{-\alpha} - pn^{-\alpha}}{n-p} - \left(\frac{n}{C_1}\right)^{-\alpha} \left( \frac{C_2C_4}{C_1-C_2} \right)\,.
\end{aligned}
\]
Thus we have
\[
\begin{aligned}
    \lim_{p \to n}\left( \sB_{\sR,0}^{\tt RFM} - f(\sB_{\sN,0}^{\tt RFM}) \right) =&~ \left(\frac{n}{C_1}\right)^{-\alpha} \left((\alpha+1)C_3 - \frac{C_2C_4}{C_1-C_2}\right)\\
    \approx&~ \left(\frac{n}{C_1}\right)^{-\alpha} \left((\alpha+1)C_4 - \frac{C_2}{C_1-C_2}C_4\right)\\
    =&~ \left(\frac{n}{C_1}\right)^{-\alpha} 2 C_4\,,
\end{aligned}
\]
and the relationship between $\sB_{\sR,0}^{\tt RFM}$ and $\sB_{\sN,0}^{\tt RFM}$ is 
\begin{equation}\label{eq:B_under_power_law_2}
    \begin{split}
        \sB_{\sR,0}^{\tt RFM} \approx&~ \left(\frac{n}{C_1}\right)^{-\alpha} \sB_{\sN,0}^{\tt RFM} - \left(\frac{n}{C_1}\right)^{-\alpha}C_4 + \left(\frac{n}{C_1}\right)^{-\alpha}2C_4\\
        \approx&~ \left(\frac{n}{C_1}\right)^{-\alpha} \sB_{\sN,0}^{\tt RFM} + \left(\frac{n}{C_1}\right)^{-\alpha}C_4\,.
    \end{split}
\end{equation}


When $p \ll n$, we discuss cases $r \in (0, \frac{1}{2})$ and $r \in (\frac{1}{2}, \infty)$ separately. 

If $r \in (0, \frac{1}{2})$, we have $\frac{n}{n-p} \approx 1$ and $\frac{p}{n-p} \approx 0$, then
\[
\begin{aligned}
\sB_{\sR,0}^{\tt RFM} 
\approx C_3 \nu_2^{2r\wedge1} \approx C_3 \left( \frac{p}{C_1} \right)^{-\alpha2r},
\end{aligned}
\]
\[
\begin{aligned}
\sB_{\sN,0}^{\tt RFM} \approx \frac{C_1C_4}{C_1-C_2} \nu_2^{(2r-1)\wedge0} \approx \frac{C_1C_4}{C_1-C_2} \left(\frac{p}{C_1}\right)^{-\alpha\left(2r-1\right)}.
\end{aligned}
\]
Then we eliminate $p$ and obtain
\[
\begin{aligned}
\sB_{\sR,0}^{\tt RFM} \approx C_3 \left(\frac{C_1-C_2}{C_1C_4}\right)^{\nicefrac{2r}{(2r-1)}} \left(\sB_{\sN,0}^{\tt RFM}\right)^{\nicefrac{2r}{(2r-1)}}.
\end{aligned}
\]

If $2r \ge 1$, we have
\[
\begin{aligned}
\sB_{\sR,0}^{\tt RFM} \approx&~ \frac{n}{n-p} C_3 \nu_2 \approx \frac{n}{n-p} C_3\left(\frac{p}{C_1}\right)^{-\alpha},
\end{aligned}
\]
\[
\begin{aligned}
\sB_{\sN,0}^{\tt RFM} \approx&~ \frac{C_1C_4}{C_1-C_2} + \frac{p}{n-p}C_3 .
\end{aligned}
\]
Then we eliminate $p$ and obtain
\[
\begin{aligned}
\sB_{\sR,0}^{\tt RFM} \approx \left(\frac{C_1C_3-C_2C_3-C_1C_4}{C_1-C_2}+\sB_{\sN,0}^{\tt RFM}\right)\left(\frac{n\left(\sB_{\sN,0}^{\tt RFM}-\frac{C_1C_4}{C_1-C_2}\right)}{C_1\left(C_3+\sB_{\sN,0}^{\tt RFM}-\frac{C_1C_4}{C_1-C_2}\right)}\right)^{-\alpha}.
\end{aligned}
\]

From \cref{eq:V_under_power_law,eq:B_under_power_law_1,eq:B_under_power_law_2}, we know that the relationship between \(\sR_0^{\tt RFM}\) and \(\sN_0^{\tt RFM}\) in the under-parameterized regime when \(p \to n\) can be written as
\[
\sR_0^{\tt RFM} \approx \left(\nicefrac{n}{C_\alpha}\right)^{-\alpha} \sN_0^{\tt RFM} + C_{n,\alpha,r,2}\,.
\]

\end{proof}



    



\section{Scaling laws}
\label{app:scaling_law}
To derive the scaling laws based on norm-based capacity, we first give the decay rate of the $\ell_2$ norm w.r.t.\ \(n\).

The rate of the deterministic equivalent of the random feature ridge regression estimator's $\ell_2$ norm  is given by
\[
\sN_\lambda^{\tt RFM} = \Theta\left(n^{-\gamma_{\sB_{\sN,\lambda}^{\tt RFM}}}+\sigma^2n^{-\gamma_{\sV_{\sN,\lambda}^{\tt RFM}}}\right) = \Theta\left(n^{-\gamma_{\sN_\lambda^{\tt RFM}}}\right)\,,
\]
where $\gamma_{\sN_\lambda^{\tt RFM}} := \gamma_{\sB_{\sN,\lambda}^{\tt RFM}} \wedge \gamma_{\sV_{\sN,\lambda}^{\tt RFM}}$ for $\sigma^2 \neq 0$.
    
\subsection{Variance term}
Using \cref{eq:rate_nu2,eq:rates:Upsilon2,eq:rates:chi}, we have
\[
\begin{aligned}
\sV_{\sN,\lambda}^{\tt RFM} =&~ \sigma^2 \frac{p}{n} \frac{\chi(\nu_2)}{1 - \Upsilon(\nu_1, \nu_2)} =~ n^{q-1}n^{-q}O\left(\nu_2^{-1-\nicefrac{1}{\alpha}}\right)\\
=&~O\left(n^{-\left(1- \left(\alpha+1\right)\left(1 \wedge q \wedge \nicefrac{\ell}{\alpha}\right)\right)}\right).
\end{aligned}
\]
Hence, the variance term of the norm decays with $n$ with rate
\[
\gamma_{\sV_{\sN,\lambda}^{\tt RFM}}(\ell, q) = 1 - \left(\alpha+1\right)\left(\frac{\ell}{\alpha}\wedge q\wedge 1\right). 
\]

\subsection{Bias term}
First, one could notice, using the integral approximation and \cref{eq:rate_T,eq:rate_nu2}, that
\[
\begin{aligned}
\frac{p}{p - {\rm df}_2(\nu_2)} =&~ \left(1 + n^{-q} O\left(\nu_2^{-\nicefrac{1}{\alpha}}\right) \right) = \left(1 + O\left(n^{-q}n^{\left(1 \wedge q \wedge \nicefrac{\ell}{\alpha}\right)}\right) \right) = O\left(1\right).
\end{aligned}
\]
Thus for the bias term, using \cref{eq:rate_T,eq:rate_nu2,eq:rates:Upsilon2,eq:rates:chi} we have
\[
\begin{aligned}
\sB_{\sN,\lambda}^{\tt RFM} =&~ \< \btheta_*, \bLambda ( \bLambda + \nu_2)^{-2} \btheta_* \> \cdot \frac{p}{p - {\rm df}_2(\nu_2)}\\
&~+ \frac{p}{n} \nu_2^2 \left( \< \btheta_*, (\bLambda + \nu_2)^{-2} \btheta_* \> + \chi(\nu_2) \< \btheta_*, \bLambda (\bLambda + \nu_2)^{-2} \btheta_* \> \right) \cdot \frac{\chi(\nu_2)}{1 - \Upsilon(\nu_1, \nu_2)}\\
=&~ T_{2r+1, 2}^1(\nu_2) + n^{q-1}\nu_2^2\left( T_{2r,2}^1(\nu_2) + \chi(\nu_2) T_{2r+1,2}^1(\nu_2)\right)\chi(\nu_2)\\
=&~ \nu_2^{(2r-1)\wedge 0} + n^{q-1} \nu_2^2 O\left(\nu_2^{(2r-2)\wedge 0} +  n^{-q}\nu_2^{-1-\nicefrac{1}{\alpha}+(2r-1)\wedge 0}\right) n^{-q}O\left(\nu_2^{-1-\nicefrac{1}{\alpha}}\right)\\
=&~ \nu_2^{(2r-1)\wedge 0} + n^{-1} O\left(\nu_2^{2r\wedge 2} +  n^{-q}\nu_2^{-\nicefrac{1}{\alpha}+2r\wedge 1}\right) O\left(\nu_2^{-1-\nicefrac{1}{\alpha}}\right)\\
=&~ O\left( n^{-\alpha \left(1 \wedge q \wedge \nicefrac{\ell}{\alpha}\right) \left[(2r-1)\wedge 0\right]} \right)\\
&~+ O\left(n^{-\alpha \left(1 \wedge q \wedge \nicefrac{\ell}{\alpha}\right) \left[(2r-1)\wedge 1\right] + \left(1 \wedge q \wedge \nicefrac{\ell}{\alpha}\right) - 1}
+ 
n^{-\alpha \left(1 \wedge q \wedge \nicefrac{\ell}{\alpha}\right) \left[(2r-1)\wedge 0\right] + 2\left(1 \wedge q \wedge \nicefrac{\ell}{\alpha}\right) - 1 - q}\right)\\
=&~ O\left( n^{-\alpha \left(1 \wedge q \wedge \nicefrac{\ell}{\alpha}\right) \left[(2r-1)\wedge 0\right]}
+ 
n^{-\alpha \left(1 \wedge q \wedge \nicefrac{\ell}{\alpha}\right) \left[(2r-1)\wedge 1\right] + \left(1 \wedge q \wedge \nicefrac{\ell}{\alpha}\right) - 1}
\right)\\
=&~ O\left( n^{-\alpha \left(1 \wedge q \wedge \nicefrac{\ell}{\alpha}\right) \left[(2r-1)\wedge 0\right]}\right).
\end{aligned}
\]
Hence, the bias term of the norm decays with $n$ with rate
\[
\begin{aligned}
\gamma_{\sB_{\sN,\lambda}^{\tt RFM}}(\ell, q) =&~ \alpha \left(1 \wedge q \wedge \nicefrac{\ell}{\alpha}\right) \left[(2r-1)\wedge 0\right].
\end{aligned}
\]
Recalling that we have
\[
\gamma_{\sN_\lambda^{\tt RFM}} := \gamma_{\sB_{\sN,\lambda}^{\tt RFM}} \wedge \gamma_{\sV_{\sN,\lambda}^{\tt RFM}}\,,
\]
according to which, we obtain the norm exponent $\gamma_{\sN_\lambda^{\tt RFM}}$ as a function of $\ell$ and $q$, showing in \cref{fig:scaling_law}. As observed in \cref{fig:scaling_law}, $\gamma_{\sN_\lambda^{\tt RFM}}$ is non-positive across all regions, indicating that the norm either increases or remains constant with \(n\) in every case.

\begin{figure}[H]
    \centering
    \includegraphics[width=0.6\textwidth]{arxiv_version/figures/Scaling_Law/scaling_law_norm.pdf} 
    \caption{The norm rate $\gamma_{\sN_\lambda^{\tt RFM}}$ as a function of $(\ell,q)$. Variance dominated region is colored by {\color{regionorange}orange}, {\color{regionyellow}yellow} and {\color{regionbrown}brown}, bias dominated region is colored by {\color{regionblue}blue} and {\color{regiongreen}green}.} 
    \label{fig:scaling_law} 
\end{figure}

Next for the condition $r \in (0, \frac{1}{2})$, we derive the scaling law under norm-based capacity.

\paragraph{Region 1: $\ell > \alpha$ and $q > 1$}

In this region, according to \citet[Corollary 4.1]{defilippis2024dimension}, we have
\[
\sR_\lambda^{\tt RFM} = \Theta\left( n^{-0} \right) = \Theta\left( 1 \right)\,,
\]
and according to \cref{fig:scaling_law}, we have
\[
\sN_\lambda^{\tt RFM} = \Theta\left( n^{\alpha} \right)\,,
\]
combing the above rate, we can obtain that
\[
\sR_\lambda^{\tt RFM} = \Theta\left( n^{-\alpha} \cdot \sN_\lambda^{\tt RFM} \right)\,.
\]

\paragraph{Region 2: $\frac{\alpha}{2\alpha r+1} < \ell < \alpha$ and $q > \frac{\ell}{\alpha}$}

In this region, according to \citet[Corollary 4.1]{defilippis2024dimension}, we have
\[
\sR_\lambda^{\tt RFM} = \Theta\left( n^{-\left(1-\frac{\ell}{\alpha}\right)} \right)\,,
\]
and according to \cref{fig:scaling_law}, we have
\[
\sN_\lambda^{\tt RFM} = \Theta\left( n^{-\left(1-\frac{(\alpha+1)\ell}{\alpha}\right)} \right)\,,
\]
combing the above rate, we can obtain that
\[
\sR_\lambda^{\tt RFM} = \Theta\left( n^{-\ell} \cdot \sN_\lambda^{\tt RFM} \right)\,.
\]

\paragraph{Region 3: $\frac{1}{2\alpha r+1} < q < 1$ and $q < \frac{\ell}{\alpha}$}

In this region, according to \citet[Corollary 4.1]{defilippis2024dimension}, we have
\[
\sR_\lambda^{\tt RFM} = \Theta\left( n^{-\left(1-q\right)} \right)\,,
\]
and according to \cref{fig:scaling_law}, we have
\[
\sN_\lambda^{\tt RFM} = \Theta\left( n^{-\left(1-(\alpha+1)q\right)} \right)\,,
\]
combing the above rate and eliminate $q$, we can obtain that
\[
\sR_\lambda^{\tt RFM} = \Theta\left( n^{-\frac{\alpha}{\alpha+1}} \cdot \left(\sN_\lambda^{\tt RFM}\right)^{\frac{1}{\alpha+1}} \right)\,.
\]

\paragraph{Region 4: $\ell < \frac{\alpha}{2\alpha r+1}$ and $q > \frac{\ell}{\alpha}$}

In this region, according to \citet[Corollary 4.1]{defilippis2024dimension}, we have
\[
\sR_\lambda^{\tt RFM} = \Theta\left( n^{-2\ell r} \right)\,,
\]
and according to \cref{fig:scaling_law}, we have
\[
\sN_\lambda^{\tt RFM} = \Theta\left( n^{-\ell(2r-1)} \right)\,,
\]
combing the above rate, we can obtain that
\[
\sR_\lambda^{\tt RFM} = \Theta\left( n^{-1} \cdot \sN_\lambda^{\tt RFM} \right)\,.
\]

\paragraph{Region 5: $q < \frac{1}{2\alpha r+1}$ and $q < \frac{\ell}{\alpha}$}

In this region, according to \citet[Corollary 4.1]{defilippis2024dimension}, we have
\[
\sR_\lambda^{\tt RFM} = \Theta\left( n^{-2\alpha q r} \right)\,,
\]
and according to \cref{fig:scaling_law}, we have
\[
\sN_\lambda^{\tt RFM} = \Theta\left( n^{-\alpha q(2r-1)} \right)\,,
\]
combing the above rate, we can obtain that
\[
\sR_\lambda^{\tt RFM} = \Theta\left( n^0 \cdot \left(\sN_\lambda^{\tt RFM}\right)^{-\frac{2r}{1-2r}} \right)\,.
\]






\subsection{Experimental Setup}
\label{section:experimental_setup}
\textbf{Datasets:} Table~\ref{tab:datasets} provides a detailed breakdown of the SOTA intrusion datasets utilized in our study. 
%For each dataset we follow the data preparation steps outlined in section~\ref{section:data_preparation}. 
% \sean{is this section necessary with reduced page limit?}
% \begin{enumerate}
%     \item X-IIoTID \cite{al2021x}: The dataset consists of 59 features which are collected with the independence of devices and connectivity, generating a holistic intrusion data set to represent the heterogeneity of IIoT systems. It includes novel IIoT connectivity protocols, activities of various devices, and attack scenarios.  
%     \item WUSTL-IIoT \cite{zolanvari2021wustl}: WUSTL-IIoT aims to emulate real-world industrial systems. The dataset is deliberately unbalanced to imitate real-world industrial control systems, consisting of 41 features and 1,194,464 observations.
%     \item CICIDS2017 \cite{Sharafaldin2018TowardGA} The CICIDS2017 dataset includes a comprehensive collection of benign and malicious network traffic. It contains 80 features and represents a broad range of attacks, such as DoS, DDoS, Brute Force, XSS, and SQL Injection, across more than 2.8 million network flows. The dataset is widely used in evaluating intrusion detection systems.
%     \item UNSW-NB15 \cite{moustafa2015unsw, moustafa2016evaluation, moustafa2017novel, moustafa2017big, sarhan2020netflow} UNSW-NB15 is a comprehensive network intrusion dataset created by the University of New South Wales. It contains 49 features representing normal and malicious activities generated using IXIA's network traffic generator, covering a variety of contemporary attack types. 
% \end{enumerate}
For IIoT intrusion, we use IIoT datasets X-IIoTID \cite{al2021x} and WUSTL-IIoT \cite{zolanvari2021wustl}. We also include commonly used network intrusion datasets CICIDS2017 \cite{Sharafaldin2018TowardGA} and UNSW-NB15 \cite{moustafa2015unsw}. For X-IIoTID \cite{al2021x}, CICIDS2017 \cite{Sharafaldin2018TowardGA}, and UNSW-NB15 \cite{moustafa2015unsw}, we split the data across five experiences such that each experience contains two to four attacks. For WUSTL-IIoT \cite{zolanvari2021wustl}, we split the data across four experiences such that each experience contains one attack. We perform this data split to simulate an evolving data stream with emerging cyber attacks over time where each experience contains different attacks. 


%%%%%%%%%%%%%%%%%%%%%%%%%%%%%%%%%%%%%%%%%%%%%%%%%%%%%%%%%%%%%%%%%%%%%%%%%%%
\begin{table}[h]
    \caption{Selected Intrusion Datasets}
    \centering
    \label{tab:datasets}
    \resizebox{.99\columnwidth}{!}{
    \begin{tabular}{c|c|c|c|c}
    \hline
    Dataset    & Size      & Normal Data & Attack Data & Attack Types \\ 
    \hline
    X-IIoTID \cite{al2021x}   & 820,502   & 421,417     & 399,417     & 18           \\
    \hline
    WUSTL-IIoT \cite{zolanvari2021wustl} & 1,194,464 & 1,107,448   & 87,016      & 4       \\
    \hline
    CICIDS2017 \cite{Sharafaldin2018TowardGA} & 2,830,743 & 2,273,097 & 557,646 & 15 \\
    \hline
    UNSW-NB15 \cite{moustafa2015unsw}
 & 257,673 & 164,673 & 93,000 & 10 \\
    \hline
    \end{tabular}}
\end{table}
%%%%%%%%%%%%%%%%%%%%%%%%%%%%%%%%%%%%%%%%%%%%%%%%%%%%%%%%%%%%%%%%%%%%%%%%%%%

\textbf{Baselines:} %Due to the novelty of this problem formulation, there are no directly comparable methods. However, the most similar widely studied problem would be unsupervised continual learning (UCL). Therefore, 
We evaluate our algorithm against two SOTA unsupervised continual learning (UCL) algorithms: the Autonomous Deep Clustering Network (\textbf{ADCN}) \cite{ashfahani2023unsupervised}, and an autoencoder paired with K-Means clustering. The autoencoder K-Means model is combined with Learning without Forgetting \cite{lwf2019Li} continual learning loss; we refer to this model as \textbf{LwF}. Note that both \textbf{ADCN} and \textbf{LwF} require a small amount of labeled normal and attack data to perform classification. We also compare our approach against SOTA ND methods: local outlier factor (\textbf{LOF})\cite{Faber_2024}, one-class support vector machine (\textbf{OC-SVM})\cite{Faber_2024}, principal component analysis (\textbf{PCA})\cite{rios2022incdfm}, and Deep Isolation Forest (\textbf{DIF}) \cite{xu2023deep}. 
%We train the ND algorithms on the clean subset of normal data, $N_c$, and evaluate their performance on the remainder of the dataset. 
Since these ND models cannot be retrained on unlabeled contaminated data, continual learning is not feasible for these methods.

%an autoencoder with K-Means clustering paired with SOTA Learning without Forgetting (LwF) continual loss (LwF) \cite{lwf2019Li}.
%Notably, many SOTA UCL algorithms rely on image-specific contrastive pairs, which is not directly applicable to intrusion detection \cite{madaan2022representational, yu2023scale, fini2022self, liu2024unsupervised}.

%%%%%%%%%%%%%%%%%%%%%%%%%%%%%%%%%%%%%%%%%%%%%%%%%%%%%%%
\begin{figure*}
    \centering
    \includegraphics[width=.95\linewidth]{figures/cl_experiments.pdf}
    \caption{Continual learning metric results of ADCN\cite{ashfahani2023unsupervised}, LwF\cite{lwf2019Li}, and \Design{}}
    \label{fig:continual_methods_results}
\end{figure*}
%%%%%%%%%%%%%%%%%%%%%%%%%%%%%%%%%%%%%%%%%%%%%%%%%%%%%%%

\textbf{Evaluation Metrics:} To evaluate the model performance, we report $F_{1}$ score. Since there is a class imbalance within these datasets, to simulate real world IDS, $F_{1}$ score gives an accurate idea on attack detection. For the continual learning methods, we evaluate their performance at the end of each training experience on all experience test sets. This generates a matrix of $F_{1}$ score results $R_{ij}$ such that $i$ is the current training experience, and $j$ is the testing experience. To summarize this matrix of results, we report widely used CL metrics \cite{diaz2018don}: average $F_{1}$ score on current experience (AVG), forward transfer (FwdTrans), and backward transfer (BwdTrans). For a matrix $R_{ij}$ with $m$ total experiences, our metrics are formulated as follows: $\text{AVG}_{F_1} = \frac{\sum_{i = j} R_{ij}}{m}$; $\text{FwdTrans}_{F_1} = \frac{\sum_{j>i} R_{ij}}{\frac{m * (m-1)}{2}}$; $\text{BwdTrans}_{F_1} = \frac{\sum_{i}^m R_{mi} - R_{ii}}{\frac{m * (m-1)}{2}}$.
AVG is the average performance on the current test experience at every point of training. FwdTrans is the average performance on ``future'' experiences, which simulates performance on zero-day attacks. Finally, BwdTrans is the average change in performance of ``past'' test experiences at a ``future'' point of training. A negative BwdTrans indicates catastrophic forgetting, whereas a positive BwdTrans  indicates the model actually improved performance on past experiences after learning a future experience. Overall, AVG measures seen attacks, FwdTrans measures zero-day attacks, and BwdTrans measures forgetting. For all metrics, a higher positive result indicates a better performance. 

We also report the threshold-free metric Precision-Recall Area Under the Curve (PR-AUC) \cite{praucDavid06}. Since \Design{} requires selecting a threshold, PR-AUC allows us to assess model performance independently of the threshold. We choose PR-AUC over Receiver Operating Characteristic Area Under the Curve (ROC-AUC) because ROC-AUC can give misleadingly high results in the presence of class imbalance \cite{praucDavid06}.

\textbf{Hyperparameters:} %For $L_{CND}$ hyperparameters are the number of K-Means clusters $K$, the reconstruction loss strength $\lambda_R$,  the continual learning loss strength $\lambda_{CL}$, and the cluster separation loss margin $m$. 
We utilize \textit{elbow method} \cite{han2011data} for determining the number of clusters $K$. 
%It tests a range of $K$ values and then selects the value   where there is a significant change in slope, called the elbow point. 
%This resulted in $K$ values between 100-500. 
We set $\lambda_R$ and $\lambda_{CL}$ to 0.1, and for $m$ we use 2 after careful experimentation. For the AE modules of \Design{}, we use 4-layer MLP with 256 neurons in the hidden layers. We train it using Adam optimizer \cite{kingma2017adammethods} with a learning rate of 0.001. For PCA, we use the explained variance method and set it to 95\% \cite{rios2022incdfm}.

\textbf{Hardware:} We run our experiments on NVIDIA GeForce RTX 3090 GPU, with a AMD EPYC 7343 16-Core processor.

\subsection{Results}

\textbf{Continual Learning Comparison:} Fig.~\ref{fig:continual_methods_results} presents the results of our approach \Design{} compared with ADCN\cite{ashfahani2023unsupervised} and LwF\cite{lwf2019Li}. \Design{} shows the best performance on both seen (AVG) and unseen (FwdTrans) attacks across all datasets. \Design{} also has the highest BwdTrans on all except one dataset (UNSW-NB15). The average BwdTrans of \Design{} (0.87\%) is higher than the average BwdTrans of both ADCN (-0.06\%) and LwF (0.09\%). Notably, the BwdTrans of \Design{} is positive for three datasets. Indicating past experiences actually improve after training on future experiences for these datasets. Given the high FwdTrans as well, our approach finds features that generalize well to future experiences. 

Table~\ref{tab:improvement} shows the improvement of \Design{} over the UCL baselines on all datasets. Bold and underlined cases indicate the best and the second best improvements with respect to each metric, respectively. These improvements were calculated by comparing the performance of \Design{} to the baselines, where the improvement values represent the proportional increase over the baseline performance. We do not include BwdTrans because a proportional increase does not make sense for a metric that can be negative. \Design{} has up to $4.50\times$ and $6.1\times$ AVG improvement on ADCN and LwF, respectively. In addition, \Design{} has up to $6.47\times$ and $3.47\times$ FwdTrans improvement on ADCN and LwF. Averaged across all datasets, \Design{} shows a $1.88\times$ and $1.78\times$ improvement on AVG, and a $2.63\times$ and $1.60\times$ improvement on FwdTrans, compared to ADCN and LwF, respectively. %These results underscore the benefit of our continual novelty detection method \Design{}. The notably high FwdTrans score emphasizes how novelty detection can be used to identify unseen anomalous data, thereby significantly enhancing performance on zero-day attacks.

Overall, these results highlight the benefit of continual ND over UCL methods for IDS. \Design{}, with its PCA-based novelty detector, excels by effectively harnessing the normal data to identify attacks. A key strength of our approach lies in the assumption that normal data forms a distinct class, while everything else is treated as anomalous. This assumption is particularly well-suited to IDS. In contrast, methods like ADCN and LwF do not make this distinction where they handle both normal and attack data similarly, limiting their ability to fully exploit the inherent structure of the data. 



% %%%%%%%%%%%%%%%%%%%%%%%%%%%%%%%%%%%%%%%%%%%%%%%%%%%%%%%
% \begin{table}[]
% \centering
% \caption{\Design{} Percentage Improvement over UCL Baselines on AVG and FwdTrans}
% \label{tab:improvement}
% \begin{tabular}{|c|c|c|c|}
% \hline
% Baseline      & Dataset    & AVG  & FwdTrans  \\ \hline
% ADCN\cite{ashfahani2023unsupervised}          & X-IIoTID   & 101.88\%        & 400.35\%        \\ \cline{2-4} 
%               & WUSTL-IIoT & 349.86\%        & 546.68\%        \\ \cline{2-4} 
%               & CICIDS2017 & 37.19\%         & 73.46\%         \\ \cline{2-4} 
%               & UNSW-NB15  & 29.25\%         & 43.90\%         \\ \hline
% LwF\cite{lwf2019Li} & X-IIoTID   & 46.43\%         & 35.39\%         \\ \cline{2-4} 
%               & WUSTL-IIoT & 510.92\%        & 246.81\%        \\ \cline{2-4} 
%               & CICIDS2017 & 92.72\%         & 163.81\%        \\ \cline{2-4} 
%               & UNSW-NB15  & 11.07\%         & 2.20\%          \\ \hline
% \end{tabular}
% \end{table}
% %%%%%%%%%%%%%%%%%%%%%%%%%%%%%%%%%%%%%%%%%%%%%%%%%%%%%%%

%%%%%%%%%%%%%%%%%%%%%%%%%%%%%%%%%%%%%%%%%%%%%%%%%%%%%%%
\begin{table}[]
\centering
\caption{\Design{} Improvement over UCL Baselines}
\label{tab:improvement}
\scalebox{1}{
\begin{tabular}{|c|c|c|c|}
\hline
Baseline      & Dataset    & AVG  & FwdTrans  \\ \hline
ADCN\cite{ashfahani2023unsupervised}  & X-IIoTID   & $\underline{2.02\times}$  & $\underline{5.00\times}$   \\ \cline{2-4} 
                                      & WUSTL-IIoT & $\mathbf{4.50\times}$  & $\mathbf{6.47\times}$   \\ \cline{2-4} 
                                      & CICIDS2017 & $1.37\times$  & $1.73\times$   \\ \cline{2-4} 
                                      & UNSW-NB15  & $1.29\times$  & $1.44\times$   \\ \hline
LwF\cite{lwf2019Li}                   & X-IIoTID   & $1.46\times$  & $1.35\times$   \\ \cline{2-4} 
                                      & WUSTL-IIoT & $\mathbf{6.11\times}$  & $\mathbf{3.47\times}$   \\ \cline{2-4} 
                                      & CICIDS2017 & $\underline{1.93\times}$  & $\underline{2.64\times}$   \\ \cline{2-4} 
                                      & UNSW-NB15  & $1.11\times$  & $1.02\times$   \\ \hline
\end{tabular}}
\end{table}

%%%%%%%%%%%%%%%%%%%%%%%%%%%%%%%%%%%%%%%%%%%%%%%%%%%%%%%

%Figure~\ref{fig:XIIoT_graph} shows the $F_{1}$ score of ADCN and \Design{} for each experience on both datasets. Similarly, we use green and red colors for \Design{} and ADCN respectively. Notably for \Design{}, the $F_{1}$ score of each experience has little change over training time. This highlights the strength of novelty detection for IDSs, as even before seeing attacks \Design{} has good performance. On the other hand, ADCN test experiences do not improve until the associated training experience, meaning ADCN does not have an ability to generalize to future attacks. ADCN utilizes a subset of labeled data to assign labels to clusters. This subset of labeled might be causing ADCN to overfit to the attacks within the current experience, therefore leading ADCN to not generalize well. We can also clearly see that our approach is consistently better (higher $F_{1}$ score) than the state-of-the-art ADCN. 

% %%%%%%%%%%%%%%%%%%%%%%%%%%%%%%%%%%%%%%%%%%%%%%%%%%%%%%%
% \begin{figure*}[t]
%     \centering
%     \begin{subfigure}[t]{\linewidth}
%         \centering
%         \includegraphics[width=\linewidth]{figures/X-IIoTID-experiences.pdf}
%         \caption{X-IIoTID}
%         \label{fig:ADCN_XIIoT_results}
%     \end{subfigure}
%     \begin{subfigure}[t]{\linewidth}
%         \centering
%         \includegraphics[width=\linewidth]{figures/WUSTL-IIoT-experiences.pdf}
%         \caption{WUSTL-IIoT}
%         \label{fig:WUSTL-}
%     \end{subfigure}
%     \caption{$F_1$ Score of ADCN and \Design{} of each test experience over training experiences.}
%     \label{fig:XIIoT_graph}
% \end{figure*}
% %%%%%%%%%%%%%%%%%%%%%%%%%%%%%%%%%%%%%%%%%%%%%%%%%%%%%%%

\textbf{Novelty Detectors Comparison:} Fig.~\ref{fig:novelty_methods_results} compares LOF\cite{Faber_2024}, OC-SVM\cite{Faber_2024}, PCA\cite{rios2022incdfm}, and DIF \cite{xu2023deep} with \Design{} on all datasets. The average $F_{1}$ score of the novelty detection methods are compared to the AVG of \Design{}.  It can be seen \Design{} outperforms all other methods across all datasets. The two best performing methods are DIF and PCA. The average $F_{1}$ score improvement across all datasets of \Design{} is $1.16\times$ and $1.08\times$ over DIF and PCA, respectively. These results highlight the critical role of leveraging information from unsupervised data streams. Unlike these ND algorithms, \Design{} is capable of continuously learning from this unsupervised data, enabling it to enhance PCA reconstruction over time. By integrating evolving data patterns, \Design{} not only adapts to new anomalies but also improves its overall detection accuracy, demonstrating a clear advantage in dynamic environments.

%Given that \Design{} employs PCA detection, this indicates that the CFE effectively extracts useful features from the unlabeled training experiences. T

%%%%%%%%%%%%%%%%%%%%%%%%%%%%%%%%%%%%%%%%%%%%%%%%%%%%%%%   
\begin{figure}
    \centering
    \includegraphics[width=0.9\linewidth]{figures/novelty_detectors_experiments.pdf}
    \caption{Average $F_1$ score on all experiences of \Design{} and novelty detection methods: LOF, OC-SVM, PCA, DIF}
    \label{fig:novelty_methods_results}
\end{figure}
%%%%%%%%%%%%%%%%%%%%%%%%%%%%%%%%%%%%%%%%%%%%%%%%%%%%%%%
%%%%%%%%%%%%%%%%%%%%%%%%%%%%%%%%%%%%%%%%%%%%%%%%%%%%%%% 
\begin{figure}
    \centering
    \includegraphics[width=0.86\linewidth]{figures/novelty_detectors_pr_auc.pdf}
    \caption{Thresholding Free Evaluation of \Design{}}
    \label{fig:thresholding_free}
\end{figure}

%%%%%%%%%%%%%%%%%%%%%%%%%%%%%%%%%%%%%%%%%%%%%%%%%%%%%%%

\textbf{Pre-threshold Evaluation:} While thresholding plays a crucial role in attack decision-making, evaluating model prediction performance before applying threshold is also important. The UCL algorithms (ADCN\cite{ashfahani2023unsupervised} and LwF\cite{lwf2019Li}) do not output anomaly scores because they select classes based on the closest labeled cluster. Therefore we compare against the two best ND methods: DIF\cite{xu2023deep} and PCA\cite{rios2022incdfm}. Fig.~\ref{fig:thresholding_free} presents the PR-AUC values of DIF, PCA, and \Design{}. It can be seen that \Design{} provides the best threshold free results, which aligns with the threshold-based results presented earlier. The strong performance of \Design{} in both pre-threshold and threshold-based evaluations demonstrates that the model is robust regardless of the decision threshold. 

\subsection{Ablation Study}

To demonstrate the impact of our loss function components, we perform an ablation study. Table~\ref{tab:ablation_loss} shows the results of \Design{} with each loss function removed to demonstrate their individual effectiveness. Bold and underlined cases indicate the best and the second best performances with respect to each metric, respectively. \Design{} without reconstruction loss ($L_R$) and \Design{} without cluster separation loss ($L_{CS}$) performs worse in all categories. \Design{} without both $L_R$ and continual learning loss ($L_{CL}$) actually performs better AVG but has worse BwdTrans and FwdTrans. AVG does not account for past experiences, so the significantly negative BwdTrans indicates \Design{} w/o $L_R$ and $L_{CL}$ forgets, and therefore would perform worse on those experiences in the future. This would make sense as a regularization loss to improve continual learning would slightly decrease performance in non-continual scenario. Overall \Design{} has the best results when taking every metric category into account. Notably the low BwdTrans and FwdTrans of \Design{} (w/o $L_R$) showcases how the reconstruction loss helps \Design{} generalize better to unseen and past data. This highlights the power of $L_R$ to provide good features for continual learning. 

%%%%%%%%%%%%%%%%%%%%%%%%%%%%%%%%%%%%%%%%%%%%%%%%%%%%%%%%%%%%%%%%%%%%%
\begin{table}[]
\caption{Ablation Study of \Design{} Loss Functions}
\label{tab:ablation_loss}
\centering
\begin{tabular}{|c|c|c|c|}
\hline
Strategy                         & AVG              & BwdTrans        & FwdTrans         \\ \hline
CND-IDS                          &\underline{76.92\%}    & \textbf{0.87\%} & \textbf{73.70\%} \\ \hline
CND-IDS (w/o $L_{CS}$)           & 66.23\%          & \underline{0.09\%}    & 70.26\%          \\ \hline
CND-IDS (w/o $L_R$)              & 72.86\%          & -5.44\%         & 67.82\%          \\ \hline
CND-IDS (w/o $L_R$ and $L_{CL}$) & \textbf{79.92\%} & -11.26\%        & \underline{71.01\%}    \\ \hline
\end{tabular}
\end{table}
%%%%%%%%%%%%%%%%%%%%%%%%%%%%%%%%%%%%%%%%%%%%%%%%%%%%%%%%%%%%%%%%%%%%%%%

\subsection{Overhead Analysis}
%%%%%%%%%%%%%%%%%%%%%%%%%%%%%%%%%%%%%%%%%%%%%%%%%%%%%%%%%%%
% \begin{table}[]
% \centering
% \caption{Average training time and inference time per sample across all datasets in milliseconds}
% \label{tab:overhead}
% \begin{tabular}{|c|c|c|}
% \hline
% Strategy               & Inference Time(ms) \\ \hline
% \Design{}                   & 0.0019             \\ \hline
% ADCN\cite{ashfahani2023unsupervised}    & 0.4061             \\ \hline
% LwF\cite{lwf2019Li}           & 0.0677             \\ \hline
% DIF\cite{xu2023deep}         & 1.0535             \\ \hline
% PCA\cite{rios2022incdfm}       & 0.0018             \\ \hline
% \end{tabular}
% \end{table}
%%%%%%%%%%%%%%%%%%%%%%%%%%%%%%%%%%%%%%%%%%%%%%%%%%%%%%%%%%%%%
\begin{table}[]
\centering

\caption{Average inference time (in ms) per test sample}
\label{tab:overhead}
\scalebox{0.95}{
\begin{tabular}{|c|c|c|c|c|c|}
\hline
Strategy           & \Design{} & ADCN   & LwF    & DIF    & PCA    \\ \hline
Inference Time (ms) & \underline{0.0019}                     & 0.4061 & 0.0677 & 1.0535 & \textbf{0.0018} \\ \hline
\end{tabular}}
\end{table}
%%%%%%%%%%%%%%%%%%%%%%%%%%%%%%%%%%%%%%%%%%%%%%%%%%%%%%%%
Table~\ref{tab:overhead} evaluates the inference overhead of \Design{} compared to ADCN \cite{ashfahani2023unsupervised}, LwF \cite{lwf2019Li}, DIF \cite{xu2023deep}, and PCA \cite{rios2022incdfm}. %, excluding OC-SVM \cite{Faber_2024} and LOF \cite{Faber_2024} due to poor performance. 
\Design{} offers the fastest inference time among continual learning methods. Out of novelty detection methods, \Design{} is second only to PCA. We attribute the efficiency of \Design{} to avoiding the clustering classification used by LwF and ADCN. %\Design{} instead uses PCA reconstruction, which is much quicker than comparing data points to clusters. In addition, 
The difference between \Design{} and PCA is minimal, only 0.0001 milliseconds slower, due to the additional but lightweight step of encoding the data. Considering that the average median flow duration across datasets is 27.77 milliseconds, the overhead introduced by \Design{} is negligible in the context of real-time traffic flow.

%In this section we analyze the inference overhead of \Design{} compared to ADCN\cite{ashfahani2023unsupervised}, LwF\cite{lwf2019Li}, DIF\cite{xu2023deep}, and PCA\cite{rios2022incdfm}. We do not include OC-SVM\cite{Faber_2024} and LOF \cite{Faber_2024} due to weak performance. Table~\ref{tab:overhead} shows the average inference time in milliseconds per sample across all datasets. \Design{} has the best inference time besides PCA. We attribute this good inference time to \Design{} not using clustering classification like LwF and ADCN. Evidently, PCA reconstruction utilized by \Design{} is more time efficient than having to compare a data point to all saved clusters. Compared to pure PCA reconstruction, \Design{} is only 0.0001 ms slower. This small increase in inference time is due to the only added computation at inference is encoding the data with the encoder, which is simply a 4 layer MLP. Across all datasets, the average median travel flow duration is 27.77 ms, and the dataset with the quickest median travel flow is UNSW with 4.29 ms. Therefore the overhead introduced by \Design{} is irrelevant compared to the speed of the traffic flow. 

%\label{section:ablation_study}
%To assess the impact of our design choices, we perform an ablation study. Our goal is to analyze (i) threshold function evaluation, and (ii) novelty detection algorithm selection. 

 

%\textbf{Threshold Function Evaluation:} AE, PCA, and \Design{} all require a threshold to classify an anomaly based on the anomaly score. In all previously reported results, we select a widely used threshold that maximizes the $F_{1}$ score on the test set, i.e., Best-F. %This is not realistic but was used to compare the effectiveness of these methods. In this section 
%Here, we analyze three different threshold methods, which we denote: Best-F \cite{su2019robust}, Top-k \cite{zong2018deep}, and validation percentile (ValPer). Best-F uses the threshold that maximizes the $F_{1}$ score on test set. Top-k utilizes the contamination ratio $r$ of the test set, such that $r$ is the percentage of anomalies within the test set. Top-k selects a threshold so that the percentile of data within the test set classified as anomalies is equal to $r$. ValPer utilizes a validation set of normal data, and selects a threshold such that 99.7\% (3 standard deviations) of the normal data is within this threshold. 
%ValPer is the most realistic method as it does not rely on any information from the test set. 
%A breakdown of the $F_{1}$ score results for the different threshold methods is show in Table~\ref{tab:thresholding_results} where the best within each category is bolded. Overall Best-F performs significantly better than the other threshold methods, which is obvious as Best-F is an upper-bound for threshold selection. However the significant gap highlights the importance of threshold selection. Most importantly, \Design{} still performs better than PCA and AE through all threshold methods. 

%%%%%%%%%%%%%%%%%%%%%%%%%%%%%%%%%%%%%%%%%%%%%%%%%%%%%%%
%\begin{table}[]
%    \centering
%    \caption{Threshold Function Evaluation}
%    \resizebox{.97\columnwidth}{!}{
%    \begin{tabular}{c|c|c|c|c}
%        \hline
%         Dataset & Stategy & Best-F & Top-k & ValPer\\
%         \hline
%         & PCA  & 70.9 & 4.03 & 3.56 \\
%         \cline{2-5}
%         X-IIoTID & AE  & 75.6 & 4.03 & 29.4 \\
%         \cline{2-5}
%         & \Design{} & \textbf{78.8} & \textbf{5.63} &  %\textbf{52.9} \\	
%         \hline
%        & PCA  & 85.6 &19.9 & 52.8\\
%         \cline{2-5}
%         WUSTL-IIoT & AE  & 79.6 &19.7 & 37.8\\
%         \cline{2-5}
%         & \Design{} & \textbf{88.2} & \textbf{21.1} & \textbf{55.6}\\	
%         \hline
%    \end{tabular}}
%    \label{tab:thresholding_results}
%\end{table}
%%%%%%%%%%%%%%%%%%%%%%%%%%%%%%%%%%%%%%%%%%%%%%%%%%%%%%%

% %%%%%%%%%%%%%%%%%%%%%%%%%%%%%%%%%%%%%%%%%%%%%%%%%%%%%%%
% \begin{figure}
%     \centering
%     \includegraphics[width=0.95\linewidth]{figures/novelty_ablation.pdf}
%     \caption{Comparison of \Design{} with PCA and AE novelty detection models}
%     \label{fig:novelty_ablation_results}
% \end{figure}
% %%%%%%%%%%%%%%%%%%%%%%%%%%%%%%%%%%%%%%%%%%%%%%%%%%%%%%%

% \textbf{Novelty Detection Algorithm Selection:} For \Design{}, we select PCA as the novelty detection algorithm. As shown in Figure~\ref{fig:novelty_methods_results}, both PCA and AE perform well for detecting intrusions. Therefore, we test both AE and PCA as the novelty detection methods for \Design{}. Figure~\ref{fig:novelty_ablation_results} illustrates the AVG performance of \Design{} with AE and PCA as the novelty detection models. It is evident that PCA outperforms AE, justifying our selection of this algorithm for novelty detection. This could be because the CFE utilizes SAEs, which generate features based on the same reconstruction loss used by AE to classify anomalies. It may be beneficial to use PCA as it deconstructs the input in a different manner, thereby identifying different features and functioning better in conjunction with the SAE-based CFE.
















\end{document}