\subsection{Preliminary results on asymptotic deterministic equivalence}
\label{app:pre_asy_deter_equiv}

For the ease of description, we include preliminary results on asymptotic deterministic equivalence here. In fact, these assumptions and results can be recovered from non-asymptotic results, e.g., \cite{misiakiewicz2024non}.

For linear regression, the asymptotic deterministic equivalence aim to find $\mathcal{B}_{\mathcal{R},\lambda}^{\tt LS} \sim \sB_{\sR, \lambda}^{\tt LS}$, $\mathcal{V}_{\mathcal{R},\lambda}^{\tt LS} \sim \sV_{\sR, \lambda}^{\tt LS}$, where $\sB_{\sR, \lambda}^{\tt LS}$ and $\sV_{\sR, \lambda}^{\tt LS}$ are some deterministic quantities.
For asymptotic results, a series of assumptions in high-dimensional statistics via random matrix theory are required, on well-behaved data, spectral properties of $\bSigma$ under nonlinear transformation in high-dimensional regime.
We put the assumption from \citet{bach2024high} here that are also widely used in previous literature \cite{dobriban2018high, richards2021asymptotics}. 

\begin{assumption}\citep{bach2024high}\label{ass:asym}
    We assume that:
    \begin{itemize}
    \item[\textbf{(A4)}] The sample size $n$ and dimension $d$ grow to infinity with $\frac{d}{n} \to \gamma > 0$.
    \item[\textbf{(A5)}] $\bX = \bT \bSigma^{1/2}$, where $\bT \in \mathbb{R}^{n \times d}$ has i.i.d.\ sub-Gaussian entries with zero mean and unit variance.
    \item[\textbf{(A6)}] $\bSigma$ is invertible with $\| \bSigma \|_{\text{op}}< \infty$ and its spectral measure $ \frac{1}{d} \sum_{i=1}^d \delta_{\sigma_i} $ converges to a compactly supported probability distribution $\mu$ on $\mathbb{R}^+$.
    \item[\textbf{(A7)}] $\|\bbeta_\ast\|_2 < \infty$ and the measure $ \sum_{i=1}^d (\bv_i^\sT \bbeta_\ast)^2 \delta_{\sigma_i} $ converges to a measure $\nu$ with bounded mass, where $\bv_i$ is the unit-norm eigenvector of $\bSigma$ related to its respective eigenvalue $\sigma_i$.
    \end{itemize}
\end{assumption}

\begin{definition}[Effective regularization]
    For $n$, $\bSigma$, and $\lambda \geq 0$, we define the \emph{effective regularization} $\lambda_*$ to be the unique non-negative solution to the self-consistent equation
\begin{equation}\label{eq:def_lambda_star_asy}
    n - \frac{\lambda}{\lambda_*} \sim \Tr ( \bSigma ( \bSigma + \lambda_* )^{-1} ).
\end{equation}
\end{definition}

\begin{definition}[Degrees of freedom]\label{def:df}
\[
{\rm df}_1(\lambda_*) := \Tr ( \bSigma ( \bSigma + \lambda_*)^{-1}), \quad {\rm df}_2(\lambda_*) := \Tr ( \bSigma^2 ( \bSigma + \lambda_*)^{-2}).
\]
\end{definition}

\begin{proposition}\citep[Restatement of Proposition 1]{bach2024high}\label{prop:spectral}
    Assume \textbf{(A4)}, \textbf{(A5)}, \textbf{(A6)}, we consider $\bA$ and $\bB$ with bounded operator norm, admitting the convergence of the empirical measures, i.e., $ \sum_{i=1}^d   \bv_i^\sT \bA \bv_i  \cdot\delta_{\sigma_i} \rightarrow \nu_A$
    and $ \sum_{i=1}^d   \bv_i^\sT \bB \bv_i  \cdot\delta_{\sigma_i} \rightarrow \nu_B$ with bounded total variation, respectively. Then, for $\lambda \geq 0$, with $\lambda_*$ satisfying Eq.~\eqref{eq:def_lambda_star_asy},
    we have the following {\bf asymptotic deterministic equivalence}
    \begin{align}
        \label{eq:trA1}
        \Tr ( \bA \bX^\sT \bX ( \bX^\sT \bX +\lambda )^{-1} ) \sim&~ \Tr ( \bA \bSigma ( \bSigma + \lambda_* )^{-1} )\,,
        \\
        \label{eq:trAB1}
        \Tr ( \bA \bX^\sT \bX ( \bX^\sT \bX + \lambda )^{-1} \bB \bX^\sT \bX ( \bX^\sT \bX + \lambda )^{-1}) \sim&~ \Tr ( \bA \bSigma ( \bSigma + \lambda_* )^{-1} \bB \bSigma ( \bSigma + \lambda_* )^{-1} ) \nonumber \\
        + \lambda_*^2 \Tr ( \bA ( \bSigma + \lambda_* )^{-2}  \bSigma ) &\cdot \Tr ( \bB ( \bSigma + \lambda_* )^{-2} \bSigma ) \cdot \frac{1}{ n -  {\rm df}_2(\lambda_*) }\,,\\
        \label{eq:trA2}
        \Tr ( \bA ( \bX^\sT \bX +\lambda )^{-1} ) \sim&~ \frac{\lambda_*}{\lambda} \Tr ( \bA ( \bSigma + \lambda_* )^{-1} )\,,
        \\
        \label{eq:trAB2}
        \Tr ( \bA ( \bX^\sT \bX + \lambda )^{-1} \bB ( \bX^\sT \bX + \lambda )^{-1}) \sim&~ \frac{\lambda_*^2}{\lambda^2} \Tr ( \bA ( \bSigma + \lambda_* )^{-1} \bB ( \bSigma + \lambda_* )^{-1} ) \nonumber \\
        + \frac{\lambda_*^2}{\lambda^2} \Tr ( \bA ( \bSigma + \lambda_* )^{-2}  \bSigma ) &\cdot \Tr ( \bB ( \bSigma + \lambda_* )^{-2} \bSigma ) \cdot \frac{1}{ n -  {\rm df}_2(\lambda_*) }\,.
    \end{align}
\end{proposition}

\begin{proposition}\citep[Restatement of Proposition 2]{bach2024high}
\label{prop:spectralK}
Assume \textbf{(A4)}, \textbf{(A5)}, \textbf{(A6)}, we consider $\bA$ and $\bB$ with bounded operator norm, admitting the convergence of the empirical measures, i.e., $ \sum_{i=1}^d   \bv_i^\sT \bA \bv_i  \cdot\delta_{\sigma_i} \rightarrow \nu_A$ and $ \sum_{i=1}^d   \bv_i^\sT \bB \bv_i  \cdot\delta_{\sigma_i} \rightarrow \nu_B$ with bounded total variation, respectively. Then, for $\lambda \in \mathbb{C} \backslash \mathbb{R}_+$, with $\lambda_*$ satisfying Eq.~\eqref{eq:def_lambda_star_asy}, we have the following {\bf asymptotic deterministic equivalence}
\begin{align}
\label{eq:trA1K}
\Tr ( \bA \bT^\sT ( \bT \bSigma \bT^\sT + \lambda )^{-1} \bT) \sim&~ \Tr ( \bA ( \bSigma + \lambda_* )^{-1} ),
\\
\label{eq:trAB1K}
\Tr ( \bA \bT^\sT ( \bT \bSigma \bT^\sT + \lambda )^{-1} \bT \bB \bT^\sT ( \bT \bSigma \bT^\sT + \lambda )^{-1} \bT) \nonumber \sim&~ \Tr ( \bA ( \bSigma + \lambda_* )^{-1} \bB ( \bSigma + \lambda_* )^{-1} )\\
+ \lambda_*^2 \Tr ( \bA ( \bSigma + \lambda_* )^{-2} )&~ \cdot \Tr ( \bB ( \bSigma + \lambda_* )^{-2} ) \cdot \frac{1}{ n -  {\rm df}_2(\lambda_*) }\,.
\end{align}
\end{proposition}

Note that the results in \cref{prop:spectral}, \ref{prop:spectralK} still hold even for the random features model.
We will explain this in details in \cref{app:proof_rf}.