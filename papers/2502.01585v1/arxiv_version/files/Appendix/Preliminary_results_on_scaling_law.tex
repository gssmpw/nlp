\subsection{Preliminary results on scaling law}\label{app:pre_scaling_law}

For the derivation of the scaling law, we use the results in \citet[Appendix D]{defilippis2024dimension}. We define $T^s_{\delta,\gamma}(\nu)$ as
\begin{equation*}
    T^s_{\delta,\gamma}(\nu) := \sum_{k = 1}^\infty \frac{k^{-s-\delta\alpha}}{(k^{-\alpha}+\nu)^{\gamma}}\,, \quad s \in {0,1},\;0\leq\delta\leq\gamma.
\end{equation*}
Under \cref{ass:powerlaw_rf}, according to \citet[Appendix D]{defilippis2024dimension}, we have the following results
\begin{equation}\label{eq:rate_T}
T_{\delta\gamma}^{s}(\nu) = O\left(\nu^{\nicefrac{1}{\alpha}\left[s-1 + \alpha(\delta-\gamma)\right]\wedge0}\right).
\end{equation}

Next, we present some rates of the quantities used in the deterministic equivalence of random feature ridge regression. The rate of $\nu_2$ is given by
\begin{equation}\label{eq:rate_nu2}
    \nu_2 \approx O\left(n^{-\alpha\left(1 \wedge q \wedge \nicefrac{\ell}{\alpha}\right)}\right),
\end{equation}
and in particular, for \(\Upsilon(\nu_1, \nu_2)\) and \(\chi (\nu_2)\), we have
\begin{equation}\label{eq:rates:Upsilon2}
    1 - \Upsilon(\nu_1, \nu_2) = O(1)\,,
\end{equation}

\begin{equation}\label{eq:rates:chi}
    \chi (\nu_2) = n^{-q}O\left(\nu_2^{-1-\nicefrac{1}{\alpha}}\right)\,.
\end{equation}