\subsection{Non-asymptotic deterministic equivalence for ridge regression}
\label{app:nonasy_deter_equiv_lr}

The deterministic equivalents for the bias and variance terms of the test risk are given by \citet{misiakiewicz2024non} as 
\[
\begin{aligned}
\sB_{\sR,\lambda}^{\tt LS} :=&~ \frac{\lambda_*^2 \langle \bbeta_*, \bSigma(\bSigma + \lambda_*\id)^{-2} \bbeta_* \rangle}{1 - n^{-1} \Tr\left(\bSigma^2 (\bSigma + \lambda_*\id)^{-2}\right)}\,, \qquad \sV_{\sR,\lambda}^{\tt LS} :=&~ \frac{\sigma_{\varepsilon}^2 \Tr\left(\bSigma^2 (\bSigma + \lambda_*\id)^{-2}\right)}{n - \Tr\left(\bSigma^2 (\bSigma + \lambda_*\id)^{-2}\right)}\,.
\end{aligned}
\]
In this section, we establish the approximation guarantees for linear ridge regression.
Instead of using the power-law assumption \cref{ass:powerlaw} in the main text, we adopt the following weaker assumption.
\begin{assumption}[\citet{defilippis2024dimension}]\label{ass:technical_LR} There exists $C>1$
\[
    \frac{\< \bbeta_*, \bSigma(\bSigma+\lambda_*)^{-1}\bbeta_* \>}{\< \bbeta_*, \bSigma^2(\bSigma+\lambda_*)^{-2}\bbeta_* \>} \leq C\,.
\]
\end{assumption}
\noindent{\bf Remark:}
This assumption holds in many settings of interest, such as power law assumptions like those in \cref{ass:powerlaw}, since under this assumption the numerator and denominator are bounded sums of finite terms. It is a technical assumption used to address the difference between two deterministic equivalents that are needed in our work for norm-based capacity.
In fact, this assumption is used for RFMs in \cite{defilippis2024dimension} as the authors also face with the issue on the difference between two deterministic equivalents.


We generalize \cref{prop:non-asy_equiv_norm_LR} as below.
\begin{theorem}[Deterministic equivalents of the $\ell_2$-norm of the estimator. Full version of \cref{prop:non-asy_equiv_norm_LR}]\label{prop:det_equiv_LR}
    Assume well-behaved data $\{ \bm x_i \}_{i=1}^n$ satisfy \cref{ass:concentrated_LR} and \cref{ass:technical_LR}. Then for any $D,K > 0$, there exist constants $\eta_* \in (0, 1/2)$ and $C_{*,D,K} > 0$ ensuring the following property holds. For any $n \geq C_{*,D,K}$, $\lambda > 0$, if the following condition is satisfied:
    \begin{equation*}
        \lambda \geq n^{-K}\,, \quad \rho_{\lambda}(n)^{5/2} \log^{3/2}(n) \leq K \sqrt{n}\,,
    \end{equation*} 
    then with probability at least $1-n^{-D}$, we have that
    \[
    \begin{aligned}
         \left|\mathcal{B}_{\mathcal{N},\lambda}^{\tt LS} - \sB_{\sN,\lambda}^{\tt LS}\right| \leq&~ C_{x, D, K} \frac{\rho_{\lambda}(n)^6 \log^{3/2}(n)}{\sqrt{n}}\sB_{\sN,\lambda}^{\tt LS}\,,\\
        \left|\mathcal{V}_{\mathcal{N},\lambda}^{\tt LS} - \sV_{\sN,\lambda}^{\tt LS}\right| \leq&~ C_{x, D, K} \frac{\rho_{\lambda}(n)^6 \log^{3/2}(n)}{\sqrt{n}} \sV_{\sN,\lambda}^{\tt LS}\,.
    \end{aligned}
    \]
\end{theorem}

\begin{proof}[Proof of \cref{prop:det_equiv_LR}] 
{\bf Part 1: Deterministic equivalents for the bias term.}

Here we prove the deterministic equivalents of $\mathcal{B}_{\mathcal{N},\lambda}^{\tt LS}$ and $\mathcal{V}_{\mathcal{N},\lambda}^{\tt LS}$. First, we decompose $\mathcal{B}_{\mathcal{N},\lambda}^{\tt LS}$ into
\[
\begin{aligned}
    \mathcal{B}_{\mathcal{N},\lambda}^{\tt LS} &= \Tr\left(\bbeta_*\bbeta_*^\sT\bX^\sT \bX (\bX^\sT \bX + \lambda)^{-1}\right) - \lambda\Tr\left(\bbeta_*\bbeta_*^\sT\bX^\sT \bX (\bX^\sT \bX + \lambda)^{-2}\right),\\
    &= \Phi_2(\bX; \tilde{\bA}_1, \lambda) - n\lambda \Phi_4(\bX; \tilde{\bA}_2, \lambda)\,,
\end{aligned}
\]
where $\tilde{\bA}_1 := \bbeta_*\bbeta_*^\sT$, $\tilde{\bA}_2 := \bSigma^{-1/2}\bbeta_*\bbeta_*^\sT\bSigma^{-1/2}$. 
Therefore, using \cref{thm:main_det_equiv_summary}, with probability at least $1-n^{-D}$, we have
\[
\begin{aligned}
    \left|\Phi_2(\bX; \tilde{\bA}_1, \lambda) - \Psi_1(\lambda_*; \tilde{\bA}_1)\right| &\leq C_{x, D, K} \frac{\rho_{\lambda}(n)^{5/2} \log^{3/2}(n)}{\sqrt{n}} \Psi_1(\lambda_*; \tilde{\bA}_1)\,,\\
    \left|n\lambda\Phi_4(\bX; \tilde{\bA}_2, \lambda) - n\lambda\Psi_2(\lambda_*; \tilde{\bA}_2)\right| &\leq C_{x, D, K} \frac{\rho_{\lambda}(n)^6 \log^{3/2}(n)}{\sqrt{n}} n\lambda\Psi_2(\lambda_*; \tilde{\bA}_2)\,.
\end{aligned}
\]
Combining the above bounds, we deduce that
\[
\begin{aligned}
    \left|\mathcal{B}_{\mathcal{N},\lambda}^{\tt LS} - \left(\Psi_1(\lambda_*; \tilde{\bA}_1) - n\lambda\Psi_2(\lambda_*; \tilde{\bA}_2)\right)\right| \leq C_{x, D, K} \frac{\rho_{\lambda}(n)^6 \log^{3/2}(n)}{\sqrt{n}}\left(\Psi_1(\lambda_*; \tilde{\bA}_1)+n\lambda\Psi_2(\lambda_*; \tilde{\bA}_2)\right).
\end{aligned}
\]
Note that
\[
\begin{aligned}
    \Psi_1(\lambda_*; \tilde{\bA}_1) - n\lambda\Psi_2(\lambda_*; \tilde{\bA}_2) = \sB_{\sN,\lambda}^{\tt LS}\,.    
\end{aligned}
\]
For $n\lambda\Psi_2(\lambda_*; \tilde{\bA}_2)$, recall that $\Psi_2(\lambda_*; \bA) := \frac{1}{n} \frac{\Tr(\bA \bSigma^2 (\bSigma + \lambda_*\id)^{-2})}{n - \Tr(\bSigma^2 (\bSigma + \lambda_*\id)^{-2})}$, and according to \cref{def:effective_regularization} 
and \cref{ass:technical_LR}, we have
\[
\begin{aligned}
    n\lambda\Psi_2(\lambda_*; \tilde{\bA}_2) =&~ \lambda\frac{\Tr(\bbeta_*\bbeta_*^\sT\bSigma (\bSigma + \lambda_*\id)^{-2})}{n - \Tr(\bSigma^2 (\bSigma + \lambda_*\id)^{-2})}\\
    \leq&~ \lambda_*\Tr(\bbeta_*\bbeta_*^\sT\bSigma (\bSigma + \lambda_*\id)^{-2})\\
    =&~ \Tr(\bbeta_*\bbeta_*^\sT \bSigma (\bSigma + \lambda_*\id)^{-1}) - \Tr(\bbeta_*\bbeta_*^\sT\bSigma^2 (\bSigma + \lambda_*\id)^{-2})\\
    \leq&~ \left(1-\frac{1}{C}\right) \Tr(\bbeta_*\bbeta_*^\sT\bSigma(\bSigma+\lambda_*)^{-1})\,,
\end{aligned}
\]
and therefore
\[
\begin{aligned}
    \Psi_1(\lambda_*; \tilde{\bA}_1)+n\lambda\Psi_2(\lambda_*; \tilde{\bA}_2) \leq&~ \left(2-\frac{1}{C}\right) \Tr(\bbeta_*\bbeta_*^\sT\bSigma(\bSigma+\lambda_*)^{-1})\\
    \leq&~ \left(2C-1\right)\frac{1}{C}\Tr(\bbeta_*\bbeta_*^\sT\bSigma(\bSigma+\lambda_*)^{-1})\\
    \leq&~ \left(2C-1\right)\left(\Psi_1(\lambda_*; \tilde{\bA}_1)-n\lambda\Psi_2(\lambda_*; \tilde{\bA}_2)\right).
\end{aligned}
\]
Then we conclude that
\[
    \left|\mathcal{B}_{\mathcal{N},\lambda}^{\tt LS} - \sB_{\sN,\lambda}^{\tt LS}\right| \leq C_{x, D, K} \frac{\rho_{\lambda}(n)^6 \log^{3/2}(n)}{\sqrt{n}}\sB_{\sN,\lambda}^{\tt LS},
\]
with probability at least $1-n^{-D}$.

{\bf Part 2: Deterministic equivalents for the variance term.} Next, we prove the deterministic equivalent of $\mathcal{V}_{\mathcal{N},\lambda}^{\tt LS}$. First, note that $\mathcal{V}_{\mathcal{N},\lambda}^{\tt LS}$ can be written in terms of the functional $\Phi_4(\bX; \bA, \lambda)$ defined in \cref{eq:Phi_4}
\[
    \mathcal{V}_{\mathcal{N},\lambda}^{\tt LS} = n\sigma_{\varepsilon}^2\Phi_4(\bX; \bSigma^{-1}, \lambda)\,.
\]
Thus, under the assumptions, we can apply \cref{thm:main_det_equiv_summary} to obtain that with probability at least $1-n^{-D}$
\[
\left|n\sigma_{\varepsilon}^2\Phi_4(\bX; \bSigma^{-1}, \lambda) - n\sigma_{\varepsilon}^2\Psi_2(\lambda_*; \bSigma^{-1})\right| \leq C_{x, D, K} \frac{\rho_{\lambda}(n)^6 \log^{3/2}(n)}{\sqrt{n}} n\sigma_{\varepsilon}^2\Psi_2(\lambda_*; \bSigma^{-1})\,.
\]
Recall that $\Psi_2(\lambda_*; \bA) := \frac{1}{n} \frac{\Tr(\bA \bSigma^2 (\bSigma + \lambda_*\id)^{-2})}{n - \Tr(\bSigma^2 (\bSigma + \lambda_*\id)^{-2})}$, then we have
\[
\left|\mathcal{V}_{\mathcal{N},\lambda}^{\tt LS} - \sV_{\sN,\lambda}^{\tt LS}\right| \leq C_{x, D, K} \frac{\rho_{\lambda}(n)^6 \log^{3/2}(n)}{\sqrt{n}} \sV_{\sN,\lambda}^{\tt LS}\,,
\]
with probability at least $1-n^{-D}$.
\end{proof}