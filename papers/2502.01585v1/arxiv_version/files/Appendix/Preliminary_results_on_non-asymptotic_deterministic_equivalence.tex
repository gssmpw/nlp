\subsection{Preliminary results on non-asymptotic deterministic equivalence}\label{app:pre_non-asy_deter_equiv}

Regarding non-asymptotic results, we require a series of notations and assumptions. We give a brief introduction here for self-completeness. More details can be found in \citet{cheng2022dimension,misiakiewicz2024non,defilippis2024dimension}.

Given $\bx \in \R^d$ with $d \in \naturals$, the associated covariance matrix is given by $\bSigma = \E[\bx \bx^\sT]$. We denote the eigenvalue of $\bSigma$ in non-increasing order as $\sigma_1 \geq \sigma_2 \geq \sigma_3 \geq \cdots \geq \sigma_d$. 

We introduce the non-asymptotic version of \cref{eq:def_lambda_star_asy} as below.
\begin{definition}[Effective regularization]\label{def:effective_regularization}
    Given $n$, $\bSigma$, and $\lambda \geq 0$, the \emph{effective regularization} $\lambda_*$ is defined as the unique non-negative solution of the following self-consistent equation
    \begin{equation*}
        n - \frac{\lambda}{\lambda_*} = \Tr \big( \bSigma ( \bSigma + \lambda_* )^{-1} \big).
    \end{equation*}
\end{definition}
\noindent{\bf Remark:} 
Existence and uniqueness of $\lambda_*$ are guaranteed since the left-hand side of the equation is monotonically increasing in $\lambda_*$, while the right-hand side is monotonically decreasing. 

In the next, we introduce the following definitions on ``effective dimension'', a metric to describe the model capacity, widely used in statistical learning theory.

Define \(r_{\bSigma}(k) := \frac{\Tr(\bSigma_{\geq k})}{\| \bSigma_{\geq k} \|_{\rm op}} = \frac{\sum_{j=k}^d \sigma_j}{\sigma_k}\) as the intrinsic dimension, we require the following definition

\begin{equation}\label{eq:rho_lambda}
    \rho_{\lambda} (n) := 1 +  \frac{n \sigma_{\lfloor \eta_* \cdot n \rfloor}}{\lambda}\left\{ 1 + \frac{r_{\bSigma} (\lfloor \eta_* \cdot n \rfloor) \vee n}{n} \log \big(r_{\bSigma} (\lfloor \eta_* \cdot n \rfloor) \vee n \big) \right\},
\end{equation}
where $\eta_* \in (0,1/2)$ is a constant that will only depend on $C_*$ defined in \cref{ass:concentrated_LR}. And we used the convention that $\sigma_{\lfloor \eta_* \cdot n \rfloor} = 0$ if $\lfloor \eta_* \cdot n \rfloor > d$.


In this section we consider functionals that depend on $\bX$ and deterministic matrices. For a general PSD~matrix $\bA \in \R^{d\times d}$, define the functionals
\begin{align}
    \Phi_1(\bX; \bA, \lambda) :=&~ \Tr \left(\bA \bSigma^{1/2} (\bX^\sT \bX + \lambda)^{-1} \bSigma^{1/2}\right),\label{eq:Phi_1}\\
    \Phi_2(\bX; \bA, \lambda) :=&~ \Tr \left(\bA\bX^\sT \bX (\bX^\sT \bX + \lambda)^{-1}\right),\label{eq:Phi_2}\\
    \Phi_3(\bX; \bA, \lambda) :=&~ \Tr \left(\bA \bSigma^{1/2} (\bX^\sT \bX + \lambda)^{-1} \bSigma (\bX^\sT \bX + \lambda)^{-1} \bSigma^{1/2}\right),\label{eq:Phi_3}\\
    \Phi_4(\bX; \bA, \lambda) :=&~ \Tr \left(\bA \bSigma^{1/2} (\bX^\sT \bX + \lambda)^{-1} \frac{\bX^\sT \bX}{n} (\bX^\sT \bX + \lambda)^{-1} \bSigma^{1/2}\right).\label{eq:Phi_4}
\end{align}
These functionals can be approximated through quantities that scale proportionally to
\begin{align}
    \Psi_1(\lambda_*; \bA) :=&~ \Tr\left(\bA \bSigma (\bSigma + \lambda_*\id)^{-1}\right),\label{eq:Psi_1}\\
    \Psi_2(\lambda_*; \bA) :=&~ \frac{1}{n} \cdot \frac{\Tr\left(\bA \bSigma^2 (\bSigma + \lambda_*\id)^{-2}\right)}{n - \Tr\left(\bSigma^2 (\bSigma + \lambda_*\id)^{-2}\right)}.\label{eq:Psi_2}
\end{align}



The following theorem gathers the approximation guarantees for the different functionals stated above, and is obtained by modifying \citet[Theorem A.2]{defilippis2024dimension}. 
We generalize \cref{eq:det_equiv_phi2_main} for any PSD matrix $\bm A$, which will be required for our results on the deterministic equivalence of $\ell_2$ norm. The proof can be found in \cref{app:proof_non-asy_results}.

\begin{theorem}[Dimension-free deterministic equivalents, Theorem A.2 of \cite{defilippis2024dimension}]\label{thm:main_det_equiv_summary}
    Assume the features $\{\bx_i\}_{i \in [n]}$ satisfy \cref{ass:concentrated_LR} with a constant $C_* > 0$. Then for any $D, K > 0$, there exist constants $\eta_* \in (0, 1/2)$, $C_{D, K} > 0$ and $C_{*, D, K} > 0$ ensuring the following property holds. For any $n \geq C_{D, K}$ and $\lambda > 0$, if the following condition is satisfied:
    \begin{equation}\label{eq:conditions_det_equiv_main}
        \lambda \cdot \rho_{\lambda}(n) \geq \|\bm{\Sigma}\|_{\mathrm{op}} \cdot n^{-K}, \quad \rho_{\lambda}(n)^{\nicefrac{5}{2}} \log^{\nicefrac{3}{2}}(n) \leq K \sqrt{n},
    \end{equation}
    then for any PSD matrix $\bA$, with probability at least $1 - n^{-D}$, we have that
    \begin{align}
        |\Phi_1(\bX; \bA, \lambda) - \frac{\lambda_*}{\lambda} \Psi_1(\lambda_*; \bA)| &\leq C_{*, D, K} \frac{\rho_{\lambda}(n)^{\nicefrac{5}{2}} \log^{\nicefrac{3}{2}}(n)}{\sqrt{n}} \cdot \frac{\lambda_*}{\lambda} \Psi_1(\lambda_*; \bA),\label{eq:det_equiv_phi1_main}\\
        |\Phi_2(\bX; \id, \lambda) - \Psi_1(\lambda_*; \id)| &\leq C_{*, D, K} \frac{\rho_{\lambda}(n)^4 \log^{\nicefrac{3}{2}}(n)}{\sqrt{n}} \Psi_1(\lambda_*; \id),\label{eq:det_equiv_phi2_main}\\
        |\Phi_3(\bX; \bA, \lambda) - \left(\frac{n \lambda_*}{\lambda}\right)^2 \Psi_2(\lambda_*; \bA)| &\leq C_{*, D, K} \frac{\rho_{\lambda}(n)^6 \log^{\nicefrac{5}{2}}(n)}{\sqrt{n}} \cdot \left(\frac{n \lambda_*}{\lambda}\right)^2 \Psi_2(\lambda_*; \bA),\label{eq:det_equiv_phi3_main}\\
        |\Phi_4(\bX; \bA, \lambda) - \Psi_2(\lambda_*; \bA)| &\leq C_{*, D, K} \frac{\rho_{\lambda}(n)^6 \log^{\nicefrac{3}{2}}(n)}{\sqrt{n}} \Psi_2(\lambda_*; \bA).\label{eq:det_equiv_phi4_main}
    \end{align}
\end{theorem}


Next, we present some of the concepts to be used in deriving random feature ridge regression. Similar to how ridge regression depends on \(\lambda_*\), as defined in \cref{def:effective_regularization}, the deterministic equivalence of random feature ridge regression relies on \(\nu_1\) and \(\nu_2\), which are the solutions to the coupled equations
\begin{equation}\label{eq:fixed_points_appendix}
    n - \frac{\lambda}{\nu_1} = \Tr\left( \bLambda ( \bLambda + \nu_2)^{-1} \right)\,, \quad p - \frac{p\nu_1}{\nu_2} = \Tr \left( \bLambda ( \bLambda + \nu_2 )^{-1} \right).
\end{equation}
Writing $\nu_1$ as a function of $\nu_2$ produces the equations as below
\begin{equation}\label{eq:def:nu}
    1 + \frac{n}{p} - \sqrt{\left(1 - \frac{n}{p}\right)^2 + 4\frac{\lambda}{p\nu_2}}  = \frac{2}{p} \Tr \left( \bLambda ( \bLambda + \nu_2 )^{-1} \right)\,, \quad \nu_1 := \frac{\nu_2}{2} \left[ 1 - \frac{n}{p} + \sqrt{\left(1 - \frac{n}{p}\right)^2 + 4\frac{\lambda}{p\nu_2}} \right].
\end{equation} 



For random features, our results also depend on the capacity of $\bLambda$. Recall the definition of \(r_\bLambda(k) := \frac{\Tr(\bLambda{\geq k})}{\| \bLambda{\geq k} \|_{\rm op}}\) as the intrinsic dimension of \(\bLambda\) at level \(k\), we sequentially define the following quantities that can be found in \citet{misiakiewicz2024non,defilippis2024dimension}.

\begin{align}
    M_\bLambda (k) =&~ 1 + \frac{r_{\bLambda} (\lfloor \eta_* \cdot k \rfloor) \vee k}{k} \log \left( r_{\bLambda} (\lfloor \eta_* \cdot k \rfloor) \vee k \right)\,,\\
    \rho_\kappa (p) =&~ 1 + \frac{p \cdot \xi^2_{\lfloor \eta_* \cdot p \rfloor}}{\kappa}  M_\bLambda (p)\,, \label{eq:def_rho_p}
    \\
    \trho_\kappa (n,p) =&~ 1 + \ind \{ n \leq p/\eta_*\} \cdot \left\{ \frac{n \xi_{\lfloor \eta_* \cdot n \rfloor}^2}{\kappa} + \frac{n}{p} \cdot \rho_\kappa (p)\right\} M_\bLambda (n)\,, \label{eq:def_trho_n_p}
\end{align}
where the constant \(\eta_* \in (0,1/2)\) only depends on \(C_*\) introduced in Assumption \ref{ass:concentrated_RFRR}. 


For an integer $\evn \in \naturals$, we split the covariance matrix $\bLambda$ into low degree part and high degree part as
\[
\bLambda_0 := \diag (\xi_1^2, \xi_2^2 , \ldots , \xi_{\evn}^2)\,, \quad \bLambda_+ := \diag (\xi_{\evn+1}^2, \xi_{\evn+2}^2 , \ldots )\,.
\]

After we define the high degree feature covariance \(\bLambda_+\), we can define the function \(\gamma (\kappa) := \kappa + \Tr(\bLambda_{+})\). To simplify the statement, we assume that we can choose $\evn$ such that $p^2 \xi_{\evn +1}^2 \leq \gamma (p\lambda/n)$, which is always satisfied under \cref{ass:concentrated_RFRR}. For convenience, we will further denote
\begin{equation}\label{eq:def_gamma_lamb_plus}
\gamma_+ := \gamma (p\nu_1) , \quad \quad \gamma_\lambda := \gamma (p\lambda / n).
\end{equation}

For random feature ridge regression, we will first demonstrate that the \(\ell_2\) norm concentrates around a quantity that depends only on \(\hbLambda_\bF\). To this end, we define the following functionals with respect to \(\bZ\).
\begin{equation}\label{eq:functionals_Z}
\begin{aligned}
\Phi_3(\bZ; \bA, \kappa) &:= \Tr \left( \bA \hbLambda_\bF^{1/2} (\bZ^\sT \bZ + \kappa)^{-1} \hbLambda_\bF (\bZ^\sT \bZ + \kappa)^{-1} \hbLambda_\bF^{1/2} \right),\\
\Phi_4(\bZ; \bA, \kappa) &:= \Tr \left( \bA \hbLambda_\bF^{1/2} (\bZ^\sT \bZ + \kappa)^{-1} \frac{\bZ^\sT \bZ}{n} (\bZ^\sT \bZ + \kappa)^{-1} \hbLambda_\bF^{1/2} \right).
\end{aligned}
\end{equation}
Given that \(\bZ\) consists of i.i.d. rows with covariance \(\hbLambda_\bF = \bF \bF^\sT / p\), we will demonstrate that the aforementioned functionals can be approximated by those of \(\bF\), which, in turn, can be represented using the following functionals:

\begin{equation}\label{eq:det_equiv_Z_F}
\begin{aligned}
\widetilde{\Phi}_5(\bF; \bA, \kappa) &:= \frac{1}{n} \cdot \frac{\widetilde{\Phi}_6(\bF; \bA, \kappa)}{n - \widetilde{\Phi}_6(\bF; \bm{I}, \kappa)},\\
\widetilde{\Phi}_6(\bF; \bA, \kappa) &:= \Tr \left( \bA (\bF \bF^\sT)^2 (\bF \bF^\sT + \kappa)^{-2} \right).\\
\end{aligned}
\end{equation}

\begin{proposition}[Deterministic equivalents for $\Phi(\bZ)$ conditional on $\bF$, Proposition B.6 of \cite{defilippis2024dimension}]\label{prop:det_Z} Assume \(\{\bz_i\}_{i \in [n]}\) and \(\{\boldf\}_{i \in [p]}\) satisfy \cref{ass:concentrated_RFRR} with a constant \(C_* > 0\), and $\bF \in \mathcal{A}_{\bF}$ defined in \citet[Eq. (79)]{defilippis2024dimension}. Then for any $D, K > 0$, there exist constants $\eta_* \in (0, 1/2)$, $C_{D, K} > 0$ and $C_{*, D, K} > 0$ ensuring the following property holds. Let $\rho_{\kappa}(p)$ and $\tilde{\rho}_{\kappa}(n, p)$ be defined as per \cref{eq:def_rho_p} and \cref{eq:def_trho_n_p}, $\gamma_+$ be defined as \cref{eq:def_gamma_lamb_plus}. For any $n \geq C_{D, K}$ and $\lambda > 0$, if the following
condition is satisfied:

\begin{equation*}
\begin{aligned}
    \lambda  \geq n^{-K}\,, \quad \quad \trho_{\lambda} (n,p)^{5/2} \log^{3/2} (n) \leq K \sqrt{n}\,, \quad \quad \trho_\lambda (n,p)^2 \cdot \rho_{\gamma_+} (p)^{5/2} \log^3 (p) \leq K \sqrt{p}\,,
    \end{aligned}
\end{equation*}
then for any PSD matrix $\bA \in \mathbb{R}^{p \times p}$ (independent of \(\bZ\) conditional on \(\bF\)), we have with probability at least $1 - n^{-D}$ that

\begin{align}
\left| \Phi_3(\bZ; \bA, \lambda) - \left( \frac{n \nu_1}{\lambda} \right)^2 \widetilde{\Phi}_5(\bF; \bA, p \nu_1) \right| &\leq C_{*, D, K} \cdot \mathcal{E}_1(n, p) \cdot \left( \frac{n \nu_1}{\lambda} \right)^2 \widetilde{\Phi}_5(\bF; \bA, p \nu_1), \\
\left| \Phi_4(\bZ; \bA, \lambda) - \widetilde{\Phi}_5(\bF; \bA, p \nu_1) \right| &\leq C_{*, D, K} \cdot \mathcal{E}_1(n, p) \cdot \widetilde{\Phi}_5(\bF; \bA, p \nu_1),
\end{align}
where the rate \( \mathcal{E}_1(n, p)\) is given by \( \cE_1 (n,p) := \frac{\trho_\lambda (n,p)^6 \log^{5/2} (n)}{\sqrt{n}} + \frac{\trho_\lambda (n,p)^2 \cdot \rho_{\gamma_+} (p)^{5/2}  \log^3 (p)}{\sqrt{p}}\).

\end{proposition}
