\section{Scaling laws}
\label{app:scaling_law}
To derive the scaling laws based on norm-based capacity, we first give the decay rate of the $\ell_2$ norm w.r.t.\ \(n\).

The rate of the deterministic equivalent of the random feature ridge regression estimator's $\ell_2$ norm  is given by
\[
\sN_\lambda^{\tt RFM} = \Theta\left(n^{-\gamma_{\sB_{\sN,\lambda}^{\tt RFM}}}+\sigma^2n^{-\gamma_{\sV_{\sN,\lambda}^{\tt RFM}}}\right) = \Theta\left(n^{-\gamma_{\sN_\lambda^{\tt RFM}}}\right)\,,
\]
where $\gamma_{\sN_\lambda^{\tt RFM}} := \gamma_{\sB_{\sN,\lambda}^{\tt RFM}} \wedge \gamma_{\sV_{\sN,\lambda}^{\tt RFM}}$ for $\sigma^2 \neq 0$.
    
\subsection{Variance term}
Using \cref{eq:rate_nu2,eq:rates:Upsilon2,eq:rates:chi}, we have
\[
\begin{aligned}
\sV_{\sN,\lambda}^{\tt RFM} =&~ \sigma^2 \frac{p}{n} \frac{\chi(\nu_2)}{1 - \Upsilon(\nu_1, \nu_2)} =~ n^{q-1}n^{-q}O\left(\nu_2^{-1-\nicefrac{1}{\alpha}}\right)\\
=&~O\left(n^{-\left(1- \left(\alpha+1\right)\left(1 \wedge q \wedge \nicefrac{\ell}{\alpha}\right)\right)}\right).
\end{aligned}
\]
Hence, the variance term of the norm decays with $n$ with rate
\[
\gamma_{\sV_{\sN,\lambda}^{\tt RFM}}(\ell, q) = 1 - \left(\alpha+1\right)\left(\frac{\ell}{\alpha}\wedge q\wedge 1\right). 
\]

\subsection{Bias term}
First, one could notice, using the integral approximation and \cref{eq:rate_T,eq:rate_nu2}, that
\[
\begin{aligned}
\frac{p}{p - {\rm df}_2(\nu_2)} =&~ \left(1 + n^{-q} O\left(\nu_2^{-\nicefrac{1}{\alpha}}\right) \right) = \left(1 + O\left(n^{-q}n^{\left(1 \wedge q \wedge \nicefrac{\ell}{\alpha}\right)}\right) \right) = O\left(1\right).
\end{aligned}
\]
Thus for the bias term, using \cref{eq:rate_T,eq:rate_nu2,eq:rates:Upsilon2,eq:rates:chi} we have
\[
\begin{aligned}
\sB_{\sN,\lambda}^{\tt RFM} =&~ \< \btheta_*, \bLambda ( \bLambda + \nu_2)^{-2} \btheta_* \> \cdot \frac{p}{p - {\rm df}_2(\nu_2)}\\
&~+ \frac{p}{n} \nu_2^2 \left( \< \btheta_*, (\bLambda + \nu_2)^{-2} \btheta_* \> + \chi(\nu_2) \< \btheta_*, \bLambda (\bLambda + \nu_2)^{-2} \btheta_* \> \right) \cdot \frac{\chi(\nu_2)}{1 - \Upsilon(\nu_1, \nu_2)}\\
=&~ T_{2r+1, 2}^1(\nu_2) + n^{q-1}\nu_2^2\left( T_{2r,2}^1(\nu_2) + \chi(\nu_2) T_{2r+1,2}^1(\nu_2)\right)\chi(\nu_2)\\
=&~ \nu_2^{(2r-1)\wedge 0} + n^{q-1} \nu_2^2 O\left(\nu_2^{(2r-2)\wedge 0} +  n^{-q}\nu_2^{-1-\nicefrac{1}{\alpha}+(2r-1)\wedge 0}\right) n^{-q}O\left(\nu_2^{-1-\nicefrac{1}{\alpha}}\right)\\
=&~ \nu_2^{(2r-1)\wedge 0} + n^{-1} O\left(\nu_2^{2r\wedge 2} +  n^{-q}\nu_2^{-\nicefrac{1}{\alpha}+2r\wedge 1}\right) O\left(\nu_2^{-1-\nicefrac{1}{\alpha}}\right)\\
=&~ O\left( n^{-\alpha \left(1 \wedge q \wedge \nicefrac{\ell}{\alpha}\right) \left[(2r-1)\wedge 0\right]} \right)\\
&~+ O\left(n^{-\alpha \left(1 \wedge q \wedge \nicefrac{\ell}{\alpha}\right) \left[(2r-1)\wedge 1\right] + \left(1 \wedge q \wedge \nicefrac{\ell}{\alpha}\right) - 1}
+ 
n^{-\alpha \left(1 \wedge q \wedge \nicefrac{\ell}{\alpha}\right) \left[(2r-1)\wedge 0\right] + 2\left(1 \wedge q \wedge \nicefrac{\ell}{\alpha}\right) - 1 - q}\right)\\
=&~ O\left( n^{-\alpha \left(1 \wedge q \wedge \nicefrac{\ell}{\alpha}\right) \left[(2r-1)\wedge 0\right]}
+ 
n^{-\alpha \left(1 \wedge q \wedge \nicefrac{\ell}{\alpha}\right) \left[(2r-1)\wedge 1\right] + \left(1 \wedge q \wedge \nicefrac{\ell}{\alpha}\right) - 1}
\right)\\
=&~ O\left( n^{-\alpha \left(1 \wedge q \wedge \nicefrac{\ell}{\alpha}\right) \left[(2r-1)\wedge 0\right]}\right).
\end{aligned}
\]
Hence, the bias term of the norm decays with $n$ with rate
\[
\begin{aligned}
\gamma_{\sB_{\sN,\lambda}^{\tt RFM}}(\ell, q) =&~ \alpha \left(1 \wedge q \wedge \nicefrac{\ell}{\alpha}\right) \left[(2r-1)\wedge 0\right].
\end{aligned}
\]
Recalling that we have
\[
\gamma_{\sN_\lambda^{\tt RFM}} := \gamma_{\sB_{\sN,\lambda}^{\tt RFM}} \wedge \gamma_{\sV_{\sN,\lambda}^{\tt RFM}}\,,
\]
according to which, we obtain the norm exponent $\gamma_{\sN_\lambda^{\tt RFM}}$ as a function of $\ell$ and $q$, showing in \cref{fig:scaling_law}. As observed in \cref{fig:scaling_law}, $\gamma_{\sN_\lambda^{\tt RFM}}$ is non-positive across all regions, indicating that the norm either increases or remains constant with \(n\) in every case.

\begin{figure}[H]
    \centering
    \includegraphics[width=0.6\textwidth]{arxiv_version/figures/Scaling_Law/scaling_law_norm.pdf} 
    \caption{The norm rate $\gamma_{\sN_\lambda^{\tt RFM}}$ as a function of $(\ell,q)$. Variance dominated region is colored by {\color{regionorange}orange}, {\color{regionyellow}yellow} and {\color{regionbrown}brown}, bias dominated region is colored by {\color{regionblue}blue} and {\color{regiongreen}green}.} 
    \label{fig:scaling_law} 
\end{figure}

Next for the condition $r \in (0, \frac{1}{2})$, we derive the scaling law under norm-based capacity.

\paragraph{Region 1: $\ell > \alpha$ and $q > 1$}

In this region, according to \citet[Corollary 4.1]{defilippis2024dimension}, we have
\[
\sR_\lambda^{\tt RFM} = \Theta\left( n^{-0} \right) = \Theta\left( 1 \right)\,,
\]
and according to \cref{fig:scaling_law}, we have
\[
\sN_\lambda^{\tt RFM} = \Theta\left( n^{\alpha} \right)\,,
\]
combing the above rate, we can obtain that
\[
\sR_\lambda^{\tt RFM} = \Theta\left( n^{-\alpha} \cdot \sN_\lambda^{\tt RFM} \right)\,.
\]

\paragraph{Region 2: $\frac{\alpha}{2\alpha r+1} < \ell < \alpha$ and $q > \frac{\ell}{\alpha}$}

In this region, according to \citet[Corollary 4.1]{defilippis2024dimension}, we have
\[
\sR_\lambda^{\tt RFM} = \Theta\left( n^{-\left(1-\frac{\ell}{\alpha}\right)} \right)\,,
\]
and according to \cref{fig:scaling_law}, we have
\[
\sN_\lambda^{\tt RFM} = \Theta\left( n^{-\left(1-\frac{(\alpha+1)\ell}{\alpha}\right)} \right)\,,
\]
combing the above rate, we can obtain that
\[
\sR_\lambda^{\tt RFM} = \Theta\left( n^{-\ell} \cdot \sN_\lambda^{\tt RFM} \right)\,.
\]

\paragraph{Region 3: $\frac{1}{2\alpha r+1} < q < 1$ and $q < \frac{\ell}{\alpha}$}

In this region, according to \citet[Corollary 4.1]{defilippis2024dimension}, we have
\[
\sR_\lambda^{\tt RFM} = \Theta\left( n^{-\left(1-q\right)} \right)\,,
\]
and according to \cref{fig:scaling_law}, we have
\[
\sN_\lambda^{\tt RFM} = \Theta\left( n^{-\left(1-(\alpha+1)q\right)} \right)\,,
\]
combing the above rate and eliminate $q$, we can obtain that
\[
\sR_\lambda^{\tt RFM} = \Theta\left( n^{-\frac{\alpha}{\alpha+1}} \cdot \left(\sN_\lambda^{\tt RFM}\right)^{\frac{1}{\alpha+1}} \right)\,.
\]

\paragraph{Region 4: $\ell < \frac{\alpha}{2\alpha r+1}$ and $q > \frac{\ell}{\alpha}$}

In this region, according to \citet[Corollary 4.1]{defilippis2024dimension}, we have
\[
\sR_\lambda^{\tt RFM} = \Theta\left( n^{-2\ell r} \right)\,,
\]
and according to \cref{fig:scaling_law}, we have
\[
\sN_\lambda^{\tt RFM} = \Theta\left( n^{-\ell(2r-1)} \right)\,,
\]
combing the above rate, we can obtain that
\[
\sR_\lambda^{\tt RFM} = \Theta\left( n^{-1} \cdot \sN_\lambda^{\tt RFM} \right)\,.
\]

\paragraph{Region 5: $q < \frac{1}{2\alpha r+1}$ and $q < \frac{\ell}{\alpha}$}

In this region, according to \citet[Corollary 4.1]{defilippis2024dimension}, we have
\[
\sR_\lambda^{\tt RFM} = \Theta\left( n^{-2\alpha q r} \right)\,,
\]
and according to \cref{fig:scaling_law}, we have
\[
\sN_\lambda^{\tt RFM} = \Theta\left( n^{-\alpha q(2r-1)} \right)\,,
\]
combing the above rate, we can obtain that
\[
\sR_\lambda^{\tt RFM} = \Theta\left( n^0 \cdot \left(\sN_\lambda^{\tt RFM}\right)^{-\frac{2r}{1-2r}} \right)\,.
\]



