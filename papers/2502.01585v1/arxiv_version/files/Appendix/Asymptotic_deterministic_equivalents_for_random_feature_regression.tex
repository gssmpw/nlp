\subsection{Asymptotic deterministic equivalence for random features ridge regression}
\label{app:asy_deter_equiv_rf}

In this section, we establish the asymptotic approximation guarantees for random feature regression in terms of its $\ell_2$-norm based capacity. Before presenting the proof of \cref{prop:asy_equiv_norm_RFRR}, we firstly give the proof of the bias-variance decomposition in \cref{lemma:biasvariance_rf}.

\begin{proof}[Proof of \cref{lemma:biasvariance_rf}]
Here we give the bias-variance decomposition of $\E_{\varepsilon}\|\hat{\ba}\|_2^2$. The formulation of $\E_{\varepsilon}\|\hat{\ba}\|_2^2$ is given by
\[
\E_{\varepsilon}\|\hat{\ba}\|_2^2 = \E_{\varepsilon} \|(\bZ^\sT \bZ + \lambda \id)^{-1} \bZ^\sT \by\|_2^2\,,
\]
which admits a similar bias-variance decomposition
\[
\begin{aligned}
    \E_{\varepsilon}\|\hat{\ba}\|_2^2 =&~ \E_{\varepsilon}\|(\bZ^\sT \bZ + \lambda \id)^{-1} \bZ^\sT (\bG \btheta_*+\bm\varepsilon)\|_2^2\\
    =&~ \|(\bZ^\sT \bZ + \lambda \id)^{-1} \bZ^\sT \bG \btheta_*\|_2^2 + \E_{\varepsilon}\|(\bZ^\sT \bZ + \lambda \id)^{-1} \bZ^\sT \bm\varepsilon\|_2^2\\
    =&~ \<\btheta_*, \bG^\sT \bZ (\bZ^\sT \bZ + \lambda\id)^{-2} \bZ^\sT \bG\btheta_* \> + \sigma^2\Tr\left(\bZ^\sT \bZ(\bZ^\sT \bZ + \lambda\id)^{-2}\right)\\
    =:&~ \mathcal{B}_{\mathcal{N},\lambda}^{\tt RFM} + \mathcal{V}_{\mathcal{N},\lambda}^{\tt RFM}\,.
\end{aligned}
\]
Accordingly, we conclude the proof.
\end{proof}

Now we are ready to present the proof of \cref{prop:asy_equiv_norm_RFRR} as below.

\begin{proof}[Proof of \cref{prop:asy_equiv_norm_RFRR}]
We give the asymptotic deterministic equivalents for the norm from the bias $\mathcal{B}_{\mathcal{N},\lambda}^{\tt RFM}$ and variance $\mathcal{V}_{\mathcal{N},\lambda}^{\tt RFM}$, respectively. We provide asymptotic expansions in two steps, by first considering the deterministic equivalent over $\bG$, and then over $\bF$.

Under \cref{ass:concentrated_RFRR}, we can apply \cref{prop:spectral,prop:spectral2,prop:spectralK,prop:spectralK2} directly in the proof below.

\paragraph{Deterministic equivalent over $\bG$:}
For the bias term, we use \cref{eq:trAB1K} in \cref{prop:spectralK} with $\bT=\bG$, $\bSigma=\bF^\sT\bF$, $\bA=\btheta_*\btheta_*^\sT$ and $\bB=\bF^\sT\bF$ and obtain
\begin{equation}\label{eq:bnrfm}
   \begin{split}
          \mathcal{B}_{\mathcal{N},\lambda}^{\tt RFM} =&~ \<\btheta_*, \bG^\sT \bZ (\bZ^\sT \bZ + \lambda\id)^{-2} \bZ^\sT \bG\btheta_* \>\\
    =&~ \Tr(\btheta_*^\sT \bG^\sT \bZ (\bZ^\sT \bZ + \lambda\id)^{-2} \bZ^\sT \bG\btheta_* )\\
    =&~ p\Tr(\btheta_* \btheta_*^\sT \bG^\sT ( \bG \bF^\sT \bF \bG^\sT + p\lambda\id)^{-1} \bG \bF^\sT \bF \bG^\sT ( \bG \bF^\sT \bF \bG^\sT + p\lambda\id)^{-1} \bG )\\
    \sim&~ p \underbrace{\Tr(\btheta_* \btheta_*^\sT ( \bF^\sT \bF + \nu_1\id)^{-1} \bF^\sT \bF ( \bF^\sT \bF + \nu_1\id)^{-1} )}_{\tt I_1} \\
    &~ + p\nu_1^2 \underbrace{\Tr(\btheta_* \btheta_*^\sT ( \bF^\sT \bF + \nu_1\id)^{-2})}_{:=I_2} \cdot \underbrace{\Tr(\bF^\sT \bF ( \bF^\sT \bF + \nu_1\id)^{-2})}_{:=I_3} \cdot \frac{1}{n-\widehat{\rm df}_2(\nu_1)} \,,
   \end{split} 
\end{equation}
where $\nu_1$ defined by $\nu_1(1-\frac{1}{n}\widehat{\rm df}_1(\nu_1)) \sim \frac{p\lambda}{n}$, $\widehat{\rm df}_1(\nu_1)$ and $\widehat{\rm df}_2(\nu_1)$ are degrees of freedom associated to $\bF^\sT \bF$ in \cref{def:df}.

For the variance term, we use \cref{eq:trA3K} with $\bT=\bG$ in \cref{prop:spectralK}, $\bA=\bF^\sT\bF$, $\bSigma=\bF^\sT\bF$ and obtain
\[
\begin{aligned}
    \mathcal{V}_{\mathcal{N},\lambda}^{\tt RFM} =&~ \sigma^2\Tr\left(\bZ^\sT \bZ(\bZ^\sT \bZ + \lambda\id)^{-2}\right) = \sigma^2\Tr\left(\bZ \bZ^\sT(\bZ \bZ^\sT + \lambda\id)^{-2}\right)\\    =&~\sigma^2p\Tr\left(\bG\bF^\sT\bF\bG^\sT(\bG\bF^\sT\bF\bG^\sT + p\lambda\id)^{-2}\right)\\
    \sim&~\sigma^2p\frac{\Tr(\bF^\sT\bF(\bF^\sT\bF+\nu_1\id)^{-2})}{n-\widehat{\rm df}_2(\nu_1)}\,.
\end{aligned}
\]

\paragraph{Deterministic equivalent over $\bF$:}

In the next, we aim to eliminate the randomness over $\bF$ in \cref{eq:bnrfm} from the bias part.
First our result depends on the asymptotic equivalents for $\widehat{\rm df}_1(\nu_1)$ and $\widehat{\rm df}_2(\nu_1)$. For $\widehat{\rm df}_1(\nu_1)$, we use \cref{eq:trA1} in \cref{prop:spectral} with $\bX=\bF$ and obtain
\[
\begin{aligned}
    \widehat{\rm df}_1(\nu_1) = \Tr(\bF^\sT \bF (\bF^\sT \bF + \nu_1\id)^{-1}) \sim \Tr(\bLambda(\bLambda + \nu_2\id)^{-1})={\rm df}_1(\nu_2)\,,
\end{aligned}
\]
where $\nu_2$ defined by $\nu_2(1-\frac{1}{p}{\rm df}_1(\nu_2)) \sim \frac{\nu_1}{p}$. Hence $\nu_1$ can be defined by $\nu_1(1-\frac{1}{n}{\rm df}_1(\nu_2))\sim\frac{p\lambda}{n}$ from \cref{eq:def_nu}.

For $\widehat{\rm df}_2(\nu_1)$, we use \cref{eq:trAB1} in \cref{prop:spectral} with $\bX=\bF$, $\bA=\bB=\id$ and obtain
\begin{equation}\label{eq:df2v1}
    \begin{split}
    \widehat{\rm df}_2(\nu_1) &=~ \Tr(\bF^\sT \bF (\bF^\sT \bF + \nu_1\id)^{-1} \bF^\sT \bF (\bF^\sT \bF + \nu_1\id)^{-1})\\
    &\sim~ \Tr(\bLambda^2(\bLambda + \nu_2\id)^{-2}) + \nu_2^2 \Tr(\bLambda(\bLambda + \nu_2\id)^{-2}) \cdot \Tr(\bLambda^2(\bLambda + \nu_2\id)^{-2}) \cdot \frac{1}{p - {\rm df}_2(\nu_2)}\\
    &=:~ n\Upsilon(\nu_1, \nu_2)\,. 
    \end{split}
\end{equation}

For $I_3:= \Tr(\bF^\sT \bF ( \bF^\sT \bF + \nu_1\id)^{-2})$, we use \cref{eq:trA3} with $\bX=\bF$ and obtain
\begin{align}\label{eq:I3}
\Tr(\bF^\sT \bF ( \bF^\sT \bF + \nu_1\id)^{-2}) \sim&~ \Tr(\bLambda(\bLambda + \nu_2\id)^{-2}) \cdot \frac{1}{p - {\rm df}_2(\nu_2)}\,.
\end{align}
Then we use \cref{eq:trA3} again with $\bX=\bF$, $\bA = \btheta_*\btheta_*^\sT$ to obtain the deterministic equivalent of $I_1$
\[
\begin{aligned}
\Tr(\btheta_* \btheta_*^\sT ( \bF^\sT \bF + \nu_1\id)^{-1} \bF^\sT \bF ( \bF^\sT \bF + \nu_1\id)^{-1}) =&~ \Tr(\btheta_* \btheta_*^\sT \bF^\sT \bF ( \bF^\sT \bF + \nu_1\id)^{-2})\\
\sim&~ \Tr(\btheta_* \btheta_*^\sT \bLambda ( \bLambda + \nu_2\id)^{-2}) \cdot \frac{1}{p - {\rm df}_2(\nu_2)}\\
=&~ \btheta_*^\sT \bLambda ( \bLambda + \nu_2\id)^{-2} \btheta_* \cdot \frac{1}{p - {\rm df}_2(\nu_2)}.
\end{aligned}
\]
Further, for $I_2$, use \cref{eq:trAB2} with $\bA=\btheta_*\btheta_*^\sT$ and $\bB=\id$, we obtain
\[
\begin{aligned}
\Tr(\btheta_* \btheta_*^\sT ( \bF^\sT \bF + \nu_1\id)^{-2}) \sim&~ \frac{\nu_2^2}{\nu_1^2}\Tr(\btheta_* \btheta_*^\sT (\bLambda + \nu_2\id)^{-2})\\
&~+ \frac{\nu_2^2}{\nu_1^2}\Tr(\btheta_* \btheta_*^\sT (\bLambda + \nu_2\id)^{-2} \bLambda) \cdot \Tr( (\bLambda + \nu_2\id)^{-2} \bLambda) \cdot \frac{1}{p - {\rm df}_2(\nu_2)}.
\end{aligned}
\]
Finally, combine the above equivalents, for the bias, we obtain
\[
\begin{aligned}
    \mathcal{B}_{\mathcal{N},\lambda}^{\tt RFM} \sim&~ p \btheta_*^\sT \bLambda ( \bLambda + \nu_2\id)^{-2} \btheta_* \cdot \frac{1}{p - {\rm df}_2(\nu_2)}\\
    &~+ p \nu_1^2 \left(\frac{\nu_2^2}{\nu_1^2}\Tr(\btheta_* \btheta_*^\sT (\bLambda + \nu_2\id)^{-2}) + \frac{\nu_2^2}{\nu_1^2}\Tr(\btheta_* \btheta_*^\sT (\bLambda + \nu_2\id)^{-2} \bLambda) \cdot \Tr( (\bLambda + \nu_2\id)^{-2} \bLambda) \cdot \frac{1}{p - {\rm df}_2(\nu_2)} \right)\\
    &~\cdot \Tr(\bLambda(\bLambda + \nu_2\id)^{-2}) \cdot \frac{1}{p - {\rm df}_2(\nu_2)} \cdot \frac{1}{n - n\Upsilon(\nu_1, \nu_2)}\\
    =&~ p\btheta_*^\sT \bLambda ( \bLambda + \nu_2\id)^{-2} \btheta_* \cdot \frac{1}{p - {\rm df}_2(\nu_2)}\\
    &~+ \frac{p}{n} \left(\nu_2^2 \btheta_*^\sT (\bLambda + \nu_2\id)^{-2} \btheta_* + \nu_2^2 \btheta_*^\sT \bLambda (\bLambda + \nu_2\id)^{-2} \btheta_* \cdot \Tr( \bLambda (\bLambda + \nu_2\id)^{-2} ) \cdot \frac{1}{p - {\rm df}_2(\nu_2)} \right)\\
    &~\cdot \Tr(\bLambda(\bLambda + \nu_2\id)^{-2}) \cdot \frac{1}{p - {\rm df}_2(\nu_2)} \cdot \frac{1}{1 - \Upsilon(\nu_1, \nu_2)}\\
    =&~\frac{p\< \btheta_*, \bLambda ( \bLambda + \nu_2\id)^{-2} \btheta_* \>}{p - \Tr\left(\bLambda^2 (\bLambda + \nu_2\id)^{-2}\right)} + \frac{p\chi(\nu_2)}{n} \cdot \frac{\nu_2^2\left[ \< \btheta_*, (\bLambda + \nu_2\id)^{-2} \btheta_* \> \!+\! \chi(\nu_2) \< \btheta_*, \bLambda (\bLambda + \nu_2\id)^{-2} \btheta_* \> \right]}{1 - \Upsilon(\nu_1, \nu_2)}\,.
\end{aligned}
\]
Similarly, for the variance, using \cref{eq:df2v1} and \cref{eq:I3} for $I_3$, we have
\[
\begin{aligned}
    \mathcal{V}_{\mathcal{N},\lambda}^{\tt RFM} \sim&~ \sigma^2 p \Tr(\bLambda(\bLambda + \nu_2\id)^{-2}) \cdot \frac{1}{p - {\rm df}_2(\nu_2)}\cdot \frac{1}{n-n\Upsilon(\nu_1,\nu_2)}\\
    \sim&~ \sigma^2 \frac{\frac{p}{n}\chi(\nu_2)}{1-\Upsilon(\nu_1, \nu_2)}\,.
\end{aligned}
\]
Accordingly, we finish the proof.
\end{proof}

In the next, we present the proof for min-$\ell_2$-norm interpolator under RFMs.

\begin{proof}[Proof of \cref{prop:asy_equiv_norm_RFRR_minnorm}]
Similar to linear regression, we separate the two regimes $p<n$ and $p>n$ as well. For both of them, we provide asymptotic expansions in two steps, first with respect to $\bG$ and then $\bF$ in the under-parameterized regime and vice-versa for the over-parameterized regime.
\paragraph{Under-parameterized regime: Deterministic equivalent over $\bG$} For the variance term, we can use \cref{eq:trA3K} with $\bT=\bG$, $\bSigma=\bF^\sT\bF$, $\bA=\bF^\sT\bF$ and obtain
\[
\begin{aligned}
\mathcal{V}_{\mathcal{N},0}^{\tt RFM} =&~ \sigma^2 \cdot \Tr(\bZ^\sT \bZ(\bZ^\sT \bZ + \lambda\id)^{-2})\\
=&~ \sigma^2 \cdot p\Tr(\bF\bG^\sT \bG \bF^\sT(\bF\bG^\sT \bG \bF^\sT + p\lambda\id)^{-2})\\
=&~ \sigma^2 \cdot p\Tr(\bF^\sT \bF\bG^\sT ( \bG \bF^\sT \bF \bG^\sT + p\lambda\id)^{-2}\bG )\\
\sim&~ \sigma^2 \cdot p\Tr(\bF^\sT \bF ( \bF^\sT \bF + \tilde\lambda\id)^{-2} ) \cdot \frac{1}{n-p}\\
\sim&~ \sigma^2 \cdot \Tr(( \bF \bF^\sT )^{-1}) \cdot \frac{p}{n-p}\,,\\
\end{aligned}
\]
where $\tilde\lambda$ is defined by
\begin{equation}\label{eq:tilde_lambda}
    \tilde\lambda(1-\frac{1}{n}\widetilde{\rm df}_1(\tilde\lambda)) \sim \frac{p\lambda}{n}\,,
\end{equation}
where $\widetilde{\rm df}_1(\tilde\lambda)$ and $\widetilde{\rm df}_2(\tilde\lambda)$ are degrees of freedom associated to $\bF^\sT \bF$. In the under-parameterized regime ($p<n$), when $\lambda$ goes to zero, we have $\tilde\lambda \to 0$ and  $\widetilde{\rm df}_2(\tilde\lambda) \to p$ \citep{bach2024high}.

For the bias term, we use \cref{eq:trAB1K} with $\bT=\bG$, $\bSigma=\bF^\sT\bF$, $\bA=\btheta_* \btheta_*^\sT$, $\bB=\bF^\sT\bF$ and then obtain
\[
\begin{aligned}
\mathcal{B}_{\mathcal{N},0}^{\tt RFM} =&~ \Tr(\btheta_*^\sT \bG^\sT \bZ (\bZ^\sT \bZ + \lambda\id)^{-2} \bZ^\sT \bG\btheta_* )\\
=&~ p\Tr(\btheta_*^\sT \bG^\sT \bG \bF^\sT (\bF \bG^\sT \bG \bF^\sT + p\lambda\id)^{-2} \bF \bG^\sT \bG \btheta_* )\\
=&~ p\Tr(\btheta_* \btheta_*^\sT \bG^\sT ( \bG \bF^\sT \bF \bG^\sT + p\lambda\id)^{-1} \bG \bF^\sT \bF \bG^\sT ( \bG \bF^\sT \bF \bG^\sT + p\lambda\id)^{-1} \bG )\\
\sim&~ p\Tr(\btheta_* \btheta_*^\sT ( \bF^\sT \bF + \tilde\lambda\id)^{-1} \bF^\sT \bF ( \bF^\sT \bF + \tilde\lambda\id)^{-1} )\\ 
&~+ p \tilde\lambda^2 \Tr(\btheta_* \btheta_*^\sT ( \bF^\sT \bF + \tilde\lambda\id)^{-2}) \cdot \Tr(\bF^\sT \bF  ( \bF^\sT \bF + \tilde\lambda\id)^{-2}) \cdot \frac{1}{n-p}\\
\sim&~ p\Tr(\btheta_* \btheta_*^\sT \bF^\sT ( \bF \bF^\sT )^{-2} \bF) + p \Tr(\btheta_* \btheta_*^\sT ( \id - \bF^\sT (\bF\bF^\sT)^{-1} \bF )) \cdot \Tr(( \bF \bF^\sT)^{-1}) \cdot \frac{1}{n-p}\,.
\end{aligned}
\]

In the next, we are ready to eliminate the randomness over $\bF$.
\paragraph{Under-parameterized regime: deterministic equivalent over $\bF$}
For the variance term, from \citet[Sec 3.2]{bach2024high} we know that $\nicefrac{1}{\lambda_p}$ is almost surely the limit of $\Tr((\bF\bF^\sT)^{-1})$, thus we have
\[
\begin{aligned}
\Tr((\bF\bF^\sT)^{-1}) \sim \frac{1}{\lambda_p}\,,
\end{aligned}
\]
where $\lambda_p$ defined by ${\rm df_1}(\lambda_p) = p$, where ${\rm df_1}(\lambda_p)$ and ${\rm df_2}(\lambda_p)$ are degrees of freedom associated to $\bLambda$. Hence we can obtain
\[
\begin{aligned}
\mathcal{V}_{\mathcal{N},0}^{\tt RFM} \sim \sigma^2 \cdot \frac{1}{\lambda_p} \cdot \frac{p}{n-p} = \frac{\sigma^2p}{\lambda_p(n-p)}\,.
\end{aligned}
\]

For the bias term, denote $\bD:=\bF\bLambda^{-1/2}$, we first use \cref{eq:trA3K} with $\bT=\bD$, $\bSigma=\bLambda$, $\bA=\bLambda^{1/2} \btheta_* \btheta_*^\sT \bLambda^{1/2}$ and obtain the deterministic equivalent of the first term in $
\mathcal{B}_{\mathcal{N},0}^{\tt RFM}$
\[
\begin{aligned}
\Tr(\btheta_* \btheta_*^\sT \bF^\sT ( \bF \bF^\sT )^{-2} \bF) = \Tr( \bLambda^{1/2} \btheta_* \btheta_*^\sT \bLambda^{1/2} \bD^\sT ( \bD \bLambda \bD^\sT )^{-2} \bD) \sim \Tr( \btheta_* \btheta_*^\sT \bLambda ( \bLambda + \lambda_p )^{-2} ) \cdot \frac{1}{n-{\rm df}_2(\lambda_p)}\,.
\end{aligned}
\]
Then we use \cref{eq:trAB1K} with $\bT=\bD$, $\bSigma=\bLambda$, $\bA=\bLambda^{1/2} \btheta_* \btheta_*^\sT \bLambda^{1/2}$ and obtain 
\[
\begin{aligned}
\Tr(\btheta_* \btheta_*^\sT \bF^\sT (\bF \bF^\sT)^{-1} \bF) = \Tr( \bLambda^{1/2} \btheta_* \btheta_*^\sT \bLambda^{1/2} \bD^\sT ( \bD \bLambda \bD^\sT )^{-1} \bD) \sim \Tr(\btheta_* \btheta_*^\sT \bLambda (\bLambda +\lambda_p)^{-1})\,,
\end{aligned}
\]
Then the deterministic equivalent of the second term in $\mathcal{B}_{\mathcal{N},0}^{\tt RFM} $ is given by
\[
\begin{aligned}
\Tr(\btheta_* \btheta_*^\sT ( \id - \bF^\sT (\bF\bF^\sT)^{-1} \bF )) \sim \lambda_p \btheta_*^\sT (\bLambda +\lambda_p)^{-1} \btheta_*.
\end{aligned}
\]
Finally, combine the above equivalents and we have
\[
\begin{aligned}
\mathcal{B}_{\mathcal{N},0}^{\tt RFM} \sim&~ \btheta_*^\sT \bLambda (\bLambda +\lambda_p)^{-2} \btheta_* \cdot \frac{p}{n-{\rm df}_2(\lambda_p)} + \btheta_*^\sT (\bLambda +\lambda_p)^{-1} \btheta_* \cdot \frac{p}{n-p}\\
=&~ \frac{p\<\btheta_*, \bLambda (\bLambda +\lambda_p)^{-2} \btheta_*\>}{n-\Tr(\bLambda^2(\bLambda+\lambda_n\id)^{-2})} + \frac{p\<\btheta_*, (\bLambda +\lambda_p)^{-1} \btheta_*\>}{n-p}\,.
\end{aligned}
\]
\paragraph{Over-parameterized regime: deterministic equivalent over $\bF$}

Denote $ \bK:=\bLambda^{1/2}\bG^\sT\bG\bLambda^{1/2}$, for the variance term, we use \cref{eq:trA3K} with $\bT=\bD$, $\bSigma=\bA=\bK$ and obtain 
\[
\begin{aligned}
\mathcal{V}_{\mathcal{N},0}^{\tt RFM} =&~ \sigma^2 \cdot p\Tr(\bF\bG^\sT \bG \bF^\sT(\bF\bG^\sT \bG \bF^\sT + p\lambda\id)^{-2})\\
=&~ \sigma^2 \cdot p\Tr(\bK \bD^\sT (\bD \bK \bD^\sT + p\lambda\id)^{-2} \bD)\\
\sim&~ \sigma^2 \cdot p\Tr(\bK (\bK + \hat\lambda\id)^{-2}) \cdot \frac{1}{p-n}\\
\sim&~ \sigma^2 \cdot \Tr( (\bG \bLambda \bG^\sT )^{-1}) \cdot \frac{p}{p-n}\,,
\end{aligned}
\]
where $\hat\lambda$ is defined by
\begin{equation}\label{eq:hat_lambda}
    \hat\lambda(1-\frac{1}{n}\widehat{\rm df}_1(\hat\lambda)) \sim \frac{p\lambda}{n}\,,
\end{equation}
where $\widehat{\rm df}_1(\hat\lambda)$ and $\widehat{\rm df}_2(\hat\lambda)$ are degrees of freedom associated to $\bK$. In the over-parameterized regime ($p>n$), when $\lambda$ goes to zero, we have $\hat\lambda \to 0$ and  $\widehat{\rm df}_2(\hat\lambda) \to n$ \citep{bach2024high}.

For the bias term, we use \cref{eq:trA3K} with $\bT=\bD$, $\bSigma=\bK$, $\bA=\bLambda^{1/2} \bG^\sT \bG \btheta_* \btheta_*^\sT \bG^\sT \bG \bLambda^{1/2}$ and obtain 
\[
\begin{aligned}
\mathcal{B}_{\mathcal{N},0}^{\tt RFM} =&~ p\Tr(\btheta_*^\sT \bG^\sT \bG \bF^\sT (\bF \bG^\sT \bG \bF^\sT + p\lambda\id)^{-2} \bF \bG^\sT \bG \btheta_* )\\
=&~ p\Tr(\bLambda^{1/2} \bG^\sT \bG \btheta_* \btheta_*^\sT \bG^\sT \bG \bLambda^{1/2} \bD (\bD \bK \bD^\sT + p\lambda\id)^{-2} \bD )\\
\sim&~ p\Tr(\bLambda^{1/2} \bG^\sT \bG \btheta_* \btheta_*^\sT \bG^\sT \bG \bLambda^{1/2} (\bK + \hat\lambda\id)^{-2} ) \cdot \frac{1}{p-n}\\
\sim&~ \Tr( \btheta_* \btheta_*^\sT \bG^\sT (\bG \bLambda \bG^\sT)^{-1} \bG ) \cdot \frac{p}{p-n}\,.
\end{aligned}
\]

\paragraph{Over-parameterized regime: deterministic equivalent over $\bG$}

For the variance term, we have
\[
\begin{aligned}
\mathcal{V}_{\mathcal{N},0}^{\tt RFM} \sim \sigma^2 \cdot \frac{1}{\lambda_n} \cdot \frac{p}{p-n} = \frac{\sigma^2p}{\lambda_n(p-n)}.
\end{aligned}
\]

For the bias term, we have
\[
\begin{aligned}
\mathcal{B}_{\mathcal{N},0}^{\tt RFM} \sim&~ \Tr( \btheta_* \btheta_*^\sT ( \bLambda + \lambda_n)^{-1} ) \cdot \frac{p}{p-n}\\
=&~ \btheta_*^\sT ( \bLambda + \lambda_n)^{-1} \btheta_* \cdot \frac{p}{p-n}\\
=&~ \frac{p\<\btheta_*, ( \bLambda + \lambda_n)^{-1} \btheta_*\>}{p-n}\,.
\end{aligned}
\]
Finally, we conclude the proof.
\end{proof}

To build the connection between the test risk and norm for the min-$\ell_2$-norm estimator for random features regression, we also need the deterministic equivalent of the test risk as below.

\begin{proposition}[Asymptotic deterministic equivalence of the test risk of the min-$\ell_2$-norm interpolator]\label{prop:asy_equiv_error_RFRR_minnorm}
    Under \cref{ass:concentrated_RFRR}, for the minimum $\ell_2$-norm estimator $\hat{\ba}_{\min}$, we have the following deterministic equivalence: for the under-parameterized regime ($p<n$), we have
    \[
    \begin{aligned}
        \mathcal{B}^{\tt RFM}_{\mathcal{R},0} \sim \frac{n\lambda_p \<\btheta_*, (\bLambda +\lambda_p\id)^{-1} \btheta_*\>}{n-p}\,,\quad \mathcal{V}^{\tt RFM}_{\mathcal{R},0} \sim&~ \frac{\sigma^2p}{n-p}\,,
    \end{aligned}
    \]
    where $\lambda_p$ is defined by $\Tr(\bLambda(\bLambda+\lambda_p\id)^{-1}) \sim p$. In the over-parameterized regime ($p>n$), we have
    \[
    \begin{aligned}
        \mathcal{B}^{\tt RFM}_{\mathcal{R},0} \sim&~ \frac{n\lambda_n^2 \<\btheta_*, ( \bLambda + \lambda_n \id)^{-2} \btheta_*\>}{ n - \Tr(\bLambda^2(\bLambda+\lambda_n\id)^{-2})} + \frac{n\lambda_n \<\btheta_*, ( \bLambda + \lambda_n\id)^{-1} \btheta_*\>}{p-n}\,,\\
        \mathcal{V}^{\tt RFM}_{\mathcal{R},0} \sim&~  \frac{\sigma^2\Tr(\bLambda^2(\bLambda+\lambda_n\id)^{-2})}{n - \Tr(\bLambda^2(\bLambda+\lambda_n\id)^{-2})} + \frac{\sigma^2n}{p-n}\,,
    \end{aligned}
    \]
    where $\lambda_n$ is defined by $\Tr(\bLambda(\bLambda+\lambda_n\id)^{-1}) \sim n$.
\end{proposition}

\begin{proof}[Proof of \cref{prop:asy_equiv_error_RFRR_minnorm}]
For the proof, we separate the two regimes $p<n$ and $p>n$. For both of them, we provide asymptotic expansions in two steps, first with respect to $\bG$ and then $\bF$ in the under-parameterized regime and vice-versa for the over-parameterized regime.


\paragraph{Under-parameterized regime: deterministic equivalent over $\bG$}

For the variance term, in the under-parameterized regime, when $\lambda \to 0$, the variance term will become $\mathcal{V}^{\tt RFM}_{\mathcal{R},0} = \sigma^2 \cdot \Tr(\widehat{\bLambda}_{\bF} (\bZ^\sT \bZ)^{-1})$. Accordingly, using \citet[Eq. (12)]{bach2024high}, we have 
\[
\begin{aligned}
\mathcal{V}^{\tt RFM}_{\mathcal{R},0} =&~ \sigma^2 \cdot \Tr(\widehat{\bLambda}_{\bF} (\bZ^\sT \bZ)^{-1})\\
=&~ \sigma^2 \cdot \Tr(\bF\bF^\sT(\bF\bG^\sT\bG\bF^\sT)^{-1})\\
\sim&~ \frac{\sigma^2}{n-p} \cdot \Tr(\bF\bF^\sT(\bF\bF^\sT)^{-1})\\
=&~\frac{\sigma^2p}{n-p}\,.
\end{aligned}
\]

For the bias term, it can be decomposed into
\[
\begin{aligned}
\mathcal{B}^{\tt RFM}_{\mathcal{R},0} =&~ \|\btheta_* - p^{-1/2} \bF^\sT (\bZ^\sT \bZ + \lambda\id)^{-1} \bZ^\sT \bm{G} \btheta_*\|_2^2\\
=&~ \btheta_*^\sT \btheta_* -2 p^{-1/2}\btheta_*^\sT \bF^\sT (\bZ^\sT \bZ + \lambda\id)^{-1} \bZ^\sT \bm{G} \btheta_* + \btheta_*^\sT \bG^\sT \bZ (\bZ^\sT \bZ + \lambda\id)^{-1} \widehat{\bLambda}_{\bF} (\bZ^\sT \bZ + \lambda\id)^{-1} \bZ^\sT \bm{G} \btheta_*.
\end{aligned}
\]
For the second term: $p^{-1/2}\btheta_*^\sT \bF^\sT (\bZ^\sT \bZ + \lambda\id)^{-1} \bZ^\sT \bm{G} \btheta_*$, we can use \cref{eq:trA1K} with $\bT=\bG$, $\bSigma=\bF^\sT\bF$, $\bA=\btheta_*\btheta_*^\sT \bF^\sT \bF$ and obtain
\[
\begin{aligned}
p^{-1/2}\btheta_*^\sT \bF^\sT (\bZ^\sT \bZ + \lambda\id)^{-1} \bZ^\sT \bm{G} \btheta_* =&~ \Tr( \btheta_*\btheta_*^\sT \bF^\sT \bF \bG^\sT (\bG\bF^\sT\bF\bG^\sT + p\lambda\id)^{-1} \bG)\\
\sim&~ \Tr( \btheta_*\btheta_*^\sT \bF^\sT \bF (\bF^\sT\bF + \tilde\lambda\id)^{-1})\\
\sim&~ \Tr( \btheta_*\btheta_*^\sT \bF^\sT (\bF \bF^\sT)^{-1}\bF)\,,
\end{aligned}
\]
where the implicit regularization parameter $\tilde\lambda$ is defined by \cref{eq:tilde_lambda}.

For the third term: $\btheta_*^\sT \bG^\sT \bZ (\bZ^\sT \bZ + \lambda\id)^{-1} \widehat{\bLambda}_{\bF} (\bZ^\sT \bZ + \lambda\id)^{-1} \bZ^\sT \bm{G} \btheta_*$, we can use \cref{eq:trAB1K} with $\bT=\bG$, $\bSigma=\bF^\sT\bF$, $\bA=\btheta_*\btheta_*^\sT$, $\bB=\bF^\sT\bF\bF^\sT\bF$ and obtain
\[
\begin{aligned}
&~\btheta_*^\sT \bG^\sT \bZ (\bZ^\sT \bZ + \lambda\id)^{-1} \widehat{\bLambda}_{\bF} (\bZ^\sT \bZ + \lambda\id)^{-1} \bZ^\sT \bm{G} \btheta_*\\
=&~\Tr(\btheta_* \btheta_*^\sT \bG^\sT \bG \bF^\sT(\bF \bG^\sT \bG \bF^\sT + p\lambda\id)^{-1} \bF\bF^\sT (\bF \bG^\sT \bG \bF^\sT + p\lambda\id)^{-1} \bF \bG^\sT \bG )\\
=&~\Tr(\btheta_* \btheta_*^\sT \bG^\sT ( \bG \bF^\sT \bF \bG^\sT + p\lambda\id)^{-1} \bG \bF^\sT \bF\bF^\sT \bF \bG^\sT ( \bG \bF^\sT \bF \bG^\sT + p\lambda\id)^{-1} \bG )\\
\sim&~ \Tr(\btheta_* \btheta_*^\sT (\bF^\sT \bF + \tilde\lambda\id)^{-1} \bF^\sT \bF\bF^\sT \bF (\bF^\sT \bF + \tilde\lambda\id)^{-1})\\
&~+ \tilde\lambda^2 \Tr(\btheta_* \btheta_*^\sT (\bF^\sT \bF + \tilde\lambda\id)^{-2}) \cdot \Tr(\bF^\sT \bF\bF^\sT \bF (\bF^\sT \bF + \tilde\lambda\id)^{-2}) \cdot \frac{1}{n-p}\\
\sim&~ \Tr(\btheta_* \btheta_*^\sT \bF^\sT (\bF \bF^\sT)^{-1} \bF) + \Tr(\btheta_* \btheta_*^\sT (\id -\bF^\sT (\bF \bF^\sT)^{-1} \bF )) \cdot \frac{p}{n-p}\,.
\end{aligned}
\]
Combining the above equivalents, we have
\[
\begin{aligned}
\mathcal{B}^{\tt RFM}_{\mathcal{R},0} =&~ \btheta_*^\sT \btheta_* -\Tr(\btheta_* \btheta_*^\sT \bF^\sT (\bF \bF^\sT)^{-1} \bF) + \Tr(\btheta_* \btheta_*^\sT (\id -\bF^\sT (\bF \bF^\sT)^{-1} \bF )) \cdot \frac{p}{n-p}\\
=&~ \btheta_*^\sT \btheta_* \cdot \frac{n}{n-p} -\Tr(\btheta_* \btheta_*^\sT \bF^\sT (\bF \bF^\sT)^{-1} \bF) \cdot \frac{n}{n-p}\,.
\end{aligned}
\]
\paragraph{Under-parameterized regime: deterministic equivalent over $\bF$}
For the bias term, we can use \cref{eq:trA1K} with $\bT = \bD := \bF \bLambda^{-1/2}$, $\bA=\bLambda^{1/2} \btheta_* \btheta_*^\sT \bLambda^{1/2}$ and obtain
\[
\begin{aligned}
\Tr(\btheta_* \btheta_*^\sT \bF^\sT (\bF \bF^\sT)^{-1} \bF) =&~ \Tr(\bLambda^{1/2} \btheta_* \btheta_*^\sT \bLambda^{1/2} \bD^\sT (\bD \bLambda \bD^\sT)^{-1} \bD )\\
\sim&~ \Tr(\bLambda^{1/2} \btheta_* \btheta_*^\sT \bLambda^{1/2} (\bLambda +\lambda_p)^{-1})\\
=&~ \btheta_*^\sT \bLambda (\bLambda +\lambda_p)^{-1} \btheta_*\,.
\end{aligned}
\]
Thus, we finally obtain
\[
\begin{aligned}
\mathcal{B}^{\tt RFM}_{\mathcal{R},0} \sim&~ \btheta_*^\sT \btheta_* \cdot \frac{n}{n-p} - \btheta_*^\sT \bLambda (\bLambda +\lambda_p)^{-1} \btheta_* \cdot \frac{n}{n-p}\\
=&~ \lambda_p \btheta_*^\sT (\bLambda +\lambda_p)^{-1} \btheta_* \cdot \frac{n}{n-p}\\
=&~ \frac{n\lambda_p \<\btheta_*, (\bLambda +\lambda_p\id)^{-1} \btheta_*\>}{n-p}\,.
\end{aligned}
\]
\paragraph{Over-parameterized regime: deterministic equivalent over $\bF$}
For the variance term, with $\bD := \bF \bLambda^{-1/2}$ and $\bK := \bLambda^{1/2} \bG^\sT \bG \bLambda^{1/2}$ we can obtain
\[
\begin{aligned}
\mathcal{V}^{\tt RFM}_{\mathcal{R},0} &= \sigma^2 \cdot \mathrm{Tr}(\widehat{\bLambda}_{\bF} \bZ^\sT \bZ (\bZ^\sT \bZ + \lambda\id)^{-2})\\
&= \sigma^2 \cdot \mathrm{Tr}(\bF \bF^\sT \bF \bG^\sT \bG \bF^\sT (\bF \bG^\sT \bG \bF^\sT + p\lambda\id)^{-2})\\
&= \sigma^2 \cdot \mathrm{Tr}(\bD \bLambda \bD^\sT \bD \bLambda^{1/2} \bG^\sT \bG \bLambda^{1/2} \bD^\sT (\bD \bLambda^{1/2} \bG^\sT \bG \bLambda^{1/2} \bD^\sT + p\lambda\id)^{-2})\\
&= \sigma^2 \cdot \mathrm{Tr}(\bLambda \bD^\sT (\bD \bK \bD^\sT + p\lambda\id)^{-1} \bD \bK \bD^\sT (\bD \bK \bD^\sT + p\lambda\id)^{-1} \bD )\,,
\end{aligned}
\]
then we directly use \cref{eq:trAB1K} with $\bT=\bD$, $\bSigma=\bK$, $\bA=\bLambda$, $\bB=\bK$ and obtain
\[
\begin{aligned}
&~\mathrm{Tr}(\bLambda \bD^\sT (\bD \bK \bD^\sT + p\lambda\id)^{-1} \bD \bK \bD^\sT (\bD \bK \bD^\sT + p\lambda\id)^{-1} \bD )\\
\sim&~ \mathrm{Tr}(\bLambda ( \bK + \hat\lambda\id)^{-1} \bK ( \bK + \hat\lambda\id)^{-1} ) + \hat\lambda^2 \mathrm{Tr}(\bLambda ( \bK + \hat\lambda\id)^{-2} ) \cdot \mathrm{Tr}( \bK ( \bK + \hat\lambda\id)^{-2} ) \cdot \frac{1}{p-n}\\
\sim&~ \Tr(\bLambda^2 \bG^\sT (\bG \bLambda \bG^\sT)^{-2} \bG ) + \mathrm{Tr}(\bLambda ( \id - \bLambda^{1/2}\bG^\sT (\bG \bLambda \bG^\sT)^{-1} \bG \bLambda^{1/2} ) ) \cdot \mathrm{Tr}( (\bG \bLambda \bG^\sT)^{-1} ) \cdot \frac{1}{p-n}\,,
\end{aligned}
\]
where the implicit regularization parameter $\hat\lambda$ is defined by \cref{eq:hat_lambda}.

For the bias term, first we have
\[
\begin{aligned}
p^{-1/2}\btheta_*^\sT \bF^\sT (\bZ^\sT \bZ + \lambda\id)^{-1} \bZ^\sT \bm{G} \btheta_* =&~ \Tr( \btheta_*\btheta_*^\sT \bF^\sT (\bF\bG^\sT \bG\bF^\sT+ p\lambda\id)^{-1} \bF \bG^\sT \bG)\\
=&~ \Tr(\bLambda^{1/2} \bG^\sT \bG \btheta_*\btheta_*^\sT \bLambda^{1/2} \bD^\sT (\bD \bK \bD^\sT+ p\lambda\id)^{-1} \bD )\,,
\end{aligned}
\]
then we use \cref{eq:trA1K} with $\bT=\bD$, $\bSigma=\bK$, $\bA=\bLambda^{1/2} \bG^\sT \bG \btheta_*\btheta_*^\sT \bLambda^{1/2}$ and obtain
\[
\begin{aligned}
\Tr(\bLambda^{1/2} \bG^\sT \bG \btheta_*\btheta_*^\sT \bLambda^{1/2} \bD^\sT (\bD \bK \bD^\sT+ p\lambda\id)^{-1} \bD ) \sim&~ \Tr( \btheta_*\btheta_*^\sT \bLambda \bG^\sT ( \bG \bLambda \bG^\sT)^{-1} \bG)\,.
\end{aligned}
\]
Furthermore, we use \cref{eq:trAB1K} with $\bT=\bD$, $\bSigma=\bK$, $\bA=\bLambda^{1/2} \bG^\sT \bG \btheta_* \btheta_*^\sT \bG^\sT \bG \bLambda^{1/2}$, $\bB=\bLambda$ and obtain
\[
\begin{aligned}
&~\btheta_*^\sT \bG^\sT \bZ (\bZ^\sT \bZ + \lambda\id)^{-1} \widehat{\bLambda}_{\bF} (\bZ^\sT \bZ + \lambda\id)^{-1} \bZ^\sT \bm{G} \btheta_*\\
=&~\Tr(\bLambda^{1/2} \bG^\sT \bG \btheta_* \btheta_*^\sT \bG^\sT \bG \bLambda^{1/2} \bD^\sT(\bD \bK \bD^\sT + p\lambda\id)^{-1} \bD \bLambda \bD^\sT (\bD \bK \bD^\sT + p\lambda\id)^{-1} \bD )\\
\sim&~ \Tr(\bLambda^{1/2} \bG^\sT \bG \btheta_* \btheta_*^\sT \bG^\sT \bG \bLambda^{1/2} ( \bK + \hat\lambda\id)^{-1} \bLambda ( \bK + \hat\lambda\id)^{-1} )\\
&~+ \hat\lambda^2 \Tr(\bLambda^{1/2} \bG^\sT \bG \btheta_* \btheta_*^\sT \bG^\sT \bG \bLambda^{1/2} ( \bK + \hat\lambda\id)^{-2} ) \cdot \Tr( \bLambda ( \bK + \hat\lambda\id)^{-2} ) \cdot \frac{1}{p-n}\\
\sim&~ \Tr( \btheta_* \btheta_*^\sT \bG^\sT ( \bG \bLambda \bG^\sT )^{-1} \bG \bLambda^2 \bG^\sT ( \bG \bLambda \bG^\sT )^{-1} \bG)\\
&~+ \Tr( \btheta_* \btheta_*^\sT \bG^\sT ( \bG \bLambda \bG^\sT)^{-1} \bG) \cdot \Tr(\bLambda ( \id - \bLambda^{1/2}\bG^\sT (\bG \bLambda \bG^\sT)^{-1} \bG \bLambda^{1/2} ) ) \cdot \frac{1}{p-n}\,.
\end{aligned}
\]
In the next, we are ready to eliminate the randomness over $\bG$.
\paragraph{Over-parameterized regime: deterministic equivalent over $\bG$}
For the variance term, we use \cref{eq:trA3K} to obtain
\[
\begin{aligned}
\Tr(\bLambda^2 \bG^\sT (\bG \bLambda \bG^\sT)^{-2} \bG ) \sim \frac{{\rm df}_2(\lambda_n)}{n - {\rm df}_2(\lambda_n)}\,.
\end{aligned}
\]
Then we use \cref{eq:trA1K} to obtain
\[
\begin{aligned}
\Tr(\bLambda^2 \bG^\sT (\bG \bLambda \bG^\sT)^{-1} \bG ) \sim \Tr(\bLambda^2(\bLambda + \lambda_n)^{-1}),
\end{aligned}
\]
where $\lambda_n$ is defined by ${\rm df_1}(\lambda_n) = n$. Hence we have
\[
\begin{aligned}
\Tr(\bLambda ( \id - \bLambda^{1/2}\bG^\sT (\bG \bLambda \bG^\sT)^{-1} \bG \bLambda^{1/2} ) ) \sim n\lambda_n.
\end{aligned}
\]
Combine the above equivalents, we have
\[
\begin{aligned}
\mathcal{V}^{\tt RFM}_{\mathcal{R},0} \sim&~  \sigma^2 \cdot  \frac{{\rm df}_2(\lambda_n)}{n - {\rm df}_2(\lambda_n)} + \sigma^2 \cdot \frac{n}{p-n}\\
=&~ \frac{\sigma^2\Tr(\bLambda^2(\bLambda+\lambda_n\id)^{-2})}{n - \Tr(\bLambda^2(\bLambda+\lambda_n\id)^{-2})} + \frac{\sigma^2n}{p-n}\,.
\end{aligned}
\]

For the bias term, we first use \cref{eq:trA1K} to obtain
\[
\begin{aligned}
\Tr( \btheta_*\btheta_*^\sT \bLambda \bG^\sT ( \bG \bLambda \bG^\sT)^{-1} \bG) \sim \Tr(\btheta_*\btheta_*^\sT \bLambda (\bLambda + \lambda_n)^{-1})\,.
\end{aligned}
\]
Moreover, we use \cref{eq:trAB1K} to obtain
\[
\begin{aligned}
\Tr( \btheta_* \btheta_*^\sT \bG^\sT ( \bG \bLambda \bG^\sT )^{-1} \bG \bLambda^2 \bG^\sT ( \bG \bLambda \bG^\sT )^{-1} \bG) \sim&~ \Tr(\btheta_* \btheta_*^\sT \bLambda^2 ( \bLambda + \lambda_n )^{-2})\\ 
&~+ \lambda_n^2 \cdot \Tr(\btheta_* \btheta_*^\sT ( \bLambda + \lambda_n )^{-2}) \cdot \frac{{\rm df}_2(\lambda_n)}{n - {\rm df}_2(\lambda_n)}.
\end{aligned}
\]
Accordingly, we finally conclude that
\[
\begin{aligned}
\mathcal{B}^{\tt RFM}_{\mathcal{R},0} \sim&~ \lambda_n^2 \btheta_*^\sT ( \bLambda + \lambda_n \id)^{-2} \btheta_* \cdot \frac{n}{ n - {\rm df}_2(\lambda_n)} + \lambda_n \btheta_*^\sT ( \bLambda+ \lambda_n\id)^{-1} \btheta_* \cdot \frac{n}{p-n}\\
=&~ \frac{n\lambda_n^2 \<\btheta_*, ( \bLambda + \lambda_n \id)^{-2} \btheta_*\>}{ n - \Tr(\bLambda^2(\bLambda+\lambda_n\id)^{-2})} + \frac{n\lambda_n \<\btheta_*, ( \bLambda + \lambda_n\id)^{-1} \btheta_*\>}{p-n}\,.
\end{aligned}
\]
\end{proof}

