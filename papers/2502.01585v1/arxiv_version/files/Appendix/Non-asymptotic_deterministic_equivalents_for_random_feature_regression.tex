\subsection{Non-asymptotic deterministic equivalence for random features ridge regression}
\label{app:nonasy_deter_equiv_rf}

Here we present the proof for the non-asymptotic results on the variance and then discuss the related results on bias due to the insufficient deterministic equivalence.

\subsubsection{Proof on the variance term}

\begin{theorem}[Deterministic equivalence of variance part of the $\ell_2$ norm]\label{prop:det_equiv_RFRR_V}
    Assume the features $\{\bz_i\}_{i \in [n]}$ and $\{\boldf_j\}_{j \in [p]}$ satisfy \cref{ass:concentrated_RFRR} with a constant $C_* > 0$. Then for any $D,K > 0$, there exist constant $\eta_* \in (0, 1/2)$ and $C_{*,D,K} > 0$ ensuring the following property holds. For any $n,p \geq C_{*,D,K}$, $\lambda > 0$, if the following condition is satisfied:
    \begin{equation*}
        \lambda  \geq n^{-K}, \quad \gamma_\lambda \geq p^{-K}, \quad \trho_{\lambda} (n,p)^{5/2} \log^{3/2} (n) \leq K \sqrt{n}\,, \quad \trho_\lambda (n,p)^2 \cdot \rho_{\gamma_+} (p)^{7} \log^4 (p) \leq K \sqrt{p}\,,
    \end{equation*}
    then with probability at least $1-n^{-D}-p^{-D}$, we have that
    \[
    \begin{aligned}
        \left|\mathcal{V}_{\mathcal{N},\lambda}^{\tt RFM} - \sV_{\sN,\lambda}^{\tt RFM}\right| \leq&~ C_{x, D, K} \cdot \mathcal{E}_V(n, p) \cdot \sV_{\sN,\lambda}^{\tt RFM}\,.
    \end{aligned}
    \]
    where the approximation rate is given by
    \[
    \mathcal{E}_V(n, p) := \frac{\widetilde{\rho}_\lambda(n, p)^6 \log^{5/2}(n)}{\sqrt{n}} + \frac{\widetilde{\rho}_\lambda(n, p)^2 \cdot \rho_{\gamma_+}(p)^7 \log^3(p)}{\sqrt{p}}.
    \]
\end{theorem}

\begin{proof}[Proof of \cref{prop:det_equiv_RFRR_V}]
First, note that $\mathcal{V}_{\mathcal{N},\lambda}^{\tt RFM}$ can be written in terms of the functional $\Phi_4$ defined in \cref{eq:functionals_Z}:
\[
\mathcal{V}_{\mathcal{N},\lambda}^{\tt RFM} = \sigma^2 \cdot n \Phi_4 ( \bZ ; \hbLambda^{-1}_\bF,\lambda).
\]
Recall that $\cA_\cF$ is the event defined in \citet[Eq. (79)]{defilippis2024dimension}. Under the assumptions, we have
\[
\P (\cA_\cF) \geq 1 - p^{-D}.
\]
Hence, applying \cref{prop:det_Z} for $\bF \in \cA_\cF$ and via union bound, we obtain that with probability at least $1 - p^{-D} - n^{-D}$,
\begin{equation}\label{eq:var_remove_Z}
\left| n\Phi_4 ( \bZ ; \hbLambda^{-1}_\bF,\lambda) - n\tPhi_5 ( \bF ; \hbLambda^{-1}_\bF, p\nu_1 )\right|  \leq C_{*,D,K} \cdot \cE_1 (p,n) \cdot n\tPhi_5 ( \bF ; \hbLambda^{-1}_\bF, p\nu_1 ),
\end{equation}
and we recall the expressions
\[
n\tPhi_5 ( \bF ; \hbLambda^{-1}_\bF, p\nu_1 ) = \frac{\tPhi_6 ( \bF ; \hbLambda^{-1}_\bF , p\nu_1) }{n - \tPhi_6 ( \bF ; \id , p\nu_1)}, \quad \quad \tPhi_6 ( \bF ; \hbLambda^{-1}_\bF , p\nu_1) = p\Tr \big( \bF \bF^\sT ( \bF \bF^\sT + p \nu_1)^{-2} \big).
\]
From \citet[Lemma B.11]{defilippis2024dimension}, we have with probability at least $1 - p^{-D}$
\[
\begin{aligned}
\left| p\Tr (\bF \bF^\sT ( \bF \bF^\sT + p\nu_1)^{-2} ) - p^2\Psi_3 ( \nu_2 ; \bLambda^{-1} )\right| \leq&~ C_{*,D,K} \cdot \rho_{\gamma_+} (p) \cdot \cE_3 (p) \cdot p^2\Psi_3 ( \nu_2 ; \bLambda^{-1} )\,,
\end{aligned}
\]
where the approximation rate \(\cE_3 (p)\) is given by
\[
    \mathcal{E}_3(p) := \frac{\rho_{\gamma_+}(p)^6 \log^3(p)}{\sqrt{p}}.
\]


Furthermore, from the proof of \citet[Theorem B.12]{defilippis2024dimension}, we have with probability at least $1 - p^{-D}$,
\[
\left| (1 - n^{-1} \tPhi_6 (\bF; \id, p\nu_1))^{-1} - (1 -  \Upsilon (\nu_1,\nu_2) )^{-1} \right| \leq C_{*,D,K} \cdot \trho_\lambda (n,p) \rho_{\gamma_+} (p) \cE_3 (p) \cdot  (1 -   \Upsilon (\nu_1,\nu_2))^{-1}.
\]
Combining those two bounds, we obtain
\[
\left| \frac{\tPhi_6 ( \bF ; \hbLambda^{-1}_\bF , p\nu_1) }{n - \tPhi_6 ( \bF ; \id , p\nu_1)} - \frac{p^2\Psi_3 ( \nu_2 ; \bLambda^{-1} )}{n - n\Upsilon (\nu_1,\nu_2)} \right| \leq C_{*,D,K} \cdot \trho_\lambda (n,p) \rho_{\gamma_+} (p) \cE_3 (p) \cdot \frac{p^2\Psi_3 ( \nu_2 ; \bLambda^{-1} )}{n - n\Upsilon (\nu_1,\nu_2)}.
\]
Finally, we can combine this bound with \cref{eq:var_remove_Z} to obtain via union bound that with probability at least $1 - n^{-D} - p^{-D}$, 
\[
\left| n\Phi_4 ( \bZ ; \hbLambda^{-1}_\bF,\lambda) - \frac{p^2\Psi_3 ( \nu_2 ; \bLambda^{-1} )}{n - n\Upsilon (\nu_1,\nu_2)} \right| \leq C_{*,D,K} \left\{ \cE_1 (p,n) + \trho_\lambda (n,p) \rho_{\gamma_+} (p) \cE_3 (p) \right\} \frac{p^2\Psi_3 ( \nu_2 ; \bLambda^{-1} )}{n - n\Upsilon (\nu_1,\nu_2)}\,.
\]
Replacing the rate $\cE_j$ by their expressions conclude the proof of this theorem.
\end{proof}

\subsubsection{Discussion on the bias term}\label{app:discuss_bias}

We present the deterministic equivalence of the bias term as an informal result, without a Existing deterministic equivalence results appear insufficient to directly establish this desired bias result. While we believe this is doable under  additional assumptions, a complete proof is beyond the scope of this paper..

In the proof of the bias term, deterministic equivalences for functionals of the form 
\[
\Tr \left( \bA \left( \bX^\sT \bX \right)^2 (\bX^\sT \bX + \lambda)^{-2} \right)
\]
are required. However, such equivalences are currently unavailable, necessitating the introduction of technical assumptions to leverage the deterministic equivalences of \(\Phi_2(\bX; \bA, \lambda)\) and \(\Phi_4(\bX; \bA, \lambda)\).

Furthermore, the proof of the bias term in \cite{defilippis2024dimension} suggests that deriving deterministic equivalences for the bias of the \(\ell_2\) norm, analogous to \citet[Proposition B.7]{defilippis2024dimension}, is also required but remains unresolved.

Addressing these gaps in deterministic equivalence is an important direction for future work, particularly to establish rigorous proofs for the currently missing results.
