\subsection{Proofs on relationship between test risk and \texorpdfstring{$\ell_2$}{L2} norm of random feature ridge regression estimator}
\label{app:relationship_rf}

To derive the relationship between test risk and norm for the random feature model, we first examine the linear relationship in the over-parameterized regime. Next, we analyze the case where \(\bLambda = \id_m\) with \(n < m < \infty\) (finite rank), followed by the relationship under the power-law assumption.



\subsubsection{Proof for min-norm interpolator in the over-parameterized regime}
According to the formulation in \cref{prop:asy_equiv_norm_RFRR_minnorm} and \cref{prop:asy_equiv_error_RFRR_minnorm}, we have for the under-parameterized regime ($p<n$), we have
\[
\begin{aligned}
    \mathcal{B}^{\tt RFM}_{\mathcal{N},0} \sim& \sB^{\tt RFM}_{\sN,0} = \frac{p\<\btheta_*, \bLambda (\bLambda +\lambda_p\id)^{-2} \btheta_*\>}{n - \Tr(\bLambda^2(\bLambda+\lambda_p\id)^{-2})} + \frac{p\<\btheta_*, (\bLambda +\lambda_p\id)^{-1} \btheta_*\>}{n-p}\,,\\
    \mathcal{V}^{\tt RFM}_{\mathcal{N},0} \sim& \sV^{\tt RFM}_{\sN,0} = \frac{\sigma^2p}{\lambda_p(n-p)}\,,
\end{aligned}
\]
\[
\begin{aligned}
    \mathcal{B}^{\tt RFM}_{\mathcal{R},0} \sim \sB^{\tt RFM}_{\sR,0} = \frac{n\lambda_p \<\btheta_*, (\bLambda +\lambda_p\id)^{-1} \btheta_*\>}{n-p}\,,\quad \mathcal{V}^{\tt RFM}_{\mathcal{R},0} \sim \sV^{\tt RFM}_{\sR,0} = \frac{\sigma^2p}{n-p}\,.
\end{aligned}
\]
In the over-parameterized regime ($p>n$), we have
\[
\begin{aligned}
    \mathcal{B}^{\tt RFM}_{\mathcal{N},0} \sim \sB^{\tt RFM}_{\sN,0} = \frac{p\<\btheta_*, ( \bLambda + \lambda_n\id)^{-1} \btheta_*\>}{p-n}\,,\quad
    \mathcal{V}^{\tt RFM}_{\mathcal{N},0} \sim \sV^{\tt RFM}_{\sN,0} = \frac{\sigma^2p}{\lambda_n(p-n)}\,,
\end{aligned}
\]
\[
\begin{aligned}
    \mathcal{B}^{\tt RFM}_{\mathcal{R},0} \sim&~ \sB^{\tt RFM}_{\sR,0} = \frac{n\lambda_n^2 \<\btheta_*, ( \bLambda + \lambda_n \id)^{-2} \btheta_*\>}{ n - \Tr(\bLambda^2(\bLambda+\lambda_n\id)^{-2})} + \frac{n\lambda_n \<\btheta_*, ( \bLambda + \lambda_n\id)^{-1} \btheta_*\>}{p-n}\,,\\
    \mathcal{V}^{\tt RFM}_{\mathcal{R},0} \sim&~  \sV^{\tt RFM}_{\sR,0} = \frac{\sigma^2\Tr(\bLambda^2(\bLambda+\lambda_n\id)^{-2})}{n - \Tr(\bLambda^2(\bLambda+\lambda_n\id)^{-2})} + \frac{\sigma^2n}{p-n}\,.
\end{aligned}
\]
With these formulations we can introduce the relationship between test risk and norm in the over-parameterized regime as follows. 

\begin{proof}[Proof of \cref{prop:relation_minnorm_overparam}]
In the over-parameterized regime ($p > n$), we have
\[
    \sN_{0}^{\tt RFM} 
    = 
    \sB^{\tt RFM}_{\sN,0} + \sV^{\tt RFM}_{\sN,0} 
    = 
    \frac{p\<\btheta_*, ( \bLambda + \lambda_n\id)^{-1} \btheta_*\>}{p-n} + \frac{\sigma^2p}{\lambda_n(p-n)} 
    = 
    \left[\<\btheta_*, ( \bLambda + \lambda_n\id)^{-1} \btheta_*\> + \frac{\sigma^2}{\lambda_n}\right] \frac{p}{p-n}\,.
\]
\[
\begin{aligned}
    \sR_{0}^{\tt RFM} 
    =&
    \frac{n\lambda_n^2 \<\btheta_*, ( \bLambda + \lambda_n \id)^{-2} \btheta_*\>}{ n - \Tr(\bLambda^2(\bLambda+\lambda_n\id)^{-2})} 
    + 
    \frac{n\lambda_n \<\btheta_*, ( \bLambda + \lambda_n\id)^{-1} \btheta_*\>}{p-n}
    + 
    \frac{\sigma^2\Tr(\bLambda^2(\bLambda+\lambda_n\id)^{-2})}{n - \Tr(\bLambda^2(\bLambda+\lambda_n\id)^{-2})} 
    + 
    \frac{\sigma^2n}{p-n}\\
    =& 
    \frac{n\lambda_n^2 \<\btheta_*, ( \bLambda + \lambda_n \id)^{-2} \btheta_*\> + \sigma^2\Tr(\bLambda^2(\bLambda+\lambda_n\id)^{-2})}{ n - \Tr(\bLambda^2(\bLambda+\lambda_n\id)^{-2})} 
    + 
    \left[n\lambda_n \<\btheta_*, ( \bLambda + \lambda_n\id)^{-1} \btheta_*\> + \sigma^2n\right]\frac{1}{p-n}\,.
\end{aligned}
\]
Then we eliminate $p$ and obtain that the deterministic equivalents of the estimator's test risk and norm, $\sR^{\tt RFM}_{0}$ and $\sN^{\tt RFM}_{0}$, in over-parameterized regimes ($p>n$) admit
\[
\sR_{0}^{\tt RFM} 
= 
\lambda_n\sN_{0}^{\tt RFM} 
- 
\left[\lambda_n\<\btheta_*, ( \bLambda + \lambda_n\id)^{-1} \btheta_*\> + \sigma^2\right] 
+ 
\frac{n\lambda_n^2 \<\btheta_*, ( \bLambda + \lambda_n \id)^{-2} \btheta_*\> + \sigma^2\Tr(\bLambda^2(\bLambda+\lambda_n\id)^{-2})}{ n - \Tr(\bLambda^2(\bLambda+\lambda_n\id)^{-2})}\,. 
\]
\end{proof}


\subsubsection{Proof on isotropic features with finite rank}
Here we present the proof of \cref{prop:relation_minnorm_id_rf} with \( \bLambda = \id_m \).

\begin{proof}[Proof of \cref{prop:relation_minnorm_id_rf}]
Here we consider the case where \( \bLambda = \id_m \). Under this condition, the definitions of \( \lambda_p \) and \( \lambda_n \) above are simplified to \( \frac{m}{1+\lambda_p} = p \) and \( \frac{m}{1+\lambda_n} = n \), respectively. Consequently, \( \lambda_p \) and \( \lambda_n \) have explicit expressions given by \( \lambda_p = \frac{m-p}{p} \) and \( \lambda_n = \frac{m-n}{n} \), respectively.

First, in the over-parameterized regime ($p>n$), we have
\[
\begin{aligned}
     \sB^{\tt RFM}_{\sN,0} = \frac{p\frac{1}{1+\lambda_n}\|\btheta_*\|_2^2}{p - n} = \frac{np}{m(p-n)} \|\btheta_*\|_2^2\,,\quad
    \sV^{\tt RFM}_{\sN,0} = \frac{\sigma^2p}{\lambda_n(p-n)} = \frac{\sigma^2np}{(m-n)(p-n)}\,.
\end{aligned}
\]
\[
\begin{aligned}
     \sB^{\tt RFM}_{\sR,0} =& \frac{n\lambda_n^2 \frac{1}{(1+\lambda_n)^2}\|\btheta_*\|_2^2}{n-\frac{m}{(1+\lambda_n)^2}} + \frac{n\lambda_n\frac{1}{1+\lambda_n}\|\btheta_*\|_2^2}{p-n} = \frac{p(m-n)}{m(p-n)} \|\btheta_*\|_2^2\,,\\ 
     \sV^{\tt RFM}_{\sR,0} =& \frac{\sigma^2\frac{m}{(1+\lambda_n)^2}}{n-\frac{m}{(1+\lambda_n)^2}}+\frac{\sigma^2n}{p-n}=\frac{\sigma^2n}{m-n} + \frac{\sigma^2n}{p-n}\,.
\end{aligned}
\]
We eliminate $p$ and obtain that the relationship between $\sV^{\tt RFM}_{\sR,0}$ and $\sV^{\tt RFM}_{\sN,0}$ is
\[
\begin{aligned}
\sV^{\tt RFM}_{\sR,0} = \frac{m-n}{n}\sV^{\tt RFM}_{\sN,0} + \frac{2n -m}{m-n} \sigma^2\,.
\end{aligned}
\]
similarly, the relationship between $\sB^{\tt RFM}_{\sR,0}$ and $\sB^{\tt RFM}_{\sN,0}$ is
\[
\begin{aligned}
\sB^{\tt RFM}_{\sR,0} = \frac{m-n}{n}\sB^{\tt RFM}_{\sN,0}\,.
\end{aligned}
\]
Combining the above two relationship, we obtain the relationship between test risk $\sR_{0}^{\tt RFM}$ and norm $\sN_{0}^{\tt RFM}$ as

\[
\begin{aligned}
\sR_{0}^{\tt RFM} = \frac{m-n}{n} \sN_{0}^{\tt RFM} +\frac{2n-m}{m-n} \sigma^2.
\end{aligned}
\]
Accordingly, in the under-parameterized regime ($p<n$), we have
\[
\begin{aligned}
     \sB^{\tt RFM}_{\sN,0} =& \frac{p\frac{1}{(1+\lambda_p)^2}\|\btheta_*\|_2^2}{n - \frac{m}{(1+\lambda_p)^2}} + \frac{p\frac{1}{1+\lambda_p}\|\btheta_*\|_2^2}{n-p} = \frac{p}{m} \left( \frac{p^2}{nm - p^2} + \frac{p}{n-p} \right) \|\btheta_*\|_2^2\,,\\
    \sV^{\tt RFM}_{\sN,0} =& \frac{\sigma^2p}{\lambda_p(p-n)} = \frac{\sigma^2p^2}{(m-p)(n-p)}\,.
\end{aligned}
\]
\[
\begin{aligned}
     \sB^{\tt RFM}_{\sR,0} = \frac{n\lambda_p \frac{1}{1+\lambda_p}\|\btheta_*\|_2^2}{n-p} = \frac{n(m-p)}{m(n-p)} \|\btheta_*\|_2^2\,,\quad \sV^{\tt RFM}_{\sR,0} = \frac{\sigma^2p}{n-p}\,.
\end{aligned}
\]
Then we eliminate $p$ and obtain that, in the under-parameterized regime ($p<n$), the relationship between $\sV^{\tt RFM}_{\sR,0}$ and $\sV^{\tt RFM}_{\sN,0}$ is
\[
\begin{aligned}
\sV^{\tt RFM}_{\sR,0} = \frac{(m-n) \sV^{\tt RFM}_{\sN,0} + \sqrt{(m-n)^2(\sV^{\tt RFM}_{\sN,0})^2 + 4nm\sigma^2\sV^{\tt RFM}_{\sN,0}}}{2n}\,,
\end{aligned}
\]
which can be further simplified as a hyperbolic function
\begin{equation*}
     \left(\sV^{\tt RFM}_{\sR,0} \right)^2 = \frac{m-n}{n} \sV^{\tt RFM}_{\sR,0} \sV^{\tt RFM}_{\sN,0} + \frac{m \sigma^2}{n} \sV^{\tt RFM}_{\sN,0}\,,
\end{equation*}
and the asymptote of this hyperbola is $\sV^{\tt RFM}_{\sR,0} = \frac{m-n}{n}\sV^{\tt RFM}_{\sN,0} + \frac{m}{m-n} \sigma^2$.

Besides, we eliminate $p$ and obtain the relationship between $\sB^{\tt RFM}_{\sR,0}$ and $\sB^{\tt RFM}_{\sN,0}$ as
\[
\begin{aligned}
&\frac{\|\btheta_*\|_2^6 n^2 \left( 2 \|\btheta_*\|_2^2 + \sB^{\tt RFM}_{\sN,0} - \frac{\sB^{\tt RFM}_{\sN,0} n}{m} \right)}{m} 
=
(\sB^{\tt RFM}_{\sR,0})^4 n 
+ 
(\sB^{\tt RFM}_{\sR,0})^2 \|\btheta_*\|_2^2 n \left( \|\btheta_*\|_2^2 + \sB^{\tt RFM}_{\sN,0} - \frac{4 \|\btheta_*\|_2^2 n}{m} - \frac{\sB^{\tt RFM}_{\sN,0} n}{m} \right)\\
& + \sB^{\tt RFM}_{\sR,0} \|\btheta_*\|_2^4 n \left( \sB^{\tt RFM}_{\sN,0} + \frac{5 \|\btheta_*\|_2^2 n}{m} - \frac{\sB^{\tt RFM}_{\sN,0} n}{m} \right)
+ 
(\sB^{\tt RFM}_{\sR,0})^3 \left( -\sB^{\tt RFM}_{\sN,0} m - 2 \|\btheta_*\|_2^2 n + \sB^{\tt RFM}_{\sN,0} n + \frac{\|\btheta_*\|_2^2 n^2}{m} \right),
\end{aligned}
\]
which can be simplified to
\[
\begin{aligned}
    \sB^{\tt RFM}_{\sN,0}(m-n)(m\sB^{\tt RFM}_{\sR,0}-n\|\btheta_*\|_2^2)(m(\sB^{\tt RFM}_{\sR,0})^2-n\|\btheta_*\|_2^4) 
    = 
    nm(\sB^{\tt RFM}_{\sR,0}-\|\btheta_*\|_2^2)^2(m(\sB^{\tt RFM}_{\sR,0})^2-2n\|\btheta_*\|_2^4+n\|\btheta_*\|_2^2\sB^{\tt RFM}_{\sR,0})\,.
\end{aligned}
\]
We can find that in this case, the relationship can be easily written as
\[
\begin{aligned}
    \sB^{\tt RFM}_{\sN,0} =& \frac{nm(\sB^{\tt RFM}_{\sR,0}-\|\btheta_*\|_2^2)^2(m(\sB^{\tt RFM}_{\sR,0})^2-2n\|\btheta_*\|_2^4+n\|\btheta_*\|_2^2\sB^{\tt RFM}_{\sR,0})}{(m-n)(m\sB^{\tt RFM}_{\sR,0}-n\|\btheta_*\|_2^2)(m(\sB^{\tt RFM}_{\sR,0})^2-n\|\btheta_*\|_2^4)}\,.
\end{aligned}
\]
Next we will show that when $p \to n$, which also implies that $\sB^{\tt RFM}_{\sN,0} \to \infty$ and $\sB^{\tt RFM}_{\sR,0} \to \infty$, this relationship is approximately linear.

Recall that the relationship between $\sB^{\tt RFM}_{\sR,0}$ and $\sB^{\tt RFM}_{\sN,0}$ is given by \(\sB^{\tt RFM}_{\sR,0} = \frac{(m-n)}{n}\sB^{\tt RFM}_{\sN,0}\), and is equivalent to \(\sB^{\tt RFM}_{\sN,0} = \frac{n}{(m-n)}\sB^{\tt RFM}_{\sR,0} := f(\sB^{\tt RFM}_{\sR,0})\). We then do a difference and get
\[
\sB^{\tt RFM}_{\sN,0} - f(\sB^{\tt RFM}_{\sR,0}) = \frac{nm(\sB^{\tt RFM}_{\sR,0}-\|\btheta_*\|_2^2)^2(m(\sB^{\tt RFM}_{\sR,0})^2-2n\|\btheta_*\|_2^4+n\|\btheta_*\|_2^2\sB^{\tt RFM}_{\sR,0})}{(m-n)(m\sB^{\tt RFM}_{\sR,0}-n\|\btheta_*\|_2^2)(m(\sB^{\tt RFM}_{\sR,0})^2-n\|\btheta_*\|_2^4)} - \frac{n}{m-n}\sB^{\tt RFM}_{\sR,0}\,,
\]
then take \(\sB^{\tt RFM}_{\sR,0} \to \infty\) and we get
\[
\lim_{\sB^{\tt RFM}_{\sR,0} \to \infty}\sB^{\tt RFM}_{\sN,0} - f(\sB^{\tt RFM}_{\sR,0}) = -\frac{2n}{m}\|\btheta_*\|_2^2\,.
\]
Finally, organizing this equation and we get
\[
    \sB^{\tt RFM}_{\sR,0} \approx \frac{m-n}{n}\sB^{\tt RFM}_{\sN,0} + \frac{2(m-n)}{m}\|\btheta_*\|_2^2\,.
\]
\end{proof}

\subsubsection{Proof on features under power law assumption}

\begin{proof}[Proof of \cref{prop:relation_minnorm_powerlaw_rf}]
First, we use integral approximation to give approximations to some quantities commonly used in deterministic equivalence to prepare for the subsequent derivations.

According to the integral approximation in \citet[Lemma 1]{simonmore}, we have
\begin{equation}\label{eq:integrate_approx1}
    \Tr(\bLambda (\bLambda + \nu_2)^{-1}) \approx C_1 \nu_2^{-\frac{1}{\alpha}},\quad \Tr(\bLambda^2 (\bLambda + \nu_2)^{-2}) \approx C_2 \nu_2^{-\frac{1}{\alpha}},\quad \Tr(\bLambda (\bLambda + \nu_2)^{-2}) \approx (C_1-C_2) \nu_2^{-\frac{1}{\alpha}-1}\,,
\end{equation}
where $C_1$ and $C_2$ are
\begin{equation}\label{eq:C1C2}
    C_1 = \frac{\pi}{\alpha \sin\left(\nicefrac{\pi}{\alpha}\right)}\,,\quad C_2 = \frac{\pi(\alpha-1)}{\alpha^2 \sin\left(\nicefrac{\pi}{\alpha}\right)}\,,\quad \text{with $C_1 > C_2$}\,.
\end{equation}
Besides, according to definition of \(T(\nu)\) \cref{app:pre_scaling_law}, we have 
\[
\begin{aligned}
\< \btheta_*, (\bLambda + \nu_2)^{-1} \btheta_* \> =&~ T^1_{2r,1}(\nu_2) \approx C_3 \nu_2^{(2r-1)\wedge0},\\
\< \btheta_*, \bLambda(\bLambda + \nu_2)^{-2} \btheta_* \> =&~ T^1_{2r+1,2}(\nu_2) \approx C_4 \nu_2^{(2r-1)\wedge0}.\\
\end{aligned}
\]
When $r \in (0, \frac{1}{2})$, according to the integral approximation, we have
\begin{equation}
    C_3 = \frac{\pi}{\alpha \sin(2\pi r)}\,,\quad C_4 = \frac{2\pi r}{\alpha \sin(2\pi r)}\,,\quad \text{with $C_3 > C_4$}.
\end{equation}
Otherwise, if $r \in [\frac{1}{2}, \infty)$, we have
\[
\frac{1}{\alpha(2r-1)}< C_3 < \frac{1}{\alpha(2r-1)}+1\,,\quad \frac{1}{\alpha(2r-1)}< C_4 < \frac{1}{\alpha(2r-1)}+1\,,\quad \text{with $C_3 > C_4$}.
\]
For $\< \btheta_*, (\bLambda + \nu_2)^{-2} \btheta_* \>$, we have to discuss its approximation in the case $r \in (0, \frac{1}{2})$, $r \in [\frac{1}{2}, 1)$ and $r \in [\frac{1}{2}, \infty)$ separately.
\[
\langle \bm{\theta}_*, (\bm{\Lambda} + \nu_2)^{-2} \bm{\theta}_* \rangle \approx
\begin{cases} 
    (C_3 - C_4) \nu_2^{2r - 2}, & \text{if } r \in (0, \frac{1}{2})\,; \\
    C_5 \nu_2^{2r-2}, & \text{if } r \in [\frac{1}{2}, 1)\,; \\
    C_6, & \text{if } r \in [1, \infty)\,,
\end{cases}
\]
where $\frac{1}{2\alpha(r-1)} < C_6 < \frac{1}{2\alpha(r-1)} + 1$.


With the results of the integral approximation above, we next derive the relationship between $\sR_0^{\tt RFM}$ and $\sN_0^{\tt RFM}$ {\bf separately in over-parameterized regime ($p > n$) and under-parameterized regime ($p < n$).}

\paragraph{The relationship in over-parameterized regime ($p > n$)}

According to the self-consistent equation
\[
\begin{aligned}
1 + \frac{n}{p} - \sqrt{\left( 1 - \frac{n}{p} \right)^2 + \frac{4\lambda}{p\nu_2}} = \frac{2}{p} \operatorname{Tr} \left( \bLambda \left( \bLambda + \nu_2 \right)^{-1} \right),
\end{aligned}
\]
\[
\begin{aligned}
\nu_1 = \frac{\nu_2}{2} \left[ 1 - \frac{n}{p} + \sqrt{\left( 1 - \frac{n}{p} \right)^2 + \frac{4\lambda}{p\nu_2}} \right],
\end{aligned}
\]
In the over-parameterized regime (\(p > n\)), as \(\lambda \to 0\), for the first equation, \(\frac{4\lambda}{p\nu_2}\) will approach \(0\), and \(\Tr(\bLambda(\bLambda + \nu_2)^{-1})\) will converge to \(n\). Consequently, by \cref{eq:integrate_approx1}, \(\nu_2\) will converge to the constant \((\frac{n}{C_1})^{-\alpha}\). Furthermore, from the second equation, \(\nu_1\) will converge to \(\nu_2(1 - \frac{n}{p})\). Thus, according to \cref{eq:integrate_approx1}, we have
\[
\begin{aligned}
\Tr(\bLambda (\bLambda + \nu_2)^{-1}) \approx n,\quad \Tr(\bLambda^2 (\bLambda + \nu_2)^{-2}) \approx \frac{C_2}{C_1}n,\quad \Tr(\bLambda (\bLambda + \nu_2)^{-2}) \approx (C_1-C_2) (\frac{n}{C_1})^{\alpha+1}.
\end{aligned}
\]
Thus, in the over-parameterized regime
\[
\begin{aligned}
\Upsilon(\nu_1, \nu_2) =&~ \frac{p}{n} \left[ \left( 1 - \frac{\nu_1}{\nu_2} \right)^2 + \left( \frac{\nu_1}{\nu_2} \right)^2 \frac{\operatorname{Tr}\left(\bLambda^2 (\bLambda + \nu_2)^{-2}\right)}{p - \operatorname{Tr}\left(\bLambda^2 (\bLambda + \nu_2)^{-2}\right)} \right]\\
\approx&~ \frac{p}{n} \left[ \left( \frac{n}{p} \right)^2 + \left( 1 - \frac{n}{p} \right)^2 \frac{\operatorname{Tr}\left(\bLambda^2 (\bLambda + \nu_2)^{-2}\right)}{p - \operatorname{Tr}\left(\bLambda^2 (\bLambda + \nu_2)^{-2}\right)} \right]\\
\approx&~ \frac{\frac{C_2}{C_1}p -2 \frac{C_2}{C_1} n + n}{p - \frac{C_2}{C_1}n}\,,
\end{aligned}
\]
\[
\begin{aligned}
\chi(\nu_2) =~ \frac{\Tr(\bLambda (\bLambda + \nu_2)^{-2})}{p - \Tr(\bLambda^2 (\bLambda + \nu_2)^{-2})} \approx~ \frac{(C_1-C_2)(\frac{n}{C_1})^{\alpha+1}}{p-\frac{C_2}{C_1}n}\,.
\end{aligned}
\]
According to the approximation, we have the deterministic equivalents of variance terms
\[
\begin{aligned}
\sV_{\sR,0}^{\tt RFM} =&~ \sigma^2 \frac{\Upsilon(\nu_1, \nu_2)}{1 - \Upsilon(\nu_1, \nu_2)} \approx \sigma^2 \frac{(C_1-2C_2)n + C_2p}{(C_1-C_2)(p-n)}\,,\\
\sV_{\sN,0}^{\tt RFM} =&~ \sigma^2 \frac{p}{n} \frac{\chi(\nu_2)}{1 - \Upsilon(\nu_1, \nu_2)} \approx \sigma^2 \frac{(\frac{n}{C_1})^\alpha p}{p-n}\,.
\end{aligned}
\]
Then recall \cref{eq:C1C2}, we eliminate $p$ and obtain
\begin{equation}\label{eq:V_over_power_law}
    \sV_{\sR,0}^{\tt RFM} \approx \left(\frac{n}{C_1}\right)^{-\alpha}\sV_{\sN,0}^{\tt RFM} + \sigma^2\frac{2C_2-C_1}{C_1-C_2} = \left(\frac{n}{C_1}\right)^{-\alpha}\sV_{\sN,0}^{\tt RFM} + \sigma^2(\alpha - 2)\,.
\end{equation}

For the bias terms, due to the varying approximation behaviors of the quantities containing $\btheta_*$ for different values of $r$, we have to discuss their approximations in the conditions $r \in (0, \frac{1}{2})$, $r \in [\frac{1}{2}, 1)$ and $r \in [\frac{1}{2}, \infty)$ separately.

\paragraph{Condition 1: $r \in (0, \frac{1}{2})$}
\[
\begin{aligned}
\sB_{\sR,0}^{\tt RFM} =&~ \frac{\nu_2^2}{1 - \Upsilon(\nu_1, \nu_2)} \left[ \< \btheta_*, (\bLambda + \nu_2)^{-2} \btheta_* \> + \chi(\nu_2) \< \btheta_*, \bLambda (\bLambda + \nu_2)^{-2} \btheta_* \> \right]\\
\approx&~\frac{\left(\frac{n}{C_1}\right)^{-2\alpha r}\left((C_1C_4 -C_2C_3)n+C_1(C_3-C_4)p\right)}{(C1-C2)(p-n)},\\
\sB_{\sN,0}^{\tt RFM} =&~ \frac{\nu_2}{\nu_1} \< \btheta_*, (\bLambda + \nu_2)^{-1} \btheta_* \> - \frac{\lambda}{n} \frac{\nu_2^2}{\nu_1^2} \frac{\< \btheta_*, (\bLambda + \nu_2)^{-2} \btheta_* \> + \chi(\nu_2) \< \btheta_*, \bLambda (\bLambda + \nu_2)^{-2} \btheta_* \>}{1 - \Upsilon(\nu_1, \nu_2)}\\
\approx&~ \frac{\nu_2}{\nu_1} \< \btheta_*, (\bLambda + \nu_2)^{-1} \btheta_* \>\\
\approx&~ \frac{\left(\frac{n}{C_1}\right)^{-\alpha(2r-1)}C_3 p}{p-n}.
\end{aligned}
\]
Then we eliminate $p$ and obtain
\begin{equation}\label{eq:B_over_power_law_1}
    \sB_{\sR,0}^{\tt RFM} \approx \left(\frac{n}{C_1}\right)^{-\alpha} \sB_{\sN,0}^{\tt RFM} + \left(\frac{n}{C_1}\right)^{-2\alpha r}\frac{C_2C_3-C_1C_4}{C_1-C_2}\,.
\end{equation}

\paragraph{Condition 2: $r \in [\frac{1}{2}, 1)$}
\[
\begin{aligned}
\sB_{\sR,0}^{\tt RFM} =&~ \frac{\nu_2^2}{1 - \Upsilon(\nu_1, \nu_2)} \left[ \< \btheta_*, (\bLambda + \nu_2)^{-2} \btheta_* \> + \chi(\nu_2) \< \btheta_*, \bLambda (\bLambda + \nu_2)^{-2} \btheta_* \> \right]\\
\approx&~\frac{\left(\frac{n}{C_1}\right)^{-\alpha}\left(C_1\left(C_4 n + C_5 \left(\frac{n}{C_1}\right)^{-\alpha (2r-1)} p\right)-C_2 n \left(C_4 + C_5 \left(\frac{n}{C_1}\right)^{-\alpha (2r-1)}\right)\right)}{(C1-C2)(p-n)},\\
\sB_{\sN,0}^{\tt RFM} =&~ \< \btheta_*, \bLambda ( \bLambda + \nu_2)^{-2} \btheta_* \> \cdot \frac{p}{p - {\rm df}_2(\nu_2)}\\
&~+ \frac{p}{n} \nu_2^2 \left( \< \btheta_*, (\bLambda + \nu_2)^{-2} \btheta_* \> + \chi(\nu_2) \< \btheta_*, \bLambda (\bLambda + \nu_2)^{-2} \btheta_* \> \right) \cdot \frac{\chi(\nu_2)}{1 - \Upsilon(\nu_1, \nu_2)}\\
\approx&~ \frac{\left(C_4+C_5\left(\frac{n}{C_1}\right)^{-\alpha(2r-1)}\right)p}{p-n}\,.
\end{aligned}
\]
Then we eliminate $p$ and obtain
\[
\begin{aligned}
\sB_{\sR,0}^{\tt RFM} \approx&~ \left(\frac{n}{C_1}\right)^{-\alpha} \sB_{\sN,0}^{\tt RFM} + \left(\frac{n}{C_1}\right)^{-\alpha}\frac{-C_1C_4+C_2C_4+C_2C_5\left(\frac{n}{C_1}\right)^{-\alpha(2r-1)}}{C_1-C_2}\\
\approx&~ \left(\frac{n}{C_1}\right)^{-\alpha} \sB_{\sN,0}^{\tt RFM} - \left(\frac{n}{C_1}\right)^{-\alpha}C_4\,.
\end{aligned}
\]
The last ``$\approx$'' holds because $\left(\frac{n}{C_1}\right)^{-\alpha(2r-1)} = o(1)$.

\paragraph{Condition 3: $r \in [1, \infty)$}
\[
\begin{aligned}
\sB_{\sR,0}^{\tt RFM} =&~ \frac{\nu_2^2}{1 - \Upsilon(\nu_1, \nu_2)} \left[ \< \btheta_*, (\bLambda + \nu_2)^{-2} \btheta_* \> + \chi(\nu_2) \< \btheta_*, \bLambda (\bLambda + \nu_2)^{-2} \btheta_* \> \right]\\
\approx&~\frac{\left(\frac{n}{C_1}\right)^{-2\alpha}\left(C_1\left(C_4 n \left(\frac{n}{C_1}\right)^{\alpha} + C_6 p\right) - C_2 n \left(C_6 + C_4 \left(\frac{n}{C_1}\right)^{\alpha}\right)\right)}{(C1-C2)(p-n)},\\
\sB_{\sN,0}^{\tt RFM} =&~ \< \btheta_*, \bLambda ( \bLambda + \nu_2)^{-2} \btheta_* \> \cdot \frac{p}{p - {\rm df}_2(\nu_2)}\\
&~+ \frac{p}{n} \nu_2^2 \left( \< \btheta_*, (\bLambda + \nu_2)^{-2} \btheta_* \> + \chi(\nu_2) \< \btheta_*, \bLambda (\bLambda + \nu_2)^{-2} \btheta_* \> \right) \cdot \frac{\chi(\nu_2)}{1 - \Upsilon(\nu_1, \nu_2)}\\
\approx&~ \frac{\left(C_4+C_6\left(\frac{n}{C_1}\right)^{-\alpha}\right)p}{p-n}\,.
\end{aligned}
\]
Then we eliminate $p$ and obtain
\[
\begin{aligned}
\sB_{\sR,0}^{\tt RFM} \approx&~ \left(\frac{n}{C_1}\right)^{-\alpha} \sB_{\sN,0}^{\tt RFM} + \left(\frac{n}{C_1}\right)^{-\alpha}\frac{-C_1C_4+C_2C_4+C_2C_6\left(\frac{n}{C_1}\right)^{-\alpha}}{C_1-C_2}\\
\approx&~ \left(\frac{n}{C_1}\right)^{-\alpha} \sB_{\sN,0}^{\tt RFM} - \left(\frac{n}{C_1}\right)^{-\alpha}C_4\,.
\end{aligned}
\]
The last ``$\approx$'' holds because $\left(\frac{n}{C_1}\right)^{-\alpha(2r-1)} = o(1)$.

Combining the above condition \(r \in [\frac{1}{2}, 1)\) and \(r \in [1, \infty)\), we have for \(r \in [\frac{1}{2}, \infty)\)

\begin{equation}\label{eq:B_over_power_law_2}
\sB_{\sR,0}^{\tt RFM} \approx \left(\frac{n}{C_1}\right)^{-\alpha} \sB_{\sN,0}^{\tt RFM} - \left(\frac{n}{C_1}\right)^{-\alpha}C_4\,.    
\end{equation}

From \cref{eq:V_over_power_law,eq:B_over_power_law_1,eq:B_over_power_law_2}, we know that the relationship between \(\sR_0^{\tt RFM}\) and \(\sN_0^{\tt RFM}\) in the over-parameterized regime can be written as
\[
\sR_0^{\tt RFM} \approx \left(\nicefrac{n}{C_\alpha}\right)^{-\alpha} \sN_0^{\tt RFM} + C_{n,\alpha,r,1}\,.
\]


\paragraph{The relationship in under-parameterized regime ($p < n$)} While in the under-parameterized regime ($p < n$), When $\lambda \to 0$, $\Tr(\bLambda(\bLambda + \nu_2)^{-1})$ will converge to $p$, which means $\nu_2$ will converge to $(\frac{p}{C_1})^{-\alpha}$ and $\nu_1$ will converge to 0, with $\frac{\lambda}{\nu_1}\to n-p$. 

Accordingly, in the under-parameterized regime
\[
\begin{aligned}
\Upsilon(\nu_1, \nu_2) = \frac{p}{n} \left[ \left( 1 - \frac{\nu_1}{\nu_2} \right)^2 + \left( \frac{\nu_1}{\nu_2} \right)^2 \frac{\operatorname{Tr}\left(\bLambda^2 (\bLambda + \nu_2)^{-2}\right)}{p - \operatorname{Tr}\left(\bLambda^2 (\bLambda + \nu_2)^{-2}\right)} \right] \to \frac{p}{n}\,,
\end{aligned}
\]
\[
\begin{aligned}
\chi(\nu_2) = \frac{\operatorname{Tr}\left(\bLambda (\bLambda + \nu_2)^{-2}\right)}{p - \operatorname{Tr}\left(\bLambda^2 (\bLambda + \nu_2)^{-2}\right)} \to \frac{1}{\nu_2} \approx (\frac{p}{C_1})^{\alpha}\,.
\end{aligned}
\]
Then we can further obtain that, for the variance
\[
\begin{aligned}
\sV_{\sR,0}^{\tt RFM} =&~ \sigma^2 \frac{\Upsilon(\nu_1, \nu_2)}{1 - \Upsilon(\nu_1, \nu_2)} \approx \sigma^2 \frac{p}{n-p},\\
\sV_{\sN,0}^{\tt RFM} =&~ \sigma^2 \frac{p}{n} \frac{\chi(\nu_2)}{1 - \Upsilon(\nu_1, \nu_2)} \approx \sigma^2 C_1^{-\alpha} \frac{p^{\alpha+1}}{n-p}.
\end{aligned}
\]
For the relationship in the under-parameterized regime, we separately consider two cases, i.e. $p \ll n$ and $p\to n$.

First, we derive the relationship in the under-parameterized regime ($p < n$) as $p \to n$, based on the relationship in the over-parameterized regime.
Recall the relationship between $\sV_{\sR,0}^{\tt RFM}$ and $\sV_{\sN,0}^{\tt RFM}$ in the over-parameterized regime, as presented in \cref{eq:V_over_power_law}, given by 
\[
\sV_{\sR,0}^{\tt RFM} \approx \left(\frac{n}{C_1}\right)^{-\alpha} \sV_{\sN,0}^{\tt RFM} + \sigma^2 (\alpha-2) =:h(\sV_{\sN,0}^{\tt RFM})\,.
\]
Substituting the expression for $\sV_{\sN,0}^{\tt RFM}$ in the under-parameterized regime into this relationship, we obtain
\[
\sV_{\sR,0}^{\tt RFM} \approx \left(\frac{n}{C_1}\right)^{-\alpha} \sigma^2 C_1^{-\alpha} \frac{p^{\alpha+1}}{n-p} + \sigma^2 (\alpha-2)\,,
\]
then we compute $\sV_{\sR,0}^{\tt RFM} - h(\sV_{\sN,0}^{\tt RFM})$ and obtain
\[
\begin{aligned}
    \sV_{\sR,0}^{\tt RFM} - h(\sV_{\sN,0}^{\tt RFM}) =&~ \sigma^2 \frac{p}{n-p} - \left(\frac{n}{C_1}\right)^{-\alpha} \sigma^2 C_1^{-\alpha} \frac{p^{\alpha+1}}{n-p} - \sigma^2 (\alpha-2)\\
    =&~ \sigma^2\left(\frac{p-p^{\alpha+1}n^{-\alpha}}{n-p}\right) - \sigma^2 (\alpha-2)\,.
\end{aligned}
\]
Taking limits on the left and right sides of the equation, we get
\[
\lim_{p \to n} \left(\sV_{\sR,0}^{\tt RFM} - h(\sV_{\sN,0}^{\tt RFM})\right) = 2\sigma^2\,.
\]
Then when $p \to n$, we have
\begin{equation}\label{eq:V_under_power_law}
    \sV_{\sR,0}^{\tt RFM} \approx \left(\frac{n}{C_1}\right)^{-\alpha} \sV_{\sN,0}^{\tt RFM} + \sigma^2 \alpha\,.
\end{equation}
For $p \ll n$, we have $\frac{1}{n-p} \approx \frac{1}{n}$, then 
\[
\begin{aligned}
\sV_{\sR,0}^{\tt RFM} =&~ \sigma^2 \frac{\Upsilon(\nu_1, \nu_2)}{1 - \Upsilon(\nu_1, \nu_2)} \approx \sigma^2 \frac{p}{n},\\
\sV_{\sN,0}^{\tt RFM} =&~ \sigma^2 \frac{p}{n} \frac{\chi(\nu_2)}{1 - \Upsilon(\nu_1, \nu_2)} \approx \sigma^2 C_1^{-\alpha} \frac{p^{\alpha+1}}{n}.
\end{aligned}
\]
Eliminate $p$ and we have
\[
\sV_{\sR,0}^{\tt RFM} \approx \left(\sigma^2\right)^{\frac{\alpha}{\alpha+1}} C_1^{\frac{\alpha}{\alpha+1}} \left(\sV_{\sR,0}^{\tt RFM}\right)^\frac{1}{\alpha+1}\,.
\]
Next, for the bias term we have
\[
\begin{aligned}
\sB_{\sR,0}^{\tt RFM} =&~ \frac{\nu_2^2}{1 - \Upsilon(\nu_1, \nu_2)} \left[ \< \btheta_*, (\bLambda + \nu_2)^{-2} \btheta_* \> + \chi(\nu_2) \< \btheta_*, \bLambda (\bLambda + \nu_2)^{-2} \btheta_* \> \right]\\
\approx&~ \frac{\nu_2}{1 - \Upsilon(\nu_1, \nu_2)} \< \btheta_*, (\bLambda + \nu_2)^{-1} \btheta_* \>\\
\approx&~ \frac{n}{n-p} C_3 \nu_2^{2r\wedge1}.\\
\sB_{\sN,0}^{\tt RFM} =&~ p \< \btheta_*, \bLambda ( \bLambda + \nu_2)^{-2} \btheta_* \> \cdot \frac{1}{p - {\rm df}_2(\nu_2)}\\
&~+ \frac{p}{n} \chi(\nu_2) \frac{\nu_2^2}{1 - \Upsilon(\nu_1, \nu_2)} \left[ \< \btheta_*, (\bLambda + \nu_2)^{-2} \btheta_* \> + \chi(\nu_2) \< \btheta_*, \bLambda (\bLambda + \nu_2)^{-2} \btheta_* \> \right] \\
\approx&~ p \< \btheta_*, \bLambda ( \bLambda + \nu_2)^{-2} \btheta_* \> \cdot \frac{1}{p - {\rm df}_2(\nu_2)} + \frac{p}{n} \chi(\nu_2) \frac{\nu_2}{1 - \Upsilon(\nu_1, \nu_2)} \< \btheta_*, (\bLambda + \nu_2)^{-1} \btheta_* \>\\
\approx&~ \frac{p}{p-\frac{C_2}{C_1}p} C_4 \nu_2^{(2r-1)\wedge0} + \frac{p}{n-p} C_3 \nu_2^{(2r-1)\wedge0}\\
\approx&~ \left(\frac{C_1C_4}{C_1-C_2} + \frac{p}{n-p}C_3\right) \nu_2^{(2r-1)\wedge0}.
\end{aligned}
\]
Then we use the approximation $\nu_2 \approx (\frac{p}{C_1})^{-\alpha}$ and obtain
\[
\begin{aligned}
\sB_{\sR,0}^{\tt RFM} 
\approx \frac{n}{n-p} C_3 \nu_2^{2r\wedge1} \approx \frac{n}{n-p} C_3 \left( \frac{p}{C_1} \right)^{-\alpha\left(2r\wedge1\right)},
\end{aligned}
\]
\[
\begin{aligned}
\sB_{\sN,0}^{\tt RFM} \approx \left(\frac{C_1C_4}{C_1-C_2}+\frac{p}{n-p}C_3\right) \nu_2^{(2r-1)\wedge0} \approx \left(\frac{C_1C_4}{C_1-C_2}+\frac{p}{n-p}C_3\right) \left(\frac{p}{C_1}\right)^{-\alpha\left[(2r-1)\wedge0\right]}.
\end{aligned}
\]
Similarly to the bias term, we derive the relationship in the under-parameterized regime ($p < n$) as $p \to n$, based on the relationship in the over-parameterized regime. And we discuss the relationship when $r \in (0, \frac{1}{2})$ and $r \in [\frac{1}{2}, \infty)$ separately.

\paragraph{Condition 1: $r \in (0, \frac{1}{2})$.}
Recall the relationship between $\sB_{\sR,0}^{\tt RFM}$ and $\sB_{\sN,0}^{\tt RFM}$ in the over-parameterized regime, as presented in \cref{eq:B_over_power_law_1}, given by: 
\[
\begin{aligned}
\sB_{\sR,0}^{\tt RFM} = \left(\frac{n}{C_1}\right)^{-\alpha} \sB_{\sN,0}^{\tt RFM} + \left(\frac{n}{C_1}\right)^{-2\alpha r}\frac{C_2C_3-C_1C_4}{C_1-C_2} =: f(\sB_{\sN,0}^{\tt RFM}).
\end{aligned}
\]
Substituting the expression for $\sB_{\sN,0}^{\tt RFM}$ in the under-parameterized regime into this relationship, we obtain:
\[
\begin{aligned}
f(\sB_{\sN,0}^{\tt RFM}) =&~ \left(\frac{n}{C_1}\right)^{-\alpha} \left(\frac{C_1C_4}{C_1-C_2}+\frac{p}{n-p}C_3\right) \left(\frac{p}{C_1}\right)^{-\alpha(2r-1)} + \left(\frac{n}{C_1}\right)^{-2\alpha r}\frac{C_2C_3-C_1C_4}{C_1-C_2},
\end{aligned}
\]
then we compute $\sB_{\sR,0}^{\tt RFM} - f(\sB_{\sN,0}^{\tt RFM})$ and obtain
\[
\begin{aligned}
\sB_{\sR,0}^{\tt RFM} - f(\sB_{\sN,0}^{\tt RFM}) = C_1^{2\alpha r}\Big(\frac{n}{n-p}C_3p^{-2\alpha r} - \frac{C_1C_4}{C_1-C_2}p^{-\alpha(2r-1)}n^{-\alpha} -\frac{p}{n-p}C_3p^{-\alpha(2r-1)}n^{-\alpha} - \frac{C_2C_3-C_1C_4}{C_1-C_2}n^{-2\alpha r}\Big).\\
\end{aligned}
\]
To simplify this equation, we begin by computing $\frac{n}{n-p}C_3p^{-2\alpha r} - \frac{p}{n-p}C_3p^{-\alpha(2r-1)}n^{-\alpha}$ and obtain
\[
\begin{aligned}
\frac{n}{n-p}C_3p^{-2\alpha r} - \frac{p}{n-p}C_3p^{-\alpha(2r-1)}n^{-\alpha} =&~ C_3p^{-\alpha(2r-1)} \left(\frac{n}{n-p}p^{-\alpha} - \frac{p}{n-p}n^{-\alpha} \right)\\
=&~ C_3p^{-\alpha(2r-1)}\frac{np^{-\alpha} - pn^{-\alpha}}{n-p},
\end{aligned}
\]
where $\frac{np^{-\alpha} - pn^{-\alpha}}{n-p}$ is monotonically decreasing in $p$ (monotonicity can be obtained by simple derivatives), and by applying L'Hôpital's rule, we have:
\[
\lim_{p \to n} \frac{np^{-\alpha} - pn^{-\alpha}}{n-p} = \lim_{p \to n} \frac{-\alpha n p^{-\alpha-1}-n^{-\alpha}}{-1} = (\alpha+1)n^{-\alpha}.
\]
Thus we have
\[
\begin{aligned}
\lim_{p \to n} C_3p^{-\alpha(2r-1)}\frac{np^{-\alpha} - pn^{-\alpha}}{n-p} = (\alpha+1)C_3n^{-2\alpha r}.
\end{aligned}
\]
Thus we have
\[
\begin{aligned}
&~\lim_{p \to n} C_1^{2\alpha r}\Big(C_3p^{-\alpha(2r-1)}\frac{np^{-\alpha} - pn^{-\alpha}}{n-p} - \frac{C_1C_4}{C_1-C_2}p^{-\alpha(2r-1)}n^{-\alpha} - \frac{C_2C_3-C_1C_4}{C_1-C_2}n^{-2\alpha r}\Big)\\
=&~ C_1^{2\alpha r}\Big((\alpha+1)C_3n^{-2\alpha r} - \frac{C_1C_4}{C_1-C_2}n^{-2\alpha r} - \frac{C_2C_3-C_1C_4}{C_1-C_2}n^{-2\alpha r}\Big)\\
=&~ C_1^{2\alpha r}C_3n^{-2\alpha r}\Big((\alpha+1) - \frac{C_2}{C_1-C_2}\Big).
\end{aligned}
\]
Recall that from \cref{eq:C1C2} we have
\[
\begin{aligned}
C_1 = \frac{\pi}{\alpha \sin\left(\nicefrac{\pi}{\alpha}\right)}\,, \quad C_2 = \frac{\pi(\alpha-1)}{\alpha^2 \sin\left(\nicefrac{\pi}{\alpha}\right)},
\end{aligned}
\]
thus 
\[
\begin{aligned}
(\alpha+1) - \frac{C_2}{C_1-C_2} = (\alpha+1) - \frac{ \frac{\pi(\alpha-1)}{\alpha^2 \sin\left(\nicefrac{\pi}{\alpha}\right)}}{\frac{\pi}{\alpha \sin\left(\nicefrac{\pi}{\alpha}\right)} - \frac{\pi(\alpha-1)}{\alpha^2 \sin\left(\nicefrac{\pi}{\alpha}\right)}} = 2.
\end{aligned}
\]
Finally, we have
\[
\begin{aligned}
\lim_{p \to n}\left( \sB_{\sR,0}^{\tt RFM} - f(\sB_{\sN,0}^{\tt RFM}) \right) = 2C_1^{2\alpha r}C_3n^{-2\alpha r} = 2C_3\left(\frac{n}{C_1}\right)^{-2\alpha r},
\end{aligned}
\]
and then the relationship between $\sB_{\sR,0}^{\tt RFM}$ and $\sB_{\sN,0}^{\tt RFM}$ is 
\begin{equation}\label{eq:B_under_power_law_1}
    \begin{split}
        \sB_{\sR,0}^{\tt RFM} \approx&~ \left(\frac{n}{C_1}\right)^{-\alpha} \sB_{\sN,0}^{\tt RFM} + \left(\frac{n}{C_1}\right)^{-2\alpha r}\frac{C_2C_3-C_1C_4}{C_1-C_2} + 2C_3\left(\frac{n}{C_1}\right)^{-2\alpha r}\\
        \approx&~ \left(\frac{n}{C_1}\right)^{-\alpha} \sB_{\sN,0}^{\tt RFM} + \left(\frac{n}{C_1}\right)^{-2\alpha r}\frac{2C_1C_3-C_2C_3-C_1C_4}{C_1-C_2}.
    \end{split}
\end{equation}


\paragraph{Condition 2: $r \in [\frac{1}{2}, \infty)$.}

In this condition, the approximation of $\sB_{\sR,0}^{\tt RFM}$ and $\sB_{\sN,0}^{\tt RFM}$ can be simplified to 
\[
\begin{aligned}
\sB_{\sR,0}^{\tt RFM} 
\approx \frac{n}{n-p} C_3 \nu_2^{2r\wedge1} \approx \frac{n}{n-p} C_3 \left( \frac{p}{C_1} \right)^{-\alpha\left(2r\wedge1\right)} = \frac{n}{n-p} C_3 \left( \frac{p}{C_1} \right)^{-\alpha}\,,
\end{aligned}
\]
\[
\begin{aligned}
\sB_{\sN,0}^{\tt RFM} \approx \left(\frac{C_1C_4}{C_1-C_2}+\frac{p}{n-p}C_3\right) \nu_2^{(2r-1)\wedge0} \approx \left(\frac{C_1C_4}{C_1-C_2}+\frac{p}{n-p}C_3\right) \left(\frac{p}{C_1}\right)^{-\alpha\left[(2r-1)\wedge0\right]} = \frac{C_1C_4}{C_1-C_2}+\frac{p}{n-p}C_3\,.
\end{aligned}
\]
Recall the relationship between $\sB_{\sR,0}^{\tt RFM}$ and $\sB_{\sN,0}^{\tt RFM}$ in the over-parameterized regime is presented in \cref{eq:B_over_power_law_2}, given by: 
\[
\begin{aligned}
\sB_{\sR,0}^{\tt RFM} \approx&~ \left(\frac{n}{C_1}\right)^{-\alpha} \sB_{\sN,0}^{\tt RFM} - \left(\frac{n}{C_1}\right)^{-\alpha}C_4 =: g(\sB_{\sN,0}^{\tt RFM})\,.
\end{aligned}
\]
Substituting the expression for $\sB_{\sN,0}^{\tt RFM}$ in the under-parameterized regime into this relationship, we obtain:
\[
\begin{aligned}
g(\sB_{\sN,0}^{\tt RFM}) =&~ \left(\frac{n}{C_1}\right)^{-\alpha} \left(\frac{C_1C_4}{C_1-C_2}+\frac{p}{n-p}C_3\right) - \left(\frac{n}{C_1}\right)^{-\alpha}C_4\,,
\end{aligned}
\]
then we compute $\sB_{\sR,0}^{\tt RFM} - g(\sB_{\sN,0}^{\tt RFM})$ and obtain
\[
\begin{aligned}
    \sB_{\sR,0}^{\tt RFM} - g(\sB_{\sN,0}^{\tt RFM}) = C_3 C_1^{\alpha} \frac{np^{-\alpha} - pn^{-\alpha}}{n-p} - \left(\frac{n}{C_1}\right)^{-\alpha} \left( \frac{C_2C_4}{C_1-C_2} \right)\,.
\end{aligned}
\]
Thus we have
\[
\begin{aligned}
    \lim_{p \to n}\left( \sB_{\sR,0}^{\tt RFM} - f(\sB_{\sN,0}^{\tt RFM}) \right) =&~ \left(\frac{n}{C_1}\right)^{-\alpha} \left((\alpha+1)C_3 - \frac{C_2C_4}{C_1-C_2}\right)\\
    \approx&~ \left(\frac{n}{C_1}\right)^{-\alpha} \left((\alpha+1)C_4 - \frac{C_2}{C_1-C_2}C_4\right)\\
    =&~ \left(\frac{n}{C_1}\right)^{-\alpha} 2 C_4\,,
\end{aligned}
\]
and the relationship between $\sB_{\sR,0}^{\tt RFM}$ and $\sB_{\sN,0}^{\tt RFM}$ is 
\begin{equation}\label{eq:B_under_power_law_2}
    \begin{split}
        \sB_{\sR,0}^{\tt RFM} \approx&~ \left(\frac{n}{C_1}\right)^{-\alpha} \sB_{\sN,0}^{\tt RFM} - \left(\frac{n}{C_1}\right)^{-\alpha}C_4 + \left(\frac{n}{C_1}\right)^{-\alpha}2C_4\\
        \approx&~ \left(\frac{n}{C_1}\right)^{-\alpha} \sB_{\sN,0}^{\tt RFM} + \left(\frac{n}{C_1}\right)^{-\alpha}C_4\,.
    \end{split}
\end{equation}


When $p \ll n$, we discuss cases $r \in (0, \frac{1}{2})$ and $r \in (\frac{1}{2}, \infty)$ separately. 

If $r \in (0, \frac{1}{2})$, we have $\frac{n}{n-p} \approx 1$ and $\frac{p}{n-p} \approx 0$, then
\[
\begin{aligned}
\sB_{\sR,0}^{\tt RFM} 
\approx C_3 \nu_2^{2r\wedge1} \approx C_3 \left( \frac{p}{C_1} \right)^{-\alpha2r},
\end{aligned}
\]
\[
\begin{aligned}
\sB_{\sN,0}^{\tt RFM} \approx \frac{C_1C_4}{C_1-C_2} \nu_2^{(2r-1)\wedge0} \approx \frac{C_1C_4}{C_1-C_2} \left(\frac{p}{C_1}\right)^{-\alpha\left(2r-1\right)}.
\end{aligned}
\]
Then we eliminate $p$ and obtain
\[
\begin{aligned}
\sB_{\sR,0}^{\tt RFM} \approx C_3 \left(\frac{C_1-C_2}{C_1C_4}\right)^{\nicefrac{2r}{(2r-1)}} \left(\sB_{\sN,0}^{\tt RFM}\right)^{\nicefrac{2r}{(2r-1)}}.
\end{aligned}
\]

If $2r \ge 1$, we have
\[
\begin{aligned}
\sB_{\sR,0}^{\tt RFM} \approx&~ \frac{n}{n-p} C_3 \nu_2 \approx \frac{n}{n-p} C_3\left(\frac{p}{C_1}\right)^{-\alpha},
\end{aligned}
\]
\[
\begin{aligned}
\sB_{\sN,0}^{\tt RFM} \approx&~ \frac{C_1C_4}{C_1-C_2} + \frac{p}{n-p}C_3 .
\end{aligned}
\]
Then we eliminate $p$ and obtain
\[
\begin{aligned}
\sB_{\sR,0}^{\tt RFM} \approx \left(\frac{C_1C_3-C_2C_3-C_1C_4}{C_1-C_2}+\sB_{\sN,0}^{\tt RFM}\right)\left(\frac{n\left(\sB_{\sN,0}^{\tt RFM}-\frac{C_1C_4}{C_1-C_2}\right)}{C_1\left(C_3+\sB_{\sN,0}^{\tt RFM}-\frac{C_1C_4}{C_1-C_2}\right)}\right)^{-\alpha}.
\end{aligned}
\]

From \cref{eq:V_under_power_law,eq:B_under_power_law_1,eq:B_under_power_law_2}, we know that the relationship between \(\sR_0^{\tt RFM}\) and \(\sN_0^{\tt RFM}\) in the under-parameterized regime when \(p \to n\) can be written as
\[
\sR_0^{\tt RFM} \approx \left(\nicefrac{n}{C_\alpha}\right)^{-\alpha} \sN_0^{\tt RFM} + C_{n,\alpha,r,2}\,.
\]

\end{proof}



    
