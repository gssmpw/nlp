\section{Main results on random feature regression}
\label{sec:rff}

In this section, we mathematically characterize \cref{fig:random_feature_risk_vs_norm} under norm-based capacity.
Similar to ridge regression, we have the following bias-variance decomposition for the norm.


\begin{lemma}[Bias-variance decomposition of $\mathcal{N}_{\lambda}^{\tt RFM}$]
\label{lemma:biasvariance_rf}
We have the bias-variance decomposition $\E_{\varepsilon}\|\hat{\ba}\|_2^2 =: \mathcal{N}_{\lambda}^{\tt RFM} = \mathcal{B}_{\mathcal{N},\lambda}^{\tt RFM} + \mathcal{V}_{\mathcal{N},\lambda}^{\tt RFM}$, where $\mathcal{B}_{\mathcal{N},\lambda}^{\tt RFM}$ and $\mathcal{V}_{\mathcal{N},\lambda}^{\tt RFM}$ are defined as 
\[
\begin{aligned}
    \mathcal{B}_{\mathcal{N},\lambda}^{\tt RFM} := \<\btheta_*, \bm{G}^\sT \bm{Z} (\bm{Z}^\sT \bm{Z} + \lambda\id)^{-2} \bm{Z}^\sT \bm{G}\btheta_* \>\,, \quad \mathcal{V}_{\mathcal{N},\lambda}^{\tt RFM} := \sigma^2\Tr\left(\bm{Z}^\sT \bm{Z}(\bm{Z}^\sT \bm{Z} + \lambda\id)^{-2}\right)\,.
\end{aligned}
\]
\end{lemma}

A main goal of this section is to prove that $\mathcal{B}^{\tt RFM}_{\mathcal{N},\lambda}$ and $\mathcal{V}^{\tt RFM}_{\mathcal{N},\lambda}$ admit the following deterministic equivalents, both  asymptotically (\cref{sec:linear_asym_rf}) and non-asymptotically (\cref{sec:linear_nonasym_rf}):
\begin{align}
    \sB_{\sN,\lambda}^{\tt RFM} :=&~ \frac{p\< \btheta_*, \bLambda ( \bLambda + \nu_2\id)^{-2} \btheta_* \>}{p - \Tr\left(\bLambda^2 (\bLambda + \nu_2\id)^{-2}\right)} + {\color{red}\frac{p\chi(\nu_2)}{n}} \notag \cdot \frac{\nu_2^2\left[ \< \btheta_*, (\bLambda + \nu_2\id)^{-2} \btheta_* \> + \chi(\nu_2) \< \btheta_*, \bLambda (\bLambda + \nu_2\id)^{-2} \btheta_* \> \right]}{1 - \Upsilon(\nu_1, \nu_2)}, \notag \\
    \sV_{\sN,\lambda}^{\tt RFM} :=&~ \sigma^2 \frac{{\color{blue}\frac{p}{n}\chi(\nu_2)}}{1-\Upsilon(\nu_1, \nu_2)}\,. \label{eq:equiv_random_feature}
\end{align}
When checking \cref{eq:de_risk_rf} and \cref{eq:equiv_random_feature}, we find that for variance, $\sB_{\sR,\lambda}^{\tt RFM}$ and $\sB_{\sN,\lambda}^{\tt RFM}$ only differ on the numerator, where $\Upsilon(\nu_1,\nu_2)$ is changed by $\frac{p}{n}\chi(\nu_2)$ ({\color{blue}in blue}).  For the bias term, we find that the second term of $\sB_{\sN,\lambda}^{\tt RFM}$ ({\color{red}in red}) rescales $\sB_{\sR,\lambda}^{\tt RFM}$ in \cref{eq:de_risk_rf} by a factor $\frac{p\chi(\nu_2)}{n}$.

Both of our asymptotic and non-asymptotic results are based on the following assumption on well-behaved data and features, but non-asymptotic results requires more technical assumptions we will deliver later.

\begin{assumption}[Concentration of the eigenfunctions \cite{defilippis2024dimension}]\label{ass:concentrated_RFRR} Recall the random vectors $\bpsi := (\xi_k \psi_k(\bx))_{k \geq 1}$ and $\bphi := (\xi_k \phi_k(\bw))_{k \geq 1}$. There exists $C_* > 0$ such that for any PSD matrix $\bA \in \mathbb{R}^{\infty \times \infty}$ with $\operatorname{Tr}(\bLambda \bA) < \infty$ and any $t\ge0$, we have
\[
\begin{aligned}
 & \mathbb{P} \left( \left| \bpsi^\sT \bA \bpsi - \Tr(\bLambda \bA) \right| \geq t \|\bLambda^{1/2} \bA \bLambda^{1/2}\|_F \right) 
\leq C_*  e^{-\frac{t}{C_*}}, \\
& \mathbb{P} \left( \left| \bphi^\sT \bA \bphi - \Tr(\bLambda \bA) \right| \geq t \|\bLambda^{1/2} \bA \bLambda^{1/2}\|_F \right) 
\leq C_*  e^{-\frac{t}{C_*}}.  
\end{aligned}
\]
\end{assumption}

This assumption requires well-behaved data---similarly to \cref{ass:concentrated_LR}---and additionally well-behaved random features. It holds for the classical sub-Gaussian case and log-Sobolev inequality or convex Lipschitz concentration \citep{cheng2022dimension}.

\vspace{-0.cm}
\subsection{Asymptotic deterministic equivalence}
\label{sec:linear_asym_rf}
\vspace{-0.cm}


Here we present the asymptotic results of $\E_{\varepsilon}\|\hat{\ba}\|_2^2$, see the proof in \cref{app:asy_deter_equiv_rf}.
\begin{proposition}[Asymptotic deterministic equivalence]\label{prop:asy_equiv_norm_RFRR}
    Given the bias-variance decomposition of $\E_{\varepsilon}\|\hat{\ba}\|_2^2$ in \cref{lemma:biasvariance_rf}, 
    under \cref{ass:concentrated_RFRR}, we have the following asymptotic deterministic equivalents $\mathcal{N}^{\tt RFM}_\lambda \sim \sN^{\tt RFM}_\lambda := \sB_{\sN,\lambda}^{\tt RFM} + \sV_{\sN,\lambda}^{\tt RFM}$ such that $\mathcal{B}^{\tt RFM}_{\mathcal{N},\lambda} \sim \sB_{\sN,\lambda}^{\tt RFM}$, $\mathcal{V}^{\tt RFM}_{\mathcal{N},\lambda} \sim \sV_{\sN,\lambda}^{\tt RFM}$, where $\sB_{\sN,\lambda}^{\tt RFM}$ and $\sV_{\sN,\lambda}^{\tt RFM}$ are defined by \cref{eq:equiv_random_feature}.
\end{proposition}
\cref{prop:asy_equiv_norm_RFRR} is numerically validated by \cref{fig:random_feature_risk_vs_norm}, supporting that $\E_{\varepsilon}\|\hat{\ba}\|_2^2$ is able to characterize the bias and variance of the excess risk.
We will quantitatively characterize this relationship in \cref{sec:relationship_rf}.

Similar to linear regression, we also need to analyze the under-/over-parameterized regimes separately for RFMs when $\lambda \rightarrow 0$.
In the under-parameterized regime, $\nu_1$ converges to $0$, and $\nu_2$ converges to a value $\lambda_p$ satisfying $\Tr(\bLambda(\bLambda + \lambda_p\id)^{-1}) = p$; while in the over-parameterized regime, $\nu_2$ converges to $\lambda_n$ satisfying $\Tr(\bLambda(\bLambda + \lambda_n\id)^{-1}) = n$, and $\nu_1$ converges to $\nu_2(1-\nicefrac{n}{p})$. 
We have the following result, see the proof in \cref{app:asy_deter_equiv_rf}. 


\begin{corollary}[Asymptotic deterministic equivalence of $\sN_{0}^{\tt RFM}$]\label{prop:asy_equiv_norm_RFRR_minnorm}
    Under \cref{ass:concentrated_RFRR}, for the min-$\ell_2$-norm estimator $\hat{\ba}_{\min}$, in the under-parameterized regime ($p<n$), we have
    \[
    \begin{aligned}
        \mathcal{B}^{\tt RFM}_{\mathcal{N},0} \sim \frac{p\<\btheta_*, \bLambda (\bLambda +\lambda_p\id)^{-2} \btheta_*\>}{n-\Tr(\bLambda^2(\bLambda +\lambda_p\id)^{-2})} + \frac{p\<\btheta_*, (\bLambda +\lambda_p\id)^{-1} \btheta_*\>}{n-p}\,, \quad 
        \mathcal{V}^{\tt RFM}_{\mathcal{N},0} \sim \frac{\sigma^2p}{\lambda_p(n-p)},
    \end{aligned}
    \]
    where $\lambda_p$ is from $\Tr(\bLambda(\bLambda+\lambda_p\id)^{-1}) \sim p$. In the over-parameterized regime ($p>n$), we have
    \[
    \begin{aligned}
        \mathcal{B}^{\tt RFM}_{\mathcal{N},0} \sim \frac{p\<\btheta_*, ( \bLambda + \lambda_n\id)^{-1} \btheta_*\>}{p-n}\,,
        \quad
        \mathcal{V}^{\tt RFM}_{\mathcal{N},0} \sim \frac{\sigma^2p}{\lambda_n(p-n)}\,,
    \end{aligned}
    \]
    where $\lambda_n$ is defined by $\Tr(\bLambda(\bLambda+\lambda_n\id)^{-1}) \sim n$.
\end{corollary}
\noindent{\bf Remark:} Notice that $\mathcal{V}^{\tt RFM}_{\mathcal{N},0}$ admits the similar formulation in under-/over-parameterized regimes but differs in $\lambda_n$ and $\lambda_p$. An interesting point to note is that, in the over-parameterized regime, $\lambda_n$ is a constant when $n$ constant. Therefore, $\mathcal{B}^{\tt RFM}_{\mathcal{N},0}$ and $\mathcal{V}^{\tt RFM}_{\mathcal{N},0}$ are proportional to each other.

\vspace{-0.cm}
\subsection{Non-asymptotic deterministic equivalence}
\label{sec:linear_nonasym_rf}
\vspace{-0.cm}

To present our non-asymptotic results, we additionally consider the following classical power-law assumption.

\begin{assumption}[Power-law, \citealt{defilippis2024dimension}]
\label{ass:powerlaw_rf}
    We assume that $\{ \xi_k^2\}_{k=1}^{\infty}$ in $\bLambda$ and $\btheta_*$ satisfy
    \[
    \xi_k^2 = k^{-\alpha}, \quad \theta_{\ast, k} = k^{-\frac{1 + 2\alpha\tau}{2}}\,, \mbox{with}~\alpha > 1,~ r>0\,.
    \]
\end{assumption}

The assumption coincides with the source condition $\|\bLambda^{-r} \btheta_*\|_2 < \infty$ ($r>0$) and capacity condition $\Tr(\bLambda^{1/\alpha}) < \infty$ ($\alpha > 1$) \citep{caponnetto2007optimal}.

We have the following non-asymptotic result on variance. 
\begin{theorem}[Non-asymptotic deterministic equivalents for variance, simplified version of \cref{prop:det_equiv_RFRR_V}]\label{prop:non-asy_equiv_norm_RFRR_V}
    Under \cref{ass:concentrated_RFRR} and \ref{ass:powerlaw_rf}, for any $D,K >0$, if $\lambda > n^{-K}$, then with probability at least $1-n^{-D}-p^{-D}$, we have
    \[
    \begin{aligned}
          \left|\mathcal{V}^{\tt RFM}_{\mathcal{N},\lambda} - \sV_{\sN,\lambda}^{\tt RFM}\right| \leq \widetilde{\mathcal{O}}(n^{-\nicefrac{1}{2}}+p^{-\nicefrac{1}{2}}) \cdot \sV_{\sN,\lambda}^{\tt RFM}\,.
    \end{aligned}
    \]
\end{theorem}
\noindent{\bf Remark:} Our results remain valid under weaker assumptions related to \emph{effective dimension} used in \citet{defilippis2024dimension}; see  \cref{app:nonasy_deter_equiv_rf}. 
We expect similar results to hold for bias as well, but additional technical assumptions and calculations may be needed; see more discussion in \cref{app:discuss_bias}.
We leave this to future work.  

\subsection{Relationship and scaling law}\label{sec:relationship_rf}

Here we discuss the risk-norm relationship for min-$\ell_2$-norm interpolator, and then build the scaling law under certain settings; see the proof in \cref{app:relationship_rf}.
\begin{proposition}[Relationship for min-$\ell_2$-norm interpolator in the {\bf over-parameterized} regime]\label{prop:relation_minnorm_overparam}
The deterministic equivalents $\sR^{\tt RFM}_{0}$ and $\sN^{\tt RFM}_{0}$, in over-parameterized regimes ($p>n$) admit the linear relationship due to $\lambda_n$ as a constant
% \begin{equation}\label{eq:rfflam0}
%   \sR_{0}^{\tt RFM} = \lambda_n\sN_{0}^{\tt RFM} + C_{n,\bLambda,\btheta_*,\sigma}\,,  
% \end{equation}
\begin{equation}\label{eq:rfflam0}
    \sR_{0}^{\tt RFM} 
    = 
    \lambda_n\sN_{0}^{\tt RFM} 
    - 
    \left[\lambda_n\<\btheta_*, ( \bLambda + \lambda_n\id)^{-1} \btheta_*\> + \sigma^2\right] 
    + 
    \frac{n\lambda_n^2 \<\btheta_*, ( \bLambda + \lambda_n \id)^{-2} \btheta_*\> + \sigma^2\Tr(\bLambda^2(\bLambda+\lambda_n\id)^{-2})}{ n - \Tr(\bLambda^2(\bLambda+\lambda_n\id)^{-2})}\,. 
\end{equation}
%where $C_{n,\bLambda,\btheta_*,\sigma}$ is a constant independent of $p$ but depending on $n$, $\bLambda$, $\btheta_*$ and $\sigma$, given in \cref{app:relationship_rf}.
\end{proposition}

The relationship in the under-parameterized regime is more complicated. We present it in the special case of isotropic features in \cref{prop:relation_minnorm_id_rf} and give an approximation in \cref{prop:relation_minnorm_powerlaw_rf} under the power-law assumption.

\begin{corollary}[Isotropic features for min-$\ell_2$-norm interpolator]\label{prop:relation_minnorm_id_rf}
    Consider covariance matrix $\bLambda = \id_m$ ($n<m<\infty$), in the over-parameterized regime ($p>n$), the deterministic equivalents $\sR^{\tt RFM}_0$ and $\sN^{\tt RFM}_0$ specifies the linear relationship in \cref{eq:rfflam0} as $\sR_{0}^{\tt RFM} = \frac{m-n}{n} \sN_{0}^{\tt RFM} +\frac{2n-m}{m-n} \sigma^2$.\\
While in the under-parameterized regime ($p<n$), we focus on bias and variance separately
    \[
    \begin{aligned}
     \mbox{Variance:}~ \left(\sV^{\tt RFM}_{\sR,0} \right)^2 = \frac{m-n}{n} \sV^{\tt RFM}_{\sR,0} \sV^{\tt RFM}_{\sN,0} + \frac{m \sigma^2}{n} \sV^{\tt RFM}_{\sN,0}\,,
    \end{aligned}
    \]
    \[
    \begin{aligned}
      \mbox{Bias:}~  &~(m-n)\sB^{\tt RFM}_{\sN,0}(m\sB^{\tt RFM}_{\sR,0}-n\|\btheta_*\|_2^2)(m(\sB^{\tt RFM}_{\sR,0})^2-n\|\btheta_*\|_2^4)\\
        &= nm(\sB^{\tt RFM}_{\sR,0} -\|\btheta_*\|_2^2)^2[m(\sB^{\tt RFM}_{\sR,0})^2 + n\|\btheta_*\|_2^2\sB^{\tt RFM}_{\sR,0} - 2n\|\btheta_*\|_2^4].
    \end{aligned}
    \]
\end{corollary}

\noindent{\bf Remark:} 
In the under-parameterized regime, $\sV^{\tt RFM}_{\sR,0}$ and $\sV^{\tt RFM}_{\sN,0}$ are related by a hyperbola, the asymptote of which is $\sV^{\tt RFM}_{\sR,0} = \frac{m-n}{n}\sV^{\tt RFM}_{\sN,0} + \frac{m}{m-n} \sigma^2$. Further, for $p \to n$, we have $\sB^{\tt RFM}_{\sR,0} \approx \frac{m-n}{n}\sB^{\tt RFM}_{\sR,0} + \frac{2(m-n)}{m}\|\btheta_*\|_2^2$, see discussion in \cref{app:relationship_rf}.


\begin{figure}[t]
    \centering
    \subfigure[$\alpha = 2.5$, $r=0.2$]{
        \includegraphics[width=0.3\textwidth]{arxiv_version/figures/Main_results_on_random_feature_regression/risk_vs_norm_ridgeless_alpha2.5_r0.2_n500.pdf}
    }
    \subfigure[$\alpha = 1.5$, $r=0.8$]{
        \includegraphics[width=0.3\textwidth]{arxiv_version/figures/Main_results_on_random_feature_regression/risk_vs_norm_ridgeless_alpha1.5_r0.8_n500.pdf}
    }
    \caption{Validation of \cref{prop:relation_minnorm_powerlaw_rf}. 
    The solid line represents the result of the deterministic equivalents, well approximated by the {\color{red}red dashed line} of \cref{eq:RORFMover} in the over-parameterized regime, and the {\color{blue}blue dashed line} of \cref{eq:RORFMunder} when $p \to n$ in the under-parameterized regime.}
    \label{fig:random_feature_risk_vs_norm_approx}\vspace{-0.05cm}
\end{figure}



Under power-law, we need to handle the self-consistent equations to approximate the infinite summation. We have the following approximation.
\begin{corollary}[Relationship for min-$\ell_2$ norm interpolator under power law assumption]\label{prop:relation_minnorm_powerlaw_rf}
    Under \cref{ass:powerlaw_rf}, The deterministic equivalents $\sR^{\tt RFM}_{0}$ and $\sN^{\tt RFM}_{0}$, in over-parameterized regimes ($p>n$) admit \footnote{The symbol $\approx$ here denotes using an integral to approximate an infinite sum when calculating $\Tr(\cdot)$.}
    \begin{equation}\label{eq:RORFMover}
            \sR_0^{\tt RFM} \approx \left(\nicefrac{n}{C_\alpha}\right)^{-\alpha} \sN_0^{\tt RFM} + C_{n,\alpha,r,1}\,,  
    \end{equation}
    while in the under-parameterized regime ($p<n$), we have
    \begin{align}\label{eq:RORFMunder}
        \sR_0^{\tt RFM} \approx \left(\nicefrac{n}{C_{\alpha}}\right)^{-\alpha}\sN_0^{\tt RFM} + C_{n,\alpha,r,2}\,, \quad \mbox{when}~p \to n \,,
    \end{align}
where \( C_{n,\alpha,r,1 (2)} \) are constants (see \cref{app:relationship_rf} for details) that only depend on $n$, $\alpha$ and $r$, and it admits that $C_{n,\alpha,r,1} <  C_{n,\alpha,r,2}$.
\end{corollary}


\noindent{\bf Remark:}
In the over-parameterized regime, the relationship between \(\sR_0^{\tt RFM}\) and \(\sN_0^{\tt RFM}\) is a monotonically increasing linear function, with a growth rate controlled by the factor decaying with $n$.
In the under-parameterized regime, as \(p \to n\) (which also leads to \(\sR_0^{\tt RFM}\) and \(\sN_0^{\tt RFM} \to \infty\)), \(\sR_0^{\tt RFM}\) still grows linearly w.r.t \(\sN_0^{\tt RFM}\), with the same growth rate factor decaying with $n$. Furthermore, since \(C_{n,\alpha,r,1} < C_{n,\alpha,r,2}\), the test risk curve shows that over-parameterization is better than under-parameterization.
This approximation is also empirically verified to be precise in \cref{fig:random_feature_risk_vs_norm_approx}.


\begin{figure}[t]
    \centering
    \includegraphics[width=0.5\textwidth]{arxiv_version/figures/Scaling_Law/scaling_law_norm_based_capacity.pdf} 
    \caption{The value of exponents $\gamma_n$ and $\gamma_\sN$ in different regions (divided by $q$ and $\ell$) for $r \in (0, \frac{1}{2})$. Variance dominated region is colored by {\color{regionorange}orange}, {\color{regionyellow}yellow} and {\color{regionbrown}brown}, bias dominated region is colored by {\color{regionblue}blue} and {\color{regiongreen}green}.} 
    \label{fig:scaling_law_norm_based_capacity} \vspace{-0.35cm}
\end{figure}


To study scaling law, we follow the same setting of \citet{defilippis2024dimension} by choosing $p = n^q$ and $\lambda = n^{-(\ell-1)}$ with $q,l \geq 0$. We have the scaling law as below; see the proof in \cref{app:scaling_law}.
\begin{proposition}\label{prop:scaling_law_norm_based_capacity}
Under \cref{ass:powerlaw_rf}, for $r \in (0, \frac{1}{2})$, taking $p = n^q$ and $\lambda = n^{-(\ell-1)}$ with $q,l \geq 0$, we formulate the scaling law under norm-based capacity in different areas is 
\begin{equation*}
    \sR_\lambda^{\tt RFM} = \Theta\left(n^{\gamma_n} \cdot \left(\sN_\lambda^{\tt RFM}\right)^{\gamma_{\sN}}\right)\,, 
\end{equation*}    
where the rate $\{ \gamma_n, \gamma_{\sN} \}$ in different areas is given in \cref{fig:scaling_law_norm_based_capacity}.
\end{proposition}

\noindent{\bf Remark:}
In regions \ding{172}, \ding{173}, \ding{174}, and \ding{175} of \cref{fig:scaling_law_norm_based_capacity}, the exponent of \(\sN_\lambda^{\tt RFM}\) is positive, i.e., $\gamma_{\sN} >0$, and  \(\sR_\lambda^{\tt RFM}\) increases monotonically with  \(\sN_\lambda^{\tt RFM}\). However, in region \ding{176}, we have $\gamma_{\sN} <0$, and \(\sR_\lambda^{\tt RFM}\) decreases monotonically with \(\sN_\lambda^{\tt RFM}\).\\