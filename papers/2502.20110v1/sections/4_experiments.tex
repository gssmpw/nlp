\section{Experiments}
\label{sec:experiments}


\subsection{Experimental Setup}
\label{ssec:experiments:setup}
\blue{\PAR{Data.} The training data is the combination of 24 publicly available datasets: A2D2~\cite{geyer2020a2d2}, Argoverse2~\cite{2021argoverse2}, ARKit-Scenes~\cite{baruch2021arkitscenes}, BEDLAM~\cite{black2023bedlam}, BlendedMVS~\cite{yao2020blendedmvs}, DL3DV~\cite{ling2024dl3dv}, DrivingStereo~\cite{yang2019drivingstereo}, DynamicReplica~\cite{karaev2023dynamicreplica}, EDEN~\cite{le2021eden}, HOI4D~\cite{liu2022hoi4d}, HM3D~\cite{ramakrishnan2021habitat}, Matterport3D~\cite{chang2017matterport3d}, Mapillary-PSD~\cite{Lopez2020mapillary}, MatrixCity~\cite{li2023matrixcity}, MegaDepth~\cite{li2018megadepth}, NianticMapFree~\cite{arnold2022mapfree}, PointOdyssey~\cite{zheng2023pointodyssey}, ScanNet~\cite{dai2017scannet}, ScanNet++~\cite{yeshwanthliu2023scannetpp}, TartanAir~\cite{wang2020tartanair}, Taskonomy~\cite{zamir2018taskonomy}, Waymo~\cite{sun2020waymo}, and WildRGBD~\cite{xia2024wildrgbd} for a total of 16M images.
We evaluate the generalizability of models by testing them on 8 datasets not seen during training, grouped in different domains that are defined based on indoor or outdoor settings. 
The indoor group corresponds to the validation splits of SUN-RGBD~\cite{Song2015sunrgbd}, IBims~\cite{koch2022ibims}, TUM-RGBD~\cite{sturm12tumrgbd}, and HAMMER~\cite{jung2022hammer}, while the outdoor group comprises ETH3D~\cite{schoeps2017eth3d}, Sintel~\cite{Butler2012sintel}, DDAD~\cite{Guizilini2020ddad}, and NuScenes~\cite{nuscenes}.}

\blue{\PAR{Evaluation Details.} All methods have been re-evaluated with a fair and consistent pipeline.
In particular, we do not exploit any test-time augmentations and we utilize the same weights for all zero-shot evaluations.
We use the checkpoint corresponding to the zero-shot model for each method, \ie not fine-tuned on KITTI or NYU.
The metrics utilized in the main experiments are $\mathrm{\delta_1^{SSI}}$, $\mathrm{F_{A}}$, and $\mathrm{\rho_{A}}$.
$\mathrm{\delta_1}$ measures the depth estimation performance.
$\mathrm{F_{A}}$ is the area under the curve (AUC) of F1-score~\cite{ornek20222metrics} up to $1/20$ of the datasets' maximum depth and evaluates 3D estimation accuracy.
$\mathrm{\rho_{A}}$ evaluates the camera performance and is the AUC of the average angular error of camera rays up to 15$^{\circ}$.
We do not use parametric evaluation of \eg{}focal length, since it is a less flexible metric across diverse camera models and perfectly unrectified images.
Moreover, we present the fine-tuning ability of \ourmodel by training the final checkpoint on KITTI and NYU-Depth V2 and evaluating in-domain, as per standard practice.}

\PAR{Implementation Details.} \ourmodel is implemented in PyTorch~\cite{pytorch} and CUDA~\cite{nickolls2008cuda}.
For training, we use the AdamW~\cite{Loshchilov2017adamw} optimizer ($\beta_1=0.9$, $\beta_2=0.999$) with an initial learning rate of $5\times{}10^{-5}$.
The learning rate is divided by a factor of 10 for the backbone weights for every experiment and weight decay is set to $0.1$.
We exploit Cosine Annealing as learning rate and weight decay scheduler to one-tenth starting from 30\% of the whole training.
\blue{We run 300k optimization iterations with a batch size of 128.
The training time amounts to 6 days on 16 NVIDIA 4090 with half precision.
The dataset sampling procedure follows a weighted sampler, where the weight of each dataset is its number of scenes.
Our augmentations are both geometric and photometric, \ie random resizing, cropping, and translation for the former type, and brightness, gamma, saturation, and hue shift for the latter.
We randomly sample the image ratio per batch between 2:1 and 1:2.}
Our ViT~\cite{Dosovitskiy2020VIT} backbone is initialized with weights from DINO-pre-trained~\cite{oquab2023dinov2} models.
For the ablations, we run 100k training steps with a ViT-S backbone, with the same training pipeline as for the main experiments.


\subsection{Comparison with The State of The Art}
\label{ssec:experiments:comparison}

\begin{table*}[t]
    \centering
    \caption{\textbf{Results for Indoor Domains.} All methods are tested in a zero-shot fashion. Missing values (\textcolor{gray}{-}) indicate the model's inability to produce the respective output. \dag: Requires ground-truth (GT) camera for 3D reconstruction. \ddag: Requires GT camera for 2D depth map inference.}
    \label{tab:results:indoor}
    \vspace{-1em}
    \resizebox{\linewidth}{!}{%
    \begin{tabular}{l|ccc|ccc|ccc|ccc}
    \toprule
    \multirow{2}{*}{\textbf{Method}} & \multicolumn{3}{c|}{SUNRGBD} & \multicolumn{3}{c|}{HAMMER} & \multicolumn{3}{c|}{IBims-1} & \multicolumn{3}{c}{TUM-RGBD} \\
     & $\mathrm{\delta_1}\uparrow$ & $\mathrm{F_A}\uparrow$ & $\mathrm{\rho_A}\uparrow$ & $\mathrm{\delta_1}\uparrow$ & $\mathrm{F_A}\uparrow$ & $\mathrm{\rho_A}\uparrow$ & $\mathrm{\delta_1}\uparrow$ & $\mathrm{F_A}\uparrow$ & $\mathrm{\rho_A}\uparrow$ & $\mathrm{\delta_1}\uparrow$ & $\mathrm{F_A}\uparrow$ & $\mathrm{\rho_A}\uparrow$ \\
    \midrule
    Metric3D\textsuperscript{\dag \ddag}~\cite{yin2023metric3d} & $1.9$ & \textcolor{gray}{-} & \textcolor{gray}{-} & $0.9$ & \textcolor{gray}{-} & \textcolor{gray}{-} & $75.1$ & \textcolor{gray}{-} & \textcolor{gray}{-} & $7.7$ & \textcolor{gray}{-} & \textcolor{gray}{-} \\
    Metric3Dv2\textsuperscript{\dag \ddag}~\cite{hu2024metric3dv2} & $81.2$ & \textcolor{gray}{-} & \textcolor{gray}{-} & $\best{65.3}$ & \textcolor{gray}{-} & \textcolor{gray}{-} & $68.4$ & \textcolor{gray}{-} & \textcolor{gray}{-} & $63.0$ & \textcolor{gray}{-} & \textcolor{gray}{-} \\
    ZoeDepth\textsuperscript{\dag}~\cite{bhat2023zoedepth} & $80.9$ & \textcolor{gray}{-} & \textcolor{gray}{-} & $0.9$ & \textcolor{gray}{-} & \textcolor{gray}{-} & $49.8$ & \textcolor{gray}{-} & \textcolor{gray}{-} & $55.6$ & \textcolor{gray}{-} & \textcolor{gray}{-} \\
    UniDepth~\cite{piccinelli2024unidepth} & $94.3$ & $78.6$ & $85.8$ & $1.8$ & $52.1$ & $55.3$ & $15.7$ & $30.3$ & $\best{76.6}$ & $72.3$ & $54.8$ & $86.8$ \\
    MASt3R~\cite{leroy2024master} & $80.1$ & $71.5$ & $\scnd{92.0}$ & $2.2$ & $38.1$ & $\best{86.5}$ & $61.0$ & $55.7$ & $76.0$ & $52.4$ & $44.1$ & $\scnd{93.7}$ \\
    DepthPro~\cite{bochkovskii2024depthpro} & $83.1$ & $71.1$ & $89.3$ & $29.4$ & $\scnd{71.0}$ & $69.1$ & $82.3$ & $62.8$ & $75.9$ & $56.9$ & $48.1$ & $\best{96.5}$ \\
    \midrule
    \ourmodel-Small & $90.8$ & $74.2$ & $87.7$ & $20.1$ & $52.6$ & $77.5$ & $86.6$ & $62.4$ & $67.5$ & $69.0$ & $50.6$ & $86.1$ \\
    \ourmodel-Base & $\scnd{94.4}$ & $\scnd{79.9}$ & $91.1$ & $30.6$ & $57.0$ & $65.6$ & $\scnd{89.7}$ & $\scnd{68.5}$ & $\scnd{76.5}$ & $\scnd{77.5}$ & $\scnd{57.3}$ & $89.4$ \\
    \ourmodel-Large & $\best{96.4}$ & $\best{84.6}$ & $\best{93.4}$ & $\scnd{64.5}$ & $\best{74.9}$ & $\scnd{78.3}$ & $\best{94.5}$ & $\best{70.9}$ & $74.1$ & $\best{90.5}$ & $\best{62.9}$ & $89.6$ \\
    \bottomrule
    \end{tabular}%
    }
\vspace{-1em}
\end{table*}
\begin{table*}[t]
    \centering
    \caption{\textbf{Results for Outdoor Domains.} All methods are tested in a zero-shot fashion. Missing values (\textcolor{gray}{-}) indicate the model's inability to produce the respective output. \dag: Requires ground-truth (GT) camera for 3D reconstruction. \ddag: Requires GT camera for 2D depth map inference.}
    \label{tab:results:outdoor}
    \vspace{-1em}
    \resizebox{\linewidth}{!}{%
    \begin{tabular}{l|ccc|ccc|ccc|ccc}
    \toprule
    \multirow{2}{*}{\textbf{Method}}  & \multicolumn{3}{c|}{ETH3D} & \multicolumn{3}{c|}{Sintel} & \multicolumn{3}{c|}{DDAD} & \multicolumn{3}{c}{NuScenes} \\
     & $\mathrm{\delta_1}\uparrow$ & $\mathrm{F_A}\uparrow$ & $\mathrm{\rho_A}\uparrow$ & $\mathrm{\delta_1}\uparrow$ & $\mathrm{F_A}\uparrow$ & $\mathrm{\rho_A}\uparrow$ & $\mathrm{\delta_1}\uparrow$ & $\mathrm{F_A}\uparrow$ & $\mathrm{\rho_A}\uparrow$ & $\mathrm{\delta_1}\uparrow$ & $\mathrm{F_A}\uparrow$ & $\mathrm{\rho_A}\uparrow$ \\
    \midrule
    Metric3D\textsuperscript{\dag \ddag}~\cite{yin2023metric3d} & $19.7$ & \textcolor{gray}{-} & \textcolor{gray}{-} & $1.4$ & \textcolor{gray}{-} & \textcolor{gray}{-} & $81.9$ & \textcolor{gray}{-} & \textcolor{gray}{-} & $75.4$ & \textcolor{gray}{-} & \textcolor{gray}{-} \\
    Metric3Dv2\textsuperscript{\dag \ddag}~\cite{hu2024metric3dv2} & $\best{90.0}$ & \textcolor{gray}{-} & \textcolor{gray}{-} & $\best{34.5}$ & \textcolor{gray}{-} & \textcolor{gray}{-} & $\scnd{87.6}$ & \textcolor{gray}{-} & \textcolor{gray}{-} & $84.1$ & \textcolor{gray}{-} & \textcolor{gray}{-} \\
    ZoeDepth\textsuperscript{\dag}~\cite{bhat2023zoedepth} & $33.8$ & \textcolor{gray}{-} & \textcolor{gray}{-} & $5.6$ & \textcolor{gray}{-} & \textcolor{gray}{-} & $27.9$ & \textcolor{gray}{-} & \textcolor{gray}{-} & $33.8$ & \textcolor{gray}{-} & \textcolor{gray}{-} \\
    UniDepth~\cite{piccinelli2024unidepth} & $18.5$ & $27.6$ & $42.6$ & $13.2$ & $40.2$ & $65.6$ & $85.8$ & $\scnd{72.8}$ & $\best{98.1}$ & $84.6$ & $\scnd{64.4}$ & $\best{97.7}$ \\
    MASt3R~\cite{leroy2024master} & $21.4$ & $28.4$ & $\scnd{92.2}$ & $17.2$ & $41.5$ & $72.2$ & $4.3$ & $22.1$ & $74.6$ & $2.7$ & $13.6$ & $78.3$ \\
    DepthPro~\cite{bochkovskii2024depthpro} & $39.7$ & $41.2$ & $77.4$ & $26.2$ & $49.7$ & $75.2$ & $29.9$ & $42.1$ & $83.0$ & $56.6$ & $46.5$ & $79.1$ \\
    \midrule
    \ourmodel-Small & $64.6$ & $44.3$ & $78.4$ & $14.6$ & $37.1$ & $73.5$ & $83.3$ & $68.5$ & $94.7$ & $82.1$ & $59.7$ & $96.2$ \\
    \ourmodel-Base & $75.4$ & $\scnd{53.5}$ & $91.4$ & $31.9$ & $\best{51.8}$ & $\scnd{75.9}$ & $86.8$ & $71.4$ & $96.1$ & $\scnd{85.3}$ & $63.6$ & $96.6$ \\
    \ourmodel-Large & $\scnd{85.2}$ & $\best{59.3}$ & $\best{92.6}$ & $\scnd{34.4}$ & $\scnd{51.4}$ & $\best{76.3}$ & $\best{88.2}$ & $\best{73.3}$ & $\scnd{96.7}$ & $\best{87.0}$ & $\best{66.7}$ & $\scnd{97.2}$ \\
    \bottomrule
    \end{tabular}%
    }
    \vspace{-1em}
\end{table*}
\begin{figure}[t]
    \centering
    \includegraphics[width=1.0\linewidth]{figures/assets/normalized_plot_resistance.pdf}
    \vspace{-2em}
    \caption{\textbf{Invariance to image shape.} \ourmodel is trained with a variable input shape pipeline in addition to random resizing for each of the image pairs. The proposed training strategy improves the robustness in terms of predicted depth scale and accuracy ($\delta_1$) to the input image's shape compared to two other state-of-the-art methods.}
    \label{fig:results:shape_invariance}
    \vspace{-1em}
\end{figure}



\blue{We evaluate our method on eight zero-shot validation sets, covering both indoor and outdoor scenes, as shown in \Cref{tab:results:indoor} and \Cref{tab:results:outdoor}, respectively. Our model performs better than or at least on par with all baselines, even outperforming methods that require ground-truth camera parameters at inference time, such as \cite{yin2023metric3d, hu2024metric3dv2}.
Notably, \ourmodel excels in 3D estimation, as reflected in the $\mathrm{F_A}$ metric, where it achieves a consistent improvement ranging from 0.5\% to 18.1\% over the second-best method. Additionally, it outperforms UniDepth~\cite{piccinelli2024unidepth} in nearly all cases, except for the $\mathrm{\rho_A}$ metric on IBims-1, DDAD, and NuScenes.
This demonstrates that our proposed version is a significant step forward in both performance and efficiency.
However, the camera parameter estimation ($\mathrm{\rho_A}$) sees only marginal improvements, indicating that the limited diversity of training cameras remains a challenge that could be addressed with additional camera-only training, as suggested in~\cite{bochkovskii2024depthpro}.
\Cref{tab:results:nyu_ft} and \Cref{tab:results:kitti_ft} show results for models fine-tuned on the NYU and KITTI training sets and evaluated on their respective validation splits, following standard protocols.
Fine-tuning performance serves as an indicator of a model's ability to specialize to specific downstream tasks and domains.
\ourmodel effectively adapts to new domains and outperforms methods that were pre-trained on large, diverse datasets before fine-tuning on NYU or KITTI, such as~\cite{bhat2023zoedepth, hu2024metric3dv2, yang2024da2},
This is particularly evident in the outdoor setting (KITTI), as shown in \Cref{tab:results:kitti_ft}.
As detailed in \Cref{ssec:method:design}, our training strategy incorporates variable image aspect ratios and resolutions within the same distributed batch.
Combined with camera conditioning and invariance learning, this approach enhances the model’s robustness to changes in input image shape.
\Cref{fig:results:shape_invariance} quantifies this effect: the y-axis represents normalized metric accuracy ($\mathrm{\delta}_1$ scaled by the method’s maximum value), while the x-axis varies the image shape.
The normalization ensures a consistent scale across models.
\ourmodel is almost invariant to image shape, demonstrating that it can effectively trade off resolution for speed without sacrificing accuracy, as clearly illustrated in \Cref{fig:results:shape_invariance}.}


\begin{table}[t]
    \centering
    \caption{\textbf{Comparison on NYU validation set.} All models are trained on NYU. The first 4 are trained only on NYU. The last 4 are fine-tuned on NYU.}
    \vspace{-1em}
    \label{tab:results:nyu_ft}
    \resizebox{\columnwidth}{!}{
    \begin{tabular}{l|ccc|ccc}
        \toprule
        \multirow{2}{*}{\textbf{Method}} & $\mathrm{\delta}_{1}$ & $\mathrm{\delta}_{2}$ & $\mathrm{\delta}_{3}$ & $\mathrm{A.Rel}$ & $\mathrm{RMS}$ & $\mathrm{Log}_{10}$\\
         & \multicolumn{3}{c|}{\textit{Higher is better}} & \multicolumn{3}{c}{\textit{Lower is better}}\\
        \toprule
        BTS~\cite{Lee2019bts} & $88.5$ & $97.8$ & $99.4$ & $10.9$ & $0.391$ & $0.046$\\
        AdaBins~\cite{Bhat2020adabins} & $90.1$ & $98.3$ & $99.6$ & $10.3$ & $0.365$ & $0.044$\\
        NeWCRF~\cite{Yuan2022newcrf} & $92.1$ & $99.1$ & $\scnd{99.8}$ & $9.56$ & $0.333$ & $0.040$\\
        iDisc~\cite{piccinelli2023idisc} & $93.8$ & $99.2$ & $\scnd{99.8}$ & $8.61$ & $0.313$ & $0.037$\\
        ZoeDepth~\cite{bhat2023zoedepth} & $95.2$ & $\scnd{99.5}$ & $\scnd{99.8}$ & $7.70$ & $0.278$ & $0.033$\\
        Metric3Dv2~\cite{hu2024metric3dv2} & $\best{98.9}$ & $\best{99.8}$ & $\best{100}$ & $\scnd{4.70}$ & $\scnd{0.183}$ & $\best{0.020}$\\
        DepthAnythingv2~\cite{yang2024da2} & $98.4$ & $\mathbf{99.8}$ & $\mathbf{100}$ & $5.60$ & $0.206$ & $\scnd{0.024}$\\
        \midrule 
        \ourmodel & $\scnd{98.8}$ & $\best{99.8}$ & $\best{100}$ & $\best{4.68}$ & $\best{0.180}$ & $\best{0.020}$\\ % 262250
        \bottomrule
    \end{tabular}}
    \vspace{-1em}
\end{table}

\begin{table}[t]
    \centering
    \caption{\textbf{Comparison on KITTI Eigen-split validation set.} All models are trained on KITTI Eigen-split training and tested on the corresponding validation split. The first 4 are trained only on KITTI. The last 4 are fine-tuned on KITTI.}
    \vspace{-1em}
    \label{tab:results:kitti_ft}
    \resizebox{\columnwidth}{!}{
    \begin{tabular}{l|ccc|ccc}
        \toprule
        \multirow{2}{*}{\textbf{Method}} & $\mathrm{\delta}_{1}$ & $\mathrm{\delta}_{2}$ & $\mathrm{\delta}_{3}$ & $\mathrm{A.Rel}$ & $\mathrm{RMS}$ & $\mathrm{RMS}_{\log}$\\
         & \multicolumn{3}{c|}{\textit{Higher is better}} & \multicolumn{3}{c}{\textit{Lower is better}}\\
        \toprule
        BTS~\cite{Lee2019bts} & $96.2$ & $99.4$ & $99.8$ & $5.63$ & $2.43$ & $0.089$\\
        AdaBins~\cite{Bhat2020adabins} & $96.3$ & $99.5$ & $99.8$ & $5.85$ & $2.38$ & $0.089$\\
        NeWCRF~\cite{Yuan2022newcrf} & $97.5$ & $\scnd{99.7}$ & $\scnd{99.9}$ & $5.20$ & $2.07$ & $0.078$\\
        iDisc~\cite{piccinelli2023idisc} & $97.5$ & $\scnd{99.7}$ & $\scnd{99.9}$ &$5.09$ & $2.07$ & $0.077$\\
        ZoeDepth~\cite{bhat2023zoedepth} & $96.5$ & $99.1$ & $99.4$ & $5.76$ & $2.39$ & $0.089$ \\
        Metric3Dv2~\cite{yin2023metric3d} & $\scnd{98.5}$ & $\best{99.8}$ & $\best{100}$ & $\scnd{4.40}$ & $1.99$ & $\scnd{0.064}$\\
        DepthAnythingv2~\cite{yang2024da2} & $98.3$ & $\best{99.8}$ & $\best{100}$ & $4.50$ & $\scnd{1.86}$ & $0.067$ \\
        \midrule
        \ourmodel & $\best{98.9}$ & $\best{99.8}$ & $\scnd{99.9}$ & $\best{3.73}$ & $\best{1.71}$ & $\best{0.061}$ \\ %263000
        \bottomrule
    \end{tabular}}
    \vspace{-1em}
\end{table}

\begin{figure}[t]
    \centering
    \includegraphics[width=1.0\linewidth]{figures/assets/normalized_plot_confidence.pdf}
    \vspace{-2em}
    \caption{\textbf{Confidence invariance.} The uncertainty output of \ourmodel represents the predicted error. The confidence is obtained as the inverse uncertainty and the output is evaluated by taking into account only the pixels with a confidence higher than the corresponding x-axis. Y-axis reports the normalized RMSE to have a consistent scale among different datasets, where normalization involves dividing the RMSE by the value with threshold 0, namely evaluating over all pixels.}
    \label{fig:results:confidence}
    \vspace{-1em}
\end{figure}



\subsection{Ablation Studies}
\label{ssec:experiments:ablations}

The importance of each new component introduced in \ourmodel in \cref{sec:method} is evaluated by ablating the method in \blue{Tables \ref{tab:results:ablations_arch}, \ref{tab:results:ablations_loss}, and \ref{tab:results:ablations_version}.}
All ablations exploit the predicted camera representation, if not stated otherwise.
\blue{\Cref{tab:results:ablations_arch} evaluates the impact of various architectural modifications compared to UniDepth~\cite{piccinelli2024unidepth}, analyzing their effects on both performance and efficiency.
\Cref{tab:results:ablations_loss} assesses the importance of the proposed loss function (\cref{ssec:method:egssi}) and examines the effect of applying the geometric invariance loss originally introduced in UniDepth~\cite{piccinelli2024unidepth} (\cref{ssec:method:consistency}) in different spaces.
The rationale behind our design choices is to maintain simplicity while maximizing effectiveness.
Additionally, in \Cref{tab:results:ablations_version} we analyze the role of camera conditioning and report results for the original UniDepth under the same training and evaluation setup as our method for a direct comparison.
The evaluation is based on four key metrics: $\mathrm{\delta}_1$, which measures metric depth accuracy; $\mathrm{SI}_{\log}$, which assesses scale-invariant scene geometry; $\mathrm{F_A}$, which captures the 3D estimation capability; and $\mathrm{\rho_A}$, which evaluates monocular camera parameter estimation.
All reported metrics correspond to the aggregated zero-shot performance across datasets, as detailed in \cref{ssec:experiments:setup}.}

\begin{figure*}[t]
    \renewcommand{\arraystretch}{1}
    \centering
    \small
    \hspace{-5pt}
    \begin{tabular}{cc|ccc}
        \multirow{1}{*}[0.5in]{\rotatebox[origin=c]{90}{KITTI}}
        & \includegraphics[width=0.22\linewidth]{figures/assets/edges/kitti_rgb.pdf}
        & \includegraphics[width=0.22\linewidth]{figures/assets/edges/kitti_v1.pdf}
        & \includegraphics[width=0.22\linewidth]{figures/assets/edges/kitti_no_loss.pdf}
        & \includegraphics[width=0.22\linewidth]{figures/assets/edges/kitti_v2.pdf}\\
        
        \multirow{1}{*}[0.5in]{\rotatebox[origin=c]{90}{Sintel}}
        & \includegraphics[width=0.22\linewidth]{figures/assets/edges/sintel_rgb.pdf}
        & \includegraphics[width=0.22\linewidth]{figures/assets/edges/sintel_v1.pdf}
        & \includegraphics[width=0.22\linewidth]{figures/assets/edges/sintel_no_loss.pdf}
        & \includegraphics[width=0.22\linewidth]{figures/assets/edges/sintel_v2.pdf}\\
        
        % \multirow{1}{*}[0.5in]{\rotatebox[origin=c]{90}{ETH3D}}
        % & \includegraphics[width=0.145\linewidth]{figures/assets/edges/eth_rgb.pdf}
        % & \includegraphics[width=0.145\linewidth]{figures/assets/edges/eth_v1.pdf}
        % & \includegraphics[width=0.145\linewidth]{figures/assets/edges/eth_no_loss.pdf}
        % & \includegraphics[width=0.145\linewidth]{figures/assets/edges/eth_v2.pdf} \\
        & RGB & UniDepth~\cite{piccinelli2024unidepth} & UniDepthV2 w/o $\mathcal{L}_\mathrm{EG-SSI}$ & \ourmodel\\
    \end{tabular}
    \vspace{-1em}
    \caption{\textbf{Comparisons of predicted edges.} Each row displays the input RGB image and the 2D depth maps predicted by compared methods, color-coded with the \textit{magma reverse} colormap with a range between 0 and 50 meters. Better viewed on a screen and zoomed in.
    }
    \label{fig:results:edges}
    \vspace{-1em}
\end{figure*}

\begin{table}[t]
    \centering
    \caption{\textbf{Architectural ablations.} The different architectural additions (``+'') and subtractions (``-'') from the original UniDepth~\cite{piccinelli2024unidepth} are reported. ``- SHE + Sine'': camera encoding via Sine encoding instead of Spherical Harmonic Transform of the pinhole-based pencil of rays. ``- Attention'': attention layers in the decoder are removed. ``+ ResNet Blocks'': the attention layers in the decoder are substituted with simpler ResNet blocks. ``+ Multi-resol.'': the decoder has lateral connections with the shallower encoder layer, rather than a simpler merging of all resolutions in the bottleneck.}
    \vspace{-1em}
    \label{tab:results:ablations_arch}
    \resizebox{\linewidth}{!}{%
    \begin{tabular}{ll|cccc|cc}
    \toprule
    & \multirow{2}{*}{\textbf{Architecture}} & \multicolumn{4}{c|}{{Performance}} & \multicolumn{2}{c}{{Efficiency}} \\
     & & $\mathrm{\delta_1}\uparrow$ & $\mathrm{SI_{\log}}\downarrow$ & $\mathrm{F_A}\uparrow$ & $\mathrm{\rho_A}\uparrow$ & Latency$\downarrow$ & Params$\downarrow$\\
    \midrule
    1 & UniDepth~\cite{piccinelli2024unidepth} & $54.5$ & $16.4$ & $56.1$ & $77.1$ & 73.2 & 35.2 \\
    2 & - SHE + Sine & $54.6$ & $16.4$ & $56.0$ & $76.9$ & 53.2 & 35.2 \\
    3 & - Attention & $50.3$ & $17.9$ & $51.0$ & $76.6$ & 20.4 & 29.0 \\
    4 & + ResNet Blocks & $52.6$ & $16.6$ & $55.0$ & $76.6$ & 24.0 & 33.5 \\
    5 & + Multi-resol. & $54.5$ & $16.3$ & $56.0$ & $77.9$ & 25.0 & 34.2 \\
    \bottomrule
    \end{tabular}%
    }
    \vspace{-1em}
\end{table}
\begin{figure*}[tb!] 
    \centering
    \begin{subfigure}[b]{0.49\linewidth}
        \centering
        \includegraphics[width=\textwidth]{figures/exp_ScalingLaw_TinyStories_f1_score.pdf}
        \caption{Character choices prediction (TinyStories).}
    \end{subfigure}
    \hfill  % Creates horizontal spacing between subfigures
    \begin{subfigure}[b]{0.49\linewidth}
        \centering
        \includegraphics[width=\textwidth]{figures/exp_ScalingLaw_ultrachat_spearman.pdf}
        \caption{Response length prediction (Ultrachat).}
    \end{subfigure}
    \vspace{-5pt}
    \caption{Scaling effects on planning capabilities. Evaluated across four model families (LLaMA-2-chat, LLaMA-3-Instruct, Qwen-2-Instruct, Qwen-2.5-Instruct; 1.5B–72B) using UltraChat and TinyStories, structure and content attributes show family-specific scaling: larger models within each family improve planning.}
    \label{fig:exp_ablation_scaling}
    \vspace{-5pt}
\end{figure*}

\begin{figure*}[tb!] 
    \centering
    \begin{subfigure}[b]{0.48\linewidth}
        \centering
        \includegraphics[width=\textwidth]{figures/exp_TinyStories_stepwise_f1_dynamic_uniform.pdf}
        \caption{Character choice prediction (TinyStories).}
    \end{subfigure}
    \hfill  % Creates horizontal spacing between subfigures
    \begin{subfigure}[b]{0.48\linewidth}
        \centering
        \includegraphics[width=\textwidth]{figures/exp_medmcqa_stepwise_f1_dynamic_uniform.pdf}
        \caption{Answer confidence prediction (MedMCQA).}
    \end{subfigure}
    \vspace{-10pt}
    \caption{U-shaped planning dynamics during generation. Probing at equidistant positions (character choice, answer confidence) shows three-phase patterns: high accuracy in early segments (global planning intent), mid-segment decline (local token focus), and late-stage recovery (contextualized refinement). This suggests models first outline global attributes, then refine locally, before finalizing coherent plans.}
    \label{fig:exp_ablation_dynamics}
    \vspace{-10pt}
\end{figure*}

\begin{figure}[tb!]
    \centering
    \includegraphics[width=0.49\textwidth]{figures/exp_reflection_analysis_ultrachat_final.pdf}
    \vspace{-30pt}
    \caption{Implicit-explicit planning discrepancy. Models struggle to explicitly predict their own response lengths, with base models showing near-zero Spearman correlation and most fine-tuned models achieving only marginal gains. This gap implies limited introspective awareness despite underlying capability.}
    \label{fig:exp_ablation_selfAware_ultrachat}
    \vspace{-18pt}
\end{figure}

\section{Ablation}

\subsection{Planning Ability Scales with Model Size}
We analyze how emergent response planning scales across different model sizes using four model families: LLama-2-chat (7B, 13B, 70B), Llama-3-Instruct (8B, 70B), Qwen-2-Instruct (7B, 72B), and Qwen-2.5-Instruct (1.5B, 32B, 72B). Using grid search over layers and hidden sizes, we identify optimal configurations and evaluate models on UltraChat and TinyStories datasets, focusing on structure and content attributes.
We exclude base models as the relatively small models have short context which limit few-shot prompts, while the same prompts fail to effectively prompt larger base models to follow instructions. We omit the behavior attribute type as larger models tend to give correct answers consistently, making it difficult to obtain balanced data for analysis.

Results shown in Fig.~\ref{fig:exp_ablation_scaling} exhibit two key insights:  \textbf{(1)} within each model family, larger models demonstrate stronger planning capabilities, and \textbf{(2)} this scaling pattern does not generalize across different model families, suggesting that other factors like architectural differences also influence planning behavior.





\subsection{Evolution of Planning Representations During Response Generation}
We analyze how planning features evolve during generation by probing at different positions in the response sequence. For each response, we collect activations from the first token up to the token before attribute-revealing keywords (e.g., animal words in story character selection tasks) or throughout the entire sequence for tasks requiring external ground-truth labels (e.g., answer confidence tasks). We divide these positions into equal segments and apply  probes previously trained with in-dataset settings at each division point.
We conduct experiments on two datasets: TinyStories for character choice prediction and MedMCQA for answer confidence prediction. Results in Fig.~\ref{fig:exp_ablation_dynamics} reveal a distinctive pattern: \textbf{probing accuracy is high initially, decreases in the middle segments, and rises again toward the end}. This pattern suggests a three-phase planning process:
(1) initial phase with strong planning that provides an overview of the intended response;
(2) middle phase with weaker planning, characterized by more local, token-by-token generation;
(3) final phase with increased planning clarity as accumulated context makes the target attributes more apparent.



\subsection{Gap Between Probing and LLMs' Self-Predicted Results}

We investigate how the models' ability to predict their own response attributes compare to probe-based predictions with the UltraChat dataset on response length task. For fine-tuned models, we prompt: "Estimate your answer length in tokens using \texttt{[TOKENS]number[/TOKENS]}, then provide your answer." For base models, we provide few-shot examples with pre-calculated lengths. To evaluate self-prediction accuracy, we compare the predicted token count collected in a separated run against the length of the previous-collected model's greedy-decoded response.

As shown in Fig.~\ref{fig:exp_ablation_selfAware_ultrachat}, base models achieve near-zero Spearman correlation when predicting token lengths, even with examples. While fine-tuned models perform marginally better, there remains a substantial gap between models' direct predictions and probe-based predictions.
This gap suggests that \textbf{models encode more planning information in their hidden representations than they can explicitly access during token-by-token generation}, indicating a discrepancy between implicit planning capabilities and explicit self-awareness.

% While our results show emergent planning capabilities in LLMs, we also investigate whether models can explicitly predict attributes of their future responses. We test this using the response length prediction task on the UltraChat dataset, comparing direct model predictions against probe-based predictions.
% For fine-tuned models, we prompt: "Given this question, first estimate the number of tokens you would need for a complete answer using format as \texttt{[TOKENS]number[/TOKENS]}, then give your answer." For base models, we provide few-shot examples with pre-calculated token lengths.

% As shown in Fig.~\ref{fig:exp_ablation_selfAware_ultrachat}, we evaluate prediction accuracy using Spearman correlation. Base models struggle to predict their token lengths even with examples, while fine-tuned models perform better. However, there is an obvious gap between models' direct predictions with probe-based predictions, suggesting that probing better captures this capability.





% \subsection{Models across Different Architectures Show Similar Deep-in Planning Patterns}


% We investigate whether representations from one model can predict the behavior of another. Specifically, given a prompt $\mathbf{x}i$, can we use Model A's representations $\mathcal{H}_{i, A} = \{ \mathbf{H}^l{\mathbf{x}_{i, A}}\}^L_{l=1}$ to predict Model B's output label $\hat{g}_{i, B} = g(\mathbf{y}_{i, B})$? Success would suggest these representations capture intrinsic patterns beyond model-specific features.

% Results on the TinyStories dataset (Figure~\ref{fig:exp_cross_TinyStories}) compare models as predictors (horizontal axis) versus targets (vertical axis). We observe that: (1) diagonal terms (same predictor and target) show highest performance, indicating models best predict their own behavior; (2) strong non-diagonal performance suggests shared representations across different architectures, implying common underlying planning mechanisms.


% \begin{figure}[tb!]
%     \centering
%     % \vspace{-18pt}
%     \includegraphics[width=1.0\columnwidth]{figures/ablation_cross_model_correlations_f1_score_TinyStories_story_continuation.png}
%     \vspace{-4pt}
%     \caption{
%     Cross-model test for chat models on TinyStories dataset, with F1 score as metric.
%     }
%     \label{fig:exp_cross_TinyStories}
%     \vspace{-7pt}
% \end{figure} 

% \DZCtodo{First collect model A's activations on model B's responses, then use features of model A to predict label of model B's response;
% Results: 1) non-diagonal terms are much worse than diagonal terms, thus the planning features are model-specific, rather than a simple problem-label mapping; 2) some non-diagonal terms are non-zero, means different models show similar planning features to some degrees}



\blue{\PAR{Architecture.} \Cref{tab:results:ablations_arch} outlines the key modifications that transform the original UniDepth~\cite{piccinelli2024unidepth} architecture into \ourmodel.
The first major change is the removal of spherical harmonics (SH)-based encoding, which is computationally inefficient.
Instead, we revert to standard Sine encoding (row 2).
While the difference in performance is minimal in our setup, we hypothesize that the encoding’s impact diminishes as the model benefits from larger and more diverse training data across different cameras.
Next, we eliminate the attention mechanism in row 3 due to its high computational cost.
This removal results in a significant performance drop, \eg{}-4.3\% for $\mathrm{\delta}_1$, but yields a greater than 2x improvement in efficiency.
In row 4, we replace the pure MLP-based decoder with ResNet blocks, introducing spatial $3\times3$ convolutions.
This modification enhances performance by leveraging local spatial structure while inducing a minimal impact on efficiency.
Finally, row 5 integrates a multi-resolution feature fusion from the encoder to the decoder, following an FPN-style design.
This final architecture significantly reduces computational cost while preserving overall performance: the final model (row 5) achieves similar performance to the original UniDepth (row 1) while requiring only one-third of the computation.}
\blue{\PAR{$\mathcal{L}_{\mathrm{EG-SSI}}$ Loss.} The effectiveness of the proposed $\mathcal{L}_{\mathrm{EG-SSI}}$ loss, detailed in \cref{ssec:method:egssi}, is evaluated in row 2 \vs row 3 of \Cref{tab:results:ablations_loss}.
Introducing this loss results in a 4.7\% improvement in $\mathrm{\delta}_1$ and a 1.8\% improvement in $\mathrm{F_A}$, demonstrating its contribution to both metric accuracy and 3D estimation.
Interestingly, despite $\mathcal{L}_{\mathrm{EG-SSI}}$ not explicitly supervising camera parameter estimation, the $\mathrm{\rho_A}$ metric also shows improvement.
This suggests that the loss contributes to a less noisy training process, leading to better feature representations in the encoder.
A qualitative comparison of the impact of $\mathcal{L}_{\mathrm{EG-SSI}}$ is presented in \cref{fig:results:edges}.
The difference between the third and fourth columns highlights the visual impact of the proposed loss, particularly in refining depth discontinuities.
Additionally, the comparison between the second and third columns illustrates the combined effect of architectural changes and increased data diversity, showing improved reconstruction of finer details, such as body parts that were previously smoothed or missed.}
\blue{\PAR{$\mathcal{L}_{\mathrm{con}}$ Output Space.} \ourmodel introduces multiple instances of camera-conditioned depth features $\mathbf{D}|\mathbf{E}$, corresponding to different decoder resolutions, as described in \cref{ssec:method:design}.
This contrasts with the original UniDepth~\cite{piccinelli2024unidepth}, which relied on a single conditioning point.
Given this architectural shift, we argue that deep conditioning may not be optimal.
Features at different resolutions encode varying levels of abstraction, and enforcing deep conditioning introduces additional design freedom.
\Cref{tab:results:ablations_loss} investigates where to apply the consistency loss ($\mathcal{L}_{\mathrm{con}}$) from~\cite{piccinelli2024unidepth}: either directly in the output space ($\mathbf{Z}$, row 2) or within the camera-conditioned features at each scale ($\mathbf{D}|\mathbf{E}$, row 1).
The results indicate minimal differences from applying the loss directly in the output space. Therefore, based on Occam's razor, we adopt the simpler and more effective design from row 2 as the final approach.}
\blue{\PAR{Conditioning Impact.} As previously explored in~\cite{piccinelli2024unidepth}, we analyze the impact of our proposed camera conditioning in \Cref{tab:results:ablations_version}.
This ablation includes both UniDepth and \ourmodel under the same conditions—without $\mathcal{L}_{\mathrm{EG-SSI}}$ and without invariance applied to deep features ($\mathbf{D}|\mathbf{E}$).
The results show that conditioning has a even stronger positive effect for \ourmodel, as evidenced by comparing row 3 \vs row 4 against the comparison of row 1 \vs row 2.}
\blue{\PAR{Confidence.} The confidence measure introduced in \cref{ssec:method:design} is evaluated on three zero-shot datasets, as shown in \cref{fig:results:confidence}.
The y-axis represents the normalized $\mathrm{RMSE}$, computed as $\mathrm{RMSE}$ divided by its per-dataset value at $x = 0$, while the x-axis corresponds to the confidence quantile.
For each quantile, the evaluation considers only pixels whose confidence exceeds the given threshold.
Ideally, confidence should be negatively correlated with error: if the confidence estimate is reliable, higher-confidence regions should exhibit lower $\mathrm{RMSE}$.
More specifically, \cref{fig:results:confidence} validates how the predicted confidence of \ourmodel negatively correlates with the error, thus showing its reliability.}
