    
    % To train an LLM that is able to converse following designated depression manifestation,
    % we design the format of a profile for each realistic conversation in the dataset to form a instruction-tuning dataset.
    % These profiles are iteratively refined through expert feedback, ensuring accuracy and clinical relevance.
    % We carefully design a psychological profile to record a client's characteristics regarding their depression manifestation. 
    % To ensure provided information in the profile is accurate and comprehensive to meet counselors' practice needs, we iteratively develop and refine the profiles by absorbing feedbacks from counselors. 
    % Authors develop the first version of profile and train a model that roleplay following the profile information, and then we invited counselors to interact with the model and evaluate the design of profile and suggest improvement. Finally we got a profile that meets their practice needs.
    % We supervised-tune an LLM using system prompts that specify the client's profile and conversation context, enabling the model to role-play with consistency and realism across different depressive manifestations
    % This approach allows practitioners to engage with a range of depressive profiles, mirroring real-world variations in symptoms and experiences.

        
    % Standard LLMs (e.g., GPT4) undergo significant alignment with general human preferences—to be positive and supportive—creating an inherent ceiling on their ability to authentically simulate some depression symptoms through language (e.g., self-harm ideation, or cognitive distortions), even with carefully crafted prompting.

   
    % First, we generate preference data using a model-based verifier, employing a novel sampling method to produce highly contrasting negative responses from noisy profile adherence. This mitigates the model’s tendency to generate only subtle deviations, ensuring clear distinctions between preferred (fully aligned) and less preferred (slightly deviated) responses. We trained model on this preference data via DPO and get into the next step.
    % Second, we recruit human counselors to contribute a handful of preference labels and  continuously optimize the DPO model trained from the first round of preference data as a final calibration.
    % This procedure perfectly balances the cost and profession.
    
    % Though the general performance of the SFT model has shown its excellent roleplay ability with a designated psychological profile, we found there is space to improve generation quality to more precisely comply all the information in the profile. 
    % To address it, we sort to preference optimization. 

    % We first leverage model-based verifiers to construct preference data. Specifically in this step, we develop a novel procedure to sample highly contrasting negative (less preferred) responses to address the model ard to generate very contrasting samples, and then we improve the model via direct preference optimization.


    % The alignment goals rooted in standard LLMs (e.g., GPT4) train them to be more like a positive and supportive character that creates an inherent ceiling on their ability to authentically express depression through (e.g., self-harm ideation, or cognitive distortions), even with carefully crafted prompting.

        % To bridge this gap, we strain our nerves to find and refine raw, uncurated depression-related conversations from existing resources. Leveraging GPT-4 labeling and pre-existing dataset labels, we systematically mine and rebalance data to ensure coverage of \textbf{diverse depressive traits and realistic conversational scenarios} and finally get \textasciitilde 3000 conversations. This curated dataset provides a solid data foundation for modeling the language and cognitive expressions of individuals experiencing depression

            
            % \item \textbf{Symptom Severity:}
            % \begin{itemize}[leftmargin=0.5cm, itemsep=0pt]
            %     \item Feelings of sadness, tearfulness, emptiness, or hopelessness: \textit{Severe}
            %     \item Tiredness and lack of energy: \textit{Moderate}
            %     \item Feelings of worthlessness or guilt: \textit{Severe}
            %     \item Frequent or recurrent thoughts of death, suicidal thoughts, or suicide: \textit{Severe}
            %     \item Becoming withdrawn, negative, or detached: \textit{Severe}
            % \end{itemize}
            
            % \item \textbf{Cognition Distortion Exhibition:}
            % \begin{itemize}[leftmargin=0.5cm, itemsep=0pt]
            %     \item Selective abstraction: \textit{Exhibited}
            %     \item Catastrophic thinking: \textit{Exhibited}
            % \end{itemize}
            
            % \item \textbf{Severity Levels:}
            % \begin{itemize}[leftmargin=0.5cm, itemsep=0pt]
            %     \item Depression severity: \textit{Severe}
            %     \item Suicidal ideation severity: \textit{\\Severe}
            %     \item Homicidal ideation severity: \textit{\\No Homicidal Ideation}
            % \end{itemize}

            % \item \textbf{Name:} \textit{N/A}
            % \item \textbf{Gender:} \textit{Female}
            % \item \textbf{Age:} \textit{25}
            % \item \textbf{Marital Status:} \textit{N/A}
            % \item \textbf{Occupation:} \textit{Unemployed}
            % \item \textbf{Situation of the Client:} \textit{\\The client has lost their job and home, feels worthless, and has turned to alcohol as a coping mechanism. They feel they have hit rock bottom and are contemplating suicide.}
            % \item \textbf{Counseling History:} \textit{\\Over the course of seeking help, the client has become more negative and less hopeful about their situation, feeling that life no longer makes sense given their circumstances. They are not making progress toward finding a job and are not actively trying to change their drinking habits.}
            % \item \textbf{Resistance Toward the Support:} \textit{Medium}

            
% We collect preference data in an online setting. We develop an interactive interface with detailed guideline. 
% In each interaction, the interface will demonstrate a psychological profile sampled from the SFT training set to the expert, and let experts chat with the simulated client which is our model as a mental health supporter. During the conversation, the client model will generate two responses and offer four options to select, "<response 1>", "<response 2>", "<equally good>", "<equally bad>" for question [Which response is more aligned with a real depressed person with the given profile?]. After making a choice, the chosen one will be the client model's response of the current turn. And the expert can send their message to continue the chat. If the experts choose equally, a random response will be chosen as the response of the current turn. Each interaction is required to be engaged with at least 5 minutes. Each expert is required to engaged with three different sampled psychological profiles.

% A total of 317 preference annotations are collected. Among them, 82.0\% indicate a clear preference for one response, suggesting room for further optimization. 16.1\% are labeled as "equally good," while only 1.9\% are "equally bad." These results indicate that after model-based preference training, the model achieves a reasonable level of expert acceptability, with further calibration space to enhance response quality.

% After deleting invalid annotations in 82\%, 250 pre

% \begin{itemize}[leftmargin=0.2cm, itemsep=0pt]
%     \item \textbf{Name:} \textit{Not Specified}
%     \item \textbf{Gender:} \textit{Female}
%     \item \textbf{Age:} \textit{25}
%     \item \textbf{Marital Status:} \textit{Not Specified}
%     \item \textbf{Occupation:} \textit{Unemployed}
%     \item \textbf{Situation of the Client:} \textit{The client has lost their job and home, feels worthless, and has turned to alcohol as a coping mechanism. They feel they have hit rock bottom and are contemplating suicide.}
%     \item \textbf{Counseling Summary:} \textit{Over the course of seeking help, the client has become more negative and less hopeful about their situation, feeling that life no longer makes sense given their circumstances. They are not making progress toward finding a job and are not actively trying to change their drinking habits.}
%     \item \textbf{Resistance Toward the Support:} \textit{Medium}
    
%     \item \textbf{Symptom Severity:}
%     \begin{itemize}[leftmargin=0.5cm,itemsep=0pt]
%         \item Feelings of sadness, tearfulness, emptiness, or hopelessness: \textit{Severe}
%         \item Tiredness and lack of energy: \textit{Moderate}
%         \item Feelings of worthlessness or guilt: \textit{Severe}
%         \item Frequent or recurrent thoughts of death, suicidal thoughts, or suicide: \textit{Severe}
%         \item Becoming withdrawn, negative, or detached: \textit{Severe}
%     \end{itemize}
    
%     \item \textbf{Cognition Distortion Exhibition:}
%     \begin{itemize}[leftmargin=0.5cm,itemsep=0pt]
%         \item Selective abstraction: \textit{Exhibited}
%         \item Catastrophic thinking: \textit{Exhibited}
%     \end{itemize}
    
%     \item \textbf{Severity Levels:}
%     \begin{itemize}[leftmargin=0.5cm,itemsep=0pt]
%         \item Depression severity: \textit{Severe Depression}
%         \item Suicidal ideation severity: \textit{Severe Suicidal Ideation}
%         \item Homicidal ideation severity: \textit{No Homicidal Ideation}
%     \end{itemize}   
% \end{itemize}


% \begin{itemize}[leftmargin=0.2cm, itemsep=0pt]
%     \item \textbf{Name:} \textit{Not Specified}
%     \item \textbf{Gender:} \textit{Female}
%     \item \textbf{Age:} \textit{25}
%     \item \textbf{Marital Status:} \textit{Not Specified}
%     \item \textbf{Occupation:} \textit{Unemployed}
%     \item \textbf{Situation of the Client:} \textit{The client has lost their job and home, feels worthless, and has turned to alcohol as a coping mechanism. They feel they have hit rock bottom and are contemplating suicide.}
%     \item \textbf{Counseling Summary:} \textit{Over the course of seeking help, the client has become more negative and less hopeful about their situation, feeling that life no longer makes sense given their circumstances. They are not making progress toward finding a job and are not actively trying to change their drinking habits.}
%     \item \textbf{Resistance Toward the Support:} \textit{Medium}
    
%     \item \textbf{Symptom Severity:}
%     \begin{itemize}[leftmargin=0.5cm,itemsep=0pt]
%         \item Feelings of sadness, tearfulness, emptiness, or hopelessness: \textit{Severe}
%         \item Angry outbursts, irritability, or frustration, even over small matters: \textit{<corresponding severity>}
%         \item Loss of interest or pleasure in most or all normal activities: \textit{Severe}
%         \item Sleep disturbances, including insomnia or sleeping too much: \textit{<corresponding severity>}
%         \item Tiredness and lack of energy: \textit{Moderate}
%         \item Changes in appetite and weight: \textit{<corresponding severity>}
%         \item Anxiety, agitation, or restlessness: \textit{<corresponding severity>}
%         \item Slowed thinking, speaking, or body movements: \textit{<corresponding severity>}
%         \item Feelings of worthlessness or guilt: \textit{Severe}
%         \item Trouble thinking, concentrating, making decisions, and remembering things: \textit{<corresponding severity>}
%         \item Frequent or recurrent thoughts of death, suicidal thoughts, or suicide: \textit{Severe}
%         \item Unexplained physical problems, such as back pain or headaches: \textit{<corresponding severity>}
%         \item Becoming withdrawn, negative, or detached: \textit{Severe}
%         \item Increased engagement in high-risk activities: \textit{<corresponding severity>}
%         \item Greater impulsivity: \textit{<corresponding severity>}
%         \item Increased use of alcohol or drugs: \textit{<corresponding severity>}
%         \item Isolating from family and friends: \textit{<corresponding severity>}
%         \item Inability to meet responsibilities of work and family: \textit{<corresponding severity>}
%     \end{itemize}
    
%     \item \textbf{Cognition Distortion Exhibition:}
%     \begin{itemize}[leftmargin=0.5cm,itemsep=0pt]
%         \item Selective abstraction: \textit{Exhibited}
%         \item Overgeneralizing: \textit{<corresponding exhibition>}
%         \item Personalization: \textit{<corresponding exhibition>}
%         \item Catastrophic thinking: \textit{Not Specified}
%         \item Minimisation: \textit{<corresponding exhibition>}
%         \item Arbitrary inference: \textit{Not Specified}
%     \end{itemize}
    
%     \item \textbf{Severity Levels:}
%     \begin{itemize}[leftmargin=0.5cm,itemsep=0pt]
%         \item Depression severity: \textit{<depression severity>}
%         \item Suicidal ideation severity: \textit{<suicidal ideation severity>}
%         \item Homicidal ideation severity: \textit{<homicidal ideation severity>}
%     \end{itemize}   
% \end{itemize}




% \begin{table}[]
%     \centering
%     \begin{tabular}{c|c}
%          &  \\
%          & 
%     \end{tabular}
%     \caption{Caption}
%     \label{tab:my_label}
% \end{table}

% Patient-$\psi$ \cite{wang2024patient} and Roleplay-doh \cite{louie2024roleplay} are two representative works of patient simulation in mental health support, which serve as two baselines.
% Patient-$\psi$ leverages a cognitive model framework from CBT in building patient profiles for prompting to achieve better fidelity and accuracy of patient simulation.
% Roleplay-doh defines a series of patient principles to authentically simulate a patient in therapy and develop a principle-adherence prompting pipeline to apply them in prompt-based solutions.
% Both works claim clear superiority of their methods over plain role-play from the GPT4 generation without any prompting strategies. 
% Thus we include them as strong baselines.


% We search public mental health forums, clinical transcripts, academic datasets, social media platforms
% RED\cite{welivita-etal-2023-empathetic} include threads from subreddit "r/depression" and "r/depressed" which are further structured as dialogs. 
% HOPE \cite{10.1145/3488560.3498509} are transcripts from publicly-available pre-recorded counseling videos on YouTube, 
% ESC\cite{liu-etal-2021-towards} is crowdsourcing emotional support conversations,

% AnnoMI-full\cite{9746035} is a dataset of transcripts of therapy session recordings demostrating motivational interviewing skills.

% These datasets provide evidence of qualified source because they are (1) public available conversations (2) related to mental health (3) one party is highly possiblely involving emotion distress though not have to be depression
% We further send all conversations into the next step of labeling and filtering.
% We get 5618 conversations.


we define a conversation is depression-related by in the conversation a depression features (losing interest of activity and depressed mood) are observed or  the existing labels in the source data has indicated the data is depression-related.
% We don't require the conversations where participants have depression disorder diagnosis as these this data is rarely  public and hard to acquire.

% We leverage GPT4 to recognize depression-related conversations from HOPE and AnnoMI-full.
% We use existing labels of ESC to get related conversations. 
% We adopt all conversations of RED.



% We try our best keep a balanced dataset where the depression severity and other detailed traits are balanced. 
% We extract depressive traits from each conversation using GPT4. The depressive traits are defined by our profile, which is introduced in Sec \ref{subsec:role-pley}.
% After extraction, we found the conversations are highly imbalanced in the distribution of symptom demonstrations and depression severity, which may bring role-play bias if we directly tunning on them. Especially moderate and severe depression compose a  larger proportion. Younger people compose a larger proportion. Symptom ""
% We filter the conversations and finally get 3044 conversations.
% Steps taken to balance diversity in depressive experiences (e.g., mild vs. severe depression
% We try have best to keep the traits balancedly demonstrated in the final dataset while not perfect. The trait distribution is demonstrated in Tab 1.




% We recruit 3 experienced mental health professionals to evaluate profile extraction by reading three conversation scripts and the corresponding extracted profile.
% We will ask their opinion on each item (e.g., age, symptoms) shown in the profile about whether the extracted information about this item is correctly reflecting the conversation.
% Among all answers, 80.2\% agree "Yes, it is directly reflected in the conversation" or "Yes, it is a reasonable inference though not directly stated in the conversation". Between these two answers, 50.6\% agree "Yes, it is a reasonable inference though not directly stated in the conversation".



% \subsection{Profile-guided Role-play}\label{subsec:role-play}
% \paragraph{Psychological Profile Construction.} 

% Key components (e.g., depression severity, cognitive distortions, emotional expression, behavioral traits).

% The design of psychological profile needs wisdom of cross-discipline collaboration ability.
% We authors first take the role of designing the preliminary version of the psychological profile.
% Because we as AI researchers know what information and how granular of information could be extracted from the conversation realistically, second we can build a model that is trained based on a preliminary version of profile so that the experts can interact with it and understand  and better imagine how the model will incorporate into their work and help their practice.
% The profile is designed as three parts: demographics, situation, disease-related manifestation. 

% Demographics will include general demographics about the client in the conversation (name, gender, occupation, etc).
% Situation includes current situation of client that brings distress and their willingness of getting support. 
% we also compile disease-related traits into the profile design: depression symptom severity, cognition distortion exhibition, functional impairments and overall depression severity.
% We refer DSM-V\cite{edition2013diagnostic} and sort out 18 depression-related symptoms from the section of Depressive Disorders. They serve as options of Depression Symptom in our profile.  Each symptom is categorized as "not exhibited", "mild", "moderate", "severe".
% We also provide 5 options   (i.e., overgeneralizing, selective abstraction, personalization, arbitrary inference, minimization) according to Aaron T. Beck's theory \cite{clak1999scientific, beck2009depression} about types of negative thought patterns can contribute to depression. Each is recognized as "not exhibited" or "exhibited".
% Functional impairments \cite{ustun2010measuring} are referred were included but finally deleted based on expert feedback. 
% Overall depression severity is categorized as "minimal", "mild", "moderate" and "severe" multi-level catergory similar to some depression inventories (PHQ-9, DBI) would adopt.  
% We use GPT-4 to generate the value for each item in the preliminary version of profile.

% using profile in the system prompt and conversation as context, then we supervised fine-tuning a  model with conversations with the extracted profile of this version. Training details are in the paragraph "instruction-tuning" of this subsection.

% The model is used to interact with experts who can customize and play with a profile and seek their feedback for further improve the profile.



% \paragraph{Expert Profile Refinement} 
% We prototype our model with 10 experts by inviting them to customize a profile by editing the preliminary version of profile items we provided, and chat with the model with their customized profile. 
% we believe this will help the counselors, our users better understand the usage of the profile so that they can futher provide in-context suggestions to improve the current profile or validate the curent design.
% The interaction interface is designed as Figure 2.
% we ask their opinions on the Existing items in the profile. they receive 80\% satisfaction while inprovement is given.

% based on their suggestions, we deleted some confusing or repetitive items like "unwillingness to express feelings" ( confusing with "resistance towards support), "emotion flunctuation" (hard to precise), "fuctional impairment" (all of them overlap with depression symptoms).
% They also suggest some missing important items which we add into the new version: marital status, counseling history, suicidal ideation.
% Though counseling history is a useful item in the pcychological profile, our data is not multi-session, all conversations are single.  
% However, we leverage over-lengthy conversations and seperate them into multi conversations. And summary the previous session as the counseling history of the next session. 
% In our dataset, we have 453 data points that have counseling history.
% An example of the profile is shown in Fig 1.
% We extracted the information required by the new version of profile again on the datasets. 
% And finally we got the profile-rich depression-related conversation dataset for instruction-tuning. 



% \paragraph{Intruction-tuning}
% The pipeline from data to instruction-tuning process is illustrated in Sec \ref{fig:instr}. Profile is used in the system prompt, and the assistant messages correspond to the depression clients' messages. The role of user is equivelant to the role of supporter. The model is trained to predict the utterence of assitant while system prompt and user's message are not count to contribute to prediction loss. 

%
% As illustrated in Fig \ref{fig:dpo}, we did such DPO training iteratively, first on model-based preference data and then on expert annotation data.

% We will optimize policy model (instruction-tuned model in the first round when training on model-based preference data, and the resulted DPO model as the policy model of second round of training on expert annotation data.) via direct preference optimization. The loss is calculated as below:

% \begin{equation}
% \mathcal{L}_{\text{DPO}}(\pi_{\theta}; \pi_{\text{ref}}) = 
% - \mathbb{E}_{(x, y_w, y_l) \sim \mathcal{D}} 
% \left[ \log \sigma \left( 
% \beta \log \frac{\pi_{\theta}(y_w \mid x)}{\pi_{\text{ref}}(y_w \mid x)} 
% - \beta \log \frac{\pi_{\theta}(y_l \mid x)}{\pi_{\text{ref}}(y_l \mid x)}
% \right) \right].
% \end{equation}

% [should denote $y_w, y_l, x$ and $\pi_{\text{ref}}$ model]




% We generate 1,933 preference data samples using a model-based verifier, employing a novel sampling method that slightly noisifies psychological profiles to produce highly contrasting negative responses. 
% Before diving into details, let me provide preliminary:

% We begin preference optimization with our model that is supervised fine-tuned (intruction-tuned) on the profile-guided dialog dataset as reference policy model $\pi_{ref}$, how we trained can be seen from Sec \ref{subsec:role-play}.
% We will optimize policy model via direct preference optimization. The loss is calculated as below:

% \begin{equation}
% \mathcal{L}_{\text{DPO}}(\pi_{\theta}; \pi_{\text{ref}}) = 
% - \mathbb{E}_{(x, y_w, y_l) \sim \mathcal{D}} 
% \left[ \log \sigma \left( 
% \beta \log \frac{\pi_{\theta}(y_w \mid x)}{\pi_{\text{ref}}(y_w \mid x)} 
% - \beta \log \frac{\pi_{\theta}(y_l \mid x)}{\pi_{\text{ref}}(y_l \mid x)}
% \right) \right].
% \end{equation}

% [should denote $y_w, y_l, x$ and $\pi_{\text{ref}}$ model]
% The preference data is a dataset of comparisons $D = \{x^{(i)},y^{(i)}_w ,y^{(i)}_l \}^N_{i=1} $.

% The classical preference generation process usually requires the $\pi_{ref}$ model to produce pairs of answers $(y_w,y_l) \sim \pi_{ref}(x|y)$ from the same source prompt $x$.

% However, it doesn't have to be, like some other works sample pairwise preferences from different input though we have different problem-specific reasons to do that \cite{pi2024strengthening,lu2024step}.
% We utilize a GPT4 as judger to verify determine whether it aligns with the client profile information one item by one item. We observed that after instruction-tuning, the model reach overall good profile-following ability while improvement is still needed to meet accurate use in clinical standards.
% On the on hand, a response averagely complies with 96.0\% information of the profile, which indicate highly profile-following ability.
% On the other side, only 31.7\%  generations  fully (reach 100\%) comply with all the items in the profile.
% This statistics is aligned with we observed from generation sampling. It is always that both of generations are fairly good or subtly/lightly flawed. It is hard to distinguished which is better.
% But we want to further improve the performance.

% Then we come up with an strange but useful idea. We slightly noisifies psychological profiles to produce highly contrasting negative responses.
% We randomly nosified 30\% items in the profiles by substituting the orifinal value with others (e.g., changing symptom "sleep disturbances" from "severe" to "mild" ). And we sample a response $y_n$ from a nosified prompt $\pi_{ref}(y|x_n)$. 

% \begin{equation}
%     y_n \sim  pi_{ref}(y|x_n) \\
%     y_o \sim  pi_{ref}(y|x_o) \\
%     x_n = nosify(x_o)
% \end{equation}

% This will help us to build preferences that have clear distinctions between preferred (highly aligned) and less preferred (slightly deviated) responses.

% However, we have to worry because $y_n$ is not sampling from policy model given original prompt $\pi_{ref}(y|x_o)$. This may bring problems, as $y_n$ can be little possible response from the distribution $\pi_{ref}(y|x_o)$, which may increase variance during training and damage the policy model, even not the generation probability of generating $y_l$ from $\pi_{ref}(y|x_o)$ could be already very low, thus this will serve as an invalid negative response which cannot contribute to optimization.


% Thus we set a selection standards for sampling $y_n$ :
% (1) GPT-4 verifier score of $y_w$ sampled from $\pi_{ref}(y|x_o)$ should higher than $y_n$ sampled from $\pi_{ref}(y|x_n)$ .
% (2) the average token probability ratio between $y_o$  and $y_n$ in the original distribution $logp(y_n|x_o)/|y_n|$ should be less than 2, which means though $y_o$  is not sampled from the original prompt, but its generation probability from original prompt is still close to the real generation from the original prompt.


% We use the training data for instruction-tuning to construct model-based preference dataset. 
% We chunk each conversation evenly into three list of messages. And randomly pick a turn from each of three list. The selected turn will be the turn to generate presponse pair. The messages inclusding system prompt will serve as context prompt $x$. Then we generate $y_o$ with original profile in the system prompt, and generate $y_n$ with a modified system prompt of modified profile.

% We collected 4778 pairwise responses with contexts, we deploy the model to inference the average token probability for both given the original input $x_o$. among them, 2542 meet the criterion (2), that the average token probability ratio between them is less than 2, which indicate their likelihood are comparable in original distribution. Then we verify they profile complying by GPT-4 verifier, we filter out the pairs where $y_0$ didn't receive higher judgement scores. 

% Finally we get 1933 prerences data. We send them to DPO training and futher for expert calibration.

% then conduct semantic analysis to evaluate language authenticity and use a \textbf{judger agent}



% In addition to human assessment, we use an \textbf{interviewer agent} that simulates a counselor’s probing questions to interact with Eeyore and baselines,  to evaluate each chatbot's consistency and adherence to psychological profiles.

%  The interviewer agent follows a structured questioning framework designed to

% \paragraph{Agent Design.} The interviewer agent follows a structured questioning framework.
% Majorly it will ask questions from 6 dimensions , for example to test if a specific symptom is correctly reflected in the role-play, it will ask "Have you been experiencing [SYMPTOM] recently?", "How much does [SYMPTOM] affect your daily life or ability to do things you enjoy?". Full question list is illustrated in Tab 3

% % make the below json into latex table
% {
%     "resistance_to_support": [
%         "How are you feeling today?",
%         "I’m here to support you. Is there something that you would like to talk through or work on together?",
%         "I believe you’re capable of handling this, even if it feels difficult right now. What do you think?",
%         "Have you tried something that’s helped in the past in situations like the one you are facing right now, or would you like to brainstorm some ideas together?",
%         "Do you feel like talking about this is helping, or would you prefer another approach?",
%         "It’s okay if this feels hard to talk about. What do you think might make it easier to open up?",
%         "Do you think working through this with someone could be helpful, or do you prefer to handle things on your own?"
%     ],
%     "symptom_severity": [
%         "Have you been experiencing SYMPTOM recently?",
%         "How often would you say SYMPTOM happens—every day, a few times a week, or only occasionally?",
%         "How much does SYMPTOM affect your daily life or ability to do things you enjoy?",
%         "When you experience SYMPTOM, how severe would you say it is—mild, moderate, or very intense?",
%         "What, if anything, helps when SYMPTOM happens? Have you found ways to manage or reduce it?",
%         "Does SYMPTOM ever affect your relationships with others, like friends, family, or coworkers?"
%     ],
%     "cognitive_distortion": [
%         "When something goes wrong, do you tend to focus only on that and overlook what went well?",
%         "Have you ever thought, 'This always happens to me,' when faced with a setback?",
%         "Do you find yourself feeling responsible for things that might not be entirely your fault?",
%         "When you face challenges, do you imagine the worst possible outcome?",
%         "When you achieve something, do you tend to dismiss it or think it is not a big deal?",
%         "Do you ever feel certain about a negative outcome, even without solid evidence to support it?"
%     ],
%     "depression_severity": [
%         "How have you been feeling emotionally over the past few weeks?",
%         "Do you still enjoy activities that you used to find fun or meaningful?",
%         "How has your energy been lately? Do you feel tired or drained often?",
%         "Do you ever feel guilty, worthless, or overly critical of yourself?",
%         "Have you had any thoughts about death, feeling hopeless, or that things won’t get better?"
%     ],
%     "suicidal_ideation_severity": [
%         "Do you ever think about hurting yourself or ending your life?",
%         "How often do these thoughts come up—daily, weekly, or only sometimes?",
%         "When these thoughts happen, how intense are they—mild, moderate, or severe?",
%         "Have you ever taken any steps towards acting on these thoughts, or have you made a plan?",
%         "Is there anything that helps you feel better or more grounded when you have these thoughts?"
%     ],
%     "homicidal_ideation_severity": [
%         "Have you been feeling unusually angry or frustrated with others recently?",
%         "Do you ever feel like hurting someone when you’re upset or overwhelmed?",
%         "When you experience these feelings, how intense do they become—mild, moderate, or severe?",
%         "How often do these thoughts occur—daily, weekly, or occasionally?",
%         "What do you typically do when these thoughts arise? Have you found ways to calm yourself or manage them?",
%         "Do these thoughts ever feel out of your control, or do you feel like you can manage them?"
%     ]
% }


% Then a judger agent will be required to read the interviewing script and give answer to six types of yes/no questions, for example, for judging a symptom, the  question is "Based on the patient’s responses, does the patient consistently acknowledge or admit to the symptom selected for symptom severity in the system prompt? Please answer the question above with a single Yes or No.". See full list in Table 3.

% We let interviewer agent interact with the model with 5 different profiles picked from the test set of profile. Finally we will sum the number of yes as the  scores the model receives.


% \section{Experiment}
% We evaluate \textbf{Eeyore} using both expert and automatic evaluation, comparing its performance to state-of-the-art baselines for patient simulation in mental health support. All evaluations are conducted in an \textbf{online testing setting}, ensuring real-time interaction between evaluators and chatbots.

% \subsection{Unseen Evaluation Profiles}
% To assess model performance across multiple dimensions, we retain 12 real-world conversations from the training set and extract \textbf{12 unseen psychological profiles} from them as evaluation seeds  . The test set covers diverse client backgrounds, with 4 cases each of severe, moderate, and mild depression. These profiles are used for both human expert evaluation and automatic evaluation.

% \subsection{Baselines}  
% We compare \textbf{Eeyore} against two representative patient simulation approaches:  \textbf{Patient-$\psi$} \cite{wang2024patient}, which constructs a structured \textit{cognitive model} based on CBT to characterize patient traits from conversational data and then augments simulation using this model.  \textbf{Roleplay-doh} \cite{louie2024roleplay}, which employs principle-adherence pipeline at each turn to ensure consistent and behaviorally accurate patient role-play.  Both baselines have demonstrated superior performance over generic GPT-4 role-playing.  

% To ensure a fair comparison, we implement the baselines as follows: 
% For \textbf{Patient-$\psi$}, we use their provided script to extract a \textit{cognitive model} from the real-world conversations that are associated with the evaluation profiles. During testing, we provide both the assigned evaluation profile and the extracted cognitive model in the system prompt. 
% For \textbf{Roleplay-doh}, we apply its principle-adherence pipeline for turn-by-turn generation while explicitly setting the evaluation profile in the system prompt.  
% This ensures that all models—including \textbf{Eeyore}—receive the same psychological profile information, allowing a \textbf{fair comparison} in evaluating profile adherence.  

% \subsection{Model Training and Inference Details}

% \subsubsection{Training Procedure}  
% We fine-tune \textbf{Eeyore} starting from the \textbf{LLaMA 3.1-8B-Instruct} model. The training process consists of:

% \textbf{Instruction-Tuning:} The model is trained for \textbf{two epochs} to ensure adaptation to profile-guided role-play while avoiding overfitting.

% \textbf{Direct Preference Optimization (DPO):} We apply DPO in two stages—first using model-generated preference data, then refining with expert annotations. Since preference accuracy reaches 100 percent after one epoch, we limit DPO training to **one epoch per stage** to prevent overfitting.

% \subsubsection{Inference Configuration}  
% We use hyperparameter settings reported in baseline works:  
% \textbf{Roleplay-doh} uses Temperature set to 0.7 and Top-p set to 1.0.  
% \textbf{Patient-$\psi$} uses Temperature set to 1.0 and Top-p set to 1.0.  
% For \textbf{Eeyore}, we set Temperature to 1.0 and Top-p to 0.8.  

% To mitigate premature \texttt{[EOS]} token generation, we apply:

% \textbf{SequenceBiasLogitsProcessor} with a **negative bias of -4.0**, discouraging early \texttt{[EOS]} token generation.  

% \textbf{ExponentialDecayLengthPenalty} with a decay factor of 1.01, gradually increasing the probability of \texttt{[EOS]} as conversation length increases.  

% These adjustments ensure smoother, more naturalistic conversations without abrupt cutoffs. Both processors are applied before the Top-p processor.





% \subsubsection{Human Expert Evaluation}

% To assess authenticity and psychological profile adherence, we conduct a human evaluation study where professional counselors and advanced psychology students interact with \textbf{Eeyore} and baseline models in real time. 

% \paragraph{Procedure.}  
% Each evaluator engages in multiple chat sessions with different models and rates them based on their alignment with real-world depressed individuals. The evaluation is conducted using an interactive annotation interface (see Fig. \ref{fig:expert_eva} for the chat interface).  

% \paragraph{Scoring Dimensions.}  
% Evaluators assess the models across five dimensions using 5-point Likert scale: \textbf{Contrast with AI-Like Responses}: "The chatbot avoids AI-like tendencies such as overly detailed or polished responses. Instead, it responds concisely, colloquially, and naturally, providing information progressively rather than all at once." \textbf{Linguistic Authenticity}: "The chatbot’s wording, phrasing, and tone closely match how individuals with depression speak." \textbf{Cognitive Pattern Authenticity}: "The chatbot realistically reflects depressive thought patterns like selective abstraction and overgeneralization without exaggeration." \textbf{Subtle Emotional Expression}: "The chatbot conveys depressive emotions realistically—neither overly dramatic nor emotionally flat." \textbf{Profile Adherence and Personalization}: "The chatbot accurately reflects the assigned psychological profile, including situation, symptom severity, and other relevances, without inconsistencies."
% Among these, the first four dimensions collectively evaluate the authenticity of the chatbot’s responses, while the final dimension assesses its ability to \textbf{adhere to assigned psychological profiles}.
% Statistical comparisons were conducted using the Wilcoxon signed-rank test. 

% \paragraph{Results.}
% [TBD]

% In addition to human assessment, we use an \textbf{interviewer agent} that simulates a counselor’s probing questions when interacting with Eeyore and the baselines to evaluate chatbot consistency and adherence to psychological profiles.


% \paragraph{Agent Design.} The interviewer agent follows a structured questioning framework to get evaluation on three disease-related dimensions in the profile, symptom severity, cognitive distortion, depression severity. For example, to assess whether a symptom is correctly reflected in the role-play, it asks: \textit{“Have you been experiencing [SYMPTOM] recently?”} or  
% \textit{“How much does [SYMPTOM] affect your daily life or ability to do things you enjoy?”} The full list of interview questions is provided in Tables \ref{tab:interview1} and \ref{tab:interview2}.
% Following the interview, the agent evaluates the chatbot’s responses and assigns 5-point Likert score to the predefined questions. 5 indicates fully alignment.  For example, in assessing symptom alignment, the agent answers:  
% \textit{“How subtly and consistently does the client reflect a mild level of symptoms in their responses?”}  
% The interviewer agent interacts with the model using evaluation profiles.
% As most of the answers are 5, which means fully alignment. 
% We compare three models based on the average rating and the percentage of score that  doesn't reach full alignment on each dimension.


% The interviewer agent follows a structured questioning framework designed to assess chatbot consistency across three clinically relevant dimensions: \textbf{Symptom Severity}: Evaluates whether the chatbot's responses appropriately reflect mild, moderate, or severe symptoms as specified in the profile. \textbf{Cognitive Distortion}: Assesses the chatbot’s ability to exhibit realistic cognitive distortions associated with depression.
% \textbf{Overall Depression Severity}: Determines whether the chatbot's responses align with the assigned level of depression severity.
% For each dimension, the agent asks targeted interview questions, for example, to assess whether a symptom is correctly reflected in the role-play, it asks: \textit{“Have you been experiencing [SYMPTOM] recently?”} or  
% \textit{“How much does [SYMPTOM] affect your daily life or ability to do things you enjoy?”} (see Tables \ref{tab:interview1} and \ref{tab:interview2}).


% After interviewing, The agent is required to read the interview script, and evaluate the alignment dased on the dimension, for example the agent answers \textit{“How subtly and consistently does the client reflect a mild level of symptoms in their responses?”} for evaluating a symptom  adherence and assigns a 5-point Likert score to rate alignment quality, where 5 indicates full alignment to the profile.

% in the first phase, we optimize the instruction-tuned model as $\pi_{\text{ref}}$ using model-generated preference data to get a DPO model as $\pi_{\theta}$. In the second phase, we further refine the resulting DPO model as $\pi_{\text{ref}}$ using expert-annotated preferences to get the final preference-optimized model as $\pi_{\theta}$.

% The policy model $\pi_{\theta}$ is optimized relative to a reference model $\pi_{\text{ref}}$ using the following DPO loss:


% During inference, for baselines we use hyperparameter settings aligned with prior works. 
% Details of our and baselinse's ifnerence setting is available in Appendix~\ref{sec:infercence}.
% It is hard to justify all the parameters we used as a formal evaluation cost is very high. These choices are based on the authors themselves' sanity check.

 % in online evaluations. %Our findings emphasize the need for careful optimization to achieve more realistic LLM-driven simulations.
% Despite its smaller model size, the Eeyore depression simulation outperforms GPT-4o while using SOTA prompting strategies, both in linguistic authenticity and profile adherence. 