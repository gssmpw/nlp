\section{Background and Related Work}
\label{related_work}

% \subsection{Cross-Chain Asset Swap and Transfer}

\noindent
\textbf{Cross-Chain Asset Swap and Transfer.}
Blockchain interoperability has gained significant attention. 
% The most extensively researched areas are cross-chain asset swaps and transfers
One widely studied area is cross-chain asset swap and transfer~\cite{2019atomicBEswap,Xu2021htlc,Luo2024crosschannel,Manevich2022ccas,tian2021enabling,herlihy2018atomic,deshpande2020privacy,thyagarajan2022universal,chen2024pacdam,yin2022sidechain,zamyatin2019xclaim}. 
Cross-chain asset swap protocols enable untrusted parties to exchange assets across different blockchain networks, while cross-chain asset transfer protocols involve locking or burning assets on the source chain and creating their equivalents on the target chain. 
% However, these protocols are limited to assets and are not applicable to smart contracts. 
While these protocols provide atomicity guarantees during cross-chain asset swaps or transfers, their protocols cannot be applied to our target scenario of CCSCI.
% Therefore, we propose the \texttt{IntegrateX} protocol, which considers the atomic and efficient cross-chain invocation of smart contracts.

% \subsection{Interoperability Protocol for General Communication}

\vspace{3pt}
\noindent
\textbf{Interoperability Protocol for General Communication.}
Besides cross-chain asset transfers and swaps, some other efforts aim to provide interoperability protocols for more general cross-chain communication~\cite{nissl2021towards,wood2016polkadot,cosmos2019,abebe2019enabling,darshan2023an,reigsbergen2023demo,ghosh2021leveraging,garoffolo2020zendoo}, enabling the transfer of general data beyond assets between blockchains. 
However, the primary limitation of these protocols is that they typically guarantee atomicity for, at most, a single cross-chain step between two blockchains. 
For general cross-chain dApps, which often require multiple rounds of cross-chain interactions across multiple chains, these protocols fail to ensure atomicity for the entire cross-chain dApp. 
In contrast to these approaches, \texttt{IntegrateX} \emph{guarantees overall atomicity of the entire multi-round CCSCI for general cross-chain dApps.}

% However, some of these protocols (e.g., IBC)~\cite{nissl2021towards,wood2016polkadot,cosmos2019,abebe2019enabling,darshan2023an,reigsbergen2023demo,ghosh2021leveraging,garoffolo2020zendoo} do not explicitly provide atomicity guarantees for cross-chain transactions. 
% Atomicity of transactions needs to be realized by extending and building application-specific protocols based on their protocols.
% Some of the other protocols ~\cite{ye2020bitxhub} can only guarantee the atomicity of a single cross-chain step between two blockchains. 
% For complex cross-chain applications that often require multiple rounds of cross-chain interactions across multiple chains, these protocols cannot guarantee the atomicity of the entire cross-chain application.
% Unlike the above works, \texttt{IntegrateX} is able to \emph{guarantee the overall atomicity of the entire multi-round CCSCI for cross-chain applications.}

% However, these approaches either do not explicitly address the atomicity of cross-chain smart contract invocations, or only support the atomicity of single-step cross-chain calls, rendering them unsuitable for complex cross-chain applications. 
% Consequently, such efforts cannot ensure the overall atomicity of multiple smart contract invocations. 
% In the \texttt{IntegrateX} system, we further focus on complex cross-chain smart contract invocation scenarios involving multi-step smart contract calls.

% Nissl et al.~\cite{nissl2021towards} proposed a cross-blockchain smart contract framework that allows parameter passing, return values, scalability, and recursive calls. However, it lacks atomic operations and support for multiple contracts.
% Wood~\cite{wood2016polkadot} introduced Polkadot, a scalable blockchain platform enabling message and value transfers between blockchains. However, it focuses on single-step interactions and lacks mechanisms for atomic smart contract invocations across blockchains. Similarly, Cosmos~\cite{cosmos2019} uses the Inter-Blockchain Communication (IBC) protocol for interoperability, supporting seamless token and data transfers, but it does not support back-and-forth interactions between smart contracts on different chains within a single transaction.
% Furthermore, BitXHub's cross-chain project~\cite{ye2020bitxhub} focuses on establishing a universal and reliable interoperability protocol to facilitate seamless interactions and information transfers between diverse blockchain networks. 
% Although these projects enable cross-chain data transmission and reliable single-step interactions, they lack specific mechanisms to ensure the atomicity and efficiency of complex cross-chain applications involving multiple smart contracts.

% \subsection{Complex Cross-Chain Contract Invocation}

\vspace{3pt}
\noindent
\textbf{CCSCI with Overall Atomicity.}
% In addition to the aforementioned work, some efforts have been made to address complex cross-chain smart contract invocations (CCSCI) while trying to ensure overall atomicity
There are some studies that attempt to ensure overall atomicity for the cross-chain dApp with complex CCSCI~\cite{liu2021hyperservice,tao2023atomicity,weterkamp2023instant,hu2024ivyredaction,Multi-Chain-Atomic-Commits,robinson2021general,atomic-ibc,chen2024atomci}. 
However, these approaches have various limitations.
For example, HyperService~\cite{liu2021hyperservice} only can provide weak atomicity, named financial atomicity, via an insurance mechanism. 
It fails to provide strong execution atomicity guarantees during CCSCI.
Some works like Heterogeneous Paxos~\cite{Multi-Chain-Atomic-Commits} provide overall atomicity for CCSCI. 
However, they usually require the underlying blockchains to make targeted modifications to fit their protocols, limiting their generality.
Some other works, like Unity~\cite{tao2023atomicity}, also ensure atomicity for a whole CCSCI process.
However, their protocols only apply to cross-chain scenarios between two blockchains and does not take into account the more general multi-chain case.
% , such as only guaranteeing the atomicity of the final result~\cite{liu2021hyperservice}, requiring modifications to the underlying blockchain~\cite{Multi-Chain-Atomic-Commits}, or supporting only specific application scenarios~\cite{tao2023atomicity,weterkamp2023instant,hu2024ivyredaction}. A few approaches have successfully achieved atomicity for complex CCSCI~\cite{robinson2021general,atomic-ibc,chen2024atomci}, but they suffer from efficiency issues due to the need to lock contracts for extended periods. Therefore, in the \texttt{IntegrateX} system, our aim is to efficiently and atomically implement complex cross-chain smart contract invocations without modifying the underlying blockchain.

% HyperService~\cite{liu2021hyperservice} is the first platform to provide both interoperability and programmability across heterogeneous blockchains, enabling the development of cross-chain applications through a unified programming framework and a secure cryptography protocol. 
% However, while HyperService provides financial atomicity through an insurance mechanism, it does not ensure the atomicity of application execution.
% Sheff~\cite{Multi-Chain-Atomic-Commits} proposed the Heterogeneous Paxos protocol, which enables atomic transactions across multiple blockchains by using overlapping validator sets and synchronously committing cross-chain transactions in a single consensus round. However, this approach faces challenges related to implementation complexity, potential performance issues, and the need to modify existing blockchain consensus protocols.
% Unity~\cite{tao2023atomicity} introduces an innovative mechanism that guarantees atomicity and confidentiality of cross-chain transactions, even in case of failures. However, it is limited to decentralized applications between two chains and is not suitable for more complex cross-chain scenarios.

% Some works have achieved atomic execution of complex cross-chain smart contract invocations across multiple blockchains. However, due to the need for long-term locking of smart contracts, these approaches result in low invocation efficiency.
There are few other solutions that guarantee overall atomicity for the cross-chain dApp with complex CCSCI. 
However, they generally suffer from inefficiency ~\cite{robinson2021general,atomic-ibc,chen2024atomci}. 
A complex cross-chain dApp typically involves multiple blockchains and several interdependent application logic components (implemented through smart contracts).
In these works, handling such complexity requires multiple rounds of cross-chain execution and cross-chain interaction in sequential order across several involved chains. 
Specifically, this involves locking relevant states, executing the logic fragment, reaching consensus, transferring intermediate states to the next blockchain responsible for executing the subsequent logic fragment, and finally unlocking and updating the states. 
% The main reason is that, these approaches usually require multiple rounds of cross-chain execution and cross-chain information transfer when handling a cross-chain application, since a cross-chain application usually involves interdependent execution logic distributed over multiple blockchains.
As a result, they typically suffer from high latency and poor concurrency handling CCSCI.
Additionally, Atomic IBC \cite{atomic-ibc} relies on the Cosmos Hub to provide security guarantees for its protocol. 
This approach is not easily adaptable to other ecosystems (e.g., EVM-compatible blockchains) and may introduce security bottlenecks.
% Robinson et al.~\cite{robinson2021general} introduced GPACT to achieve atomic cross-chain transactions for Ethereum-based blockchains, but it has several limitations.
% Firstly, it locks entire contracts during updates, reducing availability and performance.
% Additionally, during the execution of cross-chain applications, GPACT requires sequentially waiting for each smart contract call to be executed on its respective blockchain and for the results to be transferred across chains, resulting in significant overhead and high latency.
% Atomic IBC~\cite{atomic-ibc} aims to achieve atomicity and low latency in cross-chain applications.
% It provides an efficient and reliable solution for multi-chain interoperability through shared validator sets and atomic transaction bundles.
% By ensuring cross-chain transactions either fully succeed or rollback, and keeping each chain's state unchanged during execution, it optimizes cross-chain communication speed and security, enhancing both developer and user experience.
% However, Atomic IBC faces performance bottlenecks because validators need to join a common engine to atomically execute cross-chain transactions. 
% This centralized execution mechanism can limit the system's scalability and concurrency, potentially becoming a performance bottleneck, especially under high transaction volumes.

Unlike these aforementioned works, \texttt{IntegrateX} is able to maintain high efficiency during CCSCI while ensuring overall atomicity. 
Moreover, it can be deployed to blockchains with the same execution environment (e.g., EVM-compatible) without requiring modifications to the underlying blockchain.

% In the \texttt{IntegrateX} system, we improve the execution efficiency of complex cross-chain smart contract invocations by migrating smart contract logic and utilizing an integrated execution mechanism, thereby avoiding long-term locking of smart contract states. At the same time, we ensure the atomicity of the overall invocation. Most importantly, this protocol does not require modifications to the underlying blockchain protocols or validators to join a shared mempool and blocks, ensuring simplicity of implementation and scalability.

% \subsection{Smart Contract Portability}

\vspace{3pt}
\noindent
\textbf{Smart Contract Portability.}
Another type of approach, known as smart contract portability~\cite{westerkamp2019verifiable,fynn2020smom,westerkamp2022smartsync}, involves frequently moving or replicating entire smart contracts with all the states associated across different blockchains. 
This allows cross-chain invocations to be converted into intra-chain invocations during contract execution. 
However, the practicality of this approach is often questionable. 
The choice of which blockchain to deploy and run a smart contract and dApp typically involves several considerations, including security, ecosystem, and business concern. 
Moving an entire smart contract from one ecosystem to another is often not favored by developers, as it may compromise security or disrupt the completeness of the ecosystem.
Besides, smart contracts often manage a significant amount of user state, and frequently relocating all of this state across chains incurs substantial overhead. 
This inefficiency further reduces the practicality of these approaches.

The philosophy of \texttt{IntegrateX} is fundamentally different from smart contract portability. 
In \texttt{IntegrateX}, the states involved in each smart contract are still maintained and updated by their original blockchains. 
This allows dApps to continue benefiting from the security and ecosystem of their original blockchain for most of the time. 
During the CCSCI process, we only temporarily lock part of the state relevant to the invocation and transfer it to a single blockchain, where the logic is integrated and executed using that blockchain's execution environment.

% In the \texttt{IntegrateX} system, we migrate smart contract logic so that the logic for complex cross-chain smart contract invocations can be entirely stored on the same chain. Some existing works have also proposed the portability of smart contracts~\cite{westerkamp2019verifiable,fynn2020smom,westerkamp2022smartsync}. However, these works primarily focus on migrating or synchronizing the state of a smart contract from the original chain to another chain, transferring the smart contract itself without involving cross-chain invocations. In \texttt{IntegrateX}, we only migrate the logic of the smart contract, and the migrated contract is used for subsequent complex cross-chain smart contract invocations.