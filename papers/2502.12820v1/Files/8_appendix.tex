\appendix

\section{Logic-state Decoupling Example}
\label{codeex}

We give a sample of logic-state decoupling. In the following code~\ref{ex}, after applying logic-state decoupling, the original Hotel contract is split into two separate contracts: the logic contract \texttt{LHotel} and the state contract \texttt{SHotel}. The \texttt{LHotel} contract contains no state variables and only includes the \texttt{book()} function, which implements the hotel reservation functionality. Since there are no variables within the \texttt{LHotel} contract, the \texttt{book()} function must take all necessary parameters as inputs.

On the other hand, the \texttt{SHotel} contract retains all the variables from the original Hotel contract and introduces an additional address variable, \texttt{addr\_lhotel}, which records the address of the \texttt{LHotel} contract. In the \texttt{book()} function of the \texttt{SHotel} contract, no reservation logic is implemented directly; instead, it calls the \texttt{book()} function from \texttt{LHotel} using the \texttt{addr\_lhotel} parameter to execute the hotel reservation functionality.

By decoupling the logic and state in this way, only the \texttt{LHotel} contract needs to be cloned. 
% during logic updates. 
Since \texttt{LHotel} contains no state variables, this approach significantly reduces gas consumption during the cross-chain clone and deployment process.

\begin{lstlisting}[language=Solidity, caption={Pseudocode of Hotel Logic-State Decoupling}, label={ex}]
contract Hotel{
    int256 price;
    int256 remain;
    mapping (address => int256) accounts;
    function getPrice() public view returns(uint256); 
    function getRemain() public view returns(uint256); 
    function book(address user_addr, uint256 num) public returns(uint256); 
    function LockState(bytes[] memory args) public returns();
    function UpdateSteate(bytes[] memory args) public returns();

contract LHotel{
    function book(uint256 price, uint256 remain, uint256 num) public returns(uint256)
}

contract SHotel{
    int256 price;
    int256 remain;
    address addr_lhotel;
    mapping (address => int256) accounts;
    function getPrice() public view returns(uint256); 
    function getRemain() public view returns(uint256); 
    function LockState(bytes[] memory args) public returns();
    function UpdateSteate(bytes[] memory args) public returns();
    function book(address user_addr, uint256 num) public returns(uint256);

\end{lstlisting}
