\section{Discussions and Limitations}\label{disscussion}

\noindent
\textbf{Potential Application Scenarios.} 
Beyond the example of the train-and-hotel problem mentioned in this paper, \texttt{IntegrateX} can be extended to a wide range of application scenarios. 
One promising use case is cross-chain flash loans \cite{tefagh2021ccfl}. 
Flash loans are atomic, uncollateralized lending protocols that allow users to borrow funds at nearly zero cost, perform other operations, and then repay the loan. 
However, these processes require the guarantee of overall atomicity, meaning that either all steps succeed or they all fail. 
Due to this requirement for atomicity, existing flash loan protocols are limited to intra-chain operations. 
With \texttt{IntegrateX}, which provides overall atomicity for cross-chain dApps, users can efficiently perform cross-chain flash loan operations.
Other promising application scenarios include, but are not limited to, cross-chain atomic arbitrage and cross-chain supply chain management.

\vspace{3pt}
\noindent
\textbf{Learning Cost for Developers.} 
The logic-state decoupling and fine-grained state lock mechanisms may slightly increase the development learning curve for smart contract developers. Fortunately, we have proposed a set of guidelines to assist developers, and the mechanisms are flexible (as discussed in Section \ref{subsec:LSD} and \ref{subsec:lock}), allowing developers to freely decide whether to implement them. Furthermore, we can provide formal documentation, SDKs, and other resources (following existing standards such as Wormhole \cite{wormhole} and IBC \cite{cosmos2019}) to guide developers in secure and efficient development and auditing, thereby reducing the learning curve. Additionally, incentive mechanisms (e.g., token rewards) are widely adopted in the industry to encourage developers to build and utilize our system. In future research, we could even explore AI-based semi-automated smart contract tools to further address this challenge.

Moreover, logic-state decoupling offers several additional benefits. For instance, it facilitates modular programming principles in smart contract development, which helps reduce subsequent upgrade and maintenance costs while improving contract security. When an issue arises in a specific contract module, developers can conduct targeted audits and resolve vulnerabilities efficiently. 


\vspace{3pt}
\noindent
\textbf{Support for Heterogeneous Chains.} 
\texttt{IntegrateX} currently supports blockchains that run different consensus protocols but share the same smart contract execution environment. 
As mentioned in the paper, to ensure cross-chain transaction security across blockchains with different consensus protocols, \texttt{IntegrateX} waits until consensus on the source chain is finalized (or highly likely to be finalized) before committing the cross-chain transaction to the target chain. 
Additionally, while this paper focuses on \texttt{IntegrateX}’s implementation on EVM-compatible blockchains, it can also be modified to operate between non-EVM-compatible blockchains that share the same smart contract execution environment.

However, a current limitation of \texttt{IntegrateX} is that it cannot operate between blockchains with different smart contract execution environments. 
Fortunately, this issue could potentially be addressed using advanced techniques such as code virtualization \cite{virtualization}. 
Expanding \texttt{IntegrateX} to support integrated execution across blockchains with different smart contract environments is a future research direction we aim to explore.

\vspace{3pt}
\noindent
\textbf{Trade-off Between Flexibility and Load Balancing.} 
In \texttt{IntegrateX}, cross-chain dApp providers have the flexibility to select any chain as the execution chain for integrated execution, based on their preferences. 
However, this flexibility may introduce a potential issue: if many cross-chain dApp providers choose the same chain as the execution chain, that chain could become a hotspot, potentially degrading its performance.
One possible solution is for a third party (e.g., \texttt{IntegrateX}) to manage load balancing by selecting the execution chain on behalf of the cross-chain dApp providers. 
However, this approach could introduce centralization risks. 
Additionally, as discussed in the paper, developers' choice of which chain to run their dApps on often involves considerations beyond performance, such as ecosystem compatibility and business partnerships.
How to better achieve load balancing and how to strike a trade-off between performance and flexibility are important questions that warrant future research.

\vspace{3pt}
\noindent
\textbf{Mitigating Malicious Application Layer Components.}
In public blockchain scenarios, there are common strategies to mitigate malicious behavior from application layer components (e.g., dApp providers, users). 
For instance, a malicious cross-chain dApp provider might attempt to maliciously lock certain states to prevent their usage by others. 
Such behavior can be countered using contract-based authorization or blacklisting mechanisms (widely used in existing dApp development \cite{etherscan}). 
For example, an intra-chain dApp provider can pre-arrange with a cross-chain dApp provider and authorize trusted cross-chain dApp providers (through their associated addresses) in their contracts, allowing only authorized cross-chain dApp providers to invoke and lock their states. 
Similarly, intra-chain dApp providers can blacklist specific cross-chain dApp providers in their contracts to block their interactions.
Additionally, gas fee mechanisms can serve as a deterrent to malicious application layer components attempting to launch flooding attacks against the blockchain.

\vspace{3pt}
\noindent
\textbf{Cross-Chain vs. Cross-Shard.}
Some existing works have explored the issue of cross-shard smart contract handling \cite{qi2024lightcross, li2022jenga}. 
However, cross-chain and cross-shard scenarios are fundamentally different, and their solutions cannot be directly applied to cross-chain contexts. 
The primary reasons are as follows: First, research on blockchain sharding typically involves modifications to the underlying system. 
In contrast, a key requirement of cross-chain protocols or systems is that they must not require modifications to the underlying blockchains, ensuring better compatibility with existing blockchain systems. 
Second, a blockchain sharding system usually has a beacon chain responsible for coordinating progress across shards, which is impractical in cross-chain protocols. 
After all, in cross-chain scenarios, each blockchain essentially belongs to a different system. 
Finally, in cross-chain contexts, blockchains are often heterogeneous (e.g., different consensus protocols), which is uncommon in blockchain sharding systems.

\vspace{3pt}
\noindent
\textbf{Inter-Chain Shared Security.}
In \texttt{IntegrateX}, each blockchain is assumed to have a proportion of malicious nodes lower than its fault tolerance threshold (i.e., they are secure). 
This is a widely accepted assumption in most existing works. 
However, recent research has begun exploring how to achieve secure cross-chain interoperability protocols in scenarios where individual blockchains may not be secure, by sharing security across multiple blockchains \cite{sheng2023trustboost}.
\texttt{IntegrateX} could adopt similar ideas through modifications to achieve shared security among blockchains. 
However, how to design and implement inter-chain shared security within \texttt{IntegrateX} while still maintaining efficiency is an important direction for future research.
