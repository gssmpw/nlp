\section{Implementation and Evaluation}\label{evaluation}

% In this section, we implement \texttt{IntegrateX} protocol with local servers and conduct comprehensive tests to evaluate the performance of the protocols in terms of run time, latency, and gas consumption.

\subsection{Implementation}
% The \texttt{IntegrateX} system incorporates on-chain components on both the invoked chain and the execution chain. 
We implement a prototype of \texttt{IntegrateX} based on a cross-chain communication project Bitxhub \cite{ye2020bitxhub} (its transport layer is similar to IBC). 
The relayers are implemented in Golang, while the bridging contracts are implemented using Solidity.
In addition, we implement several EVM-compatible blockchains in Golang leveraging go-ethereum clients.
For performance comparison, we also implement a baseline CCSCI protocol GPACT~\cite{robinson2021general}. 
As mentioned in the paper, GPACT is one of the most advanced existing protocols that can guarantee overall atomicity for complex CCSCI. 
However, GPACT has limitations in efficiency, as it requires multiple rounds of cross-chain execution and message exchange.
Furthermore, we use Solidity to implement the Train-and-Hotel example mentioned in the paper, along with smart contracts in various other scenarios, to demonstrate the \texttt{IntegrateX}'s performance in real use cases and under different conditions.

% utilized the open-source project Bitxhub \cite{ye2020bitxhub} to implement the cross-chain communication component of the \texttt{IntegrateX} system. 
% % In BitXHub, cross-chain information is transmitted by the relayer through the cross-chain light client. 
% We implemented the light client in Golang. We implemented bridging contracts and various dApp applications using Solidity, including an application that increases call depth, the train-hotel application, and others. Additionally, we implemented versions of the aforementioned applications under GPACT~\cite{robinson2021general} using Solidity. In GPACT, CCSCI requires multiple rounds of execution. By comparing the latency and gas consumption of the IntegrateX system and GPACT under CCSCI, we can better demonstrate the efficiency of the IntegrateX system in CCSCI. 

% \subsection{Experimental Setup}

We test the performance of \texttt{IntegrateX} on multiple servers.
Each server is equipped with an Intel® Core(TM) i5-10400F CPU @ 2.90 GHz and 15.5 GB RAM running 64-bit Ubuntu 22.04 LTS. 
% We build the underlying EVM-compatible blockchain in Golang. 
We deploy 4 relayers via Docker containers on multiple servers.
We also deploy 3 blockchains. 
Each blockchain contains 4 nodes and uses the Proof-of-Authority consensus algorithm. 
The block time is set to be 5 seconds. 
% We deploy serveral Docker containers on multiple servers to run the system, including 4 relayers and 3 cross-chain light clients. 
Additionally, the gas calculation method in the experiments is the same as that used in Ethereum. 
For every experiment, we select Blockchain 1 (bc1) as the execution chain in the \texttt{IntegrateX} system, with the other chains as invoked chains.
% Detailed descriptions of the experimental setup will be provided before presenting the performance evaluation results.

\begin{figure}[t]
    \centering
    % 第一个子图
    \begin{subfigure}[b]{0.24\textwidth}
        \centering
        \includegraphics[width=\textwidth]{Figures/integratex_t.png}  % 使用你的图片路径
        \caption{Latency}
        \label{ixtime}
    \end{subfigure}
    % \hspace{3pt}
    \hfill
     % 第二个子图
    \begin{subfigure}[b]{0.24\textwidth}
        \centering
        \includegraphics[width=\textwidth]{Figures/integratex_g.png}  % 使用你的图片路径
        \caption{Gas consumption}
        \label{ixgas}  % 子图标签,用于引用
    \end{subfigure}
    \vspace{-18pt}
    \caption{Latency and gas consumption in the Train-and-Hotel use case.}
    % TA: transaction aggregation.}
    \label{integratex}  % 整体图的标签
 
\end{figure}
\subsection{Experimental Results}
\subsubsection{Performance under Train-and-Hotel Use Case}
% In this experiment, 
We now compare the latency and gas consumption during the Train-and-Hotel CCSCI process for \texttt{IntegrateX} and GPACT. 
% The results are shown in Figure~\ref{integratex}.

As shown in Figure~\ref{ixtime}, \texttt{IntegrateX} decreases total latency by 51.2\% compared to GPACT. 
On bc1 (execution chain), the latency of \texttt{IntegrateX} is reduced by 46.2\% compared to GPACT.
The reason is that, due to the integrated execution, \texttt{IntegrateX} could integrate and execute all the related execution logic within one block, avoiding multiple rounds of cross-chain execution during CCSCI.
On bc2, \texttt{IntegrateX} also reduces the latency by 60\% compared to GPACT.
% This is because that the Train Contract is invoked multiple times, 
This is because the train contract is invoked multiple times, requiring multiple rounds of state transfers in GPACT, whereas in \texttt{IntegrateX}, only a single round of state transfer is needed.
% Train-and-Hotel use case invokes the Train Contract state on bc2 multiple times, the transaction aggregation mechanism in \texttt{IntegrateX} locks all the required states in a single transaction, further enhancing efficiency. 
% Overall, in the Train-and-Hotel use case, compared to GPACT, \texttt{IntegrateX} achieves a 51.2\% improvement in efficiency with nearly the same gas consumption.

The results in Figure \ref{ixgas} show that the total gas consumption of \texttt{IntegrateX} and GPACT is nearly identical in the Train-and-Hotel use case. 
On bc1, the gas consumption in \texttt{IntegrateX} is larger than GPACT (41.9\%).
This is mainly because in \texttt{IntegrateX}, the execution chain (bc1) handles all the logic during CCSCI, leading to a higher gas.
However, on the invoked chains (bc2 and bc3), \texttt{IntegrateX} achieves lower gas cost compared to GPACT (35.4\%, 35.2\%, respectively).
The main reason is that in \texttt{IntegrateX}, the invoked chains only need to lock and update states without executing any logic. 

% The gas consumption on bc1 increases in \texttt{IntegrateX} compared to GPACT, while the gas consumption on bc2 and bc3 decreases. 
% In \texttt{IntegrateX}, the execution chain (bc1) handles all the logic in CCSCI, meaning that the invoked chains (bc2 and bc3) only need to lock and update states. 

% The results show that in this use case, the average latency for GPACT is 76.35 seconds, while the average latency for \texttt{IntegrateX} and \texttt{IntegrateX} without transaction aggregation is 37.25 seconds and 36.68 seconds, respectively. Additionally, \texttt{IntegrateX} consumes less gas compared to GPACT. In the Train-Hotel use case, \texttt{IntegrateX} reduced gas consumption by 1.2\%, while \texttt{IntegrateX} without transaction aggregation (TA) consumed 4.3\% more gas compared to GPACT. Overall, in the train-hotel application, while \texttt{IntegrateX} decreases gas consumption by 1.2\%, it achieves a 51.2\% improvement in efficiency.

\subsubsection{Performance under Different Invocation Complexity}
We evaluate the performance of latency and gas consumption of \texttt{IntegrateX} and GPACT with different CCSCI complexity.
% We evaluate the performance of \texttt{IntegrateX} without transaction aggregation at the same time.
% Since there are only three chains, multiple cross-chain contracts are invoked when the call tree depth exceeds 2. 
We vary the call tree depth to represent different levels of complexity in cross-chain invocations.
% adjust the complexity of cross-chain invocations and evaluate the gas consumption and latency of \texttt{IntegrateX} and GPACT at different depths. 
% The comparison results are shown in Figure~\ref{depth}.

\begin{figure}[t]
    \centering
    % 第一个子图
    \begin{subfigure}[b]{0.24\textwidth}
        \centering
        \includegraphics[width=\textwidth]{Figures/dpTime.png}  % 使用你的图片路径
        \caption{Latency}
        \label{dptime}
    \end{subfigure}
    % \hspace{3pt}
    \hfill
     % 第二个子图
    \begin{subfigure}[b]{0.24\textwidth}
        \centering
        \includegraphics[width=\textwidth]{Figures/dpGas.png}  % 使用你的图片路径
        \caption{Gas consumption}
        \label{dpgas}  % 子图标签,用于引用
    \end{subfigure}   
    \vspace{-18pt}
    \caption{Latency and gas consumption under different call tree depths.}
    % Gas consumption and latency of IntegrateX and GPACT of different call tree depth.}
    \label{depth}  % 整体图的标签

\end{figure}

The latency results are shown in Figure~\ref{dptime}. Due to the atomic integrated execution mechanism, the latency of \texttt{IntegrateX} and \texttt{IntegrateX} without transaction aggregation (TA) remain stable at around 35 seconds. 
The latency of GPACT, however, increases linearly as the complexity of CCSCI grows.
Specifically, when the call tree depth is 4, \texttt{IntegrateX} significantly reduces latency by 61.2\% compared to GPACT. 
Furthermore, as can be inferred, with invocation complexity further increases, \texttt{IntegrateX} will show greater improvements compared to GPACT. 
% , indicating that with further increases in invocation complexity, \texttt{IntegrateX} will show greater improvements compared to GPACT. 
% When the call tree depth is 2, \texttt{IntegrateX} reduces latency by 33.3\% compared to GPACT. 
% When the call tree depth is 4, \texttt{IntegrateX} reduces latency by 61.2\% compared to GPACT. 

The gas consumption results are shown in Figure~\ref{dpgas}. 
Due to the lower invocation complexity, the gas savings from transaction aggregation are not significant,
when the call tree depth is 3, the gas consumption for CCSCI in \texttt{IntegrateX} is nearly the same compared to GPACT. As the invocation complexity increases, when the call tree depth reaches 4, transaction aggregation reduces gas consumption in \texttt{IntegrateX} by 9.8\% compared to GPACT. And with further increases in invocation complexity, transaction aggregation will reduces more gas consumption in \texttt{IntegrateX}. We also evaluate the gas consumption between \texttt{IntegrateX} and \texttt{IntegrateX} without transaction aggregation. The experimental results show that as the complexity of CCSCI increases, the transaction aggregation mechanism significantly reduces gas consumption in CCSCI. When the call tree depth is 3, the transaction aggregation mechanism reduces gas consumption by 5.4\%, and when the call tree depth is 4, it reduces gas consumption by 19.3\%. This suggests that the transaction aggregation mechanism achieves greater improvements in gas consumption as the complexity of CCSCI increases.

% Overall, as the complexity of cross-chain invocations increases, \texttt{IntegrateX} not only demonstrates significant performance improvements but also exhibits progressively lower gas consumption compared to GPACT.

\subsubsection{Concurrency Performance with Fine-Grained State Lock}
We now measure the concurrency performance of \texttt{IntegrateX} with and without the fine-grained state lock mechanism. 
% In this experiment, to show the concurrency performance, we compare the latency of \texttt{IntegrateX} with and without the fine-grained state lock mechanism. 
% We ensured that the locked state could successfully handle all the calls, thus avoiding failures due to incomplete calls. 
To show the concurrency performance, we initiate multiple cross-chain calls on the same state simultaneously and compare the time consumed by the cross-chain dApp. 
The experimental results are shown in the Figure~\ref{fgl_t}.

\begin{figure}[t]
    \centering
    \includegraphics[width=0.24\textwidth]{Figures/fglTime.png}

    \caption{Latency of handling CCSCI in \texttt{IntegrateX} with/without Fine-Grained State Lock (FGSL).}
    \label{fgl_t}
\end{figure} 

The experimental results show that with fine-grained state lock mechanism, the latency of \texttt{IntegrateX} remains stable at around 35 seconds regardless of the number of simultaneous calls.
This demonstrates that the fine-grained state lock can maintain good transaction concurrency.
% When the number of simultaneous invocations is 2, the average latency of the \texttt{IntegrateX} system without fine-grained state lock is 68.6 seconds, representing an increase of approximately 102.3\%.
On the other side, as the number of simultaneous calls increases, the latency increases linearly in \texttt{IntegrateX} without fine-grained state lock. 
When the number of simultaneous invocations reached 6, the average latency of the \texttt{IntegrateX} system rises to 178.8 seconds, representing an increase of approximately 400.1\%.
% Without fine-grained state lock mechanism, the latency increased linearly with the number of simultaneous invocations. 



\begin{table}[tbh]
\centering
\caption{Gas consumption during cross-chain deployment.}
%\vspace{-10pt}
\begin{tabular}{|c|c|c|}
\hline
\textbf{  Contract Type  }  & \textbf{  w/o LSD 
(Gas)  } & \textbf{  w/ LSD (Gas)  } \\ \hline
Train Contract          & 1,459,626                                & 749,383                              \\ \hline
Hotel Contract          & 1,524,151                                & 389,383                              \\ \hline
\end{tabular}
\label{lsd}
%\vspace{-10pt}
\end{table}

\subsubsection{Logic-State Decoupling}
We evaluate the gas consumption during the cross-chain smart contract deployment, using the Train-and-Hotel contracts as an example.
We compare the results of the train and hotel contracts with and without logic-state decoupling.
% logic clone for smart contracts, comparing the logic-state decoupling version with the non-decoupled version. 
% In the train-hotel example, we test two contracts: the train contract, which has more complex logic, and the hotel contract, which stores more states. The experimental results are presented in Table~\ref{lsd}.
The experimental results in Table~\ref{lsd} indicate that, after applying logic-state decoupling, gas consumption during off-chain clone and deployment is reduced by 48.6\% for the train contract and 74.5\% for the hotel contract. 
The main reason is that, the train contract is implemented with more complex logic, while the hotel contract stores more states. 
As can been seen, for contracts with substantial state but relatively simple logic, logic-state decoupling can significantly reduce gas consumption.

