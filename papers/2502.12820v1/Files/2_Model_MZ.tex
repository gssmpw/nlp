\section{System Overview}

In this section, we provide an overview of \texttt{IntegrateX}. 
\texttt{IntegrateX}'s architecture and workflow overview is shown in Figure \ref{overview}.
For a detailed analysis of \texttt{IntegrateX}'s security and further discussion, please refer to Section~\ref{security_analysis} and Section~\ref{disscussion}.

\subsection{System and Threat Model}

In \texttt{IntegrateX}, there are $n$ (variable) blockchains that share the same smart contract execution environment (e.g., EVM). 
These blockchains are developed by their respective projects (e.g., Ethereum, BSC, etc.~\cite{eth,bsc_whitepaper}) and operated by their own blockchain nodes, which handle transaction processing, consensus within the blockchain, and other tasks. 
\texttt{IntegrateX} also features $m$ (variable) \emph{relayers} responsible for trustless cross-chain communication between blockchains, similar to many mainstream interoperability protocols \cite{atomic-ibc, cosmos2019, sheng2023trustboost, tas2023interchain}. 
Each relayer has its own public-private key pair and signs cross-chain transactions.
Additionally, \texttt{IntegrateX}'s \emph{bridging smart contracts} are deployed on each blockchain (similar to on-chain light clients \cite{cosmos2019, atomic-ibc}). 
% The functionality of these smart contracts is similar to the on-chain light clients in some well-known cross-chain communication protocols (e.g., IBC)~\cite{cosmos2019, atomic-ibc}. 
% These contracts serve as verifiable bridges that connect on-chain smart contracts with off-chain environments. 
The bridging contracts mainly serve to verify state transitions on the other blockchains, and send information externally via event emissions.
% On-chain contracts use these bridging contracts to send information externally via event emissions.
% Cross-chain transactions must also go through these bridging contracts to verify their validity and interact with other on-chain smart contracts.
% Furthermore, multiple nodes on each blockchain run \texttt{IntegrateX}’s \emph{cross-chain light clients}, which are primarily responsible for synchronizing block headers across chains and assisting the bridging smart contracts in verification. 
The above \texttt{IntegrateX}'s transport layer architecture is similar to some well-known cross-chain communication protocols (e.g., IBC)~\cite{cosmos2019, atomic-ibc}, which guarantee basic security during cross-chain communication. 
However, \texttt{IntegrateX} achieves efficiency and overall atomicity at the application layer, which these other protocols do not.

\texttt{IntegrateX} also includes several \emph{intra-chain dApp providers}. 
These providers can freely choose to deploy their contracts and run their intra-chain dApps on different blockchains. 
Additionally, there are \emph{cross-chain dApp providers} who can similarly flexibly choose to deploy their cross-chain dApps across various blockchains. 
A cross-chain dApp typically consists of multiple intra-chain dApps distributed across different blockchains, and relies on \texttt{IntegrateX}'s cross-chain protocol to ensure efficient CCSCI and maintain the overall atomicity of the cross-chain dApp.
Finally, there are also \emph{users} within the \texttt{IntegrateX} ecosystem. 
These users interact with the cross-chain dApps by sending transactions to the blockchains in order to use these cross-chain dApps.

\vspace{3pt}
\noindent
\textbf{Threat Model.}
In \texttt{IntegrateX}, we make only minimal trust assumptions about the relayers, assuming that at least one relayer is honest and functioning correctly. 
This assumption is consistent with those made in many existing secure interoperability protocols \cite{atomic-ibc, cosmos2019, sheng2023trustboost, tas2023interchain}. 
For each blockchain, the proportion of Byzantine nodes is assumed to be less than the fault tolerance threshold $t$ of the respective blockchain network (e.g., for blockchains using BFT-type consensus protocols under partial-synchronous network, $t=1/3$~\cite{pbft}).
% ; for blockchains using Nakamoto-type consensus protocols, $t=1/2$~\cite{ren2019analysis}). 
This ensures the safety and liveness of consensus within each blockchain.
As for the dApp providers and users—who represent the application layer components—we make no specific threat assumptions, similar to most existing works~\cite{cosmos2019,chen2024atomci,robinson2021general}. 
However, we do discuss common countermeasures for dealing with malicious behavior from these components in Section~\ref{disscussion}.

\subsection{Objective}

\texttt{IntegrateX} aims to achieve the following primary objectives:
\begin{itemize}[left=0pt]
\item \textbf{Efficiency}: During the process of cross-chain smart contract deployment and invocation, \texttt{IntegrateX} seeks to reduce latency, lower gas consumption, and increase transaction concurrency.
\item \textbf{Overall Atomicity}: \texttt{IntegrateX} aims to ensure overall atomicity for cross-chain dApps that require it during CCSCI. This means guaranteeing that the series of CCSCI operations required by cross-chain dApp providers either all succeed or all fail.
\end{itemize}

In addition, \texttt{IntegrateX} aims to possess the following desirable properties. 
\emph{Reliability}: \texttt{IntegrateX} ensures that cross-chain transactions can still be completed even in the presence of malicious relayers. 
\emph{Verifiability}: \texttt{IntegrateX} guarantees that cross-chain transactions can be verified for authenticity, completeness, and validity, even if malicious relayers are involved. 
\emph{Consistency}: \texttt{IntegrateX} ensures that the state changes across the blockchains involved in the cross-chain transaction remain coordinated and consistent. 

The proofs and experiments related to aforementioned properties are detailed in Section~\ref{security_analysis} and Section~\ref{evaluation}.



\begin{figure}[t]
    \centering
    \includegraphics[width=0.51\textwidth]{Figures/overview3.png}
    \vspace{-18pt}
    \caption{An overview of \texttt{IntegrateX}. 
    % Execution Chain: the chain responsible for integrated execution.
    % ; invoked chain: the chain where the invoked contracts are deployed.
    %four roles: intra-chain DApp providers, cross-chain DApp providers, users, and relayer.
    }
    %\vspace{-10pt}
    \label{overview}
\end{figure} 



\subsection{Primary Workflow}

The operation of \texttt{IntegrateX} consists of three main phases: \emph{Smart Contract Preparation}, \emph{Cross-Chain Smart Contract (CCSC) Deployment}, and \emph{Cross-Chain Smart Contract (CCSC) Integrated Execution}, as illustrated in Figure \ref{overview}.

% The smart contract preparation phase only occurs when dApp providers need to develop or upgrade smart contracts, so the frequency of this phase is relatively low. 
% Similarly, the cross-chain logic contract deployment phase only runs when cross-chain dApp providers need to deploy logic contracts across chains onto the same blockchain, which also occurs infrequently. 
% In contrast, the cross-chain smart contract invocation and integrated execution phase runs every time a user needs to interact with a cross-chain dApp, making it the most frequent phase of operation.

\vspace{3pt}
\noindent
\textbf{Smart Contract Preparation.} 
This phase only occurs when dApp providers need to develop or upgrade smart contracts, so the frequency of this phase is relatively low. 
In this phase, dApp providers can flexibly develop logic contracts and state contracts according to our defined logic-state decoupling guidelines (Section \ref{subsec:LSD}). 
This decoupling is primarily intended to reduce gas consumption and minimize storage costs on the target chain by cloning only the logic contracts during the subsequent CCSC deployment phase. 
Additionally, dApp providers can develop their contracts according to our fine-grained state lock guidelines (Section \ref{subsec:lock}), which enhances transaction concurrency during the CCSC integrated execution phase. 
Once the smart contract development is complete, dApp providers can freely choose a blockchain to deploy their contracts. 

\vspace{3pt}
\noindent
\textbf{CCSC Deployment.} 
This phase only runs when cross-chain dApp providers need to deploy logic contracts across chains onto the same blockchain, which also occurs infrequently. 
In this phase, cross-chain dApp providers can flexibly choose a target chain for deployment. 
By sending a transaction, they can clone and deploy the specified logic contracts from other designated chains onto the chosen target chain based on their cross-chain dApp needs. 
\texttt{IntegrateX} introduces an off-chain clone approach (Section \ref{subsec:migration}) to reduce gas consumption, while on-chain cross-chain verification (Section \ref{subsec:verification}) ensures that the cloned contracts are identical to the original ones.

\vspace{3pt}
\noindent
\textbf{CCSC Integrated Execution.}
This phase runs every time a user needs to interact with a cross-chain dApp, making it the most frequent phase of operation.
In this phase, users interact with cross-chain dApps by sending transactions to the target chain, invoking cross-chain smart contracts based on the dApp’s application logic. 
\texttt{IntegrateX} employs an atomic integrated execution scheme (Section \ref{subsec:execution}), similar to the 2PC protocol, to ensure the overall atomicity of the series of CCSCI involved in a cross-chain dApp.
This scheme first locks the relevant states of the smart contracts involved in the cross-chain dApp on the respective chains at a fine-grained level, 
then transmits these states across chains to the target chain. 
Since the target chain already contains all the execution logic required for the cross-chain dApp (from the earlier phase), it can integrate and execute all logic within one transaction once the necessary states have been received. 
After execution, the new states are returned to the corresponding chains for unlocking and updating.
Additionally, \texttt{IntegrateX} employs a transaction aggregation mechanism (Section \ref{subsec:aggregation}) during this phase to reduce cross-chain overhead.

In the subsequent descriptions within this paper, we will refer to the target chain selected by a cross-chain dApp for deployment—i.e., the chain responsible for integrated execution—as the \textbf{\emph{execution chain}}. 
The other chains associated with the cross-chain dApp will be collectively referred to as \textbf{\emph{invoked chains}}.