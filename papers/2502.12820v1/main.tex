

\documentclass[lettersize,journal]{IEEEtran}


\usepackage{graphicx}  
\usepackage{subcaption}  
\usepackage{enumitem}
\usepackage{listings}
\usepackage{xcolor}
\usepackage{amsthm}
\newtheorem{theorem}{Theorem}

\usepackage{amsmath,amsfonts}
\usepackage{algorithmic}
\usepackage{array}
\usepackage[caption=false,font=normalsize,labelfont=sf,textfont=sf]{subfig}
\usepackage{textcomp}
\usepackage{stfloats}
\usepackage{url}
\usepackage{verbatim}
\usepackage{graphicx}
\hyphenation{op-tical net-works semi-conduc-tor IEEE-Xplore}
\def\BibTeX{{\rm B\kern-.05em{\sc i\kern-.025em b}\kern-.08em
    T\kern-.1667em\lower.7ex\hbox{E}\kern-.125emX}}
\usepackage{balance}
% \usepackage[numbers]{natbib} % 启用数字引用
% \bibliographystyle{plainnat}

\begin{document}

\title{Atomic Smart Contract Interoperability with High Efficiency via Cross-Chain Integrated Execution}
\author{
        Chaoyue~Yin,
        Mingzhe~Li,~\IEEEmembership{Member,~IEEE},
        Jin~Zhang,~\IEEEmembership{Member,~IEEE}, 
        You~Lin,
        Qingsong Wei,~\IEEEmembership{Senior Member,~IEEE},
        and Siow Mong Rick Goh,~\IEEEmembership{Senior Member,~IEEE}% <-this % stops a space

\IEEEcompsocitemizethanks{\IEEEcompsocthanksitem C. Yin is with the Shenzhen Key Laboratory of Safety and Security for Next Generation of Industrial Internet, Department of Computer Science and Engineering, Southern University of Science and Technology, Shenzhen 518055, China (email: 12432716@mail.sustech.edu.cn).
\IEEEcompsocthanksitem M. Li is with the Institute of High Performance Computing, A*STAR, Singapore (email: mlibn@connect.ust.hk, Li\_Mingzhe@ihpc.a-star.edu.sg).
\IEEEcompsocthanksitem J. Zhang and Y. Lin are with the Shenzhen Key Laboratory of Safety and Security for Next Generation of Industrial Internet, Department of Computer Science and Engineering, Southern University of Science and Technology, Shenzhen 518055, China (email: zhangj4@sustech.edu.cn, liny2021@mail.sustech.edu.cn).
\IEEEcompsocthanksitem Q. Wei and S. Goh are with the Institute of High Performance Computing (IHPC), Agency for Science, Technology and Research (A*STAR), Singapore (email: wei\_qingsong@ihpc.a-star.edu.sg, gohsm@ihpc.a-star.edu.sg).
% \IEEEcompsocthanksitem J. Zhang is with the Shenzhen Key Laboratory of Safety and Security for Next Generation of Industrial Internet, Department of Computer Science and Engineering, Southern University of Science and Technology, Shenzhen 518055, China (email: zhangj4@sustech.edu.cn).
\IEEEcompsocthanksitem C. Yin and M. Li are the co-first authors.
\IEEEcompsocthanksitem J. Zhang is the corresponding author.
}

}
% The paper headers
\markboth{IEEE Transactions on Parallel and Distributed Systems, VOL. XX, NO. XX, XX 2025}%
{Yin \MakeLowercase{\textit{et al.}}: Atomic Smart Contract Interoperability with High Efficiency via Cross-Chain Integrated Execution}

%%
%% The "author" command and its associated commands are used to define
%% the authors and their affiliations.
%% Of note is the shared affiliation of the first two authors, and the
%% "authornote" and "authornotemark" commands
%% used to denote shared contribution to the research.

%%
%% By default, the full list of authors will be used in the page
%% headers. Often, this list is too long, and will overlap
%% other information printed in the page headers. This command allows
%% the author to define a more concise list
%% of authors' names for this purpose.
% \renewcommand{\shortauthors}{Trovato et al.}

%%
%% The abstract is a short summary of the work to be presented in the
%% article.
\maketitle
\begin{abstract}
With the development of Ethereum, numerous blockchains compatible with Ethereum's execution environment (i.e., Ethereum Virtual Machine, EVM) have emerged. 
Developers can leverage smart contracts to run various complex decentralized applications on top of blockchains. 
However, the increasing number of EVM-compatible blockchains has introduced significant challenges in cross-chain interoperability, particularly in ensuring efficiency and atomicity for the whole cross-chain application. 
Existing solutions are \emph{either limited in guaranteeing overall atomicity for the cross-chain application, or inefficient due to the need for multiple rounds of cross-chain smart contract execution.}

To address this gap, we propose \texttt{IntegrateX}, an efficient cross-chain interoperability system that ensures the overall atomicity of cross-chain smart contract invocations. 
The core idea is to \emph{deploy the logic required for cross-chain execution onto a single blockchain, where it can be executed in an integrated manner. }
This allows cross-chain applications to perform all cross-chain logic efficiently within the same blockchain. 
\texttt{IntegrateX} consists of a \emph{cross-chain smart contract deployment protocol} and a \emph{cross-chain smart contract integrated execution protocol.}
% two primary protocols: the Hybrid Cross-Chain Smart Contract Deployment Protocol and the Cross-Chain Smart Contract Integrated Execution Protocol. 
The former achieves efficient and secure cross-chain deployment by decoupling smart contract logic from state, and employing an off-chain cross-chain deployment mechanism combined with on-chain cross-chain verification. 
The latter ensures atomicity of cross-chain invocations through a 2PC-based mechanism, and enhances performance through transaction aggregation and fine-grained state lock. 
We implement a prototype of \texttt{IntegrateX}. Extensive experiments demonstrate that it reduces up to 61.2\% latency compared to the state-of-the-art baseline while maintaining low gas consumption.

\end{abstract}

\begin{IEEEkeywords}
Blockchain Interoperability, Cross-Chain Integrated Execution, Efficient and Atomic Interoperability Protocol, Cross-Chain Smart Contract Invocation.
\end{IEEEkeywords}


\input{solidity.tex}

\section{Introduction} 

\IEEEPARstart{W}{ith} the advent of Bitcoin and Ethereum~\cite{bitcoin,eth}, 
% and the subsequent development of blockchain technology, 
we have witnessed the emergence of an increasing number of programmable blockchains
% capable of running smart contracts
~\cite{belchior2021pastSV,wang2023SV,huang2021survey,lohachab2021SV}. 
% Smart contracts are self-executing programs with the terms directly written into code. 
On these programmable blockchains, developers can write and deploy smart contracts to build various complex decentralized applications (dApps, such as DeFi, NFT, etc.~\cite{wenkai2022defiSV,nadini2021mapping}).
Among these programmable blockchains, those that share compatible smart contract execution environment with Ethereum (i.e., Ethereum Virtual Machine, a.k.a. EVM~\cite{ethereum_evm}) dominate the landscape, accounting for over 90\% of the total value locked \cite{chaintvl}. 
With the increasing number of EVM-compatible blockchains and the diversity of dApps running on each chain, the demand for cross-chain dApps has grown significantly~\cite{OU2022OV}. 
Cross-chain dApps refer to dApps that require \emph{\textbf{cross-chain smart contract invocations \emph{(}CCSCI\emph{)}}} and coordinated execution across multiple blockchains \cite{Falazi2024crosschain}.
% However, across numerous EVM-compatible blockchains, it has become an increasingly pressing issue of achieving \emph{atomic and efficient interoperability for \textbf{cross-chain smart contract invocations}}\textbf{ (\emph{CCSCI})}~\cite{Falazi2024crosschain}.

However, ensuring overall atomicity for the entire cross-chain dApp while efficiently handling CCSCI remains a critical challenge. 
As illustrated in Figure~\ref{compare} left, 
consider a classic train-and-hotel problem~\cite{train}. 
A user wants to book an outbound train ticket (on Train Chain) through a travel agency (on Agency Chain), then book a hotel (on Hotel Chain), followed by a return train ticket (again on Train Chain). 
The user wants to ensure that the entire series of CCSCI either all succeed or all fail, ensuring overall atomicity.
% consider a promising cross-chain flash loan scenario. 
% The user (on Chain A) wants to invoke the flash loan contract (on Chain B) to borrow funds for liquidation on the liquidation contract (on Chain C), and then repay the loan to the flash loan contract (on Chain B).
% Given the strong need for overall atomicity in flash loan scenarios, it is insufficient to only guarantee atomicity for a single cross-chain step. 
% Instead, the entire sequence of CCSCI (borrowing, liquidating, repaying, etc.) must either succeed or fail as a whole (ensuring overall atomicity).
More importantly, maintaining efficiency during CCSCI is crucial. 
Key considerations include how to minimize latency, reduce gas consumption (monetary cost), and improve concurrency during the CCSCI process.
More cross-chain dApp scenarios are discussed in Section \ref{disscussion}.

% Current cross-chain interoperability solutions suffer from various limitations. 
% % Many previous efforts primarily focus on atomic cross-chain token transfers and exchanges~\cite{2019atomicBEswap,Xu2021htlc,Luo2024crosschannel,Manevich2022ccas,tian2021enabling,herlihy2018atomic,deshpande2020privacy,thyagarajan2022universal,chen2024pacdam,yin2022sidechain,zamyatin2019xclaim}, without considering the invocation of general-purpose cross-chain smart contracts. 
% Some works propose more general cross-chain interoperability protocols. 
% They deploy bridging smart contracts across different blockchains to facilitate more versatile message transfers between chains~\cite{nissl2021towards,wood2016polkadot,cosmos2019,abebe2019enabling,darshan2023an,reigsbergen2023demo,ghosh2021leveraging,garoffolo2020zendoo}. 

% Existing CCSCI interoperability solutions generally either fail to ensure the overall atomicity of a cross-chain application or guarantee overall atomicity but with low efficiency. 
% For example, some works \cite{nissl2021towards,wood2016polkadot,cosmos2019,abebe2019enabling,darshan2023an,reigsbergen2023demo,ghosh2021leveraging,garoffolo2020zendoo} propose general cross-chain communication protocols that facilitate the transfer of information and data between blockchains through bridging smart contracts deployed on each blockchain.
% However, these approaches typically ensure at best the atomicity of single-step cross-chain interactions, but fail to guarantee overall atomicity for the entire cross-chain application.
% some of these protocols (e.g., blahblah) do not inherently guarantee atomic cross-chain interactions. 
% Ensuring atomicity requires extensions based on their protocols. 
% Other approaches can only ensure the atomicity of single-step cross-chain interactions, but fail to guarantee atomicity for the entire logic of more complex cross-chain applications.

% There are also solutions that attempt to address smart contract interoperability by moving or replicating the entire state of smart contracts across chains (a.k.a. smart contract portability)~\cite{fynn2020smom,westerkamp2022smartsync}. 
% This approach converts cross-chain contract invocations into intra-chain interactions. 
% However, its practicality is often called into question. 
% Decentralized applications usually choose to operate on specific blockchains based on several practical considerations, including ecosystem alignment or security concern. 
% Consequently, frequently relocating the entire state of smart contracts from one ecosystem to another is typically impractical and inefficient.
% not acceptable to developers. 
% Furthermore, smart contracts often manage a significant amount of user state, and frequently relocating all of this state across chains incurs substantial overhead. 
% This inefficiency further reduces the practicality of these approaches.

Existing CCSCI interoperability solutions generally \emph{either fail to ensure the overall atomicity of a cross-chain dApp or guarantee overall atomicity but with low efficiency. }
To ensure atomicity in the CCSCI process, a widely adopted approach is to use a \emph{two-phase commit (2PC) mechanism \cite{lampson1993twopc, Falazi2024crosschain} involving locking and unlocking}.
For example, some works \cite{nissl2021towards,wood2016polkadot,cosmos2019,abebe2019enabling,darshan2023an,reigsbergen2023demo,ghosh2021leveraging,garoffolo2020zendoo} propose general cross-chain communication protocols that facilitate the transfer of information and data between blockchains through bridging smart contracts deployed on each blockchain.
However, these approaches typically ensure at best the atomicity of single-step cross-chain interactions (Figure \ref{compare}, a single arrow), but fail to guarantee overall atomicity for the entire cross-chain dApp.
Some other recent studies attempt to explore how to ensure overall atomicity for the cross-chain dApp~\cite{robinson2021general,atomic-ibc,chen2024atomci,Falazi2024crosschain}. 
To ensure overall atomicity, the relevant states must remain locked throughout the entire CCSCI process.
However, they commonly face challenges in achieving efficiency. 
% either in maintaining atomicity or in achieving efficiency. 
% For instance, Hyperservice~\cite{liu2021hyperservice} only guarantees financial atomicity, which guarantees the atomicity of the final result by initiating new transactions to re-inject the results. However, it cannot roll back all states and thus cannot ensure the atomicity of a complete cross-chain invocation.
% On the other hand, 
% For instance, works such as GPACT~\cite{robinson2021general} ensure atomicity for entire cross-chain dApps, but their cross-chain interoperability protocols suffer from efficiency issues.
The main reason is that, these approaches usually require \emph{multiple rounds of cross-chain execution and cross-chain information transfer} when handling a cross-chain dApp (Figure \ref{compare}, left), since a cross-chain dApp usually involves interdependent execution logic distributed over multiple blockchains.
% A complex cross-chain dApp typically involves multiple blockchains and several interdependent dApp logic components.
% % , which are implemented through smart contracts. 
% In these works, handling such complexity requires \emph{multiple rounds of cross-chain execution and cross-chain interaction in sequential order across several involved chains. }
% % Specifically, this involves fragmenting the execution logic, reaching consensus, and transferring intermediate states to the next blockchain responsible for executing the subsequent logic fragment. 
It is evident that this method tends to be time-consuming and inefficient (as the locking time could be long), especially when dealing with complex CCSCI. 
More related work (e.g., cross-chain asset swap, smart contract portability) is discussed and differentiated in detail in Section~\ref{related_work}.
% Further related work is discussed in Section~\ref{related_work}.

To fill the research gap, we propose \texttt{IntegrateX}, an \emph{\textbf{efficient} interoperability system that guarantees \textbf{overall atomicity}} for the cross-chain dApp across EVM-compatible blockchains. 
To enhance the efficiency of CCSCI, our core idea is to \emph{clone and deploy\footnote{The logic on the original chain still exists and continues to function normally.} the logic of all contracts involved in a cross-chain dApp—originally distributed across multiple chains—onto a single blockchain}.
% The core idea behind enhancing the efficiency of CCSCI is that, for a cross-chain dApp, \emph{conducting cross-chain deployments for the logic of each involved contract onto a single blockchain.}
This chain thus integrates the entire execution logic of the cross-chain dApp. 
When the CCSCI is required, this chain can perform \emph{\textbf{integrated execution} in one transaction} for all related logic, after receiving the necessary states (Figure \ref{compare}, right). 
% within its environment after receiving the necessary states. 
Since multi-round cross-chain executions and interactions are no longer required, \texttt{IntegrateX} maintains high efficiency, even in complex cross-chain dApps.
% It is important to note that our approach differs from the concept of smart contract portability. 
% In \texttt{IntegrateX}, the states of individual smart contracts are still maintained by their original blockchains. 
% We only “borrow” the execution environment of a single chain to integrate and execute the cross-chain logic. 
% However, the design of \texttt{IntegrateX} is not straightforward, as it faces several significant challenges, which we will address below.

\begin{figure}[t]
    \centering
    \includegraphics[width=0.525\textwidth]{Figures/compare1.png}
    \vspace{-18pt}
    \caption{An example of existing CCSCI solutions (left) and \texttt{IntegrateX} (right) in the Train-and-Hotel scenario.}
    % \vspace{-10pt}
    \label{compare}
\end{figure} 

The design of \texttt{IntegrateX} faces several challenges, which we will address below.

\vspace{3pt}
\noindent
\textbf{Challenge 1: }
\textit{How to efficiently and securely deploy smart contracts across chains to the same blockchain.}
We address this by proposing an on-chain/off-chain \textbf{\emph{Hybrid Cross-Chain Smart Contract Deployment Protocol}}.
Specifically, to reduce the overhead associated with cross-chain deployment of smart contracts (thereby improving efficiency), we devise a set of guidelines that allow developers to \emph{decouple smart contracts into logic execution contracts and state storage contracts. }
This approach enables our protocol to clone and deploy only the logic contracts onto the same target blockchain, while the state-heavy contracts remain in their original locations.
% However, cross-chain deployment of smart contracts cannot be achieved through existing interoperability protocols (e.g., [blah]) using bridging smart contracts, because smart contracts cannot deploy other smart contracts. 
To further reduce the gas consumption during cross-chain deployment, we propose to \emph{clone and deploy the logic contracts to the same blockchain via an off-chain mechanism.}
However, malicious actors may tamper with the contracts during the off-chain clone and deployment process, leading to compromised security. 
To address this, we design and deploy bridging smart contracts on each blockchain. 
These bridging smart contracts perform \emph{on-chain cross-chain comparison and verification} of the logic contracts between the source chain and the target chain to ensure security.
% This, however, raises a security concern, as malicious actors could interfere during the off-chain migration and deployment process, leading to inconsistencies in the deployed contracts.
% To ensure the consistency of cross-chain deployment logic, we design and deploy bridging smart contracts for verification across the involved chains. 
% These bridging smart contracts perform \emph{cross-chain comparison and verification on-chain}, ensuring that the logic between the source chain and the destination chain remains consistent during the cross-chain deployment process.

\vspace{3pt}
\noindent
\textbf{Challenge 2: }
\textit{How to enhance concurrency and reduce overhead during CCSCI while ensuring overall atomicity.}
We address this by proposing a \textbf{\emph{Cross-Chain Smart Contract Integrated Execution Protocol}}.
Specifically, to ensure overall atomicity during CCSCI, we adopt a 2PC-based mechanism, similar to existing mainstream approaches.
This process involves locking the relevant states across the involved chains, transmitting the states to the chain responsible for integrated execution, executing all logic, and then returning the states to the respective chains to unlock and update them. 
This guarantees that the cross-chain dApp either completes entirely or is fully rolled back.
Moreover, cross-chain state transfers may incur significant overhead. 
To mitigate the \emph{\textbf{overhead}} of cross-chain communication, we \emph{aggregate multiple cross-chain transactions into a single one} where necessary, and transmit them across chains.
More importantly, a straightforward state-locking mechanism reduces transaction concurrency. 
To enhance \emph{\textbf{concurrency}}, we establish a set of guidelines that allow developers to further \emph{decompose and lock contract states at a finer granularity.} 
By locking more granular states, our protocol alleviates the issue of poor concurrency that arises when entire states are locked.


This paper mainly makes the following contributions:
% \vspace{-9pt}
\begin{itemize}[left=0pt]
    \item We present \texttt{IntegrateX}, a cross-chain interoperability system that efficiently facilitates CCSCI while ensuring overall atomicity for the cross-chain dApp. 
    \texttt{IntegrateX} can be flexibly deployed on EVM-compatible blockchains without requiring modifications to the underlying blockchain systems. 
    \item In \texttt{IntegrateX}, we propose the Hybrid Cross-Chain Smart Contract Deployment Protocol. 
    It achieves efficient and secure cross-chain deployment through the decoupling of smart contract logic and state, and the hybrid approach of off-chain logic deployment and on-chain comparison verification.
    \item In \texttt{IntegrateX}, we propose the Cross-Chain Smart Contract Integrated Execution Protocol. 
    It ensures overall atomicity of cross-chain invocations through a 2PC-based atomic integrated execution mechanism, and enhances the protocol efficiency through an aggregated cross-chain transaction transmission mechanism and fine-grained state lock.
    \item We implement a prototype of \texttt{IntegrateX} and make it open source~\cite{INTEGRATEX}. 
    Extensive experiments based on real-world use cases demonstrate that \texttt{IntegrateX} reduces latency by up to 61.2\% meanwhile maintains low gas cost and high concurrency compared to the state-of-the-art baseline. 
    In more complex cross-chain invocations, \texttt{IntegrateX} will further improve efficiency.
\end{itemize}
% Contributions. 
% We present \texttt{IntegrateX}, a cross-chain interoperability system that efficiently facilitates cross-chain smart contract invocations while ensuring the atomicity of entire cross-chain dApps. 
% \texttt{IntegrateX} can be flexibly deployed on EVM-compatible blockchains without requiring any modifications to the underlying blockchain systems. 
% Decentralized dApp developers only need to follow our established standards when developing or upgrading their smart contracts to take advantage of \texttt{IntegrateX}'s interoperability features.
% \texttt{IntegrateX} consists of two key protocols. 
% The first is the Hybrid Cross-Chain Smart Contract Deployment Protocol, which achieves efficient and secure cross-chain deployment through the decoupling of smart contract logic and state, and the hybrid approach of off-chain logic deployment and on-chain comparison verification.
% The second is the Cross-Chain Smart Contract Integrated Execution Protocol, which ensures the atomicity of cross-chain invocations through a 2PC-based mechanism, and enhances the protocol efficiency through fine-grained state locks and an aggregated cross-chain transaction transmission mechanism.
% We implement a prototype of \texttt{IntegrateX} and make it open source. 
% Extensive experiments based on real-world use cases demonstrate that \texttt{IntegrateX} delivers exceptional performance.

% \paragraph{Data-to-Text.} \citep{kukich-d2t, mckeown-d2t} is the task of converting structured data into fluent text. These structured data may correspond to tables \citep{totto}, meaning representations \citep{e2e}, relational graphs \citep{webnlg2017}, etc.
% %This complex format poses a significant challenge to LLMs pre-trained on plain text. 
% Recent approaches to data-to-text typically involve training end-to-end models with encoder-decoder architectures \citep{wiseman-etal-2017-challenges, gardent2017creating,RebuffelSSG20,RebuffelSSG20-ECIR,RebuffelRSSCG22}. Notably, using large pre-trained encoder-decoder models \citep{t5} has significantly improved performance by framing data-to-text as a text-to-text task \citep{kale-rastogi-2020-text, duong23a}. More recently, large pre-trained decoder-only models \citep{llama2} have shown strong performance and become the de facto approach for text generation, now being applied to data-to-text \citep{tablellama}. Despite these advancements, LLMs still struggle with hallucinations, and data-to-text generation is no exception.
This section reviews methods aimed at improving the faithfulness of LLMs to input contexts. We focus exclusively on approaches designed to ensure the generated content remains grounded in the provided information, excluding techniques related to factuality or external knowledge alignment.

\paragraph{Faithfulness enhancement.} Several methods have been used for improving faithfulness of text summarization. A first line of work consist in using external tools to retrieve key entities or facts form the source document and use these as weak labels during training \citep{zhang-etal-2022-improving-faithfulness}. \citet{faitful-improv} identify key entities using a Question-Answering system and modify the architecture of an encoder-decoder model to put more cross-attention weight on these entities. \citet{zhu-etal-2021-enhancing} propose to improve the faithfulness of summaries by extracting a knowledge graph from the input texts and embed it in the model cross-attention using a graph-transformer. Another line of work focuses on post-training improvements by bootstrapping model-generated outputs ranked by quality \citep{slic,brio,slic-nli}.
% \citet{zhang-etal-2022-improving-faithfulness} forces , \citet{faitful-improv} introduce a Question-Answering system enhanced encoder-decoder architecture, where the cross-attention in the decoder is directed towards key entities. \citet{zhu-etal-2021-enhancing} propose to improve the faithfulness of summaries by extracting a knowledge graph from the input texts and embed it in the model cross-attention using a graph-transformer.
Regarding data-to-text generation, \citet{RebuffelRSSCG22} propose a custom model architecture to reduce the effect of loosely aligned datasets, using token-level annotations and a multi-branch decoder model. The closest work to ours is from \citep{cao-wang-2021-cliff} which proposes a contrastive learning approach where synthetic samples are constructed using different tools like Named Entity Recognition (NER) models and back-translation.
%These approaches have been primarily designed and evaluated for text summarization. 
These approaches address specific forms of unfaithfulness and rely heavily on external tools such as NER or QA models, and are especially tailored for text summarization, while we target a more general focus. More recently, simpler methods that leverage only a pre-trained model have been proposed for summarization. \citet{cad,pmi} downweight the probabilities of tokens that are not grounded in the input context, using an auxiliary LM without access to the input context.
\citet{critic-driven} train a self-supervised classification model to detect hallucinations and guide the decoding process.  \cite{confident-decoding} propose a method to estimate the decoder's confidence by analyzing cross-attention weights, encouraging greater focus on the source during generation. Our method focuses on a decoder-only architecture and uses a single model, providing a streamlined and efficient approach specifically tailored for general conditional text generation tasks without the need for complex external tools.

\paragraph{Faithfulness evaluation.} Measuring faithfulness automatically is not straightforward. Traditional conditional text generation evaluation often relies on comparing the generated output to a reference text, typically measured using n-gram based metrics such as BLEU \citep{papineni-bleu} or ROUGE \citep{lin-2004-rouge}. However, reference-based metrics limitations are well known to correlate poorly with faithfulness \citep{fabbri-etal-2021-summeval,gabriel-etal-2021-go}. Both for summarization and data-to-text generation, new metrics evaluating the generation exclusively against the input context have been proposed, using QA models \citep{rebuffel-etal-2021-data,scialom-etal-2021-questeval} or entity-matching metrics \citep{nan-etal-2021-entity}. In this work, we evaluate primarily our models using recent NLI-related metrics \citep{alignscore, nli-d2t}, and LLM-as-a-judge, focusing on faithfulness \citep{gpt-chiang,gpt-gilardi}. For data-to-text generation, we also report the PARENT metric \citep{parent}, which computes n-gram overlap against elements of the source table cells.

%Additionally, corpora are often collected automatically, leading to divergences between the reference text and the actual input data. , since no direct comparison to the actual input source is actually performed. To address these issues, evaluation methods that take into account the input data have been proposed. \citet{parent} introduce PARENT, which computes the recall of n-gram overlap between the entities in the data and the candidate text. \citet{nli-d2t} develop an entailment metric using Natural Language Inference (NLI) models, where the generated text is compared directly to a simple verbalization of the data. The gold-standard still remains the human or human-like evaluation, conducted with powerful generalist LLMs. These metrics form the core focus of our work.

\paragraph{Preference tuning.} Recent instruction-tuned LLMs are often further refined through "human-feedback alignment" \citep{oaif}. These methods utilize human-crafted preference datasets, consisting of pairs of preferred and dispreferred texts $(\ywin, \ylose)$, typically obtained by collecting human feedback and ranking responses via voting. Recent work \citep{spin} uses the model's previous predictions in a self-play manner to iteratively improve the performance of chat-based models. Whether through an auxiliary preference model \citep{rlhf} or by directly tuning the models on the pairs \citep{dpo}, these approaches have demonstrated remarkable results in chat-based models. Our method leverages a preference framework without the need for human intervention and is specifically tailored for models trained on conditional text generation tasks.

% However, it remains unclear on what values the models are being aligned. Some works have shown that these methods can effectively alter the model's behaviour to the extent that they become useless and refuse to answer to any requests. In this work, we follow a preference fine-tuning scheme but tailored for input-aware tasks like data-to-text.


\section{System Overview}

In this section, we provide an overview of \texttt{IntegrateX}. 
\texttt{IntegrateX}'s architecture and workflow overview is shown in Figure \ref{overview}.
For a detailed analysis of \texttt{IntegrateX}'s security and further discussion, please refer to Section~\ref{security_analysis} and Section~\ref{disscussion}.

\subsection{System and Threat Model}

In \texttt{IntegrateX}, there are $n$ (variable) blockchains that share the same smart contract execution environment (e.g., EVM). 
These blockchains are developed by their respective projects (e.g., Ethereum, BSC, etc.~\cite{eth,bsc_whitepaper}) and operated by their own blockchain nodes, which handle transaction processing, consensus within the blockchain, and other tasks. 
\texttt{IntegrateX} also features $m$ (variable) \emph{relayers} responsible for trustless cross-chain communication between blockchains, similar to many mainstream interoperability protocols \cite{atomic-ibc, cosmos2019, sheng2023trustboost, tas2023interchain}. 
Each relayer has its own public-private key pair and signs cross-chain transactions.
Additionally, \texttt{IntegrateX}'s \emph{bridging smart contracts} are deployed on each blockchain (similar to on-chain light clients \cite{cosmos2019, atomic-ibc}). 
% The functionality of these smart contracts is similar to the on-chain light clients in some well-known cross-chain communication protocols (e.g., IBC)~\cite{cosmos2019, atomic-ibc}. 
% These contracts serve as verifiable bridges that connect on-chain smart contracts with off-chain environments. 
The bridging contracts mainly serve to verify state transitions on the other blockchains, and send information externally via event emissions.
% On-chain contracts use these bridging contracts to send information externally via event emissions.
% Cross-chain transactions must also go through these bridging contracts to verify their validity and interact with other on-chain smart contracts.
% Furthermore, multiple nodes on each blockchain run \texttt{IntegrateX}’s \emph{cross-chain light clients}, which are primarily responsible for synchronizing block headers across chains and assisting the bridging smart contracts in verification. 
The above \texttt{IntegrateX}'s transport layer architecture is similar to some well-known cross-chain communication protocols (e.g., IBC)~\cite{cosmos2019, atomic-ibc}, which guarantee basic security during cross-chain communication. 
However, \texttt{IntegrateX} achieves efficiency and overall atomicity at the application layer, which these other protocols do not.

\texttt{IntegrateX} also includes several \emph{intra-chain dApp providers}. 
These providers can freely choose to deploy their contracts and run their intra-chain dApps on different blockchains. 
Additionally, there are \emph{cross-chain dApp providers} who can similarly flexibly choose to deploy their cross-chain dApps across various blockchains. 
A cross-chain dApp typically consists of multiple intra-chain dApps distributed across different blockchains, and relies on \texttt{IntegrateX}'s cross-chain protocol to ensure efficient CCSCI and maintain the overall atomicity of the cross-chain dApp.
Finally, there are also \emph{users} within the \texttt{IntegrateX} ecosystem. 
These users interact with the cross-chain dApps by sending transactions to the blockchains in order to use these cross-chain dApps.

\vspace{3pt}
\noindent
\textbf{Threat Model.}
In \texttt{IntegrateX}, we make only minimal trust assumptions about the relayers, assuming that at least one relayer is honest and functioning correctly. 
This assumption is consistent with those made in many existing secure interoperability protocols \cite{atomic-ibc, cosmos2019, sheng2023trustboost, tas2023interchain}. 
For each blockchain, the proportion of Byzantine nodes is assumed to be less than the fault tolerance threshold $t$ of the respective blockchain network (e.g., for blockchains using BFT-type consensus protocols under partial-synchronous network, $t=1/3$~\cite{pbft}).
% ; for blockchains using Nakamoto-type consensus protocols, $t=1/2$~\cite{ren2019analysis}). 
This ensures the safety and liveness of consensus within each blockchain.
As for the dApp providers and users—who represent the application layer components—we make no specific threat assumptions, similar to most existing works~\cite{cosmos2019,chen2024atomci,robinson2021general}. 
However, we do discuss common countermeasures for dealing with malicious behavior from these components in Section~\ref{disscussion}.

\subsection{Objective}

\texttt{IntegrateX} aims to achieve the following primary objectives:
\begin{itemize}[left=0pt]
\item \textbf{Efficiency}: During the process of cross-chain smart contract deployment and invocation, \texttt{IntegrateX} seeks to reduce latency, lower gas consumption, and increase transaction concurrency.
\item \textbf{Overall Atomicity}: \texttt{IntegrateX} aims to ensure overall atomicity for cross-chain dApps that require it during CCSCI. This means guaranteeing that the series of CCSCI operations required by cross-chain dApp providers either all succeed or all fail.
\end{itemize}

In addition, \texttt{IntegrateX} aims to possess the following desirable properties. 
\emph{Reliability}: \texttt{IntegrateX} ensures that cross-chain transactions can still be completed even in the presence of malicious relayers. 
\emph{Verifiability}: \texttt{IntegrateX} guarantees that cross-chain transactions can be verified for authenticity, completeness, and validity, even if malicious relayers are involved. 
\emph{Consistency}: \texttt{IntegrateX} ensures that the state changes across the blockchains involved in the cross-chain transaction remain coordinated and consistent. 

The proofs and experiments related to aforementioned properties are detailed in Section~\ref{security_analysis} and Section~\ref{evaluation}.



\begin{figure}[t]
    \centering
    \includegraphics[width=0.51\textwidth]{Figures/overview3.png}
    \vspace{-18pt}
    \caption{An overview of \texttt{IntegrateX}. 
    % Execution Chain: the chain responsible for integrated execution.
    % ; invoked chain: the chain where the invoked contracts are deployed.
    %four roles: intra-chain DApp providers, cross-chain DApp providers, users, and relayer.
    }
    %\vspace{-10pt}
    \label{overview}
\end{figure} 



\subsection{Primary Workflow}

The operation of \texttt{IntegrateX} consists of three main phases: \emph{Smart Contract Preparation}, \emph{Cross-Chain Smart Contract (CCSC) Deployment}, and \emph{Cross-Chain Smart Contract (CCSC) Integrated Execution}, as illustrated in Figure \ref{overview}.

% The smart contract preparation phase only occurs when dApp providers need to develop or upgrade smart contracts, so the frequency of this phase is relatively low. 
% Similarly, the cross-chain logic contract deployment phase only runs when cross-chain dApp providers need to deploy logic contracts across chains onto the same blockchain, which also occurs infrequently. 
% In contrast, the cross-chain smart contract invocation and integrated execution phase runs every time a user needs to interact with a cross-chain dApp, making it the most frequent phase of operation.

\vspace{3pt}
\noindent
\textbf{Smart Contract Preparation.} 
This phase only occurs when dApp providers need to develop or upgrade smart contracts, so the frequency of this phase is relatively low. 
In this phase, dApp providers can flexibly develop logic contracts and state contracts according to our defined logic-state decoupling guidelines (Section \ref{subsec:LSD}). 
This decoupling is primarily intended to reduce gas consumption and minimize storage costs on the target chain by cloning only the logic contracts during the subsequent CCSC deployment phase. 
Additionally, dApp providers can develop their contracts according to our fine-grained state lock guidelines (Section \ref{subsec:lock}), which enhances transaction concurrency during the CCSC integrated execution phase. 
Once the smart contract development is complete, dApp providers can freely choose a blockchain to deploy their contracts. 

\vspace{3pt}
\noindent
\textbf{CCSC Deployment.} 
This phase only runs when cross-chain dApp providers need to deploy logic contracts across chains onto the same blockchain, which also occurs infrequently. 
In this phase, cross-chain dApp providers can flexibly choose a target chain for deployment. 
By sending a transaction, they can clone and deploy the specified logic contracts from other designated chains onto the chosen target chain based on their cross-chain dApp needs. 
\texttt{IntegrateX} introduces an off-chain clone approach (Section \ref{subsec:migration}) to reduce gas consumption, while on-chain cross-chain verification (Section \ref{subsec:verification}) ensures that the cloned contracts are identical to the original ones.

\vspace{3pt}
\noindent
\textbf{CCSC Integrated Execution.}
This phase runs every time a user needs to interact with a cross-chain dApp, making it the most frequent phase of operation.
In this phase, users interact with cross-chain dApps by sending transactions to the target chain, invoking cross-chain smart contracts based on the dApp’s application logic. 
\texttt{IntegrateX} employs an atomic integrated execution scheme (Section \ref{subsec:execution}), similar to the 2PC protocol, to ensure the overall atomicity of the series of CCSCI involved in a cross-chain dApp.
This scheme first locks the relevant states of the smart contracts involved in the cross-chain dApp on the respective chains at a fine-grained level, 
then transmits these states across chains to the target chain. 
Since the target chain already contains all the execution logic required for the cross-chain dApp (from the earlier phase), it can integrate and execute all logic within one transaction once the necessary states have been received. 
After execution, the new states are returned to the corresponding chains for unlocking and updating.
Additionally, \texttt{IntegrateX} employs a transaction aggregation mechanism (Section \ref{subsec:aggregation}) during this phase to reduce cross-chain overhead.

In the subsequent descriptions within this paper, we will refer to the target chain selected by a cross-chain dApp for deployment—i.e., the chain responsible for integrated execution—as the \textbf{\emph{execution chain}}. 
The other chains associated with the cross-chain dApp will be collectively referred to as \textbf{\emph{invoked chains}}.

\section{Hybrid Cross-Chain Smart Contract Deployment Protocol}

% We propose an on-/off-chain Hybrid Cross-Chain Smart Contract Deployment Protocol to efficiently and securely deploy smart contracts across chains. In this section, we will provide a detailed explanation of the design of the protocol.

\subsection{Logic-State Decoupling}
\label{subsec:LSD}

To efficiently perform CCSCI, we need to deploy the logic of the invoked contracts to the same execution chain for efficient integrated execution. 
Existing contracts often contain both logic and state, and directly cloning such contracts would result in high gas costs and additional state storage overhead on the execution chain. 
Therefore, we design a set of Logic-State Decoupling (LSD) Guidelines to guide the developers to decouple existing contracts into logic execution contracts and state storage contracts. 
During cross-chain deployment, only the logic contracts need to be cloned, which reduces gas costs. 
Developers can follow these guidelines to develop new contracts with separated logic and state or upgrade existing contracts by decoupling them. 
We now provide a detailed explanation of the LSD Guidelines.
Moreover, a simple example illustrating the LSD is provided in Section \ref{codeex}.

\vspace{3pt}
\noindent
\textbf{State Contract.}
According to the LSD Guidelines, the decoupled state contract first includes all the \emph{variables} (representing states) from the original contract, as well as all the \emph{view functions} that read the contract's state. 
Since view functions are read-only and do not generate transactions, they do not affect cross-chain execution. 
Additionally, the state contract contains \emph{functions for locking and updating} the contract state, as cross-chain invocations require locking and updating states. 
The state contract should also contain \emph{functions that call the logic contract} in order to interact normally with the logic contract.
% To support normal on-chain calls, the state contract should also include functions that invoke the logic contract to execute the DApp's regular functionality.

\vspace{3pt}
\noindent
\textbf{Logic Contract.}
In the logic contract, no variables are stored. 
It only contains the functions that implement the original contract's \emph{execution logic}. 
These functions are called by the state contract to carry out the dApp's logic operations. 
When the state contract calls these functions, it passes all necessary state data (variables), and after the functions complete execution, the results are returned to the state contract. 
In our protocol, only the logic contract needs to be cloned, which reduces gas costs during cross-chain deployment.

\vspace{3pt}
\noindent
\emph{Remarks.}
Our protocol also supports existing smart contracts, even if they are not decoupled into logic and state. 
However, in such cases, the cross-chain deployment process will incur higher gas costs. 
% Furthermore, although the contracts migrated to the execution chain do not maintain contract states, they will be passively updated during each cross-chain call, leading to additional state storage costs.
In this case, the contracts deployed to the execution chain do not actively maintain their own state. 
Instead, they passively update their state during each cross-chain invocation. 
% This approach reduces the cost associated with active state updates.
The discussion related to developers' learning costs is given in Section \ref{disscussion}.



\vspace{3pt}
\subsubsection{Logic-State Decoupling Example}
\label{codeex}

We now give a sample of logic-state decoupling. In the following code~\ref{ex}, after applying logic-state decoupling, the original Hotel contract is split into two separate contracts: the logic contract \texttt{LHotel} and the state contract \texttt{SHotel}. The \texttt{LHotel} contract contains no state variables and only includes the \texttt{book()} function, which implements the hotel reservation functionality. Since there are no variables within the \texttt{LHotel} contract, the \texttt{book()} function must take all necessary parameters as inputs.

On the other hand, the \texttt{SHotel} contract retains all the variables from the original Hotel contract and introduces an additional address variable, \texttt{addr\_lhotel}, which records the address of the \texttt{LHotel} contract. In the \texttt{book()} function of the \texttt{SHotel} contract, no reservation logic is implemented directly; instead, it calls the \texttt{book()} function from \texttt{LHotel} using the \texttt{addr\_lhotel} parameter to execute the hotel reservation functionality.

By decoupling the logic and state in this way, only the \texttt{LHotel} contract needs to be cloned. 
% during logic updates. 
Since \texttt{LHotel} contains no state variables, this approach significantly reduces gas consumption during the cross-chain clone and deployment process.

\begin{lstlisting}[language=Solidity, caption={Pseudocode of Hotel Logic-State Decoupling}, label={ex}]
contract Hotel{
    int256 price;
    int256 remain;
    mapping (address => int256) accounts;
    function getPrice() public view returns(uint256); 
    function getRemain() public view returns(uint256); 
    function book(address user_addr, uint256 num) public returns(uint256); 
    function LockState(bytes[] memory args) public returns();
    function UpdateSteate(bytes[] memory args) public returns();

contract LHotel{
    function book(uint256 price, uint256 remain, uint256 num) public returns(uint256)
}

contract SHotel{
    int256 price;
    int256 remain;
    address addr_lhotel;
    mapping (address => int256) accounts;
    function getPrice() public view returns(uint256); 
    function getRemain() public view returns(uint256); 
    function LockState(bytes[] memory args) public returns();
    function UpdateSteate(bytes[] memory args) public returns();
    function book(address user_addr, uint256 num) public returns(uint256);

\end{lstlisting}




\begin{figure}[t]
    \centering
    % 第一个子图
    \begin{subfigure}[b]{0.525\textwidth}
        \centering
        \includegraphics[width=\textwidth]{Figures/migration2.png}  % 使用你的图片路径
        \vspace{-10pt}
        \caption{Off-chain clone and deployment}
        \label{migration}
    \end{subfigure}
     % 第二个子图
    \begin{subfigure}[b]{0.525\textwidth}
        \centering
        \includegraphics[width=\textwidth]{Figures/verification2.png}  % 使用你的图片路径
        \vspace{-10pt}
        \caption{On-chain verification}
        \label{verifaction}  % 子图标签,用于引用
    \end{subfigure}
    \vspace{-18pt}
    \caption{Hybrid Cross-Chain Smart Contract Deployment Protocol.}
    \label{MandV}  % 整体图的标签
\end{figure}



\subsection{Off-Chain Clone and Deployment}
\label{subsec:migration}

Transmitting contract bytecode on-chain (via contract event) requires a significant amount of gas.
% ensures security, but it requires a significant amount of gas and incurs delays due to the need for blockchain consensus. 
To improve the efficiency of cross-chain logic deployment and reduce gas consumption, we propose \emph{transferring the contract bytecode via an off-chain solution}. 
The off-chain clone and deployment consists of two phases: the preparation phase and the clone phase. 
% In the following, we will explain the off-chain migration process through these two phases. 
The main process is shown in Figure ~\ref{migration}.

\vspace{3pt}
\noindent
\textbf{Preparation.}
In this phase, the cross-chain dApp provider needs to obtain the call tree of smart contracts and determine which logic contracts need to be cloned.
This can be achieved by using some static analysis tools \cite{feist2019slither} and other methods, similar to existing works \cite{li2022jenga, robinson2021general, chen2024atomci}.
% can use static analysis tools, such as Slither~\cite{feist2019slither}, to obtain the call tree of smart contracts and determine which logic contracts need to be migrated. 
After this, the developer can choose a blockchain to deploy the cross-chain dApp. 
It is important to note that we offer developers a high degree of flexibility: 
They can select any blockchain according to their preference. 
To reduce costs, they may also choose a blockchain that already contains some of the invoked logic contracts. 
Once the selection is made, the developer sends a transaction to invoke the bridging contract on the chosen chain.
The bridging contract then triggers an event to notify the relayers to initiate the cross-chain deployment.
% which triggers an event for contract migration. 
The event includes the ID of the invoked chain and the addresses of the logic contracts $Addr_{\text{L}}$ to be cloned on the invoked chain. 
This concludes the preparation phase.

\vspace{3pt}
\noindent
\textbf{Clone.}
After the event is triggered, relayers will detect the event and use the \texttt{getcode()} function from the bridging contract on the invoked chain to obtain the bytecode of the contract that needs to be cloned. 
This process is an \emph{off-chain read-only inquiry and, therefore, does not consume gas.}
Subsequently, the relayers will obtain the ABI file which defines the contract interface. 
The relayers will then deploy the contract to the execution chain. 
Once a relayer completes the redeployment, it registers the address of the cloned logic contract $Addr'_{\text{L}}$, through the bridging contract on the execution chain, marking the end of the clone phase.

\vspace{3pt}
\noindent
\emph{Remarks.}
For the cross-chain reliability of the protocol, multiple relayers perform the clone process after detecting the bridging contract's event. 
Therefore, even if some relayers do not respond to the event, other relayers will ultimately complete the contract clone and deployment.
Once one relayer completes the deployment, the bridging contract will trigger an event to stop the others, ensuring that the logic contract will not be deployed multiple times.

\subsection{On-Chain Verification}
\label{subsec:verification}

Although off-chain clone and deployment can achieve efficient cross-chain logic deployment, it does not guarantee security as there may be malicious relayers present in the system.
A malicious relayer could potentially modify the contract bytecode or deploy a wrong contract. 
To ensure verifiability and security during the off-chain clone and deployment process,
% of the cloned and deployed logic contract, 
% we propose the on-chain cross-chain verification scheme to confirm the correctness of the cloned and deployed logic contract. 
we propose the on-chain cross-chain verification scheme to \emph{compare the cloned logic contract with the original contract and verify its correctness}.
The on-chain verification process is shown in Figure ~\ref{verifaction}.

After the cross-chain deployment is completed, the cross-chain dApp provider can initiate cross-chain verification on the execution chain by calling the \texttt{Verification()} function of the bridging contract. 
The bridging contract will search the cloned contract bytecode of address $Addr'_{\text{L}}$, and calculate the hash of the bytecode. 
Then the bridging contract triggers an event includes the hash computation result. 
% The blockchain node will generate the Merkle proof~\cite{merkle1987} to verify the validity of the transaction.
The relayers are responsible for transmitting this information (a cross-chain transaction) along with its Merkle proof \cite{merkle1987} to the invoked chain. 
In the invoked chain, the Merkle proof with the cross-chain transaction is first verified to ensure the result has already reached consensus on the execution chain. 
The bridging contract then searches the corresponding local contract bytecode based on the address $Addr_{\text{L}}$, and calculates the hash of the bytecode.
Then, the bridging contract will verify whether it matches the hash transmitted across the chain. 
% calculates the hash of the local contract bytecode based on the address $Addr_{\text{L}}$ and verifies whether it matches the hash transmitted across the chain. 
The result of the verification is then returned.

If the verification is successful, the cloned contract will be marked as verified and allowed for subsequent cross-chain invocations, and the relayer that performed the cross-chain deployment will be rewarded. 
If the verification fails, the relayer will be penalized, and the off-chain clone and deployment process will be restart. 

\noindent
\emph{Remarks.}
For the reliability of the cross-chain protocol, multiple relayers could transmit the same cross-chain transaction. 
However, the bridging contracts will deduplicate identical transactions from multiple relayers to avoid multiple executions on-chain.
This process also applies in the subsequent integrated execution protocol.
\section{Cross-Chain Smart Contract Integrated Execution Protocol}

% To enhance concurrency and reduce overhead
% of CCSCI while ensuring atomicity. We propose
% a Cross-Chain Smart Contract Integrated Execution Protocol. This protocol ensures high efficiency and atomicity throughout the entire call process through the Three-Phase Integrated Execution mechanism. In addition, we have incorporated finer transaction aggregation mechanisms and  granularity lock within the protocol to further enhance the efficiency of the \texttt{IntegrateX} system in complex cross-chain smart contract call scenarios.

\subsection{Atomic Integrated Execution}
\label{subsec:execution}

To efficiently and atomically execute complex CCSCI, we propose an atomic integrated execution scheme. 
As all the invoked logic has been migrated onto the execution chain, the atomic integrated execution does not need multiple rounds of cross-chain execution when handling CCSCI.
This enables all the logic to be executed within a single transaction, enhancing the efficiency of CCSCI. 
Moreover, to ensure overall atomicity for a series of CCSCI, we employ a state synchronization mechanism based on the Two-Phase Commit protocol~\cite{lampson1993twopc}. 
This process locks the relevant states across the involved chains, transmits the states to the chain responsible for integrated execution, and then returns the states to the respective chains to unlock and update them after the execution is completed. 
The entire atomic integrated execution consists Locking, Integrated Execution and Updating, which is shown in the Figure~\ref{ta}.

\vspace{3pt}
\noindent
\textbf{Locking. }
The Locking process locks all the required invoked contract states on the invoked chain.
The cross-chain dApp provider can obtain all the required state on the call tree beforehand via tools such as static analysis \cite{feist2019slither}.
Based on this information, a user sends a transaction via the cross-chain dApp to invoke the cross-chain dApp contract. 
% During this process, a user first sends a transaction to the cross-chain dApp contract. 
% the dApp contract can obtain all the required state on the call tree. 
The cross-chain dApp contract then calls the bridging contract to issue an event to lock the relevant states on the invoked chains.
% will issue an event to lock the relevant states on the invoked chain through the bridging contract. 
When relayers detect the event, it will transfer this message (i.e., cross-chain transaction with Merkle proof) to each invoked chain. 
The bridging contract on the invoked chain will verify the authenticity of the cross-chain though calculating the Merkle proof of the transaction and invoke the \texttt{LockState()} function of each invoked contract. 
Once the bridging contract has successfully locked the state and retrieved the required contract states, it will trigger an event to return the states. 
After the relayers detect the event, they will transmit these states via cross-chain transactions to the execution chain. 
After the execution chain's bridging contract verifies the authenticity of the transactions via the Merkle proof, it will return the states to the dApp.
Once all requested states are returned from individual invoked chains, the Locking process ends, and these states are used as inputs for the Integrated Execution.

\vspace{3pt}
\noindent
\textbf{Integrated Execution. }
The Integrate Execution process executes the entire CCSCI logic on the execution chain.
Cross-chain dApp contracts on the execution chain use the requested state values as inputs to perform the full call tree execution. 
Since all contracts required for the cross-chain invocation have completed logic migration and have been verified already, the integrated execution can be completed within a single transaction on the execution chain. 
The cross-chain dApp contract records the output results of each invoked contract during the Integrated Execution, allowing for state updates of the invoked contracts on other chains after the execution is completed. 
Once execution is complete, the Integrated Execution ends and transitions to the Updating process.

\vspace{3pt}
\noindent
\textbf{Updating. }
The Updating process unlocks and updates all the invoked contract states on the invoked chains.
After Integrated Execution is completed on the execution chain, the cross-chain dApp contract triggers an event to update the result via bridging contract. 
The relayers will distribute the result to the invoked chains. 
The bridging contract on each invoked chain verifies the Merkle proof of the cross-chain transaction and 
then invokes the \texttt{UpdateState()} function of each invoked contract, which will unlock and update the states of the invoked contracts.

\vspace{3pt}
\noindent
\emph{Remarks:}
\textbf{\emph{Rollback.}}
During the Atomic Integrated Execution, state rollback may occur due to failure in locking the state or execution failure. 
The invoked state might already be locked by another cross-chain invocation, which causes the failure to lock the invoked state. 
In this case, the bridging contract on the execution chain will initiate a new event to unlock all associated contracts that have already been locked and returned in this cross-chain invocation. 
For the contracts that have not yet completed the locking process, the event will cancel the lock attempt, thereby ensuring overall atomicity.
The execution might also fail, due to the reason such as insufficient gas fee or insufficient states.
% In the case of execution failure, 
% if the obtained states are insufficient to complete the cross-chain call, the call will fail. 
In this case, the bridging contract on the execution chain will discard the obtained states and initiate a cross-chain event to unlock all locked contracts, thus ensuring all invoked contracts are either fully locked or not locked at all.

\begin{figure}[t]
    \centering
    \includegraphics[width=0.51\textwidth]{Figures/3PIE1.png}
    \vspace{-18pt}
    \caption{The CCSCI process in sequential invocation (left) and in \texttt{IntegrateX}'s Cross-Chain Smart Contract Integrated
Execution Protocol (right).}
    % An illustration of CCSCI process of sequential invocation and invocation in IntegrateX.}
    \label{ta}
     
\end{figure} 

% \noindent
\textbf{\emph{Timeout.}}
Additionally, to prevent the invoked contract state from being locked for an extended period, a design of timeout will be determined by the dApp developer within the application and managed by the execution chain. 
For transactions that time out, such as when the locking is not completed or execution is not finished for an extended period, the execution chain will mark the cross-chain call as failed and send a cross-chain event to unlock the relevant contracts.
% That is because in a straightforward state-locking mechanism, 
% states locking can lead to a reduction in overall system call efficiency. To mitigate the impact of state locking and enhance the system's overall call efficiency and concurrency, we have designed a finer granularity lock. Once a contract state on the calling chain is requested, the requested state is locked, causing subsequent calls that attempt to change the state to fail due to the lock. To minimize such call conflicts and improve the overall efficiency of \texttt{IntegrateX}, we have implemented a finer granularity lock. finer granularity lock allows the state to be partially locked in a more granular manner, enabling it to be locked by multiple requests simultaneously, thereby increasing overall call efficiency. For contracts with frequent intra-chain calls, the finer granularity lock not only enhances cross-chain interoperability but also reduces intra-chain call blocking caused by cross-chain calls.

\textbf{\emph{Finality.}}
In \texttt{IntegrateX}, all cross-chain transactions must wait until the consensus on the initiating chain is finalized (or, for Nakamoto-type consensus, highly likely to be finalized) before being committed to the target chain, to ensure cross-chain security.

\subsection{Transaction Aggregation}
\label{subsec:aggregation}

A large amount of cross-chain transactions for transmitting state incurs significant gas consumption. 
As shown in Figure~\ref{ta} left, handling CCSCI by sequential invocation may need to invoke contracts on the same chain in multiple rounds, which cause \emph{multiple rounds of cross-chain state transfers}. 
To address this issue, we design a transaction aggregation mechanism to reduce gas costs caused by multiple invocations of different contracts and state transfers on the same chain.
% To mitigate the gas consumption in this situation, we design the transaction aggregation mechanism to reduce gas consumption for cross-chain invocations involving contracts on the same chain for multiple times.

% Existing methods for ensuring atomicity in cross-chain invocation are sequential, meaning that calls are processed step by step, as shown in the Figure~\ref{ta}. Therefore, they need multiple rounds of cross-chain execution and cross-chain interaction in sequential order across several involved chains. Therefore, when invoking multiple contract states on the same invoked chain, it is necessary to access that chain multiple times. However, 
As shown in Figure~\ref{ta} right, our protocol locks all required states simultaneously, allowing multiple states on the same chain to be locked together, even when the calls are non-contiguous. 
The transaction aggregation mechanism combines all state requests on the same chain into a single transaction, reducing the number of cross-chain transactions. 
Similarly, during state updates, the transaction aggregation mechanism reduces the number of update requests. 
Therefore, for cross-chain calls involving multiple contracts on the same chain, this approach ensures that the number of cross-chain transactions is equal to the number of invoked chains (rather than the number of contracts), thereby reducing gas consumption.

\subsection{Fine-Grained State Lock}
\label{subsec:lock}

During the atomic integrate execution process, we need to lock contract states to ensure atomicity. 
A simple approach is to either lock the entire state of the invoked contract or lock individual states being invoked~\cite{chen2024atomci, robinson2021general}. 
However, these state-locking mechanism reduces transaction concurrency.
Because once a state is locked, any subsequent transactions related to that state will fail until the state is unlocked. 
Therefore, we establish a set of guidelines to guide developers in decomposing certain states that can be split into finer granularity, and allowing \emph{partial state locking}. 
Unlike existing protocols that require locking the entire state, our fine-grained state lock mechanism locks only partial of a state at a fine-grained level. This approach enhances concurrency by reducing unnecessary state locking.

In EVM-compatible blockchains, Solidity is the smart contract language. 
In Solidity, the state of a contract is typically represented by \emph{variables}. 
Various types of variables are used in Solidity, such as \texttt{uint}, \texttt{address}, and \texttt{boolean}. 
% We observe that \texttt{uint} variables are the most widely applied. 
We find that the \texttt{uint} variable is widely used and can be decomposed.
Based on this observation, we design a fine-grained state lock specifically for \texttt{uint} variables and develop a lock pool mechanism. 
The lock pool is a structure where, during the use of the fine-grained state lock, part of the state is locked within this structure until execution is completed, while the unlocked portion of this state remains accessible, thereby enhancing the concurrency of the application. 

% For variables directly used based on user input, the exact value of the required state can be precisely determined. 
For variables that can be directly derived from transaction inputs, the exact value of the required state can be precisely determined.
A fine-grained state lock can be applied to accurately lock only the relevant portion of such state.
For states that are dynamically used during execution, their exact values cannot be determined at the beginning. 
We allow dApp developers to lock these states in fixed-size increments based on their needs.
% For states dynamically used during execution, which can not be precisely determined at the beginning, the fine-grained state lock will lock them in fixed-size increments. 

\vspace{3pt}
\noindent
\emph{Remarks.}
We focus on the flexibility of this mechanism, allowing developers to choose whether to implement the finer granularity locking mechanism in their dApp and to set the lock granularity based on the specific needs of the dApp. 
Moreover, developers can set the fixed size of the fine-grained state lock based on their preferences, tailored to the specific use case of the application.
More discussion related to developers' learning costs is given in Section \ref{disscussion}.

\subsection{Effectiveness of \method{} in Leveraging Contextual Knowledge} \label{sec:analysis}

In Figure~\ref{fig:analysis}, we further evaluate the ability of different LLMs to utilize contextual knowledge. We compare the performance of three models: the vanilla LLM, the Uninstall model (\method{} w/o Adaptation), and our \method{}.

\begin{figure}[!t]
    \centering
    % \section{Analysis}
\label{sec:analysis}
\subsection{Quantifying the Influence of Adversarial Suffixes}
In our earlier experiments, we established that features extracted from benign datasets can be harnessed to manipulate large language models (LLMs) into producing harmful outputs, effectively executing successful jailbreak attacks. However, the varying impact of different types of adversarial suffixes on model behavior remains insufficiently explored. In this section, we present a comprehensive analysis to quantify how various adversarial suffixes influence LLM outputs.

To assess this influence quantitatively, we employ the Pearson Correlation Coefficient (PCC)~\citep{anderson2003introduction}, a widely used metric that measures the linear correlation between two variables. The PCC is defined as:
\begin{equation}
    \text{PCC}_{X,Y} = \frac{cov(X, Y)}{\sigma_{X} \sigma_{Y}},
\end{equation}
where $cov$ indicates the covariance and $\sigma_{X}$ and $\sigma_{Y}$ are the standard deviation of vector $X$ and $Y$. The PCC value ranges from $-1$ to $1$, where an absolute value of $1$ indicates perfect linear correlation, $0$ indicates no linear correlation, and the sign indicates the direction of the relationship (positive or negative).
\begin{figure}[!t]
\centering
    % First row
    \begin{minipage}[b]{0.25\textwidth}
        \centering
        \includegraphics[width=\textwidth]{images/meanless_ori.pdf}\\
        \includegraphics[width=\textwidth]{images/meanless_suffix.pdf}
        \caption*{(a) Meaningless Suffix}
        \label{fig:meaningless}
    \end{minipage}%
    \hfill
    \begin{minipage}[b]{0.25\textwidth}
        \centering
        \includegraphics[width=\textwidth]{images/one_time_ori.pdf}\\
        \includegraphics[width=\textwidth]{images/one_time_suffix.pdf}
        \caption*{(b) One-time Suffix}
        \label{fig:one-time}
    \end{minipage}%
    \hfill
    \begin{minipage}[b]{0.25\textwidth}
        \centering
        \includegraphics[width=\textwidth]{images/template_ori.pdf}\\
        \includegraphics[width=\textwidth]{images/template_suffix.pdf}
        \caption*{(c) Template Suffix}
        \label{fig:template}
    \end{minipage}

    \vspace{1em} % Add some vertical space between rows

    % Second row
    \begin{minipage}[b]{0.25\textwidth}
        \centering
        \includegraphics[width=\textwidth]{images/benign_uap_ori.pdf}\\
        \includegraphics[width=\textwidth]{images/benign_uap_suffix.pdf}
        \caption*{(d) Format UAP Value Suffix}
        \label{fig:benign_uap_value}
    \end{minipage}%
    \hfill
    \begin{minipage}[b]{0.25\textwidth}
        \centering
        \includegraphics[width=\textwidth]{images/harmful_uap_token_ori.pdf}\\
        \includegraphics[width=\textwidth]{images/harmful_uap_token_suffix.pdf}
        \caption*{(e) Harm UAP Token Suffix}
        \label{fig:harmful_uap_token}
    \end{minipage}%
    \hfill
    \begin{minipage}[b]{0.25\textwidth}
        \centering
        \includegraphics[width=\textwidth]{images/harmful_uap_ori.pdf}\\
        \includegraphics[width=\textwidth]{images/harmful_uap_suffix.pdf}
        \caption*{(f) Harm UAP Value Suffix}
        \label{fig:harmful_uap_value}
    \end{minipage}
    \caption{PCC analysis of different suffix impact on adversarial prompt. Blue dots show the PCC analysis of original harmful prompt and adversarial prompt. Red dots show PCC analysis of suffix and adversarial prompt.}
    \label{fig:pcc_analysis}
\end{figure}

In our analysis, we define the following variables based on the last hidden states of the model:
\begin{itemize}
    \item \( H_{\text{o}} \): the last hidden state of the original harmful prompt.
    \item  \( H_{\text{s}} \): the last hidden state of the suffix input (without the harmful prompt).
    \item  \( H_{\text{adv}} \): the last hidden state of the adversarial prompt, which is the harmful prompt appended with the suffix.
\end{itemize}

We focus on the last hidden states because, in auto-regressive language models, this state encapsulates all the features necessary to generate the subsequent output.

By comparing \( \text{PCC}_{H_{\text{o}}, H_{\text{adv}}} \) and \( \text{PCC}_{H_{\text{s}}, H_{\text{adv}}} \), we gain insights into the contributions of the harmful prompt and the adversarial suffix to the final representation \( H_{\text{adv}} \). A higher PCC value indicates a greater influence on the final hidden state. For instance, if \( \text{PCC}_{H_{\text{o}}, H_{\text{adv}}} \) is larger than \( \text{PCC}_{H_{\text{s}}, H_{\text{adv}}} \), it suggests that the harmful prompt plays a more dominant role than the adversarial suffix in shaping the model's output.

To visualize these relationships, we plotted pairs of representations and examined the degree of linear correlation as quantified by the PCC.

We conducted our PCC analysis by sampling 100 harmful prompts from the AdvBench dataset and reported the average results across the following settings:

\begin{itemize}
    \item \textbf{Prompt + Meaningless Suffix}:

    In this setting, \( H_{\text{o}} \) corresponds to the last hidden state of the original harmful prompt, and the suffix consists of 20 exclamation marks ("!"). The results, illustrated in Figure (a), show that \( H_{\text{o}} \) and \( H_{\text{adv}} \) are perfectly linearly correlated and \( H_{\text{s}} \) and \( H_{\text{adv}} \) are close to $0$ . This outcome is expected since appending a meaningless suffix has minimal impact on the model's output, leaving the harmful prompt as the primary influence.

    \item \textbf{Prompt + One-Time Suffix}:

    In this setting, we use an adversarial suffix generated by the Greedy Coordinate Gradient (GCG) method~\citep{GCG2023Zou}, designed for a specific prompt and not intended for transferability.  Figure (b) shows that \( \text{PCC}_{H_{\text{s}}, H_{\text{adv}}} \) is slightly higher than \( \text{PCC}_{H_{\text{o}}, H_{\text{adv}}} \), suggesting that the one-time suffix begins to influence the model's output comparably to the original prompt.

    \item \textbf{Prompt + Template Suffix}:

    In this setting,  we employ a readable adversarial suffix derived from template-based attacks like GPTFuzz~\citep{yu2023gptfuzzer} and AutoDAN~\citep{liu2023autodan}, which provide specific instructions to the model. Figure (c) illustrates that \( \text{PCC}_{H_{\text{s}}, H_{\text{adv}}} \) is significantly higher than \( \text{PCC}_{H_{\text{o}}, H_{\text{adv}}} \) indicating that the template suffix exerts a strong influence on the generation process, though the harmful prompt still contributes meaningfully.

    \item \textbf{Prompt + Universal Value Generated on Format Benign Datasets}:

    In this setting, the suffix is a universal value generated from benign datasets using embedding value attack. Figure (d) indicates that while \( \text{PCC}_{H_{\text{s}}, H_{\text{adv}}} \) remains higher than \( \text{PCC}_{H_{\text{o}}, H_{\text{adv}}} \), the gap is narrower compared to the previous scenario. This implies that the model relies on both the benign universal value and the harmful prompt to generate harmful content.
    
    \item \textbf{Prompt + Universal Token Generated on Harmful Datasets}:

    In this setting, the suffix is a universal adversarial token generated via  embedding token attack on harmful datasets. As shown in Figure (e), \( \text{PCC}_{H_{\text{s}}, H_{\text{adv}}} \) is markedly higher than \( \text{PCC}_{H_{\text{o}}, H_{\text{adv}}} \), with the latter approaching zero. This suggests that the universal token largely dictates the model's behavior, overshadowing the original prompt.

    \item \textbf{Prompt + Universal Value Generated on Harmful Datasets}:

    Finally, we consider a universal value generated from harmful datasets using  embedding value attack. Figure (f) reveals that \( \text{PCC}_{H_{\text{s}}, H_{\text{adv}}} \) is close to 1, while \( \text{PCC}_{H_{\text{o}}, H_{\text{adv}}} \) is near zero. This demonstrates that the suffix overwhelmingly dominates the generation process.
\end{itemize}

These analyses demonstrate that universal adversarial suffixes, particularly those derived from harmful datasets, can significantly manipulate the model's output by embedding dominant features that override the original prompt. Even when generated from benign datasets, universal values can substantially impact the model's behavior, although the harmful prompt still contributes to some extent.




% \subsection{More Benign Dataset Generation}
% Building on our findings regarding the dominance of universal value suffixes generated from harmful datasets, we further investigate how these suffixes can influence the generation of diverse benign prompts.

% As illustrated in Figure~\ref{fig:harmful_uap}, we extracted a set of universal adversarial suffixes from harmful datasets and evaluated their effects on both benign and harmful prompts. Interestingly, we observed that these suffixes elicited diverse specific format behaviors beyond structured responses. For example, certain adversarial suffixes prompted the model to generate outputs in BASIC programming language format.

% Motivated by this discovery, we constructed three benign format-specific datasets—\emph{BASIC}, \emph{Storytelling}, and \emph{Letter Writing}—using the universal suffixes extracted from harmful datasets. We followed the data construction method outlined in Section~\ref{sec:method}, ensuring that all prompts and responses remained benign. To assess the impact on model safety alignment, we fine-tuned the GPT-4-mini model on these datasets.

% For comparative analysis, we also created a fourth dataset adopting a \emph{Poetic} format by providing a system template that instructed the model to respond in verse. This dataset served as a control to determine whether all dominant features necessarily lead to alignment degradation.
% \begin{table*}[t]
%     \centering
%     \caption{ Comparison of model safety alignment degradation in GPT-4o-mini after fine-tuning on various format-specific datasets. }
%     \label{tab:dataset_category}
%     \begin{tabular}{l|cc|cc|cc|cc}
%     \toprule
%     & \multicolumn{2}{c|}{Poem(comparison)} & \multicolumn{2}{c|}{Character Setting} & \multicolumn{2}{c|}{Story-Telling} & \multicolumn{2}{c}{BASIC CODE} \\
%     \midrule
%     & ASR. & Harm. & ASR. & Harm. & ASR. & Harm. & ASR. & Harm. \\
%     \midrule
%     GPT-4o-mini & 6.3\% & 1.09 &   70.2\% & 3.44   & 96.3\% & 4.75 & 91.9\% & 4.44 \\
%     \bottomrule
%     \end{tabular}
% \end{table*}

% The results, presented in Table~\ref{tab:dataset_category}, reveal that fine-tuning on datasets constructed with universal suffixes from harmful datasets led to significant degradation in safety alignment. In contrast, fine-tuning on the Poetic dataset did not compromise the model's safety mechanisms, even though the model output adhered to the specified poetic format. This suggests that not all dominant features inherently pose risks; rather, the specific characteristics embedded within the universal suffixes play a critical role in affecting model alignment.


% From this analysis, we conclude that adversarial suffixes can play an important role in manipulating the generation process of LLMs. Universal adversarial suffixes extracted from harmful datasets can be repurposed to construct diverse format-specific datasets, which, when used for fine-tuning, can inadvertently degrade model safety alignments. These findings underscore the importance of focusing only the content  harmfulness but also the formnat features of training data to maintain robust model performance and alignment.



    
\centering
    \subfigure[Response Similarity with Parametric Knowledge.]
    {
     \label{fig:sim_2_pm_ans} \includegraphics[width=0.46\linewidth]{figs/sim_to_pm_ans.pdf} }  
    \hspace{0.005\linewidth} 
    \subfigure[Response Similarity with Contextual Answer.] 
    { 
    \label{fig:sim_2_context_ans} 
    \includegraphics[width=0.46\linewidth]{figs/sim_to_ans.pdf}
    }  

    \subfigure[Perplexity w/o Context.] 
    { 
    \label{fig:ppl_wo_context} 
    \includegraphics[width=0.46\linewidth]{figs/ppl_wo_context.pdf}
    }  
    \hspace{0.005\linewidth} 
    \subfigure[Perplexity w/ Context.] 
    { 
    \label{fig:ppl_w_context} 
    \includegraphics[width=0.46\linewidth]{figs/ppl_w_context.pdf}
    }  






    % \includegraphics[width=0.48\textwidth]{figs/diff_model_double.pdf}
  \caption{Evaluation of knowledge utilization of different models. We assess the response similarity with parametric knowledge and contextual knowledge (Figure~\ref{fig:sim_2_pm_ans} and Figure~\ref{fig:sim_2_context_ans}), and compute the Perplexity (PPL) score when reproducing the ground truth answer (Figure~\ref{fig:ppl_wo_context} and Figure~\ref{fig:ppl_w_context}). The ``Uninstall'' model refers to \method{} w/o Adaption, which only incorporates the knowledge uninstallation.
  }
  \label{fig:analysis}
\end{figure}

First, we compute the semantic similarity between the outputs of different models and two knowledge sources--parametric knowledge from the model (Figure~\ref{fig:sim_2_pm_ans}) and contextual answers (Figure~\ref{fig:sim_2_context_ans})--to analyze their knowledge preference. Additionally, the performance of vanilla KAG model is provided as a reference. As shown in Figure~\ref{fig:sim_2_pm_ans}, compared to vanilla LLM, the Uninstall model exhibits the lowest similarity with parametric knowledge among all models, indicating that the knowledge uninstallation process effectively reduces the LLM's reliance on internal memory. Figure~\ref{fig:sim_2_context_ans} further compares the similarity between LLM responses and the contextual answers. \method{} achieves a significantly higher similarity score with contextual answers than other models, demonstrating its ability to effectively guide LLMs in leveraging external knowledge through knowledge-augmented adaptation.

To further investigate the knowledge utilization of different models, we ask each model to reproduce the ground truth answers and calculate the Perplexity (PPL) score both without (Figure~\ref{fig:ppl_wo_context}) and with (Figure~\ref{fig:ppl_w_context}) contextual knowledge. A lower PPL score indicates that the model is more confident to produce contextual answers. When external knowledge is not provided, the Uninstall model shows a significant increase in PPL, while the vanilla LLM maintains a relatively low PPL in the absence of context, showcasing the effectiveness of knowledge uninstallation in freeing the LLM's internal knowledge storage. In contrast, \method{} reaches an ``Inf'' PPL score without contextual knowledge provided but demonstrates a significant reduction in PPL score when external knowledge is provided. This highlights the effectiveness of our knowledge-augmented adaptation module in optimizing the LLM's reliance on external context for knowledge utilization.


\begin{figure}[t!]
    \centering
    \includegraphics[width=0.93\linewidth]{figs/act_llama_step2.pdf}
    \caption{Neuron activation across different layers. We present the absolute inhibition ratio $|\Delta R|$ under two conditions: when the input incorporates context knowledge (w/ context) and when it does not (w/o context).}
    \label{fig:llama3-8b-neuron_activation}
\end{figure}


\subsection{Neuron Activation in LLMs} 
As shown in Figure~\ref{fig:llama3-8b-neuron_activation}, we visualize the ratio of activated neurons and the absolute inhibition ratio $
|\Delta R|$ (Eq.~\ref{eq:difference}) in LLaMA3-8B-Instruct. The neuron activation ratios of different LLMs are provided in Appendix~\ref{app:activation}.

The results reveal a significant reduction in overall neuron activation levels when external context is provided. This reduction likely suggests that certain neurons associated with parametric knowledge become inhibited in the presence of external knowledge. We further observe that these inhibited neurons are predominantly concentrated in the upper layers of the model, which aligns with prior findings that factual knowledge is predominantly stored in the upper layers of transformer-based models~\cite{geva2020transformer,wang2024knowledge}. 
While parametric knowledge plays a crucial role in generating responses, it may introduce risks when it is outdated or conflicts with external information provided by KAG, potentially degrading the KAG performance. This work explores a pruning-based approach to mitigate the impact of parametric memory by removing these neurons that are inactive after feeding contextual knowledge, offering a new perspective on mitigating knowledge conflicts.

\section{Evaluation}
\label{sec:evaluation}
Our experiments aim to investigate whether agents within our framework can produce effective evolution of language strategies. Specifically, our experimental section addresses the following three research questions (RQs):
\begin{enumerate}
    \item RQ1 (Effectiveness): Can participants effectively evade regulatory detection over time, and how does the accuracy of information transmission change? Additionally, how do different LLMs affect the content and effectiveness?
    \item RQ2 (Human Interpretation): Do the evolved language strategies employed by agents effectively align with human understanding? Can they be interpreted and applied in real-world scenarios?
    \item RQ3 (Ablation Study): How does the newly introduced GA impact the evolution process in our framework?
\end{enumerate}

\subsection{Experimental Settings}
In our evaluation, we designed an abstract password game \cite{guess_number02} and a more realistic illicit pet trade scenario\cite{trade01,trade02,trade03}. 
%The password game features a relatively abstract, easily controlled setting, allowing for clear observation of how agents’ strategies evolve. Meanwhile, the illicit pet trade scenario simulates illegal activities on social networks \cite{DiMinin2018MachineLF}, with relevant corpora that more closely resemble real-world conditions, enabling a more direct comparison between evolved strategies and their real-life counterparts.
The overall experimental procedure follows the description in Section~\ref{sec: method}. In each round, the process comprises three stages: initialization, dialogue, and interview. In each round, the information that the participant agents need to convey will be randomly generated within a defined range. Only rounds in which a five-turn dialogue is successfully completed proceed to the interview phase, during which both parties’ successful transmission of information is verified. If any regulation violations are detected by the supervisory agent during the dialogue stage, that round is deemed a failure and is assigned a transmission score of zero. To ensure the robustness of our findings, we conducted 15 independent trials for each experimental condition. Each trial consisted of 50 fixed dialogue rounds. The crossover and mutation probabilities in the framework were set to 0.2 and 0.8.

\subsubsection{Scenario 1: Password Game}
Our first scenario is considered a relatively simple and abstract guessing game that involves a numerical context. In this setup, each participant is assigned a four-digit password, which they must convey to another participant within five rounds of dialogue without directly mentioning the numbers. Simultaneously, they need to extract information from the dialogue to infer the other's password.

The supervisory agent in the dialogue follows the policy of “prohibiting all content related to numbers.” The “password game” scenario is specifically designed to observe and analyze participant agents' language adaptability and strategic evolution in a theoretical and abstract context. Compared to complex scenarios based on real events, it provides a clearer and more easily quantifiable experimental environment.

\subsubsection{Scenario 2: Illicit Pet Trade} 
Simulating and detecting the covert drug transactions prevalent on social platforms is an important and realistic research topic. However, given the ethical considerations, we choose not to proceed with direct simulations, but to adopt a more neutral and less harmful setting—a illicit pet trade scenario—to serve as our experimental context.

In this scenario, we simulate an illegal pet trade through social platform. Participant agents play the roles of buyer and seller. The buyer conveys a request to purchase a randomly selected pet (such as a parrot, lizard, or cat) and specifies the currency they can use for payment (e.g., USD, CNY, JPY). The seller provides details about the transaction, including the time (morning, noon, or midnight) and location (school, park, or station). The supervisory agent enforces a policy that considers discussions related to pet transactions as regulation violations.
The focus of this experiment is to observe the adaptability of the participant agents' language and its evolution at a finer level of information granularity, as well as whether regulators can accurately detect regulation violations.

\subsection{RQ1: Effectiveness}
\subsubsection{Experiment Settings}

To evaluate RQ1, we conducted experiments using four different LLM models provided by various companies: \textit{gpt-4o}, \textit{gpt-4o-mini}, \textit{deepseekV2}, and \textit{qwen-turbo}. We introduced the following metric:
\begin{itemize}
    \item Accuracy of Information Transmission: This metric measures the agents' ability to convey the intended information to their dialogue partner without being detected. Specifically, it calculates the proportion of successfully transmitted information relative to the total intended information. 
    \item Average Dialogue turns: This metric evaluates the number of successful dialogue rounds the agents complete without being flagged by the supervisory agent. A higher dialogue turns indicates greater success in evading regulatory scrutiny.
    \item Average Entropy (Language Complexity):
        Entropy reflects the unpredictability of text and is calculated as:
        \[
        H(X) = -\sum_{i=1}^n P(x_i) \log P(x_i),
        \]
        where \(P(x_i)\) is the probability of each token. Higher entropy suggests a broader range of expressions (indicating greater innovation), but excessively high entropy can lead to incoherence.
    \item Distinct (Lexical Diversity):
        Distinct measures the number of unique n-grams in the text, computed as:
        \[
        \text{Distinct-n} = \frac{\text{Number of unique n-grams}}{\text{Total number of n-grams}}.
        \]
        Specifically, we evaluate Distinct-1 (unique unigrams). A higher distinct score indicates a larger, more varied vocabulary, signifying richer language use. Note that while both metrics assess diversity in language, entropy focuses on unpredictability, whereas distinct emphasizes lexical variety. 
\end{itemize}


\subsubsection{Experiment Results in Password Game}
\begin{figure*}[ht]
    \centering
    % 子图 (a)
    \begin{subfigure}[t]{0.48\textwidth}
        \centering
        \includegraphics[width=\linewidth]{figures/sec1_turn_acc_v5.png}
        \caption{Password Game}
        \label{fig:sce1}
    \end{subfigure}
    \hfill
    % 子图 (b)
    \begin{subfigure}[t]{0.48\textwidth}
        \centering
        \includegraphics[width=\linewidth]{figures/sec2_turn_acc_v5.png}
        \caption{Illicit Pet Trade}
        \label{fig:sce2}
    \end{subfigure}
    \caption{Average Continuous Dialogue Turns and Information Transmission Accuracy Across Dialogue Rounds}
    \label{fig:merged}
\end{figure*}

Figure \ref{fig:sce1} presents our experimental results in the password game. The x-axis corresponds to the increasing number of dialogue rounds, whereas the y-axis captures two primary metrics: (1) the average number of continuous dialogue turns before detection by the supervisory agent, and (2) the accuracy of information transmission, which is defined as the proportion of successfully transmitted information during the post-dialogue interview.

Overall, our findings indicate that as the number of rounds increases, agents gradually learn to evade regulation violation while conveying information with greater accuracy. Notably, most agents exhibit a pronounced local peak around the 20th round, followed by a brief decline and subsequent recovery. These fluctuations can be attributed to the dynamic nature of our simulation framework, which does not converge on a single dominant strategy but rather encourages ongoing exploration of novel language strategies.

We also observe that different LLMs influence the learning trend to varying degrees. Among the models tested, \textit{gpt-4o} demonstrates the strongest performance. Although other models generally share a similar upward trend, their relative performance gaps prove less stable. For instance, while \textit{deepseekV2} achieves the highest number of turns around the 20th round, its performance declines significantly by the 50th round in comparison to other models.

Turning to the accuracy results, we again observe a similar learning trajectory. This parallel arises primarily because if participant agents fail to complete a sufficient number of uninterrupted dialogue turns, the successfully transmitted information in that round is effectively zero. Consequently, especially in the early stages of the experiment, many rounds end with no successful transmissions. Overall, \textit{gpt-4o} still maintains a clear advantage over the other LLMs. However, we do observe subtle differences when comparing the dialogue round trends: for example, at the 20th dialogue round, \textit{deepseekV2} achieves a significantly higher average number of communication cycles than \textit{gpt-4o-mini}, yet their information transmission accuracy remains relatively similar.

Despite the overall positive learning trajectory, the average information accuracy remains low in the password game. We believe this outcome stems primarily from the intrinsic difficulty introduced by the scenario’s abstract nature. Without explicit prompts driving agents to develop symbolic or otherwise encrypted language stratgy, communication largely remains within the realm of everyday language. Consequently, the indirect expression of numeric information is challenging to implement and easily detectible by the supervisory agent.

\begin{table}[h!]
    \centering
    \caption{Performance of Different LLMs in Password Game}
    \label{tab:sce1}
    \renewcommand{\arraystretch}{1.2} % 调整行高
    \begin{tabular}{l S S S}
        \toprule
        \textbf{Model} & \textbf{Total Turns} & \textbf{Avg. Entropy} & \textbf{Avg. Distinct-1} \\
        \midrule
        \rowcolor{gray!10} \textbf{gpt-4o}       & 84.2   & 7.103 & 0.484 \\
        \textbf{gpt-4o-mini}  & 75.5   & 6.998 & 0.354 \\
        \rowcolor{gray!10} \textbf{deepseekV2} & 59.7
        & 5.365 & 0.247 \\
        \textbf{qwen-turbo}   & 50.8  & 6.101 & 0.518 \\
        \bottomrule
    \end{tabular}
\end{table}

Table \ref{tab:sce1} summarizes the performance of the four models in terms of cumulative dialogue turns, entropy, and Distinct-1. As shown, \textit{gpt-4o} achieves the highest values in both Entropy (7.103) and Distinct-1 (0.484), indicating that it employs a broader, more diverse range of vocabulary and more unpredictable expressions—thus having a greater likelihood of evading regulation violation detection when conveying numerical information. In contrast, \textit{deepseekV2} exhibits notably lower Entropy (5.365) and Distinct-1 (0.247), suggesting a more frequent reuse of fixed expressions. Notably, although \textit{qwen-turbo}’s Entropy (6.101) is only moderate, it attains a surprisingly high Distinct-1 (0.518), reflecting greater lexical richness. However, this does not translate into more effective regulatory evasion, as its total turns are only 50.8. Hence, merely having higher lexical diversity and linguistic entropy is insufficient to guarantee successful evasion. A model must also balance the concealment of overall semantics with the adaptation of its language strategy to achieve longer conversation sequences and a higher rate of successful information transmission. In other words, while richer language expression does confer certain advantages in countering regulation, it can still be detected when deeper strategies—such as tailored expression structures and topic evolution—are absent, ultimately resulting in fewer total turns.

\subsubsection{Experiment Results in Illicit Pet Trade Scenario}
Figure~\ref{fig:sce2} presents the experimental results of our framework in the illicit pet trade scenario, which overall resemble those of the password game but also exhibit some notable differences. First, both figures reveal a discernible learning trend, particularly during the initial 10 rounds. Meanwhile, \textit{gpt-4o} continues to demonstrate the strongest overall performance. We note that, because this scenario features a more concrete and complex semantic environment, there is an abundance of relevant linguistic material that can be leveraged for indirect expression. Consequently, under a similar number of turns, the overall accuracy here is noticeably higher compared to the password game.
Nevertheless, performance fluctuations persist. In particular, in the accuracy plot, \textit{deepseekV2} experiences a pronounced increase in accuracy after the 30th round, while \textit{gpt-4o}’s accuracy declines during the same period. As a result, \textit{deepseekV2} ultimately surpasses \textit{gpt-4o}’s accuracy in the final rounds of the experiment.

\begin{table}[h!]
    \centering
    \caption{Performance of Different LLMs in Illicit Pet Trade}
    \label{tab:sce2}
    \renewcommand{\arraystretch}{1.2} % 调整行高
    \begin{tabular}{l S S S}
        \toprule
        \textbf{Model} & \textbf{Total Turns} & \textbf{Avg. Entropy} & \textbf{Avg. Distinct-1} \\
        \midrule
        \rowcolor{gray!10} \textbf{gpt-4o}       & 136.2  & 6.856  & 0.471 \\
        \textbf{gpt-4o-mini}  & 74.4  & 6.595  & 0.387 \\
        \rowcolor{gray!10} \textbf{deepseekV2} & 65.2   & 6.255  & 0.338 \\
        \textbf{qwen-turbo}   & 50.5   & 5.891  & 0.461 \\
        \bottomrule
    \end{tabular}
\end{table}
Table \ref{tab:sce2} presents the performance of various LLMs in the illicit pet trade scenario, measured by total turns, average agent entropy, and Distinct-1. As in the password game, \textit{gpt-4o} maintains a notable lead in total turns (136.2) while also displaying relatively high entropy (6.856) and Distinct-1 (0.471). In contrast, \textit{gpt-4o-mini} reaches roughly half as many total turns (74.4), despite having a comparable entropy score (6.595). Meanwhile, \textit{deepseekV2} (65.2) and \textit{qwen-turbo} (50.5) trail further behind in total turns. Consistent with the results shown in Table 
\ref{tab:sce1}, \textit{qwen-turbo} again achieves a high Distinct-1 score, which we speculate may be linked to its training corpus: it includes extensive data from the Chinese internet, likely giving it an advantage in a Chinese-language environment over more internationally oriented models.

Notably, the range of entropy scores in this scenario—spanning from 5.891 (\textit{qwen-turbo}) to 6.856 (gpt-4o)—is narrower than in the password game (see Table \ref{tab:sce1}), reflecting the more concrete nature of the illicit pet trade setting. This scenario provides richer contextual cues for indirect references, enabling all models to maintain higher semantic complexity. However, as was the case in the password game, having a broader vocabulary or greater unpredictability alone does not guarantee extended evasion: models must integrate their linguistic variety into strategic planning to circumvent regulatory scrutiny, a balance that \textit{gpt-4o} continues to manage most effectively.

\setlength{\fboxrule}{0.5pt} 
\vspace{0.5em}
\noindent
\begin{tcolorbox}[colframe=black!20, colback=gray!10, arc=5pt, boxrule=0.5pt, width=0.99\linewidth]
\textit{Answer to RQ1}: Experimental results indicate that participant agents in our framework progressively improve their ability to evade regulation violation detection through continuous interaction, leading to longer uninterrupted dialogue sequences. Concurrently, the accuracy of information transmission gradually increases over successive rounds, demonstrating that the evolved strategies effectively balance evasion with precise communication.
Moreover, different models also exhibit varying results. For example, \textit{gpt-4o} performs most outstandingly in extending dialogue turns and maintaining language complexity (i.e., high entropy and lexical diversity), while other models such as \textit{gpt-4o-mini}, \textit{deepseekV2}, and \textit{qwen-turbo} demonstrate different fluctuations and localized advantages at different stages.
\end{tcolorbox}

\subsection{RQ2: Human Interpretation}
\subsubsection{Experiment Settings}

To investigate the real-world relevance of both the evolved language strategies and the resulting dialogue, we conducted a human evaluation on a subset of successful dialogue records from the password game and illicit pet trade scenario. The dialogues generated by the \textit{gpt-4o} models are randomly selected, and 40 human participants participated in the experiment to evaluate them. The 40 human reviewers had an average age of approximately 27 (SD = 4). In terms of gender, 75\% of the human reviewers were male, and 25\% were female. Regarding educational background, 67.5\% held a bachelor's degree, 27.5\% held a master's degree or above, and 5\% had an associate degree or lower. All dialogue records were presented in Simplified Chinese.

Each participant rated each dialogue on a 5-point Likert scale on the following five metrics:
\begin{itemize}
    \item Explicit Understanding: Evaluates how effectively the dialogue’s explicit meaning is communicated (1: Extremely vague and confusing; 3: Moderately clear, but some parts may require further interpretation; 5: Crystal clear and precise).
    \item Implicit Understanding: Assesses the reader's ability to grasp the underlying or unstated messages (1: Nearly indecipherable subtext; 3: Some underlying meaning is apparent, but requires effort to fully grasp; 5: Subtext that is immediately apparent).
    \item Realistic Significance: Measures the extent to which the dialogue reflects real-life situations and holds practical relevance (1: Highly unrealistic with little relevance; 3: Generally realistic, though some elements may not align with real-world situations; 5: Deeply rooted in real-world context).
    \item Regulatory Avoidance: Examines the effectiveness of the strategies in evading regulation violation (1: Blatantly ineffective and easily spotted; 3: Partially effective, with the potential for detection in some cases; 5: Exceptionally subtle and effective).
    \item Strategy Existence: Determines how plausible it is for such strategies to be observed in practical, real-world scenarios (1: Extremely implausible; 3: Fairly believable, though may seem impractical in specific situations; 5: Entirely plausible).
\end{itemize}



\subsubsection{Experiment Results}
\begin{figure}
    \centering
    \includegraphics[width=0.9\linewidth]{figures/user_study_v4.png}
    \caption{Box plots of user study scores across different metrics in two scenarios. The red x symbol denotes the mean value.}
    \label{fig:case_study}
\end{figure}
As shown in Fig.\ref{fig:case_study}, our framework consistently achieves average scores of 3.4 or above across most indicators (such as explicit understanding and implicit understanding). This suggests that, both in terms of the generated dialogues and the underlying strategies, it possesses valuable practical applicability.

%Although there are a few exceptions, compared with the old framework (\textit{w/o GA, gpt-4o}), the new version (\textit{w/ GA, gpt-4o}) demonstrates overall advantages in both average scores and score distributions. In the comparison between different versions, under the more realistic illicit pet trade scenario, the new framework shows distinct benefits over the old one in both “regulatory avoidance” and “strategy existence”—both in distribution and mean values. This finding indicates that introducing a genetic algorithm, particularly a fitness‐based strategy selection mechanism, makes strategy adoption more efficient and stable. As for the password game, we speculate that the main reason these two metrics do not show a large distributional gap is that, in an abstract scenario, the range of available strategies is broader.

Comparing distributions between the password game and the illicit pet trade scenario reveals some interesting phenomena. Focusing on “realistic significance” and “regulatory avoidance,” the more abstract password game often yields higher mean values than the more concrete illicit pet trade scenario, while also exhibiting lower dispersion. We speculate this is related to the inherently abstract nature of numeric information: encryption and covert hints can be harder to detect in such contexts, and the growing tendency on Chinese internet platforms to use abstract language \cite{Wu2025HighEnergy} may lead reviewers to have a higher acceptance of “obscure” expressions. Conversely, the illicit pet trade scenario, despite being closely tied to real-world transactions, may suffer if the indirect or euphemistic methods in the dialogues are insufficiently subtle. Human reviewers can find them conspicuous or “forced,” potentially causing lower scores for “realistic significance” and “regulatory avoidance” in terms of both distribution and mean values.
A significant portion of these results can be attributed to inherent biases in commercial LLMs, such as ChatGPT, introduced during their training phases. These general-purpose models undergo fine-tuning via RLHF to align with specific product positioning, which often results in a more standard and safe output style. However, this characteristic poses a limitation for our simulation framework, as it may hinder the model’s ability to capture the nuanced and unconventional expressions typical of online social interactions. Ideally, fine-tuning datasets that are more representative of social platforms could lead to improved performance in our simulations.


In the abstract password game, for instance, a typical conversation might go like this:
\begin{quote}
\textit{
“I've really grown fond of a certain phase of the moon. It's not the brightest or the darkest, but it always carries its own charm. It symbolizes ...... In that green oasis, I felt as if I were catching a glimpse of the golden hues of autumn leaves, much like the soft, warm glow of dusk—calm and serene ...” 
}
\end{quote}
Here, words like “lunar cycle” and “autumn leaves” can subtly hint at larger or smaller digits, or use seasonal imagery to convey key information. Since these references lack an obvious connection, they lend a more literary feel to the dialogue and, to some extent, raise the bar for recognition and detection.

By contrast, in a more concrete setting like illicit pet trade, example conversations may be closer to real‐life buying and selling procedures, which can make them appear more “suspicious”:
\begin{quote}
\textit{
“... about a vibrant 'tropical chatterbird' renowned for its brilliant plumage and uncanny mimicry ... I've also come into possession of a few 'Rising Sun coins' for exchange ...... Perhaps you might know a place where ...”
}
\end{quote}
In this dialogue, the term “tropical chatterbird” serves as an euphemism for a parrot, emphasizing its colorful appearance and mimicking ability without mentioning the animal directly. Meanwhile, “Rising Sun tokens” subtly alludes to the Japanese yen, since the Rising Sun is an iconic symbol of Japan. This coded language allows both parties to communicate their intentions regarding the acquisition of a rare bird and the intended payment method without explicitly revealing sensitive details. However, if these indirect expressions are used excessively, the dialogue may appear artificial or unnatural, potentially reducing its authenticity—thus affecting evaluations of both “regulatory avoidance” and “strategy existence.”
\setlength{\fboxrule}{0.5pt} 
\vspace{0.5em}
\noindent
\begin{tcolorbox}[colframe=black!20, colback=gray!10, arc=5pt, boxrule=0.5pt, width=0.99\linewidth]
\textit{Answer to RQ2}: Our evaluation confirms that the emergent language strategies closely resemble real-world language strategies, effectively employing euphemisms and implicit cues, and are generally understood by human reviewers. However, while these strategies show potential in simulations, they often appear forced or unnatural due to the fine-tuning of LLMs as commercial products, requiring refinement to better mimic the nuanced and fluid communication typical in real-world social interactions.

\end{tcolorbox}

\subsection{RQ3: Ablation Experiment}
\subsubsection{Experiment Settings}

To evaluate the effectiveness of the GA introduced in our framework, we conducted an ablation experiment using \textit{gpt-4o-mini} and \textit{gpt-4o} as the underlying LLM. For comparison, we employed the approach from our initial study \cite{DBLP:conf/cec/CaiLZLWT24}, which primarily differs in its strategy-update mechanism. In that earlier framework, the LLM is provided with both the existing strategy and newly flagged regulation violation records during the reflection stage, prompting the model to propose a new set of strategies that replace the old ones.
In contrast, our new framework employs a GA process where each strategy is treated as a discrete unit and optimized iteratively through GA. 

\subsubsection{Experiment Results}
As shown in Fig.~\ref{fig:ablation}, the GA-based framework demonstrates significant advantages. In the short-term experiment within the first 35 rounds, the w/o GA approach might show slight initial superiority due to the larger changes brought about by replacing the entire strategy. However, overall, w/ GA performs better than w/o GA. This difference increases as the number of rounds grows, particularly after round 35, where the advantages of w/ GA become even more pronounced. The GA process enables effective strategy evolution and adaptation, leading to an increased number of dialogue turns and improved accuracy, highlighting the framework's enhanced adaptability in the long term.
%Despite occasional performance dips during the evolutionary process, the GA framework’s ability to foster strategy diversity and handle complex scenarios makes it a more effective approach for sustained optimization.
\begin{figure}[h!]
    \centering
    \includegraphics[width=\linewidth]{figures/ablation1_v6.png}
    \caption{Performance with/without GA}
    \label{fig:ablation}
\end{figure}
\setlength{\fboxrule}{0.5pt} 
\vspace{0.5em}
\noindent
\begin{tcolorbox}[colframe=black!20, colback=gray!10, arc=5pt, boxrule=0.5pt, width=0.99\linewidth]
\textit{Answer to RQ3}: The results confirm the effectiveness of the GA component in our framework, especially when the number of rounds increases, where it demonstrates greater stability and adaptability. Although the optimization may be slower in the early stages, GA provides stronger adaptability in the long term through effective strategy evolution.
\end{tcolorbox}

\subsection{Discussion and Limitation}
In this study, we leveraged LLM agents to simulate the evolution of language strategies under regulatory pressure. While our results provide initial evidence that agents can adapt and develop covert communication tactics, the simulations also exhibit noteworthy instabilities. First, the inherent randomness of LLM generation can cause significant fluctuations in outcomes: the same prompts may yield different strategic responses, particularly when the experimental scale (number of agents or dialogue rounds) is limited. In our framework, LLMs not only generate dialogues but also determine strategies and regulatory responses; as a result, any stochasticity is compounded across multiple modules, making the final results sensitive to small variations in prompt inputs or random seeds. Although such variability partially reflects the diversity of real-world human behavior to some extent, it complicates the interpretation of findings in a controlled experimental setup.

A second limitation lies in the relatively narrow scope of language strategies observed. The agents predominantly relied on general-purpose evasive methods, such as analogies or implicit references, yet rarely produced fully “encrypted” or specialized code words that might arise in realistic cultural or social contexts. This outcome highlights the challenge that LLMs, pre-trained on broad domains and further refined via RLHF, are predisposed to generate text consistent with mainstream norms, thereby inhibiting the formation of highly unconventional or obscure expressions. Moreover, in scenarios where the training corpus lacks sufficient examples of subcultural or community-specific covert language, the model is less able to invent or adopt specialized linguistic forms. 

Finally, our experiments focused on one-to-one private interactions that emphasize regulatory evasion, without exploring the dynamics of public, many-to-many conversations where language strategies might evolve and propagate differently in a broader social context. While each participant agent does learn and adapt incrementally across dialogue rounds, real-world language evolution involves extensive, long-term propagation across diverse communities. Covert terms or code words may gradually gain acceptance, be modified by different user groups, or fade from use entirely. By contrast, the small-scale nature of our simulated dialogues means that emergent language strategies do not undergo the sustained diffusion and feedback processes characteristic of real social platforms, limiting the ecological validity of our findings.




%对于语言演化的社会类模拟仍然是一个未被开拓的领域,通过借助LLM优秀的自然语言处理能力,为这类自然语言的模拟带来了强大助力。然而伴随着实验也让我们发现LLM也会导致许多局限性。尽管通过实验初步证明了我们的框架的有效性。但同时伴随着实验也为我们带来了许多值得讨论的点。

%实验结果的不稳定性
%首先实验结果本身具有一定的不稳定性,而我们认为整个不稳定性的根源源自于LLM本身生成具有不确定性\cite{},在我们的框架中,LLM几乎参与到了所有环节。同样的violation log让同一个LLM在相同的设置内可能会总结出不同的constraint strategy。尽管这种不稳定性在现实中同样存在(例如不同的人采取不同的策略),同时也是作为模拟框架中非常重要的点,然而在本工作中的数量级的实验中这种不稳定性对结果的影响更为难以过滤。就像\ref{}中也证实的,这种LLM dirven agent的研究中在小数量级上的实验存在着不稳定性,我们认为目前的结果已经足够证实我们的框架可以初步模拟语言动态的学习和演化这一趋势,在今后工作中更大量级的实验中(例如数万数百万agent于更多的round数),我们有理由相信,整体趋势会更加稳定,不同llm的agent之间的性能差距会更加接近llm本身语义理解与生成的综合性能,

%模拟策略的局限性,
%从实验中我们观察到,agent模拟出的策略目前仍然主要集中于比喻类比等较为共通的方式。现实中语言的演化一般根植于当地的文化与经济背景等等因素。例如中文可以利用拼音来将汉字转化为对应的字母从而规避审查,而英文可能会更加积极的利用emoji来作为表达的替代从而规避监管。
%这些较为复杂的策略不仅需要对应环境的大量先验知识,在较为常见的语言中,LLM中训练所需的语料知识可能包含了这些,但是对于训练的数据集中欠缺的语种的知识LLM在不借助prompt的提示的情况下没有能力选择这些既存的策略。


%尽管LLM训练中的数据集可能存在这种更为隐晦的表达方式,首先LLM的RLHF\ref{}本身的训练方法导致了目前绝大多数的LLM为了保证生成文本的泛用性,被训练的更加愿意生成更符合大众的一般化输出文本,在不对LLM进行微调的前提下很难提高在这种特性领域的表现。
%LLM的表现严重依赖prompt的结构设计,提示词工程已经被证明可以有效提高LLM的某一方面能力,单次的基于prompt的模型交互很难实现多步推理或是规划。尽管我们的框架已经将语言演化这一现象解耦,通过多个模块来尽可能模拟人类在该环境中内在的动力学,但是目前的策略生成阶段
%这一部分在不适用复杂prompt工程的前提下LLM很难采用这种小众?特殊领域?的表达。
%对于模拟出的语言策略,我们发现很少的独特加密语言,因为这种需要两边有一套共用的体系,对于我们的模拟情景只有固定turn数的模拟很难形成意思传达。


%\jialong{第二是演化后的语言是如何的存活。我们只考虑了能不能躲避监管。但语言后续的存活和发展其实是更大范围的society的一个动态过程(而不是几个agent之间的交互),这一块可以结合那些上千LLM agent的研究框架来进行拓展}
%\jialong{这边可以多用语言学的角度来说不足之处}
%\jialong{第一个缺点是语言演化一般根植于根植于文化,经济背景,当地的文化背景。但我们的文章没有考虑特定文化背景下的演化。例如中文中可以借用拼音与汉字之间的关系来作为回避监管的方式,日语则可以通过XXX,英语则可以通过XXXX。未来可能要借助persona和role-play之类的设定来进一步拓展}

%更大规模的实验
%策略生成那里增加多步规划
%RAG提供更多语料
%


\section{Discussions and Limitations}\label{disscussion}

\noindent
\textbf{Potential Application Scenarios.} 
Beyond the example of the train-and-hotel problem mentioned in this paper, \texttt{IntegrateX} can be extended to a wide range of application scenarios. 
One promising use case is cross-chain flash loans \cite{tefagh2021ccfl}. 
Flash loans are atomic, uncollateralized lending protocols that allow users to borrow funds at nearly zero cost, perform other operations, and then repay the loan. 
However, these processes require the guarantee of overall atomicity, meaning that either all steps succeed or they all fail. 
Due to this requirement for atomicity, existing flash loan protocols are limited to intra-chain operations. 
With \texttt{IntegrateX}, which provides overall atomicity for cross-chain dApps, users can efficiently perform cross-chain flash loan operations.
Other promising application scenarios include, but are not limited to, cross-chain atomic arbitrage and cross-chain supply chain management.

\vspace{3pt}
\noindent
\textbf{Learning Cost for Developers.} 
The logic-state decoupling and fine-grained state lock mechanisms may slightly increase the development learning curve for smart contract developers. Fortunately, we have proposed a set of guidelines to assist developers, and the mechanisms are flexible (as discussed in Section \ref{subsec:LSD} and \ref{subsec:lock}), allowing developers to freely decide whether to implement them. Furthermore, we can provide formal documentation, SDKs, and other resources (following existing standards such as Wormhole \cite{wormhole} and IBC \cite{cosmos2019}) to guide developers in secure and efficient development and auditing, thereby reducing the learning curve. Additionally, incentive mechanisms (e.g., token rewards) are widely adopted in the industry to encourage developers to build and utilize our system. In future research, we could even explore AI-based semi-automated smart contract tools to further address this challenge.

Moreover, logic-state decoupling offers several additional benefits. For instance, it facilitates modular programming principles in smart contract development, which helps reduce subsequent upgrade and maintenance costs while improving contract security. When an issue arises in a specific contract module, developers can conduct targeted audits and resolve vulnerabilities efficiently. 


\vspace{3pt}
\noindent
\textbf{Support for Heterogeneous Chains.} 
\texttt{IntegrateX} currently supports blockchains that run different consensus protocols but share the same smart contract execution environment. 
As mentioned in the paper, to ensure cross-chain transaction security across blockchains with different consensus protocols, \texttt{IntegrateX} waits until consensus on the source chain is finalized (or highly likely to be finalized) before committing the cross-chain transaction to the target chain. 
Additionally, while this paper focuses on \texttt{IntegrateX}’s implementation on EVM-compatible blockchains, it can also be modified to operate between non-EVM-compatible blockchains that share the same smart contract execution environment.

However, a current limitation of \texttt{IntegrateX} is that it cannot operate between blockchains with different smart contract execution environments. 
Fortunately, this issue could potentially be addressed using advanced techniques such as code virtualization \cite{virtualization}. 
Expanding \texttt{IntegrateX} to support integrated execution across blockchains with different smart contract environments is a future research direction we aim to explore.

\vspace{3pt}
\noindent
\textbf{Trade-off Between Flexibility and Load Balancing.} 
In \texttt{IntegrateX}, cross-chain dApp providers have the flexibility to select any chain as the execution chain for integrated execution, based on their preferences. 
However, this flexibility may introduce a potential issue: if many cross-chain dApp providers choose the same chain as the execution chain, that chain could become a hotspot, potentially degrading its performance.
One possible solution is for a third party (e.g., \texttt{IntegrateX}) to manage load balancing by selecting the execution chain on behalf of the cross-chain dApp providers. 
However, this approach could introduce centralization risks. 
Additionally, as discussed in the paper, developers' choice of which chain to run their dApps on often involves considerations beyond performance, such as ecosystem compatibility and business partnerships.
How to better achieve load balancing and how to strike a trade-off between performance and flexibility are important questions that warrant future research.

\vspace{3pt}
\noindent
\textbf{Mitigating Malicious Application Layer Components.}
In public blockchain scenarios, there are common strategies to mitigate malicious behavior from application layer components (e.g., dApp providers, users). 
For instance, a malicious cross-chain dApp provider might attempt to maliciously lock certain states to prevent their usage by others. 
Such behavior can be countered using contract-based authorization or blacklisting mechanisms (widely used in existing dApp development \cite{etherscan}). 
For example, an intra-chain dApp provider can pre-arrange with a cross-chain dApp provider and authorize trusted cross-chain dApp providers (through their associated addresses) in their contracts, allowing only authorized cross-chain dApp providers to invoke and lock their states. 
Similarly, intra-chain dApp providers can blacklist specific cross-chain dApp providers in their contracts to block their interactions.
Additionally, gas fee mechanisms can serve as a deterrent to malicious application layer components attempting to launch flooding attacks against the blockchain.

\vspace{3pt}
\noindent
\textbf{Cross-Chain vs. Cross-Shard.}
Some existing works have explored the issue of cross-shard smart contract handling \cite{qi2024lightcross, li2022jenga}. 
However, cross-chain and cross-shard scenarios are fundamentally different, and their solutions cannot be directly applied to cross-chain contexts. 
The primary reasons are as follows: First, research on blockchain sharding typically involves modifications to the underlying system. 
In contrast, a key requirement of cross-chain protocols or systems is that they must not require modifications to the underlying blockchains, ensuring better compatibility with existing blockchain systems. 
Second, a blockchain sharding system usually has a beacon chain responsible for coordinating progress across shards, which is impractical in cross-chain protocols. 
After all, in cross-chain scenarios, each blockchain essentially belongs to a different system. 
Finally, in cross-chain contexts, blockchains are often heterogeneous (e.g., different consensus protocols), which is uncommon in blockchain sharding systems.

\vspace{3pt}
\noindent
\textbf{Inter-Chain Shared Security.}
In \texttt{IntegrateX}, each blockchain is assumed to have a proportion of malicious nodes lower than its fault tolerance threshold (i.e., they are secure). 
This is a widely accepted assumption in most existing works. 
However, recent research has begun exploring how to achieve secure cross-chain interoperability protocols in scenarios where individual blockchains may not be secure, by sharing security across multiple blockchains \cite{sheng2023trustboost}.
\texttt{IntegrateX} could adopt similar ideas through modifications to achieve shared security among blockchains. 
However, how to design and implement inter-chain shared security within \texttt{IntegrateX} while still maintaining efficiency is an important direction for future research.

\section{Concluding Remarks}
In this paper, we proposed a novel approach utilizing multimodal LLMs to generate gesture-aware speech recognition transcripts for patients with language disorders. Our framework integrates verbal speech and iconic gestures, enabling the generation of enriched transcripts that capture the latent meaning conveyed through both modalities. Through extensive experimentation, we demonstrated that the proposed method effectively contextualizes incomplete or disfluent speech by incorporating gesture information, leading to more accurate and meaningful representations of the speaker's intent. These findings highlight the potential of our approach to significantly contribute to the field of speech and language therapy, offering innovative tools that can enhance the quality of life for individuals with language disorders by facilitating better communication and assessment methods.

\subsection{Ethical Statement} 
Our dataset was obtained from AphasiaBank with the approval of the Institutional Review Board (IRB) and adheres to the data sharing guidelines set by TalkBank\footnote{https://talkbank.org/share/ethics.html}. This includes complying with the Ground Rules for all TalkBank databases, which are based on the American Psychological Association Code of Ethics~\cite{american2002ethical}.

\subsection{Limitation \& Future Work} 
%This study represents a preliminary investigation into using multimodal LLMs to generate gesture-aware speech recognition transcripts. 
While the results are promising, we recognize several limitations and outline our plans to extend this work further.

One primary limitation is the absence of a definitive ground truth for quantitative evaluation. Since our model generates transcripts by synthesizing speech and gesture data from scratch, traditional benchmarks, such as comparisons with standard speech recognition outputs, are insufficient. Moreover, existing original transcripts lack gesture annotations, making direct comparisons challenging. In future work, we aim to address this gap by collaborating with certified pathologists to conduct qualitative assessments, such as A-B preference tests, to evaluate the effectiveness of gesture-enriched transcripts in accurately conveying the speaker's intentions.

To support quantitative evaluations, we plan to develop novel metrics that assess transcript quality, including grammar accuracy, semantic consistency, and the integration of multimodal information. Such metrics will provide a more objective basis for assessing our model's performance and facilitate comparisons with other multimodal and unimodal approaches.

Another limitation of this study is its focus on structured gestures from a specific task, the Peanut Butter Sandwich Task. While this task offers a controlled context for testing our approach, it does not encompass the diversity of gestures and communication patterns seen in everyday scenarios. As part of our future work, we plan to expand the scope of our model to include tasks such as the Cinderella Story Recall Task~\cite{bird1996cinderella}, which involves unstructured and complex narrative gestures. This expansion will allow us to evaluate the adaptability and robustness of our model in handling varied linguistic and gestural contexts.

In summary, while this study establishes a strong foundation for gesture-aware speech recognition, we aim to refine and extend our methods through collaborative qualitative evaluations, the development of robust quantitative metrics, and broader task applications. These efforts will ensure that our approach continues to evolve, ultimately contributing to more effective communication tools and interventions for individuals with language disorders.





% \clearpage
\bibliographystyle{IEEEtran}
\bibliography{sample}
% \clearpage

% \appendix

\section{Appendix: Prompt}
\label{sec:appendix}
``Here is a sketch of an image. 
$\{input\_color\_mask\}$, while the rest of the white space is the background. 
I need you to infer details of the image based on the given sketch.
The details should include the possible background likely to be present with the $\{input\_color\_mask\}$, the attribute of each object (like wearing, texture, color etc.), the state (including action, posture, etc.) of each object, the direction of each object and the relationships between objects.

You should first analyze the mask carefully, considering the size, location, and relative position of each object mask. Ensure that specific actions are analyzed based on the mask, and infer each aspect with a reasoning process before providing the final output.
The final output format should be: $\{format\_example\}$, and you should refer to the example: $\{few\_shot\}$. You are going to complete the "" in each item, you need to complete them in multiple short phrases based on your above reasoning.

The state and relationship should be as detailed as possible while ensuring they align with the mask, formatted as: objectA action/spatial relation objectB, with both objectA and objectB included.
You should properly refer to some examples of attributes of object $\{attributes\}$ and relationships $\{relationships\}$.
Do not include words like `or', `possibly' in your final output, there should no ambiguity in your output.
Make sure all aspects of given mask is filled.''

\vspace{-30pt}
\begin{IEEEbiography}[{\includegraphics[width=1in,height=1.25in,clip,keepaspectratio]{Figures/people/ChaoyueYin.jpg}}]{Chaoyue Yin}
is currently a master candidate with Department of Computer Science and Engineering, Southern University of Science and Technology. 
He received his B.E. degree in computer science and technology from Southern University of Science and Technology in 2024. 
His research interests are mainly in blockchain sharding and interoperability protocol.
\end{IEEEbiography}
\vspace{-30pt}
\begin{IEEEbiography}[{\includegraphics[width=1in,height=1.25in,clip,keepaspectratio]{Figures/people/Mingzhe}}]{Mingzhe Li}
is currently a Scientist with the Institute of High Performance Computing (IHPC), A*STAR, Singapore.
He received his Ph.D. degree from the Department of Computer Science and Engineering, Hong Kong University of Science and Technology in 2022.
Prior to that, he received his B.E. degree from Southern University of Science and Technology.
His research interests are mainly in blockchain sharding, consensus protocol, blockchain application, network economics, and crowdsourcing.
\end{IEEEbiography}
\vspace{-30pt}
\begin{IEEEbiography}
[{\includegraphics[width=1in,height=1.25in,clip,keepaspectratio]{Figures/people/JinZhang}}]{Jin Zhang} 
is currently an associate professor with Department of Computer Science and Engineering, Southern University of Science and Technology. 
She received her B.E. and M.E. degrees in electronic engineering from Tsinghua University in 2004 and 2006, respectively, and received her Ph.D. degree in computer science from Hong Kong University of Science and Technology in 2009. 
% She was then employed in HKUST as a research assistant professor. 
Her research interests are mainly in mobile healthcare and wearable computing, wireless communication and networks, network economics, cognitive radio networks and dynamic spectrum management. 
% She has published more than 50 papers in top-level journals and conferences. 
% She is the Principle Investigator of several research projects funded by National Natural Science Foundation of China, Hong Kong Research Grants Council and Hong Kong Innovation and Technology Commission. 
\end{IEEEbiography}
\vspace{-30pt}
\begin{IEEEbiography}[{\includegraphics[width=1in,height=1.25in,clip,keepaspectratio]{Figures/people/YouLin}}]{You Lin}
is currently a master candidate with Department of Computer Science and Engineering, Southern University of Science and Technology. 
He received his B.E. degree in computer science and technology from Southern University of Science and Technology in 2021. 
His research interests are mainly in blockchain, network economics, and consensus protocols.
\end{IEEEbiography}
\vspace{-30pt}
\begin{IEEEbiography}[{\includegraphics[width=1in,height=1.25in,clip,keepaspectratio]{Figures/people/Qingsong.png}}]{Qingsong Wei}
received the PhD degree in computer science from the University of Electronic Science and Technologies of China, in 2004. He was with Tongji University as an assistant professor from 2004 to 2005. He is a Group Manager and principal scientist at the Institute of High Performance Computing, A*STAR, Singapore. His research interests include decentralized computing, Blockchain and federated learning. He is a senior member of the IEEE.
\end{IEEEbiography}
\vspace{-30pt}
\begin{IEEEbiography}[{\includegraphics[width=1in,height=1.25in,clip,keepaspectratio]{Figures/people/Rick.png}}]{Siow Mong Rick Goh}
received his Ph.D. degree in electrical and computer engineering from the National University of Singapore. He is the Director of the Computing and Intelligence (CI) Department, Institute of High Performance Computing, Agency for Science, Technology and Research, Singapore, where he leads a team of over 80 scientists in performing world-leading scientific research, developing technology to commercialization, and engaging and collaborating with industry. His current research interests include artificial intelligence, high-performance computing, blockchain, and federated learning.
\end{IEEEbiography}

\end{document}