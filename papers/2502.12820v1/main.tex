

\documentclass[lettersize,journal]{IEEEtran}


\usepackage{graphicx}  
\usepackage{subcaption}  
\usepackage{enumitem}
\usepackage{listings}
\usepackage{xcolor}
\usepackage{amsthm}
\newtheorem{theorem}{Theorem}

\usepackage{amsmath,amsfonts}
\usepackage{algorithmic}
\usepackage{array}
\usepackage[caption=false,font=normalsize,labelfont=sf,textfont=sf]{subfig}
\usepackage{textcomp}
\usepackage{stfloats}
\usepackage{url}
\usepackage{verbatim}
\usepackage{graphicx}
\hyphenation{op-tical net-works semi-conduc-tor IEEE-Xplore}
\def\BibTeX{{\rm B\kern-.05em{\sc i\kern-.025em b}\kern-.08em
    T\kern-.1667em\lower.7ex\hbox{E}\kern-.125emX}}
\usepackage{balance}
% \usepackage[numbers]{natbib} % 启用数字引用
% \bibliographystyle{plainnat}

\begin{document}

\title{Atomic Smart Contract Interoperability with High Efficiency via Cross-Chain Integrated Execution}
\author{
        Chaoyue~Yin,
        Mingzhe~Li,~\IEEEmembership{Member,~IEEE},
        Jin~Zhang,~\IEEEmembership{Member,~IEEE}, 
        You~Lin,
        Qingsong Wei,~\IEEEmembership{Senior Member,~IEEE},
        and Siow Mong Rick Goh,~\IEEEmembership{Senior Member,~IEEE}% <-this % stops a space

\IEEEcompsocitemizethanks{\IEEEcompsocthanksitem C. Yin is with the Shenzhen Key Laboratory of Safety and Security for Next Generation of Industrial Internet, Department of Computer Science and Engineering, Southern University of Science and Technology, Shenzhen 518055, China (email: 12432716@mail.sustech.edu.cn).
\IEEEcompsocthanksitem M. Li is with the Institute of High Performance Computing, A*STAR, Singapore (email: mlibn@connect.ust.hk, Li\_Mingzhe@ihpc.a-star.edu.sg).
\IEEEcompsocthanksitem J. Zhang and Y. Lin are with the Shenzhen Key Laboratory of Safety and Security for Next Generation of Industrial Internet, Department of Computer Science and Engineering, Southern University of Science and Technology, Shenzhen 518055, China (email: zhangj4@sustech.edu.cn, liny2021@mail.sustech.edu.cn).
\IEEEcompsocthanksitem Q. Wei and S. Goh are with the Institute of High Performance Computing (IHPC), Agency for Science, Technology and Research (A*STAR), Singapore (email: wei\_qingsong@ihpc.a-star.edu.sg, gohsm@ihpc.a-star.edu.sg).
% \IEEEcompsocthanksitem J. Zhang is with the Shenzhen Key Laboratory of Safety and Security for Next Generation of Industrial Internet, Department of Computer Science and Engineering, Southern University of Science and Technology, Shenzhen 518055, China (email: zhangj4@sustech.edu.cn).
\IEEEcompsocthanksitem C. Yin and M. Li are the co-first authors.
\IEEEcompsocthanksitem J. Zhang is the corresponding author.
}

}
% The paper headers
\markboth{IEEE Transactions on Parallel and Distributed Systems, VOL. XX, NO. XX, XX 2025}%
{Yin \MakeLowercase{\textit{et al.}}: Atomic Smart Contract Interoperability with High Efficiency via Cross-Chain Integrated Execution}

%%
%% The "author" command and its associated commands are used to define
%% the authors and their affiliations.
%% Of note is the shared affiliation of the first two authors, and the
%% "authornote" and "authornotemark" commands
%% used to denote shared contribution to the research.

%%
%% By default, the full list of authors will be used in the page
%% headers. Often, this list is too long, and will overlap
%% other information printed in the page headers. This command allows
%% the author to define a more concise list
%% of authors' names for this purpose.
% \renewcommand{\shortauthors}{Trovato et al.}

%%
%% The abstract is a short summary of the work to be presented in the
%% article.
\maketitle
\begin{abstract}
With the development of Ethereum, numerous blockchains compatible with Ethereum's execution environment (i.e., Ethereum Virtual Machine, EVM) have emerged. 
Developers can leverage smart contracts to run various complex decentralized applications on top of blockchains. 
However, the increasing number of EVM-compatible blockchains has introduced significant challenges in cross-chain interoperability, particularly in ensuring efficiency and atomicity for the whole cross-chain application. 
Existing solutions are \emph{either limited in guaranteeing overall atomicity for the cross-chain application, or inefficient due to the need for multiple rounds of cross-chain smart contract execution.}

To address this gap, we propose \texttt{IntegrateX}, an efficient cross-chain interoperability system that ensures the overall atomicity of cross-chain smart contract invocations. 
The core idea is to \emph{deploy the logic required for cross-chain execution onto a single blockchain, where it can be executed in an integrated manner. }
This allows cross-chain applications to perform all cross-chain logic efficiently within the same blockchain. 
\texttt{IntegrateX} consists of a \emph{cross-chain smart contract deployment protocol} and a \emph{cross-chain smart contract integrated execution protocol.}
% two primary protocols: the Hybrid Cross-Chain Smart Contract Deployment Protocol and the Cross-Chain Smart Contract Integrated Execution Protocol. 
The former achieves efficient and secure cross-chain deployment by decoupling smart contract logic from state, and employing an off-chain cross-chain deployment mechanism combined with on-chain cross-chain verification. 
The latter ensures atomicity of cross-chain invocations through a 2PC-based mechanism, and enhances performance through transaction aggregation and fine-grained state lock. 
We implement a prototype of \texttt{IntegrateX}. Extensive experiments demonstrate that it reduces up to 61.2\% latency compared to the state-of-the-art baseline while maintaining low gas consumption.

\end{abstract}

\begin{IEEEkeywords}
Blockchain Interoperability, Cross-Chain Integrated Execution, Efficient and Atomic Interoperability Protocol, Cross-Chain Smart Contract Invocation.
\end{IEEEkeywords}



% Copyright 2017 Sergei Tikhomirov, MIT License
% https://github.com/s-tikhomirov/solidity-latex-highlighting/



\definecolor{verylightgray}{rgb}{.97,.97,.97}

\lstdefinelanguage{Solidity}{
	keywords=[1]{anonymous, assembly, assert, balance, break, call, callcode, case, catch, class, constant, continue, constructor, contract, debugger, default, delegatecall, delete, do, else, emit, event, experimental, export, external, false, finally, for, function, gas, if, implements, import, in, indexed, instanceof, interface, internal, is, length, library, log0, log1, log2, log3, log4, memory, modifier, new, payable, pragma, private, protected, public, pure, push, require, return, returns, revert, selfdestruct, send, solidity, storage, struct, suicide, super, switch, then, this, throw, transfer, true, try, typeof, using, value, view, while, with, addmod, ecrecover, keccak256, mulmod, ripemd160, sha256, sha3}, % generic keywords including crypto operations
	keywordstyle=[1]\color{blue}\bfseries,
	keywords=[2]{address, bool, byte, bytes, bytes1, bytes2, bytes3, bytes4, bytes5, bytes6, bytes7, bytes8, bytes9, bytes10, bytes11, bytes12, bytes13, bytes14, bytes15, bytes16, bytes17, bytes18, bytes19, bytes20, bytes21, bytes22, bytes23, bytes24, bytes25, bytes26, bytes27, bytes28, bytes29, bytes30, bytes31, bytes32, enum, int, int8, int16, int24, int32, int40, int48, int56, int64, int72, int80, int88, int96, int104, int112, int120, int128, int136, int144, int152, int160, int168, int176, int184, int192, int200, int208, int216, int224, int232, int240, int248, int256, mapping, string, uint, uint8, uint16, uint24, uint32, uint40, uint48, uint56, uint64, uint72, uint80, uint88, uint96, uint104, uint112, uint120, uint128, uint136, uint144, uint152, uint160, uint168, uint176, uint184, uint192, uint200, uint208, uint216, uint224, uint232, uint240, uint248, uint256, var, void, ether, finney, szabo, wei, days, hours, minutes, seconds, weeks, years},	% types; money and time units
	keywordstyle=[2]\color{teal}\bfseries,
	keywords=[3]{block, blockhash, coinbase, difficulty, gaslimit, number, timestamp, msg, data, gas, sender, sig, value, now, tx, gasprice, origin},	% environment variables
	keywordstyle=[3]\color{violet}\bfseries,
	identifierstyle=\color{black},
	sensitive=true,
	comment=[l]{//},
	morecomment=[s]{/*}{*/},
	commentstyle=\color{gray}\ttfamily,
	stringstyle=\color{red}\ttfamily,
	morestring=[b]',
	morestring=[b]"
}

\lstset{
	language=Solidity,
	backgroundcolor=\color{verylightgray},
	extendedchars=true,
	basicstyle=\footnotesize\ttfamily,
	showstringspaces=false,
	showspaces=false,
	numbers=none,
	numberstyle=\footnotesize,
	numbersep=9pt,
	tabsize=2,
	breaklines=true,
	showtabs=false,
	captionpos=b
}

\section{Introduction} 

\IEEEPARstart{W}{ith} the advent of Bitcoin and Ethereum~\cite{bitcoin,eth}, 
% and the subsequent development of blockchain technology, 
we have witnessed the emergence of an increasing number of programmable blockchains
% capable of running smart contracts
~\cite{belchior2021pastSV,wang2023SV,huang2021survey,lohachab2021SV}. 
% Smart contracts are self-executing programs with the terms directly written into code. 
On these programmable blockchains, developers can write and deploy smart contracts to build various complex decentralized applications (dApps, such as DeFi, NFT, etc.~\cite{wenkai2022defiSV,nadini2021mapping}).
Among these programmable blockchains, those that share compatible smart contract execution environment with Ethereum (i.e., Ethereum Virtual Machine, a.k.a. EVM~\cite{ethereum_evm}) dominate the landscape, accounting for over 90\% of the total value locked \cite{chaintvl}. 
With the increasing number of EVM-compatible blockchains and the diversity of dApps running on each chain, the demand for cross-chain dApps has grown significantly~\cite{OU2022OV}. 
Cross-chain dApps refer to dApps that require \emph{\textbf{cross-chain smart contract invocations \emph{(}CCSCI\emph{)}}} and coordinated execution across multiple blockchains \cite{Falazi2024crosschain}.
% However, across numerous EVM-compatible blockchains, it has become an increasingly pressing issue of achieving \emph{atomic and efficient interoperability for \textbf{cross-chain smart contract invocations}}\textbf{ (\emph{CCSCI})}~\cite{Falazi2024crosschain}.

However, ensuring overall atomicity for the entire cross-chain dApp while efficiently handling CCSCI remains a critical challenge. 
As illustrated in Figure~\ref{compare} left, 
consider a classic train-and-hotel problem~\cite{train}. 
A user wants to book an outbound train ticket (on Train Chain) through a travel agency (on Agency Chain), then book a hotel (on Hotel Chain), followed by a return train ticket (again on Train Chain). 
The user wants to ensure that the entire series of CCSCI either all succeed or all fail, ensuring overall atomicity.
% consider a promising cross-chain flash loan scenario. 
% The user (on Chain A) wants to invoke the flash loan contract (on Chain B) to borrow funds for liquidation on the liquidation contract (on Chain C), and then repay the loan to the flash loan contract (on Chain B).
% Given the strong need for overall atomicity in flash loan scenarios, it is insufficient to only guarantee atomicity for a single cross-chain step. 
% Instead, the entire sequence of CCSCI (borrowing, liquidating, repaying, etc.) must either succeed or fail as a whole (ensuring overall atomicity).
More importantly, maintaining efficiency during CCSCI is crucial. 
Key considerations include how to minimize latency, reduce gas consumption (monetary cost), and improve concurrency during the CCSCI process.
More cross-chain dApp scenarios are discussed in Section \ref{disscussion}.

% Current cross-chain interoperability solutions suffer from various limitations. 
% % Many previous efforts primarily focus on atomic cross-chain token transfers and exchanges~\cite{2019atomicBEswap,Xu2021htlc,Luo2024crosschannel,Manevich2022ccas,tian2021enabling,herlihy2018atomic,deshpande2020privacy,thyagarajan2022universal,chen2024pacdam,yin2022sidechain,zamyatin2019xclaim}, without considering the invocation of general-purpose cross-chain smart contracts. 
% Some works propose more general cross-chain interoperability protocols. 
% They deploy bridging smart contracts across different blockchains to facilitate more versatile message transfers between chains~\cite{nissl2021towards,wood2016polkadot,cosmos2019,abebe2019enabling,darshan2023an,reigsbergen2023demo,ghosh2021leveraging,garoffolo2020zendoo}. 

% Existing CCSCI interoperability solutions generally either fail to ensure the overall atomicity of a cross-chain application or guarantee overall atomicity but with low efficiency. 
% For example, some works \cite{nissl2021towards,wood2016polkadot,cosmos2019,abebe2019enabling,darshan2023an,reigsbergen2023demo,ghosh2021leveraging,garoffolo2020zendoo} propose general cross-chain communication protocols that facilitate the transfer of information and data between blockchains through bridging smart contracts deployed on each blockchain.
% However, these approaches typically ensure at best the atomicity of single-step cross-chain interactions, but fail to guarantee overall atomicity for the entire cross-chain application.
% some of these protocols (e.g., blahblah) do not inherently guarantee atomic cross-chain interactions. 
% Ensuring atomicity requires extensions based on their protocols. 
% Other approaches can only ensure the atomicity of single-step cross-chain interactions, but fail to guarantee atomicity for the entire logic of more complex cross-chain applications.

% There are also solutions that attempt to address smart contract interoperability by moving or replicating the entire state of smart contracts across chains (a.k.a. smart contract portability)~\cite{fynn2020smom,westerkamp2022smartsync}. 
% This approach converts cross-chain contract invocations into intra-chain interactions. 
% However, its practicality is often called into question. 
% Decentralized applications usually choose to operate on specific blockchains based on several practical considerations, including ecosystem alignment or security concern. 
% Consequently, frequently relocating the entire state of smart contracts from one ecosystem to another is typically impractical and inefficient.
% not acceptable to developers. 
% Furthermore, smart contracts often manage a significant amount of user state, and frequently relocating all of this state across chains incurs substantial overhead. 
% This inefficiency further reduces the practicality of these approaches.

Existing CCSCI interoperability solutions generally \emph{either fail to ensure the overall atomicity of a cross-chain dApp or guarantee overall atomicity but with low efficiency. }
To ensure atomicity in the CCSCI process, a widely adopted approach is to use a \emph{two-phase commit (2PC) mechanism \cite{lampson1993twopc, Falazi2024crosschain} involving locking and unlocking}.
For example, some works \cite{nissl2021towards,wood2016polkadot,cosmos2019,abebe2019enabling,darshan2023an,reigsbergen2023demo,ghosh2021leveraging,garoffolo2020zendoo} propose general cross-chain communication protocols that facilitate the transfer of information and data between blockchains through bridging smart contracts deployed on each blockchain.
However, these approaches typically ensure at best the atomicity of single-step cross-chain interactions (Figure \ref{compare}, a single arrow), but fail to guarantee overall atomicity for the entire cross-chain dApp.
Some other recent studies attempt to explore how to ensure overall atomicity for the cross-chain dApp~\cite{robinson2021general,atomic-ibc,chen2024atomci,Falazi2024crosschain}. 
To ensure overall atomicity, the relevant states must remain locked throughout the entire CCSCI process.
However, they commonly face challenges in achieving efficiency. 
% either in maintaining atomicity or in achieving efficiency. 
% For instance, Hyperservice~\cite{liu2021hyperservice} only guarantees financial atomicity, which guarantees the atomicity of the final result by initiating new transactions to re-inject the results. However, it cannot roll back all states and thus cannot ensure the atomicity of a complete cross-chain invocation.
% On the other hand, 
% For instance, works such as GPACT~\cite{robinson2021general} ensure atomicity for entire cross-chain dApps, but their cross-chain interoperability protocols suffer from efficiency issues.
The main reason is that, these approaches usually require \emph{multiple rounds of cross-chain execution and cross-chain information transfer} when handling a cross-chain dApp (Figure \ref{compare}, left), since a cross-chain dApp usually involves interdependent execution logic distributed over multiple blockchains.
% A complex cross-chain dApp typically involves multiple blockchains and several interdependent dApp logic components.
% % , which are implemented through smart contracts. 
% In these works, handling such complexity requires \emph{multiple rounds of cross-chain execution and cross-chain interaction in sequential order across several involved chains. }
% % Specifically, this involves fragmenting the execution logic, reaching consensus, and transferring intermediate states to the next blockchain responsible for executing the subsequent logic fragment. 
It is evident that this method tends to be time-consuming and inefficient (as the locking time could be long), especially when dealing with complex CCSCI. 
More related work (e.g., cross-chain asset swap, smart contract portability) is discussed and differentiated in detail in Section~\ref{related_work}.
% Further related work is discussed in Section~\ref{related_work}.

To fill the research gap, we propose \texttt{IntegrateX}, an \emph{\textbf{efficient} interoperability system that guarantees \textbf{overall atomicity}} for the cross-chain dApp across EVM-compatible blockchains. 
To enhance the efficiency of CCSCI, our core idea is to \emph{clone and deploy\footnote{The logic on the original chain still exists and continues to function normally.} the logic of all contracts involved in a cross-chain dApp—originally distributed across multiple chains—onto a single blockchain}.
% The core idea behind enhancing the efficiency of CCSCI is that, for a cross-chain dApp, \emph{conducting cross-chain deployments for the logic of each involved contract onto a single blockchain.}
This chain thus integrates the entire execution logic of the cross-chain dApp. 
When the CCSCI is required, this chain can perform \emph{\textbf{integrated execution} in one transaction} for all related logic, after receiving the necessary states (Figure \ref{compare}, right). 
% within its environment after receiving the necessary states. 
Since multi-round cross-chain executions and interactions are no longer required, \texttt{IntegrateX} maintains high efficiency, even in complex cross-chain dApps.
% It is important to note that our approach differs from the concept of smart contract portability. 
% In \texttt{IntegrateX}, the states of individual smart contracts are still maintained by their original blockchains. 
% We only “borrow” the execution environment of a single chain to integrate and execute the cross-chain logic. 
% However, the design of \texttt{IntegrateX} is not straightforward, as it faces several significant challenges, which we will address below.

\begin{figure}[t]
    \centering
    \includegraphics[width=0.525\textwidth]{Figures/compare1.png}
    \vspace{-18pt}
    \caption{An example of existing CCSCI solutions (left) and \texttt{IntegrateX} (right) in the Train-and-Hotel scenario.}
    % \vspace{-10pt}
    \label{compare}
\end{figure} 

The design of \texttt{IntegrateX} faces several challenges, which we will address below.

\vspace{3pt}
\noindent
\textbf{Challenge 1: }
\textit{How to efficiently and securely deploy smart contracts across chains to the same blockchain.}
We address this by proposing an on-chain/off-chain \textbf{\emph{Hybrid Cross-Chain Smart Contract Deployment Protocol}}.
Specifically, to reduce the overhead associated with cross-chain deployment of smart contracts (thereby improving efficiency), we devise a set of guidelines that allow developers to \emph{decouple smart contracts into logic execution contracts and state storage contracts. }
This approach enables our protocol to clone and deploy only the logic contracts onto the same target blockchain, while the state-heavy contracts remain in their original locations.
% However, cross-chain deployment of smart contracts cannot be achieved through existing interoperability protocols (e.g., [blah]) using bridging smart contracts, because smart contracts cannot deploy other smart contracts. 
To further reduce the gas consumption during cross-chain deployment, we propose to \emph{clone and deploy the logic contracts to the same blockchain via an off-chain mechanism.}
However, malicious actors may tamper with the contracts during the off-chain clone and deployment process, leading to compromised security. 
To address this, we design and deploy bridging smart contracts on each blockchain. 
These bridging smart contracts perform \emph{on-chain cross-chain comparison and verification} of the logic contracts between the source chain and the target chain to ensure security.
% This, however, raises a security concern, as malicious actors could interfere during the off-chain migration and deployment process, leading to inconsistencies in the deployed contracts.
% To ensure the consistency of cross-chain deployment logic, we design and deploy bridging smart contracts for verification across the involved chains. 
% These bridging smart contracts perform \emph{cross-chain comparison and verification on-chain}, ensuring that the logic between the source chain and the destination chain remains consistent during the cross-chain deployment process.

\vspace{3pt}
\noindent
\textbf{Challenge 2: }
\textit{How to enhance concurrency and reduce overhead during CCSCI while ensuring overall atomicity.}
We address this by proposing a \textbf{\emph{Cross-Chain Smart Contract Integrated Execution Protocol}}.
Specifically, to ensure overall atomicity during CCSCI, we adopt a 2PC-based mechanism, similar to existing mainstream approaches.
This process involves locking the relevant states across the involved chains, transmitting the states to the chain responsible for integrated execution, executing all logic, and then returning the states to the respective chains to unlock and update them. 
This guarantees that the cross-chain dApp either completes entirely or is fully rolled back.
Moreover, cross-chain state transfers may incur significant overhead. 
To mitigate the \emph{\textbf{overhead}} of cross-chain communication, we \emph{aggregate multiple cross-chain transactions into a single one} where necessary, and transmit them across chains.
More importantly, a straightforward state-locking mechanism reduces transaction concurrency. 
To enhance \emph{\textbf{concurrency}}, we establish a set of guidelines that allow developers to further \emph{decompose and lock contract states at a finer granularity.} 
By locking more granular states, our protocol alleviates the issue of poor concurrency that arises when entire states are locked.


This paper mainly makes the following contributions:
% \vspace{-9pt}
\begin{itemize}[left=0pt]
    \item We present \texttt{IntegrateX}, a cross-chain interoperability system that efficiently facilitates CCSCI while ensuring overall atomicity for the cross-chain dApp. 
    \texttt{IntegrateX} can be flexibly deployed on EVM-compatible blockchains without requiring modifications to the underlying blockchain systems. 
    \item In \texttt{IntegrateX}, we propose the Hybrid Cross-Chain Smart Contract Deployment Protocol. 
    It achieves efficient and secure cross-chain deployment through the decoupling of smart contract logic and state, and the hybrid approach of off-chain logic deployment and on-chain comparison verification.
    \item In \texttt{IntegrateX}, we propose the Cross-Chain Smart Contract Integrated Execution Protocol. 
    It ensures overall atomicity of cross-chain invocations through a 2PC-based atomic integrated execution mechanism, and enhances the protocol efficiency through an aggregated cross-chain transaction transmission mechanism and fine-grained state lock.
    \item We implement a prototype of \texttt{IntegrateX} and make it open source~\cite{INTEGRATEX}. 
    Extensive experiments based on real-world use cases demonstrate that \texttt{IntegrateX} reduces latency by up to 61.2\% meanwhile maintains low gas cost and high concurrency compared to the state-of-the-art baseline. 
    In more complex cross-chain invocations, \texttt{IntegrateX} will further improve efficiency.
\end{itemize}
% Contributions. 
% We present \texttt{IntegrateX}, a cross-chain interoperability system that efficiently facilitates cross-chain smart contract invocations while ensuring the atomicity of entire cross-chain dApps. 
% \texttt{IntegrateX} can be flexibly deployed on EVM-compatible blockchains without requiring any modifications to the underlying blockchain systems. 
% Decentralized dApp developers only need to follow our established standards when developing or upgrading their smart contracts to take advantage of \texttt{IntegrateX}'s interoperability features.
% \texttt{IntegrateX} consists of two key protocols. 
% The first is the Hybrid Cross-Chain Smart Contract Deployment Protocol, which achieves efficient and secure cross-chain deployment through the decoupling of smart contract logic and state, and the hybrid approach of off-chain logic deployment and on-chain comparison verification.
% The second is the Cross-Chain Smart Contract Integrated Execution Protocol, which ensures the atomicity of cross-chain invocations through a 2PC-based mechanism, and enhances the protocol efficiency through fine-grained state locks and an aggregated cross-chain transaction transmission mechanism.
% We implement a prototype of \texttt{IntegrateX} and make it open source. 
% Extensive experiments based on real-world use cases demonstrate that \texttt{IntegrateX} delivers exceptional performance.

% \paragraph{Data-to-Text.} \citep{kukich-d2t, mckeown-d2t} is the task of converting structured data into fluent text. These structured data may correspond to tables \citep{totto}, meaning representations \citep{e2e}, relational graphs \citep{webnlg2017}, etc.
% %This complex format poses a significant challenge to LLMs pre-trained on plain text. 
% Recent approaches to data-to-text typically involve training end-to-end models with encoder-decoder architectures \citep{wiseman-etal-2017-challenges, gardent2017creating,RebuffelSSG20,RebuffelSSG20-ECIR,RebuffelRSSCG22}. Notably, using large pre-trained encoder-decoder models \citep{t5} has significantly improved performance by framing data-to-text as a text-to-text task \citep{kale-rastogi-2020-text, duong23a}. More recently, large pre-trained decoder-only models \citep{llama2} have shown strong performance and become the de facto approach for text generation, now being applied to data-to-text \citep{tablellama}. Despite these advancements, LLMs still struggle with hallucinations, and data-to-text generation is no exception.
This section reviews methods aimed at improving the faithfulness of LLMs to input contexts. We focus exclusively on approaches designed to ensure the generated content remains grounded in the provided information, excluding techniques related to factuality or external knowledge alignment.

\paragraph{Faithfulness enhancement.} Several methods have been used for improving faithfulness of text summarization. A first line of work consist in using external tools to retrieve key entities or facts form the source document and use these as weak labels during training \citep{zhang-etal-2022-improving-faithfulness}. \citet{faitful-improv} identify key entities using a Question-Answering system and modify the architecture of an encoder-decoder model to put more cross-attention weight on these entities. \citet{zhu-etal-2021-enhancing} propose to improve the faithfulness of summaries by extracting a knowledge graph from the input texts and embed it in the model cross-attention using a graph-transformer. Another line of work focuses on post-training improvements by bootstrapping model-generated outputs ranked by quality \citep{slic,brio,slic-nli}.
% \citet{zhang-etal-2022-improving-faithfulness} forces , \citet{faitful-improv} introduce a Question-Answering system enhanced encoder-decoder architecture, where the cross-attention in the decoder is directed towards key entities. \citet{zhu-etal-2021-enhancing} propose to improve the faithfulness of summaries by extracting a knowledge graph from the input texts and embed it in the model cross-attention using a graph-transformer.
Regarding data-to-text generation, \citet{RebuffelRSSCG22} propose a custom model architecture to reduce the effect of loosely aligned datasets, using token-level annotations and a multi-branch decoder model. The closest work to ours is from \citep{cao-wang-2021-cliff} which proposes a contrastive learning approach where synthetic samples are constructed using different tools like Named Entity Recognition (NER) models and back-translation.
%These approaches have been primarily designed and evaluated for text summarization. 
These approaches address specific forms of unfaithfulness and rely heavily on external tools such as NER or QA models, and are especially tailored for text summarization, while we target a more general focus. More recently, simpler methods that leverage only a pre-trained model have been proposed for summarization. \citet{cad,pmi} downweight the probabilities of tokens that are not grounded in the input context, using an auxiliary LM without access to the input context.
\citet{critic-driven} train a self-supervised classification model to detect hallucinations and guide the decoding process.  \cite{confident-decoding} propose a method to estimate the decoder's confidence by analyzing cross-attention weights, encouraging greater focus on the source during generation. Our method focuses on a decoder-only architecture and uses a single model, providing a streamlined and efficient approach specifically tailored for general conditional text generation tasks without the need for complex external tools.

\paragraph{Faithfulness evaluation.} Measuring faithfulness automatically is not straightforward. Traditional conditional text generation evaluation often relies on comparing the generated output to a reference text, typically measured using n-gram based metrics such as BLEU \citep{papineni-bleu} or ROUGE \citep{lin-2004-rouge}. However, reference-based metrics limitations are well known to correlate poorly with faithfulness \citep{fabbri-etal-2021-summeval,gabriel-etal-2021-go}. Both for summarization and data-to-text generation, new metrics evaluating the generation exclusively against the input context have been proposed, using QA models \citep{rebuffel-etal-2021-data,scialom-etal-2021-questeval} or entity-matching metrics \citep{nan-etal-2021-entity}. In this work, we evaluate primarily our models using recent NLI-related metrics \citep{alignscore, nli-d2t}, and LLM-as-a-judge, focusing on faithfulness \citep{gpt-chiang,gpt-gilardi}. For data-to-text generation, we also report the PARENT metric \citep{parent}, which computes n-gram overlap against elements of the source table cells.

%Additionally, corpora are often collected automatically, leading to divergences between the reference text and the actual input data. , since no direct comparison to the actual input source is actually performed. To address these issues, evaluation methods that take into account the input data have been proposed. \citet{parent} introduce PARENT, which computes the recall of n-gram overlap between the entities in the data and the candidate text. \citet{nli-d2t} develop an entailment metric using Natural Language Inference (NLI) models, where the generated text is compared directly to a simple verbalization of the data. The gold-standard still remains the human or human-like evaluation, conducted with powerful generalist LLMs. These metrics form the core focus of our work.

\paragraph{Preference tuning.} Recent instruction-tuned LLMs are often further refined through "human-feedback alignment" \citep{oaif}. These methods utilize human-crafted preference datasets, consisting of pairs of preferred and dispreferred texts $(\ywin, \ylose)$, typically obtained by collecting human feedback and ranking responses via voting. Recent work \citep{spin} uses the model's previous predictions in a self-play manner to iteratively improve the performance of chat-based models. Whether through an auxiliary preference model \citep{rlhf} or by directly tuning the models on the pairs \citep{dpo}, these approaches have demonstrated remarkable results in chat-based models. Our method leverages a preference framework without the need for human intervention and is specifically tailored for models trained on conditional text generation tasks.

% However, it remains unclear on what values the models are being aligned. Some works have shown that these methods can effectively alter the model's behaviour to the extent that they become useless and refuse to answer to any requests. In this work, we follow a preference fine-tuning scheme but tailored for input-aware tasks like data-to-text.


\section{System Overview}

In this section, we provide an overview of \texttt{IntegrateX}. 
\texttt{IntegrateX}'s architecture and workflow overview is shown in Figure \ref{overview}.
For a detailed analysis of \texttt{IntegrateX}'s security and further discussion, please refer to Section~\ref{security_analysis} and Section~\ref{disscussion}.

\subsection{System and Threat Model}

In \texttt{IntegrateX}, there are $n$ (variable) blockchains that share the same smart contract execution environment (e.g., EVM). 
These blockchains are developed by their respective projects (e.g., Ethereum, BSC, etc.~\cite{eth,bsc_whitepaper}) and operated by their own blockchain nodes, which handle transaction processing, consensus within the blockchain, and other tasks. 
\texttt{IntegrateX} also features $m$ (variable) \emph{relayers} responsible for trustless cross-chain communication between blockchains, similar to many mainstream interoperability protocols \cite{atomic-ibc, cosmos2019, sheng2023trustboost, tas2023interchain}. 
Each relayer has its own public-private key pair and signs cross-chain transactions.
Additionally, \texttt{IntegrateX}'s \emph{bridging smart contracts} are deployed on each blockchain (similar to on-chain light clients \cite{cosmos2019, atomic-ibc}). 
% The functionality of these smart contracts is similar to the on-chain light clients in some well-known cross-chain communication protocols (e.g., IBC)~\cite{cosmos2019, atomic-ibc}. 
% These contracts serve as verifiable bridges that connect on-chain smart contracts with off-chain environments. 
The bridging contracts mainly serve to verify state transitions on the other blockchains, and send information externally via event emissions.
% On-chain contracts use these bridging contracts to send information externally via event emissions.
% Cross-chain transactions must also go through these bridging contracts to verify their validity and interact with other on-chain smart contracts.
% Furthermore, multiple nodes on each blockchain run \texttt{IntegrateX}’s \emph{cross-chain light clients}, which are primarily responsible for synchronizing block headers across chains and assisting the bridging smart contracts in verification. 
The above \texttt{IntegrateX}'s transport layer architecture is similar to some well-known cross-chain communication protocols (e.g., IBC)~\cite{cosmos2019, atomic-ibc}, which guarantee basic security during cross-chain communication. 
However, \texttt{IntegrateX} achieves efficiency and overall atomicity at the application layer, which these other protocols do not.

\texttt{IntegrateX} also includes several \emph{intra-chain dApp providers}. 
These providers can freely choose to deploy their contracts and run their intra-chain dApps on different blockchains. 
Additionally, there are \emph{cross-chain dApp providers} who can similarly flexibly choose to deploy their cross-chain dApps across various blockchains. 
A cross-chain dApp typically consists of multiple intra-chain dApps distributed across different blockchains, and relies on \texttt{IntegrateX}'s cross-chain protocol to ensure efficient CCSCI and maintain the overall atomicity of the cross-chain dApp.
Finally, there are also \emph{users} within the \texttt{IntegrateX} ecosystem. 
These users interact with the cross-chain dApps by sending transactions to the blockchains in order to use these cross-chain dApps.

\vspace{3pt}
\noindent
\textbf{Threat Model.}
In \texttt{IntegrateX}, we make only minimal trust assumptions about the relayers, assuming that at least one relayer is honest and functioning correctly. 
This assumption is consistent with those made in many existing secure interoperability protocols \cite{atomic-ibc, cosmos2019, sheng2023trustboost, tas2023interchain}. 
For each blockchain, the proportion of Byzantine nodes is assumed to be less than the fault tolerance threshold $t$ of the respective blockchain network (e.g., for blockchains using BFT-type consensus protocols under partial-synchronous network, $t=1/3$~\cite{pbft}).
% ; for blockchains using Nakamoto-type consensus protocols, $t=1/2$~\cite{ren2019analysis}). 
This ensures the safety and liveness of consensus within each blockchain.
As for the dApp providers and users—who represent the application layer components—we make no specific threat assumptions, similar to most existing works~\cite{cosmos2019,chen2024atomci,robinson2021general}. 
However, we do discuss common countermeasures for dealing with malicious behavior from these components in Section~\ref{disscussion}.

\subsection{Objective}

\texttt{IntegrateX} aims to achieve the following primary objectives:
\begin{itemize}[left=0pt]
\item \textbf{Efficiency}: During the process of cross-chain smart contract deployment and invocation, \texttt{IntegrateX} seeks to reduce latency, lower gas consumption, and increase transaction concurrency.
\item \textbf{Overall Atomicity}: \texttt{IntegrateX} aims to ensure overall atomicity for cross-chain dApps that require it during CCSCI. This means guaranteeing that the series of CCSCI operations required by cross-chain dApp providers either all succeed or all fail.
\end{itemize}

In addition, \texttt{IntegrateX} aims to possess the following desirable properties. 
\emph{Reliability}: \texttt{IntegrateX} ensures that cross-chain transactions can still be completed even in the presence of malicious relayers. 
\emph{Verifiability}: \texttt{IntegrateX} guarantees that cross-chain transactions can be verified for authenticity, completeness, and validity, even if malicious relayers are involved. 
\emph{Consistency}: \texttt{IntegrateX} ensures that the state changes across the blockchains involved in the cross-chain transaction remain coordinated and consistent. 

The proofs and experiments related to aforementioned properties are detailed in Section~\ref{security_analysis} and Section~\ref{evaluation}.



\begin{figure}[t]
    \centering
    \includegraphics[width=0.51\textwidth]{Figures/overview3.png}
    \vspace{-18pt}
    \caption{An overview of \texttt{IntegrateX}. 
    % Execution Chain: the chain responsible for integrated execution.
    % ; invoked chain: the chain where the invoked contracts are deployed.
    %four roles: intra-chain DApp providers, cross-chain DApp providers, users, and relayer.
    }
    %\vspace{-10pt}
    \label{overview}
\end{figure} 



\subsection{Primary Workflow}

The operation of \texttt{IntegrateX} consists of three main phases: \emph{Smart Contract Preparation}, \emph{Cross-Chain Smart Contract (CCSC) Deployment}, and \emph{Cross-Chain Smart Contract (CCSC) Integrated Execution}, as illustrated in Figure \ref{overview}.

% The smart contract preparation phase only occurs when dApp providers need to develop or upgrade smart contracts, so the frequency of this phase is relatively low. 
% Similarly, the cross-chain logic contract deployment phase only runs when cross-chain dApp providers need to deploy logic contracts across chains onto the same blockchain, which also occurs infrequently. 
% In contrast, the cross-chain smart contract invocation and integrated execution phase runs every time a user needs to interact with a cross-chain dApp, making it the most frequent phase of operation.

\vspace{3pt}
\noindent
\textbf{Smart Contract Preparation.} 
This phase only occurs when dApp providers need to develop or upgrade smart contracts, so the frequency of this phase is relatively low. 
In this phase, dApp providers can flexibly develop logic contracts and state contracts according to our defined logic-state decoupling guidelines (Section \ref{subsec:LSD}). 
This decoupling is primarily intended to reduce gas consumption and minimize storage costs on the target chain by cloning only the logic contracts during the subsequent CCSC deployment phase. 
Additionally, dApp providers can develop their contracts according to our fine-grained state lock guidelines (Section \ref{subsec:lock}), which enhances transaction concurrency during the CCSC integrated execution phase. 
Once the smart contract development is complete, dApp providers can freely choose a blockchain to deploy their contracts. 

\vspace{3pt}
\noindent
\textbf{CCSC Deployment.} 
This phase only runs when cross-chain dApp providers need to deploy logic contracts across chains onto the same blockchain, which also occurs infrequently. 
In this phase, cross-chain dApp providers can flexibly choose a target chain for deployment. 
By sending a transaction, they can clone and deploy the specified logic contracts from other designated chains onto the chosen target chain based on their cross-chain dApp needs. 
\texttt{IntegrateX} introduces an off-chain clone approach (Section \ref{subsec:migration}) to reduce gas consumption, while on-chain cross-chain verification (Section \ref{subsec:verification}) ensures that the cloned contracts are identical to the original ones.

\vspace{3pt}
\noindent
\textbf{CCSC Integrated Execution.}
This phase runs every time a user needs to interact with a cross-chain dApp, making it the most frequent phase of operation.
In this phase, users interact with cross-chain dApps by sending transactions to the target chain, invoking cross-chain smart contracts based on the dApp’s application logic. 
\texttt{IntegrateX} employs an atomic integrated execution scheme (Section \ref{subsec:execution}), similar to the 2PC protocol, to ensure the overall atomicity of the series of CCSCI involved in a cross-chain dApp.
This scheme first locks the relevant states of the smart contracts involved in the cross-chain dApp on the respective chains at a fine-grained level, 
then transmits these states across chains to the target chain. 
Since the target chain already contains all the execution logic required for the cross-chain dApp (from the earlier phase), it can integrate and execute all logic within one transaction once the necessary states have been received. 
After execution, the new states are returned to the corresponding chains for unlocking and updating.
Additionally, \texttt{IntegrateX} employs a transaction aggregation mechanism (Section \ref{subsec:aggregation}) during this phase to reduce cross-chain overhead.

In the subsequent descriptions within this paper, we will refer to the target chain selected by a cross-chain dApp for deployment—i.e., the chain responsible for integrated execution—as the \textbf{\emph{execution chain}}. 
The other chains associated with the cross-chain dApp will be collectively referred to as \textbf{\emph{invoked chains}}.

\section{Hybrid Cross-Chain Smart Contract Deployment Protocol}

% We propose an on-/off-chain Hybrid Cross-Chain Smart Contract Deployment Protocol to efficiently and securely deploy smart contracts across chains. In this section, we will provide a detailed explanation of the design of the protocol.

\subsection{Logic-State Decoupling}
\label{subsec:LSD}

To efficiently perform CCSCI, we need to deploy the logic of the invoked contracts to the same execution chain for efficient integrated execution. 
Existing contracts often contain both logic and state, and directly cloning such contracts would result in high gas costs and additional state storage overhead on the execution chain. 
Therefore, we design a set of Logic-State Decoupling (LSD) Guidelines to guide the developers to decouple existing contracts into logic execution contracts and state storage contracts. 
During cross-chain deployment, only the logic contracts need to be cloned, which reduces gas costs. 
Developers can follow these guidelines to develop new contracts with separated logic and state or upgrade existing contracts by decoupling them. 
We now provide a detailed explanation of the LSD Guidelines.
Moreover, a simple example illustrating the LSD is provided in Section \ref{codeex}.

\vspace{3pt}
\noindent
\textbf{State Contract.}
According to the LSD Guidelines, the decoupled state contract first includes all the \emph{variables} (representing states) from the original contract, as well as all the \emph{view functions} that read the contract's state. 
Since view functions are read-only and do not generate transactions, they do not affect cross-chain execution. 
Additionally, the state contract contains \emph{functions for locking and updating} the contract state, as cross-chain invocations require locking and updating states. 
The state contract should also contain \emph{functions that call the logic contract} in order to interact normally with the logic contract.
% To support normal on-chain calls, the state contract should also include functions that invoke the logic contract to execute the DApp's regular functionality.

\vspace{3pt}
\noindent
\textbf{Logic Contract.}
In the logic contract, no variables are stored. 
It only contains the functions that implement the original contract's \emph{execution logic}. 
These functions are called by the state contract to carry out the dApp's logic operations. 
When the state contract calls these functions, it passes all necessary state data (variables), and after the functions complete execution, the results are returned to the state contract. 
In our protocol, only the logic contract needs to be cloned, which reduces gas costs during cross-chain deployment.

\vspace{3pt}
\noindent
\emph{Remarks.}
Our protocol also supports existing smart contracts, even if they are not decoupled into logic and state. 
However, in such cases, the cross-chain deployment process will incur higher gas costs. 
% Furthermore, although the contracts migrated to the execution chain do not maintain contract states, they will be passively updated during each cross-chain call, leading to additional state storage costs.
In this case, the contracts deployed to the execution chain do not actively maintain their own state. 
Instead, they passively update their state during each cross-chain invocation. 
% This approach reduces the cost associated with active state updates.
The discussion related to developers' learning costs is given in Section \ref{disscussion}.



\vspace{3pt}
\subsubsection{Logic-State Decoupling Example}
\label{codeex}

We now give a sample of logic-state decoupling. In the following code~\ref{ex}, after applying logic-state decoupling, the original Hotel contract is split into two separate contracts: the logic contract \texttt{LHotel} and the state contract \texttt{SHotel}. The \texttt{LHotel} contract contains no state variables and only includes the \texttt{book()} function, which implements the hotel reservation functionality. Since there are no variables within the \texttt{LHotel} contract, the \texttt{book()} function must take all necessary parameters as inputs.

On the other hand, the \texttt{SHotel} contract retains all the variables from the original Hotel contract and introduces an additional address variable, \texttt{addr\_lhotel}, which records the address of the \texttt{LHotel} contract. In the \texttt{book()} function of the \texttt{SHotel} contract, no reservation logic is implemented directly; instead, it calls the \texttt{book()} function from \texttt{LHotel} using the \texttt{addr\_lhotel} parameter to execute the hotel reservation functionality.

By decoupling the logic and state in this way, only the \texttt{LHotel} contract needs to be cloned. 
% during logic updates. 
Since \texttt{LHotel} contains no state variables, this approach significantly reduces gas consumption during the cross-chain clone and deployment process.

\begin{lstlisting}[language=Solidity, caption={Pseudocode of Hotel Logic-State Decoupling}, label={ex}]
contract Hotel{
    int256 price;
    int256 remain;
    mapping (address => int256) accounts;
    function getPrice() public view returns(uint256); 
    function getRemain() public view returns(uint256); 
    function book(address user_addr, uint256 num) public returns(uint256); 
    function LockState(bytes[] memory args) public returns();
    function UpdateSteate(bytes[] memory args) public returns();

contract LHotel{
    function book(uint256 price, uint256 remain, uint256 num) public returns(uint256)
}

contract SHotel{
    int256 price;
    int256 remain;
    address addr_lhotel;
    mapping (address => int256) accounts;
    function getPrice() public view returns(uint256); 
    function getRemain() public view returns(uint256); 
    function LockState(bytes[] memory args) public returns();
    function UpdateSteate(bytes[] memory args) public returns();
    function book(address user_addr, uint256 num) public returns(uint256);

\end{lstlisting}




\begin{figure}[t]
    \centering
    % 第一个子图
    \begin{subfigure}[b]{0.525\textwidth}
        \centering
        \includegraphics[width=\textwidth]{Figures/migration2.png}  % 使用你的图片路径
        \vspace{-10pt}
        \caption{Off-chain clone and deployment}
        \label{migration}
    \end{subfigure}
     % 第二个子图
    \begin{subfigure}[b]{0.525\textwidth}
        \centering
        \includegraphics[width=\textwidth]{Figures/verification2.png}  % 使用你的图片路径
        \vspace{-10pt}
        \caption{On-chain verification}
        \label{verifaction}  % 子图标签,用于引用
    \end{subfigure}
    \vspace{-18pt}
    \caption{Hybrid Cross-Chain Smart Contract Deployment Protocol.}
    \label{MandV}  % 整体图的标签
\end{figure}



\subsection{Off-Chain Clone and Deployment}
\label{subsec:migration}

Transmitting contract bytecode on-chain (via contract event) requires a significant amount of gas.
% ensures security, but it requires a significant amount of gas and incurs delays due to the need for blockchain consensus. 
To improve the efficiency of cross-chain logic deployment and reduce gas consumption, we propose \emph{transferring the contract bytecode via an off-chain solution}. 
The off-chain clone and deployment consists of two phases: the preparation phase and the clone phase. 
% In the following, we will explain the off-chain migration process through these two phases. 
The main process is shown in Figure ~\ref{migration}.

\vspace{3pt}
\noindent
\textbf{Preparation.}
In this phase, the cross-chain dApp provider needs to obtain the call tree of smart contracts and determine which logic contracts need to be cloned.
This can be achieved by using some static analysis tools \cite{feist2019slither} and other methods, similar to existing works \cite{li2022jenga, robinson2021general, chen2024atomci}.
% can use static analysis tools, such as Slither~\cite{feist2019slither}, to obtain the call tree of smart contracts and determine which logic contracts need to be migrated. 
After this, the developer can choose a blockchain to deploy the cross-chain dApp. 
It is important to note that we offer developers a high degree of flexibility: 
They can select any blockchain according to their preference. 
To reduce costs, they may also choose a blockchain that already contains some of the invoked logic contracts. 
Once the selection is made, the developer sends a transaction to invoke the bridging contract on the chosen chain.
The bridging contract then triggers an event to notify the relayers to initiate the cross-chain deployment.
% which triggers an event for contract migration. 
The event includes the ID of the invoked chain and the addresses of the logic contracts $Addr_{\text{L}}$ to be cloned on the invoked chain. 
This concludes the preparation phase.

\vspace{3pt}
\noindent
\textbf{Clone.}
After the event is triggered, relayers will detect the event and use the \texttt{getcode()} function from the bridging contract on the invoked chain to obtain the bytecode of the contract that needs to be cloned. 
This process is an \emph{off-chain read-only inquiry and, therefore, does not consume gas.}
Subsequently, the relayers will obtain the ABI file which defines the contract interface. 
The relayers will then deploy the contract to the execution chain. 
Once a relayer completes the redeployment, it registers the address of the cloned logic contract $Addr'_{\text{L}}$, through the bridging contract on the execution chain, marking the end of the clone phase.

\vspace{3pt}
\noindent
\emph{Remarks.}
For the cross-chain reliability of the protocol, multiple relayers perform the clone process after detecting the bridging contract's event. 
Therefore, even if some relayers do not respond to the event, other relayers will ultimately complete the contract clone and deployment.
Once one relayer completes the deployment, the bridging contract will trigger an event to stop the others, ensuring that the logic contract will not be deployed multiple times.

\subsection{On-Chain Verification}
\label{subsec:verification}

Although off-chain clone and deployment can achieve efficient cross-chain logic deployment, it does not guarantee security as there may be malicious relayers present in the system.
A malicious relayer could potentially modify the contract bytecode or deploy a wrong contract. 
To ensure verifiability and security during the off-chain clone and deployment process,
% of the cloned and deployed logic contract, 
% we propose the on-chain cross-chain verification scheme to confirm the correctness of the cloned and deployed logic contract. 
we propose the on-chain cross-chain verification scheme to \emph{compare the cloned logic contract with the original contract and verify its correctness}.
The on-chain verification process is shown in Figure ~\ref{verifaction}.

After the cross-chain deployment is completed, the cross-chain dApp provider can initiate cross-chain verification on the execution chain by calling the \texttt{Verification()} function of the bridging contract. 
The bridging contract will search the cloned contract bytecode of address $Addr'_{\text{L}}$, and calculate the hash of the bytecode. 
Then the bridging contract triggers an event includes the hash computation result. 
% The blockchain node will generate the Merkle proof~\cite{merkle1987} to verify the validity of the transaction.
The relayers are responsible for transmitting this information (a cross-chain transaction) along with its Merkle proof \cite{merkle1987} to the invoked chain. 
In the invoked chain, the Merkle proof with the cross-chain transaction is first verified to ensure the result has already reached consensus on the execution chain. 
The bridging contract then searches the corresponding local contract bytecode based on the address $Addr_{\text{L}}$, and calculates the hash of the bytecode.
Then, the bridging contract will verify whether it matches the hash transmitted across the chain. 
% calculates the hash of the local contract bytecode based on the address $Addr_{\text{L}}$ and verifies whether it matches the hash transmitted across the chain. 
The result of the verification is then returned.

If the verification is successful, the cloned contract will be marked as verified and allowed for subsequent cross-chain invocations, and the relayer that performed the cross-chain deployment will be rewarded. 
If the verification fails, the relayer will be penalized, and the off-chain clone and deployment process will be restart. 

\noindent
\emph{Remarks.}
For the reliability of the cross-chain protocol, multiple relayers could transmit the same cross-chain transaction. 
However, the bridging contracts will deduplicate identical transactions from multiple relayers to avoid multiple executions on-chain.
This process also applies in the subsequent integrated execution protocol.
\section{Cross-Chain Smart Contract Integrated Execution Protocol}

% To enhance concurrency and reduce overhead
% of CCSCI while ensuring atomicity. We propose
% a Cross-Chain Smart Contract Integrated Execution Protocol. This protocol ensures high efficiency and atomicity throughout the entire call process through the Three-Phase Integrated Execution mechanism. In addition, we have incorporated finer transaction aggregation mechanisms and  granularity lock within the protocol to further enhance the efficiency of the \texttt{IntegrateX} system in complex cross-chain smart contract call scenarios.

\subsection{Atomic Integrated Execution}
\label{subsec:execution}

To efficiently and atomically execute complex CCSCI, we propose an atomic integrated execution scheme. 
As all the invoked logic has been migrated onto the execution chain, the atomic integrated execution does not need multiple rounds of cross-chain execution when handling CCSCI.
This enables all the logic to be executed within a single transaction, enhancing the efficiency of CCSCI. 
Moreover, to ensure overall atomicity for a series of CCSCI, we employ a state synchronization mechanism based on the Two-Phase Commit protocol~\cite{lampson1993twopc}. 
This process locks the relevant states across the involved chains, transmits the states to the chain responsible for integrated execution, and then returns the states to the respective chains to unlock and update them after the execution is completed. 
The entire atomic integrated execution consists Locking, Integrated Execution and Updating, which is shown in the Figure~\ref{ta}.

\vspace{3pt}
\noindent
\textbf{Locking. }
The Locking process locks all the required invoked contract states on the invoked chain.
The cross-chain dApp provider can obtain all the required state on the call tree beforehand via tools such as static analysis \cite{feist2019slither}.
Based on this information, a user sends a transaction via the cross-chain dApp to invoke the cross-chain dApp contract. 
% During this process, a user first sends a transaction to the cross-chain dApp contract. 
% the dApp contract can obtain all the required state on the call tree. 
The cross-chain dApp contract then calls the bridging contract to issue an event to lock the relevant states on the invoked chains.
% will issue an event to lock the relevant states on the invoked chain through the bridging contract. 
When relayers detect the event, it will transfer this message (i.e., cross-chain transaction with Merkle proof) to each invoked chain. 
The bridging contract on the invoked chain will verify the authenticity of the cross-chain though calculating the Merkle proof of the transaction and invoke the \texttt{LockState()} function of each invoked contract. 
Once the bridging contract has successfully locked the state and retrieved the required contract states, it will trigger an event to return the states. 
After the relayers detect the event, they will transmit these states via cross-chain transactions to the execution chain. 
After the execution chain's bridging contract verifies the authenticity of the transactions via the Merkle proof, it will return the states to the dApp.
Once all requested states are returned from individual invoked chains, the Locking process ends, and these states are used as inputs for the Integrated Execution.

\vspace{3pt}
\noindent
\textbf{Integrated Execution. }
The Integrate Execution process executes the entire CCSCI logic on the execution chain.
Cross-chain dApp contracts on the execution chain use the requested state values as inputs to perform the full call tree execution. 
Since all contracts required for the cross-chain invocation have completed logic migration and have been verified already, the integrated execution can be completed within a single transaction on the execution chain. 
The cross-chain dApp contract records the output results of each invoked contract during the Integrated Execution, allowing for state updates of the invoked contracts on other chains after the execution is completed. 
Once execution is complete, the Integrated Execution ends and transitions to the Updating process.

\vspace{3pt}
\noindent
\textbf{Updating. }
The Updating process unlocks and updates all the invoked contract states on the invoked chains.
After Integrated Execution is completed on the execution chain, the cross-chain dApp contract triggers an event to update the result via bridging contract. 
The relayers will distribute the result to the invoked chains. 
The bridging contract on each invoked chain verifies the Merkle proof of the cross-chain transaction and 
then invokes the \texttt{UpdateState()} function of each invoked contract, which will unlock and update the states of the invoked contracts.

\vspace{3pt}
\noindent
\emph{Remarks:}
\textbf{\emph{Rollback.}}
During the Atomic Integrated Execution, state rollback may occur due to failure in locking the state or execution failure. 
The invoked state might already be locked by another cross-chain invocation, which causes the failure to lock the invoked state. 
In this case, the bridging contract on the execution chain will initiate a new event to unlock all associated contracts that have already been locked and returned in this cross-chain invocation. 
For the contracts that have not yet completed the locking process, the event will cancel the lock attempt, thereby ensuring overall atomicity.
The execution might also fail, due to the reason such as insufficient gas fee or insufficient states.
% In the case of execution failure, 
% if the obtained states are insufficient to complete the cross-chain call, the call will fail. 
In this case, the bridging contract on the execution chain will discard the obtained states and initiate a cross-chain event to unlock all locked contracts, thus ensuring all invoked contracts are either fully locked or not locked at all.

\begin{figure}[t]
    \centering
    \includegraphics[width=0.51\textwidth]{Figures/3PIE1.png}
    \vspace{-18pt}
    \caption{The CCSCI process in sequential invocation (left) and in \texttt{IntegrateX}'s Cross-Chain Smart Contract Integrated
Execution Protocol (right).}
    % An illustration of CCSCI process of sequential invocation and invocation in IntegrateX.}
    \label{ta}
     
\end{figure} 

% \noindent
\textbf{\emph{Timeout.}}
Additionally, to prevent the invoked contract state from being locked for an extended period, a design of timeout will be determined by the dApp developer within the application and managed by the execution chain. 
For transactions that time out, such as when the locking is not completed or execution is not finished for an extended period, the execution chain will mark the cross-chain call as failed and send a cross-chain event to unlock the relevant contracts.
% That is because in a straightforward state-locking mechanism, 
% states locking can lead to a reduction in overall system call efficiency. To mitigate the impact of state locking and enhance the system's overall call efficiency and concurrency, we have designed a finer granularity lock. Once a contract state on the calling chain is requested, the requested state is locked, causing subsequent calls that attempt to change the state to fail due to the lock. To minimize such call conflicts and improve the overall efficiency of \texttt{IntegrateX}, we have implemented a finer granularity lock. finer granularity lock allows the state to be partially locked in a more granular manner, enabling it to be locked by multiple requests simultaneously, thereby increasing overall call efficiency. For contracts with frequent intra-chain calls, the finer granularity lock not only enhances cross-chain interoperability but also reduces intra-chain call blocking caused by cross-chain calls.

\textbf{\emph{Finality.}}
In \texttt{IntegrateX}, all cross-chain transactions must wait until the consensus on the initiating chain is finalized (or, for Nakamoto-type consensus, highly likely to be finalized) before being committed to the target chain, to ensure cross-chain security.

\subsection{Transaction Aggregation}
\label{subsec:aggregation}

A large amount of cross-chain transactions for transmitting state incurs significant gas consumption. 
As shown in Figure~\ref{ta} left, handling CCSCI by sequential invocation may need to invoke contracts on the same chain in multiple rounds, which cause \emph{multiple rounds of cross-chain state transfers}. 
To address this issue, we design a transaction aggregation mechanism to reduce gas costs caused by multiple invocations of different contracts and state transfers on the same chain.
% To mitigate the gas consumption in this situation, we design the transaction aggregation mechanism to reduce gas consumption for cross-chain invocations involving contracts on the same chain for multiple times.

% Existing methods for ensuring atomicity in cross-chain invocation are sequential, meaning that calls are processed step by step, as shown in the Figure~\ref{ta}. Therefore, they need multiple rounds of cross-chain execution and cross-chain interaction in sequential order across several involved chains. Therefore, when invoking multiple contract states on the same invoked chain, it is necessary to access that chain multiple times. However, 
As shown in Figure~\ref{ta} right, our protocol locks all required states simultaneously, allowing multiple states on the same chain to be locked together, even when the calls are non-contiguous. 
The transaction aggregation mechanism combines all state requests on the same chain into a single transaction, reducing the number of cross-chain transactions. 
Similarly, during state updates, the transaction aggregation mechanism reduces the number of update requests. 
Therefore, for cross-chain calls involving multiple contracts on the same chain, this approach ensures that the number of cross-chain transactions is equal to the number of invoked chains (rather than the number of contracts), thereby reducing gas consumption.

\subsection{Fine-Grained State Lock}
\label{subsec:lock}

During the atomic integrate execution process, we need to lock contract states to ensure atomicity. 
A simple approach is to either lock the entire state of the invoked contract or lock individual states being invoked~\cite{chen2024atomci, robinson2021general}. 
However, these state-locking mechanism reduces transaction concurrency.
Because once a state is locked, any subsequent transactions related to that state will fail until the state is unlocked. 
Therefore, we establish a set of guidelines to guide developers in decomposing certain states that can be split into finer granularity, and allowing \emph{partial state locking}. 
Unlike existing protocols that require locking the entire state, our fine-grained state lock mechanism locks only partial of a state at a fine-grained level. This approach enhances concurrency by reducing unnecessary state locking.

In EVM-compatible blockchains, Solidity is the smart contract language. 
In Solidity, the state of a contract is typically represented by \emph{variables}. 
Various types of variables are used in Solidity, such as \texttt{uint}, \texttt{address}, and \texttt{boolean}. 
% We observe that \texttt{uint} variables are the most widely applied. 
We find that the \texttt{uint} variable is widely used and can be decomposed.
Based on this observation, we design a fine-grained state lock specifically for \texttt{uint} variables and develop a lock pool mechanism. 
The lock pool is a structure where, during the use of the fine-grained state lock, part of the state is locked within this structure until execution is completed, while the unlocked portion of this state remains accessible, thereby enhancing the concurrency of the application. 

% For variables directly used based on user input, the exact value of the required state can be precisely determined. 
For variables that can be directly derived from transaction inputs, the exact value of the required state can be precisely determined.
A fine-grained state lock can be applied to accurately lock only the relevant portion of such state.
For states that are dynamically used during execution, their exact values cannot be determined at the beginning. 
We allow dApp developers to lock these states in fixed-size increments based on their needs.
% For states dynamically used during execution, which can not be precisely determined at the beginning, the fine-grained state lock will lock them in fixed-size increments. 

\vspace{3pt}
\noindent
\emph{Remarks.}
We focus on the flexibility of this mechanism, allowing developers to choose whether to implement the finer granularity locking mechanism in their dApp and to set the lock granularity based on the specific needs of the dApp. 
Moreover, developers can set the fixed size of the fine-grained state lock based on their preferences, tailored to the specific use case of the application.
More discussion related to developers' learning costs is given in Section \ref{disscussion}.
% 在Hybrid Cross-Chain Smart Contract Deployment Protocol 协议中,relayer对需要被跨链调用intra-chain DApp的逻辑合约进行了迁移。relayer在bridging合约中通过已经部署的合约地址获得其字节码,合约的 API 文件用来定义与合约交互的接口,通常在智能合约部署时就会公开,因此可以直接获取。通过 API 文件以及链上合约的字节码,relayer可以重新在其他区块链上部署该合约。在IntegratedX系统中,Hybrid Cross-Chain Smart Contract Deployment Protocol 协议内的任意relayer都可以进行迁移操作,然而这些relayer并不一定是可信任的,这就导致了可能有恶意的relayer在迁移过程中故意更改字节码造成安全性漏洞,因此我们通过on-chain verification进行链上合约验证来保证字节码没有被更改。
% 我们设计使用bridging合约比较迁移后合约的字节码哈希值和原合约的字节码哈希值。合约字节码哈希值可以通过智能合约的链内调用获得,并且将会由relayer传输到intra-chain DApp所在的链上,由于存在至少一个可以运行的relayer,该哈希值最终会被传输到目标链上并由目标链进行验证并将验证结果回传。由于区块链数据的不可篡改性,使用链上智能合约验证可以保证验证结果无法被更改。只有通过验证的合约才会完成完整的登记流程并可以被跨链应用链内调用。由于区块链数据具有可追溯性,因此对于验证通过的合约,进行该次迁移的relayer将会获得奖励,如果验证失败,将由新的relayer进行迁移,并对迁移失败的relayre进行惩罚。惩罚措施包含建立黑名单机制,一定次数的验证失败将无法进行迁移且此次迁移的费用将会由relayer承担。通过这样的on-chain verification,可以保证逻辑合约迁移后的正确性。由于至少存在一个relayer可以正常工作,因此该逻辑合约最后一定会被成功迁移。
\section{Security Analysis}\label{security_analysis}
% In this section, we analyzed how \texttt{IntegrateX} can safely and efficiently perform complex cross-chain calls.

\subsection{Security in Hybrid Cross-Chain Smart Contract Deployment Protocol}

\begin{theorem}
The Hybrid Cross-Chain Smart Contract Deployment Protocol ensures reliability, verifiability, and consistency, if the proportion of Byzantine nodes in each blockchain is less than its fault tolerance threshold, and at least one functional relayer is present.
% As long as the number of Byzantine nodes in each blockchain remains within the Byzantine fault tolerance limits, and at least one functional relayer is present, the reliability, verifiability, and consistency of the Hybrid Cross-Chain Smart Contract Deployment Protocol can thus be assured.
\end{theorem}

\begin{proof}
\noindent
\textbf{Reliability. }
The Hybrid Cross-Chain Smart Contract Deployment Protocol ensures reliability, meaning that when an execution chain issues a request for cross-chain deployment of a contract, the requested contract will eventually be deployed to the execution chain. 
Within the Hybrid Cross-Chain Smart Contract Deployment Protocol, multiple relayers listen for such requests. Even if malicious relayers intentionally ignore the requests, assuming that at least one functional relayer exists, this relayer will ultimately handle the logic clone and deployment of the requested contract. Thus, even in the worst-case scenario, the protocol ensures that the contract will be successfully deployed to the execution chain.

\vspace{3pt}
\noindent
\textbf{Verifiability. }
The Hybrid Cross-Chain Smart Contract Deployment Protocol ensures verifiability, meaning that both the execution chain and the invoked chain are able to verify the cross-chain transactions transmitted by the relayers. 
Additionally, both chains can verify the hash values of the contract bytecode before and after the cross-chain clone and deployment to ensure that the contract has been deployed correctly. 

In the Hybrid Cross-Chain Smart Contract Deployment Protocol, multiple relayers listen to the request and relay the messages. 
% However, relayers are not responsible for verifying the cross-chain transactions. 
% Instead, 
The invoked chain (bridging contract) validates the authenticity of the transactions using the Merkle proof attached with the cross-chain transactions, thereby preventing the relayers from altering the cross-chain data. 
By parsing the transactions, the invoked chain can obtain the bytecode hash of the cloned contract and compare it with the bytecode hash of the original contract on the chain to ensure the correctness of the cross-chain deployment. 
Additionally, the nonce value associated with each transaction prevents malicious replay attacks.

\vspace{3pt}
\noindent
\textbf{Consistency. }
The Hybrid Cross-Chain Smart Contract Deployment Protocol ensures consistency, meaning that during the off-chain clone and deployment as well as the on-chain verification process, both the execution chain and the invoked chain reach agreement on the outcome of the cross-chain requests.
% meaning that the agreement between the execution chain and the invoked chain on the outcome of the cross-chain request during off-chain clone and deployment and on-chain verification. 
In the Hybrid Cross-Chain Smart Contract Deployment Protocol, the proportion of malicious nodes on both the execution chain and the invoked chain remains within the fault tolerance threshold.
Moreover, the protocol requires to wait until consensus on one chain is finalized (or highly likely to be finalized) before committing the cross-chain transaction to another chain. 
% both the execution chain and the invoked chain maintain within the Byzantine fault tolerance limits. 
As a result, even in the presence of Byzantine nodes, each chain can still achieve consensus on cross-chain transactions and finalize them, ensuring that consistency is not undermined by malicious nodes. 
This guarantees that the outcomes of cross-chain requests remain consistent.
\end{proof}

\subsection{Security in Cross-Chain Smart Contract Integrated Execution Protocol}
%Cross-Chain Smart Contract Integrated Execution Protocol实现了跨链调用的原子执行,IntegrateX系统通过类似于 2PC 的状态同步机制对invoked的合约进行状态锁定和更新,在cross-chain应用中当用户在execution chain上的调用cross-chain DApp合约后,bridging合约将向所有invoked chain上的invoked合约发送状态请求,如果任何一个invoked合约未能锁定状态,将会返回失败信息,bridging合约将会取消本次调用并且发送跨链请求解锁所有其他合约,被锁定的合约将被解锁,而当尚未完成锁定的合约先接收到了解锁请求,将会在忽略后续接收到的该次锁定请求,以此保证合约状态锁定的一致性。
% 当所有被请求的合约都成功锁定状态后,cross-chain DApp合约将在execution chain上进行集成执行并将结果返回给所有的invoked合约,invoked 合约的新的状态将会首先返回给invoked chain上的bridging合约,并由bridging合约返回状态更新成功信息,如果任意一条invoked chain上的状态更新失败,所有链上的bridging合约都会丢弃更新的状态,此次跨链调用将失败,只有所有invoked chain上都完成了状态更新,cross-chain DApp合约才会输出最终结果,invoked合约解锁状态并从本链的bridging合约进行状态更新,标志着此次调用成功,由此保证完整的跨链调用原子性。
\begin{theorem}
The Hybrid Cross-Chain Smart Contract Integrated Execution Protocol ensures overall atomicity, reliability, verifiability, and consistency, if the proportion of Byzantine nodes in each blockchain is less than its fault tolerance threshold, and at least one functional relayer is present.
% As long as the number of Byzantine nodes in each blockchain remains within the Byzantine fault tolerance limits, and at least one functional relayer is present, the atomicity, reliability, verifiability, and consistency of the Cross-Chain Smart Contract Integrated Execution Protocol can thus be assured.
\end{theorem}

\begin{proof}
\noindent
\textbf{Overall Atomicity. }
The Cross-Chain Smart Contract Integrated Execution Protocol guarantees overall atomicity, which means that in the selected CCSCI process, state changes on both the execution chain and the invoked chain either all succeed or all fail, preventing any situation where one chain's state changes while the other does not. 
We employ the atomic integrated execution mechanism, similar to the 2PC scheme. 
For one cross-chain dApp, during this process, all the states required by the invocation on the invoked chains will be locked. 
If any contract has already been locked by another invocation, this invocation will fail, and all other invoked contracts will be unlocked. 
If the execution chain obtains all the necessary states but the execution fails due to insufficient gas or other reasons, the execution will be aborted, and all related locked contracts will be unlocked without any state changes. 
Furthermore, the protocol incorporates a timeout scheme: 
If any of the invoked chains fails to return the required state within the specified time frame by dApp, or if the execution transaction on the execution chain fails to complete within the specified time limit, the execution chain will abort the invocation and unlock all related contract states, ignoring any subsequent state responses from the invoked chains. 
% In cases where an invoked chain receives a lock request after receiving an unlock request, it will also ignore the lock request, ensuring that all invoked contract states are either locked or unlocked simultaneously. 
As a result, only when the execution chain has successfully acquired all required states and completed execution will it issue state updates to all related invoked chains, thereby ensuring the atomicity of the entire CCSCI process.

\vspace{3pt}
\noindent
\textbf{Reliability. }
The Cross-Chain Smart Contract Integrated Execution Protocol ensures reliability, which means that when an execution chain initiates a CCSCI request, the invoked chain will eventually receive the cross-chain transaction, and the state returned by the invoked chain will likewise be received by the execution chain. 
In the Cross-Chain Smart Contract Integrated Execution Protocol, multiple relayers monitor CCSCI requests. 
Even in the presence of malicious relayers who deliberately fail to respond to the request, the assumption of at least one functional relayer ensures that, in the worst-case scenario, this relayer will relay the cross-chain transaction to the invoked chain, ensuring that the transaction is eventually received. 
Similarly, even in the worst-case scenario, at least one relayer will transmit the state returned by the invoked chain back to the execution chain, thereby guaranteeing the reliability of the CCSCI process.

\vspace{3pt}
\noindent
\textbf{Verifiability. }
The Cross-Chain Smart Contract Integrated Execution Protocol ensures verifiability, which means the ability of the invoked chain to verify the authenticity of cross-chain transactions transmitted by relayers from the execution chain, while the execution chain can also verify the transactions returned by the invoked chain. 
In this protocol, multiple relayers listen for the request and relay messages.
% , but relayers themselves are not responsible for verifying the cross-chain transactions. 
Both the execution chain and the invoked chain can independently validate the authenticity of the cross-chain transactions using the Merkle proof attached with the transactions. 
Additionally, the use of transaction nonce values prevents malicious replay attacks, ensuring the integrity of the cross-chain interaction.

\vspace{3pt}
\noindent
\textbf{Consistency. }
The Cross-Chain Smart Contract Integrated Execution Protocol guarantees consistency, ensuring that both the execution chain and the invoked chain agree on the result of the requested operation during a cross-chain smart contract invocation. 
In this protocol, the proportion of malicious nodes on both chains remains within the fault tolerance threshold, allowing consensus to be achieved even if there exist Byzantine nodes. 
Additionally, the protocol requires waiting until consensus on one chain has been finalized before committing the cross-chain transaction to the other chain.
Therefore, cross-chain transactions on the blockchain cannot be maliciously altered, ensuring that both the execution chain and the invoked chain reach a unified agreement on the outcome of cross-chain operations.
\end{proof}


% \subsection{Efficiency Between \texttt{IntegrateX} and GPACT}
% %在\texttt{IntegrateX}中通过迁移合约逻辑将跨链调用转换为链内调用可以提高调用效率,因为通过集成执行的方式,合约的状态将会被同步锁定,提高了不同invoked chain上对合约状态并行处理的效率,从而减少了链式调用的等待时间,从而达到减少整体跨链调用latency的目的。考虑在GPACT协议agency应用的调用过程,如图所示。
% %在GPACT中进行跨链调用的过程中,整个跨链调用过程是一个串联调用过程,由于一次跨链调用需要发送receipt来确定信息是否已经送达,因此一次跨链通信时间至少需要2个区块时间,分别为接收方获得跨链信息所需的1个区块时间和发送方确认receipt所需的1个区块时间。在GPACT中跨链调用必须从调用树的根节点合约进行状态锁定,并在到达叶子节点合约时遵从叶子节点到根节点的执行顺序进行执行,假设调用树的深度为d,则此过程需要2d次信息传递,在执行结束后的更新合约状态则需要进行1次跨链信息传递,因此总共需要2d+1次信息传递,即4d+2个区块时间,考虑在调用根节点合约时还需要一个区块时间处理该交易,因此总共需要4d+3个区块时间。在agency应用中,调用树的深度为2,因此需要进行5次跨链信息传递 ,总共需要11个区块时间。
% In \texttt{IntegrateX}, by migrating contract logic, cross-chain calls can be converted into intra-chain calls, which improves call efficiency. This is because, through integrated execution, contract states are synchronously locked, enhancing the efficiency of parallel processing of contract states on different invoked chains. This reduces the waiting time associated with chained calls, ultimately lowering the overall latency of cross-chain calls. Consider the call process in the agency application under the GPACT protocol, as illustrated in Fig. \ref{gpact}.
% \begin{figure}[htbp]
%     \centering
%     \includegraphics[width=0.35\textwidth]{Figures/GPACT_process.png}
%     \caption{The process of the agency application in the GPACT protocol.}
%     \label{gpact}
% \end{figure} 

% In GPACT, the cross-chain call process is sequential. Since each cross-chain call requires sending a receipt to confirm whether the information has been delivered, a single cross-chain communication requires at least 2 block times: 1 block time for the recipient to receive the cross-chain information and 1 block time for the sender to confirm the receipt. In GPACT, cross-chain calls must begin by locking the state at the root contract of the call tree, and the execution follows the sequence from the leaf node contract back to the root node contract. Assuming the depth of the call tree is $d$, this process requires $2d$ transmissions of information. After the execution, updating the contract state requires an additional cross-chain transmission, making a total of $2d+1$ transmissions, which corresponds to $4d+2$ block times. Considering that processing the transaction at the root contract also requires 1 block time, the total time required is $4d+3$ block times. In the agency application, where the depth of the call tree is 2, 5 cross-chain transmissions are needed, resulting in a total of 11 block times.

% %在\texttt{IntegrateX}中进行跨链调用的过程中,整个跨链调用过程是一个并行调用过程,如图所示。在\texttt{IntegrateX}中,跨链调用的调用树的根节点合约将同时对所有调用树中的其他节点合约发起状态锁定请求,并在获得所有其他节点合约状态后在本链进行integrated execution,并在执行结束后同时更新其他节点合约的状态。假设调用树的深度为d,则锁定和获取其他合约状态需要2次信息传递,在执行结束后的更新合约状态则需要进行1次跨链信息传递,因此总共需要3次信息传递,即6个区块时间,考虑在调用根节点合约时还需要一个区块时间处理该交易,因此总共需要7个区块时间,是一个常数,与调用树深度无关。
% In \texttt{IntegrateX}, the cross-chain call process is parallel, as illustrated in Fig. \ref{IntegrateX}. In \texttt{IntegrateX}, the root contract of the call tree simultaneously sends state lock requests to all other node contracts in the call tree. After obtaining the states of all other node contracts, integrated execution is performed on the same chain, and after execution, the states of the other node contracts are updated simultaneously. Assuming the depth of the call tree is $d$, locking and obtaining the states of other contracts require 2 transmissions of information. After execution, updating the contract states requires an additional cross-chain transmission, making a total of 3 transmissions, corresponding to 6 block times. Considering that processing the transaction at the root contract requires 1 block time, the total time needed is 7 block times, which is a constant and independent of the depth of the call tree.
% \begin{figure}[htbp]
%     \centering
%     \includegraphics[width=0.35\textwidth]{Figures/IntegrateX_process.png}
%     \caption{The process of the agency application in the \texttt{IntegrateX} protocol.}
%     \label{IntegrateX}
% \end{figure}

% %在agency应用中,\texttt{IntegrateX}协议完整跨链调用只需要7个区块时间,而GPACT需要11个区块时间,由于处理根节点合约的调用并不需要一个完整的区块时间,因此\texttt{IntegrateX}的最短时间将会介于6-7个区块时间,而GPACT则是介于4d+2-4d+3个区块时间。考虑到这只是一个调用树深度为2跨链调用,在 INtegratedX 中,跨链等待时间并不会随着调用树深度的增加而增大而在GPACT中跨链调用的latency会随着调用树的深度增加而增大。因此随着调用树的深度增加,\texttt{IntegrateX}将会减少更多的时间。然而在实际场景中,由于存在网络波动,跨链信息传递的发送和接收并不能保证一定会在一个区块时间内完成,因此\texttt{IntegrateX}和GPACT所需要的区块时间将会大于最短区块时间。
% In the agency application, the \texttt{IntegrateX} protocol requires only 7 block times for a complete cross-chain call, whereas GPACT requires 11 block times. Since processing the root contract's call does not necessarily require a full block time, the minimum time for \texttt{IntegrateX} will range between 6 to 7 block times, while for GPACT, it will range between $4d+2$ to $4d+3$ block times. Considering that this example is a cross-chain call with a call tree depth of 2, in \texttt{IntegrateX}, the cross-chain waiting time does not increase with the depth of the call tree, whereas in GPACT, the latency of cross-chain calls increases with the depth of the call tree. Therefore, as the depth of the call tree increases, \texttt{IntegrateX} will save more time. However, in real-world scenarios, due to network fluctuations, the transmission and reception of cross-chain information cannot always be guaranteed to complete within a single block time. Consequently, the actual block time required for \texttt{IntegrateX} and GPACT will be greater than the minimum block time.

\section{Evaluation}
\label{sec:evaluation}
Our experiments aim to investigate whether agents within our framework can produce effective evolution of language strategies. Specifically, our experimental section addresses the following three research questions (RQs):
\begin{enumerate}
    \item RQ1 (Effectiveness): Can participants effectively evade regulatory detection over time, and how does the accuracy of information transmission change? Additionally, how do different LLMs affect the content and effectiveness?
    \item RQ2 (Human Interpretation): Do the evolved language strategies employed by agents effectively align with human understanding? Can they be interpreted and applied in real-world scenarios?
    \item RQ3 (Ablation Study): How does the newly introduced GA impact the evolution process in our framework?
\end{enumerate}

\subsection{Experimental Settings}
In our evaluation, we designed an abstract password game \cite{guess_number02} and a more realistic illicit pet trade scenario\cite{trade01,trade02,trade03}. 
%The password game features a relatively abstract, easily controlled setting, allowing for clear observation of how agents’ strategies evolve. Meanwhile, the illicit pet trade scenario simulates illegal activities on social networks \cite{DiMinin2018MachineLF}, with relevant corpora that more closely resemble real-world conditions, enabling a more direct comparison between evolved strategies and their real-life counterparts.
The overall experimental procedure follows the description in Section~\ref{sec: method}. In each round, the process comprises three stages: initialization, dialogue, and interview. In each round, the information that the participant agents need to convey will be randomly generated within a defined range. Only rounds in which a five-turn dialogue is successfully completed proceed to the interview phase, during which both parties’ successful transmission of information is verified. If any regulation violations are detected by the supervisory agent during the dialogue stage, that round is deemed a failure and is assigned a transmission score of zero. To ensure the robustness of our findings, we conducted 15 independent trials for each experimental condition. Each trial consisted of 50 fixed dialogue rounds. The crossover and mutation probabilities in the framework were set to 0.2 and 0.8.

\subsubsection{Scenario 1: Password Game}
Our first scenario is considered a relatively simple and abstract guessing game that involves a numerical context. In this setup, each participant is assigned a four-digit password, which they must convey to another participant within five rounds of dialogue without directly mentioning the numbers. Simultaneously, they need to extract information from the dialogue to infer the other's password.

The supervisory agent in the dialogue follows the policy of “prohibiting all content related to numbers.” The “password game” scenario is specifically designed to observe and analyze participant agents' language adaptability and strategic evolution in a theoretical and abstract context. Compared to complex scenarios based on real events, it provides a clearer and more easily quantifiable experimental environment.

\subsubsection{Scenario 2: Illicit Pet Trade} 
Simulating and detecting the covert drug transactions prevalent on social platforms is an important and realistic research topic. However, given the ethical considerations, we choose not to proceed with direct simulations, but to adopt a more neutral and less harmful setting—a illicit pet trade scenario—to serve as our experimental context.

In this scenario, we simulate an illegal pet trade through social platform. Participant agents play the roles of buyer and seller. The buyer conveys a request to purchase a randomly selected pet (such as a parrot, lizard, or cat) and specifies the currency they can use for payment (e.g., USD, CNY, JPY). The seller provides details about the transaction, including the time (morning, noon, or midnight) and location (school, park, or station). The supervisory agent enforces a policy that considers discussions related to pet transactions as regulation violations.
The focus of this experiment is to observe the adaptability of the participant agents' language and its evolution at a finer level of information granularity, as well as whether regulators can accurately detect regulation violations.

\subsection{RQ1: Effectiveness}
\subsubsection{Experiment Settings}

To evaluate RQ1, we conducted experiments using four different LLM models provided by various companies: \textit{gpt-4o}, \textit{gpt-4o-mini}, \textit{deepseekV2}, and \textit{qwen-turbo}. We introduced the following metric:
\begin{itemize}
    \item Accuracy of Information Transmission: This metric measures the agents' ability to convey the intended information to their dialogue partner without being detected. Specifically, it calculates the proportion of successfully transmitted information relative to the total intended information. 
    \item Average Dialogue turns: This metric evaluates the number of successful dialogue rounds the agents complete without being flagged by the supervisory agent. A higher dialogue turns indicates greater success in evading regulatory scrutiny.
    \item Average Entropy (Language Complexity):
        Entropy reflects the unpredictability of text and is calculated as:
        \[
        H(X) = -\sum_{i=1}^n P(x_i) \log P(x_i),
        \]
        where \(P(x_i)\) is the probability of each token. Higher entropy suggests a broader range of expressions (indicating greater innovation), but excessively high entropy can lead to incoherence.
    \item Distinct (Lexical Diversity):
        Distinct measures the number of unique n-grams in the text, computed as:
        \[
        \text{Distinct-n} = \frac{\text{Number of unique n-grams}}{\text{Total number of n-grams}}.
        \]
        Specifically, we evaluate Distinct-1 (unique unigrams). A higher distinct score indicates a larger, more varied vocabulary, signifying richer language use. Note that while both metrics assess diversity in language, entropy focuses on unpredictability, whereas distinct emphasizes lexical variety. 
\end{itemize}


\subsubsection{Experiment Results in Password Game}
\begin{figure*}[ht]
    \centering
    % 子图 (a)
    \begin{subfigure}[t]{0.48\textwidth}
        \centering
        \includegraphics[width=\linewidth]{figures/sec1_turn_acc_v5.png}
        \caption{Password Game}
        \label{fig:sce1}
    \end{subfigure}
    \hfill
    % 子图 (b)
    \begin{subfigure}[t]{0.48\textwidth}
        \centering
        \includegraphics[width=\linewidth]{figures/sec2_turn_acc_v5.png}
        \caption{Illicit Pet Trade}
        \label{fig:sce2}
    \end{subfigure}
    \caption{Average Continuous Dialogue Turns and Information Transmission Accuracy Across Dialogue Rounds}
    \label{fig:merged}
\end{figure*}

Figure \ref{fig:sce1} presents our experimental results in the password game. The x-axis corresponds to the increasing number of dialogue rounds, whereas the y-axis captures two primary metrics: (1) the average number of continuous dialogue turns before detection by the supervisory agent, and (2) the accuracy of information transmission, which is defined as the proportion of successfully transmitted information during the post-dialogue interview.

Overall, our findings indicate that as the number of rounds increases, agents gradually learn to evade regulation violation while conveying information with greater accuracy. Notably, most agents exhibit a pronounced local peak around the 20th round, followed by a brief decline and subsequent recovery. These fluctuations can be attributed to the dynamic nature of our simulation framework, which does not converge on a single dominant strategy but rather encourages ongoing exploration of novel language strategies.

We also observe that different LLMs influence the learning trend to varying degrees. Among the models tested, \textit{gpt-4o} demonstrates the strongest performance. Although other models generally share a similar upward trend, their relative performance gaps prove less stable. For instance, while \textit{deepseekV2} achieves the highest number of turns around the 20th round, its performance declines significantly by the 50th round in comparison to other models.

Turning to the accuracy results, we again observe a similar learning trajectory. This parallel arises primarily because if participant agents fail to complete a sufficient number of uninterrupted dialogue turns, the successfully transmitted information in that round is effectively zero. Consequently, especially in the early stages of the experiment, many rounds end with no successful transmissions. Overall, \textit{gpt-4o} still maintains a clear advantage over the other LLMs. However, we do observe subtle differences when comparing the dialogue round trends: for example, at the 20th dialogue round, \textit{deepseekV2} achieves a significantly higher average number of communication cycles than \textit{gpt-4o-mini}, yet their information transmission accuracy remains relatively similar.

Despite the overall positive learning trajectory, the average information accuracy remains low in the password game. We believe this outcome stems primarily from the intrinsic difficulty introduced by the scenario’s abstract nature. Without explicit prompts driving agents to develop symbolic or otherwise encrypted language stratgy, communication largely remains within the realm of everyday language. Consequently, the indirect expression of numeric information is challenging to implement and easily detectible by the supervisory agent.

\begin{table}[h!]
    \centering
    \caption{Performance of Different LLMs in Password Game}
    \label{tab:sce1}
    \renewcommand{\arraystretch}{1.2} % 调整行高
    \begin{tabular}{l S S S}
        \toprule
        \textbf{Model} & \textbf{Total Turns} & \textbf{Avg. Entropy} & \textbf{Avg. Distinct-1} \\
        \midrule
        \rowcolor{gray!10} \textbf{gpt-4o}       & 84.2   & 7.103 & 0.484 \\
        \textbf{gpt-4o-mini}  & 75.5   & 6.998 & 0.354 \\
        \rowcolor{gray!10} \textbf{deepseekV2} & 59.7
        & 5.365 & 0.247 \\
        \textbf{qwen-turbo}   & 50.8  & 6.101 & 0.518 \\
        \bottomrule
    \end{tabular}
\end{table}

Table \ref{tab:sce1} summarizes the performance of the four models in terms of cumulative dialogue turns, entropy, and Distinct-1. As shown, \textit{gpt-4o} achieves the highest values in both Entropy (7.103) and Distinct-1 (0.484), indicating that it employs a broader, more diverse range of vocabulary and more unpredictable expressions—thus having a greater likelihood of evading regulation violation detection when conveying numerical information. In contrast, \textit{deepseekV2} exhibits notably lower Entropy (5.365) and Distinct-1 (0.247), suggesting a more frequent reuse of fixed expressions. Notably, although \textit{qwen-turbo}’s Entropy (6.101) is only moderate, it attains a surprisingly high Distinct-1 (0.518), reflecting greater lexical richness. However, this does not translate into more effective regulatory evasion, as its total turns are only 50.8. Hence, merely having higher lexical diversity and linguistic entropy is insufficient to guarantee successful evasion. A model must also balance the concealment of overall semantics with the adaptation of its language strategy to achieve longer conversation sequences and a higher rate of successful information transmission. In other words, while richer language expression does confer certain advantages in countering regulation, it can still be detected when deeper strategies—such as tailored expression structures and topic evolution—are absent, ultimately resulting in fewer total turns.

\subsubsection{Experiment Results in Illicit Pet Trade Scenario}
Figure~\ref{fig:sce2} presents the experimental results of our framework in the illicit pet trade scenario, which overall resemble those of the password game but also exhibit some notable differences. First, both figures reveal a discernible learning trend, particularly during the initial 10 rounds. Meanwhile, \textit{gpt-4o} continues to demonstrate the strongest overall performance. We note that, because this scenario features a more concrete and complex semantic environment, there is an abundance of relevant linguistic material that can be leveraged for indirect expression. Consequently, under a similar number of turns, the overall accuracy here is noticeably higher compared to the password game.
Nevertheless, performance fluctuations persist. In particular, in the accuracy plot, \textit{deepseekV2} experiences a pronounced increase in accuracy after the 30th round, while \textit{gpt-4o}’s accuracy declines during the same period. As a result, \textit{deepseekV2} ultimately surpasses \textit{gpt-4o}’s accuracy in the final rounds of the experiment.

\begin{table}[h!]
    \centering
    \caption{Performance of Different LLMs in Illicit Pet Trade}
    \label{tab:sce2}
    \renewcommand{\arraystretch}{1.2} % 调整行高
    \begin{tabular}{l S S S}
        \toprule
        \textbf{Model} & \textbf{Total Turns} & \textbf{Avg. Entropy} & \textbf{Avg. Distinct-1} \\
        \midrule
        \rowcolor{gray!10} \textbf{gpt-4o}       & 136.2  & 6.856  & 0.471 \\
        \textbf{gpt-4o-mini}  & 74.4  & 6.595  & 0.387 \\
        \rowcolor{gray!10} \textbf{deepseekV2} & 65.2   & 6.255  & 0.338 \\
        \textbf{qwen-turbo}   & 50.5   & 5.891  & 0.461 \\
        \bottomrule
    \end{tabular}
\end{table}
Table \ref{tab:sce2} presents the performance of various LLMs in the illicit pet trade scenario, measured by total turns, average agent entropy, and Distinct-1. As in the password game, \textit{gpt-4o} maintains a notable lead in total turns (136.2) while also displaying relatively high entropy (6.856) and Distinct-1 (0.471). In contrast, \textit{gpt-4o-mini} reaches roughly half as many total turns (74.4), despite having a comparable entropy score (6.595). Meanwhile, \textit{deepseekV2} (65.2) and \textit{qwen-turbo} (50.5) trail further behind in total turns. Consistent with the results shown in Table 
\ref{tab:sce1}, \textit{qwen-turbo} again achieves a high Distinct-1 score, which we speculate may be linked to its training corpus: it includes extensive data from the Chinese internet, likely giving it an advantage in a Chinese-language environment over more internationally oriented models.

Notably, the range of entropy scores in this scenario—spanning from 5.891 (\textit{qwen-turbo}) to 6.856 (gpt-4o)—is narrower than in the password game (see Table \ref{tab:sce1}), reflecting the more concrete nature of the illicit pet trade setting. This scenario provides richer contextual cues for indirect references, enabling all models to maintain higher semantic complexity. However, as was the case in the password game, having a broader vocabulary or greater unpredictability alone does not guarantee extended evasion: models must integrate their linguistic variety into strategic planning to circumvent regulatory scrutiny, a balance that \textit{gpt-4o} continues to manage most effectively.

\setlength{\fboxrule}{0.5pt} 
\vspace{0.5em}
\noindent
\begin{tcolorbox}[colframe=black!20, colback=gray!10, arc=5pt, boxrule=0.5pt, width=0.99\linewidth]
\textit{Answer to RQ1}: Experimental results indicate that participant agents in our framework progressively improve their ability to evade regulation violation detection through continuous interaction, leading to longer uninterrupted dialogue sequences. Concurrently, the accuracy of information transmission gradually increases over successive rounds, demonstrating that the evolved strategies effectively balance evasion with precise communication.
Moreover, different models also exhibit varying results. For example, \textit{gpt-4o} performs most outstandingly in extending dialogue turns and maintaining language complexity (i.e., high entropy and lexical diversity), while other models such as \textit{gpt-4o-mini}, \textit{deepseekV2}, and \textit{qwen-turbo} demonstrate different fluctuations and localized advantages at different stages.
\end{tcolorbox}

\subsection{RQ2: Human Interpretation}
\subsubsection{Experiment Settings}

To investigate the real-world relevance of both the evolved language strategies and the resulting dialogue, we conducted a human evaluation on a subset of successful dialogue records from the password game and illicit pet trade scenario. The dialogues generated by the \textit{gpt-4o} models are randomly selected, and 40 human participants participated in the experiment to evaluate them. The 40 human reviewers had an average age of approximately 27 (SD = 4). In terms of gender, 75\% of the human reviewers were male, and 25\% were female. Regarding educational background, 67.5\% held a bachelor's degree, 27.5\% held a master's degree or above, and 5\% had an associate degree or lower. All dialogue records were presented in Simplified Chinese.

Each participant rated each dialogue on a 5-point Likert scale on the following five metrics:
\begin{itemize}
    \item Explicit Understanding: Evaluates how effectively the dialogue’s explicit meaning is communicated (1: Extremely vague and confusing; 3: Moderately clear, but some parts may require further interpretation; 5: Crystal clear and precise).
    \item Implicit Understanding: Assesses the reader's ability to grasp the underlying or unstated messages (1: Nearly indecipherable subtext; 3: Some underlying meaning is apparent, but requires effort to fully grasp; 5: Subtext that is immediately apparent).
    \item Realistic Significance: Measures the extent to which the dialogue reflects real-life situations and holds practical relevance (1: Highly unrealistic with little relevance; 3: Generally realistic, though some elements may not align with real-world situations; 5: Deeply rooted in real-world context).
    \item Regulatory Avoidance: Examines the effectiveness of the strategies in evading regulation violation (1: Blatantly ineffective and easily spotted; 3: Partially effective, with the potential for detection in some cases; 5: Exceptionally subtle and effective).
    \item Strategy Existence: Determines how plausible it is for such strategies to be observed in practical, real-world scenarios (1: Extremely implausible; 3: Fairly believable, though may seem impractical in specific situations; 5: Entirely plausible).
\end{itemize}



\subsubsection{Experiment Results}
\begin{figure}
    \centering
    \includegraphics[width=0.9\linewidth]{figures/user_study_v4.png}
    \caption{Box plots of user study scores across different metrics in two scenarios. The red x symbol denotes the mean value.}
    \label{fig:case_study}
\end{figure}
As shown in Fig.\ref{fig:case_study}, our framework consistently achieves average scores of 3.4 or above across most indicators (such as explicit understanding and implicit understanding). This suggests that, both in terms of the generated dialogues and the underlying strategies, it possesses valuable practical applicability.

%Although there are a few exceptions, compared with the old framework (\textit{w/o GA, gpt-4o}), the new version (\textit{w/ GA, gpt-4o}) demonstrates overall advantages in both average scores and score distributions. In the comparison between different versions, under the more realistic illicit pet trade scenario, the new framework shows distinct benefits over the old one in both “regulatory avoidance” and “strategy existence”—both in distribution and mean values. This finding indicates that introducing a genetic algorithm, particularly a fitness‐based strategy selection mechanism, makes strategy adoption more efficient and stable. As for the password game, we speculate that the main reason these two metrics do not show a large distributional gap is that, in an abstract scenario, the range of available strategies is broader.

Comparing distributions between the password game and the illicit pet trade scenario reveals some interesting phenomena. Focusing on “realistic significance” and “regulatory avoidance,” the more abstract password game often yields higher mean values than the more concrete illicit pet trade scenario, while also exhibiting lower dispersion. We speculate this is related to the inherently abstract nature of numeric information: encryption and covert hints can be harder to detect in such contexts, and the growing tendency on Chinese internet platforms to use abstract language \cite{Wu2025HighEnergy} may lead reviewers to have a higher acceptance of “obscure” expressions. Conversely, the illicit pet trade scenario, despite being closely tied to real-world transactions, may suffer if the indirect or euphemistic methods in the dialogues are insufficiently subtle. Human reviewers can find them conspicuous or “forced,” potentially causing lower scores for “realistic significance” and “regulatory avoidance” in terms of both distribution and mean values.
A significant portion of these results can be attributed to inherent biases in commercial LLMs, such as ChatGPT, introduced during their training phases. These general-purpose models undergo fine-tuning via RLHF to align with specific product positioning, which often results in a more standard and safe output style. However, this characteristic poses a limitation for our simulation framework, as it may hinder the model’s ability to capture the nuanced and unconventional expressions typical of online social interactions. Ideally, fine-tuning datasets that are more representative of social platforms could lead to improved performance in our simulations.


In the abstract password game, for instance, a typical conversation might go like this:
\begin{quote}
\textit{
“I've really grown fond of a certain phase of the moon. It's not the brightest or the darkest, but it always carries its own charm. It symbolizes ...... In that green oasis, I felt as if I were catching a glimpse of the golden hues of autumn leaves, much like the soft, warm glow of dusk—calm and serene ...” 
}
\end{quote}
Here, words like “lunar cycle” and “autumn leaves” can subtly hint at larger or smaller digits, or use seasonal imagery to convey key information. Since these references lack an obvious connection, they lend a more literary feel to the dialogue and, to some extent, raise the bar for recognition and detection.

By contrast, in a more concrete setting like illicit pet trade, example conversations may be closer to real‐life buying and selling procedures, which can make them appear more “suspicious”:
\begin{quote}
\textit{
“... about a vibrant 'tropical chatterbird' renowned for its brilliant plumage and uncanny mimicry ... I've also come into possession of a few 'Rising Sun coins' for exchange ...... Perhaps you might know a place where ...”
}
\end{quote}
In this dialogue, the term “tropical chatterbird” serves as an euphemism for a parrot, emphasizing its colorful appearance and mimicking ability without mentioning the animal directly. Meanwhile, “Rising Sun tokens” subtly alludes to the Japanese yen, since the Rising Sun is an iconic symbol of Japan. This coded language allows both parties to communicate their intentions regarding the acquisition of a rare bird and the intended payment method without explicitly revealing sensitive details. However, if these indirect expressions are used excessively, the dialogue may appear artificial or unnatural, potentially reducing its authenticity—thus affecting evaluations of both “regulatory avoidance” and “strategy existence.”
\setlength{\fboxrule}{0.5pt} 
\vspace{0.5em}
\noindent
\begin{tcolorbox}[colframe=black!20, colback=gray!10, arc=5pt, boxrule=0.5pt, width=0.99\linewidth]
\textit{Answer to RQ2}: Our evaluation confirms that the emergent language strategies closely resemble real-world language strategies, effectively employing euphemisms and implicit cues, and are generally understood by human reviewers. However, while these strategies show potential in simulations, they often appear forced or unnatural due to the fine-tuning of LLMs as commercial products, requiring refinement to better mimic the nuanced and fluid communication typical in real-world social interactions.

\end{tcolorbox}

\subsection{RQ3: Ablation Experiment}
\subsubsection{Experiment Settings}

To evaluate the effectiveness of the GA introduced in our framework, we conducted an ablation experiment using \textit{gpt-4o-mini} and \textit{gpt-4o} as the underlying LLM. For comparison, we employed the approach from our initial study \cite{DBLP:conf/cec/CaiLZLWT24}, which primarily differs in its strategy-update mechanism. In that earlier framework, the LLM is provided with both the existing strategy and newly flagged regulation violation records during the reflection stage, prompting the model to propose a new set of strategies that replace the old ones.
In contrast, our new framework employs a GA process where each strategy is treated as a discrete unit and optimized iteratively through GA. 

\subsubsection{Experiment Results}
As shown in Fig.~\ref{fig:ablation}, the GA-based framework demonstrates significant advantages. In the short-term experiment within the first 35 rounds, the w/o GA approach might show slight initial superiority due to the larger changes brought about by replacing the entire strategy. However, overall, w/ GA performs better than w/o GA. This difference increases as the number of rounds grows, particularly after round 35, where the advantages of w/ GA become even more pronounced. The GA process enables effective strategy evolution and adaptation, leading to an increased number of dialogue turns and improved accuracy, highlighting the framework's enhanced adaptability in the long term.
%Despite occasional performance dips during the evolutionary process, the GA framework’s ability to foster strategy diversity and handle complex scenarios makes it a more effective approach for sustained optimization.
\begin{figure}[h!]
    \centering
    \includegraphics[width=\linewidth]{figures/ablation1_v6.png}
    \caption{Performance with/without GA}
    \label{fig:ablation}
\end{figure}
\setlength{\fboxrule}{0.5pt} 
\vspace{0.5em}
\noindent
\begin{tcolorbox}[colframe=black!20, colback=gray!10, arc=5pt, boxrule=0.5pt, width=0.99\linewidth]
\textit{Answer to RQ3}: The results confirm the effectiveness of the GA component in our framework, especially when the number of rounds increases, where it demonstrates greater stability and adaptability. Although the optimization may be slower in the early stages, GA provides stronger adaptability in the long term through effective strategy evolution.
\end{tcolorbox}

\subsection{Discussion and Limitation}
In this study, we leveraged LLM agents to simulate the evolution of language strategies under regulatory pressure. While our results provide initial evidence that agents can adapt and develop covert communication tactics, the simulations also exhibit noteworthy instabilities. First, the inherent randomness of LLM generation can cause significant fluctuations in outcomes: the same prompts may yield different strategic responses, particularly when the experimental scale (number of agents or dialogue rounds) is limited. In our framework, LLMs not only generate dialogues but also determine strategies and regulatory responses; as a result, any stochasticity is compounded across multiple modules, making the final results sensitive to small variations in prompt inputs or random seeds. Although such variability partially reflects the diversity of real-world human behavior to some extent, it complicates the interpretation of findings in a controlled experimental setup.

A second limitation lies in the relatively narrow scope of language strategies observed. The agents predominantly relied on general-purpose evasive methods, such as analogies or implicit references, yet rarely produced fully “encrypted” or specialized code words that might arise in realistic cultural or social contexts. This outcome highlights the challenge that LLMs, pre-trained on broad domains and further refined via RLHF, are predisposed to generate text consistent with mainstream norms, thereby inhibiting the formation of highly unconventional or obscure expressions. Moreover, in scenarios where the training corpus lacks sufficient examples of subcultural or community-specific covert language, the model is less able to invent or adopt specialized linguistic forms. 

Finally, our experiments focused on one-to-one private interactions that emphasize regulatory evasion, without exploring the dynamics of public, many-to-many conversations where language strategies might evolve and propagate differently in a broader social context. While each participant agent does learn and adapt incrementally across dialogue rounds, real-world language evolution involves extensive, long-term propagation across diverse communities. Covert terms or code words may gradually gain acceptance, be modified by different user groups, or fade from use entirely. By contrast, the small-scale nature of our simulated dialogues means that emergent language strategies do not undergo the sustained diffusion and feedback processes characteristic of real social platforms, limiting the ecological validity of our findings.




%对于语言演化的社会类模拟仍然是一个未被开拓的领域,通过借助LLM优秀的自然语言处理能力,为这类自然语言的模拟带来了强大助力。然而伴随着实验也让我们发现LLM也会导致许多局限性。尽管通过实验初步证明了我们的框架的有效性。但同时伴随着实验也为我们带来了许多值得讨论的点。

%实验结果的不稳定性
%首先实验结果本身具有一定的不稳定性,而我们认为整个不稳定性的根源源自于LLM本身生成具有不确定性\cite{},在我们的框架中,LLM几乎参与到了所有环节。同样的violation log让同一个LLM在相同的设置内可能会总结出不同的constraint strategy。尽管这种不稳定性在现实中同样存在(例如不同的人采取不同的策略),同时也是作为模拟框架中非常重要的点,然而在本工作中的数量级的实验中这种不稳定性对结果的影响更为难以过滤。就像\ref{}中也证实的,这种LLM dirven agent的研究中在小数量级上的实验存在着不稳定性,我们认为目前的结果已经足够证实我们的框架可以初步模拟语言动态的学习和演化这一趋势,在今后工作中更大量级的实验中(例如数万数百万agent于更多的round数),我们有理由相信,整体趋势会更加稳定,不同llm的agent之间的性能差距会更加接近llm本身语义理解与生成的综合性能,

%模拟策略的局限性,
%从实验中我们观察到,agent模拟出的策略目前仍然主要集中于比喻类比等较为共通的方式。现实中语言的演化一般根植于当地的文化与经济背景等等因素。例如中文可以利用拼音来将汉字转化为对应的字母从而规避审查,而英文可能会更加积极的利用emoji来作为表达的替代从而规避监管。
%这些较为复杂的策略不仅需要对应环境的大量先验知识,在较为常见的语言中,LLM中训练所需的语料知识可能包含了这些,但是对于训练的数据集中欠缺的语种的知识LLM在不借助prompt的提示的情况下没有能力选择这些既存的策略。


%尽管LLM训练中的数据集可能存在这种更为隐晦的表达方式,首先LLM的RLHF\ref{}本身的训练方法导致了目前绝大多数的LLM为了保证生成文本的泛用性,被训练的更加愿意生成更符合大众的一般化输出文本,在不对LLM进行微调的前提下很难提高在这种特性领域的表现。
%LLM的表现严重依赖prompt的结构设计,提示词工程已经被证明可以有效提高LLM的某一方面能力,单次的基于prompt的模型交互很难实现多步推理或是规划。尽管我们的框架已经将语言演化这一现象解耦,通过多个模块来尽可能模拟人类在该环境中内在的动力学,但是目前的策略生成阶段
%这一部分在不适用复杂prompt工程的前提下LLM很难采用这种小众?特殊领域?的表达。
%对于模拟出的语言策略,我们发现很少的独特加密语言,因为这种需要两边有一套共用的体系,对于我们的模拟情景只有固定turn数的模拟很难形成意思传达。


%\jialong{第二是演化后的语言是如何的存活。我们只考虑了能不能躲避监管。但语言后续的存活和发展其实是更大范围的society的一个动态过程(而不是几个agent之间的交互),这一块可以结合那些上千LLM agent的研究框架来进行拓展}
%\jialong{这边可以多用语言学的角度来说不足之处}
%\jialong{第一个缺点是语言演化一般根植于根植于文化,经济背景,当地的文化背景。但我们的文章没有考虑特定文化背景下的演化。例如中文中可以借用拼音与汉字之间的关系来作为回避监管的方式,日语则可以通过XXX,英语则可以通过XXXX。未来可能要借助persona和role-play之类的设定来进一步拓展}

%更大规模的实验
%策略生成那里增加多步规划
%RAG提供更多语料
%


\section{Discussions and Limitations}\label{disscussion}

\noindent
\textbf{Potential Application Scenarios.} 
Beyond the example of the train-and-hotel problem mentioned in this paper, \texttt{IntegrateX} can be extended to a wide range of application scenarios. 
One promising use case is cross-chain flash loans \cite{tefagh2021ccfl}. 
Flash loans are atomic, uncollateralized lending protocols that allow users to borrow funds at nearly zero cost, perform other operations, and then repay the loan. 
However, these processes require the guarantee of overall atomicity, meaning that either all steps succeed or they all fail. 
Due to this requirement for atomicity, existing flash loan protocols are limited to intra-chain operations. 
With \texttt{IntegrateX}, which provides overall atomicity for cross-chain dApps, users can efficiently perform cross-chain flash loan operations.
Other promising application scenarios include, but are not limited to, cross-chain atomic arbitrage and cross-chain supply chain management.

\vspace{3pt}
\noindent
\textbf{Learning Cost for Developers.} 
The logic-state decoupling and fine-grained state lock mechanisms may slightly increase the development learning curve for smart contract developers. Fortunately, we have proposed a set of guidelines to assist developers, and the mechanisms are flexible (as discussed in Section \ref{subsec:LSD} and \ref{subsec:lock}), allowing developers to freely decide whether to implement them. Furthermore, we can provide formal documentation, SDKs, and other resources (following existing standards such as Wormhole \cite{wormhole} and IBC \cite{cosmos2019}) to guide developers in secure and efficient development and auditing, thereby reducing the learning curve. Additionally, incentive mechanisms (e.g., token rewards) are widely adopted in the industry to encourage developers to build and utilize our system. In future research, we could even explore AI-based semi-automated smart contract tools to further address this challenge.

Moreover, logic-state decoupling offers several additional benefits. For instance, it facilitates modular programming principles in smart contract development, which helps reduce subsequent upgrade and maintenance costs while improving contract security. When an issue arises in a specific contract module, developers can conduct targeted audits and resolve vulnerabilities efficiently. 


\vspace{3pt}
\noindent
\textbf{Support for Heterogeneous Chains.} 
\texttt{IntegrateX} currently supports blockchains that run different consensus protocols but share the same smart contract execution environment. 
As mentioned in the paper, to ensure cross-chain transaction security across blockchains with different consensus protocols, \texttt{IntegrateX} waits until consensus on the source chain is finalized (or highly likely to be finalized) before committing the cross-chain transaction to the target chain. 
Additionally, while this paper focuses on \texttt{IntegrateX}’s implementation on EVM-compatible blockchains, it can also be modified to operate between non-EVM-compatible blockchains that share the same smart contract execution environment.

However, a current limitation of \texttt{IntegrateX} is that it cannot operate between blockchains with different smart contract execution environments. 
Fortunately, this issue could potentially be addressed using advanced techniques such as code virtualization \cite{virtualization}. 
Expanding \texttt{IntegrateX} to support integrated execution across blockchains with different smart contract environments is a future research direction we aim to explore.

\vspace{3pt}
\noindent
\textbf{Trade-off Between Flexibility and Load Balancing.} 
In \texttt{IntegrateX}, cross-chain dApp providers have the flexibility to select any chain as the execution chain for integrated execution, based on their preferences. 
However, this flexibility may introduce a potential issue: if many cross-chain dApp providers choose the same chain as the execution chain, that chain could become a hotspot, potentially degrading its performance.
One possible solution is for a third party (e.g., \texttt{IntegrateX}) to manage load balancing by selecting the execution chain on behalf of the cross-chain dApp providers. 
However, this approach could introduce centralization risks. 
Additionally, as discussed in the paper, developers' choice of which chain to run their dApps on often involves considerations beyond performance, such as ecosystem compatibility and business partnerships.
How to better achieve load balancing and how to strike a trade-off between performance and flexibility are important questions that warrant future research.

\vspace{3pt}
\noindent
\textbf{Mitigating Malicious Application Layer Components.}
In public blockchain scenarios, there are common strategies to mitigate malicious behavior from application layer components (e.g., dApp providers, users). 
For instance, a malicious cross-chain dApp provider might attempt to maliciously lock certain states to prevent their usage by others. 
Such behavior can be countered using contract-based authorization or blacklisting mechanisms (widely used in existing dApp development \cite{etherscan}). 
For example, an intra-chain dApp provider can pre-arrange with a cross-chain dApp provider and authorize trusted cross-chain dApp providers (through their associated addresses) in their contracts, allowing only authorized cross-chain dApp providers to invoke and lock their states. 
Similarly, intra-chain dApp providers can blacklist specific cross-chain dApp providers in their contracts to block their interactions.
Additionally, gas fee mechanisms can serve as a deterrent to malicious application layer components attempting to launch flooding attacks against the blockchain.

\vspace{3pt}
\noindent
\textbf{Cross-Chain vs. Cross-Shard.}
Some existing works have explored the issue of cross-shard smart contract handling \cite{qi2024lightcross, li2022jenga}. 
However, cross-chain and cross-shard scenarios are fundamentally different, and their solutions cannot be directly applied to cross-chain contexts. 
The primary reasons are as follows: First, research on blockchain sharding typically involves modifications to the underlying system. 
In contrast, a key requirement of cross-chain protocols or systems is that they must not require modifications to the underlying blockchains, ensuring better compatibility with existing blockchain systems. 
Second, a blockchain sharding system usually has a beacon chain responsible for coordinating progress across shards, which is impractical in cross-chain protocols. 
After all, in cross-chain scenarios, each blockchain essentially belongs to a different system. 
Finally, in cross-chain contexts, blockchains are often heterogeneous (e.g., different consensus protocols), which is uncommon in blockchain sharding systems.

\vspace{3pt}
\noindent
\textbf{Inter-Chain Shared Security.}
In \texttt{IntegrateX}, each blockchain is assumed to have a proportion of malicious nodes lower than its fault tolerance threshold (i.e., they are secure). 
This is a widely accepted assumption in most existing works. 
However, recent research has begun exploring how to achieve secure cross-chain interoperability protocols in scenarios where individual blockchains may not be secure, by sharing security across multiple blockchains \cite{sheng2023trustboost}.
\texttt{IntegrateX} could adopt similar ideas through modifications to achieve shared security among blockchains. 
However, how to design and implement inter-chain shared security within \texttt{IntegrateX} while still maintaining efficiency is an important direction for future research.

\paragraph{Summary}
Our findings provide significant insights into the influence of correctness, explanations, and refinement on evaluation accuracy and user trust in AI-based planners. 
In particular, the findings are three-fold: 
(1) The \textbf{correctness} of the generated plans is the most significant factor that impacts the evaluation accuracy and user trust in the planners. As the PDDL solver is more capable of generating correct plans, it achieves the highest evaluation accuracy and trust. 
(2) The \textbf{explanation} component of the LLM planner improves evaluation accuracy, as LLM+Expl achieves higher accuracy than LLM alone. Despite this improvement, LLM+Expl minimally impacts user trust. However, alternative explanation methods may influence user trust differently from the manually generated explanations used in our approach.
% On the other hand, explanations may help refine the trust of the planner to a more appropriate level by indicating planner shortcomings.
(3) The \textbf{refinement} procedure in the LLM planner does not lead to a significant improvement in evaluation accuracy; however, it exhibits a positive influence on user trust that may indicate an overtrust in some situations.
% This finding is aligned with prior works showing that iterative refinements based on user feedback would increase user trust~\cite{kunkel2019let, sebo2019don}.
Finally, the propensity-to-trust analysis identifies correctness as the primary determinant of user trust, whereas explanations provided limited improvement in scenarios where the planner's accuracy is diminished.

% In conclusion, our results indicate that the planner's correctness is the dominant factor for both evaluation accuracy and user trust. Therefore, selecting high-quality training data and optimizing the training procedure of AI-based planners to improve planning correctness is the top priority. Once the AI planner achieves a similar correctness level to traditional graph-search planners, strengthening its capability to explain and refine plans will further improve user trust compared to traditional planners.

\paragraph{Future Research} Future steps in this research include expanding user studies with larger sample sizes to improve generalizability and including additional planning problems per session for a more comprehensive evaluation. Next, we will explore alternative methods for generating plan explanations beyond manual creation to identify approaches that more effectively enhance user trust. 
Additionally, we will examine user trust by employing multiple LLM-based planners with varying levels of planning accuracy to better understand the interplay between planning correctness and user trust. 
Furthermore, we aim to enable real-time user-planner interaction, allowing users to provide feedback and refine plans collaboratively, thereby fostering a more dynamic and user-centric planning process.


% \clearpage
\bibliographystyle{IEEEtran}
\bibliography{sample}
% \clearpage

% % E5数据集介绍,数据集处理过程
% 基线模型介绍

\definecolor{titlecolor}{rgb}{0.9, 0.5, 0.1}
\definecolor{anscolor}{rgb}{0.2, 0.5, 0.8}
\definecolor{labelcolor}{HTML}{48a07e}
\begin{table*}[h]
	\centering
	
 % \vspace{-0.2cm}
	
	\begin{center}
		\begin{tikzpicture}[
				chatbox_inner/.style={rectangle, rounded corners, opacity=0, text opacity=1, font=\sffamily\scriptsize, text width=5in, text height=9pt, inner xsep=6pt, inner ysep=6pt},
				chatbox_prompt_inner/.style={chatbox_inner, align=flush left, xshift=0pt, text height=11pt},
				chatbox_user_inner/.style={chatbox_inner, align=flush left, xshift=0pt},
				chatbox_gpt_inner/.style={chatbox_inner, align=flush left, xshift=0pt},
				chatbox/.style={chatbox_inner, draw=black!25, fill=gray!7, opacity=1, text opacity=0},
				chatbox_prompt/.style={chatbox, align=flush left, fill=gray!1.5, draw=black!30, text height=10pt},
				chatbox_user/.style={chatbox, align=flush left},
				chatbox_gpt/.style={chatbox, align=flush left},
				chatbox2/.style={chatbox_gpt, fill=green!25},
				chatbox3/.style={chatbox_gpt, fill=red!20, draw=black!20},
				chatbox4/.style={chatbox_gpt, fill=yellow!30},
				labelbox/.style={rectangle, rounded corners, draw=black!50, font=\sffamily\scriptsize\bfseries, fill=gray!5, inner sep=3pt},
			]
											
			\node[chatbox_user] (q1) {
				\textbf{System prompt}
				\newline
				\newline
				You are a helpful and precise assistant for segmenting and labeling sentences. We would like to request your help on curating a dataset for entity-level hallucination detection.
				\newline \newline
                We will give you a machine generated biography and a list of checked facts about the biography. Each fact consists of a sentence and a label (True/False). Please do the following process. First, breaking down the biography into words. Second, by referring to the provided list of facts, merging some broken down words in the previous step to form meaningful entities. For example, ``strategic thinking'' should be one entity instead of two. Third, according to the labels in the list of facts, labeling each entity as True or False. Specifically, for facts that share a similar sentence structure (\eg, \textit{``He was born on Mach 9, 1941.''} (\texttt{True}) and \textit{``He was born in Ramos Mejia.''} (\texttt{False})), please first assign labels to entities that differ across atomic facts. For example, first labeling ``Mach 9, 1941'' (\texttt{True}) and ``Ramos Mejia'' (\texttt{False}) in the above case. For those entities that are the same across atomic facts (\eg, ``was born'') or are neutral (\eg, ``he,'' ``in,'' and ``on''), please label them as \texttt{True}. For the cases that there is no atomic fact that shares the same sentence structure, please identify the most informative entities in the sentence and label them with the same label as the atomic fact while treating the rest of the entities as \texttt{True}. In the end, output the entities and labels in the following format:
                \begin{itemize}[nosep]
                    \item Entity 1 (Label 1)
                    \item Entity 2 (Label 2)
                    \item ...
                    \item Entity N (Label N)
                \end{itemize}
                % \newline \newline
                Here are two examples:
                \newline\newline
                \textbf{[Example 1]}
                \newline
                [The start of the biography]
                \newline
                \textcolor{titlecolor}{Marianne McAndrew is an American actress and singer, born on November 21, 1942, in Cleveland, Ohio. She began her acting career in the late 1960s, appearing in various television shows and films.}
                \newline
                [The end of the biography]
                \newline \newline
                [The start of the list of checked facts]
                \newline
                \textcolor{anscolor}{[Marianne McAndrew is an American. (False); Marianne McAndrew is an actress. (True); Marianne McAndrew is a singer. (False); Marianne McAndrew was born on November 21, 1942. (False); Marianne McAndrew was born in Cleveland, Ohio. (False); She began her acting career in the late 1960s. (True); She has appeared in various television shows. (True); She has appeared in various films. (True)]}
                \newline
                [The end of the list of checked facts]
                \newline \newline
                [The start of the ideal output]
                \newline
                \textcolor{labelcolor}{[Marianne McAndrew (True); is (True); an (True); American (False); actress (True); and (True); singer (False); , (True); born (True); on (True); November 21, 1942 (False); , (True); in (True); Cleveland, Ohio (False); . (True); She (True); began (True); her (True); acting career (True); in (True); the late 1960s (True); , (True); appearing (True); in (True); various (True); television shows (True); and (True); films (True); . (True)]}
                \newline
                [The end of the ideal output]
				\newline \newline
                \textbf{[Example 2]}
                \newline
                [The start of the biography]
                \newline
                \textcolor{titlecolor}{Doug Sheehan is an American actor who was born on April 27, 1949, in Santa Monica, California. He is best known for his roles in soap operas, including his portrayal of Joe Kelly on ``General Hospital'' and Ben Gibson on ``Knots Landing.''}
                \newline
                [The end of the biography]
                \newline \newline
                [The start of the list of checked facts]
                \newline
                \textcolor{anscolor}{[Doug Sheehan is an American. (True); Doug Sheehan is an actor. (True); Doug Sheehan was born on April 27, 1949. (True); Doug Sheehan was born in Santa Monica, California. (False); He is best known for his roles in soap operas. (True); He portrayed Joe Kelly. (True); Joe Kelly was in General Hospital. (True); General Hospital is a soap opera. (True); He portrayed Ben Gibson. (True); Ben Gibson was in Knots Landing. (True); Knots Landing is a soap opera. (True)]}
                \newline
                [The end of the list of checked facts]
                \newline \newline
                [The start of the ideal output]
                \newline
                \textcolor{labelcolor}{[Doug Sheehan (True); is (True); an (True); American (True); actor (True); who (True); was born (True); on (True); April 27, 1949 (True); in (True); Santa Monica, California (False); . (True); He (True); is (True); best known (True); for (True); his roles in soap operas (True); , (True); including (True); in (True); his portrayal (True); of (True); Joe Kelly (True); on (True); ``General Hospital'' (True); and (True); Ben Gibson (True); on (True); ``Knots Landing.'' (True)]}
                \newline
                [The end of the ideal output]
				\newline \newline
				\textbf{User prompt}
				\newline
				\newline
				[The start of the biography]
				\newline
				\textcolor{magenta}{\texttt{\{BIOGRAPHY\}}}
				\newline
				[The ebd of the biography]
				\newline \newline
				[The start of the list of checked facts]
				\newline
				\textcolor{magenta}{\texttt{\{LIST OF CHECKED FACTS\}}}
				\newline
				[The end of the list of checked facts]
			};
			\node[chatbox_user_inner] (q1_text) at (q1) {
				\textbf{System prompt}
				\newline
				\newline
				You are a helpful and precise assistant for segmenting and labeling sentences. We would like to request your help on curating a dataset for entity-level hallucination detection.
				\newline \newline
                We will give you a machine generated biography and a list of checked facts about the biography. Each fact consists of a sentence and a label (True/False). Please do the following process. First, breaking down the biography into words. Second, by referring to the provided list of facts, merging some broken down words in the previous step to form meaningful entities. For example, ``strategic thinking'' should be one entity instead of two. Third, according to the labels in the list of facts, labeling each entity as True or False. Specifically, for facts that share a similar sentence structure (\eg, \textit{``He was born on Mach 9, 1941.''} (\texttt{True}) and \textit{``He was born in Ramos Mejia.''} (\texttt{False})), please first assign labels to entities that differ across atomic facts. For example, first labeling ``Mach 9, 1941'' (\texttt{True}) and ``Ramos Mejia'' (\texttt{False}) in the above case. For those entities that are the same across atomic facts (\eg, ``was born'') or are neutral (\eg, ``he,'' ``in,'' and ``on''), please label them as \texttt{True}. For the cases that there is no atomic fact that shares the same sentence structure, please identify the most informative entities in the sentence and label them with the same label as the atomic fact while treating the rest of the entities as \texttt{True}. In the end, output the entities and labels in the following format:
                \begin{itemize}[nosep]
                    \item Entity 1 (Label 1)
                    \item Entity 2 (Label 2)
                    \item ...
                    \item Entity N (Label N)
                \end{itemize}
                % \newline \newline
                Here are two examples:
                \newline\newline
                \textbf{[Example 1]}
                \newline
                [The start of the biography]
                \newline
                \textcolor{titlecolor}{Marianne McAndrew is an American actress and singer, born on November 21, 1942, in Cleveland, Ohio. She began her acting career in the late 1960s, appearing in various television shows and films.}
                \newline
                [The end of the biography]
                \newline \newline
                [The start of the list of checked facts]
                \newline
                \textcolor{anscolor}{[Marianne McAndrew is an American. (False); Marianne McAndrew is an actress. (True); Marianne McAndrew is a singer. (False); Marianne McAndrew was born on November 21, 1942. (False); Marianne McAndrew was born in Cleveland, Ohio. (False); She began her acting career in the late 1960s. (True); She has appeared in various television shows. (True); She has appeared in various films. (True)]}
                \newline
                [The end of the list of checked facts]
                \newline \newline
                [The start of the ideal output]
                \newline
                \textcolor{labelcolor}{[Marianne McAndrew (True); is (True); an (True); American (False); actress (True); and (True); singer (False); , (True); born (True); on (True); November 21, 1942 (False); , (True); in (True); Cleveland, Ohio (False); . (True); She (True); began (True); her (True); acting career (True); in (True); the late 1960s (True); , (True); appearing (True); in (True); various (True); television shows (True); and (True); films (True); . (True)]}
                \newline
                [The end of the ideal output]
				\newline \newline
                \textbf{[Example 2]}
                \newline
                [The start of the biography]
                \newline
                \textcolor{titlecolor}{Doug Sheehan is an American actor who was born on April 27, 1949, in Santa Monica, California. He is best known for his roles in soap operas, including his portrayal of Joe Kelly on ``General Hospital'' and Ben Gibson on ``Knots Landing.''}
                \newline
                [The end of the biography]
                \newline \newline
                [The start of the list of checked facts]
                \newline
                \textcolor{anscolor}{[Doug Sheehan is an American. (True); Doug Sheehan is an actor. (True); Doug Sheehan was born on April 27, 1949. (True); Doug Sheehan was born in Santa Monica, California. (False); He is best known for his roles in soap operas. (True); He portrayed Joe Kelly. (True); Joe Kelly was in General Hospital. (True); General Hospital is a soap opera. (True); He portrayed Ben Gibson. (True); Ben Gibson was in Knots Landing. (True); Knots Landing is a soap opera. (True)]}
                \newline
                [The end of the list of checked facts]
                \newline \newline
                [The start of the ideal output]
                \newline
                \textcolor{labelcolor}{[Doug Sheehan (True); is (True); an (True); American (True); actor (True); who (True); was born (True); on (True); April 27, 1949 (True); in (True); Santa Monica, California (False); . (True); He (True); is (True); best known (True); for (True); his roles in soap operas (True); , (True); including (True); in (True); his portrayal (True); of (True); Joe Kelly (True); on (True); ``General Hospital'' (True); and (True); Ben Gibson (True); on (True); ``Knots Landing.'' (True)]}
                \newline
                [The end of the ideal output]
				\newline \newline
				\textbf{User prompt}
				\newline
				\newline
				[The start of the biography]
				\newline
				\textcolor{magenta}{\texttt{\{BIOGRAPHY\}}}
				\newline
				[The ebd of the biography]
				\newline \newline
				[The start of the list of checked facts]
				\newline
				\textcolor{magenta}{\texttt{\{LIST OF CHECKED FACTS\}}}
				\newline
				[The end of the list of checked facts]
			};
		\end{tikzpicture}
        \caption{GPT-4o prompt for labeling hallucinated entities.}\label{tb:gpt-4-prompt}
	\end{center}
\vspace{-0cm}
\end{table*}

% \begin{figure}[t]
%     \centering
%     \includegraphics[width=0.9\linewidth]{Image/abla2/doc7.png}
%     \caption{Improvement of generated documents over direct retrieval on different models.}
%     \label{fig:comparison}
% \end{figure}

\begin{figure}[t]
    \centering
    \subfigure[Unsupervised Dense Retriever.]{
        \label{fig:imp:unsupervised}
        \includegraphics[width=0.8\linewidth]{Image/A.3_fig/improvement_unsupervised.pdf}
    }
    \subfigure[Supervised Dense Retriever.]{
        \label{fig:imp:supervised}
        \includegraphics[width=0.8\linewidth]{Image/A.3_fig/improvement_supervised.pdf}
    }
    
    % \\
    % \subfigure[Comparison of Reasoning Quality With Different Method.]{
    %     \label{fig:reasoning} 
    %     \includegraphics[width=0.98\linewidth]{images/reasoning1.pdf}
    % }
    \caption{Improvements of LLM-QE in Both Unsupervised and Supervised Dense Retrievers. We plot the change of nDCG@10 scores before and after the query expansion using our LLM-QE model.}
    \label{fig:imp}
\end{figure}
\section{Appendix}
\subsection{License}
The authors of 4 out of the 15 datasets in the BEIR benchmark (NFCorpus, FiQA-2018, Quora, Climate-Fever) and the authors of ELI5 in the E5 dataset do not report the dataset license in the paper or a repository. We summarize the licenses of the remaining datasets as follows.

MS MARCO (MIT License); FEVER, NQ, and DBPedia (CC BY-SA 3.0 license); ArguAna and Touché-2020 (CC BY 4.0 license); CQADupStack and TriviaQA (Apache License 2.0); SciFact (CC BY-NC 2.0 license); SCIDOCS (GNU General Public License v3.0); HotpotQA and SQuAD (CC BY-SA 4.0 license); TREC-COVID (Dataset License Agreement).

All these licenses and agreements permit the use of their data for academic purposes.

\subsection{Additional Experimental Details}\label{app:experiment_detail}
This subsection outlines the components of the training data and presents the prompt templates used in the experiments.


\textbf{Training Datasets.} Following the setup of \citet{wang2024improving}, we use the following datasets: ELI5 (sample ratio 0.1)~\cite{fan2019eli5}, HotpotQA~\cite{yang2018hotpotqa}, FEVER~\cite{thorne2018fever}, MS MARCO passage ranking (sample ratio 0.5) and document ranking (sample ratio 0.2)~\cite{bajaj2016ms}, NQ~\cite{karpukhin2020dense}, SQuAD~\cite{karpukhin2020dense}, and TriviaQA~\cite{karpukhin2020dense}. In total, we use 808,740 training examples.

\textbf{Prompt Templates.} Table~\ref{tab:prompt_template} lists all the prompts used in this paper. In each prompt, ``query'' refers to the input query for which query expansions are generated, while ``Related Document'' denotes the ground truth document relevant to the original query. We observe that, in general, the model tends to generate introductory phrases such as ``Here is a passage to answer the question:'' or ``Here is a list of keywords related to the query:''. Before using the model outputs as query expansions, we first filter out these introductory phrases to ensure cleaner and more precise expansion results.



\subsection{Query Expansion Quality of LLM-QE}\label{app:analysis}
This section evaluates the quality of query expansion of LLM-QE. As shown in Figure~\ref{fig:imp}, we randomly select 100 samples from each dataset to assess the improvement in retrieval performance before and after applying LLM-QE.

Overall, the evaluation results demonstrate that LLM-QE consistently improves retrieval performance in both unsupervised (Figure~\ref{fig:imp:unsupervised}) and supervised (Figure~\ref{fig:imp:supervised}) settings. However, for the MS MARCO dataset, LLM-QE demonstrates limited effectiveness in the supervised setting. This can be attributed to the fact that MS MARCO provides higher-quality training signals, allowing the dense retriever to learn sufficient matching signals from relevance labels. In contrast, LLM-QE leads to more substantial performance improvements on the NQ and HotpotQA datasets. This indicates that LLM-QE provides essential matching signals for dense retrievers, particularly in retrieval scenarios where high-quality training signals are scarce.


\subsection{Case Study}\label{app:case_study}
\begin{figure}[htb]
\small
\begin{tcolorbox}[left=3pt,right=3pt,top=3pt,bottom=3pt,title=\textbf{Conversation History:}]
[human]: Craft an intriguing opening paragraph for a fictional short story. The story should involve a character who wakes up one morning to find that they can time travel.

...(Human-Bot Dialogue Turns)... \textcolor{blue}{(Topic: Time-Travel Fiction)}

[human]: Please describe the concept of machine learning. Could you elaborate on the differences between supervised, unsupervised, and reinforcement learning? Provide real-world examples of each.

...(Human-Bot Dialogue Turns)... \textcolor{blue}{(Topic: Machine learning Concepts and Types)}


[human]: Discuss antitrust laws and their impact on market competition. Compare the antitrust laws in US and China along with some case studies

...(Human-Bot Dialogue Turns)... \textcolor{blue}{(Topic: Antitrust Laws and Market Competition)}

[human]: The vertices of a triangle are at points (0, 0), (-1, 1), and (3, 3). What is the area of the triangle?

...(Human-Bot Dialogue Turns)... \textcolor{blue}{(Topic: Area and Circumcircle of Triangle)}

...(Human-Bot Dialogue Interaction on Other Topics)... 

\end{tcolorbox}
\begin{tcolorbox}[left=3pt,right=3pt,top=3pt,bottom=3pt,title=\textbf{Question:}]
What title reflects the thematic essence of the Doctor's experience with time travel as described in the story?
\end{tcolorbox}
\begin{tcolorbox}[left=3pt,right=3pt,top=3pt,bottom=3pt,title=\textbf{Ground-Truth Answer:}]
Awakening to Eternity: The Doctor's New Dawn
\end{tcolorbox}
\begin{tcolorbox}[left=3pt,right=3pt,top=3pt,bottom=3pt,title=\textbf{Retrieved Memory (Turn-Level):}]
[human]: Describe the process and write out the balanced chemical equation for the reaction that occurs when solid calcium carbonate reacts with hydrochloric acid to form aqueous calcium chloride, carbon dioxide, and water. What type of reaction is this, and what observations might indicate that the reaction is taking place?

[bot]: When solid calcium carbonate ($\text{CaCO}_{3}$) reacts with hydrochloric acid (HCl), the products formed are aqueous calcium chloride ($\text{CaCl}_{2}$), carbon dioxide ($\text{CO}_{2}$), and water ($\text{H}_{2}\text{O}$). The balanced chemical equation for this reaction is: ... \textcolor{blue}{(Topic: Calcium Carbonate and Hydrochloric Acid Reaction)}

\textcolor{brown}{
[human]: Write a title for your story.}

\textcolor{brown}{
[bot]: "Awakening to Eternity: The Doctor's New Dawn".
} \textcolor{blue}{(Topic: Time-Travel Fiction)}

[human]: Analyze the following customer reviews from different sources for three different smartphones - the latest iPhone, Samsung Galaxy, and Google Pixel - and provide an overall rating for each phone on a scale of 1 to 10. Consider the following complex and contradictory reviews: - TechRadar's review of the latest iPhone: The new iPhone is a stunning triumph of engineering that sets a new bar for smartphone performance and camera quality. However, the incremental design and high price mean it lacks the 'wow' factor of previous iPhones. Still, its power and intelligence are unrivaled. - CNET's review of the latest Samsung Galaxy: ... Return the answer as a JSON object with the overall ratings for each phone out of 10, to one decimal place.

[bot]: ```json { "iPhone": 8.5, "Samsung Galaxy": 8.0, "Google Pixel": 7.5 } ```

\textcolor{blue}{(Topic: Smartphone Review Ratings Analysis)}
\end{tcolorbox}
\begin{tcolorbox}[left=3pt,right=3pt,top=3pt,bottom=3pt,title=\textbf{Answer (Turn-Level Retrieval):}]
As an AI language model, I do not have access to the story you mentioned earlier. Please provide me with the story or the necessary details so that I can assist you better.

\textbf{\textcolor{red}{(The next part of this figure~\ref{fig: case_study_segment_vs_turn} will be continued on the next page.)}}
\end{tcolorbox}
\end{figure}

\begin{figure}[htb]
% \ContinuedFloat
\small
\begin{tcolorbox}[left=3pt,right=3pt,top=3pt,bottom=3pt,title=\textbf{Retrieved Memory (Ours):}]
[human]: Craft an intriguing opening paragraph for a fictional short story. The story should involve a character who wakes up one morning to find that they can time travel.

[bot]: In the hushed pre-dawn hours, when the world still slumbers under the cloak of night, Alex found themselves caught in the fragile boundary between sleep and wakefulness. ...

[human]: Summarize the story with three bullet points using only nouns and adjectives, without verbs.

[bot]: - Alex, pre-dawn awakening, time-travel discovery - Disorientation, vertigo, reality's malleability - Sunlight, new day, astonishing truth

[human]: Is there any famous fictional character that can time travel? If so, can you rewrite your story with him or her?

[bot]: - \textcolor{brown}{Doctor Who, a renowned time-traveling character from British television, is a perfect fit for this scenario.} In the silent embrace of early morning, where shadows merge with the fading night, the Doctor found themselves in the thin veil separating dreams from consciousness. It was here, amidst the soft whispers of time, that an unusual sensation enveloped them—a sense of time's fabric slipping through their fingers like grains of sand. As dawn's light broke through the darkness, painting the room in hues of gold and amber, the Doctor's eyes opened to what appeared to be an ordinary day. \textcolor{brown}{However, upon standing, a wave of dizziness overwhelmed them, the surroundings blurred, and in that moment of confusion, the Doctor realized a profound truth: they had not only awakened to a new dawn but had also stumbled upon the incredible ability to navigate the currents of time at will.}

\textcolor{brown}{[human]: Write a title for your story.}

\textcolor{brown}{[bot]: "Awakening to Eternity: The Doctor's New Dawn"} \textcolor{blue}{(Topic: Time-Travel Fiction)}
\end{tcolorbox}
\begin{tcolorbox}[left=3pt,right=3pt,top=3pt,bottom=3pt,title=\textbf{Answer (Ours):}]
The title "Awakening to Eternity: The Doctor's New Dawn" reflects the thematic essence of the Doctor's experience with time travel as described in the story. It captures the idea of a new beginning, a sense of eternity, and the awakening of a new power that will change the course of the Doctor's life forever.
\end{tcolorbox}
\caption{Case study comparing our \sysname\ with the \textit{turn-level} memory baseline. When constructing the memory bank at the \textit{turn-level}, the agent fails to provide a satisfactory response due to fragmentation: the retrieved dialogue turns can not capture a complete human-agent interaction related to the query, missing critical information highlighted in \textcolor{brown}{brown}. The dialogue topic is emphasized in \textcolor{blue}{blue}. Some context is omitted (replaced by ...) due to length constraints.}
\label{fig: case_study_segment_vs_turn}
\end{figure}

\newpage

\begin{figure}[htbp]
\small
\begin{tcolorbox}[left=3pt,right=3pt,top=3pt,bottom=3pt,title=\textbf{Conversation History:}]
[human]: Explain what's base rate fallacy and list five specific examples of ...

... (Human-Bot Dialogue Turns) ... \textcolor{blue}{(Topic: Base Rate Fallacy In Politics)}


[human]: How many integers are in the solution of the inequality $|x + 5| < 10$?

... (Human-Bot Dialogue Turns) ... \textcolor{blue}{(Topic: Number of Integers in Inequalities)}

[human]: How have the Alps and Rhine River influenced settlement and agriculture in Western Europe?

... (Human-Bot Dialogue Turns) ... \textcolor{blue}{(Topic: Alps and Rhine River Impact)}

[human]: In a survey conducted at a local high school, preferences for a new school color were measured: 58\% of students liked the color blue, 45\% preferred green, and 22\% liked both colors. If we randomly pick a student from the school, what's the probability that they would like neither blue nor green?

... (Human-Bot Dialogue Turns) ... \textcolor{blue}{(Topic: Probability of Liking Colors in Survey)}

[human]: Imagine you are participating in a race with a group of people. If you have just overtaken the second person, what's your current position? Where is the person you just overtook?

... (Human-Bot Dialogue Turns) ... \textcolor{blue}{(Topic: Overtaking Positions in Race)}
\end{tcolorbox}
\begin{tcolorbox}[left=3pt,right=3pt,top=3pt,bottom=3pt,title=\textbf{Question:}]
[human]: In the inequality question, how many positive integers are in the solution of the second inequality?
\end{tcolorbox}
\begin{tcolorbox}[left=3pt,right=3pt,top=3pt,bottom=3pt,title=\textbf{Ground-Truth Answer:}]
In the second inequality, $(|x + 10| < 5)$, the solution range is $(-15 < x < -5)$. Within this range, there are no positive integers, as all values of $(x)$ that satisfy the inequality are negative.
\end{tcolorbox}
\begin{tcolorbox}[left=3pt,right=3pt,top=3pt,bottom=3pt,title=\textbf{Retrieved Memory (Session-Level):}]
[human]: Explain what's base rate fallacy and list five specific examples of ...

... (Human-Bot Dialogue Turns) ... \textcolor{blue}{(Topic: Base Rate Fallacy In Politics)}

[human]: How many integers are in the solution of the inequality $|x + 5| < 10$?

[bot]: To solve the inequality $|x + 5| < 10$, ... 

\textcolor{brown}{
[human]: What about $|x + 10| < 5$?
}

\textcolor{brown}{
[bot]: To solve the inequality $|x + 10| < 5$, we ...} \textcolor{gray}{(``overwhelmed'' by other irrelevant, redundant content)}

... (Human-Bot Dialogue Turns) ...  \textcolor{blue}{(Topic: Number of Integers in Inequalities)}

% \textcolor{brown}{
% [human]: What about $|x + 10| < 5$
% }

% \textcolor{brown}{
% [bot]: To solve the inequality $|x + 10| < 5$, ...
% } \textcolor{blue}{(Topic: Number of Integers in Inequalities)}

[human]: How have the Alps and Rhine River influenced settlement and agriculture in Western Europe?

... (Human-Bot Dialogue Turns) ... \textcolor{blue}{(Topic: Alps and Rhine River Impact)}

[human]: In a survey conducted at a local high school, ...

... (Human-Bot Dialogue Turns) ... \textcolor{blue}{(Topic: Probability of Liking Colors in Survey)}
\end{tcolorbox}
\begin{tcolorbox}[left=3pt,right=3pt,top=3pt,bottom=3pt,title=\textbf{Answer (Session-Level Retrieval):}]
I apologize for the confusion earlier. To answer your question, the second inequality was not provided in our previous conversation. If you could provide the inequality, I would be happy to help you find the solution and the number of positive integers in the solution set.

\textbf{\textcolor{red}{(The next part of this figure~\ref{fig: case_study_segment_vs_session} will be continued on the next page.)}}
\end{tcolorbox}
\end{figure}
\begin{figure}[htbp]
% \ContinuedFloat
\small
\begin{tcolorbox}[left=3pt,right=3pt,top=3pt,bottom=3pt,title=\textbf{Retrieved Memory (Ours):}]
[human]: How many integers are in the solution of the inequality $|x + 5| < 10$?

[bot]: To solve the inequality $|x + 5| < 10$, we break it into two cases based on the definition of absolute value: 
1. $x + 5 < 10$ 
2. $-(x + 5) < 10$, which simplifies to $x + 5 > -10$ 
Solving each case: 
1. $x < 5$ 
2. $x > -15$ 
Combining these, we get $-15 < x < 5$. This inequality means that $x$ can be any number greater than $-15$ and less than $5$. To find how many integers satisfy this condition, we list them: $-14, -13, -12, -11, -10, -9, -8, -7, -6, -5, -4, -3, -2, -1, 0, 1, 2, 3, 4$. There are 19 integers that satisfy the inequality $|x + 5| < 10$.

\textcolor{brown}{
[human]: What about $|x + 10| < 5$?
}

\textcolor{brown}{
[bot]: To solve the inequality $|x + 10| < 5$, we again consider two cases: 
1. $x + 10 < 5$ 
2. $-(x + 10) < 5$, which simplifies to $x + 10 > -5$ 
Solving each case: 
1. $x < -5$ 
2. $x > -15$ 
Combining these, we get $-15 < x < -5$. This inequality means that $x$ can be any number greater than $-15$ and less than $-5$. To find how many integers satisfy this condition, we list them: $-14, -13, -12, -11, -10, -9, -8, -7, -6$. There are 9 integers that satisfy the inequality $|x + 10| < 5$.
}

[human]: What about $|x + 10| < 10$?

[bot]: To solve the inequality $|x + 10| < 10$, we consider two cases: 
1. $x + 10 < 10$ 
2. $-(x + 10) < 10$, which simplifies to $x + 10 > -10$ 
Solving each case: 
1. $x < 0$ 
2. $x > -20$ 
Combining these, we get $-20 < x < 0$. This inequality means that $x$ can be any number greater than $-20$ and less than $0$. To find how many integers satisfy this condition, we list them: $-19, -18, -17, -16, -15, -14, -13, -12, -11, -10, -9, -8, -7, -6, -5, -4, -3, -2, -1$. There are 19 integers that satisfy the inequality $|x + 10| < 10$. \textcolor{blue}{(Topic: Number of Integers in Inequalities)}
\end{tcolorbox}
\begin{tcolorbox}[left=3pt,right=3pt,top=3pt,bottom=3pt,title=\textbf{Answer (Ours):}]
To find the number of positive integers that satisfy the inequality $(|x + 10| < 5)$, we need to look for positive integers in the solution set $(-15 < x < -5)$. Since there are no positive integers in this interval, the answer is 0.

\end{tcolorbox}
\caption{Case study comparing our \sysname\ with the \textit{session-level} memory baseline. When constructing the memory bank at the \textit{session-level}, the agent fails to provide a satisfactory response due to the inclusion of excessive irrelevant information within a session. This distracts the generator from focusing on the truly useful information highlighted in \textcolor{brown}{brown}. The dialogue topic is emphasized in \textcolor{blue}{blue}. Some context is omitted (replaced by ...) due to length constraints.}
\label{fig: case_study_segment_vs_session}
\end{figure}



\begin{figure}[htb]
\small
\begin{tcolorbox}[left=3pt,right=3pt,top=3pt,bottom=3pt,title=\textbf{Conversation History:}]
[human]: Photosynthesis is a vital process for life on Earth. Could you outline the two main stages of photosynthesis, including where they take place within the chloroplast, and the primary inputs and outputs for each stage? ... (Human-Bot Dialogue Turns)... \textcolor{blue}{(Topic: Photosynthetic Energy Production)}

[human]: Please assume the role of an English translator, tasked with correcting and enhancing spelling and language. Regardless of the language I use, you should identify it, translate it, and respond with a refined and polished version of my text in English. 

... (Human-Bot Dialogue Turns)...  \textcolor{blue}{(Topic: Language Translation and Enhancement)}

[human]: Suggest five award-winning documentary films with brief background descriptions for aspiring filmmakers to study.

\textcolor{brown}{[bot]: ...
5. \"An Inconvenient Truth\" (2006) - Directed by Davis Guggenheim and featuring former United States Vice President Al Gore, this documentary aims to educate the public about global warming. It won two Academy Awards, including Best Documentary Feature. The film is notable for its straightforward yet impactful presentation of scientific data, making complex information accessible and engaging, a valuable lesson for filmmakers looking to tackle environmental or scientific subjects.}

... (Human-Bot Dialogue Turns)... 
\textcolor{blue}{(Topic: Documentary Films Recommendation)}

[human]: Given the following records of stock prices, extract the highest and lowest closing prices for each month in the year 2022. Return the results as a CSV string, with one line allocated for each month. Date,Open,High,Low,Close,Volume ... ... (Human-Bot Dialogue Turns)...  \textcolor{blue}{(Topic: Stock Prices Analysis)}

[human]: The city of Vega intends to build a bridge that will span the Vegona River, covering a distance of 1.8 kilometers. The proposed location falls within a seismically active area that has experienced several high-magnitude earthquakes. Given these circumstances, what would be the best approach to constructing the bridge? ... (Human-Bot Dialogue Turns)...  \textcolor{blue}{(Topic: Seismic Bridge Construction)}
\end{tcolorbox}
\begin{tcolorbox}[left=3pt,right=3pt,top=3pt,bottom=3pt,title=\textbf{Question:}]
Which documentary won two Academy Awards and what lesson does it offer to filmmakers?
\end{tcolorbox}
\begin{tcolorbox}[left=3pt,right=3pt,top=3pt,bottom=3pt,title=\textbf{Ground-Truth Answer:}]
``An Inconvenient Truth" (2006) won two Academy Awards. It offers the lesson of making complex information accessible and engaging, a valuable lesson for filmmakers looking to tackle environmental or scientific subjects.
\end{tcolorbox}
\begin{tcolorbox}[left=3pt,right=3pt,top=3pt,bottom=3pt,title=\textbf{Memory generated by RecurSum~\citep{wang2023recursively}:}]
The dialogue provided detailed information on various topics, including the stages of photosynthesis, energy production estimates through photosynthesis for different organisms, translations and enhancements of phrases in different languages, recommendations for documentary films, a pitch for a film about overcoming adversity, analysis of stock prices, and considerations for constructing a bridge in a seismically active area. Here's a summary of the key points: ...

\textbf{Documentary Film Recommendations}: Suggested five award-winning documentary films for aspiring filmmakers to study, with brief background descriptions for each. The response included:

-  "March of the Penguins" (2005) - Emperor penguins' journey in Antarctica; noted for its cinematography and portrayal of the penguins' life cycle.

-  "Citizenfour" (2014) - Edward Snowden and NSA spying scandal; known for its real-time exploration of Snowden's revelations.

-  "20 Feet from Stardom" (2013) - Stories of backup singers behind musical legends; celebrates unsung heroes in the music industry.

-  "Man on Wire" (2008) - Philippe Petit's 1974 high-wire walk between the Twin Towers; combines personal footage, interviews, and re-enactments.

-  \textcolor{brown}{"An Inconvenient Truth" (2006) - Al Gore's documentary on global warming; known for its presentation of scientific data.}
\end{tcolorbox}
\begin{tcolorbox}[left=3pt,right=3pt,top=3pt,bottom=3pt,title=\textbf{Answer (RSum):}]
I'm sorry, but based on the updated context you provided, I don't see any previous discussion related to the question you asked. However, to answer your question, the documentary that won two Academy Awards is "Free Solo," which won for Best Documentary Feature and Best Achievement in Film Editing in 2019. 
% The lesson it offers to filmmakers is the importance of taking risks and pushing boundaries in order to create something truly unique and impactful.
....
The film follows climber Alex Honnold as he attempts to free solo climb El Capitan in Yosemite National Park, and the filmmakers had to take risks themselves in order to capture the footage and tell the story.

\textbf{\textcolor{red}{(The next part of this figure~\ref{fig: case_study_segment_vs_rsum} will be continued on the next page.)}}
\end{tcolorbox}
\end{figure}
\begin{figure}[htbp]
% \ContinuedFloat
\small
\begin{tcolorbox}[left=3pt,right=3pt,top=3pt,bottom=3pt,title=\textbf{Retrieved Memory (Ours):}]
[human]: Suggest five award-winning documentary films with brief background descriptions for aspiring filmmakers to study.

[bot]: 1. **"March of the Penguins" (2005)** - Directed by Luc Jacquet, this French feature-length nature documentary delves into the annual journey of Emperor penguins across the Antarctic. It won the Academy Award for Best Documentary Feature. The film is celebrated for its stunning cinematography and its detailed, heartfelt portrayal of the penguins' life cycle, making it a must-study for filmmakers interested in wildlife documentaries. 2. **"Citizenfour" (2014)** ... 5. \textcolor{brown}{**"An Inconvenient Truth" (2006)** - Directed by Davis Guggenheim and featuring former United States Vice President Al Gore, this documentary aims to educate the public about global warming. It won two Academy Awards, including Best Documentary Feature.} The film is notable for its straightforward yet impactful presentation of scientific data, making complex information accessible and engaging, \textcolor{brown}{a valuable lesson for filmmakers looking to tackle environmental or scientific subjects.}

[human]: With the spirit in the first film, craft a succinct and persuasive pitch for a film about overcoming adversity.

[bot]: Drawing inspiration from the resilience and majestic journey of Emperor penguins in "March of the Penguins," our documentary, "Unyielding Spirits: The Human Odyssey," will explore the profound stories of individuals around the globe who have faced and overcome staggering adversities. Through breathtaking cinematography and intimate storytelling, we ...
\textcolor{blue}{(Topic: Documentary Films Recommendation)}
\end{tcolorbox}
\begin{tcolorbox}[left=3pt,right=3pt,top=3pt,bottom=3pt,title=\textbf{Answer (Ours):}]
``An Inconvenient Truth" won two Academy Awards, including Best Documentary Feature. The film offers a valuable lesson to filmmakers looking to tackle environmental or scientific subjects, as it presents complex information in a straightforward yet impactful way, making it accessible and engaging to a wider audience.


\end{tcolorbox}
\caption{Case study comparing our \sysname\ with RecurSum~\citep{wang2023recursively}. When using RecurSum, the agent fails to provide a satisfactory answer due to information loss during the summarization process. Summarization omits essential details needed to answer the user's new request, as highlighted in \textcolor{brown}{brown}. The dialogue topic is emphasized in \textcolor{blue}{blue}. Some context is omitted (replaced by ...) due to length constraints.}
\label{fig: case_study_segment_vs_rsum}
\end{figure}




\begin{figure}[htb]
\small
\begin{tcolorbox}[left=3pt,right=3pt,top=3pt,bottom=3pt,title=\textbf{Conversation History:}]
[human]: Photosynthesis is a vital process for life on Earth. Could you outline the two main stages of photosynthesis, including where they take place within the chloroplast, and the primary inputs and outputs for each stage? ... (Human-Bot Dialogue Turns)... \textcolor{blue}{(Topic: Photosynthetic Energy Production)}

[human]: Please assume the role of an English translator, tasked with correcting and enhancing spelling and language. Regardless of the language I use, you should identify it, translate it, and respond with a refined and polished version of my text in English. 

... (Human-Bot Dialogue Turns)...  \textcolor{blue}{(Topic: Language Translation and Enhancement)}

[human]: Suggest five award-winning documentary films with brief background descriptions for aspiring filmmakers to study.

\textcolor{brown}{[bot]: ...
5. \"An Inconvenient Truth\" (2006) - Directed by Davis Guggenheim and featuring former United States Vice President Al Gore, this documentary aims to educate the public about global warming. It won two Academy Awards, including Best Documentary Feature. The film is notable for its straightforward yet impactful presentation of scientific data, making complex information accessible and engaging, a valuable lesson for filmmakers looking to tackle environmental or scientific subjects.}

... (Human-Bot Dialogue Turns)... 
\textcolor{blue}{(Topic: Documentary Films Recommendation)}

[human]: Given the following records of stock prices, extract the highest and lowest closing prices for each month in the year 2022. Return the results as a CSV string, with one line allocated for each month. Date,Open,High,Low,Close,Volume ... ... (Human-Bot Dialogue Turns)...  \textcolor{blue}{(Topic: Stock Prices Analysis)}

[human]: The city of Vega intends to build a bridge that will span the Vegona River, covering a distance of 1.8 kilometers. The proposed location falls within a seismically active area that has experienced several high-magnitude earthquakes. Given these circumstances, what would be the best approach to constructing the bridge? ... (Human-Bot Dialogue Turns)...  \textcolor{blue}{(Topic: Seismic Bridge Construction)}
\end{tcolorbox}
\begin{tcolorbox}[left=3pt,right=3pt,top=3pt,bottom=3pt,title=\textbf{Question:}]
Which documentary won two Academy Awards and what lesson does it offer to filmmakers?
\end{tcolorbox}
\begin{tcolorbox}[left=3pt,right=3pt,top=3pt,bottom=3pt,title=\textbf{Ground-Truth Answer:}]
"An Inconvenient Truth" (2006) won two Academy Awards. It offers the lesson of making complex information accessible and engaging, a valuable lesson for filmmakers looking to tackle environmental or scientific subjects.
\end{tcolorbox}
\begin{tcolorbox}[left=3pt,right=3pt,top=3pt,bottom=3pt,title=\textbf{Memory generated by ConditionMem~\citep{yuan2023evolving}:}]
- \textcolor{brown}{The user is seeking recommendations for award-winning documentary films, along with brief background descriptions, for educational purposes.} This request likely stems from a desire to learn from successful examples in the documentary filmmaking field, possibly to improve their own filmmaking skills or to gain inspiration for their projects. 

- \textcolor{brown}{User is interested in award-winning documentary films for educational purposes, aiming to study their backgrounds and filmmaking techniques.}

- The user's request for crafting a succinct and persuasive pitch for a film about overcoming adversity is inspired by the spirit of "March of the Penguins," a documentary that beautifully captures the Emperor penguins' annual journey across the Antarctic, highlighting themes of endurance and survival against the odds. The user seeks to emulate the emotional depth and cinematic quality that made "March of the Penguins" compelling, aiming to create a pitch that resonates with similar themes of resilience and triumph over challenges. 

- User is interested in creating content that focuses on themes of overcoming adversity, inspired by the emotional depth and cinematic excellence of documentaries like "March of the Penguins." They value storytelling that captures the essence of endurance and survival against the odds.
% : The user's request to write a budget plan for the video production directly follows the task of creating a video script aimed at training new employees of a car wash business in Japan, focusing on business etiquette norms.
% The user is likely seeking to understand the financial requirements to bring the previously crafted script to life, indicating a move from the planning and content development phase to the practical aspects of production. 

... (Other Irrelevant Memory Records) ...
\end{tcolorbox}
\begin{tcolorbox}[left=3pt,right=3pt,top=3pt,bottom=3pt,title=\textbf{Answer (Conditional Memory):}]
The documentary that won two Academy Awards is "Free Solo," which follows climber Alex Honnold as he attempts to free solo climb El Capitan in Yosemite National Park. The film offers a lesson to filmmakers on the importance of capturing authentic and compelling human stories, as well as the technical challenges of filming in extreme environments.

\textbf{\textcolor{red}{(The next part of this figure~\ref{fig: case_study_segment_vs_condmem} will be continued on the next page.)}}
\end{tcolorbox}
\end{figure}
\begin{figure}[htbp]
% \ContinuedFloat
\small
\begin{tcolorbox}[left=3pt,right=3pt,top=3pt,bottom=3pt,title=\textbf{Retrieved Memory (Ours):}]
[human]: Suggest five award-winning documentary films with brief background descriptions for aspiring filmmakers to study.

[bot]: 1. **"March of the Penguins" (2005)** - Directed by Luc Jacquet, this French feature-length nature documentary delves into the annual journey of Emperor penguins across the Antarctic. It won the Academy Award for Best Documentary Feature. The film is celebrated for its stunning cinematography and its detailed, heartfelt portrayal of the penguins' life cycle, making it a must-study for filmmakers interested in wildlife documentaries. 2. **"Citizenfour" (2014)** ... 5. \textcolor{brown}{**"An Inconvenient Truth" (2006)** - Directed by Davis Guggenheim and featuring former United States Vice President Al Gore, this documentary aims to educate the public about global warming. It won two Academy Awards, including Best Documentary Feature.} The film is notable for its straightforward yet impactful presentation of scientific data, making complex information accessible and engaging, a valuable lesson for filmmakers looking to tackle environmental or scientific subjects.

[human]: With the spirit in the first film, craft a succinct and persuasive pitch for a film about overcoming adversity.

[bot]: Drawing inspiration from the resilience and majestic journey of Emperor penguins in "March of the Penguins," our documentary, "Unyielding Spirits: The Human Odyssey," will explore the profound stories of individuals around the globe who have faced and overcome staggering adversities. Through breathtaking cinematography and intimate storytelling, we ...
\textcolor{blue}{(Topic: Documentary Films Recommendation)}
\end{tcolorbox}
\begin{tcolorbox}[left=3pt,right=3pt,top=3pt,bottom=3pt,title=\textbf{Answer (Ours):}]
``An Inconvenient Truth" won two Academy Awards, including Best Documentary Feature. The film offers a valuable lesson to filmmakers looking to tackle environmental or scientific subjects, as it presents complex information in a straightforward yet impactful way, making it accessible and engaging to a wider audience.
\end{tcolorbox}
\caption{Case study comparing our \sysname\ with ConditionMem~\citep{yuan2023evolving}. When using ConditionMem, the agent fails to provide a satisfactory answer due to (1) information loss during the summarization process and (2) the incorrect discarding of turns that are actually useful, as highlighted in \textcolor{brown}{brown}. The dialogue topic is emphasized in \textcolor{blue}{blue}. Some context is omitted (replaced by ...) due to length constraints.}
\label{fig: case_study_segment_vs_condmem}
\end{figure}


To further demonstrate the effectiveness of LLM-QE, we conduct a case study by randomly sampling a query from the evaluation dataset. We then compare retrieval performance using the raw queries, expanded queries by vanilla LLM, and expanded queries by LLM-QE.

As shown in Table~\ref{tab:case_study}, query expansion significantly improves retrieval performance compared to using the raw query. Both vanilla LLM and LLM-QE generate expansions that include key phrases, such as ``temperature'', ``humidity'', and ``coronavirus'', which provide crucial signals for document matching. However, vanilla LLM produces inconsistent results, including conflicting claims about temperature ranges and virus survival conditions. In contrast, LLM-QE generates expansions that are more semantically aligned with the golden passage, such as ``the virus may thrive in cooler and more humid environments, which can facilitate its transmission''. This further demonstrates the effectiveness of LLM-QE in improving query expansion by aligning with the ranking preferences of both LLMs and retrievers.



\vspace{-30pt}
\begin{IEEEbiography}[{\includegraphics[width=1in,height=1.25in,clip,keepaspectratio]{Figures/people/ChaoyueYin.jpg}}]{Chaoyue Yin}
is currently a master candidate with Department of Computer Science and Engineering, Southern University of Science and Technology. 
He received his B.E. degree in computer science and technology from Southern University of Science and Technology in 2024. 
His research interests are mainly in blockchain sharding and interoperability protocol.
\end{IEEEbiography}
\vspace{-30pt}
\begin{IEEEbiography}[{\includegraphics[width=1in,height=1.25in,clip,keepaspectratio]{Figures/people/Mingzhe}}]{Mingzhe Li}
is currently a Scientist with the Institute of High Performance Computing (IHPC), A*STAR, Singapore.
He received his Ph.D. degree from the Department of Computer Science and Engineering, Hong Kong University of Science and Technology in 2022.
Prior to that, he received his B.E. degree from Southern University of Science and Technology.
His research interests are mainly in blockchain sharding, consensus protocol, blockchain application, network economics, and crowdsourcing.
\end{IEEEbiography}
\vspace{-30pt}
\begin{IEEEbiography}
[{\includegraphics[width=1in,height=1.25in,clip,keepaspectratio]{Figures/people/JinZhang}}]{Jin Zhang} 
is currently an associate professor with Department of Computer Science and Engineering, Southern University of Science and Technology. 
She received her B.E. and M.E. degrees in electronic engineering from Tsinghua University in 2004 and 2006, respectively, and received her Ph.D. degree in computer science from Hong Kong University of Science and Technology in 2009. 
% She was then employed in HKUST as a research assistant professor. 
Her research interests are mainly in mobile healthcare and wearable computing, wireless communication and networks, network economics, cognitive radio networks and dynamic spectrum management. 
% She has published more than 50 papers in top-level journals and conferences. 
% She is the Principle Investigator of several research projects funded by National Natural Science Foundation of China, Hong Kong Research Grants Council and Hong Kong Innovation and Technology Commission. 
\end{IEEEbiography}
\vspace{-30pt}
\begin{IEEEbiography}[{\includegraphics[width=1in,height=1.25in,clip,keepaspectratio]{Figures/people/YouLin}}]{You Lin}
is currently a master candidate with Department of Computer Science and Engineering, Southern University of Science and Technology. 
He received his B.E. degree in computer science and technology from Southern University of Science and Technology in 2021. 
His research interests are mainly in blockchain, network economics, and consensus protocols.
\end{IEEEbiography}
\vspace{-30pt}
\begin{IEEEbiography}[{\includegraphics[width=1in,height=1.25in,clip,keepaspectratio]{Figures/people/Qingsong.png}}]{Qingsong Wei}
received the PhD degree in computer science from the University of Electronic Science and Technologies of China, in 2004. He was with Tongji University as an assistant professor from 2004 to 2005. He is a Group Manager and principal scientist at the Institute of High Performance Computing, A*STAR, Singapore. His research interests include decentralized computing, Blockchain and federated learning. He is a senior member of the IEEE.
\end{IEEEbiography}
\vspace{-30pt}
\begin{IEEEbiography}[{\includegraphics[width=1in,height=1.25in,clip,keepaspectratio]{Figures/people/Rick.png}}]{Siow Mong Rick Goh}
received his Ph.D. degree in electrical and computer engineering from the National University of Singapore. He is the Director of the Computing and Intelligence (CI) Department, Institute of High Performance Computing, Agency for Science, Technology and Research, Singapore, where he leads a team of over 80 scientists in performing world-leading scientific research, developing technology to commercialization, and engaging and collaborating with industry. His current research interests include artificial intelligence, high-performance computing, blockchain, and federated learning.
\end{IEEEbiography}

\end{document}