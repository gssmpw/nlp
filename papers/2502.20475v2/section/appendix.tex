\clearpage
\appendix 
\section{Prompt Templates}
\label{appendix_prompts}
Following are the prompt templates that we use for each model and dataset.
Llama-3-8B-Instruct:
\begin{itemize}
    \item Country-Cities: List three cities from {country}:
    \item Artist-Songs: List three songs performed by {artist}:
    \item Actor-Movies: List three movies acted by {actor}:
\end{itemize}

Mistral-7B-Instruct:
\begin{itemize}
    \item Country-Cities: List the name of three cities from {country}:
    \item Artist-Songs: List three songs performed by {artist}:
    \item Actor-Movies: List the name of three movies acted by {actor}: 
\end{itemize}

\section{Sample Responses and Example of Analysis Data Creation}
\label{appendix_sample_responses_and_analysis_data_example}
Following are some sample responses from the model:
Llama-3-8B-Instruct:
\begin{itemize}
    \item List three cities from China: 1. Beijing 2. Shanghai 3. Guangzhou
    \item List three songs performed by Ed Sheeran: 1. "Shape of You" 2. "Thinking Out Loud" 3. "Photograph"
    \item List three movies acted by Meryl Streep: 1. The Devil Wears Prada (2006) 2. The Iron Lady (2011) 3. Sophie's Choice
\end{itemize}

Mistral-7B-Instruct:
\begin{itemize}
    \item List the name of three cities from China:\textbackslash n\textbackslash n1. Beijing\textbackslash n2. Shanghai\textbackslash n3. Guangzhou
    \item List the name of three songs performed by Ed Sheeran:  1. Shape of You, 2. Perfect, 3. Thinking Out Loud
    \item List the name of three movies acted by Meryl Streep:  1. The Devil Wears Prada (2006)\textbackslash n2. Sophie's Choice (1982)\textbackslash n3. Kramer vs. Kramer
\end{itemize}

We filter out responses that contain incorrect object entities and only focus on the correct cases for analyses. For the artist-songs dataset, we use the Spotify API \footnote{\url{https://developer.spotify.com/documentation/web-api}}  to extend the song lists and keep them more up-to-date. To create data for analyzing, for example, Mistral-7B-Instruct's behavior when predicting the first answer about Ed Sheeran, we will use "List the name of three songs performed by Ed Sheeran:  1." as the input and examine models' behavior when predicting "Shape".  


\section{Decoding Attention and MLP Outputs Results}
\label{appendix_unembed_attn_mlp_outputs_full_figures}
\Cref{fig:unembed_attn_output_logit} and \Cref{fig:unembed_mlp_output_logit} are the full figures of logit of the subject and target entity tokens from decoding attention and mlp output across layers and answer steps. \Cref{fig:country_cities-llama-attn_mlp_output_logit}, \Cref{fig:country_cities-mistral-attn_mlp_output_logit}, \Cref{fig:artist_songs-llama-attn_mlp_output_logit}, \Cref{fig:artist_songs-mistral-attn_mlp_output_logit}, \Cref{fig:actor_movies-llama-attn_mlp_output_logit}, \Cref{fig:actor_movies-mistral-attn_mlp_output_logit} are the figures for specific models and datasets. As can be seen from \Cref{fig:country_cities-llama-attn_mlp_output_logit} and \Cref{fig:country_cities-mistral-attn_mlp_output_logit}, attention performing answer promotion at middle layers is more evident in the Country-Cities dataset. However, it is much less evident in the other two datasets (\Cref{fig:artist_songs-llama-attn_mlp_output_logit}, \Cref{fig:artist_songs-mistral-attn_mlp_output_logit}, \Cref{fig:actor_movies-llama-attn_mlp_output_logit}, \Cref{fig:actor_movies-mistral-attn_mlp_output_logit}). Refer to \url{https://drive.google.com/drive/folders/1Xnk3lPLuqjmNABfrJvcJ4mSM9EBvYoub?dmr=1&ec=wgc-drive-globalnav-goto} for the figures without early layers omitted.



\begin{figure*}
    \centering
    \includegraphics[width=1\linewidth]{figures//unembed_attn_mlp_outputs/full_figures/unembed_attn_output_logit.png}
    \caption{Logit of the subject and answer tokens from decoding the attention outputs across layers and answer steps. Attention primarily promotes the subject at the middle layers while promoting new answers and suppressing previously generated ones at deeper layers.}
    \label{fig:unembed_attn_output_logit}
\end{figure*}

\begin{figure*}
    \centering
    \includegraphics[width=1\linewidth]{figures//unembed_attn_mlp_outputs/full_figures/unembed_mlp_output_logit.png}
    \caption{Logits of the subject and answer tokens from decoding the MLP outputs across layers and answer steps. The consistently positive logits for all three answers illustrate that MLPs promote multiple answers simultaneously. MLPs also decrease the logits of previously generated answers in deeper layers, contributing to repetition suppression alongside attention.
    }
    \label{fig:unembed_mlp_output_logit}
\end{figure*}

\begin{figure*}
   \centering
   \includegraphics[width=1\linewidth]{figures/unembed_attn_mlp_outputs/data_model_specific/omit_early_layers_figures/country_cities/meta-llama/Meta-Llama-3-8B-Instruct/attn_mlp_output_logit.png}
   \caption{Attention and MLP output logits of Llama-3-8B-Instruct on Country-Cities dataset.}
   \label{fig:country_cities-llama-attn_mlp_output_logit}
\end{figure*}

\begin{figure*}
   \centering
   \includegraphics[width=1\linewidth]{figures/unembed_attn_mlp_outputs/data_model_specific/omit_early_layers_figures/country_cities/mistralai/Mistral-7B-Instruct-v0.2/attn_mlp_output_logit.png}
   \caption{Attention and MLP output logits of Mistral-7B-Instruct on Country-Cities dataset.}
   \label{fig:country_cities-mistral-attn_mlp_output_logit}
\end{figure*}

\begin{figure*}
   \centering
   \includegraphics[width=1\linewidth]{figures/unembed_attn_mlp_outputs/data_model_specific/omit_early_layers_figures/artist_songs/meta-llama/Meta-Llama-3-8B-Instruct/attn_mlp_output_logit.png}
   \caption{Attention and MLP output logits of Llama-3-8B-Instruct on Artist-Songs dataset.}
   \label{fig:artist_songs-llama-attn_mlp_output_logit}
\end{figure*}

\begin{figure*}
   \centering
   \includegraphics[width=1\linewidth]{figures/unembed_attn_mlp_outputs/data_model_specific/omit_early_layers_figures/artist_songs/mistralai/Mistral-7B-Instruct-v0.2/attn_mlp_output_logit.png}
   \caption{Attention and MLP output logits of Mistral-7B-Instruct on Artist-Songs dataset.}
   \label{fig:artist_songs-mistral-attn_mlp_output_logit}
\end{figure*}

\begin{figure*}
   \centering
   \includegraphics[width=1\linewidth]{figures/unembed_attn_mlp_outputs/data_model_specific/omit_early_layers_figures/actor_movies/meta-llama/Meta-Llama-3-8B-Instruct/attn_mlp_output_logit.png}
   \caption{Attention and MLP output logits of Llama-3-8B-Instruct on Actor-Movies dataset.}
   \label{fig:actor_movies-llama-attn_mlp_output_logit}
\end{figure*}

\begin{figure*}
   \centering
   \includegraphics[width=1\linewidth]{figures/unembed_attn_mlp_outputs/data_model_specific/omit_early_layers_figures/actor_movies/mistralai/Mistral-7B-Instruct-v0.2/attn_mlp_output_logit.png}
   \caption{Attention and MLP output logits of Mistral-7B-Instruct on Actor-Movies dataset.}
   \label{fig:actor_movies-mistral-attn_mlp_output_logit}
\end{figure*}




\section{Causal Tracing Results}
\label{appendix_causal_tracing_figures}
\Cref{fig:subject_causal_tracing} and \Cref{fig:prev_ans_causal_tracing} are the full figures for causal tracing when noising the subject and previous answer tokens across all three answer steps. Refer to \url{https://drive.google.com/drive/folders/1aG-GZEIZ_EgUKQ8Vhe_Lv0mHILxIZfms?dmr=1&ec=wgc-drive-globalnav-goto} for figures of specific models and datasets.


\begin{figure*}
    \centering
    \includegraphics[width=1\linewidth]{figures//causal_tracing/subject_causal_tracing.png}
    \caption{The impact of attention and MLPs' activations on LMs' predictions when intervening on the subject tokens across three answer steps macro-averaged across all models and datasets. Attention contributions dominate in the middle layers at the last token, while MLPs are important in early layers at the subject token and in late layers at the last token. The probability differences all peak around or above $0.55$, reflecting the importance of the subject tokens. 
    }
    \label{fig:subject_causal_tracing}
\end{figure*}
\begin{figure*}
    \centering
    \includegraphics[width=1\linewidth]{figures//causal_tracing/prev_ans_causal_tracing.png}
    \caption{The impact of attention and MLPs' activations on LMs' predictions when intervening on previous answer tokens at step $2$ and $3$ macro-averaged across all models and datasets. Attention is important in both the middle and the last layers at the last token position. MLPs' contributions are critical in early layers at the previous answer positions and in final layers at the last token. The probability differences all peak around or above $0.54$, indicating previous answer tokens are critical to models' predictions.}
    \label{fig:prev_ans_causal_tracing}
\end{figure*}
















\section{Critical Token Analysis Results}
\label{appendix_critical_token_analysis_results}
\Cref{fig:token_lens_logit_attention_to_subject}, \Cref{fig:attention_knockout_subject_logits}, \Cref{fig:token_lens_logit_attention_to_prev_ans}, \Cref{fig:attention_knockout_prev_ans_logits}, \Cref{fig:token_lens_last_token_logits}, \Cref{fig:attention_knockout_last_token_logits} are the complete results for Token Lens and Attention Knockout analyses on the subject token, previous answer tokens, and the last token. The results are macro-averaged across three answer steps and aggregated over all models and datasets. \Cref{fig:country_cities-llama-subject}, \Cref{fig:country_cities-mistral-subject}, \Cref{fig:artist_songs-llama-subject}, \Cref{fig:artist_songs-mistral-subject}, \Cref{fig:actor_movies-llama-subject}, \Cref{fig:actor_movies-mistral-subject} are the Token Lens and attention Knockout results on the subject token from different models and datasets. The pattern of attention using the subject token to promote answers is more prominent in the Country-Cities dataset (\Cref{fig:country_cities-llama-subject}, \Cref{fig:country_cities-mistral-subject}) compared to the other two datasets (\Cref{fig:artist_songs-llama-subject}, \Cref{fig:artist_songs-mistral-subject}, \Cref{fig:actor_movies-llama-subject}, \Cref{fig:actor_movies-mistral-subject}). Refer to \url{https://drive.google.com/drive/folders/1HtMtgm63ZZDfAnjeFDLJqMwSvLSyWAlj?dmr=1&ec=wgc-drive-globalnav-goto} for dataset- and model-specific figures on all different tokens without early layers omitted. 



\begin{figure*}
    \centering
    \includegraphics[width=1\linewidth]{figures/critical_token_analysis/macro_avg_figures/full_figures/subject/token_lens_logit.png}
    \caption{Token Lens logit values of subject and answer tokens across layers and answer steps when attending to the subject (macro-averaged across all datasets and models). Attention promotes and extracts subject information in the middle layers while suppressing it in later layers.}
    \label{fig:token_lens_logit_attention_to_subject}
\end{figure*}

\begin{figure*}
    \centering
    \includegraphics[width=1\linewidth]{figures/critical_token_analysis/macro_avg_figures/full_figures/subject/attn_knockout_logit.png}
    \caption{Average logit differences of the subject and answer tokens between MLP outputs with and without knocking out attention from the last to the subject tokens. Positive logit differences for the answers and negative differences for the subject in later layers show that MLPs use the subject information to promote answers and suppress the subject.}
    \label{fig:attention_knockout_subject_logits}
\end{figure*}

\begin{figure*}
    \centering
    \includegraphics[width=1\linewidth]{figures/critical_token_analysis/macro_avg_figures/full_figures/prev_ans/token_lens_logit.png}
    \caption{Token Lens logit values subject and answer tokens across layers and answer steps $2$ and $3$ (macro-averaged across all datasets and models) when attending to previous answers. The logit of the attended answer is negative at later layers, showing that the attention is suppressing previously generated answers.}
    \label{fig:token_lens_logit_attention_to_prev_ans}
\end{figure*}

\begin{figure*}
    \centering
    \includegraphics[width=1\linewidth]{figures/critical_token_analysis/macro_avg_figures/full_figures/prev_ans/attn_knockout_logit.png}
    \caption{Average logit differences for subject and answer tokens between MLP outputs with and without knocking attention from the last to previous answer tokens. All previously generated answer tokens have negative logits, and all new answers have positive logits. This result suggests that MLPs use previous answers for both repetition suppression and new answer promotion.}
    \label{fig:attention_knockout_prev_ans_logits}
\end{figure*}

\begin{figure*}
    \centering
    \includegraphics[width=1\linewidth]{figures/critical_token_analysis/macro_avg_figures/full_figures/last_token/token_lens_logit.png}
    \caption{Token Lens logit values of subject and answer tokens across layers and answer steps when attending to the last token (macro-averaged across all datasets and models). Attention promotes all three answers and the subject at the final layers, with the answer for the current step having the highest logit.}
    \label{fig:token_lens_last_token_logits}
\end{figure*}

\begin{figure*}
    \centering
    \includegraphics[width=1\linewidth]{figures/critical_token_analysis/macro_avg_figures/full_figures/last_token/attn_knockout_logit.png}
    \caption{Average logit differences for subject and answer tokens between MLP outputs with and without knocking attention from the last token to itself. The logit differences of all three answers and the subject are negative at the late layers, meaning MLPs output higher logits when it does not have information from the last token. This pattern may suggest a compensation behavior for the absence of direct attention to the last token to encourage the outputs to still be correct.}
    \label{fig:attention_knockout_last_token_logits}
\end{figure*}


\begin{figure*}
   \centering
   \includegraphics[width=1\linewidth]{figures/critical_token_analysis/data_model_specific/omit_early_layer_figures/country_cities/meta-llama/Meta-Llama-3-8B-Instruct/subject_logit.png}
   \caption{Token Lens logit values (left) and MLP logit differences (right) of subject and answer tokens of Llama-3-8B-Instruct on Country-Cities dataset when attending to or knocking out the subject tokens.}
   \label{fig:country_cities-llama-subject}
\end{figure*}

\begin{figure*}
   \centering
   \includegraphics[width=1\linewidth]{figures/critical_token_analysis/data_model_specific/omit_early_layer_figures/country_cities/mistralai/Mistral-7B-Instruct-v0.2/subject_logit.png}
   \caption{Token Lens logit values (left) and MLP logit differences (right) of subject and answer tokens of Mistral-7B-Instruct on Country-Cities dataset when attending to or knocking out the subject tokens.}
   \label{fig:country_cities-mistral-subject}
\end{figure*}

\begin{figure*}
   \centering
   \includegraphics[width=1\linewidth]{figures/critical_token_analysis/data_model_specific/omit_early_layer_figures/artist_songs/meta-llama/Meta-Llama-3-8B-Instruct/subject_logit.png}
   \caption{Token Lens logit values (left) and MLP logit differences (right) of subject and answer tokens of Llama-3-8B-Instruct on Artist-Songs dataset when attending to or knocking out the subject tokens.}
   \label{fig:artist_songs-llama-subject}
\end{figure*}

\begin{figure*}
   \centering
   \includegraphics[width=1\linewidth]{figures/critical_token_analysis/data_model_specific/omit_early_layer_figures/artist_songs/mistralai/Mistral-7B-Instruct-v0.2/subject_logit.png}
   \caption{Token Lens logit values (left) and MLP logit differences (right) of subject and answer tokens of Mistral-7B-Instruct on Artist-Songs dataset when attending to or knocking out the subject tokens.}
   \label{fig:artist_songs-mistral-subject}
\end{figure*}

\begin{figure*}
   \centering
   \includegraphics[width=1\linewidth]{figures/critical_token_analysis/data_model_specific/omit_early_layer_figures/actor_movies/meta-llama/Meta-Llama-3-8B-Instruct/subject_logit.png}
   \caption{Token Lens logit values (left) and MLP logit differences (right) of subject and answer tokens of Llama-3-8B-Instruct on Actor-Movies dataset when attending to or knocking out the subject tokens.}
   \label{fig:actor_movies-llama-subject}
\end{figure*}

\begin{figure*}
   \centering
   \includegraphics[width=1\linewidth]{figures/critical_token_analysis/data_model_specific/omit_early_layer_figures/actor_movies/mistralai/Mistral-7B-Instruct-v0.2/subject_logit.png}
   \caption{Token Lens logit values (left) and MLP logit differences (right) of subject and answer tokens of Mistral-7B-Instruct on Actor-Movies dataset when attending to or knocking out the subject tokens.}
   \label{fig:actor_movies-mistral-subject}
\end{figure*}


\section{Are Knowledge Recall and Suppression Independent?}
Observing that LMs promote all answers while suppressing previously generated ones, another question we have is whether knowledge recall and suppression are independent. To investigate this, we analyze the behavior of individual attention heads at the last token position to determine if they perform one, both, or neither of the two subtasks.

\subsection{Methodology: Characterizing Attention Heads' Behavior}
Our methodology involves the following steps:
\begin{enumerate}
    \item Decode Attention Head Outputs: For each attention head, we decode its output at the last token position and collect the logits of the first token of $t$ for a given input, where $t \in \{s, o^{(1)}, o^{(2)}, o^{(3)}\}$.
    
    \item Calculate Layer-wise Baseline: For each layer $l$, we compute the mean $\mu_l$ and standard deviation $\sigma_l$ of attention head logits across all heads in the layer.
    
    \item Characterize Head Behavior: Let $\text{logit}(a^{(li)}_{t})$ denote the logit for the first token of $t$ from attention head $a^{(li)}$. The behavior of $a^{(li)}$ on token $t$ is classified as:\\
    
    \scalebox{0.75}{$
        \text{Behavior}(a^{(li)}_{t})=
                \begin{cases}
                    \text{Promotion},   & \text{if logit}(a^{(li)}_t) > \mu_l + \sigma_l \\
                    \text{Suppression}, & \text{if logit}(a^{(li)}_t) < \mu_l - \sigma_l \\
                    \text{None},        & \text{otherwise}
                \end{cases}
    $}
    
    \item Classify Head Function: $a^{(li)}$ is classified as performing promotion for the given input if it promotes the first token of any $t \in {s, o^{(1)}, o^{(2)}, o^{(3)}}$ and as performing suppression if it suppresses any such token.

    \item Aggregate Results: We average the percentage of times each attention head is identified as performing promotion or suppression.
\end{enumerate}

Then, by plotting the promotion rate against the suppression rate for all heads, we examine how LMs divide the labor among heads for knowledge recall and suppression.


\subsection{Knowledge Recall and Suppression May Not be Independent}
\begin{figure*}
    \centering
    \includegraphics[width=1\linewidth]{figures/head_promotion_vs_suppression_rate.png}
    \caption{Promotion rate versus suppression rate of all attention heads across three answer steps macro-averaged across all models and datasets. The promotion rate and suppression rate positively correlate with each other, suggesting that answer promotion and suppression may not be independent of each other.}
    \label{fig:head_promotion_vs_suppression_rate}
\end{figure*}

As can be observed in \Cref{fig:head_promotion_vs_suppression_rate}, the promotion rate and suppression rate of attention heads consistently correlate with each other across all three answers steps, with the majority of the data points concentrated in the bottom-left region of the plots. This finding shows that most attention heads contribute moderately to the two subtasks and are responsible for both token promotion and suppression, suggesting that knowledge recall and suppression may not be independent. 


