\section{Are Knowledge Recall and Suppression Independent?}
Observing that LMs promote all answers while suppressing previously generated ones, another question we have is whether knowledge recall and suppression are independent. To investigate this, we analyze the behavior of individual attention heads at the last token position to determine if they perform one, both, or neither of the two subtasks.

\subsection{Methodology: Characterizing Attention Heads' Behavior}
Our methodology involves the following steps:
\begin{enumerate}
    \item Decode Attention Head Outputs: For each attention head, we decode its output at the last token position and collect the logits of the first token of $t$ for a given input, where $t \in \{s, o^{(1)}, o^{(2)}, o^{(3)}\}$.
    
    \item Calculate Layer-wise Baseline: For each layer $l$, we compute the mean $\mu_l$ and standard deviation $\sigma_l$ of attention head logits across all heads in the layer.
    
    \item Characterize Head Behavior: Let $\text{logit}(a^{(li)}_{t})$ denote the logit for the first token of $t$ from attention head $a^{(li)}$. The behavior of $a^{(li)}$ on token $t$ is classified as:\\
    
    \scalebox{0.75}{$
        \text{Behavior}(a^{(li)}_{t})=
                \begin{cases}
                    \text{Promotion},   & \text{if logit}(a^{(li)}_t) > \mu_l + \sigma_l \\
                    \text{Suppression}, & \text{if logit}(a^{(li)}_t) < \mu_l - \sigma_l \\
                    \text{None},        & \text{otherwise}
                \end{cases}
    $}
    
    \item Classify Head Function: $a^{(li)}$ is classified as performing promotion for the given input if it promotes the first token of any $t \in {s, o^{(1)}, o^{(2)}, o^{(3)}}$ and as performing suppression if it suppresses any such token.

    \item Aggregate Results: We average the percentage of times each attention head is identified as performing promotion or suppression.
\end{enumerate}

Then, by plotting the promotion rate against the suppression rate for all heads, we examine how LMs divide the labor among heads for knowledge recall and suppression.


\subsection{Knowledge Recall and Suppression May Not be Independent}
\begin{figure*}
    \centering
    \includegraphics[width=1\linewidth]{figures/head_promotion_vs_suppression_rate.png}
    \caption{Promotion rate versus suppression rate of all attention heads across three answer steps macro-averaged across all models and datasets. The promotion rate and suppression rate positively correlate with each other, suggesting that answer promotion and suppression may not be independent of each other.}
    \label{fig:head_promotion_vs_suppression_rate}
\end{figure*}

As can be observed in \Cref{fig:head_promotion_vs_suppression_rate}, the promotion rate and suppression rate of attention heads consistently correlate with each other across all three answers steps, with the majority of the data points concentrated in the bottom-left region of the plots. This finding shows that most attention heads contribute moderately to the two subtasks and are responsible for both token promotion and suppression, suggesting that knowledge recall and suppression may not be independent. 

