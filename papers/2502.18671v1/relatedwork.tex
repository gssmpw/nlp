\section{Literature review}
The literature review focuses on WSN data synchronization using Node-MCU memory for precision agriculture applications. It explores existing research and studies related to the synchronization of data in wireless sensor networks (WSNs) specifically using Node-MCU memory as a synchronization mechanism. The review aims to depict the existing state of knowledge, research gaps, and potential options in leveraging Node-MCU memory for achieving accurate and reliable data synchronization in the context of precision agriculture applications. By analyzing the previous researches, this analysis provides a comprehensive understanding of the advantages, limitations, and future directions of utilizing Node-MCU memory for WSN data synchronization in precision agriculture.

Dependable tools for agricultural management are provided by wireless sensor networks. A cost-effective and energy-efficient wireless sensing system tailored for agricultural settings was introduced \cite{tagarakis2021low}. Geared towards small to medium-sized fields, this system incorporates a pseudo device functioning as a Coordinator. This Coordinator processes and transmits data packets to the cloud via a 4G cellular network. Powered by solar energy, the system functions autonomously, exhibiting commendable performance in terms of reliability, communication range, energy self-sufficiency, and uninterrupted operation. This technology holds the potential to effectively monitor environmental, soil, and crop parameters.

New technological development has led to the advancement of precision agriculture systems, which offer significant benefits in cost reduction and energy conservation for the agricultural sector. In a study \cite{rodriguez2020autonomous}, a prototype precision agriculture system tailored for small and medium-sized plantations was introduced. This system integrates Wireless Sensor Networks (WSNs), sensor devices, Internet of Things (IoT) technologies, and data analysis to improve agricultural practices. The proposed system focuses on optimizing energy management and maintaining affordability while enhancing the efficiency of farming operations. It collects environmental data through a network of sensors, which is then analyzed to make informed decisions about irrigation. The system activates the watering system based on real-time data, ensuring that water is used efficiently and only when needed. The study provides a comprehensive analysis of the agricultural monitoring system, detailing its design, implementation, and the outcomes achieved. By combining these technologies, the system aims to offer practical solutions that enhance agricultural productivity while minimizing resource usage and operational costs.

Accurate time synchronization is crucial for various functions like data fusion in Wireless sensor network mainly in the decentralized systems, TDMA schedules, and synchronized sleep periods. Previous time synchronization techniques were not initially designed for wireless sensor networks and thus need extension or redesign. Time synchronization technique was designed for Wireless sensor networks \cite{sichitiu2003simple}. Minimal complexity was suggested by proposed method to achieve high accuracy in terms of network Bandwidth, storage and processing. It is highly suitable for sensor networks, offering tight, deterministic bounds on both offsets and clock drifts. A technique was proposed to synchronize the whole network for the preparation of data fusion with the execution of proposed methodology evaluated based on real world experiment of Wireless ad-hock network in the composition of data fusion.

Full time stamp is required in Wireless sensor networks in previous existing clock synchronization approaches to accurately determine clock skew and offset. However, in many scenarios aimed at reducing energy consumption, minimizing communication overhead, or enhancing security, timestamps are not swapped during the coordination process. This leads to nodes lacking full time information, presenting a significant challenge in jointly estimating clock skew and offset with limited observations. A Partial Timestamp synchronization scheme was proposed in which they used only local timestamps and the relationships obtained from basic synchronization protocol as addressed in \cite{wang2023clock}. When stochastic delays followed a Gaussian distribution based on PTS, a 
joint Maximum Likelihood Estimator was established to estimate clock skew and offset. Additionally, they present a Best Linear Unbiased 
Estimator for PTS with exponential random delays, capable of simultaneously estimating clock skew and offset. Finally, simulations demonstrate the effectiveness of the proposed algorithms.

Experimental research often requires recording rapid processes that are difficult to localize precisely in space and time, necessitating a network of recording devices to capture critical events and gather sufficient data. Synchronizing data from these spatially distributed devices is crucial, as discrepancies in timekeeping can distort the understanding of the studied process. A solution was proposed by synchronizing the local clocks of wireless nodes with an accuracy of  \SI{12}{\micro\second}/second using low-cost commercial components \cite{zuev2023wireless}. This method, tested with ESP family microprocessors, offers a practical and cost-effective approach. It has been applied to study high-velocity injuries using artificial body simulators and is versatile enough for measuring vibrations in electrical machines, engines, and structural health monitoring, demonstrating its broad utility and effectiveness in various fields.

The reason why localization within wireless sensor networks has received quite a lot of attention from analysts is because it has many real time applications that help identify the origin of events. Wireless sensors also transmit multimedia data such as audio, video and images in addition to scalar data. These WSNs are therefore known as WMSN. Consequently, various techniques for sensing node identification in wireless sensor network have been developed by researchers through simulating real-time situations or using simulations. In the article \cite{wajgi2024localization}, clustering-based localization strategies for WSN (Wireless Sensor Network) and Wireless Multimedia Sensor Networks) was discussed in depth. Clustering is one of the ways used in wireless sensor networks and wireless multimedia sensor networks to provide distributed computing and parallel processing with each group operating independently with its own small subgroups. Clustering techniques would be highly useful to improve traditional localization procedures’ performance. By dividing the network into partitions that can operate independently thus reducing communication cost, clustering reduces communication overheads significantly. Additionally, distributed computing improves throughput of a network through clustering. The paper also raises future concerns and issues so as to know how new researchers should go about future studies on WSN.

Precision agriculture is increasingly essential for feeding the growing global population, minimizing environmental impact, and optimizing scarce resources. However, the high cost of technology adoption poses a significant challenge. This paper addresses this issue by introducing a low-cost wireless sensor network comprising soil moisture sensors. These sensors, built with coils wound on plastic pipes, enable moisture measurement at different depths. Through mutual induction, the sensors monitor soil water content, aiding farmers in optimizing irrigation processes. Several prototypes underwent testing, and the most effective among them showcased winding ratios of 1:2, 15, and 30 spires, functioning at a frequency of 93 kHz \cite{lloret2021wireless}. A specialized communication protocol was devised to bolster the overall performance of the system. A experiment was performed on citrus tree area, in which wireless network was assessed and some of the evaluations including the coverage and received signal strength indication to measure potential signal losses just because of vegetation.

In another study, it was presented that wireless sensor networks are widely used in fields like engineering, science, agriculture, surveillance, military applications, and smart cars \cite{thakur2019applicability}. Precision agriculture extensively adopts WSNs to measure environmental parameters, enhance crop quality and quantity, and conserve natural resources. This review identifies WSN technologies for precision agriculture, analyzes their impact, and explores various environmental parameters, crops, communication technologies, and sensors used. Several research questions are addressed to understand the influence of WSNs in agriculture.

Time synchronization poses challenges in wireless sensor networks, due to clock deviations and the need to associate sensor measurements with reference clock time \cite{skiadopoulos2019synchronization}. In IoT environments, limited capabilities, energy constraints, robustness, and 
extreme conditions (such as deployment in soil) prompt researchers to explore synchronization schemes that relax requirements. This article presents a lightweight synchronization algorithm tailored for wireless sensor networks. The algorithm primarily concentrates on synchronizing measurements on a per-hop basis within the data packets sent to the sink node. The core objective is to achieve synchronization of data rather than synchronization of node clocks. This approach minimizes unnecessary overhead by eliminating the requirement for supplementary messages and re-synchronization intervals. The analysis of the proposed algorithm illustrates its reliance on residual time, distance, and average skew deviation. Through simulations, the algorithm's effectiveness is thoroughly exhibited, and analytical results are validated. A comparison with a traditional synchronization algorithm shows that the proposed approach is suitable for demanding IoT systems.

Efficient data collection is vital for precision agriculture and smart agriculture, which rely on wireless sensor networks (WSNs). However, agricultural WSNs face challenges like multitasking, data quality, and latency. To address these issues an innovative solution using edge computing-enabled Wireless sensor networks was proposed \cite{li2020edge}. They introduced a novel data collection framework that integrates Wireless sensor networks and edge computing, considering multiple sensors and tasks. Utilizing a double selection strategy, the edge computing server identifies the optimal node and sensor network by considering task demands and sensor interconnections. Furthermore, an algorithm for data collection is developed to enhance data quality through predefined values optimization. Through simulations, the proposed strategy is evaluated, showing superior performance compared to traditional methods across various metrics.

Within precision agriculture, Wireless Sensor Networks (WSNs) play a pivotal role by enabling the effective utilization of natural resources through real-time data collection on farm premises. This data empowers farmers with the knowledge needed to make well-informed and intelligent choices, ultimately leading to improved crop yields and enhanced profitability. Another work introduced an innovative method for monitoring agricultural lands through the utilization of a Wireless Sensor and Actuator Network (WSAN) \cite{tiglao2020agrinex}. The Agrinex system employs a mesh-like arrangement of in-field nodes that function both as sensors, measuring parameters like soil moisture, temperature, and humidity, and as actuators, granting control over drip irrigation valves to enhance water management efficiency. The network is designed to dynamically self-reorganize when changes occur. The Agrinex system demonstrates great potential as a WSAN framework for various agricultural applications.

The data collection though Wireless sensor is the basic source of energy dissipation in a low-cost wireless sensor network. Motivated by this, compressive data collection has become an important tool for minimizing energy consumption. Compressive data collection. Compressive sensing was used for minimizing the amount of transmitted data to reduce the energy dissipation. Keeping in view, physical interference model was used. They are jointly in a disconnected network and it investigates the compressive sensing-based data collection and link scheduling. With the network disconnected, mobile collets have been developed to enable data collection within the network. Compressive sensing is when active sensors are triggered to send compressed data that mobile collec tors then recover against all the sensed data. Another study targeted two metrics, end-to-end latency and number of transmissions for data collection with this work \cite{ghosh2023energy}. Since the joint problem is NP-Hard, researchers have tried to find heuristic-based solutions for tree construction as well as link scheduling. The feasibility of the proposed algorithm is validated by simulation results in comparison with some existing algorithms.

Over the past few decades, Wireless Sensor Network (WSN) technology has gained extensive acceptance across diverse scientific domains, enabling accurate monitoring of climate phenomena such as air pollution, as well as catastrophic incidents like landslides. Its applications in agriculture for field monitoring have been particularly notable. WSN represents an emerging technology that has proven to be a valuable tool in exploring the intricate world behind environmental phenomena using 
small sensor nodes. "Expert Advisory System" developed to enhance farmers' productivity, save time, and improve crop efficiency \cite{muzafarov2019wireless}. The system utilizes WSN for real-time monitoring of crop fields and delivers essential information to farmers through the Internet. Based on the received information, farmers can take necessary remedial actions. The study also includes the simulation of WSN using the Contiki Simulator tool and considers queuing models for WSN in the research.

Another technique used to reduce the mass and design complexity of cable harnesses, the space industry seeks to deploy reliable Wireless Sensor Networks (WSNs) in spacecraft and launcher systems \cite{lubken2024adaptive}. Low-latency sensor data acquisition in these systems depends on synchronized local clocks across network nodes, making time synchronization essential. Controlling the local clock on a node between synchronization points provided by a time distribution protocol is challenging, especially in harsh environmental conditions like temperature fluctuations and electromagnetic interference, which cause rapid clock drift. Existing methods struggle to mitigate these large drift rates and cope with varying drift amounts. This article introduces the Adaptive Local Clock Control (ALCC) algorithm, which minimizes deviation from a reference clock under extreme conditions, outperforming existing methods by nearly an order of magnitude. A formal model of Adaptive Local Clock Control is compared to current solutions in simulations replicating real-world scenarios, and the algorithm is verified on a hardware test platform, demonstrating its effectiveness in maintaining synchronized local clocks under harsh conditions.

In the Industrial Internet of Things (IoT), energy efficiency is critical for operational longevity, and traditional approaches like flooding for time synchronization often waste energy through redundant message transmissions. Intelligent neighbor-knowledge synchronization (INKS) method was introduced to address this issue by leveraging each node's knowledge of its neighbors to optimize the synchronization process, thereby reducing the total number of synchronization messages and conserving energy \cite{jia2024low}. Implemented and evaluated using real wireless sensor networks with various configurations, intelligent neighbor-knowledge synchronization demonstrates superior performance to existing techniques such as rapid flooding multiple one-way broadcast time synchronization (RMTS). Simulations on large-scale networks, particularly with a four-way grid topology, reveal that intelligent neighbor-knowledge synchronization reduces transmitted messages by approximately 72\% compared to RMTS and matches the efficiency of scheduling-based low-energy synchronization for IoT, highlighting its potential for enhancing energy efficiency in IoT applications.

The elucidation of agricultural automation presented \cite{shafi2019precision} via IoT has the potential to revolutionize the sector, imbuing it with dynamism and efficiency. At the heart of this transformation lie Precision Agriculture and Wireless Sensor Networks. The study in question significantly contributes by meticulously examining how sensor networks and remote sensing technologies synergistically enable real-time surveillance of crop well-being and environmental conditions within agriculture. Furthermore, it proposes an IoT-driven solution that harmonizes wireless sensors and remote sensing platforms to accomplish effective crop health monitoring. Additionally, the research incorporates a case study demonstration, addresses the existing challenges, and outlines the future prospects within this domain.

A cost-effective and energy-efficient solution is proposed for information monitoring in large-scale agricultural farms. The system utilized sensor-based soil properties measurement to automate farm operations and improve results compared to manual methods \cite{saqib2020low}. The paper presented an innovative low-power, long-range communication module with a tree-based mechanism for efficient agricultural monitoring. Sensor nodes equipped with essential components enable successful communication up to 750 meters with minimal packet loss, making it a valuable reference model for large-scale agricultural applications.

In another research, authors proposed an innovative architecture designed to enhance security protocols within wireless sensor networks (WSNs) and Internet of Things (IoT) frameworks \cite{hasan2023efficient}. Their study addresses 
several critical aspects, including energy consumption, mobility, information transmission, quality of service (QoS), and overall security. By 
focusing on the data link layer of WSN frameworks, the research tackles existing challenges, limitations, and protocol issues that are often overlooked in the broader literature on WSNs. The authors highlight that, despite extensive research on WSNs, there is limited exploration of how the data link layer impacts network performance. Their work emphasizes practical solutions that incorporate machine learning to improve various aspects of WSN performance. Key challenges identified in their study include managing flow control, ensuring QoS, maintaining security, and optimizing overall performance in WSN applications. The proposed architecture aims to address these challenges by integrating advanced security protocols and optimizing network parameters, thereby enhancing the efficiency and robustness of WSN and IoT systems.

In another study, domain-agnostic monitoring and control framework (MCF) was introduced for IoT systems which is built on building blocks that can be implemented by the five-layer architecture of IoT system and includes three sub-domain subsystems: Monitoring, Controlling, Computing \cite{senoo2023monitoring}. They have seenn it successful in smart agriculture using commodity sensors and code, and an affordable technique. Our MCF offers scalable, reusable, and interoperable solutions, costing up to 20 times less than commercial alternatives. It operates efficiently on rechargeable batteries and solar power, with stable data exchange and minimal power consumption. The framework supports reliable data from multiple sensors, ensuring accurate and consistent performance over time.

Extensive blueprint and execution of an economical Wireless Sensor Network (WSN) system designed specifically for intelligent agriculture, with a focus on synchronization \cite{zervopoulos2020wireless}. The study's objective was to create a cost-effective and proficient answer for real-time surveillance within agricultural environments. Through utilizing the sink node's clock as a reference point, a straightforward yet impactful synchronization mechanism was put forth. This approach facilitated the harmonious synchronization of data measurements across the entirety of the WSN. The system's performance was rigorously tested in an olive grove, demonstrating successful measurement synchronization even in the presence of slight time differences resulting from transmission delays. The findings underscore the system's capability to provide crucial environmental data in precision agriculture applications, paving the way for more informed decision-making and resource optimization in farming practices. The study's cost-effectiveness and reliability make it a valuable contribution to the advancement of smart agriculture technologies.
   
\subsection*{Research Gap}
Achieving accurate data synchronization and efficient storage in 
Wireless Sensor Networks for precision agriculture applications involves overcoming several complex challenges. Timestamp-based solutions often lead to issues such as record duplication and storage inefficiencies, which compromise the effectiveness of data analysis and decision-making. Advanced synchronization protocols are essential to ensure precise time alignment of sensor data, thereby eliminating redundant records and maintaining data integrity. Additionally, optimized data storage solutions are necessary to handle the large volumes of high-frequency data generated by sensors, which can otherwise lead to inefficiencies and bottlenecks. Internet congestion exacerbates the problem by causing packet loss and incomplete data transmission, further complicating reliable data management. To address these issues, it is crucial to develop and implement sophisticated techniques for data synchronization, duplication prevention, and robust transmission protocols, ensuring that data is accurate, complete, and efficiently managed for improved agricultural productivity and resource optimization.

\subsection*{Problem Statement}
Accurate data synchronization among sensor nodes in Wireless Sensor Networks (WSNs) is crucial for precision agriculture applications, as reliable data storage is now highly demanded for local monitoring stations and online live servers through the internet. In precision agriculture, accurate and synchronized data collection is crucial for effective analysis and decision-making. Current solutions often save records using timestamps, which can lead to record duplication and storage wastage, impacting analysis and decision-making processes. Enhanced synchronization protocols and data duplication removing techniques are needed to ensure accurate time alignment of data from different sensors and to identify and remove redundant records. Additionally, optimized data storage solutions are essential to handle high-frequency data efficiently, thereby improving data quality, analysis accuracy, and decision-making in precision agriculture.