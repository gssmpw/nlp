\documentclass[journal]{vgtc}
\vgtccategory{Research}
\vgtcpapertype{Technique}

\usepackage[svgnames]{xcolor}%
\usepackage{graphicx}%
\usepackage{booktabs}%
\usepackage{multirow}%
\usepackage{amsmath,amssymb,amsfonts}%
\usepackage{amsthm}%
\usepackage{mathrsfs}%
\usepackage[title]{appendix}%
\usepackage{textcomp}%
\usepackage{manyfoot}%
\usepackage{algorithm}%
\usepackage{algorithmicx}%
\usepackage{algpseudocode}%
\usepackage{listings}%
%%%%

\usepackage{caption}
\usepackage{subcaption}
\usepackage{tikz}
\usepackage{pgfplots}
\usepackage{array}
\usepackage{amssymb}

%%%%

\newcommand{\shortcite}[1]{\cite{{#1}}}
\newcommand{\Skip}[1]{}
\newcommand\tab[1][1cm]{\hspace*{#1}}

\newcommand{\DS}[1]{
   \textcolor{blue}{\bfseries{DS: {#1}}}
}

\newcommand{\YW}[1]{
   \textcolor{red}{\bfseries{YW: {#1}}}
}

%\newcommand{\revision}[1]{\textcolor{red}{#1}}
\newcommand{\revision}[1]{{#1}}

\newcommand{\TODO}[1]{
   \textcolor{red}{\bfseries{TODO: {#1}}}
}

\newcommand{\ToCheck}[1]{{#1}}
%\newcommand{\ToCheck}[1]{\textcolor{red}{#1}}

\newcommand{\XX}{\textcolor{red}{XX}
}

\newcommand{\addRef}{\textcolor{red}{[ref]}
}

\aboverulesep=0ex  % added
\belowrulesep=0ex  % added

%%%%


\raggedbottom

\begin{document}


\title{RTPD: Penetration Depth calculation using Hardware accelerated Ray-Tracing}

\author{%
  \authororcid{YoungWoo Kim}{0000-0003-3341-1714},
  \authororcid{Sungmin Kwon}{0000-0000-0000-0000}, and
  \authororcid{Duksu Kim}{0000-0002-9075-3983}
}

\authorfooter{
  %% insert punctuation at end of each item
  \item
  	YoungWoo Kim is with Korea University of Technology and Education(KOREATECH).
  	E-mail: aister9@koreatech.ac.kr.
  \item
  	Sungmin Kwon is with Korea University of Technology and Education(KOREATECH).
  	E-mail: 00kwonsm@koreatech.ac.kr.

  \item Duksu Kim is with Korea University of Technology and Education(KOREATECH).
  	E-mail: bluekdct@gmail.com.
}

%%==================================%%
%% Sample for unstructured abstract %%
%%==================================%%

\begin{abstract}
  In this work, we present a novel technique for GPU-accelerated Boolean satisfiability (SAT) sampling. Unlike conventional sampling algorithms that directly operate on conjunctive normal form (CNF), our method transforms the logical constraints of SAT problems by factoring their CNF representations into simplified multi-level, multi-output Boolean functions. It then leverages gradient-based optimization to guide the search for a diverse set of valid solutions. Our method operates directly on the circuit structure of refactored SAT instances, reinterpreting the SAT problem as a supervised multi-output regression task. This differentiable technique enables independent bit-wise operations on each tensor element, allowing parallel execution of learning processes. As a result, we achieve GPU-accelerated sampling with significant runtime improvements ranging from $33.6\times$ to $523.6\times$ over state-of-the-art heuristic samplers. We demonstrate the superior performance of our sampling method through an extensive evaluation on $60$ instances from a public domain benchmark suite utilized in previous studies. 


  
  % Generating a wide range of diverse solutions to logical constraints is crucial in software and hardware testing, verification, and synthesis. These solutions can serve as inputs to test specific functionalities of a software program or as random stimuli in hardware modules. In software verification, techniques like fuzz testing and symbolic execution use this approach to identify bugs and vulnerabilities. In hardware verification, stimulus generation is particularly vital, forming the basis of constrained-random verification. While generating multiple solutions improves coverage and increases the chances of finding bugs, high-throughput sampling remains challenging, especially with complex constraints and refined coverage criteria. In this work, we present a novel technique that enables GPU-accelerated sampling, resulting in high-throughput generation of satisfying solutions to Boolean satisfiability (SAT) problems. Unlike conventional sampling algorithms that directly operate on conjunctive normal form (CNF), our method refines the logical constraints of SAT problems by transforming their CNF into simplified multi-level Boolean expressions. It then leverages gradient-based optimization to guide the search for a diverse set of valid solutions.
  % Our method specifically takes advantage of the circuit structure of refined SAT instances by using GD to learn valid solutions, reinterpreting the SAT problem as a supervised multi-output regression task. This differentiable technique enables independent bit-wise operations on each tensor element, allowing parallel execution of learning processes. As a result, we achieve GPU-accelerated sampling with significant runtime improvements ranging from $10\times$ to $1000\times$ over state-of-the-art heuristic samplers. Specifically, we demonstrate the superior performance of our sampling method through an extensive evaluation on $60$ instances from a public domain benchmark suite utilized in previous studies.

\end{abstract}

\begin{IEEEkeywords}
Boolean Satisfiability, Gradient Descent, Multi-level Circuits, Verification, and Testing.
\end{IEEEkeywords}

\keywords{GPUs and Multi-core Architectures, Special Purpose Hardware, Geometry-based Techniques, Proximity queries}


\teaser{
  \centering
  \includegraphics[width=\linewidth, alt={Temporal image for teaser.}]{Image/teaser.pdf}
    \caption{
        Comparison of processing times for penetration depth calculation using the CPU-based implementation based on the state-of-the-art method ($CPU_{Tang}$~\cite{SIG09HIST}), CUDA-based GPU implementation ($GPU_{CUDA}$), and our ray-tracing core-based method ($RTPD$).
        Left: Bunny benchmark. Right: Lucy benchmark.
        The black lines represent the ground truth, while the white lines indicate the results from our method.
        Our method achieved speedups of up to 37.66x and 5.33x compared to $CPU_{Tang}$ and $GPU_{CUDA}$, respectively.
  }
  \label{fig:teaser}
}

\date{February 2025}

%%\pacs[JEL Classification]{D8, H51}

%%\pacs[MSC Classification]{35A01, 65L10, 65L12, 65L20, 65L70}

\maketitle

\section{Introduction}\label{sec:Intro} 


Novel view synthesis offers a fundamental approach to visualizing complex scenes by generating new perspectives from existing imagery. 
This has many potential applications, including virtual reality, movie production and architectural visualization \cite{Tewari2022NeuRendSTAR}. 
An emerging alternative to the common RGB sensors are event cameras, which are  
 bio-inspired visual sensors recording events, i.e.~asynchronous per-pixel signals of changes in brightness or color intensity. 

Event streams have very high temporal resolution and are inherently sparse, as they only happen when changes in the scene are observed. 
Due to their working principle, event cameras bring several advantages, especially in challenging cases: they excel at handling high-speed motions 
and have a substantially higher dynamic range of the supported signal measurements than conventional RGB cameras. 
Moreover, they have lower power consumption and require varied storage volumes for captured data that are often smaller than those required for synchronous RGB cameras \cite{Millerdurai_3DV2024, Gallego2022}. 

The ability to handle high-speed motions is crucial in static scenes as well,  particularly with handheld moving cameras, as it helps avoid the common problem of motion blur. It is, therefore, not surprising that event-based novel view synthesis has gained attention, although color values are not directly observed.
Notably, because of the substantial difference between the formats, RGB- and event-based approaches require fundamentally different design choices. %

The first solutions to event-based novel view synthesis introduced in the literature demonstrate promising results \cite{eventnerf, enerf} and outperform non-event-based alternatives for novel view synthesis in many challenging scenarios. 
Among them, EventNeRF \cite{eventnerf} enables novel-view synthesis in the RGB space by assuming events associated with three color channels as inputs. 
Due to its NeRF-based architecture \cite{nerf}, it can handle single objects with complete observations from roughly equal distances to the camera. 
It furthermore has limitations in training and rendering speed: 
the MLP used to represent the scene requires long training time and can only handle very limited scene extents or otherwise rendering quality will deteriorate. 
Hence, the quality of synthesized novel views will degrade for larger scenes. %

We present Event-3DGS (E-3DGS), i.e.,~a new method for novel-view synthesis from event streams using 3D Gaussians~\cite{3dgs} 
demonstrating fast reconstruction and rendering as well as handling of unbounded scenes. 
The technical contributions of this paper are as follows: 
\begin{itemize}
\item With E-3DGS, we introduce the first approach for novel view synthesis from a color event camera that combines 3D Gaussians with event-based supervision. 
\item We present frustum-based initialization, adaptive event windows, isotropic 3D Gaussian regularization and 3D camera pose refinement, and demonstrate that high-quality results can be obtained. %

\item Finally, we introduce new synthetic and real event datasets for large scenes to the community to study novel view synthesis in this new problem setting. 
\end{itemize}
Our experiments demonstrate systematically superior results compared to EventNeRF \cite{eventnerf} and other baselines. 
The source code and dataset of E-3DGS are released\footnote{\url{https://4dqv.mpi-inf.mpg.de/E3DGS/}}. 





\section{Background and Related Work}\label{sec:related}

\paragraph{\textbf{Privacy of Human-Centered Systems}}
Ensuring privacy in human-centric ML-based systems presents inherent conflicts among service utility, cost, and personal and institutional privacy~\cite{sztipanovits2019science}. Without appropriate incentives for societal information sharing, we may face decision-making policies that are either overly restrictive or that compromise private information, leading to adverse selection~\cite{jin2016enabling}. Such compromises can result in privacy violations, exacerbating societal concerns regarding the impact of emerging technology trends in human-centric systems~\cite{mulligan2016privacy,fox2021exploring,goldfarb2012shifts}. Consequently, several studies have aimed to establish privacy guarantees that allow auditing and quantifying compromises to make these systems more acceptable~\cite{jagielski2020auditing, raji2020saving}. ML models in decision-making systems have also been shown to leak significant amounts of private information that requires auditing platforms~\cite{hamon2022bridging}. Various studies focused on privacy-preserving machine learning techniques targeting decision-making systems~\cite{abadi2016deep, cummings2019compatibility, taherisadr2023adaparl, taherisadr2024hilt}. Recognizing that perfect privacy is often unattainable, this paper examines privacy from an equity perspective. We investigate how to ensure a fair distribution of harm when privacy leaks occur, addressing the technical challenges alongside the ethical imperatives of equitable privacy protection.


\paragraph{\textbf{\acf{fl}}}
\ac{fl} is an approach in machine learning that enables the collaborative training of models across multiple devices or institutions without requiring data to be centralized. This decentralized setup is particularly beneficial in fields where data-sharing restrictions are enforced by privacy regulations, such as healthcare and finance. \ac{fl} allows organizations to derive insights from data distributed across various locations while adhering to legal constraints, including the General Data Protection Regulation (GDPR) \cite{BG_Survey2,BG_Survey1}.

One of the most widely adopted methods in \ac{fl} is \ac{fedavg}, which operates through iterative rounds of communication between a central server and participating clients to collaboratively train a shared model. During each communication round, the server sends the current global model to each client, which uses their locally stored data to perform optimization steps. These optimized models are subsequently sent back to the server, where they are aggregated to update the global model. The process repeats until the model converges. Known for its simplicity and effectiveness, \ac{fedavg} serves as the primary technique for coordinating model updates across distributed clients in our work. Additionally, we specifically employ horizontal federated learning, where data is distributed across entities with similar feature spaces but distinct user groups \cite{BG_HorizontalFL}.

\paragraph{\textbf{Privacy Risks in \ac{fl}}}
Privacy risks are a critical concern in \ac{fl}, as collaborative training on decentralized data can inadvertently expose sensitive information. A primary threat is the \ac{mia}, where adversaries determine whether specific data records were part of the model's training set \cite{shokri2017membership,BG_MIA}. Researchers have since demonstrated \ac{mia}'s effectiveness across various machine learning models, including \ac{fl}, showing, for example, that adversaries can infer if a specific location profile contributed to an FL model \cite{BG_MIA_1,BG_MIA_2}. However, while \ac{mia} identifies training members, it does not reveal the client that contributed the data. \ac{sia}, introduced in \cite{BG_SIA_2}, extends \ac{mia} by identifying which client owns a training record, thus posing significant security risks by exposing client-specific information in \ac{fl} settings.

The \ac{noniid} nature of data in federated learning presents additional privacy challenges, as variations in data distributions across clients heighten the risk of privacy leakage. When data distributions differ widely among clients, individual model updates become more distinguishable, potentially allowing attackers to infer sensitive information \cite{BG_NON_IID}. This distinctiveness in updates can make federated models more susceptible to inference attacks, such as \ac{mia} and \ac{sia}, as malicious actors may exploit these distributional differences to trace updates back to specific clients. This vulnerability is especially relevant in our work, as we use the \ac{har} dataset, which is inherently \ac{noniid} across clients, thus posing an increased risk for privacy leakage.




\paragraph{\textbf{Fairness in \ac{fl}}}
Fairness in \ac{fl} is crucial due to the varied data distributions among clients, which can lead to biased outcomes favoring certain groups \cite{BG_Fairness_2}. Achieving fairness involves balancing the global model's benefits across clients despite the decentralized nature of the data. Approaches include group fairness, ensuring performance equity across client groups, and performance distribution fairness, which focuses on fair accuracy distribution~\cite{selialia2024mitigating}. Additional types are selection fairness (equitable client participation), contribution fairness (rewards based on contributions), and expectation fairness (aligning performance with client expectations) \cite{BG_Fairness}. Achieving fairness in \ac{fl} across these various dimensions remains challenging due to the inherent heterogeneity of client data and environments. In response to this heterogeneity, personalization has emerged as a strategy to tailor models to individual clients, enhancing local performance~\cite{BG_Personalization,BG_Personalization_2, BG_FairnessPrivacy}.   

When considering fairness in FL, it is crucial to address the interplay with privacy. Specifically, ensuring an equitable distribution of privacy risks across clients is paramount, preventing any group from being disproportionately vulnerable to privacy leakage, particularly under attacks such as source inference attacks (SIAs).



\section{Overview}

\revision{In this section, we first explain the foundational concept of Hausdorff distance-based penetration depth algorithms, which are essential for understanding our method (Sec.~\ref{sec:preliminary}).
We then provide a brief overview of our proposed RT-based penetration depth algorithm (Sec.~\ref{subsec:algo_overview}).}



\section{Preliminaries }
\label{sec:Preliminaries}

% Before we introduce our method, we first overview the important basics of 3D dynamic human modeling with Gaussian splatting. Then, we discuss the diffusion-based 3d generation techniques, and how they can be applied to human modeling.
% \ZY{I stopp here. TBC.}
% \subsection{Dynamic human modeling with Gaussian splatting}
\subsection{3D Gaussian Splatting}
3D Gaussian splatting~\cite{kerbl3Dgaussians} is an explicit scene representation that allows high-quality real-time rendering. The given scene is represented by a set of static 3D Gaussians, which are parameterized as follows: Gaussian center $x\in {\mathbb{R}^3}$, color $c\in {\mathbb{R}^3}$, opacity $\alpha\in {\mathbb{R}}$, spatial rotation in the form of quaternion $q\in {\mathbb{R}^4}$, and scaling factor $s\in {\mathbb{R}^3}$. Given these properties, the rendering process is represented as:
\begin{equation}
  I = Splatting(x, c, s, \alpha, q, r),
  \label{eq:splattingGA}
\end{equation}
where $I$ is the rendered image, $r$ is a set of query rays crossing the scene, and $Splatting(\cdot)$ is a differentiable rendering process. We refer readers to Kerbl et al.'s paper~\cite{kerbl3Dgaussians} for the details of Gaussian splatting. 



% \ZY{I would suggest move this part to the method part.}
% GaissianAvatar is a dynamic human generation model based on Gaussian splitting. Given a sequence of RGB images, this method utilizes fitted SMPLs and sampled points on its surface to obtain a pose-dependent feature map by a pose encoder. The pose-dependent features and a geometry feature are fed in a Gaussian decoder, which is employed to establish a functional mapping from the underlying geometry of the human form to diverse attributes of 3D Gaussians on the canonical surfaces. The parameter prediction process is articulated as follows:
% \begin{equation}
%   (\Delta x,c,s)=G_{\theta}(S+P),
%   \label{eq:gaussiandecoder}
% \end{equation}
%  where $G_{\theta}$ represents the Gaussian decoder, and $(S+P)$ is the multiplication of geometry feature S and pose feature P. Instead of optimizing all attributes of Gaussian, this decoder predicts 3D positional offset $\Delta{x} \in {\mathbb{R}^3}$, color $c\in\mathbb{R}^3$, and 3D scaling factor $ s\in\mathbb{R}^3$. To enhance geometry reconstruction accuracy, the opacity $\alpha$ and 3D rotation $q$ are set to fixed values of $1$ and $(1,0,0,0)$ respectively.
 
%  To render the canonical avatar in observation space, we seamlessly combine the Linear Blend Skinning function with the Gaussian Splatting~\cite{kerbl3Dgaussians} rendering process: 
% \begin{equation}
%   I_{\theta}=Splatting(x_o,Q,d),
%   \label{eq:splatting}
% \end{equation}
% \begin{equation}
%   x_o = T_{lbs}(x_c,p,w),
%   \label{eq:LBS}
% \end{equation}
% where $I_{\theta}$ represents the final rendered image, and the canonical Gaussian position $x_c$ is the sum of the initial position $x$ and the predicted offset $\Delta x$. The LBS function $T_{lbs}$ applies the SMPL skeleton pose $p$ and blending weights $w$ to deform $x_c$ into observation space as $x_o$. $Q$ denotes the remaining attributes of the Gaussians. With the rendering process, they can now reposition these canonical 3D Gaussians into the observation space.



\subsection{Score Distillation Sampling}
Score Distillation Sampling (SDS)~\cite{poole2022dreamfusion} builds a bridge between diffusion models and 3D representations. In SDS, the noised input is denoised in one time-step, and the difference between added noise and predicted noise is considered SDS loss, expressed as:

% \begin{equation}
%   \mathcal{L}_{SDS}(I_{\Phi}) \triangleq E_{t,\epsilon}[w(t)(\epsilon_{\phi}(z_t,y,t)-\epsilon)\frac{\partial I_{\Phi}}{\partial\Phi}],
%   \label{eq:SDSObserv}
% \end{equation}
\begin{equation}
    \mathcal{L}_{\text{SDS}}(I_{\Phi}) \triangleq \mathbb{E}_{t,\epsilon} \left[ w(t) \left( \epsilon_{\phi}(z_t, y, t) - \epsilon \right) \frac{\partial I_{\Phi}}{\partial \Phi} \right],
  \label{eq:SDSObservGA}
\end{equation}
where the input $I_{\Phi}$ represents a rendered image from a 3D representation, such as 3D Gaussians, with optimizable parameters $\Phi$. $\epsilon_{\phi}$ corresponds to the predicted noise of diffusion networks, which is produced by incorporating the noise image $z_t$ as input and conditioning it with a text or image $y$ at timestep $t$. The noise image $z_t$ is derived by introducing noise $\epsilon$ into $I_{\Phi}$ at timestep $t$. The loss is weighted by the diffusion scheduler $w(t)$. 
% \vspace{-3mm}

\subsection{Overview of the RTPD Algorithm}\label{subsec:algo_overview}
Fig.~\ref{fig:Overview} presents an overview of our RTPD algorithm.
It is grounded in the Hausdorff distance-based penetration depth calculation method (Sec.~\ref{sec:preliminary}).
%, similar to that of Tang et al.~\shortcite{SIG09HIST}.
The process consists of two primary phases: penetration surface extraction and Hausdorff distance calculation.
We leverage the RTX platform's capabilities to accelerate both of these steps.

\begin{figure*}[t]
    \centering
    \includegraphics[width=0.8\textwidth]{Image/overview.pdf}
    \caption{The overview of RT-based penetration depth calculation algorithm overview}
    \label{fig:Overview}
\end{figure*}

The penetration surface extraction phase focuses on identifying the overlapped region between two objects.
\revision{The penetration surface is defined as a set of polygons from one object, where at least one of its vertices lies within the other object. 
Note that in our work, we focus on triangles rather than general polygons, as they are processed most efficiently on the RTX platform.}
To facilitate this extraction, we introduce a ray-tracing-based \revision{Point-in-Polyhedron} test (RT-PIP), significantly accelerated through the use of RT cores (Sec.~\ref{sec:RT-PIP}).
This test capitalizes on the ray-surface intersection capabilities of the RTX platform.
%
Initially, a Geometry Acceleration Structure (GAS) is generated for each object, as required by the RTX platform.
The RT-PIP module takes the GAS of one object (e.g., $GAS_{A}$) and the point set of the other object (e.g., $P_{B}$).
It outputs a set of points (e.g., $P_{\partial B}$) representing the penetration region, indicating their location inside the opposing object.
Subsequently, a penetration surface (e.g., $\partial B$) is constructed using this point set (e.g., $P_{\partial B}$) (Sec.~\ref{subsec:surfaceGen}).
%
The generated penetration surfaces (e.g., $\partial A$ and $\partial B$) are then forwarded to the next step. 

The Hausdorff distance calculation phase utilizes the ray-surface intersection test of the RTX platform (Sec.~\ref{sec:RT-Hausdorff}) to compute the Hausdorff distance between two objects.
We introduce a novel Ray-Tracing-based Hausdorff DISTance algorithm, RT-HDIST.
It begins by generating GAS for the two penetration surfaces, $P_{\partial A}$ and $P_{\partial B}$, derived from the preceding step.
RT-HDIST processes the GAS of a penetration surface (e.g., $GAS_{\partial A}$) alongside the point set of the other penetration surface (e.g., $P_{\partial B}$) to compute the penetration depth between them.
The algorithm operates bidirectionally, considering both directions ($\partial A \to \partial B$ and $\partial B \to \partial A$).
The final penetration depth between the two objects, A and B, is determined by selecting the larger value from these two directional computations.

%In the Hausdorff distance calculation step, we compute the Hausdorff distance between given two objects using a ray-surface-intersection test. (Sec.~\ref{sec:RT-Hausdorff}) Initially, we construct the GAS for both $\partial A$ and $\partial B$ to utilize the RT-core effectively. The RT-based Hausdorff distance algorithms then determine the Hausdorff distance by processing the GAS of one object (e.g. $GAS_{\partial A}$) and set of the vertices of the other (e.g. $P_{\partial B}$). Following the Hausdorff distance definition (Eq.~\ref{equation:hausdorff_definition}), we compute the Hausdorff distance to both directions ($\partial A \to \partial B$) and ($\partial B \to \partial A$). As a result, the bigger one is the final Hausdorff distance, and also it is the penetration depth between input object $A$ and $B$.


%the proposed RT-based penetration depth calculation pipeline.
%Our proposed methods adopt Tang's Hausdorff-based penetration depth methods~\cite{SIG09HIST}. The pipeline is divided into the penetration surface extraction step and the Hausdorff distance calculation between the penetration surface steps. However, since Tang's approach is not suitable for the RT platform in detail, we modified and applied it with appropriate methods.

%The penetration surface extraction step is extracting overlapped surfaces on other objects. To utilize the RT core, we use the ray-intersection-based PIP(Point-In-Polygon) algorithms instead of collision detection between two objects which Tang et al.~\cite{SIG09HIST} used. (Sec.~\ref{sec:RT-PIP})
%RT core-based PIP test uses a ray-surface intersection test. For purpose this, we generate the GAS(Geometry Acceleration Structure) for each object. RT core-based PIP test takes the GAS of one object (e.g. $GAS_{A}$) and a set of vertex of another one (e.g. $P_{B}$). Then this computes the penetrated vertex set of another one (e.g. $P_{\partial B}$). To calculate the Hausdorff distance, these vertex sets change to objects constructed by penetrated surface (e.g. $\partial B$). Finally, the two generated overlapped surface objects $\partial A$ and $\partial B$ are used in the Hausdorff distance calculation step.
\section{Methods}
This work uses qualitative methods to explore the communication behaviors of players in \textit{League of Legends (LoL)}. By qualitatively observing and inquiring about player communication decisions as they occur, we aim to extract insights into players' reasoning, strategies, and the underlying factors influencing their choices during the actual conditions of gameplay. We observe \textit{LoL} players during real ranked games, asking them in-the-moment questions as well as follow-up interview questions after the matches have ended to capture the nuances of their communication decisions. 

\subsection{Participants}
We recruited participants for the study through university forums and social media in South Korea. Participants were required to be 18 or older and active players of Solo Ranked mode in \textit{LoL} with a valid rank during the current season at the time of the experiment (Season 2024). The recruitment post notified participants of the observational nature of the study and informed them that they would be expected to speak out loud and answer questions during their play sessions. A total of 36 players completed the recruitment survey, which asked for a self-report of their age, game history, preferred roles, and current rank.

We conducted in-person interview studies with a sample of 22 players. This sample excluded players who had played for less than a year, who were not willing to answer questions during the game, or who were unable to participate in person. From the remaining pool, players were chosen to maximize the diversity and representativeness of the player base based on their rank, experience, and roles. If several players shared similar profiles, we randomly selected between the participants. We conducted 17 interviews through this sampling method. Out of the first 17 participants, 16 participants identified as male, and one identified as female. Consequently, to increase the gender diversity of the sample and ensure that the results reflect a broad range of player experiences and perspectives, we specifically recruited female \textit{LoL} players through snowball sampling, while maintaining diversity in preferred roles and rank. We recruited and interviewed participants from the survey responders until qualitative saturation was reached, following the definition by Braun and Clark~\cite{braun2021saturation}. The final sample consisted of 16 male ($72.7\%$) and 6 female ($27.3\%$) participants. This ratio approximates the imbalanced gender demographic of \textit{LoL}, where estimates have suggested that $80$-$90\%$ of the player base is male~\cite{kordyaka2023gender}. We address the influence of gender identity on communication in Section ~\ref{discussion} and ~\ref{limitations}. The full list of participants and their information is shown in Table ~\ref{table:participant_info}.

The players' age ranged from 20 to 32 years old (mean=$23.7$ years, SD=$3.3$ years) and players' \textit{LoL} experience ranged from 2 to 13 years (mean=$7.7$ years, SD=$3.9$ years). The Solo queue ranks of the players were 2 Iron ($9.1\%$), 2 Bronze ($9.1\%$), 5 Silver ($22.7\%$), 7 Gold ($31.8\%$), 5 Platinum ($22.7\%$), and 1 Emerald ($4.5\%$) at the time of the study. Though the distribution is not as even as the Solo queue rank distribution in the Korean \textit{LoL} server ($12\%$ Iron, $19\%$ Bronze, $16\%$ Silver, $15\%$ Gold, $18\%$ Platinum, $13\%$ Emerald, and $5\%$ Diamond and above~\cite{log2024}), it encompasses the diverse range of skills of most \textit{LoL} players. Thus, the selected players reflected a comprehensive sample of engaged players with varying experience and skill levels. 


\begin{table*}
\centering
\caption{Participant Information and Game Session Information of \textit{League of Legends} Players}
    \label{table:participant_info}
\begin{tabular}{c|c|c|c|c|c|c} 
\toprule
\textbf{ID} & \textbf{Gender} & \textbf{Age} & \textbf{Experience} & \textbf{Solo Rank Tier} & \textbf{Role Played} & \textbf{Game Outcome}  \\ 
\hline
P1          & Male            & 24           & 12 years            & Silver                  & Jungle               & Win                    \\ 
\hline
P2          & Male            & 23           & 3 years             & Silver                  & Top                  & Loss                   \\ 
\hline
P3          & Male            & 32           & 12 years            & Bronze                  & Mid                  & Loss                   \\ 
\hline
P4          & Male            & 29           & 10 years            & Silver                  & Jungle               & Win                    \\ 
\hline
P5          & Male            & 27           & 11 years            & Emerald                 & Bot                  & Loss                   \\ 
\hline
P6          & Male            & 21           & 11 years            & Platinum                & Top                  & Win                    \\ 
\hline
P7          & Male            & 27           & 3 years             & Gold                    & Jungle               & Win                    \\ 
\hline
P8          & Male            & 20           & 9 years             & Gold                    & Bot                  & Win                    \\ 
\hline
P9          & Male            & 25           & 9 years             & Bronze                  & Jungle               & Loss                   \\ 
\hline
P10         & Male            & 23           & 7 years             & Silver                  & Jungle               & Loss                   \\ 
\hline
P11         & Male            & 21           & 12 years            & Gold                    & Mid                  & Loss                   \\ 
\hline
P12         & Male            & 22           & 11 years            & Platinum                & Mid                  & Loss                   \\ 
\hline
P13         & Female          & 24           & 3 years             & Gold                    & Support              & Loss                   \\ 
\hline
P14         & Male            & 25           & 10 years            & Platinum                & Bot                  & Win                    \\ 
\hline
P15         & Male            & 26           & 10 years            & Gold                    & Jungle               & Win                    \\ 
\hline
P16         & Male            & 26           & 13 years            & Gold                    & Mid                  & Loss                   \\ 
\hline
P17         & Male            & 25           & 8 years             & Platinum                & Support              & Loss                   \\
\hline
P18         & Female            & 21           & 3 years             & Platinum                & Support              & Loss                   \\
\hline
P19         & Female            & 20           & 6 years             & Gold                & Support              & Loss                   \\
\hline
P20         & Female            & 20           & 2 years             & Iron                & Top              & Loss                   \\
\hline
P21         & Female            & 20           & 3 years             & Iron                & Jungle              & Win                   \\
\hline
P22         & Female            & 21           & 2 years             & Silver                & Support              & Loss                   \\
\bottomrule
\end{tabular}
\end{table*}

\subsection{Procedure}
To capture the in-game mechanics of communication patterns and dynamically changing communication behavior in \textit{LoL}, we conducted an in-person observation and interview study. The study was conducted with the approval of the Institutional Review Board at the first author's research institution.

\subsubsection{Study Environment}

Each participant was asked to play a Solo Ranked game of \textit{LoL} while researchers observed and inquired about their actions in real time. In the process of study design, alternative study setups were considered. An initial plan to observe participants remotely through screen sharing was discarded as pilot studies revealed that latency and network issues were significantly disruptive to the researchers’ ability to observe and ask questions in a timely manner. The research team also decided not to recruit participants to observe in their own homes due to concerns about privacy and safety. Thus, to better gather responses to in-the-moment inquiries from the participants without any latency, the study was held in person within a controlled research environment to enhance the clarity and quality of the question-and-answer process. This approach also allowed researchers to note the participant's gaze, hesitation, and other subtle movements that would be missed in remote settings. Though an in-lab study does not perfectly recreate the in-home environment, the research team determined that this option best balanced the various tradeoffs.


The research team worked to ensure that the study environment best suited each participant’s preferences in the following steps. We first conducted several pilot interviews to tailor the play space to prioritize player comfort and approximate real-life conditions. The study took place in an enclosed room --- frequently used for interviews and user studies --- equipped with large desks and office chairs. We limited natural light and turned on overhead lights for visibility. Moreover, players were individually asked if the environment felt natural and comfortable, and adjustments were made based on their feedback. We set up the study environment with equipment (mouse, keyboard, monitor) designed specifically for online gaming. The players were also permitted to bring personal equipment if they wished. Before entering the game, the players were instructed to adjust both the equipment (e.g., mouse sensitivity) and in-game settings (e.g., shortcut keys). All players were given as much time as needed until they expressed satisfaction with the setup. At the end of the study, we asked players if anything about the setup or procedure had negatively impacted their gameplay. Three players reported feeling some discomfort from using unfamiliar equipment but noted that it did not affect their typical playstyle or their answers. The participants were compensated 20,000 KRW (approximately 15 USD) for completing the study. 


\subsubsection{Interview Process}

The researchers observed the game session through a separate screen connected to the player's monitor and noted any communication actions, attempts, and responses by the player. During the study orientation, researchers emphasized that participants should play and communicate naturally, including using offensive language, muting or reporting other players, or forfeiting the game if desired. Participants were assured that all data would be anonymized for analysis. To minimize distractions, they were informed that they could skip questions if they found them intrusive or preferred not to respond. During intense in-game situations in which the participant could not answer, the researchers documented the context and either repeated the question once the game state had stabilized or immediately after the match ended. We discuss the limitations of using an observational approach to study in-game communication, such as social desirability bias~\cite{grimm2010social}, in Section ~\ref{limitations}.

The researcher observed the communication between teammates, noting what triggered the communication or what communication medium was used for different purposes. Based on these observations, the participant was asked questions about why they did or did not perform certain communication actions to understand their assessment and perception of the communication. Some of the questions were asked to all participants, such as the reasoning behind the frequency of using certain communication media and their perception towards teammates who engage in certain forms of communication. Other non-structured questions were asked when the player triggered a certain action (``\textit{You just pinged your ally with Enemy Missing ping multiple times. What was the purpose?''}) or responded to (or ignored) their team's communication (``\textit{It seems that you opted to not vote for the surrender vote. Why is this so?}'').

The researchers recorded the gameplay and thoroughly transcribed observations and game states during the game. The observations included types of communication media used (chat, ping, votes, emotes), the target of the communication (if unclear, then players were asked to clarify the target), player reactions to ongoing discourse or communication usage by their team, and physical reactions such as spoken utterances and body gestures. After completing the game, participants engaged in a 15 to 20-minute post-game interview. They were asked to reflect on their in-game communication behavior and perceptions, including the motivations behind their communication tendencies and choices. The interview also inquired how their teammates' communication behaviors influenced their perception of those teammates, as well as the overall impact of such interactions on their mental state or performance. Participants were further invited to share suggestions for improving communication in \textit{LoL}, such as the potential addition of voice chat. The full list of the interview questions is provided in Appendix ~\ref{appendix2}.

Overall, a total of 24 games were played, of which two were forfeited within 20 minutes. Players were asked to play another game if their first game ended within 20 minutes due to a surrender vote from either team as these games did not demonstrate communication across all stages of the game. The other 22 games lasted a minimum of 24 minutes, most of which were played to completion without forfeit from either team or with a forfeit when the victor was very clearly determined.


\subsection{Thematic Analysis}
We conducted an inductive thematic analysis applying the methodology from Elo and  Kyngäs~\cite{elo2008qualitative}. The researchers gathered the transcript of the in-game and post-interview, video recording and the replay file of the game, and observational notes for each participant. We incorporated the notes on participant behavior, game states, and other players' communication patterns into the transcript at its corresponding times, providing contextual information on what was happening at the time of the question or the player's reaction.

Before initiating coding, the first, second, and third authors familiarized themselves with the data collected. The three authors then independently performed line-by-line open coding on eight participants' data to identify preliminary themes. After the initial coding was completed, the authors shared the codes to combine convergent ideas and discuss any differing perspectives. The first author then validated the codes on the remaining data, engaging in discussions with the second and third authors to iterate on the codebook. The final codebook contained 55 codes with 13 categories, organized by themes that answer each research question in depth. For RQ1, we find themes of communication types and what triggers or deters a player's decision to communicate. For RQ2, we categorize factors used by players to assess communication opportunities as well as reactions based on such assessment. For RQ3, we find themes on how the team's communication behaviors affect a player's perception towards other teammates during the game. We provide the final codebook in Appendix ~\ref{appendix}. 
\section{Experiment}

This section evaluates IRIS's capability to create demonstrations, focusing on the efficiency and intuitiveness of data collection.
It is followed by a performance profile analysis.

\subsection{Data Collection Evaluation in Simulation}

% IRIS is promising to collect demonstration in the simualtion,
% To verify the data collection efficiency of IRIS,
% a study was conducted to evaluate it.
% LIBERO was selected and 4 tasks is picked which represent four different movement.
% which includes
% \textit{close the microwave},
% \textit{turn off the stove},
% \textit{pick up the book in the middle and place it on the cabinet shelf},
% and
% \textit{turn on the stove and put the frying pan on it}.


% To make a fair comparison, we use the task from LIBERO and the data rather than designing other task.
% The baselines are two control interfaces from LIBERO (keyboards and 3D mouse),
% From the prior study from \cite{jiang2024comprehensive} \textcolor{red}{find another one support research} and our try,
% the hand tracking is not stable comparing to motion controller. so we choose KT and MC to conduct the user study.

% There are 8 participants in this user study, we evaluate each interface by using objective metrics ans subjective metrics.
% The objective metrics includes success rate and time consumed in each tasks,
% and the subjective metrics are conducted by questionnaires includes 
% \textit{Experience}, \textit{Usefulness}, \textit{Intuitiveness}, and \textit{Efficiency}.
% Every participant will use each interface to collect demonstration five times for each task,
% they will give a score in these four dimensions from 1 to 7.

% 
\begin{figure}[h]
    \centering
    \setlength{\tabcolsep}{2pt} % Reduce space between columns
    \renewcommand{\arraystretch}{.5} % Adjust row spacing
    \begin{tabular}{cccc}
        \includegraphics[width=0.115\textwidth]{image/user_study_task/t1.png} & 
        \includegraphics[width=0.115\textwidth]{image/user_study_task/t2.png} & 
        \includegraphics[width=0.115\textwidth]{image/user_study_task/t3.png} & 
        \includegraphics[width=0.115\textwidth]{image/user_study_task/t4.png} \\
        \scriptsize (a1) & 
        \scriptsize (b1) & 
        \scriptsize (c1) & 
        \scriptsize (d1) \\
        \includegraphics[width=0.115\textwidth]{image/user_study_task/libt1.png} & 
        \includegraphics[width=0.115\textwidth]{image/user_study_task/libt2.png} & 
        \includegraphics[width=0.115\textwidth]{image/user_study_task/libt3.png} & 
        \includegraphics[width=0.115\textwidth]{image/user_study_task/libt4.png} \\
        \scriptsize (a2) & 
        \scriptsize (b2) & 
        \scriptsize (c2) & 
        \scriptsize (d2) \\
    \end{tabular}
    \caption{
    Four tasks from LIBERO in simulation (top row: a1, b1, c1, d1) and corresponding view from Meta Quest 3 (bottom row: a2, b2, c2, d2): (a) Close the microwave, (b) Turn off the stove, (c) Pick up the book and place it on the shelf, (d) Turn on the stove and place the frying pan on it.}
    \label{fig:images_grid}
\end{figure}


To assess the data collection efficiency of IRIS, a user study was conducted by using tasks from the LIBERO benchmark \cite{liu2024libero}, which represents diverse types of movements.
Four representative tasks (see Figure~\ref{fig:images_grid}) were selected in the dimension of translation, rotation, and compound movement:
\begin{itemize}
\item[-] Task 1: \textit{close the microwave}
\item[-] Task 2: \textit{turn off the stove}
\item[-] Task 3: \textit{pick up the book in the middle and place it on the cabinet shelf}
\item[-] Task 4: \textit{turn on the stove and put the frying pan on it}.
\end{itemize}


\begin{figure}[h]
    \centering
    \setlength{\tabcolsep}{2pt} % Reduce space between columns
    \renewcommand{\arraystretch}{.5} % Adjust row spacing
    \begin{tabular}{cccc}
        \includegraphics[width=0.115\textwidth]{image/user_study_task/t1.png} & 
        \includegraphics[width=0.115\textwidth]{image/user_study_task/t2.png} & 
        \includegraphics[width=0.115\textwidth]{image/user_study_task/t3.png} & 
        \includegraphics[width=0.115\textwidth]{image/user_study_task/t4.png} \\
        \scriptsize (a1) & 
        \scriptsize (b1) & 
        \scriptsize (c1) & 
        \scriptsize (d1) \\
        \includegraphics[width=0.115\textwidth]{image/user_study_task/libt1.png} & 
        \includegraphics[width=0.115\textwidth]{image/user_study_task/libt2.png} & 
        \includegraphics[width=0.115\textwidth]{image/user_study_task/libt3.png} & 
        \includegraphics[width=0.115\textwidth]{image/user_study_task/libt4.png} \\
        \scriptsize (a2) & 
        \scriptsize (b2) & 
        \scriptsize (c2) & 
        \scriptsize (d2) \\
    \end{tabular}
    \caption{
    Four tasks from LIBERO in simulation (top row: a1, b1, c1, d1) and corresponding view from Meta Quest 3 (bottom row: a2, b2, c2, d2): (a) Close the microwave, (b) Turn off the stove, (c) Pick up the book and place it on the shelf, (d) Turn on the stove and place the frying pan on it.}
    \label{fig:images_grid}
\end{figure}



To ensure a fair comparison, the tasks and data were taken directly from LIBERO \cite{liu2024libero} without any modification.
The baselines for this study were two standard control interfaces provided by LIBERO: the Keyboard (KB) and the 3D Mouse (3M).
Based on findings from prior research \cite{jiang2024comprehensive},
hand tracking was found to be less stable than motion controllers. Therefore, we selected Kinesthetic Teaching and Motion Controller as the interfaces for the user study.

The study involved eight participants who evaluated the efficiency and intuitiveness of each interface for collecting demonstrations using both objective and subjective metrics.
The objective metrics included the success rate and the average time taken per task.
To ensure successful demonstrations, 
participants executed tasks at a very slow pace, which diluted efficiency measurements (as all interfaces appeared efficient when tasks were performed slowly), which biased participants against later interfaces \cite{jiang2024comprehensive}. 
To mitigate these biases and ensure high-quality data collection, a time limit per task was introduced,
and the time limits for four tasks are 20s, 20s, 30s, and 40s, which are quite enough for finishing the task.
The subjective metrics were assessed through a questionnaire evaluating four dimensions: \textit{Experience}, \textit{Usefulness}, \textit{Intuitiveness}, and \textit{Efficiency}.
Each participant performed each task five times using all interfaces. After finishing using one interface, they provided ratings on the subjective dimensions using a 7-point Likert scale.


\begin{table*}[!h]
\caption{Mean success rate and standard deviation per platform from the data in Fig.\ref{fig:box success rates}}
\centering
\begin{tabular}{c|c c c c}
\toprule
\textbf{Platform} & \textbf{Procedural} & \textbf{Diffusion}& \textbf{Diffusion} & \textbf{Diffusion} \\
\textbf{Heights (cm)} & \textbf{Generator} & & \textbf{VF-Offline} & \textbf{VF-Online}\\
\hline
0.10 & 0.88 $\pm$ 0.06 & 0.97 $\pm$ 0.04 & 0.98 $\pm$ 0.02 & 0.97 $\pm$ 0.02 \\
0.15 & 0.90 $\pm$ 0.07 & 0.96 $\pm$ 0.02 & 0.99 $\pm$ 0.02 & 0.99 $\pm$ 0.02 \\
0.20 & 0.91 $\pm$ 0.02 & 0.93 $\pm$ 0.05 & 0.98 $\pm$ 0.02 & 0.98 $\pm$ 0.04 \\
0.25 & 0.81 $\pm$ 0.11 & 0.93 $\pm$ 0.07 & 0.95 $\pm$ 0.08 & 0.96 $\pm$ 0.04 \\
0.30 & 0.67 $\pm$ 0.08 & 0.83 $\pm$ 0.09 & 0.86 $\pm$ 0.06 & 0.95 $\pm$ 0.03 \\
0.35 & 0.55 $\pm$ 0.13 & 0.80 $\pm$ 0.08 & 0.95 $\pm$ 0.04 & 0.93 $\pm$ 0.04 \\
0.40 & 0.62 $\pm$ 0.07 & 0.87 $\pm$ 0.09 & 0.94 $\pm$ 0.02 & 0.95 $\pm$ 0.03 \\
0.45 & 0.41 $\pm$ 0.10 & 0.79 $\pm$ 0.06 & 0.95 $\pm$ 0.05 & 0.96 $\pm$ 0.04 \\
0.50 & 0.36 $\pm$ 0.10 & 0.81 $\pm$ 0.09 & 0.97 $\pm$ 0.04 & 0.97 $\pm$ 0.04 \\
0.55 & 0.33 $\pm$ 0.05 & 0.76 $\pm$ 0.07 & 0.93 $\pm$ 0.04 & 0.98 $\pm$ 0.04 \\
0.60 & 0.18 $\pm$ 0.02 & 0.64 $\pm$ 0.05 & 0.82 $\pm$ 0.07 & 0.88 $\pm$ 0.04 \\
0.65 & 0.05 $\pm$ 0.03 & 0.21 $\pm$ 0.05 & 0.39 $\pm$ 0.06 & 0.68 $\pm$ 0.07 \\
\bottomrule
\end{tabular}
\label{tab:merged_success_rates}
\end{table*}


\begin{figure}[h!]
    \centering
    % \includegraphics[width=0.45\textwidth]{image/objective_study.pdf}
    \includegraphics[width=\linewidth]{image/objective_study_new.pdf}
    \caption{This graph shows the average task completion time (in seconds) for each interface across tasks. The KT and MC interfaces consistently perform more efficiently, while the Keyboard and 3D Mouse interfaces result in longer completion times, particularly on more complex tasks.}
    \label{fig:objective_study}
\end{figure}
For the objective metrics, Table \ref{tab:success_rate} presents the success rates for each interface across the tasks. 
A trial was considered unsuccessful if the participant failed to complete the task or exceeded the time limit. The time limits were set based on task difficulty: 20 seconds for Task 1 and Task 2, 30 seconds for Task 3, and 40 seconds for Task 4.
The data shows a success rate of over $90\%$ across all four tasks when using the KT and MC interfaces from IRIS. In contrast, the Keyboard and 3D Mouse methods from LIBERO often resulted in failures. For example, the 3D Mouse interface achieved only a $37.5\%$ success rate on task 3.
Figure \ref{fig:objective_study} shows the average time consumed for each task across four interfaces. The KT and MC interfaces consistently demonstrate lower task completion times, indicating higher efficiency. In contrast, the Keyboard and 3D Mouse interfaces show significantly higher completion times, particularly for Task 3 and Task 4, where the 3D Mouse method approaches or exceeds the task's time limit. These results align with the observed lower success rates for these interfaces, highlighting their inefficiency in time-critical tasks.


\begin{figure}[h!]
    \centering
    % \includegraphics[width=0.45\textwidth]{image/subjective_study.pdf}
    \includegraphics[width=\linewidth]{image/subjective_study_new.pdf}
    \caption{Subjective evaluation scores for usefulness, experience, intuitiveness, and efficiency across four interfaces. The KT and MC interfaces perform favorably in all categories, while the Keyboard and 3D Mouse interfaces receive lower ratings, particularly in intuitiveness and efficiency.}
    \label{fig:subjective_study}
\end{figure}

Figure \ref{fig:subjective_study} compares subjective scores for four interfaces based on usefulness, experience, intuitiveness, and efficiency.
The KT and MC interfaces consistently receive high scores across all criteria, indicating positive user perception and ease of use. 
In contrast, the Keyboard and 3D Mouse interfaces receive significantly lower ratings, particularly in intuitiveness and efficiency, reflecting the participants' difficulties in using these methods to control the robot.

The results of this user study show that IRIS received higher scores than the baseline interfaces across both objective and subjective metrics.
This indicates that the system offers a more intuitive and efficient approach for data collection.

\subsection{Performance Analysis}

The performance analysis focuses on two key aspects: network latency and headset FPS (frames per second).

Since IRIS uses asynchronous bidirectional data transfer, network latency is low, averaging around 20-30 ms. Transmission bandwidth depends on Wi-Fi capacity. Even for large scenes from RoboCasa, which contain over 200 MB of compressed assets, it takes no more than 5 seconds to transfer and generate a full scene with more than 300 objects. For real robot teleoperation, the system handles point cloud data efficiently, achieving a transmission speed of 10,000 points at 60 Hz—exceeding the camera’s frame rate.

The FPS performance depends on the hardware. On the Meta Quest 3, a scene with one robot runs at approximately 70 FPS, while on HoloLens 2, it runs around 40 FPS. As the scene size increases, FPS gradually decreases. With around 200 objects, the Meta Quest 3 headsets struggle to keep up with head movement, causing virtual objects to lag or become stuck. However, performance is additionally influenced by the complexity of the meshes and textures, as these require significant computational resources from the XR headset.

In our experiments by using Meta Quest 3, IRIS successfully handled all benchmark scenarios listed in the paper, except for some scenes from RoboCasa \cite{nasiriany2024robocasa}. These scenes have over 700 MB of assets in one single instance, which is closed to the Meta Quest 3 RAM limit. Nonetheless, IRIS was able to manage most of scenes from RoboCasa without any significant performance issues.
For real robot data collection, the optimal point cloud size is around 10,000 points, which achieves a balance between point cloud quality and FPS, maintaining a frame rate of approximately 40 FPS.


% \textcolor{red}{should be a table of performance here}



We present RiskHarvester, a risk-based tool to compute a security risk score based on the value of the asset and ease of attack on a database. We calculated the value of asset by identifying the sensitive data categories present in a database from the database keywords. We utilized data flow analysis, SQL, and Object Relational Mapper (ORM) parsing to identify the database keywords. To calculate the ease of attack, we utilized passive network analysis to retrieve the database host information. To evaluate RiskHarvester, we curated RiskBench, a benchmark of 1,791 database secret-asset pairs with sensitive data categories and host information manually retrieved from 188 GitHub repositories. RiskHarvester demonstrates precision of (95\%) and recall (90\%) in detecting database keywords for the value of asset and precision of (96\%) and recall (94\%) in detecting valid hosts for ease of attack. Finally, we conducted an online survey to understand whether developers prioritize secret removal based on security risk score. We found that 86\% of the developers prioritized the secrets for removal with descending security risk scores.


%\bmhead{Supplementary information}

%If your article has accompanying supplementary file/s please state so here. 

%Authors reporting data from electrophoretic gels and blots should supply the full unprocessed scans for key as part of thei
%Supplementary information. This may be requested by the editorial team/s if it is missing.

%Please refer to Journal-level guidance for any specific requirements.

%\acknowledgments{Acknowledgements}
%This research was supported by Basic Science Research Program through the National Research Foundation of Korea (NRF) funded by the Ministry of Education under grant  2021R1I1A3048263 (70\%) and RS-2024-00410511 (30\%).

%Acknowledgements are not compulsory. Where included they should be brief. Grant or contribution numbers may be acknowledged.

%Please refer to Journal-level guidance for any specific requirements.

%\section*{Declarations}

%Some journals require declarations to be submitted in a standardised format. Please check the Instructions for Authors of the journal to which you are submitting to see if you need to complete this section. If yes, your manuscript must contain the following sections under the heading `Declarations':


%\begin{appendices}

%\section{Section title of first appendix}\label{secA1}

%An appendix contains supplementary information that is not an essential part of the text itself but which may be helpful in providing a more comprehensive understanding of the research problem or it is information that is too cumbersome to be included in the body of the paper.

%%=============================================%%
%% For submissions to Nature Portfolio Journals %%
%% please use the heading ``Extended Data''.   %%
%%=============================================%%

%%=============================================================%%
%% Sample for another appendix section			       %%
%%=============================================================%%

%% \section{Example of another appendix section}\label{secA2}%
%% Appendices may be used for helpful, supporting or essential material that would otherwise 
%% clutter, break up or be distracting to the text. Appendices can consist of sections, figures, 
%% tables and equations etc.

%\end{appendices}

\bibliographystyle{abbrv-doi-hyperref}
\bibliography{ref}% common bib file
%% if required, the content of .bbl file can be included here once bbl is generated
%%\input sn-article.bbl

\end{document}