\documentclass[journal]{vgtc}
\vgtccategory{Research}
\vgtcpapertype{Technique}

\usepackage[svgnames]{xcolor}%
\usepackage{graphicx}%
\usepackage{booktabs}%
\usepackage{multirow}%
\usepackage{amsmath,amssymb,amsfonts}%
\usepackage{amsthm}%
\usepackage{mathrsfs}%
\usepackage[title]{appendix}%
\usepackage{textcomp}%
\usepackage{manyfoot}%
\usepackage{algorithm}%
\usepackage{algorithmicx}%
\usepackage{algpseudocode}%
\usepackage{listings}%
%%%%

\usepackage{caption}
\usepackage{subcaption}
\usepackage{tikz}
\usepackage{pgfplots}
\usepackage{array}
\usepackage{amssymb}

%%%%

\newcommand{\shortcite}[1]{\cite{{#1}}}
\newcommand{\Skip}[1]{}
\newcommand\tab[1][1cm]{\hspace*{#1}}

\newcommand{\DS}[1]{
   \textcolor{blue}{\bfseries{DS: {#1}}}
}

\newcommand{\YW}[1]{
   \textcolor{red}{\bfseries{YW: {#1}}}
}

%\newcommand{\revision}[1]{\textcolor{red}{#1}}
\newcommand{\revision}[1]{{#1}}

\newcommand{\TODO}[1]{
   \textcolor{red}{\bfseries{TODO: {#1}}}
}

\newcommand{\ToCheck}[1]{{#1}}
%\newcommand{\ToCheck}[1]{\textcolor{red}{#1}}

\newcommand{\XX}{\textcolor{red}{XX}
}

\newcommand{\addRef}{\textcolor{red}{[ref]}
}

\aboverulesep=0ex  % added
\belowrulesep=0ex  % added

%%%%


\raggedbottom

\begin{document}


\title{RTPD: Penetration Depth calculation using Hardware accelerated Ray-Tracing}

\author{%
  \authororcid{YoungWoo Kim}{0000-0003-3341-1714},
  \authororcid{Sungmin Kwon}{0000-0000-0000-0000}, and
  \authororcid{Duksu Kim}{0000-0002-9075-3983}
}

\authorfooter{
  %% insert punctuation at end of each item
  \item
  	YoungWoo Kim is with Korea University of Technology and Education(KOREATECH).
  	E-mail: aister9@koreatech.ac.kr.
  \item
  	Sungmin Kwon is with Korea University of Technology and Education(KOREATECH).
  	E-mail: 00kwonsm@koreatech.ac.kr.

  \item Duksu Kim is with Korea University of Technology and Education(KOREATECH).
  	E-mail: bluekdct@gmail.com.
}

%%==================================%%
%% Sample for unstructured abstract %%
%%==================================%%

End-to-end imitation learning offers a promising approach for training robot policies. However, generalizing to new settings—such as unseen scenes, tasks, and object instances—remains a significant challenge. Although large-scale robot demonstration datasets have shown potential for inducing generalization, they are resource-intensive to scale. In contrast, human video data is abundant and diverse, presenting an attractive alternative. Yet, these human-video datasets lack action labels, complicating their use in imitation learning. Existing methods attempt to extract grounded action representations (e.g., hand poses), but resulting policies struggle to bridge the embodiment gap between human and robot actions.
% our approach
We propose an alternative approach: leveraging language-based reasoning from human videos - essential for guiding robot actions - to train generalizable robot policies. Building on recent advances in reasoning-based policy architectures, we introduce Reasoning through Action-free Data (RAD). RAD learns from both robot demonstration data (with reasoning and action labels) and action-free human video data (with only reasoning labels). The robot data teaches the model to map reasoning to low-level actions, while the action-free data enhances reasoning capabilities. Additionally, we will release a new dataset of 3,377 human-hand demonstrations compatible with the Bridge V2 benchmark. This dataset includes chain-of-thought reasoning annotations and hand-tracking data to help facilitate future work on reasoning-driven robot learning.
% experiments
Our experiments demonstrate that RAD enables effective transfer across the embodiment gap, allowing robots to perform tasks seen only in action-free data. Furthermore, scaling up action-free reasoning data significantly improves policy performance and generalization to novel tasks. These results highlight the promise of reasoning-driven learning from action-free datasets for advancing generalizable robot control. 
% releasing dataset
Website: \href{https://rad-generalization.github.io}{here}.


\keywords{GPUs and Multi-core Architectures, Special Purpose Hardware, Geometry-based Techniques, Proximity queries}


\teaser{
  \centering
  \includegraphics[width=\linewidth, alt={Temporal image for teaser.}]{Image/teaser.pdf}
    \caption{
        Comparison of processing times for penetration depth calculation using the CPU-based implementation based on the state-of-the-art method ($CPU_{Tang}$~\cite{SIG09HIST}), CUDA-based GPU implementation ($GPU_{CUDA}$), and our ray-tracing core-based method ($RTPD$).
        Left: Bunny benchmark. Right: Lucy benchmark.
        The black lines represent the ground truth, while the white lines indicate the results from our method.
        Our method achieved speedups of up to 37.66x and 5.33x compared to $CPU_{Tang}$ and $GPU_{CUDA}$, respectively.
  }
  \label{fig:teaser}
}

\date{February 2025}

%%\pacs[JEL Classification]{D8, H51}

%%\pacs[MSC Classification]{35A01, 65L10, 65L12, 65L20, 65L70}

\maketitle

\section{Introduction}
\label{sec:intro}


\ps{Challenges of technology scaling}

The growing demand for computing performance has always been met by increasing the number of transistors per chip, which is only possible due to CMOS technology scaling.
However, as we keep pushing the boundaries of technology scaling, we encounter multiple challenges.
Firstly, whenever we transition to a more advanced technology node, the non-recurring cost due to physical design, verification, software, mask sets, and prototyping almost doubles \cite{cost-tech-node}.
As a result, designing a chip in an advanced technology node is only economically viable if the chip is manufactured in vast quantities.
Secondly, many chip components such as I/O drivers, analog circuits, or \gls{srams} have reached their scaling limit.
This means that we cannot shrink these components further, even if we use a more advanced technology with a smaller feature size.
Thirdly, advanced technology nodes suffer from high defect rates, diminishing the yield and inflating the recurring cost.
To tackle these challenges, new chip-design paradigms have been developed.

\ps{Why 2.5D integration?}

One of these new paradigms is 2.5D integration, where multiple silicon dies called chiplets are integrated into the same package.
Once designed, a single chiplet can be reused in multiple 2.5D stacked chips, which increases the ratio of production volume to non-recurring cost.
Another advantage is that multiple chiplets - fabricated in different technologies - can be integrated into the same package.
This means that only components that can take full advantage of technology scaling are built in bleeding-edge technologies.
Components that have reached their scaling limit are fabricated in more mature and hence less costly technology nodes.
Furthermore, chiplets are smaller than monolithic chips.
Therefore, manufacturing chiplets results in less silicon area loss due to fabrication defects and hence a higher yield.
Due to these economic advantages, chip vendors such as AMD \cite{amd-chiplet} and NVIDIA \cite{chiplet-book} have adopted the 2.5D integration paradigm.  

\ps{Challenges of 2.5D integration}

An important challenge when designing 2.5D stacked chips is the construction of a low-latency and high-throughput \gls{ici}. 
To build an \gls{ici}, we connect different chiplets using \gls{d2d} links.
These links are fabricated in an organic package substrate, silicon bridge, or silicon interposer, and they are connected to the chiplets using \gls{c4} bumps or microbumps.
The number of bumps per chiplet is limited, and so is the bandwidth of \gls{d2d} links.
In addition to having lower bandwidth than links in monolithic chips, \gls{d2d} links also have higher latency.
This latency is caused by wire delay and by \gls{phys} that are necessary in both the sending and the receiving chiplet.
\gls{phys} are needed to convert between protocols, voltage levels, and frequencies, which are usually different between on-chiplet links and \gls{d2d} links.
Due to these limitations, the \gls{ici} can quickly become a bottleneck.

\ps{How we solve these challenges differently than the related work does.}

Existing approaches to maximize the performance of the \gls{ici} either optimize the placement of chiplets (with potentially heterogeneous shapes) for a predetermined \gls{ici} topology 
\cite{ho,liu,seemuth,eris,osmolovskyi,tap25d,chiou}, select one topology out of a set of candidates \cite{coskun-1, coskun-2}, or they optimize the \gls{ici} topology for a 2D grid of homogeneously shaped chiplets on an active interposer \cite{butterdonut, cluscross, kite}.
To the best of our knowledge, there is no prior work on \gls{ici} topologies for chips with heterogeneously shaped chiplets or with passive silicon interposers or silicon bridges.
To fill this gap, we propose \name, a novel optimization methodology to jointly optimize the chiplet placement and \gls{ici} topology of such architectures.
\ifnb
\else
\newpage
\fi

\ps{Details on \name~and the key idea}

The key idea is as follows: 
We optimize the chiplet placement without a predetermined topology.
For each placement generated by an optimization algorithm, we infer a placement-based \gls{ici} topology by connecting chiplets that are in close proximity in that specific placement.
We then compute the latency and throughput of this combination of placement and topology for different traffic types.
These latencies and throughputs together with the total chip area are used to compute a user-defined quality-score of the placement, which is returned to the optimization algorithm.
Based on this quality score, the algorithm can further optimize the placement.
By following this iterative process, we jointly optimize the chiplet placement and the \gls{ici} topology.

\ps{Short evaluation-summary}

We provide our open-source framework implementing the proposed placement and topology co-optimization methodology, which we evaluate using both synthetic traffic and traffic traces.
A 2D grid of chiplets with a mesh topology is used as a baseline since many proposals for 2.5D stacked chips \cite{dataflow_accel_dnn, cifher, simba, hecaton, dojo} use such an architecture.
We reduce the latency of synthetic L1-to-L2 and L2-to-memory traffic, the two most important traffic types for cache coherency traffic, by up to 28\% and 62\% respectively.
For real traffic traces, we reduce the average packet latency for almost all traces and architectures considered (reduced by an 8\% or 18\% on average depending on the configuration of \gls{phys} within a chiplet).

\section{Related Work}

\subsection{Penetration Depth Computation}

The computation of penetration depth often utilizes the Minkowski sum, a well-regarded algorithm documented in Dobkin et al.'s work~\cite{dobkin1993computing}.
This method shows high efficacy for convex shapes, where the simplicity of the objects allows for accurate and computationally efficient penetration depth calculations~\cite{dobkin1993computing,varadhan2004accurate,hachenberger2009exact}.
However, applying this algorithm to concave shapes significantly increases computational complexity.  
As a result, research has focused on developing methods to approximate penetration depth more efficiently for these shapes~\cite{cameron1997enhancing,bergen1999fast,lien2010simple,je2012polydepth}.  

Beyond the Minkowski sum, other methods have been explored, including techniques such as utilizing distance fields or the Hausdorff distance for penetration depth calculations~\cite{fisher2001fast,sud2006fast,SIG09HIST}.

Tang et al.\cite{SIG09HIST} devised an efficient algorithm for calculating the Hausdorff distance between two objects within a given error bound.
They also demonstrated that the proposed algorithm can accelerate penetration depth computation by focusing on the Hausdorff distance in overlapping regions of objects.
Building upon Tang et al.'s method, Zheng et al.\cite{zheng2022economic} improved performance using a BVH-based framework with a four-point strategy.
This method has achieved a performance improvement of up to 20 times compared to Tang et al.'s technique~\cite{SIG09HIST}.
\revision{A common feature of these works, known as the culling-based method, is computing bounds for the Hausdorff distance and reducing the search space.}

\revision{Although culling-based methods have demonstrated significant performance gains, they face challenges in leveraging parallel hardware.  
Updating and sharing bounds require synchronization, which is not well-suited for massively parallel processing architectures such as GPUs.}

\revision{In this work, we propose a GPU-based penetration depth algorithm that specifically accelerates two key processes using RT core technology:  
(1) detecting the overlapping volume and (2) calculating the Hausdorff distance.  
To highlight the effectiveness of our approach, we also implemented a CPU-based penetration depth algorithm based on Tang et al.~\cite{SIG09HIST} and Zheng et al.~\cite{zheng2022economic} for performance comparison.}

%In this work, we propose a GPU-based penetration depth algorithm, specifically accelerating two key processes with RT core technology:  
%first, detecting the overlapping volume; and second, calculating the Hausdorff distance.  
%To highlight our method's effectiveness, we also implemented a CPU-based penetration depth algorithm based on Tang et al.~\cite{SIG09HIST} and Zheng et al.~\cite{zheng2022economic} for performance comparison.  

%utilize a Hausdorff distance-based method for penetration depth calculation, accelerating two key processes with RT core technology: 

%A notable development in this area is the work of Tang et al., who devised algorithms for the rapid calculation of the Hausdorff distance between two objects~\cite{SIG09HIST}.
%Their approach is geared towards efficient penetration depth calculation by focusing on the Hausdorff distance in overlapping object regions.


%One of the algorithms for calculating penetration depth is the Minkowski sum.\cite{dobkin1993computing} The Minkowski sum is useful to compute penetration depth between two convex objects because they have a simple shape so the Minkowski sum can calculate accurate penetration depth with low computational complexity~\cite{dobkin1993computing,varadhan2004accurate,hachenberger2009exact}.
%However, applying the Minkowski sum in cases involving concave objects is challenging due to higher computational complexity. As a result, prior research has focused on quickly computing an approximate penetration depth in these scenarios~\cite{cameron1997enhancing,bergen1999fast,lien2010simple,je2012polydepth}.

%Instead of the Minkowski sum method, there have also been attempts to calculate the penetration depth based on the distance field or the vertices that make up the objects~\cite{fisher2001fast,sud2006fast,SIG09HIST}. Tang et al.~\cite{SIG09HIST} proposed the algorithms that compute the Hausdorff distance between two objects quickly and showed that can be computed penetration depth to fast by calculating the Hausdorff distance for the overlapping area of two objects.

%In this paper, the proposed method is based on Tang's methods~\cite{SIG09HIST}, and then partially divided into steps detecting overlapping volume step and the Hausdorff distance step. These two steps accelerated with RT core.

\subsection{Ray-Tracing Core-Based Acceleration}

\revision{Recent advancements in GPU technology have led to the integration of dedicated ray-tracing cores (RT cores), enabling hardware-accelerated ray tracing.
These cores optimize intersection checks between rays and objects, allowing for efficient ray-bounding box and ray-triangle intersection tests.
To utilize RT cores, various frameworks such as DXR, OptiX~\cite{parker2010optix}, and Vulkan have been developed.
RT cores primarily accelerate ray intersection tasks by efficiently traversing acceleration hierarchies.}

%The Ray-Tracing Core (RT-core) is NVIDIA’s specialized hardware for accelerating ray tracing.
%Integrated into RTX GPUs like the GeForce RTX series

%\revision{Notably, OptiX~\cite{parker2010optix} is an NVIDIA-supported SDK.
%The ray-tracing core primarily facilitates two tasks: building an acceleration hierarchy and executing ray intersection tasks with traversal.}
%OptiX operates by launching a CUDA kernel and invoking a ray generation ($ray_{gen}$) shader.
%Each CUDA core thread makes requests to the ray-tracing core, which then executes appropriate shaders like intersection ($IS$), miss($miss_{hit}$), closest hit($closest$), and any hit($any_{hit}$).
%Consequently, OptiX enables access to the results of ray-primitive intersection tests.

While the core purpose of ray-tracing cores is to expedite ray tracing, recent studies have explored their application beyond this traditional scope~\cite{wald2019rtx,zhu2022rtnn,thoman2022multi,nagarajan2023rt,meneses2023accelerating,morrical2023attribute}.
Wald et al.~\cite{wald2019rtx} addressed the problem of locating points within tetrahedra using ray-tracing cores.
Zhu et al.~\cite{zhu2022rtnn} introduced a K-Nearest Neighbor (K-NN) algorithm utilizing ray-tracing cores, achieving performance improvements of 2.2 to 65.0 times compared to previous GPU-based nearest neighbor search algorithms.
Thoman et al.~\cite{thoman2022multi} employed RT cores for Room Impulse Response (RIR) simulation.
Nagarajan et al.~\cite{nagarajan2023rt} implemented RT core-based DBSCAN clustering, reporting up to 4 times higher performance enhancement.
Meneses et al.~\cite{meneses2023accelerating} proposed RT core-based Range Minimum Query (RMQ) algorithms, yielding performance up to 2.3 times faster than existing RMQ methods.

\revision{
For collision detection between objects, one of the fundamental proximity queries, researchers have explored ray-tracing approaches even before the introduction of RT-core technology.
Hermann et al.\cite{hermann2008ray} proposed ray-tracing-based collision detection methods for deformable bodies.
Youngjun et al.\cite{kim2010mesh} applied Hermann's idea to medical simulation.
Lehericey et al.\cite{lehericey2015gpu} introduced GPU ray-traced collision detection algorithms for cloth simulation.
Recently, these approaches have been extended to utilize RT cores, as demonstrated by Sui et al.\cite{sui2024hardware}, who proposed discrete and continuous collision detection algorithms using ray-tracing cores.
Unlike these works, which focus on determining when and where collisions occur, our work focuses on calculating penetration depth.
}

In line with these advancements, this study uniquely applies RT-core technology to compute penetration depth, diverging from traditional ray-tracing applications and thereby contributing a novel approach to this field.

%\subsection{Collision detection with Ray-tracing}

%\YW{There have been attempts to apply the ray tracing approaches for collision detection~\cite{hermann2008ray, kim2010mesh, lehericey2015gpu}. Hermann et al~\cite{hermann2008ray} proposed ray tracing collision detection methods for deformable bodies. Youngjun et al~\cite{kim2010mesh} apply Hermann's idea for Medical simulation. Lehericey et al~\cite{lehericey2015gpu} introduced GPU ray-traced collision detection algorithms for cloth simulation.
%However, these methods proposed deformable objects, not solid- or discrete- objects, and there is no report about the result using ray tracing core yet. Therefore, our research implements the penetration depth algorithm with ray tracing methods and reports the benefit of ray tracing core.}

%\YW{Sui et al~\cite{sui2024hardware} proposed the method for discrete and continuous collision detection with ray tracing core. They generate the ray candidate as much as the edge of the source mesh and investigate the intersections to solve discrete collision detection. And also, to solve continuous collision detection, they build sphere-swept volumes with OptiX B-Spline curves using continuous trajectory points that are pre-computed and trace the ray samely. However, their implementation only considers non-penetrating collision, and because of that reason, there need for other approaches to compute penetration cases.}

%\YW{To address this issue, our approacthe has propose the methods to find penetration surface with RT core (that called RT-PPE). Not only that, our methods report the penetration depth as computing the Hausdorff distance between the penetration surface.}

%Recently, modern GPU embedded ray tracing core for hardware accelerated ray tracing.

%To access the ray tracing core, we can use DXR, OptiX~\cite{parker2010optix}, and Vulkan.
%Above all, OptiX~\cite{parker2010optix} is NVIDIA NVIDIA-supported SDK. The ray tracing core actually works about two tasks. One is a built acceleration hierarchy, and another is ray intersection task with traversal. Therefore OptiX launches one CUDA kernel and called $ray\_gen$ shader. Each CUDA core thread requests to ray tracing core, and then ray tracing core executes a suitable shader such as $IS$, $miss\_hit$, $closest$, $any\_hit$ shader.
%Finally, we can access ray-primitive intersection test results using OptiX shader.

%While the ray tracing core is designed for accelerating ray tracing, recent research tried using the ray tracing core for other purposes~\cite{wald2019rtx,zhu2022rtnn,thoman2022multi,nagarajan2023rt,meneses2023accelerating,morrical2023attribute}.
%%Beyond ray tracing
%Wald et al~\cite{wald2019rtx} solved the point in location of tetrahedron problem using ray tracing cores.
%%RTNN
%Zhu et al~\cite{zhu2022rtnn} proposed K-NN(K-Nearest Neighbor) algorithms using ray tracing cores. They achieved a performance of 2.2-65.0 times faster than prior GPU-based nearest neighbor search algorithms.
%%RIR Simulation
%Thoman et al~\cite{thoman2022multi} utilized the RT core to RIR(Room impulse response) simulation,
%%RT-DBSCAN
%Nagarajan et al~\cite{nagarajan2023rt} implemented DBSCAN clustering with RT core and achieved performance up to 4x times.
%%RTX-RMQ
%Meneses et al~\cite{meneses2023accelerating} proposed RT core-based RMQ(Range minimum query) algorithms, and they got performance up to 2.3x than state-of-the-art RMQ algorithms.

%%
%Similar to prior research, this study is distinguished by utilizing RT-core for computing penetration depth, as opposed to conventional ray tracing problems.



\begin{figure*}[t]
\begin{center}
\includegraphics[width=.85\linewidth]{fig_overview_v3.pdf}
\end{center}
\caption{
FastAtlas Overview: In each frame, we compute charts spanning fully or partially visible triangles (a), determine texture space bounding boxes for the visible portions of the view-space projections of each chart, and tightly pack these boxes into atlases (b, here $2K \times 2K$). We simultaneously bijectively parameterize and shade the charts into their atlas boxes, obtaining high quality texture space shading (c), and use this shading to render the shaded frames (d).}
\label{fig:overview}
\label{fig:alg_overview}
\end{figure*}

\section{Overview}
\label{sec:overview}
Our work has two core contributions: a real-time, GPU-based algorithm for tight packing of general parameterized charts into compact atlases; and a real-time TSS method that
utilizes this packing.  

\paragraph*{FastAtlas Packing.}
FastAtlas runs entirely on the GPU as a series of compute shaders. It takes the bounding boxes of parameterized charts as input, and packs them into an atlas (Fig~\ref{fig:overview}b, Sec.~\ref{sec:pack}). As such, the only input it requires are the dimensions of the bounding boxes.
Its outputs are deterministic; identical input charts are packed into identical atlases. This is critical for TSS and similar applications, as it ensures that consecutive frames taken from the same camera view have the same shading. Even minute shading differences across such frames can cause sampling jitter, leading to undesirable flicker \cite{baker2012rock}. 
While prior methods such as \cite{mueller2018shading,hladky2019tessellated,hladky2021snakebinning,Neff2022MSA} cap the dimensions of the charts that can be packed as-is for a given atlas size, and scale down all charts that exceed these dimensions, we scale all charts by the same factor, and do so only when strictly necessary to achieve packing success (Figs~\ref{fig:atlas},~\ref{fig:sas_issues}). 

\paragraph*{TSS using FastAtlas.}
Our end-to-end TSS atlas generation method combines the packing method above with a novel approach for computing seamless per-frame charts. 
We define our charts as the connected components of the visible surfaces in each frame (Fig.~\ref{fig:overview}a), and efficiently compute them using a parallel union-find algorithm (Sec.~\ref{sec:visible}). Since the boundaries of these charts coincide with the contours of the rendered surface, they are {\em invisible} to the viewer. This approach 
eliminates the artifacts caused by shading discontinuities along visible seams (Fig.~\ref{fig:seams}). 

\begin{parWithWrapFigure}
\begin{wrapfigure}{l}{.27\columnwidth}%
\includegraphics[width=\linewidth]{fig_inset_view_plane.pdf}%
\end{wrapfigure}
We bijectively parametrize the {\em visible portions} of our charts by projecting them to view space (inset). This maps a constant number of texels to each pixel in the final rendered output, evenly distributing residual undersampling error across all image pixels. While conceptually straightforward, efficiently parameterizing charts containing partially visible triangles using viewspace projection is non-trivial, as the visible portions may no longer be triangular (e.g. green triangle in the inset); applying naive projection to triangles with vertices behind the camera may produce ill-posed results. Clipping triangles before projection is both computationally expensive and significantly complicates downstream operations. We avoid explicit clipping by observing that all that is required for atlas packing is the dimensions of, potentially conservative, bounding boxes of these projected visible portions. We compute such bounding boxes without explicit chart clipping by adapting a conservative screen coverage estimator \shortcite{Blinn:CalculatingScreenCoverage} (Sec.~\ref{sec:box}). We then pack the computed boxes using FastAtlas. 
\end{parWithWrapFigure}

Finally, we shade the visible portion of each chart into its corresponding atlas bounding box (Fig~\ref{fig:overview}c). 
The resulting texture is then used during rasterization as a standard texture map (Fig. ~\ref{fig:overview}d). 
Our framework is compatible with all existing approaches for texture space shading, including forward shading, raytraced illumination, or deferred shading in texture space \cite{baker:2016}. In the examples shown, we use the standard forward shading based rendering pipeline included in the G3D Innovation Engine \cite{G3D17}, a commercial grade renderer.

\section{Methods}
\label{sec:method}
Given a model $\varepsilon_{\theta}$, fine-tuned by (\ref{eq:finetuning}) for a specific concept, we can identify two distinct sampling approaches, each maximizing one of the objectives: concept fidelity or editability:

Sampling with concept (Base sampling):
\begin{equation} 
  \label{eq:concept_sampling}
  \tilde{\varepsilon}_{\theta}(p^C) = \varepsilon_{\theta} + \omega(\varepsilon_{\theta}(p^C) - \varepsilon_{\theta}) = \varepsilon_{\theta} + \omega \Delta\varepsilon_{\theta}^{C}
\end{equation}
Sampling with superclass:
\begin{equation}
  \label{eq:superclass_sampling}
  \tilde{\varepsilon}_{\theta}(p^S) = \varepsilon_{\theta} + \omega( \varepsilon_{\theta}(p^S) - \varepsilon_{\theta}) = \varepsilon_{\theta} + \omega \Delta\varepsilon_{\theta}^{S}
\end{equation} 
% Here, $p^C$ represents a concept prompt embedding (for example, \textit{"a V* with a city in the background"}) and $p^S$ indicates a superclass prompt embedding (\textit{"a backpack with a city in the background"}) where the concept token $V^*$ is replaced by a superclass token (\textit{"backpack"}).
Here, $p^C$ represents a concept prompt embedding (for example, \textit{"a V* with a city in the background"}), and $p^S$ indicates a superclass prompt embedding (\textit{"a backpack with a city in the background"}) where the concept token $V^*$ has been replaced by a superclass token (\textit{"backpack"}).

\begin{figure*}[ht!]
  \centering
  \vspace{-0.02in}
  \includegraphics[trim={0 7cm 0 7cm},clip,width=\linewidth]{imgs/sampling_new.pdf}
  \vspace{-0.20in}
  \caption{\textbf{Visualization of Different Sampling Strategies.} (a) Usual sampling with concept reproduces the concept but does not align closely with the text prompt. (b) Generation with superclass effectively captures the context obtained from the prompt but produces a random superclass representative (e.g., dog). (c-d) Mixed and Switching sampling strategies improve context preservation while maintaining the concept's identity.}
  \label{fig:sampling}
  \vspace{-0.20in}
\end{figure*}

The extended fine-tuning of the model \(\varepsilon_{\theta}\) enhances its ability to accurately reproduce the concept generated via~(\ref{eq:concept_sampling}). However, this improvement comes at the cost of overlooking the contextual information supplied by the prompt $P^C$ (see Figure~\ref{fig:sampling}a). Conversely, the generation via~(\ref{eq:superclass_sampling}) ensures the highest alignment with the text prompt, though at the expense of preserving the concept's identity (see Figure~\ref{fig:sampling}b).

% This raises the question of whether we can integrate the two sampling strategies~(\ref{eq:concept_sampling}) and~(\ref{eq:superclass_sampling}) to obtain the optimal balance between the high fidelity of the learned concept identity and its adaptability to various contexts.

This consideration raises the question of whether we can integrate the two sampling strategies~(\ref{eq:concept_sampling}) and~(\ref{eq:superclass_sampling}) to obtain the optimal balance between the high fidelity of the learned concept identity and its adaptability to various contexts.

\subsection{Mixed sampling} \label{sec:mixed_sampling}
One reasonable approach for incorporating superclass into the generation process~\citep{profusion} is to modify the sampling strategy by adding guidance to the superclass prompt (see Figure~\ref{fig:sampling}c):
\begin{align}\label{eq:mixed_sampling}
    \tilde{\varepsilon}^{MX}_{\theta}(p^S, p^C) = \varepsilon_{\theta} + \omega_s \Delta\varepsilon_{\theta}^{S} + \omega_c\Delta\varepsilon_{\theta}^{C}
\end{align}
% By adjusting the ratio between the concept guidance scale $\omega_c$ and the superclass guidance scale $\omega_s$, we can either amplify or diminish the influence of the concept or superclass, thus varying the trade-off between concept and context fidelity. In Figure~\ref{fig:visual}, you can observe how the generated output alters with increasing superclass influence. For instance, in the teapot example, as we raise the superclass guidance scale, the context, which was initially poorly represented through sampling with the concept, gradually becomes more accurate. However, excessive superclass influence may result in a loss of concept identity preservation, as illustrated in the dog example.

Adjusting the ratio between the concept guidance scale $\omega_c$ and the superclass guidance scale $\omega_s$ amplifies or diminishes the influence of the concept or superclass, varying the trade-off between concept and context fidelity. Figure~\ref{fig:visual} shows how the generated output changes with increasing superclass influence. For instance, in the teapot example, as we raise the superclass guidance scale, the context, which was initially poorly represented through sampling with the concept, gradually becomes more accurate. However, excessive superclass influence may reduce concept identity preservation, as shown in the dog example.

\subsection{Switching sampling} \label{sec:switching_sampling}
Another solution for how to combine the superclass sampling trajectory with the concept sampling trajectory is to condition several steps on the superclass prompt embedding $p^S$, then at the \textit{switching step} $t_{sw}$ switch to the concept prompt embedding $p^C$  (see Figure~\ref{fig:sampling}d). In this case~(\ref{eq:sampling}) will be rewritten in the following form:
\begin{align}\label{eq:switching_sampling}
    \tilde{\varepsilon}^{SW}_{\theta}(p^S, p^C, t_{sw}) = \varepsilon_{\theta} +
    \begin{cases}
         \omega\Delta\varepsilon_{\theta}^{S}, & t > T - t_{sw}\\
        \omega\Delta\varepsilon_{\theta}^{C}, &\text{otherwise}
    \end{cases} 
\end{align}
By increasing the \textit{switching step} $t_{sw}$, we can amplify the influence of the superclass and thus improve context preservation.
Up to 10 steps can effectively recover context that has been poorly generated through Base sampling, as demonstrated in the teapot example in Figure~\ref{fig:visual}. Nonetheless, this strategy may result in notable degradation of the concept's identity. The effect of the superclass can be so intense that the concept loses its original attributes and takes on excessive characteristics from the superclass, as evidenced by the dog example in Figure~\ref{fig:visual}.

% This sampling procedure is similar to Photoswap~\cite{photoswap} approach adapted to the personalization task. The main difference is that in switched sampling we take the noise predictions entirely from the superclass trajectory for the first $t_{sw}$ steps, whereas in Photoswap only the self- and cross-attention maps and features are taken from the superclass for the first $t_{sw}$ steps. However, as we show in Section~\ref{sec:experiments}, the results of these two methods are almost indistinguishable.

This sampling procedure is similar to Photoswap~\cite{photoswap} but adapted to the personalization task. The main difference is that switched sampling takes noise predictions entirely from the superclass trajectory for the first $t_{sw}$ steps, whereas Photoswap uses only self- and cross-attention maps, and features are taken from the superclass for the first $t_{sw}$ steps. However, as we show in Section~\ref{sec:experiments}, the results of both methods are almost indistinguishable.

The aforementioned methods can be flexibly combined, we refer to this type of sampling as \textit{multi-stage sampling}:
\begin{align}\label{eq:multistage_sampling}
\tilde{\varepsilon}^{MS}_{\theta}(p^S, p^C) = \varepsilon_{\theta} +
    \begin{cases}
         (\omega_s + \omega_c)\Delta\varepsilon_{\theta}^{S} &t > T - t_{sw}\\[-0pt]
         \omega_s \Delta\varepsilon_{\theta}^{S} + \omega_c\Delta\varepsilon_{\theta}^{C}&\text{otherwise}
    \end{cases}
\end{align}
This combination enables a greater influence of the superclass on the generated output and enhances alignment with the text prompt. However, it is important to consider that as the influence of the superclass increases, the more the concept's identity is lost.

\subsection{Masked sampling}

\begin{figure*}[t!]
  \centering
  \includegraphics[trim={0 6.7cm 0 6.7cm},clip,width=\linewidth]{imgs/visual_new_v2.pdf}
  \vspace{-0.19in}
  \caption{\textbf{Effects of Superclass Influence on Different Sampling Methods.} 
  % For Mixed Sampling, the influence is adjusted by varying the superclass guidance scale $\omega_s = [1.0, 3.5, 5.0]$ with $\omega_c = 7.0 - \omega_s$. For Switching Sampling, we vary the switching step $t_{sw} = [3, 7, 20]$ . For Masked Sampling, the mask is modified by altering the thresholding quantile $q = [0.3, 0.5, 0.9]$. 
  For Mixed Sampling, the influence is adjusted by varying the superclass guidance scale $\omega_s$ with $\omega_c = 7.0 - \omega_s$. For Switching Sampling, we vary the switching step $t_{sw}$ . For Masked Sampling, the mask is modified by altering the concept mask thresholding quantile $q$.
  }
  \label{fig:visual}
  \vspace{-0.19in}
\end{figure*}

Sampling with a superclass prompt hinders the preservation of concept identity, whereas sampling with a concept prompt disrupts contextual adaptation. To address this challenge, restricting the image regions impacted by each sampling approach could be beneficial. This can be effectively achieved through masking.

Suppose at each diffusion step we could obtain a concept mask $M_t$, then we can use it in the Mixed sampling. Specifically, we apply this mask to the concept trajectory, ensuring it only influences relevant regions:
\begin{align}\label{eq:masked_base}
    \varepsilon^{M}_{\theta}(p^S, p^C) = \varepsilon_{\theta} + \omega\Delta\varepsilon_{\theta}^{C} \odot M_t + \omega\Delta\varepsilon_{\theta}^{S} \odot \overline{M_t} 
\end{align}
Moreover, to enhance the alignment between regions inside and outside the mask, and to gently amplify the influence of the superclass within the mask -- especially in cases where prompts alter the object's appearance (like color or outfit) -- we can apply Mixed sampling within the mask:
\begin{align}\label{eq:masked_mixed}
    &\varepsilon^{M}_{\theta}(p^S, p^C) = \varepsilon_{\theta} + \\ &+ \omega_c\Delta\varepsilon_{\theta}^{C} \odot M_t + 
    \omega_s\Delta\varepsilon_{\theta}^{S} \odot M_t +(\omega_c + \omega_s)\Delta\varepsilon_{\theta}^{S} \odot \overline{M_t}  \notag
\end{align}
The generation process starts with Mixed sampling for a limited number of steps, thereby enhancing the robustness of mask generation. Then, we apply masked sampling as described in (\ref{eq:masked_mixed}), using the concept mask $M_t(q)$. This mask is derived by averaging the cross-attention maps associated with the concept identifier token across all U-Net layers and binarizing it using a threshold determined by the quantile $q$:
\begin{equation}\label{eq:masked_sampling}
\tilde{\varepsilon}^{M}_{\theta}(p^S, p^C) = 
    \begin{cases}
         \tilde{\varepsilon}^{MX}_{\theta}(p^S, p^C, \omega_c^0, \omega_s^0), & t > T - t_{sw}\\
         \varepsilon^{M}_{\theta}(p^S, p^C, \omega_c, \omega_s, q),
    &\text{otherwise,}
    \end{cases} 
\end{equation}
where $\varepsilon^{M}_{\theta}(p^S, p^C, \omega_c, \omega_s, q)$ is computed as in (\ref{eq:masked_mixed}). 

Equation~\ref{eq:masked_sampling} summarizes the complete Masked sampling algorithm. Increasing the quantile $q$ reduces the area influenced by the concept, thereby expanding the region impacted by the superclass (see Appendix~\ref{app:cross_attn}) and enhancing the influence of the context, as illustrated in Figure~\ref{fig:visual}.

\subsection{Other approaches}
\textbf{ProFusion} The main contribution of the Profusion~\citep{profusion} sampling method is a novel technique to ensure the concept's preservation combined with Mixed Sampling. A sampling step in this approach consists of the following stages: (1) we predict $x_t \rightarrow \tilde{x}_{t-1}$ through the usual diffusion backward sampling process with concept (2) after that we make a forward diffusion step $\tilde{x}_{t-1}\rightarrow \tilde{x_t}$ (3) finally, we again make a backward step with the Mixed sampling  $\tilde{x_{t}} \rightarrow x_{t-1}$. The first two steps define Fusion Step and have a hyperparameter $r$ that controls its intensity (e.g. the influence on the result). In case $r=0$ we get Mixed sampling.

\textbf{Photoswap} In this method, the author proposes to replace self-attention features, cross-attention maps, and self-attention maps in the concept trajectory with maps from the superclass at several initial steps. Thus, the method has three hyperparameters: (1) $t_{SF}$ the number of initial steps during which the self-attention features are replaced, (2) $t_{CM}$ the same parameter for cross-attention maps, and (3) $t_{SM}$ for self-attention maps.

\subsection{Evaluation protocol for sampling techniques}
The study of sampling methods involves several key steps. 

% The first step is to select a fundamental fine-tune model on the basis of which we can compare different sampling techniques. For each model, we propose constructing a complete Pareto front of the Mixed sampling. We chose Mixed sampling as our baseline because it is the simplest efficient method, characterized by a single hyperparameter.

The first step is to select a fundamental fine-tuned model that will be used as a baseline for comparing different sampling techniques. For each model, we propose to construct the full Pareto front of Mixed sampling, which we selected as our baseline because it is the simplest yet efficient method, defined by a single hyperparameter.

% It is essential to select a model whose Pareto frontier exhibits a sufficiently large length; this allows for a clearer distinction between the varying parameters. Additionally, this front should lie within the optimal balance between concept fidelity and editability comparing to other fine-tuning methods. By doing so, we can examine sampling not only in scenarios where the model performs poorly but also ensure that sampling does not undermine performance in cases where the model excels.

It is crucial to select a model whose Pareto frontier is of sufficient length, enabling a clearer distinction between the varying parameters. Additionally, this frontier should lie within the optimal balance between concept fidelity and editability compared to other fine-tuning methods. This ensures that we can study sampling in scenarios where the model performs poorly while also confirming that it does not degrade performance when the model excels.

\begin{figure*}[ht!]
\centering
\begin{minipage}{.477\textwidth}
  \centering
  \includegraphics[trim={3cm 10cm 3cm 10cm},clip,width=\linewidth]{imgs/multi-stage.pdf}
  \vspace{-0.19in}
  \captionof{figure}{Pareto Frontier curves for Mixed, Switching and Multi-stage Sampling methods. Each Multi-stage sampling curve is generated by fixing the switching step while varying the superclass guidance scale $\omega_s = [1.0, 3.0, 5.0]$.}
  \label{fig:multi-stage}
\end{minipage}%
\hfill
\begin{minipage}{.477\textwidth}
  \centering
  \includegraphics[trim={3cm 10cm 3cm 10cm},clip,width=\linewidth]{imgs/masked.pdf}
  \vspace{-0.19in}
  \captionof{figure}{Pareto frontiers curves for  Masked sampling. Each Masked sampling curve is derived by varying the quantile \( q = [ 0.3, 0.5, 0.7, 0.9 ] \), which controls the mask binarization threshold; \( t_{sw} = 3, \omega_s = 3.5\) are fixed.}
  \label{fig:masked}
\end{minipage}
\vspace{-0.19in}
\end{figure*}

\begin{figure}[h!]
    \includegraphics[trim={3cm 10cm 3cm 10cm},clip,width=\linewidth]{imgs/all_mixed.pdf}
  \vspace{-0.23in}
    \caption{Mixed sampling Pareto frontiers for different fine-tuning methods.}
    \label{fig:all_mixed}
    \vspace{-0.24in}
\end{figure}
Once the base model is chosen, we fix it and proceed to compare different sampling techniques. For each method, we demonstrate its behaviour at different hyperparameter values. We illustrate the optimal points with generation examples and prove our findings with a user study.

It is important to note that the choice of sampling that maximizes editability can be approached in different ways. For example, one option is to use the model weights before fine-tuning, $\theta^{\text{orig}}$, in (\ref{eq:superclass_sampling}) instead of the fine-tuned weights, $\theta$. Additionally, we can vary the superclass prompts. One extreme option is to remove the superclass token entirely, allowing the model to focus solely on the scene's context (e.g., $p^{\hat{S}} = \textit{"with a city in the background"}$). These hyperparameters affect all sampling methods simultaneously. We analyze this dependency in Appendix~\ref{app:hyper_theta}.
% Tips: for each results section, when you write each paragraph, try to follow this:
% (1) motivation, why are you doing this?
% (2) preset result (pure data)
% (3) give interpretation


\begin{figure}[t]
	\centering
	\includegraphics[width=0.49\textwidth]{figs/Fig5_exp_platforms.pdf}
    % \includesvg[width=0.49\textwidth, inkscapelatex=false]{figs/Fig5_exp_platforms.svg} 
	\caption{Tasks for the simulation experiments. Each task is tested with various feedback types, including accurate demonstrations, noisy demonstrations, and relative corrective feedback. For the TwoArm-Lift task, partial feedback is also tested by applying feedback only to one of the robots.}
	\label{fig:tasks}
\end{figure}



% \begin{table*}
% \footnotesize
% % \setlength{\tabcolsep}{10pt}
% \caption{Experimental results in simulation under various types of feedback data. SR indicates the success rate, and CT represents the convergence timestep. A ‘$\diagdown$’ symbol denotes that the algorithm did not converge. For calculating CT, $\diagdown$ entries are replaced with the maximum allowable timestep.}
% \label{tab:data_folder_presence}
% \begin{center}
% \begin{tabular}{lcccccccccccc}
% \Xhline{0.75pt}
% Method & \multicolumn{2}{c}{CLIC-Half } & \multicolumn{2}{c}{Diffusion} & \multicolumn{2}{c}{Implicit BC} & \multicolumn{2}{c}{PVP} & \multicolumn{2}{c}{CLIC-Simplified} & \multicolumn{2}{c}{HG-DAgger/D-COACH } \\
% \hline
%  \textbf{Accurate data}& SR & CT & SR & CT & SR & CT & SR & CT &SR & CT & SR & CT \vspace{2.5pt}\\
% PushT & \textbf{0.931} & 25.6 & 0.915 & 24.1 & 0.890 & 31.8 & 0.440 & 28.5 &  0.765 & 35.6 & 0.710 & 38.6  \\
% Square & 0.930 & 43.0 & \textbf{0.953} & 48.4 & 0.732 & 54.6 &  0.000 & $\diagdown$   &0.634 & 65.9 & 0.420 & 67.0 \\
% PickCan & 0.983 & 37.8 & 0.963 & 36.1 & 0.688 & 44.2 &  0.000 & $\diagdown$    & 0.995 & 42.5 & 0.990 & 35.7 \\
% TwoArmLift & 0.970 & 34.9 & 0.990 & 23.0 & 0.000 & 2.1 &  0.000 & $\diagdown$    & 0.902 & 14.0 & 0.982 & 14.9  \\
% \hline
% \textbf{Gaussian noise} & SR & CT & SR & CT & SR & CT & SR & CT &
% SR & CT & SR & CT\vspace{2.5pt}\\ 
% PushT & 0.880 & 35.6 & \textbf{0.893} & 49.0 & 0.735 & 42.1 & 0.155 & 45.8 & 0.663 & 41.4 & 0.598 & 41.0 \\
% Square & \textbf{0.925} & 64.5 & 0.000 & 2.1 & 0.000 & 2.1 &   0.000 & $\diagdown$    & 0.238 & 71.0 & 0.060 & 77.2 \\
% PickCan & \textbf{0.973} & 37.8 & 0.467 & 68.2 & 0.070 & 70.4 &   0.000 & $\diagdown$    & 0.800 & 69.5 & 0.028 & 23.1 \\
% TwoArmLift & \textbf{0.847} & 68.0 & 0.000 & 2.1 &   0.000 & $\diagdown$    &  0.000 & $\diagdown$    & 0.433 & 39.3 & 0.008 & 63.8 \\
% \hline
% \textbf{Partial feedback} & SR & CT & SR & CT & SR & CT & SR & CT &SR & CT & SR & CT\vspace{2.5pt}\\
% TwoArmLift & \textbf{0.990} & 26.9 & 0.897 & 29.7 & 0.000 & $\diagdown$  & 0.000 & $\diagdown$ & 0.780 & 19.1 & 0.687 & 25.7 \\
% \hline
% \textbf{Relative correction}  & SR & CT & SR & CT & SR & CT & SR & CT & SR & CT & SR & CT \vspace{2.5pt}\\
% PushT & \textbf{0.853} & 40.8 & 0.060 & 72.0 & 0.400 & 58.8 &   0.110	& 50.4   & 0.733 & 43.5 & 0.520 & 49.0 \\
% Square & \textbf{0.817} & 69.0 & 0.000 & 2.1 & 0.005 & 56.3 &   0.000 & $\diagdown$    & 0.065 & 66.1 & 0.243 & 79.7 \\
% PickCan & 0.870 & 40.8 & 0.000 & 2.0 & 0.310 & 81.7 &   0.000 & $\diagdown$    & \textbf{0.890} & 67.2 & 0.693 & 62.8 \\
% TwoArmLift & \textbf{0.860} & 31.1    &  0.000 & $\diagdown$   & 0.000  & $\diagdown$  &   0.000 & $\diagdown$   & 0.613 & 18.9 & 0.115 & 64.7 \\
% \hline
% \textbf{Direction noise}  & SR & CT & SR & CT & SR & CT & SR & CT & SR & CT & SR & CT \vspace{2.5pt}\\
% PushT & 0.700 & 48.1 & 0.187 & 67.9 & 0.574 & 55.5 & NaN & NaN & NaN & NaN & NaN & NaN \\
% Square & 0.870 & 63.6 & 0.125 & 75.8 & 0.230 & 70.4 & NaN & NaN & NaN & NaN & NaN & NaN \\
% PickCan & 0.850 & 33.6 & NaN & NaN & 0.482 & 81.4 & NaN & NaN & NaN & NaN & NaN & NaN \\
% TwoArmLift & 0.907 & 20.8 & 0.885 & 46.6 & NaN & NaN & NaN & NaN & NaN & NaN & NaN & NaN \\
% \hline
% Average & \textbf{0.928} & \textbf{42.1} &  0.675 & 46.3 & 0.346  & 48.9  & 0.066 & 62.7 & & & & \\
% \Xhline{0.75pt}
% \end{tabular}
% \end{center}
% \end{table*}


% \begin{table*}
% \footnotesize
% % \setlength{\tabcolsep}{10pt}
% \caption{Experimental results in simulation under various types of feedback data. SR indicates the success rate, and CT represents the convergence timestep. A ‘$\diagdown$’ symbol denotes that the algorithm did not converge. For calculating CT, $\diagdown$ entries are replaced with the maximum allowable timestep.}
% \label{tab:data_folder_presence}
% \begin{center}
% \begin{tabular}{lcccccccc|cccc}
% \Xhline{0.75pt}
% Method & \multicolumn{2}{c}{CLIC-Half } & \multicolumn{2}{c}{Diffusion} & \multicolumn{2}{c}{Implicit BC} & \multicolumn{2}{c}{PVP} & \multicolumn{2}{c}{CLIC-Simplified} & \multicolumn{2}{c}{HG-DAgger/D-COACH } \\
%  & SR & CT & SR & CT & SR & CT & SR & CT &SR & CT & SR & CT \\
%  \hline  \textbf{Accurate data}  && & & & & & & & && & \\
% PushT & \textbf{0.931} & 25.6 & 0.915 & 24.1 & 0.890 & 31.8 & 0.440 & 28.5 &  0.765 & 35.6 & 0.710 & 38.6  \\
% Square & 0.930 & 43.0 & \textbf{0.953} & 48.4 & 0.732 & 54.6 &  0.000 & $\diagdown$   &0.634 & 65.9 & 0.420 & 67.0 \\
% PickCan & 0.983 & 37.8 & 0.963 & 36.1 & 0.688 & 44.2 &  0.000 & $\diagdown$    & 0.995 & 42.5 & 0.990 & 35.7 \\
% TwoArmLift & 0.970 & 34.9 & 0.990 & 23.0 & 0.000 & 2.1 &  0.000 & $\diagdown$    & 0.902 & 14.0 & 0.982 & 14.9  \\
% \hline
% \textbf{Gaussian noise } && & & & & & & & && &\\ 
% PushT & 0.880 & 35.6 & \textbf{0.893} & 49.0 & 0.735 & 42.1 & 0.155 & 45.8 & 0.663 & 41.4 & 0.598 & 41.0 \\
% Square & \textbf{0.925} & 64.5 & 0.000 & 2.1 & 0.000 & 2.1 &   0.000 & $\diagdown$    & 0.238 & 71.0 & 0.060 & 77.2 \\
% PickCan & \textbf{0.973} & 37.8 & 0.467 & 68.2 & 0.070 & 70.4 &   0.000 & $\diagdown$    & 0.800 & 69.5 & 0.028 & 23.1 \\
% TwoArmLift & \textbf{0.847} & 68.0 & 0.000 & 2.1 &   0.000 & $\diagdown$    &  0.000 & $\diagdown$    & 0.433 & 39.3 & 0.008 & 63.8 \\
% \hline
% \textbf{Partial feedback } && & & & & & & & && & \\
% TwoArmLift & \textbf{0.990} & 26.9 & 0.897 & 29.7 & 0.000 & $\diagdown$  & 0.000 & $\diagdown$ & 0.780 & 19.1 & 0.687 & 25.7 \\
% \hline
% \textbf{Relative correction} && & & & & & & & && &  \\
% PushT & \textbf{0.853} & 40.8 & 0.060 & 72.0 & 0.400 & 58.8 &   0.110	& 50.4   & 0.733 & 43.5 & 0.520 & 49.0 \\
% Square & \textbf{0.817} & 69.0 & 0.000 & 2.1 & 0.005 & 56.3 &   0.000 & $\diagdown$    & 0.065 & 66.1 & 0.243 & 79.7 \\
% PickCan & 0.870 & 40.8 & 0.000 & 2.0 & 0.310 & 81.7 &   0.000 & $\diagdown$    & \textbf{0.890} & 67.2 & 0.693 & 62.8 \\
% TwoArmLift & \textbf{0.860} & 31.1    &  0.000 & $\diagdown$   & 0.000  & $\diagdown$  &   0.000 & $\diagdown$   & 0.613 & 18.9 & 0.115 & 64.7 \\
% \hline
% \textbf{Direction noise }   && & & & & & & & && & \\
% PushT & 0.700 & 48.1 & 0.187 & 67.9 & 0.574 & 55.5 & NaN & NaN & NaN & NaN & NaN & NaN \\
% Square & 0.870 & 63.6 & 0.125 & 75.8 & 0.230 & 70.4 & NaN & NaN & NaN & NaN & NaN & NaN \\
% PickCan & 0.850 & 33.6 & NaN & NaN & 0.482 & 81.4 & NaN & NaN & NaN & NaN & NaN & NaN \\
% TwoArmLift & 0.907 & 20.8 & 0.885 & 46.6 & NaN & NaN & NaN & NaN & NaN & NaN & NaN & NaN \\
% \hline
% Average & \textbf{0.928} & \textbf{42.1} &  0.675 & 46.3 & 0.346  & 48.9  & 0.066 & 62.7 & & & & \\
% \Xhline{0.75pt}
% \end{tabular}
% \end{center}
% \end{table*}


\begin{table*}[t!]
\footnotesize
% \setlength{\tabcolsep}{10pt}
\caption{Experimental results in simulation under accurate feedback data. SR indicates the success rate, and CT represents the convergence timestep ($\times 10^3$). A ‘$\diagdown$’ symbol denotes that the algorithm did not converge. 
% For calculating CT, $\diagdown$ entries are replaced with the maximum allowable timestep.
}
\label{tab:sim_exp_accurate}
\begin{center}
\begin{tabular}{lcccccccccc|cccc}
\Xhline{0.75pt}
Method & \multicolumn{2}{c}{CLIC-Half (ours) } &  \multicolumn{2}{c}{CLIC-Circular (ours)} & \multicolumn{2}{c}{Diffusion Policy} & \multicolumn{2}{c}{Implicit BC} & \multicolumn{2}{c}{PVP} & \multicolumn{2}{c}{CLIC-Explicit (ours)} & \multicolumn{2}{c}{HG-DAgger} \\
 & SR & CT & SR & CT & SR & CT & SR & CT & SR & CT &SR & CT & SR & CT \\ \hline
 %  \textbf{Accurate data}  && & & & & & & & & & && & \\
Push-T & 0.931 & 25.6  & \textbf{0.955} & 28.3 & 0.915 & 24.1 & 0.890 & 31.8 & 0.440 & 28.5 &  0.765 & 35.6 & 0.710 & 38.6  \\
Square & 0.930 & 43.0 & \textbf{0.960} & 55.5 & 0.953 & 48.4 & 0.732 & 54.6 &  0.000 & $\diagdown$   &0.634 & 65.9 & 0.420 & 67.0 \\
Pick-Can & 0.983 & 37.8 & \textbf{0.990} &  38.4 & 0.980 & 36.1 & 0.688 & 44.2 &  0.000 & $\diagdown$    & 0.995 & 42.5 & 0.990 & 35.7 \\
TwoArm-Lift & 0.970 & 18.5 &  \textbf{0.990} & 12.6 & \textbf{0.990} & 23.0 & 0.000 & $\diagdown$ &  0.000 & $\diagdown$    & 0.932 & 14.7 & 0.982 & 14.9  \\
\hline
Average & {0.954} & 31.2 & \textbf{0.974} & 33.7 &  0.960 & 32.9 & 0.578  & $\diagdown$  & 0.066 & $\diagdown$ & 0.836 & 39.7 & 0.776 & 39.1 \\
\Xhline{0.75pt}
\end{tabular}
\end{center}
\end{table*}

% \vspace{-20pt}

\section{Experiments}
\label{sec:experiments}

In the experiment section, we demonstrate the effectiveness of our CLIC method through a series of simulations and real-world experiments. In Section \ref{sec:exp:simulation}, we compare CLIC with state-of-the-art methods under various types of feedback in the robot action space. Section \ref{sec:exp:ablation} presents an ablation study, analyzing the impact of key parameters and design choices. In Section \ref{sec:exp:toy_exp}, a 2D toy experiment highlights how CLIC prevents overfitting. Finally, Section \ref{sec:exp:real_rotbo} showcases the performance of CLIC in real-robot experiments.


\subsection{Simulation Experiments}
\label{sec:exp:simulation}
\textbf{Baselines} We compare CLIC with multiple baselines.
For explicit policies, we consider HG-DAgger \cite{2019_HG_DAgger} and D-COACH \cite{2019_Rodrigo_D_COACH}, which are IIL algorithms that learn from demonstration data and relative correction data, respectively. These two methods are refined for better performance, as reported in Appendix \ref{appendix:baselines}.
For implicit policies, the baselines include IBC \cite{2022_implicit_BC}, PVP \cite{2023_NIPS_PVP}, and Diffusion Policy \cite{2023_diffusionpolicy}. 
As IBC and Diffusion Policy are originally offline IL methods, to make fair comparisons, we adapt them to the IIL framework to ensure fair comparisons. 
Within this IIL framework, all methods share the same structure, differing only in their specific policy update methods.
PVP \cite{2023_NIPS_PVP} employs a loss function to assign low energy values to human actions $\bm a^h$ and high energy values to robot actions $\bm a^r$, given an observed action pair $(\bm a^r, \bm a^h)$ at state $\bm s$.
IBC, detailed in Section \ref{sec:Preliminaries}, and PVP are closely related to our method because they both involve training energy-based models. 
The Diffusion Policy is a counterpart method to IBC when learning from demonstrations. It outperforms IBC because of the improved training stability offered by the diffusion model. 
To ensure fair comparisons, we use a consistent velocity control scheme across all methods. Specifically, each method outputs a velocity command for the robot's end-effector and, if applicable, the gripper as the action at each time step.

% \textbf{Baselines} We compare CLIC with multiple baselines.
% For implicit policies, the baselines include Implicit BC (IBC) \cite{2022_implicit_BC}, PVP \cite{2023_NIPS_PVP}, and Diffusion Policy \cite{2023_diffusionpolicy}. For explicit policies, the baselines include HG-DAgger \cite{2019_HG_DAgger} and D-COACH \cite{2019_Rodrigo_D_COACH}.
% HG-DAgger and D-COACH are Interactive IL algorithms that learn from demonstration data and relative corrective data, respectively, both assuming an explicit Gaussian policy\footnote{The methods are refined for better performance, reported in Appendix \ref{appendix:HGDAgger}.}.
% When adapting offline IL baselines to the Interactive IL framework, we modify the policy update process while retaining the overall IIL structure.
% The Proxy Value Propagation (PVP) \cite{2023_NIPS_PVP} method utilizes a Proxy Value loss to assign high Q-values to human actions $\bm a^h$ and low Q-values to robot actions $\bm a^r$ that trigger human intervention.
% The IBC method, detailed in Section \ref{sec:Preliminaries}, and PVP are closely related to our method, as they both involve training energy-based models. 
% The Diffusion Policy is a counterpart method to IBC when learning from demonstrations, which outperforms IBC because of improved training stability offered by the diffusion model. 
% To ensure fair comparisons, we use a consistent velocity control scheme across all methods. Specifically, each method outputs a velocity command for the robot's end-effector as the action at each time step.



\textbf{Tasks and metrics} We compared these methods across four simulated tasks, including a Push-T task introduced in \cite{2023_diffusionpolicy} and three manipulation tasks from the robosuite benchmark \citep{2020_robosuite}, as illustrated in Fig. \ref{fig:tasks} and described in Appendix \ref{appendix:simulated_experiments_task details}.
The agent is trained by each method using an IIL framework, where the agent interacts with the environment and receives feedback from a \textbf{simulated teacher}.
This simulated teacher is employed to guarantee repeatability and fairness in training, because human teachers may not provide consistent feedback across different experimental trials. 
This is an expert policy that compares its actions with those of the learner every $n$ time steps. If the distance between these actions exceeds a threshold (set at 0.2 for all tasks), the simulated teacher provides feedback to the learner.  For all the simulation tasks, we set $n=2$.
 Each method was run for 160 episodes in every experiment, with this entire procedure repeated 3 times to calculate average final success rates and convergence time steps. 
Specifically, for each individual experiment, we calculated the final success rate by averaging the success rates of the last 8 episodes, each determined by evaluating the learned policy 10 times at the end of that episode.
We defined the convergence time step as the earliest time step when the success rate exceeded 90\% of the final success rate.

\begin{table}[t!]
\caption{Various feedback in the action space}
\centering
% \begin{tabular}{@{}ll@{}}
\begin{tabularx}{0.49\textwidth}{@{}lX@{}}
\toprule
\textbf{Type of Feedback Data}      & \textbf{Definition} \\ \midrule
Accurate absolute correction              & \( \bm a^h = \bm a^* \) \\
Gaussian noise             & \( \bm a^h = \bm a^* + \bm \omega, \bm \omega \sim \mathcal{N}(\bm0, ||\bm a^* - \bm a^r||) \) \\
Partial feedback           & \( \bm a^h \in \{[\bm a^*_{r1}, \bm a_{r2}], [\bm a_{r1}, \bm a_{r2}^*]\} \) \\ 
\hline
Accurate relative correction              & \(\bm  a^h = \bm a^r + e\bm h^*, \bm h^*= \frac{\bm a^* - \bm a^r}{||\bm a^* - \bm a^r||}  \) \\
Direction noise            & \( \bm a^h = \bm a^r + e \bm h_r \), \( \angle (\bm h_r, \bm h^*) = \beta \in [0, 90^\circ) \) \\
\bottomrule
\end{tabularx}
\label{tab:feedback-definitions}
\end{table}
% briefly show different types of feedback
\textbf{Feedback types}
In addition to accurate absolute and relative corrections, Table \ref{tab:feedback-definitions} summarizes other common types of feedback humans provide. These feedback types are also utilized in the simulation experiments. Here, the optimal action $\bm a^*$ is the original action taken by the simulated teacher. The partial feedback is utilized in the TwoArm-Lift task, where $\bm a_{r\,i}, i\in\{1, 2\}$ denotes each robot's action, and $\bm a_{ri}^*$ denotes its optimal action. 


% \begin{table*}
% \footnotesize
% % \setlength{\tabcolsep}{10pt}
% \caption{Experimental results in simulation under various types of feedback data. SR indicates the success rate, and CT represents the convergence timestep. A ‘$\diagdown$’ symbol denotes that the algorithm did not converge. For calculating CT, $\diagdown$ entries are replaced with the maximum allowable timestep.}
% \label{tab:data_folder_presence}
% \begin{center}
% \begin{tabular}{lcccccccccccc}
% \Xhline{0.75pt}
% Method & \multicolumn{2}{c}{CLIC-Half } & \multicolumn{2}{c}{Diffusion} & \multicolumn{2}{c}{Implicit BC} & \multicolumn{2}{c}{PVP} & \multicolumn{2}{c}{CLIC-Simplified} & \multicolumn{2}{c}{HG-DAgger/D-COACH } \\
% \hline
%  \textbf{Accurate data}& SR & CT & SR & CT & SR & CT & SR & CT &SR & CT & SR & CT \vspace{2.5pt}\\
% PushT & \textbf{0.931} & 25.6 & 0.915 & 24.1 & 0.890 & 31.8 & 0.440 & 28.5 &  0.765 & 35.6 & 0.710 & 38.6  \\
% Square & 0.930 & 43.0 & \textbf{0.953} & 48.4 & 0.732 & 54.6 &  0.000 & $\diagdown$   &0.634 & 65.9 & 0.420 & 67.0 \\
% PickCan & 0.983 & 37.8 & 0.963 & 36.1 & 0.688 & 44.2 &  0.000 & $\diagdown$    & 0.995 & 42.5 & 0.990 & 35.7 \\
% TwoArmLift & 0.970 & 34.9 & 0.990 & 23.0 & 0.000 & 2.1 &  0.000 & $\diagdown$    & 0.902 & 14.0 & 0.982 & 14.9  \\
% \hline
% \textbf{Gaussian noise} & SR & CT & SR & CT & SR & CT & SR & CT &
% SR & CT & SR & CT\vspace{2.5pt}\\ 
% PushT & 0.880 & 35.6 & \textbf{0.893} & 49.0 & 0.735 & 42.1 & 0.155 & 45.8 & 0.663 & 41.4 & 0.598 & 41.0 \\
% Square & \textbf{0.925} & 64.5 & 0.000 & 2.1 & 0.000 & 2.1 &   0.000 & $\diagdown$    & 0.238 & 71.0 & 0.060 & 77.2 \\
% PickCan & \textbf{0.973} & 37.8 & 0.467 & 68.2 & 0.070 & 70.4 &   0.000 & $\diagdown$    & 0.800 & 69.5 & 0.028 & 23.1 \\
% TwoArmLift & \textbf{0.847} & 68.0 & 0.000 & 2.1 &   0.000 & $\diagdown$    &  0.000 & $\diagdown$    & 0.433 & 39.3 & 0.008 & 63.8 \\
% \hline
% \textbf{Partial feedback} & SR & CT & SR & CT & SR & CT & SR & CT &SR & CT & SR & CT\vspace{2.5pt}\\
% TwoArmLift & \textbf{0.990} & 26.9 & 0.897 & 29.7 & 0.000 & $\diagdown$  & 0.000 & $\diagdown$ & 0.780 & 19.1 & 0.687 & 25.7 \\
% \hline
% \textbf{Relative correction}  & SR & CT & SR & CT & SR & CT & SR & CT & SR & CT & SR & CT \vspace{2.5pt}\\
% PushT & \textbf{0.853} & 40.8 & 0.060 & 72.0 & 0.400 & 58.8 &   0.110	& 50.4   & 0.733 & 43.5 & 0.520 & 49.0 \\
% Square & \textbf{0.817} & 69.0 & 0.000 & 2.1 & 0.005 & 56.3 &   0.000 & $\diagdown$    & 0.065 & 66.1 & 0.243 & 79.7 \\
% PickCan & 0.870 & 40.8 & 0.000 & 2.0 & 0.310 & 81.7 &   0.000 & $\diagdown$    & \textbf{0.890} & 67.2 & 0.693 & 62.8 \\
% TwoArmLift & \textbf{0.860} & 31.1    &  0.000 & $\diagdown$   & 0.000  & $\diagdown$  &   0.000 & $\diagdown$   & 0.613 & 18.9 & 0.115 & 64.7 \\
% \hline
% \textbf{Direction noise}  & SR & CT & SR & CT & SR & CT & SR & CT & SR & CT & SR & CT \vspace{2.5pt}\\
% PushT & 0.700 & 48.1 & 0.187 & 67.9 & 0.574 & 55.5 &   &   &   &   &   &   \\
% Square & 0.870 & 63.6 & 0.125 & 75.8 & 0.230 & 70.4 &   &   &   &   &   &   \\
% PickCan & 0.850 & 33.6 &   &   & 0.482 & 81.4 &   &   &   &   &   &   \\
% TwoArmLift & 0.907 & 20.8 & 0.885 & 46.6 &   &   &   &   &   &   &   &   \\
% \hline
% Average & \textbf{0.928} & \textbf{42.1} &  0.675 & 46.3 & 0.346  & 48.9  & 0.066 & 62.7 & & & & \\
% \Xhline{0.75pt}
% \end{tabular}
% \end{center}
% \end{table*}


% \begin{table*}
% \footnotesize
% % \setlength{\tabcolsep}{10pt}
% \caption{Experimental results in simulation under various types of feedback data. SR indicates the success rate, and CT represents the convergence timestep. A ‘$\diagdown$’ symbol denotes that the algorithm did not converge. For calculating CT, $\diagdown$ entries are replaced with the maximum allowable timestep.}
% \label{tab:data_folder_presence}
% \begin{center}
% \begin{tabular}{lcccccccc|cccc}
% \Xhline{0.75pt}
% Method & \multicolumn{2}{c}{CLIC-Half } & \multicolumn{2}{c}{Diffusion} & \multicolumn{2}{c}{Implicit BC} & \multicolumn{2}{c}{PVP} & \multicolumn{2}{c}{CLIC-Simplified} & \multicolumn{2}{c}{HG-DAgger/D-COACH } \\
%  & SR & CT & SR & CT & SR & CT & SR & CT &SR & CT & SR & CT \\
%  \hline  \textbf{Accurate data}  && & & & & & & & && & \\
% PushT & \textbf{0.931} & 25.6 & 0.915 & 24.1 & 0.890 & 31.8 & 0.440 & 28.5 &  0.765 & 35.6 & 0.710 & 38.6  \\
% Square & 0.930 & 43.0 & \textbf{0.953} & 48.4 & 0.732 & 54.6 &  0.000 & $\diagdown$   &0.634 & 65.9 & 0.420 & 67.0 \\
% PickCan & 0.983 & 37.8 & 0.963 & 36.1 & 0.688 & 44.2 &  0.000 & $\diagdown$    & 0.995 & 42.5 & 0.990 & 35.7 \\
% TwoArmLift & 0.970 & 34.9 & 0.990 & 23.0 & 0.000 & 2.1 &  0.000 & $\diagdown$    & 0.902 & 14.0 & 0.982 & 14.9  \\
% \hline
% \textbf{Gaussian noise } && & & & & & & & && &\\ 
% PushT & 0.880 & 35.6 & \textbf{0.893} & 49.0 & 0.735 & 42.1 & 0.155 & 45.8 & 0.663 & 41.4 & 0.598 & 41.0 \\
% Square & \textbf{0.925} & 64.5 & 0.000 & 2.1 & 0.000 & 2.1 &   0.000 & $\diagdown$    & 0.238 & 71.0 & 0.060 & 77.2 \\
% PickCan & \textbf{0.973} & 37.8 & 0.467 & 68.2 & 0.070 & 70.4 &   0.000 & $\diagdown$    & 0.800 & 69.5 & 0.028 & 23.1 \\
% TwoArmLift & \textbf{0.847} & 68.0 & 0.000 & 2.1 &   0.000 & $\diagdown$    &  0.000 & $\diagdown$    & 0.433 & 39.3 & 0.008 & 63.8 \\
% \hline
% \textbf{Partial feedback } && & & & & & & & && & \\
% TwoArmLift & \textbf{0.990} & 26.9 & 0.897 & 29.7 & 0.000 & $\diagdown$  & 0.000 & $\diagdown$ & 0.780 & 19.1 & 0.687 & 25.7 \\
% \hline
% \textbf{Relative correction} && & & & & & & & && &  \\
% PushT & \textbf{0.853} & 40.8 & 0.060 & 72.0 & 0.400 & 58.8 &   0.110	& 50.4   & 0.733 & 43.5 & 0.520 & 49.0 \\
% Square & \textbf{0.817} & 69.0 & 0.000 & 2.1 & 0.005 & 56.3 &   0.000 & $\diagdown$    & 0.065 & 66.1 & 0.243 & 79.7 \\
% PickCan & 0.870 & 40.8 & 0.000 & 2.0 & 0.310 & 81.7 &   0.000 & $\diagdown$    & \textbf{0.890} & 67.2 & 0.693 & 62.8 \\
% TwoArmLift & \textbf{0.860} & 31.1    &  0.000 & $\diagdown$   & 0.000  & $\diagdown$  &   0.000 & $\diagdown$   & 0.613 & 18.9 & 0.115 & 64.7 \\
% \hline
% \textbf{Direction noise }   && & & & & & & & && & \\
% PushT & 0.700 & 48.1 & 0.187 & 67.9 & 0.574 & 55.5 &   &   &   &   &   &   \\
% Square & 0.870 & 63.6 & 0.125 & 75.8 & 0.230 & 70.4 &   &   &   &   &   &   \\
% PickCan & 0.850 & 33.6 &   &   & 0.482 & 81.4 &   &   &   &   &   &   \\
% TwoArmLift & 0.907 & 20.8 & 0.885 & 46.6 &   &   &   &   &   &   &   &   \\
% \hline
% Average & \textbf{0.928} & \textbf{42.1} &  0.675 & 46.3 & 0.346  & 48.9  & 0.066 & 62.7 & & & & \\
% \Xhline{0.75pt}
% \end{tabular}
% \end{center}
% \end{table*}


\begin{table*}[t!]
\footnotesize
% \setlength{\tabcolsep}{10pt}
\caption{Simulation results under noisy demonstration data. SR: success rate, CT: convergence timestep ($\times 10^3$). }
\label{tab:sim_exp_noise}
\begin{center}
\begin{tabular}{lcccccccccc|cccc}
\Xhline{0.75pt}
Method & \multicolumn{2}{c}{CLIC-Half  } & \multicolumn{2}{c}{CLIC-Circular } & \multicolumn{2}{c}{Diffusion Policy} & \multicolumn{2}{c}{Implicit BC} & \multicolumn{2}{c}{PVP} & \multicolumn{2}{c}{CLIC-Explicit} & \multicolumn{2}{c}{HG-DAgger} \\
 & SR & CT & SR & CT & SR & CT & SR & CT &SR & CT & SR & CT & SR & CT \\
%  \hline  \textbf{Accurate data}  && & & & & & & & && & \\
% PushT & \textbf{0.931} & 25.6 & 0.915 & 24.1 & 0.890 & 31.8 & 0.440 & 28.5 &  0.765 & 35.6 & 0.710 & 38.6  \\
% Square & 0.930 & 43.0 & \textbf{0.953} & 48.4 & 0.732 & 54.6 &  0.000 & $\diagdown$   &0.634 & 65.9 & 0.420 & 67.0 \\
% PickCan & 0.983 & 37.8 & 0.963 & 36.1 & 0.688 & 44.2 &  0.000 & $\diagdown$    & 0.995 & 42.5 & 0.990 & 35.7 \\
% TwoArmLift & 0.970 & 34.9 & 0.990 & 23.0 & 0.000 & 2.1 &  0.000 & $\diagdown$    & 0.902 & 14.0 & 0.982 & 14.9  \\
\hline
\textbf{Gaussian noise } && & & & & & & & & & && &\\ 
Push-T & 0.880 & 35.6 &  \textbf{0.960} & 29.2 & 0.893 & 49.0 & 0.735 & 42.1 & 0.155 & 45.8 & 0.663 & 41.4 & 0.598 & 41.0 \\
Square & \textbf{0.925} & 64.5 & 0.855 & 63.9 & 0.000 &  $\diagdown$ & 0.000 &  $\diagdown$ &   0.000 & $\diagdown$    & 0.238 & 71.0 & 0.060 & 77.2 \\
Pick-Can & 0.973 & 37.8 & \textbf{1.000} & 42.8 & 0.467 & 68.2 & 0.070 & 70.4 &   0.000 & $\diagdown$    & 0.800 & 69.5 & 0.028 & 23.1 \\
TwoArm-Lift & 0.847 & 47.0 & \textbf{0.945} &  19.2& 0.000 & $\diagdown$ &   0.000 & $\diagdown$    &  0.000 & $\diagdown$    & 0.433 & 39.3 & 0.008 & 63.8 \\
\hline
% \textbf{Partial feedback } && & & & & & & & && & \\
% TwoArmLift & \textbf{0.990} & 26.9 & 0.897 & 29.7 & 0.000 & $\diagdown$  & 0.000 & $\diagdown$ & 0.780 & 19.1 & 0.687 & 25.7 \\
% \hline
% \textbf{Relative correction} && & & & & & & & && &  \\
% PushT & \textbf{0.853} & 40.8 & 0.060 & 72.0 & 0.400 & 58.8 &   0.110	& 50.4   & 0.733 & 43.5 & 0.520 & 49.0 \\
% Square & \textbf{0.817} & 69.0 & 0.000 & 2.1 & 0.005 & 56.3 &   0.000 & $\diagdown$    & 0.065 & 66.1 & 0.243 & 79.7 \\
% PickCan & 0.870 & 40.8 & 0.000 & 2.0 & 0.310 & 81.7 &   0.000 & $\diagdown$    & \textbf{0.890} & 67.2 & 0.693 & 62.8 \\
% TwoArmLift & \textbf{0.860} & 31.1    &  0.000 & $\diagdown$   & 0.000  & $\diagdown$  &   0.000 & $\diagdown$   & 0.613 & 18.9 & 0.115 & 64.7 \\
% \hline
% \hline
Average & 0.906 & 46.2 & \textbf{0.933} &  42.0 &  0.340 & $\diagdown$ & 0.268  & $\diagdown$  & 0.039 & $\diagdown$ & 0.566 & 53.3 & 0.172& 35.7 \\
\hline
\textbf{Direction noise }   && & & & & & & & & & && & \\
Push-T & 0.700 & 48.1 &  \textbf{0.950} & 27.3  & 0.187 & 67.9 & 0.574 & 55.5 & 0.000  &  $\diagdown$ & 0.638 & 44.6 & 0.473 & 43.6 \\
Square & 0.870 & 63.6 &  \textbf{0.910} & 58.9 & 0.125 & 75.8 & 0.230 & 70.4 & 0.000  &  $\diagdown$ & 0.161 & 66.9 & 0.128 & 71.6 \\
Pick-Can & \textbf{1.000} & 43.1 & \textbf{1.000} &  39.0 &   0.000  & $\diagdown$   & 0.482 & 81.4 & 0.000  & $\diagdown$  & 0.867 & 55.4 & 0.342 & 72.0 \\
TwoArm-Lift & 0.965 & 18.7 & \textbf{0.980} & 16.5  & 0.885 & 46.6 &   0.000 & $\diagdown$   &   0.000  & $\diagdown$   & 0.807 & 21.0 & 0.157 & 31.4 \\
\hline
Average & 0.884 & 43.4 &    \textbf{0.960} & 35.4 & 0.399  & $\diagdown$  & 0.429 & $\diagdown$ & 0.000 & $\diagdown$ &  0.618& 47.0 & 0.275 &  54.7 \\
\Xhline{0.75pt}
\end{tabular}
\end{center}
\end{table*}

\subsubsection{Experiments with accurate feedback}
\label{sec:exp:accurate_feedback}
% briefly introduce the tasks (or in appendix)

Table \ref{tab:sim_exp_accurate} shows the results when the teacher's feedback has no noise. 

\textbf{CLIC-Half outperforms IBC, PVP, and performs on par with Diffusion Policy} 
The results shown in Table \ref{tab:sim_exp_accurate} indicate that CLIC-Half constantly outperforms IBC and PVP in terms of success rate and convergence timesteps. 
PVP fails at the robosuite tasks because its loss function only considers the energy value of observed action pairs and cannot effectively shape the EBM. 
The optimal action assumption of IBC and Diffusion Policy is valid when the teacher's demonstration feedback is noise-free.
Under such ideal conditions, this assumption should provide more informative guidance than the assumption used by CLIC-Half.
However, IBC performance decreases as the action dimension of the task increases, with zero success rate in the TwoArm-Lift task. 
Notably, CLIC-Half achieves a $37.6 \%$ higher average success rate compared to IBC. 
Besides, CLIC-Half achieves a similar performance to that of Diffusion Policy.
These results highlight the effectiveness of training EBMs using half-space desired action spaces.

% IBC performance decreases as the action dimension of the task increases, with zero success rate in the TwoArm-Lift task. 
% CLIC-Half achieves a $37.6 \%$ higher average success rate compared to IBC. 
% Besides, CLIC-Half achieves a similar performance to that of Diffusion Policy.
% These results are remarkable because the optimal action assumption of IBC and Diffusion Policy holds true when the teacher's demonstration feedback is noise-free.
% In such ideal conditions, this assumption should provide more informative guidance than the assumption used by CLIC-Half.


 
% Therefore, this result further highlights a promising alternative approach to training policies through the estimation of the optimal actions using desired action spaces.


% motivation, why are you doing this?
% preset result (pure data)
% give interpretation
\textbf{CLIC-Circular outperforms all the baselines} 
% motivation, why are you doing this?
CLIC-Circular is the version of CLIC that mostly closely resembles IBC, as it utilizes a circular desired action space under the assumption of demonstration data. 
It reduces to IBC if all the three following conditions are met: (1) a very small radius defines the circular desired action space, (2) the temperature of the sigmoid function goes to zero, and (3) the uniform Bayes loss is used instead of policy-weighted Bayes loss.
However, CLIC-Circular outperforms IBC by a large margin. Notably,  CLIC-Circular can achieve a 99$\%$ success rate in the TwoArm-Lift task while IBC achieves zero.
CLIC-Circular also outperforms CLIC-Half as it has a more strict assumption. 
Moreover, the fact that CLIC-Circular outperforms Diffusion Policy underscores the capacity of policies represented by EBMs to surpass their diffusion-based counterparts.
Therefore, while previous studies \cite{2022_arxiv_IBC_gaps, 2023_diffusionpolicy} highlight the challenges of training EBMs, our results indicate that EBM-based policies can be trained reliably using demonstration data, by leveraging the concept of the desired action space.






\begin{table*}
\footnotesize
% \setlength{\tabcolsep}{10pt}
\caption{Simulation results under partial and relative feedback data. SR: success rate, CT: convergence timestep ($\times 10^3$). 
% A ‘$\diagdown$’ symbol denotes that the algorithm did not converge. For calculating CT, $\diagdown$ entries are replaced with the maximum allowable timestep.
}
\label{tab:sim_exp_relative_partial}
\begin{center}
\begin{tabular}{lcccccccccc|cccc}
\Xhline{0.75pt}
Method & \multicolumn{2}{c}{CLIC-Half } & \multicolumn{2}{c}{CLIC-Circular } & \multicolumn{2}{c}{Diffusion Policy} & \multicolumn{2}{c}{Implicit BC} & \multicolumn{2}{c}{PVP} & \multicolumn{2}{c}{CLIC-Explicit} & \multicolumn{2}{c}{D-COACH} \\
 & SR & CT & SR & CT & SR & CT & SR & CT &SR & CT & SR & CT & SR & CT \\\hline
\textbf{Partial feedback } && & & & & & & & && & \\
TwoArm-Lift & \textbf{0.990} & 26.9 & 0.920 &  17.8   & 0.897 & 29.7 & 0.000 & $\diagdown$  & 0.000 & $\diagdown$ & 0.863 & 18.1 & 0.687 & 25.7 \\
\hline
\textbf{Relative correction} && & &  & & & & & & & && &  \\
Push-T & \textbf{0.853} & 40.8  & 0.000 & $\diagdown$    & 0.060 & 72.0 & 0.400 & 58.8 &   0.110	& 50.4   & 0.733 & 43.5 & 0.520 & 49.0 \\
Square & \textbf{0.940} & 65.6 & 0.000 & $\diagdown$    & 0.000 & $\diagdown$ & 0.005 & 56.3 &   0.000 & $\diagdown$    & 0.065 & 66.1 & 0.243 & 79.7 \\
Pick-Can & \textbf{0.983} & 41.9 &  0.000 & $\diagdown$    & 0.000 & $\diagdown$ & 0.310 & 81.7 &   0.000 & $\diagdown$    & 0.890 & 67.2 & 0.693 & 62.8 \\
TwoArm-Lift & \textbf{0.955} & 25.3    &  0.000 & $\diagdown$    & 0.000 & $\diagdown$   & 0.000  & $\diagdown$  &   0.000 & $\diagdown$   & 0.920 & 16.8 & 0.115 & 64.7 \\
\hline
Average & \textbf{0.933} & \textbf{43.4} &  0.000 & $\diagdown$    & 0.015 & $\diagdown$ & 0.346  & $\diagdown$ & 0.066 & $\diagdown$ & 0.652 & 48.4 & 0.393 & 64.1 \\
\Xhline{0.75pt}
\end{tabular}
\end{center}
\end{table*}

\textbf{CLIC-Explicit achieves good results in uni-modal tasks, whereas PVP performs poorly}
The losses used by PVP and CLIC-Explicit share a critical similarity: they both increase the probability of human actions and decrease that of robot actions.
We denote this loss type as \textit{point-based loss} as it calculates the loss only on observed action pairs. In contrast, the losses for CLIC-Half, CLIC-Circular, and IBC are termed \textit{set-based loss}.
These losses utilize not only observed action pairs but also additional actions sampled from the EBM for loss calculation. 
Notably, PVP has zero success rate at robosuite tasks, whereas CLIC-Explicit performs well in Pick-Can and TwoArm-Lift—tasks that are both unimodal and align with the Gaussian policy assumption. 
Since PVP employs an EBM as its policy and CLIC-Explicit uses a Gaussian-parametrized policy, 
the result suggests that point-based loss is only effective for policies with simple forms, such as those based on Gaussian distributions, and fails to shape complex policies like EBMs. 
In contrast, the set-based loss provides richer information and is more effective for training EBMs. 





\subsubsection{Experiments with noisy demonstration feedback}
% motivation, why are you doing this?
% preset result (pure data)
% give interpretation

Noise is common when human teachers provide feedback to robots, either for absolute or relative corrections.
This can arise due to factors such as human fatigue or the limitations of teleoperation devices. Therefore, it is essential for algorithms to account for such noise.
To evaluate the ability of CLIC and baseline methods to learn from noisy feedback, we implemented two types of noise in simulation, as defined in Table \ref{tab:feedback-definitions}. 
For absolute corrections, we use a Gaussian distribution to model the noise, which is added to the original demonstration data. 
For relative corrections, the feedback is derived from absolute correction with a known magnitude. To introduce noise, we perturb the original direction signal by $45^\circ$ while maintaining its magnitude.
% we add direction noise to the original direction signal, resulting in a noisy direction signal with the same magnitude but has a $ 45 ^ \circ$ angle between the accurate signal. 
% To make the direction noise less challenging for the baselines, the relative correction is derived from absolute correction with known magnitude. 

% \textcolor{red}{detail more}

% \textbf{Implicit CLIC also outperforms Diffusion Policy when feedback is noisy or partial} 
\textbf{CLIC remains robust while baselines degrade under noisy feedback}
The results in Table \ref{tab:sim_exp_noise} show that, as feedback transitions from accurate to noisy, CLIC-Half and CLIC-Circular experience much smaller performance drops compared to the other baselines.
This can be observed by comparing Table \ref{tab:sim_exp_noise} with Table \ref{tab:sim_exp_accurate}.
In contrast, methods like Diffusion Policy and IBC perform significantly worse under noisy conditions.
This difference arises because these baselines depend on the strict assumption of having accurate demonstrations, making them unable to handle noisy feedback effectively. In comparison, CLIC allows adjusting the desired action space through hyperparameters, ensuring that the true optimal action remains within the desired action space even under noisy feedback.
This capability helps maintain robust performance under noisy conditions. 

\subsubsection{Experiments with relative or partial feedback}
\label{sec:exp:simulation_relative_partial}
% motivation, why are you doing this?
% Previous behavior cloning methods rely on accurate, high-quality demonstrations, limiting their use to scenarios with experts and precise teleoperation devices.  
When providing demonstrations is not possible, humans can provide feedback in more flexible ways, offering valuable information to guide improvement.
One such scenario involves partial feedback, where limitations in the control interface or a large action space make it challenging to provide complete demonstrations. In this case, we evaluated all methods on the TwoArm-Lift task, in which the teacher provides demonstration feedback to only one robot at a time.
Another scenario involves relative corrective feedback.
This feedback type is easier to provide than demonstration, because it does not require the teacher to know precisely which action should be taken; instead, it only necessitates an understanding of the general behavior the robot should exhibit.
To assess how effectively the methods handle this feedback type, we conducted experiments across four simulation tasks.


\textbf{CLIC-Half and CLIC-Explicit effectively learn from partial feedback}
The results of partial feedback experiments are shown in Table \ref{tab:sim_exp_relative_partial}.
In these experiments, human actions consist of two parts: actions on the \textit{feedback dimensions} (where human feedback is provided) and actions on the \textit{non-feedback dimensions} (which may be suboptimal).
While CLIC-Half maintains its success rate when transitioning from accurate demonstration to partial feedback, Diffusion Policy suffers from lower success rates and longer convergence times.  
This difference arises because the BC loss in Diffusion Policy attempts to imitate the entire teacher action, including non-feedback dimensions, potentially leading to suboptimal behaviors.
In contrast, CLIC-Half imitates a desired action space rather than a single action label.
It focuses on improving actions on the feedback dimensions while leaving the non-feedback dimensions unconstrained. 
As a result, CLIC-Half is robust to partial feedback as long as the optimal action lies within the desired action space.
The same reasoning explains the results of CLIC-Explicit outperforming HG-DAgger.
On the other hand, CLIC-Circular's performance drops because its circular desired action space might not include the optimal action. 
This result highlights the importance of ensuring the assumption of CLIC aligns with the data. 
% By using an intersected half-space that includes the optimal action, CLIC-Half avoids such issues, demonstrating consistent performance in partial feedback scenarios.

\textbf{CLIC-Half and CLIC-Explicit can learn from relative corrective feedback}
The results of relative corrective feedback are reported in Table  \ref{tab:sim_exp_relative_partial}. 
In these experiments, human actions improve upon robot actions but are not optimal.
CLIC-Half and CLIC-Explicit show only small performance drops compared to results in absolute corrections in Table \ref{tab:sim_exp_accurate}, because the correction magnitude information is unknown in relative correction.
In contrast, all demonstration-based baselines fail completely, achieving near-zero success rates. 
This failure occurs because the BC loss can mislead policy updates, especially when the current policy’s actions are better than the human actions in the dataset. 
Meanwhile, CLIC-Half and CLIC-Explicit construct desired action spaces to update the policy; as the policy improves, these spaces constructed from the dataset do not conflict with it and are still useful for policy improvement. 
Additionally, CLIC-Circular fails with a zero success rate, as its circular desired action space fails to include the optimal action.
Overall, the ability of CLIC-Half and CLIC-Explicit to learn from relative corrective feedback highlights their distinct advantages in such scenarios.
% As previously discussed in the analysis of partial feedback failures, the same limitations of behavior cloning methods contribute to their inability to handle relative corrective feedback effectively. CLIC’s ability to adapt highlights its distinct advantage in such scenarios.


\begin{figure}[t]
    \centering
    \includegraphics[width = 0.49\textwidth]{figs/Fig10_effects_of_alpha.pdf}
    % \includesvg[width=0.49\textwidth, inkscapelatex=false]{figs/Fig10_effects_of_alpha.svg} 
	\caption{Hyperparameter analysis of the directional certainty parameter 
$\alpha$ for CLIC-Half. The right figure visualizes how different values of 
$\alpha$ adjust the desired action space in 3D.}
 \label{fig:Fig10_effects_of_alpha}
\end{figure}


\subsection{Ablation Study}
\label{sec:exp:ablation}
In this section, we analyze the impact of various hyperparameters and loss design choices on the performance of the CLIC method.  Specifically, we focus on the directional certainty parameter $\alpha$, the temperature $T$ used in the sigmoid function of the observation model, and the different assumptions regarding the prior probability $p(\bm{a}|\bm{s})$.
During these experiments, CLIC-Half is utilized.


\subsubsection{Effects of directional certainty $\alpha$ }
The angle $\alpha$ is the main parameter utilized to control the shape of the desired action space for CLIC-Half, as shown in the right part of Fig. \ref{fig:Fig10_effects_of_alpha}.
In this experiment, we carried out experiments in the Square task. Two feedback types were considered, one with accurate feedback and another with direction noise (noise angle $\beta=45^{\circ}$). The results are reported in the left part of Fig. \ref{fig:Fig10_effects_of_alpha}.
For accurate feedback cases, the success rate decreases when $\alpha$ is larger than $120^\circ$. 
 This occurs because increasing $\alpha$ expands the desired action space to include more undesired actions, thereby providing less useful information for updating the EBM.
For direction noise, the success rate decreases for $\alpha < 2 \beta = 90^\circ$. 
This is because for any given feedback with $\alpha < 2 \beta$, the desired action space fails to include the optimal action and misguides the EBM in its update process.
These findings highlight the importance of carefully selecting $\alpha$ to balance the trade-off: maintaining an informative desired action space and ensuring that it includes the optimal action.

\begin{figure}[t]
    \centering
    \includegraphics[width = 0.49\textwidth]{figs/Fig13_exp_ablation_2.pdf}
    % \includesvg[width=0.49\textwidth, inkscapelatex=false]{figs/Fig13_exp_ablation_2.svg}
	\caption{Ablation study: (1) effects of the temperature parameter $T$. (2) Policy-weighted Bayes loss vs uniform Bayes loss.}
 \label{fig:Fig13_exp_ablation_2}
\end{figure}

\begin{figure*}[t]
	\centering
	% \includegraphics[width=\textwidth]{figs/Fig1_Illustration_generalization_across_state_2.png}
    \includegraphics[width=\textwidth]{figs/Fig1_Illustration_generalization_across_state_3.pdf}
    % \includesvg[width=\textwidth, inkscapelatex=false]{figs/Fig1_Illustration_generalization_across_state_3.svg} 
	\caption{ \textbf{Learned EBM landscapes across different trials}. The figure compares the energy landscapes learned by CLIC, PVP, and IBC after training in a 2D action space. Each row corresponds to the resulting EBMs of each trial. 
    In the middle part, we visualize the process of how CLIC-Circular reduces to IBC as $\varepsilon$ increases.
    CLIC-Circular ( with $\varepsilon=0.5$) effectively trains EBM across different trials, leading to consistent minima close to the true optimal action. In contrast, IBC overfits human actions and fails to estimate the true optimal action. Three evaluation metrics are shown in the right part of the figure.}
 \label{Fig1_Illustration_generalization_across_state}
\end{figure*}


\subsubsection{Effects of temperature $T$}
In Section \ref{section:sub:prob_desired_action_space}, the temperature $T$ is utilized to control the sharpness of the probability distribution $\text{Pr} [\bm a \in \mathcal{A} {(\bm a^r, \bm a^h)} | \bm a , \bm s] $.
We study the effects of $T$ in this experiment on the performance of CLIC, where four different values of $T$ are tested in the Square task. 
The results are presented in the left side of Fig. \ref{fig:Fig13_exp_ablation_2}.
When $T$ is very small ($\log_{10} T = -3$), the success rate drops sharply. At this extreme, the observation model becomes binary (0/1), creating a sharp boundary that is difficult for the neural network to learn. Conversely, when $T$ is too large ($T = 1$), the success rate also declines. In this case, the probabilities of actions belonging or not belonging to $\mathcal{A} {(\bm a^r, \bm a^h)}$ become nearly indistinguishable, offering limited information for policy improvement.
$T = 0.1$ proves to be a good balance between these extremes and is selected across all experiments for CLIC-Half. 


% motivation, why are you doing this?
% preset result (pure data)
% give interpretation
\subsubsection{Policy-weighted Bayes loss vs Uniform Bayes loss}
As described in Section \ref{sec:sub:sub:uniform_bayes_loss}, the uniform Bayes loss treats all actions within the desired action space equally, whereas the policy-weighted Bayes loss prioritizes actions closer to the current robot's policy.
To evaluate the effectiveness of the policy-weighted Bayes loss, we compared it against the uniform Bayes loss.  We implement the uniform variant of CLIC and evaluate it across four simulation tasks with accurate demonstration feedback. The results, shown in the right side of Fig. \ref{fig:Fig13_exp_ablation_2}, demonstrate that the uniform Bayes loss leads to significantly poorer performance compared to the policy-weighted Bayes loss.
This highlights the importance of incremental policy updates. Since the desired action space may include some undesired actions, staying close to the current policy helps avoid imitating unintended behaviors, resulting in a more stable training process.

\subsection{Toy Experiments on Noisy Feedback}
\label{sec:exp:toy_exp}




In this section, we present an example to illustrate the improved performance of CLIC over IBC. The toy task in this example consists of a single constant state with a 2D action space, where the optimal action is set to $\bm{0}$ (see Fig. \ref{Fig1_Illustration_generalization_across_state}). 
The objective is to estimate the optimal action through multiple corrective feedback.
We carried out experiments over 10 trials. In each trial, we generated a randomly sampled dataset consisting of 6 or 7 data points $(\bm s, \bm a^r, \bm a^h)$, where human actions were drawn from a Gaussian distribution centered at the optimal action. 
Each method was trained for 1,000 steps in an offline IL setting, and we visualized the trained EBMs for each method for the first two trials in Fig. \ref{Fig1_Illustration_generalization_across_state}.

To evaluate the methods, we introduced three metrics: (1) the mean square error (MSE) to optimal action: this measures MSE between each local minimum action of the EBM and the optimal action. (2) MSE to human action: this calculates the average MSE between each local minimum action of the EBM and its nearest human action. A smaller value indicates that the EBM is overfitting to the human action. (3) Variance across trials: this evaluates the variance of the EBM values over the entire action space across ten different trials.
These metrics are computed by averaging the results over the 10 trials and are reported in the right side of Fig. \ref{Fig1_Illustration_generalization_across_state}.



\textbf{CLIC learns consistent EBM landscapes across different trials}
% \cite{2017_NIPS_understanding_noise_generalization}
 The PVP-trained EBM tends to be over-optimistic about favorable actions by outputting low energy values for a large region of actions that are not present in the dataset, as shown in the left side of Fig. \ref{Fig1_Illustration_generalization_across_state}.
 This occurs because PVP calculates its loss using only observed action pairs, leaving the energy values of other actions in the action space uncontrolled. 
  On the other hand, IBC's loss function encourages human actions and discourages all other actions, even actions that are very similar to human actions. Consequently, the IBC-trained EBM overfits the data, learning minima corresponding to individual human actions. 
  This overfitting results in EBM landscapes with high variance across different trials, as shown in the bottom-right figure of Fig. \ref{Fig1_Illustration_generalization_across_state}.
 In contrast, the CLIC-trained EBM maintains consistent landscapes, with minima close to the true optimal action and low variance across different trials.
 This explains the superior performance of  CLIC over IBC and PVP, as observed in Section \ref{sec:exp:accurate_feedback}.
 


\textbf{CLIC-Circular reduces to IBC under stricter assumptions}
To illustrate how CLIC-Circular reduces to IBC, we progressively decrease the radius of the circular desired action spaces by increasing the value of $\varepsilon$ (see the middle part of Fig. \ref{Fig1_Illustration_generalization_across_state}). 
As $\varepsilon$ increases, the CLIC-trained EBM starts to split into several clusters and overfit data labels with smaller MSE to human actions, as reported in the right part of Fig. \ref{Fig1_Illustration_generalization_across_state}.
 This overfitting also leads to a larger MSE to the optimal action, indicating that the EBM becomes less effective at identifying the optimal action.
Eventually, as $\varepsilon \rightarrow 1$,  the EBM landscape closely resembles the one trained using IBC.
When the radius becomes nearly zero, imitating a circular desired action space reduces to imitating a human action label, leading to overfitting and performance drops. 
This observation highlights the key distinction between CLIC-Circular and IBC: imitating a circular desired action space rather than a single action. This distinction is crucial for training EBMs stably.
% We demonstrate the transition from CLIC-Circular to IBC by progressively adjusting its hyperparameters and analyzing their impact on the energy-based model.
% (1) For CLIC-Circular with $\varepsilon = 0.5$, by changing the policy-weighted Bayes loss to uniform Bayes loss, the learned EBM tends to imitate the whole desired action space and leads to the EBM landscape quite flat and over-optimistic.





\begin{figure*}[t]
    \centering
    \includegraphics[width = 1.0\textwidth]{figs/Fig7_InsertT_exp_results_traj_3.pdf}
    % \includesvg[width=1.0\textwidth, inkscapelatex=false]{figs/Fig7_InsertT_exp_results_traj_2.svg}
	\caption{Examples of CLIC-Half policy rollout for the Insert-T task after training. At each step, the transparent figure shows the initial state, and the orange arrow indicates the end-effector’s trajectory. The solid figure illustrates the resulting end state, which becomes the initial state for the next step.}
 \label{fig:Fig7_InsertT_exp_results_traj}
\end{figure*}

\begin{figure}[t]
    \centering
    \includegraphics[width = 0.49\textwidth]{figs/Fig7_InsertT_exp_results.pdf}
    % \includesvg[width=0.49\textwidth, inkscapelatex=false]{figs/Fig7_InsertT_exp_results.svg}
	\caption{Experiment results for the Insert-T task, categorized by difficulty levels (easy, medium, and hard). Each column shows the performance metrics for a given difficulty level, along with examples of initial states for that level.
    ``CLIC-Half (offline)'' denotes results for CLIC-Half trained offline.}
 \label{fig:Fig7_InsertT_exp_results}
\end{figure}

\subsection{Real-robot Validations}
\label{sec:exp:real_rotbo}

% \textcolor{red}{explain why we use CLIC-simplified }

Here, we use three tasks to demonstrate the practical applicability of CLIC.
The experiments include a long-horizon multi-modal Insert-T task, a dynamic ball-catching task, and a water-pouring task that necessitates precise control of the robot's end effector position and orientation. 
For the Insert-T task, we employ CLIC-Half and compare its performance against IBC and Diffusion Policy.
For the ball-catching and water-pouring tasks, we use CLIC-Explicit because it performs well in uni-modal tasks, as demonstrated in Section \ref{sec:exp:simulation}, and is more time-efficient compared to CLIC with an EBM policy (Details in Appendix \ref{appendix:time_efficiency_comparision}.).
% \footnote{Details on the time efficiency comparison between CLIC-Explicit and CLIC with an EBM policy are provided in the Appendix \ref{appendix:time_efficiency_comparision}.}
The experiments were carried out using a 7-DoF KUKA iiwa manipulator. When required, an underactuated robotic hand (1-dimensional action space) was attached to its end effector. A 6D space mouse was employed to provide feedback on the pose of the robot's end effector. Furthermore, in the ball-catching task, a keyboard provided feedback on the gripper's actuation. 
The setup of each task is detailed in Appendix \ref{appendix:real_robot_experiments_task details}, and the time durations used are reported in Appendix \ref{appendix:time_duration}. 
% The learned policy is evaluated every 5 episodes for the water-pouring task, every 10 episodes for the ball-catching task, and every 20 episodes for the Insert-T task.
% Results in Fig. \ref{fig:real_exp_figs_combined_all} show that the success rate for all tasks
% exhibits an overall improving trend as the timestep increases, demonstrating the practical applicability of CLIC. 
The experiment results are reported as follows:



\subsubsection{Insert-T—a comparison between state-of-the-art methods}
The Insert-T task requires the robot to insert a T-shaped object into a U-shaped object by pushing to adjust their positions and orientations.
Compared to the Push-T task in simulation experiments, Insert-T is more complex due to two factors: (1) it involves two objects, introducing multi-modal decisions about which object to manipulate first; and (2) it has an increased task horizon. This makes Insert-T a valuable benchmark for evaluating CLIC's performance in long-horizon, multi-modal tasks compared to state-of-the-art methods. 
To better analyze the performance of each method, we categorize the task into three difficulty levels based on the initial state of the objects and the number of contact changes required to complete the task (according to the teacher’s policy).  Examples of these categories are shown in the upper part of Fig. \ref{fig:Fig7_InsertT_exp_results}.
Tasks requiring fewer than 1 contact change are classified as ``easy”, fewer than 5 as ``medium”, and 5 or more as ``hard”. 
During each evaluation, 10 different initial states are tested for each category. To ensure a fair comparison, all methods are evaluated using the same set of initial states.
In the experiment, human-provided demonstration feedback is used to train CLIC-Half within the IIL framework. The collected data is also used to train baseline methods (Diffusion and IBC) offline. For a fair comparison, CLIC is additionally trained offline on the same dataset as the baselines.

Results for different difficulty levels are shown in Fig. \ref{fig:Fig7_InsertT_exp_results}. For easy tasks, baseline methods perform similarly to CLIC but converge more slowly. For medium and hard tasks, CLIC achieves significantly higher success rates. This is particularly evident for hard tasks, where CLIC achieves 80$\%$ success compared to 30$\%$ for Diffusion and 10$\%$ for IBC. 
The results demonstrate CLIC's ability to handle complex multi-modal tasks, thanks to the powerful encoding capabilities of EBMs and CLIC's stable EBM training. 
Furthermore, as the task difficulty increases, CLIC outperforms Diffusion and IBC by a large margin. This suggests that, for training policies, using a desired action space is more robust and efficient in real-robot tasks than relying on a single action label.
 Examples of post-training policy rollouts for CLIC in hard tasks are shown in Fig. \ref{fig:Fig7_InsertT_exp_results_traj}.

Note that CLIC is an interactive learning method, whereas Diffusion and IBC are offline methods in this experiment. Figure \ref{fig:Fig7_InsertT_exp_results} also includes results for CLIC trained with offline data, showing a similar final success rate to the online version. This indicates that CLIC can be employed to learn from offline data as well.
While CLIC is primarily based on the IIL framework, the core ideas proposed here could also benefit offline methods. We believe exploring offline training of CLIC is a promising direction for future work.



\subsubsection{Ball catching—quick coordination and partial feedback}
 The ball-catching task is challenging because of its highly dynamic nature. 
This complexity makes it difficult to provide successful demonstrations of the task to the robot, thus ruling out demonstration-based IIL methods for solving it\footnote{This limitation could be overcome with a highly reactive and precise teleoperation device. However, this also makes the solution more expensive.}.
Instead, relative corrective feedback is more intuitive and easier for this task as humans can provide direction signals occasionally to improve the robot's policy \cite{2020_DCOACH_temporal}.  
Besides, for a successful grasp, the robot must coordinate precisely the ball's motion, the end-effector's motion, along with the gripper's actuation.
This requirement makes it challenging to provide feedback on the complete action space at any given moment, and makes partial feedback suitable for this task.
With partial feedback, relative corrections can be independently provided for either the end-effector's motion or the gripper's actuation.
Therefore, this task enables testing CLIC's ability to learn effective policies from relative corrective feedback that is also partial. 
% The robot is also expected to react after an unsuccessful attempt by trying to grasp the ball again. 
% As a result, this task is particularly interesting for evaluating CLIC's performance, as it allows for the analysis of its ability to quickly coordinate multiple variables in a problem.
% Additionally, the necessary coordination between end-effector motion and gripper actuation makes it challenging to provide feedback on the complete action space at any given moment. 
% Therefore, this enables testing CLIC's ability to learn effective policies from partial feedback in real-world problems, where feedback is independently provided for either the end-effector's motion or the gripper's actuation.

% \textcolor{red}{add results analysis.}
 \begin{figure}
    \centering
    \includegraphics[width = 0.46\textwidth]{figs/Fig8_CatchBall02.pdf}
    % \includesvg[width=0.49\textwidth, inkscapelatex=false]{figs/Fig8_CatchBall02.svg}
	\caption{Experiment results for the ball-catching task.}
 \label{fig:Fig8_CatchBall}
\end{figure}

\begin{figure}
    \centering
    \includegraphics[width = 0.45\textwidth]{figs/Fig9_waterpouring.pdf}
    % \includesvg[width=0.49\textwidth, inkscapelatex=false]{figs/Fig9_waterpouring.svg}
	\caption{Experiment results for the water-pouring task.}
 \label{fig:Fig9_waterpouring}
\end{figure}

Fig. \ref{fig:Fig8_CatchBall} shows the experiment results of the ball-catching task, reporting the success rate of catching the ball within one, two, and three attempts. 
By the end of training, the robot achieves a 1.0 success rate for catching the ball within two attempts, and its first-attempt success rate continues to improve to 0.4.
One post-training policy rollout of a successful first-attempt catch is shown in Fig. \ref{fig:Fig8_CatchBall}, where the ball is caught within 1.5 seconds, an impressive result given the actuation delay of the robot hand.
This experiment demonstrates that CLIC can leverage both the relative corrective feedback and partial feedback effectively to learn challenging high-frequency tasks. 




% \begin{figure}
%     \centering
%     \includegraphics[width = 0.49\textwidth]{figs/Fig7_InsertT_exp_results_traj.png}
% 	\caption{Example of one CLIC-implicit policy rollout for the InsertT task. At each step, The transparent figure represents the initial state, the green arrow visualizes the trajectory of the end-effector, and the solid figure shows the corresponding end state, which becomes the initial state for the next step.}
%  \label{fig:Fig7_InsertT_exp_results_traj}
% \end{figure}



\subsubsection{Water pouring—learning full pose control with CLIC}
% In this experiment, the CLIC-Absolute and CLIC-Relative methods are combined to teach the robot, allowing the human teacher to select (by pressing a button) which teaching mode (absolute or relative correction) to use. 
% This approach showcases the flexibility of CLIC, which can learn from relative or absolute corrections depending on which is more appropriate at each moment. 
The water-pouring task requires the robot to control the pose of a bottle to precisely pour liquid (represented with marbles) into a small bowl. 
CLIC-Explicit is utilized in this experiment. 
The human teacher can provide either absolute or relative corrective feedback and has the flexibility to switch between these modes by pressing a specific keyboard button. 
Initially, absolute feedback was preferred as the policy was learned from scratch, and it was easier to intervene in a 6D action space. As the policy improved, relative corrections made it easier to refine the policy in specific regions of the state space.
% Furthermore, since CLIC allows partial feedback, the corrective signal could be provided in three ways: (1) position only, (2) rotation only, and (3) both position and rotation. This is useful when one part of the robot's action is correct while the other still needs improvement.
% An example
% of the learned policy and the results are shown in Fig. \ref{fig:Fig9_waterpouring}.

The experimental data is shown in Fig. \ref{fig:Fig9_waterpouring}. From Episode $1$ to Episode $16$, the teacher's feedback is provided in an absolute correction format. From Episode $16$ onward, the teacher's feedback is given as relative corrections to make small adjustments to the robot's policy. The success rate exhibits an overall improving trend, consistently increasing from 0.6 in Episode 26 to 0.9 by Episode 41. An example of the policy rollouts after training is illustrated in Fig. \ref{fig:Fig9_waterpouring}.
This experiment demonstrates the effectiveness of CLIC for learning precise control over position and orientation.














% \begin{figure*}[t]
%  \captionsetup{skip=0.2pt} % Adjust skip parameter for this figure
% 	\centering
% 	\includegraphics[width = 0.99\textwidth]{figs/real_exp_figs_combined_all.jpg}
%  % \vspace{2pt}
% 	\caption{Experiment results of the three real-world tasks, with success rate evolution in time and timesteps.
%  In the ball-catching task, success rates are recorded over different numbers of attempts.
%  }	\label{fig:real_exp_figs_combined_all}
% \end{figure*}

 





Software development is increasingly conceived as a collaboration activity between developers and AIs. Indeed, IDEs already implement features to enable interactive development, with AI suggesting implementations that are reused by developers.

Although multiple studies show this interaction can be successful, there is still limited understanding of how the models must be configured and used in the context of code generation tasks. This study addresses this gap, systematically investigating the impact of several key parameters, including the repeated submission of a prompt to accommodate for the non-deterministic nature of the models.

Our study reveals several key findings about the usage of ChatGPT. In particular, we discovered how creativity, although up to a limited extent, is useful to increase the range of methods whose code can be generated correctly. A major role is played by parameter top-p, which is commonly underrated, and instead has a major impact on the correctness of the results, with lower values producing better results. Finally, prompts should be submitted multiple times, with $5$ repetitions combined with a temperature of $1.2$ resulting in an effective configuration in our experiments.  

Future work concerns two main research directions. One is about replicating this experiment with other AI assistants, to validate our findings in multiple contexts. The second research direction concerns finding strategies to deal with the need to submit the same prompt multiple times to obtain a useful result, and thus developing approaches able to select or merge multiple responses automatically. 


%\bmhead{Supplementary information}

%If your article has accompanying supplementary file/s please state so here. 

%Authors reporting data from electrophoretic gels and blots should supply the full unprocessed scans for key as part of thei
%Supplementary information. This may be requested by the editorial team/s if it is missing.

%Please refer to Journal-level guidance for any specific requirements.

%\acknowledgments{Acknowledgements}
%This research was supported by Basic Science Research Program through the National Research Foundation of Korea (NRF) funded by the Ministry of Education under grant  2021R1I1A3048263 (70\%) and RS-2024-00410511 (30\%).

%Acknowledgements are not compulsory. Where included they should be brief. Grant or contribution numbers may be acknowledged.

%Please refer to Journal-level guidance for any specific requirements.

%\section*{Declarations}

%Some journals require declarations to be submitted in a standardised format. Please check the Instructions for Authors of the journal to which you are submitting to see if you need to complete this section. If yes, your manuscript must contain the following sections under the heading `Declarations':


%\begin{appendices}

%\section{Section title of first appendix}\label{secA1}

%An appendix contains supplementary information that is not an essential part of the text itself but which may be helpful in providing a more comprehensive understanding of the research problem or it is information that is too cumbersome to be included in the body of the paper.

%%=============================================%%
%% For submissions to Nature Portfolio Journals %%
%% please use the heading ``Extended Data''.   %%
%%=============================================%%

%%=============================================================%%
%% Sample for another appendix section			       %%
%%=============================================================%%

%% \section{Example of another appendix section}\label{secA2}%
%% Appendices may be used for helpful, supporting or essential material that would otherwise 
%% clutter, break up or be distracting to the text. Appendices can consist of sections, figures, 
%% tables and equations etc.

%\end{appendices}

\bibliographystyle{abbrv-doi-hyperref}
\bibliography{ref}% common bib file
%% if required, the content of .bbl file can be included here once bbl is generated
%%\input sn-article.bbl

\end{document}