\section{Overview}

\revision{In this section, we first explain the foundational concept of Hausdorff distance-based penetration depth algorithms, which are essential for understanding our method (Sec.~\ref{sec:preliminary}).
We then provide a brief overview of our proposed RT-based penetration depth algorithm (Sec.~\ref{subsec:algo_overview}).}

%\section{Preliminaries}\
\subsection{Preliminary: Hausdorff Distance-based Penetration Depth}
\label{sec:preliminary}

%In this section, we introduce the foundational concept of Hausdorff distance-based penetration depth algorithms, which are crucial for understanding our proposed ray tracing-based penetration depth methods.
%and ray tracing core technology, 

%\subsection{Hausdorff distance-based penetration depth}

The Hausdorff distance, as defined in Eq.~\ref{equation:hausdorff_definition}, plays a pivotal role in our approach.
Consider $A$ and $B$ as sets of vertices forming each \revision{object}, and let $d(\cdot, \cdot)$ denote the Euclidean distance between any two vertices.

%equation
\begin{equation}
    H(A,B) = \max \left( \max_{a \in A} \min_{b \in B} d(a,b) ,
    \max_{b \in B} \min_{a \in A} d(b,a) \right)
    \label{equation:hausdorff_definition}
\end{equation}
%

To compute penetration depth using the Hausdorff distance, the process involves several steps.
First, the overlapping volume $V$ between \revision{objects} $A$ and $B$ is computed.
Next, the surfaces of the overlapping volume, $\partial A$ and $\partial B$, contained within each \revision{object}, are extracted.
The final step involves computing the Hausdorff distance $H(\partial A, \partial B)$ between these surfaces.
The resulting distance $H(\partial A, \partial B)$ represents the penetration depth between the two objects.

%In this work, we introduce methods that utilize RT core technology for efficiently computing each step of the Hausdorff distance-based penetration depth calculation.

%\paragraph{BVH-Based Hausdorff Distance Computation:}
The brute-force computation of Hausdorff distance has a time complexity of $O(nm)$, where $n$ and $m$ represent the number of vertices in the two objects, as it requires evaluating all vertex pairs.
To reduce this computational burden, Bounding Volume Hierarchy (BVH) is employed, offering rapid localization of target polygons for distance assessment.
%
Tang et al.~\cite{SIG09HIST} constructed BVHs for two objects, $A$ and $B$.
Their approach begins by computing the Hausdorff distance from $A$ to $B$ (denoted as $h(A,B) = \max_{a \in A} \min_{b \in B} d(a,b)$) through a depth-first traversal of $BVH_A$.
Leveraging the property $h(A', B) \leq h(A, B) \leq h(A, B')$, where $A' \subseteq A$, this step determines the upper bound of the Hausdorff distance, $\overline{h}(A,B)$.
Subsequently, the lower bound $\underline{h}(A,B)$ is determined using $h(B,A)$.
This process yields an approximate Hausdorff distance bound satisfying $\overline{h}(A, B) - \underline{h}(A, B) \leq \epsilon$.
%
Building upon this method, Zheng et al.~\cite{zheng2022economic} implemented a four-point strategy, sampling four points on triangles (three vertices and one center point) to enable more efficient BVH traversal.
This approach is based on the observation that computing the distance between a triangle and a point is computationally less expensive than computing the distance between two triangles.

\revision{These prior methods, which focused on reducing the search space, achieved significant performance improvements for Hausdorff distance computation.
However, they are not well-suited for parallel processing on GPUs, as they require synchronization for updating and sharing the upper and lower bounds.}

\revision{Departing from previous methods, our approach exploits parallel processing on GPUs while leveraging the intrinsic capabilities of RT cores, with a unique emphasis on the ray-triangle intersection test, which is significantly accelerated by the RT core.
In alignment with Tang et al.~\cite{SIG09HIST}, our method approximates the Hausdorff distance.
However, we place greater emphasis on a ray sampling strategy designed to balance accuracy and performance.
Additionally, while previous methods focused on Hausdorff distance computation and presented penetration depth as an application, we accelerate the entire penetration depth computation process on the GPU.}

%Our approach, leveraging the intrinsic capabilities of the RT-core, also employs BVH, but with a unique emphasis on the ray-triangle intersection test, significantly accelerated by the RT-core.
%This strategy represents a departure from previous BVH-based Hausdorff methods, which predominantly focused on reducing search space by computing upper and lower bounds that requires synchronization process for updating the bounds while it is not appropriate for parallel processing.

%improving traversal efficiency during distance computation.

%This method has achieved a performance improvement of up to 20 times compared to Tang et al.'s technique~\cite{SIG09HIST}.

%This strategy involves sampling four points on triangles (three vertices and one center point) to facilitate BVH traversal.


%General algorithms for calculating the Hausdorff distance, when implemented without any additional culling strategies, exhibit a time complexity of at least $O(n^2)$ for triangular meshes~\cite{}. In the huge scenario, there is too much time cost.
%Tang et al. ~\cite{SIG09HIST} proposed the BVH-based framework for reducing the time complexity.
%They build the BVH(Bounding volume hierarchy) for model $A$ and $B$ and traverse the BVH tree of model $A$ in depth-first order. At the first traverse sequence, this algorithm gets sorted by the upper bound of the Hausdorff distance to Model $B$. Since $h(A', B) \leq h(A, B) \leq h(A, B')$ when $A' \subseteq A$ and $B' \subseteq B$, they use this upper bound to compute lower bound when reduction process. In the reduction process, they find the suitable primitive triangles to reduce the Hausdorff distance bound $\overline{h}(A, B) - \underline{h}(A, B) \leq \epsilon$. Finally, they achieve the approximated Hausdorff distance $h(A, B)$ in error bound $\epsilon$.

%Tang's method, Zheng's method
%\subsubsection{Acceleration of Hausdorff distance}

%\subsection{Ray-Tracing Core}
%\DS{Will move to realted work section}

%The Ray-Tracing Core (RT-core) is a specialized hardware component developed by NVIDIA, specifically designed for accelerating ray tracing-based rendering.
%This technology is a key feature of NVIDIA's RTX platform GPUs, such as the GeForce RTX series.
%At the heart of ray tracing is the fundamental operation of intersection checking between rays and objects.
%The RTX platform enhances this process by providing RT-core-based functionalities for efficient ray-bounding box and ray-triangle intersection tests.

%To exploit this powerful hardware for penetration depth computation, we designed RT-core based algorithms for two core tasks: firstly, to extract the overlapping region between two objects, and secondly, to calculate the Hausdorff distance for penetration depth computation.

%The ray tracing core(RT-core) is a specialized hardware on the RTX platform GPU (e.g., GeForce RTX) that was developed by NVIDIA for acceleration of ray tracing-based rendering.
%The basic operation of ray tracing is an intersection test between the ray-bounding box or ray-surface, RT core has a powerful performance up to two to four times higher than CUDA-based computation performance in these operations.

%In this work, to compute the penetration depth, we change the two main algorithms to ray-intersection-based operation to use these powerful RTX platforms.


\subsection{Overview of the RTPD Algorithm}\label{subsec:algo_overview}
Fig.~\ref{fig:Overview} presents an overview of our RTPD algorithm.
It is grounded in the Hausdorff distance-based penetration depth calculation method (Sec.~\ref{sec:preliminary}).
%, similar to that of Tang et al.~\shortcite{SIG09HIST}.
The process consists of two primary phases: penetration surface extraction and Hausdorff distance calculation.
We leverage the RTX platform's capabilities to accelerate both of these steps.

\begin{figure*}[t]
    \centering
    \includegraphics[width=0.8\textwidth]{Image/overview.pdf}
    \caption{The overview of RT-based penetration depth calculation algorithm overview}
    \label{fig:Overview}
\end{figure*}

The penetration surface extraction phase focuses on identifying the overlapped region between two objects.
\revision{The penetration surface is defined as a set of polygons from one object, where at least one of its vertices lies within the other object. 
Note that in our work, we focus on triangles rather than general polygons, as they are processed most efficiently on the RTX platform.}
To facilitate this extraction, we introduce a ray-tracing-based \revision{Point-in-Polyhedron} test (RT-PIP), significantly accelerated through the use of RT cores (Sec.~\ref{sec:RT-PIP}).
This test capitalizes on the ray-surface intersection capabilities of the RTX platform.
%
Initially, a Geometry Acceleration Structure (GAS) is generated for each object, as required by the RTX platform.
The RT-PIP module takes the GAS of one object (e.g., $GAS_{A}$) and the point set of the other object (e.g., $P_{B}$).
It outputs a set of points (e.g., $P_{\partial B}$) representing the penetration region, indicating their location inside the opposing object.
Subsequently, a penetration surface (e.g., $\partial B$) is constructed using this point set (e.g., $P_{\partial B}$) (Sec.~\ref{subsec:surfaceGen}).
%
The generated penetration surfaces (e.g., $\partial A$ and $\partial B$) are then forwarded to the next step. 

The Hausdorff distance calculation phase utilizes the ray-surface intersection test of the RTX platform (Sec.~\ref{sec:RT-Hausdorff}) to compute the Hausdorff distance between two objects.
We introduce a novel Ray-Tracing-based Hausdorff DISTance algorithm, RT-HDIST.
It begins by generating GAS for the two penetration surfaces, $P_{\partial A}$ and $P_{\partial B}$, derived from the preceding step.
RT-HDIST processes the GAS of a penetration surface (e.g., $GAS_{\partial A}$) alongside the point set of the other penetration surface (e.g., $P_{\partial B}$) to compute the penetration depth between them.
The algorithm operates bidirectionally, considering both directions ($\partial A \to \partial B$ and $\partial B \to \partial A$).
The final penetration depth between the two objects, A and B, is determined by selecting the larger value from these two directional computations.

%In the Hausdorff distance calculation step, we compute the Hausdorff distance between given two objects using a ray-surface-intersection test. (Sec.~\ref{sec:RT-Hausdorff}) Initially, we construct the GAS for both $\partial A$ and $\partial B$ to utilize the RT-core effectively. The RT-based Hausdorff distance algorithms then determine the Hausdorff distance by processing the GAS of one object (e.g. $GAS_{\partial A}$) and set of the vertices of the other (e.g. $P_{\partial B}$). Following the Hausdorff distance definition (Eq.~\ref{equation:hausdorff_definition}), we compute the Hausdorff distance to both directions ($\partial A \to \partial B$) and ($\partial B \to \partial A$). As a result, the bigger one is the final Hausdorff distance, and also it is the penetration depth between input object $A$ and $B$.


%the proposed RT-based penetration depth calculation pipeline.
%Our proposed methods adopt Tang's Hausdorff-based penetration depth methods~\cite{SIG09HIST}. The pipeline is divided into the penetration surface extraction step and the Hausdorff distance calculation between the penetration surface steps. However, since Tang's approach is not suitable for the RT platform in detail, we modified and applied it with appropriate methods.

%The penetration surface extraction step is extracting overlapped surfaces on other objects. To utilize the RT core, we use the ray-intersection-based PIP(Point-In-Polygon) algorithms instead of collision detection between two objects which Tang et al.~\cite{SIG09HIST} used. (Sec.~\ref{sec:RT-PIP})
%RT core-based PIP test uses a ray-surface intersection test. For purpose this, we generate the GAS(Geometry Acceleration Structure) for each object. RT core-based PIP test takes the GAS of one object (e.g. $GAS_{A}$) and a set of vertex of another one (e.g. $P_{B}$). Then this computes the penetrated vertex set of another one (e.g. $P_{\partial B}$). To calculate the Hausdorff distance, these vertex sets change to objects constructed by penetrated surface (e.g. $\partial B$). Finally, the two generated overlapped surface objects $\partial A$ and $\partial B$ are used in the Hausdorff distance calculation step.