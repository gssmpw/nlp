\abstract{
Penetration depth calculation quantifies the extent of overlap between two objects and is crucial in fields like simulations, the metaverse, and robotics.
Recognizing its significance, efforts have been made to accelerate this computation using parallel computing resources, such as CPUs and GPUs.
Unlike traditional GPU cores, modern GPUs incorporate specialized ray-tracing cores (RT-cores) primarily used for rendering applications.
We introduce a novel algorithm for penetration depth calculation that leverages RT-cores.
Our approach includes a ray-tracing based algorithm for penetration surface extraction and another for calculating Hausdorff distance, optimizing the use of RT-cores.
We tested our method across various generations of RTX GPUs with different benchmark scenes.
The results demonstrated that our algorithm outperformed a state-of-the-art penetration depth calculation method and conventional GPU implementations by up to 37.66 and 5.33 times, respectively. 
These findings demonstrate the efficiency of our RT core-based method and suggest broad applicability for RT-cores in diverse computational tasks.
}
  %
  %% We recommend that you link to your supplemental material here in the abstract, as well
  %% as in the Supplemental Materials section at the end.
  %A free copy of this paper and all supplemental materials are available at \url{https://OSF.IO/2NBSG}.