\section{Conclusion}

We introduced RTPD, a novel algorithm for calculating penetration depth using hardware-accelerated ray-tracing cores (RT-cores).
Our approach leverages the specialized capabilities of RT-cores to efficiently perform penetration surface extraction and Hausdorff distance calculations.
We also presented a GPU-based algorithm for generating penetration surfaces, ensuring that RTPD operates entirely on GPU platforms.
Through extensive testing on various generations of RTX GPUs and benchmark scenes, our algorithm demonstrated significant performance improvements, outperforming a state-of-the-art penetration depth method and conventional GPU implementations by up to 37.66 times and 5.33 times, respectively.

These results highlight the potential of RT-cores beyond their traditional rendering applications, suggesting broad applicability in diverse computational tasks, including simulations, the metaverse, and robotics. 
The efficiency and scalability of RTPD make it a promising solution for applications that require precise and rapid penetration depth calculations.

\textbf{Limitations and Future Work:}
Since our RT-based Hausdorff distance computation uses a sampling method, the results may not match exact penetration depth values.
Future work will focus on finding more efficient sampling strategies to reduce errors further.
We also aim to develop RT-based penetration depth algorithms that guarantee exact results.
Additionally, we plan to explore the application of RT-core-based algorithms in other domains, expanding their use in various computational tasks.


\Skip{
In this work, we propose penetration depth measurement algorithms using RT cores.
TO compute the penetration depth and accelerate it on the RTX platform, we employed a Hausdorff-based penetration depth measurement algorithm and implemented it to separate into three steps on the RTX platform.
We changed the algorithms to RT-based point-in-polygon algorithms for penetration surface detection algorithms because the collision detection and hole-filling algorithms are too expensive and not suitable for using RT core. 
%\TODO{Do we need to implement the CPU-side penetration surface detection using collision detection and hole-filling?}
And we implemented the GPU-based penetration surface generation algorithms to compute the Hausdorff distance, due to this, we obtained the benefit of the effect of pipeline optimization to run pipelines on GPUs only.
Then, we proposed the RT-based Hausdorff distance calculation algorithm using ray-sampling methods.
We implemented the methods on the four different GPU and one CPU computing systems and compared the performance between our method and CPU-based penetration depth measurement methods. Also, we examined and reported how efficient each step of the proposed method was than the CPU-based approach.
As a result, our RT-based penetration depth measurement algorithms obtained the performance up to \ToCheck{XX} times with an average error rate \ToCheck{XX}\% for a huge polygon model.

\paragraph{Limitation:} While the previous study computed the penetration depth for the dynamic scene, Our proposed method is tested based on static scenes. Due to this, we had limitations in computing the acceleration structure and reporting the results every time for each frame of dynamic scenes. 
And we discovered our proposed RT-based Hausdorf distance algorithm rather performs worse than conventional algorithms when the number of vertices on the penetrating surface becomes less than the number of samples.
~\\

\paragraph{Future work:}

To apply the method to generalized simulation, We will expand our algorithms to perform in dynamic scenes.
And we also intend to study memory-efficient algorithms so that the proposed method works well for huge data.

~\\
}