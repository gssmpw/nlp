\section{Problem Definition}
Aiming to enhance the capability of LLMs in handling knowledge stored in structured data with hypergraphs, in this paper, we consider tables as the structured data sources to illustrate our \name framework. We construct a hypergraph with the structured knowledge in table $\mathcal{T}$. Each table is formally represented as $\mathcal{T}=\{o, h_i, v_{m,n}|0\leq i \leq N,0\leq m \leq M,0\leq n\geq N\}$, where $o$ is the table caption, $h_i$ represents the header for the $i^{th}$ column, $v_{m,n}$ represents the cell at the $m^{th}$ row (denoted as $r_{m}\in\mathcal{R}$), and the $n^{th}$ column (denoted as $c_{n}\in\mathcal{C}$). As depicted in the upper left of Figure \ref{fig:method}, the very upper-left cell is denoted as $v_{0,0}$. The task description prompt $x$, provided in natural language to the LLMs, includes a textual representation of the table $\mathcal{T}$ (\eg{in markdown format}) and the essential inquiry $\omega$ regarding this table, following a specific template, \ie{$\omega\subset x$}. Specifically, the essential inquiry $\omega$ can be claims in fact verification or questions in question answering. For tasks requiring the knowledge stored in $\mathcal{T}$, we aim to help pretrained LLMs (denoted as $LLM(\cdot)$) to understand and extract the structured knowledge relevant to the inquiry $\omega$ stored in $\mathcal{T}$, thereby improving the effectiveness of LLM's final generations. 

% \noindent
% \begin{problem}[Enhancing LLM on Structured Knowledge]
% For each task $t$, given the Large Language Model $LLM(\cdot)$, structured data $\mathcal{T}$, we aim at finding a learner $f$ to extract the relavent structured knowledge from $\mathcal{T}$ to enhance the capability of LLM.
% \end{problem}
