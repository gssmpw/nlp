\section{The EMI model} \label{sec:emi}
As in \cite{2D-proof}, we consider $N$ myocytes immersed in extracellular liquid, together forming the cardiac tissue $\Omega$ while focusing on the case that $\Omega \subset \R^3$. We assign the extracellular subdomain to be $\Omega_0$ and each of the $N$ cardiac cells to be separate subdomains $\Omega_1,\dots,\Omega_N$, interacting with the surrounding extracellular space through ionic currents and with their neighboring myocytes via gap junctions, which are special protein channels allowing for the passage of ions directly between two cells \cite{ionic}. We assume a partition of the cardiac tissue $\Omega$ into $N + 1$ non-overlapping subdomains $\Omega_i$ such that $\overline \Omega = \cup_{i = 0}^N \overline \Omega_i$, $\Omega_i \cap \Omega_j = \emptyset$ if $i \neq j$.

\begin{figure}
    \centering
    \includegraphics[width=0.5\textwidth]{figures/two_cells}
    \caption{Visualization of two cells $\Omega_1$ (green) and $\Omega_2$ (blue) floating in extracellular liquid $\Omega_0$ (grey). For this three-dimensional example, we have to consider interface terms for ionic currents between extracellular space and the cells over $F_{01} = \partial\Omega_0 \cap \partial\Omega_1$ and $F_{02} = \partial \Omega_0 \cap \partial \Omega_1$ and ionic currents through gap junctions between the cells over $F_{12} = \partial\Omega_1\cap\partial\Omega_2$. For the BDDC preconditioner, it is important to also consider edge terms on $E_{0, \{1, 2\}} = \partial\Omega_0\cap\partial\Omega_1\cap\partial\Omega_2$.}
    \label{fig:two_cells}
\end{figure}

The EMI model is described by the equations
\begin{equation}
\label{eq:emi}
    \begin{cases}
        -\text{div}(\sigma_i \nabla u_i) = 0 & \text{in } \Omega_i, i = 0,\dots,N \\
        -n_i^T\sigma_i\nabla u_i = C_m\frac{\partial v_{ij}}{\partial t} + F(v_{ij}, c, w) & \text{on } F_{ij} = \partial\Omega_i \cap \partial\Omega_j, i \neq j \\
        n^T\sigma_i\nabla u_i = 0 & \text{on } \partial\Omega_i \cap \partial\Omega \\
        \frac{\partial c}{\partial t}-C(v_{ij}, w, c) = 0, \quad \frac{\partial w}{\partial t}-R(v_{ij},w)=0&
    \end{cases}
\end{equation}
with conductivity coefficients $\sigma_i$ in $\Omega_i$, outward normals $n_i$ on $\partial \Omega_i$ and membrane capacitance for unit area $C_m$ on the membrane surface. $v_{ij} = u_i - u_j$ describes the transmembrane voltage, i.e. the discontinuities of the electric potentials between to neighboring subdomains and $F(v_{ij}, c, w)$ stands for either ionic current $I_{\text{ion}}(v_{ij},c,w)$ or gap junction current $G(v_{ij})$, depending on whether the two neighbors are both cells or one is the extracellular domain. We assume $G(v)$ to be linear in the potential jumps $v$ and note that generally, $F(v_{ij}, c, w) = - F(v_{ji}, c, w)$. The last two terms model the ion flow with ordinary differential equations describing the time evolution of ion concentrations $c$ and gating variables $w$. More details on derivation and analysis of the EMI model can be found in \cite{EMI-1, EMI-4, veneroni2006micro}.

We note that since the solution of (\ref{eq:emi}) is only unique up to a constant, we require a zero average on the extracellular solution $u_0$. Additionally, we mention that we consider a splitting strategy in time for the solution of (\ref{eq:emi}), first solving the ionic model with jumps $v_{ij}$ known from the previous time step and then updating the model with the solutions $c$ and $w$ and solving it for the electric potential. For brevity, we will from now on write $F(v_{ij}) \coloneqq F(v_{ij}, c, w)$, omitting $c$ and $w$.

\subsection{Weak Formulation}

On each subdomain $\Omega_i$, integrating the first equations in (\ref{eq:emi}) by parts and substituting in the second equation on the cell membrane, the $i$-th sub-problem reads: find $u_i \in H^1(\Omega_i)$ such that for all $\phi_i \in H^1(\Omega_i)$ the following holds:
\begin{equation*}
    \begin{split}
        0 & = -\int_{\Omega_i}\text{div}(\sigma_i\nabla u_i) \phi_idx \\
        & = \int_{\Omega_i}\sigma_i\nabla u_i \nabla\phi_idx - \int_{\partial \Omega_i} n_i^T\sigma_i\nabla u_i \phi_ids \\
        & = \int_{\Omega_i}\sigma_i\nabla u_i \nabla\phi_idx + \sum_{i \neq j}\int_{F_{ij}} \big(C_m\frac{\partial v_{ij}}{\partial t} + F(v_{ij})\big)\phi_ids.
    \end{split}
\end{equation*} 

Summing up the contributions of all subdomains, we get the global problem
\[\sum_{i = 0}^N\int_{\Omega_i}\sigma_i\nabla u_i\nabla\phi_idx + \frac{1}{2} \sum_{i = 0}^N\sum_{j \neq i}\int_{F_{ij}}\big(C_m\frac{\partial[\![u]\!]_{ij}}{\partial t} + F([\![u]\!]_{ij})\big)[\![\phi]\!]_{ij}ds = 0,\]
where $[\![u]\!]_{ij} = u_i - u_j$ denotes the jump in value of the electric potential $u_i$ and its neighboring $u_j$ from the subdomain $\Omega_j$ along the boundary face $F_{ij}\subset \partial\Omega_i$.

\subsection{Space and Time Discretization}

Let $V_i(\overline{\Omega}_i)$ be the regular finite element space of piece-wise linear functions in $\overline{\Omega}_i$ and define the global finite element space as $V(\Omega) \coloneqq V_0(\overline{\Omega}_0)\times\cdots\times V_N(\overline{\Omega}_N)$. Similar as in \cite{sarkis-3D} we denote:
\begin{itemize}
    \item The \textbf{subdomain face} shared between subdomains $\Omega_i$ and $\Omega_j$ is symbolized by $\overline{F}_{ij} \coloneqq \partial \Omega_i \cap \partial \Omega_j$. We note that geometrically speaking, $\overline{F}_{ij}$ and $\overline{F}_{ji}$ are identical, but to allow for different triangulations on either subdomain \cite{sarkis-3D}, we treat them separately.
    \item $\F_i^0$ describes the set of indices $j$ for which $\Omega_i$ and $\Omega_j$ share a face $F_{ij}$ with non-vanishing two-dimensional measure.
    \item $\N_x$ refers to the set of indexes of subdomains with $x$ in the closure of the subdomain.
\end{itemize}
Just like in \cite{2D-proof}, we consider an implicit-explicit (IMEX) time discretization scheme, treating the diffusion term implicitly and the reaction term explicitly. We split the time interval $[0,T]$ into $K$ intervals. With $\tau = t^{k + 1} - t^k, k = 0,\dots,K$ we derive the following scheme:
\begin{equation*}
    \begin{split}
        \frac{1}{2}\sum_{i = 0}^N\sum_{j \in \F_i^0}\int_{F_{ij}}C_m\frac{[\![u^{k + 1}]\!]_{ij} - [\![u^k]\!]_{ij}}{\tau}[\![\phi]\!]_{ij}ds + \sum_{i = 0}^N\int_{\Omega_i}\sigma_i\nabla u_i^{k+1}\nabla\phi_idx \\
        =-\frac{1}{2}\sum_{i=0}^N\sum_{j \in \F_i^0}\int_{F_{ij}}F([\![u^k]\!]_{ij})[\![\phi]\!]_{ij}ds.
    \end{split}
\end{equation*}

Rearranging the terms such that we only have $[\![u^{k+1}]\!]_{ij}$ on the left hand side, we get
\begin{equation}
\begin{split}
\frac{1}{2}\sum_{i=0}^N\sum_{j\in \F_i^0}\int_{F_{ij}}C_m[\![u^{k+1}]\!]_{ij}[\![\phi]\!]_{ij}ds + \tau \sum_{i=0}^N\int_{\Omega_i}\sigma_i\nabla u_i^{k+1}\nabla\phi_idx \\
=\frac{1}{2}\sum_{i=0}^N\sum_{j\in \F_i^0}\int_{F_{ij}}C_m[\![u^k]\!]_{ij}[\![\phi]\!]_{ij}ds - \frac{1}{2}\tau\sum_{i=0}^N\sum_{j \in \F_i^0}\int_{F_{ij}}F([\![u^k]\!]_{ij})[\![\phi]\!]_{ij}ds.
\end{split}
\end{equation}

This now lets us define the following local (bi-)linear forms:

\begin{equation}
\label{eq:bilinear-forms}
\begin{split}
    a_i(u_i, \phi_i) & \coloneqq \int_{\Omega_i} \sigma_i \nabla u_i \nabla \phi_i dx, \\
    p_i(u_i, \phi_i) & \coloneqq \frac{1}{2} \sum_{j \neq i} \int_{F_{ij}} C_m [\![u]\!]_{ij}[\![\phi]\!]_{ij} ds, \\
    f_i(\phi_i) & \coloneqq \frac{1}{2} \sum_{j \neq i} \int_{F_{ij}} (C_m [\![u^k]\!]_{ij}[\![\phi]\!]_{ij} - \tau F([\![u^k]\!]_{ij})[\![\phi]\!]_{ij}) ds, \\
    d_i(u_i, \phi_i) & \coloneqq \tau a_i(u_i, \phi_i) + p_i(u_i, \phi_i).
\end{split}
\end{equation}

The global problem now reads: Find $u = \{u_i\}_{i=0}^N \in V(\Omega)$ such that
\begin{equation} \label{eq:bilinear-global}
    d_h(u, \phi) = f(\phi), \quad \forall \phi = \{\phi_i\}_{i=0}^N \in V(\Omega),
\end{equation}

where $d_h(u, \phi) \coloneqq \sum_{i=0}^Nd_i(u_i,\phi_i) = \sum_{i=0}^N(\tau a_i(u_i,\phi_i) + p_i(u_i,\phi_i))$ and $f(\phi) \coloneqq \sum_{i=0}^Nf_i(\phi_i)$. With local stiffness matrices $A_i$ and mass matrices $M_i$, (\ref{eq:bilinear-global}) corresponding to $a_i$ and $p_i$, respectively, can be written in matrix form:
\begin{equation} \label{eq:matrix-form}
    Ku=f, \quad \text{where } K = \sum_{i=0}^NK_i, \quad K_i = \tau A_i + M_i.
\end{equation}