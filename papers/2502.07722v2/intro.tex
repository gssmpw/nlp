\section{Introduction}

This work aims at generalizing a previous convergence analysis of BDDC preconditioners for cardiac cell-by-cell models \cite{2D-proof} to three spatial dimensions. Such models on the microscopic level have been studied over the past years in order to understand events in aging and structurally diseased hearts which macroscopic Monodomain and Bidomain models \cite{potse2006comparison, bidomain-1, bidomain-2} that are based on a homogenized description of the cardiac tissue fail to adequately represent. In such events, reduced electrical coupling leads to large differences in behavior between neighboring cells, which requires careful modeling of each individual cardiac cell.

We consider the EMI (Extracellular space, cell Membrane and Intracellular space) model, which has been introduced and analyzed in \cite{potse2017cinc, EMI-3, EMI-2, EMI-1, EMI-4, EMI-5} and has been at the core of the EuroHPC project MICROCARD \cite{microcard}. Throughout the course of this project, Balancing Domain Decomposition by Constraints (BDDC) preconditioners (we refer to \cite{pechstein2017bddc, dd-book} for extensive explanations of such algorithms) have been identified to be an efficient choice for composite Discontinuous Galerkin (DG) type discretizations of cardiac cell-by-cell models and a theoretical analysis of this method has been presented in \cite{2D-proof}, while a preliminary study on its integration with time-stepping methods can be found in \cite{chegini2023coupled}. Here, a key step was the careful construction of extended dual and primal spaces for the degrees of freedom, allowing for continuous mapping in the BDDC splitting while still honoring the discontinuities across cell boundaries as they occur in the EMI model. Up to now, the theoretical analysis in three dimensions remains an open question, as the introduction of edge terms into the primal space requires additional constraints. In this work, we leverage results from \cite{sarkis-3D} as well as standard sub-structuring theory arguments from \cite{dd-book} in order to close this gap. Other paths for the solution of EMI models have been investigted, ranging from boundary element methods \cite{bem2024} as discretization choice, to multigrd solvers \cite{budivsa2024algebraic}, overlapping Schwarz preconditioners \cite{huynh2025gdsw} and other iterative solvers build upon ad-hoc spectral analysis \cite{benedusi2024modeling}. Other approaches to domain decomposition preconditioning for DG type problems have been studied e.g. in \cite{antonietti2007esaim} and \cite{ayuso2014multilevel}.

For the sake of completeness, we will first give an overview of a simplified EMI model and its time and space discretizations in \Cref{sec:emi}, referring to \cite{EMI-1, EMI-4, EMI-5} for details on the EMI model and to \cite{2D-proof} for the derivation of the time and space composite-DG discretizations. We introduce the finite element spaces, describe the BDDC preconditioner for this application, and recall the Lemmas from literature necessary to prove its convergence in \Cref{spaces} before providing a convergence proof, the main contribution of this paper, in \Cref{sec:proof}. Finally, \Cref{results} presents a numerical study supporting the derived theory with practical results.

