\section{BDDC and Function Spaces} \label{spaces}

In this section, we will first construct the function spaces necessary for our preconditioner, extending the theory in \cite{2D-proof} to the three-dimensional case. Next, we will construct a family of BDDC preconditioners with different classes of primal constraints, building on results from \cite{sarkis-2D, sarkis-2D-deluxe, sarkis-3D,2D-proof} before concluding the section with some technical tools required for the proof of a condition number bound in the next section.

\subsection{Function Spaces}
\label{sec:spaces}
Considering a family of partitionings such that each subdomain is the union of shape-regular conforming finite elements, we define the global interface $\Gamma$ as the set of points belonging to at least two subdomains:
\[\Gamma \coloneqq \bigcup_{i = 0}^N \Gamma_i, \quad \Gamma_i \coloneqq \overline{\partial\Omega_i\setminus\partial\Omega}.\]
As in \cite{2D-proof}, we consider discontinuous Galerkin type discretizations in order to correctly treat jumps of the electric potentials along the cell membrane and gap junctions of each myocyte, which means that the degrees of freedom on the interface $\Gamma$ will have a multiplicity depending on the number of subdomains that share them.

We further assume that the finite elements are of diameter $h$ and that all subdomains are shape regular with a characteristic diameter $H_i$. We denote $H = \max_i H_i$.

Considering the multiplicity of degrees of freedom on the interface, we denote with $\Omega_i' = \overline{\Omega}_i \cup \bigcup_{j \in \F_i^0} \overline{F}_{ji}$ the union of nodes in $\Omega_i$ and on faces $\overline{F}_{ji} \subset \partial \Omega_j$, $j \in \F_i^0$ and we define the according local finite element spaces
\[W_i(\Omega_i') \coloneqq V_i(\overline{\Omega}_i) \times \prod_{j \in \F_i^0} W_i(\overline{F}_{ji}).\]
Here, $W_i(\overline{F}_{ji})$ is the trace of the space $V_j(\overline{\Omega}_j)$ on $F_{ji} \subset \partial \Omega_j$ for all $j \in \F_i^0$. See \Cref{fig:fe-space} for a visualization of $W_i(\Omega_i')$. Each function $u_i \in W_i(\Omega_i')$ in this space can then be written as 
\begin{equation}
    u_i = \{(u_i)_i, \{(u_i)_j\}_{j \in \F_i^0} \}
    \label{u_1}
\end{equation}
where $(u_i)_i$ and $(u_i)_j$ are the restrictions of $u_i$ to $\overline{\Omega}_i$ and $\overline{F}_{ji}$, respectively.

\begin{figure}
    \centering
    \includegraphics[width=.7\textwidth]{figures/FE_space}
    \caption{Schematic visualization of the FE space for a tetrahedral substructure $\Omega_i$ (grey) with two neighboring substructures. $u_i \in W(\Omega_i')$ will consist of $(u_i)_i$ on the substructure $\Omega_i$ and of $(u_i)_j$ and $(u_i)_k$, the traces of $V_{\{j, k\}}(\Omega_{\{j, k\}})$ on the faces $F_{ji}$ (green) and $F_{ki}$ (blue), respectively.}
    \label{fig:fe-space}
\end{figure}

Again as in \cite{2D-proof}, we partition $W_i(\Omega_i')$ into its interior part $W_i(I_i)$ and the finite element trace space $W_i(\Gamma_i')$, so
\[W_i(\Omega_i') = W_i(I_i) \times W_i(\Gamma_i')\]
where $\Gamma_i' = \Gamma_i \cup \{\bigcup_{j \in \F_i^0} \overline{F}_{ji}\}$ denotes the local interface nodes in $\Omega_i'$. This allows a representation of (\ref{u_1}) as
\begin{equation}
    u_i = (u_{i,I}, u_{i,\Gamma'}),
\end{equation}
where $u_{i,I}$ represents the values of $u_i$ at the interior nodes on $I_i$ and $u_{i, \Gamma'}$ denotes the values at the nodes on $\Gamma_i'$. We consider the product spaces
\[W(\Omega') \coloneqq \prod_{i = 0}^N W_i(\Omega_i'), \quad W(\Gamma') \coloneqq \prod_{i = 0}^N W_i(\Gamma_i').\]
Here, $u \in W(\Omega')$ means that $u = \{u_i\}_{i = 0}^N$ with $u_i \in W_i(\Omega_i')$ and similalrly $u_{\Gamma'} \in W(\Gamma')$ means that $u_{\Gamma'} = \{u_{i, \Gamma'}\}_{i = 0}^N$ with $u_{i, \Gamma'} \in W_i(\Gamma_i')$ where $\Gamma' = \prod_{i = 0}^N \Gamma_i'$ denotes the global broken interface.

\textbf{Subdomain Edges} will be denoted by $E_{ijk} \coloneqq \partial F_{ij} \cap \partial F_{ik}$ for two faces $F_{ij}$ and $F_{ik}$ of $\Omega_i$. Similar to the faces, we will treat the three geometrically identical edges $E_{ijk}$, $E_{jik}$ and $E_{kij}$ separately and we define $\E_i^0 \coloneqq \{(j, k)|E_{ijk} \text{ is an edge of } \Omega_i\}$.

Finally we define \textbf{Subdomain Vertices} as $\V_i \coloneqq \{\cup_{(j,k) \in \E_i^0} \partial E_{ijk}\}$ and $\V_i'$ as the union of $\V_i$ with all vertices from other subdomains that $\Omega_i$ has a share in. We say that $u = \{u_i\}_{i = 0}^N \in W(\Omega')$ is continuous at the corners $\V_i$ if
\[(u_i)_i(x) = (u_j)_i(x) \text{ at all } x \in \V_i \text{ for all } j \in \N_x.\]


\begin{definition}
    The discrete harmonic extension $\HH_i'$ in the sense of $d_i$ as defined in (\ref{eq:bilinear-forms}) is defined as
    \begin{equation*}
    \HH_i': W_i(\Gamma') \rightarrow W_i(\Omega_i'), 
    \begin{cases}
        d_i(\HH_i'u_{i,\Gamma'}, v_i) = 0 & \forall v_i \in W_i(\Omega_i') \\
        \HH_i'u_{i,\Gamma'} = (u_i)_i & \text{on } \partial \Omega_i \\
        \HH_i'u_{i,\Gamma'} = (u_i)_j & \text{on } F_{ji} \subset \partial \Omega_j \\
        & \text{and on } E_{j, \E} \subset \partial \Omega_j.
    \end{cases}
    \end{equation*}
\end{definition}

With a notion of faces, edges, and vertices, we can define the function spaces relevant for our preconditioner as in \cite{dd-book, sarkis-3D}:

\begin{definition}[Subspaces $\widetilde{W}(\Omega')$ and $\widetilde{W}(\Gamma')$]
\label{sec2:VEF}
We define the space $\widetilde{W}(\Omega')$ as the subspace of functions $u = \{u_i\}_{i=0}^N \in W(\Omega')$ for which the following conditions hold for all $0 \leq i \leq N$:
\begin{itemize}
    \item $u$ is continuous at all corners $\V_i$.
    \item On all edges $E_{ijk}$ for $(j,k) \in \E_i^0$
    \[(\overline{u}_i)_{i,E_{ijk}} = (\overline{u}_j)_{i,E_{ijk}} = (\overline{u}_k)_{i, E_{ijk}}.\]
    \item On all faces $F_{ij}$ for $j \in \F_i^0$
    \[(\overline{u}_i)_{i,F_{ij}} = (\overline{u}_j)_{i,F_{ij}}.\]
\end{itemize}

Here, 
\[(\overline{u}_j)_{i,E_{ijk}} = \frac{1}{|E_{ijk}|} \int_{E_{ijk}}(u_j)_i ds, \quad (\overline{u}_j)_{i,F_{ij}} = \frac{1}{|F_{ij}|} \int_{F_{ij}}(u_j)_i ds.\]
We denote with $\widetilde{W}(\Gamma')$ the subspace of $\widetilde{W}(\Omega')$ of functions which are discrete harmonic in the sense of $\HH_i'$.
\end{definition}

\begin{figure}
    \centering
    \includegraphics[width=.5\textwidth]{figures/primal_DOFs.png}
    \caption{Representation of primal constraints connected to a tetrahedral substructure $\Omega_i$ (grey) surrounded by three neighbors (red, green and blue), with the fourth face intersecting with $\partial \Omega$ (the global Neumann boundary). For each face, edge and vertex, one primal constraint per involved substructure is created and on each of them, the according averaging constraints are imposed. In this particular example, $\Omega_i$ will contribute to the primal space with 6 face average, 9 edge average and 4 vertex constraints.}
    \label{fig:primal-constraints}
\end{figure}

As indicated in a simplified way by \Cref{fig:primal-constraints}, the primal space $\widetilde{W}(\Omega')$ can grow in dimension very quickly, which increases the computational cost of setting up a BDDC preconditioner dramatically, making it desirable to reduce the amount of primal constraints. We therefore also consider the following setting, removing the face averages from the primal space:

\begin{definition}[Subspaces $\widetilde{W}_{VE}(\Omega')$ and $\widetilde{W}_{VE}(\Gamma')$]
\label{sec2:VE}
We define the space \newline
$\widetilde{W}_{VE}(\Omega')$ as the subspace of functions $u = \{u_i\}_{i=0}^N \in W(\Omega')$ for which the following conditions hold for all $0 \leq i \leq N$:
\begin{itemize}
    \item $u$ is continuous at all corners $\V_i$.
    \item On all edges $E_{ijk}$ for $(j,k) \in \E_i^0$
    \[(\overline{u}_i)_{i,E_{ijk}} = (\overline{u}_j)_{i,E_{ijk}} = (\overline{u}_k)_{i, E_{ijk}}.\]
\end{itemize}

We define $\widetilde{W}_{VE}(\Gamma')$ in the same way as $\widetilde{W}(\Gamma')$ above.
\end{definition}

A function $u \in \widetilde W(\Omega')$ (or $u \in \widetilde{W}_{VE}(\Omega')$) can be represented as $u = (u_I, u_{\Delta}, u_{\Pi})$ where $I = \prod_{i = 0}^N I_i$ represents the degrees of freedom on \textit{interior} nodes, $\Pi$, which we will call \textit{primal}, denotes degrees of freedom at the vertices $\V_i'$ and the average face and edge values (or only average edge values in the case of $\widetilde{W}_{VE}(\Omega')$). $\Delta$ refers to the remaining nodal degrees of freedom on $\Gamma_i' \setminus \V_i'$ with zero interface averages for the involved interfaces. We will refer to them as \textit{dual}.

Introducing the spaces
\[W_{\Delta}(\Gamma') = \prod_{i = 0}^N W_{i, \Delta}(\Gamma_i') \quad\text{and}\quad \widetilde W_{\Pi}(\Gamma')\]
% \mw{Is there the defining term for the right term missing? Yes, support on primal constraints, continuous on vertices, discrete harmonic extension of those}
where $W_{i,\Delta}(\Gamma_i')$ are the local spaces associated with the dual degrees of freedom and $\widetilde W_{\Pi}(\Gamma')$ refers to the space associated with the primal degrees of freedom, we can decompose $\widetilde W(\Gamma')$ into
\[\widetilde W(\Gamma') = W_{\Delta}(\Gamma') \times \widetilde W_{\Pi}(\Gamma')\]
and get the representation
\[u_{\Gamma'} \in \widetilde W(\Gamma'), \quad u_{\Gamma'} = (u_{\Delta}, u_{\Pi})\]
with $u_{\Pi} \in \widetilde W_{\Pi}(\Gamma')$ and $u_{\Delta} = \{u_{i, \Delta}\}_{i = 0}^N \in W_{\Delta}(\Gamma')$. Note that we can write $u_{i, \Delta}$ as
\[u_{i, \Delta} = \{\{u_{i, F_{ij}}, u_{i, F_{ji}}\}_{j \in \F_i^0}, \{u_{i, E_{ijk}}, u_{i, E_{jik}}, u_{i, E_{kij}}\}_{(j, k) \in \E_i^0}\}\]
where $u_{i, F}$ is the restriction of $u_{i, \Delta}$ to the face $F$, and $u_{i, E}$ is the restriction of $u_{i, \Delta}$ to the edges $E$. For convenience, \Cref{tab:spaces} gives an overview of the above mentioned function spaces.

{
\renewcommand{\arraystretch}{1.2}
\begin{table}[h]
    \centering
    \begin{tabular}{l p{9cm}}
         \hline 
         \textbf{Space Symbol} & \textbf{Short Description} \\\hline
         $V_i(\overline{\Omega}_i)$ & local FE space on $\overline{\Omega}_i$ \\
         $W_i(\overline{F}_{ji})$ & trace of $V_j(\overline{\Omega}_j)$ on the face $F_{ji}$ between \newline subdomains $\Omega_i$ and $\Omega_j$ \\
         $W_i(\Omega_i')$ & local FE space including duplicated face degrees of freedom \\
         $W_i(I_i)$ & interior part of $W_i(\Omega_i')$ \\
         $W_i(\Gamma_i')$ & part of $W_i(\Omega_i')$ on the broken interface $\Gamma_i'$ \\
         $W(\Omega'), W(\Gamma')$ & global product spaces of the $W_i(\Omega_i')$ and $W_i(\Gamma_i')$ \\
         $\widetilde{W}(\Omega')$ & functions $u \in W(\Omega')$ continuous on vertices \newline with edge and face average constraints \\
         $\widetilde{W}_{VE}(\Omega')$ & functions $u \in W(\Omega')$ continuous on vertices \newline with edge but no face average constraints \\
         $\widetilde{W}(\Gamma'),\widetilde{W}_{VE}(\Gamma')$ & subspaces of $\widetilde{W}(\Omega')$, $\widetilde{W}_{VE}(\Omega')$ of functions \newline which are discrete harmonic in the sense of $\HH_i'$ \\
         $W_{i, \Delta}(\Gamma_i')$ & local space associated with nodal degrees of freedom on\newline $\Gamma_i'\setminus \V_i'$ with zero interface averages for the involved interfaces \\
         $W_{\Delta}(\Gamma')$ & global product space of the $W_{i, \Delta}(\Gamma_i')$ \\
         $\widetilde{W}_{\Pi}(\Gamma')$ & space of degrees of freedom associated with primal constraints (vertex values and face / edge averages)\\\hline
    \end{tabular}
    \caption{A list of the function spaces mentioned in \Cref{sec:spaces} with short descriptions.}
    \label{tab:spaces}
\end{table}
}
\subsection{Schur Bilinear Form, Restriction and Scaling Operators}

Assuming that in algebraic form, we can write the local problems as $K_i'u_i = f_i$, we can order the degrees of freedom such that the local matrices read
\begin{equation}
    \label{local-matrix}
    K_i' = \begin{bmatrix} K_{i, II}' & K_{i, I\Gamma'}' \\ K_{i, \Gamma' I}' & K_{i, \Gamma' \Gamma'}' \end{bmatrix}.
\end{equation}
By eliminating the interior degrees of freedom (\textit{static condensation}), our preconditioner will work only on the unknowns on the interface $\Gamma'$. In order to do this, we need the local Schur complement systems
\[S_i' \coloneqq K_{i, \Gamma' \Gamma'}' - K_{i, \Gamma'}'(K_{i, II}')^{-1}K_{i, I \Gamma'}'\]
with which we define the \textit{unassembled} global Schur complement matrix \[S' = \text{diag}[S_0',\dots,S_N'].\] 

Let $R_{\Gamma'}^{(i)}: W(\Omega') \rightarrow W_i(\Gamma_i')$ denote the restriction operators returning the local interface components and define $R_{\Gamma'} \coloneqq \sum_{i = 0}^N R_{\Gamma'}^{(i)}$. The global Schur complement matrix is then given by $\widehat S_{\Gamma'} = R_{\Gamma'}^TS'R_{\Gamma'}$.

Hence, instead of solving the global linear system $Ku = f$, we can first eliminate the interior degrees of freedom to retrieve a right-hand side $\widehat f_{\Gamma'}$ on the interface $\Gamma'$, then solve the Schur complement system
\[\widehat S_{\Gamma'} u_{\Gamma'} = \widehat f_{\Gamma'}\]
and use the solution $u_{\Gamma'}$ on the interface to recover the interior solution as
\[u_{i,I} = (K_{i,II}')^{-1}(f_{i,I} - K_{i,I\Gamma'}'u_{\Gamma'}).\]

The Schur bilinear form can now be defined as 
\[d_i(\HH_i'u_{i,\Gamma'},\HH_i'v_{i,\Gamma'}) = v_{i,\Gamma'}^TS_i'u_{i,\Gamma'} = s_i'(u_{i,\Gamma'},v_{i,\Gamma'})\]
and it has the property
\begin{equation}
    \label{schur-min-property}
    s_i'(u_{i,\Gamma'},u_{i,\Gamma'}) = \min_{v_i|_{\partial \Omega_i \cap \Gamma'} = u_{i,\Gamma'}} d_i(v_i,v_i)
\end{equation}
which allows us to work with discrete harmonic extensions rather than functions only defined on $\Gamma'$.

The function spaces from the previous chapter will be equipped with the following restriction operators:
\begin{equation*}
    \begin{split}
        R_{i, \Delta}: W_{\Delta}(\Gamma')\rightarrow W_{i, \Delta}(\Gamma'), \quad \quad & R_{\Gamma' \Delta}: W(\Gamma') \rightarrow W_{\Delta}(\Gamma'), \\
        R_{i, \Pi}: \widetilde W_{\Pi}(\Gamma') \rightarrow W_{i, \Pi}(\Gamma_i'), \quad \quad & R_{\Gamma' \Pi}: W(\Gamma') \rightarrow\widetilde W_{\Pi}(\Gamma').
    \end{split}
\end{equation*}
We further define the direct sums $R_{\Delta} = \oplus R_{i, \Delta}, R_{\Pi} = \oplus R_{i, \Pi}$ and $\widetilde R_{\Gamma'} = R_{\Gamma' \Pi} \oplus R_{\Gamma' \Delta}$.

For the EMI model, we consider $\rho$-scaling for the dual variables. For $x \in \overline{\Omega}_i$ it is defined by the pseudoinverses
\begin{equation}
    \label{rho-scaling}
    \delta_i^{\dag}(x) \coloneqq \frac{\sigma_i}{\sum_{j \in \N_x} \sigma_j}.
\end{equation}
For $x \notin \overline{\Omega}_i$ we define $\delta_i^{\dag}(x) = 0$. We note that the $\delta_i^{\dag}$ form a partition of unity, so $\sum_{i = 0}^N \delta_i^{\dag} (x) = 1$ for all $x \in \Omega$.

Recall the following inequality ((6.19) in \cite{dd-book}), it will be an important tool later in this paper:
\begin{equation}
    \label{sigma-ineq}
    \sigma_i (\delta_j^\dag(x))^2 \leq \min\{\sigma_i, \sigma_j\} \quad \forall j \in \N_x.
\end{equation}

We define the local scaling operators on each subdomain $\Omega_i$ as the diagonal matrix
\begin{equation}
    D_i \coloneqq \text{diag}(\delta_i^{\dag}),
\end{equation}
so the $i$-th scaling matrix contains the coefficients (\ref{rho-scaling}) evaluated on the nodal points of $\Omega_i$ along the diagonal. With the scaling operators, we can define scaled local restriction operators
\[R_{i, D, \Gamma'} \coloneqq D_iR_{i, \Gamma'},\quad R_{i,D,\Delta} \coloneqq R_{i,\Gamma'\Delta}R_{i,D,\Gamma'},\]
$R_{D,\Delta}$ as the direct sum of the $R_{i,D,\Delta}$ and finally the global scaled restriction operator 
\[\widetilde R_{D,\Gamma'} \coloneqq R_{\Gamma'\Pi} \oplus R_{D,\Delta}R_{\Gamma'\Delta}.\]

\subsection{The BDDC preconditioner}
Introduced in \cite{dohrmann}, Balancing Domain Decomposition by Constraints (BDDC) is a two-level preconditioner for the Schur complement system $\widehat S_{\Gamma'}u_{\Gamma'} = \widehat f_{\Gamma'}$. Partitioning the degrees of freedom in each subdomain $\Omega_i$ into interior ($I$), dual ($\Delta$) and primal ($\Pi$) degrees of freedom, we can further partition (\ref{eq:matrix-form}) into
\[K_i' = \begin{bmatrix}
    K_{i,II}' & K_{i,I\Delta}' & K_{i,I\Pi}' \\
    K_{i,\Delta I}' & K_{i,\Delta \Delta}' & K_{i, \Delta \Pi}' \\
    K_{i,\Gamma I}' & K_{i, \Gamma\Delta}' & K_{i, \Gamma\Gamma}'
\end{bmatrix}.\]

The action of the BDDC preconditioner is now given by 
\[M^{-1}_{\text{BDDC}}x = \widetilde R^T_{D, \Gamma'}(\widetilde S_{\Gamma'})^{-1}\widetilde R_{D, \Gamma'}x,\quad \widetilde S_{\Gamma'} = \widetilde R_{\Gamma'} S' \widetilde R_{\Gamma'}^T,\]
where the inverse of $\widetilde S^{-1}_{\Gamma'}$ does not have to be computed explicitly but can be evaluated with Block-Cholesky elimination via
\[\widetilde S^{-1}_{\Gamma'} = \widetilde R^T_{\Gamma' \Delta} \big( \sum_{i = 0}^N \begin{bmatrix} 0 & R^T_{i, \Delta} \end{bmatrix}\begin{bmatrix} K_{i, II}' & K_{i, I\Delta}' \\ K_{i, \Delta I}' & K_{i, \Delta \Delta}'\end{bmatrix}^{-1}\begin{bmatrix}0 \\ R_{i, \Delta}\end{bmatrix} \big) \widetilde R_{\Gamma'\Delta} + \Phi S_{\Pi\Pi}^{-1}\Phi^T.\]
The first term above is the sum of independently computed local solvers on each subdomain $\Omega_i'$, and the second term is a coarse solver for the primal variables (which is computed independently of the local solvers), where
\[\Phi = R^T_{\Gamma'\Pi} - R^T_{\Gamma'\Delta}\sum_{i = 0}^N\begin{bmatrix} 0 & R^T_{i, \Delta} \end{bmatrix}\begin{bmatrix} K_{i, II}' & K_{i, I\Delta}' \\ K_{i, \Delta I}' & K_{i, \Delta \Delta}'\end{bmatrix}^{-1}\begin{bmatrix}K_{i,I\Pi} \\ R_{i, \Delta\Pi}\end{bmatrix}R_{i,\Pi},\]
\[S_{\Pi\Pi} = \sum_{i=0}^N R_{i,\Pi}^T\big(K_{i, \Pi\Pi}' - \begin{bmatrix} K_{i,\Pi I} & K_{i,\Pi\Delta}' \end{bmatrix}\begin{bmatrix} K_{i, II}' & K_{i, I\Delta}' \\ K_{i, \Delta I}' & K_{i, \Delta \Delta}'\end{bmatrix}^{-1}\begin{bmatrix}K_{i,I\Pi} \\ R_{i, \Delta\Pi}\end{bmatrix}\big)R_{i,\Pi}.\]

\subsection{Technical Tools and Assumptions}
As in \cite{2D-proof}, we will utilize the following Lemma that is proven considering the continuity and coercivity of the standard Laplacian bilinear form $a_i$:
\begin{lemma}
    For the bilinear form $d_i(u_i, v_i) = \tau a_i(u_i, v_i) + p_i(u_i, v_i)$ with $a_i$ and $p_i$ as defined in (\ref{eq:bilinear-forms}), the following bounds hold:
    \[d_i(u_i, u_i) \leq \tau \sigma_M |u_i|^2_{H^1(\Omega_i)} + \sum_{j \neq i} \frac{C_m}{2}\|u_i - u_j\|^2_{L^2(F_{ij})},\]
    \[d_i(u_i, u_i) \geq \tau \sigma_m|u_i|^2_{H^1(\Omega_i)} + \sum_{j \neq i} \frac{C_m}{2}\|u_i - u_j\|^2_{L^2(F_{ij})},\]
    for all $u_i \in V_i(\Omega_i)$ with $\sigma_m$ and $\sigma_M$ being the minimum and maximum values of the coefficients $\sigma_i$, respectively.
\end{lemma}

We assume that the cardiac domain $\Omega \subset \R^3$ is subdivided into substructures $\Omega_i \subset \R^3$ which have Lipschitz-continuous boundaries. We will work with Sobolev spaces on subsets $\Lambda \subset \partial \Omega_i$ of the subdomain boundaries which have non-vanishing two-dimensional (faces) or one-dimensional (edges) measure and and are relatively open to $\partial \Omega_i$. We will mostly consider the space $H^{1/2}(\Lambda)$ of functions $u \in L^2(\Lambda)$ with finite semi-norm $|u|_{H^{1/2}(\Lambda)} < \infty$ and norm 
\[\|u\|^2_{H^{1/2}(\Lambda)} = \|u\|^2_{L^2(\Lambda)} + |u|^2_{H^{1/2}(\Lambda)} < \infty.\]

We will also use the set of functions in $H^{1/2}(\Lambda)$ which extend to zero from $\Lambda$ to $\partial \Omega_i$ by the extension operator $\E_{\text{ext}}$
\[\E_{\text{ext}}: \Lambda \rightarrow \partial \Omega_i,\quad \E_{\text{ext}}u = \begin{cases}
    0 & \text{on } \partial\Omega_i \setminus \Lambda \\
    u & \text{on } \Lambda.
\end{cases}\]

We will denote this space by
\[H^{1/2}_{00}(\Lambda) = \big\{ u \in H^{1/2}(\Lambda): \E_{\text{ext}}u \in H^{1/2}(\partial\Omega_i)\big\}.\]

\begin{remark}
    For notational convenience, we will write $A \lesssim B$ whenever $A \leq cB$ where $c$ is some constant independent on problem parameters (like e.g. subdomain sizes, mesh size, or conductivity coefficients).
\end{remark}

The following Lemmas will be used in the theoretical analysis in the next chapter, their proofs can be found in \cite{dd-book} (Lemma 4.17, Lemma 4.19 and Lemma 4.26).

\begin{lemma}
\label{sec2:edge-face}
    Let $\overline{u}_{E_{ijk}}$ be the average value of $u$ over $E_{ijk}$, an edge of the face $F_{ij}$. Then,
    \[\|u\|^2_{L^2(E_{ijk})} \lesssim \bigg( 1 + \log\frac{H}{h} \bigg) \|u\|^2_{H^{1/2}(F_{ij})}\]
    and
    \[\|u - \overline{u}_{E_{ijk}}\|^2_{L^2(E_{ijk})} \lesssim \bigg(1 + \log\frac{H}{h}\bigg) |u|^2_{H^{1/2}(F_{ij})}.\]
\end{lemma}

In short, \Cref{sec2:edge-face} will enable us to bound edge terms by face terms of an adjacent face. Another inequality we will leverage for the edge terms is
\begin{lemma}
\label{sec2:boundary-edge}
    Let $E_{ijk}$ be an edge of a subdomain $\Omega_i$ and let $u \in V^h$. Then, 
    \[|\HH_i'(\Theta_{E_{ijk}} u)|_{H^{1/2}(\partial \Omega_i)}^2 \lesssim \|I^h\Theta_{E_{ijk}}u\|_{L^2(E_{ijk})}^2,\]
    where $\Theta_{E_{ijk}}$ is the characteristic finite element function on the edge $E_{ijk}$ and $I^h$ is the usual finite element interpolant.
\end{lemma}

The following Lemma will be useful for the proof considering reduced primal space only containing vertex and edge average constraints:

\begin{lemma}
    Let $\overline{u}_{\E}$ be the average of $u$ over $\E$, an edge of subdomain $\Omega_i$. Let $H_{\E}$ be the diameter of this edge. Then,
    \[(\overline{u}_{\E})^2 \lesssim \frac{1}{H_{\E}} \|u\|^2_{L^2(\E)}.\]
    \label{edge-average}
\end{lemma}

For face terms, we will use the following inequality:

\begin{lemma}
\label{sec2:face}
    Let $F_{ij}$ be a face of a subdomain $\Omega_i$, let $u \in V^h$ and $\overline{u}_{F_{ij}}$ be the average of $u$ over $F_{ij}$. Then,
    \[\|\Theta_{F_{ij}}\|_{H^{1/2}_{00}(F_{ij})} \lesssim \bigg(1 + \log\frac{H}{h}\bigg)H\]
    and
    \[\|I^h\Theta_{F_{ij}}(u - \overline{u}_{F_{ij}})\|^2_{H^{1/2}_{00}(F_{ij})} \lesssim \bigg(1 + \log\frac{H}{h}\bigg)^2|u|^2_{H^{1/2}(\partial\Omega_i)}.\]
\end{lemma}

Finally, we will consider the following classical, well-known results throughout the proof. They can be found for example in Appendix A of \cite{dd-book}.

\begin{lemma}
    Let $\Lambda \subset \partial\Omega_i$. Then, for $u \in H^{1/2}_{00}$ it holds that 
    \[\|\E_{\text{ext}}u\|^2_{H^{1/2}(\partial \Omega_i)} \lesssim \|u\|^2_{H^{1/2}_{00}(\Lambda)} \lesssim \|\E_{\text{ext}}u\|^2_{H^{1/2}(\partial\Omega_i)}.\]
\end{lemma}

\begin{theorem}[Trace theorem] Let $\Omega_i$ be a polyhedral domain. Then,
\[|u|_{H^{1/2}(\Gamma_i)} \sim |\HH^{\Delta}_iu_{\Gamma}|^2_{H^1(\Omega_i)}.\]
    
\end{theorem}