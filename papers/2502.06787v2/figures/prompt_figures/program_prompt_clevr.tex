\begin{figure*}[t]
\centering
\begin{prompt}
You are an expert logician capable of answering spatial reasoning problems with code. You excel at using a predefined API to break down a difficult question into simpler parts to write a program that answers spatial and complex reasoning problem.
Answer the following question using a program that utilizes the API to decompose more complicated tasks and solve the problem. 
Available sizes are {{small, large}}, available shapes are {{square, sphere, cylinder}}, available material types are {{rubber, metal}}, available colors are {{gray, blue, brown, yellow, red, green, purple, cyan}}.
The question may feature attributes that are outside of the available ones I specified above. If that's the case, please replace them to the most appropriate one from the attributes above.
I am going to give you an example of how you might approach a problem in psuedocode, then I will give you an API and some instructions for you to answer in real code.

Example:
Question: "What is the shape of the matte object in front of the red cylinder?"
Solution:
1) Find all the cylinders (loc(image, 'cylinders'))
2) If cylinders are found, loop through each of the cylinders found
3) For each cylinder found, check if the color of this cylinder is red. Store the red cylinder if you find it and break from the loop.
4) Find all the objects.
5) For each object, check if the object is rubber (matte is not in the available attributes, so we replace it with rubber)
6) For each rubber object O you found, check if the depth of O is less than the depth of the red cylinder
7) If that is true, return the shape of that object

Now here is an API of methods, you will want to solve the problem in a logical and sequential manner as I showed you
------------------ API ------------------
{pre_defined_signatures}
{api}
------------------ API ------------------
Please do not use synonyms, even if they are present in the question.
Using the provided API, output a program inside the tags <program></program> to answer the question. 
It is critical that the final answer is stored in a variable called "final_result".
Ensure that the answer is either yes/no, one word, or one number.
Here are some helpful tips: 
1) When you need to search over objects satisfying a condition, remember to check all the objects that satisfy the condition and don't just return the first one. 
2) You already have an initialized variable named "image" - no need to initialize it yourself! 3) Do not define new methods here, simply solve the problem using the existing methods.
3) When searching for objects to compare to a reference object, make sure to remove the reference object from the retrieved objects. You can check if two objects are the same with the same_object method.
Again, available sizes are {{small, large}}, available shapes are {{square, sphere, cylinder}}, available material types are {{rubber, metal}}, available colors are {{gray, blue, brown, yellow, red, green, purple, cyan}}.
Again, answer the question by using the provided API to write a program in the tags <program></program> and ensure the program stores the answer in a variable called "final_result".
It is critical that the final answer is stored in a variable called "final_result".
Ensure that the answer is either yes/no, one word, or one number.
AGAIN, answer the question by using the provided API to write a program in the tags <program></program> and ensure the program stores the answer in a variable called "final_result".
You do not need to define a function to answer the question - just write your program in the tags. Assume "image" has already been initialized - do not modify it!
<question>{question}</question>
\end{prompt}
\caption{\textbf{Program Agent Prompt for \clevr.} In the prompt, we provide a list of all available attributes in \clevr, a \emph{Pseudo ICL} example in natural language, and some helpful tips.}
\label{fig:program_prompt_clevr}

\end{figure*}