\documentclass[conference]{IEEEtran}
\usepackage{times}
\usepackage[T1]{fontenc}
\usepackage[utf8]{inputenc}

% numbers option provides compact numerical references in the text. 
\usepackage[numbers]{natbib}
\usepackage{multicol}
% \usepackage[bookmarks=true]{hyperref}
\usepackage[pagebackref=true,breaklinks=true,bookmarks=true,colorlinks]{hyperref}
%%%%%%%%%%%%%%%%%%%%%%%%%%%%%%%%%%%%%%%%%%%%%%%%%%%%%%%%%%%%%%%%%%%%%%%%%%%%%%

%% Beautiful mathematics
\usepackage{amsmath, amssymb, amsfonts} 
\usepackage{nicefrac}
\usepackage{mathtools}
\usepackage{bm, bbm}
\usepackage[scr=boondoxo,scrscaled=1.05]{mathalfa}

%% References in the correct format 
%\usepackage[square,numbers]{natbib}
%\def\bibfont{\footnotesize} % fix to have the same font size as without natbib

\usepackage[sort, compress, space]{cite}            


%% Enumerate nicely 
\usepackage{enumitem}

%% Different color comments and commenting large parts of the text
\usepackage{xcolor}
\usepackage{comment}
\usepackage{soul}

%% Hyper references
\usepackage{hyperref}
\usepackage{cleveref}
%\usepackage[numbers]{natbib}

\usepackage{tikz}
%\usepackage{thm-restate}
%% Appendix package
%\usepackage{appendix}

%% Random text to test spacing 
\usepackage{blindtext}

\usepackage{afterpage}

\usepackage{algorithm, algorithmic}    



\usepackage{dsfont}

\usepackage{tikz}
\usepackage{graphicx}
\usepackage{tikzscale}
\usepackage{pgfplots}
\pgfplotsset{compat=newest}
\usepackage{xfrac}

\usepackage{thm-restate}

%\usepackage{subcaption}

\usepackage{balance}

\usepackage{cite}
\usepackage{amsmath,amssymb,amsfonts}
\usepackage{balance}
\usepackage{algorithmic}
\usepackage{graphicx}
\usepackage{textcomp}
\usepackage{xcolor}
\usepackage{amsmath}
\usepackage{amssymb}
\usepackage[mathscr]{euscript}
\usepackage{comment}
\usepackage{xcolor}
\usepackage{enumitem} 
\usepackage{amsthm}


\newcommand{\thought}[1]{{\color[rgb]{0.2,0.39,0.66}(#1)}}
\newcommand{\todo}[1]{{\color[rgb]{1.0,0.0,0.0}(#1)}}
\newcommand{\hsh}[1]{{\color{green!50!black} Henrik: #1}}
\newcommand{\st}[1]{{\color{red!50!black} Sebastian: #1}}

\newcommand{\ulm}[1]{_{\scaleto{\mathrm{#1}}{3pt}}}
\newcommand\at[2]{\left.#1\right|_{#2}}











\newtheorem{assumption}{Assumption}

\DeclareMathOperator*{\argmax}{arg\,max}
\DeclareMathOperator*{\argmin}{arg\,min}

\newcommand{\swname}[1]{\texttt{#1}}
\newcommand{\ie}{i\/.\/e\/.,\/~}
\newcommand{\eg}{e\/.\/g\/.,\/~}
\newcommand{\cf}{cf\/.\/~}

\newcommand{\fig}{Fig\/.\/~}
\newcommand{\defn}{Def\/.\/~}
\newcommand{\sect}{Sec\/.\/~}
\newcommand{\tabl}{Tab\/.\/~}
\newcommand{\algo}{Algorithm~}
\newcommand{\theo}{Theorem~}

\newcommand{\bnnl}{3 hidden layers}
\newcommand{\bnnn}{50 neurons}
\newcommand{\bnna}{tanh activations}

\newcommand{\capt}[1]{\mdseries{\emph{#1}}}

\newcommand{\videolink}{at \url{https://youtu.be/_d7AqTRjz6g}}
\newcommand{\codelink}{\url{https://github.com/wheelbot/mini-wheelbot}}

\newcommand{\fakepar}[1]{\vspace{0mm}\noindent\textbf{#1.}}

\newcommand{\needref}{\textcolor{red}{[REF]}}

\newcommand{\plotfontsize}{9pt}


\pdfinfo{
   /Author (Homer Simpson)
   /Title  (Robots: Our new overlords)
   /CreationDate (D:20101201120000)
   /Subject (Robots)
   /Keywords (Robots;Overlords)
}

\begin{document}

% paper title
\title{Diffusion Policy}
\title{Diffusion Policy: \\ \LARGE{ Visuomotor Policy Learning via Action Diffusion}}
%\title{Diffusion Policy: \\ \LARGE{Learning Visuomotor Policies with Conditional Diffusion Process}}
% You will get a Paper-ID when submitting a pdf file to the conference system
% \author{Author Names Omitted for Anonymous Review. Paper-ID [6]}
\author{
Cheng Chi$^1$, Siyuan Feng$^2$, Yilun Du$^3$, Zhenjia Xu$^1$, Eric Cousineau$^2$, Benjamin Burchfiel$^2$, Shuran Song$^1$ \\ 
$^1$ Columbia University \quad\quad 
$^2$ Toyota Research Institute \quad\quad
$^3$ MIT
\\ \href{https://diffusion-policy.cs.columbia.edu}{https://diffusion-policy.cs.columbia.edu}
}

%\maketitle

% \begin{figure*}[t]
%     \centering
%     \includegraphics[width=\linewidth]{figure/DP_teaser.pdf}
%     \caption{\textbf{Different Form of Policy Representations.} a) Explicit policy with different types of action representations.  b) Implicit policy learns an energy function conditioned on both action and observation and optimizes for actions that minimize the energy landscape c) Diffusion policy learns a gradient field to refine actions. \todo{update the figure with the same example. }  }
%     \label{fig:policy_rep}
%     %https://docs.google.com/drawings/d/1SNd5_khk3RsYuE9JCwUVjmRED-eF3UrO78XnzbOwE4Y/edit?usp=sharing
% \end{figure*}

\twocolumn[{%
	\renewcommand\twocolumn[1][]{#1}%
	\maketitle
        \vspace{-5mm}
	\begin{center}
		\includegraphics[width=0.95\textwidth]{figure/DP_teaser.pdf}
		\captionof{figure}{\textbf{Policy Representations.} \label{fig:policy_rep} a) Explicit policy with different types of action representations.  b) Implicit policy learns an energy function conditioned on both action and observation and optimizes for actions that minimize the energy landscape c) Diffusion policy refines noise into actions via a learned gradient field. This formulation provides stable training, allows the learned policy to accurately model multimodal action distributions, and accommodates high-dimensional action sequences. 
        % \todo{update figure to make b and c consistent, both 2D or both 3D. Make it clear c is gradient of b change J (a) -> E (a)}
        } 
	%https://docs.google.com/drawings/d/1SNd5_khk3RsYuE9JCwUVjmRED-eF3UrO78XnzbOwE4Y/edit?usp=sharing
	\end{center}
}]


\begin{abstract}
%This consistent performance boost provides strong evidence of the Diffusion Policy's effectiveness, which is attributed to its inherited properties of the diffusion formulation. 
%By learning the gradient field of an implicit action score and performing Stochastic Langevin Dynamics sampling on this gradient field, Diffusion policy is able to accurately model the multimodal action distribution, scalable to high-dimensional action space, and achieve stable training.  
This paper introduces Diffusion Policy, a new way of generating robot behavior by representing a robot's visuomotor policy as a conditional denoising diffusion process. We benchmark Diffusion Policy across 12 different tasks from 4 different robot manipulation benchmarks and find that it consistently outperforms existing state-of-the-art robot learning methods with an average improvement of 46.9\%. 
Diffusion Policy learns the gradient of the action-distribution score function and iteratively optimizes with respect to this gradient field during inference via a series of stochastic Langevin dynamics steps.
We find that the diffusion formulation yields powerful advantages when used for robot policies, including gracefully handling multimodal action distributions, being suitable for high-dimensional action spaces, and exhibiting impressive training stability.
To fully unlock the potential of diffusion models for visuomotor policy learning on physical robots, this paper presents a set of key technical contributions including the incorporation of receding horizon control, visual conditioning, and the time-series diffusion transformer. 
We hope this work will help motivate a new generation of policy learning techniques that are able to leverage the powerful generative modeling capabilities of diffusion models. Code, data, and training details will be publicly available.

% that are often designed specifically for those benchmarks.  
%we introduce a new form of policy that represents a robot's visuomotor policy as a ``conditional diffusion denoising process on action'', Diffusion Policy. In this formulation, the policy learns to infer the gradient field of the action score for K denoising iterations conditioned on its visual observation. The output action sequence can be then inferred by performing Stochastic Langevin Dynamics sampling on this learned gradient field. 
% This formulation allows the learned policy to accurately model the multimodal action distribution, scalable to high-dimensional action space, and achieve stable training.
% We systematically evaluate the algorithm over 11 tasks from 4 different benchmarks that range from both simulated and real-world environments, single- and multi-task benchmarks, and fully- and under-actuated systems.
% The consistent performance boost against state-of-the-art methods across all benchmarks, tasks, and scenarios provides strong evidence of the effectiveness of the Diffusion Policy. We also provide detailed analysis to carefully examine the characteristics of the proposed algorithm and the impacts of the key design decisions. 
% The code, data, and training details will be publicly available.
\end{abstract}

\IEEEpeerreviewmaketitle



\documentclass[../main.tex]{subfiles}
\graphicspath{{../images/}}
\makeatletter
\def\input@path{{../images/}}
\makeatother
\begin{document}
\section{Introduction}
\begin{figure}
\centering
\begin{tikzpicture}
\node[inner sep=0pt] (ws) at (0, 0) {
\includegraphics[height=.4\textwidth, trim={10cm 0 10cm 0},clip]{world_space.png}};
\node[inner sep=0pt] (cs) at (6,0) {\includegraphics[height=.4\textwidth, trim={10cm 1cm 10cm 4cm},clip]{conf_space.png}};
\end{tikzpicture}
\vspace{-5pt}
\label{fig:pbrm_intro}
\caption{\textbf{Left}: Shows world space obstacles as grey spheres. Robots start and goal configuration is colored red and green, respectively. Configurations along the computed path are colored transparent blue. \textbf{Right:} Mapped world space scenario to configuration space. Obstacle region is the grey mesh. Red spheres are collision-free regions computed by the neural SCDF. The optimized shortest path in the convex corridor is the blue curve.}
\vspace{-25pt}
\end{figure}
Motion planning is the problem of finding a collision-free trajectory that connects a given start and goal configuration. The planning takes place in the configuration space of the robot. For single body robots, like mobile robots or drones, the configuration space and the world space are usually the same. This simplifies the planning, since explicit obstacle representations are available which enables geometrical tools like separating hyperplanes, smallest distance to obstacles etc., to be used when designing motion planning algorithms. For multi-body robots like manipulators, the situation is completely different. The world space obstacles are usually mapped to non-convex regions, and to make the problem even harder, the mapping is usually not known. Forming explicit representations of the obstacle region in the configuration space is usually too expensive or intractable. Despite all of this, sampling based planners are used with great success, which mainly is due to their use of implicit representations of the obstacle region. The basic idea is to construct a graph in the configuration space that covers and connects the collision-free region. From this graph, a path can be extracted that connects a given start and goal configuration. The approach is computationally expensive, since the graph is constructed with the smallest geometrical building block available, points, which represents a collision-check. Furthermore, the extracted paths from the graph are non-smooth and jagged due to the stochastic nature of the approach. This adds an additional post-processing step to the process, where the paths are shortcutted and smoothened, before the path can be used for tracking. Clearly a lot of time is invested to form this graph and produce smooth paths. Thus, if the obstacles start to move, then all of this work is done in no use, since all points that make up this graph need to be re-verified, which is simply too time consuming to be done in real time.
\\\\
In this work, we want to address the existing drawbacks of the sampling based planners. Our main contribution is an improved motion planner where each vertex in the graph covers a collision-free region in the form of a sphere instead of a point and where the edges are formed with neighboring intersecting spheres. This representation has the advantage of instead of returning piecewise linear paths, returning a sequence of overlapping spheres, i.e. a convex corridor, that connects a given start and goal configuration, illustrated in Figure \ref{fig:pbrm_intro}. This convex corridor allows us to use convex optimization to produce smooth trajectories, instead of computationally expensive post-processing methods. The representation further allows us to estimate the coverage of the collision-free space, which gives us awareness and feedback in the offline roadmap construction phase. Finally, our representation is simple to adapt to moving obstacles, simply requery for the new radii and recheck for intersections. 
\\\\
The spherical collision-free regions are formed using a signed distance function (SDF), which is a function that returns the smallest distance from an arbitrary point to the boundary of an obstacle. As the name implies, the distance is signed, thus if the point is inside the obstacle it is negative otherwise positive. If the distance is positive, a sphere with radius equal to the distance is guaranteed to cover a collision-free region. Using an SDF in motion planning is not new, but what is novel about our approach is that we express the distance in the configuration space instead of the world space and by doing so allows us to form these convex collision-free regions. We refer to the resulting SDF as a signed configuration distance function (SCDF). Computing an SCDF analytically is non-trivial, our approach is therefore to parameterize the SCDF with a deep neural network and learn the mapping by supervised learning. Our resulting neural SCDF can compute distances for different parameter values of obstacle shapes and we also show how multiple distances can be combined, thus making our approach flexible.
\section{Related work}
Motion planning algorithms can roughly be divided into three families, grid-based, sampling based and optimization based methods. Grid-based methods (GBM) discretize the planning space from which a graph is then compiled. A standard search method is A$^\star$ \citep{a_star}, which is classified as an \textit{informed} search method, since it employs a heuristic function to speed up the search. A$^\star$ guarantees to return an optimal path at the level of discretization used. GBMs usually discretize the planning space by a regular lattice and this limits the GBMs to problems with low dimensionality due to the curse of dimensionality. Thus, GBMs are usually limited to single-body robots where the degrees of freedom (DOF) are low. To overcome the inherent scaling problem with the GBMs, stochastic methods are usually used for multi-body robots. These methods are termed as sampling-based methods (SBM) and core members within this family are the rapidly-exploring random trees (RRT) \citep{rrt} and the probabilistic roadmap (PRM) \citep{prm}. RRT grows a tree from the start configuration and explores the collision-free region in a rapid way until it is able to connect to the goal region. RRT is usually improved by bi-directional planning \citep{rrt_connect}, i.e. an additional tree is grown from the goal configuration and the trees are tested for connection after any tree has been expanded. RRT is a single-query method, thus it searches for a path from scratch each time it is queried. Contrary to this, PRM is a multi-query method, which solves for multiple queries without starting from scratch. PRM does this by creating a roadmap (graph) that covers the collision-free space as an offline step. The graph is then used to solve for multiple queries. PRMs are used in cases where the environment does not change since the extra offline step is too computationally costly and needs to be re-done if the environment is changed. In our work, we address this inherent issue by using a different roadmap representation. Our vertices in the graph cover a collision-free region in the form of spheres and we form the edges by checking for intersecting spheres. If something in the environment changes, we recompute the spheres radii and recheck the intersections, without relying on collision detection. We use a trained neural network to compute the sphere radius, therefore querying for the radius can be done fast, hence our representation enables the PRM for dynamic environments.
\\\\
In the recent decades, optimization based methods (OBM) \citep{chomp, schulman, itomp, stomp} have been introduced as an alternative to SBM for multi-body robots. Like the SBM, the OBMs scale well to higher dimensional problems and produce smoother motion. It is common to use a SDF in the optimization since it is a smooth function, thus enabling gradient-based methods. However, the standard way of expressing the SDF is in world space. The distance therefore needs to be mapped to the configuration space by the forward kinematics. This mapping makes the optimization problem a non-linear program (NLP), which is computationally expensive to solve. Recently, a different approach has been proposed. In \cite{mp_gcs} motion planning is formulated as a convex optimization problem by using the graph of convex sets framework \citep{gcs}. The underlying idea is to decompose the collision-free space into intersecting convex sets from which a convex optimization problem is formulated. In cases where an explicit representation of the obstacles in the configuration space exists, like for single-body robots, creating collision-free convex regions can be done fast \citep{iris}. For multi-body robots, this is non-trivial. Existing work does this successfully \citep{iris_nlp, iris_c} by an optimization based approach, but the methods are still too time consuming to be used in the presence of moving obstacles. Our approach is instead to use deep learning to learn an SDF expressed in the configuration space. With this, we can query for shortest distances to the collision boundary, which allows us to expand spherical regions which are collision-free. Our approach is fast and therefore enables our suggested roadmap planner to be used in dynamic environments.
\\\\
Recent research has focused on learning collision detection \citep{fk_kernel_distance, diffco, graphdistnet} by predicting the signed distance between the robot links and the surrounding obstacles in the world space. The learned SDF is used in trajectory optimization but since the distance is expressed in the world space, the problem becomes an NLP and therefore takes a long time to solve. We take a novel approach and suggest to instead express the signed distance in the configuration space. This allows us to improve the PRM at the same time as it enables convex optimization for trajectory optimization, which runs faster and is more reliable than NLP solvers. In \cite{cspf} a learned signed distance function in the configuration space is proposed similar to our approach. However, their approach is restricted to point cloud representations, while we propose to represent the obstacles as parameterized geometric shapes, e.g. spheres. Furthermore, we also show how to use our learned SCDF to improve an existing roadmap planner.
\section{Problem formulation}
A robot is located in the world space, $\W \subset \R^3 $. The unique location of the robot is given by its configuration $\q \in \C$, where $\C$ is the configuration space. The set of points covered by the robots bodies at a certain configuration is expressed as $\B(\q) \subset \W$. The robot is surrounded by $\NrObst$ obstacles $\O = \bigcup_{i=1}^{\NrObst} \O_i$, where  $\O_i \subset \W$. The representation of the obstacle in the configuration space is the set $\C\O_i = \{\q \in \C \: |\: \B(\q) \cap \O_i \neq \emptyset \}$. The obstacle space is formed as $\Co = \bigcup_{i=1}^{\NrObst} \C \O_i$. The complement is referred to as the free space, $\Cf = \C \setminus \Co$. The path planning problem is a tuple, ($\Cf$, $\qStart$, $\qGoal$), where we want to connect a query pair, consisting of a start, $\qStart$, and goal configuration, $\qGoal$, with a geometric path, $\q(s): [0, 1] \mapsto \Cf$, such that $\q(0)=\qStart$ and $\q(1)=\qGoal$, or report correctly when such a path does not exist.
\end{document}

% \begin{figure}
%     \centering
%     \includegraphics[width=0.5\linewidth]{Move_teaser.pdf}
%     \caption{Comparison of different dynamic compute approaches. length of arrow indicates residual transformation per token while width indicates velocity of transformation.}
%     \label{fig:enter-label}
% \end{figure}

\section{Method}
\label{sec:method}
Residual connections play a crucial role in shaping token representations, yet their dynamics remain underexplored in the context of efficient decoding. In this work, we delve deeper into transformer residual dynamics and investigate how modulating residual transformation velocity can improve inference efficiency in token-level processing, optimizing both dense and sparse MoE transformers.


\subsection{Residual Dynamics and Motivation for Multi-rate Residuals} \label{sec:motivation}

To analyze how hidden representations evolve across different layers of a transformer architecture, it's crucial to consider the effect of residual connections. Each transformer decoder layer typically has residual connections across attention and MLP submodules. As the residual stream $h_i$ traverses from interval $E_j$ to $E_{j+1}$, it undergoes a residual transformation given by:  
% \begin{equation}
% \label{eq:slow_residual_transformation}
% H_{E_{j+1}} = H_{E_j} \prod_{i=E_j}^{E_{j+1}} \left( I + \mathcal{A}_i \right) \left( I + \mathcal{M}_i \right) \quad \text{where} \quad \mathcal{A}_i = f(c_i, h_{i}), \mathcal{M}_i = g(h_i)
% \end{equation}

\begin{equation} \label{eq:slow_residual_transformation}
h_{E_{j+1}} = h_{E_j} + \sum_{i=E_j}^{E_{j+1}-1} \left( \mathcal{A}_i(h_i) + \mathcal{M}_i(h_i + \mathcal{A}_i(h_i)) \right) \quad \text{where} \quad \mathcal{A}_i = f(c_i, h_{i}), \mathcal{M}_i = g(h_i). 
\end{equation}

Here, \( \mathcal{A}_i \) denotes the non-linear transformation introduced by the multi-head attention mechanism at layer \( i \), while \( \mathcal{M}_i \) corresponds to the non-linear transformation of the MLP block at the same layer. These transformations depend on the input residual stream \( h_i \) and, in the case of \( \mathcal{A}_i \), the previous contextual representation \( c_i \).\footnote{Normalization layers are typically applied in practice but are omitted here for simplicity of the argument.}


% For easy tokens, the magnitude and direction of this delta transformation become progressively smaller with each successive layer as shown in \cref{fig:delta_transformation}. Consequently, it is feasible to predict these tokens after only a few residual connections, whereas harder tokens necessitate more extensive processing through additional layers.

\begin{figure}[ht]
    \centering
    \begin{subfigure}{0.48\textwidth}
        \centering
        \includegraphics[width=\textwidth]{sections/figures/residual_change.pdf}
        \caption{}
        \label{fig:residual_change}
    \end{subfigure}%
    \hfill
    \begin{subfigure}{0.48\textwidth}
        \centering
        \includegraphics[width=\textwidth]{sections/figures/alignment_wrt_dedicated_model.pdf}
        \caption{}
    \label{fig:alignment_wrt_dedicated_model}
    \end{subfigure}
    \caption{(a) As residual streams propagate through the model, the directional shifts in the residuals become progressively smaller. (b) A dedicated model with $k$ layers achieves a faster rate of change in residual streams and higher alignment than base model leveraging early exit mechanisms at layer $k$.}
    \label{fig}
\end{figure}


To examine whether residual transformations can be accelerated across layers, we conducted experiments using a diverse set of prompts on a pre-trained Phi3 model~\cite{phi3_report}. As illustrated in \cref{fig:residual_change}, we measured the directional shift in residual states as \( 1 - \mathcal{C}(h_{i-1}, h_i) \), where \(\mathcal{C}\) denotes normalized cosine similarity. This shift is notably higher in the initial layers, gradually decreasing in subsequent layers. This behavior allows traditional early exit approaches to effectively accelerate decoding by enabling earlier exits for simpler tokens. However, these approaches typically rely on a distance-based approximation, where the full residual transformation of the model is approximated by the residual transformations of the initial layers. To gain deeper insights into the distance versus velocity aspects of residual transformation, we conducted a comparative study. Specifically, we trained an early exit head at layer $k$ of the Phi3 model, which consists of 32 layers, restricting the distance traveled by each token. To accelerate the residual transformation relative to number of layers, we trained a smaller model consisting of only $k$ layers, while keeping all other hyperparameters consistent. We then compared the next-token prediction accuracy of the early exit head of the base model with that of the smaller model. To ensure an equal number of trainable parameters, we inserted low-rank adapters into the smaller model and trained only these adapters, whereas, in the distance-based approach, we trained solely the early exit head. In addition, to accelerate the residual transformation in smaller model, we distilled the residual streams from the larger model by incorporating a distillation loss ~\cite{sanh2019distilbert} between the residual state at layer \(i\) of the smaller model and the residual state at layer \(4 \times i\) of the larger model. As shown in ~\cref{fig:alignment_wrt_dedicated_model} the smaller model demonstrates a significantly faster rate of change in residual streams, leading to higher next token prediction accuracy after $k$ layers compared to the base model that employs traditional early exit mechanisms after $k$ layers \cite{schuster2022confident, chen2023eellm, varshney-etal-2024-investigating}. This experimental setup, which modifies only the rate of change in residual streams while keeping other factors constant, suggests that dense transformers, trained with a fixed number of layers, may inherently possess a slow residual transformation bias.

This observation raises an intriguing question: if the rate of change in residual streams could be accelerated relative to the number of layers, is it possible to facilitate earlier alignment for a greater proportion of tokens? Earlier alignment would be beneficial to not only facilitate dynamic computation but also for generating speculative tokens efficiently with high acceptance rates in speculative decoding setups ~\cite{leviathan2023fast, chen2023accelerating}. 

%thereby enhancing the efficiency of early exiting? 
 % This bias likely constrains the effectiveness of early exiting, particularly for easier tokens. By addressing this limitation through accelerated residual transformations, we hypothesize that it is possible to substantially improve the efficiency and accuracy of early exit strategies in transformer models.

\subsection{Multi-Rate Residual Transformation} \label{m2r2_method}

To address the slow residual transformation bias described in ~\cref{sec:motivation}, we introduce \textit{accelerated residual streams} that operate at rate $R$ relative to original slow residual stream. We pair slow residual stream, $h$ with an accelerated residual stream, $p$, which has an intrinsic bias towards earlier alignment. Relative to ~\cref{eq:slow_residual_transformation}, accelerated residual transformation from interval $E_j$ to $E_{j+1}$ can be represented as: 

% \begin{equation}
% \label{eq:fast_residual_transformation}
% P_{E_{j+1}} = P_{E_j} \prod_{i=E_j}^{E_{j+1}} \left( I + \hat{\mathcal{A}_i} \right) \left( I + \hat{\mathcal{M}_i} \right) \quad \text{where} \quad \hat{\mathcal{A}_i} = \hat{f}(c_i, P_{i}), \hat{\mathcal{M}_i} = \hat{g}(P_{i})
% \end{equation}


\begin{equation} \label{eq:fast_residual_transformation}
p_{E_{j+1}} = p_{E_j} + \sum_{i=E_j}^{E_{j+1}-1} \left( \hat{\mathcal{A}_i}(p_i) + \hat{\mathcal{M}_i}(p_i + \hat{\mathcal{A}_i}(p_i)) \right) \quad \text{where} \quad \hat{\mathcal{A}_i} = \hat{f}(c_i, p_{i}), \hat{\mathcal{M}_i} = \hat{g}(h_i), 
\end{equation}



where $\hat{\mathcal{A}_i}$ and $\hat{\mathcal{M}_i}$ denote non-linear transformation added by layer $i$ to previous accelerated residual $p_{i}$. Similar to $\mathcal{A}_i$, non-linear transformation $\hat{\mathcal{A}_i}$ attends to same context $c_i$ but uses a different transformation $\hat{f}$ for accelerating $p_{E_j}$ relative to $h_{E_j}$. 

We integrate accelerated residual transformation directly into the base network using parallel accelerator adapters such that rank of accelerator adapters $R_p << d$ where $d$ denotes base model hidden dimension. This setup allows the slow residual stream $h_{E_j}$ to pass through the base model layers while the accelerated residual stream $p_{E_j}$ utilizes these parallel adapters as shown in ~\cref{fig:m2r2_main}. Both slow and accelerated residuals are processed in same forward pass via attention masking and incur negligible additional inference latency in memory bound decoding setups, while in compute bound decoding setups where FLOPs optimization is essential, accelerated residual stream utilizes a fraction of attention heads that of slow residual (see ~\cref{sec:flops_optimization}). Additionally, to maximize the utility of accelerated residual transformations without introducing dedicated KV caches, we propose a shared caching mechanism between the slow and accelerated streams which minimally impact alignment benefits of our approach while offering substantial memory savings (see ~\cref{fig:koala_alignment}). Specifically, the attention operation on the slow residuals \( \text{MHA}(h_t, h_{\leq t}, h_{\leq t}) \) is redefined for accelerated residuals as 
\[
\hat{\mathcal{A}} = MHA(p_t, h_{<t} \oplus p_t, h_{<t} \oplus p_t),
\]
where the accelerated residual at time-step $t$, \( p_t \) attends to the slow residual’s KV cache, facilitating the reuse of contextual information across both residual streams without incurring additional caching costs. Here, \(MHA(q, k, v) \) represents multi-head attention between query \( q \), key \( k \), and value \( v \).

\begin{figure}
    \centering
    \includegraphics[width=0.8\linewidth]{sections//figures/m2r2_main2.pdf}
    \caption{Multi-rate Residuals Framework: Slow residual stream of base model is accompanied by a faster stream that operates at a $2-(J+1)\times$ rate relative to the slow stream, undergoing transformations via accelerator adapters as detailed in \cref{m2r2_method}, where J denotes number of early exit intervals. Colors within the slow and fast residual streams indicate similarity, with matching colors representing the most closely aligned residual states. At the beginning of the forward pass and at each exit point, the accelerated residual state is initialized from the corresponding slow residual state to avoid gradient conflict during training (see ~\cref{sec:grad_conflict}). Early exiting decisions are informed by the Accelerated Residual Latent Attention (ARLA) mechanism, described in \cref{method_arla}, which evaluates residual dynamics across consecutive exit gates.}
    \label{fig:m2r2_main}
\end{figure}

% Furthermore. to maximize the benefits of fast residual transformations without using dedicated KV caches, we propose sharing the fast network’s cache with the slow network. Formally speaking, We modify attention operation on slow residuals $MHA(H_t, H_{<=t}, H_{<=t})$ as $MHA(P_{t}, H_{<t} \oplus P_t, H_{<t}  \oplus P_t)$ such that accelerated residuals attend to previous slow context KV cache, where $MHA(q,k,v)$ denotes multi head attention between query, $q$, key $k$ and value $v$.


\subsection{Enhanced Early Residual Alignment}
Early residual alignment is instrumental in optimizing early exiting, speculative decoding, and Mixture-of-Experts (MoE) inference mechanisms. In this section, we provide a detailed analysis of how accelerated residuals enhance these inference setups.

% By aligning the residual states of intermediate layers with the final output representations, the model can maintain high prediction accuracy even when computations are truncated at earlier layers. This enables more reliable early exiting, reducing the overall computational cost while preserving performance. Additionally, in speculative decoding, early residual alignment allows the model to make confident predictions using faster, partial computations, thereby accelerating inference without sacrificing output quality.


\subsubsection{Early Exiting} \label{method_early_exiting}

A prevalent strategy for enabling early exiting at an intermediate layer $E_{j}$ involves approximating the residual transformation between $E_{j}$ and the final layer $N-1$ using a linear, context independent mapping, $\mathcal{T}$, such that $H_{N-1} \approx \mathcal{T}(H_{E_{j}})$. This approximation has been extensively employed in conventional approaches ~\cite{schuster2022confident, chen2023eellm, varshney-etal-2024-investigating}, providing a computationally efficient means to project the output of deeper layers from intermediate states. Specifically, residual state of layer $N-1$ with this approximation can be expressed as:


% \begin{equation}
% \label{eq: vanila_ea_assumption}
% \Phi(H_{E_{j}}) \sim H_{E_{j}} \prod_{i=E_{j}}^{N}\left( I + \mathcal{A}_i \right) \left( I + \mathcal{M}_i \right) \quad \text{where} \quad \Phi \perp C
% \end{equation}

\begin{equation} \label{eq:early_exiting}
h_{E_j} + \sum_{i=E_j}^{N-1} \left( \mathcal{A}_i(h_i) + \mathcal{M}_i(h_i + \mathcal{A}_i(h_i)) \right) \sim \mathcal{T}(h_{E_{j}})  \quad \text{where} \quad \mathcal{T} \perp c. 
\end{equation}


Here, $\mathcal{A}_i$ and $\mathcal{M}_i$ represent the residual contributions of the multi-head attention and MLP layers, respectively, while $\mathcal{T}$ remains independent of $c$, the preceding context.

This approach is inherently limited by two major factors: first, the assumption of linearity between $h_{E_{j}}$ and $h_{N-1}$ may not hold uniformly for all tokens, particularly when $E_j \ll N$. Second, the linear transformation $\mathcal{T}$ disregards the influence of the context $c$ and fails to account for the latent representations of previous contextual states. In contrast, M2R2 accelerated residual states mitigate both of these challenges by approximating the slow residual transformation of all layers via a faster residual transformation of fewer layers as:
% \begin{equation}
% H_{E_j} \prod_{i=E_j}^{N}\left( I + \mathcal{A}_i \right) \left( I + \mathcal{M}_i \right) \sim P_{E_j} \prod_{i=E_j}^{E_j+1}\left( I + \hat{\mathcal{A}_i} \right) \left( I + \hat{\mathcal{M}_i} \right)
% \end{equation}


\begin{equation} \label{eq:m2r2_approximating_ea}
h_{E_j} + \sum_{i=E_j}^{N-1} \left( \mathcal{A}_i(h_i) + \mathcal{M}_i(h_i + \mathcal{A}_i(h_i)) \right) \sim p_{E_j} + \sum_{i=E_j}^{E_{j+1}-1} \left( \hat{\mathcal{A}_i}(p_i) + \hat{\mathcal{M}_i}(p_i + \hat{\mathcal{A}_i}(p_i)) \right), 
\end{equation}

% \begin{equation} \label{eq:fast_residual_transformation}
% p_{E_{j+1}} = p_{E_j} + \sum_{i=E_j}^{E_{j+1}-1} \left( \hat{\mathcal{A}_i}(p_i) + \hat{\mathcal{M}_i}(p_i + \hat{\mathcal{A}_i}(p_i)) \right) \quad \text{where} \quad \hat{\mathcal{A}_i} = \hat{f}(c_i, p_{i}), \hat{\mathcal{M}_i} = \hat{g}(h_i) 
% \end{equation}






where $p_{E_j}$ is initialized from the slow residual state $h_{E_j}$ at each early exit interval $E_j$ using an identity transformation (see ~\cref{fig:m2r2_main}). As shown in ~\cref{fig:m2r2_residual_sim}, accelerated residuals offer a smoother, more consistent shift in residual direction across layers, in contrast to the abrupt changes typically seen at early exit points in standard early exit methods. Moreover, the normalized cosine similarity between accelerated states at early exit intervals and final residual states is substantially higher compared to traditional early exit techniques, highlighting improved alignment with final layer representations. Traditional adaptive compute methods are constrained by two principal factors: the number of tokens eligible for early exit at intermediate layers and the precision of early exit decision. If residual streams fail to saturate early, the majority of tokens remain ineligible for exit, thereby diminishing potential speedups. Additionally, imprecise delineations between tokens suitable for early exit can lead to underthinking (premature exits that adversely affect accuracy) or overthinking (unnecessary processing that compromises efficiency) ~\cite{zhou2020self, dai2020dynamic}. Enhanced early alignment using ~\cref{eq:m2r2_approximating_ea} helps to address  first issue. To address the second issue we introduce Accelerated Residual Latent Attention, which dynamically assesses the saturation of the residual stream, allowing for a more precise differentiation between tokens that can exit early and those requiring further processing.

% This results in uniform change in residual direction    
% % We keep $\mathcal{A} = \hat{\mathcal{A}}$, while $\hat{\mathcal{M}}$ is accelerated by a factor of $2 - (N_{E}+1)X$ relative to the slower residual transformation $\mathcal{M}$, where $N_E$ represents number of early exiting intervals.
% Figure~\cref{fig:rate_change_comparison} illustrates the comparative rate of change between these transformation streams.



% fig:rate_change_comparison
% - grid plot x axis -> layer id (0, 8) , y axis -> layer id -> dark color cell for max similarity , lighter for lower 
% 
-------------------------------------------------------
Let's consider residual stream $h_i$ traverses through interval $E_j$ to $E_{j+1}$ and undergoes residual transformation given by 
\begin{equation}
h_{E_{j+1}} = h_{E_j} \prod_{i=E_j}^{E_{j+1}} \left( 1 + \delta_i \right)    
\end{equation}

where $\delta_i$ denotes non-linear transformation added by layer $i$. Each non-linear transformation of layer $i$ is a function of previous contextual representation, $c_i$ and input residual stream $h_i-1$ as
$\delta_i = f(c_i, h_{i-1})$ 

One way to exit early at exit $E_j+1$ is to assume that residual transformation from $E_j+1$ to final layer $N-1$ can be approximated by a linear function $\phi$ as $h_{N-1} \sim \Phi(h_{E_j+1})$ and most conventional approaches such as \todo{cite EA papers} use this approach. In other words, 

\begin{equation}
\Phi(h_{E_j+1} \sim h_{E_j+1} \prod_{i=E_j+1}^{N} \left( 1 + \delta_i \right)   
\end{equation}

This approach suffers from two primary issues, linearity assumption from $h_E_j+1$ to $H_N-1$ if often incorrect, particularly when $E_j << N$. More importantly, linear transformation $\Phi$ doesn't consider effect of context $C_i$. M2R2  effectively addresses these issues as accelerated residual stream at interval $E_j+1$ can be represented as 

\begin{equation}
r_{E_{j+1}} = r_{E_j} \prod_{i=E_j}^{E_{j+1}} \left( 1 + \gamma_i \right)    
\end{equation}

where $\gamma_i$ denotes non-linear transformation added by layer $i$ to previous accelerated residual $r_i-1$. Similar to $\delta_i$, non-linear transformation $\gamma_i$ considers context $C_i$ as 
$\gamma_i = g(c_i, r_{i-1})$. So in summary, slow residual transformation is approximated by accelerated residual as: 

\begin{equation}
h_{E_j} \prod_{i=E_j}^{N} \left( 1 + \delta_i \right) \sim h_{E_j} \prod_{i=E_j}^{E_j+1} \left( 1 + \gamma_i \right)
\end{equation}

It's worth noting that accelerated residual $r_i$ and slow residual $h_i$ are processed concurrently at layer $i$ by constructing proper attention mask such as attention of slow residual is represented as 

$MHA(H_it, H_{i<=t}, H_{i<=t}$ while attention of fast residual is computed as 

$MHA(r_it, H_{i<=t}, H_{i<=t}$ where $MHA(q,k,v$ denotes multi head attention between query, $q$, key $k$ and value $v$.


------------------------------------------------------------------

Vertical latent attention on accelerated residual is computed as 
$MHA(S_mt, S(Ej<=i<=m)t, S(Ej<=i<=m)t)$ where $Smt$ denotes query/key/value projection in latent domain at layer $m$ at time $t$. 
------------------------------------------------------------------

Gradient conflict Avoidance: 

Let's consider $w_j$ is a trainable parameter that belongs to a layer between $E_j$ and $E_j+1$. Consider early exit loss at gate $E_j+1$, $L_j+1$, gradient propagation of $w_j$ at another trainable parameter $w_j-n$ can be gives as 

$\sum_{k=E_j-n}^{E_j} \beta_k \frac{\partial L_{E_k}}{\partial w_k}$

where $\beta_j$ denotes backward transformation coefficient for weight $w_j$ to reach gate $E_j$. 
 
On the other hand, gradient propagation in proposed approach can be represented as 

\[
\frac{\partial L_{E_j}}{\partial w_j} = 
\begin{cases} 
\beta_j \frac{\partial L_{E_j}}{\partial w_j} & \text{if } E_j \leq w_j \leq E_{j+1} \\
0 & \text{otherwise}
\end{cases}
\]







% \begin{figure}[ht]
%     \centering
%     \includegraphics[width=0.8\textwidth, height=5cm]{rate_change_comparison.png}
%     \caption{Rate of change comparison between fast and slow residual streams.}
%     \label{fig:rate_change_comparison}
% \end{figure}

%vary k and and plot EA accuracy for larger and smaller models. 

% \begin{figure}[ht]
%     \centering
%     \includegraphics[width=0.5\textwidth,height=5cm]{sections/figures/alignment_comparison_dialogsum.pdf}
%     \caption{Alignment of exited tokens for different early exit layers using traditional early exiting heads, dedicated faster networks, and faster residuals.}
%     \label{fig:small_model_early_exiting}
% \end{figure}


\textbf{Accelerated Residual Latent Attention} \label{method_arla}

In the context of residual streams, we observe that the decision to exit at a given layer can be more effectively informed by analyzing the dynamics of residual stream transformations, instead of solely relying on a classification head applied at the early exit interval $E_j$. To capture the subtle dynamics of residual acceleration, we propose a \textit{Accelerated Residual Latent Attention} (ARLA) mechanism. This approach involves making the exit decision at gate $E_j$ by attending to the residuals spanning from gate $E_{j-1}$ to $E_j$, rather than considering only the residual at gate $E_j$. To minimize the computational overhead associated with exit decision-making, the attention mechanism operates within the latent domain as depicted in ~\cref{fig:arla_arch}. Formally, for each interval $[E_j, E_{j+1}]$, the accelerated residuals are projected into Query ($Q^s_{E_j}, \ldots, Q^s_{E_{j+1}}$), Key ($K^s_{E_j}, \ldots, K^s_{E_{j+1}}$), and Value ($V^s_{E_j}, \ldots, V^s_{E_{j+1}}$) vectors, with latent dimension $d^s$ for $Q^s$, $K^s$, and $V^s$ being significantly smaller than hidden dimension of $p$.\footnote{We use $d^s = 64$ for experiments described in ~\cref{sec:experiments}.} Notably, when the router is allowed to make exit decisions at gate $E_j$ based on residual change dynamics, we observe that the attention is not confined to the residual state at $E_j$ but is distributed across residual states from $E_{j-1}$ to $E_j$, %as illustrated in Figure~\ref{fig:vertical_latent_attention_dynamics}. 
This broader focus on residual dynamics significantly reduces decision ambiguity in early exits, as demonstrated in Figure~\ref{fig:roc_arla}, which contrasts routers based on the last hidden state, and the proposed ARLA router.

%show R -> S transformation. 
%show parameter and flop overhead as compared to adapter on last hidden state.

% \begin{figure}[ht]
%     \centering
%     \includegraphics[width=0.5\textwidth,height=5cm]{sections/figures/roc_arla.pdf}
%     \caption{ROC curves of early exit decision strategies: confidence-based methods (CALM/LITE), routers based on the accelerated hidden state, and latent attention routers.}
%     \label{fig:decision_making_comparison}
% \end{figure}

% \begin{figure}[ht]
%     \centering
%     \includegraphics[width=0.5\textwidth,height=5cm]{vertical_latent_attention.png}
%     \caption{Vertical latent attention mechanism for optimizing early exit decisions by considering residuals from gate \(M\) through \(M-1\).}
%     \label{fig:vertical_latent_attention}
% \end{figure}

\begin{figure}[ht]
    \centering
    \begin{subfigure}{0.52\textwidth}
        \centering
        \includegraphics[width=\textwidth, height = 4cm]{sections/figures/arla_arch.pdf}
        \caption{Accelerated Residual Latent Attention (ARLA): Accelerated residuals between early exit gates are projected into latent domain and attention over residual states within the interval is computed to capture residual dynamics and exit decision is made based on residual saturation.}
        \label{fig:arla_arch}
    \end{subfigure}%
    \hfill
    \begin{subfigure}{0.45\textwidth}
        \centering
        \includegraphics[width=\textwidth, height = 4.5cm]{sections/figures/vla_roc.pdf}
        \caption{ROC classification curves of early exit decision strategies using a linear router used on last residual state ~\cite{schuster2022confident, varshney-etal-2024-investigating, chen2023eellm}  and using ARLA approach that considers residual dynamics. }
        \label{fig:roc_arla}
    \end{subfigure}
    \caption{Effectiveness of ARLA in capturing residual dynamics for early exiting decisions.}


\end{figure}



% \begin{figure}[ht]
%     \centering
%     \includegraphics[width=1\textwidth,height=5cm]{sections/figures/arla.pdf}
%     \caption{fig that plots 32 rows 2 cols heatmap showing attention at each gate}
%     \label{fig:vertical_latent_attention_dynamics}
% \end{figure}

\subsubsection{Self Speculative Decoding} \label{method_self_speculative_decoding}

An alternative means to exploit the early alignment properties of our approach is through the use of accelerated residual states for speculative token sampling to accelerate autoregressive decoding. Speculative decoding aims to speed up memory-bound transformer inference by employing a lightweight draft model to predict candidate tokens, while verifying speculated tokens in parallel and advancing token generation by more than one token per full model invocation \cite{leviathan2023fast, chen2023accelerating, xia2023speculative, miao2023specinfer}. Despite its effectiveness in accelerating large language models (LLMs), speculative decoding introduces substantial complexity in both deployment and training. A separate draft model must be specifically trained and aligned with the target model for each application, which increases the training load and operational complexity ~\cite{chen2023accelerating}. Additionally, this approach is resource-inefficient, as it requires both the draft and target models to be simultaneously maintained in memory during inference \cite{leviathan2023fast, chen2023accelerating}. 

One strategy to address this inefficiency is to leverage the initial layers of the target model itself to generate speculative candidates, as depicted in ~\cite{Tang2024}. While this method reduces the autoregressive overhead associated with speculation, it suffers from suboptimal acceptance rates. This occurs because the linear transformation employed for translating hidden states from layer $k$ to the final layer $N$ is typically a poor approximation, as discussed in ~\cref{sec:motivation} and ~\cref{method_early_exiting}. Our approach resolves this limitation by utilizing accelerated residuals, which demonstrate higher fidelity to their slower counterparts. By utilizing accelerated residuals operating at a rate of $N/k$, where $k$ denotes the number of layers used for candidate speculation, we are able to efficiently generate speculative tokens for decoding.\footnote{We typically set $k = 4$ to balance the trade-off between autoregressive drafting overhead and acceptance rate, as discussed in~\cref{sec:experiments}.}
 This technique not only obviates the need for multiple models during inference but also improves the overall efficiency and effectiveness of speculative decoding.

\begin{figure}
    \centering    \includegraphics[width=1\linewidth]{sections/figures/m2r2_aot_loading.pdf}
    \caption{Ahead-of-Time Expert Loading: M2R2 accelerated residual stream predicts experts required for future layers, reducing reliance on on-demand lazy loading. Speculative pre-loading is efficiently overlapped with computation of multi-head attention (MHA) and MLP transformations. Only incorrectly speculated experts are loaded lazily, resulting in faster inference steps and improved computational efficiency. Here, H indicates LBM Host while D indicates HBM Device.}
    \label{fig:moe_expert_aot_loading}
\end{figure}


\subsubsection{Ahead of Time Expert Loading:} \label{method_aot_expert_loading}

Recent advancements in sparse Mixture-of-Experts (MoE) architectures ~\cite{shazeer2017outrageously, fedus2022switch, artetxe2019massively, lepikhin2020gshard, zoph2022designing} have introduced a paradigm shift in token generation by dynamically activating only a subset of experts per input, achieving superior efficiency in comparison to dense models, particularly under memory-bound constraints of autoregressive decoding \cite{fedus2022switch, zoph2022designing}. This sparse activation approach enables MoE-based language models to generate tokens more swiftly, leveraging the efficiency of selective expert usage and avoiding the overhead of full dense layer invocation. In dense transformer models, pre-loading layers is a common strategy to enhance throughput, as computations of current layer can be overlapped with pre-loading of next layer parameters ~\cite{narayanan2021efficient, shoeybi2020megatron}. However, MoE models face a unique challenge: expert selection occurs dynamically based on previous layer’s output, making it infeasible to preload next layer’s experts in parallel. This limitation results in inherent latency, as expert loading becomes a sequential, on-demand process ~\cite{lepikhin2020gshard, fedus2022switch}.

To address this inefficiency, our method introduces a mechanism with \textit{accelerated residuals}, which not only captures key characteristics of base slower residual states but also exhibit high cosine similarity with their final counterparts (as illustrated in \cref{fig:m2r2_residual_sim}). By employing accelerated residual streams, we can effectively predict the necessary experts for future layers well in advance of their actual invocation. Specifically, using a $2\times$ accelerated residual, the experts needed for layers $2i+2$ and $2i+3$ can be identified while still computing in layer $i$, thus overcoming the bottleneck of sequential, on-demand expert selection and mitigating latency in the decoding pipeline, as shown in \cref{fig:moe_expert_aot_loading}. Note that, we use fixed set of accelerator adapters for transforming accelerated residuals (as discussed in ~\cref{m2r2_method}) while slow residual is transformed via expert routing mechanism. 

Furthermore, our approach integrates a Least Recently Used (LRU) caching strategy, which enhances memory efficiency by replacing the least recently used experts with speculated experts that are anticipated to be needed in upcoming layers. This hybrid approach of preemptive expert loading with LRU caching yields substantial improvements over traditional on-demand loading or standalone caching strategies. By minimizing cache misses and efficiently managing memory, this approach addresses both compute and memory bottlenecks, leading to faster, more resource-efficient token generation in MoE architectures. A comprehensive evaluation of this strategy, in relation to state-of-the-art methods, is provided in \cref{experiments_aot}, and the compute and memory traces on an A100 GPU are detailed in \cref{fig:moe_aot_cuda_trace}.



% Recent advancements in sparse Mixture-of-Experts (MoE) architectures have introduced the concept of utilizing distinct computational paths for different tokens \cite{shazeer2017outrageously}. This approach, wherein only a subset of experts are activated per input, enables MoE-based language models to generate tokens more swiftly compared to their dense counterparts due to memory-bound nature of auto-regressive decoding. In dense models, pre-loading layers in advance is a common strategy to enhance computational efficiency. However, this technique is not applicable to MoE models, where expert selection occurs dynamically based on the outputs of previous layers, preventing parallel pre-fetching of experts.

% Our proposed method addresses this inefficiency. Accelerated residuals, which are highly similar to their slower counterparts (see \cref{fig:similarity}), can reliably predict the necessary experts ahead of time. For instance, by utilizing $2X$ accelerated residual stream, we can predict the experts needed for the layer $2i+1$ and $2i+3$ while carrying out computation in layer $i$. This enables us to commence expert loading significantly earlier, as illustrated in \cref{expert_loading}, effectively mitigating the delays observed with the naive on-demand expert loading. Additionally, our method benefits from incorporating a Least Recently Used (LRU) strategy, where speculated experts replace those that are least recently utilized, resulting in improved performance compared to using either strategy alone. For a comprehensive evaluation, refer to \cref{moe_trace}, which provides a CUDA compute and memory trace of our approach executed on <>.



% A naive solution involves using the residual state of the previous layer along with the gating function of the next layer to predict which experts need to be loaded, and initiating the expert loading process in parallel with the attention computation of the next layer. Yet, as shown in \cref{fig:MOE_attn_vs_loading_time}, the attention computation for medium to long contexts is considerably faster than the expert loading time, making this approach inefficient.




\subsection{Training} \label{method_training}
% This approach is feasible due to the absence of gradient conflicts, as discussed in \cref{sec:grad_conflict}.

To accelerate residual streams, we employ parallel accelerator adapters as described in \cref{m2r2_method}.  For the early exiting use-case outlined in \cref{method_early_exiting}, we define the training objective for these adapters using the following loss function, which combines cross-entropy loss at each exit $E_j$ with distillation loss at each layer $i$. Loss weights coefficients $\alpha_0$ and $\alpha_1$ are employed to balance contribution of corresponding losses.

\begin{align} \label{eq:mr_loss}
L_{\text{m2r2}} = \underbrace{-\alpha_0 \sum_{j=1}^{J} \sum_{t=1}^{T} \log p_{\theta} \left( \hat{y}_t^{E_j} \mid y_{<t}, x \right)}_{\text{cross-entropy loss}} 
+ \underbrace{\alpha_1\sum_{i=1}^{E_{J-1}} \sum_{t=1}^{T} \| \mathbf{p}_{t}^{i} - \mathbf{h}_{t}^{((i - E_{j(i)}) \cdot R_i) + E_{j(i)})} \|^2}_{\text{distillation loss}}.
\end{align}

where $\hat{y}_t^{E_j}$ denotes the predictions from the accelerated residual stream at layer $E_j$ and time step $t$, $y_t$ represents the corresponding ground truth tokens, and $x$ indicates previous context tokens. The distillation loss at each layer $i$ is computed by comparing accelerated residuals at layer $i$ with slow residuals at layer $(i - E_{j(i)}) \cdot R_i + E_{j(i)}$, where $R_i$ denotes the rate of accelerated residuals at layer $i$ while $E_{j(i)}$ represents the most recent gate layer index such that $E_{j(i)} <= i$. \( J \) represents the total number of early exit gates, N denotes number of hidden layers and $E_j$ denotes layer index corresponding to gate index $j$ and \( T \) denotes the sequence length. 

In dynamic compute settings, after training of accelerator adapters, we optimize the query, key, and value parameters governing the ARLA routers (see ~\cref{method_arla}) across all exits in parallel on binary cross entropy loss between predicted decision and ground truth exiting decision. The ground truth labels for the router are determined based on whether the application of the final logit head on $\hat{y}_t^{E_j}$ yields the correct next-token prediction. 


% The objective for this optimization is defined by the following loss function:


%TODO are equations required ? 
% \begin{equation} \label{eq:arla_loss_combined}\small
%     L_{\text{arla}} = -\frac{1}{N} \sum_{t=1}^{T} \left( \sum_{j=1}^{E_n} \left[ O_t^{E_j} \log(\hat{O}_t^{E_j}) + (1 - O_t^{E_j}) \log(1 - \hat{O}_t^{E_j}) \right] \right), \quad \text{where} \quad 
%     O_t^{E_j} = \begin{cases} 
%     1, & \text{if } L(\hat{y}_t^{E_j}) = y_t^{E_j} \\
%     0, & \text{otherwise}
%     \end{cases}
% \end{equation}

% where $\hat{O}_t^{E_j}$ represents the binary predicted logits produced by the vertical latent attention router, as described in \cref{sec:arla}, at gate $E_j$ and time step $t$, and $O_t^{E_j}$ denotes the corresponding ground truth labels. The ground truth labels for the router are determined based on whether the application of the logit head on $\hat{y}_t^{E_j}$ yields the correct next-token prediction. The parameters controlling vertical latent attention are trained concurrently to ensure consistency and efficient use of computational resources.

For self-speculative decoding, as described in \cref{method_self_speculative_decoding}, the training objective remains the same as \cref{eq:mr_loss}, but with the number of intervals set to $J = 1$ and the rate of residual transformation set to $R_n = N/k$, where the first $k$ layers generate speculative candidate tokens. In the context of Ahead-of-Time Expert Loading for Mixture-of-Experts (MoE) models (see \cref{method_aot_expert_loading}), setting the rate of residual transformation to $R_n = 2$ typically offers a good trade-off between the accuracy of expert speculation and AoT pre-loading of experts. 

% Thus, we set $J = 1$ and $E_1 = 16$.


~\subsection{FLOPs Optimization} \label{sec:flops_optimization}

Naively implemented, M2R2 incurs higher FLOP overhead compared to traditional speculative decoding and early exiting approaches such as ~\cite{medusa, schuster2022confident, Tang2024}. However, modern accelerators demonstrate compute bandwidth that exceeds memory access bandwidth by an order of magnitude or more~\cite{databricksLLMInference2023, jouppi2021ten}, meaning increased FLOPs do not necessarily translate to increased decoding latency. Nevertheless, to ensure fair comparison and efficiency in compute bound scenarios, we introduce targeted optimizations.

~\textbf{Attention FLOPs Optimization} For medium-to-long context lengths, attention computation dominates FLOPs in the self-attention layer, surpassing the contribution from MLP layers. Specifically, matrix multiplications involving queries, cached keys, and cached values scale with $l_{kv} * l_{q}$ where $l_{kv}$ denotes previous context length and $l_q$ denotes current query length. Since M2R2 pairs accelerated residuals with slow residuals, a naive implementation results in twice the FLOPs consumption compared to a standard attention layer. To address this, we limit the attention of accelerated residual stream to selectively attend to the top-k most relevant tokens, identified by the slow residual stream based on top attention coefficients\footnote{We set to k = 64 and attend to top 64 tokens as identified by the slow residual stream.}. This is possible since slow and accelerated residual streams are processed in same forward pass and accelerated streams have access to attention coefficients of slow stream. Note that, the faster residual stream still retains the flexibility to assign distinct attention coefficients to these tokens. Furthermore, we design the faster residual stream to employ only 8 attention heads, compared to the 32 heads used in the slow residual stream of the Phi-3 model, reducing query, key, value, and output projection FLOPs by a factor of 1/4. ~\cref{fig:m2r2_num_heads_ablation} indicates effect of using a slicker stream on alignment. As depicted, using $\hat{n}_h = 8$ offers a good trade-off between alignment and FLOPs overhead. 

~\textbf{MLP FLOPs Optimization} The accelerator adapters operating on the accelerated residual stream are intentionally designed with lower rank than their counterparts in the base model. This reduces FLOP overhead by a factor proportional to $hiddenSize / rank$. Additionally, since the faster residual stream uses only 8 attention heads (compared to 32 in the slow residual stream of Phi-3), the subsequent MLP layers process a smaller set of activations, further reducing FLOPs by another factor of 1/4.

These optimizations significantly reduce the FLOP overhead per speculative draft generation, as illustrated in ~\cref{fig:flops_optmization}. Notably, while traditional early-exiting speculative approaches such as DEED require propagating the full slow residual state through the initial layers, incurring substantial computational costs, M2R2 achieves efficient token generation via slimmer, low-rank faster residual streams. In contrast, Medusa introduces considerable FLOP overhead due to per-head computations scaling with $d^2+dv$\footnote{Here $d$ denotes hidden state dimension while $v$ denotes vocab size.}, whereas M2R2 employs low-rank layers for both MLP and language modeling heads, maintaining computational efficiency. All experiments involving the M2R2 approach, as detailed in ~\cref{sec:experiments}, are conducted using these FLOPs optimizations.









% \[
% O_t^{E_j} = 
% \begin{cases} 
% 1, & \text{if } L(\hat{y}_t^{E_j}) = y_t^{E_j} \\
% 0, & \text{otherwise}
% \end{cases}
% \]




%add distillation
% We train accelerator adapters described in \cref{m2r2_method} to accelerate residual streams on next token prediction all in parallel since there are no gradient conflict issues as described in \cref{sec:grad_conflict}.

% \begin{align} \label{eq:mr_loss}
% L_{mr} =  & -\sum_{j = 1}^{E_n} (\sum_{t=1}^{T}\log p_{\theta} (\hat{y}_t^{E_j} | \hat{y}_{<t}, x)) \nonumber
% \end{align}

% where $\hat{y_t^{E_j}}$ denotes predicted logits obtained from accelerated residual stream at gate $E_j$ and time-step $t$ while $y_t^{E_j}$ denotes corresponding truth tokens. 

% Upon training of adapters responsible for accelerating residual streams, we train query, key, value parameters responsible for vertical latent attention of all gates in parallel as

% \begin{equation} \label{eq:arla_loss}
%     L_{arla} = -\frac{1}{N} (\sum_{t=1}^{T}(1\sum_{j=1}^{E_n} \left[ O_t^{E_j} \log(\hat{O}_t^{E_j}) + (1 - o_t^{E_j}) \log(1 - \hat{o_t}_{E_j}) \right]))
% \end{equation}

% where $\hat{O_t^{E_j}}$ denotes binary predicted logits obtained from vertical latent attention router described in \cref{sec:arla} at gate $E_j$ and timestep $t$ while $O_t^{E_j}$ denotes corresponding truth label. Truth labels for router are obtained by computing whether logit head application on $\hat{y}_t^j$ results in true next token prediction. Formally speaking, 

% $O_t^{E_j} = 1 if L(\hat{y_t^{E_j}}) == y_t^{E_j} , 0 otherwise$. 

% Parameters responsible for vertical latent attention are also trained in parallel as well. 

%todo: training slow and fast residuals together and distillation can be two training mdoes. 
%Distillation can be an ablation. 




% Although transformer decoding is memory bound on most mainstream accelerators, there could be scenarios where flop savings are crucial. For instance, on on-device settings power consumption is directly correlated with flops per decoding step and reducing flops does help with overall energy consumption. Vanilla early exiting methods help with flop reduction but suffer from mismatch between training and inference due to early exited tokens. If token at decoding step $t$, $T_t$ exited at layer $E_i$, while token $T_{t+k}$ exits at layer $E_j$ such that $E_i < E_j$, hidden state $H_{t+k}l$ does not have corresponding hidden state $H_tl$ to attend to where $E_i < l <= E_j$. One solution that's often used in literature is to rely on last hidden state available, $H_t{E_j}$, however it tends to be sub-optimal and does affect generation quality \cite{ref}.  To alleviate this mismatch while reducing flops, we train router such that attention mask between token $T_{t+k}$ and token $T_{<t+k}$ is given by: 

% \begin{equation}
%     a_{T_{{t+k}{T_{<t+k}}} = 1 if  E_{T_{<t+k}} >= E{T_{t+k}}
%     else 0
% \end{equation}

% This attention mask enables router to account for exited tokens and get trained accordingly. Since attention mechanism during decoding remains exactly same as that during training, impact on generation quality tends to be minimal as noted in \cref{fig:gen_auality_with_and_without_recompute_attention_show_flops}.  Although MoD does not suffer from training and inference mismatch, we observe that it suffers from discountinuity between pre-training and super-vised fine-tuning resulting in sub-optimal perplexity. On the other hand, our method doesn't not require pre-training , doesn't suffer from discountinuity, and achieves much better perplexity in super-vised fine-tuning and instruction tuning setups as shown in \cref{fig:Mod_vs_m2r2_loss_curves}.






% Our techniques are directly applicable in such scenarios.    




%expert loading with cuda streams in experiments
\input{text/evaluation_v2.tex}
\section{Related Work}
Alongside a discussion of what is meant by LLM harmfulness,
this section covers two distinct strands of related work: measuring types of harm in LLMs, and LLMs for diverse annotation tasks. %First,

%Different kinds of 
Diverse undesirable LLM outputs, from toxic language to privacy invasion, have been discussed in the observed \cite{banko-etal-2020-unified}. Here we review the ones we include in our definition of ``harm.'' %definition. Plus, we review LLMs as judges. 
Toxic content can be elicited from both generative  \cite{deshpande2023toxicity} and masked LLMs \cite{ousidhoum-etal-2021-probing}. 
%Among ways 
To measure toxic or hateful language, some use APIs such as PerspectiveAPI \cite{lees2022new} or HateBERT \cite{caselli-etal-2021-hatebert}. \citet{openai2024gpt4technicalreport} report that GPT4 produces toxic content 0.78\% of the time, versus 6.48\% in GPT3.5.
%as opposed to GPT3.5 with 6.48\%. On the other hand,
\citet{dubey2024llama} report that llama3-70B produces harmful content 5\% of the time, %whereas the 405B model generates harm 3\% of the time. 
compared to 3\% in the 405B model.
Instead of %single value classifiers to measure harm, 
reporting an absolute rate, we focus on relative harmfulness of different LLMs. %, so we point to recent work on LLMs for annotation.

The first category of harm we consider is social stereotyping and bias. %discrimination. It has been shown that 
LLMs can perpetuate social bias based on gender, race, religion etc. \cite{lin-etal-2022-gendered,bender2021dangers,field-etal-2021-survey,gupta-etal-2024-sociodemographic,andriushchenko2024agentharm,mazeika2024harmbench}. This can marginalize these groups more, and results in less fair model performance. \citet{guo2024hey} designed a competition to elicit biased output from LLMs to assess the perception of bias from non-expert users. %The first part of our work is similar to this analysis, but 
We also intentionally elicit harmful output, going %we look at other types of harms besides bias.
beyond social bias.

%When the models become stronger, they become more robust to jailbreaking attacks to elicit harmful content. However, there are datasets that can still jailbreak models to produce harmful content \cite{andriushchenko2024agentharm,mazeika2024harmbench}.

Our second category of harm is offensiveness and toxicity, which %. As opposed to stereotyping or social discrimination, this harm 
%is more subjective and harder to define than the previous category, so there 
lacks an established definition due to its greater subjectivity \cite{dev-etal-2022-measures,korre-etal-2023-harmful}. We include hate speech (HS) and abusive language as toxic content. HS can be defined as expressions of offensive and discriminatory discourse towards a group or an individual based on characteristics such as race, religion, nationality, or other group characteristics \cite{john2000hate,jahan2023systematic,basile2019semeval,davidson2017automated}. It includes racism, negative stereotyping, and sexist language. On the other hand, abusive language is content with inappropriate words such as profanity or disrespectful terms. It also includes psychological threats such as humiliation. %or constant criticism. %Toxic content can be elicited from both generative models \cite{deshpande2023toxicity} and masked language models \cite{ousidhoum-etal-2021-probing}.

%In addition to obvious toxic content, LLMs can generate diverse implicit toxic outputs using reinforcement learning with favoring toxic content in the reward function \cite{wen-etal-2023-unveiling}.  Regarding the subjectivity of this task, \cite{korre-etal-2023-harmful} reannotate the existing datasets with different definitions of toxicity and show that broader definitions result in more robust annotations, but interannotator agreements are still lower than 0.5. \cite{dev-etal-2022-measures} also point out the lack of definition for bias and harm in general and propose a framework to guide researchers during the development of bias measures.

Harm can be implicit, such as privacy invasion
%We are also interested in privacy invasion,
where there is 
leakage of personal information. %leakage from the model. 
%LLMs can memorize details of the training data and then leak private information such as 
This includes social security numbers, phone numbers, or bank account information \cite{carlini2021extracting,brown2022does}. 
%There are several frameworks to test the privacy of LLMs \cite{li2024llm} and generate data for personal attribute inference \cite{yukhymenko2024synthetic,kim2024propile}.

%Our definition of harm includes hate speech (HS) as well. HS can be defined as \textcolor{red}{expressions of} hatred towards a social group, the humiliation of the members of a group, or %communication disparaging  extreme disparagement of a person or a group based on race, color, ethnicity, gender, sexual orientation, nationality, religion, or other group characteristics .

For data annotation, LLMs
%Besides text generation, 
%LLMs have been used to annotate data because they 
can %be comparable to 
replace humans for some tasks, %and make the annotation process faster and cheaper 
with gains in efficiency and economy \cite{tan2024large}. They have been used for sociological annotations such as for classification of stance, bots or humor  \cite{ziems2024can,zhu2023can}. For tasks such as topic and frame detection or sentence segmentation they can surpass crowd-workers
%Some works show that they can surpass crowd-workers for some tasks such as topic and frame detection or sentence segmentation %into research aspects 
\cite{he2024if,gilardi2023chatgpt}. Some have argued that human-LLM collaboration results in more reliable annotation \cite{he2024if,zhang2023llmaaa,kim2024meganno+}. In addition to more objective tasks,
%LLMs have been used to annotate data %even 
they have been applied to subjective annotations such as offensiveness and abusiveness \cite{pavlovic-poesio-2024-effectiveness,zhu2023can,he2023annollm}, %. For example, LLMs are used as judges to rank responses from different LLMs 
or to rank outputs from different LLMs based on helpfulness, accuracy, or relevance \cite{zheng2023judging,lin2024wildbench,dubois2024length}. These works tend to focus on human-large LLM interactions, whereas we focus on single-turn responses from smaller LLMs. We inspire from \citet{zheng2023judging} but we only measure harm instead of overall performance. Plus, we use 3 LLMs to evaluate smaller LLMs.
\section*{Conclusion}
This paper aims to enhance our understanding of the computational complexity of computing various Shapley value variants. We found that for various ML models --- including decision trees, regression tree ensembles, weighted automata, and linear regression --- both local and global interventional and baseline SHAP can be computed in polynomial time under HMM modeled distributions. This extends popular algorithms, such as TreeSHAP, beyond their empirical distributional scope. We also establish strict complexity gaps between the various SHAP variants (baseline, interventional, and conditional) and prove the intractability of computing SHAP for tree ensembles and neural networks in simplified scenarios. Overall, we present SHAP as a versatile framework whose complexity depends on four key factors: \begin{inparaenum}[(i)] \item model type, \item SHAP variant, \item distribution modeling approach, \item and local vs. global explanations\end{inparaenum}. We believe this perspective provides deeper insight into the computational complexity of SHAP, paving the way for future work.




%We believe that our framework provides a more intricate understanding of SHAP computation complexity across different models, distributions, and variants, paving the way for further research.

Our work opens promising directions for future research. First, expanding our computational analysis to other SHAP-related metrics, such as asymmetric SHAP~\citep{frye20} and SAGE~\citep{covert2020understanding}, would be valuable. Additionally, we aim to explore more expressive distribution classes and relaxed assumptions beyond those in Section \ref{sec:tractable} while maintaining tractable SHAP computation. Finally, when exact computation is intractable (Section \ref{sec:intractable}), investigating the approximability of SHAP metrics through approximation and parameterized complexity theory~\citep{downey2012parameterized} is an important direction.

%Our work opens several promising avenues for future research on the computational properties of explainable AI methods, with a particular focus on SHAP. First, it would be interesting to broaden the computational analysis conducted in this work to include other popular SHAP-related metrics in the literature, such as asymmetric SHAP \cite{frye20} and SAGE \cite{covert2020understanding}. Also, in the future, we aim to explore more expressive distribution classes and relaxed distributional assumptions—extending beyond those examined in Section \ref{sec:tractable} —that still yield tractable SHAP computation. Finally, when exact computation proves intractable (Section \ref{sec:intractable}), it is worthwhile to theoretically investigate the question of the approximability of computing the SHAP metrics across various configurations, through the lens of approximation and parametrized complexity theory \cite{arora2009computational}.

%This paper aims to deepen our understanding of the computational complexity involved in obtaining different Shapley value variants. We found that for a variety of ML models, including decision trees, tree ensembles for regression, weighted automata, and linear regression models — computing both local and global interventional and baseline SHAP can be done in polynomial time when distributions are modeled by HMMs. This extends the distributional scope of popular algorithms like TreeSHAP, which is limited to empirical distributions. Additionally, we demonstrate a strict complexity gap between SHAP variants, showing that interventional and baseline SHAP can be strictly easier to compute than conditional SHAP. Despite these positive results, we uncovered intractability for various SHAP variants in neural networks and tree ensembles. Finally, we provided generalized complexity relations across SHAP variants. We believe that our framework offers a deeper understanding of the complexity involved in computing SHAP across various variants, models, distributions, as well as in both local and global computations, laying the groundwork for future research.


\bibliographystyle{plainnat}
\bibliography{references}

\appendix
% ----------------------------------------------
\newpage
\section*{supplementary material}
In this supplementary material, we will provide a theoretical analysis to the proposed memory efficient Transformer adapter (META) in Section~\ref{secS1}, provide a detailed description of the experimental datasets in Section~\ref{secS2}, provide a detailed description of the experimental settings in Section~\ref{secS3},
provide more result comparisons under different pre-trained weights in Section~\ref{secS4},
provide more ablation study results in Section~\ref{secS5}, show class activation map comparisons of instance segmentation before and after adding the Conv branch in Section~\ref{secS6},qualitative visualizations of instance segmentation and semantic segmentation results in Section~\ref{secS7},  as well as the pseudo-code for when the stripe size is set to $2$ in Section~\ref{secS8}. 
% -------------------------------------------
\section{Theoretical Analysis of META}
\label{secS1}
% -------------------------------------------
{\color{red}{\emph{This supplementary is for Section~3 of the main paper.}}} In this section, we will prove that META exhibits superior generalization capability and stronger adaptability compared to existing ViT adapters. 
%
To achieve this goal, we will prove that the proposed memory efficient adapter (MEA) block possesses larger information entropy (IE) than the existing attention-based ViT adapters~\citep{hu2022lora,jie2023fact,chen2022vision,ma2024segment,luo2023forgery,shao2023deepfake}, which provides evidence that the MEA block has more comprehensive feature representations. Then, based on the maximum mean discrepancy (MMD) theory~\citep{cheng2021neural,arbel2019maximum,wang2021rethinking}, larger IE in the ViT adapter framework leads to superior generalization capability and stronger adaptability. The detailed theoretical analysis process is as follows:

\begin{lemma}
% ---------------------------------
In any case of mutual information, the MEA block will gain larger information entropy after fusing $\textbf{X}_{vit}$ and $\textbf{X}_{con}$.
% ---------------------------------
\end{lemma}
% ---------------------------------
\begin{proof}
As introduced in Section~3.2 of the main paper, the proposed MEA block can be viewed as an operation that integrates the ViT features (\ie, the Attn branch and the FFN branch) and the convolution features (\ie, the Conv branch). Therefore, we begin by formalizing the obtained features into the following two basic elements: the ViT features and the convolution features. To formalize the learning setting, we express the ViT features as $\textbf{X}_{vit}$ and the convolution features as $\textbf{X}_{con}$. It is evident that if $\textbf{X}_{vit}$ and $\textbf{X}_{con}$ are extracted from the same image, then $\textbf{X}_{vit}$ and $\textbf{X}_{con}$ are not independently distributed, and there exists some mutual information between them~\citep{zhang2022graph,wu2021cvt,zhang2023cae,peng2021conformer}. Therefore, the IE of the fused feature of $\textbf{X}_{vit}$ and $\textbf{X}_{con}$ within the MEA block can be expressed as:
% ---------------------------------------------------
\begin{equation}
\begin{split}
\label{eqs:1}
\textrm{H}(\textbf{X}_{vit}, \textbf{X}_{con}) = \textrm{H}(\textbf{X}_{vit}) + \textrm{H}(\textbf{X}_{con}) - \textrm{I}(\textbf{X}_{vit}; \textbf{X}_{con}),
\end{split}
\end{equation}
% ---------------------------------------------------
where $\textrm{H}(\cdot)$ is utilized to calculate the IE of the given variate, which can be formulated as:
% ---------------------------------------------------
\begin{equation}
\begin{split}
\label{eqs:2}
\textrm{H}(\textbf{X}_{vit}) = -\sum P(\textbf{x}_{vit}) log(P(\textbf{x}_{vit})),\\
\textrm{H}(\textbf{X}_{con}) = -\sum P(\textbf{x}_{con}) log(P(\textbf{x}_{con})),
\end{split}
\end{equation}
% ---------------------------------------------------
where $P(\textbf{x}_{vit})$ represents the probability of $\textbf{X}_{vit}$ taking on the value of $\textbf{x}_{vit}$. The similar definition of $P(\textbf{x}_{con})$. $\textrm{I}(\cdot;\cdot)$ in Eq.~\eqref{eqs:1} is used to compute the mutual information between $\textbf{X}_{vit}$ and $\textbf{X}_{con}$, which can be expressed as:
% ---------------------------------------------------
\begin{equation}
\begin{split}
\label{eqs:3}
\textrm{I}(\textbf{X}_{vit}; \textbf{X}_{con}) = \sum\sum \textrm{P}(\textbf{X}_{vit}, \textbf{X}_{con}) \textrm{log}(\textrm{P}(\textbf{X}_{vit}, \textbf{X}_{con}) (\textrm{P}(\textbf{X}_{vit}), \textrm{P}(\textbf{X}_{con}))),
\end{split}
\end{equation}
% ---------------------------------------------------
where $\textrm{P}(\textbf{X}_{vit}, \textbf{X}_{con})$ is their joint probability distribution. 
%\textrm{P}(\textbf{X}_{vit})$ and $\textrm{P}(\textbf{X}_{con})$ are the marginal probability distributions of $\textbf{X}_{vit}$ and $\textbf{X}_{con}$, respectively. 
Since $\textrm{I}(\textbf{X}_{vit}; \textbf{X}_{con})$ is always non-negative, $\textrm{H}(\textbf{X}_{vit}, \textbf{X}_{con})$ may still be greater than $\textrm{H}(\textbf{X}_{vit})$ or $\textrm{H}(\textbf{X}_{con})$~\citep{paninski2003estimation,gabrie2018entropy}. This suggests that the IE of the features extracted by MEA is always greater than the feature representation extracted by either of them separately.

Specifically, if $\textrm{I}(\textbf{X}_{vit}; \textbf{X}_{con})$ is small, the IE gain after fusion may still be significant, which is beneficial for improving the generalization capability and adaptability of the block. However, when $\textrm{I}(\textbf{X}_{vit}; \textbf{X}_{con})$ is large, the IE gain after fusion may be reduced. This means that $\textrm{I}(\textbf{X}_{vit}; \textbf{X}_{con})$ may affect the IE improvement of the fused model. Next, we will discuss the impact of $\textbf{X}_{vit}$ and $\textbf{X}_{con}$ on improving the IE of the adapter based on the size of $\textrm{I}(\textbf{X}_{vit}; \textbf{X}_{con})$, which can be divided into the following three cases:

\begin{itemize}
% --------------------------
\item {{Small} $\textrm{I}(\textbf{X}_{vit}; \textbf{X}_{con})$.} This is an ideal state. When the dependency between $\textbf{X}_{vit}$ and $\textbf{X}_{con}$ is small, it indicates that $\textrm{I}(\textbf{X}_{vit}; \textbf{X}_{con})$ is small, that is, $\textbf{X}_{vit}$ and $\textbf{X}_{con}$ respectively represent different information of the image. In this case, fusing $\textbf{X}_{vit}$ and $\textbf{X}_{con}$ can bring a significant increase in IE, which is beneficial to improving the adapter's generalization capability and adaptability.
% --------------------------
\item {{Medium} $\textrm{I}(\textbf{X}_{vit}; \textbf{X}_{con})$.} When $\textrm{I}(\textbf{X}_{vit}; \textbf{X}_{con})$ is between small and large, it indicates that there is a certain degree of correlation between them. In this case, fusing $\textbf{X}_{vit}$ and $\textbf{X}_{con}$ may still bring some IE gain. The specific improvement effect depends on the degree of correlation between $\textbf{X}_{vit}$ and $\textbf{X}_{con}$ and their complementarity in image representations. Fortunately~\citep{zhang2022graph,zhang2023cae,marouf2024mini,liu2023efficientvit}, a large amount of work has validated that ViT and convolutional layers can extract distinctive information from images. Therefore, in this case, fusing $\textbf{X}_{vit}$ and $\textbf{X}_{con}$ can still bring IE gains.
\item {{Large} \myparagraph{$\textrm{I}(\textbf{X}_{vit}; \textbf{X}_{con})$}.} When $\textrm{I}(\textbf{X}_{vit}; \textbf{X}_{con})$ between $\textbf{X}_{vit}$ and $\textbf{X}_{con}$ is large, it indicates that there is a high correlation between them, \ie, global ViT and local convolution features may represent similar or overlapping information of the image. In this case, the IE gain brought by fusing $\textbf{X}_{vit}$ and $\textbf{X}_{con}$ may decrease because there is a lot of information overlap between them. However, in our case, the probability of such a scenario occurring is almost non-existent, fusing $\textbf{X}_{vit}$ and $\textbf{X}_{con}$ may still improve the performance of the model to some extent, because they may capture the detailed information of the image to varying degrees.
% --------------------------
\end{itemize}
% --------------------------

Based on the aforementioned theoretical analysis, we can conclude that the proposed MEA block has a larger IE than existing ViT adapters (which are primarily based on the attention mechanism) under any scenario. This provides evidence that the MEA block has more comprehensive feature representations. 
% ---------------------------------
\end{proof}
% ---------------------------------
As the MEA block includes a parallel convolutional branch, it can better capture local inductive biases compared to the traditional ViT adapter, which mainly uses self-attention~\citep{hu2022lora,jie2023fact,chen2022vision,ma2024segment,luo2023forgery,shao2023deepfake,mercea2024time}. 
%
Therefore, the MEA block's feature space should be more capable of distinguishing different samples, resulting in a larger MMD value. 
%
Our MEA block's feature space is obtained by combining the attention branch, the feed-forward network branch, and the local convolutional branch, enabling it to capture both local and global inductive biases of the given image. 
%
In contrast, the traditional ViT adapter's feature space is mainly obtained through self-attention and may not be able to capture local features well. Therefore, according to the MMD theory~\citep{cheng2021neural,arbel2019maximum,wang2021rethinking}, we can conclude that if the MEA block's feature space is more discriminative than the traditional ViT adapter's feature space, then the MEA block's feature space is more suitable for adapter feature space and can better improve the model's generalization capability and adaptability.

% -------------------------------------------
\section{Introduction of the Experimental Datasets}
\label{secS2}
% -------------------------------------------
{\color{red}{\emph{This supplementary is for Section~4.1 of the main paper.}}}
In our paper, two representative datasets are used to evaluate the effectiveness and efficiency of our method, including MS-COCO~\citep{caesar2018coco} for ODet and ISeg, and ADE20K~\citep{zhou2017scene} for SSeg. Below are the details of the used datasets:

% -------------------------------
\begin{itemize}
% -------------------------------
\item MS-COCO~\citep{caesar2018coco} is a representative yet challenging dataset for common scene IS and object detection, which consists of $118$k, $5$k and $20$k images for the \emph{training} set, the \emph{val} set and the \emph{test} set, respectively. In our experiments, the model is trained on the \emph{training} set and evaluated on the \emph{val} set.
% -------------------------------
\item ADE20K~\citep{zhou2017scene} is a scene parsing dataset with $20$k images and $150$ object categories. Each image has pixel-level annotations for SS of objects and regions within the scene. The dataset is divided into $20$k, $2$k, and $3$k images for \emph{training}, \emph{val} and \emph{test}, respectively. Our model is trained on the \emph{training} set and evaluated on the \emph{val} set.
% -------------------------------
\end{itemize}
% -------------------------------
For data augmentation, random horizontal flip, brightness jittering and random scaling within the range of $[0.5, 2]$ are used in training as in~\citep{chen2022vision,luo2023forgery,zhang2023cae,mercea2024time}. By default, the inference results are obtained at a single scale, unless explicitly specified otherwise.    


% -------------------------------------------
\section{Introduction of the Experimental Settings}
\label{secS3}
% -------------------------------------------
{\color{red}{\emph{This supplementary is for Section~4.2 of the main paper.}}} Experiments on object detection and instance segmentation are conducted using the open-source MMDetection framework~\citep{chen2019mmdetection}. The training batch size is set to $16$, and AdamW~\citep{loshchilov2017decoupled} is used as the optimizer with the initial learning rate of $1 \times 10^{-4}$ and the weight decay of $0.05$. The layer-wise learning rate decay is used and set to $0.9$, and the drop path rate is set to $0.4$. Following~\citep{xiong2024efficient,wang2021pyramid,chen2022vision,liu2022convnet}, to ensure a fair result comparison, we choose two training schedules, 1$\times$ (\ie, $12$ training epochs) and 3$\times$ (\ie, $36$ training epochs). For the 1$\times$ training schedule, images are resized to the shorter side of 800 pixels, with the longer side not exceeding $1,333$ pixels. In inference, the shorter side of images is consistently set to 800 pixels by default. For the 3$\times$ training schedule, the multi-scale training strategy is also used as in~\citep{chen2022vision}, and the shorter side is resized to $480$ to $800$ pixels, while the longer side remains capped at $1,333$ pixels.

{\color{red}{\emph{This supplementary is for Section~4.3 of the main paper.}}} Experiments on semantic segmentation are conducted using the MMSegmentation framework~\citep{mmseg2020}. The input images are cropped to a fix size of 512 $\times$ 512 pixels as in~\citep{xiong2024efficient,chen2022vision}. The training batch size is set to $16$, and AdamW~\citep{loshchilov2017decoupled} is used as the optimizer with the initial learning rate of $1 \times 10^{-5}$ and the weight decay of $0.05$. Following~\citep{li2022exploring,liu2021swin}, the layer-wise learning rate decay is set to $0.9$ and the drop path rate is set to $0.4$. We report the experimental results on both single scale training and multi-scale training strategies. 
% -------------------------------
\begin{table}[t]
\centering
\small
\renewcommand\arraystretch{1.2}
\setlength{\tabcolsep}{6pt}{
\begin{tabular}{r|r|ccl}
\hline \hline 
Methods & Pre-Trained & Params.$\downarrow$ & AP$^\textrm{m}$ $\uparrow$ \\
\hline 
Swin-B~\citep{liu2021swin} & ImageNet-1k~\citep{deng2009imagenet} & 107.1 &  43.3 \\
ViT-Adapter-B~\citep{chen2022vision} & ImageNet-1k~\citep{deng2009imagenet} & 120.2 & 43.6 \\
\cellcolor[gray]{.95}\textbf{META-B$_{{\textrm{(Ours)}}}$} & \cellcolor[gray]{.95}ImageNet-1k~\citep{deng2009imagenet} & \cellcolor[gray]{.95}115.3 & \cellcolor[gray]{.95}44.3$_{\color{red}{+0.7}}$ \\
\cdashline{1-4}[0.8pt/2pt]
Swin-B~\citep{liu2021swin} & ImageNet-22k~\citep{steiner2021train} & 107.1 & 44.3\\
ViT-Adapter-B~\citep{chen2022vision} & ImageNet-22k~\citep{steiner2021train} & 120.2 & 44.6 \\
\cellcolor[gray]{.95}\textbf{META-B$_{{\textrm{(Ours)}}}$} & \cellcolor[gray]{.95}ImageNet-22k~\citep{steiner2021train} & \cellcolor[gray]{.95}115.3  & \cellcolor[gray]{.95}45.2$_{\color{red}{+0.6}}$ \\
\cdashline{1-4}[0.8pt/2pt]
Swin-B~\citep{liu2021swin} & Multi-Modal~\citep{zhu2022uni} & 107.1 &   -- \\
ViT-Adapter-B~\citep{chen2022vision} & Multi-Modal~\citep{zhu2022uni} & 120.2  & 45.3 \\
\cellcolor[gray]{.95}\textbf{META-B$_{{\textrm{(Ours)}}}$} & \cellcolor[gray]{.95}Multi-Modal~\citep{zhu2022uni} & \cellcolor[gray]{.95}115.3  & \cellcolor[gray]{.95}45.9$_{\color{red}{+0.6}}$ \\
\hline \hline 
\end{tabular}
\caption{Result comparisons on Params. (\textbf{M}) and AP (\%) under different pre-trained weights with Mask R-CNN ($3 \times$ +MS schedule)~\citep{he2017mask} as the baseline model on the \emph{val} set of MS-COCO~\citep{caesar2018coco}. ``--'' denotes there is no such a result in its paper.}
\label{tab3}}
\end{table}
% -------------------------------

% -------------------------------------------
\section{Result Comparisons under Different Weights}
\label{secS4}
% -------------------------------------------
{\color{red}{\emph{This supplementary is for Section~4.2 of the main paper.}}} In this section, we present the experimental results of META on object detection and instance segmentation with different pre-trained weights and compare them with other state-of-the-art methods including SwinViT~\citep{liu2021swin} and ViT-Adapter~\citep{chen2022vision} as in~\citep{chen2022vision}. 
Mask R-CNN~\citep{he2017mask} is used as the baseline, and ViT-B~\citep{li2022exploring} is used as the backbone. The 3$\times$ training schedule with MS training strategy is used. The obtained experimental results are given in Table~\ref{tab3}.
%
From this table, we can observe that our method is applicable to different pre-trained weights (\ie, ImageNet-1k~\citep{deng2009imagenet}, ImageNet-22k~\citep{steiner2021train}, and Multi-Modal~\citep{zhu2022uni}), and achieves more accurate AP with fewer model parameters compared to ViT-Adapter~\citep{chen2022vision}, across different pre-trained weights.  

% -------------------------------------------
\section{More Ablation Study Results}
\label{secS5}
% -------------------------------------------
{\color{red}{\emph{This supplementary is for Section~4.4 of the main paper.}}} In our main paper, we present the experimental results of deploying adapters with Attn branch and FFN branch as components on ViT-B~\citep{li2022exploring}. It is noteworthy that the layer normalization operation has been shared between the Attn branch and the FFN branch to reduce the memory access costs associated with the normalization operations. In this section, we demonstrate a result comparison between the experimental results of using shared layer normalization operation and those of not using it in the traditional setting (\ie, the non-shared normalization). The obtained experimental results are shown in Table~\ref{tab:s1}. It can be observed that sharing layer normalization does not significantly improve the performance in terms of AP. However, compared to FPS, FLOPs, MC, our approach can achieve satisfactory performance gains.
% --------------------------
\begin{table*}[t]
\centering
\renewcommand\arraystretch{1.2}
\setlength{\tabcolsep}{1pt}{
\begin{tabular}{r|ccccc|ccccc}
\hline \hline 
Settings & ViT-B & Attn & FFN & Conv & Cascade & AP$^\textrm{m}$ $\uparrow$ & FPS$\uparrow$ & Params.$\downarrow$ & FLOPs$\downarrow$ & MC$\downarrow$ \\
\hline 
Baseline model & \cmark & \xmark & \xmark & \xmark & \xmark & 41.3 & 11.5 & 113.6\textbf{M} & 719\textbf{G} & NA\\
\cdashline{1-11}[0.8pt/2pt]
\cellcolor[gray]{.95}Shared normalization & \cmark & \cmark & \cmark & \xmark & \xmark & \cellcolor[gray]{.95}43.4 & \cellcolor[gray]{.95}11.3 & \cellcolor[gray]{.95}114.4\textbf{M} & \cellcolor[gray]{.95}719\textbf{G} & \cellcolor[gray]{.95}7.5\textbf{GB}\\
Non-shared normalization & \cmark & \cmark & \cmark & \xmark & \xmark & 43.2 & 10.5 & 114.4\textbf{M} & 737\textbf{G} & 8.8\textbf{GB}\\
\hline \hline 
\end{tabular}
\caption{Ablation study results on shared layer normalization.}
\label{tab:s1}}
\end{table*}
% --------------------------

% -------------------------------
{\color{red}{\emph{This supplementary is for Section~4.4 of the main paper.}}} META is proposed as a simple and fast ViT adapter by minimizing inefficient memory access operations. In this section, we compare META with other efficient attention methods and advanced adapter methods~\citep{marouf2024mini,xia2022vision,sung2022vl}. All methods are used with their default settings and the same settings as the injector and extractor in ViT-adapter~\citep{chen2022vision}. Following the same setup as in~\citep{chen2022vision}, the attention mechanism is utilized as the ViT-adapter layer. Therefore, during the experimental comparisons, we replace the attention mechanism in the ViT-adapter with alternative attention mechanisms to ensure a fair comparison. 
The obtained experimental results are given in Table~\ref{tab6}. We can observe that compared to these methods, META achieves new state-of-the-art performance in both accuracy and efficiency. We ultimately achieve an AP of $44.3\%$ with $115.3$\textbf{M} parameters, $720$\textbf{G} FLOPs, $17.4$ FPS, and 8.1 \textbf{GB} MC. 
% -------------------------------
\begin{table}[t]
\centering
\footnotesize
\renewcommand\arraystretch{1.2}
\setlength{\tabcolsep}{5pt}{
\begin{tabular}{r|ccccc}
\hline \hline 
Methods & AP$\uparrow$ & FPS$\uparrow$ & Params. (\textbf{M})$\downarrow$ & FLOPs (\textbf{G})$\downarrow$  & Momory (\textbf{GB})$\downarrow$ \\
\hline 
WindowAtt~\citep{liu2021swin} & 41.2 & 11.6 & 145.0 & 982 & 18.5 \\
PaleAttention~\citep{wu2022pale} & 42.8 & 14.4 & 155.2 & 1,029 & 16.7\\
Attention~\citep{vaswani2017attention} & 43.1 & 5.2 & 188.4 & 1,250 & 18.3 \\
CSWindow~\citep{dong2022cswin}& 43.1 & 13.7 & 144.6 & 990 & 12.9\\
SimplingAtte~\citep{he2023simplifying} & 43.3 & 12.2 & 126.3 & 994 & 17.1\\
DeformableAtt~\citep{xia2022vision} & 43.7 & 13.5 & 166.0 & 988 & 15.2 \\
\cdashline{1-6}[0.8pt/2pt]
MiniAdapters~\citep{marouf2024mini} & 41.9 & 15.0 & 131.8 & 995 & 12.2 \\
VL-Adapter~\citep{sung2022vl} & 42.7 & 14.5 & 167.2 & 993  & 14.0\\
\cellcolor[gray]{.95}\textbf{META-B$_{{\textrm{(Ours)}}}$} & \cellcolor[gray]{.95}44.3 & \cellcolor[gray]{.95}17.4 & \cellcolor[gray]{.95}115.3 & \cellcolor[gray]{.95}720 & \cellcolor[gray]{.95}8.1\\
\hline \hline 
\end{tabular}
\caption{Result comparisons with different adapters.}
\label{tab6}}
\end{table}
% -------------------------------

% -------------------------------------------
\section{Visualizations under the Conv branch}
\label{secS6}
% -------------------------------------------
{\color{red}{\emph{This supplementary is for Section~3.2 of the main paper.}}} In this section, to observe if the adapter has learned local inductive biases through the Conv branch, we visualize the model's class activation maps. The obtained visualizations are given in Figure~\ref{figs1}. From this figure, it can be observed that after adding the Conv branch, the model focuses more on the specific object area (\eg,`` the dog'' and ``the person'') rather than the surrounding area that may extend beyond the object itself, as was the case before adding the Conv branch. This indicates that our method effectively learns local inductive biases after incorporating the Conv branch.
% -------------------------------------------
% This file was created by matlab2tikz.
%
%The latest updates can be retrieved from
%  http://www.mathworks.com/matlabcentral/fileexchange/22022-matlab2tikz-matlab2tikz
%where you can also make suggestions and rate matlab2tikz.
%
\definecolor{mycolor1}{rgb}{0.21569,0.54902,0.72157}%
\definecolor{mycolor2}{rgb}{0.80784,0.16863,0.12157}%
%
\begin{tikzpicture}

\begin{axis}[%
width=0.898in,
height=1.5in,%3.603in,
at={(0.766in,0.486in)},
scale only axis,
xmin=0,
xmax=10,
ymin=0,
ymax=0.8,
xlabel= \phantom{$z$},
ylabel=$p(g_{z^*}|Y)$,
ylabel near ticks,
title={Linearization-based\\ approach},
title style={align=left}, 
axis background/.style={fill=white},
axis x line*=bottom,
axis y line*=left,
legend style={legend cell align=left, align=left, draw=white!15!black}
]
\addplot[ybar interval, fill=mycolor1, fill opacity=0.4, draw=mycolor1, area legend] table[row sep=crcr] {%
x	y\\
3.36	0.0144927536231884\\
3.429	0.0289855072463768\\
3.498	0.0869565217391305\\
3.567	0.217391304347825\\
3.636	0.391304347826087\\
3.705	0.565217391304348\\
3.774	0.449275362318841\\
3.843	0.405797101449276\\
3.912	0.666666666666667\\
3.981	0.420289855072464\\
4.05	0.478260869565218\\
4.119	0.289855072463768\\
4.188	0.289855072463768\\
4.257	0.347826086956522\\
4.326	0.246376811594203\\
4.395	0.304347826086953\\
4.464	0.20289855072464\\
4.533	0.217391304347823\\
4.602	0.246376811594203\\
4.671	0.289855072463768\\
4.74	0.246376811594203\\
4.809	0.188405797101449\\
4.878	0.231884057971015\\
4.947	0.27536231884058\\
5.016	0.391304347826087\\
5.085	0.246376811594203\\
5.154	0.27536231884058\\
5.223	0.217391304347826\\
5.292	0.347826086956518\\
5.361	0.231884057971018\\
5.43	0.2463768115942\\
5.499	0.260869565217395\\
5.568	0.275362318840576\\
5.637	0.289855072463772\\
5.706	0.304347826086953\\
5.775	0.289855072463772\\
5.844	0.362318840579706\\
5.913	0.289855072463768\\
5.982	0.463768115942029\\
6.051	0.420289855072464\\
6.12	0.492753623188406\\
6.189	0.463768115942029\\
6.258	0.463768115942029\\
6.327	0.420289855072459\\
6.396	0.246376811594206\\
6.465	0.202898550724635\\
6.534	0.0869565217391316\\
6.603	0.0579710144927529\\
6.672	0.0289855072463772\\
6.741	0.0144927536231882\\
6.81	0.0144927536231882\\
};
%\addlegendentry{ground truth}

\addplot [color=mycolor2, line width=2.0pt]
  table[row sep=crcr]{%
0	0.00495647934021539\\
0.01	0.00503120737369003\\
0.02	0.00510691052511148\\
0.03	0.00518359894024572\\
0.04	0.00526128282899922\\
0.05	0.00533997246516116\\
0.06	0.00541967818613227\\
0.07	0.00550041039264009\\
0.08	0.00558217954844049\\
0.09	0.00566499618000535\\
0.1	0.00574887087619621\\
0.11	0.00583381428792371\\
0.12	0.00591983712779276\\
0.13	0.00600695016973323\\
0.14	0.006095164248616\\
0.15	0.00618449025985429\\
0.16	0.00627493915898998\\
0.17	0.00636652196126503\\
0.18	0.0064592497411776\\
0.19	0.0065531336320228\\
0.2	0.00664818482541805\\
0.21	0.0067444145708128\\
0.22	0.0068418341749824\\
0.23	0.00694045500150619\\
0.24	0.00704028847022945\\
0.25	0.00714134605670927\\
0.26	0.00724363929164408\\
0.27	0.00734717976028668\\
0.28	0.00745197910184082\\
0.29	0.00755804900884096\\
0.3	0.00766540122651533\\
0.31	0.00777404755213196\\
0.32	0.00788399983432761\\
0.33	0.00799526997241961\\
0.34	0.00810786991570027\\
0.35	0.00822181166271396\\
0.36	0.00833710726051655\\
0.37	0.00845376880391716\\
0.38	0.00857180843470236\\
0.39	0.00869123834084212\\
0.4	0.00881207075567811\\
0.41	0.00893431795709363\\
0.42	0.00905799226666551\\
0.43	0.00918310604879767\\
0.44	0.00930967170983618\\
0.45	0.00943770169716588\\
0.46	0.00956720849828854\\
0.47	0.00969820463988202\\
0.48	0.00983070268684103\\
0.49	0.00996471524129867\\
0.5	0.0101002549416293\\
0.51	0.0102373344614323\\
0.52	0.0103759665084967\\
0.53	0.0105161638237465\\
0.54	0.0106579391801673\\
0.55	0.0108013053817126\\
0.56	0.010946275262192\\
0.57	0.0110928616841387\\
0.58	0.0112410775376588\\
0.59	0.0113909357392594\\
0.6	0.0115424492306589\\
0.61	0.0116956309775763\\
0.62	0.0118504939685016\\
0.63	0.0120070512134457\\
0.64	0.0121653157426715\\
0.65	0.012325300605404\\
0.66	0.012487018868522\\
0.67	0.0126504836152284\\
0.68	0.0128157079437017\\
0.69	0.0129827049657272\\
0.7	0.0131514878053083\\
0.71	0.0133220695972577\\
0.72	0.0134944634857689\\
0.73	0.0136686826229678\\
0.74	0.0138447401674437\\
0.75	0.0140226492827617\\
0.76	0.0142024231359536\\
0.77	0.0143840748959895\\
0.78	0.0145676177322305\\
0.79	0.0147530648128596\\
0.8	0.0149404293032943\\
0.81	0.0151297243645786\\
0.82	0.0153209631517556\\
0.83	0.0155141588122201\\
0.84	0.0157093244840515\\
0.85	0.0159064732943274\\
0.86	0.0161056183574168\\
0.87	0.016306772773255\\
0.88	0.0165099496255975\\
0.89	0.0167151619802561\\
0.9	0.0169224228833143\\
0.91	0.017131745359324\\
0.92	0.0173431424094836\\
0.93	0.0175566270097958\\
0.94	0.0177722121092072\\
0.95	0.0179899106277294\\
0.96	0.0182097354545399\\
0.97	0.018431699446066\\
0.98	0.0186558154240489\\
0.99	0.0188820961735898\\
1	0.0191105544411779\\
1.01	0.0193412029326999\\
1.02	0.0195740543114314\\
1.03	0.0198091211960107\\
1.04	0.0200464161583947\\
1.05	0.0202859517217969\\
1.06	0.0205277403586084\\
1.07	0.0207717944883015\\
1.08	0.0210181264753159\\
1.09	0.0212667486269282\\
1.1	0.0215176731911045\\
1.11	0.0217709123543368\\
1.12	0.0220264782394623\\
1.13	0.0222843829034672\\
1.14	0.0225446383352741\\
1.15	0.0228072564535138\\
1.16	0.0230722491042814\\
1.17	0.0233396280588773\\
1.18	0.0236094050115328\\
1.19	0.0238815915771207\\
1.2	0.0241561992888519\\
1.21	0.0244332395959568\\
1.22	0.0247127238613533\\
1.23	0.0249946633592999\\
1.24	0.0252790692730363\\
1.25	0.0255659526924097\\
1.26	0.0258553246114888\\
1.27	0.026147195926164\\
1.28	0.0264415774317359\\
1.29	0.0267384798204914\\
1.3	0.0270379136792673\\
1.31	0.0273398894870028\\
1.32	0.0276444176122807\\
1.33	0.0279515083108568\\
1.34	0.0282611717231799\\
1.35	0.0285734178718998\\
1.36	0.0288882566593667\\
1.37	0.0292056978651199\\
1.38	0.0295257511433677\\
1.39	0.0298484260204579\\
1.4	0.0301737318923396\\
1.41	0.0305016780220172\\
1.42	0.0308322735369956\\
1.43	0.0311655274267189\\
1.44	0.0315014485400004\\
1.45	0.0318400455824475\\
1.46	0.0321813271138789\\
1.47	0.032525301545736\\
1.48	0.0328719771384891\\
1.49	0.0332213619990379\\
1.5	0.0335734640781073\\
1.51	0.0339282911676388\\
1.52	0.034285850898178\\
1.53	0.0346461507362581\\
1.54	0.0350091979817809\\
1.55	0.0353749997653947\\
1.56	0.0357435630458696\\
1.57	0.0361148946074719\\
1.58	0.0364890010573359\\
1.59	0.0368658888228363\\
1.6	0.0372455641489584\\
1.61	0.0376280330956701\\
1.62	0.0380133015352924\\
1.63	0.0384013751498731\\
1.64	0.0387922594285599\\
1.65	0.0391859596649766\\
1.66	0.0395824809546015\\
1.67	0.0399818281921483\\
1.68	0.040384006068951\\
1.69	0.0407890190703521\\
1.7	0.0411968714730958\\
1.71	0.0416075673427256\\
1.72	0.0420211105309874\\
1.73	0.0424375046732393\\
1.74	0.0428567531858665\\
1.75	0.0432788592637046\\
1.76	0.0437038258774694\\
1.77	0.0441316557711956\\
1.78	0.0445623514596833\\
1.79	0.0449959152259539\\
1.8	0.0454323491187165\\
1.81	0.0458716549498434\\
1.82	0.0463138342918566\\
1.83	0.0467588884754263\\
1.84	0.047206818586881\\
1.85	0.0476576254657295\\
1.86	0.0481113097021966\\
1.87	0.0485678716347722\\
1.88	0.0490273113477746\\
1.89	0.0494896286689283\\
1.9	0.0499548231669573\\
1.91	0.0504228941491942\\
1.92	0.0508938406592059\\
1.93	0.0513676614744362\\
1.94	0.0518443551038658\\
1.95	0.0523239197856913\\
1.96	0.0528063534850216\\
1.97	0.0532916538915952\\
1.98	0.0537798184175165\\
1.99	0.0542708441950129\\
2	0.0547647280742127\\
2.01	0.0552614666209458\\
2.02	0.0557610561145651\\
2.03	0.0562634925457924\\
2.04	0.0567687716145866\\
2.05	0.0572768887280368\\
2.06	0.0577878389982796\\
2.07	0.0583016172404419\\
2.08	0.0588182179706094\\
2.09	0.0593376354038221\\
2.1	0.0598598634520962\\
2.11	0.0603848957224738\\
2.12	0.0609127255151013\\
2.13	0.0614433458213363\\
2.14	0.0619767493218837\\
2.15	0.0625129283849627\\
2.16	0.063051875064503\\
2.17	0.0635935810983738\\
2.18	0.0641380379066434\\
2.19	0.0646852365898719\\
2.2	0.0652351679274361\\
2.21	0.0657878223758891\\
2.22	0.0663431900673527\\
2.23	0.0669012608079454\\
2.24	0.0674620240762456\\
2.25	0.0680254690217902\\
2.26	0.06859158446361\\
2.27	0.0691603588888017\\
2.28	0.0697317804511384\\
2.29	0.0703058369697168\\
2.3	0.070882515927644\\
2.31	0.0714618044707637\\
2.32	0.0720436894064212\\
2.33	0.0726281572022699\\
2.34	0.0732151939851174\\
2.35	0.0738047855398146\\
2.36	0.0743969173081847\\
2.37	0.0749915743879969\\
2.38	0.0755887415319811\\
2.39	0.0761884031468874\\
2.4	0.0767905432925892\\
2.41	0.0773951456812309\\
2.42	0.0780021936764202\\
2.43	0.0786116702924671\\
2.44	0.0792235581936674\\
2.45	0.079837839693634\\
2.46	0.0804544967546748\\
2.47	0.0810735109872175\\
2.48	0.0816948636492834\\
2.49	0.0823185356460092\\
2.5	0.0829445075292169\\
2.51	0.0835727594970344\\
2.52	0.084203271393565\\
2.53	0.0848360227086074\\
2.54	0.0854709925774259\\
2.55	0.0861081597805728\\
2.56	0.0867475027437606\\
2.57	0.0873889995377879\\
2.58	0.0880326278785163\\
2.59	0.0886783651269002\\
2.6	0.0893261882890704\\
2.61	0.0899760740164704\\
2.62	0.0906279986060465\\
2.63	0.0912819380004926\\
2.64	0.0919378677885496\\
2.65	0.0925957632053583\\
2.66	0.0932555991328697\\
2.67	0.0939173501003085\\
2.68	0.0945809902846947\\
2.69	0.0952464935114191\\
2.7	0.0959138332548775\\
2.71	0.0965829826391593\\
2.72	0.0972539144387952\\
2.73	0.0979266010795609\\
2.74	0.0986010146393384\\
2.75	0.0992771268490357\\
2.76	0.0999549090935642\\
2.77	0.100634332412874\\
2.78	0.101315367503047\\
2.79	0.101997984717452\\
2.8	0.102682154067954\\
2.81	0.103367845226185\\
2.82	0.104055027524874\\
2.83	0.104743669959238\\
2.84	0.105433741188425\\
2.85	0.106125209537028\\
2.86	0.106818042996652\\
2.87	0.107512209227537\\
2.88	0.108207675560252\\
2.89	0.108904408997439\\
2.9	0.109602376215622\\
2.91	0.110301543567076\\
2.92	0.111001877081754\\
2.93	0.111703342469276\\
2.94	0.112405905120979\\
2.95	0.113109530112027\\
2.96	0.113814182203577\\
2.97	0.114519825845015\\
2.98	0.115226425176241\\
2.99	0.115933944030024\\
3	0.116642345934409\\
3.01	0.117351594115192\\
3.02	0.118061651498449\\
3.03	0.118772480713129\\
3.04	0.119484044093701\\
3.05	0.120196303682873\\
3.06	0.120909221234355\\
3.07	0.121622758215693\\
3.08	0.122336875811159\\
3.09	0.123051534924699\\
3.1	0.123766696182943\\
3.11	0.124482319938267\\
3.12	0.125198366271925\\
3.13	0.12591479499723\\
3.14	0.126631565662796\\
3.15	0.127348637555838\\
3.16	0.128065969705534\\
3.17	0.128783520886434\\
3.18	0.129501249621939\\
3.19	0.130219114187826\\
3.2	0.130937072615837\\
3.21	0.131655082697321\\
3.22	0.132373101986931\\
3.23	0.133091087806378\\
3.24	0.133808997248241\\
3.25	0.13452678717983\\
3.26	0.135244414247106\\
3.27	0.135961834878651\\
3.28	0.136679005289694\\
3.29	0.137395881486191\\
3.3	0.138112419268956\\
3.31	0.138828574237848\\
3.32	0.139544301796001\\
3.33	0.140259557154114\\
3.34	0.14097429533479\\
3.35	0.141688471176923\\
3.36	0.142402039340136\\
3.37	0.143114954309265\\
3.38	0.143827170398901\\
3.39	0.144538641757969\\
3.4	0.145249322374359\\
3.41	0.145959166079606\\
3.42	0.146668126553618\\
3.43	0.147376157329438\\
3.44	0.148083211798065\\
3.45	0.148789243213316\\
3.46	0.149494204696722\\
3.47	0.150198049242482\\
3.48	0.150900729722448\\
3.49	0.151602198891158\\
3.5	0.152302409390905\\
3.51	0.153001313756856\\
3.52	0.153698864422194\\
3.53	0.154395013723319\\
3.54	0.155089713905069\\
3.55	0.155782917125992\\
3.56	0.156474575463644\\
3.57	0.157164640919932\\
3.58	0.157853065426486\\
3.59	0.158539800850068\\
3.6	0.15922479899801\\
3.61	0.159908011623694\\
3.62	0.160589390432051\\
3.63	0.161268887085104\\
3.64	0.16194645320753\\
3.65	0.16262204039226\\
3.66	0.163295600206099\\
3.67	0.163967084195381\\
3.68	0.164636443891645\\
3.69	0.165303630817341\\
3.7	0.165968596491557\\
3.71	0.166631292435772\\
3.72	0.167291670179631\\
3.73	0.167949681266742\\
3.74	0.168605277260496\\
3.75	0.169258409749901\\
3.76	0.169909030355445\\
3.77	0.170557090734967\\
3.78	0.171202542589549\\
3.79	0.171845337669428\\
3.8	0.17248542777992\\
3.81	0.173122764787353\\
3.82	0.173757300625023\\
3.83	0.174388987299154\\
3.84	0.175017776894874\\
3.85	0.1756436215822\\
3.86	0.176266473622029\\
3.87	0.176886285372142\\
3.88	0.177503009293208\\
3.89	0.178116597954805\\
3.9	0.178727004041434\\
3.91	0.179334180358544\\
3.92	0.179938079838555\\
3.93	0.180538655546892\\
3.94	0.181135860688006\\
3.95	0.181729648611403\\
3.96	0.182319972817676\\
3.97	0.182906786964519\\
3.98	0.183490044872755\\
3.99	0.184069700532347\\
4	0.184645708108407\\
4.01	0.185218021947199\\
4.02	0.185786596582135\\
4.03	0.186351386739756\\
4.04	0.186912347345709\\
4.05	0.187469433530713\\
4.06	0.188022600636505\\
4.07	0.188571804221778\\
4.08	0.189117000068111\\
4.09	0.189658144185865\\
4.1	0.190195192820082\\
4.11	0.190728102456354\\
4.12	0.191256829826677\\
4.13	0.191781331915283\\
4.14	0.192301565964453\\
4.15	0.192817489480306\\
4.16	0.193329060238567\\
4.17	0.193836236290307\\
4.18	0.194338975967659\\
4.19	0.19483723788951\\
4.2	0.195330980967159\\
4.21	0.195820164409952\\
4.22	0.196304747730887\\
4.23	0.196784690752181\\
4.24	0.197259953610816\\
4.25	0.197730496764038\\
4.26	0.198196280994839\\
4.27	0.198657267417387\\
4.28	0.19911341748243\\
4.29	0.199564692982658\\
4.3	0.200011056058033\\
4.31	0.200452469201067\\
4.32	0.200888895262076\\
4.33	0.201320297454377\\
4.34	0.201746639359454\\
4.35	0.202167884932071\\
4.36	0.202583998505353\\
4.37	0.202994944795806\\
4.38	0.203400688908302\\
4.39	0.203801196341013\\
4.4	0.204196432990295\\
4.41	0.204586365155525\\
4.42	0.204970959543887\\
4.43	0.205350183275104\\
4.44	0.205724003886121\\
4.45	0.206092389335735\\
4.46	0.206455308009166\\
4.47	0.206812728722581\\
4.48	0.207164620727553\\
4.49	0.207510953715471\\
4.5	0.207851697821886\\
4.51	0.208186823630804\\
4.52	0.208516302178914\\
4.53	0.208840104959761\\
4.54	0.209158203927852\\
4.55	0.209470571502708\\
4.56	0.209777180572843\\
4.57	0.210078004499687\\
4.58	0.210373017121445\\
4.59	0.210662192756884\\
4.6	0.21094550620906\\
4.61	0.211222932768978\\
4.62	0.211494448219179\\
4.63	0.211760028837266\\
4.64	0.212019651399358\\
4.65	0.212273293183472\\
4.66	0.212520931972839\\
4.67	0.212762546059146\\
4.68	0.212998114245707\\
4.69	0.213227615850562\\
4.7	0.213451030709505\\
4.71	0.213668339179034\\
4.72	0.21387952213923\\
4.73	0.214084560996561\\
4.74	0.214283437686611\\
4.75	0.214476134676732\\
4.76	0.214662634968617\\
4.77	0.214842922100803\\
4.78	0.215016980151093\\
4.79	0.215184793738895\\
4.8	0.21534634802749\\
4.81	0.21550162872622\\
4.82	0.215650622092589\\
4.83	0.215793314934298\\
4.84	0.215929694611184\\
4.85	0.216059749037091\\
4.86	0.216183466681654\\
4.87	0.216300836572001\\
4.88	0.21641184829438\\
4.89	0.21651649199569\\
4.9	0.216614758384949\\
4.91	0.21670663873466\\
4.92	0.216792124882109\\
4.93	0.21687120923057\\
4.94	0.216943884750432\\
4.95	0.21701014498024\\
4.96	0.217069984027651\\
4.97	0.21712339657031\\
4.98	0.217170377856636\\
4.99	0.217210923706529\\
5	0.21724503051199\\
5.01	0.217272695237652\\
5.02	0.217293915421236\\
5.03	0.217308689173911\\
5.04	0.217317015180578\\
5.05	0.217318892700066\\
5.06	0.217314321565234\\
5.07	0.217303302183007\\
5.08	0.217285835534308\\
5.09	0.217261923173913\\
5.1	0.217231567230225\\
5.11	0.217194770404953\\
5.12	0.217151535972715\\
5.13	0.217101867780549\\
5.14	0.217045770247346\\
5.15	0.216983248363191\\
5.16	0.216914307688628\\
5.17	0.21683895435383\\
5.18	0.216757195057696\\
5.19	0.216669037066854\\
5.2	0.216574488214588\\
5.21	0.216473556899676\\
5.22	0.216366252085147\\
5.23	0.216252583296955\\
5.24	0.216132560622568\\
5.25	0.216006194709479\\
5.26	0.215873496763628\\
5.27	0.215734478547749\\
5.28	0.21558915237963\\
5.29	0.215437531130296\\
5.3	0.215279628222107\\
5.31	0.215115457626779\\
5.32	0.214945033863324\\
5.33	0.21476837199591\\
5.34	0.214585487631638\\
5.35	0.21439639691825\\
5.36	0.21420111654175\\
5.37	0.213999663723949\\
5.38	0.213792056219935\\
5.39	0.213578312315465\\
5.4	0.213358450824282\\
5.41	0.213132491085352\\
5.42	0.212900452960033\\
5.43	0.212662356829162\\
5.44	0.212418223590075\\
5.45	0.212168074653546\\
5.46	0.211911931940665\\
5.47	0.211649817879627\\
5.48	0.211381755402469\\
5.49	0.21110776794172\\
5.5	0.210827879426989\\
5.51	0.210542114281486\\
5.52	0.210250497418463\\
5.53	0.209953054237604\\
5.54	0.209649810621332\\
5.55	0.209340792931057\\
5.56	0.209026028003361\\
5.57	0.208705543146111\\
5.58	0.208379366134512\\
5.59	0.2080475252071\\
5.6	0.207710049061664\\
5.61	0.207366966851112\\
5.62	0.207018308179279\\
5.63	0.206664103096667\\
5.64	0.206304382096132\\
5.65	0.205939176108512\\
5.66	0.205568516498194\\
5.67	0.20519243505863\\
5.68	0.204810964007791\\
5.69	0.204424135983574\\
5.7	0.204031984039149\\
5.71	0.203634541638252\\
5.72	0.203231842650435\\
5.73	0.202823921346256\\
5.74	0.202410812392424\\
5.75	0.201992550846891\\
5.76	0.201569172153899\\
5.77	0.201140712138983\\
5.78	0.200707207003916\\
5.79	0.200268693321625\\
5.8	0.199825208031051\\
5.81	0.199376788431969\\
5.82	0.198923472179769\\
5.83	0.198465297280192\\
5.84	0.198002302084028\\
5.85	0.197534525281773\\
5.86	0.197062005898254\\
5.87	0.196584783287207\\
5.88	0.196102897125829\\
5.89	0.195616387409288\\
5.9	0.195125294445206\\
5.91	0.1946296588481\\
5.92	0.194129521533804\\
5.93	0.193624923713846\\
5.94	0.193115906889808\\
5.95	0.192602512847652\\
5.96	0.192084783652017\\
5.97	0.191562761640493\\
5.98	0.19103648941787\\
5.99	0.19050600985036\\
6	0.189971366059799\\
6.01	0.189432601417827\\
6.02	0.188889759540044\\
6.03	0.188342884280149\\
6.04	0.187792019724062\\
6.05	0.187237210184025\\
6.06	0.186678500192687\\
6.07	0.186115934497179\\
6.08	0.185549558053167\\
6.09	0.184979416018895\\
6.1	0.184405553749221\\
6.11	0.183828016789635\\
6.12	0.183246850870269\\
6.13	0.182662101899903\\
6.14	0.182073815959955\\
6.15	0.181482039298473\\
6.16	0.180886818324118\\
6.17	0.18028819960014\\
6.18	0.179686229838354\\
6.19	0.179080955893116\\
6.2	0.17847242475529\\
6.21	0.177860683546225\\
6.22	0.177245779511726\\
6.23	0.176627760016027\\
6.24	0.176006672535775\\
6.25	0.175382564654008\\
6.26	0.174755484054144\\
6.27	0.174125478513977\\
6.28	0.173492595899676\\
6.29	0.1728568841598\\
6.3	0.172218391319313\\
6.31	0.171577165473615\\
6.32	0.170933254782588\\
6.33	0.170286707464643\\
6.34	0.169637571790795\\
6.35	0.168985896078742\\
6.36	0.168331728686962\\
6.37	0.167675118008831\\
6.38	0.167016112466753\\
6.39	0.166354760506308\\
6.4	0.165691110590427\\
6.41	0.16502521119358\\
6.42	0.164357110795988\\
6.43	0.163686857877857\\
6.44	0.163014500913636\\
6.45	0.162340088366297\\
6.46	0.161663668681645\\
6.47	0.160985290282647\\
6.48	0.160305001563796\\
6.49	0.159622850885492\\
6.5	0.158938886568466\\
6.51	0.15825315688822\\
6.52	0.157565710069504\\
6.53	0.156876594280826\\
6.54	0.156185857628989\\
6.55	0.155493548153663\\
6.56	0.154799713821993\\
6.57	0.154104402523241\\
6.58	0.153407662063457\\
6.59	0.152709540160195\\
6.6	0.152010084437263\\
6.61	0.151309342419508\\
6.62	0.15060736152764\\
6.63	0.1499041890731\\
6.64	0.149199872252962\\
6.65	0.148494458144878\\
6.66	0.147787993702068\\
6.67	0.147080525748341\\
6.68	0.146372100973175\\
6.69	0.145662765926825\\
6.7	0.144952567015486\\
6.71	0.144241550496492\\
6.72	0.14352976247357\\
6.73	0.14281724889213\\
6.74	0.142104055534609\\
6.75	0.141390228015859\\
6.76	0.140675811778582\\
6.77	0.139960852088818\\
6.78	0.139245394031472\\
6.79	0.138529482505904\\
6.8	0.137813162221557\\
6.81	0.137096477693644\\
6.82	0.136379473238879\\
6.83	0.135662192971265\\
6.84	0.134944680797929\\
6.85	0.134226980415018\\
6.86	0.133509135303637\\
6.87	0.132791188725845\\
6.88	0.132073183720711\\
6.89	0.131355163100412\\
6.9	0.130637169446397\\
6.91	0.129919245105597\\
6.92	0.129201432186696\\
6.93	0.128483772556459\\
6.94	0.127766307836106\\
6.95	0.127049079397757\\
6.96	0.126332128360921\\
6.97	0.12561549558905\\
6.98	0.124899221686146\\
6.99	0.124183346993427\\
7	0.123467911586051\\
7.01	0.122752955269899\\
7.02	0.122038517578414\\
7.03	0.1213246377695\\
7.04	0.12061135482248\\
7.05	0.119898707435113\\
7.06	0.119186734020668\\
7.07	0.118475472705061\\
7.08	0.117764961324047\\
7.09	0.117055237420477\\
7.1	0.116346338241611\\
7.11	0.11563830073649\\
7.12	0.114931161553372\\
7.13	0.114224957037223\\
7.14	0.113519723227275\\
7.15	0.112815495854637\\
7.16	0.112112310339969\\
7.17	0.111410201791219\\
7.18	0.110709205001419\\
7.19	0.110009354446535\\
7.2	0.109310684283388\\
7.21	0.108613228347628\\
7.22	0.107917020151773\\
7.23	0.107222092883299\\
7.24	0.106528479402805\\
7.25	0.105836212242227\\
7.26	0.105145323603112\\
7.27	0.104455845354962\\
7.28	0.103767809033623\\
7.29	0.103081245839751\\
7.3	0.102396186637321\\
7.31	0.101712661952208\\
7.32	0.101030701970822\\
7.33	0.100350336538802\\
7.34	0.0996715951597692\\
7.35	0.0989945069941437\\
7.36	0.0983191008580132\\
7.37	0.0976454052220644\\
7.38	0.0969734482105724\\
7.39	0.0963032576004468\\
7.4	0.0956348608203369\\
7.41	0.0949682849497942\\
7.42	0.0943035567184924\\
7.43	0.093640702505504\\
7.44	0.0929797483386348\\
7.45	0.0923207198938145\\
7.46	0.0916636424945427\\
7.47	0.0910085411113931\\
7.48	0.0903554403615708\\
7.49	0.0897043645085274\\
7.5	0.0890553374616292\\
7.51	0.0884083827758814\\
7.52	0.0877635236517063\\
7.53	0.0871207829347749\\
7.54	0.0864801831158936\\
7.55	0.0858417463309428\\
7.56	0.0852054943608689\\
7.57	0.0845714486317291\\
7.58	0.083939630214788\\
7.59	0.0833100598266659\\
7.6	0.0826827578295388\\
7.61	0.0820577442313888\\
7.62	0.0814350386863051\\
7.63	0.0808146604948357\\
7.64	0.0801966286043874\\
7.65	0.0795809616096762\\
7.66	0.0789676777532255\\
7.67	0.0783567949259133\\
7.68	0.0777483306675663\\
7.69	0.0771423021676022\\
7.7	0.0765387262657182\\
7.71	0.0759376194526259\\
7.72	0.0753389978708321\\
7.73	0.0747428773154653\\
7.74	0.0741492732351464\\
7.75	0.0735582007329042\\
7.76	0.0729696745671344\\
7.77	0.0723837091526021\\
7.78	0.071800318561487\\
7.79	0.0712195165244711\\
7.8	0.0706413164318677\\
7.81	0.0700657313347914\\
7.82	0.0694927739463699\\
7.83	0.0689224566429943\\
7.84	0.0683547914656103\\
7.85	0.0677897901210473\\
7.86	0.067227463983387\\
7.87	0.0666678240953688\\
7.88	0.0661108811698337\\
7.89	0.0655566455912037\\
7.9	0.0650051274169981\\
7.91	0.0644563363793857\\
7.92	0.0639102818867709\\
7.93	0.0633669730254157\\
7.94	0.0628264185610945\\
7.95	0.0622886269407826\\
7.96	0.0617536062943775\\
7.97	0.0612213644364519\\
7.98	0.060691908868039\\
7.99	0.0601652467784482\\
8	0.0596413850471111\\
8.01	0.0591203302454577\\
8.02	0.0586020886388212\\
8.03	0.0580866661883726\\
8.04	0.0575740685530815\\
8.05	0.0570643010917059\\
8.06	0.0565573688648083\\
8.07	0.0560532766367973\\
8.08	0.0555520288779961\\
8.09	0.0550536297667352\\
8.1	0.0545580831914697\\
8.11	0.0540653927529206\\
8.12	0.0535755617662392\\
8.13	0.0530885932631942\\
8.14	0.0526044899943808\\
8.15	0.052123254431452\\
8.16	0.0516448887693693\\
8.17	0.0511693949286748\\
8.18	0.0506967745577832\\
8.19	0.050227029035292\\
8.2	0.0497601594723113\\
8.21	0.0492961667148101\\
8.22	0.0488350513459819\\
8.23	0.0483768136886253\\
8.24	0.0479214538075419\\
8.25	0.0474689715119493\\
8.26	0.0470193663579096\\
8.27	0.0465726376507721\\
8.28	0.0461287844476299\\
8.29	0.0456878055597898\\
8.3	0.0452496995552554\\
8.31	0.0448144647612222\\
8.32	0.0443820992665838\\
8.33	0.0439526009244504\\
8.34	0.0435259673546765\\
8.35	0.0431021959463999\\
8.36	0.0426812838605893\\
8.37	0.042263228032601\\
8.38	0.041848025174744\\
8.39	0.0414356717788534\\
8.4	0.0410261641188703\\
8.41	0.0406194982534291\\
8.42	0.0402156700284508\\
8.43	0.0398146750797419\\
8.44	0.039416508835599\\
8.45	0.0390211665194178\\
8.46	0.0386286431523058\\
8.47	0.0382389335556999\\
8.48	0.0378520323539862\\
8.49	0.0374679339771224\\
8.5	0.0370866326632633\\
8.51	0.0367081224613873\\
8.52	0.0363323972339243\\
8.53	0.0359594506593843\\
8.54	0.0355892762349867\\
8.55	0.0352218672792885\\
8.56	0.034857216934813\\
8.57	0.0344953181706767\\
8.58	0.0341361637852141\\
8.59	0.0337797464086021\\
8.6	0.0334260585054804\\
8.61	0.0330750923775696\\
8.62	0.0327268401662861\\
8.63	0.0323812938553532\\
8.64	0.0320384452734075\\
8.65	0.0316982860966018\\
8.66	0.0313608078512014\\
8.67	0.0310260019161767\\
8.68	0.030693859525789\\
8.69	0.0303643717721696\\
8.7	0.0300375296078941\\
8.71	0.0297133238485476\\
8.72	0.0293917451752839\\
8.73	0.0290727841373765\\
8.74	0.0287564311547617\\
8.75	0.0284426765205727\\
8.76	0.0281315104036655\\
8.77	0.0278229228511355\\
8.78	0.027516903790824\\
8.79	0.0272134430338157\\
8.8	0.0269125302769252\\
8.81	0.0266141551051738\\
8.82	0.0263183069942544\\
8.83	0.0260249753129863\\
8.84	0.0257341493257578\\
8.85	0.0254458181949572\\
8.86	0.0251599709833921\\
8.87	0.0248765966566958\\
8.88	0.0245956840857209\\
8.89	0.0243172220489209\\
8.9	0.0240411992347177\\
8.91	0.0237676042438558\\
8.92	0.0234964255917429\\
8.93	0.0232276517107765\\
8.94	0.0229612709526558\\
8.95	0.0226972715906801\\
8.96	0.0224356418220311\\
8.97	0.0221763697700415\\
8.98	0.0219194434864471\\
8.99	0.0216648509536248\\
9	0.021412580086814\\
9.01	0.0211626187363224\\
9.02	0.0209149546897159\\
9.03	0.0206695756739917\\
9.04	0.0204264693577359\\
9.05	0.0201856233532632\\
9.06	0.0199470252187409\\
9.07	0.0197106624602952\\
9.08	0.0194765225341003\\
9.09	0.0192445928484506\\
9.1	0.0190148607658151\\
9.11	0.0187873136048738\\
9.12	0.018561938642537\\
9.13	0.0183387231159461\\
9.14	0.0181176542244563\\
9.15	0.0178987191316015\\
9.16	0.01768190496704\\
9.17	0.0174671988284827\\
9.18	0.0172545877836021\\
9.19	0.0170440588719226\\
9.2	0.0168355991066919\\
9.21	0.0166291954767339\\
9.22	0.0164248349482823\\
9.23	0.0162225044667949\\
9.24	0.0160221909587488\\
9.25	0.0158238813334166\\
9.26	0.015627562484623\\
9.27	0.0154332212924815\\
9.28	0.015240844625113\\
9.29	0.0150504193403427\\
9.3	0.0148619322873796\\
9.31	0.0146753703084749\\
9.32	0.0144907202405607\\
9.33	0.0143079689168702\\
9.34	0.0141271031685362\\
9.35	0.0139481098261715\\
9.36	0.0137709757214283\\
9.37	0.0135956876885382\\
9.38	0.0134222325658323\\
9.39	0.0132505971972412\\
9.4	0.013080768433775\\
9.41	0.0129127331349834\\
9.42	0.0127464781703963\\
9.43	0.0125819904209438\\
9.44	0.0124192567803563\\
9.45	0.0122582641565455\\
9.46	0.0120989994729642\\
9.47	0.0119414496699472\\
9.48	0.011785601706032\\
9.49	0.0116314425592593\\
9.5	0.0114789592284545\\
9.51	0.0113281387344882\\
9.52	0.0111789681215181\\
9.53	0.0110314344582108\\
9.54	0.0108855248389431\\
9.55	0.0107412263849851\\
9.56	0.0105985262456626\\
9.57	0.0104574115995006\\
9.58	0.0103178696553468\\
9.59	0.010179887653476\\
9.6	0.0100434528666753\\
9.61	0.00990855260130987\\
9.62	0.00977517419836908\\
9.63	0.00964330503449418\\
9.64	0.00951293252298657\\
9.65	0.00938404411479698\\
9.66	0.00925662729949601\\
9.67	0.00913066960622556\\
9.68	0.00900615860463183\\
9.69	0.00888308190577935\\
9.7	0.00876142716304676\\
9.71	0.00864118207300384\\
9.72	0.00852233437627048\\
9.73	0.00840487185835706\\
9.74	0.00828878235048692\\
9.75	0.0081740537304007\\
9.76	0.00806067392314265\\
9.77	0.00794863090182918\\
9.78	0.0078379126883997\\
9.79	0.00772850735434974\\
9.8	0.00762040302144666\\
9.81	0.00751358786242797\\
9.82	0.00740805010168226\\
9.83	0.00730377801591319\\
9.84	0.00720075993478624\\
9.85	0.00709898424155877\\
9.86	0.00699843937369316\\
9.87	0.00689911382345334\\
9.88	0.00680099613848483\\
9.89	0.00670407492237845\\
9.9	0.00660833883521762\\
9.91	0.00651377659410972\\
9.92	0.00642037697370137\\
9.93	0.00632812880667791\\
9.94	0.00623702098424713\\
9.95	0.00614704245660751\\
9.96	0.00605818223340095\\
9.97	0.00597042938415035\\
9.98	0.00588377303868186\\
9.99	0.00579820238753227\\
10	0.00571370668234159\\
};
%\addlegendentry{linearization}

\addplot [color=mycolor2, line width=2.0pt, forget plot]
  table[row sep=crcr]{%
5.04791147756762	0\\
5.04791147756762	0.6\\
};
\addplot [color=mycolor1, dashed, line width=2.0pt, forget plot]
  table[row sep=crcr]{%
5.0284309151552	0\\
5.0284309151552	0.6\\
};
\end{axis}

\begin{axis}[%
width=0.898in,
height=1.5in,%3.603in,
at={(1.981in,0.486in)},
scale only axis,
xmin=0,
xmax=10,
ymin=0,
ymax=0.8,
axis background/.style={fill=white},
title={Exact moment \\ matching},
title style={align=left}, 
axis x line*=bottom,
axis y line*=left,
legend style={legend cell align=left, align=left, draw=white!15!black}
]
\addplot[ybar interval, fill=mycolor1, fill opacity=0.4, draw=mycolor1, area legend] table[row sep=crcr] {%
x	y\\
3.36	0.0144927536231884\\
3.429	0.0289855072463768\\
3.498	0.0869565217391305\\
3.567	0.217391304347825\\
3.636	0.391304347826087\\
3.705	0.565217391304348\\
3.774	0.449275362318841\\
3.843	0.405797101449276\\
3.912	0.666666666666667\\
3.981	0.420289855072464\\
4.05	0.478260869565218\\
4.119	0.289855072463768\\
4.188	0.289855072463768\\
4.257	0.347826086956522\\
4.326	0.246376811594203\\
4.395	0.304347826086953\\
4.464	0.20289855072464\\
4.533	0.217391304347823\\
4.602	0.246376811594203\\
4.671	0.289855072463768\\
4.74	0.246376811594203\\
4.809	0.188405797101449\\
4.878	0.231884057971015\\
4.947	0.27536231884058\\
5.016	0.391304347826087\\
5.085	0.246376811594203\\
5.154	0.27536231884058\\
5.223	0.217391304347826\\
5.292	0.347826086956518\\
5.361	0.231884057971018\\
5.43	0.2463768115942\\
5.499	0.260869565217395\\
5.568	0.275362318840576\\
5.637	0.289855072463772\\
5.706	0.304347826086953\\
5.775	0.289855072463772\\
5.844	0.362318840579706\\
5.913	0.289855072463768\\
5.982	0.463768115942029\\
6.051	0.420289855072464\\
6.12	0.492753623188406\\
6.189	0.463768115942029\\
6.258	0.463768115942029\\
6.327	0.420289855072459\\
6.396	0.246376811594206\\
6.465	0.202898550724635\\
6.534	0.0869565217391316\\
6.603	0.0579710144927529\\
6.672	0.0289855072463772\\
6.741	0.0144927536231882\\
6.81	0.0144927536231882\\
};
%\addlegendentry{ground truth}

\addplot [color=mycolor2, line width=2.0pt]
  table[row sep=crcr]{%
0	1.57796213037878e-07\\
0.01	1.6732895884297e-07\\
0.02	1.7741695560253e-07\\
0.03	1.88091260969029e-07\\
0.04	1.99384592349323e-07\\
0.05	2.11331411084634e-07\\
0.06	2.239680106454e-07\\
0.07	2.37332609018677e-07\\
0.08	2.51465445472897e-07\\
0.09	2.66408881892191e-07\\
0.1	2.82207508880122e-07\\
0.11	2.98908256840605e-07\\
0.12	3.16560512251946e-07\\
0.13	3.35216239358466e-07\\
0.14	3.54930107512825e-07\\
0.15	3.75759624411323e-07\\
0.16	3.97765275473713e-07\\
0.17	4.21010669628791e-07\\
0.18	4.45562691776965e-07\\
0.19	4.71491662211309e-07\\
0.2	4.98871503289274e-07\\
0.21	5.27779913658211e-07\\
0.22	5.58298550349138e-07\\
0.23	5.90513219064949e-07\\
0.24	6.24514073001227e-07\\
0.25	6.6039582055038e-07\\
0.26	6.982579422525e-07\\
0.27	7.38204917369618e-07\\
0.28	7.80346460473661e-07\\
0.29	8.24797768452321e-07\\
0.3	8.71679778351659e-07\\
0.31	9.21119436488881e-07\\
0.32	9.73249979284244e-07\\
0.33	1.02821122627662e-06\\
0.34	1.08614988580351e-06\\
0.35	1.14721987384284e-06\\
0.36	1.2115826465312e-06\\
0.37	1.27940754689035e-06\\
0.38	1.35087216631238e-06\\
0.39	1.42616272137205e-06\\
0.4	1.50547444655403e-06\\
0.41	1.58901200350241e-06\\
0.42	1.6769899074197e-06\\
0.43	1.76963297126329e-06\\
0.44	1.86717676840831e-06\\
0.45	1.96986811446753e-06\\
0.46	2.07796556898125e-06\\
0.47	2.19173995771237e-06\\
0.48	2.31147491630579e-06\\
0.49	2.4374674560944e-06\\
0.5	2.57002855285888e-06\\
0.51	2.70948375937301e-06\\
0.52	2.8561738425921e-06\\
0.53	3.01045544636769e-06\\
0.54	3.172701780599e-06\\
0.55	3.34330333775836e-06\\
0.56	3.52266863775558e-06\\
0.57	3.71122500213506e-06\\
0.58	3.90941935862813e-06\\
0.59	4.11771907711229e-06\\
0.6	4.33661283806007e-06\\
0.61	4.56661153458997e-06\\
0.62	4.80824920926364e-06\\
0.63	5.06208402680582e-06\\
0.64	5.32869928395452e-06\\
0.65	5.6087044576838e-06\\
0.66	5.90273629307341e-06\\
0.67	6.21145993213505e-06\\
0.68	6.53557008493801e-06\\
0.69	6.87579224441405e-06\\
0.7	7.23288394625544e-06\\
0.71	7.60763607535751e-06\\
0.72	8.00087422029293e-06\\
0.73	8.41346007734252e-06\\
0.74	8.84629290564536e-06\\
0.75	9.30031103506883e-06\\
0.76	9.77649342843766e-06\\
0.77	1.02758612998007e-05\\
0.78	1.07994797904525e-05\\
0.79	1.13484597044685e-05\\
0.8	1.19239593055503e-05\\
0.81	1.25271861770208e-05\\
0.82	1.31593991468468e-05\\
0.83	1.38219102796108e-05\\
0.84	1.45160869373932e-05\\
0.85	1.5243353911568e-05\\
0.86	1.60051956275572e-05\\
0.87	1.68031584246318e-05\\
0.88	1.7638852912887e-05\\
0.89	1.85139564095642e-05\\
0.9	1.9430215456933e-05\\
0.91	2.03894484239879e-05\\
0.92	2.13935481942593e-05\\
0.93	2.24444849420767e-05\\
0.94	2.35443089996663e-05\\
0.95	2.46951538175072e-05\\
0.96	2.58992390204102e-05\\
0.97	2.71588735618243e-05\\
0.98	2.8476458978919e-05\\
0.99	2.98544927510274e-05\\
1	3.12955717640784e-05\\
1.01	3.28023958836825e-05\\
1.02	3.43777716395763e-05\\
1.03	3.60246160241672e-05\\
1.04	3.77459604079575e-05\\
1.05	3.95449545746657e-05\\
1.06	4.14248708788915e-05\\
1.07	4.33891085292137e-05\\
1.08	4.54411979996362e-05\\
1.09	4.75848055723336e-05\\
1.1	4.98237380146774e-05\\
1.11	5.21619473935505e-05\\
1.12	5.4603536029991e-05\\
1.13	5.71527615972265e-05\\
1.14	5.98140423651886e-05\\
1.15	6.25919625946185e-05\\
1.16	6.54912780838939e-05\\
1.17	6.85169218717338e-05\\
1.18	7.16740100989391e-05\\
1.19	7.49678480323591e-05\\
1.2	7.84039362542751e-05\\
1.21	8.19879770204012e-05\\
1.22	8.5725880789721e-05\\
1.23	8.96237729293646e-05\\
1.24	9.36880005977504e-05\\
1.25	9.79251398092017e-05\\
1.26	0.00010234200268325\\
1.27	0.000106945644881829\\
1.28	0.000111743373237537\\
1.29	0.000116742753576166\\
1.3	0.000121951618736632\\
1.31	0.000127378076791452\\
1.32	0.000133030519470876\\
1.33	0.000138917630793744\\
1.34	0.000145048395908117\\
1.35	0.000151432110144684\\
1.36	0.000158078388285897\\
1.37	0.000164997174053755\\
1.38	0.000172198749819085\\
1.39	0.000179693746535117\\
1.4	0.000187493153898097\\
1.41	0.000195608330737591\\
1.42	0.000204051015639069\\
1.43	0.000212833337801297\\
1.44	0.000221967828130926\\
1.45	0.000231467430576641\\
1.46	0.000241345513705077\\
1.47	0.000251615882520645\\
1.48	0.000262292790531263\\
1.49	0.000273390952061908\\
1.5	0.000284925554817743\\
1.51	0.000296912272698468\\
1.52	0.000309367278865386\\
1.53	0.000322307259062539\\
1.54	0.000335749425193115\\
1.55	0.000349711529152155\\
1.56	0.000364211876916429\\
1.57	0.000379269342892173\\
1.58	0.000394903384521187\\
1.59	0.0004111340571456\\
1.6	0.000427982029131418\\
1.61	0.000445468597250736\\
1.62	0.000463615702322315\\
1.63	0.000482445945109952\\
1.64	0.00050198260247786\\
1.65	0.000522249643802037\\
1.66	0.000543271747636321\\
1.67	0.000565074318631597\\
1.68	0.000587683504706312\\
1.69	0.000611126214466206\\
1.7	0.000635430134870859\\
1.71	0.000660623749144346\\
1.72	0.000686736354927006\\
1.73	0.00071379808266499\\
1.74	0.000741839914233901\\
1.75	0.00077089370179259\\
1.76	0.000800992186862692\\
1.77	0.000832169019629243\\
1.78	0.00086445877845729\\
1.79	0.000897896989619027\\
1.8	0.000932520147225665\\
1.81	0.000968365733357731\\
1.82	0.00100547223838721\\
1.83	0.00104387918148447\\
1.84	0.00108362713130246\\
1.85	0.00112475772683027\\
1.86	0.00116731369840771\\
1.87	0.00121133888889208\\
1.88	0.00125687827496791\\
1.89	0.0013039779885898\\
1.9	0.0013526853385483\\
1.91	0.00140304883214798\\
1.92	0.00145511819698661\\
1.93	0.0015089444028236\\
1.94	0.00156457968352566\\
1.95	0.00162207755907679\\
1.96	0.00168149285763936\\
1.97	0.00174288173765267\\
1.98	0.00180630170995434\\
1.99	0.00187181165990996\\
2	0.00193947186953536\\
2.01	0.00200934403959567\\
2.02	0.00208149131166451\\
2.03	0.00215597829012625\\
2.04	0.00223287106410366\\
2.05	0.00231223722929271\\
2.06	0.00239414590968566\\
2.07	0.002478667779163\\
2.08	0.00256587508293435\\
2.09	0.00265584165880763\\
2.1	0.0027486429582654\\
2.11	0.00284435606732658\\
2.12	0.00294305972717127\\
2.13	0.00304483435450568\\
2.14	0.00314976206164373\\
2.15	0.00325792667628116\\
2.16	0.00336941376093756\\
2.17	0.00348431063204102\\
2.18	0.00360270637862959\\
2.19	0.0037246918806432\\
2.2	0.00385035982677906\\
2.21	0.003979804731883\\
2.22	0.00411312295384885\\
2.23	0.00425041270999705\\
2.24	0.00439177409290354\\
2.25	0.00453730908564923\\
2.26	0.00468712157645985\\
2.27	0.00484131737270557\\
2.28	0.00500000421422928\\
2.29	0.00516329178597171\\
2.3	0.00533129172986171\\
2.31	0.00550411765593867\\
2.32	0.00568188515267452\\
2.33	0.00586471179646173\\
2.34	0.00605271716023359\\
2.35	0.00624602282118281\\
2.36	0.00644475236754353\\
2.37	0.00664903140440251\\
2.38	0.00685898755850369\\
2.39	0.00707475048201145\\
2.4	0.00729645185519606\\
2.41	0.00752422538800619\\
2.42	0.0077582068204918\\
2.43	0.00799853392204129\\
2.44	0.00824534648939642\\
2.45	0.00849878634340852\\
2.46	0.00875899732449895\\
2.47	0.00902612528678757\\
2.48	0.00930031809085181\\
2.49	0.00958172559508029\\
2.5	0.00987049964558341\\
2.51	0.010166794064625\\
2.52	0.0104707646375381\\
2.53	0.0107825690980882\\
2.54	0.0111023671122482\\
2.55	0.0114303202603494\\
2.56	0.0117665920175712\\
2.57	0.0121113477327361\\
2.58	0.0124647546053737\\
2.59	0.0128269816610191\\
2.6	0.0131981997247123\\
2.61	0.0135785813926636\\
2.62	0.0139683010020526\\
2.63	0.0143675345989288\\
2.64	0.0147764599041794\\
2.65	0.0151952562775363\\
2.66	0.0156241046795887\\
2.67	0.0160631876317733\\
2.68	0.0165126891743121\\
2.69	0.0169727948220707\\
2.7	0.0174436915183084\\
2.71	0.0179255675862951\\
2.72	0.0184186126787698\\
2.73	0.0189230177252151\\
2.74	0.0194389748769265\\
2.75	0.0199666774498538\\
2.76	0.0205063198651942\\
2.77	0.0210580975877172\\
2.78	0.0216222070618048\\
2.79	0.0221988456451888\\
2.8	0.0227882115403718\\
2.81	0.023390503723716\\
2.82	0.0240059218721913\\
2.83	0.0246346662877682\\
2.84	0.02527693781945\\
2.85	0.0259329377829358\\
2.86	0.0266028678779091\\
2.87	0.0272869301029493\\
2.88	0.0279853266680635\\
2.89	0.0286982599048398\\
2.9	0.0294259321742241\\
2.91	0.0301685457719252\\
2.92	0.0309263028314531\\
2.93	0.031699405224802\\
2.94	0.0324880544607849\\
2.95	0.0332924515810362\\
2.96	0.0341127970536951\\
2.97	0.0349492906647884\\
2.98	0.0358021314073319\\
2.99	0.0366715173681734\\
3	0.0375576456125996\\
3.01	0.0384607120667382\\
3.02	0.0393809113977787\\
3.03	0.0403184368920498\\
3.04	0.041273480330984\\
3.05	0.0422462318650073\\
3.06	0.0432368798853951\\
3.07	0.0442456108941351\\
3.08	0.045272609371842\\
3.09	0.0463180576437729\\
3.1	0.0473821357439927\\
3.11	0.0484650212777422\\
3.12	0.0495668892820659\\
3.13	0.0506879120847558\\
3.14	0.0518282591616751\\
3.15	0.052988096992522\\
3.16	0.0541675889151041\\
3.17	0.0553668949781894\\
3.18	0.0565861717930083\\
3.19	0.057825572383479\\
3.2	0.0590852460352369\\
3.21	0.0603653381435441\\
3.22	0.0616659900601659\\
3.23	0.0629873389392968\\
3.24	0.0643295175826253\\
3.25	0.0656926542836283\\
3.26	0.0670768726711883\\
3.27	0.0684822915526288\\
3.28	0.0699090247562672\\
3.29	0.0713571809735846\\
3.3	0.0728268636011179\\
3.31	0.0743181705821772\\
3.32	0.0758311942484995\\
3.33	0.0773660211619461\\
3.34	0.0789227319563588\\
3.35	0.0805014011796868\\
3.36	0.0821020971365045\\
3.37	0.0837248817310358\\
3.38	0.0853698103108073\\
3.39	0.0870369315110538\\
3.4	0.0887262870999983\\
3.41	0.0904379118251351\\
3.42	0.0921718332606424\\
3.43	0.093928071656055\\
3.44	0.0957066397863258\\
3.45	0.0975075428034124\\
3.46	0.0993307780895168\\
3.47	0.10117633511212\\
3.48	0.103044195280937\\
3.49	0.104934331806946\\
3.5	0.106846709563605\\
3.51	0.10878128495042\\
3.52	0.110738005758985\\
3.53	0.112716811041641\\
3.54	0.114717630982898\\
3.55	0.116740386773744\\
3.56	0.118784990489004\\
3.57	0.120851344967873\\
3.58	0.122939343697765\\
3.59	0.125048870701625\\
3.6	0.127179800428835\\
3.61	0.129331997649863\\
3.62	0.131505317354778\\
3.63	0.133699604655791\\
3.64	0.135914694693932\\
3.65	0.138150412550017\\
3.66	0.14040657316004\\
3.67	0.142682981235104\\
3.68	0.144979431186045\\
3.69	0.147295707052865\\
3.7	0.149631582439104\\
3.71	0.151986820451277\\
3.72	0.154361173643503\\
3.73	0.156754383967443\\
3.74	0.159166182727665\\
3.75	0.161596290542557\\
3.76	0.164044417310898\\
3.77	0.166510262184199\\
3.78	0.168993513544918\\
3.79	0.17149384899066\\
3.8	0.174010935324459\\
3.81	0.176544428551235\\
3.82	0.179093973880531\\
3.83	0.181659205735608\\
3.84	0.184239747768995\\
3.85	0.186835212884572\\
3.86	0.189445203266253\\
3.87	0.192069310413369\\
3.88	0.194707115182791\\
3.89	0.197358187837878\\
3.9	0.200022088104301\\
3.91	0.2026983652328\\
3.92	0.20538655806893\\
3.93	0.208086195129822\\
3.94	0.210796794688034\\
3.95	0.213517864862485\\
3.96	0.216248903716537\\
3.97	0.218989399363226\\
3.98	0.221738830077677\\
3.99	0.224496664416695\\
4	0.227262361345569\\
4.01	0.230035370372056\\
4.02	0.232815131687571\\
4.03	0.235601076315548\\
4.04	0.238392626266977\\
4.05	0.241189194703077\\
4.06	0.243990186105084\\
4.07	0.24679499645112\\
4.08	0.249603013400101\\
4.09	0.252413616482629\\
4.1	0.255226177298835\\
4.11	0.258040059723083\\
4.12	0.260854620115493\\
4.13	0.263669207540208\\
4.14	0.266483163990311\\
4.15	0.269295824619319\\
4.16	0.272106517979162\\
4.17	0.274914566264549\\
4.18	0.277719285563608\\
4.19	0.280519986114707\\
4.2	0.283315972569318\\
4.21	0.286106544260821\\
4.22	0.288890995479109\\
4.23	0.291668615750864\\
4.24	0.294438690125349\\
4.25	0.297200499465598\\
4.26	0.299953320744823\\
4.27	0.302696427347894\\
4.28	0.305429089377727\\
4.29	0.308150573966407\\
4.3	0.310860145590876\\
4.31	0.313557066393003\\
4.32	0.316240596503845\\
4.33	0.318909994371925\\
4.34	0.321564517095314\\
4.35	0.324203420757327\\
4.36	0.326825960765618\\
4.37	0.32943139219449\\
4.38	0.332018970130165\\
4.39	0.334587950018842\\
4.4	0.337137588017281\\
4.41	0.339667141345718\\
4.42	0.342175868642858\\
4.43	0.344663030322738\\
4.44	0.347127888933195\\
4.45	0.349569709515727\\
4.46	0.351987759966485\\
4.47	0.354381311398166\\
4.48	0.356749638502546\\
4.49	0.359092019913418\\
4.5	0.361407738569672\\
4.51	0.363696082078269\\
4.52	0.365956343076854\\
4.53	0.36818781959575\\
4.54	0.37038981541908\\
4.55	0.372561640444749\\
4.56	0.374702611043043\\
4.57	0.376812050413566\\
4.58	0.378889288940274\\
4.59	0.380933664544337\\
4.6	0.382944523034563\\
4.61	0.384921218455149\\
4.62	0.386863113430472\\
4.63	0.38876957950668\\
4.64	0.390639997489835\\
4.65	0.392473757780324\\
4.66	0.394270260703325\\
4.67	0.396028916835047\\
4.68	0.397749147324504\\
4.69	0.399430384210594\\
4.7	0.401072070734212\\
4.71	0.402673661645181\\
4.72	0.404234623503753\\
4.73	0.405754434976448\\
4.74	0.407232587126003\\
4.75	0.408668583695207\\
4.76	0.410061941384398\\
4.77	0.411412190122402\\
4.78	0.412718873330714\\
4.79	0.413981548180693\\
4.8	0.415199785843588\\
4.81	0.416373171733188\\
4.82	0.417501305740899\\
4.83	0.418583802463076\\
4.84	0.419620291420413\\
4.85	0.420610417269232\\
4.86	0.421553840004482\\
4.87	0.422450235154311\\
4.88	0.423299293966035\\
4.89	0.424100723583361\\
4.9	0.424854247214724\\
4.91	0.425559604292603\\
4.92	0.426216550623681\\
4.93	0.426824858529737\\
4.94	0.42738431697915\\
4.95	0.427894731708904\\
4.96	0.428355925337014\\
4.97	0.42876773746526\\
4.98	0.429130024772156\\
4.99	0.429442661096082\\
5	0.429705537508501\\
5.01	0.429918562377215\\
5.02	0.430081661419596\\
5.03	0.430194777745755\\
5.04	0.430257871891611\\
5.05	0.430270921841844\\
5.06	0.430233923042688\\
5.07	0.43014688840459\\
5.08	0.43000984829469\\
5.09	0.429822850519175\\
5.1	0.429585960295475\\
5.11	0.429299260214361\\
5.12	0.428962850191953\\
5.13	0.42857684741169\\
5.14	0.428141386256298\\
5.15	0.427656618229819\\
5.16	0.427122711869768\\
5.17	0.426539852649476\\
5.18	0.425908242870709\\
5.19	0.425228101546648\\
5.2	0.424499664275325\\
5.21	0.423723183103612\\
5.22	0.422898926381879\\
5.23	0.422027178609439\\
5.24	0.421108240270897\\
5.25	0.420142427663547\\
5.26	0.419130072715939\\
5.27	0.418071522797781\\
5.28	0.416967140521312\\
5.29	0.415817303534313\\
5.3	0.414622404304924\\
5.31	0.413382849898431\\
5.32	0.412099061746202\\
5.33	0.410771475406965\\
5.34	0.409400540320603\\
5.35	0.407986719554673\\
5.36	0.406530489543842\\
5.37	0.405032339822449\\
5.38	0.403492772750393\\
5.39	0.401912303232586\\
5.4	0.400291458432158\\
5.41	0.398630777477662\\
5.42	0.396930811164499\\
5.43	0.395192121650785\\
5.44	0.393415282147916\\
5.45	0.391600876606045\\
5.46	0.389749499394726\\
5.47	0.387861754978974\\
5.48	0.38593825759097\\
5.49	0.383979630897678\\
5.5	0.381986507664611\\
5.51	0.379959529416016\\
5.52	0.377899346091709\\
5.53	0.375806615700843\\
5.54	0.373682003972844\\
5.55	0.371526184005785\\
5.56	0.369339835912456\\
5.57	0.367123646464386\\
5.58	0.364878308734077\\
5.59	0.362604521735712\\
5.6	0.360302990064595\\
5.61	0.35797442353558\\
5.62	0.355619536820747\\
5.63	0.353239049086587\\
5.64	0.350833683630945\\
5.65	0.348404167519972\\
5.66	0.345951231225358\\
5.67	0.343475608262067\\
5.68	0.340978034826848\\
5.69	0.338459249437751\\
5.7	0.335919992574906\\
5.71	0.333361006322791\\
5.72	0.330783034014235\\
5.73	0.328186819876394\\
5.74	0.325573108678915\\
5.75	0.322942645384542\\
5.76	0.320296174802359\\
5.77	0.317634441243908\\
5.78	0.314958188182398\\
5.79	0.312268157915207\\
5.8	0.309565091229886\\
5.81	0.306849727073878\\
5.82	0.304122802228139\\
5.83	0.301385050984853\\
5.84	0.298637204829441\\
5.85	0.295879992127034\\
5.86	0.293114137813595\\
5.87	0.290340363091857\\
5.88	0.287559385132253\\
5.89	0.28477191677898\\
5.9	0.281978666261385\\
5.91	0.279180336910783\\
5.92	0.276377626882882\\
5.93	0.273571228885938\\
5.94	0.270761829914778\\
5.95	0.267950110990813\\
5.96	0.265136746908157\\
5.97	0.26232240598598\\
5.98	0.259507749827182\\
5.99	0.256693433083513\\
6	0.253880103227204\\
6.01	0.25106840032923\\
6.02	0.248258956844258\\
6.03	0.24545239740238\\
6.04	0.242649338607682\\
6.05	0.23985038884372\\
6.06	0.237056148085968\\
6.07	0.234267207721267\\
6.08	0.231484150374343\\
6.09	0.228707549741418\\
6.1	0.225937970430952\\
6.11	0.223175967811532\\
6.12	0.220422087866947\\
6.13	0.217676867058438\\
6.14	0.214940832194157\\
6.15	0.212214500305811\\
6.16	0.209498378532504\\
6.17	0.206792964011768\\
6.18	0.20409874377775\\
6.19	0.201416194666557\\
6.2	0.198745783228713\\
6.21	0.196087965648708\\
6.22	0.193443187671595\\
6.23	0.1908118845366\\
6.24	0.188194480917686\\
6.25	0.185591390871025\\
6.26	0.183003017789326\\
6.27	0.180429754362946\\
6.28	0.17787198254772\\
6.29	0.175330073539448\\
6.3	0.172804387754944\\
6.31	0.170295274819593\\
6.32	0.1678030735613\\
6.33	0.165328112010778\\
6.34	0.162870707408052\\
6.35	0.160431166215103\\
6.36	0.158009784134541\\
6.37	0.155606846134221\\
6.38	0.153222626477669\\
6.39	0.150857388760237\\
6.4	0.148511385950859\\
6.41	0.146184860439295\\
6.42	0.14387804408875\\
6.43	0.141591158293747\\
6.44	0.139324414043124\\
6.45	0.137078011988046\\
6.46	0.134852142514887\\
6.47	0.132646985822869\\
6.48	0.130462712006318\\
6.49	0.128299481141408\\
6.5	0.126157443377264\\
6.51	0.124036739031276\\
6.52	0.121937498688509\\
6.53	0.11985984330505\\
6.54	0.117803884315168\\
6.55	0.115769723742151\\
6.56	0.113757454312658\\
6.57	0.111767159574483\\
6.58	0.109798914017555\\
6.59	0.107852783198056\\
6.6	0.105928823865511\\
6.61	0.104027084092708\\
6.62	0.102147603408304\\
6.63	0.100290412931992\\
6.64	0.0984555355120719\\
6.65	0.096642985865296\\
6.66	0.0948527707188535\\
6.67	0.093084888954349\\
6.68	0.0913393317536445\\
6.69	0.0896160827464269\\
6.7	0.0879151181593685\\
6.71	0.0862364069667458\\
6.72	0.0845799110423867\\
6.73	0.0829455853128137\\
6.74	0.0813333779114575\\
6.75	0.0797432303338104\\
6.76	0.0781750775933965\\
6.77	0.0766288483784348\\
6.78	0.0751044652090717\\
6.79	0.0736018445950657\\
6.8	0.072120897193804\\
6.81	0.0706615279685363\\
6.82	0.0692236363467128\\
6.83	0.0678071163783132\\
6.84	0.066411856894059\\
6.85	0.0650377416634018\\
6.86	0.063684649552183\\
6.87	0.0623524546798617\\
6.88	0.0610410265762144\\
6.89	0.059750230337404\\
6.9	0.0584799267813295\\
6.91	0.0572299726021594\\
6.92	0.0560002205239619\\
6.93	0.0547905194533451\\
6.94	0.0536007146310212\\
6.95	0.0524306477822145\\
6.96	0.0512801572658352\\
6.97	0.0501490782223411\\
6.98	0.049037242720213\\
6.99	0.0479444799009781\\
7	0.0468706161227059\\
7.01	0.0458154751019185\\
7.02	0.0447788780538475\\
7.03	0.0437606438309811\\
7.04	0.0427605890598424\\
7.05	0.0417785282759456\\
7.06	0.0408142740568774\\
7.07	0.0398676371534559\\
7.08	0.0389384266189196\\
7.09	0.0380264499361035\\
7.1	0.0371315131425608\\
7.11	0.0362534209535923\\
7.12	0.0353919768831473\\
7.13	0.0345469833625628\\
7.14	0.0337182418571098\\
7.15	0.0329055529803198\\
7.16	0.0321087166060638\\
7.17	0.031327531978362\\
7.18	0.0305617978189021\\
7.19	0.0298113124322489\\
7.2	0.0290758738087266\\
7.21	0.0283552797249624\\
7.22	0.027649327842078\\
7.23	0.0269578158015187\\
7.24	0.0262805413185151\\
7.25	0.0256173022731688\\
7.26	0.0249678967991615\\
7.27	0.024332123370084\\
7.28	0.0237097808833873\\
7.29	0.023100668741957\\
7.3	0.0225045869333163\\
7.31	0.0219213361064622\\
7.32	0.0213507176463456\\
7.33	0.0207925337460005\\
7.34	0.0202465874763387\\
7.35	0.0197126828536178\\
7.36	0.0191906249046004\\
7.37	0.0186802197294185\\
7.38	0.0181812745621606\\
7.39	0.0176935978292\\
7.4	0.0172169992052859\\
7.41	0.0167512896674144\\
7.42	0.0162962815465073\\
7.43	0.0158517885769169\\
7.44	0.0154176259437855\\
7.45	0.014993610328283\\
7.46	0.0145795599507499\\
7.47	0.0141752946117745\\
7.48	0.0137806357312301\\
7.49	0.0133954063853056\\
7.5	0.0130194313415551\\
7.51	0.0126525370920022\\
7.52	0.0122945518843264\\
7.53	0.0119453057511672\\
7.54	0.0116046305375768\\
7.55	0.0112723599266554\\
7.56	0.0109483294634039\\
7.57	0.0106323765768272\\
7.58	0.0103243406003238\\
7.59	0.0100240627903965\\
7.6	0.00973138634372004\\
7.61	0.00944615641260181\\
7.62	0.00916822011887066\\
7.63	0.00889742656623163\\
7.64	0.00863362685112175\\
7.65	0.00837667407210451\\
7.66	0.00812642333783901\\
7.67	0.00788273177366106\\
7.68	0.00764545852681277\\
7.69	0.00741446477035737\\
7.7	0.00718961370581621\\
7.71	0.00697077056456436\\
7.72	0.00675780260802152\\
7.73	0.00655057912667446\\
7.74	0.0063489714379676\\
7.75	0.0061528528830971\\
7.76	0.00596209882274507\\
7.77	0.00577658663178883\\
7.78	0.00559619569302082\\
7.79	0.00542080738991404\\
7.8	0.00525030509846783\\
7.81	0.00508457417816796\\
7.82	0.00492350196209547\\
7.83	0.00476697774621747\\
7.84	0.00461489277789316\\
7.85	0.004467140243628\\
7.86	0.00432361525610802\\
7.87	0.00418421484054641\\
7.88	0.00404883792037373\\
7.89	0.00391738530230245\\
7.9	0.00378975966079661\\
7.91	0.00366586552197609\\
7.92	0.00354560924698525\\
7.93	0.00342889901485462\\
7.94	0.00331564480488402\\
7.95	0.00320575837857484\\
7.96	0.00309915326113891\\
7.97	0.0029957447226103\\
7.98	0.00289544975858655\\
7.99	0.0027981870706246\\
8	0.0027038770463165\\
8.01	0.00261244173906925\\
8.02	0.00252380484761261\\
8.03	0.00243789169525811\\
8.04	0.00235462920893187\\
8.05	0.00227394589800328\\
8.06	0.00219577183293114\\
8.07	0.00212003862374783\\
8.08	0.00204667939840225\\
8.09	0.00197562878098082\\
8.1	0.00190682286982598\\
8.11	0.0018401992155706\\
8.12	0.00177569679910627\\
8.13	0.00171325600950294\\
8.14	0.00165281862189675\\
8.15	0.00159432777536226\\
8.16	0.00153772795078486\\
8.17	0.00148296494874843\\
8.18	0.00142998586745307\\
8.19	0.00137873908067667\\
8.2	0.00132917421579415\\
8.21	0.00128124213186715\\
8.22	0.00123489489781688\\
8.23	0.00119008577069181\\
8.24	0.00114676917404195\\
8.25	0.00110490067641056\\
8.26	0.00106443696995373\\
8.27	0.0010253358491979\\
8.28	0.000987556189944738\\
8.29	0.0009510579283326\\
8.3	0.000915802040062941\\
8.31	0.000881750519800031\\
8.32	0.00084886636075151\\
8.33	0.000817113534437219\\
8.34	0.000786456970653031\\
8.35	0.000756862537636212\\
8.36	0.000728297022438306\\
8.37	0.000700728111511244\\
8.38	0.000674124371511893\\
8.39	0.00064845523032998\\
8.4	0.000623690958343928\\
8.41	0.000599802649908723\\
8.42	0.000576762205079736\\
8.43	0.000554542311575931\\
8.44	0.000533116426985664\\
8.45	0.000512458761217927\\
8.46	0.000492544259201582\\
8.47	0.000473348583834874\\
8.48	0.000454848099187134\\
8.49	0.000437019853954429\\
8.5	0.000419841565170547\\
8.51	0.000403291602174477\\
8.52	0.000387348970835326\\
8.53	0.000371993298035319\\
8.54	0.000357204816411329\\
8.55	0.000342964349355168\\
8.56	0.000329253296272633\\
8.57	0.000316053618101072\\
8.58	0.000303347823085131\\
8.59	0.000291118952810022\\
8.6	0.00027935056849159\\
8.61	0.000268026737522192\\
8.62	0.000257132020271303\\
8.63	0.000246651457139548\\
8.64	0.000236570555864774\\
8.65	0.000226875279078558\\
8.66	0.000217552032111468\\
8.67	0.00020858765104525\\
8.68	0.000199969391009976\\
8.69	0.000191684914724091\\
8.7	0.000183722281275182\\
8.71	0.000176069935139188\\
8.72	0.000168716695435684\\
8.73	0.000161651745416754\\
8.74	0.000154864622186933\\
8.75	0.000148345206651577\\
8.76	0.000142083713690956\\
8.77	0.000136070682557319\\
8.78	0.000130296967492101\\
8.79	0.00012475372856038\\
8.8	0.000119432422699661\\
8.81	0.000114324794980009\\
8.82	0.000109422870072488\\
8.83	0.000104718943922871\\
8.84	0.00010020557562752\\
8.85	9.58755795083083e-05\\
8.86	9.17220173834464e-05\\
8.87	8.77381910310472e-05\\
8.88	8.39176348422412e-05\\
8.89	8.0254108660648e-05\\
8.9	7.67415908050039e-05\\
8.91	7.33742712717205e-05\\
8.92	7.01465451141713e-05\\
8.93	6.70530059954835e-05\\
8.94	6.40884399116234e-05\\
8.95	6.12478190815682e-05\\
8.96	5.85262960013651e-05\\
8.97	5.59191976588905e-05\\
8.98	5.3422019906128e-05\\
8.99	5.10304219858156e-05\\
9	4.87402212093111e-05\\
9.01	4.65473877825589e-05\\
9.02	4.4448039777055e-05\\
9.03	4.24384382427343e-05\\
9.04	4.05149824597284e-05\\
9.05	3.86742053259695e-05\\
9.06	3.69127688776458e-05\\
9.07	3.52274599395366e-05\\
9.08	3.36151859023003e-05\\
9.09	3.20729706238066e-05\\
9.1	3.05979504516463e-05\\
9.11	2.91873703639862e-05\\
9.12	2.78385802259674e-05\\
9.13	2.65490311588883e-05\\
9.14	2.53162720194479e-05\\
9.15	2.413794598636e-05\\
9.16	2.30117872516976e-05\\
9.17	2.19356178143529e-05\\
9.18	2.09073443730507e-05\\
9.19	1.99249553163848e-05\\
9.2	1.89865178073933e-05\\
9.21	1.80901749602264e-05\\
9.22	1.7234143106505e-05\\
9.23	1.64167091490052e-05\\
9.24	1.5636228000354e-05\\
9.25	1.48911201044538e-05\\
9.26	1.41798690384032e-05\\
9.27	1.35010191927194e-05\\
9.28	1.28531735277123e-05\\
9.29	1.22349914038993e-05\\
9.3	1.16451864843962e-05\\
9.31	1.10825247072582e-05\\
9.32	1.05458223257856e-05\\
9.33	1.00339440148562e-05\\
9.34	9.54580104138018e-06\\
9.35	9.08034949702025e-06\\
9.36	8.63658859135671e-06\\
9.37	8.21355900371943e-06\\
9.38	7.81034129194826e-06\\
9.39	7.4260543563832e-06\\
9.4	7.05985395742368e-06\\
9.41	6.71093128503779e-06\\
9.42	6.37851157863778e-06\\
9.43	6.06185279577802e-06\\
9.44	5.76024432816823e-06\\
9.45	5.47300576353238e-06\\
9.46	5.1994856918794e-06\\
9.47	4.93906055478872e-06\\
9.48	4.69113353634792e-06\\
9.49	4.45513349441633e-06\\
9.5	4.23051393092132e-06\\
9.51	4.01675199992853e-06\\
9.52	3.81334755226055e-06\\
9.53	3.61982221547129e-06\\
9.54	3.43571850801515e-06\\
9.55	3.26059898648207e-06\\
9.56	3.0940454248004e-06\\
9.57	2.9356580243395e-06\\
9.58	2.78505465387479e-06\\
9.59	2.64187011840634e-06\\
9.6	2.50575545585127e-06\\
9.61	2.37637726065802e-06\\
9.62	2.25341703341841e-06\\
9.63	2.13657055558007e-06\\
9.64	2.0255472883885e-06\\
9.65	1.92006979521328e-06\\
9.66	1.81987318643892e-06\\
9.67	1.72470458612492e-06\\
9.68	1.63432261966389e-06\\
9.69	1.54849692169029e-06\\
9.7	1.46700766351513e-06\\
9.71	1.38964509938443e-06\\
9.72	1.31620913088152e-06\\
9.73	1.24650888881384e-06\\
9.74	1.18036233194682e-06\\
9.75	1.11759586196676e-06\\
9.76	1.05804395407504e-06\\
9.77	1.00154880263508e-06\\
9.78	9.47959981312304e-07\\
9.79	8.9713411716567e-07\\
9.8	8.48934578167194e-07\\
9.81	8.0323117364319e-07\\
9.82	7.59899867147783e-07\\
9.83	7.18822501295873e-07\\
9.84	6.79886534098404e-07\\
9.85	6.42984786358577e-07\\
9.86	6.0801519970248e-07\\
9.87	5.74880604832494e-07\\
9.88	5.43488499605858e-07\\
9.89	5.1375083655478e-07\\
9.9	4.8558381947767e-07\\
9.91	4.58907708744353e-07\\
9.92	4.33646634970517e-07\\
9.93	4.09728420729044e-07\\
9.94	3.87084409977666e-07\\
9.95	3.65649304893996e-07\\
9.96	3.45361009820035e-07\\
9.97	3.26160482029178e-07\\
9.98	3.07991589039127e-07\\
9.99	2.90800972204353e-07\\
10	2.74537916331524e-07\\
};
%\addlegendentry{moment matching}

\addplot [color=mycolor2, line width=2.0pt, forget plot]
  table[row sep=crcr]{%
5.04760739417037	0\\
5.04760739417037	0.6\\
};
\addplot [color=mycolor1, dashed, line width=2.0pt, forget plot]
  table[row sep=crcr]{%
5.0284309151552	0\\
5.0284309151552	0.6\\
};
\end{axis}

\begin{axis}[%
width=0.898in,
height=1.5in,%3.603in,
at={(3.196in,0.486in)},
scale only axis,
xmin=0,
xmax=10,
ymin=0,
ymax=0.8,
title={Sigma-point \\ propagation},
title style={align=left}, 
axis background/.style={fill=white},
axis x line*=bottom,
axis y line*=left,
legend style={legend cell align=right, align=right, draw=white!15!black, cells={align=right}, font=\tiny}
]
\addplot[ybar interval, fill=mycolor1, fill opacity=0.4, draw=mycolor1, area legend] table[row sep=crcr] {%
x	y\\
3.36	0.0144927536231884\\
3.429	0.0289855072463768\\
3.498	0.0869565217391305\\
3.567	0.217391304347825\\
3.636	0.391304347826087\\
3.705	0.565217391304348\\
3.774	0.449275362318841\\
3.843	0.405797101449276\\
3.912	0.666666666666667\\
3.981	0.420289855072464\\
4.05	0.478260869565218\\
4.119	0.289855072463768\\
4.188	0.289855072463768\\
4.257	0.347826086956522\\
4.326	0.246376811594203\\
4.395	0.304347826086953\\
4.464	0.20289855072464\\
4.533	0.217391304347823\\
4.602	0.246376811594203\\
4.671	0.289855072463768\\
4.74	0.246376811594203\\
4.809	0.188405797101449\\
4.878	0.231884057971015\\
4.947	0.27536231884058\\
5.016	0.391304347826087\\
5.085	0.246376811594203\\
5.154	0.27536231884058\\
5.223	0.217391304347826\\
5.292	0.347826086956518\\
5.361	0.231884057971018\\
5.43	0.2463768115942\\
5.499	0.260869565217395\\
5.568	0.275362318840576\\
5.637	0.289855072463772\\
5.706	0.304347826086953\\
5.775	0.289855072463772\\
5.844	0.362318840579706\\
5.913	0.289855072463768\\
5.982	0.463768115942029\\
6.051	0.420289855072464\\
6.12	0.492753623188406\\
6.189	0.463768115942029\\
6.258	0.463768115942029\\
6.327	0.420289855072459\\
6.396	0.246376811594206\\
6.465	0.202898550724635\\
6.534	0.0869565217391316\\
6.603	0.0579710144927529\\
6.672	0.0289855072463772\\
6.741	0.0144927536231882\\
6.81	0.0144927536231882\\
};
%\addlegendentry{Numerical \\ approx.}

\addplot [color=mycolor2, line width=2.0pt]
  table[row sep=crcr]{%
0	8.87764380319432e-05\\
0.01	9.1700822136384e-05\\
0.02	9.47154517333371e-05\\
0.03	9.78228998119414e-05\\
0.04	0.000101025805675457\\
0.05	0.000104326876427423\\
0.06	0.000107728888484391\\
0.07	0.000111234689115493\\
0.08	0.000114847198009143\\
0.09	0.000118569408867125\\
0.1	0.000122404391026334\\
0.11	0.00012635529110845\\
0.12	0.000130425334697778\\
0.13	0.000134617828047533\\
0.14	0.000138936159814801\\
0.15	0.000143383802824432\\
0.16	0.000147964315862095\\
0.17	0.000152681345496739\\
0.18	0.000157538627932674\\
0.19	0.000162539990891505\\
0.2	0.000167689355524124\\
0.21	0.000172990738352978\\
0.22	0.000178448253244802\\
0.23	0.000184066113414019\\
0.24	0.000189848633456982\\
0.25	0.000195800231417249\\
0.26	0.000201925430882044\\
0.27	0.000208228863110072\\
0.28	0.000214715269190831\\
0.29	0.000221389502235576\\
0.3	0.000228256529600039\\
0.31	0.000235321435139054\\
0.32	0.000242589421493169\\
0.33	0.000250065812407359\\
0.34	0.000257756055081929\\
0.35	0.000265665722555657\\
0.36	0.000273800516121267\\
0.37	0.000282166267773259\\
0.38	0.000290768942688146\\
0.39	0.000299614641737107\\
0.4	0.000308709604031073\\
0.41	0.000318060209498228\\
0.42	0.000327672981493914\\
0.43	0.0003375545894429\\
0.44	0.00034771185151395\\
0.45	0.000358151737326627\\
0.46	0.000368881370690256\\
0.47	0.000379908032374928\\
0.48	0.00039123916291442\\
0.49	0.000402882365440907\\
0.5	0.000414845408551292\\
0.51	0.000427136229204983\\
0.52	0.000439762935652922\\
0.53	0.000452733810397627\\
0.54	0.000466057313184048\\
0.55	0.000479742084020933\\
0.56	0.000493796946232453\\
0.57	0.000508230909539774\\
0.58	0.000523053173172236\\
0.59	0.00053827312900782\\
0.6	0.000553900364742502\\
0.61	0.000569944667088102\\
0.62	0.000586416024998223\\
0.63	0.000603324632921818\\
0.64	0.000620680894083919\\
0.65	0.000638495423793025\\
0.66	0.000656779052774621\\
0.67	0.000675542830530304\\
0.68	0.000694798028721899\\
0.69	0.000714556144579973\\
0.7	0.000734828904336119\\
0.71	0.000755628266678342\\
0.72	0.000776966426228842\\
0.73	0.000798855817043487\\
0.74	0.000821309116132199\\
0.75	0.000844339246999491\\
0.76	0.000867959383204324\\
0.77	0.000892182951938454\\
0.78	0.000917023637622338\\
0.79	0.000942495385517763\\
0.8	0.000968612405356183\\
0.81	0.000995389174981824\\
0.82	0.00102284044400852\\
0.83	0.00105098123748923\\
0.84	0.00107982685959722\\
0.85	0.00110939289731764\\
0.86	0.00113969522414855\\
0.87	0.00117075000380997\\
0.88	0.00120257369395996\\
0.89	0.00123518304991627\\
0.9	0.00126859512838244\\
0.91	0.00130282729117678\\
0.92	0.00133789720896316\\
0.93	0.00137382286498187\\
0.94	0.00141062255877942\\
0.95	0.00144831490993543\\
0.96	0.0014869188617855\\
0.97	0.0015264536851381\\
0.98	0.00156693898198413\\
0.99	0.00160839468919738\\
1	0.00165084108222428\\
1.01	0.00169429877876111\\
1.02	0.00173878874241707\\
1.03	0.00178433228636116\\
1.04	0.00183095107695133\\
1.05	0.00187866713734369\\
1.06	0.00192750285108013\\
1.07	0.00197748096565218\\
1.08	0.00202862459603924\\
1.09	0.00208095722821903\\
1.1	0.00213450272264824\\
1.11	0.00218928531771122\\
1.12	0.0022453296331345\\
1.13	0.00230266067336505\\
1.14	0.00236130383090986\\
1.15	0.00242128488963473\\
1.16	0.0024826300280197\\
1.17	0.00254536582236912\\
1.18	0.00260951924997343\\
1.19	0.0026751176922208\\
1.2	0.00274218893765555\\
1.21	0.00281076118498133\\
1.22	0.00288086304600608\\
1.23	0.00295252354852655\\
1.24	0.00302577213914932\\
1.25	0.00310063868604605\\
1.26	0.00317715348163996\\
1.27	0.00325534724522087\\
1.28	0.0033352511254861\\
1.29	0.00341689670300425\\
1.3	0.00350031599259913\\
1.31	0.00358554144565092\\
1.32	0.00367260595231155\\
1.33	0.0037615428436315\\
1.34	0.00385238589359484\\
1.35	0.00394516932105972\\
1.36	0.00403992779160099\\
1.37	0.00413669641925212\\
1.38	0.00423551076814321\\
1.39	0.00433640685403186\\
1.4	0.0044394211457239\\
1.41	0.00454459056638071\\
1.42	0.00465195249470999\\
1.43	0.00476154476603654\\
1.44	0.00487340567325009\\
1.45	0.00498757396762667\\
1.46	0.00510408885952039\\
1.47	0.00522299001892209\\
1.48	0.00534431757588181\\
1.49	0.00546811212079153\\
1.5	0.00559441470452481\\
1.51	0.00572326683843017\\
1.52	0.00585471049417453\\
1.53	0.00598878810343347\\
1.54	0.00612554255742495\\
1.55	0.00626501720628298\\
1.56	0.00640725585826777\\
1.57	0.00655230277880913\\
1.58	0.00670020268937946\\
1.59	0.00685100076619316\\
1.6	0.00700474263872876\\
1.61	0.0071614743880705\\
1.62	0.00732124254506602\\
1.63	0.00748409408829659\\
1.64	0.00765007644185648\\
1.65	0.00781923747293826\\
1.66	0.00799162548922045\\
1.67	0.00816728923605427\\
1.68	0.00834627789344614\\
1.69	0.00852864107283249\\
1.7	0.00871442881364377\\
1.71	0.00890369157965414\\
1.72	0.00909648025511389\\
1.73	0.00929284614066099\\
1.74	0.00949284094900887\\
1.75	0.00969651680040709\\
1.76	0.00990392621787196\\
1.77	0.0101151221221837\\
1.78	0.0103301578266475\\
1.79	0.0105490870316149\\
1.8	0.0107719638187632\\
1.81	0.0109988426451295\\
1.82	0.0112297783368967\\
1.83	0.0114648260829282\\
1.84	0.0117040414280498\\
1.85	0.0119474802660738\\
1.86	0.0121951988325657\\
1.87	0.0124472536973477\\
1.88	0.0127037017567393\\
1.89	0.0129646002255305\\
1.9	0.0132300066286863\\
1.91	0.0134999787927796\\
1.92	0.0137745748371512\\
1.93	0.0140538531647934\\
1.94	0.0143378724529564\\
1.95	0.0146266916434741\\
1.96	0.0149203699328093\\
1.97	0.0152189667618148\\
1.98	0.0155225418052091\\
1.99	0.0158311549607656\\
2	0.0161448663382137\\
2.01	0.0164637362478493\\
2.02	0.0167878251888549\\
2.03	0.0171171938373269\\
2.04	0.0174519030340098\\
2.05	0.0177920137717355\\
2.06	0.0181375871825679\\
2.07	0.0184886845246504\\
2.08	0.0188453671687578\\
2.09	0.0192076965845497\\
2.1	0.0195757343265274\\
2.11	0.0199495420196917\\
2.12	0.0203291813449033\\
2.13	0.020714714023945\\
2.14	0.0211062018042852\\
2.15	0.0215037064435448\\
2.16	0.0219072896936656\\
2.17	0.0223170132847825\\
2.18	0.0227329389087996\\
2.19	0.0231551282026703\\
2.2	0.0235836427313844\\
2.21	0.0240185439706604\\
2.22	0.0244598932893471\\
2.23	0.0249077519315344\\
2.24	0.0253621809983751\\
2.25	0.0258232414296198\\
2.26	0.0262909939848663\\
2.27	0.0267654992245267\\
2.28	0.0272468174905128\\
2.29	0.0277350088866436\\
2.3	0.0282301332587772\\
2.31	0.0287322501746691\\
2.32	0.029241418903561\\
2.33	0.0297576983955025\\
2.34	0.0302811472604087\\
2.35	0.0308118237468586\\
2.36	0.0313497857206356\\
2.37	0.0318950906430169\\
2.38	0.0324477955488128\\
2.39	0.0330079570241622\\
2.4	0.0335756311840884\\
2.41	0.0341508736498181\\
2.42	0.0347337395258714\\
2.43	0.0353242833769239\\
2.44	0.0359225592044501\\
2.45	0.0365286204231498\\
2.46	0.0371425198371657\\
2.47	0.0377643096160964\\
2.48	0.0383940412708106\\
2.49	0.0390317656290701\\
2.5	0.0396775328109659\\
2.51	0.040331392204175\\
2.52	0.0409933924390448\\
2.53	0.0416635813635106\\
2.54	0.0423420060178543\\
2.55	0.043028712609312\\
2.56	0.0437237464865357\\
2.57	0.0444271521139196\\
2.58	0.0451389730457957\\
2.59	0.0458592519005098\\
2.6	0.0465880303343831\\
2.61	0.0473253490155696\\
2.62	0.0480712475978175\\
2.63	0.048825764694142\\
2.64	0.0495889378504206\\
2.65	0.0503608035189175\\
2.66	0.0511413970317481\\
2.67	0.0519307525742921\\
2.68	0.0527289031585651\\
2.69	0.0535358805965574\\
2.7	0.0543517154735524\\
2.71	0.0551764371214307\\
2.72	0.0560100735919743\\
2.73	0.0568526516301781\\
2.74	0.0577041966475806\\
2.75	0.0585647326956243\\
2.76	0.0594342824390564\\
2.77	0.0603128671293805\\
2.78	0.0612005065783718\\
2.79	0.0620972191316639\\
2.8	0.0630030216424231\\
2.81	0.0639179294451165\\
2.82	0.0648419563293903\\
2.83	0.0657751145140666\\
2.84	0.0667174146212718\\
2.85	0.0676688656507095\\
2.86	0.0686294749540878\\
2.87	0.069599248209715\\
2.88	0.0705781893972751\\
2.89	0.0715663007727959\\
2.9	0.0725635828438214\\
2.91	0.0735700343448021\\
2.92	0.0745856522127145\\
2.93	0.0756104315629238\\
2.94	0.0766443656653014\\
2.95	0.0776874459206114\\
2.96	0.0787396618371763\\
2.97	0.0798010010078386\\
2.98	0.0808714490872275\\
2.99	0.0819509897693467\\
3	0.0830396047654938\\
3.01	0.0841372737825264\\
3.02	0.0852439745014865\\
3.03	0.0863596825565973\\
3.04	0.0874843715146442\\
3.05	0.088618012854755\\
3.06	0.0897605759485901\\
3.07	0.0909120280409574\\
3.08	0.0920723342308645\\
3.09	0.0932414574530208\\
3.1	0.0944193584598023\\
3.11	0.0956059958036935\\
3.12	0.0968013258202172\\
3.13	0.0980053026113663\\
3.14	0.0992178780295499\\
3.15	0.100439001662067\\
3.16	0.101668620816118\\
3.17	0.102906680504373\\
3.18	0.104153123431098\\
3.19	0.105407889978861\\
3.2	0.10667091819583\\
3.21	0.10794214378367\\
3.22	0.109221500086047\\
3.23	0.110508918077769\\
3.24	0.111804326354552\\
3.25	0.113107651123439\\
3.26	0.114418816193881\\
3.27	0.115737742969483\\
3.28	0.117064350440432\\
3.29	0.118398555176626\\
3.3	0.11974027132149\\
3.31	0.121089410586524\\
3.32	0.122445882246559\\
3.33	0.123809593135755\\
3.34	0.125180447644343\\
3.35	0.126558347716114\\
3.36	0.127943192846677\\
3.37	0.129334880082487\\
3.38	0.13073330402065\\
3.39	0.13213835680952\\
3.4	0.133549928150096\\
3.41	0.134967905298216\\
3.42	0.136392173067578\\
3.43	0.137822613833567\\
3.44	0.139259107537919\\
3.45	0.140701531694221\\
3.46	0.142149761394247\\
3.47	0.143603669315142\\
3.48	0.145063125727469\\
3.49	0.146527998504106\\
3.5	0.14799815313001\\
3.51	0.149473452712855\\
3.52	0.15095375799454\\
3.53	0.152438927363579\\
3.54	0.153928816868373\\
3.55	0.155423280231368\\
3.56	0.156922168864102\\
3.57	0.158425331883151\\
3.58	0.159932616126957\\
3.59	0.161443866173569\\
3.6	0.162958924359271\\
3.61	0.164477630798113\\
3.62	0.16599982340235\\
3.63	0.167525337903775\\
3.64	0.169054007875955\\
3.65	0.170585664757372\\
3.66	0.172120137875463\\
3.67	0.17365725447156\\
3.68	0.175196839726729\\
3.69	0.176738716788501\\
3.7	0.17828270679851\\
3.71	0.179828628921003\\
3.72	0.181376300372256\\
3.73	0.182925536450863\\
3.74	0.184476150568913\\
3.75	0.186027954284037\\
3.76	0.187580757332328\\
3.77	0.18913436766213\\
3.78	0.190688591468681\\
3.79	0.192243233229616\\
3.8	0.193798095741313\\
3.81	0.195352980156075\\
3.82	0.196907686020157\\
3.83	0.198462011312601\\
3.84	0.200015752484904\\
3.85	0.201568704501475\\
3.86	0.203120660880904\\
3.87	0.204671413738017\\
3.88	0.206220753826697\\
3.89	0.20776847058349\\
3.9	0.209314352171954\\
3.91	0.210858185527757\\
3.92	0.212399756404511\\
3.93	0.21393884942032\\
3.94	0.215475248105045\\
3.95	0.217008734948253\\
3.96	0.21853909144786\\
3.97	0.220066098159426\\
3.98	0.22158953474612\\
3.99	0.223109180029311\\
4	0.224624812039785\\
4.01	0.226136208069575\\
4.02	0.227643144724377\\
4.03	0.229145397976542\\
4.04	0.230642743218632\\
4.05	0.232134955317513\\
4.06	0.233621808668973\\
4.07	0.235103077252852\\
4.08	0.236578534688654\\
4.09	0.238047954291633\\
4.1	0.239511109129332\\
4.11	0.240967772078555\\
4.12	0.242417715882746\\
4.13	0.243860713209771\\
4.14	0.245296536710069\\
4.15	0.246724959075156\\
4.16	0.248145753096463\\
4.17	0.24955869172449\\
4.18	0.250963548128245\\
4.19	0.252360095754955\\
4.2	0.253748108390029\\
4.21	0.255127360217246\\
4.22	0.256497625879142\\
4.23	0.25785868053759\\
4.24	0.259210299934527\\
4.25	0.260552260452831\\
4.26	0.261884339177307\\
4.27	0.263206313955762\\
4.28	0.264517963460149\\
4.29	0.265819067247764\\
4.3	0.267109405822455\\
4.31	0.268388760695835\\
4.32	0.269656914448466\\
4.33	0.270913650790995\\
4.34	0.27215875462522\\
4.35	0.273392012105055\\
4.36	0.274613210697374\\
4.37	0.275822139242714\\
4.38	0.2770185880158\\
4.39	0.278202348785892\\
4.4	0.279373214876894\\
4.41	0.280530981227242\\
4.42	0.281675444449509\\
4.43	0.282806402889732\\
4.44	0.283923656686419\\
4.45	0.285027007829217\\
4.46	0.286116260217221\\
4.47	0.287191219716896\\
4.48	0.28825169421959\\
4.49	0.289297493698606\\
4.5	0.290328430265828\\
4.51	0.291344318227856\\
4.52	0.292344974141641\\
4.53	0.293330216869595\\
4.54	0.294299867634142\\
4.55	0.2952537500717\\
4.56	0.29619169028607\\
4.57	0.297113516901198\\
4.58	0.298019061113301\\
4.59	0.298908156742324\\
4.6	0.29978064028272\\
4.61	0.300636350953513\\
4.62	0.301475130747636\\
4.63	0.302296824480521\\
4.64	0.303101279837916\\
4.65	0.303888347422908\\
4.66	0.30465788080214\\
4.67	0.30540973655119\\
4.68	0.306143774299102\\
4.69	0.306859856772047\\
4.7	0.307557849836088\\
4.71	0.308237622539045\\
4.72	0.308899047151428\\
4.73	0.309541999206423\\
4.74	0.310166357538923\\
4.75	0.310772004323572\\
4.76	0.311358825111819\\
4.77	0.311926708867954\\
4.78	0.312475548004121\\
4.79	0.313005238414284\\
4.8	0.313515679507133\\
4.81	0.314006774237922\\
4.82	0.314478429139213\\
4.83	0.314930554350523\\
4.84	0.315363063646858\\
4.85	0.315775874466112\\
4.86	0.31616890793534\\
4.87	0.316542088895868\\
4.88	0.316895345927247\\
4.89	0.317228611370034\\
4.9	0.317541821347388\\
4.91	0.317834915785474\\
4.92	0.318107838432669\\
4.93	0.318360536877553\\
4.94	0.318592962565682\\
4.95	0.318805070815141\\
4.96	0.318996820830853\\
4.97	0.319168175717664\\
4.98	0.319319102492166\\
4.99	0.319449572093287\\
5	0.319559559391609\\
5.01	0.319649043197439\\
5.02	0.319718006267613\\
5.03	0.319766435311032\\
5.04	0.31979432099293\\
5.05	0.319801657937876\\
5.06	0.319788444731501\\
5.07	0.319754683920947\\
5.08	0.319700382014057\\
5.09	0.319625549477275\\
5.1	0.319530200732294\\
5.11	0.319414354151413\\
5.12	0.319278032051646\\
5.13	0.319121260687552\\
5.14	0.31894407024281\\
5.15	0.318746494820531\\
5.16	0.318528572432322\\
5.17	0.318290344986095\\
5.18	0.318031858272638\\
5.19	0.317753161950944\\
5.2	0.317454309532312\\
5.21	0.317135358363221\\
5.22	0.316796369606996\\
5.23	0.316437408224253\\
5.24	0.316058542952159\\
5.25	0.315659846282491\\
5.26	0.315241394438521\\
5.27	0.31480326735073\\
5.28	0.314345548631364\\
5.29	0.313868325547838\\
5.3	0.313371688995012\\
5.31	0.312855733466336\\
5.32	0.312320557023887\\
5.33	0.311766261267315\\
5.34	0.311192951301694\\
5.35	0.310600735704312\\
5.36	0.309989726490407\\
5.37	0.309360039077857\\
5.38	0.30871179225085\\
5.39	0.308045108122549\\
5.4	0.307360112096763\\
5.41	0.306656932828639\\
5.42	0.305935702184406\\
5.43	0.305196555200171\\
5.44	0.304439630039798\\
5.45	0.30366506795188\\
5.46	0.302873013225835\\
5.47	0.302063613147121\\
5.48	0.301237017951625\\
5.49	0.300393380779205\\
5.5	0.299532857626443\\
5.51	0.298655607298603\\
5.52	0.297761791360826\\
5.53	0.296851574088582\\
5.54	0.295925122417399\\
5.55	0.294982605891887\\
5.56	0.294024196614091\\
5.57	0.293050069191175\\
5.58	0.292060400682484\\
5.59	0.291055370545979\\
5.6	0.290035160584095\\
5.61	0.288999954889018\\
5.62	0.287949939787436\\
5.63	0.286885303784746\\
5.64	0.285806237508785\\
5.65	0.284712933653071\\
5.66	0.283605586919601\\
5.67	0.28248439396122\\
5.68	0.281349553323585\\
5.69	0.280201265386755\\
5.7	0.279039732306417\\
5.71	0.27786515795479\\
5.72	0.276677747861214\\
5.73	0.275477709152461\\
5.74	0.274265250492784\\
5.75	0.273040582023736\\
5.76	0.271803915303767\\
5.77	0.270555463247651\\
5.78	0.269295440065736\\
5.79	0.268024061203064\\
5.8	0.266741543278376\\
5.81	0.26544810402302\\
5.82	0.264143962219805\\
5.83	0.2628293376418\\
5.84	0.261504450991124\\
5.85	0.260169523837733\\
5.86	0.258824778558242\\
5.87	0.257470438274798\\
5.88	0.25610672679402\\
5.89	0.254733868546051\\
5.9	0.253352088523716\\
5.91	0.251961612221839\\
5.92	0.250562665576713\\
5.93	0.249155474905767\\
5.94	0.247740266847442\\
5.95	0.246317268301297\\
5.96	0.244886706368375\\
5.97	0.24344880829184\\
5.98	0.242003801397917\\
5.99	0.24055191303714\\
6	0.239093370525952\\
6.01	0.237628401088657\\
6.02	0.23615723179975\\
6.03	0.234680089526651\\
6.04	0.233197200872859\\
6.05	0.231708792121536\\
6.06	0.230215089179558\\
6.07	0.228716317522038\\
6.08	0.22721270213734\\
6.09	0.22570446747261\\
6.1	0.224191837379833\\
6.11	0.22267503506244\\
6.12	0.221154283022476\\
6.13	0.21962980300835\\
6.14	0.218101815963184\\
6.15	0.216570541973764\\
6.16	0.215036200220133\\
6.17	0.213499008925817\\
6.18	0.21195918530871\\
6.19	0.210416945532633\\
6.2	0.208872504659575\\
6.21	0.207326076602634\\
6.22	0.205777874079673\\
6.23	0.204228108567692\\
6.24	0.202676990257946\\
6.25	0.201124728011806\\
6.26	0.199571529317377\\
6.27	0.198017600246892\\
6.28	0.196463145414885\\
6.29	0.194908367937152\\
6.3	0.193353469390517\\
6.31	0.191798649773403\\
6.32	0.190244107467226\\
6.33	0.188690039198609\\
6.34	0.187136640002441\\
6.35	0.185584103185767\\
6.36	0.184032620292533\\
6.37	0.182482381069184\\
6.38	0.180933573431129\\
6.39	0.179386383430064\\
6.4	0.177840995222173\\
6.41	0.17629759103721\\
6.42	0.174756351148454\\
6.43	0.173217453843555\\
6.44	0.17168107539627\\
6.45	0.170147390039088\\
6.46	0.16861656993675\\
6.47	0.167088785160669\\
6.48	0.165564203664248\\
6.49	0.164042991259093\\
6.5	0.162525311592137\\
6.51	0.161011326123657\\
6.52	0.159501194106197\\
6.53	0.157995072564391\\
6.54	0.156493116275693\\
6.55	0.154995477751992\\
6.56	0.153502307222147\\
6.57	0.1520137526154\\
6.58	0.15052995954569\\
6.59	0.149051071296865\\
6.6	0.147577228808773\\
6.61	0.146108570664244\\
6.62	0.144645233076952\\
6.63	0.143187349880159\\
6.64	0.141735052516324\\
6.65	0.14028847002759\\
6.66	0.138847729047126\\
6.67	0.137412953791335\\
6.68	0.135984266052907\\
6.69	0.134561785194725\\
6.7	0.133145628144603\\
6.71	0.131735909390866\\
6.72	0.130332740978749\\
6.73	0.128936232507617\\
6.74	0.127546491129\\
6.75	0.126163621545422\\
6.76	0.124787726010036\\
6.77	0.123418904327037\\
6.78	0.122057253852859\\
6.79	0.120702869498137\\
6.8	0.119355843730431\\
6.81	0.1180162665777\\
6.82	0.116684225632517\\
6.83	0.115359806057013\\
6.84	0.114043090588543\\
6.85	0.112734159546067\\
6.86	0.111433090837222\\
6.87	0.11013995996609\\
6.88	0.108854840041642\\
6.89	0.10757780178685\\
6.9	0.106308913548457\\
6.91	0.105048241307385\\
6.92	0.103795848689784\\
6.93	0.102551796978697\\
6.94	0.10131614512634\\
6.95	0.100088949766973\\
6.96	0.0988702652303605\\
6.97	0.0976601435558059\\
6.98	0.0964586345067445\\
6.99	0.0952657855858868\\
7	0.0940816420508967\\
7.01	0.0929062469305942\\
7.02	0.091739641041668\\
7.03	0.0905818630058867\\
7.04	0.0894329492677946\\
7.05	0.088292934112881\\
7.06	0.087161849686207\\
7.07	0.0860397260114811\\
7.08	0.0849265910105658\\
7.09	0.0838224705234065\\
7.1	0.0827273883283668\\
7.11	0.081641366162959\\
7.12	0.0805644237449543\\
7.13	0.0794965787938631\\
7.14	0.0784378470527686\\
7.15	0.0773882423105042\\
7.16	0.0763477764241583\\
7.17	0.0753164593418964\\
7.18	0.0742942991260859\\
7.19	0.0732813019767105\\
7.2	0.0722774722550628\\
7.21	0.0712828125077007\\
7.22	0.0702973234906556\\
7.23	0.0693210041938795\\
7.24	0.0683538518659187\\
7.25	0.0673958620388005\\
7.26	0.0664470285531221\\
7.27	0.0655073435833272\\
7.28	0.0645767976631603\\
7.29	0.063655379711284\\
7.3	0.0627430770570496\\
7.31	0.0618398754664067\\
7.32	0.0609457591679419\\
7.33	0.0600607108790333\\
7.34	0.0591847118321099\\
7.35	0.0583177418010041\\
7.36	0.0574597791273863\\
7.37	0.0566108007472685\\
7.38	0.0557707822175694\\
7.39	0.0549396977427251\\
7.4	0.0541175202013394\\
7.41	0.0533042211728588\\
7.42	0.0524997709642644\\
7.43	0.0517041386367693\\
7.44	0.0509172920325106\\
7.45	0.0501391978012268\\
7.46	0.0493698214269105\\
7.47	0.0486091272544258\\
7.48	0.0478570785160813\\
7.49	0.04711363735815\\
7.5	0.0463787648673247\\
7.51	0.0456524210971024\\
7.52	0.0449345650940864\\
7.53	0.0442251549241985\\
7.54	0.0435241476987926\\
7.55	0.0428314996006617\\
7.56	0.0421471659099276\\
7.57	0.0414711010298096\\
7.58	0.0408032585122601\\
7.59	0.0401435910834611\\
7.6	0.0394920506691755\\
7.61	0.0388485884199429\\
7.62	0.0382131547361147\\
7.63	0.0375856992927223\\
7.64	0.0369661710641687\\
7.65	0.0363545183487397\\
7.66	0.0357506887929274\\
7.67	0.0351546294155594\\
7.68	0.0345662866317275\\
7.69	0.0339856062765119\\
7.7	0.0334125336284925\\
7.71	0.0328470134330438\\
7.72	0.0322889899254083\\
7.73	0.0317384068535412\\
7.74	0.0311952075007242\\
7.75	0.0306593347079419\\
7.76	0.0301307308960168\\
7.77	0.0296093380874989\\
7.78	0.029095097928305\\
7.79	0.0285879517091046\\
7.8	0.0280878403864484\\
7.81	0.0275947046036344\\
7.82	0.0271084847113114\\
7.83	0.0266291207878126\\
7.84	0.0261565526592195\\
7.85	0.0256907199191516\\
7.86	0.02523156194828\\
7.87	0.0247790179335609\\
7.88	0.0243330268871889\\
7.89	0.0238935276652653\\
7.9	0.0234604589861821\\
7.91	0.0230337594487173\\
7.92	0.022613367549841\\
7.93	0.0221992217022314\\
7.94	0.0217912602514977\\
7.95	0.0213894214931097\\
7.96	0.0209936436890328\\
7.97	0.0206038650840679\\
7.98	0.0202200239218939\\
7.99	0.0198420584608144\\
8	0.0194699069892059\\
8.01	0.019103507840669\\
8.02	0.0187427994088807\\
8.03	0.0183877201621492\\
8.04	0.0180382086576705\\
8.05	0.0176942035554866\\
8.06	0.0173556436321467\\
8.07	0.0170224677940708\\
8.08	0.0166946150906171\\
8.09	0.0163720247268531\\
8.1	0.0160546360760318\\
8.11	0.0157423886917736\\
8.12	0.0154352223199552\\
8.13	0.0151330769103064\\
8.14	0.0148358926277161\\
8.15	0.0145436098632491\\
8.16	0.0142561692448743\\
8.17	0.0139735116479077\\
8.18	0.0136955782051693\\
8.19	0.0134223103168583\\
8.2	0.0131536496601461\\
8.21	0.0128895381984906\\
8.22	0.0126299181906737\\
8.23	0.0123747321995629\\
8.24	0.0121239231006\\
8.25	0.0118774340900205\\
8.26	0.0116352086928026\\
8.27	0.0113971907703519\\
8.28	0.0111633245279221\\
8.29	0.0109335545217749\\
8.3	0.0107078256660808\\
8.31	0.0104860832395661\\
8.32	0.0102682728919047\\
8.33	0.0100543406498618\\
8.34	0.00984423292318888\\
8.35	0.00963789651027523\\
8.36	0.00943527860355735\\
8.37	0.00923632679469026\\
8.38	0.00904098907948317\\
8.39	0.00884921386260283\\
8.4	0.00866094996204767\\
8.41	0.00847614661339551\\
8.42	0.00829475347382863\\
8.43	0.00811672062593896\\
8.44	0.00794199858131661\\
8.45	0.00777053828392544\\
8.46	0.00760229111326832\\
8.47	0.00743720888734606\\
8.48	0.00727524386541275\\
8.49	0.0071163487505313\\
8.5	0.00696047669193228\\
8.51	0.00680758128717966\\
8.52	0.00665761658414656\\
8.53	0.00651053708280472\\
8.54	0.00636629773683088\\
8.55	0.00622485395503358\\
8.56	0.00608616160260388\\
8.57	0.00595017700219317\\
8.58	0.00581685693482179\\
8.59	0.00568615864062177\\
8.6	0.00555803981941703\\
8.61	0.00543245863114457\\
8.62	0.00530937369612007\\
8.63	0.00518874409515118\\
8.64	0.00507052936950197\\
8.65	0.00495468952071215\\
8.66	0.00484118501027384\\
8.67	0.00472997675916999\\
8.68	0.00462102614727719\\
8.69	0.00451429501263655\\
8.7	0.00440974565059588\\
8.71	0.00430734081282629\\
8.72	0.00420704370621683\\
8.73	0.00410881799165\\
8.74	0.00401262778266171\\
8.75	0.00391843764398876\\
8.76	0.00382621259000694\\
8.77	0.00373591808306305\\
8.78	0.00364752003170384\\
8.79	0.0035609847888051\\
8.8	0.00347627914960381\\
8.81	0.00339337034963658\\
8.82	0.00331222606258721\\
8.83	0.00323281439804643\\
8.84	0.00315510389918684\\
8.85	0.00307906354035585\\
8.86	0.00300466272458949\\
8.87	0.00293187128105012\\
8.88	0.0028606594623906\\
8.89	0.00279099794204787\\
8.9	0.00272285781146872\\
8.91	0.00265621057727016\\
8.92	0.00259102815833753\\
8.93	0.00252728288286249\\
8.94	0.00246494748532389\\
8.95	0.0024039951034138\\
8.96	0.00234439927491129\\
8.97	0.00228613393450666\\
8.98	0.00222917341057807\\
8.99	0.0021734924219235\\
9	0.00211906607445008\\
9.01	0.00206586985782318\\
9.02	0.00201387964207761\\
9.03	0.00196307167419305\\
9.04	0.00191342257463601\\
9.05	0.00186490933387047\\
9.06	0.00181750930883927\\
9.07	0.0017712002194183\\
9.08	0.00172596014484571\\
9.09	0.00168176752012786\\
9.1	0.00163860113242423\\
9.11	0.00159644011741304\\
9.12	0.00155526395563949\\
9.13	0.00151505246884854\\
9.14	0.00147578581630389\\
9.15	0.00143744449109497\\
9.16	0.00140000931643379\\
9.17	0.00136346144194302\\
9.18	0.00132778233993727\\
9.19	0.00129295380169897\\
9.2	0.00125895793375037\\
9.21	0.00122577715412341\\
9.22	0.00119339418862865\\
9.23	0.0011617920671249\\
9.24	0.00113095411979088\\
9.25	0.00110086397340032\\
9.26	0.00107150554760171\\
9.27	0.00104286305120415\\
9.28	0.00101492097847045\\
9.29	0.000987664105418677\\
9.3	0.000961077486133441\\
9.31	0.000935146449087961\\
9.32	0.000909856593478026\\
9.33	0.000885193785569004\\
9.34	0.000861144155056849\\
9.35	0.000837694091444159\\
9.36	0.000814830240432241\\
9.37	0.000792539500330117\\
9.38	0.000770809018481389\\
9.39	0.000749626187709831\\
9.4	0.000728978642784507\\
9.41	0.000708854256905303\\
9.42	0.000689241138209577\\
9.43	0.000670127626300705\\
9.44	0.000651502288799236\\
9.45	0.000633353917917341\\
9.46	0.000615671527057199\\
9.47	0.000598444347433983\\
9.48	0.000581661824723986\\
9.49	0.000565313615738524\\
9.5	0.000549389585124119\\
9.51	0.000533879802089483\\
9.52	0.000518774537159811\\
9.53	0.000504064258958826\\
9.54	0.000489739631019037\\
9.55	0.000475791508620601\\
9.56	0.000462210935659209\\
9.57	0.000448989141543317\\
9.58	0.000436117538121123\\
9.59	0.000423587716637555\\
9.6	0.000411391444721603\\
9.61	0.000399520663404253\\
9.62	0.000387967484167287\\
9.63	0.000376724186023164\\
9.64	0.000365783212626227\\
9.65	0.00035513716941539\\
9.66	0.000344778820788498\\
9.67	0.00033470108730853\\
9.68	0.000324897042941733\\
9.69	0.000315359912327858\\
9.7	0.000306083068082558\\
9.71	0.000297060028132048\\
9.72	0.000288284453080093\\
9.73	0.00027975014360736\\
9.74	0.000271451037903185\\
9.75	0.000263381209129764\\
9.76	0.000255534862918768\\
9.77	0.000247906334900376\\
9.78	0.000240490088264704\\
9.79	0.000233280711355582\\
9.8	0.000226272915296628\\
9.81	0.000219461531649576\\
9.82	0.000212841510104734\\
9.83	0.000206407916203538\\
9.84	0.000200155929093061\\
9.85	0.000194080839312381\\
9.86	0.000188178046610692\\
9.87	0.000182443057797005\\
9.88	0.000176871484621314\\
9.89	0.000171459041687076\\
9.9	0.000166201544394823\\
9.91	0.000161094906916755\\
9.92	0.000156135140202134\\
9.93	0.000151318350013282\\
9.94	0.000146640734991985\\
9.95	0.000142098584756127\\
9.96	0.000137688278026307\\
9.97	0.000133406280782254\\
9.98	0.000129249144448798\\
9.99	0.000125213504111178\\
10	0.000121296076759446\\
};
%\addlegendentry{sigma points}

\addplot [color=mycolor2, line width=2.0pt, forget plot]
  table[row sep=crcr]{%
5.04857024213078	0\\
5.04857024213078	0.6\\
};
\addplot [color=mycolor1, dashed, line width=2.0pt, forget plot]
  table[row sep=crcr]{%
5.0284309151552	0\\
5.0284309151552	0.6\\
};
\end{axis}
\end{tikzpicture}%
% -------------------------------------------

% -------------------------------------------
\section{Qualitative Visualization results}
\label{secS7}
% -------------------------------------------
{\color{red}{\emph{This supplementary is for Section~4.2 and~4.3 of the main paper.}}} In this section, we show qualitative results on both instance segmentation and semantic segmentation. To demonstrate the superiority of our method, we present visualization results of ablation studies on instance segmentation, as well as comparisons with state-of-the-art methods on both instance segmentation and semantic segmentation. 
% 
The obtained visualization results are shown in Figure~\ref{figs2}. From the results, it can be observed that compared to other methods, our method can achieve more accurate object masks that better fit the actual boundaries of the objects themselves.
% -------------------------------------------
% \begin{figure}[t]
%     \centering
%     \includegraphics[width=\linewidth]{figure/images/concept_affordance_prompt_v2.pdf}
%     \caption{Results of two different levels of prompts.}
%     \label{fig:prompt}
% \end{figure}
% -------------------------------------------

as well as the pseudo-code for when the stripe size is set to $2$ in Section~\ref{secS8}. 
% -------------------------------------------
\section{Pseudo-code fo stripe size = $2$}
\label{secS8}
% -------------------------------------------
In this code snippet, stripe size is set to 2, and relevant features are directly obtained using the gather function instead of reshaping them with img2windows. This operation can reduce unnecessary reshaping operations and improves the efficiency of the code.
% -------------------------------------------
\begin{python}
function cross_shaped_window_attention(x, num_heads, window_size):
    # x: given feature
    # num_heads: head number
    # window_size: window size

    # Get dimensions
    (batch_size, seq_length, d_model) = shape(x)

    # Split into multiple heads
    Q, K, V = split_heads(x, num_heads)

    # Initialize attention output
    attention_output = zeros(batch_size, seq_length, d_model)

    # Initialize previous head's output for cascaded attention
    previous_Q = zeros(batch_size, seq_length, d_model)
    previous_K = zeros(batch_size, seq_length, d_model)
    previous_V = zeros(batch_size, seq_length, d_model)

    # Calculate attention for each head
    for head in range(num_heads):
        for position in range(seq_length):
            # Get cross-shaped window indices
            window_indices = get_cross_shaped_window_indices(position, window_size)

            # Gather Q, K, V for the current window
            Q_window = gather(Q[head], window_indices)
            K_window = gather(K[head], window_indices)
            V_window = gather(V[head], window_indices)

            # Incorporate previous head's output for cascaded attention
            if head > 0:
                Q_window += previous_Q
                K_window += previous_K
                V_window += previous_V

            # Calculate attention scores
            attention_scores = softmax(Q_window * K_window^T / sqrt(d_k))

            # Compute the attention output for the current position
            attention_output[position] = attention_scores * V_window

        # Update previous head's output for the next head
        previous_Q = Q_window
        previous_K = K_window
        previous_V = V_window

    # Final linear transformation
    attention_output = linear_transform(attention_output)
    return attention_output

function feed_forward_network(x):
    # Feed Forward Network
    x = ReLU(linear(x))
    x = linear(x)
    return x
\end{python}

\begin{python}
def get_cross_shaped_window_indices(position, window_size, seq_length):
    # Initialize the list of indices
    indices = []

    # Add the current position
    indices.append(position)

    # Add vertical neighbors (up and down)
    for offset in range(-window_size, window_size + 1):
        if position + offset >= 0 and position + offset < seq_length:
            indices.append(position + offset)

    # Remove duplicates and sort the indices
    indices = list(set(indices))
    indices.sort()

    return indices
\end{python}
% ----------------------------------------------


\end{document}


