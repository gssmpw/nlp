\section{Literature Review and Research Gaps}
The power curtailment of RES, driven by the coupling between heat and electricity demand, has garnered attention from researchers ____. Various solutions have been proposed from multiple perspectives, including mechanical improvements to CHP generators ____, enhanced dispatch models ____, fuel cell integration ____, and heat demand response strategies ____. Among these, the configuration of additional heat sources emerges as a critical factor.

In ____, two types of thermal energy storage (TES) systems are proposed to address multi-scale uncertainties in cold regions, each designed to operate on different time scales.
%____ propose an flexibility improvement technology of coal-fire power plant that combines steam extraction and TES. 
%____ propose a hybrid energy system of “solar air collector + air source heat pump + energy storage” for cold region. 
____ propose a hybrid energy system combining a solar air collector, an air-source heat pump, and energy storage for use in cold regions. In ____, an energy storage system incorporating both TES and battery energy storage is installed in buildings, enabling participation in demand response programs.
Current research on heat source planning encompasses a range of equipment on both the generation and consumption sides. To maximize societal welfare, a collaborative capacity planning model is required, which simultaneously considers the planning of these equipment.
\nomenclature[AT]{TES}{Thermal energy storage}

In power system planning, particularly concerning the planning of heat sources, achieving a balance between economic viability and sustainability often requires trade-offs, as accomplishing the two objectives simultaneously is highly challenging. Traditional models typically utilize a weighted sum approach to integrate economic and sustainability objectives—such as investment ____, RES power curtailment ____, and carbon taxation ____—into a single objective function. While this method is straightforward, it may overlook complex trade-offs and introduce subjectivity, as planners must determine the weights based on personal judgment.

Conversely, multi-objective optimization (MOO) tackles the complexity of these factors by identifying Pareto-optimal solutions, which offer a range of options instead of a single compromise. MOO empowers planners to make more informed decisions by integrating insights from the Pareto front with their expertise ____.
\nomenclature[AM]{MOO}{Multi-objective optimization}

MOO algorithms used in capacity planning problems are predominantly categorized into two types. The first type employs adaptive weights for objectives, exemplified by the adaptive weighted-sum algorithm ____ and its enhanced variant ____. These dynamic weights eliminate the subjectivity of planners but often result in an uneven distribution of Pareto front points and necessitate manual adjustment of certain parameters. The second type is derived from heuristic search methods, such as NSGA-II ____ and MOAVOA ____.
\textcolor{revise_blue}{
        Both of these types can be integrated with Bayesian optimization to form MOBO algorithms. For instance, in ____, a modified version of PBO  ____ is proposed to address high-dimensional optimization problems in transportation systems. Similarly, in  ____, basic Bayesian optimization is combined with varying weights to identify optimal charging protocols.
        Despite these advancements, the application of MOBO within the power system domain remains largely underexplored. Moreover, the utilization of Bayesian optimization in power systems has primarily focused on hyperparameter tuning for machine learning algorithms ____ rather than directly optimizing power system models.
    }

While the heuristic algorithms are effective in handling nonlinear problems, they face challenges in determining the appropriate population size ____, computational efficiency, and dealing with the overlapping points in the Pareto fronts ____. Importantly, neither category directly addresses the distribution of Pareto fronts, which is crucial for facilitating a more informed decision-making process.

In addition to optimization methods, heat source planning encounters several challenges. Capacity planning models are crucial for preparing power systems to meet future demands more effectively. However, the actual scenarios these systems will encounter are unpredictable, leading to deviations between computer simulations and real-world performance. This issue is particularly pronounced in systems with high penetration of RESs, whose output power is highly variable. Traditional methods often rely on \textit{typical scenarios} ____ or \textit{extended dispatch cycles} ____ to mitigate this variability, but these approaches tend to compromise either objectivity or computational efficiency. Furthermore, few studies address the operational scenarios specific to cold regions, where heat load, electric load, and RES output are deeply interdependent. This underscores the necessity for a specially designed time series scenario generation method to effectively tackle these unique challenges.

%To address these limitations, this study proposes a noise-modeling approach that integrates the inevitable deviations between simulation and real-world environments into the planning model as a noise function, providing a more robust response to the unpredictable and variable nature of RES scenarios.