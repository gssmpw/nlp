\section{Related Work}
\label{sec:rw}


\subsection{Point-based Semantic Segmentation}

PointNet~\cite{qi2017pointnet} pioneers the 3D semantic segmentation, which directly works on irregular point clouds. This network processes individual points with shared MLPs to aggregate global features. However, its performances are limited because of the lack of considering local spatial relations in the point cloud structure. Following PointNet, PointNet++~\cite{qi2017pointnet++} develops a hierarchical spatial structure on local regions with MLPs, termed the set abstraction block. In MLPs-based philosophy, follow-up methods develop novel modules~\cite{zhang2024geoauxnet,liu2020closer}. PointNeXt~\cite{qian2022pointnext} revisits training and scaling strategies, tweaking the set abstraction block. The recently proposed method, PointMetaBase~\cite{lin2023meta}, designs building blocks into four meta functions for point cloud analysis. Compared with convolutional kernels~\cite{thomas2019kpconv},~\cite{xu2021paconv},~\cite{liu2020semantic}, graph structures~\cite{landrieu2018large},~\cite{qian2021pu},~\cite{tao2022seggroup}, and transformer architectures~\cite{zhao2021point},~\cite{park2022fast}, the highly-optimized MLPs are conceptually simpler to reduce computational and memory costs and achieve results on par or better.


\subsection{Contrastive Learning}

Contrastive learning is widely used to pull together feature embeddings from the same class and push away the feature embeddings from different classes~\cite{gutmann2010noise},~\cite{oord2018representation},~\cite{khosla2020supervised},~\cite{ong2023quad}. Works that follow this path design various contrastive objectives on 3D tasks in unsupervised approach~\cite{xie2020pointcontrast}, weakly-supervised approach~\cite{li2022hybridcr}, semi-supervised approach~\cite{jiang2021guided} and supervised approach~\cite{tang2022contrastive}. However, they only conduct fixed contrast on feature embeddings while ignoring adaptive ambiguities from position embeddings. 



\subsection{Margin-based Training Objective}

The typical networks use the cross-entropy objective during training. 2D tasks have witnessed a surge regarding decision margins to adjust the objective and strengthen the discriminating power~\cite{deng2019arcface},~\cite{wang2018cosface}. Recent works propose dynamic margins that are proven effective~\cite{li2022towards},~\cite{li2020boosting}, yet they are mostly constrained on positive margins to heighten objectives. This direction is essentially an under-explored aspect of 3D tasks. Meanwhile, considering the intrinsic properties of point clouds, one-sided margins are restrictive. Our method deviates from one-sided margins by exploring adaptive margins involving a diversity of positive, zero, and negative values to generate adaptive objectives.