%File: anonymous-submission-latex-2025.tex
\documentclass[letterpaper]{article} % DO NOT CHANGE THIS
\usepackage{aaai25}  % DO NOT CHANGE THIS
\usepackage{times}  % DO NOT CHANGE THIS
\usepackage{helvet}  % DO NOT CHANGE THIS
\usepackage{courier}  % DO NOT CHANGE THIS
\usepackage[hyphens]{url}  % DO NOT CHANGE THIS
\usepackage{graphicx} % DO NOT CHANGE THIS
\urlstyle{rm} % DO NOT CHANGE THIS
\def\UrlFont{\rm}  % DO NOT CHANGE THIS
\usepackage{natbib}  % DO NOT CHANGE THIS AND DO NOT ADD ANY OPTIONS TO IT
\usepackage{caption} % DO NOT CHANGE THIS AND DO NOT ADD ANY OPTIONS TO IT
\frenchspacing  % DO NOT CHANGE THIS
\setlength{\pdfpagewidth}{8.5in} % DO NOT CHANGE THIS
\setlength{\pdfpageheight}{11in} % DO NOT CHANGE THIS
%
% These are recommended to typeset algorithms but not required. See the subsubsection on algorithms. Remove them if you don't have algorithms in your paper.
% \usepackage{algorithm}
% \usepackage{algorithmic}
\usepackage[ruled,noline,nofillcomment]{algorithm2e}

%%%%%%%%%%%%%%%%%%%%%%%%%%%%%%%%%%%%%%%%%
\usepackage{float}
\usepackage{subcaption}

% 2.表格内多行
\usepackage{multirow}

% 3.表格内多列
\usepackage{multicol}

% 4.表格内颜色
\usepackage{color, colortbl}
% \definecolor{Gray}{gray}{0.9}
\usepackage{xcolor}

\usepackage{pifont} % support more "x" style
\usepackage{bbding} 
\newcommand{\cmark}{\ding{51}}%
\newcommand{\xmark}{\ding{55}}%

%% 自定义命令
\newcommand{\yj}[1]{\textcolor{red}{#1}}        % \yj{x}   批注 1
\newcommand{\red}[1]{\textcolor{red}{#1}}       % \red{x}  会将 x 调整成红色
\newcommand{\blue}[1]{\textcolor{blue}{#1}}     % \blue{x} 会将 x 调整成蓝色
\newcommand{\teal}[1]{\textcolor{teal}{#1}} 
\newcommand{\gray}[1]{\textcolor{gray}{#1}} 
\newcommand{\cyan}[1]{\textcolor{cyan}{#1}} 
\newcommand{\ua}[1]{#1$\uparrow$}               % \u{x} 会在 x 右边加上箭头
\newcommand{\da}[1]{#1$\downarrow$}             % \u{x} 会在 x 右边加下箭头

%% 邮箱
% \newcommand{\emaila}[1]{\texttt{\{#1\}@shanghaitech.edu.cn}}
% \newcommand{\emailb}[1]{\texttt{\{#2\}@cs.hku.hk}}

\usepackage{ragged2e}
\usepackage{blindtext}
% \usepackage{stfloats}
\usepackage{amsmath}
\usepackage{mathtools}
\usepackage{booktabs}
\usepackage{amssymb}
\usepackage{makecell}
% \usepackage{ulem}
%%%%%%%%%%%%%%%%%%%%%%%%%%%%%%%%%%%%%%%%%%

%
% These are recommended to typeset listings but not required. See the subsubsection on listing. Remove this block if you don't have listings in your paper.
\usepackage{newfloat}
\usepackage{listings}
\DeclareCaptionStyle{ruled}{labelfont=normalfont,labelsep=colon,strut=off} % DO NOT CHANGE THIS
\lstset{%
	basicstyle={\footnotesize\ttfamily},% footnotesize acceptable for monospace
	numbers=left,numberstyle=\footnotesize,xleftmargin=2em,% show line numbers, remove this entire line if you don't want the numbers.
	aboveskip=0pt,belowskip=0pt,%
	showstringspaces=false,tabsize=2,breaklines=true}
\floatstyle{ruled}
\newfloat{listing}{tb}{lst}{}
\floatname{listing}{Listing}
%
% Keep the \pdfinfo as shown here. There's no need
% for you to add the /Title and /Author tags.
\pdfinfo{
/TemplateVersion (2025.1)
}

% DISALLOWED PACKAGES
% \usepackage{authblk} -- This package is specifically forbidden
% \usepackage{balance} -- This package is specifically forbidden
% \usepackage{color (if used in text)
% \usepackage{CJK} -- This package is specifically forbidden
% \usepackage{float} -- This package is specifically forbidden
% \usepackage{flushend} -- This package is specifically forbidden
% \usepackage{fontenc} -- This package is specifically forbidden
% \usepackage{fullpage} -- This package is specifically forbidden
% \usepackage{geometry} -- This package is specifically forbidden
% \usepackage{grffile} -- This package is specifically forbidden
% \usepackage{hyperref} -- This package is specifically forbidden
% \usepackage{navigator} -- This package is specifically forbidden
% (or any other package that embeds links such as navigator or hyperref)
% \indentfirst} -- This package is specifically forbidden
% \layout} -- This package is specifically forbidden
% \multicol} -- This package is specifically forbidden
% \nameref} -- This package is specifically forbidden
% \usepackage{savetrees} -- This package is specifically forbidden
% \usepackage{setspace} -- This package is specifically forbidden
% \usepackage{stfloats} -- This package is specifically forbidden
% \usepackage{tabu} -- This package is specifically forbidden
% \usepackage{titlesec} -- This package is specifically forbidden
% \usepackage{tocbibind} -- This package is specifically forbidden
% \usepackage{ulem} -- This package is specifically forbidden
% \usepackage{wrapfig} -- This package is specifically forbidden
% DISALLOWED COMMANDS
% \nocopyright -- Your paper will not be published if you use this command
% \addtolength -- This command may not be used
% \balance -- This command may not be used
% \baselinestretch -- Your paper will not be published if you use this command
% \clearpage -- No page breaks of any kind may be used for the final version of your paper
% \columnsep -- This command may not be used
% \newpage -- No page breaks of any kind may be used for the final version of your paper
% \pagebreak -- No page breaks of any kind may be used for the final version of your paperr
% \pagestyle -- This command may not be used
% \tiny -- This is not an acceptable font size.
% \vspace{- -- No negative value may be used in proximity of a caption, figure, table, section, subsection, subsubsection, or reference
% \vskip{- -- No negative value may be used to alter spacing above or below a caption, figure, table, section, subsection, subsubsection, or reference

\setcounter{secnumdepth}{0} %May be changed to 1 or 2 if section numbers are desired.

% The file aaai25.sty is the style file for AAAI Press
% proceedings, working notes, and technical reports.
%

% Title

% Your title must be in mixed case, not sentence case.
% That means all verbs (including short verbs like be, is, using,and go),
% nouns, adverbs, adjectives should be capitalized, including both words in hyphenated terms, while
% articles, conjunctions, and prepositions are lower case unless they
% directly follow a colon or long dash
\title{UniDemoiré: Towards Universal Image Demoiréing with Data Generation and Synthesis}
\author{
    %Authors
    % All authors must be in the same font size and format.
    % Written by AAAI Press Staff\textsuperscript{\rm 1}\\
    % AAAI Style Contributions by Pater Patel Schneider,
    % Sunil Issar,\\
    Zemin Yang\textsuperscript{\rm 1,}\equalcontrib,
    Yujing Sun\textsuperscript{\rm 2,}\equalcontrib,
    Xidong Peng\textsuperscript{\rm 1},
    Siu Ming Yiu\textsuperscript{\rm 2},
    Yuexin Ma\textsuperscript{\rm 1,}\thanks{Corresponding author.}
}
\affiliations{
    %Afiliations
    \textsuperscript{\rm 1}ShanghaiTech University \\
    \textsuperscript{\rm 2}The University of Hong Kong\\
    % If you have multiple authors and multiple affiliations
    % use superscripts in text and roman font to identify them.
    % For example,

    % Sunil Issar\textsuperscript{\rm 2},
    % J. Scott Penberthy\textsuperscript{\rm 3},
    % George Ferguson\textsuperscript{\rm 4},
    % Hans Guesgen\textsuperscript{\rm 5}
    % Note that the comma should be placed after the superscript

    % 1101 Pennsylvania Ave, NW Suite 300\\
    % Washington, DC 20004 USA\\
    % email address must be in roman text type, not monospace or sans serif
    % proceedings-questions@aaai.org, 
    \{csyangzm, mayuexin\}@shanghaitech.edu.cn, \{yjsun, smyiu\}@cs.hku.uk
    % \emaila{csyangzm, mayuexin}, \emailb{yjsun, smyiu}
%
% See more examples next
}

%Example, Single Author, ->> remove \iffalse,\fi and place them surrounding AAAI title to use it
\iffalse
\title{My Publication Title --- Single Author}
\author {
    Author Name
}
\affiliations{
    Affiliation\\
    Affiliation Line 2\\
    name@example.com
}
\fi

\iffalse
%Example, Multiple Authors, ->> remove \iffalse,\fi and place them surrounding AAAI title to use it
\title{My Publication Title --- Multiple Authors}
\author {
    % Authors
    First Author Name\textsuperscript{\rm 1},
    Second Author Name\textsuperscript{\rm 2},
    Third Author Name\textsuperscript{\rm 1}
}
\affiliations {
    % Affiliations
    \textsuperscript{\rm 1}Affiliation 1\\
    \textsuperscript{\rm 2}Affiliation 2\\
    firstAuthor@affiliation1.com, secondAuthor@affilation2.com, thirdAuthor@affiliation1.com
}
\fi


% REMOVE THIS: bibentry
% This is only needed to show inline citations in the guidelines document. You should not need it and can safely delete it.
% \usepackage{bibentry}
% END REMOVE bibentry

\begin{document}

\maketitle

% Large language model (LLM)-based agents have shown promise in tackling complex tasks by interacting dynamically with the environment. 
Existing work primarily focuses on behavior cloning from expert demonstrations and preference learning through exploratory trajectory sampling. However, these methods often struggle in long-horizon tasks, where suboptimal actions accumulate step by step, causing agents to deviate from correct task trajectories.
To address this, we highlight the importance of \textit{timely calibration} and the need to automatically construct calibration trajectories for training agents. We propose \textbf{S}tep-Level \textbf{T}raj\textbf{e}ctory \textbf{Ca}libration (\textbf{\model}), a novel framework for LLM agent learning. 
Specifically, \model identifies suboptimal actions through a step-level reward comparison during exploration. It constructs calibrated trajectories using LLM-driven reflection, enabling agents to learn from improved decision-making processes. These calibrated trajectories, together with successful trajectory data, are utilized for reinforced training.
Extensive experiments demonstrate that \model significantly outperforms existing methods. Further analysis highlights that step-level calibration enables agents to complete tasks with greater robustness. 
Our code and data are available at \url{https://github.com/WangHanLinHenry/STeCa}.
%% Abstract
\begin{abstract}
Image demoiréing poses one of the most formidable challenges in image restoration, primarily due to the unpredictable and anisotropic nature of moiré patterns. Limited by the quantity and diversity of training data, current methods tend to overfit to a single moiré domain, resulting in performance degradation for new domains and restricting their robustness in real-world applications. In this paper, we propose a universal image demoiréing solution, \textbf{UniDemoiré}, which has superior generalization capability. Notably, we propose innovative and effective data generation and synthesis methods that can automatically provide vast high-quality moiré images to train a universal demoiréing model. Our extensive experiments demonstrate the cutting-edge performance and broad potential of our approach for generalized image demoiréing. 
% \textbf{Upon publication of this paper, we will release both our code and dataset.}
% Code and dataset are available at https://github.com/4DVLab/UniDemoire.
% Code: https://github.com/4DVLab/UniDemoire.
% Project page: https://github.com/4DVLab/UniDemoire.
% Project page: https://yizhifengyeyzm.github.io/UniDemoire-page.
\end{abstract}


\begin{links} 
    \link{Code}{https://github.com/4DVLab/UniDemoire} 
\end{links}

%% Introduction
% \section{Introduction}

Despite the remarkable capabilities of large language models (LLMs)~\cite{DBLP:conf/emnlp/QinZ0CYY23,DBLP:journals/corr/abs-2307-09288}, they often inevitably exhibit hallucinations due to incorrect or outdated knowledge embedded in their parameters~\cite{DBLP:journals/corr/abs-2309-01219, DBLP:journals/corr/abs-2302-12813, DBLP:journals/csur/JiLFYSXIBMF23}.
Given the significant time and expense required to retrain LLMs, there has been growing interest in \emph{model editing} (a.k.a., \emph{knowledge editing})~\cite{DBLP:conf/iclr/SinitsinPPPB20, DBLP:journals/corr/abs-2012-00363, DBLP:conf/acl/DaiDHSCW22, DBLP:conf/icml/MitchellLBMF22, DBLP:conf/nips/MengBAB22, DBLP:conf/iclr/MengSABB23, DBLP:conf/emnlp/YaoWT0LDC023, DBLP:conf/emnlp/ZhongWMPC23, DBLP:conf/icml/MaL0G24, DBLP:journals/corr/abs-2401-04700}, 
which aims to update the knowledge of LLMs cost-effectively.
Some existing methods of model editing achieve this by modifying model parameters, which can be generally divided into two categories~\cite{DBLP:journals/corr/abs-2308-07269, DBLP:conf/emnlp/YaoWT0LDC023}.
Specifically, one type is based on \emph{Meta-Learning}~\cite{DBLP:conf/emnlp/CaoAT21, DBLP:conf/acl/DaiDHSCW22}, while the other is based on \emph{Locate-then-Edit}~\cite{DBLP:conf/acl/DaiDHSCW22, DBLP:conf/nips/MengBAB22, DBLP:conf/iclr/MengSABB23}. This paper primarily focuses on the latter.

\begin{figure}[t]
  \centering
  \includegraphics[width=0.48\textwidth]{figures/demonstration.pdf}
  \vspace{-4mm}
  \caption{(a) Comparison of regular model editing and EAC. EAC compresses the editing information into the dimensions where the editing anchors are located. Here, we utilize the gradients generated during training and the magnitude of the updated knowledge vector to identify anchors. (b) Comparison of general downstream task performance before editing, after regular editing, and after constrained editing by EAC.}
  \vspace{-3mm}
  \label{demo}
\end{figure}

\emph{Sequential} model editing~\cite{DBLP:conf/emnlp/YaoWT0LDC023} can expedite the continual learning of LLMs where a series of consecutive edits are conducted.
This is very important in real-world scenarios because new knowledge continually appears, requiring the model to retain previous knowledge while conducting new edits. 
Some studies have experimentally revealed that in sequential editing, existing methods lead to a decrease in the general abilities of the model across downstream tasks~\cite{DBLP:journals/corr/abs-2401-04700, DBLP:conf/acl/GuptaRA24, DBLP:conf/acl/Yang0MLYC24, DBLP:conf/acl/HuC00024}. 
Besides, \citet{ma2024perturbation} have performed a theoretical analysis to elucidate the bottleneck of the general abilities during sequential editing.
However, previous work has not introduced an effective method that maintains editing performance while preserving general abilities in sequential editing.
This impacts model scalability and presents major challenges for continuous learning in LLMs.

In this paper, a statistical analysis is first conducted to help understand how the model is affected during sequential editing using two popular editing methods, including ROME~\cite{DBLP:conf/nips/MengBAB22} and MEMIT~\cite{DBLP:conf/iclr/MengSABB23}.
Matrix norms, particularly the L1 norm, have been shown to be effective indicators of matrix properties such as sparsity, stability, and conditioning, as evidenced by several theoretical works~\cite{kahan2013tutorial}. In our analysis of matrix norms, we observe significant deviations in the parameter matrix after sequential editing.
Besides, the semantic differences between the facts before and after editing are also visualized, and we find that the differences become larger as the deviation of the parameter matrix after editing increases.
Therefore, we assume that each edit during sequential editing not only updates the editing fact as expected but also unintentionally introduces non-trivial noise that can cause the edited model to deviate from its original semantics space.
Furthermore, the accumulation of non-trivial noise can amplify the negative impact on the general abilities of LLMs.

Inspired by these findings, a framework termed \textbf{E}diting \textbf{A}nchor \textbf{C}ompression (EAC) is proposed to constrain the deviation of the parameter matrix during sequential editing by reducing the norm of the update matrix at each step. 
As shown in Figure~\ref{demo}, EAC first selects a subset of dimension with a high product of gradient and magnitude values, namely editing anchors, that are considered crucial for encoding the new relation through a weighted gradient saliency map.
Retraining is then performed on the dimensions where these important editing anchors are located, effectively compressing the editing information.
By compressing information only in certain dimensions and leaving other dimensions unmodified, the deviation of the parameter matrix after editing is constrained. 
To further regulate changes in the L1 norm of the edited matrix to constrain the deviation, we incorporate a scored elastic net ~\cite{zou2005regularization} into the retraining process, optimizing the previously selected editing anchors.

To validate the effectiveness of the proposed EAC, experiments of applying EAC to \textbf{two popular editing methods} including ROME and MEMIT are conducted.
In addition, \textbf{three LLMs of varying sizes} including GPT2-XL~\cite{radford2019language}, LLaMA-3 (8B)~\cite{llama3} and LLaMA-2 (13B)~\cite{DBLP:journals/corr/abs-2307-09288} and \textbf{four representative tasks} including 
natural language inference~\cite{DBLP:conf/mlcw/DaganGM05}, 
summarization~\cite{gliwa-etal-2019-samsum},
open-domain question-answering~\cite{DBLP:journals/tacl/KwiatkowskiPRCP19},  
and sentiment analysis~\cite{DBLP:conf/emnlp/SocherPWCMNP13} are selected to extensively demonstrate the impact of model editing on the general abilities of LLMs. 
Experimental results demonstrate that in sequential editing, EAC can effectively preserve over 70\% of the general abilities of the model across downstream tasks and better retain the edited knowledge.

In summary, our contributions to this paper are three-fold:
(1) This paper statistically elucidates how deviations in the parameter matrix after editing are responsible for the decreased general abilities of the model across downstream tasks after sequential editing.
(2) A framework termed EAC is proposed, which ultimately aims to constrain the deviation of the parameter matrix after editing by compressing the editing information into editing anchors. 
(3) It is discovered that on models like GPT2-XL and LLaMA-3 (8B), EAC significantly preserves over 70\% of the general abilities across downstream tasks and retains the edited knowledge better.
\section{Introduction}
\label{sec:intro}
Digital screens have become essential devices for displaying information in our daily work and life.
However, images captured from screens frequently suffer from frustrating moiré patterns, significantly degrading image quality and hindering content extraction. 
Therefore, it becomes crucial to effectively remove such moiré artifacts to help users obtain high-quality images from their digital imaging devices and to support industries in maintaining high-standard product visual presentation and digital archiving.
However, moiré patterns are characterized as anisotropic and multi-scale, as well as involving considerable shape variations and color distortions~\cite{amidror2009theory}. Such traits are seldom seen in other types of artifacts, like noise, rain streaks, fog, blurring, etc., posing a significant challenge for even the most advanced image restoration methods~\cite{luo2023refusion,zhu2023denoising,fei2023generative}. 

Hence, many methods have been proposed to tackle the problem of demoiréing in recent years~\cite{sun2018moire,liu2020wavelet,luo2020deep,he2019mop,he2020fhde,wang2023coarse,yue2022recaptured,yu2022towards}. Nevertheless, the effectiveness of such supervised methods heavily depends on the volume of training data, consisting of pairs of moiré images and their clean counterparts. As we know, collecting such data is a daunting task and it requires precise calibration between natural images and moiré patterns. The limitations of the data lead to the limitations of the methods, resulting in poor generalization of the network model, which performs poorly on the data containing new moiré patterns or new natural images. In order to expand the quantity and diversity of the training data in a convenient way, some methods have started to explore the synthesis of moiré patterns. LCDMoiré~\cite{yuan2019aim} deigns handcraft mathematical models. However, it could not represent complex features of moiré patterns and leads to a substantial discrepancy between the synthetic data and actual moiré images. To enhance realism, recent studies~\cite{cyclic, undem} extract moiré patterns from existing real images and combine them with clean images for data synthesis. Nevertheless, these methods do not escape from the moiré domains of the existing training data, bringing limited performance improvement on new moiré domains. To develop a universal model for image demoiréing with greater generalization capability and practicality, two critical challenges emerge: \textit{how to generate a vast amount of diverse data, and how to ensure the authenticity of the data?}

\begin{figure*}[ht]
  \centering   
  \includegraphics[width=1.0\linewidth]{Images/Pipeline.pdf}
  \caption{The workflow of our proposed UniDemoiré.}
  \label{fig:pipeline}
\end{figure*}

To address the above challenges, we propose a universal image demoiréing solution, \textbf{UniDemoiré}, capable of generating a vast amount of realistic-looking training data to enhance the generalization capabilities of the image demoiréing model, as Figure~\ref{fig:pipeline} shows. 
First, inspired by the fact that the moiré pattern is unrelated to the content of the image, we introduce a novel, large-scale \textbf{Moiré Pattern Dataset} by capturing moiré patterns against a plain white background. Unlike previous moiré datasets that capture nature images with moiré, our pure moiré patterns can be applied to arbitrary nature images to scale up the data domain automatically. Moreover, our dataset does not need calibrations between the moiré image and the clean image, which can avoid the effect of calibration errors and facilitate the learning process of the model. In particular, our dataset introduces more pattern diversity by considering various previously overlooked factors~\cite{yang2023doing}, including zooming rate, CMOS technology, pixel size, and panel types. Second, building on this real-captured moiré pattern dataset, we propose a diffusion model-based \textbf{Moiré Pattern Generation} method to further increase the diversity of moiré patterns. Specifically, we implement a multi-scale cropping strategy to accommodate different input image sizes and an effective data filtering strategy to ensure the quality of training data for the diffusion model. 
Third, we propose a \textbf{Moiré Image Synthesis} method to create a sufficient amount of diverse and realistic-looking moiré images by blending the generated moiré patterns with clean natural images. In particular, to improve the authenticity of our synthesized data, we develop an effective learnable network and three effective losses to closely mimic the real captured moiré images in terms of color and brightness. Finally, our synthesized abundant moiré images serve to train an \textbf{Image Demoiréing Model} that achieves superior performance and promising generalization capabilities for zero-shot image demoiréing and cross-domain evaluations.

Our contribution can be summarized as follows:
\begin{itemize}
    \item We propose a universal demoiréing solution, which substantially enlarges the knowledge domain and improves the generalization capability of demoiréing models.
    \item We collect a large-scale and high-resolution moiré pattern dataset and develop an effective moiré pattern generator to further increase the diversity of moiré patterns.
    \item We present a novel moiré image synthesis approach, providing a large amount of realistic-looking and high-quality moiré image samples, facilitating the training of a universal image demoiréing model.    
\end{itemize}


%% Related Work
\section{Related Work}
\label{sec:related-work} 

\subsection{Image Restoration and Demoiréing}
The inherent complexity of moiré patterns presents a unique challenge compared to other artifacts such as noise~\cite{xing2021end}, haze~\cite{li2021dehazeflow}, blur~\cite{lee2021iterative}, multiple artifacts in one go~\cite{luo2023refusion,zhu2023denoising,fei2023generative, zhang2023all}, etc. Consequently, these methods may not effectively solve the moiré issue. Current mainstream methods for image demoreing are learning based~\cite{sun2018moire,liu2020wavelet,luo2020deep,he2019mop,he2020fhde,niu2023progressive,wang2023coarse, yue2022recaptured, liu2024video,zheng2020image,zheng2021learning,yu2022towards}, greatly outperforming early handcraft feature based approaches~\cite{ sun2014scanned,liu2015moire,yang2017textured,yang2017demoireing}. However, they exhibit poor generalization capability due to insufficient diverse and realistic training data, and researchers have thus begun exploring the potential of synthesized data.


% Datasets
\begin{table*}[ht]
\centering
\small
\setlength{\tabcolsep}{2.37mm}
\scalebox{0.9}{
\begin{tabular}{cc|ccccccc}
\toprule[1.25pt]
\multicolumn{2}{c|}{Datasets} & \multirow{2.5}{*}{Avg. Resolution} & \multirow{2.5}{*}{Size} & \multicolumn{5}{c}{Capture settings}   \\ 
\cmidrule(lr){1-2} \cmidrule(lr){5-9} 
Type & Name &       &      & Phone  & Screen & Multi-zooming rate & Multi-camera / CMOS  & Screen Panel \\ 
\midrule
\midrule
\multirow{4}{*}{\begin{tabular}[c]{@{}c@{}} Moiré Image \\ Dataset\end{tabular}}  
& TIP2018(R) & 256 $\times$ 256   & 135000 & 3     & 3      & \ding{55}(1x-only) & \ding{55}(Main-only) & IPS-only     \\
& FHDMi(R)     & 1024 $\times$ 1024 & 12000  & 3     & 2      & \ding{55}(1x-only) & \ding{55}(Main-only) & IPS-only     \\
& UHDM(R)   & 4328 $\times$ 3248 & 5000   & 3     & 3      & \ding{55}(1x-only) & \ding{55}(Main-only) & IPS-only     \\ 
& LCDMoiré(S) & 1024 $\times$ 1024 & 10200  & -     & -      & -          & -            & -            \\ 
\midrule
\multirow{2.5}{*}{\begin{tabular}[c]{@{}c@{}} Moiré Pattern \\ Dataset\end{tabular}} 
& MoireSpace(R)  & 2160 $\times$ 1286 & 18147  & 3     & 3      & \ding{55}(1x-only) & \ding{55}(Main-only) & IPS-only     \\ 
\cmidrule{2-9}
& \textbf{Ours(R)} & \textbf{3840} $\times$ \textbf{2160} & \textbf{150000} & \textbf{6} & \textbf{6} & \textbf{\checkmark(1x,2x,3x)} & \textbf{\checkmark(Main,Telephoto)} & \textbf{IPS, SVA}  \\ 
\bottomrule[1.25pt]
\end{tabular}
}
\caption{Comparisons of different moiré datasets. The ``R'' denotes the real dataset, and the ``S'' denotes the synthetic dataset.}
\label{tab:datasets-overview}
\end{table*}

\subsection{Moiré Image Synthesis}
An important category focuses on extracting moiré patterns from existing moiré images. Cyclic~\cite{cyclic} and UnDeM~\cite{undem} utilized GAN-based networks to generate moiré images from unpaired real moiré image datasets, resembling moiré patterns found in moiré images while retaining details from moiré-free images. However, they are unstable and constrained by the moiré patterns present in the real image datasets. 
Another category directly simulates moiré patterns on natural images. Shooting~\cite{shooting} simulated the interference of image processing to produce moiré patterns on natural images while Yang et al.~\shortcite{yang2023doing} collected background-independent moiré patterns and then superimposes the natural image with the collected pattern to synthesize moiré images. 
Unfortunately, due to the real-to-synthetic discrepancy, their model performance is limited in real-world applications. 
In contrast, our solution can produce realistic-looking and diverse data to greatly improve demoiréing models' performance.

\subsection{Moiré Dataset} 
TIP18~\cite{sun2018moire}, FHDMi~\cite{he2020fhde}, UHDM~\cite{yu2022towards} are the most widely-used real-world moiré image dataset with increased resolutions 256, 1080P, and 4K, respectively. To lessen the burden of huge human efforts, a synthetic moiré image dataset LCDMoiré~\shortcite{yuan2019aim} has been generated through shooting simulation.
% However, such a synthetic dataset often fails to accurately mimic the real imaging process, resulting in the consequence that demoiréing models trained on synthetic data struggle in real-world situations. 
However, synthetic datasets often fail to accurately replicate real imaging processes, making it difficult for demoiréing models trained on them to perform well in real-world situations.
More recently, MoireSpace~\cite{yang2023doing} collects background-independent moiré pattern data for a different task, moiré detection. Inspired by it, we propose to collect a real moiré pattern dataset for image demoiréing. Taking inspiration from this effort, we introduce a real moiré pattern dataset specifically tailored for image demoiréing. Comparatively, our dataset boasts a larger volume and greater diversity of data.


%% Methods
\section{Method}

\subsection{Overview} 
The generalization ability of SOTA demoiréing models is greatly limited by the scarcity of data. Therefore, we mainly face two challenges to obtain a universal model with improved generalization capability: To obtain a vast amount of 1) diverse and 2) realistic-looking moiré data.
Notice that traditional moiré image datasets contain real data, but continuously expanding their size to involve more diversity is extremely time-consuming and impractical. 
While current synthesized datasets/methods struggle to synthesize realistic-looking moiré images.
Hence, to tackle these challenges, we introduce a universal solution, UniDemoiré (Figure~\ref{fig:pipeline}). The data diversity challenge is solved by collecting a more diverse moiré pattern dataset and presenting a moiré pattern generator to increase further pattern variations. 
Meanwhile, the data realistic-looking challenge is undertaken by a moiré image synthesis module.  
Finally, our solution can produce realistic-looking moiré images of sufficient diversity, substantially enhancing the zero-shot and cross-domain performance of demoiréing models.

\subsection{Moiré Pattern Dataset}
\label{subsec: Moiré_Patterns_Collection}

% \paragraph{Moiré Pattern Dataset V.S. Moiré Image Dataset}
% \textit{Moiré Pattern Dataset V.S. Moiré Image Dataset} \quad 
The traditional demoiréing datasets~\cite{sun2018moire,he2020fhde,yu2022towards} typically exhibit a 1-1 correspondence, 1 clean image corresponds to only 1 moiré-contaminated image. However, in the real world, an image may be affected by various moiré patterns.
Meanwhile, aligning moiré images with clean images often introduces errors because of the non-linear distortions and moiré artifacts within cameras.
Therefore, we propose to collect a moiré pattern dataset rather than a moiré image dataset, with no need for image alignment and can easily synthesize multiple moiré counterparts of a single natural image. 
The collection of such a dataset is inspired by MoireSpace, which is designed to address the problem of detecting the presence of moiré rather than to eliminate moiré artifacts.  

\paragraph{Capturing Process}
% \textit{Capturing Process} \quad  
We capture videos of real-world moiré patterns on a pure white screen with a mobile phone to minimize color distortion in the moiré patterns. After recording, frames are uniformly extracted from each video to constitute our dataset. The setup is shown in Figure~\ref{fig:data_collection}-left.

\paragraph{Data Diversity}
% \textit{Data Diversity} \quad 
To enhance pattern diversity, we build our dataset by considering additional factors that influence moiré formation, which were overlooked in previous moiré datasets, including zooming rate, camera types, CMOS, and screen panel types. Besides, we doubled the number of mobile devices and display screens compared to existing datasets.
% ~\cite{sun2018moire,he2020fhde,yu2022towards}
A detailed comparison of ours and others is shown in Table~\ref{tab:datasets-overview}.
In summary, our dataset showcases an expanded size, 150000 moiré patterns, in standard 4K resolution with increased diversity. More dataset details are in the appendix. 

\begin{figure}[t]
  \centering
    \includegraphics[width=1.0\linewidth]{Images/Data_Collection.pdf}
  \caption{ Data collection setup (left), and examples of moiré patterns in our dataset captured at different zoom rates and screen panel (middle), and our generated patterns (right). } 
  \label{fig:data_collection}
  % \vspace{-1.5ex}
\end{figure}

\subsection{Moiré Pattern Generation}
\label{subsec: Moiré_Pattern_Generator}
 
Although we have collected a large scale of diverse data, it cannot encompass all conceivable moiré patterns.   
Inspired by recent diffusion models, which have been successfully trained towards diverse image generation in many tasks~\cite{dhariwal2021diffusionmodelsbeatgans}, we propose to use diffusion models to further sample more diverse moiré patterns by sufficiently learning the structural, textural, and color representations of real moiré patterns. 
In this stage, we propose a multi-scale cropping strategy and a colorfulness-sharpness selection strategy to filter high-quality real data. Then we learn the distribution of real moiré patterns in the latent space to generate diverse patterns (Figure~\ref{fig:data_collection}-right).



\paragraph{Multi-Scale Cropping}
Demoiréing models typically employ image patches cropped from the entire image for training. However, given the significant variation of image size in different demoiréing datasets, the scale of content in cropped image patches of the same size also varies greatly. 
Hence, to simulate this process and enhance the diversity of the training data, we perform multi-scale cropping (Figure~\ref{fig:generation} up). 
In particular, 4k images are resized to different sizes, from which we extract and randomly select image patches of uniform size as training data.
In this way, the patches extracted from low and high-resolution images emphasize overall patterns and finer details, respectively.

\paragraph{Colorfulness-Sharpness Selection}
We notice that certain patches involve visually invisible patterns (with a ``\ding{55}'' mark in Figure~\ref{fig:generation}). They potentially confuse the generator during training, aiming to generate moiré pattern images rather than to reproduce plain white images. Hence, we filter out such patches based on colorfulness and sharpness. 
As depicted in Figure~\ref{fig:generation} lower-right, an increased sharpness value indicates more visible moiré patterns, while an increased colorfulness value signifies patterns with richer colors.
The sharpness metric is calculated as the standard deviation of grayscaled input image processed with an edge filter, while the colorfulness metric is calculated as the average standard deviations of A and B channels in image LAB color space.


\subsubsection{Learning Moiré Patterns in the Latent Space} 
As shown in Figure~\ref{fig:data_collection}(middle), plenty of pixels in the moiré pattern appear pure white. 
This leads to a polarization in the pixel distribution of the moiré pattern images, where informative data is concentrated in a few pixels with high values while the rest contains little information.
Based on this observation, we choose to compress the moiré pattern into the latent space through an autoencoder for a more compact and efficient representation of its structural, textural, and color information. For better stability and controllability, we utilize the Latent Diffusion Model~\cite{Rombach2022LDM} to effectively model the complex distribution of the moiré pattern in the latent space.
Examples of generated moiré patterns are shown in Figure~\ref{fig:data_collection} right.
More examples are in the appendix.




\subsection{Moiré Image Synthesis}
\label{subsec: Moiré_Image_Synthesis}

Via data collection and generation, we obtain a vast number of diverse moiré patterns. 
Then, we need to composite moiré patterns with clean images $I_{n}$ to form moiré images. To make the synthesized images realistic-looking,
We first create handcraft rules to produce initial moiré images in the Moiré Image Blending (MIB) module, then design a Tone Refinement Network (TRN) to further faithfully replicate the color and brightness variations observed in real scenes that cannot be fully formulated in those handcraft rules. 
The proposed synthesis process is illustrated in Figure~\ref{fig:systhesis}.



\begin{figure}[t]
  \centering
    \includegraphics[width=1.0\linewidth]{Images/Data_Prepocessing.pdf}
  \caption{Data preprocessing for moiré pattern generation. }
  \label{fig:generation}
   % \vspace{-1.5ex}
\end{figure}


\begin{figure*}[ht]
  \centering
    \includegraphics[width=0.95\linewidth]{Images/Moire_Image_Synthesis.pdf}
  \caption{Overview of the Moiré Image Synthesis stage (a). It involves a Moiré Image Blending module (b) for initial moiré image synthesis and a Tone Refinement Network (c) to refine for more realistic results.}
  \label{fig:systhesis}
  % \vspace{-1.5ex}
\end{figure*}


\subsubsection{Moiré Image Blending}
We blend the clean natural image $I_{n}$ (background layer) with the moiré pattern $I_{mp}$ (foreground layer) to form our initial moiré image $I_{mib}$. Notice that MoireSpace~\cite{yang2023doing} synthesized their moiré image $I'_{sm} $via a Multiply Strategy $M(\cdot, \cdot)$,
\begin{equation}
     I'_{sm} = M(I_{mp}, I_{n}) = I_{mp} \odot I_{n},
 \end{equation}
where ``$\odot$'' denotes element-wise multiplication.
However, the result produced by MoireSpace~\cite{yang2023doing} tends to be dark and cannot replicate the desired contrast and color distortion, as shown in Figure~\ref{fig:synthesis_result}. 
% More results are in the appendix. 
Therefore, we design the following handcraft rules to make the blending more realistic (Figure~\ref{fig:systhesis}b). 
we first incorporate an additional blending strategy, Grain Merge ~\cite{LayerModes} $G(\cdot, \cdot)$. Such a brighter strategy can balance the darker result from $M(\cdot, \cdot)$: 
\begin{equation}
    G(I_{mp}, I_{n}) =  I_{mp} + I_{n} - 0.5.
\end{equation} 
Then, we incorporate transparency of the layers using alpha blending~\cite{Porter_Duff_1984} to obtain $I_{comp}^{M}$ and $I_{comp}^{G}$ :
\begin{align}    
I_{comp}^{M} &= r_m \cdot M(I_{mp}, I_{n}) + [1-r_m] \cdot I_{n},\\
I_{comp}^{G} &= r_g \cdot G(I_{mp}, I_{n}) + [1-r_g] \cdot I_{n}.
\end{align}
where $r_m$ and $r_g$ represent the composition ratio parameter of foreground layer $M(I_{mp}, I_{n})$ and $G(I_{mp}, I_{n})$:
\begin{equation}
    r_x = \frac{op_{x}}{op_{x}+(1-op_{x})\cdot op_{n}}, x \in \{m,g\}.
\end{equation}
where $op_m$, $op_g$ represent the opacity of the output layers from the Multiply and the Grain Merge strategies, and $op_{n}$ represent the opacity of the background layer $I_{n}$.
Finally, we perform a weighted (weight $\omega_m$, $\omega_g$) combination of $I_{comp}^{M}$ and $I_{comp}^{G}$ to obtain $I_{mib}$ :
\begin{equation}    
I_{mib} = \omega_m \cdot I_{comp}^{M} + \omega_g \cdot I_{comp}^{G}.
\end{equation}
A visual comparison of MoireSpace result $I'_{sm}$, and our $I_{mib}$ is shown in Figure~\ref{fig:synthesis_result}, showing the superior of $I_{mib}$ over $I'_{sm}$.
Please refer to the appendix for more visual results.



\subsubsection{Tone Refinement Network} 
Though the moiré image blending module creates a preliminary moiré image $I_{mib}$, such a synthesized result based on handcraft rules still struggles to replicate accurate color and brightness changes. 
Comparatively, networks are more powerful in capturing such unknown changes and distortion by progressive learning.
Hence, we present a learnable refinement network to synthesize more realistic results.

The Tone Refinement Network (TRN) proposed here is built on a U-shaped transformer backbone~\cite{Wang2022Uformer} incorporating multiple refine blocks, illustrated in Figure~\ref{fig:systhesis} (c). 
It takes $I_{mib}$ as input, applies pixel-wise tone adjustment to $I_{mib}$, and minimizes the tone gap between the output $I_{trn}$ and the given real moiré images $I_{rm}$. 
To be clear,
TRN firstly applies a 3$\times$3 convolutional layer with LeakyReLU to extract tone features $F_{mib}^{(0)}$, $F_{rm}^{(0)}$. 
Next, the feature maps $F_{mib}^{(0)}$ and $F_{rm}^{(0)}$ are passed through $N$ encoder phases and $N$ decoder phases with skip connections. 
Each phase contains a refine block to capture long-range dependencies, benefiting from the self-attention in Transformer. 

Inspired by research in style transfer and domain generalization~\cite{Ulyanov2016IN, Huang2017Arbitrary, Zhou2021Mixstyle}, we design a tone feature fusion block within each refine block to better fuse the tone feature statistics between $I_{mib}$ and corresponding $I_{rm}$. 
It mixes the feature statistics of two instances with a random convex weight. 
As illustrated in Figure~\ref{fig:systhesis} (c), the computations inside a fusion block module in the $k$-th refine block can be summarized into two steps. 
First, given two sets of feature maps $f^{(k)}$ and $f_r^{(k)}$ for $I_{mib}$ and $I_{rm}$, the fusion block generates a mixture of feature statistics, 
\begin{align}
    \gamma_{mix} &= \lambda \cdot \sigma(f^{(k)}) + (1-\lambda) \cdot \sigma(f^{(k)}_r), \\
    \beta_{mix}  &= \lambda \cdot \mu(f^{(k)})    + (1-\lambda) \cdot \mu(f^{(k)}_r).
\end{align}
where $\mu$ and $\sigma$ represent the mean and variance of feature maps, while $\lambda$ is a random weight sampled from the beta distribution, $\lambda \in \text{Beta}(\alpha, \alpha)$ with $\alpha \in (0, \infty)$ being a hyper-parameter.
Then, the mixture of feature statistics is applied to the tone-normalized $F_{mib}^{(k+1)}$:
\begin{equation}
    F_{mib}^{(k+1)} = \gamma_{mix} \odot \frac{f^{(k)}-\mu(f^{(k)})}{\sigma(f^{(k)})} + \beta_{mix}.
\end{equation}
The fusion block can effectively utilize the moiré feature information of $I_{rm}$ and greatly helps reduce the moiré domain gap between the final synthesized image $I_{trn}$ and real moiré image $I_{rm}$, which is one significant innovation.
After the $N$ decoder stages, we apply a 3$\times$3 convolution layer on feature maps $F_{mib}^{(2N)}$ to obtain a tone refinement matrix $M_{trn}$. 
Finally, the synthetic image is obtained by $I_{trn}=I_{mib} \odot M_{trn}$ after color normalization, where ``$\odot$'' represents element-wise multiplication. 
Notice that the fusion block is solely utilized in the training phase, and $I_{rm}$ is exclusively fed into the network during training. 
% More details of TRN can be found in the appendix.
Figure~\ref{fig:synthesis_result} compares the initial blending result $I_{mib}$ with the final synthesized result $I_{trn}$. Please refer to the appendix for more results.

\begin{figure}[t]
  \centering
    \includegraphics[width=1.0\linewidth]{Images/Synthesis_Result.pdf}
  \caption{Visualization of our intermediate synthetic results.  
  % Final synthesis results $I_{trn}$ best resemble real moiré images.
  } 
  \label{fig:synthesis_result}
  % \vspace{-3ex}
\end{figure}


 





% % \begin{table}[t]
% %% Results of the demoireing experiment using the case of multiple dataset training. 
% \caption{Influence of data volume on target dataset TIP~\cite{sun2018moire}.
% %Quantitative results of cross-domain demoireing on the target dataset TIP with incrementally adding new source data. 
% ``F'' represents the FHDMi dataset~\cite{he2020fhde}. ``\dag'' indicates that only natural images are being used.}
% \centering
% % \renewcommand\tabcolsep{5.0pt}
% % \vspace{-8pt}
% % \resizebox{10cm}{!}       % \textwidth, 10cm, 这里用 10cm 好像效果好于 \textwidth
% \scalebox{0.72}
% {       
% \begin{tabular}{ccc|ccc}
% \toprule    % 顶部粗线
% \multirow{2}{*}{Model}    & \multicolumn{2}{c|}{Source Datasets}                        & \multicolumn{3}{c}{Metric}            \\ 
%                                                                                  \cmidrule{2-6} 
%                           & Synthesis             & Demoiréing          & \ua{PSNR } & \ua{SSIM } & \da{LPIPS } \\
% \midrule

% % \multirow{6}{*}{ESDNet-L} & \multirow{5}{*}{UHDM} & UHDM                         &            &            &             \\
% %                           &                       & UHDM + FHDMi\dag {[}25\%{]}  &            &            &             \\
% %                           &                       & UHDM + FHDMi\dag {[}50\%{]}  &            &            &             \\
% %                           &                       & UHDM + FHDMi\dag {[}75\%{]}  &            &            &             \\
% %                           &                       & UHDM + FHDMi\dag {[}100\%{]} &            &            &             \\ 
% %                           \cline{2-6} 
% %                           & UHDM+FHDMi            & UHDM + FHDMi                 &            &            &             \\ 

% \multirow{4}{*}{ESDNet-L} & \multirow{4}{*}{UHDM} & UHDM                     & 19.9983    & 0.7052     & 0.3301      \\
%                           &                       & UHDM + F\dag {[}25\%{]}  & 20.8691    & 0.7491     & 0.2456      \\
%                           % &                       & UHDM + F\dag {[}50\%{]}  & 21.0504    & 0.7579     & 0.2357      \\
%                           &                       & UHDM + F\dag {[}75\%{]}  & 21.0882    & 0.7606     & 0.2245      \\
%                           &                       & UHDM + F\dag {[}100\%{]} & 21.1460    & 0.7672     & 0.2163      \\ 
%                           % \cmidrule{2-6} 
%                           % & UHDM+F                & UHDM + F                 & 21.1558    & 0.7766     & 0.2316      \\ 

% \bottomrule  % 底部粗线
% \end{tabular}
% }
% \label{tab:Exp_multi_datasets}
% % \vspace{-2ex}
% \end{table}


% \begin{table}[t]

% Demoireing results on the UHDM dataset training with synthesized data only. 
% Since both our method and shooting~\cite{shooting} borrow none real moire patterns in dataset UHDM, for fairness, we instruct UnDem~\cite{undem} to utilize the moire patterns from FHDMi~\cite{he2020fhde}. Method "Paired" indicates training using the original real moire data in the training set of UHDM.
% The "C.S." denotes size in the random crop. $(\uparrow)$ denotes the larger, the better, and $(\downarrow)$ denotes the smaller, the better.

%% Quantitative results of cross-domain demoireing on the target dataset TIP with incrementally adding new source data and 1-N synthetic moire pattern 


% \caption{Quantitative results of cross-domain demoireing on the target dataset TIP with 1-N synthetic moire pattern. The ``B'' represents the dataset used in the blending process, and the ``DT'' represents the dataset used in the demoireing training process.}
% \centering
% \renewcommand\tabcolsep{5.0pt}
% \vspace{-8pt}
% % \resizebox{10cm}{!}       % \textwidth, 10cm, 这里用 10cm 好像效果好于 \textwidth
% \scalebox{0.895}
% {       
% % \begin{tabular}{ccc|ccc}
% % \toprule    % 顶部粗线
% % \multirow{2}{*}{Model}    & \multirow{2}{*}{1 to N} & \multirow{2}{*}{Demoireing Training Datasets} & \multicolumn{3}{c}{Metric} \\ \cline{4-6} 
% %                           &                         &                                               &\ua{PSNR }&\ua{SSIM }&\da{LPIPS }   \\ \hline
% % \multirow{8}{*}{ESDNet-L} & \multirow{4}{*}{1 to 1} & UHDM + FHDMi\dag {[}0\%{]}                        &         &        &         \\
% %                           &                         & UHDM + FHDMi\dag {[}25\%{]}                       &         &        &         \\
% %                           &                         & UHDM + FHDMi\dag {[}50\%{]}                       &         &        &         \\
% %                           &                         & UHDM + FHDMi\dag {[}100\%{]}                      &         &        &         \\ 
% %                           \cline{2-6} 
% %                           & \multirow{4}{*}{1 to 3} & UHDM + FHDMi\dag {[}0\%{]}                        &         &        &         \\
% %                           &                         & UHDM + FHDMi\dag {[}25\%{]}                       &         &        &         \\
% %                           &                         & UHDM + FHDMi\dag {[}50\%{]}                       &         &        &         \\
% %                           &                         & UHDM + FHDMi\dag {[}100\%{]}                      &         &        &         \\ 
% % \bottomrule  % 底部粗线
% % \end{tabular}


% % \begin{tabular}{cccc|ccc}
% % \toprule
% % \multirow{2}{*}{Model}     & \multirow{2}{*}{1 to N} & \multicolumn{2}{c|}{Datasets}                        & \multicolumn{3}{c}{Metric} \\ \cline{3-7} 
% %                            &                         & Blending              & Demoireing Training          &\ua{PSNR }&\ua{SSIM }&\da{LPIPS }\\ 
% % %\midrule
% % \hline
% % \multirow{10}{*}{ESDNet-L} & \multirow{5}{*}{1 to 1} & \multirow{4}{*}{UHDM} & UHDM                         &         &        &         \\
% %                            &                         &                       & UHDM + FHDMi\dag {[}25\%{]}  &         &        &         \\
% %                            &                         &                       & UHDM + FHDMi\dag {[}50\%{]}  &         &        &         \\
% %                            &                         &                       & UHDM + FHDMi\dag {[}100\%{]} &         &        &         \\ \cline{3-7} 
% %                            &                         & UHDM+FHDMi            & UHDM + FHDMi                 &         &        &         \\ \cline{2-7} 
% %                            & \multirow{5}{*}{1 to 3} & \multirow{4}{*}{UHDM} & UHDM                         &         &        &         \\
% %                            &                         &                       & UHDM + FHDMi\dag {[}25\%{]}  &         &        &         \\
% %                            &                         &                       & UHDM + FHDMi\dag {[}50\%{]}  &         &        &         \\
% %                            &                         &                       & UHDM + FHDMi\dag {[}100\%{]} &         &        &         \\ \cline{3-7} 
% %                            &                         & UHDM+FHDMi            & UHDM + FHDMi                 &         &        &         \\ 
% % \bottomrule
% % \end{tabular}

% \begin{tabular}{lcc|ccc}
% \toprule
% \multirow{2}{*}{Model}    & \multirow{2}{*}{Datasets (B \& DT)} & \multirow{2}{*}{1 to N} & \multicolumn{3}{c}{Metric}            \\ 
% \cmidrule{4-6} 
%                           &                                  &                         & \ua{PSNR } & \ua{SSIM } & \da{LPIPS } \\
% \midrule
% \multirow{3}{*}{ESDNet-L} & \multirow{3}{*}{UHDM}            & 1 to 1                  & 19.9983    & 0.7052     & 0.3301      \\
%                           &                                  & 1 to 3                  &            &            &             \\
%                           &                                  & 1 to 5                  & 20.5058    & 0.7008     & 0.3292      \\
% \bottomrule
% \end{tabular}
% }
% \label{tab:Exp_1_to_N}
% \end{table}



% Please add the following required packages to your document preamble:
% \usepackage{multirow}
% \begin{table}[h]


% % %Quantitative results of cross-domain demoireing on the target dataset TIP with incrementally adding new source data. 
% % ``F'' represents the FHDMi dataset~\cite{he2020fhde}. ``\dag'' indicates that only natural images are being used.}
% \centering
% \setlength{\tabcolsep}{0.9mm}
% % \scalebox{0.67}
% % {
% % \begin{tabular}{l @{\hskip +0em}|c@{\hskip +2em} c  @{\hskip +2em} c   @{\hskip +2em} c  @{\hskip +2em} c @{\hskip 2em}c}
% \scalebox{0.85}
% {
% \begin{tabular}{l|cccccc}
% \toprule    % 顶部粗线
% % \multicolumn{1}{c}{Method}
% % \multirow{2.5}{*}{Method}
% % \multicolumn{1}{c}{\multirow{2.5}{*}{Method}}
% \multicolumn{1}{c|}{\multirow{2.5}{*}{Method}}    & \multicolumn{3}{c}{UHDM}       & \multicolumn{3}{c}{FHDMi}   \\ 
% \cmidrule(lr){2-4} \cmidrule(lr){5-7} 
%                            &\ua{PSNR}& \ua{SSIM} & \da{LPIPS}     &\ua{PSNR}& \ua{SSIM} & \da{LPIPS}    \\ 
% % \midrule
% % \multicolumn{7}{c}{\textit{Demoiréing Model: \textbf{ESDNet-L}, trained with \textbf{MHRNID}, 50 Epochs}} \\
% \midrule
% Shooting                   & 10.2568          & 0.5664          & 0.5130          & 11.6022  & 0.5425 & 0.4515  \\
% MoireSpace$\ddagger^{1}$   &           &           &           &   &  &   \\ 
% UnDeM$\dagger$             & 15.2269          & 0.5873          & 0.4190          &          &        &         \\
% UnDeM$\ddagger^{1}$        & 15.2947          & 0.5777          & 0.4241          &          &        &         \\
% \midrule
% Ours$\ddagger^{1}$         & \textbf{17.2524} & \textbf{0.6454} & \textbf{0.3238} & \textbf{19.8128} & \textbf{0.7319} & \textbf{0.2134} \\
% Ours$\ddagger^{3}$         &                  &                 &                 &          &        &         \\ 
% \bottomrule  % 底部粗线
% \end{tabular}
% }
% \caption{Quantitative results of zero-shot demoiréing trained with synthesized data only. ``$\dagger$'', ``$\ddagger^{1}$'' and ``$\ddagger^{3}$'' indicate that the synthetic model is trained with a TIP dataset, a 1-to-1 moiré pattern, and a 1-to-3 moiré pattern. }

% \label{tab:Exp_multi_datasets}
% % \vspace{-2ex}
% \end{table}
		

		


% Please add the following required packages to your document preamble:
% \usepackage{multirow}
% \usepackage[table,xcdraw]{xcolor}
% Beamer presentation requires \usepackage{colortbl} instead of \usepackage[table,xcdraw]{xcolor}
\begin{table*}[ht]
\small
\centering
\setlength{\tabcolsep}{2.7mm}
\scalebox{0.85}{
\begin{tabular}{cc|ccccc|ccccc}
\toprule[1.25pt]
% \multicolumn{2}{c}{Zero   Shot}& \multirow{2.5}{*}{Metric} & \multicolumn{5}{c|}{Demoire Network: MBCNN}  & \multicolumn{5}{c}{Demoire Network:   ESDNet-L}  \\ 
% \cmidrule(lr){1-2} \cmidrule(lr){4-8}  \cmidrule(lr){9-13}

\multirow{2.5}{*}{\makecell{Test \\ Dataset}} & \multirow{2.5}{*}{Metric} & \multicolumn{5}{c|}{Demoiréing Network: \textbf{MBCNN}}  & \multicolumn{5}{c}{Demoiréing Network:   \textbf{ESDNet-L}}  \\ 
 \cmidrule(lr){3-7}  \cmidrule(lr){8-12}
&           & Shooting & UnDeM$^\dagger$ & UnDeM$^\ddagger$ & MoireSpace  & Ours & Shooting & UnDeM$^\dagger$ & UnDeM$^\ddagger$ & MoireSpace & Ours \\ 
\midrule
\midrule
% \multirow{6.5}{*}{MHRNID} 
 \multirow{3}{*}{UHDM}  &\ua{PSNR } & 9.2284  & 13.4256 & 14.5237 & 14.7826 & \textbf{17.9162} & 10.2568 & 15.2269 & 15.2947 & 14.7989 & \textbf{17.2524} \\
                        &\ua{SSIM } & 0.5180  & 0.3973  & 0.4425  & 0.4724  & \textbf{0.6280}  & 0.5664  & 0.5873  & 0.5777  & 0.4859  & \textbf{0.6454}  \\
                        &\da{LPIPS }& 0.6664  & 0.6489  & 0.6332  & 0.5568  & \textbf{0.4162}  & 0.5130  & 0.4190  & 0.4241  & 0.5254  & \textbf{0.3238}  \\ 
\cmidrule{1-12} 
 \multirow{3}{*}{FHDMi} &\ua{PSNR } & 10.6750  & 17.8355 & 18.1652 & 18.5523 & \textbf{19.0094} & 11.6022 & 18.4335 & 18.5390 & 18.0763 & \textbf{19.8128} \\
                        &\ua{SSIM } & 0.4478  & 0.6802  & 0.6999  & 0.7094  & \textbf{0.7137}  & 0.5425  & 0.6900  & 0.6812  & 0.7189  & \textbf{0.7319}  \\
                        &\da{LPIPS }& 0.5978  & 0.2606  & 0.2472  & 0.2742  & \textbf{0.2390}  & 0.4515  & 0.2877  & 0.2986  & 0.2616  & \textbf{0.2134}  \\ 
\bottomrule[1.25pt]
\end{tabular}
}
\caption{Quantitative results of zero-shot demoiréing trained with synthesized data only. 
``$\dagger$'' indicates UnDem uses moiré patterns retrieved from real data in TIP for inference.  ``$\ddagger$''  indicates UnDem uses our generated moiré pattern for inference.
}
\label{tab:Exp_multi_datasets}
\end{table*}










% \begin{table*}[t]
% \small
% \centering
% \setlength{\tabcolsep}{0.9mm}
% \begin{tabular}{ccc|ccccc|ccccc}
% \toprule[1.25pt]
% \multicolumn{2}{c}{Zero   Shot}& \multirow{2.5}{*}{Metric} & \multicolumn{5}{c|}{Demoire Network: MBCNN}  & \multicolumn{5}{c}{Demoire Network:   ESDNet-L}  \\ 
% \cmidrule(lr){1-2} \cmidrule(lr){4-8}  \cmidrule(lr){9-13}
% Source   & Target              &           & Shooting & UnDeM$^\dagger$ & UnDeM$^\ddagger$ & MoireSpace  & Ours & Shooting & UnDeM$^\dagger$ & UnDeM$^\ddagger$ & MoireSpace & Ours \\ 
% \midrule
% \midrule
% \multirow{6.5}{*}{MHRNID} 
% & \multirow{3}{*}{UHDM}  &\ua{PSNR} & 00.0000 & 00.0000 & 00.0000 & 00.0000 & \textbf{00.0000} & 10.2568 & 15.2269 & 15.2947 & 00.0000 & \textbf{17.2524} \\
% &                        &\ua{SSIM} &         &         &         &         &                  & 0.5664  & 0.5873  & 0.5777  &         & \textbf{0.6454}  \\
% &                        &\da{LPIPS}&         &         &         &         &                  & 0.5130  & 0.4190  & 0.4241  &         & \textbf{0.3238}  \\ 
% \cmidrule{2-13} 
% & \multirow{3}{*}{FHDMi} &\ua{PSNR} &         &         &         &         &                  & 11.6022 &         &         &         & \textbf{19.8128} \\
% &                        &\ua{SSIM} &         &         &         &         &                  & 0.5425  &         &         &         & \textbf{0.7319}  \\
% &                        &\da{LPIPS}&         &         &         &         &                  & 0.4515  &         &         &         & \textbf{0.2134}  \\ 
% \bottomrule[1.25pt]
% \end{tabular}
% \caption{Quantitative results of zero-shot demoiréing trained with synthesized data only. 
% ``$\dagger$'' and ``$\ddagger$'' indicate that the UnDem uses real moiré images from TIP or pure moiré patterns for inference.
% }
% \label{tab:Exp_multi_datasets}
% \end{table*}
		
		
 %% Zero-shot 实验
\begin{table*}[ht]
\small
\centering
\setlength{\tabcolsep}{2.26mm}
\scalebox{0.9}{
\begin{tabular}{cc|ccccc|ccccc}
\toprule[1.25pt]
\multirow{2.5}{*}{\makecell{Test \\ Dataset}} & \multirow{2.5}{*}{Metric} & \multicolumn{5}{c|}{Demoiréing Network: \textbf{MBCNN}}  & \multicolumn{5}{c}{Demoiréing Network:   \textbf{ESDNet-L}}  \\ 
 \cmidrule(lr){3-7}  \cmidrule(lr){8-12}
&           & Shooting & UnDeM$^\dagger$ & UnDeM$^\ddagger$ & MoireSpace  & Ours & Shooting & UnDeM$^\dagger$ & UnDeM$^\ddagger$ & MoireSpace & Ours \\ 
\midrule
\midrule
 \multirow{3}{*}{UHDM}  &\ua{PSNR } & 9.2284  & 13.4256 & 14.5237 & 14.7826 & \textbf{17.9162} & 10.2568 & 15.2269 & 15.2947 & 14.7989 & \textbf{17.2524} \\
                        &\ua{SSIM } & 0.5180  & 0.3973  & 0.4425  & 0.4724  & \textbf{0.6280}  & 0.5664  & 0.5873  & 0.5777  & 0.4859  & \textbf{0.6454}  \\
                        &\da{LPIPS }& 0.6664  & 0.6489  & 0.6332  & 0.5568  & \textbf{0.4162}  & 0.5130  & 0.4190  & 0.4241  & 0.5254  & \textbf{0.3238}  \\ 
\cmidrule{1-12} 
 \multirow{3}{*}{FHDMi} &\ua{PSNR } & 10.6750  & 17.8355 & 18.1652 & 18.5523 & \textbf{19.0094} & 11.6022 & 18.4335 & 18.5390 & 18.0763 & \textbf{19.8128} \\
                        &\ua{SSIM } & 0.4478  & 0.6802  & 0.6999  & 0.7094  & \textbf{0.7137}  & 0.5425  & 0.6900  & 0.6812  & 0.7189  & \textbf{0.7319}  \\
                        &\da{LPIPS }& 0.5978  & 0.2606  & 0.2472  & 0.2742  & \textbf{0.2390}  & 0.4515  & 0.2877  & 0.2986  & 0.2616  & \textbf{0.2134}  \\ 
\bottomrule[1.25pt]
\end{tabular}
}
\caption{Quantitative results of zero-shot demoiréing trained with synthesized data only. 
``$\dagger$'' indicates UnDem uses moiré patterns retrieved from real data in TIP for inference.  ``$\ddagger$''  indicates UnDem uses our generated moiré pattern for inference.
}
\label{tab:Exp_multi_datasets}
\end{table*}


% % \begin{table}[t]

% % Demoireing results on the UHDM dataset training with synthesized data only. 
% % Since both our method and shooting~\cite{shooting} borrow none real moire patterns in dataset UHDM, for fairness, we instruct UnDem~\cite{undem} to utilize the moire patterns from FHDMi. Method "Paired" indicates training using the original real moire data in the training set of UHDM.
% % The "C.S." denotes size in the random crop. $(\uparrow)$ denotes the larger, the better, and $(\downarrow)$ denotes the smaller, the better.
% \caption{Cross-dataset Evaluations %The ``Paired'' indicates training demoiréing network using paired data from the source dataset, and the ``UnDeM'' indicates training demoiréing network using pseudo-paired data generated by the GAN-based generator trained on the source dataset.
% }
% \centering
% \newcolumntype{g}{>{\columncolor{Gray}}c}
% % \renewcommand\tabcolsep{5.0pt}
% % \vspace{-8pt}
% % \resizebox{10cm}{!}       % \textwidth, 10cm, 这里用 10cm 好像效果好于 \textwidth
% %\scalebox{0.85}{    
% \resizebox{.48\textwidth}{!}{
% \begin{tabular}{cccc|ccc}
% \toprule[1.25pt]    % 顶部粗线
% Model                      & Source                 & Target                 & Metric    & Original  & UnDeM   &  \textbf{Ours} \\ 
% \midrule
% \multirow{18}{*}{MBCNN}    & \multirow{6}{*}{UHDM}  & \multirow{3}{*}{FHDMi} &\ua{PSNR } & 19.3848 & 19.4676 & \textbf{20.3260} \\
%                            &                        &                        &\ua{SSIM } & 0.7536  & 0.7555  & \textbf{0.7812}  \\
%                            &                        &                        &\da{LPIPS }& 0.2902  & 0.2864  & \textbf{0.2355}  \\ 
%                            \cmidrule{3-7} 
%                            &                        & \multirow{3}{*}{TIP}   &\ua{PSNR } & 17.8107 & 19.6674 & \textbf{20.7684} \\
%                            &                        &                        &\ua{SSIM } & 0.6627  & 0.6911  & \textbf{0.7465}  \\
%                            &                        &                        &\da{LPIPS }& 0.3580  & 0.3909  & \textbf{0.2871}  \\ 
%                            \cmidrule{2-7} 
%                            & \multirow{6}{*}{FHDMi} & \multirow{3}{*}{UHDM}  &\ua{PSNR } & 17.1331 & 17.4870 & \textbf{18.2999} \\
%                            &                        &                        &\ua{SSIM } & 0.6159  & 0.6331  & \textbf{0.6772}  \\
%                            &                        &                        &\da{LPIPS }& 0.4470  & 0.4285  & \textbf{0.3822}  \\ 
%                            \cmidrule{3-7} 
%                            &                        & \multirow{3}{*}{TIP}   &\ua{PSNR } & 20.2161 & 20.8261 & \textbf{21.1324} \\
%                            &                        &                        &\ua{SSIM } & 0.7340  & 0.7381  & \textbf{0.7449}  \\
%                            &                        &                        &\da{LPIPS }& 0.2879  & 0.2791  & \textbf{0.2770}  \\ 
%                            \cmidrule{2-7} 
%                            & \multirow{6}{*}{TIP}   & \multirow{3}{*}{UHDM}  &\ua{PSNR } & 17.3409 & 17.4407 & \textbf{17.7890} \\
%                            &                        &                        &\ua{SSIM } & 0.6144  & 0.6066  & \textbf{0.6549}  \\
%                            &                        &                        &\da{LPIPS }& 0.4726  & 0.4473  & \textbf{0.4286}  \\
%                            \cmidrule{3-7}
%                            &                        & \multirow{3}{*}{FHDMi} &\ua{PSNR } & 18.9458 & 19.0336 & \textbf{19.2419} \\
%                            &                        &                        &\ua{SSIM } & 0.7369  & 0.7215  & \textbf{0.7441}  \\
%                            &                        &                        &\da{LPIPS }& 0.2494  & 0.2452  & \textbf{0.2256}  \\
% \midrule
% \midrule
% \multirow{18}{*}{ESDNet-L} & \multirow{6}{*}{UHDM}  & \multirow{3}{*}{FHDMi} &\ua{PSNR } & 20.3422 & 20.4014 & \textbf{22.0638} \\
%                            &                        &                        &\ua{SSIM } & 0.7699  & 0.7710  & \textbf{0.8021}  \\
%                            &                        &                        &\da{LPIPS }& 0.2525  & 0.2509  & \textbf{0.1707}  \\ 
%                            \cmidrule{3-7} 
%                            &                        & \multirow{3}{*}{TIP}   &\ua{PSNR } & 18.8040 & 19.3545 & \textbf{19.9983} \\
%                            &                        &                        &\ua{SSIM } & 0.6921  & 0.6998  & \textbf{0.7052}  \\
%                            &                        &                        &\da{LPIPS }& 0.3524  & 0.3601  & \textbf{0.3301}  \\ 
%                            \cmidrule{2-7} 
%                            & \multirow{6}{*}{FHDMi} & \multirow{3}{*}{UHDM}  &\ua{PSNR } & 18.0049 & 17.9574 & \textbf{18.6718} \\
%                            &                        &                        &\ua{SSIM } & 0.5755  & 0.5857  & \textbf{0.6867}  \\
%                            &                        &                        &\da{LPIPS }& 0.4420  & 0.4460  & \textbf{0.3621}  \\ 
%                            \cmidrule{3-7} 
%                            &                        & \multirow{3}{*}{TIP}   &\ua{PSNR } & 20.6647 & 20.4663 & \textbf{21.2096} \\
%                            &                        &                        &\ua{SSIM } & 0.7504  & 0.7278  & \textbf{0.7602}  \\
%                            &                        &                        &\da{LPIPS }& 0.2359  & 0.2998  & \textbf{0.2231}  \\ 
%                            \cmidrule{2-7} 
%                            & \multirow{6}{*}{TIP}   & \multirow{3}{*}{UHDM}  &\ua{PSNR } & 17.4332 & 16.8402 & \textbf{18.0582} \\
%                            &                        &                        &\ua{SSIM } & 0.5523  & 0.5692  & \textbf{0.6586}  \\
%                            &                        &                        &\da{LPIPS }& 0.4987  & 0.4532  & \textbf{0.3716}  \\ 
%                            \cmidrule{3-7} 
%                            &                        & \multirow{3}{*}{FHDMi} &\ua{PSNR } & 19.2368 & 19.2112 & \textbf{19.4041} \\
%                            &                        &                        &\ua{SSIM } & 0.7354  & 0.7499  & \textbf{0.7573}  \\
%                            &                        &                        &\da{LPIPS }& 0.2316  & 0.2130  & \textbf{0.2118}  \\ 
% \bottomrule  % 底部粗线
% \end{tabular}
% }
% \label{tab:Exp_cross_datasets}
% \end{table}





% Please add the following required packages to your document preamble:
% \usepackage{multirow}
% \begin{table}[t]

% \caption{Cross-dataset Evaluations}
% % \setlength{\tabcolsep}{1mm}
% % \setlength{\tabcolsep}{1mm}
% \centering
% \resizebox{.474\textwidth}{!}{
% \begin{tabular}{lcccccc}
% \toprule[1.25pt]    % 顶部粗线
% % \multirow{2}{*}{\makecell{Cross-dataset \\(Train→Test)}}
% % \multirow{2}{*}{\begin{tabular}[c]{@{}c@{}}Cross-dataset \\(Train→Test)\end{tabular}}
% % \makecell{Cross-dataset \\ (Train→Test)}

% \multirow{2.5}{*}{\multicolumn{1}{c}{\makecell{Cross \\ Dataset}}} & \multirow{2.5}{*}{Metric} & \multirow{2.5}{*}{Baseline} & \multicolumn{4}{c}{Moire Image Synthesis   Method} \\ 
% \cmidrule(lr){4-7} 
% &      &  
% & Shooting    & UnDeM      & MoireSpace    & Ours    \\ 
% \midrule
% \midrule
% \multicolumn{7}{c}{\textit{Demoiréing Model: \textbf{MBCNN}}}\\
% % \cmidrule(lr){3-5} 
% \midrule
% % UHDM $\rightarrow$ FHDMi 
% % UHDM \\ $\downarrow$ \\ FHDMi
% % UHDM \\  $\rightarrow$ FHDMi 
% % \begin{tabular}[c]{@{}c@{}}UHDM \\$\rightarrow$ FHDMi\end{tabular}
% \multirow{3}{*}{\begin{tabular}[c]{@{}l@{}}UHDM \\ $\rightarrow$ FHDMi\end{tabular}} 
% & PSNR  & 19.3848 & 19.2032 & 19.4676 &         &         \\
% & SSIM  & 0.7536  & 0.7459  & 0.7555  &         &         \\
% & LPIPS & 0.2902  & 0.2975  & 0.2864  &         &         \\
% \midrule
% % UHDM→TIP
% \multirow{3}{*}{\begin{tabular}[c]{@{}l@{}}UHDM \\ $\rightarrow$ TIP\end{tabular}}   
% & PSNR  & 17.8107 & 18.373  & 19.6674 &         &         \\
% & SSIM  & 0.6627  & 0.6888  & 0.6911  &         &         \\
% & LPIPS & 0.3580  & 0.3886  & 0.3909  &         &         \\
% \midrule
% % FHDMi→UHDM
% \multirow{3}{*}{\begin{tabular}[c]{@{}l@{}}FHDMi \\ $\rightarrow$ UHDM\end{tabular}} 
% & PSNR  & 17.1331 &         & 17.487  &         &         \\
% & SSIM  & 0.6159  &         & 0.6331  &         &         \\
% & LPIPS & 0.4470  &         & 0.4285  &         &         \\
% \midrule
% % FHDMi→TIP
% \multirow{3}{*}{\begin{tabular}[c]{@{}l@{}}FHDMi \\ $\rightarrow$ TIP\end{tabular}}  
% & PSNR  & 20.2161 &         & 20.8261 &         &         \\
% & SSIM  & 0.7340  &         & 0.7381  &         &         \\
% & LPIPS & 0.2879  &         & 0.2791  &         &         \\
% \midrule
% % TIP→UHDM
% \multirow{3}{*}{\begin{tabular}[c]{@{}l@{}}TIP \\ $\rightarrow$ UHDM\end{tabular}} 
% & PSNR  & 17.3409 &         & 17.4407 &         &         \\
% & SSIM  & 0.6144  &         & 0.6066  &         &         \\
% & LPIPS & 0.4726  &         & 0.4473  &         &         \\
% \midrule
% % TIP→FHDMi
% \multirow{3}{*}{\begin{tabular}[c]{@{}l@{}}TIP \\ $\rightarrow$ FHDMi\end{tabular}}
% & PSNR  & 18.9458 &         & 19.0336 &         &         \\
% & SSIM  & 0.7369  &         & 0.7215  &         &         \\
% & LPIPS & 0.2494  &         & 0.2452  &         &         \\ 
% \midrule
% \midrule
% % \textit{Demoiréing Model: \textbf{ESDNet-L}}
% % \multirow{1}{*}{\textit{Demoiréing Model: \textbf{ESDNet-L}}}
% % \multirow{0.7}{*}{\textit{Demoiréing Model: \textbf{ESDNet-L}}}
% \multicolumn{7}{c}{\textit{Demoiréing Model: \textbf{ESDNet-L}}} \\
% \midrule
% % \cmidrule{1-2}
% \multirow{3}{*}{\begin{tabular}[c]{@{}l@{}}UHDM \\ $\rightarrow$ FHDMi\end{tabular}}                              
% & PSNR  & 20.3422 & 20.2407 & 20.4014 &         &         \\
% & SSIM  & 0.7699  & 0.7579  & 0.771   &         &         \\
% & LPIPS & 0.2525  & 0.2632  & 0.2509  &         &         \\
% \midrule
% % UHDM→TIP
% \multirow{3}{*}{\begin{tabular}[c]{@{}l@{}}UHDM \\ $\rightarrow$ TIP\end{tabular}}
% & PSNR  & 18.804  & 18.4543 & 19.3545 &         &         \\
% & SSIM  & 0.6921  & 0.693   & 0.6998  &         &         \\
% & LPIPS & 0.3524  & 0.3849  & 0.3601  &         &         \\
% \midrule
% % FHDMi→UHDM
% \multirow{3}{*}{\begin{tabular}[c]{@{}l@{}}FHDMi \\ $\rightarrow$ UHDM\end{tabular}} 
% & PSNR  & 18.0049 &         & 17.9574 &         &         \\
% & SSIM  & 0.5755  &         & 0.5857  &         &         \\
% & LPIPS & 0.442   &         & 0.446   &         &         \\
% \midrule
% % FHDMi→TIP
% \multirow{3}{*}{\begin{tabular}[c]{@{}l@{}}FHDMi \\ $\rightarrow$ TIP \end{tabular}}  
% & PSNR  & 20.6647 &         & 20.4663 &         &         \\
% & SSIM  & 0.7504  &         & 0.7278  &         &         \\
% & LPIPS & 0.2359  &         & 0.2998  &         &         \\
% \midrule
% % TIP→UHDM
% \multirow{3}{*}{\begin{tabular}[c]{@{}l@{}}TIP \\ $\rightarrow$ UHDM\end{tabular}}  
% & PSNR  & 17.4332 &         & 16.8402 &         &         \\
% & SSIM  & 0.5523  &         & 0.5692  &         &         \\
% & LPIPS & 0.4987  &         & 0.4532  &         &         \\
% \midrule
% % TIP→FHDMi
% \multirow{3}{*}{\begin{tabular}[c]{@{}l@{}}TIP \\ $\rightarrow$ FHDMi\end{tabular}} 
% & PSNR  & 19.2368 &         & 19.2112 &         &         \\
% & SSIM  & 0.7354  &         & 0.7499  &         &         \\
% & LPIPS & 0.2316  &         & 0.213   &         &         \\ 
% \bottomrule[1.25pt]  % 底部粗线
% \end{tabular}
% }
% \label{tab:Exp_cross_datasets}

% \end{table}



% Please add the following required packages to your document preamble:
% \usepackage{multirow}
\begin{table*}[t]
\small
\centering
\setlength{\tabcolsep}{2.3mm}
% \scalebox{0.65}{
% \begin{tabular}{c @{\hskip +3em}c@{\hskip +3em}c@{\hskip +3em}|c@{\hskip +3em}c @{\hskip +3em} c @{\hskip +3em}c @{\hskip +3em} c@{\hskip +3em}| c @{\hskip +3em}c @{\hskip +3em}c @{\hskip +3em}c @{\hskip +3em}c}
\scalebox{0.85}{
\begin{tabular}{ccc|ccccc|ccccc}
\toprule[1.25pt]    % 顶部粗线
% \multicolumn{2}{c}{\multirow{2.5}{*}{\makecell{Cross Dataset \\ Evaluations}}} & \multirow{4}{*}{Metric} & \multicolumn{5}{c|}{Demoiréing Model: \textbf{MBCNN}} & \multicolumn{5}{c}{Demoiréing Model: \textbf{ESDNet-L}} \\ 
% \cmidrule(lr){4-8} \cmidrule(lr){9-13} 
% \multicolumn{2}{c}{} &     & \multirow{2}{*}{Baseline} & \multicolumn{4}{c|}{Moiré Image Synthesis Method} & \multirow{2}{*}{Baseline} & \multicolumn{4}{c}{Moiré Image Synthesis Method} \\ 
% \cmidrule(lr){1-2} \cmidrule(lr){5-8} \cmidrule(lr){10-13} 

% \multicolumn{1}{c}{Cross Dataset} & \multirow{2.5}{*}{Metric} & \multicolumn{5}{c|}{Demoiréing Model: \textbf{MBCNN}} & \multicolumn{5}{c}{Demoiréing Model: \textbf{ESDNet-L}} \\ 
% \cmidrule(lr){4-8} \cmidrule(lr){9-13} 
% \multicolumn{2}{c}{} &     & \multirow{2}{*}{Baseline} & \multicolumn{4}{c|}{Moiré Image Synthesis Method} & \multirow{2}{*}{Baseline} & \multicolumn{4}{c}{Moiré Image Synthesis Method} \\ 
% \cmidrule(lr){1-2} \cmidrule(lr){5-8} \cmidrule(lr){10-13} 

\multicolumn{2}{c}{Cross Dataset}& \multirow{2.5}{*}{Metric} & \multicolumn{5}{c|}{Demoiréing Network: \textbf{MBCNN}}  & \multicolumn{5}{c}{Demoiréing Network:   \textbf{ESDNet-L}}  \\ 
\cmidrule(lr){1-2} \cmidrule(lr){4-8}  \cmidrule(lr){9-13}

Source & Target &        & Baseline & Shooting & UnDeM & MoireSpace & Ours  & Baseline & Shooting & UnDeM & MoireSpace   & Ours   \\ 
\midrule
\midrule
\multirow{6.5}{*}{UHDM}    
& \multirow{3}{*}{FHDMi} &\ua{PSNR } & 19.3848 & 19.2032 & 19.4676 & 19.4531 & \textbf{19.8625} & 20.3422 & 20.2407 & 20.4014 & 20.2806 & \textbf{20.7543} \\
&                        &\ua{SSIM } & 0.7436  & 0.7459  & 0.7455  & 0.7496  & \textbf{0.7525}  & 0.7599  & 0.7579  & 0.7510  & 0.7603  & \textbf{0.7653}  \\
&                        &\da{LPIPS }& 0.3002  & 0.2975  & 0.2964  & 0.2993  & \textbf{0.2842}  & 0.2525  & 0.2632  & 0.2509  & 0.2324  & \textbf{0.2136}  \\ 
\cmidrule{2-13} 
& \multirow{3}{*}{TIP}   &\ua{PSNR } & 17.8107 & 18.3730 & 18.6674 & 18.9214 & \textbf{19.3922} & 18.8040 & 18.4543 & 19.3545 & 19.3964 & \textbf{19.5009} \\
&                        &\ua{SSIM } & 0.6627  & 0.6888  & 0.6911  & 0.6996  & \textbf{0.7022}  & 0.6921  & 0.6930  & 0.6998  & 0.7111  & \textbf{0.7149}  \\
&                        &\da{LPIPS }& 0.3580  & 0.3886  & 0.3909  & 0.3829  & \textbf{0.3781}  & 0.3524  & 0.3849  & 0.3601  & 0.3522  & \textbf{0.3495}  \\ 
\midrule
\multirow{6.5}{*}{FHDMi}   
& \multirow{3}{*}{UHDM}  &\ua{PSNR } & 17.1331 & 17.5326 & 17.4870 & 17.6050 & \textbf{18.7931} & 18.0049 & 18.4189 & 17.9574 & 17.9751 & \textbf{18.9240} \\
&                        &\ua{SSIM } & 0.6159  & 0.6334  & 0.6331  & 0.6642  & \textbf{0.7186}  & 0.5755  & 0.5780  & 0.5857  & 0.5548  & \textbf{0.6658}  \\
&                        &\da{LPIPS }& 0.4470  & 0.4350  & 0.4285  & 0.4020  & \textbf{0.3508}  & 0.4420  & 0.4279  & 0.4460  & 0.4579  & \textbf{0.3405}  \\ 
\cmidrule{2-13} 
& \multirow{3}{*}{TIP}   &\ua{PSNR } & 20.2161 & 20.7793 & 20.8261 & 20.1194 & \textbf{21.0694} & 20.6647 & 20.8678 & 20.4663 & 20.8107 & \textbf{21.5786} \\
&                        &\ua{SSIM } & 0.7340  & 0.7304  & 0.7381  & 0.7347  & \textbf{0.7494}  & 0.7504  & 0.7606  & 0.7278  & 0.7582  & \textbf{0.7668}  \\
&                        &\da{LPIPS }& 0.2979  & 0.2884  & 0.2891  & 0.2961  & \textbf{0.2832}  & 0.2459  & 0.2450  & 0.2998  & 0.2468  & \textbf{0.2310}  \\ 
\midrule
\multirow{6.5}{*}{TIP}     
& \multirow{3}{*}{UHDM}  &\ua{PSNR } & 17.3409 & 17.4011 & 17.4407 & 17.4987 & \textbf{18.2937} & 17.4332 & 16.1836 & 16.8402 & 16.6296 & \textbf{18.4978} \\
&                        &\ua{SSIM } & 0.6144  & 0.6062  & 0.6066  & 0.6059  & \textbf{0.6913}  & 0.5523  & 0.5511  & 0.5692  & 0.5748  & \textbf{0.6866}  \\
&                        &\da{LPIPS }& 0.4726  & 0.4487  & 0.4473  & 0.4412  & \textbf{0.3990}  & 0.4987  & 0.4723  & 0.4532  & 0.4387  & \textbf{0.3231}  \\ 
\cmidrule{2-13} 
& \multirow{3}{*}{FHDMi} &\ua{PSNR } & 18.9458 & 19.2731 & 19.0336 & 19.1101 & \textbf{20.1053} & 19.2368 & 18.1936 & 19.2112 & 18.8385 & \textbf{19.9971} \\
&                        &\ua{SSIM } & 0.7369  & 0.7399  & 0.7215  & 0.7321  & \textbf{0.7725}  & 0.7354  & 0.7297  & 0.7499  & 0.7389  & \textbf{0.7580}  \\
&                        &\da{LPIPS }& 0.2494  & 0.2447  & 0.2452  & 0.2382  & \textbf{0.2315}  & 0.2316  & 0.2320  & 0.2130  & 0.2228  & \textbf{0.1915}  \\ 
\bottomrule[1.25pt]  % 底部粗线
\end{tabular}
}
\caption{Quantitative results of cross-dataset evaluations.}
\vspace{-2ex}
\label{tab:Exp_cross_datasets}
\end{table*}


















































































 %% Cross-dataset 实验
\begin{table*}[t]
\small
\centering
\setlength{\tabcolsep}{1.89mm}
\scalebox{0.9}{
\begin{tabular}{ccc|ccccc|ccccc}
\toprule[1.25pt]    % 顶部粗线

\multicolumn{2}{c}{Cross Dataset}& \multirow{2.5}{*}{Metric} & \multicolumn{5}{c|}{Demoiréing Network: \textbf{MBCNN}}  & \multicolumn{5}{c}{Demoiréing Network:   \textbf{ESDNet-L}}  \\ 
\cmidrule(lr){1-2} \cmidrule(lr){4-8}  \cmidrule(lr){9-13}

Source & Target &        & Baseline & Shooting & UnDeM & MoireSpace & Ours  & Baseline & Shooting & UnDeM & MoireSpace   & Ours   \\ 
\midrule
\midrule
\multirow{6.5}{*}{UHDM}    
& \multirow{3}{*}{FHDMi} &\ua{PSNR } & 19.3848 & 19.2032 & 19.4676 & 19.4531 & \textbf{19.8625} & 20.3422 & 20.2407 & 20.4014 & 20.2806 & \textbf{20.7543} \\
&                        &\ua{SSIM } & 0.7436  & 0.7459  & 0.7455  & 0.7496  & \textbf{0.7525}  & 0.7599  & 0.7579  & 0.7510  & 0.7603  & \textbf{0.7653}  \\
&                        &\da{LPIPS }& 0.3002  & 0.2975  & 0.2964  & 0.2993  & \textbf{0.2842}  & 0.2525  & 0.2632  & 0.2509  & 0.2324  & \textbf{0.2136}  \\ 
\cmidrule{2-13} 
& \multirow{3}{*}{TIP}   &\ua{PSNR } & 17.8107 & 18.3730 & 18.6674 & 18.9214 & \textbf{19.3922} & 18.8040 & 18.4543 & 19.3545 & 19.3964 & \textbf{19.5009} \\
&                        &\ua{SSIM } & 0.6627  & 0.6888  & 0.6911  & 0.6996  & \textbf{0.7022}  & 0.6921  & 0.6930  & 0.6998  & 0.7111  & \textbf{0.7149}  \\
&                        &\da{LPIPS }& 0.3580  & 0.3886  & 0.3909  & 0.3829  & \textbf{0.3781}  & 0.3524  & 0.3849  & 0.3601  & 0.3522  & \textbf{0.3495}  \\ 
\midrule
\multirow{6.5}{*}{FHDMi}   
& \multirow{3}{*}{UHDM}  &\ua{PSNR } & 17.1331 & 17.5326 & 17.4870 & 17.6050 & \textbf{18.7931} & 18.0049 & 18.4189 & 17.9574 & 17.9751 & \textbf{18.9240} \\
&                        &\ua{SSIM } & 0.6159  & 0.6334  & 0.6331  & 0.6642  & \textbf{0.7186}  & 0.5755  & 0.5780  & 0.5857  & 0.5548  & \textbf{0.6658}  \\
&                        &\da{LPIPS }& 0.4470  & 0.4350  & 0.4285  & 0.4020  & \textbf{0.3508}  & 0.4420  & 0.4279  & 0.4460  & 0.4579  & \textbf{0.3405}  \\ 
\cmidrule{2-13} 
& \multirow{3}{*}{TIP}   &\ua{PSNR } & 20.2161 & 20.7793 & 20.8261 & 20.1194 & \textbf{21.0694} & 20.6647 & 20.8678 & 20.4663 & 20.8107 & \textbf{21.5786} \\
&                        &\ua{SSIM } & 0.7340  & 0.7304  & 0.7381  & 0.7347  & \textbf{0.7494}  & 0.7504  & 0.7606  & 0.7278  & 0.7582  & \textbf{0.7668}  \\
&                        &\da{LPIPS }& 0.2979  & 0.2884  & 0.2891  & 0.2961  & \textbf{0.2832}  & 0.2459  & 0.2450  & 0.2998  & 0.2468  & \textbf{0.2310}  \\ 
\midrule
\multirow{6.5}{*}{TIP}     
& \multirow{3}{*}{UHDM}  &\ua{PSNR } & 17.3409 & 17.4011 & 17.4407 & 17.4987 & \textbf{18.2937} & 17.4332 & 16.1836 & 16.8402 & 16.6296 & \textbf{18.4978} \\
&                        &\ua{SSIM } & 0.6144  & 0.6062  & 0.6066  & 0.6059  & \textbf{0.6913}  & 0.5523  & 0.5511  & 0.5692  & 0.5748  & \textbf{0.6866}  \\
&                        &\da{LPIPS }& 0.4726  & 0.4487  & 0.4473  & 0.4412  & \textbf{0.3990}  & 0.4987  & 0.4723  & 0.4532  & 0.4387  & \textbf{0.3231}  \\ 
\cmidrule{2-13} 
& \multirow{3}{*}{FHDMi} &\ua{PSNR } & 18.9458 & 19.2731 & 19.0336 & 19.1101 & \textbf{20.1053} & 19.2368 & 18.1936 & 19.2112 & 18.8385 & \textbf{19.9971} \\
&                        &\ua{SSIM } & 0.7369  & 0.7399  & 0.7215  & 0.7321  & \textbf{0.7725}  & 0.7354  & 0.7297  & 0.7499  & 0.7389  & \textbf{0.7580}  \\
&                        &\da{LPIPS }& 0.2494  & 0.2447  & 0.2452  & 0.2382  & \textbf{0.2315}  & 0.2316  & 0.2320  & 0.2130  & 0.2228  & \textbf{0.1915}  \\ 
\bottomrule[1.25pt]  % 底部粗线
\end{tabular}
}
\caption{Quantitative results of cross-dataset evaluations.}
% \vspace{-2ex}
\label{tab:Exp_cross_datasets}
\end{table*}


 % \vspace{-1.5ex}
\subsubsection{Loss Functions}
The tone adjustment network aims to adjust the overall color tone and contrast of $I_{trn}$ in a way that it resembles $I_{rm}$ without affecting moiré pattern $I_{mp}$. 

First, moiré patterns can disrupt image structures by generating strip-shaped artifacts~\cite{yu2022towards}. Therefore, comparing two moiré images directly in pixel space is less effective. 
Thus, we adopt the perceptual loss~\cite{johnson2016perceptual} $\mathcal{L}_{per}$ to 
%ensure the moiré artifacts of the $I_{trn}$ are aligned with the $I_{mib}$ by 
optimize the $\mathcal{L}_1$ distance between the extracted content features of $I_{mib}$ and $I_{trn}$:
\begin{small}
\begin{equation}
\mathcal{L}_{per}(I_{trn}, I_{mib})=\sum_{j=1}^{N_L}\frac{\left\|\phi_j(I_{trn})-\phi_j(I_{mib})\right\|_1}{C_j H_j W_j}, 
\end{equation}
\end{small}
\normalsize 
where $\phi_j(I)$ is the activations of the $j$-th layer of the VGG16 network~\cite{simonyan2014very}, and $N_L$ represents the number of convolutional layers in VGG16.

In addition, to effectively tune the tone of $I_{trn}$, we take advantage of color differentiable RGB-uv histogram features $H(I_{trn})$ and $H(I_{rm})$ in log chromaticity space, inspired by color constancy method~\cite{barron2015ccc, Afifi2021histogan}, 
%and transform $I_{trn}$ and $I_{rm}$ into color differentiable histogram features $H(I_{trn})$ and $H(I_{rm})$ in log chromaticity space, 
as shown in Figure~\ref{fig:systhesis} (a). Such RGB-uv histograms have proven efficient in color transfer tasks~\cite{Eibenberger2012log}. 
We optimize color loss 
using the differentiable Hellinger distances 
%Then, we follow HistoGAN~\cite{Afifi2021histogan} to optimize the RGB-uv histogram features $H(I_{trn})$ and $H(I_{rm})$ to obtain the color loss using the differentiable Hellinger distances:
\begin{equation}
    \mathcal{L}_{color}\left(I_{trn}, I_{rm}\right)=\left\|H(I_{trn})^{1 / 2}-H(I_{rm})^{1 / 2}\right\|_2,
\end{equation}
where $\left\| \cdot \right\|_2$ is the standard Euclidean norm and $\cdot ^{1/2}$ is an element-wise square root. 
%More details of color loss and RGB-uv histogram features can be found in the appendix.

Finally, 
%following prior work on style transfer~\cite{johnson2016perceptual}, and image blending~\cite{Zhang2020deepimageblend}, 
we use total variation regularizer $\mathcal{L}_{tv}$ to remove unwanted details while encouraging spatial smoothness: %in the output image $I_{trn}$:
\begin{small}
\begin{equation}
\mathcal{L}_{tv}(I_{trn})=\sum_{i=1}^H \sum_{j=1}^W\left|I_{trn}^{i+1, j}-I_{trn}^{i, j}\right|+\left|I_{trn}^{i, j+1}-I_{trn}^{i, j}\right|
\end{equation}
\end{small}
\normalsize


Total loss $\mathcal{L}$ is then defined as a weighted compound of $\mathcal{L}_{per}$, $\mathcal{L}_{color}$ and $\mathcal{L}_{tv}$:
\begin{equation}
    \mathcal{L} = \lambda_{per}\mathcal{L}_{per} + \lambda_{color}\mathcal{L}_{color} + \lambda_{tv}\mathcal{L}_{tv}.
\end{equation}

\subsection{Image Demoiréing}
\label{subsec: Image_Dmoiréing}
Our contributions mainly lie in the above three stages. %establishing a novel moiré pattern dataset, developing a moiré pattern generator, and proposing the moiré image synthesis strategy.
Then, diverse and realistic-looking data synthesized by our solution can be seamlessly integrated with demoiréing models to improve their performance.


%% Experiments
% 


\section{Experiments}
\seclabel{experiments}
Our experiments are designed to test a) the extent to which open loop execution is an issue for precise mobile manipulation tasks, b) how effective are blind proprioceptive correction techniques, c) do object detectors and point trackers perform reliably enough in wrist camera images for reliable control, d) is occlusion by the end-effector an issue and how effectively can it be mitigated through the use of video in-painting models, and e) how does our proposed \name methodology compare to large-scale imitation learning? 


\subsection{Tasks and Experimental Setup}
We work with the Stretch RE2 robot. Stretch RE2 is a commodity mobile manipulator with a 5DOF arm mounted on top of a non-holomonic base. We upgrade the robot to use the Dex Wrist 3, which has an eye-in-hand RGB-D camera (Intel D405). 
We consider 3 task families for a total
of 6 different tasks: a) holding a knob to pull open a cabinet or drawer, b) holding a
handle to pull open a cabinet, and c) pushing on objects (light buttons, books
in a book shelf, and light switches). Our focus is on generalization. {\it
Therefore, we exclusively test on previously unseen instances, not used during
development in any way.} 
\figref{tasks} shows the instances that we test on. 

All tasks involve some precise manipulation, followed by execution of a motion
primitive. {\bf For the pushing tasks}, the precise motion is to get the
end-effector exactly at the indicated point and the motion primitive is to push
in the direction perpendicular to the surface and retract the end-effector 
upon contact. The robot is positioned such
that the target position is within the field of view of the wrist camera. A user
selects the point of pushing via a mouse click on the wrist camera image. The
goal is to push at the indicated location. Success is determined by whether the
push results in the desired outcome (light turns on / off or book gets pushed in). 
The original rubber gripper bends upon contact, we use a rigid known tool
that sticks out a bit. We take the geometry of the tool into account while servoing.

{\bf For the opening articulated object tasks}, the precise manipulation is grasping the
knob / handle, while the motion primitive is the whole-body motion that opens
the cupboard. Computing and executing this full body motion is difficult. We
adopt the modular approach to opening articulated objects (MOSART) from Gupta \etal~\cite{gupta2024opening} and invoke it
after the gripper has been placed around the knob / handle. The whole tasks 
starts out with the robot about 1.5m way from the target object, with the 
target object in view
from robot's head mounted camera. We use MOSART to compute articulation
parameters and convey the robot to a pre-grasp
location with the target handle in view of the wrist camera. At this point,
\name (or baseline) is used to center the gripper around the knob / handle, 
before resuming MOSART: extending the gripper till contact, close the gripper, and play rest of the predicted motion plan. Success is 
determined by whether the cabinet opens by more than $60^\circ$
or the drawer is pulled out by more than $24cm$, similar to the criteria used in \cite{gupta2024opening}.


For the precise manipulation part, all baselines consume the current and
previous RGB-D images from the wrist camera and output full body motor
commands.

% % Please add the following required packages to your document preamble:
% % \usepackage{graphicx}
% \begin{table*}[!ht]
% \centering
% \caption{}
% \label{tab:my-table}
% \resizebox{\textwidth}{!}{%
% \begin{tabular}{lcccccc}
% \toprule
%  & \multicolumn{2}{c}{ours} & \multicolumn{2}{c}{Gurobi} & \multicolumn{2}{c}{MOSEK} \\
%  & \multicolumn{1}{l}{time (s)} & \multicolumn{1}{l}{optimality gap (\%)} & \multicolumn{1}{l}{time (s)} & \multicolumn{1}{l}{optimality gap (\%)} & \multicolumn{1}{l}{time (s)} & \multicolumn{1}{l}{optimality gap (\%)} \\ \hline
% \begin{tabular}[c]{@{}l@{}}Linear Regression\\ Synthetic \\ (n=16000, p=16000)\end{tabular} & 57 & 0.0 & 3351 & - & 2148 & - \\ \hline
% \begin{tabular}[c]{@{}l@{}}Linear Regression\\ Cancer Drug Response\\ (n=822, p=2300)\end{tabular} & 47 & 0.0 & 1800 & 0.31 & 212 & 0.0 \\ \hline
% \begin{tabular}[c]{@{}l@{}}Logistic Regression\\ Synthetic\\ (n=16000, p=16000)\end{tabular} & 271 & 0.0 & N/A & N/A & 1800 & - \\ \hline
% \begin{tabular}[c]{@{}l@{}}Logistic Regression\\ Dorothea\\ (n=1150, p=91598)\end{tabular} & 62 & 0.0 & N/A & N/A & 600 & 0.0 \\
% \bottomrule
% \end{tabular}%
% }
% \end{table*}

% Please add the following required packages to your document preamble:
% \usepackage{multirow}
% \usepackage{graphicx}
\begin{table*}[]
\centering
\caption{Certifying optimality on large-scale and real-world datasets.}
\vspace{2mm}
\label{tab:my-table}
\resizebox{\textwidth}{!}{%
\begin{tabular}{llcccccc}
\toprule
 &  & \multicolumn{2}{c}{ours} & \multicolumn{2}{c}{Gurobi} & \multicolumn{2}{c}{MOSEK} \\
 &  & time (s) & opt. gap (\%) & time (s) & opt. gap (\%) & time (s) & opt. gap (\%) \\ \hline
\multirow{2}{*}{Linear Regression} & \begin{tabular}[c]{@{}l@{}}synthetic ($k=10, M=2$)\\ (n=16k, p=16k, seed=0)\end{tabular} & 79 & 0.0 & 1800 & - & 1915 & - \\ \cline{2-8}
 & \begin{tabular}[c]{@{}l@{}}Cancer Drug Response ($k=5, M=5$)\\ (n=822, p=2300)\end{tabular} & 41 & 0.0 & 1800 & 0.89 & 188 & 0.0 \\ \hline
\multirow{2}{*}{Logistic Regression} & \begin{tabular}[c]{@{}l@{}}Synthetic ($k=10, M=2$)\\ (n=16k, p=16k, seed=0)\end{tabular} & 626 & 0.0 & N/A & N/A & 2446 & - \\ \cline{2-8}
 & \begin{tabular}[c]{@{}l@{}}DOROTHEA ($k=15, M=2$)\\ (n=1150, p=91598)\end{tabular} & 91 & 0.0 & N/A & N/A & 634 & 0.0 \\
 \bottomrule
\end{tabular}%
}
% \vspace{-3mm}
\end{table*}

\begin{figure*}
\insertW{1.0}{figures/figure_6_cropped_brighten.pdf}
\caption{{\bf Comparison of \name with the open loop (eye-in-hand) baseline} for opening a cabinet with a knob. Slight errors in getting to the target cause the end-effector to slip off, leading to failure for the baseline, where as our method is able to successfully complete the task.}
\figlabel{rollout}
\end{figure*}

\begin{table}
\setlength{\tabcolsep}{8pt}
  \centering
  \resizebox{\linewidth}{!}{
  \begin{tabular}{lcccg}
  \toprule
                              & \multicolumn{2}{c}{\bf Knobs} & \bf Handle & \bf \multirow{2}{*}{\bf Total} \\
                              \cmidrule(lr){2-3} \cmidrule(lr){4-4}
                              & \bf Cabinets & \bf Drawer & \bf Cabinets & \\
  \midrule
  RUM~\cite{etukuru2024robot}  & 0/3    & 1/4         & 1/3         & 2/10 \\
  \name (Ours) & 2/3    & 2/4         & 3/3     &  7/10 \\
  \bottomrule
  \end{tabular}}
  \caption{Comparison of \name \vs RUM~\cite{etukuru2024robot}, a recent large-scale end-to-end imitation learning method trained on 1200 demos for opening cabinets and 525 demos for opening drawers across 40 different environments. Our evaluation spans objects from three environments across two buildings.}
  \tablelabel{rum}
\end{table}

\subsection{Baselines}
We compare against three other methods for the precise manipulation part of
these tasks. 
\subsubsection{Open Loop (Eye-in-Hand)} To assess the precision requirements of
the tasks and to set it in context with the manipulation capabilities of the
robot platform, this baseline uses open loop execution starting from estimates
for the 3D target position from the first wrist camera image.
\subsubsection{MOSART~\cite{gupta2024opening}}
The recent modular system for opening cabinets and drawers~\cite{gupta2024opening}
reports impressive performance with open-loop control (using the head camera from 1.5m away), combined with proprioception-based feedback to 
compensate for errors in perception and control when interacting with handles. 
We test if such correction is also sufficient for interacting with knobs. Note 
that such correction is not possible for the smaller buttons and pliable books.

\subsubsection{\name (no inpainting)} To understand how much of an issue
occlusion due to the end-effector is during manipulation, we ablate the use of
inpainting. %

\subsubsection{Robot Utility Models (RUM)~\cite{etukuru2024robot}}
For the opening articulated object tasks, we also compare to Robot Utility Models (RUM), 
a closed-loop imitation learning method recently proposed by Etukuru et al. \cite{etukuru2024robot}.
RUM is trained on a substantial dataset comprising expert demonstrations, including 
1,200 instances of cabinet opening and 525 of drawer opening, gathered from roughly 
40 different environments.
This dataset stands as the most extensive imitation 
learning dataset for articulated object manipulation to date, establishing RUM as a 
strong baseline for our evaluation.

Similar to our method, we use MOSART to compute articulation
parameters and convey the robot to a pre-grasp location
with the target handle in view of the wrist camera.
One of the assumptions of RUM is a good view of the handle.
To benefit RUM, we try out three different heights of the wrist camera,
and \textit{report the best result for RUM.}

\begin{figure*}
\insertW{1.0}{figures/figure_9_cropped_brighten.pdf}
\caption{{\bf \name \vs open loop (eye-in-hand) baseline for pushing on user-clicked points}. Slight errors in getting to the target cause failure, where as \name successfully turns the lights off. Note the quality of CoTracker's track ({\color{blue} blue dot}).}
\figlabel{rollout_v2}
\end{figure*}

\begin{figure*}
\insertW{1.0}{figures/figure_5_v2_cropped_brighten.pdf}
\caption{{\bf Comparison of \name with and without inpainting}. Erroneous detection without inpainting causes execution to fail, where as with inpainting the target is correctly detected leading to a successful grasp and a successful execution.}
\figlabel{rollouts2}
\end{figure*}


\subsection{Results}
\tableref{results} presents results from our experiments. 
Our training-free approach \name successfully 
solves over 85\% of task instances that we test on.
As noted, all these
tests were conducted on unseen object instances in unseen
environments that were not used for development in any way. We discuss our key
experimental findings below.

\subsubsection{Closing the loop is necessary for these precise tasks} 
While the proprioception-based strategies proposed in MOSART~\cite{gupta2024opening}
work out for handles, they are inadequate for targets like knobs and just
don't work for tasks like pushing buttons. Using estimates from the wrist
camera is better, but open loop execution still fails for knobs and pushing
buttons. 

\subsubsection{Vision models work reasonably well even on wrist camera images}
Inpainting works well on wrist camera images (see \figref{occlusion} and \figref{inpainting}).
Closing the loop using feedback from vision detectors and point trackers on
wrist camera images also work well, particularly when we use in-painted images.
See some examples detections and point tracks in \figref{rollout} and \figref{rollout_v2}. 
Detic~\cite{zhou2022detecting} was able to reliably detect the knobs and
handles and CoTracker~\cite{karaev2023cotracker} was able to successfully track
the point of interaction letting us solve 24/28 task instances.

\subsubsection{Erroneous detections without inpainting hamper performance on 
handles and our end-effector out-painting strategy effectively mitigates it} 
As shown in \figref{rollouts2}, presence of the end-effector caused the object
detector to miss fire leading to failed execution. Our out painting approach
mitigates this issue leading to a higher success rate than the 
approach without out-painting. Interestingly, CoTracker~\cite{karaev2023cotracker} is quite robust
to occlusion (possibly because it tracks multiple points) and doesn't benefit
from in-painting. 


\subsubsection{Closed-loop imitation learning struggles on novel objects}
As presented in \tableref{rum}, \name significantly outperforms RUM in a paired evaluation on unseen objects across three novel environments. A common failure mode of RUM is its inability to grasp the object's handle, even when it approaches it closely.
Another failure mode we observe is RUM misidentifying keyholes or cabinet edges as handles, also resulting in failed grasp attempts.
These result demonstrate that a modular approach that leverages the broad generalization capabilities of vision foundation models is able to generalize much better than an end-to-end imitation learning approach trained on 1000+ demonstrations, which must learn all aspects of the task from scratch.



\section{Experiments}
\subsection{Experimental Setups} 
For all compared methods, we used their released code. Thorough implementation details are in the appendix. 

\begin{figure}[t]
  \centering
\includegraphics[width=1.0\linewidth]{Images/Demoire_Result.pdf}
  \caption{Comparisons of demoiréing results.} 
  \label{fig:demoiré_result}
  % \vspace{-3ex}
\end{figure}

\subsubsection{Datasets and Metrics.}
\textbf{\emph{1) }Moiré Pattern Dataset} is used to train our moiré pattern generator. 
\textbf{\emph{2) }Real Moiré Image Dataset}, TIP~\cite{sun2018moire}, FHDMi~\cite{he2020fhde}, and UHDM~\cite{yu2022towards}, are used to demonstrate our ability in restoring real moiré images. 
\textbf{\emph{3) }Evaluation Metrics.}  We evaluate demoiréing performance on the Peak-Signal-to-Noise Ratio (PSNR), Structural Similarity Index (SSIM)~\cite{wang2004image}, and LPIPS~\cite{zhang2018unreasonable}.

\subsubsection{Comparison Methods}
We compare UniDemoiré to the SOTA synthesis methods in 3 current modalities: the simulation method ``Shooting"~\cite{shooting}, the implicit moiré synthesis approach ``UnDeM"~\cite{undem}, which employs a neural network, and the explicit synthesis method termed ``MoireSpace"~\cite{yang2023doing}, which utilizes its moiré pattern dataset.

\subsubsection{Demoiréing Models}
We test on the most effective SOTA demoiréing models,  MBCNN~\cite{zheng2020image} and ESDNet-L~\cite{yu2022towards}.

\subsection{Zero-Shot Demoiréing with Synthesized Data Only} 
We first demonstrate demoiréing results on real moiré images trained on purely synthesized data by SOTA moiré synthesis methods. To avoid data overlap in training sets and test sets, we have collected a comprehensive Mixed High-Resolution Natural Image Dataset (MHRNID), based on which, moiré images are synthesized for training demoiréing models. 
Quantitative comparisons can be found in Table~\ref{tab:Exp_multi_datasets}. Visual comparisons on demoiréing real data in UHDM are illustrated in Figure~\ref{fig:demoiré_result}.
Due to that UnDeM relies on existing moiré images in both the training (fusion networks) and inference phase, %and our method also requires real moiré images to guide the tone refinement network, 
we trained their networks on the TIP dataset and showed the result of UnDeM using the real moiré in the TIP dataset (``$\dagger$" in Table  \ref{tab:Exp_multi_datasets}) and our sampled moiré pattern (``$\ddagger$" in Table \ref{tab:Exp_multi_datasets}) during inference, respectively.
% For a fair comparison, we also use real moiré images from TIP dataset to train our tone refinement network.
For a fair comparison, we also use real moiré images from TIP dataset to train our TRN.
Notice that UnDeM and our method only use real moiré images to guide the synthesis, and neither of us uses such real data to train demoiréing models directly.

From the quantitative perspective (Table~\ref{tab:Exp_multi_datasets}), our method substantially outperforms all other approaches, particularly by more than 3.2 dB and 2.0 dB for MBCNN and ESDNet-L on the UHDM dataset,  respectively.
Besides, UnDeM$^\ddagger$ using our generated moiré patterns outperforms UnDeM$^\dagger$ using real moiré patterns in all experiments, proving our effectiveness further. 
From the qualitative perspective (Figure~\ref{fig:demoiré_result}),  our method demonstrates strong capability even when images in the target domain are contaminated by severe moiré patterns, which other synthesis methods fail to address.
We attribute our superiority to the diversity and realism of our synthetic data. Such high-quality data by our UniDemoiré enables the demoiréing model to learn moiré characteristics better, improving performance in removing unseen moiré artifacts. 
More visual results are in the appendix.

\subsection{Cross-Dataset Evaluation} 
We then demonstrate our ability to improve the performance of demoiréing models across domains. 
Quantitative results are shown in Table ~\ref{tab:Exp_cross_datasets}. Note that ``Baseline" means that the demoiréing models (MBCNN and ESDNet-L) are trained with the original source real moiré datasets and tested on the target dataset. 
For each synthesis approach, a demoiréing model is trained with combined original real data in the source dataset and corresponding synthesized data. 

As shown, the Shooting method struggles with real data due to differences between synthetic and real moiré. 
% UnDeM relies on a GAN-based network but can be unscrossdata and quality-dependent. 
UnDeM relies on a GAN network but can be inconsistent depending on the dataset and quality. 
The MoireSpace method performs better than UnDeM but has inferior moiré patterns and synthesis quality, resulting in lower experimental metrics. 
Thanks to the realistic and diverse synthesized data, our method outperforms all previous methods across every experiment.
Visual comparisons in Figure~\ref{fig:demoiré_result} (lower, Source: UHDM, Target: FHDMi) demonstrate our effectiveness. 
% More visualizations are in the appendix.



%% Ablation table
% 
% \begin{table}[h]
% \centering
% \setlength{\tabcolsep}{2.5mm}
% \scalebox{0.9}{
% \begin{tabular}{lccc}
% \toprule
% \multirow{2.5}{*}{Components}  & \multicolumn{3}{c}{Test set: FHDMi}  \\ 
%                              \cmidrule(l){2-4} 
%                              &\ua{PSNR }   &\ua{SSIM } &\da{LPIPS } \\ 
% \midrule
% \multicolumn{4}{l}{\textit{Using:   \textbf{ESDNet-L}, trained with \textbf{UHDM}, 50 Epochs}} \\ 
% \midrule
% ALL                          & \textbf{12.3456} & \textbf{0.1234} & \textbf{0.1234} \\
% $w/o$ MPG                    &                  &                 &                 \\
% $w/o$ TRN                    &                  &                 &                 \\
% TRN ($w/o$ $\mathcal{L}_{per}$)    &                  &                 &                 \\
% TRN ($w/o$ $\mathcal{L}_{color}$)  &                  &                 &                 \\
% TRN ($w/o$ $\mathcal{L}_{tv}$)     &                  &                 &                 \\
% TRN ($w/o$ fusion block)           &                  &                 &                 \\ 
% \bottomrule
% \end{tabular}
% }
% \caption{Ablation studies. UHDM is the source dataset for training, while FHDMi is the target dataset for testing.}
% \label{tab:Exp_ablation}
% \end{table}


\begin{table}[h]
\centering
\setlength{\tabcolsep}{2.8mm}
\scalebox{0.8}{
\begin{tabular}{l  @{\hskip +3em} c  @{\hskip +3em} c  @{\hskip +3em} c}
%\scalebox{0.9}{
%\begin{tabular}{lccc}
\toprule
% \multirow{2.5}{*}{Components}  & \multicolumn{3}{c}{Test set: FHDMi}  \\ \cmidrule(l){2-4} 
Components                     &\ua{PSNR}   &\ua{SSIM} &\da{LPIPS} \\ 
% \midrule
% \multicolumn{4}{l}{\textit{Using:   \textbf{ESDNet-L}, trained with \textbf{UHDM}, 50 Epochs}} \\ 
\midrule
ALL                          & \textbf{20.7543} & \textbf{0.7653} & \textbf{0.2136} \\

$w/o$ MPG                    & 20.1607          & 0.7326          & 0.2456          \\
$w/o$ TRN                    & 20.1691          & 0.7372          & 0.2544          \\
TRN ($w/o$ $\mathcal{L}_{per}$)    & 20.3076          & 0.7508          & 0.2558          \\
TRN ($w/o$ $\mathcal{L}_{color}$)  & 20.2692          & 0.7406          & 0.2301          \\
TRN ($w/o$ $\mathcal{L}_{tv}$)     & 20.3961          & 0.7451          & 0.2324          \\
TRN ($w/o$ fusion block)           & 20.2868          & 0.7370          & 0.2311          \\ 
\bottomrule
\end{tabular}
}
\caption{Ablation studies. Source: UHDM, Target: FHDMi.}
\vspace{-2.5ex}
\label{tab:Exp_ablation}
\end{table}

		
		
		
		
		
	  	

 
\begin{table}[t]
\centering
\small
\setlength{\tabcolsep}{2mm}
% \scalebox{0.9}{
\begin{tabular}{lccc}
\toprule
Components                     &\ua{PSNR}   &\ua{SSIM} &\da{LPIPS} \\ 
\midrule
ALL                          & \textbf{20.7543} & \textbf{0.7653} & \textbf{0.2136} \\

$w/o$ MPG                    & 20.1607          & 0.7326          & 0.2456          \\
$w/o$ TRN                    & 20.1691          & 0.7372          & 0.2544          \\
TRN ($w/o$ $\mathcal{L}_{per}$)    & 20.3076          & 0.7508          & 0.2558          \\
TRN ($w/o$ $\mathcal{L}_{color}$)  & 20.2692          & 0.7406          & 0.2301          \\
TRN ($w/o$ $\mathcal{L}_{tv}$)     & 20.3961          & 0.7451          & 0.2324          \\
TRN ($w/o$ fusion block)           & 20.2868          & 0.7370          & 0.2311          \\ 
\bottomrule
\end{tabular}
% }
\caption{Ablation studies. Source: UHDM, Target: FHDMi.}
% \vspace{-2.5ex}
\label{tab:Exp_ablation}
\end{table}


\subsection{Ablation Study}
We individually ablate submodules in our proposed method to analyze their contribution. All these experiments are trained with the UHDM dataset and validated on the FHDMi dataset. Experimental results in Table~\ref{tab:Exp_ablation} verify that all components in our UniDemoiré solution are crucial for achieving the desired demoiréing performance. Removing any component such as the Moiré Pattern Generator (MPG), Tone Refinement Network (TRN), loss functions, and feature fusion block leads to a significant performance decline. 
More ablation studies are provided in the appendix.

% Conclusion
% \label{sec::end}
With this work, we presented an improved version of the original dataset proposed by Angelopoulos in~\cite{angelopoulos2020event}, together with the pipeline used to generate improved annotations at 200Hz, providing what we believe could be a valuable resource for the scientific community.

The semi-automatic pipeline presented provides a method that reduces both the time required to annotate an event-camera dataset and the potential human error that might arise from a completely manual annotation process. This method can also be applied to future datasets, giving more precise annotations that can help develop and test new algorithms for event-based eye-tracking.

Simultaneously, with the increasing interest in event-based technologies, particularly in low-power applications, we aim to expedite research in real-time eye-tracking using purely event-driven approaches.


\begin{comment}
\[
\mathbf{X} =
\begin{bmatrix}
x \\
y \\
\dot{x} \\
\dot{y}
\end{bmatrix}
\]

\[
\mathbf{F} =
\begin{bmatrix}
1 &  0 & \Delta t & 0 \\
0 & 1 & 0 & \Delta_t\\
0 & 0 & 1 & 0\\
0 & 0 & 0 & 1
\end{bmatrix}
\]

\[
\mathbf{H} =
\begin{bmatrix}
1 & 0 & 0 & 0\\
0 & 1 & 0 & 0\\
\end{bmatrix}
\]

\[
\mathbf{Q} =
\begin{bmatrix}
1 & 0 & 0 & 0\\
0 & 1 & 0 & 0\\
0 & 0 & 1 & 0\\
0 & 0 & 0 & 1
\end{bmatrix}
\]

\[
\mathbf{R} =
\begin{bmatrix}
1 & 0\\
0 & 1\\
\end{bmatrix}
\]

\[
\mathbf{B} = 0
\]
\end{comment}
\section{Conclusion}
By addressing the issue of data diversity and realism, our universal solution, UniDemoiré, tackles one of the most important bottlenecks in image demoiréing problems. It showcases significant performance in zero-shot demoiréing and demonstrates a strong capability of enhancing the cross-domain performance of existing demoiréing models. 
More importantly, our method holds the potential to generate billions of moiré data and to significantly expand demoiréing models with a vast increase in parameters. 
Our limitations are discussed in the appendix.


% Acknowledgment
\section{Acknowledgments}
This work is supported by NSFC (No.62206173), Shanghai Frontiers Science Center of Human-centered Artificial Intelligence (ShangHAI), MoE Key Laboratory of Intelligent Perception and Human-Machine Collaboration (KLIP-HuMaCo).
This work is also partially supported by HKU-SCF FinTech Academy,
HKRGC Theme-based research scheme project T35-710/20-R, and SZ-HK-Macau Technology Research Programme \#SGDX20210823103537030.

\bibliography{aaai25}

% \newpage
% \clearpage
% \fbox{\begin{minipage}{38em}

\subsubsection*{Scaling Law Reproducilibility Checklist}\label{sec:checklist}


\small

\begin{minipage}[t]{0.48\textwidth}
\raggedright
\paragraph{Scaling Law Hypothesis (\S\ref{sec:power-law-form})}

\begin{itemize}[leftmargin=*]
    \item What is the form of the power law?
    \item What are the variables related by (included in) the power law?
    \item What are the parameters to fit?
    \item On what principles is this form derived?
    \item Does this form make assumptions about how the variables are related?
    % \item How are each of these variables counted? (For example, how is compute cost/FLOPs counted, if applicable? How are parameters of the model counted?)
    % \item Are code/code snippets provided for calculating these variables if applicable? 
\end{itemize}


\paragraph{Training Setup (\S\ref{sec:model_training})}
\begin{itemize}[leftmargin=*]
    \item How many models are trained?
    \item At which sizes?
    \item On how much data each? On what data? Is any data repeated within the training for a model?
    \item How are model size, dataset size, and compute budget size counted? For example, how are parameters of the model counted? Are any parameters excluded (e.g., embedding layers)?
    \item Are code/code snippets provided for calculating these variables if applicable?
    % embedding  For example, how is compute cost counted, if applicable? 
    \item How are hyperparameters chosen (e.g., optimizer, learning rate schedule, batch size)? Do they change with scale?
    \item What other settings must be decided (e.g., model width vs. depth)? Do they change with scale?
    \item Is the training code open source?
    % \item How is the correctness of the scaling law considered SHOULD WE?
\end{itemize}

\end{minipage}
\begin{minipage}[t]{0.48\textwidth}
\raggedright


\paragraph{Data Collection(\S\ref{sec:data})}
\begin{itemize}[leftmargin=*]
    \item Are the model checkpoints provided openly?
    % \item Are these checkpoints modified in any way before evaluation? (say, checkpoint averaging)
    % \item If the above is done, is code for modifying the checkpoints provided?
    \item How many checkpoints per model are evaluated to fit each scaling law?
    \item What evaluation metric is used? On what dataset?
    \item Are the raw evaluation metrics modified, e.g., through loss interpolation, centering around a mean, scaling logarithmically, etc?
    \item If the above is done, is code for modifying the metric provided? 
\end{itemize}

\paragraph{Fitting Algorithm (\S\ref{sec:opt})}
\begin{itemize}[leftmargin=*]
    \item What objective (loss) is used?
    \item What algorithm is used to fit the equation?
    \item What hyperparameters are used for this algorithm?
    \item How is this algorithm initialized?
    \item Are all datapoints collected used to fit the equations? For example, are any outliers dropped? Are portions of the datapoints used to fit different equations?
    \item How is the correctness of the scaling law considered? Extrapolation, Confidence Intervals, Goodness of Fit?
\end{itemize}

\end{minipage}

% \paragraph{Other}
% \begin{itemize}
%     \item Is code for 
% \end{itemize}

\end{minipage}}
% \setcounter{secnumdepth}{2}
\newpage
\clearpage
\appendix
\setcounter{secnumdepth}{2} %May be changed to 1 or 2 if section numbers are desired.
\section*{Technical Appendix}
\label{sec:sup_outline}

This document supplements the main body of our paper with additional details, discussions, and results. In Section~\ref{sec:sup_dataset}, we present more details of the Moiré Pattern Dataset collection, including a brief analysis of various previously overlooked factors affecting moiré pattern diversity. In Section~\ref{sec:sup_method}, we will provide a detailed explanation of the two stages involved in implementing UniDemoiré: Moiré Pattern Generator and Moiré Image Synthesis. In Section~\ref{sec:sup_esperiments}, we provide more implementation details of experiments and show more qualitative results. Furthermore, as shown in Section~\ref{sec:sup_ablation}, we performed additional ablation experiments on the blending strategy in the Moiré Image Blending (MIB) module and the design of the upsampling block and the loss function in the Tone Refinement Network (TRN).


\section{Dataset Capture and Analysis}
\label{sec:sup_dataset}

\begin{table*}[t]

\centering
% \renewcommand\tabcolsep{5.0pt}
\vspace{-8pt}
%\resizebox{10cm}{!}
\scalebox{0.95}
{      % \textwidth, 10cm, 这里用 10cm 好像效果好于 \textwidth 
\begin{tabular}{c|ccccc}
\toprule[1.25pt]    % 顶部粗线
% \textbf{Mobile Phone} & \textbf{Camera} & \textbf{CMOS} & \textbf{MegaPixel($\text{MP}$)} & \textbf{Optical format($\text{Inches}$)} & \textbf{Pixel Size($\mathbf{\mu m}$)}   \\
Mobile Phone & Camera & CMOS & MegaPixel ($\text{MP}$) & Optical format ($\text{Inches}$) & Pixel Size ($\mu m$)   \\
\midrule
\multirow{1}{*}{iPhone 12}     & Main      & SONY   IMX503            & 12   & 1/2.55 & 1.40  \\
%\hline
\multirow{1}{*}{iPhone 13}     & Main      & SONY IMX603              & 12   & 1/1.88 & 1.70  \\
%\hline
\multirow{1}{*}{Honor 90}      & Main      & ISOCELL HP3              & 200  & 1/1.40 & 0.56  \\
%\hline
\multirow{1}{*}{Xiaomi 10s}    & Main      & ISOCELL HMX              & 108  & 1/1.33 & 0.80  \\
%\hline \cline
\midrule
\multirow{2}{*}{iPhone 12 Pro} & Main      & SONY IMX503              & 12   & 1/2.55 & 1.40  \\
                               & Telephoto & SONY IMX613 (2x zoom)    & 12.2 & 1/3.40 & 1.00  \\
\midrule
\multirow{2}{*}{iPhone 15 Pro} & Main      & SONY IMX803              & 48   & 1/1.28 & 1.22  \\
                               % \cline{2-6}
                               & Telephoto & SONY IMX713 (2x/4x zoom) & 12   & 1/3.40 & 1.00  \\


\bottomrule[1.25pt]  % 底部粗线
\end{tabular}
}
\caption{The mobile phone we apply to get the moiré patterns}
\label{tab:Datasets-Phone}
\end{table*}


\begin{table*}[t]

\centering
% \renewcommand\tabcolsep{5.0pt}
\vspace{-8pt}
% \resizebox{10cm}{!}
\scalebox{1.0}
{      % \textwidth, 10cm, 这里用 10cm 好像效果好于 \textwidth 
\begin{tabular}{c|ccccc}
%\toprule[1.2pt]    % 顶部粗线
\toprule[1.25pt]
% \textbf{Digital Screen}  & \textbf{Size($\text{Inches}$)} & \textbf{Panel type}  & \textbf{Resolution} & \textbf{Brightness($\mathbf{cd/m^2}$)} & \textbf{Coating} \\
Digital Screen    & Size ($Inches$) & Panel type        & Resolution         & Brightness ($cd/m^2$) & Coating \\
\midrule
DELL D2720DS      & 27              & IPS(LED)          & 2560 $\times$ 1440 & 350                   & Matte      \\
Macbook Air 2022  & 13.3            & IPS(LED)          & 2560 $\times$ 1600 & 500                   & Glossy     \\
AOC 27G2G8        & 27              & IPS(W-LED)        & 2560 $\times$ 1440 & 250                   & Matte      \\
Philips 27E1N5500 & 27              & IPS(LED)          & 2560 $\times$ 1440 & 300                   & Matte      \\
Xiaomi C34WQBA-RG & 34              & Curved SVA(W-LED) & 3440 $\times$ 1440 & 300                   & Matte      \\
ViewSonic VX2771-4K-HD & 27         & IPS(LED)          & 3840 $\times$ 2160 & 350                   & Matte      \\
%\bottomrule[1.2pt]  % 底部粗线
\bottomrule[1.25pt]
\end{tabular}
}
\caption{The screen we apply to get the moiré patterns}
\label{tab:Datasets-Screen}
\end{table*}


% \begin{table}[]
% \centering
% \caption{The mobile phone(left) and screen(right) we apply to get the moiré patterns}
% % \begin{subtable}{0.5\linewidth}
% \resizebox{\linewidth}{!}{
% \begin{tabular}{c|cc}
% \toprule    % 顶部粗线
% Mobile Phone & Camera & CMOS \\
% \midrule
% \multirow{1}{*}{iPhone 12}     & Main      & SONY   IMX503        \\
% %\hline
% \multirow{1}{*}{iPhone 13}     & Main      & SONY IMX603          \\
% %\hline
% \multirow{1}{*}{Honor 90}      & Main      & ISOCELL HP3          \\
% %\hline
% \multirow{1}{*}{Xiaomi 10s}    & Main      & ISOCELL HMX           \\
% %\hline \cline
% \midrule
% \multirow{2}{*}{iPhone 12 Pro} & Main      & SONY IMX503            \\
%                                & Telephoto & SONY IMX613 (2x zoom)  \\
% \midrule
% \multirow{2}{*}{iPhone 15 Pro} & Main      & SONY IMX803            \\
%                                % \cline{2-6}
%                                & Telephoto & SONY IMX713 (2x/4x zoom)  \\


% \bottomrule  % 底部粗线
% \end{tabular}
% }
% % \end{subtable}
% % \hfill
% % \begin{subtable}[t]{0.5\linewidth}
% \resizebox{\linewidth}{!}{
% \begin{tabular}{c|cc}
% \toprule
% Digital Screen    & Size ($Inches$) & Panel type        \\
% \midrule
% DELL D2720DS      & 27              & IPS(LED)          \\
% Macbook Air 2022  & 13.3            & IPS(LED)          \\
% AOC 27G2G8        & 27              & IPS(W-LED)        \\
% Philips 27E1N5500 & 27              & IPS(LED)          \\
% Xiaomi C34WQBA-RG & 34              & Curved SVA(W-LED)               \\
% ViewSonic VX2771-4K-HD & 27         & IPS(LED)          \\
% \bottomrule
% \end{tabular}
% }
% % \end{subtable}

% \label{tab:Datasets-Screen-Phone}
% \end{table}

In this section, we first present a brief introduction of various previously overlooked factors of devices that affect moiré pattern diversity. Then, we provide more details about our capture settings.

\subsection{The Impact of Device on Moiré Pattern Diversity}

% When taking a picture of a screen with a camera, there are two different frequencies to consider. The first one is the sampling frequency or resolution, which refers to the level of detail captured by the camera's sensor. The second one is the signal frequency of the screen itself, which determines how quickly the screen refreshes its content. Due to the optics of the lens and the color filters of the camera's CMOS sensor, the frequency at which we sample during a shot is reduced. According to the \blue{Shannon-Nyquist sampling theorem}, when the sampling frequency is less than twice the signal frequency, the sampled information becomes aliased, resulting in the moiré effect. 

Previous studies~\cite{yu2022towards, yang2023doing} have indicated that the geometric correlation between the screen and the camera significantly influences the features of the moiré pattern. However, such studies have overlooked that some aspects of the camera and the screen can also impact the moiré pattern.

For cameras, the two most critical factors affecting the moiré pattern are the CMOS and the lens used. The pixel density of a CMOS sensor (i.e., the number of pixels per unit area) determines its maximum sampling frequency, also known as the Nyquist frequency. The higher the pixel density, the higher the sampling frequency of the sensor and the higher the frequency of the signal that can be sampled, resulting in a higher frequency of moiré produced by the aliasing effect, which impacts the moiré pattern. In addition, the lens's focal length also affects the formation of moiré. In cell phone photography, lenses with shorter focal lengths (e.g., wide lenses/main camera lenses) usually have wider angles of view and can capture more of the scene content. Lenses with longer focal lengths (such as telephoto or telescopic lenses), on the other hand, offer a narrower angle of view and greater magnification for capturing distant details. When the screen is photographed with lenses of different focal lengths, the relative positional relationship between the pixels on the sensor and the pixels on the screen changes, which may cause the moiré pattern to appear or disappear.


\begin{figure}[t]
  \centering
    \includegraphics[width=1.0\linewidth]{Images/Distribution.pdf}
  \caption{Comparison of sharpness and colorfulness between our Moiré Pattern Dataset and MoireSpace.} 
  \label{fig:distribution}
\end{figure}


\begin{figure*}[!p]
  \centering
    \includegraphics[width=1.0\linewidth]{Images/Capture_Data.pdf}
  \caption{Samples from MoireSpace~\cite{yang2023doing} and our 4K Moiré Pattern Dataset.}
  \label{fig:capture_data}
\end{figure*}

Furthermore, the layout and the distance of pixels dots in the panel used can also significantly impact the formation of moiré on the display screen. The frequency of detail that a screen can display depends on how the pixel dots are arranged. Various arrangements result in distinct frequencies of detail, which impacts the formation of moiré patterns. The distance between pixel dots on the screen then affects the shooting distance. Larger pixel dot spacing will make the distance at which the moiré is formed to be photographed farther away. Conversely, the smaller the distance between pixel dots, the closer the distance needed to photograph the molded moiré.

\subsection{More Details about Capture Settings}
Based on the above analysis, we take screen images through different camera viewpoints to generate diverse moiré patterns. Specifically, we apply six mobile phones and six digital screens, as shown in Table~\ref{tab:Datasets-Phone} and ~\ref{tab:Datasets-Screen} ($\text{6} \times \text{6} = \text{36}$ combinations). 
% Figure~\ref{fig:capture_data} shows a comparison of some of our captured 4K moiré patterns with some samples from the MoireSpace dataset. It can be seen that our captured moiré patterns are much better than the MoireSpace dataset in terms of the clarity of the moiré texture and the vividness of the colors.
Figure~\ref{fig:distribution} presents a scatter plot comparison between our 4K Moiré Pattern Dataset and the MoireSpace dataset in terms of texture definition and color vibrancy, where each data point represents an individual sample. As illustrated in the visualization, our captured moiré patterns demonstrate a more comprehensive coverage in both the sharpness of textural details and the chromatic saturation range. This comparative analysis is further substantiated by the representative samples juxtaposed in Figure~\ref{fig:capture_data}, where side-by-side comparisons visually confirm that our dataset encompasses a broader diversity of morphological textures and a wider gamut of color expressions.

\subsubsection{Mobile Phones} 
We chose six mobile phones with varying camera specifications to capture diverse moiré patterns, as shown in Table~\ref{tab:Datasets-Phone}. Our selection criteria included the camera type, CMOS category, and number of megapixels. For the regular main camera with moderate resolution, we picked the iPhone 12 and iPhone 13. For electronic zooming at 2x and 3x, we selected the Honor 90 and Xiaomi 10s, which have high pixels. Additionally, we picked two iPhone 12 Pro and iPhone 15 Pro models with different CMOS specifically for telephoto lenses. These models use the telephoto lens for optical zoom at fixed magnifications of 2x and 3x.

\subsubsection{Display Screens} 
To capture a wider variety of moiré patterns in different forms, we selected display screens based on size, panel type, and resolution guidelines to maximize pixel point layouts and spacing on the screen. As shown in Table~\ref{tab:Datasets-Screen}, we have selected three 27-inch IPS panel LED matte screen monitors with a 2K resolution - DELL D2720DS, AOC 27G2G8, and Philips 27E1N5500. This specification is the most common among the available options. The AOC 27G2G8 is a W-LED monitor with an RGBW pixel layout. This IPS screen has white sub-pixels in addition to the standard RGB arrangement, creating a more varied pixel point layout. To capture the moiré pattern on the glossy display, we opted for a 13.3-inch IPS panel with a 2K resolution MacBook Air notebook. Finally, we selected two high-resolution displays: the Xiaomi C34WQBA-RG and the ViewSonic VX2771-4K-HD. These displays were explicitly chosen to capture moiré patterns with smaller pixel dot spacing. The Xiaomi C34WQBA-RG is a 34-inch curved display with an SVA panel and W-LED technology. It boasts a 3K resolution. On the other hand, the ViewSonic VX2771-4K-HD is a 27-inch matte screen display with an IPS panel and LED technology. It offers a standard 4K resolution.

\section{Method}

Our literature search focused on AI-assisted creative research tools in contrast to AI-assisted writing tools, creative research tools assist with the co-creation of concepts and ideas in the research process rather than merely improving stylistic or rhetorical choices in written research. In total, we surveyed 11 systems papers published in top HCI venues (i.e., CHI, CSCW, UIST, and ToCHI) over the last three years (2022-2024)\footnote{We include a relevant CHI 2025 paper made available on arXiv.}; details can be found in Table~\ref{tab:tool_classification}. 

As LLMs became widely used in 2022 with the release of ChatGPT \cite{openai2022chatgpt}, this timeframe was chosen to reflect the period of significant growth in LLM popularity and adoption, allowing us to capture the most relevant and impactful developments in GenAI-driven research tools. Of the systems surveyed, six of the systems integrate LLM-based functionalities, while the other five represent a more traditional AI approach and employ machine learning techniques (e.g., Seq2Seq, BERT, RNN). By examining both GenAI and traditional AI approaches, we aim to understand to what extent GenAI tools represent a fundamental shift in capabilities and design considerations compared to more established AI approaches.

The thematic dimensions presented in Section~\ref{sec:design-space} resulted from an iterative process among the authors. We engaged in extensive internal discussions and consulted with an external expert specializing in knowledge spaces and the role of AI in fostering creativity and sensemaking. This collaborative and iterative approach resulted in the four dimensions presented in the next section.
\section{Experiments}

\subsection{Consistent Update RMS}
\label{sec:exp:rms}

As discussed in Sec \ref{sec:analysis:rms}, we aim to match the update RMS across all matrix parameters and also match it with that of AdamW. We experimented with two methods to control the Muon update RMS among parameters and compared them to a baseline that only maintains a consistent RMS with AdamW:

\begin{enumerate}
    \item \texttt{Baseline.} We multiplied the update matrix by $0.2\cdot \sqrt{H}$ ($H$ is the model hidden size) to maintain a consistent update RMS with AdamW. Note that $\max(A,B)$ equals to $H$ for most matrices.
    \begin{align}
    \mathbf{W}_t = \mathbf{W}_{t-1} - \eta_t (0.2\cdot\mathbf{O}_t\cdot\sqrt{H} + \lambda \mathbf{W}_{t-1})
    \end{align}
    \item \texttt{Update Norm.} We can directly normalize the updates calculated via Newton-Schulz iterations so its RMS strictly becomes 0.2;
    \begin{align}
    \mathbf{W}_t = \mathbf{W}_{t-1} - \eta_t (0.2\cdot\mathbf{O}_t/\mathop{\text{RMS}}(\mathbf{O}_t) + \lambda \mathbf{W}_{t-1})
    \end{align}
    \item \texttt{Adjusted LR.} For each update matrix, we can scale its learning rate by a factor of $0.2 \cdot \sqrt{\max(A, B)}$ based on its shape. 
    \begin{align}
    \mathbf{W}_t = \mathbf{W}_{t-1} - \eta_t (0.2\cdot\mathbf{O}_t\cdot\sqrt{\max(A,B)} + \lambda \mathbf{W}_{t-1})
    \end{align}
\end{enumerate}


\paragraph{Analysis}
We designed experiments to illustrate the impact of Muon update RMS at an early training stage, because we observed that unexpected behaviors happened very quickly when training models at larger scale. We experimented with small scale 800M models as described in \ref{sec:exp:moonscalinglaw}. The problem of inconsistent update RMS is more pronounced when the disparity between matrix dimensions increases. To highlight the problem for further study, we slightly modify the model architecture by replacing the Swiglu MLP with a standard 2-layer MLP, changing the shape of its matrix parameters from $[H, 2.6H]$ to $[H, 4H]$. We evaluated the model's loss and monitored a few of its parameters' RMS, specifically, attention query (shape $[H, H]$) and MLP (shape $[H, 4H]$). We evaluated the model after training for 4B tokens out of a 20B-token schedule. From Table~\ref{tab:muon-params-rms}, we observed several interesting findings:


\begin{table}[t]
\small
\centering
\caption{Controlling Muon's Update RMS Across Different Model Params}
\label{tab:muon-params-rms}
\begin{tabular}{c|c|c|c|c}
\toprule
Methods & Training loss & Validation loss & query weight RMS & MLP weight RMS \\
\midrule
Baseline & 2.734 & 2.812 & 3.586e-2 & 2.52e-2 \\
Update Norm & \textbf{2.72} & \textbf{2.789} & 4.918e-2 & 5.01e-2 \\
Adjusted LR & 2.721 & \textbf{2.789} & 3.496e-2 & 4.89e-2 \\
\bottomrule
\end{tabular}
\end{table}

\begin{enumerate}
    \item Both \texttt{Update Norm} and \texttt{Adjusted LR} achieved better performances than \texttt{Baseline};
    
    \item For the MLP weight matrix of shape $[H, 4H]$, both \texttt{Update Norm} and \texttt{Adjusted LR} obtain a weight RMS that is roughly doubled comparing to \texttt{Baseline}. This is reasonable as $\sqrt{\text{max}(H,4H)} / \sqrt{H} = 2$, so the update RMS of \texttt{Update Norm} and \texttt{Adjusted LR} is roughly two times of \texttt{Baseline};
    
    \item For the attention query weight matrix of shape $[H, H]$, \texttt{Update Norm} still norms the update, while \texttt{Adjusted LR} does not because $\sqrt{\text{max}(H,H)} / \sqrt{H} = 1$. As a result, \texttt{Adjusted LR} results in a similar weight RMS as \texttt{Baseline}, but \texttt{Update Norm} has a larger weight rms similar to its MLP.
\end{enumerate}

Based on these findings, we choose the \texttt{Adjusted LR} method for future experiments because it has lower cost.

\subsection{Scaling Law of Muon}
\label{sec:exp:moonscalinglaw}

For a fair comparison with AdamW, we performed scaling law experiments on a series of dense models in Llama \citep{grattafiori2024llama3herdmodels} architecture. Building a strong baseline is of crucial importance in optimizer research. Hence, we perform a grid search for hyper-parameters of AdamW, following the compute-optimal training setup \citep{kaplan2020scalinglawsneurallanguage} (the grid search experiments can be found in Appendix~\ref{sec:appendix:scaling}). Details of the model architecture and hyper-parameters can be found in Table~\ref{tab:model-specs}. For Muon, as discussed in Sec~\ref{sec:analysis:rms}, since we matched Muon's update RMS to AdamW, we directly reused the hyper-parameters that are optimal for the AdamW baseline.

\begin{table}[t]
\small
\centering
\caption{Scaling Law Models and Hyper-Parameters}
\label{tab:model-specs}
\begin{tabular}{c|c|c|c|c|c|c}
\toprule
\# Params. w/o Embedding & Head & Layer & Hidden & Tokens & LR & Batch Size* \\
\midrule
399M & 12 & 12 & 1536 & 8.92B  & 9.503e-4 & 96  \\
545M & 14 & 14 & 1792 & 14.04B & 9.143e-4 & 128 \\
822M & 16 & 16 & 2048 & 20.76B & 8.825e-4 & 160 \\
1.1B & 18 & 18 & 2304 & 28.54B & 8.561e-4 & 192 \\
1.5B & 20 & 20 & 2560 & 38.91B & 8.305e-4 & 256 \\
\bottomrule
\end{tabular}
\\ \footnotesize{\small *In terms of number of examples in 8K context length.} 
\end{table}


\begin{figure}[h]
    \centering
    \includegraphics[width=0.6\textwidth]{figures/chinlaw_8k.pdf}
    \caption{Fitted scaling law curves for Muon and AdamW optimizers.}
    \label{fig:scaling_lm_loss_fitting}
\end{figure}

    
The fitted scaling law curve can be found in figure \ref{fig:scaling_lm_loss_fitting}, and the fitted equations are detailed in table \ref{tab:fit}. As shown in Figure~\ref{fig:scaling_lm_loss}, Muon only requires about 52\% training FLOPs to match the performance of AdamW under compute-optimal setting. 


\begin{table}
\centering
\caption{Fitted parameters of the scaling law curves}
\label{tab:fit}
\begin{tabular}{c|l|l}
\toprule
 & Muon & AdamW \\
\midrule
LM loss (seqlen=8K) & $2.506 \times C^{-0.052}$ & $2.608 \times C^{-0.054}$ \\
\bottomrule
\end{tabular}
\end{table}


\subsection{Pretraining with Muon}
\label{sec:exp:pretrain}

\paragraph{Model Architecture} To evaluate Muon against contemporary model architectures, we pretrained from scratch using the deepseek-v3-small architecture \citep{deepseekai2024deepseekv3technicalreport} as it demonstrates strong performance and the original results serve as a reference for comparison. Our pretrained model has 2.24B activated and 15.29B total parameters (3B activated and 16B total when including embedding). Minor modifications to the architecture are detailed in Appendix~\ref{sec:appendix:modelarch}.


\paragraph{Pretraining Data} Our pretraining data details can be found in \cite{k1p5}. The maximum context length during pretraining is 8K.


\paragraph{Pretraining} The model is trained in several stages. We use a 1e-3 auxfree bias update rate in stage 1 and 2, and 0.0 auxfree bias update rate in stage 3. The weight decay is set to 0.1 for all stages. More details and discussions of model training can be found in the Appendix \ref{sec:appendix:stability}.

\begin{enumerate}
    \item \texttt{0 to 33B tokens:} In this stage, the learning rate linearly increases to 4.2e-4 in 2k steps. The batch size is kept at 2048 examples;
    \item \texttt{33B to 5.2T tokens:} In this stage, the learning rate decays from 4.2e-4 to 4.2e-5 in a cosine style. We keep the batch size at 2048 until 200B tokens, and then doubled to 4096 for the remaining;
    \item \texttt{5.2T to 5.7T tokens:} In this stage (also referred as the cooldown stage), the learning rate increases to 1e-4 in in 100 steps, and then linearly decays to 0 in 500B tokens, and we keep a constant 4096 batch size. In this stage, we use the highest quality data, focusing on math, code, and reasoning.
\end{enumerate}

\paragraph{Evaluation Benchmarks} Our evaluation encompasses four primary categories of benchmarks, each designed to assess distinct capabilities of the model:

\begin{itemize}
    \item \textbf{English Language Understanding and Reasoning}: MMLU(5-shot)\citep{hendrycks2021measuringmassivemultitasklanguage}, MMLU-pro(5-shot) \citep{wang2024mmluprorobustchallengingmultitask}, BBH(3-shot) \citep{suzgun2022challengingbigbenchtaskschainofthought}, TriviaQA(5-shot) \citep{joshi2017triviaqalargescaledistantly}

    \item \textbf{Code Generation}: HumanEval(pass@1) \citep{chen2021codex}, MBPP(pass@1)\citep{austin2021programsynthesislargelanguage}
    
    \item  \textbf{Mathematical Reasoning}: GSM8K(4-shot) \citep{cobbe2021trainingverifierssolvemath} MATH \citep{hendrycks2021measuringmathematicalproblemsolving}, CMATH \citep{wei2023cmathlanguagemodelpass}

    \item \textbf{Chinese Language Understanding and Reasoning}: C-Eval(5-shot) \citep{huang2023cevalmultilevelmultidisciplinechinese}, CMMLU(5-shot)\citep{li2024cmmlumeasuringmassivemultitask}
\end{itemize}


\paragraph{Performance} We named our model trained with Muon ``Moonlight''. We compared Moonlight with different public models on a similar scale. We first evaluated Moonlight at 1.2T tokens and compared it with the following models that have the same architecture and trained with comparable number of tokens:

\begin{itemize}    
    \item \texttt{Deepseek-v3-Small } (\cite{deepseekai2024deepseekv3technicalreport}) is a  2.4B/16B-parameter MoE model trained with 1.33T tokens;
    \item \texttt{Moonlight-A} follows the same training settings as Moonlight, except that it uses the AdamW optimizer.
\end{itemize}

 For Moonlight and Moonlight-A, we used the intermediate 1.2T token checkpoint of the total 5.7T pretraining, where the learning rate is not decayed to minimal and the model has not gone through the cooldown stage yet.

\begin{table}[!ht]
    \small
    \centering
    \caption{Comparison of different models at around 1.2T tokens.}
    \setlength{\tabcolsep}{4pt}
    \begin{tabular}{@{}c l c c c c@{}}
    \toprule
    & \textbf{Benchmark (Metric)}  & \textbf{DSV3-Small} & \textbf{Moonlight-A@1.2T} & \textbf{Moonlight@1.2T} \\
    \midrule
    & Activated Params$^{\dagger}$ & 2.24B & 2.24B & 2.24B \\
    & Total Params$^{\dagger}$ & 15.29B & 15.29B & 15.29B \\
    & Training Tokens & 1.33T & 1.2T & 1.2T \\
    & Optimizer & AdamW & AdamW & Muon \\
    \midrule
    \multirow{4}{*}{English} 
    & MMLU & 53.3 & 60.2 & \textbf{60.4} \\
    & MMLU-pro & - & 26.8 & \textbf{28.1} \\
    & BBH & 41.4 & \textbf{45.3} & 43.2 \\
    & TriviaQA & -  & 57.4 & \textbf{58.1} \\
    \midrule
    \multirow{2}{*}{Code} & HumanEval & 26.8 & 29.3 & \textbf{37.2} \\
    & MBPP & 36.8 & 49.2 & \textbf{52.9} \\
    \midrule
    \multirow{3}{*}{Math} & GSM8K & 31.4 &  43.8 & \textbf{45.0} \\
    & MATH & 10.7 & 16.1 & \textbf{19.8} \\
    & CMath & - & 57.8 & \textbf{60.2} \\
    \midrule
    \multirow{2}{*}{Chinese} 
    & C-Eval & - &  57.2 & \textbf{59.9} \\
    & CMMLU & - & 58.2 & \textbf{58.8} \\
    \bottomrule
    \end{tabular}
    
    \footnotesize{\small $^{\dagger}$ The reported parameter counts exclude the embedding parameters.} 
    \label{tab:1.33Tresults}
\end{table}

As shown in Table \ref{tab:1.33Tresults}, Moonlight-A, our AdamW-trained baseline model, demonstrates strong performance compared to similar public models. Moonlight performs significantly better than Moonlight-A, proving the scaling effectiveness of Muon. We observed that Muon especially excels on Math and Code related tasks, and we encourage the research community to further investigate this phenomena. After Moonlight is fully trained to 5.7T tokens, we compared it with public models at similar scale and showed the results in Table \ref{tab:5.7Tresults_full}:

\begin{itemize}
    \item \texttt{LLAMA3-3B} from \cite{grattafiori2024llama3herdmodels} is a 3B-parameter dense model trained with 9T tokens. 
    \item \texttt{Qwen2.5-3B} from \cite{qwen2.5} is a 3B-parameter dense model trained with 18T tokens.
    \item \texttt{Deepseek-v2-Lite} from \cite{deepseekv2} is a 2.4B/16B-parameter MOE model trained with 5.7T tokens.
\end{itemize}


\begin{table}[!ht]
    \small
    \centering
    \caption{Comparison of different models on various benchmarks.}
    \setlength{\tabcolsep}{4pt}
    \begin{tabular}{@{}c l c c c c@{}}
    \toprule
    & \textbf{Benchmark (Metric)} & \textbf{Llama3.2-3B} & \textbf{Qwen2.5-3B} & \textbf{DSV2-Lite} & \textbf{Moonlight} \\
    \midrule
    & Activated Param$^{\dagger}$ & 2.81B & 2.77B & 2.24B & 2.24B \\
    & Total Params$^{\dagger}$ & 2.81B & 2.77B & 15.29B & 15.29B \\
    & Training Tokens  & 9T & 18T & 5.7T & 5.7T \\
    & Optimizer & AdamW  & Unknown & AdamW & Muon \\
    \midrule
    \multirow{4}{*}{English}
    & MMLU & 54.7 & 65.6 & 58.3 & \textbf{70.0} \\
    & MMLU-pro & 25.0 & 34.6 & 25.5 & \textbf{42.4} \\
    & BBH & 46.8 & 56.3 & 44.1 & \textbf{65.2} \\
    & TriviaQA$^{\ddagger}$ & 59.6 & 51.1 & 65.1 & \textbf{66.3} \\
    \midrule
    \multirow{2}{*}{Code} & HumanEval & 28.0 & 42.1 & 29.9 & \textbf{48.1} \\
    & MBPP & 48.7 & 57.1 & 43.2 & \textbf{63.8} \\
    \midrule
    \multirow{3}{*}{Math} & GSM8K & 34.0 & \textbf{79.1} & 41.1 & 77.4 \\
    & MATH & 8.5 & 42.6 & 17.1 & \textbf{45.3} \\
    & CMath & - & 80.0 & 58.4 & \textbf{81.1} \\
    \midrule
    \multirow{2}{*}{Chinese}
    & C-Eval & - & 75.0 & 60.3 & \textbf{77.2} \\
    & CMMLU & - & 75.0 & 64.3 & \textbf{78.2} \\
    \bottomrule
    \end{tabular}
    
    \footnotesize{$^{\dagger}$ The reported parameter counts exclude the embedding parameters.$^{\ddagger}$ We tested all listed models with the full set of TriviaQA.}
    \label{tab:5.7Tresults_full}
\end{table}


As shown in Table~\ref{tab:5.7Tresults_full}, Moonlight outperforms models with similar architectures trained with an equivalent number of tokens. Even when compared to dense models trained on substantially larger datasets, Moonlight maintains competitive performance. Detailed comparisons can be found in Appendix~\ref{sec:appendix:comparisons}. The performance of Moonlight is further compared with other well-known language models on MMLU and GSM8k, as illustrated in Figure~\ref{fig:mmlu} and Appendix~\ref{sec:appendix:comparisons} Figure~\ref{Fig:model_perf}.\footnote{Performance metrics and computational requirements (FLOPs) for baseline models are sourced from~\citep{olmo20242}}. Notably, Moonlight lies on the Pareto frontier of model performance versus training budget, outperforming many other models across various sizes. 




\subsection{Dynamics of Singular Spectrum}
In order to validate the intuition that Muon can optimize the weight matrices in more diverse directions, we conducted a spectral analysis of the weight matrices trained with Muon and AdamW. For a weight matrix with singular values $\sigma = (\sigma_1, \sigma_2, \cdots, \sigma_n)$, we calculate the SVD entropy~\citep{svd_entropy, effectiverank} of this matrix as follows:
\begin{equation}
    H(\sigma) = -\frac{1}{\log n}\sum_{i=1}^n \frac{\sigma^2_i}{\sum_{j=1}^n \sigma^2_j} \log \frac{\sigma^2_i}{\sum_{j=1}^n \sigma^2_j} \notag
\end{equation}
As shown in Figure~\ref{fig_svd_entropy}, we visualized the average SVD entropy of the weight matrices across different training checkpoints during pretraining with 1.2T tokens. We can see that across all training checkpoints and all groups of weight matrices, the SVD entropy of Muon is higher than that of AdamW, which verifies the intuition that Muon can provide a more diverse spectrum of updates for the weight matrices. This discrepancy is more significant in the router weights for expert selection, which indicates that mixture-of-expert models can benefit more from Muon.

Moreover, we visualized the singular value distributions of each weight matrix at the checkpoint trained with 1.2T tokens as demonstrated in Appendix~\ref{sec:appendix:svd}. We find that, for over 90\% of the weight matrices, the SVD entropy when optimized by Muon is higher than that of AdamW, providing strong empirical evidence for Muon's superior capability in exploring diverse optimization directions.


\begin{figure}[t]
    \centering
    \includegraphics[width=\textwidth]{figures/fig_svd_entropy_final_v2.pdf}
    \caption{SVD entropy of weight matrices across different training iterations. We categorize the weight matrices into 6 different groups: 1) AttnQO denotes the weight matrices related to the query and output projection in the attention layer; 2) AttnKV denotes the weight matrices related to the key and value projection in the attention layer; 3) Experts denotes the weight matrices in expert models; 4) SharedExperts denotes the weight matrices in shared expert models; 5) Router denotes the weight matrices in the router; 6) Dense denotes the weight matrices in the first dense layer. The SVD entropy is calculated as the macro-average of the weight matrices in each group across all layers. For weights in expert models, we only calculate 3 out of 64 experts in different layers for efficiency.} 
    \label{fig_svd_entropy} 
\end{figure}


\subsection{Supervised Finetuning (SFT) with Muon}


In this section, we present ablation studies on the Muon optimizer within the standard SFT stage of LLM training. Our findings demonstrate that the benefits introduced by Muon persist during the SFT stage. Specifically, a model that is both Muon-pretrained and Muon-finetuned outperforms others in the ablation studies. However, we also observe that when the SFT optimizer differs from the pretraining optimizer, SFT with Muon does not show a significant advantage over AdamW. This suggests that there is still considerable room for further exploration, which we leave for future work.

\subsubsection{Ablation Studies on the Interchangeability of Pretrain and SFT Optimizers}

To further investigate Muon’s potential, we finetuned Moonlight@1.2T and Moonlight-A@1.2T using both the Muon and AdamW optimizers. These models were finetuned for two epochs on the open-source tulu-3-sft-mixture dataset (\cite{lambert2024tulu3}), which contains 4k sequence length data. The learning rate followed a linear decay schedule, starting at $5 \times 10^{-5}$ and gradually reducing to $0$. The results, shown in Table \ref{tab:optim-interchangeability}, highlight the superior performance of Moonlight@1.2T compared to Moonlight-A@1.2T.


\begin{table}[ht]
\small
\centering
\caption{Examining the impact of optimizer interchangeability between pretraining and SFT phases.}
\label{tab:optim-interchangeability}
\begin{tabular}{l c|c|c|c|c}
\toprule
\textbf{Benchmark (Metric)} & \textbf{\# Shots} & \multicolumn{4}{|c}{\textbf{Moonlight-1.2T}} \\
\midrule
Pretraining Optimizer & - & Muon & AdamW & Muon & AdamW \\
SFT Optimzier & - & Muon & Muon & AdamW & AdamW \\
\midrule
MMLU (EM) & 0-shot (CoT) & \textbf{55.7} & 55.3 & 50.2 & 52.0 \\
HumanEval (Pass@1) & 0-shot & \textbf{57.3} & 53.7 & 52.4 & 53.1 \\
MBPP (Pass@1) & 0-shot & \textbf{55.6} & 55.5 & 55.2 & 55.2 \\
GSM8K (EM) & 5-shot & \textbf{68.0} & 62.1 & 64.9 & 64.6 \\
\bottomrule
\end{tabular}

\end{table}

\subsubsection{SFT with Muon on public pretrained models}

We further applied Muon to the supervised fine-tuning (SFT) of a public pretrained model, specifically the Qwen2.5-7B base model (\cite{qwen2.5}), using the open-source tulu-3-sft-mixture dataset (\cite{lambert2024tulu3}). The dataset was packed with an 8k sequence length, and we employed a cosine decay learning rate schedule, starting at $2 \times 10^{-5}$ and gradually decreasing to $2 \times 10^{-6}$. The results are presented in Table \ref{tab:public-model-SFT-results}. For comparison, we show that the Muon-finetuned model achieves performance on par with the Adam-finetuned model. These results indicate that for optimal performance, it is more effective to apply Muon during the pretraining phase rather than during supervised fine-tuning.

\begin{table}[ht]
\small
\centering
\caption{Comparison of Adam and Muon optimizers applied to the SFT of the Qwen2.5-7B pretrained model.}
\label{tab:public-model-SFT-results}
\begin{tabular}{l c|c|c}
\toprule
\textbf{Benchmark (Metric)} & \textbf{\# Shots} & \textbf{Adam-SFT} & \textbf{Muon-SFT} \\
\midrule
Pretrained Model & - & \multicolumn{2}{|c}{Qwen2.5-7B} \\
\midrule
MMLU (EM) & 0-shot (CoT) & \textbf{71.4} & 70.8 \\
HumanEval (Pass@1) & 0-shot & \textbf{79.3} & 77.4 \\
MBPP (Pass@1) & 0-shot & \textbf{71.9} & 71.6 \\
GSM8K (EM) & 5-shot & \textbf{89.8} & 85.8 \\
\bottomrule
\end{tabular}
\end{table}






% \subsection{Lloyd-Max Algorithm}
\label{subsec:Lloyd-Max}
For a given quantization bitwidth $B$ and an operand $\bm{X}$, the Lloyd-Max algorithm finds $2^B$ quantization levels $\{\hat{x}_i\}_{i=1}^{2^B}$ such that quantizing $\bm{X}$ by rounding each scalar in $\bm{X}$ to the nearest quantization level minimizes the quantization MSE. 

The algorithm starts with an initial guess of quantization levels and then iteratively computes quantization thresholds $\{\tau_i\}_{i=1}^{2^B-1}$ and updates quantization levels $\{\hat{x}_i\}_{i=1}^{2^B}$. Specifically, at iteration $n$, thresholds are set to the midpoints of the previous iteration's levels:
\begin{align*}
    \tau_i^{(n)}=\frac{\hat{x}_i^{(n-1)}+\hat{x}_{i+1}^{(n-1)}}2 \text{ for } i=1\ldots 2^B-1
\end{align*}
Subsequently, the quantization levels are re-computed as conditional means of the data regions defined by the new thresholds:
\begin{align*}
    \hat{x}_i^{(n)}=\mathbb{E}\left[ \bm{X} \big| \bm{X}\in [\tau_{i-1}^{(n)},\tau_i^{(n)}] \right] \text{ for } i=1\ldots 2^B
\end{align*}
where to satisfy boundary conditions we have $\tau_0=-\infty$ and $\tau_{2^B}=\infty$. The algorithm iterates the above steps until convergence.

Figure \ref{fig:lm_quant} compares the quantization levels of a $7$-bit floating point (E3M3) quantizer (left) to a $7$-bit Lloyd-Max quantizer (right) when quantizing a layer of weights from the GPT3-126M model at a per-tensor granularity. As shown, the Lloyd-Max quantizer achieves substantially lower quantization MSE. Further, Table \ref{tab:FP7_vs_LM7} shows the superior perplexity achieved by Lloyd-Max quantizers for bitwidths of $7$, $6$ and $5$. The difference between the quantizers is clear at 5 bits, where per-tensor FP quantization incurs a drastic and unacceptable increase in perplexity, while Lloyd-Max quantization incurs a much smaller increase. Nevertheless, we note that even the optimal Lloyd-Max quantizer incurs a notable ($\sim 1.5$) increase in perplexity due to the coarse granularity of quantization. 

\begin{figure}[h]
  \centering
  \includegraphics[width=0.7\linewidth]{sections/figures/LM7_FP7.pdf}
  \caption{\small Quantization levels and the corresponding quantization MSE of Floating Point (left) vs Lloyd-Max (right) Quantizers for a layer of weights in the GPT3-126M model.}
  \label{fig:lm_quant}
\end{figure}

\begin{table}[h]\scriptsize
\begin{center}
\caption{\label{tab:FP7_vs_LM7} \small Comparing perplexity (lower is better) achieved by floating point quantizers and Lloyd-Max quantizers on a GPT3-126M model for the Wikitext-103 dataset.}
\begin{tabular}{c|cc|c}
\hline
 \multirow{2}{*}{\textbf{Bitwidth}} & \multicolumn{2}{|c|}{\textbf{Floating-Point Quantizer}} & \textbf{Lloyd-Max Quantizer} \\
 & Best Format & Wikitext-103 Perplexity & Wikitext-103 Perplexity \\
\hline
7 & E3M3 & 18.32 & 18.27 \\
6 & E3M2 & 19.07 & 18.51 \\
5 & E4M0 & 43.89 & 19.71 \\
\hline
\end{tabular}
\end{center}
\end{table}

\subsection{Proof of Local Optimality of LO-BCQ}
\label{subsec:lobcq_opt_proof}
For a given block $\bm{b}_j$, the quantization MSE during LO-BCQ can be empirically evaluated as $\frac{1}{L_b}\lVert \bm{b}_j- \bm{\hat{b}}_j\rVert^2_2$ where $\bm{\hat{b}}_j$ is computed from equation (\ref{eq:clustered_quantization_definition}) as $C_{f(\bm{b}_j)}(\bm{b}_j)$. Further, for a given block cluster $\mathcal{B}_i$, we compute the quantization MSE as $\frac{1}{|\mathcal{B}_{i}|}\sum_{\bm{b} \in \mathcal{B}_{i}} \frac{1}{L_b}\lVert \bm{b}- C_i^{(n)}(\bm{b})\rVert^2_2$. Therefore, at the end of iteration $n$, we evaluate the overall quantization MSE $J^{(n)}$ for a given operand $\bm{X}$ composed of $N_c$ block clusters as:
\begin{align*}
    \label{eq:mse_iter_n}
    J^{(n)} = \frac{1}{N_c} \sum_{i=1}^{N_c} \frac{1}{|\mathcal{B}_{i}^{(n)}|}\sum_{\bm{v} \in \mathcal{B}_{i}^{(n)}} \frac{1}{L_b}\lVert \bm{b}- B_i^{(n)}(\bm{b})\rVert^2_2
\end{align*}

At the end of iteration $n$, the codebooks are updated from $\mathcal{C}^{(n-1)}$ to $\mathcal{C}^{(n)}$. However, the mapping of a given vector $\bm{b}_j$ to quantizers $\mathcal{C}^{(n)}$ remains as  $f^{(n)}(\bm{b}_j)$. At the next iteration, during the vector clustering step, $f^{(n+1)}(\bm{b}_j)$ finds new mapping of $\bm{b}_j$ to updated codebooks $\mathcal{C}^{(n)}$ such that the quantization MSE over the candidate codebooks is minimized. Therefore, we obtain the following result for $\bm{b}_j$:
\begin{align*}
\frac{1}{L_b}\lVert \bm{b}_j - C_{f^{(n+1)}(\bm{b}_j)}^{(n)}(\bm{b}_j)\rVert^2_2 \le \frac{1}{L_b}\lVert \bm{b}_j - C_{f^{(n)}(\bm{b}_j)}^{(n)}(\bm{b}_j)\rVert^2_2
\end{align*}

That is, quantizing $\bm{b}_j$ at the end of the block clustering step of iteration $n+1$ results in lower quantization MSE compared to quantizing at the end of iteration $n$. Since this is true for all $\bm{b} \in \bm{X}$, we assert the following:
\begin{equation}
\begin{split}
\label{eq:mse_ineq_1}
    \tilde{J}^{(n+1)} &= \frac{1}{N_c} \sum_{i=1}^{N_c} \frac{1}{|\mathcal{B}_{i}^{(n+1)}|}\sum_{\bm{b} \in \mathcal{B}_{i}^{(n+1)}} \frac{1}{L_b}\lVert \bm{b} - C_i^{(n)}(b)\rVert^2_2 \le J^{(n)}
\end{split}
\end{equation}
where $\tilde{J}^{(n+1)}$ is the the quantization MSE after the vector clustering step at iteration $n+1$.

Next, during the codebook update step (\ref{eq:quantizers_update}) at iteration $n+1$, the per-cluster codebooks $\mathcal{C}^{(n)}$ are updated to $\mathcal{C}^{(n+1)}$ by invoking the Lloyd-Max algorithm \citep{Lloyd}. We know that for any given value distribution, the Lloyd-Max algorithm minimizes the quantization MSE. Therefore, for a given vector cluster $\mathcal{B}_i$ we obtain the following result:

\begin{equation}
    \frac{1}{|\mathcal{B}_{i}^{(n+1)}|}\sum_{\bm{b} \in \mathcal{B}_{i}^{(n+1)}} \frac{1}{L_b}\lVert \bm{b}- C_i^{(n+1)}(\bm{b})\rVert^2_2 \le \frac{1}{|\mathcal{B}_{i}^{(n+1)}|}\sum_{\bm{b} \in \mathcal{B}_{i}^{(n+1)}} \frac{1}{L_b}\lVert \bm{b}- C_i^{(n)}(\bm{b})\rVert^2_2
\end{equation}

The above equation states that quantizing the given block cluster $\mathcal{B}_i$ after updating the associated codebook from $C_i^{(n)}$ to $C_i^{(n+1)}$ results in lower quantization MSE. Since this is true for all the block clusters, we derive the following result: 
\begin{equation}
\begin{split}
\label{eq:mse_ineq_2}
     J^{(n+1)} &= \frac{1}{N_c} \sum_{i=1}^{N_c} \frac{1}{|\mathcal{B}_{i}^{(n+1)}|}\sum_{\bm{b} \in \mathcal{B}_{i}^{(n+1)}} \frac{1}{L_b}\lVert \bm{b}- C_i^{(n+1)}(\bm{b})\rVert^2_2  \le \tilde{J}^{(n+1)}   
\end{split}
\end{equation}

Following (\ref{eq:mse_ineq_1}) and (\ref{eq:mse_ineq_2}), we find that the quantization MSE is non-increasing for each iteration, that is, $J^{(1)} \ge J^{(2)} \ge J^{(3)} \ge \ldots \ge J^{(M)}$ where $M$ is the maximum number of iterations. 
%Therefore, we can say that if the algorithm converges, then it must be that it has converged to a local minimum. 
\hfill $\blacksquare$


\begin{figure}
    \begin{center}
    \includegraphics[width=0.5\textwidth]{sections//figures/mse_vs_iter.pdf}
    \end{center}
    \caption{\small NMSE vs iterations during LO-BCQ compared to other block quantization proposals}
    \label{fig:nmse_vs_iter}
\end{figure}

Figure \ref{fig:nmse_vs_iter} shows the empirical convergence of LO-BCQ across several block lengths and number of codebooks. Also, the MSE achieved by LO-BCQ is compared to baselines such as MXFP and VSQ. As shown, LO-BCQ converges to a lower MSE than the baselines. Further, we achieve better convergence for larger number of codebooks ($N_c$) and for a smaller block length ($L_b$), both of which increase the bitwidth of BCQ (see Eq \ref{eq:bitwidth_bcq}).


\subsection{Additional Accuracy Results}
%Table \ref{tab:lobcq_config} lists the various LOBCQ configurations and their corresponding bitwidths.
\begin{table}
\setlength{\tabcolsep}{4.75pt}
\begin{center}
\caption{\label{tab:lobcq_config} Various LO-BCQ configurations and their bitwidths.}
\begin{tabular}{|c||c|c|c|c||c|c||c|} 
\hline
 & \multicolumn{4}{|c||}{$L_b=8$} & \multicolumn{2}{|c||}{$L_b=4$} & $L_b=2$ \\
 \hline
 \backslashbox{$L_A$\kern-1em}{\kern-1em$N_c$} & 2 & 4 & 8 & 16 & 2 & 4 & 2 \\
 \hline
 64 & 4.25 & 4.375 & 4.5 & 4.625 & 4.375 & 4.625 & 4.625\\
 \hline
 32 & 4.375 & 4.5 & 4.625& 4.75 & 4.5 & 4.75 & 4.75 \\
 \hline
 16 & 4.625 & 4.75& 4.875 & 5 & 4.75 & 5 & 5 \\
 \hline
\end{tabular}
\end{center}
\end{table}

%\subsection{Perplexity achieved by various LO-BCQ configurations on Wikitext-103 dataset}

\begin{table} \centering
\begin{tabular}{|c||c|c|c|c||c|c||c|} 
\hline
 $L_b \rightarrow$& \multicolumn{4}{c||}{8} & \multicolumn{2}{c||}{4} & 2\\
 \hline
 \backslashbox{$L_A$\kern-1em}{\kern-1em$N_c$} & 2 & 4 & 8 & 16 & 2 & 4 & 2  \\
 %$N_c \rightarrow$ & 2 & 4 & 8 & 16 & 2 & 4 & 2 \\
 \hline
 \hline
 \multicolumn{8}{c}{GPT3-1.3B (FP32 PPL = 9.98)} \\ 
 \hline
 \hline
 64 & 10.40 & 10.23 & 10.17 & 10.15 &  10.28 & 10.18 & 10.19 \\
 \hline
 32 & 10.25 & 10.20 & 10.15 & 10.12 &  10.23 & 10.17 & 10.17 \\
 \hline
 16 & 10.22 & 10.16 & 10.10 & 10.09 &  10.21 & 10.14 & 10.16 \\
 \hline
  \hline
 \multicolumn{8}{c}{GPT3-8B (FP32 PPL = 7.38)} \\ 
 \hline
 \hline
 64 & 7.61 & 7.52 & 7.48 &  7.47 &  7.55 &  7.49 & 7.50 \\
 \hline
 32 & 7.52 & 7.50 & 7.46 &  7.45 &  7.52 &  7.48 & 7.48  \\
 \hline
 16 & 7.51 & 7.48 & 7.44 &  7.44 &  7.51 &  7.49 & 7.47  \\
 \hline
\end{tabular}
\caption{\label{tab:ppl_gpt3_abalation} Wikitext-103 perplexity across GPT3-1.3B and 8B models.}
\end{table}

\begin{table} \centering
\begin{tabular}{|c||c|c|c|c||} 
\hline
 $L_b \rightarrow$& \multicolumn{4}{c||}{8}\\
 \hline
 \backslashbox{$L_A$\kern-1em}{\kern-1em$N_c$} & 2 & 4 & 8 & 16 \\
 %$N_c \rightarrow$ & 2 & 4 & 8 & 16 & 2 & 4 & 2 \\
 \hline
 \hline
 \multicolumn{5}{|c|}{Llama2-7B (FP32 PPL = 5.06)} \\ 
 \hline
 \hline
 64 & 5.31 & 5.26 & 5.19 & 5.18  \\
 \hline
 32 & 5.23 & 5.25 & 5.18 & 5.15  \\
 \hline
 16 & 5.23 & 5.19 & 5.16 & 5.14  \\
 \hline
 \multicolumn{5}{|c|}{Nemotron4-15B (FP32 PPL = 5.87)} \\ 
 \hline
 \hline
 64  & 6.3 & 6.20 & 6.13 & 6.08  \\
 \hline
 32  & 6.24 & 6.12 & 6.07 & 6.03  \\
 \hline
 16  & 6.12 & 6.14 & 6.04 & 6.02  \\
 \hline
 \multicolumn{5}{|c|}{Nemotron4-340B (FP32 PPL = 3.48)} \\ 
 \hline
 \hline
 64 & 3.67 & 3.62 & 3.60 & 3.59 \\
 \hline
 32 & 3.63 & 3.61 & 3.59 & 3.56 \\
 \hline
 16 & 3.61 & 3.58 & 3.57 & 3.55 \\
 \hline
\end{tabular}
\caption{\label{tab:ppl_llama7B_nemo15B} Wikitext-103 perplexity compared to FP32 baseline in Llama2-7B and Nemotron4-15B, 340B models}
\end{table}

%\subsection{Perplexity achieved by various LO-BCQ configurations on MMLU dataset}


\begin{table} \centering
\begin{tabular}{|c||c|c|c|c||c|c|c|c|} 
\hline
 $L_b \rightarrow$& \multicolumn{4}{c||}{8} & \multicolumn{4}{c||}{8}\\
 \hline
 \backslashbox{$L_A$\kern-1em}{\kern-1em$N_c$} & 2 & 4 & 8 & 16 & 2 & 4 & 8 & 16  \\
 %$N_c \rightarrow$ & 2 & 4 & 8 & 16 & 2 & 4 & 2 \\
 \hline
 \hline
 \multicolumn{5}{|c|}{Llama2-7B (FP32 Accuracy = 45.8\%)} & \multicolumn{4}{|c|}{Llama2-70B (FP32 Accuracy = 69.12\%)} \\ 
 \hline
 \hline
 64 & 43.9 & 43.4 & 43.9 & 44.9 & 68.07 & 68.27 & 68.17 & 68.75 \\
 \hline
 32 & 44.5 & 43.8 & 44.9 & 44.5 & 68.37 & 68.51 & 68.35 & 68.27  \\
 \hline
 16 & 43.9 & 42.7 & 44.9 & 45 & 68.12 & 68.77 & 68.31 & 68.59  \\
 \hline
 \hline
 \multicolumn{5}{|c|}{GPT3-22B (FP32 Accuracy = 38.75\%)} & \multicolumn{4}{|c|}{Nemotron4-15B (FP32 Accuracy = 64.3\%)} \\ 
 \hline
 \hline
 64 & 36.71 & 38.85 & 38.13 & 38.92 & 63.17 & 62.36 & 63.72 & 64.09 \\
 \hline
 32 & 37.95 & 38.69 & 39.45 & 38.34 & 64.05 & 62.30 & 63.8 & 64.33  \\
 \hline
 16 & 38.88 & 38.80 & 38.31 & 38.92 & 63.22 & 63.51 & 63.93 & 64.43  \\
 \hline
\end{tabular}
\caption{\label{tab:mmlu_abalation} Accuracy on MMLU dataset across GPT3-22B, Llama2-7B, 70B and Nemotron4-15B models.}
\end{table}


%\subsection{Perplexity achieved by various LO-BCQ configurations on LM evaluation harness}

\begin{table} \centering
\begin{tabular}{|c||c|c|c|c||c|c|c|c|} 
\hline
 $L_b \rightarrow$& \multicolumn{4}{c||}{8} & \multicolumn{4}{c||}{8}\\
 \hline
 \backslashbox{$L_A$\kern-1em}{\kern-1em$N_c$} & 2 & 4 & 8 & 16 & 2 & 4 & 8 & 16  \\
 %$N_c \rightarrow$ & 2 & 4 & 8 & 16 & 2 & 4 & 2 \\
 \hline
 \hline
 \multicolumn{5}{|c|}{Race (FP32 Accuracy = 37.51\%)} & \multicolumn{4}{|c|}{Boolq (FP32 Accuracy = 64.62\%)} \\ 
 \hline
 \hline
 64 & 36.94 & 37.13 & 36.27 & 37.13 & 63.73 & 62.26 & 63.49 & 63.36 \\
 \hline
 32 & 37.03 & 36.36 & 36.08 & 37.03 & 62.54 & 63.51 & 63.49 & 63.55  \\
 \hline
 16 & 37.03 & 37.03 & 36.46 & 37.03 & 61.1 & 63.79 & 63.58 & 63.33  \\
 \hline
 \hline
 \multicolumn{5}{|c|}{Winogrande (FP32 Accuracy = 58.01\%)} & \multicolumn{4}{|c|}{Piqa (FP32 Accuracy = 74.21\%)} \\ 
 \hline
 \hline
 64 & 58.17 & 57.22 & 57.85 & 58.33 & 73.01 & 73.07 & 73.07 & 72.80 \\
 \hline
 32 & 59.12 & 58.09 & 57.85 & 58.41 & 73.01 & 73.94 & 72.74 & 73.18  \\
 \hline
 16 & 57.93 & 58.88 & 57.93 & 58.56 & 73.94 & 72.80 & 73.01 & 73.94  \\
 \hline
\end{tabular}
\caption{\label{tab:mmlu_abalation} Accuracy on LM evaluation harness tasks on GPT3-1.3B model.}
\end{table}

\begin{table} \centering
\begin{tabular}{|c||c|c|c|c||c|c|c|c|} 
\hline
 $L_b \rightarrow$& \multicolumn{4}{c||}{8} & \multicolumn{4}{c||}{8}\\
 \hline
 \backslashbox{$L_A$\kern-1em}{\kern-1em$N_c$} & 2 & 4 & 8 & 16 & 2 & 4 & 8 & 16  \\
 %$N_c \rightarrow$ & 2 & 4 & 8 & 16 & 2 & 4 & 2 \\
 \hline
 \hline
 \multicolumn{5}{|c|}{Race (FP32 Accuracy = 41.34\%)} & \multicolumn{4}{|c|}{Boolq (FP32 Accuracy = 68.32\%)} \\ 
 \hline
 \hline
 64 & 40.48 & 40.10 & 39.43 & 39.90 & 69.20 & 68.41 & 69.45 & 68.56 \\
 \hline
 32 & 39.52 & 39.52 & 40.77 & 39.62 & 68.32 & 67.43 & 68.17 & 69.30  \\
 \hline
 16 & 39.81 & 39.71 & 39.90 & 40.38 & 68.10 & 66.33 & 69.51 & 69.42  \\
 \hline
 \hline
 \multicolumn{5}{|c|}{Winogrande (FP32 Accuracy = 67.88\%)} & \multicolumn{4}{|c|}{Piqa (FP32 Accuracy = 78.78\%)} \\ 
 \hline
 \hline
 64 & 66.85 & 66.61 & 67.72 & 67.88 & 77.31 & 77.42 & 77.75 & 77.64 \\
 \hline
 32 & 67.25 & 67.72 & 67.72 & 67.00 & 77.31 & 77.04 & 77.80 & 77.37  \\
 \hline
 16 & 68.11 & 68.90 & 67.88 & 67.48 & 77.37 & 78.13 & 78.13 & 77.69  \\
 \hline
\end{tabular}
\caption{\label{tab:mmlu_abalation} Accuracy on LM evaluation harness tasks on GPT3-8B model.}
\end{table}

\begin{table} \centering
\begin{tabular}{|c||c|c|c|c||c|c|c|c|} 
\hline
 $L_b \rightarrow$& \multicolumn{4}{c||}{8} & \multicolumn{4}{c||}{8}\\
 \hline
 \backslashbox{$L_A$\kern-1em}{\kern-1em$N_c$} & 2 & 4 & 8 & 16 & 2 & 4 & 8 & 16  \\
 %$N_c \rightarrow$ & 2 & 4 & 8 & 16 & 2 & 4 & 2 \\
 \hline
 \hline
 \multicolumn{5}{|c|}{Race (FP32 Accuracy = 40.67\%)} & \multicolumn{4}{|c|}{Boolq (FP32 Accuracy = 76.54\%)} \\ 
 \hline
 \hline
 64 & 40.48 & 40.10 & 39.43 & 39.90 & 75.41 & 75.11 & 77.09 & 75.66 \\
 \hline
 32 & 39.52 & 39.52 & 40.77 & 39.62 & 76.02 & 76.02 & 75.96 & 75.35  \\
 \hline
 16 & 39.81 & 39.71 & 39.90 & 40.38 & 75.05 & 73.82 & 75.72 & 76.09  \\
 \hline
 \hline
 \multicolumn{5}{|c|}{Winogrande (FP32 Accuracy = 70.64\%)} & \multicolumn{4}{|c|}{Piqa (FP32 Accuracy = 79.16\%)} \\ 
 \hline
 \hline
 64 & 69.14 & 70.17 & 70.17 & 70.56 & 78.24 & 79.00 & 78.62 & 78.73 \\
 \hline
 32 & 70.96 & 69.69 & 71.27 & 69.30 & 78.56 & 79.49 & 79.16 & 78.89  \\
 \hline
 16 & 71.03 & 69.53 & 69.69 & 70.40 & 78.13 & 79.16 & 79.00 & 79.00  \\
 \hline
\end{tabular}
\caption{\label{tab:mmlu_abalation} Accuracy on LM evaluation harness tasks on GPT3-22B model.}
\end{table}

\begin{table} \centering
\begin{tabular}{|c||c|c|c|c||c|c|c|c|} 
\hline
 $L_b \rightarrow$& \multicolumn{4}{c||}{8} & \multicolumn{4}{c||}{8}\\
 \hline
 \backslashbox{$L_A$\kern-1em}{\kern-1em$N_c$} & 2 & 4 & 8 & 16 & 2 & 4 & 8 & 16  \\
 %$N_c \rightarrow$ & 2 & 4 & 8 & 16 & 2 & 4 & 2 \\
 \hline
 \hline
 \multicolumn{5}{|c|}{Race (FP32 Accuracy = 44.4\%)} & \multicolumn{4}{|c|}{Boolq (FP32 Accuracy = 79.29\%)} \\ 
 \hline
 \hline
 64 & 42.49 & 42.51 & 42.58 & 43.45 & 77.58 & 77.37 & 77.43 & 78.1 \\
 \hline
 32 & 43.35 & 42.49 & 43.64 & 43.73 & 77.86 & 75.32 & 77.28 & 77.86  \\
 \hline
 16 & 44.21 & 44.21 & 43.64 & 42.97 & 78.65 & 77 & 76.94 & 77.98  \\
 \hline
 \hline
 \multicolumn{5}{|c|}{Winogrande (FP32 Accuracy = 69.38\%)} & \multicolumn{4}{|c|}{Piqa (FP32 Accuracy = 78.07\%)} \\ 
 \hline
 \hline
 64 & 68.9 & 68.43 & 69.77 & 68.19 & 77.09 & 76.82 & 77.09 & 77.86 \\
 \hline
 32 & 69.38 & 68.51 & 68.82 & 68.90 & 78.07 & 76.71 & 78.07 & 77.86  \\
 \hline
 16 & 69.53 & 67.09 & 69.38 & 68.90 & 77.37 & 77.8 & 77.91 & 77.69  \\
 \hline
\end{tabular}
\caption{\label{tab:mmlu_abalation} Accuracy on LM evaluation harness tasks on Llama2-7B model.}
\end{table}

\begin{table} \centering
\begin{tabular}{|c||c|c|c|c||c|c|c|c|} 
\hline
 $L_b \rightarrow$& \multicolumn{4}{c||}{8} & \multicolumn{4}{c||}{8}\\
 \hline
 \backslashbox{$L_A$\kern-1em}{\kern-1em$N_c$} & 2 & 4 & 8 & 16 & 2 & 4 & 8 & 16  \\
 %$N_c \rightarrow$ & 2 & 4 & 8 & 16 & 2 & 4 & 2 \\
 \hline
 \hline
 \multicolumn{5}{|c|}{Race (FP32 Accuracy = 48.8\%)} & \multicolumn{4}{|c|}{Boolq (FP32 Accuracy = 85.23\%)} \\ 
 \hline
 \hline
 64 & 49.00 & 49.00 & 49.28 & 48.71 & 82.82 & 84.28 & 84.03 & 84.25 \\
 \hline
 32 & 49.57 & 48.52 & 48.33 & 49.28 & 83.85 & 84.46 & 84.31 & 84.93  \\
 \hline
 16 & 49.85 & 49.09 & 49.28 & 48.99 & 85.11 & 84.46 & 84.61 & 83.94  \\
 \hline
 \hline
 \multicolumn{5}{|c|}{Winogrande (FP32 Accuracy = 79.95\%)} & \multicolumn{4}{|c|}{Piqa (FP32 Accuracy = 81.56\%)} \\ 
 \hline
 \hline
 64 & 78.77 & 78.45 & 78.37 & 79.16 & 81.45 & 80.69 & 81.45 & 81.5 \\
 \hline
 32 & 78.45 & 79.01 & 78.69 & 80.66 & 81.56 & 80.58 & 81.18 & 81.34  \\
 \hline
 16 & 79.95 & 79.56 & 79.79 & 79.72 & 81.28 & 81.66 & 81.28 & 80.96  \\
 \hline
\end{tabular}
\caption{\label{tab:mmlu_abalation} Accuracy on LM evaluation harness tasks on Llama2-70B model.}
\end{table}

%\section{MSE Studies}
%\textcolor{red}{TODO}


\subsection{Number Formats and Quantization Method}
\label{subsec:numFormats_quantMethod}
\subsubsection{Integer Format}
An $n$-bit signed integer (INT) is typically represented with a 2s-complement format \citep{yao2022zeroquant,xiao2023smoothquant,dai2021vsq}, where the most significant bit denotes the sign.

\subsubsection{Floating Point Format}
An $n$-bit signed floating point (FP) number $x$ comprises of a 1-bit sign ($x_{\mathrm{sign}}$), $B_m$-bit mantissa ($x_{\mathrm{mant}}$) and $B_e$-bit exponent ($x_{\mathrm{exp}}$) such that $B_m+B_e=n-1$. The associated constant exponent bias ($E_{\mathrm{bias}}$) is computed as $(2^{{B_e}-1}-1)$. We denote this format as $E_{B_e}M_{B_m}$.  

\subsubsection{Quantization Scheme}
\label{subsec:quant_method}
A quantization scheme dictates how a given unquantized tensor is converted to its quantized representation. We consider FP formats for the purpose of illustration. Given an unquantized tensor $\bm{X}$ and an FP format $E_{B_e}M_{B_m}$, we first, we compute the quantization scale factor $s_X$ that maps the maximum absolute value of $\bm{X}$ to the maximum quantization level of the $E_{B_e}M_{B_m}$ format as follows:
\begin{align}
\label{eq:sf}
    s_X = \frac{\mathrm{max}(|\bm{X}|)}{\mathrm{max}(E_{B_e}M_{B_m})}
\end{align}
In the above equation, $|\cdot|$ denotes the absolute value function.

Next, we scale $\bm{X}$ by $s_X$ and quantize it to $\hat{\bm{X}}$ by rounding it to the nearest quantization level of $E_{B_e}M_{B_m}$ as:

\begin{align}
\label{eq:tensor_quant}
    \hat{\bm{X}} = \text{round-to-nearest}\left(\frac{\bm{X}}{s_X}, E_{B_e}M_{B_m}\right)
\end{align}

We perform dynamic max-scaled quantization \citep{wu2020integer}, where the scale factor $s$ for activations is dynamically computed during runtime.

\subsection{Vector Scaled Quantization}
\begin{wrapfigure}{r}{0.35\linewidth}
  \centering
  \includegraphics[width=\linewidth]{sections/figures/vsquant.jpg}
  \caption{\small Vectorwise decomposition for per-vector scaled quantization (VSQ \citep{dai2021vsq}).}
  \label{fig:vsquant}
\end{wrapfigure}
During VSQ \citep{dai2021vsq}, the operand tensors are decomposed into 1D vectors in a hardware friendly manner as shown in Figure \ref{fig:vsquant}. Since the decomposed tensors are used as operands in matrix multiplications during inference, it is beneficial to perform this decomposition along the reduction dimension of the multiplication. The vectorwise quantization is performed similar to tensorwise quantization described in Equations \ref{eq:sf} and \ref{eq:tensor_quant}, where a scale factor $s_v$ is required for each vector $\bm{v}$ that maps the maximum absolute value of that vector to the maximum quantization level. While smaller vector lengths can lead to larger accuracy gains, the associated memory and computational overheads due to the per-vector scale factors increases. To alleviate these overheads, VSQ \citep{dai2021vsq} proposed a second level quantization of the per-vector scale factors to unsigned integers, while MX \citep{rouhani2023shared} quantizes them to integer powers of 2 (denoted as $2^{INT}$).

\subsubsection{MX Format}
The MX format proposed in \citep{rouhani2023microscaling} introduces the concept of sub-block shifting. For every two scalar elements of $b$-bits each, there is a shared exponent bit. The value of this exponent bit is determined through an empirical analysis that targets minimizing quantization MSE. We note that the FP format $E_{1}M_{b}$ is strictly better than MX from an accuracy perspective since it allocates a dedicated exponent bit to each scalar as opposed to sharing it across two scalars. Therefore, we conservatively bound the accuracy of a $b+2$-bit signed MX format with that of a $E_{1}M_{b}$ format in our comparisons. For instance, we use E1M2 format as a proxy for MX4.

\begin{figure}
    \centering
    \includegraphics[width=1\linewidth]{sections//figures/BlockFormats.pdf}
    \caption{\small Comparing LO-BCQ to MX format.}
    \label{fig:block_formats}
\end{figure}

Figure \ref{fig:block_formats} compares our $4$-bit LO-BCQ block format to MX \citep{rouhani2023microscaling}. As shown, both LO-BCQ and MX decompose a given operand tensor into block arrays and each block array into blocks. Similar to MX, we find that per-block quantization ($L_b < L_A$) leads to better accuracy due to increased flexibility. While MX achieves this through per-block $1$-bit micro-scales, we associate a dedicated codebook to each block through a per-block codebook selector. Further, MX quantizes the per-block array scale-factor to E8M0 format without per-tensor scaling. In contrast during LO-BCQ, we find that per-tensor scaling combined with quantization of per-block array scale-factor to E4M3 format results in superior inference accuracy across models. 


\end{document}