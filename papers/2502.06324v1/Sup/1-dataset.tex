\section{Dataset Capture and Analysis}
\label{sec:sup_dataset}

\begin{table*}[t]

\centering
% \renewcommand\tabcolsep{5.0pt}
\vspace{-8pt}
%\resizebox{10cm}{!}
\scalebox{0.95}
{      % \textwidth, 10cm, 这里用 10cm 好像效果好于 \textwidth 
\begin{tabular}{c|ccccc}
\toprule[1.25pt]    % 顶部粗线
% \textbf{Mobile Phone} & \textbf{Camera} & \textbf{CMOS} & \textbf{MegaPixel($\text{MP}$)} & \textbf{Optical format($\text{Inches}$)} & \textbf{Pixel Size($\mathbf{\mu m}$)}   \\
Mobile Phone & Camera & CMOS & MegaPixel ($\text{MP}$) & Optical format ($\text{Inches}$) & Pixel Size ($\mu m$)   \\
\midrule
\multirow{1}{*}{iPhone 12}     & Main      & SONY   IMX503            & 12   & 1/2.55 & 1.40  \\
%\hline
\multirow{1}{*}{iPhone 13}     & Main      & SONY IMX603              & 12   & 1/1.88 & 1.70  \\
%\hline
\multirow{1}{*}{Honor 90}      & Main      & ISOCELL HP3              & 200  & 1/1.40 & 0.56  \\
%\hline
\multirow{1}{*}{Xiaomi 10s}    & Main      & ISOCELL HMX              & 108  & 1/1.33 & 0.80  \\
%\hline \cline
\midrule
\multirow{2}{*}{iPhone 12 Pro} & Main      & SONY IMX503              & 12   & 1/2.55 & 1.40  \\
                               & Telephoto & SONY IMX613 (2x zoom)    & 12.2 & 1/3.40 & 1.00  \\
\midrule
\multirow{2}{*}{iPhone 15 Pro} & Main      & SONY IMX803              & 48   & 1/1.28 & 1.22  \\
                               % \cline{2-6}
                               & Telephoto & SONY IMX713 (2x/4x zoom) & 12   & 1/3.40 & 1.00  \\


\bottomrule[1.25pt]  % 底部粗线
\end{tabular}
}
\caption{The mobile phone we apply to get the moiré patterns}
\label{tab:Datasets-Phone}
\end{table*}


\begin{table*}[t]

\centering
% \renewcommand\tabcolsep{5.0pt}
\vspace{-8pt}
% \resizebox{10cm}{!}
\scalebox{1.0}
{      % \textwidth, 10cm, 这里用 10cm 好像效果好于 \textwidth 
\begin{tabular}{c|ccccc}
%\toprule[1.2pt]    % 顶部粗线
\toprule[1.25pt]
% \textbf{Digital Screen}  & \textbf{Size($\text{Inches}$)} & \textbf{Panel type}  & \textbf{Resolution} & \textbf{Brightness($\mathbf{cd/m^2}$)} & \textbf{Coating} \\
Digital Screen    & Size ($Inches$) & Panel type        & Resolution         & Brightness ($cd/m^2$) & Coating \\
\midrule
DELL D2720DS      & 27              & IPS(LED)          & 2560 $\times$ 1440 & 350                   & Matte      \\
Macbook Air 2022  & 13.3            & IPS(LED)          & 2560 $\times$ 1600 & 500                   & Glossy     \\
AOC 27G2G8        & 27              & IPS(W-LED)        & 2560 $\times$ 1440 & 250                   & Matte      \\
Philips 27E1N5500 & 27              & IPS(LED)          & 2560 $\times$ 1440 & 300                   & Matte      \\
Xiaomi C34WQBA-RG & 34              & Curved SVA(W-LED) & 3440 $\times$ 1440 & 300                   & Matte      \\
ViewSonic VX2771-4K-HD & 27         & IPS(LED)          & 3840 $\times$ 2160 & 350                   & Matte      \\
%\bottomrule[1.2pt]  % 底部粗线
\bottomrule[1.25pt]
\end{tabular}
}
\caption{The screen we apply to get the moiré patterns}
\label{tab:Datasets-Screen}
\end{table*}


% \begin{table}[]
% \centering
% \caption{The mobile phone(left) and screen(right) we apply to get the moiré patterns}
% % \begin{subtable}{0.5\linewidth}
% \resizebox{\linewidth}{!}{
% \begin{tabular}{c|cc}
% \toprule    % 顶部粗线
% Mobile Phone & Camera & CMOS \\
% \midrule
% \multirow{1}{*}{iPhone 12}     & Main      & SONY   IMX503        \\
% %\hline
% \multirow{1}{*}{iPhone 13}     & Main      & SONY IMX603          \\
% %\hline
% \multirow{1}{*}{Honor 90}      & Main      & ISOCELL HP3          \\
% %\hline
% \multirow{1}{*}{Xiaomi 10s}    & Main      & ISOCELL HMX           \\
% %\hline \cline
% \midrule
% \multirow{2}{*}{iPhone 12 Pro} & Main      & SONY IMX503            \\
%                                & Telephoto & SONY IMX613 (2x zoom)  \\
% \midrule
% \multirow{2}{*}{iPhone 15 Pro} & Main      & SONY IMX803            \\
%                                % \cline{2-6}
%                                & Telephoto & SONY IMX713 (2x/4x zoom)  \\


% \bottomrule  % 底部粗线
% \end{tabular}
% }
% % \end{subtable}
% % \hfill
% % \begin{subtable}[t]{0.5\linewidth}
% \resizebox{\linewidth}{!}{
% \begin{tabular}{c|cc}
% \toprule
% Digital Screen    & Size ($Inches$) & Panel type        \\
% \midrule
% DELL D2720DS      & 27              & IPS(LED)          \\
% Macbook Air 2022  & 13.3            & IPS(LED)          \\
% AOC 27G2G8        & 27              & IPS(W-LED)        \\
% Philips 27E1N5500 & 27              & IPS(LED)          \\
% Xiaomi C34WQBA-RG & 34              & Curved SVA(W-LED)               \\
% ViewSonic VX2771-4K-HD & 27         & IPS(LED)          \\
% \bottomrule
% \end{tabular}
% }
% % \end{subtable}

% \label{tab:Datasets-Screen-Phone}
% \end{table}

In this section, we first present a brief introduction of various previously overlooked factors of devices that affect moiré pattern diversity. Then, we provide more details about our capture settings.

\subsection{The Impact of Device on Moiré Pattern Diversity}

% When taking a picture of a screen with a camera, there are two different frequencies to consider. The first one is the sampling frequency or resolution, which refers to the level of detail captured by the camera's sensor. The second one is the signal frequency of the screen itself, which determines how quickly the screen refreshes its content. Due to the optics of the lens and the color filters of the camera's CMOS sensor, the frequency at which we sample during a shot is reduced. According to the \blue{Shannon-Nyquist sampling theorem}, when the sampling frequency is less than twice the signal frequency, the sampled information becomes aliased, resulting in the moiré effect. 

Previous studies~\cite{yu2022towards, yang2023doing} have indicated that the geometric correlation between the screen and the camera significantly influences the features of the moiré pattern. However, such studies have overlooked that some aspects of the camera and the screen can also impact the moiré pattern.

For cameras, the two most critical factors affecting the moiré pattern are the CMOS and the lens used. The pixel density of a CMOS sensor (i.e., the number of pixels per unit area) determines its maximum sampling frequency, also known as the Nyquist frequency. The higher the pixel density, the higher the sampling frequency of the sensor and the higher the frequency of the signal that can be sampled, resulting in a higher frequency of moiré produced by the aliasing effect, which impacts the moiré pattern. In addition, the lens's focal length also affects the formation of moiré. In cell phone photography, lenses with shorter focal lengths (e.g., wide lenses/main camera lenses) usually have wider angles of view and can capture more of the scene content. Lenses with longer focal lengths (such as telephoto or telescopic lenses), on the other hand, offer a narrower angle of view and greater magnification for capturing distant details. When the screen is photographed with lenses of different focal lengths, the relative positional relationship between the pixels on the sensor and the pixels on the screen changes, which may cause the moiré pattern to appear or disappear.


\begin{figure}[t]
  \centering
    \includegraphics[width=1.0\linewidth]{Images/Distribution.pdf}
  \caption{Comparison of sharpness and colorfulness between our Moiré Pattern Dataset and MoireSpace.} 
  \label{fig:distribution}
\end{figure}


\begin{figure*}[!p]
  \centering
    \includegraphics[width=1.0\linewidth]{Images/Capture_Data.pdf}
  \caption{Samples from MoireSpace~\cite{yang2023doing} and our 4K Moiré Pattern Dataset.}
  \label{fig:capture_data}
\end{figure*}

Furthermore, the layout and the distance of pixels dots in the panel used can also significantly impact the formation of moiré on the display screen. The frequency of detail that a screen can display depends on how the pixel dots are arranged. Various arrangements result in distinct frequencies of detail, which impacts the formation of moiré patterns. The distance between pixel dots on the screen then affects the shooting distance. Larger pixel dot spacing will make the distance at which the moiré is formed to be photographed farther away. Conversely, the smaller the distance between pixel dots, the closer the distance needed to photograph the molded moiré.

\subsection{More Details about Capture Settings}
Based on the above analysis, we take screen images through different camera viewpoints to generate diverse moiré patterns. Specifically, we apply six mobile phones and six digital screens, as shown in Table~\ref{tab:Datasets-Phone} and ~\ref{tab:Datasets-Screen} ($\text{6} \times \text{6} = \text{36}$ combinations). 
% Figure~\ref{fig:capture_data} shows a comparison of some of our captured 4K moiré patterns with some samples from the MoireSpace dataset. It can be seen that our captured moiré patterns are much better than the MoireSpace dataset in terms of the clarity of the moiré texture and the vividness of the colors.
Figure~\ref{fig:distribution} presents a scatter plot comparison between our 4K Moiré Pattern Dataset and the MoireSpace dataset in terms of texture definition and color vibrancy, where each data point represents an individual sample. As illustrated in the visualization, our captured moiré patterns demonstrate a more comprehensive coverage in both the sharpness of textural details and the chromatic saturation range. This comparative analysis is further substantiated by the representative samples juxtaposed in Figure~\ref{fig:capture_data}, where side-by-side comparisons visually confirm that our dataset encompasses a broader diversity of morphological textures and a wider gamut of color expressions.

\subsubsection{Mobile Phones} 
We chose six mobile phones with varying camera specifications to capture diverse moiré patterns, as shown in Table~\ref{tab:Datasets-Phone}. Our selection criteria included the camera type, CMOS category, and number of megapixels. For the regular main camera with moderate resolution, we picked the iPhone 12 and iPhone 13. For electronic zooming at 2x and 3x, we selected the Honor 90 and Xiaomi 10s, which have high pixels. Additionally, we picked two iPhone 12 Pro and iPhone 15 Pro models with different CMOS specifically for telephoto lenses. These models use the telephoto lens for optical zoom at fixed magnifications of 2x and 3x.

\subsubsection{Display Screens} 
To capture a wider variety of moiré patterns in different forms, we selected display screens based on size, panel type, and resolution guidelines to maximize pixel point layouts and spacing on the screen. As shown in Table~\ref{tab:Datasets-Screen}, we have selected three 27-inch IPS panel LED matte screen monitors with a 2K resolution - DELL D2720DS, AOC 27G2G8, and Philips 27E1N5500. This specification is the most common among the available options. The AOC 27G2G8 is a W-LED monitor with an RGBW pixel layout. This IPS screen has white sub-pixels in addition to the standard RGB arrangement, creating a more varied pixel point layout. To capture the moiré pattern on the glossy display, we opted for a 13.3-inch IPS panel with a 2K resolution MacBook Air notebook. Finally, we selected two high-resolution displays: the Xiaomi C34WQBA-RG and the ViewSonic VX2771-4K-HD. These displays were explicitly chosen to capture moiré patterns with smaller pixel dot spacing. The Xiaomi C34WQBA-RG is a 34-inch curved display with an SVA panel and W-LED technology. It boasts a 3K resolution. On the other hand, the ViewSonic VX2771-4K-HD is a 27-inch matte screen display with an IPS panel and LED technology. It offers a standard 4K resolution.
