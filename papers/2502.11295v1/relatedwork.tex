\section{Related work}
Game-theoretic approaches to decision-making have been extensively studied in various domains, including artificial intelligence, economics, and cybersecurity. Classical works in game theory, such as those by von Neumann and Morgenstern, laid the foundation for strategic reasoning in adversarial settings \cite{berliner1980backgammon}. More recent advances have focused on leveraging machine learning to enhance decision-making in competitive environments \cite{hassabis2016alphago,mikolov2013efficient}.

In the field of artificial intelligence, Minimax search with Alpha-Beta pruning has been widely used in strategic decision-making, particularly in board games such as chess and Go \cite{berliner1980backgammon,hassabis2016alphago}. Monte Carlo Tree Search (MCTS) has emerged as a powerful alternative, demonstrating remarkable success in complex, high-dimensional decision spaces \cite{hassabis2016alphago}.

From a business and economics perspective, adversarial game models have been applied to strategic planning and market competition. Researchers have explored how organizations can anticipate competitor actions and optimize decision-making through predictive modeling \cite{boella2008reasoning,allen1983maintaining}. Process mining techniques have also been utilized to extract actionable insights from business process data \cite{ProM,HSRS}.

The role of requirements engineering in adversarial decision-making has also been explored. Studies have investigated how non-functional requirements, such as security and compliance, influence strategy formulation \cite{cleland2006detection,JDDS,GSSD}. Additionally, the use of data-driven approaches to inform business strategies has gained prominence, with studies emphasizing the importance of mining customer requirements and analyzing behavioral patterns \cite{QZJZ,SCJM,GP,KDFR}.

Multi-agent systems and reasoning about norms have been key areas of research in adversarial settings, particularly in regulatory compliance and institutional decision-making \cite{blevi2015discovery,teinemaa2015diagnostics}. The study of sequential pattern mining has also provided insights into identifying strategic trends over time \cite{FZLC,SG}.

In summary, while significant research has been conducted in game theory, machine learning, and strategic decision-making, our work uniquely integrates these perspectives into a unified computational framework for adversarial decision-making in business strategy. By leveraging both classical and modern game-theoretic techniques, we aim to enhance the resilience of strategic execution plans in competitive environments.

The next section presents our proposed framework, detailing its theoretical underpinnings and methodological approach.