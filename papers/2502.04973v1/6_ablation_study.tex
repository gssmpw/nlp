\section{Ablation Study}
\label{sec:ablation_study}
In this section, we analyze the contribution of each component of the proposed DE-PADA model.
To achieve this, we perform an ablation study by systematically removing individual components from the final model to assess their impact on the results for both the sit and exercise phases.

As a reminder, the training of the DE-PADA model consists of two stages: in Stage \uppercase\expandafter{\romannumeral 1}, the backbone PQRS and ST models were trained separately on the Target set with personalized augmentation to optimize feature extraction; in Stage \uppercase\expandafter{\romannumeral 2}, the classifier was trained with domain adaptation on genuine, non-augmented data to enhance its ability to learn invariant features across different conditions.

The ablation study is conducted in two primary scenarios. In the first scenario, we evaluate our approach by training the classifier only with domain adaptation, without augmentation, consistent with the methodology used up to this point.
In the second scenario, we examine the results when the classifier is trained with both domain adaptation and personalized augmentation. The augmentation was limited to the Target set since the Auxiliary set has partial representation across sessions and lacks the required data for T-peak range calculations, which are integral to our personalized augmentation method. It's important to note that all the evaluations were conducted on the Target set, consistently with previous evaluations, with the reported IDR being the average of 10 runs.

The results are reported for four different cases, to isolate the effects of each component:
\begin{itemize}
    \item \textbf{DE-PADA}: The complete DE-PADA model, trained with all proposed components, serves as the reference for comparison.
    \item \textbf{DE-PADA\textbackslash DE}: The Standard CNN model is used instead of the DE model, removing the benefit of handling different segments separately.
    \item \textbf{DE-PADA\textbackslash PA}: The DE-PADA model was trained without personalized augmentation at any stage.
    \item \textbf{DE-PADA\textbackslash DA}: The DE-PADA model was trained without using domain adaptation.
\end{itemize}

\subsection{Non-Augmented Classifier}
The results of the ablation study, when the classifier is trained without augmentation, are presented in Table~\ref{tab:ablation_notaugmented}.
This table illustrates the impact of removing various components from the DE-PADA model without using augmented data in the classifier's training stage.

Comparing the DE-PADA\textbackslash DE model to the full DE-PADA model, we observe a significant decrease in performance for the sit position (95.55\% vs. 98.12\%), along with a slight increase in performance for Ex\_P1 (69.59\% vs. 68.95\%).
This suggests that the DE model is particularly effective at preserving performance in low-variability scenarios, such as the sit position.
In high-variability conditions, the DE model offers only marginal improvements in Ex\_P2 and a slight reduction in Ex\_P1.
These findings highlight the advantage of processing different ECG segments separately in the DE model, which is particularly valuable for maintaining robustness across various conditions.

For the DE-PADA\textbackslash PA model, where training was conducted without personalized augmentation, the highest performance is observed in the sit condition (98.52\%), slightly outperforming the full DE-PADA model (98.12\%). This aligns with the presented results on conventional augmentation, which suggest that augmentation in general can negatively impact performance in scenarios with sufficient training data and low variability. However, in this case, the reduction in performance is relatively small. Notably, the absence of personalized augmentation causes a substantial decline in performance for exercise phases, with Ex\_P2 at 81.34\% and Ex\_P1 at 59.18\%. These results underscore the importance of personalized augmentation in managing heart rate variability, as its exclusion significantly compromises the model's effectiveness in high-variability settings.

The DE-PADA\textbackslash DA model, which was trained without domain adaptation, also shows reduced performance compared to the full DE-PADA model, particularly in Ex\_P2 (80.70\%) and Ex\_P1 (59.48\%), with a minor decrease in the sit condition (97.94\%). This indicates that domain adaptation, similarly to augmentation, is crucial for enhancing the model's generalization ability, especially in conditions involving elevated heart rates and varying postures.
%%%%%%%%%%%%%%% Table Start %%%%%%%%%%%%%%%
\begin{table}[!t]
    \centering
    \caption{Ablation study results when the classifier is trained without augmentation.}
    \label{tab:ablation_notaugmented}
    \begin{tabular}{lccc}
        \hline
        Method                   & Sit              & Exercise Phase 2  & Exercise Phase 1  \\
        \hline
        DE-PADA                  & \textbf{98.12}\% & \textbf{86.45\%} & 68.95\%          \\
        DE-PADA\textbackslash DE & 95.55\%          & 84.94\%          & \textbf{69.59\%} \\
        DE-PADA\textbackslash PA & \textbf{98.52\%} & 81.34\%          & 59.18\%          \\
        DE-PADA\textbackslash DA & 97.94\%          & 80.70\%          & 59.48\%          \\
        \hline
    \end{tabular}
\end{table}
%%%%%%%%%%%%%%% Table End %%%%%%%%%%%%%%%
\subsection{Augmented Classifier}
In the second scenario, the classifier was trained with both domain adaptation and personalized augmentation, with the augmentation limited to the Target set as mentioned above.

The results presented in Table~\ref{tab:ablation_augmented} show that the full DE-PADA model achieves the highest performance for exercise phases, with IDR values of 86.28\% for Ex\_P2 and 71.78\% for Ex\_P1. This demonstrates the effectiveness of combining domain adaptation with personalized augmentation in managing high-variability data. The model also surpasses the non-augmented variant from the previous section in Ex\_P1 due to the inclusion of augmentation for the classifier, and the generation of synthetic elevated heart rate examples. However, the decrease in performance for the sit condition reflects the trade-off discussed in Subsection~\ref{chap:sit_ex_results}.

The importance of the DE model is further highlighted in this scenario, as the DE-PADA\textbackslash DE model exhibits a decline in performance across all conditions.
Similarly, the model trained without domain adaptation shows reduced performance, although the reduction in exercise phases is less pronounced than in the non-augmented scenario, as the inclusion of augmentation helps the classifier handle exercise phases more effectively.

Notably, in all models where Stage \uppercase\expandafter{\romannumeral2} included augmentation, performance on the sit position decreased. The DE-PADA\textbackslash PA model, identical to the one in the previous section, achieves the highest performance on the sit position.
%%%%%%%%%%%%%%% Table Start %%%%%%%%%%%%%%%
\begin{table}[!t]
    \centering
    \caption{Ablation study results when the classifier is trained with personalized augmentation.}
    \label{tab:ablation_augmented}
    \begin{tabular}{lccc}
        \hline
        Method                   & Sit              & Exercise Phase 2  & Exercise Phase 1  \\
        \hline
        DE-PADA                  & 96.39\%          & \textbf{86.28\%} & \textbf{71.78\%} \\
        DE-PADA\textbackslash DE & 95.37\%          & 82.57\%          & 67.09\%          \\
        DE-PADA\textbackslash PA & \textbf{98.52\%} & 81.34\%          & 59.18\%          \\
        DE-PADA\textbackslash DA & 96.33\%          & 84.32\%          & 65.33\%          \\
        \hline
    \end{tabular}
\end{table}
%%%%%%%%%%%%%%% Table End %%%%%%%%%%%%%%%


We hypothesize that the partial accounting for heart rate changes introduced through augmentation is particularly beneficial in scenarios with large performance gaps, as observed in related studies that use augmentation in low-data-availability settings. This explains the observed improvement in exercise performance.
However, in cases where the training set contains adequate amounts of representative data, achieving a high initial IDR, the augmentation's inability to fully replicate the authentic changes in the ECG waveform may hinder performance.