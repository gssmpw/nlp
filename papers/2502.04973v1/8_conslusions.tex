\section{Conclusions}
In this paper, we addressed the challenge of ECG-based user identification across varying body postures and physiological states, particularly under post-exercise conditions with elevated heart rates. We proposed a comprehensive approach that combines a novel Dual Expert (DE) model with Personalized Augmentation and Domain Adaptation (DE-PADA) to effectively handle the intra-subject variability of ECG signals across diverse conditions. Each of these three components leverages the morphological characteristics of ECG signals to achieve robust identification performance, and combined, they significantly surpassed the reference models in all tested scenarios.

We proposed a Dual Expert (DE) architecture that separately attended to the PQRS and ST intervals, effectively preserving performance in resting states. We introduced a Personalized Augmentation algorithm that augments the ST interval within predicted subject-specific ranges, significantly improving identification under exercise conditions. Additionally, we presented a domain adaptation variant that utilizes data from additional subjects with both resting and active state data. This approach enabled the classifier to learn patterns common to the population including the Target set subjects, thereby enhancing its generalization ability.

The DE-PADA model consistently outperformed the baseline models across all tested conditions. It achieved notable improvements in identification rates, increasing from 77.38\% to 86.45\% for Exercise Phase 2 and from 54.4\% to 68.95\% for Exercise Phase 1 compared to the standard reference model. In addition, the DE-PADA model maintained high accuracy in stable resting conditions such as sitting, achieving an identification rate of 98.12\%, which not only countered the reduction observed in the augmented reference model but also surpassed the baseline performance.

Furthermore, we analyzed the effect of personalized augmentation on the feature space of the ST model and demonstrated its effectiveness in reducing some of the intra-subject variability and creating a more compact feature space. However, after t-SNE dimensionality reduction, it remains evident that features from the sitting position and exercise phases are still clustered separately, indicating that additional methods, such as the proposed domain adaptation, can further reduce this gap.