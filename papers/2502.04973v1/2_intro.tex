\section{Introduction}

\IEEEPARstart{E}{lectrocardiogram} (ECG) signals have been widely recognized as a highly promising biometric modality due to their hidden nature and unique liveness detection capabilities, which provide a significant advantage over traditional biometric systems such as fingerprint or facial recognition \cite{Uwaechia2021}. These traditional systems are particularly vulnerable to sophisticated spoofing attacks, as they lack intrinsic liveness verification, which is crucial for ensuring that the biometric trait being analyzed belongs to a living individual.

The physiological and anatomical differences in heart structure, reflected in distinctive ECG waveform morphology, make ECG signals inherently unique and strong candidates for biometric applications \cite{interindividualECG, 8392675}. Their non-invasive nature and low-cost acquisition further reinforce their potential for widespread use.

Despite these advantages, ECG-based biometric systems face significant challenges in physiologically variable conditions.
While prior studies demonstrate high accuracy when users are in a controlled, resting state sitting position, their performance degrades in other conditions, with a significant degradation in non-resting states, such as post-exercise \cite{Wahabi2014, Hwang2021, Jyotishi2022}.
This decline in performance is primarily due to variability in ECG waveforms caused by numerous factors, including body posture, physical activity, and emotional state, all of which can induce substantial changes in the signal's morphology. These variations present considerable challenges for pattern recognition algorithms, which are expected to accurately identify individuals despite these fluctuations.
Pathoumvanh et al. \cite{Pathoumvanh} demonstrated that the performance decline is strongly correlated with heart rate changes, showing that even a moderate 40\% increase in heart rate due to physical activity can lead to a performance decrease of over 30\%.

In this work we address the challenge of user identification under elevated heart rates, 
using post-exercise conditions as a representative scenario. Such conditions are essential to consider for the development of practical ECG-based biometric systems, as they reflect the physiological changes that naturally occur during everyday activities.

%%%%%%%%%%%%%%%%
The variability in ECG signals is mainly observed in the duration of heartbeat intervals and their amplitudes \cite{sornmo2005bioelectrical}, with the interval duration being inversely correlated with heart rate.
Various approaches have been proposed to address this variability, leveraging both traditional signal processing techniques and advanced machine learning models, particularly deep learning. Several studies have adjusted the duration of specific heartbeat intervals to mitigate intra-subject variability \cite{Hwang2021, heartID, Arteaga-Falconi2016, Choi2020, 5580317}. In contrast, deep learning-based methods often utilize data augmentation strategies to increase the variability in training data, enhancing model robustness to high variability \cite{8219706, Um_2017, Kim2022}.
However, most of these studies have focused on scenarios with limited data availability in resting conditions or relied on private databases that include exercise conditions.

The main contributions of this work are as follows.

\begin{itemize}

    \item Introduction of a Dual-Expert Model: We propose a novel Dual-Expert model that incorporates prior knowledge on ECG into the deep neural network architecture, enhancing its ability to handle the complexities of ECG signal variability.
    
    \item Personalized Augmentation: We introduce a subject-specific augmentation technique that mitigates the performance degradation observed in resting states with conventional augmentation methods, while also improving computational efficiency.
    
    \item Domain Adaptation for various conditions: We propose a domain adaptation variant to enhance the classifier’s generalization across diverse physiological conditions.

\end{itemize}

The combination of the contributions above form The Dual Expert with Personalized Augmentation and Domain Adaptation (DE-PADA) model, which demonstrates notable performance improvements on the University of Toronto ECG Database (UofTDB).
DE-PADA achieves a relative increase of 26.75\% in identification rates in the post-exercise initial recovery phase and 11.72\% in the late recovery phase, while maintaining a 98.12\% identification rate in the sitting position.
The effectiveness of each contribution is further detailed in Section~\ref{sec:ablation_study}, highlighting their integral role in enhancing the model's performance across diverse physiological states.