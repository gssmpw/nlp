\documentclass[journal]{IEEEtran}
\usepackage{cite}
\usepackage[pdftex]{graphicx}
\usepackage{amsmath}
\usepackage{algorithmic}
\ifCLASSOPTIONcompsoc
 \usepackage[caption=false,font=normalsize,labelfont=sf,textfont=sf]{subfig}
\else
 \usepackage[caption=false,font=footnotesize]{subfig}
\fi

%%%%%%%%%%%%%%%%%% Manually Added Start %%%%%%%%%%%%%%%%%%
\usepackage{amssymb,amsfonts}
\usepackage{xcolor}
\usepackage{booktabs}
\usepackage{algorithm}
\usepackage[hidelinks]{hyperref}
\usepackage{orcidlink}
\newcommand{\myhyperlabel}[1]{\hypertarget{#1}{}}
\newcommand{\figref}[2][]{\hyperlink{#2}{\figurename~\ref*{fig:#2}#1}}
%% Algo-Figure counter workaround
\newcounter{algocf}
% New command for algorithm captions (mimicking \caption)
\newcommand{\algcaption}[1]{%
  \begingroup % Start a local group
  \addtocounter{algocf}{1}% increase the algo counter
  \renewcommand{\figurename}{Algorithm} % temporarily override \figurename
  \renewcommand{\thefigure}{\thealgocf} % temporarily override the counter
  \caption{#1}
  \addtocounter{figure}{-1}% decrease the figure counter
  \endgroup % End the local group
}
%%%%%%%%%%%%%%%%%% Manually Added End %%%%%%%%%%%%%%%%%%


\begin{document}

\title{DE-PADA: Personalized Augmentation and Domain Adaptation for ECG Biometrics Across Physiological States}


\author{Amro~{Abu~Saleh}\textsuperscript{1}\,\orcidlink{0009-0004-3759-2270},
        Elliot~Sprecher\textsuperscript{2}\,\orcidlink{0000-0001-8564-1090},
        Kfir~Y.~Levy\textsuperscript{1}\,\orcidlink{0000-0003-1236-2626},
        and~Daniel~H.~Lange\textsuperscript{1}\,\orcidlink{0009-0008-2650-6153}% 
\thanks{

\textsuperscript{1} Department of Electrical and Computer Engineering, Technion Israel Institute of Technology}%
\thanks{\textsuperscript{2} Independent Researcher.}% <-this % stops a space
\thanks{Corresponding author: samro@campus.technion.ac.il}% <-this % stops a space
}

% make the title area
\maketitle

\begin{abstract}
\begin{abstract}

We introduce \ours, a novel framework for scene-level appearance transfer from a single style image to a real-world scene represented by multiple views. The method combines explicit semantic correspondences with multi-view consistency to achieve precise and coherent stylization.
Unlike conventional stylization methods that apply a reference style globally, \ours uses open-vocabulary segmentation to establish dense, instance-level correspondences between the style and real-world images. This ensures that each object is stylized with semantically matched textures.
\ours first transfers the style to a single view using a training-free semantic-attention mechanism in a diffusion model.
It then lifts the stylization to additional views via a learned warp-and-refine network guided by monocular depth and pixel-wise correspondences.
Experiments show that \ours consistently outperforms prior methods in structure preservation, perceptual style similarity, and multi-view coherence.
User studies further validate its ability to produce photo-realistic, semantically faithful results.
Our code, pretrained models, and dataset will be publicly released, to support new applications in interior design, virtual staging, and 3D-consistent stylization.

\end{abstract}

\end{abstract}

\begin{IEEEkeywords}
Electrocardiogram (ECG), biometrics, deep learning, personalized  augmentation, domain adaptation
\end{IEEEkeywords}

\section{Introduction}

\par Autonomous Vehicles (AV) employ various sensors for comprehensive navigation and environmental perception, each contributing distinct advantages and limitations. RGB cameras are attractive due to their affordability and high accuracy in optimal lighting conditions but they struggle in low-visibility scenarios like darkness, heavy rain, or fog, which impede obstacle detection. Lidars offer precise 3D mapping and depth measurements, yet they consume high power, are susceptible to adverse weather conditions, and sometimes require mechanical parts for rotational scanning, raising the overall manufacturing and operational costs. Radars, however, stand out for their robustness in adverse weather, cost-effectiveness, low power consumption, and can offer a theoretical angular resolution comparable to lidars. 

\begin{figure}[t!]
  \centering
  \includegraphics[width=\linewidth]{figures/pixel_image_space1.pdf}
  \caption{Conceptual figure depicting the proposed spectrum-based transformation. The 'RGB' and 'Depth Maps' subspaces represent natural images captured by conventional cameras and the windshield view depth maps that lidars and radars produce, respectively. The 'Spatial Spectrum' subspace includes the special frequency 2D spectrum of bases that constitute images in the other subspaces.}
  \label{fig: image space}
  \vspace{-22px} % Adjust this value to reduce the space
\end{figure}

\par Automotive radars measure range, azimuth, and velocity, with 4D radars adding elevation to their traditional measurements. Unlike 3D radars, 4D radars are capable of estimating object heights without using speculative models. For such radars, extracting precise angular information involves a two-stage process: spatial spectrum estimations followed by Constant False Alarm Rate (CFAR) detectors like cell-averaging CFAR (CA-CFAR) and order-statistic CFAR (OS-CFAR) \cite{richards2010principles}. However, these conventional methods struggle in complex vehicular settings, producing sparse point clouds that limit accurate environmental representation \cite{khan2022comprehensive}. To address these limitations, data-driven approaches using deep neural networks (DNNs) have been reported \cite{brodeski2019deep, cheng2022novel, roldan2024see}. For instance, \cite{roldan2024see} employs a ResNet18 network \cite{chen2017rethinking} trained on dense lidar point clouds, generating denser radar point clouds that more accurately represent object shapes and sizes.

\par In this paper, we introduce a data-driven approach for generating radar depth maps by integrating radar point clouds with camera images. Leveraging the similar field-of-view (FoV) between radar and camera images, employing a non-linear frequency pixel positional encoding algorithm and Bartlett's spatial spectrum estimation \cite{Bartlett1948} transforms radar depth maps and camera RGB images into a shared spatial spectrum subspace, as shown in Figure \ref{fig: image space}. This transformation can resolve the differences between the 4D radar image and camera images, thus enabling spectrum-based learning. The method enables the use of high-resolution cameras to effectively train radar depth map generators. After this off-line training, the 4D radar model can operate independently of the camera, generating sharper and denser depth maps that are critical for perception, tracking, and rendering in AVs. Our contributions can be summarized as:

\begin{itemize}
\item We propose a pixel positional encoding algorithm that helps resolve the differences between a 4D radar image and RGB camera image, thus enabling spectrum-based learning for 4D radar images.

% \item We introduce a holistic framework for producing depth map generative models that are based on the newly developed DNN detector in \cite{roldan2024see} as a sub-module. Although the DNN detector is trained on lidar data, our framework does not require further training on lidar data and camera images for operation.

\item We present experimental results for high-resolution spectrum estimations and depth map generations. Our results show that our approach is capable of producing qualitatively sharp depth maps and significantly outperforms the state-of-the-art (SOTA), resulting in a reduction of 21.13\%, 7.9\% and 27.95\% in Mean Absolute Error (MAE), Relative Absolute Error (REL), and Unidirectional Chamfer Distance (UCD), respectively, which is quite significant.\cite{zhang2021unsupervised}. We also show that the estimated spectrum of camera and radar images results in an increase of the Pearson correlation and mutual information by a factor of 3.88 and 76.69, respectively.

\end{itemize}
\section{Related Work}
\subsection{Deep Learning based Weather Forecasting}
\textbf{Global Weather Forecasting.} Global weather forecasting has seen significant progress with deep learning models. FourCastNet, based on Fourier neural operators, provides global forecasts comparable to traditional numerical methods like IFS, but at much higher speeds~\cite{pathak2022fourcastnet}. Pangu, utilizing the Swin Transformer, exceeds NWP methods, incorporating earth-specific location embeddings for better performance~\cite{bi2023accurate}. The Spherical Fourier Neural Operator (SFNO) extends Fourier methods using spherical harmonics, offering more stable long-term predictions~\cite{bonev2023spherical}. FuXi focuses on long-term forecasting, achieving a 15-day forecasts comparable to ECMWF~\cite{chen2023fuxi}. GraphCast leverages message-passing networks to improve efficiency and forecasting accuracy~\cite{lam2023learning}, and GenCast builds on this to enhance ensemble forecasting~\cite{price2023gencast}. Further, diffusion models like those in~\cite{li2024generative} generate probabilistic ensembles by sampling, while NeuralGCM~\cite{kochkov2024neural} focuses on atmospheric circulation with a dynamic core, offering climate simulation capabilities but at higher training and inference costs. 

\textbf{Regional Weather Forecasting.} The goal of regional weather forecasting is to enhance local prediction accuracy with high-resolution models. CorrDiff~\cite{mardani2023generative} combines U-Net and diffusion models to improve local forecasts. MetaWeather~\cite{kim2024metaweather} adapts global forecasts to regional contexts using meta-learning. GNNs are also widely applied in regional forecasting, with Graphcast~\cite{lam2023learning} enhancing accuracy by modeling complex spatial dependencies. MetNet-3~\cite{espeholt2022deep} offers high-accuracy forecasts for weather variables, such as precipitation, temperature, and wind speed, at 2-minute intervals and 1–4 km resolution, outperforming traditional models like HRRR. NowcastNet~\cite{zhang2023skilful} and DGMR~\cite{ravuri2021skilful} excel in short-term extreme precipitation forecasts using deep generative models and radar data. In spatiotemporal prediction, NMO~\cite{wu2024neural} models the evolution of physical dynamics, providing new insights for local weather forecasting. Similarly, SimVP~\cite{gao2022simvp} and PastNet~\cite{wu2024pastnet} achieve good results in forecasting local precipitation evolution using spatiotemporal convolution methods.
    
% Despite these advances, none of these methods effectively address the challenge of balancing global and regional high-resolution forecasts or capturing the fine-grained, dynamic interactions important for extreme event prediction.
    
\subsection{Numerical analysis methods}
Multigrid methods~\cite{mccormick1987multigrid,wesseling1995introduction,hackbusch2013multi,bramble2019multigrid,hiptmair1998multigrid,brandt1983multigrid,borzi2009multigrid} and nested grid strategies~\cite{miyakoda1977one,zhang2012nested,sullivan1996grid} are widely used to solve PDEs and handle multi-scale problems~\cite{debreu2008two,xue2000advanced}. Multigrid methods use grids of different resolutions to transfer information and accelerate iterations. They efficiently solve large-scale problems and improve computational accuracy. By eliminating low-frequency errors on coarse grids and high-frequency errors on fine grids, multigrid methods effectively handle error convergence at different scales~\cite{he2019mgnet,he2023mgno,shao2022fast}. Nested grid strategies embed higher-resolution fine grids into regions of interest based on a global coarse grid to capture local complex physical phenomena in detail. In weather forecasting, this method provides large-scale background fields on a global scale while refining the grid for target regions to accurately simulate the evolution of local weather systems and the occurrence of extreme events~\cite{bacon2000dynamically}. 

% Our proposed neural nested grid method helps address challenges like boundary information loss in regional forecasting and multi-scale feature capture.

\section{Additional Results}
%
We present more additional results in Figure \ref{fig_0.25-day}, \ref{fig_0.5-day}, \ref{fig_1.0-day} \ref{fig_1.5-day}, \ref{fig_2.0-day}, \ref{fig_2.5-day}, \ref{fig_3.0-day}, \ref{fig_3.5-day}, \ref{fig_4.0-day}, \ref{fig_4.5-day}, \ref{fig_5.0-day}, \ref{fig_5.5-day}, \ref{fig_6.0-day}, \ref{fig_6.5-day}, \ref{fig_7.0-day}, \ref{fig_7.5-day},
\ref{fig_8.0-day}, \ref{fig_8.5-day}, \ref{fig_9.0-day}, \ref{fig_9.5-day},
\ref{fig_10.0-day}, including 18 variables that are importmant to weather forecasting, each with results ranging from 6 hours to 10 days. These additional results further demonstrate the effectiveness of OneForecast. Same as the Figure \ref{fig:visual_results}
, the initial conditions is 00:00 UTC, 1 January 2020.


\begin{figure*}[h]
\centering
\includegraphics[width=1\linewidth]{figures/fig_0.25-day.jpg}
\vspace{-20pt}
\caption{6-hour forecast results of different models.}
\label{fig_0.25-day}
\end{figure*}

\begin{figure*}[h]
\centering
\includegraphics[width=1\linewidth]{figures/fig_0.5-day.jpg}
\vspace{-20pt}
\caption{0.5-day forecast results of different models.}
\label{fig_0.5-day}
\end{figure*}

\begin{figure*}[h]
\centering
\includegraphics[width=1\linewidth]{figures/fig_1.0-day.jpg}
\vspace{-20pt}
\caption{1-day forecast results of different models.}
\label{fig_1.0-day}
\end{figure*}

\begin{figure*}[h]
\centering
\includegraphics[width=1\linewidth]{figures/fig_1.5-day.jpg}
\vspace{-20pt}
\caption{1.5-day forecast results of different models.}
\label{fig_1.5-day}
\end{figure*}

\begin{figure*}[h]
\centering
\includegraphics[width=1\linewidth]{figures/fig_2.0-day.jpg}
\vspace{-20pt}
\caption{2-day forecast results of different models.}
\label{fig_2.0-day}
\end{figure*}


\begin{figure*}[h]
\centering
\includegraphics[width=1\linewidth]{figures/fig_2.5-day.jpg}
\vspace{-20pt}
\caption{2.5-day forecast results of different models.}
\label{fig_2.5-day}
\end{figure*}

\begin{figure*}[h]
\centering
\includegraphics[width=1\linewidth]{figures/fig_3.0-day.jpg}
\vspace{-20pt}
\caption{3-day forecast results of different models.}
\label{fig_3.0-day}
\end{figure*}

\begin{figure*}[h]
\centering
\includegraphics[width=1\linewidth]{figures/fig_3.5-day.jpg}
\vspace{-20pt}
\caption{3.5-day forecast results of different models.}
\label{fig_3.5-day}
\end{figure*}

\begin{figure*}[h]
\centering
\includegraphics[width=1\linewidth]{figures/fig_4.0-day.jpg}
\vspace{-20pt}
\caption{4-day forecast results of different models.}
\label{fig_4.0-day}
\end{figure*}

\begin{figure*}[h]
\centering
\includegraphics[width=1\linewidth]{figures/fig_4.5-day.jpg}
\vspace{-20pt}
\caption{4.5-day forecast results of different models.}
\label{fig_4.5-day}
\end{figure*}


\begin{figure*}[h]
\centering
\includegraphics[width=1\linewidth]{figures/fig_5.0-day.jpg}
\vspace{-20pt}
\caption{5.0-day forecast results of different models.}
\label{fig_5.0-day}
\end{figure*}

\begin{figure*}[h]
\centering
\includegraphics[width=1\linewidth]{figures/fig_5.5-day.jpg}
\vspace{-20pt}
\caption{5.5-day forecast results of different models.}
\label{fig_5.5-day}
\end{figure*}

\begin{figure*}[h]
\centering
\includegraphics[width=1\linewidth]{figures/fig_6.0-day.jpg}
\vspace{-20pt}
\caption{6.0-day forecast results of different models.}
\label{fig_6.0-day}
\end{figure*}

\begin{figure*}[h]
\centering
\includegraphics[width=1\linewidth]{figures/fig_6.5-day.jpg}
\vspace{-20pt}
\caption{6.5-day forecast results of different models.}
\label{fig_6.5-day}
\end{figure*}

\begin{figure*}[h]
\centering
\includegraphics[width=1\linewidth]{figures/fig_7.0-day.jpg}
\vspace{-20pt}
\caption{7.0-day forecast results of different models.}
\label{fig_7.0-day}
\end{figure*}

\begin{figure*}[h]
\centering
\includegraphics[width=1\linewidth]{figures/fig_7.5-day.jpg}
\vspace{-20pt}
\caption{7.5-day forecast results of different models.}
\label{fig_7.5-day}
\end{figure*}

\begin{figure*}[h]
\centering
\includegraphics[width=1\linewidth]{figures/fig_8.0-day.jpg}
\vspace{-20pt}
\caption{8.0-day forecast results of different models.}
\label{fig_8.0-day}
\end{figure*}

\begin{figure*}[h]
\centering
\includegraphics[width=1\linewidth]{figures/fig_8.5-day.jpg}
\vspace{-20pt}
\caption{8.5-day forecast results of different models.}
\label{fig_8.5-day}
\end{figure*}

\begin{figure*}[h]
\centering
\includegraphics[width=1\linewidth]{figures/fig_9.0-day.jpg}
\vspace{-20pt}
\caption{9.0-day forecast results of different models.}
\label{fig_9.0-day}
\end{figure*}

\begin{figure*}[h]
\centering
\includegraphics[width=1\linewidth]{figures/fig_9.5-day.jpg}
\vspace{-20pt}
\caption{9.5-day forecast results of different models.}
\label{fig_9.5-day}
\end{figure*}

\begin{figure*}[h]
\centering
\includegraphics[width=1\linewidth]{figures/fig_10.0-day.jpg}
\vspace{-20pt}
\caption{10.0-day forecast results of different models.}
\label{fig_10.0-day}
\end{figure*}
\section{Verification via Confined Boxes}
\label{sec:method}

Towards formally verifying recourse over an entire region, we formulate a \emph{mixed-integer quadratically constrained program} (MIQCP) to solve the RVP. 

\paragraph{Characterizing Regions with Boxes}
We focus on a special case of the RVP that finds the largest confined \textit{box}. A box is a set defined by simple upper and lower bound constraints on each dimension. Let $U_j = \max_{x \in {\cal R}}x_j$, $L_j = \min_{x \in {\cal R}}x_j$ be the upper and lower bound for each feature $j$ in the region. Given an upper bound, $\mathbf{u} \in \mathbb{R}^d: \mathbf{u} \leq \mathbf{U}$, and lower bound, $\mathbf{l} \in \mathbb{R}^d: \mathbf{l} \geq \mathbf{L}$, a box $B_{\cal R}(\mathbf{u},\mathbf{l})$ is defined as
$
B_{\cal R}(\mathbf{u},\mathbf{l}) = \{\mathbf{x} \in {\cal R}: \mathbf{l} \leq x \leq \mathbf{u}\}
$.
We focus on boxes due their interpretability, which can help model developers understand the source of fixed predictions. Boxes can be viewed as a type of \emph{decision rule}, which have been widely studied for their interpretability within the broader ML community (e.g., \cite{lawless2023interpretable, lawless2022interpretable, lawless2023cluster}). For ease of notation we drop the explicit dependence on ${\cal R}$ and refer to boxes as $B(\mathbf{u}, \mathbf{l})$. We define the size of a box $B(\mathbf{u},\mathbf{l})$ as the sum of the normalized ranges of each feature:%
\vspace{-0.5em}
\begin{equation} \label{def:boxsize}
\text{Size}(B(\mathbf{u}, \mathbf{l})) = \sum_{j=1}^d \frac{u_j - l_j}{U_j - L_j}
\end{equation}

\paragraph{Generating Confined Boxes}
We start by formulating the related problem of auditing whether a given box $B(\mathbf{u}, \mathbf{l})$ in region ${\cal R}$ contains any data points with recourse, which we denote the \emph{Region Recourse Existence Problem (REP)}. Let $\mathbf{x} \in \mathbb{R}^{d-q} \times \mathbb{Z}^q$ be a decision variable representing an individual, and $\mathbf{a} \in \mathbb{R}^{d-q} \times \mathbb{Z}^q$ represent an action. We assume that the region ${\cal R}$, feature space ${\cal X}$, and action set ${\cal A}$ can be represented by a set of constraints over a mixed-integer set (see \cref{fig:summary} for an example). This general assumption encompasses a variety of potential regions and feature sets. We model the REP as a mixed-integer linear program (MILP) over $\mathbf{x}$ and $\mathbf{a}$ (see Appendix \ref{app:rep_form} for details). 

Recall that the RVP can be cast as an optimization problem to find the largest confined region within ${\cal R}$. By definition the REP is infeasible for \emph{every confined box}. To certify that the REP is infeasible for a given box, and by extension certify that the box is confined, we leverage a classical result from linear optimization called Farkas' lemma: 

\begin{theorem}[\citet{farkas}]\normalfont
Let $A \in \mathbb{R}^{m \times n}$ and $b \in \mathbb{R}^m$. Then exactly one of the following two assertions is true:
\begin{enumerate}[label={\Roman*.},leftmargin=*,itemsep=0.1em]
    \item There exists $x \in \mathbb{R}^n$ such that $Ax \leq b$
    \item There exists $y \geq 0$ such that $A^T y = 0$ and $b^\top y = -1$
\end{enumerate}
\end{theorem}

Farkas' lemma states that we can certify that a system of inequalities over continuous variables $Ax \leq b$ is infeasible by finding a \emph{Farkas certificate} $y \geq 0$ such that $A^\top y = 0$ and $b^\top y = -1$. In our context, we can thus view the problem of finding a confined box as a joint problem of selecting a box and finding an associated Farkas certificate for the REP. However, Farkas' lemma only applies to \emph{continuous} variables, and the REP can include discrete variables.

We extend Farkas' certificates to the discrete setting using a simple strategy that simultaneously generates certificates for all possible continuous restrictions of the REP. A \emph{continuous restriction} of a MILP is a restricted version of the optimization problem where all discrete variables are fixed to specific values. Note that a box is confined if and only if every continuous restriction of the REP is infeasible.

Let  ${\cal C}$ be the set of continuous restrictions, where each restriction $c \in {\cal C}$ corresponds to a specific set of fixed values for the discrete variables (e.g., $x_1 = 1, x_2 = 2$ for a problem with two discrete variables $x_1, x_2 \in \mathbb{Z}^2$). Note that the set ${\cal C}$ is finite, from the assumption ${\cal R}$ is bounded and only discrete variables are fixed, but grows exponentially with respect to the number of discrete variables. If there are no discrete variables in the REP, there is a single continuous restriction representing the full problem with no fixed values. In settings where there are a large number of discrete variables, enumerating all possible continuous restrictions may become computationally intractable. However, we prove in \cref{sec:scaling} that under very general constraints and minimal assumptions we can relax many if not all of the discrete variables in the REP. Under these new theoretical results, the set of restrictions that the algorithm must consider is often incredibly small (e.g., $|{\cal C}| \leq 4$ for all the datasets and actionability constraints considered in \citet{kothari2023prediction}). 

We formulate a continuous restriction $c \in {\cal C}$ of the REP as a linear program (LP) (see Appendix \ref{app:rep_form}), which we represent in the following standard form:
\begin{align*}
C_c\mathbf{x} + D_c\mathbf{a} \leq b_c(\mathbf{u}, \mathbf{l})
\end{align*}
where where $C_c$ and $D_c$ are $m \times d$ matrices and $b_c(u,l)$ is a $m$-dimensional vector that is a linear function of the box upper and lower bounds $\mathbf{u}, \mathbf{l}$. Here $m$ represents the number of constraints in the continuous restriction of the REP.

\paragraph{MIQCP Formulation} We can now formulate the RVP as MIQCP that finds the largest box with Farkas certificates of infeasibility for every continuous restriction. Let $\mathbf{y}_c \in \mathbb{R}^{m}$ be decision variables representing the Farkas certificate for a continuous restriction $c \in {\cal C}$, and $\mathbf{u}, \mathbf{l} \in \mathbb{Z}^d$ represent the upper and lower bounds of a box. Note that there is one variable in $\mathbf{y}$ for every constraint in the continuous restriction. We can now find the largest confined box $B(\mathbf{u}, \mathbf{l})$ with associated certificates of infeasibility $y_c$ for $c \in {\cal C}$ using the \emph{Farkas Certificate Problem (FCP)}:
\begin{subequations}
\allowdisplaybreaks
\begin{align}
	\maximize_{\mathbf{y}_c, \mathbf{u}, \mathbf{l}}\quad&& \sum_d \frac{u_d - l_d}{U_d - L_d} \label{obj:f_size}\\[.1cm]
	\st\quad&& b_c(\mathbf{u}, \mathbf{l})^\top \mathbf{y}_c &= -1 ~~&& \forall c \in {\cal C} \label{const:f_neg_ray}\\
	&& C_c^\top \mathbf{y}_c &= 0 && \forall c \in {\cal C} \label{const:f_dual_feas_a} \\
	&& D_c^\top \mathbf{y}_c &= 0 && \forall c \in {\cal C}\label{const:f_dual_feas_b} \\
	&& \mathbf{y}_c &\geq 0 && \forall c \in {\cal C}\label{const:f_non_neg_y} \\
	&& \mathbf{L} \leq \mathbf{l} \leq \mathbf{u} &\leq \mathbf{U} && \label{const:f_box_bounds} \\
	&& \mathbf{u}, \mathbf{l} &\in \mathbb{Z}^d \label{const:f_ul_int}
\end{align}
\end{subequations}
The objective of the problem is to maximize the size of the box, as defined in Equation \eqref{def:boxsize}. Constraints \eqref{const:f_neg_ray}-\eqref{const:f_non_neg_y} follow from Farkas' lemma and ensure that $y_c$ is a valid certificate of infeasibility for the continuous restriction $c$ of the REP. Constraint \eqref{const:f_box_bounds} ensures the FCP generates a valid box within the region ${\cal R}$. We restrict $\mathbf{u}, \mathbf{l}$ to be integer variables to prevent numerical precision issues when solving this MIQCP in practice. This is not an onerous assumption as any continuous variable $x_j$ with a desired precision $10^{-p}$ can be re-scaled and rounded to an integer variable $10^p x_j$. The problem is quadratically constrained due to the inner product of $b_c(\mathbf{u},\mathbf{l})$ and $\mathbf{y}_c$ in constraint \eqref{const:f_neg_ray}. While MIQCPs are often more computationally demanding than MILPs, the FCP can be solved in seconds on real-world datasets using commercial solvers~\citep[e.g.,][]{achterberg2019gurobi}, as the problem scales with the number of features and actionability constraints (which are typically small) rather than the number of data points in the data set. 

When verifying recourse over a \emph{fixed} box $B(\mathbf{u}, \mathbf{l})$ the FCP can be decomposed into $|{\cal C}|$ problems (solved independently for each continuous restriction). If the FCP is infeasible for any continuous restriction $c$, then the RVP is infeasible for the box. If the FCP is feasible for all continuous restrictions $c \in {\cal C}$, then the box is responsive. Alas, when optimizing over potential boxes, the FCP cannot be decomposed as the variables $\mathbf{u}, \mathbf{l}$ link all the continuous restrictions. 

\paragraph{Generating Multiple Boxes} Solving an instance of the FCP generates a single confined box or certifies that the region is responsive. However, in practice, a given region may contain multiple confined regions. To provide model developers and stakeholders with a comprehensive view of individuals with fixed predictions, the FCP can be run sequentially to enumerate multiple (or all) confined boxes with the region. It does so by iteratively adding \emph{no-good cuts} to exclude previously discovered confined regions from ${\cal R}$ (see \cref{app:multi_boxes} for details).

\subsection{Handling Discrete Variables} \label{sec:scaling}
\begin{table*}[t]
    \centering
    \resizebox{\linewidth}{!}{
    \begin{tabular}{l@{\hspace*{4mm}}R{0.4\linewidth}lR{0.6\linewidth}}
         \textbf{Class} &
         \textbf{Description} &
         \textbf{Formulation} &
         \textbf{Discussion} 
         \\
    \cmidrule(lr){1-4} %\cmidrule(lr){2-4} \cmidrule(lr){5-7}

    $K$-Hot Constraint &
    Preserves that the unweighted sum of a set of variables $\{v_j\}_{j \in J}$ is at most $K \in \mathbb{Z}$. &
    $
    \sum_{j \in J}  \pm~v_j \leq K.
    $ &
    Generalizes the popular \emph{one-hot encoding} for categorical variables. \\
    \cmidrule(lr){1-4} %\cmidrule(lr){2-4} \cmidrule(lr){5-7}

    \makecell{Directional Linkage Constraints}&
    Ensures that one feature, $v_{j}$ is greater than or equal to another feature $v_{k}$ &
    $v_{j} \leq v_{k}.$ &
    Ensures a broad class of non-separable constraints (i.e., constraints that act on multiple features) including thermometer encodings, and deterministic causal constraints (e.g., increasing years of account history implies a commensurate increase in Age). \\
    \cmidrule(lr){1-4} %\cmidrule(lr){2-4} \cmidrule(lr){5-7}
    
    Integer Bound Constraints&
    Places an integer upper or lower bound on a variable &
    $
L_j \leq  v_j \leq U_j.$
&
    Encompasses a wide range of separable constraints including monotonicity, actionability, and bounds on the action step size \cite{kothari2023prediction} \\
    \cmidrule(lr){1-4} %\cmidrule(lr){2-4} \cmidrule(lr){5-7}


    \end{tabular}
    

    }
    \caption{Linear Recourse Constraints Classes. Variables $v_j$ used in the constraints may represent $x$ variables (i.e., constrain the region), $a$ variables (i.e., constrain the actions), or $x + a$ (i.e., constrain the resulting feature vector). This restricted set of constraints encompasses a broad set of existing actionability constraints considered in previous literature.} \label{tab:linear_recourse_const}
\end{table*}

In the preceding section, the RVP was solved by enumerating and finding Farkas' certificates for all continuous restrictions of the REP. However, this approach scales exponentially with respect to the number of discrete variables in the REP. In this section, we show that under a very broad set of actionability constraints and general assumptions we can relax all the discrete variables in the REP and still verify recourse over the entire region.

\paragraph{Linear Recourse Constraints} 
We consider a restricted set of constraints, which we call \emph{linear recourse constraints} (detailed in \cref{tab:linear_recourse_const}). These constraints include a broad class of actionability constraints such as monotonicity, categorical encodings, and immutability. They can be used to define the feature space ${\cal X}$, the region ${\cal R}$, or the action set $A$. Linear recourse constraints encompass many actionability constraints considered in previous literature including all the constraints in \citep{ustun2019actionable, russell2019efficient,kothari2023prediction}. We denote an action set comprised only of these constraints as \textit{linear recourse constraints}. %These constraints can act on either $x$ variables (i.e., constrain the region), $a$ variables (i.e., constrain the actions), or $x + a$ (i.e., constrain the resulting feature vector). Let $v_j$ represent a set of variables corresponding to feature $j$ (i.e., $x_j, a_j$, or $x_j+a_j$). 

%and all but 2 of the 100+ constraints used in the experiments of \citet{}. 
%
% \begin{assumption}[A1, Informal\label{a1:onehot}] 
% All variables participate in at most one $K$-hot constraint.
% \end{assumption}

% \begin{assumption}[A2, Informal\label{a2:directional_linkage}] 
% The set of directional linkage constraints do not imply any relationships between variables participating in $K$-hot constraints.
% \end{assumption}
%
\paragraph{Key Result} 

\cref{thm:tum} shows that we can recover the solution to the REP by solving a \emph{linear relaxation} if:
%
\begin{enumerate}[label={A.\arabic*}, itemsep=0pt]
\item 
% All variables participate in at most one $K$-hot constraint.\label{a1:onehot} 
No variable appears in more than one $K$-hot constraint.\label{a1:onehot} 
\item The directional linkage constraints do not enforce relationships between variables appearing in $K$-hot constraints.\label{a2:directional_linkage}
\item The directional linkage constraints do not imply any circular relationships between variables. \label{a3:cycles}
\end{enumerate}
%
%of the REP (i.e., the problem with only \emph{continuous variables}) is equivalent to solving the original REP. 
Practically, \cref{thm:tum} shows we can solve the FCP with a single continuous restriction (i.e., $|{\cal C}| = 1$),
relaxing all discrete variables in the problem.
%
\begin{theorem}\label{thm:tum}
Under Assumptions \ref{a1:onehot}- \ref{a3:cycles}, the linear relaxation of the REP is feasible iff the REP is feasible for any problem with linear recourse constraints.
\end{theorem}
%
For a full proof and formal definitions of the assumptions, see \cref{app:tum_pf}. The assumptions for \cref{thm:tum} are general and hold in many realistic settings. For instance, $K$-hot constraints are often used to encode categorical features (e.g., via a one-hot encoding). Assumption \ref{a1:onehot} holds in this setting as each associated variable only corresponds to one encoding (i.e., one $K$-hot constraint). Similarly, Assumption \ref{a2:directional_linkage} holds as long as there are no logical implications between the categorical features. Finally, Assumption \ref{a3:cycles} holds as long as there are no circular implications between variables. Circular implications between variables represent flaws in constructing the action set and should be caught prior to solving the RVP.

%Assumption \ref{a1:onehot} holds if $K$-hot constraints are used to encode categorical features, as each feature is only represented in one encoding (i.e., one $K$-hot constraint). Assumption \ref{a2:directional_linkage} holds as long as there are not logical implications between categorical variables encoded using $K$-hot constraints. 
%\textit{Proof Sketch.} We prove this result by showing that the polyhedron defining feasible $x$ and actions $a$ under linear recourse constraints is \emph{totally unimodular}, which means that all extreme points of the polyhedron are integral. 
%Consequently, the linear relaxation of the REP is feasible 
%if and only the discrete REP is feasible. For a full proof and formal definitions of the assumptions, see \cref{app:tum_pf}.

\cref{thm:tum} holds under linear recourse constraints
% which encompass a broad class of potential actionability constraints, 
but not under more general constraints. 
In \cref{app:relax_disc} we discuss how to extend our approach to general constraints, and provide practical guidelines on how to select continuous restrictions to include in the FCP.

\section{Experiment}
In this section, we conduct extensive experiments to evaluate the performance of various LLMs on our Hellaswag-Pro benchmark. Our study is guided by three key research questions:
\textbf{RQ1}: How do different LLMs perform across all variants?
\textbf{RQ2}: What is the relative difficulty of different variants?
\textbf{RQ3}: How robust are LLMs to diverse prompts during evaluation?

\subsection{Experiment Setup} 
\subsubsection{Model Selection and Implementation Details}
We select 41 representative commercial and open-source models, including English LLMs, such as GPT-4o, Claude-3.5-Sonnet, Gemini-1.5-Pro,Mistral series, Llama3 series and Chinese LLMs, like Qwen-Max,  Qwen2.5 series, InternLM-2.5 series, Yi-1.5 series, Baichuan-2 series and DeepSeek series.

We integrate both Chinese HellaSwag and HellaSwagPro into the lm-evaluation-harness platform. For the open-source models, we use the default settings of lm-evaluation-harness: do\_sample is set to false and the temperature is set to the default value of the hugging-face library. For the closed-source models, we set the temperature to 0.7. In addition, we set the maximum output length to 1024.

\subsubsection{Prompt Strategy}
Taking into account the influence of language and shot, we design 9 prompting strategies, including Direct, CN-CoT, EN-CoT, CN-XLT and EN-XLT. The last four setups include both zero-shot and few-shot variants.\footnote{
For open-source models, Direct adopts an approach similar to the official implementation of HellaSwag, computing the log-likelihood for each option and selecting the one with the highest log-likelihood. And we report normalized accuracy that accounts for the impact of option length. Other prompting strategies use a generation setup and report accuracy based on exact match.}
\textbf {(1)Direct}: LLMs makes the selection directly without any CoT process.
\textbf{(2)CN-CoT}: LLMs performs CoT in Chinese, regardless of dataset language.
\textbf{(3)EN-CoT}: Similar to CN-CoT, but CoT is conducted in English. 
\textbf{(4)CN-XLT}: LLMs are instructed to first translate English questions and options to Chinese, and then reason in Chinese.
\textbf{(5)EN-XLT}: Similar to CN-XLT, but translates from Chinese dataset to English and reasons in English. 

%\textbf {CN-CoT}: LLMs perform Chinese reasoning and then output the answer and 3 shots are provided.
%\textbf {CN-CoT}: Similar as CNCoTFewShot without any shots.
%\textbf {EN-CoT}: The reasoning process in English is executed and then the answer is output and 3 shots are provided.
%\textbf {CN-XLT}: Inspired by this, we instruct LLMs to translate questions in Chinese and then output the answer after performing reasoning in Chinese too. And 3 shots are provided.
%\textbf {EN-XLT}: Inspired by this, we instruct LLMs to translate questions in Englsih and then output the answer after performing reasoning in Englsih too. Three shots are provided.

\subsubsection{Evaluation metric}

To comprehensively evaluate the robustness of each LLM, we consider four metrics: 
% Original Accuracy (\textbf{OA}), Average Robust Accuracy (\textbf{ARA}), Robust Loss Accuracy (\textbf{RLA}), and  Consistent Robust Accuracy (\textbf{CRA}).
\noindent %
\textbf{- Original Accuracy (OA)} measures accuracy on original problems.
\begin{equation}\label{eq1}
OA=\frac{\sum_{(x, y) \in D} \mathds{1}[L M(x), y]}{|D|}.
\end{equation}
\noindent %
\textbf{- Average Robust Accuracy  (ARA)} represents average accuracy across all variants, gauging overall performance on the robustness tasks.
\begin{equation}\label{eq2}
ARA=\frac{\sum_{\left(x^{\prime}, y^{\prime}\right) \in D_{R}} \mathds{1}\left(L M\left(x^{\prime}, y^{\prime}\right)\right.}{\left|D_{R}\right|}.
\end{equation}

\noindent %
\textbf{- Robust Loss Accuracy (RLA)} is the difference between ARA and OA, indicating performance degradation on robustness data versus original data.
%\begin{tiny}
%\begin{equation}\label{eq3}
%RLA=\frac{\sum_{\left(x^{\prime}, y^{\prime}\right) \in D_{R}} %\mathds{1}\left(L M\left(x^{\prime}, y^{\prime}\right)\right.}{\left|D_{R}\right|}-\frac{\sum_{(x, y) \in D}\mathds{1}[L M(x), y]}{|D|}
%\end{equation}
%\end{tiny}
\begin{equation}\label{eq3}
RLA= OA - ARA.
\end{equation}
\noindent %
\textbf{- Consistent Robust Accuracy (CRA)} shows accuracy when the model correctly answers both original and variant data, reflecting the model do understand the problem.
% consistency in problem-solving.
\begin{equation}\label{eq4}
CRA=\frac{\sum_{x, y, x^{\prime}, y^{\prime}}\mathds{1}[L M(x), y] \cdot \mathds{1}[L M(x^{\prime}), y^{\prime}]}{\left|D_{R}\right|}.
\end{equation}
For all equation above, $D$ denotes the original dataset, where $x$ represents the input question and options, and $y$ represents the correct label, while $D_{R}$ is the robust dataset with $x^{\prime}$ and $y^{\prime}$ representing similar to $x$ and $y$.


\begin{table*}[ht]
\centering
\setlength{\tabcolsep}{5pt}
% \footnotesize
\scalebox{0.6}{
% Please add the following required packages to your document preamble:
% \usepackage{multirow}
% \usepackage[table,xcdraw]{xcolor}
% Beamer presentation requires \usepackage{colortbl} instead of \usepackage[table,xcdraw]{xcolor}
% Please add the following required packages to your document preamble:
% \usepackage{multirow}
% \usepackage[table,xcdraw]{xcolor}
% Beamer presentation requires \usepackage{colortbl} instead of \usepackage[table,xcdraw]{xcolor}
\begin{tabular}{ccccccccccccc}
\hline
\multicolumn{1}{c|}{{ }}& \multicolumn{4}{c|}{Chinese}& \multicolumn{4}{c|}{English}& \multicolumn{4}{c}{AVG}\\ \cline{2-13} 
\multicolumn{1}{c|}{\multirow{-2}{*}{{ Model}}} & { OA(\%)$\uparrow$}& { ARA(\%)$\uparrow$} & {RLA(\%)$\downarrow$}& \multicolumn{1}{l|}{{CRA(\%)$\uparrow$}} & { OA(\%)$\uparrow$}& { ARA(\%)$\uparrow$} & { RLA(\%)$\downarrow$}& \multicolumn{1}{l|}{{CRA(\%)$\uparrow$}} & {OA(\%)$\uparrow$}& { ARA(\%)$\uparrow$} & {RLA(\%)$\downarrow$}& { CRA(\%)$\uparrow$} \\ \hline
\multicolumn{1}{c|}{{ Human}} & 96.41& 97.79& -1.38 & \multicolumn{1}{l|}{92.03}& 95.56& 96.04& -0.48 & \multicolumn{1}{l|}{90.02}& 95.99 & 96.92 & -0.93& 91.03 \\ \hline
\multicolumn{13}{c}{\textit{Close-source LLMs}}\\ 
\multicolumn{1}{c|}{{ GPT-4o}}& { 91.37} & { 81.97} & { 9.40}& \multicolumn{1}{l|}{{ 75.55}} & { \textbf{88.63}} & { \textbf{70.17}} & { \textbf{18.46}} & \multicolumn{1}{l|}{{ \textbf{63.06}}} & { 90.00} & { \textbf{76.07}} & { \textbf{13.93}} & { \textbf{69.31}} \\
\multicolumn{1}{c|}{{ Claude3.5}}& { \textbf{95.37}} & { 80.15} & { 15.22} & \multicolumn{1}{l|}{{ 75.04}} & { 85.11} & { 66.02} & { 19.08} & \multicolumn{1}{l|}{{ 57.20}} & { 90.24} & { 73.09} & { 17.15} & { 66.12} \\
\multicolumn{1}{c|}{{ Gemini-1.5-Pro}}& { 90.62} & { 78.36} & { 12.26} & \multicolumn{1}{l|}{{ 70.48}} & { 87.75} & { 60.74} & { 27.01} & \multicolumn{1}{l|}{{ 58.27}} & { 89.19} & { 69.55} & { 19.63} & { 64.38} \\
\multicolumn{1}{c|}{{ Qwen-Max}}& { 93.50} & { \textbf{84.82}} & { \textbf{8.68}}& \multicolumn{1}{l|}{{ \textbf{78.91}}} & { 87.60} & { 62.61} & { 24.99} & \multicolumn{1}{l|}{{ 59.65}} & { \textbf{90.55}} & { 73.72} & { 16.83} & { 69.28} \\ \hline
\multicolumn{13}{c}{\textit{Chinese open-source LLMs}} \\ 
\multicolumn{1}{c|}{{ Qwen2.5-0.5B}}& { 60.75} & { 45.18} & { \textbf{15.57}} & \multicolumn{1}{l|}{{ 28.70}} & { 49.50} & { 38.21} & { \textbf{11.29}} & \multicolumn{1}{l|}{{ 20.57}} & { 55.13} & { 41.70} & { \textbf{13.43}} & { 24.64} \\
\multicolumn{1}{c|}{{ Qwen2.5-1.5B}}& { 63.25} & { 46.16} & { 17.09} & \multicolumn{1}{l|}{{ 29.89}} & { 56.88} & { 39.57} & { 17.30} & \multicolumn{1}{l|}{{ 23.48}} & { 60.06} & { 42.87} & { 17.20} & { 26.69} \\
\multicolumn{1}{c|}{{ Qwen2.5-3B}}& { 67.50} & { 48.75} & { 18.75} & \multicolumn{1}{l|}{{ 33.79}} & { 61.75} & { 39.98} & { 21.77} & \multicolumn{1}{l|}{{ 25.75}} & { 64.63} & { 44.37} & { 20.26} & { 29.77} \\
\multicolumn{1}{c|}{{ Qwen2.5-7B}}& { 67.63} & { 50.59} & { 17.04} & \multicolumn{1}{l|}{{ 35.62}} & { 65.63} & { 43.93} & { 21.70} & \multicolumn{1}{l|}{{ 30.77}} & { 66.63} & { 47.26} & { 19.37} & { 33.20} \\
\multicolumn{1}{c|}{{ Qwen2.5-14B}} & { 69.00} & { 51.41} & { 17.59} & \multicolumn{1}{l|}{{ 35.84}} & { 68.50} & { 45.20} & { 23.30} & \multicolumn{1}{l|}{{ 32.12}} & { 68.75} & { 48.30} & { 20.45} & { 33.98} \\
\multicolumn{1}{c|}{{ Qwen2.5-32B}} & { 69.75} & { 53.11} & { 16.64} & \multicolumn{1}{l|}{{ 37.54}} & { 70.00} & { 46.10} & { 23.90} & \multicolumn{1}{l|}{{ 32.68}} & { 69.88} & { 49.61} & { 20.27} & { 35.11} \\
\multicolumn{1}{c|}{{ Qwen2.5-72B}} & { \textbf{70.87}} & { \textbf{54.75}} & { 16.12} & \multicolumn{1}{l|}{{ \textbf{39.64}}} & { \textbf{72.00}} & { \textbf{47.75}} & { 24.25} & \multicolumn{1}{l|}{{\textbf{ 35.12}}} & { \textbf{71.44}} & { \textbf{51.25}} & {20.19} & { \textbf{37.38}} \\ \hdashline[0.5pt/5pt]
\multicolumn{1}{c|}{{ Baichuan2-7B}}& { 67.00} & { 46.16} & { 20.84} & \multicolumn{1}{l|}{{ 31.50}} & { 60.62} & { 39.04} & { 21.58} & \multicolumn{1}{l|}{{ 25.21}} & { 63.81} & { 42.60} & { 21.21} & { 28.36} \\
\multicolumn{1}{c|}{{ Baichua2-13B}}& { 69.13} & { 46.98} & { 22.15} & \multicolumn{1}{l|}{{ 33.45}} & { 64.62} & { 38.82} & { 25.80} & \multicolumn{1}{l|}{{ 26.07}} & { 66.88} & { 42.90} & { 23.97} & { 29.76} \\ \hdashline[0.5pt/5pt]
\multicolumn{1}{c|}{{ DeepSeek-7B}} & { 68.13} & { 47.96} & { 20.17} & \multicolumn{1}{l|}{{ 33.30}} & { 63.38} & { 40.39} & { 22.99} & \multicolumn{1}{l|}{{ 26.70}} & { 65.76} & { 44.18} & { 21.58} & { 30.00} \\
\multicolumn{1}{c|}{{ DeepSeek-67B}}& { 71.50} & { 49.21} & { 22.29} & \multicolumn{1}{l|}{{ 35.89}} & { 71.37} & { 40.63} & { 30.75} & \multicolumn{1}{l|}{{ 29.71}} & { 71.44} & { 44.92} & { 26.52} & { 32.80} \\ \hdashline[0.5pt/5pt]
\multicolumn{1}{c|}{{ InternLM2.5-1.8B}}& { 61.62} & { 42.07} & { 19.55} & \multicolumn{1}{l|}{{ 26.99}} & { 55.37} & { 38.46} & { 16.91} & \multicolumn{1}{l|}{{ 22.61}} & { 58.50} & { 40.27} & { 18.23} & { 24.80} \\
\multicolumn{1}{c|}{{ InternLM2.5-7B}}& { 67.25} & { 49.77} & { 17.48} & \multicolumn{1}{l|}{{ 34.57}} & { 69.50} & { 40.89} & { 28.61} & \multicolumn{1}{l|}{{ 29.75}} & { 68.38} & { 45.33} & { 23.04} & { 32.16} \\
\multicolumn{1}{c|}{{ InternLM2.5-20B}} & { 67.37} & { 48.08} & { 19.29} & \multicolumn{1}{l|}{{ 33.21}} & { 73.62} & { 41.11} & { 32.51} & \multicolumn{1}{l|}{{ 31.23}} & { 70.50} & { 44.60} & { 25.90} & { 32.22} \\ \hdashline[0.5pt/5pt]
\multicolumn{1}{c|}{{ Yi-1.5-6B}} & { 67.00} & { 49.59} & { 17.41} & \multicolumn{1}{l|}{{ 34.27}} & { 64.38} & { 39.37} & { 25.01} & \multicolumn{1}{l|}{{ 26.62}} & { 65.69} & { 44.48} & { 21.21} & { 30.45} \\
\multicolumn{1}{c|}{{ Yi-1.5-9B}} & { 68.50} & { 50.18} & { 18.32} & \multicolumn{1}{l|}{{ 35.55}} & { 66.37} & { 39.58} & { 26.79} & \multicolumn{1}{l|}{{ 27.48}} & { 67.44} & { 44.88} & { 22.56} & { 31.52} \\
\multicolumn{1}{c|}{{ Yi-1.5-34B}}& { 71.00} & { 52.23} & { 18.77} & \multicolumn{1}{l|}{{ 38.09}} & { 71.00} & { 40.75} & { 30.25} & \multicolumn{1}{l|}{{ 29.91}} & { 71.00} & { 46.49} & { 24.51} & { 34.00} \\ \hline
\multicolumn{13}{c}{\textit{English open-source LLMs}} \\ 
\multicolumn{1}{c|}{{ Llama3-8B}} & { 59.13} & { 46.62} & { 12.51} & \multicolumn{1}{l|}{{ 28.23}} & { 66.25} & { 40.21} & { 26.04} & \multicolumn{1}{l|}{{ 27.34}} & { 62.69} & { 43.42} & { 19.27} & { 27.79} \\
\multicolumn{1}{c|}{{ Llama3-70B}}& { 65.75} & { 48.63} & { 17.12} & \multicolumn{1}{l|}{{ 32.70}} & { \textbf{72.50}} & { 41.27} & { 31.23} & \multicolumn{1}{l|}{{\textbf{ 30.63}}} & {\textbf{ 69.13}} & { 44.95} & { 24.18} & { 31.67} \\ \hdashline[0.5pt/5pt]
\multicolumn{1}{c|}{{ Mistral-7B-v0.2}} & { 57.75} & { 46.25} & { \textbf{11.50}} & \multicolumn{1}{l|}{{ 27.57}} & { 67.50} & { \textbf{41.52}} & { 25.98} & \multicolumn{1}{l|}{{ 28.93}} & { 62.63} & { 43.88} & { 18.74} & { 28.25} \\
\multicolumn{1}{c|}{{ Mixtral-8x7B-v0.1}} & { 63.62} & { 46.80} & { 16.82} & \multicolumn{1}{l|}{{ 30.82}} & { 69.75} & { 41.21} & { 28.54} & \multicolumn{1}{l|}{{ 29.39}} & { 66.69} & { 44.01} & { 22.68} & { 30.11} \\
\multicolumn{1}{c|}{{ Mixtral-8x22B-v0.1}}& { 66.00} & {\textbf{ 50.73}} & { 15.27} & \multicolumn{1}{l|}{{ \textbf{34.32}}} & { 72.12} & { 41.25} & { 30.87} & \multicolumn{1}{l|}{{ 30.61}} & { 69.06} & { \textbf{45.99}} & { 23.07} & { \textbf{32.47}} \\ \hdashline[0.5pt/5pt]
\multicolumn{1}{c|}{{ Gemma-2-2B}}& { 61.88} & { 45.38} & { 16.51} & \multicolumn{1}{l|}{{ 29.02}} & { 59.62} & { 39.13} & { \textbf{20.50}} & \multicolumn{1}{l|}{{ 24.88}} & { 60.75} & { 42.25} & {\textbf{ 18.50}} & { 26.95} \\
\multicolumn{1}{c|}{{ Gemma-2-9B}}& { \textbf{69.13}} & { 46.75} & { 22.38} & \multicolumn{1}{l|}{{ 33.29}} & { 64.88} & { 39.80} & { 25.08} & \multicolumn{1}{l|}{{ 26.91}} & { 67.01} & { 43.28} & { 23.73} & { 30.10} \\
\multicolumn{1}{c|}{{ Gemma-2-27B}} & { 63.38} & { 48.52} & { 14.86} & \multicolumn{1}{l|}{{ 31.96}} & { 71.88} & { 40.91} & { 30.97} & \multicolumn{1}{l|}{{ 30.25}} & { 67.63} & { 44.71} & { 22.92} & { 31.11} \\ \hline
\end{tabular}
}
\caption{TODO: bolded is not result. Results of existing LLMs on our HellaSwag-Pro dataset using \textbf{Direct} prompt. ``AVG'' indicates the average performance of each model on Chinese and English parts of the dataset.
The best results for each metric in each model category are \textbf{bolded}. }
\label{tab:main experiment.}
\end{table*}

\subsection{Model Performance (RQ1)}
\paragraph{Overall Performance}
Table \ref{tab:main experiment.} provides a comprehensive evaluation of various LLMs across four performance metrics\footnote{The results of instruct/chat models of Qwen2.5, Llama3 and Mixtral latest series are shown in Appendix.}. The main observations are as follow:
\begin{itemize}[leftmargin=*,topsep=0pt]
% \setlength{}{0}
    \item Upon evaluating all available models, we found that all performed well in overall accuracy (e.g., GPT-4 scored 90.00 in AVG OA, Claude 3.5 scored 90.24 in AVG OA). However, all models struggled with variations of the questions, as evidenced by a positive RLA value for each model. In contrast, humans received a negative RLA value, suggesting that the question variants were not more challenging than the originals. This disparity further illustrates that current LLMs lack a true understanding of the reasoning process and can easily be misled by question variants.
    \item When comparing open-source and close-source models, the close-source models demonstrate stronger capabilities in both OA and ARA scores, similar to most existing benchmarks. Overall, the RLA values for close-source models are also smaller, indicating that they are more robust in commonsense reasoning tasks compared to open-source models.
    \item When we compare models within the same series (e.g., Qwen, Llama), we observe that larger models often achieve higher scores on OA, ARA, and CRA. However, they are also more susceptible to variations, i.e., they have higher RLA values, a phenomenon particularly evident in English datasets. We attribute this phenomenon to the fact that larger models, compared to smaller ones, may have memorized more data, allowing them to rely on memorization to solve some problems more easily and making them more prone to the influence of variations~\cite{}.
\end{itemize}
% 1. When evaluating all available models, We find although 
% 2. When comparing the opensource LLMs and close source LLMs, 
% 3. When looking into each serious details
% \noindent
% \textbf{Overall Model Performance.}
% 1. close-source > open-source 2. the large the better 3. all have a performance decline when meeting varients.

% To evaluate the performance of various models, we observed patterns consistent with current mainstream trends: closed-source models generally outperform open-source models across metrics. 
% For instance, the closed-source model GPT-4o achieved scores of 90.00 in OA, 76.07 in ARA, and 69.31 in CRA, whereas the open-source model Qwen2.5-72B scored 71.44, 51.25, and 37.38, respectively. 
% Furthermore, within each model series, performance tends to improve with larger model sizes. 
% Nevertheless, even the strongest closed-source models struggle with variations in questions, as indicated by positive values in RLA for all models. In contrast, human performance yields a negative RLA value, highlighting that current LLMs do not genuinely grasp the reasoning process and are prone to falling into traps set by question variants. 
% This suggests that there is still significant room for improvement in developing models that can robustly understand and reason through complex linguistic challenges.
% It reveals a consistent pattern across Chinese, English, and average scores, with close-sourced LLMs generally outperforming open-sourced models. 
% However, all models exhibit a significant drop in performance when faced with robust variants, as indicated by RLA and CRA. Among closed-source models, GPT-4o demonstrates the highest ARA of 76.07\% in average scores, demonstrating its overwhelming superiority. Among open-sourced models, larger models tend to perform better, with Qwen2.5-72B achieving the highest OA (71.44\%) and ARA (51.25\%) in the average scores. However, even these top performers still struggle with robustness, as evidenced by the substantial RLA of 13.93\% for GPT-4o and 20.19\% for Qwen2.5-72B. Interestingly, some English open-sourced models, such as Llama3-70B and Mixtral-8x22B-v0.1, show competitive performance in English tasks but lag in Chinese tasks, highlighting the importance of language-specific training.

% \noindent
% \textbf{Chinese Models vs English Models.}
% Chinese models generally demonstrate higher OA in Chinese tasks compared to English tasks, with Qwen-Max achieving 93.50\% OA in Chinese versus 87.60\% in English. Conversely, English models tend to perform better in English tasks, exemplified by Llama3-70B's 72.50\% OA in English compared to 65.75\% in Chinese. 
% However, both Chinese and English models exhibit important drops in ARA across languages, indicating challenges in maintaining performance when faced with variations. This trend suggests that while models may excel in their primary language, they struggle with robustness across linguistic boundaries. 
% Notably, larger models tend to achieve higher ARA scores but also experience more substantial RLA, as seen with Qwen2.5-0.5B (41.70\% ARA, 13.43\% RLA in total) and Qwen2.5-72B (51.25\% ARA, 20.19\% RLA in total). 
% This pattern indicates that while increased model size enhances overall performance, it doesn't necessarily improve robustness proportionally. 
% The discrepancy between OA and ARA across languages underscores the need for improved cross-lingual robustness in language models, particularly as they scale in size and capability.


% \noindent
% \textbf{Comparison between Chinese and English datasets.}
% Generally, models demonstrate higher accuracy on the Chinese dataset compared to the English one, as evidenced by the consistently higher OA, ARA and CRA scores. For instance, GPT-4o achieves an OA of 91.37\%, an ARA of 81.97\% , an CRA of 75.55\% on the Chinese dataset, compared to 88.63\% and 70.17\% respectively on the English dataset. This trend is observed across most models, suggesting that the Chinese dataset is easier than English one. Moreover, the RLA values are typically lower for Chinese, indicating smaller performance drops when dealing with robust variants of Chinese questions. For example, Qwen-Max shows an RLA of 8.68\% for Chinese versus 24.99\% for English, highlighting a more consistent performance in Chinese. The CRA scores further reinforce this observation, with models generally maintaining higher consistency in correct answers for both original and variant Chinese questions.
% We attribute this phenomenon to the fact that blablabla

\noindent
\textbf{Reasoning Transferable Capability.}
% 为了进一步
To further analyze whether the model can transfer reasoning ability from the original question to its variant, Figure \ref{consis} presents the distribution of model performance on the original question and variant pairs. For all models, the pairs of (HellaSwag \ding{51} HellaSwag-Pro \ding{55}) occupy a significant proportion, indicating a challenge in transferring reasoning capabilities for current LLMs to more complex scenarios. Looking deeply, closed-source models like GPT-4 and Qwen-Max achieve around a 69\% portion of (HellaSwag \ding{51} HellaSwag-Pro \ding{51}) and a 3\% portion of (HellaSwag \ding{55} HellaSwag-Pro \ding{55}), while in contrast, open-source models struggle with around a 30\% portion of (HellaSwag \ding{51} HellaSwag-Pro \ding{51}) and a 20\% portion of (HellaSwag \ding{55} HellaSwag-Pro \ding{55}), further indicating the robustness of reasoning abilities in closed-source models.
% If a model can get both the original question and the variant right, we consider it to have transferable reasoning ability. Table \ref{consis} presents the distribution of model performance on the original question and variant pairs. Among all models, the pairs of (HellaSwag \ding{51}HellaSwag-Pro \ding{55}) account for a considerable proportion, i 
% The closed-source models like GPT-4o and Qwen-Max achieve around 69\% portion of (HellaSwag \ding{51}HellaSwag-Pro \ding{51}) and 3\% portion of (HellaSwag \ding{55} HellaSwag-Pro \ding{55}), indicating stronger reasoning transfer ability than other models. In contrast, open-source models struggle more, with around 30\% portion of (HellaSwag \ding{51}HellaSwag-Pro \ding{51}) and 20\% portion of (HellaSwag \ding{55} HellaSwag-Pro \ding{55}). 
% A notable trend is observed among the Qwen2.5 series, where increasing model size from 7B to 72B parameters correlates with improved performance on correct answers for both datasets (33.20\% to 37.38\%) and decreased failure rates (17.69\% to 14.7\%). It underscores the importance of model size in commonsense reasoning tasks.

\begin{figure}[t]
\centering
\setlength{\abovecaptionskip}{0.1cm}
\setlength{\belowcaptionskip}{0cm}
\includegraphics[width=\linewidth,scale=1.00]{images/consis.pdf}
\caption{Analysis of the transferable ability of model reasoning based on question pair performance. The green part, where both the original and the variant data are right, represents the transferable performance of model reasoning.}
\label{consis}
\vspace{-15pt}
\end{figure}

\begin{figure*}[ht]
\centering
\setlength{\abovecaptionskip}{0.1cm}
\setlength{\belowcaptionskip}{0cm}
\includegraphics[width=\linewidth,scale=1.00]{images/xing.pdf}
\caption{The impact of different few-shot prompts on model performance. With - as the separator, the first two parts of the legend represent the prompt name, and the third part represents the language of the dataset.}
\label{xing}
\vspace{-15pt}
\end{figure*}

\begin{figure}[ht]
\centering
\setlength{\abovecaptionskip}{0.1cm}
\setlength{\belowcaptionskip}{0cm}
\includegraphics[width=1.05\linewidth,scale=1.05]{images/zhu.pdf}
\caption{The RLA Distribution for 7 variants of commonsense reasoning. Parts below the 0 axis indicate that the model’s performance on the variant is improved compared to the original problem.}
\label{fig:zhu}
\vspace{-15pt}
\end{figure}


\subsection{Variant Analysis (RQ2)}
To further analyze the impact of different variants, we assessed the contribution of each variant to the RLA score. A higher contribution indicates that the model is more likely to make errors in that type. Figure~\ref{fig:zhu} presents the overall results, and the key observations are as follows:
\begin{itemize}[leftmargin=*]
    \item For problem restatement, causal inference, and sentence ordering, these three categories are the least challenging. Almost all models, particularly the close-source and Qwen series models, perform well on these variants, indicating that current LLMs can effectively handle these forms and we do not pay more attention on this kind of varients.
    \item For reverse conversion and critical testing, these two varients each contribute about 10\% to the RLA score. This indicates that current LLMs struggle to fully generalize to these simple scenarios, possibly because these types of questions are not commonly encountered, and reaserchers should pay some attention to this type of varients.
    \item For negative transformation and scenario refinement, this are the two most difficult tasks, with negative transformation being particularly challenging. For almost all models, these two varients accounts for more than 50\% of the RLA score. This may be due to intuitively counterintuitive questions—such as the use of "will not"  or counterfactual scenarios in scenario refinement. These setups are less common in LLM training data and cannot be easily tackled through memory alone. Only those LLMs which truely understand the question could answer the varient correctly, wihch better reflect the true performance of the model.. In the future, researchers should focus more on enhancing LLM's capability to address such types of questions.
\end{itemize}

% 1. Problem restCausal Inference 
% To further analysis the impact of different varients, we further 
% Figure \ref{fig: zhu} presents a comprehensive analysis of various LLMs' performance across different variant types. Negative transformation emerges as the most challenging task for all models, with scores consistently above 50.00\% and peaking at 78.38\% for Gemini-1.5-Pro. Conversely, problem restatement appears to be the least challenging, with most models scoring in the negative range. Intriguingly, smaller models like Qwen2.5-0.5B demonstrate unexpected strengths in certain areas, such as sentence sorting (7.75\%), outperforming some larger counterparts. A detailed analysis of each variant type follows.

% \noindent
% \textbf{Causal inference.} In this category, scores vary widely from -4.73\% for Qwen-Max to 12.25\% for Baichuan2-13B, illustrating differing degrees of sensitivity to causal reasoning among the models. Smaller models, such as Qwen2.5-0.5B and Qwen2.5-1.5B, achieve better scores, indicating relatively stronger robustness in causal reasoning. Conversely, larger models, like Baichuan2-13B, have higher scores, suggesting greater sensitivity to the challenges of inferring causality.

% \noindent
% \textbf{Critical testing.} Larger models, including Qwen2.5-72B and DeepSeek-67B, exhibit higher RLA scores of 30.50\% and 31.37\%, respectively, suggesting increased sensitivity when dealing with incomplete key information. In contrast, GPT-4o achieves the lowest score, highlighting its superior robustness in critical reasoning. This trend indicates that more complex models might struggle to handle incomplete contexts, underscoring potential areas for improvement in sophisticated architectures.

% \noindent
% \textbf{Negative transformation.} This aspect remains consistently challenging for all models, with scores ranging from 48.88\% to 78.38\%. Advanced commercial models like Gemini-1.5-Pro and Claude-3.5 also score higher (78.38\% and 76.43\%, respectively), indicating a prevalent sensitivity issue in reasoning processes when handling negations, irrespective of model size or architecture.

% \noindent
% \textbf{Problem restatement.} The negative values in this category for nearly all models suggest it is not particularly challenging. This is surprising, given that previous models were quite sensitive to sentence representation.

% \noindent
% \textbf{Reverse conversion.} This variation, which involves swapping the roles of the question and answer, seems to specifically impact larger models. For example, Qwen2.5-72B and DeepSeek-67B exhibit higher RLA scores of 24.38\% and 27.43\%, respectively, indicating heightened sensitivity to reverse reasoning compared to their performance on original questions.

% \noindent
% \textbf{Scenario refinement.} The scores range from 16.06\% for Gemma-2-2B to 32.56\% for Qwen2.5-72B, with larger models displaying more sensitivity in adapting to counterfactual predictions. This suggests that larger models may rely more heavily on general commonsense rather than flexibly adapting to specific contexts. Consequently, increased model complexity might adversely affect adaptability to scenario changes, underscoring the need for enhanced flexibility in advanced models.

% \noindent
% \textbf{Sentence sorting.} This category exhibits the most varied results across models. Some larger models like DeepSeek-67B and InternLM2.5-20B display higher scores (26.69\% and 26.68\%), indicating sensitivity, while others like Qwen2.5-72B and Gemini-1.5-Pro excel with lower scores (-9.88\% and -1.07\%, respectively). This suggests that sentence sorting ability may depend more on specific training approaches rather than being solely contingent on model size.


\subsection{Prompt Robustness (RQ3)}
% To investigate how prompt  influence our benchmark, we apply sereral prompt strategy on our datasets and showcase the average performance of all models on different kind of prompt strategies.
% Table~\ref{prompt} illustrates the final results. For both Chinese and English datasets, CN LLMs achieve the highest performance using CN-CoT-Few-Shot, followed closely by EN-CoT-Few-Shot, with overall performance scores of 67.36\% and 67.03\%, respectively. In contrast, English LLMs perform best with the EN-CoT-Few-Shot, reaching 67.55\% on the Chinese dataset and 60.36\% on the English dataset.
% Contrary to previous findings, translating the dataset to the model's advantage language before performing reasoning does not enhance performance. Moreover, Figure~\ref{xing} also shows the similar phenomenon. Conducting CoT reasoning in the model’s advantage language generally leads to better outcomes compared to Direct. Additionally, increasing the number of shots consistently improves performance across most configurations, highlighting the benefits of exposing models to multiple examples. 
To explore the impact of various prompt strategies on our benchmarks, we evaluated several approaches across our datasets and present the average performance of all models using different prompting techniques. Table~\ref{prompt} summarizes the results. For both Chinese and English datasets, Chinese LLMs performed best with the CN-CoT-Few-Shot strategy, followed closely by EN-CoT-Few-Shot, achieving overall scores of 67.36\% and 67.03\%, respectively. Conversely, English LLMs showed optimal performance with the EN-CoT-Few-Shot approach, attaining 67.55\% on the Chinese dataset and 60.36\% on the English dataset.
Besides, translating datasets into the model's native language before reasoning did not enhance performance. This phenomenon is further illustrated in Figure~\ref{xing}. Conducting CoT reasoning in the model's native language generally yields better results compared to direct reasoning. Furthermore, increasing the number of examples (shots) consistently boosts performance across most configurations, emphasizing the advantages of exposing models to multiple examples.
% Overall, the interaction between question language, prompt language, and the number of shots underscores the importance of aligning these factors to optimize task performance and robustness in LLMs.



% Please add the following required packages to your document preamble:
% \usepackage{multirow}
% Please add the following required packages to your document preamble:
% \usepackage{multirow}
\begin{table}[t]
\setlength{\tabcolsep}{8pt}
% \footnotesize
\scalebox{0.65}{
\begin{tabular}{c|l|lll}
\hline
\multicolumn{1}{l|}{Dataset}  & Prompt  & CN LLMs & EN LLMs &  LLMs \\ \hline
\multirow{7}{*}{\begin{tabular}[c]{@{}c@{}}Chinese\\ HellaSwag-Pro\end{tabular}} & Direct  & 48.95& 41.16& 45.06  \\
& CN-CoT-Few  & \textbf{71.04}& 51.90& 61.47  \\
& EN-CoT-Few  & 70.95& \textbf{67.55}& \textbf{69.25}  \\
& EN-XLT-Few  & 41.48& 28.69& 35.09  \\
& CN-CoT-Zero & 44.82& 23.89& 34.36  \\
& EN-CoT-Zero & 45.38& 31.39& 38.39  \\
& EN-XLT-Zero & 28.57& 12.93& 20.75  \\ \hline
\multirow{7}{*}{\begin{tabular}[c]{@{}c@{}}English\\ HellaSwag-Pro\end{tabular}} & Direct  & 47.46& 40.66& 44.06  \\
& CN-CoT-Few  & \textbf{63.67}& 47.24& 55.46  \\
& EN-CoT-Few  & 63.12& \textbf{60.36}& \textbf{61.74}  \\
& CN-XLT-Few  & 48.77& 16.61& 32.69  \\
& CN-CoT-Zero & 34.89& 18.25& 26.57  \\
& EN-CoT-Zero & 42.41& 31.03& 36.72  \\
& CN-XLT-Zero & 16.36& 11.22& 13.79  \\ \hline
\multirow{9}{*}{HellaSwag-Pro}& Direct  & 48.21& 40.91& 44.83  \\
& CN-CoT-Few  & \textbf{67.36}& 49.57& 58.46  \\
& EN-CoT-Few  & 67.03& \textbf{63.95}& \textbf{65.49}  \\
& CN-XLT-Few  & 59.91& 34.26& 47.08  \\
& EN-XLT-Few  & 52.30& 44.52& 48.41  \\
& CN-CoT-Zero & 39.86& 21.07& 30.46  \\
& EN-CoT-Zero & 43.90& 31.21& 37.55  \\
& CN-XLT-Zero & 30.59& 17.55& 24.07  \\
& EN-XLT-Zero & 35.49& 21.98& 28.74  \\ \hline
\end{tabular}
}
\caption{Average ARA of all open-source models on different prompts. CN-LLMs contains 17 LLMs, and EN-LLMs contains 7 LLMs. The bast results for each dataset are \textbf{bolded}.}
\label{prompt}
\end{table}




\section{Ablation Study}
\label{sec:ablation_study}
In this section, we analyze the contribution of each component of the proposed DE-PADA model.
To achieve this, we perform an ablation study by systematically removing individual components from the final model to assess their impact on the results for both the sit and exercise phases.

As a reminder, the training of the DE-PADA model consists of two stages: in Stage \uppercase\expandafter{\romannumeral 1}, the backbone PQRS and ST models were trained separately on the Target set with personalized augmentation to optimize feature extraction; in Stage \uppercase\expandafter{\romannumeral 2}, the classifier was trained with domain adaptation on genuine, non-augmented data to enhance its ability to learn invariant features across different conditions.

The ablation study is conducted in two primary scenarios. In the first scenario, we evaluate our approach by training the classifier only with domain adaptation, without augmentation, consistent with the methodology used up to this point.
In the second scenario, we examine the results when the classifier is trained with both domain adaptation and personalized augmentation. The augmentation was limited to the Target set since the Auxiliary set has partial representation across sessions and lacks the required data for T-peak range calculations, which are integral to our personalized augmentation method. It's important to note that all the evaluations were conducted on the Target set, consistently with previous evaluations, with the reported IDR being the average of 10 runs.

The results are reported for four different cases, to isolate the effects of each component:
\begin{itemize}
    \item \textbf{DE-PADA}: The complete DE-PADA model, trained with all proposed components, serves as the reference for comparison.
    \item \textbf{DE-PADA\textbackslash DE}: The Standard CNN model is used instead of the DE model, removing the benefit of handling different segments separately.
    \item \textbf{DE-PADA\textbackslash PA}: The DE-PADA model was trained without personalized augmentation at any stage.
    \item \textbf{DE-PADA\textbackslash DA}: The DE-PADA model was trained without using domain adaptation.
\end{itemize}

\subsection{Non-Augmented Classifier}
The results of the ablation study, when the classifier is trained without augmentation, are presented in Table~\ref{tab:ablation_notaugmented}.
This table illustrates the impact of removing various components from the DE-PADA model without using augmented data in the classifier's training stage.

Comparing the DE-PADA\textbackslash DE model to the full DE-PADA model, we observe a significant decrease in performance for the sit position (95.55\% vs. 98.12\%), along with a slight increase in performance for Ex\_P1 (69.59\% vs. 68.95\%).
This suggests that the DE model is particularly effective at preserving performance in low-variability scenarios, such as the sit position.
In high-variability conditions, the DE model offers only marginal improvements in Ex\_P2 and a slight reduction in Ex\_P1.
These findings highlight the advantage of processing different ECG segments separately in the DE model, which is particularly valuable for maintaining robustness across various conditions.

For the DE-PADA\textbackslash PA model, where training was conducted without personalized augmentation, the highest performance is observed in the sit condition (98.52\%), slightly outperforming the full DE-PADA model (98.12\%). This aligns with the presented results on conventional augmentation, which suggest that augmentation in general can negatively impact performance in scenarios with sufficient training data and low variability. However, in this case, the reduction in performance is relatively small. Notably, the absence of personalized augmentation causes a substantial decline in performance for exercise phases, with Ex\_P2 at 81.34\% and Ex\_P1 at 59.18\%. These results underscore the importance of personalized augmentation in managing heart rate variability, as its exclusion significantly compromises the model's effectiveness in high-variability settings.

The DE-PADA\textbackslash DA model, which was trained without domain adaptation, also shows reduced performance compared to the full DE-PADA model, particularly in Ex\_P2 (80.70\%) and Ex\_P1 (59.48\%), with a minor decrease in the sit condition (97.94\%). This indicates that domain adaptation, similarly to augmentation, is crucial for enhancing the model's generalization ability, especially in conditions involving elevated heart rates and varying postures.
%%%%%%%%%%%%%%% Table Start %%%%%%%%%%%%%%%
\begin{table}[!t]
    \centering
    \caption{Ablation study results when the classifier is trained without augmentation.}
    \label{tab:ablation_notaugmented}
    \begin{tabular}{lccc}
        \hline
        Method                   & Sit              & Exercise Phase 2  & Exercise Phase 1  \\
        \hline
        DE-PADA                  & \textbf{98.12}\% & \textbf{86.45\%} & 68.95\%          \\
        DE-PADA\textbackslash DE & 95.55\%          & 84.94\%          & \textbf{69.59\%} \\
        DE-PADA\textbackslash PA & \textbf{98.52\%} & 81.34\%          & 59.18\%          \\
        DE-PADA\textbackslash DA & 97.94\%          & 80.70\%          & 59.48\%          \\
        \hline
    \end{tabular}
\end{table}
%%%%%%%%%%%%%%% Table End %%%%%%%%%%%%%%%
\subsection{Augmented Classifier}
In the second scenario, the classifier was trained with both domain adaptation and personalized augmentation, with the augmentation limited to the Target set as mentioned above.

The results presented in Table~\ref{tab:ablation_augmented} show that the full DE-PADA model achieves the highest performance for exercise phases, with IDR values of 86.28\% for Ex\_P2 and 71.78\% for Ex\_P1. This demonstrates the effectiveness of combining domain adaptation with personalized augmentation in managing high-variability data. The model also surpasses the non-augmented variant from the previous section in Ex\_P1 due to the inclusion of augmentation for the classifier, and the generation of synthetic elevated heart rate examples. However, the decrease in performance for the sit condition reflects the trade-off discussed in Subsection~\ref{chap:sit_ex_results}.

The importance of the DE model is further highlighted in this scenario, as the DE-PADA\textbackslash DE model exhibits a decline in performance across all conditions.
Similarly, the model trained without domain adaptation shows reduced performance, although the reduction in exercise phases is less pronounced than in the non-augmented scenario, as the inclusion of augmentation helps the classifier handle exercise phases more effectively.

Notably, in all models where Stage \uppercase\expandafter{\romannumeral2} included augmentation, performance on the sit position decreased. The DE-PADA\textbackslash PA model, identical to the one in the previous section, achieves the highest performance on the sit position.
%%%%%%%%%%%%%%% Table Start %%%%%%%%%%%%%%%
\begin{table}[!t]
    \centering
    \caption{Ablation study results when the classifier is trained with personalized augmentation.}
    \label{tab:ablation_augmented}
    \begin{tabular}{lccc}
        \hline
        Method                   & Sit              & Exercise Phase 2  & Exercise Phase 1  \\
        \hline
        DE-PADA                  & 96.39\%          & \textbf{86.28\%} & \textbf{71.78\%} \\
        DE-PADA\textbackslash DE & 95.37\%          & 82.57\%          & 67.09\%          \\
        DE-PADA\textbackslash PA & \textbf{98.52\%} & 81.34\%          & 59.18\%          \\
        DE-PADA\textbackslash DA & 96.33\%          & 84.32\%          & 65.33\%          \\
        \hline
    \end{tabular}
\end{table}
%%%%%%%%%%%%%%% Table End %%%%%%%%%%%%%%%


We hypothesize that the partial accounting for heart rate changes introduced through augmentation is particularly beneficial in scenarios with large performance gaps, as observed in related studies that use augmentation in low-data-availability settings. This explains the observed improvement in exercise performance.
However, in cases where the training set contains adequate amounts of representative data, achieving a high initial IDR, the augmentation's inability to fully replicate the authentic changes in the ECG waveform may hinder performance.
%%%%%%%%%%%%%%%%%%%%%%%%%%%%%%%%%%%%%%%%%%%%%%%%%%%%%%%%%%%%%%%%%%%%%%%%%%%%%%%%%%%%%%%%%%%%%
\section{Augmentation Effect on Feature Space}
%%%%%%%%%%%%%%%%%%%%%%%%%%%%%%%%%%%%%%%%%%%%%%%%%%%%%%%%%%%%%%%%%%%%%%%%%%%%%%%%%%%%%%%%%%%%%
To gain deeper insights into the impact of augmentation on our models and whether ST interval normalization remains relevant when training the model with augmentation, we conducted a detailed analysis of the ST model feature space. The ST model, which is the component of DE-PADA responsible for extracting ST interval features (\figref[b]{standard_dual_models}), is the only part affected by the augmentation, therefore we can ignore the PQRS model features in this analysis.
The analysis was performed on the test set, to analyze exercise data and avoid any bias from the training process.\\
The analysis involved the following steps:

\begin{enumerate}
    \item \textbf{Normalization of ST Interval}: We began by normalizing the ST interval for the test data similarly to \cite{Hwang2021}, but accordingly to each individual’s specific fit, rather than a global fit. First, the duration of the ST interval is calculated from the linear fit at the average heart rate of the subject's training data. Then, the T-wave of all the data corresponding to the subject is resampled to the calculated duration.

    \item \textbf{Feature Extraction with Non-Augmented ST Model}: We extracted features from both the original and normalized test data using an ST model that was trained without any augmentation.

    \item \textbf{Feature Extraction with Augmented ST Model}: Next, we repeated the feature extraction process using an ST model that was trained with personalized augmentation.

    \item \textbf{Dimensionality Reduction with t-SNE}: To visualize the extracted features, we applied t-SNE for dimensionality reduction \cite{vandermaaten08a}. This technique allows us to explore the clustering behavior and the distribution of the features in a two-dimensional space.
    Since t-SNE is a stochastic iterative algorithm, which can result in a different reduction on each run, we grouped the features resulting from each model and applied t-SNE to each group.

    \item \textbf{Comparison of Normalization Effects}: Finally, we compared the effects of normalization on each model's feature space and how these effects differ between the two models.
\end{enumerate}

\figref{tsne_graphs} presents the features after t-SNE dimensionality reduction for the last eight subjects in the Target set. The shape of the data points represents the condition of each sample, while the colors distinguish different subjects. 
The background color is a convex hull that groups all data points of each subject; it serves as a visual aid and does not correspond to the classifier's decision boundaries.
\subsection{Non-Augmented ST Model}
\figref[a]{tsne_graphs} illustrates the feature space of the non-augmented model. On the left side, representing the original data, it is evident that most subjects form more than two clusters in the feature space, with subject 40 (blue) displaying four to five clusters. However, the features of the normalized data show a reduction in the number of clusters for each subject, indicating that normalization leads to a more compact feature space.

\subsection{Augmented ST Model}
Upon examining the features of the original data in \figref[b]{tsne_graphs}, it is observed that the feature space is initially compact, similar to the compactness seen in the normalized data features of the non-augmented model.

Since each model underwent a separate dimensionality reduction, a direct comparison of the data point locations between (a) and (b) is not possible. However, the changes in locations between the left and right sides of both models can be compared.
For the non-augmented model, most subjects show a noticeable shift in the location of their data points after normalization. However, for the augmented model, the feature space exhibits minimal changes for most subjects, suggesting that the model's feature space is robust to T-wave variability, as it is largely unaffected by T-wave normalization.

%%%%%%%%%%%%%%% Figure Start %%%%%%%%%%%%%%%
\begin{figure}[!t]
    \myhyperlabel{tsne_graphs}
    \centering
    \includegraphics[width=\columnwidth]{abusa8.pdf}
    \caption[t-SNE 2D Feature Reduction of Original and Normalized Data]
        {
            t-SNE 2D Feature Reduction of Original and Normalized Data.
            (a) Heartbeat normalization improves feature compactness, and the feature mapping changes considerably after normalization.
            (b) Features are compact prior to normalization, and the feature space is hardly affected by it.
        }
    \label{fig:tsne_graphs}
\end{figure}
%%%%%%%%%%%%%%% Figure End %%%%%%%%%%%%%%%
\subsection{Observations}
Our experiments using data normalization in conjunction with augmentation did not provide any additional benefit, which aligns with the observations in \figref[b]{tsne_graphs}.
Furthermore, visualizing the feature spaces of both models reveals that even at the ST interval level, neither augmentation nor normalization fully compensates for the changes occurring in Ex\_P1 and Ex\_P2.
Despite t-SNE being an unsupervised dimensionality reduction algorithm, sit and exercise data points were clustered separately, with each subject having at least two clusters.

\section{Conclusions}
In this paper, we addressed the challenge of ECG-based user identification across varying body postures and physiological states, particularly under post-exercise conditions with elevated heart rates. We proposed a comprehensive approach that combines a novel Dual Expert (DE) model with Personalized Augmentation and Domain Adaptation (DE-PADA) to effectively handle the intra-subject variability of ECG signals across diverse conditions. Each of these three components leverages the morphological characteristics of ECG signals to achieve robust identification performance, and combined, they significantly surpassed the reference models in all tested scenarios.

We proposed a Dual Expert (DE) architecture that separately attended to the PQRS and ST intervals, effectively preserving performance in resting states. We introduced a Personalized Augmentation algorithm that augments the ST interval within predicted subject-specific ranges, significantly improving identification under exercise conditions. Additionally, we presented a domain adaptation variant that utilizes data from additional subjects with both resting and active state data. This approach enabled the classifier to learn patterns common to the population including the Target set subjects, thereby enhancing its generalization ability.

The DE-PADA model consistently outperformed the baseline models across all tested conditions. It achieved notable improvements in identification rates, increasing from 77.38\% to 86.45\% for Exercise Phase 2 and from 54.4\% to 68.95\% for Exercise Phase 1 compared to the standard reference model. In addition, the DE-PADA model maintained high accuracy in stable resting conditions such as sitting, achieving an identification rate of 98.12\%, which not only countered the reduction observed in the augmented reference model but also surpassed the baseline performance.

Furthermore, we analyzed the effect of personalized augmentation on the feature space of the ST model and demonstrated its effectiveness in reducing some of the intra-subject variability and creating a more compact feature space. However, after t-SNE dimensionality reduction, it remains evident that features from the sitting position and exercise phases are still clustered separately, indicating that additional methods, such as the proposed domain adaptation, can further reduce this gap.


\bibliographystyle{IEEEtran}
\bibliography{references}

\end{document}


