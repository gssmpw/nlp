\section{Experiments}
\subsection{Database and Evaluation}
The University of Toronto ECG Database (UofTDB) \cite{Wahabi2014} consists of 1019 subjects, with recordings collected over a period of six months in up to five different conditions.
However, the majority of subjects participated only in the first session, which included recordings in the sitting position only. Only 100 subjects were recorded in subsequent sessions, with 55 of them having recordings in all five conditions.

The signals were recorded with dry electrodes in an off-the-person setting \cite{8392675}, captured from subjects' fingertips similar to Lead \uppercase\expandafter{\romannumeral 1} configuration at a sample frequency of 200Hz.

For our evaluation, we include the 55 subjects with samples recorded across all the conditions to ensure consistency in a multi-session setting. While our training set contains data only from sitting and standing positions, incorporating all conditions in the evaluation is important for achieving a thorough evaluation of the model's robustness and effectiveness under different conditions, including exercise, supine, and tripod.

To rigorously evaluate our approach, we preprocessed the exercise data to ensure that participants exhibited ECG responses consistent with normal physiological variations under physical stress. Atypical ECG waveforms, particularly those specific to physical stress conditions, may indicate underlying pathologies or anomalies. Since our methods leverage the natural variations in ECG waveforms, we excluded such cases from this study to maintain the validity of our findings and focus on healthy individuals.
Out of the initial 55 subjects, our analysis identified several subjects with atypical ECG patterns, including those exhibiting atypical changes exclusively during exercise. Using statistical measures detailed in the supplementary
material, we classified 8 of these subjects as outliers and excluded them from the study, resulting in a refined Target set of 47 subjects.

We use sessions S1, S2, S4, and S6 as our training data which consists of recordings in sitting and standing positions. For performance evaluation we use sit and exercise conditions from S3 along with supine and tripod positions from S5. The data split is summarized in Table~\ref{tab:session_conditions}.

Additionally, 15 subjects had exercise condition recordings but were excluded from the Target set due to missing recordings in at least one other position.
Their data was utilized to form an Auxiliary set, which is used for domain adaptation to improve the generalization of the MLP classifier.

%%%%%%%%%%%%%%% Table Start %%%%%%%%%%%%%%%
\begin{table}[!t]
    \centering
    \caption{Train and Test Data Split}
    \label{tab:session_conditions}
    \begin{tabular}{|c|p{0.6cm}|p{0.7cm}|p{0.6cm}|p{0.7cm}|p{1.05cm}|p{1.05cm}|}
    \cline{2-7}
    \multicolumn{1}{c}{} & \multicolumn{4}{|c|}{\textbf{Train Sessions}} % Merging the last five columns
    & \multicolumn{2}{|c|}{\textbf{Test Sessions}} \\ % Merging the last five columns
    \hline
    \textbf{Session} & \textbf{S1} & \textbf{S2} & \textbf{S4} & \textbf{S6} & \textbf{S3} & \textbf{S5} \\ 
    \hline
    \textbf{Condition} & sit & sit,\newline stand &  sit & sit,\newline stand & sit,\newline exercise & supine,\newline tripod  \\ 
    \hline
    \end{tabular}
    \vspace{0.5em}
\end{table}
%%%%%%%%%%%%%%% Table End %%%%%%%%%%%%%%%

We evaluate our performance in the closed-set Identification setting, where we have a fixed set of $N$ subjects $S=\{s_1,s_2,...,s_n\}$.
Given an input signal $sig$, the task is to predict the identity of the corresponding subject $s_{i}\in S$.
This setting is formulated as a multi-class classification problem, where the system assigns each input to one of the pre-defined classes representing the subjects.
The evaluation metrics for identification are the Identification Rate (IDR) and False Identification Rate (FIR) also known as accuracy and error rate respectively. They are defined as follows:

%%%%%%%%%%%%%%% Equation Start %%%%%%%%%%%%%%%
\begin{equation}
    \text{IDR} = \frac{\text{No. of correct predictions}}{\text{Total No. of trials}} = \frac{1}{N} \sum_{i=1}^{N} \mathbb{I}(\hat{y}_i = y_i)
    \label{eq:ir}
\end{equation}
\begin{equation}
    \text{FIR} = \frac{\text{No. of incorrect predictions}}{\text{Total No. of trials}} = 1-\text{IDR}
    \label{eq:fir}
\end{equation}
where
\begin{description}
    \item[$N$] \qquad\enspace total number of samples;
    \item[$\hat{y}_i$] \qquad\enspace predicted label for the $i$-th sample;
    \item[$y_i$] \qquad\enspace true label for the $i$-th sample;
    \item[$\mathbb{I}(\hat{y}_i=y_i)$] \qquad\enspace indicator function that equals $1$ if $\hat{y}_i = y_i$ and \text{\qquad\enspace}$0$ otherwise.
\end{description}
%%%%%%%%%%%%%%% Equation End %%%%%%%%%%%%%%%
Among the numerous databases used in the field, UofTDB stands out as the singular non-private database that includes subjects with both resting and exercise conditions data \cite{8392675, Wahabi2014}.
To the best of our knowledge, \cite{Jyotishi2022} is the only study to explicitly report IDR for exercise conditions from UofTDB, whereas other studies evaluating exercise data use private, non-publicly accessible databases.
Both our study and \cite{Jyotishi2022} conduct evaluations in an inter-session setting, however, the authors' approach involved training on data from the sit position only, resulting in a low IDR of 7.98\% for exercise conditions.
Our approach differs as we use multi-session training with data from both sitting and standing positions, making direct comparisons between the results of the two studies not applicable.

Therefore, due to the lack of comparable studies, we create two baseline models based on the Standard CNN architecture. One model is trained without augmentation, while the other incorporates augmentation techniques inspired by state-of-the-art methods, providing reference points for evaluating our proposed approaches. Further details on these models are provided in the following sections.

In the following experiments, we apply a sliding window of size $W=10$ when averaging heartbeats to reduce noise and improve SNR.
To ensure a fair comparison, all the methods were tested on the same data with a consistent preparation process. Additionally, the DE model architecture is based on the Standard CNN architecture, ensuring that any observed differences in performance reflect the methodological enhancements rather than fundamental architectural differences.

The train/validation data split is 80/20 respectively, with stratification by subject identity. The test data is taken from distinct sessions, as shown in Table~\ref{tab:session_conditions}.

All reported results are averaged over 10 runs to ensure reliability and consistency.
The CNN models were implemented with Pytorch\cite{paszke2019pytorchimperativestylehighperformance} and optimized with Adam optimizer\cite{kingma2017adammethodstochasticoptimization}, learning rate $10^{-3}$, cross-entropy loss, and early stopping of 20 epochs.

%%%%%%%%%%% Subsection
\subsection{Standard CNN Reference}
\label{chap:scr}
The first baseline model uses the Standard CNN architecture (\figref[a]{standard_dual_models}), and is referred to as SCR.
The SCR model is trained on the Target set data without augmentation, and its hyperparameters were tuned using the validation set.
These optimized hyperparameters were consistently applied across all subsequent experiments.

The purpose of the SCR model is to establish baseline results based on our preprocessing flow and data composition, serving as a reference point for evaluating the performance of our proposed methods.

%%%%%%%%%%% Subsection
\subsection{Augmented CNN Reference}
\label{chap:acr}
To investigate the efficacy of conventional augmentation methods, we establish the second baseline by training the SCR model with ST interval augmentation, referred to as ACR.

In line with previous studies, this augmentation follows \hyperlink{individual_augmentation}{Algorithm 1}, however, instead of calculating individual ranges (as detailed in lines \ref{lst:line:pa_start}-\ref{lst:line:pa_end} of \hyperlink{individual_augmentation}{Algorithm 1}), it uses predefined uniform ranges for all subjects.
The used predefined ranges are $T_{peak_{\text{min}}} = 25$ and $T_{peak_{\text{max}}} = 80$ sampling points relative to R-peak location, corresponding to 125 ms to 400 ms following the R-peak at a sampling rate of 200 Hz.
The training and validation data are then augmented, and the model is trained in the same manner as the SCR model.

%%%%%%%%%%% Subsection
\subsection{DE Model with Personalized Augmentation and Domain Adaptation}
\label{chap:dem_trainig}
The Dual Expert with Personalized Augmentation and Domain Adaptation (DE-PADA), as the name suggests, integrates the DE architecture with personalized augmentation and domain adaptation of the classifier. The training process is illustrated in \figref{overall_methods}.

%%%%%%%%%%%%%%% Figure Start %%%%%%%%%%%%%%%
\begin{figure}[!t]
\myhyperlabel{overall_methods}
\centering
\includegraphics[width=0.9\columnwidth]{abusa5.pdf}
\caption[Overview of DE-PADA training process]
{
 DE-PADA Training Process. The Target data is augmented with Personalized Augmentation, followed by separate training of the PQRS and ST models on the augmented data (Stage I). The MLP classifiers are then removed, retaining the CNN backbones with frozen weights as feature extractors. In Stage II, the DE classifier is trained using genuine data from both the target and auxiliary sets, with auxiliary classes removed from the output layer at the end.
}
\label{fig:overall_methods}
\end{figure}
%%%%%%%%%%%%%%% Figure End %%%%%%%%%%%%%%%

First, the Target data is augmented using the proposed Personalized Augmentation, after which the PQRS and ST models are trained separately on the classification task. Following this training, the FC layers are removed, leaving only the CNN backbone networks with frozen weights to serve as feature extractors for the DE model. These backbone networks, having been trained on high-variability data through augmentation, are capable of extracting meaningful features from high heart rate data, including exercise conditions.

To enhance the classifier's ability to learn condition invariant features, we incorporate the Auxiliary set into the training process. 
This set includes an additional 15 subjects with recordings that capture both low heart rates during resting states and high heart rates following exercise, providing a comprehensive range of data for each individual in the Auxiliary set.
By expanding the training data with these diverse examples, we aim to improve the classifier's robustness and performance by leveraging the inherent similarities in how ECG signals change between different conditions across individuals.
For instance, we expect to observe some similar patterns in ECG signal changes among individuals before and after exercise.

%%%%%%%%%%% Subsection
\subsection{Results}
The dataset primarily consists of low heart rate and post-exercise records, with a noticeable lack of moderate heart rate ranges.
To address this and provide a more detailed analysis of performance at elevated heart rates, we split the exercise recordings into two distinct phases, rather than evaluating them under a single category.

Each exercise recording was captured for 2 minutes, immediately following 3-5 minutes of physical activity.
As shown in \figref{hr_change}, the first minute of the recording represents the Initial Recovery Phase, characterized by a high heart rate and a rapid decline as the body begins to recover from physical stress. The second minute represents the Late Recovery Phase, where the heart rate is moderate and mostly stable. Although the duration of the initial recovery may vary between subjects, we uniformly set it as one minute for consistency in our analysis.
Separating these phases allows us to analyze identification performance more effectively across different heart rate levels and variability, especially given the scarcity of moderate heart rate samples in the dataset.

We will refer to the initial and late recovery phases as Ex\_P1 and Ex\_P2, respectively.

%%%%%%%%%%%%%%% Figure Start %%%%%%%%%%%%%%%
\begin{figure}[!t]
\myhyperlabel{hr_change}
\centering
\includegraphics[width=\columnwidth]{abusa6.pdf}
\caption{Heart rate changes during exercise recovery, with a rapid initial decrease in the first minute, followed by a gradual stabilization in the second minute.}
\label{fig:hr_change}
\end{figure}
%%%%%%%%%%%%%%% Figure End %%%%%%%%%%%%%%%

\subsubsection{Sit and Exercise Performance}
\label{chap:sit_ex_results}
In this section, we present the results for the sit position, the most commonly studied and prevalent resting condition in real-life scenarios, and for exercise, which is the primary focus of this work.

The SCR model, as shown in Table~\ref{tab:aug_baseline_performance}, achieved high performance for the sit position with an Identification Rate (IDR) of 97.76\%, demonstrating its effectiveness in low-variability, resting state data.
However, its performance drops significantly in exercise conditions, with IDR of 77.38\% in Ex\_P2 and 54.40\% in Ex\_P1.
This decline underscores the limitations of the SCR model in handling high-variability scenarios typical of elevated heart rates and exercise, with a more severe decline at higher heart rates.

To improve performance under exercise conditions, the ACR model employs conventional augmentation.
This approach resulted in improved IDR of 81.15\% and 66.63\% for Ex\_P2 and Ex\_P1, respectively, demonstrating the benefits of augmentation in addressing the increased variability of exercise data.
However, this improvement came at the cost of reduced performance for sit position, where the IDR decreased to 94.58\%.
As displayed in \figref{sit_error_final}, the False Identification Rate (FIR) for the sit position more than doubled, highlighting a significant trade-off associated with applying conventional augmentation methods without accounting for individual subject characteristics.
%%%%%%%%%%%%%%% Figure Start %%%%%%%%%%%%%%%
\begin{figure}[!t]
\myhyperlabel{sit_error_final}
\centering
\includegraphics[width=0.9\columnwidth]{abusa7.pdf}
\caption[FIR on sit position]{The FIR results for the sit position more than doubled with the use of conventional augmentation. The DE-PADA not only maintains the performance in the sit position but also achieves the lowest FIR.}
\label{fig:sit_error_final}
\end{figure}
%%%%%%%%%%%%%%% Figure End %%%%%%%%%%%%%%%

While this trade-off may initially seem acceptable, such a decrease in performance for the most common and fundamental position has a substantial impact on the overall usability of the system.

The proposed DE-PADA model demonstrates superior performance under exercise conditions, achieving IDR of 86.45\% in Ex\_P2 and 68.95\% in Ex\_P1, outperforming both baseline models.
Most importantly, this improvement in exercise performance does not come at the cost of performance for the sit position.
The DE-PADA model achieved a marginally improved IDR of 98.12\% on the sit position, surpassing both baselines and ensuring robust identification across both resting and active states.
These results emphasize the model’s capability to effectively manage high-variability data while maintaining high performance for sit position.

%%%%%%%%%%%%%%% Table Start %%%%%%%%%%%%%%%
\begin{table}[!t]
    \centering
    \caption[Performance on sit position and exercise conditions]{Sit and exercise IDR compared to the baseline models.}
    \label{tab:aug_baseline_performance}
    \begin{tabular}{lccc}
        \hline
        Method                   & Sit              & Exercise Phase 2  & Exercise Phase 1  \\
        \hline
        SCR (\ref{chap:scr})                     & 97.76\%          & 77.38\%           & 54.40\%           \\
        ACR (\ref{chap:acr})                     & 94.58\%          & 81.15\%           & 66.63\%           \\
        \textbf{DE-PADA (Ours)}  & \textbf{98.12\%} & \textbf{86.45\%}  & \textbf{68.95\%}  \\
        \hline
    \end{tabular}%
\end{table}
%%%%%%%%%%%%%%% Table End %%%%%%%%%%%%%%%

\subsubsection{Overall Performance}
The overall performance across all conditions, including sit, exercise phases, supine, and tripod, provides a comprehensive evaluation of each model's robustness and generalizability. As shown in Table~\ref{tab:final_performance}, the DE-PADA model consistently outperforms the baseline models across all conditions, demonstrating its effectiveness in handling the variability inherent in diverse postures and heart rate levels.

For the supine position, the DE-PADA model achieved an IDR of 96.08\%, surpassing both the SCR (94.81\%) and ACR (93.73\%) baseline models. This indicates the DE-PADA model's capability to maintain high performance even in less frequently encountered conditions. In the tripod position, DE-PADA also outperformed with an IDR of 87.34\%, compared to 85.77\% for SCR and 82.71\% for ACR, highlighting its resilience in scenarios with altered body positioning, which often pose challenges due to changes in ECG signal morphology.

This comprehensive improvement underscores the strength of the proposed DE-PADA model in maintaining superior performance in exercise phases while not compromising on any resting positions. DE-PADA demonstrates its potential for real-world applications, where biometric systems must reliably operate across diverse conditions and activities.
%%%%%%%%%%%%%%% Table Start %%%%%%%%%%%%%%%
\begin{table}[!t]
    \centering
    \caption{Overall IDR results for all the conditions.}
    \label{tab:final_performance}
    \resizebox{\columnwidth}{!}{%
        \begin{tabular}{lccccc}
            \hline
            Method                  & Sit              & Exercise         & Exercise         & Supine           & Tripod           \\
                                    &                  & Phase 2          & Phase 1          &                  &                  \\
            \hline
            SCR                     & 97.76\%          & 77.38\%          & 54.40\%          & 94.81\%          & 85.77\%          \\
            ACR                     & 94.58\%          & 81.15\%          & 66.63\%          & 93.73\%          & 82.71\%          \\
            \textbf{DE-PADA (Ours)} & \textbf{98.12\%} & \textbf{86.45\%} & \textbf{68.95\%} & \textbf{96.08\%} & \textbf{87.34\%} \\
            \hline
        \end{tabular}
    }
\end{table}
%%%%%%%%%%%%%%% Table End %%%%%%%%%%%%%%%