%%%%%%%%%%%%%%%%%%%%%%%%%%%%%%%%%%%%%%%%%%%%%%%%%%%%%%%%%%%%%%%%%%%%%%%%%%%%%%%%%%%%%%%%%%%%%
\section{Augmentation Effect on Feature Space}
%%%%%%%%%%%%%%%%%%%%%%%%%%%%%%%%%%%%%%%%%%%%%%%%%%%%%%%%%%%%%%%%%%%%%%%%%%%%%%%%%%%%%%%%%%%%%
To gain deeper insights into the impact of augmentation on our models and whether ST interval normalization remains relevant when training the model with augmentation, we conducted a detailed analysis of the ST model feature space. The ST model, which is the component of DE-PADA responsible for extracting ST interval features (\figref[b]{standard_dual_models}), is the only part affected by the augmentation, therefore we can ignore the PQRS model features in this analysis.
The analysis was performed on the test set, to analyze exercise data and avoid any bias from the training process.\\
The analysis involved the following steps:

\begin{enumerate}
    \item \textbf{Normalization of ST Interval}: We began by normalizing the ST interval for the test data similarly to \cite{Hwang2021}, but accordingly to each individual’s specific fit, rather than a global fit. First, the duration of the ST interval is calculated from the linear fit at the average heart rate of the subject's training data. Then, the T-wave of all the data corresponding to the subject is resampled to the calculated duration.

    \item \textbf{Feature Extraction with Non-Augmented ST Model}: We extracted features from both the original and normalized test data using an ST model that was trained without any augmentation.

    \item \textbf{Feature Extraction with Augmented ST Model}: Next, we repeated the feature extraction process using an ST model that was trained with personalized augmentation.

    \item \textbf{Dimensionality Reduction with t-SNE}: To visualize the extracted features, we applied t-SNE for dimensionality reduction \cite{vandermaaten08a}. This technique allows us to explore the clustering behavior and the distribution of the features in a two-dimensional space.
    Since t-SNE is a stochastic iterative algorithm, which can result in a different reduction on each run, we grouped the features resulting from each model and applied t-SNE to each group.

    \item \textbf{Comparison of Normalization Effects}: Finally, we compared the effects of normalization on each model's feature space and how these effects differ between the two models.
\end{enumerate}

\figref{tsne_graphs} presents the features after t-SNE dimensionality reduction for the last eight subjects in the Target set. The shape of the data points represents the condition of each sample, while the colors distinguish different subjects. 
The background color is a convex hull that groups all data points of each subject; it serves as a visual aid and does not correspond to the classifier's decision boundaries.
\subsection{Non-Augmented ST Model}
\figref[a]{tsne_graphs} illustrates the feature space of the non-augmented model. On the left side, representing the original data, it is evident that most subjects form more than two clusters in the feature space, with subject 40 (blue) displaying four to five clusters. However, the features of the normalized data show a reduction in the number of clusters for each subject, indicating that normalization leads to a more compact feature space.

\subsection{Augmented ST Model}
Upon examining the features of the original data in \figref[b]{tsne_graphs}, it is observed that the feature space is initially compact, similar to the compactness seen in the normalized data features of the non-augmented model.

Since each model underwent a separate dimensionality reduction, a direct comparison of the data point locations between (a) and (b) is not possible. However, the changes in locations between the left and right sides of both models can be compared.
For the non-augmented model, most subjects show a noticeable shift in the location of their data points after normalization. However, for the augmented model, the feature space exhibits minimal changes for most subjects, suggesting that the model's feature space is robust to T-wave variability, as it is largely unaffected by T-wave normalization.

%%%%%%%%%%%%%%% Figure Start %%%%%%%%%%%%%%%
\begin{figure}[!t]
    \myhyperlabel{tsne_graphs}
    \centering
    \includegraphics[width=\columnwidth]{abusa8.pdf}
    \caption[t-SNE 2D Feature Reduction of Original and Normalized Data]
        {
            t-SNE 2D Feature Reduction of Original and Normalized Data.
            (a) Heartbeat normalization improves feature compactness, and the feature mapping changes considerably after normalization.
            (b) Features are compact prior to normalization, and the feature space is hardly affected by it.
        }
    \label{fig:tsne_graphs}
\end{figure}
%%%%%%%%%%%%%%% Figure End %%%%%%%%%%%%%%%
\subsection{Observations}
Our experiments using data normalization in conjunction with augmentation did not provide any additional benefit, which aligns with the observations in \figref[b]{tsne_graphs}.
Furthermore, visualizing the feature spaces of both models reveals that even at the ST interval level, neither augmentation nor normalization fully compensates for the changes occurring in Ex\_P1 and Ex\_P2.
Despite t-SNE being an unsupervised dimensionality reduction algorithm, sit and exercise data points were clustered separately, with each subject having at least two clusters.
