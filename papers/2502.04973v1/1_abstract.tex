Electrocardiogram (ECG)-based biometrics offer a promising method for user identification, combining intrinsic liveness detection with morphological uniqueness. However, elevated heart rates introduce significant physiological variability, posing challenges to pattern recognition systems and leading to a notable performance gap between resting and post-exercise conditions. Addressing this gap is critical for advancing ECG-based biometric systems for real-world applications.
We propose DE-PADA, a Dual Expert model with Personalized Augmentation and Domain Adaptation, designed to enhance robustness across diverse physiological states. The model is trained primarily on resting-state data from the evaluation dataset, without direct exposure to their exercise data. To address variability, DE-PADA incorporates ECG-specific innovations, including heartbeat segmentation into the PQRS interval, known for its relative temporal consistency, and the heart rate-sensitive ST interval, enabling targeted feature extraction tailored to each region's unique characteristics. Personalized augmentation simulates subject-specific T-wave variability across heart rates using individual T-wave peak predictions to adapt augmentation ranges. Domain adaptation further improves generalization by leveraging auxiliary data from supplementary subjects used exclusively for training, including both resting and exercise conditions.
Experiments on the University of Toronto ECG Database (UofTDB) demonstrate the model’s effectiveness.
DE-PADA achieves relative improvements in post-exercise identification rates of 26.75\% in the initial recovery phase and 11.72\% in the late recovery phase, while maintaining a 98.12\% identification rate in the sitting position.
These results highlight DE-PADA’s ability to address intra-subject variability and enhance the robustness of ECG-based biometric systems across diverse physiological states.