%%
%% This is file `sample-sigconf-authordraft.tex',
%% generated with the docstrip utility.
%%
%% The original source files were:
%%
%% samples.dtx  (with options: `all,proceedings,bibtex,authordraft')
%% 
%% IMPORTANT NOTICE:
%% 
%% For the copyright see the source file.
%% 
%% Any modified versions of this file must be renamed
%% with new filenames distinct from sample-sigconf-authordraft.tex.
%% 
%% For distribution of the original source see the terms
%% for copying and modification in the file samples.dtx.
%% 
%% This generated file may be distributed as long as the
%% original source files, as listed above, are part of the
%% same distribution. (The sources need not necessarily be
%% in the same archive or directory.)
%%
%%
%% Commands for TeXCount
%TC:macro \cite [option:text,text]
%TC:macro \citep [option:text,text]
%TC:macro \citet [option:text,text]
%TC:envir table 0 1
%TC:envir table* 0 1
%TC:envir tabular [ignore] word
%TC:envir displaymath 0 word
%TC:envir math 0 word
%TC:envir comment 0 0
%%
%%
%% The first command in your LaTeX source must be the \documentclass
%% command.
%%
%% For submission and review of your manuscript please change the
%% command to \documentclass[manuscript, screen, review]{acmart}.
%%
%% When submitting camera ready or to TAPS, please change the command
%% to \documentclass[sigconf]{acmart} or whichever template is required
%% for your publication.
%%
%%
% \documentclass[sigconf,authordraft]{acmart}
% \documentclass[sigconf,review,anonymous]{acmart}
% \documentclass[manuscript,review,anonymous]{acmart}
\documentclass[sigconf]{acmart}

%%
%% \BibTeX command to typeset BibTeX logo in the docs
\AtBeginDocument{%
  \providecommand\BibTeX{{%
    Bib\TeX}}}

%% Rights management information.  This information is sent to you
%% when you complete the rights form.  These commands have SAMPLE
%% values in them; it is your responsibility as an author to replace
%% the commands and values with those provided to you when you
%% complete the rights form.
\setcopyright{acmlicensed}
\copyrightyear{2025}
\acmYear{2025}
\acmDOI{XXXXXXX.XXXXXXX}

%% These commands are for a PROCEEDINGS abstract or paper.
\acmConference[IUI'25]{30th Annual ACM Conference on Intelligent User Interfaces}{March 24--27,
  2025}{Cagliari, Italy}
%%
%%  Uncomment \acmBooktitle if the title of the proceedings is different
%%  from ``Proceedings of ...''!
%%
%%\acmBooktitle{Woodstock '18: ACM Symposium on Neural Gaze Detection,
%%  June 03--05, 2018, Woodstock, NY}
\acmISBN{978-1-4503-XXXX-X/18/06}


%%
%% Submission ID.
%% Use this when submitting an article to a sponsored event. You'll
%% receive a unique submission ID from the organizers
%% of the event, and this ID should be used as the parameter to this command.
%%\acmSubmissionID{123-A56-BU3}

%%
%% For managing citations, it is recommended to use bibliography
%% files in BibTeX format.
%%
%% You can then either use BibTeX with the ACM-Reference-Format style,
%% or BibLaTeX with the acmnumeric or acmauthoryear sytles, that include
%% support for advanced citation of software artefact from the
%% biblatex-software package, also separately available on CTAN.
%%
%% Look at the sample-*-biblatex.tex files for templates showcasing
%% the biblatex styles.
%%

%%
%% The majority of ACM publications use numbered citations and
%% references.  The command \citestyle{authoryear} switches to the
%% "author year" style.
%%
%% If you are preparing content for an event
%% sponsored by ACM SIGGRAPH, you must use the "author year" style of
%% citations and references.
%% Uncommenting
%% the next command will enable that style.
%%\citestyle{acmauthoryear}

\usepackage{listings}
\usepackage{color} % Needed for colors

\usepackage{array}
\usepackage{colortbl}
\usepackage{xcolor}
\usepackage{booktabs}
\usepackage{tabularx}

% \usepackage[margin=1in]{geometry}
% \usepackage{forest}
% \usepackage{tikz}

\usepackage{siunitx} 
\usepackage{verbatim}

% \usepackage{tcolorbox}
\usepackage{courier}  % For better monospace font
\usepackage{setspace}

% Define colors
\definecolor{lightgray}{RGB}{245,245,245}
\definecolor{orange}{RGB}{240,148,0}
\definecolor{codeblue}{RGB}{100,149,237}
\definecolor{codered}{RGB}{205,92,92}




% \usepackage{xcolor}
% \usepackage{soul} % For highlighting

% % Define custom colors
% \definecolor{lightgray}{rgb}{0.95, 0.95, 0.95}
% \definecolor{darkred}{RGB}{172,41,0}

% Define a light gray background color

% Define a very light blue-gray background color
\definecolor{lightbg}{RGB}{248,250,252}


\definecolor{codegreen}{rgb}{0,0.6,0}
\definecolor{codegray}{rgb}{0.5,0.5,0.5}
\definecolor{codepurple}{rgb}{0.58,0,0.82}





\lstdefinestyle{mystyle}{
    backgroundcolor=\color{backcolour},
    commentstyle=\color{codegreen},
    keywordstyle=\color{magenta},
    numberstyle=\tiny\color{codegray},
    stringstyle=\color{codepurple},
    basicstyle=\ttfamily\footnotesize,
    breakatwhitespace=false,
    breaklines=true,
    captionpos=b,
    keepspaces=true,
    numbers=left,
    numbersep=5pt,
    showspaces=false,
    showstringspaces=false,
    showtabs=false,
    tabsize=2
}


\lstset{style=mystyle}



%%
%% end of the preamble, start of the body of the document source.
\begin{document}

%%
%% The "title" command has an optional parameter,
%% allowing the author to define a "short title" to be used in page headers.
% \title{{\tool}: Annotating Text-to-SQL Data via Mixed-initIative Query Alignment}
% \title{{\tool}: Interactive Text-to-SQL Data Annotation via Query Alignment}

\title{Text-to-SQL Domain Adaptation via Human-LLM Collaborative Data Annotation}


%
% The "author" command and its associated commands are used to define
% the authors and their affiliations.
% Of note is the shared affiliation of the first two authors, and the
% "authornote" and "authornotemark" commands
% used to denote shared contribution to the research.

\author{Yuan Tian}
\authornote{This work was done during the first author's internship at Adobe.}
\email{tian211@purdue.edu}
% \orcid{1234-5678-9012}
\affiliation{%
  \institution{Purdue University}
  \city{West Lafayette}
  \state{Indiana}
  \country{USA}
}

\author{Daniel Lee}
\email{dlee1@adobe.com}
\affiliation{%
  \institution{Adobe Inc.}
  \city{San Jose}
  \state{California}
  \country{USA}
}

\author{Fei Wu}
\email{feiw@adobe.com}
\affiliation{%
  \institution{Adobe Inc.}
  \city{Seattle}
  \state{Washington}
  \country{USA}
}

\author{Tung Mai}
\email{tumai@adobe.com}
\affiliation{%
  \institution{Adobe Inc.}
  \city{San Jose}
  \state{California}
  \country{USA}
}

\author{Kun Qian}
\email{kunq@adobe.com}
\affiliation{%
  \institution{Adobe Inc.}
  \city{Seattle}
  \state{Washington}
  \country{USA}
}

\author{Siddhartha Sahai}
\email{siddharthas@adobe.com}
\affiliation{%
  \institution{Adobe Inc.}
  \city{Seattle}
  \state{Washington}
  \country{USA}
}

\author{Tianyi Zhang}
\email{tianyi@purdue.edu}
% \orcid{1234-5678-9012}
\affiliation{%
  \institution{Purdue University}
  \city{West Lafayette}
  \state{Indiana}
  \country{USA}
}


\author{Yunyao Li}
\email{yunyaol@adobe.com}
\affiliation{%
  \institution{Adobe Inc.}
  \city{San Jose}
  \state{California}
  \country{USA}
}










%%
%% By default, the full list of authors will be used in the page
%% headers. Often, this list is too long, and will overlap
%% other information printed in the page headers. This command allows
%% the author to define a more concise list
%% of authors' names for this purpose.
% \renewcommand{\shortauthors}{Trovato et al.}



\newcommand{\tool}{\textsc{SQLsynth}}
\newcommand{\todo}[1]{\textcolor{red}{#1}}
\newcommand{\circled}[1]{{\large \textcircled{\footnotesize #1}}}

% \newcommand{\edit}[1]{\hl{#1}}
% \newcommand{\edit}[1]{\textcolor{red}{#1}}
\newcommand{\edit}[1]{#1}

Large language model (LLM)-based agents have shown promise in tackling complex tasks by interacting dynamically with the environment. 
Existing work primarily focuses on behavior cloning from expert demonstrations and preference learning through exploratory trajectory sampling. However, these methods often struggle in long-horizon tasks, where suboptimal actions accumulate step by step, causing agents to deviate from correct task trajectories.
To address this, we highlight the importance of \textit{timely calibration} and the need to automatically construct calibration trajectories for training agents. We propose \textbf{S}tep-Level \textbf{T}raj\textbf{e}ctory \textbf{Ca}libration (\textbf{\model}), a novel framework for LLM agent learning. 
Specifically, \model identifies suboptimal actions through a step-level reward comparison during exploration. It constructs calibrated trajectories using LLM-driven reflection, enabling agents to learn from improved decision-making processes. These calibrated trajectories, together with successful trajectory data, are utilized for reinforced training.
Extensive experiments demonstrate that \model significantly outperforms existing methods. Further analysis highlights that step-level calibration enables agents to complete tasks with greater robustness. 
Our code and data are available at \url{https://github.com/WangHanLinHenry/STeCa}.

%%
%% The code below is generated by the tool at http://dl.acm.org/ccs.cfm.
%% Please copy and paste the code instead of the example below.
%%
\begin{CCSXML}
<ccs2012>
   <concept>
       <concept_id>10003120.10003121.10003129</concept_id>
       <concept_desc>Human-centered computing~Interactive systems and tools</concept_desc>
       <concept_significance>500</concept_significance>
       </concept>
   <concept>
       <concept_id>10010147.10010257</concept_id>
       <concept_desc>Computing methodologies~Machine learning</concept_desc>
       <concept_significance>500</concept_significance>
       </concept>
 </ccs2012>
\end{CCSXML}

\ccsdesc[500]{Human-centered computing~Interactive systems and tools}
\ccsdesc[500]{Computing methodologies~Machine learning}




%%
%% Keywords. The author(s) should pick words that accurately describe
%% the work being presented. Separate the keywords with commas.
\keywords{Natural Language Interface, Text-to-SQL, Databases, Domain Adaptation, Interactive Data Annotation, LLMs, PCFG}
%% A "teaser" image appears between the author and affiliation
%% information and the body of the document, and typically spans the
%% page.
\begin{teaserfigure}
  \centering
  \includegraphics[width=\textwidth]{figure/teaser.png}
  \caption{An overview of {\tool}: \textbf{(A)} Given a database schema, {\tool} populates a database and sample SQL queries for the database. \textbf{(B)} After a step-by-step analysis of each sampled SQL query, {\tool} translates the query into a natural language (NL) question. \textbf{(C)} The large language model (LLM) aligns the NL question with the SQL query through the step-by-step analysis. Users can hover over each step to check the corresponding span in both SQL and NL. \textbf{(D)} {\tool} detects potential errors in the NL and highlights them in red. Users can repair the NL by injecting missing steps or deleting redundant text. \textbf{(E)} Users monitor and visualize the composition of the annotated dataset, thereby controlling the annotation process.
    }
  \label{fig:teaser}
\end{teaserfigure}

% \received{20 February 2007}
% \received[revised]{12 March 2009}
% \received[accepted]{5 June 2009}

%%
%% This command processes the author and affiliation and title
%% information and builds the first part of the formatted document.
\maketitle

\section{Introduction}

Despite the remarkable capabilities of large language models (LLMs)~\cite{DBLP:conf/emnlp/QinZ0CYY23,DBLP:journals/corr/abs-2307-09288}, they often inevitably exhibit hallucinations due to incorrect or outdated knowledge embedded in their parameters~\cite{DBLP:journals/corr/abs-2309-01219, DBLP:journals/corr/abs-2302-12813, DBLP:journals/csur/JiLFYSXIBMF23}.
Given the significant time and expense required to retrain LLMs, there has been growing interest in \emph{model editing} (a.k.a., \emph{knowledge editing})~\cite{DBLP:conf/iclr/SinitsinPPPB20, DBLP:journals/corr/abs-2012-00363, DBLP:conf/acl/DaiDHSCW22, DBLP:conf/icml/MitchellLBMF22, DBLP:conf/nips/MengBAB22, DBLP:conf/iclr/MengSABB23, DBLP:conf/emnlp/YaoWT0LDC023, DBLP:conf/emnlp/ZhongWMPC23, DBLP:conf/icml/MaL0G24, DBLP:journals/corr/abs-2401-04700}, 
which aims to update the knowledge of LLMs cost-effectively.
Some existing methods of model editing achieve this by modifying model parameters, which can be generally divided into two categories~\cite{DBLP:journals/corr/abs-2308-07269, DBLP:conf/emnlp/YaoWT0LDC023}.
Specifically, one type is based on \emph{Meta-Learning}~\cite{DBLP:conf/emnlp/CaoAT21, DBLP:conf/acl/DaiDHSCW22}, while the other is based on \emph{Locate-then-Edit}~\cite{DBLP:conf/acl/DaiDHSCW22, DBLP:conf/nips/MengBAB22, DBLP:conf/iclr/MengSABB23}. This paper primarily focuses on the latter.

\begin{figure}[t]
  \centering
  \includegraphics[width=0.48\textwidth]{figures/demonstration.pdf}
  \vspace{-4mm}
  \caption{(a) Comparison of regular model editing and EAC. EAC compresses the editing information into the dimensions where the editing anchors are located. Here, we utilize the gradients generated during training and the magnitude of the updated knowledge vector to identify anchors. (b) Comparison of general downstream task performance before editing, after regular editing, and after constrained editing by EAC.}
  \vspace{-3mm}
  \label{demo}
\end{figure}

\emph{Sequential} model editing~\cite{DBLP:conf/emnlp/YaoWT0LDC023} can expedite the continual learning of LLMs where a series of consecutive edits are conducted.
This is very important in real-world scenarios because new knowledge continually appears, requiring the model to retain previous knowledge while conducting new edits. 
Some studies have experimentally revealed that in sequential editing, existing methods lead to a decrease in the general abilities of the model across downstream tasks~\cite{DBLP:journals/corr/abs-2401-04700, DBLP:conf/acl/GuptaRA24, DBLP:conf/acl/Yang0MLYC24, DBLP:conf/acl/HuC00024}. 
Besides, \citet{ma2024perturbation} have performed a theoretical analysis to elucidate the bottleneck of the general abilities during sequential editing.
However, previous work has not introduced an effective method that maintains editing performance while preserving general abilities in sequential editing.
This impacts model scalability and presents major challenges for continuous learning in LLMs.

In this paper, a statistical analysis is first conducted to help understand how the model is affected during sequential editing using two popular editing methods, including ROME~\cite{DBLP:conf/nips/MengBAB22} and MEMIT~\cite{DBLP:conf/iclr/MengSABB23}.
Matrix norms, particularly the L1 norm, have been shown to be effective indicators of matrix properties such as sparsity, stability, and conditioning, as evidenced by several theoretical works~\cite{kahan2013tutorial}. In our analysis of matrix norms, we observe significant deviations in the parameter matrix after sequential editing.
Besides, the semantic differences between the facts before and after editing are also visualized, and we find that the differences become larger as the deviation of the parameter matrix after editing increases.
Therefore, we assume that each edit during sequential editing not only updates the editing fact as expected but also unintentionally introduces non-trivial noise that can cause the edited model to deviate from its original semantics space.
Furthermore, the accumulation of non-trivial noise can amplify the negative impact on the general abilities of LLMs.

Inspired by these findings, a framework termed \textbf{E}diting \textbf{A}nchor \textbf{C}ompression (EAC) is proposed to constrain the deviation of the parameter matrix during sequential editing by reducing the norm of the update matrix at each step. 
As shown in Figure~\ref{demo}, EAC first selects a subset of dimension with a high product of gradient and magnitude values, namely editing anchors, that are considered crucial for encoding the new relation through a weighted gradient saliency map.
Retraining is then performed on the dimensions where these important editing anchors are located, effectively compressing the editing information.
By compressing information only in certain dimensions and leaving other dimensions unmodified, the deviation of the parameter matrix after editing is constrained. 
To further regulate changes in the L1 norm of the edited matrix to constrain the deviation, we incorporate a scored elastic net ~\cite{zou2005regularization} into the retraining process, optimizing the previously selected editing anchors.

To validate the effectiveness of the proposed EAC, experiments of applying EAC to \textbf{two popular editing methods} including ROME and MEMIT are conducted.
In addition, \textbf{three LLMs of varying sizes} including GPT2-XL~\cite{radford2019language}, LLaMA-3 (8B)~\cite{llama3} and LLaMA-2 (13B)~\cite{DBLP:journals/corr/abs-2307-09288} and \textbf{four representative tasks} including 
natural language inference~\cite{DBLP:conf/mlcw/DaganGM05}, 
summarization~\cite{gliwa-etal-2019-samsum},
open-domain question-answering~\cite{DBLP:journals/tacl/KwiatkowskiPRCP19},  
and sentiment analysis~\cite{DBLP:conf/emnlp/SocherPWCMNP13} are selected to extensively demonstrate the impact of model editing on the general abilities of LLMs. 
Experimental results demonstrate that in sequential editing, EAC can effectively preserve over 70\% of the general abilities of the model across downstream tasks and better retain the edited knowledge.

In summary, our contributions to this paper are three-fold:
(1) This paper statistically elucidates how deviations in the parameter matrix after editing are responsible for the decreased general abilities of the model across downstream tasks after sequential editing.
(2) A framework termed EAC is proposed, which ultimately aims to constrain the deviation of the parameter matrix after editing by compressing the editing information into editing anchors. 
(3) It is discovered that on models like GPT2-XL and LLaMA-3 (8B), EAC significantly preserves over 70\% of the general abilities across downstream tasks and retains the edited knowledge better.
\section{Related Work}

\subsection{Personalization and Role-Playing}
Recent works have introduced benchmark datasets for personalizing LLM outputs in tasks like email, abstract, and news writing, focusing on shorter outputs (e.g., 300 tokens for product reviews \citep{kumar2024longlamp} and 850 for news writing \citep{shashidhar-etal-2024-unsupervised}). These methods infer user traits from history for task-specific personalization \citep{sun-etal-2024-revealing, sun-etal-2025-persona, pal2024beyond, li2023teach, salemi2025reasoning}. In contrast, we tackle the more subjective problem of long-form story writing, with author stories averaging 1500 tokens. Unlike prior role-playing approaches that use predefined personas (e.g., Tony Stark, Confucius) \citep{wang-etal-2024-rolellm, sadeq-etal-2024-mitigating, tu2023characterchat, xu2023expertprompting}, we propose a novel method to infer story-writing personas from an author’s history to guide role-playing.


\subsection{Story Understanding and Generation}  
Prior work on persona-aware story generation \citep{yunusov-etal-2024-mirrorstories, bae-kim-2024-collective, zhang-etal-2022-persona, chandu-etal-2019-way} defines personas using discrete attributes like personality traits, demographics, or hobbies. Similarly, \citep{zhu-etal-2023-storytrans} explore story style transfer across pre-defined domains (e.g., fairy tales, martial arts, Shakespearean plays). In contrast, we mimic an individual author's writing style based on their history. Our approach differs by (1) inferring long-form author personas—descriptions of an author’s style from their past works, rather than relying on demographics, and (2) handling long-form story generation, averaging 1500 tokens per output, exceeding typical story lengths in prior research.
\section{Formative Study}

To understand the specific requirements for text-to-SQL dataset annotation, we conducted a formative study by interviewing 5 engineers from Adobe. These interviewees have experienced annotating text-to-SQL datasets in their work.
We describe our interview process in Section~\ref{sec:interview}. Based on these interviews, we identified five major user needs in Section~\ref{sec:user_needs}. 
Finally, we discuss our design rationale in Section~\ref{sec:design}, aiming to address the user needs.

\subsection{Interview}
\label{sec:interview}

We conducted 20-minute semi-structured interviews with each interviewee through a conversational and think-aloud process. 
During these interviews, we first asked about the \textbf{motivation} for text-to-SQL annotation in their use cases, specifically about the schemas they worked on and why obtaining more data was important.
Interviewees reported that when deploying a new service, companies often needed to introduce new entities and restructure the original schema.
However, after updating the schema, they typically found that model performance dropped dramatically. Their regression tests showed an overall accuracy drop of 13.3\% for newly added columns and 9.1\% for new tables. As the schema was further updated, performance continued to decline. Moreover, as the schema changed significantly, they needed a large amount of new data on the updated schema to ensure a robust evaluation.

Second, we asked about their detailed \textbf{workflow} and whether they used any tools to assist with data annotation. Interviewees reported that they did not use any specific tool for annotation, although they sometimes asked ChatGPT to generate initial data. 
Additionally, they often leveraged previous datasets by adapting previous queries to the new schema, such as replacing an outdated column name with a new one.
After annotation, their colleagues performed peer reviews to check and refine the data.





Third, we asked about \textbf{challenges} they had met and the speed of their dataset annotation. Overall, they considered annotation to be very expensive. 
Interviewees mentioned that one engineer could only annotate 50 effective SQL and NL pairs per day in their use case. 
They often lost track and felt overwhelmed during annotation. 
Despite the peer review, they still felt a lack of confidence in the quality of the annotated data. 
They pointed out that randomness existed throughout the entire procedure. 
We summarize more challenges as user needs in Section~\ref{sec:user_needs}.






\subsection{User Needs}
\label{sec:user_needs}

\noindent \textbf{\textit{N1: Effective Schema Comprehension.}} 
Text-to-SQL annotation assumes that users can easily understand the database schema specified in a certain format (e.g., Data Definition Language). However, our interviews indicate that it is cumbersome and error-prone for users to navigate and comprehend complex schemas from such a specification format.



\noindent \textbf{\textit{N2: Creating New Queries.}}
Creating SQL queries requires a deep understanding of both database schema and SQL grammar. When creating a text-to-SQL dataset, users need to continually come up with new, diverse SQL queries. However, it is challenging for them to break free from preconceptions shaped by existing queries they have seen before.

\noindent \textbf{\textit{N3: Detecting Errors in the Annotated Data.}} 
An annotated dataset may include errors, which can deteriorate model performance and evaluation results.
Our interviews suggest that annotators need an effective mechanism for detecting potential errors or ambiguity in the constructed queries.


\noindent \textbf{\textit{N4: Efficiently Correcting the Detected Errors.}} After identifying errors, users need an efficient way to correct these errors to ensure the accuracy and reliability of the dataset. They need to ensure the SQL query is syntactically correct, and the NL is semantically equivalent to the SQL query.
%However, correcting errors in the annotated data is often as labor-intensive as annotating new data.


\noindent \textbf{\textit{N5: Improve Dataset Diversity.}}
Dataset diversity is crucial for improving model performance and ensuring rigorous evaluation.
Human annotation can easily introduce biases due to individual knowledge gaps and a lack of holistic understanding of the dataset composition. 
Thus, interviewees reported the need for an effective way to improve diversity and eliminate biases in the dataset. 




\subsection{Design Rationale}
\label{sec:design}


To support \textbf{N1}, our approach visualizes the database schema as a dynamic, editable graph. This enables users to quickly grasp the overall structure of the database and the relationships between entities. Users can explore detailed information such as data type through further interactions with the graph.


To support \textbf{N2}, our approach alleviates the burden of manually creating new SQL queries. We design an algorithm to randomly sample SQL query templates based on SQL grammar, then fill out this template with entities and values retrieved from the database. We make the SQL generation highly configurable---users can manually adjust keyword probability, or automatically tune the probability by an existing dataset.

To support \textbf{N3}, our approach renders the alignment between the SQL query and the NL question via a step-by-step analysis. Our approach then prompts the LLM to highlight potential misalignments to users.
Subsequently, our approach performs a textual analysis to check the equivalence of the SQL query and NL question and offers users a confidence score about their consistency.

To support \textbf{N4}, we handle two common errors---missing information and including irrelevant information in the NL question. Our approach allows users to fix errors by injecting missing information or removing irrelevant details based on LLM-generated suggestions.

To support \textbf{N5}, our approach first enables users to sample SQL queries based on a probability distribution learned from real-world data rather than creating them manually. 
Furthermore, our approach supports visualizing various dataset compositions through diagrams. 
For example, users can view a bar chart displaying the distribution of column counts in SQL queries. This feature allows users to monitor dataset composition during annotation, maintaining control over the annotation direction and improving data diversity.


\section{Implementation}

\sssec{Searchable Parameters}.
We implemented \sysname's searched parallel strategy on a MegatronLM backend.
Since we need to search out different parameters for different parallel strategies, we listed our parameter search space in Appendix Table \ref{tab:parameter}.

\sssec{Searchable GPUs}.
We implemented our system on Nvidia A800. Our hardware configurations enable each node to consist of 8 GPUs with each one connecting with Nvilink. Then for the cross-node GPU communication, we use PCIE.
\section{Usage Scenario}

Bob is a data scientist at a rapidly growing technology company. 
Now, his task is to create a high-quality text-to-SQL dataset for training and evaluating the natural language (NL) interface of the company's recently updated database system.
Bob faces several challenges that make this task particularly daunting.
First, the company has just completed a major update to its database schema, introducing new tables and relationships to accommodate its expanding business needs. 
This update makes previous datasets obsolete and incompatible with the current schema. 
Consequently, it is impossible to accurately evaluate the performance of the NL interface based on the updated database.
Adding to the complexity, the schema now becomes highly complex, with numerous tables and reference relationships. Manually updating previous datasets to reflect these changes would be impractical. Bob realizes that he needs a solution that can handle this complexity efficiently and accurately.
Furthermore, Bob needs to create diverse, unbiased SQL queries and their corresponding NL questions at scale. Doing this manually would be prohibitively time-consuming and challenging, especially given the complexity of the new schema.
Recognizing these challenges, Bob decides to use the newly developed text-to-SQL data annotation tool, {\tool}, to streamline his workflow and ensure the creation of a controllable, high-quality dataset.


\textbf{Schema Comprehension.}
Bob begins by uploading a JSON file containing the company's updated database schema to {\tool}. As the schema loads onto the drag-and-drop canvas, Bob is immediately impressed by how {\tool} transforms the complex JSON structure into an intuitive visual representation. Tables appear as clearly defined boxes with columns listed inside, while relationships between tables are displayed as animated dashed lines.
The visual layout allows Bob to quickly grasp the overall structure of the database, saving him hours of time that would have been spent mentally parsing the JSON file.
Using the intuitive interface, Bob makes necessary adjustments. He double-clicks to edit table names, drags lines to establish reference relationships, and documents the meaning of an abbreviated column name. The ability to zoom in and out further helps Bob navigate the complex structure. 
As he works, Bob realizes the significant improvement in efficiency compared to editing the schema through the original schema definition file directly. What might have taken hours of painstaking work is now being accomplished in minutes, with much greater accuracy and confidence. Finally, Bob downloads the updated schema. He feels confident that this new well-documented schema will serve as a valuable foundation for future projects.

\textbf{Database Population.}
For more convenient data annotation, Bob populates the database with synthetic records with {\tool}. He specifies a need for 1,000 employee records. Upon clicking the SYNTHESIZE button, {\tool} instantly creates these records. Bob reviews the generated data and notices that the synthetic employee names look diverse. He proceeds to generate records for the other tables. Satisfied with the data generation, Bob downloads the database for future use and moves on to the next step.


\textbf{Data Annotation.}
Given the database, Bob is ready to annotate text-to-SQL data by creating SQL queries and their corresponding NL questions.
% He navigates to the \textit{Query Synthesizer} page to begin this process.
% ## Generating SQL Queries
Bob finds manually creating a SQL query from scratch challenging, so he decides to generate a random SQL query by {\tool}:

\begin{center}
\texttt{\textcolor[RGB]{172,41,0}{SELECT} Employees.name} \texttt{\textcolor[RGB]{172,41,0}{FROM} Employees} \\
\texttt{\textcolor[RGB]{172,41,0}{WHERE} Employees.department\_id = 5 \textcolor[RGB]{172,41,0}{AND} Employees.salary > 50000}
\end{center}



% ## Step-by-step explanation generation
Bob finds the SQL query reasonable. To confirm his understanding, Bob clicks the ANALYZE SQL button, and {\tool} shows a step-by-step analysis:

\begin{enumerate}
    \item \texttt{\textcolor[RGB]{172,41,0}{FROM} Employees} $\rightarrow$ \textit{Which data source should we care about?} 
          \\ \colorbox[rgb]{0.95,0.95,0.95}{In employees} 
        
    
    \item \texttt{\textcolor[RGB]{172,41,0}{WHERE} Employees.department\_id = 5} $\rightarrow$ \textit{Which department are employees from?} 
          \\ \colorbox[rgb]{0.95,0.95,0.95}{Filter employees from department 5} 
          
    
    \item \texttt{\textcolor[RGB]{172,41,0}{AND} Employees.salary > 50000} $\rightarrow$ \textit{What salary range do we care about?} 
          \\ \colorbox[rgb]{0.95,0.95,0.95}{Keep employees with salary exceeding \$50,000} 
          
    
    \item \texttt{\textcolor[RGB]{172,41,0}{SELECT} Employees.name} $\rightarrow$ \textit{What information should be returned?} 
          \\ \colorbox[rgb]{0.95,0.95,0.95}{Return the names of employees} 
        
\end{enumerate}

As Bob hovers over each step, a corresponding sub-question is rendered in the tooltip and the corresponding SQL component is highlighted.
{\tool} then generates a suggested NL question for this query:
\begin{center}
``\textit{\textbf{Who are the employees in the marketing department with a salary higher than \$50,000 and have been with the company for over 5 years?}}''
\end{center}
However, Bob notices that this question does not perfectly match the SQL query and decides to use the alignment feature to refine it.
Bob clicks the CHECK ALIGNMENT button, eager to see how well the generated question matches the SQL query.
He is immediately drawn to a phrase in the question highlighted in red: ``\textit{marketing department}'', suggesting there is no corresponding element in the SQL query.
Bob realizes this information is irrelevant and needs to be removed.
To better understand the quality of this suggested query, Bob hovers over the step-by-step explanation. To his surprise, {\tool} further visually corresponds each explanation step to sub-strings of the NL question through simultaneous highlighting.
He notices one explanation step, ``\textit{Filter employees from department 5}'', is highlighted in red.
This visual cue tells Bob that this step is not reflected in the current question.

Bob decides to address these issues one by one. 
First, he removes the irrelevant information by deleting the red-highlighted phrase ``\textit{marketing}'' and the unrelated condition ``\textit{and have been with the company for over 5 years}'' from the question.
Next, he turns his attention to the missing information about the department. He hovers over the red-highlighted explanation step, ``{\em Filter employees from department 5}'', and an INJECT button appears. Bob clicks this button and the current NL question is updated by incorporating this step.
The question now becomes:
\begin{center}
\textit{\textbf{Who are the employees in Department 5 with a salary higher than \$50,000?}}
\end{center}
Excited to see the results of his edits, Bob clicks the CHECK ALIGNMENT button again. 
This time, Bob notices that there is no red highlight in either the explanation or the NL question.
% Each step successfully maps to a SQL component and relevant sub-strings in the NL question.
As a final check, Bob hovers over the explanation steps. He watches with satisfaction as each step successfully maps to a SQL component and sub-strings in the NL question.


\begin{table*}[htbp]
\centering
\small
\begin{tabularx}{\textwidth}{>{\raggedright\arraybackslash}p{0.33\textwidth}>{\raggedright\arraybackslash}p{0.33\textwidth}>{\raggedright\arraybackslash}X}
\toprule
\textbf{Explanation Step} & \textbf{SQL Query Component} & \textbf{Question Sub-string} \\
\midrule
(1) In employees & \texttt{\textcolor{brown}{FROM} Employees} & \textit{... the employees ...} \\
(2) Filter employees from department 5 & \texttt{\textcolor{brown}{WHERE} Employees.department\_id = 5} & \textit{... in department 5 ...} \\
(3) Keep employees with salary exceeding \$50,000 & \texttt{\textcolor{brown}{AND} Employees.salary > 50000} & \textit{... with a salary higher than \$50,000...} \\
(4) Return the names of employees & \texttt{\textcolor{brown}{SELECT} Employees.name} & \textit{Who are the employees ...} \\
\bottomrule
\end{tabularx}
\label{tab:mapping}
\end{table*}


% ## Final Review and Acceptance
According to the visual alignment, Bob is confident that the NL question matches the SQL query. He further validates it by clicking the POST-SYNTHESIS button. 
{\tool} then reports the equivalence analysis in a short paragraph, along with a high confidence score of 98. Pleased with the result, Bob accepts this text-to-SQL instance and collects it to the right panel. He appreciates how the interactive alignment feature and intuitive triple-linkage visualization help him efficiently identify and correct misalignments with high confidence in the data's quality.

As Bob progresses, he periodically uses {\tool} to analyze the dataset composition to ensure he creates a diverse and balanced dataset. He notices that queries involving the newly added tables and relationships are underrepresented, so he adjusts the query generation parameters to increase their frequency.
By the end of the day, Bob creates a substantial, high-quality text-to-SQL dataset that accurately reflects the company's updated database schema. This new dataset will be invaluable for both training and evaluating their natural language interface, something that was not possible before due to the lack of relevant evaluation data.
Bob feels a sense of accomplishment. He successfully updates and documents a complex schema that would have been extremely time-consuming and error-prone to modify manually. He creates a dataset specific to the company's current database schema, including new tables, columns, and relationships. More importantly, the dataset provides a strong evaluation benchmark for the updated schema, allowing the team to accurately evaluate the performance of their NL interface.
The interactive nature of the tool allows Bob to leverage his domain knowledge while benefiting from automated generation and analysis features. He appreciates how the tool transforms a typically tedious and challenging process into an efficient and engaging one, ultimately contributing to the improvement of the company's data interaction capabilities.
\section{User Study}
To investigate the usability and effectiveness of {\tool}, we conducted a within-subjects user study with 12 participants. The study compared {\tool} with manual annotation and the use of a conversational AI assistant. 



\subsection{Participants}
We recruited 12 participants (4 female, 8 male) from Adobe. They worked in different roles including Machine Learning Engineers, Research Scientists, Data Scientists, and Product Managers.
Their works were directly or indirectly related to querying data in the database.
All of them had either Master's or PhD degrees.
Participants self-rated their proficiency in SQL (\textit{3 Beginner}, \textit{3 Basic}, 4 \textit{Intermediate}, \textit{2 Advanced}) and usage frequency of LLMs (6 \textit{Yearly}, 2 \textit{Monthly}, 2 \textit{Weekly}, 3 \textit{Daily}).

\subsection{Tasks}

% \noindent \textbf{Task 1: Text-to-SQL Creation.} 
We randomly sampled 9 schemas on the widely used text-to-SQL benchmark, Spider~\cite{spider}. 
% This pool includes 3 simple schemas, 3 medium schemas, and 3 complex schemas, categorized based on the number of entities and references in each schema.
We provided these schemas in JSON format, whose syntax was comprehensible to all participants.
Based on the schema, participants were asked to annotate text-to-SQL data while optimizing both the data quantity and quality.


% \noindent \textbf{Task 2: Schema Customization.} To assess schema customization performance, we created a pool of schema editing tasks. For each sampled schema, we manually created 30 tasks requiring edits over the existing schema, e.g., "\textit{Add a new column 'Founded' (date) to the 'airlines' table.}"
% Participants were expected to complete these tasks sequentially, as some tasks depended on the completion of previous ones. 
% We maintained a consistent distribution of task types (e.g., the number of "add column" tasks) across different schemas. 
% \todo{Does this task only require editing the schema or does it also require the regeneration of the text-to-SQL data? If the former, I don't find it relevant to the goal of this work.}


\subsection{Comparison Baselines}
To the best of our knowledge, no text-to-SQL data annotation tools were readily available for comparison at the time of the user study. Thus, we compared {\tool} with two commonly applied scenarios for text-to-SQL dataset annotation in the industry, manual annotating and using an AI assistant. 

\noindent \textbf{Manual.} We asked participants to manually review and customize the schema, create SQL queries, and write corresponding NL questions. They recorded the results in an Excel sheet. 

%\vspace{1.5mm}
\noindent \textbf{AI Assistant.} We gave participants access to using a state-of-the-art conversational AI assistant, ChatGPT (GPT-4). 
For example, participants could directly upload the entire schema file in their conversation with ChatGPT and request the generation of sufficient text-to-SQL data.
We did not impose any restrictions on how participants should use ChatGPT.

\subsection{Protocol}
Each study began with a demographic survey and study introduction. Participants then watched a 4-minute tutorial video of {\tool} and spent 3 minutes practicing to get familiar with it. Meanwhile, we collected quality feedback from users.

After participants were familiar with {\tool}, they proceeded to annotate in the assigned condition (i.e., Manual, AI assistant, {\tool}). Each task consisted of three 5-minute sessions, one for each condition. 
We randomized the order of assigned conditions to mitigate learning effects.
For each session, participants were provided with a database schema and asked to annotate as many text-to-SQL datasets as possible. We asked participants to focus on not only the quantity but also the quality of annotated data.

After each session, participants completed a post-task survey to rate their experience with the assigned condition. The survey included the System Usability Score (SUS)~\cite{sus} and NASA Task Load Index (TLX)~\cite{NASA-TLX} questionnaires, using 7-point Likert scales to assess their perceptions. At the end of the study, participants filled out a final survey sharing their experiences, opinions, and thoughts. The entire study took approximately 70 minutes.


\section{Results and Discussion}
We now discuss our experimental results and findings. 

\begin{table*}[htbp]
\centering
\caption{Percentage win-rate for Faithfulness to Writing History evaluated by GPT-4o. Each cell (`X-Y') shows `X' as the method win-rate and `Y' as the Average Author win-rate, with ties as the remainder. Best win-rates for each source is underlined.}
\label{tab:faith-auth-history}
\small
\begin{tabular}{p{2cm} p{1.5cm} p{1.5cm} p{1.5cm} p{1.5cm} p{1.5cm} p{1.5cm} p{1.75cm}}
\toprule
\textbf{Source} & \textbf{Oracle} & \textbf{RAG} & \textbf{Delta} & \textbf{Writing Sheet} & \textbf{Writing Summary} & \textbf{Writing Sheet nP} & \textbf{Writing Summary nP} \\
\midrule
AO3 & 52-32 & 31-54 & 49-41 & \underline{74}-20 & \underline{80}-10 & 68-21 & 72-19 \\
Reddit & 74-21 & 35-56 & 58-26 & \underline{86}-7 & \underline{89}-7 & 74-19 & 72-19 \\
Storium & 50-40 & 42-45 & 45-42 & 68-15 & \underline{75}-8 & \underline{70}-20 & 68-20 \\
N.Magazine & 71-14 & 57-21 & 50-36 & \underline{79}-7 & 71-21 & 71-21 & 71-0 \\
New Yorker & 53-40 & 27-67 & 53-40 & \underline{67}-20 & \underline{80}-20 & 67-27 & \underline{80}-13 \\
Overall & 60-29 & 38-49 & 51-37 & \underline{75}-14 & \underline{79}-13 & 70-22 & 73-14 \\
\bottomrule
\end{tabular}
\end{table*}

\begin{table*}[htbp]
\centering
\caption{Percentage win-rate for Similarity with Author Story evaluated by GPT-4o and Human (last column). Each cell (`X-Y') shows `X' as the method win-rate and `Y' as the Average Author win-rate, with ties as the remainder. Best win-rates for each source is underlined.}
\label{tab:sim-auth-story}
\small
\begin{tabular}{p{2cm} p{1cm} p{1cm} p{1cm} p{1.25cm} p{1.5cm} p{1.25cm} p{1.75cm} p{1.25cm}}
\toprule
\textbf{Source} & \textbf{Oracle} & \textbf{RAG} & \textbf{Delta} & \textbf{Writing Sheet} & \textbf{Writing Summary} & \textbf{Writing Sheet nP} & \textbf{Writing Summary nP} & \textbf{Human Eval} \\
\midrule
AO3 & 71-18 & 30-54 & 45-32 & 40-41 & \underline{60}-30 & 46-38 & 45-40 & 37-33 \\
Reddit & 91-5 & 37-46 & 49-35 & \underline{54}-35 & 49-42 & 47-40 & 42-44 & 52-37 \\
Storium & 60-30 & 45-40 & 35-35 & 52-32 & 50-35 & \underline{57}-14 & 14-29 & 30-30 \\
N.Magazine & 64-21 & 43-36 & 36-43 & 50-21 & \underline{57}-21 & 43-43 & 50-29 & 48-30 \\
New Yorker & 67-13 & 27-47 & 40-40 & 27-60 & 27-47 & 40-60 & \underline{47}-40 & 70-15 \\
Overall & 71-17 & 36-45 & 41-37 & 45-38 & \underline{49}-35 & \underline{47}-39 & 40-36 & 47-29 \\
\bottomrule
\end{tabular}
\end{table*}




\subsection{Automatic Evaluation}
Our automatic evaluation results show that, first, Writing Sheet and Summary consistently outperform other methods. Second, persona descriptions improve faithfulness to writing history but improve similarity to the ground truth only for select sources like Reddit. Third, personalization is more effective for sources like Reddit than AO3 and Storium, and for narrative categories like Creativity, and Language Use, compared to Plot.


\subsubsection{Faithfulness to Writing History}


Table~\ref{tab:faith-auth-history} shows results for faithfulness to writing history using GPT-4o as \(\text{LLM\textsubscript{story}}\) for our personalization methods. We observe the following:

\paragraph{Writing Sheet and Summary achieve best faithfulness to writing history:}  

Writing Sheet and Summary achieve the highest win-rates across all sources, outperforming Average Author by 64\%. The strong performance of the Writing Summary follows from its reliance on the Author Writing Summary as both a reference for evaluation and a generation constraint. Writing Sheet, despite not being explicitly conditioned on the Author Writing Summary, generalizes well, likely due to its format being the same as the Author Writing Summary. Oracle outperforms Delta and RAG, benefiting from direct conditioning on ground-truth story rules, which enhances adherence to the author's writing history. However, Oracle underperforms the Writing Sheet and Summary, likely due to its lack of explicit conditioning on the Author Writing Summary. Another possible cause is that the author's writing style possibly changes over time, which is not fully captured by the ground-truth story rules. See Appendix~\ref{app:faith-auth-history} for an example.


\paragraph{Reddit shows the highest win-rates among sources:} 
Reddit exhibits the highest win-rates across all methods, suggesting its stories are more conducive to personalization. This trend likely stems from Reddit's crowd-sourced writing prompts, where authors create prompts for others. Therefore, there are a range of diverse topics, ranging from power, survival, and mystery to humor and the supernatural. The broad thematic variety may have enabled personalization methods to capture distinct stylistic elements more effectively, resulting in higher win-rates (See the paragraph Themes in Appendix~\ref{app:dataset}).

\paragraph{Persona descriptions improve faithfulness to writing history:}
We see that adding persona descriptions in the system prompt for Writing Sheet and Summary improves win-rates by around 5\% over their nP variants. This result likely follows from personas enhancing the LLM’s role-playing capabilities, allowing for a more faithful representation of the author’s writing style \citep{jiang2024evaluating, wang-etal-2024-rolellm, wallace2024instruction}.


\subsubsection{Similarity to Author Story}

Table~\ref{tab:sim-auth-story} shows results for similarity to the author story using GPT-4o as \(\text{LLM\textsubscript{story}}\) for our personalization methods. We observe the following:

\paragraph{Writing Sheet and Summary outperform baselines of RAG and Delta:} 
Oracle achieves the highest win-rates across all methods as its story rules come from the ground-truth author story. Reddit shows the highest Oracle win-rate (91\%), followed by 65\% on average for other methods, indicating easier personalization likely due to the broad thematic variety in the writing prompts. Writing Sheet and Summary outperform RAG and Delta by 11\% and 6\%, highlighting the benefit of summarizing the author's writing history \citep{richardson2023integrating}.


\paragraph{Persona descriptions help in sources with broad thematic variety in writing prompts:} 
Writing Summary with persona descriptions outperforms its nP variant by 9\% across all sources. However, Writing Sheet shows no significant difference from its nP variant except for Reddit, where it surpasses it by 7\%. This result suggests that persona descriptions enhance personalization in sources that are easier to personalize, such as Reddit, but have limited impact elsewhere \citep{zheng-etal-2024-helpful}. The broad thematic variety in Reddit's writing prompts creates a greater divergence between author styles and the Average Story, leading to more informative Author Writing Sheets and persona descriptions that further aid personalization. See Appendix~\ref{sec:app-persona-sim-story} for an example.


\paragraph{Creativity and Language Use are the easiest categories to personalize:}


\begin{figure}[htbp]
\centering
\includegraphics[width=0.8\linewidth]{figures/cat_wise/Reddit.pdf}
\caption{Win-rate of personalization methods vs. Average Author for Reddit across four narrative categories. D: Delta, S: Writing Summary, WS: Writing Sheet.}\label{fig:reddit_win_rates}
\end{figure}

Table~\ref{tab:category_comparison_column} (in the Appendix) shows that Delta, Writing Summary, and Writing Sheet outperform Average Author in Creativity (23\% higher) and Language Use (15\% higher) \citep{huot2024agents}, but perform worse in Plot and slightly worse in Development \citep{tian-etal-2024-large-language, xu2024echoes}. Creativity and Language Use are less dependent on the writing prompt, making them easier to transfer across prompts, whereas Plot and Development are more closely tied to specific prompts, making generalization difficult, especially without thematic overlap.  

To better understand personalization performance in Reddit, the easiest source to personalize, Figure~\ref{fig:reddit_win_rates} shows that Writing Sheet is the best-performing method across all categories, with a 35\% advantage over Delta and 20\% over Writing Summary in Creativity. Writing Sheet explicitly summarizes differences between the ground-truth author story and the Average Story, providing more informative guidance for personalization. This advantage is particularly evident in Creativity, which is less constrained by the writing prompt. See Appendix~\ref{app:cat-wise-results} for plots of other data sources.

\subsubsection{Traditional Metrics}  
Table~\ref{tab:traditional-metrics} (in the Appendix) shows results with the traditional metrics. 

\paragraph{Lexical Overlap and Diversity Metrics yield similar values:}
We observe that Lexical Overlap and Diversity metrics yield similar values, as all methods use the same generation model, leading to overlapping lexical distributions and limiting these metrics' ability to capture nuanced stylistic differences \citep{zheng2023judging, xie-etal-2023-next}. 

\paragraph{Writing Sheet outperforms other methods, especially for Style metrics:} We observe that Writing Sheet nP consistently performs best across all metrics, particularly in Style metrics for both Author History and Author Story, as measured by LUAR \citep{rivera-soto-etal-2021-learning}. This improvement likely results from the Writing Sheet explicitly summarizing an author's stylistic deviations from an Average Author, enhancing personalization.


\subsection{Human Evaluation}

Table~\ref{tab:sim-auth-story} shows win-rates from human evaluation (Section~\ref{sec:human-eval-story-gen}) under the Human Eval column, showing that personalization methods—Delta, Writing Sheet, and Summary—outperform the Average Author method by \emph{18\% points} absolute win-rate in terms of similarity to ground-truth writings. See Appendix~\ref{app:human-eval-story-gen} for details on experiment design, and qualitative examples. 

\paragraph{Low inter-annotator agreement due to task subjectivity:}
Fleiss’ Kappa for inter-annotator agreement ranges from 0.2 to 0.4, indicating low but above-random agreement, likely due to the challenges of annotating long-form subjective texts \citep{subbiah-etal-2024-storysumm}. Specifically, different annotators prioritize different narrative categories when determining which story in a pair better reflects the author’s style, leading to disagreement. 

\paragraph{Sources like Reddit and New Yorker have better win-rates than others:} 
Reddit and New Yorker achieve the highest win-rates, with Reddit benefiting from a broad thematic variety in writing prompts and New Yorker from advanced narrative devices like subtext, which humans assess more reliably than LLMs \citep{subbiah-etal-2024-reading}. In contrast, AO3 and Storium exhibit lower win-rates due to LLMs' familiarity with fanfiction tropes (AO3)\footnote{\url{https://archiveofourown.org/admin_posts/25888}} and the lack of strong stylistic differentiation in Storium beyond its open-ended storytelling structure \citep{xu2024echoes, tian-etal-2024-large-language}. These findings suggest that personalization is most effective when authors engage with diverse topics or possess distinct stylistic traits (such as complex narrative structures, creative elements, and advanced language use). These characteristics, reflected in their past writing history, provide clear signals for us to mimic their writing style.

\paragraph{Narrative categories like Creativity, and Language use are more conducive to personalization:} 
Annotators favor personalized stories for Creativity, and Language Use, noting their stronger use of deeper symbolism, thematic richness, layered narratives, and expressive language. However, Average Author was preferred when personalized methods incorrectly introduce elements not present in the author's ground-truth story, suggesting deviations from the author's usual style. 
\section{Discussion}
\label{sec:discussion}

This work introduces DRIFT, a framework designed to leverage the intrinsic low-rank properties of large diffusion policy models for efficiency while preserving the benefits of overparameterization. To achieve this, we propose rank modulation and rank scheduler, which dynamically adjust trainable ranks using SVD and a decay function. We instantiate DRIFT within an interactive IL algorithm, DRIFT-DAgger, and show this efficacy of this method through extensive experiments and ablation studies in both simulation and real-world settings. Our results demonstrate that DRIFT-DAgger reduces training time and improves sample efficiency while maintaining performance on par with full-rank policies trained from scratch.

\subsection{Limitations}
This work evaluates and demonstrates the DRIFT framework using DRIFT-DAgger as an instantiation within the IL paradigm. However, prior to the adoption of large models, online reinforcement learning (RL) approaches \cite{schulman2017proximal, haarnoja2018soft} were also a popular area of research. This work does not explore the application of the DRIFT framework in the online RL paradigm. Investigating the potential of DRIFT within online RL could serve as an valuable direction for future research.

Additionally, while we have tested and evaluated various decay functions for the rank scheduler, the current implementation of dynamic rank adjustment in the DRIFT framework follows a monotonic schedule. Although we have conducted ablation studies on decay functions and terminal ranks, the impact of these design choices is likely task-dependent. Furthermore, the rank adjustment of different convolutional blocks in this work is applied uniformly throughout the U-Net backbone, even though different blocks may have varying highest possible ranks. Future research could explore more adaptive and intelligent strategies for adjusting trainable ranks to enhance training efficiency and performance, as well as identify suitable decay functions and terminal ranks for scenarios beyond those covered in this work.

\subsection{Implications}
As discussed in \S\ref{sec:related_works}, prior to the era of large models, innovations in robot learning primarily focused on learning processes with interaction. However, the increasing size of models has resulted in significantly longer training times, making many previous innovations in online interactive learning less practical due to the time required for policy updates. While the machine learning community has made progress in leveraging the intrinsic rank of large models to improve training efficiency, most of these methods are tailored for fine-tuning rather than training from scratch. This distinction arises from the availability of foundation models in general machine learning, whereas robotics often requires training policies from scratch to address scenario-specific tasks.

Despite years of research in the machine learning community, the concepts of overparameterization and intrinsic ranks remain relatively underexplored in robotics. This work introduces reduced-rank training as a means to address the challenges of training efficiency, thereby making online interactive learning methods more feasible in the era of large models for robot learning. By bridging these gaps, we aim to raise awareness within the robotics community about leveraging overparameterization and intrinsic ranks to design more efficient learning methods while preserving the powerful representations afforded by overparameterized models.

%To speculate on the potential impact of our work on large-scale training scenarios, we extrapolate the results from \S\ref{sec:abl_rm} to a large dataset such as Open-X Embodiment \cite{o2024open}, which contains over one million trajectories. Assuming the same training configuration of 100 offline epochs and 50 online rollouts, a batch size equal to the average trajectory length, and a terminal rank of $r_{min}=256$ remaining sufficient for high quality training, the proposed DRIFT-DAgger with rank modulation and rank scheduler could possibly save about 1250 hours (52 days) of training time compared to full-rank training methods.  While testing on such a large dataset was beyond the scope of our current work, we are eager to explore these possibilities ourselves and enable such investigations in the community going forward.     

\section{Conclusion}
This paper presents {\tool}, a novel interactive text-to-SQL annotation system that enables users to create high-quality, schema-specific text-to-SQL datasets.
{\tool} integrates multiple functionalities, including schema customization, database synthesis, query alignment, dataset analysis, and additional features such as confidence scoring.
A user study with 12 participants demonstrates that by combining these features, {\tool} significantly reduces annotation time while improving the quality of the annotated data.
{\tool} effectively bridges the gap resulting from insufficient training and evaluation datasets for new or unexplored schemas.




%%
%% The acknowledgments section is defined using the "acks" environment
%% (and NOT an unnumbered section). This ensures the proper
%% identification of the section in the article metadata, and the
%% consistent spelling of the heading.
\begin{acks}
We thank the anonymous reviewers for their helpful and detailed feedback, as well as the time and care they dedicated to reviewing our work. We also express our gratitude to all the participants in the interview and user study for their valuable comments. This work was supported by Adobe during the first author's internship.
\end{acks}

%%
%% The next two lines define the bibliography style to be used, and
%% the bibliography file.
\bibliographystyle{ACM-Reference-Format}
\bibliography{reference}


\section{Dataset Examples}
\label{app:dataset-eg}
Figure \ref{fig:dataset-eg} illustrates example data instances from MemeCap, NewYorker, and YesBut.

\begin{figure*}[t]
  \includegraphics[width=\linewidth]{figures/dataset-eg.pdf} \hfill
  \caption {Dataset Examples on MemeCap, NewYorker, and YesBut.}
  \label{fig:dataset-eg}
\end{figure*}


\section{SentenceSHAP}
\label{app:sentence-shap}
In this section, we introduce SentenceSHAP, an adaptation of TokenSHAP \cite{horovicz-goldshmidt-2024-tokenshap}. While TokenSHAP calculates the importance of individual tokens, SentenceSHAP estimates the importance of individual sentences in the input prompt. The importance score is calculated using Monte Carlo Shapley Estimation, following the same principles as TokenSHAP.

Given an input prompt \( X = \{x_1, x_2, \dots, x_n\} \), where \( x_i \) represents a sentence, we generate all possible combinations of \( X \) by excluding each sentence \( x_i \) (i.e., \( X - \{x_i\} \)). Let \( Z \) represent the set of all combinations where each \( x_i \) is removed. To estimate Shapley values efficiently, we randomly sample from \( Z \) with a specified sampling ratio, resulting in a subset \( Z_s = \{X_1, X_2, \dots, X_s\} \), where each \( X_i = X - \{x_i\} \).

Next, we generate a base response \( r_0 \) using a VLM (or LLM) with the original prompt \( X \), and a set of responses \( R_s = \{r_1, r_2, \dots, r_s\} \), each generated by a prompt from one of the sampled combinations in \( Z_s \).

We then compute the cosine similarity between the base response \( r_0 \) and each response in \( R_s \) using Sentence Transformer (\texttt{BAAI/bge-large-en-v1.5}). The average similarity between combinations with and without \( x_i \) is computed, and the difference between these averages gives the Shapley value for sentence \( x_i \). This is expressed as:
\begin{align}
\notag
\phi(x_i) = \\ \notag
&\frac{1}{s} \sum_{j=1}^{s} \left( \text{cos}(r_0, r_j \mid x_i) - \text{cos}(r_0, r_j \mid \neg x_i) \right)
\end{align}
where \( \phi(x_i) \) represents the Shapley value for sentence \( x_i \), $\text{cos}(r_0, r_j \mid x_i)$ is the cosine similarity between the base response and the response that includes sentence $x_i$, $\text{cos}(r_0, r_j \mid \neg x_i)$ is the cosine similarity between the base response and the response that excludes sentence $x_i$, and $s$ is the number of sampled combinations in $Z_s$.

\section{Error Analysis Based on SentenceSHAP}
Figure \ref{fig:error-analysis} presents two examples of negative impacts from implications: dilution of focus and the introduction of irrelevant information.
\label{app:error-analysis-shap}
\begin{figure*}[t]
  \includegraphics[width=\linewidth]{figures/error-analysis.pdf} \hfill
  \caption {Examples of negative impact from implications from Phi (top) and GPT4o (bottom).}
  \label{fig:error-analysis}
\end{figure*}

\section{Details on human anntations}
\label{app:cloudresearch}
We present the annotation interface on CloudResearch used for human evaluation to validate our evaluation metric in Figure \ref{fig:cloud-research}. Refer to Sec.~\ref{sec:ethics} for details on annotator selection criteria and compensation.

\begin{figure*}[t]
  \includegraphics[width=\linewidth]{figures/cloud-research.pdf} \hfill
  \caption {Annotation interface on CloudResearch used for human evaluation to validate our evaluation metric.}
  \label{fig:cloud-research}
\end{figure*}



\section{Generation Prompts for Selection and Refinement}
\label{app:gen-prompts}
Figures \ref{fig:desc-prompt}, \ref{fig:seed-imp-prompt}, and \ref{fig:nonseed-imp-prompt} show the prompts used for generating image descriptions, seed implications (1st hop), and non-seed implications (2nd hop onward). Figure \ref{fig:cand-prompt} displays the prompt used to generate candidate and final explanations. Image descriptions are used for candidate explanations when existing data is insufficient but are not used for final explanations. For calculating Cross Entropy values (used as a relevance term), we use the prompt in Figure \ref{fig:cand-prompt}, substituting the image with image descriptions, as LLM is used to calculate the cross entropies.

\begin{figure*}[h]
\small
\begin{tcolorbox}[
    title=Prompt for Image Descriptions,
    colback=white,
    colframe=CadetBlue,
    arc=0pt,        % Remove rounded corners
    outer arc=0pt   % Remove outer rounded corners (important for some styles)    
]

Describe the image by focusing on the noun phrases that highlight the actions, expressions, and interactions of the main visible objects, facial expressions, and people.\\
\\
Here are some guidelines when generating image descriptions:\\
* Provide specific and detailed references to the objects, their actions, and expressions. Avoid using pronouns in the description.\\
* Do not include trivial details such as artist signatures, autographs, copyright marks, or any unrelated background information.\\
* Focus only on elements that directly contribute to the meaning, context, or main action of the scene.\\
* If you are unsure about any object, action, or expression, do not make guesses or generate made-up elements.\\
* Write each sentence on a new line.\\
* Limit the description to a maximum of 5 sentences, with each focusing on a distinct and relevant aspect that directly contribute to the meaning, context, or main action of the scene.\\
\\
Here are some examples of desired output:
---\\
\text{[Description]} (example of newyorker cartoon image):\\
Through a window, two women with surprised expressions gaze at a snowman with human arms.\\
---\\
\text{[Description]} (example of newyorker cartoon image):\\
A man and a woman are in a room with a regular looking bookshelf and regular sized books on the wall.\\
In the middle of the room the man is pointing to text written on a giant open book which covers the entire floor.\\
He is talking while the woman with worried expression watches from the doorway.\\
---\\
\text{[Description]} (example of meme):\\
The left side shows a woman angrily pointing with a distressed expression, yelling ``You said memes would work!''.\\
The right side shows a white cat sitting at a table with a plate of food in front of it, looking indifferent or smug with the text above the cat reads, ``I said good memes would work''.\\
---\\
\text{[Description]} (example of yesbut image):\\
The left side shows a hand holding a blue plane ticket marked with a price of ``\$50'', featuring an airplane icon and a barcode, indicating it's a flight ticket.\\
The right side shows a hand holding a smartphone displaying a taxi app, showing a route map labeled ``Airport'' and a price of ``\$65''.\\
---\\

Proceed to generate the description.\\
\text{[Description]}:

\end{tcolorbox}
\caption{A prompt used to generate image descriptions.} % Add a caption to the figure
\label{fig:desc-prompt}
\end{figure*}


%%%%%%%%%%%%%%%%%%%%%%%%%%% Prompt for implications %%%%%%%%%%%%%%%%%%%%%%%%%%%
\begin{figure*}[t]
\small
\begin{tcolorbox}[
    title=Prompt for Seed Implications,
    colback=white,
    colframe=Green,
    arc=0pt,        % Remove rounded corners
    outer arc=0pt,  % Remove outer rounded corners (important for some styles)    
    % breakable,
]

You are provided with the following inputs:\\
- \text{[}Image\text{]}: An image (e.g. meme, new yorker cartoon, yes-but image)\\
- \text{[}Caption\text{]}: A caption written by a human.\\
- \text{[}Descriptions\text{]}: Literal descriptions that detail the image.\\
\\
\#\#\# Your Task:\\
\texttt{[ One-sentence description of the ultimate goal of your task. Customize based on the task. ]}\\
Infer implicit meanings, cultural references, commonsense knowledge, social norms, or contrasts that connect the caption to the described objects, concepts, situations, or facial expressions.\\
\\
\#\#\# Guidelines:\\
- If you are unsure about any details in the caption, description, or implication, refer to the original image for clarification.\\
- Identify connections between the objects, actions, or concepts described in the inputs.\\
- Explore possible interpretations, contrasts, or relationships that arise naturally from the scene, while staying grounded in the provided details.\\
- Avoid repeating or rephrasing existing implications. Ensure each new implication introduces fresh insights or perspectives.\\
- Each implication should be concise (one sentence) and avoid being overly generic or vague.\\
- Be specific in making connections, ensuring they align with the details provided in the caption and descriptions.\\
- Generate up to 3 meaningful implications.\\
\\
\#\#\# Example Outputs:\\
\#\#\#\# Example 1 (example of newyorker cartoon image):\\
\text{[}Caption\text{]}: ``This is the most advanced case of Surrealism I've seen.''\\
\text{[}Descriptions\text{]}: A body in three parts is on an exam table in a doctor's office with the body's arms crossed as though annoyed.\\
\text{[}Connections\text{]}:\\
1. The dismembered body is illogical and impossible, much like Surrealist art, which often explores the absurd.\\
2. The body’s angry posture adds a human emotion to an otherwise bizarre scenario, highlighting the strange contrast.\\
\\
\#\#\#\# Example 2 (example of newyorker cartoon image):\\
\text{[}Caption\text{]}: ``He has a summer job as a scarecrow.''\\
\text{[}Descriptions\text{]}: A snowman with human arms stands in a field.\\
\text{[}Connections\text{]}:\\
1. The snowman, an emblem of winter, represents something out of place in a summer setting, much like a scarecrow's seasonal function.\\
2. The human arms on the snowman suggest that the role of a scarecrow is being played by something unexpected and seasonal.\\
\\
\#\#\#\# Example 3 (example of yesbut image):\\
\text{[}Caption\text{]}: ``The left side shows a hand holding a blue plane ticket marked with a price of `\$50'.''\\
\text{[}Descriptions\text{]}: The screen on the right side shows a route map labeled ``Airport'' and a price of `\$65'.\\
\text{[}Connections\text{]}:\\
1. The discrepancy between the ticket price and the taxi fare highlights the often-overlooked costs of travel beyond just booking a flight.\\
2. The image shows the hidden costs of air travel, with the extra fare representing the added complexity of budgeting for transportation.\\
\\
\#\#\#\# Example 4 (example of meme):\\
\text{[}Caption\text{]}: ``You said memes would work!''\\
\text{[}Descriptions\text{]}: A cat smirks with the text ``I said good memes would work.''\\
\text{[}Connections\text{]}:\\
1. The woman's frustration reflects a common tendency to blame concepts (memes) instead of the quality of execution, as implied by the cat’s response.\\
2. The contrast between the angry human and the smug cat highlights how people often misinterpret success as simple, rather than a matter of quality.\\
\\
\#\#\# Now, proceed to generate output:\\
\text{[}Caption\text{]}: \texttt{[ Caption ]}\\
\\
\text{[}Descriptions\text{]}:\\
\texttt{[ Descriptions ]}\\
\\
\text{[}Connections\text{]}:

\end{tcolorbox}
\caption{A prompt used to generate seed implications.} % Add a caption to the figure
\label{fig:seed-imp-prompt}
\end{figure*}


%%%%%%%%%%%%%%%%%%%%%%%%%%% Prompt for nonseed implications %%%%%%%%%%%%%%%%%%%%%%%%%%%
\begin{figure*}[t]
\small
%  \begin{tcolorbox}[
%  width=\textwidth,
%  colback={white},
%  title={Title},
%  colbacktitle={DarkGreen},
%  coltitle=white,
%  colframe={DarkGreen},
%  breakable
% ]
 % \parskip=5pt

\begin{tcolorbox}[
    % breakable,
    title=Prompt for Non-Seed Implications (2nd hop onward),
    colback=white,
    colframe=Green,
    arc=0pt,        % Remove rounded corners
    outer arc=0pt,  % Remove outer rounded corners (important for some styles)    
    % breakable,
]

You are provided with the following inputs:\\
- \text{[}Image\text{]}: An image (e.g. meme, new yorker cartoon, yes-but image)\\
- \text{[}Caption\text{]}: A caption written by a human.\\
- \text{[}Descriptions\text{]}: Literal descriptions that detail the image.\\
- \text{[}Implication\text{]}: A previously generated implication that suggests a possible connection between the objects or concepts in the caption and description.\\
\\
\#\#\# Your Task:\\
\texttt{[ One-sentence description of the ultimate goal of your task. Customize based on the task. ]}\\
Infer implicit meanings across the objects, concepts, situations, or facial expressions found in the caption, description, and implication. Focus on identifying relevant commonsense knowledge, social norms, or underlying connections.\\
\\
\#\#\# Guidelines:\\
- If you are unsure about any details in the caption, description, or implication, refer to the original image for clarification.\\
- Identify potential connections between the objects, actions, or concepts described in the inputs.\\
- Explore interpretations, contrasts, or relationships that naturally arise from the scene while remaining grounded in the inputs.\\
- Avoid repeating or rephrasing existing implications. Ensure each new implication provides fresh insights or perspectives.\\
- Each implication should be concise (one sentence) and avoid overly generic or vague statements.\\
- Be specific in the connections you make, ensuring they align closely with the details provided.\\
- Generate up to 3 meaningful implications that expand on the implicit meaning of the scene.\\
\\
\#\#\# Example Outputs:\\
\#\#\#\# Example 1 (example of newyorker cartoon image):\\
\text{[}Caption\text{]}: "This is the most advanced case of Surrealism I've seen."\\
\text{[}Descriptions\text{]}: A body in three parts is on an exam table in a doctor's office with the body's arms crossed as though annoyed.\\
\text{[}Implication\text{]}: Surrealism is an art style that emphasizes strange, impossible, or unsettling scenes.\\
\text{[}Connections\text{]}:\\
1. A body in three parts creates an unsettling juxtaposition with the clinical setting, which aligns with Surrealist themes.\\
2. The body’s crossed arms add humor by assigning human emotion to an impossible scenario, reflecting Surrealist absurdity.\\
... \\
\texttt{[ We used sample examples from the prompt for generating seed implications (see Figure \ref{fig:seed-imp-prompt}), following the above format, which includes [Implication]:. ]}
\\
---\\
\\
\#\#\# Proceed to Generate Output:\\
\text{[}Caption\text{]}: \texttt{[ Caption ]}\\
\\
\text{[}Descriptions\text{]}:\\
\texttt{[ Descriptions ]}\\
\\
\text{[}Implication\text{]}:\\
\texttt{[ Implication ]}\\
\\
\text{[}Connections\text{]}:
\end{tcolorbox}
\caption{A prompt used to generate non-seed implications.} % Add a caption to the figure
\label{fig:nonseed-imp-prompt}
\end{figure*}


%%%%%%%%%%%%%%%%%%%%%%%%%%% Prompt for nonseed implications %%%%%%%%%%%%%%%%%%%%%%%%%%%
\begin{figure*}[t]
\small
%  \begin{tcolorbox}[
%  width=\textwidth,
%  colback={white},
%  title={Title},
%  colbacktitle={DarkGreen},
%  coltitle=white,
%  colframe={DarkGreen},
%  breakable
% ]
 % \parskip=5pt

\begin{tcolorbox}[
    % breakable,
    title=Prompt for Candidate and Final Explanations,
    colback=white,
    colframe=RedViolet,
    arc=0pt,        % Remove rounded corners
    outer arc=0pt,  % Remove outer rounded corners (important for some styles)    
    % breakable,
]

You are provided with the following inputs:\\
- **\text{[}Image\text{]}:** A New Yorker cartoon image.\\
- **\text{[}Caption\text{]}:** A caption written by a human to accompany the image.\\
- **\text{[}Image Descriptions\text{]}:** Literal descriptions of the visual elements in the image.\\
- **\text{[}Implications\text{]}:** Possible connections or relationships between objects, concepts, or the caption and the image.\\
- **\text{[}Candidate Answers\text{]}:** Example answers generated in a previous step to provide guidance and context.\\
\\
\#\#\# Your Task:\\
Generate **one concise, specific explanation** that clearly captures why the caption is funny in the context of the image. Your explanation must provide detailed justification and address how the humor arises from the interplay of the caption, image, and associated norms or expectations.\\
\\
\#\#\# Guidelines for Generating Your Explanation:\\
1. **Clarity and Specificity:**  \\
   - Avoid generic or ambiguous phrases.  \\
   - Provide specific details that connect the roles, contexts, or expectations associated with the elements in the image and its caption.  \\
\\
2. **Explain the Humor:**  \\
- Clearly connect the humor to the caption, image, and any cultural, social, or situational norms being subverted or referenced.  \\
- Highlight why the combination of these elements creates an unexpected or amusing contrast.\\
\\
3. **Prioritize Clarity Over Brevity:**  \\
- Justify the humor by explaining all important components clearly and in detail.  \\
- Aim to keep your response concise and under 150 words while ensuring no critical details are omitted.  \\
\\
4. **Use Additional Inputs Effectively:**\\
- **\text{[}Image Descriptions\text{]}:** Provide a foundation for understanding the visual elements."   \\
- **\text{[}Implications\text{]}:** Assist in understanding relationships and connections but do not allow them to dominate or significantly alter the central idea.\\
- **\text{[}Candidate Answers\text{]}:** Adapt your reasoning by leveraging strengths or improving upon weaknesses in the candidate answers.\\
\\
Now, proceed to generate your response based on the provided inputs.\\
\\
\#\#\# Inputs:\\
\text{[}Caption\text{]}: \texttt{\text{[} Caption \text{]}}\\
\\
\text{[}Descriptions\text{]}:\\
\texttt{\text{[} Top-K Implications \text{]}}\\
\\
\text{[}Implications\text{]}:\\
\texttt{\text{[} Top-K Implications \text{]}}\\
\\
\text{[}Candidate Anwers\text{]}:\\
\texttt{\text{[} Top-K Candidate Explanations \text{]}}\\
\\
\text{[}Output\text{]}:\\

\end{tcolorbox}
\caption{A prompt used to generate candidate and final explanations.} % Add a caption to the figure
\label{fig:cand-prompt}
\end{figure*}


\section{Evaluation Prompts}
\label{app:eval-prompts}
Figures \ref{fig:recall-prompt} and \ref{fig:precision-prompt} present the prompts used to calculate recall and precision scores in our LLM-based evaluation, respectively.

%%%%%%%%%%%%%%%%%%%%%%%%%%% Prompt for nonseed implications %%%%%%%%%%%%%%%%%%%%%%%%%%%
\begin{figure*}[t]
\small
\begin{tcolorbox}[
    % breakable,
    title=Prompt for Evaluating Recall Score,
    colback=white,
    colframe=MidnightBlue,
    arc=0pt,        % Remove rounded corners
    outer arc=0pt,  % Remove outer rounded corners (important for some styles)    
    % breakable,
]

Your task is to assess whether \text{[}Sentence1\text{]} is conveyed in \text{[}Sentence2\text{]}. \text{[}Sentence2\text{]} may consist of multiple sentences.\\
\\
Here are the evaluation guidelines:\\
1. Mark 'Yes' if \text{[}Sentence1\text{]} is conveyed in \text{[}Sentence2\text{]}.\\
2. Mark 'No' if \text{[}Sentence2\text{]} does not convey the information in \text{[}Sentence1\text{]}.\\
\\
Proceed to evaluate. \\
\\
\text{[}Sentence1\text{]}: \texttt{[ One Atomic Sentence from Decomposed Reference Explanation ]} \\
\\
\text{[}Sentence2\text{]}: \texttt{[ Predicted Explanation ]}\\
\\
\text{[}Output\text{]}:

\end{tcolorbox}
\caption{Prompt for evaluating recall score.} % Add a caption to the figure
\label{fig:recall-prompt}
\end{figure*}


\begin{figure*}[t]
\small
\begin{tcolorbox}[
    % breakable,
    title=Prompt for Evaluating Precision Score,
    colback=white,
    colframe=MidnightBlue,
    arc=0pt,        % Remove rounded corners
    outer arc=0pt,  % Remove outer rounded corners (important for some styles)    
    % breakable,
]

Your task is to assess whether \text{[}Sentence1\text{]} is inferable from \text{[}Sentence2\text{]}. \text{[}Sentence2\text{]} may consist of multiple sentences.\\
\\
Here are the evaluation guidelines:\\
1. Mark "Yes" if \text{[}Sentence1\text{]} can be inferred from \text{[}Sentence2\text{]} — whether explicitly stated, implicitly conveyed, reworded, or serving as supporting information.\\
2. Mark 'No' if \text{[}Sentence1\text{]} is absent from \text{[}Sentence2\text{]}, cannot be inferred, or contradicts it.\\
\\
Proceed to evaluate. \\
\\
\text{[}Sentence1\text{]}: \texttt{[ One Atomic Sentence from Decomposed Predicted Explanation ]}\\
\\
\text{[}Sentence2\text{]}: \texttt{[ Reference Explanation ]}\\
\\
\text{[}Output\text{]}:


\end{tcolorbox}
\caption{Prompt for evaluating precision score.} % Add a caption to the figure
\label{fig:precision-prompt}
\end{figure*}

\section{Prompts for Baselines}
\label{app:base-prompts}

Figure \ref{fig:base-prompt} presents the prompt used for the ZS, CoT, and SR Generator methods. While the format remains largely the same, we adjust it based on the baseline being tested (e.g., CoT requires generating intermediate reasoning, so we add extra instructions for that).
Figure \ref{fig:critic-prompt} shows the prompt used in the SR critic model. The critic's criteria include: (1) \textit{correctness}, measuring whether the explanation directly addresses why the caption is humorous in relation to the image and its caption; (2) \textit{soundness}, evaluating whether the explanation provides a well-reasoned interpretation of the humor; (3) \textit{completeness}, ensuring all important aspects in the caption and image contributing to the humor are considered; (4) \textit{faithfulness}, verifying that the explanation is factually consistency with the image and caption; and (5) \textit{clarity}, ensuring the explanation is clear, concise, and free from unnecessary ambiguity.
\begin{figure*}
\small
\begin{tcolorbox}[
    % breakable,
    title=Prompt for Baselines,
    colback=white,
    colframe=Black,
    arc=0pt,        % Remove rounded corners
    outer arc=0pt,  % Remove outer rounded corners (important for some styles)    
    % breakable,
]

You are provided with the following inputs:\\
- **\text{[}Image\text{]}:** A New Yorker cartoon image.\\
- **\text{[}Caption\text{]}:** A caption written by a human to accompany the image.\\
\texttt{[ if Self-Refine with Critic is True: ]} \\
- **\text{[}Feedback for Candidate Answer\text{]}:** Feedback that points out some weakness in the current candidate responses.\\
\texttt{[ if Self-Refine is True: ]} \\
- **\text{[}Candidate Answers\text{]}:** Example answers generated in a previous step to provide guidance and context.\\
\\
\#\#\# Your Task:\\
Generate **one concise, specific explanation** that clearly captures why the caption is funny in the context of the image. Your explanation must provide detailed justification and address how the humor arises from the interplay of the caption, image, and associated norms or expectations.\\
\\
\#\#\# Guidelines for Generating Your Explanation:\\
1. **Clarity and Specificity:**  \\
   - Avoid generic or ambiguous phrases.  \\
   - Provide specific details that connect the roles, contexts, or expectations associated with the elements in the image and its caption.  \\
\\
2. **Explain the Humor:**  \\
- Clearly connect the humor to the caption, image, and any cultural, social, or situational norms being subverted or referenced.  \\
- Highlight why the combination of these elements creates an unexpected or amusing contrast.\\
\\
3. **Prioritize Clarity Over Brevity:**  \\
- Justify the humor by explaining all important components clearly and in detail.  \\
- Aim to keep your response concise and under 150 words while ensuring no critical details are omitted.  \\
\\
\texttt{[ if Self-Refine is True: ]}\\
4. **Use Additional Inputs Effectively:**\\
- **[Candidate Answers]:** Adapt your reasoning by leveraging strengths or improving upon weaknesses in candidate answers. \\
\texttt{[ if Self-Refine with Critic is True: ]}\\
- **[Feedback for Candidate Answer]:** Feedback that points out some weaknesses in the current candidate responses.\\
\\
\texttt{ [ if CoT is True: ]} \\
Begin by analyzing the image and the given context, and explain your reasoning briefly before generating your final response. \\
\\
Here is an example format of the output: \\
\{\{ \\
    "Reasoning": "...", \\
    "Explanation": "..."   \\
\}\} \\

Now, proceed to generate your response based on the provided inputs.\\
\\
\#\#\# Inputs:\\
\text{[}Caption\text{]}: \texttt{\text{[} Caption \text{]}}\\
\\
\text{[}Candidate Answers\text{]}: \texttt{\text{[} Candidate Explanations \text{]}}\\
\\
\text{[}[Feedback for Candidate Answer]:\text{]}: \texttt{\text{[} Feedback for Candidate Explanations \text{]}}\\
\\
\text{[}Output\text{]}:\\

\end{tcolorbox}
\caption{A prompt used for baseline methods, with conditions added based on the specific baseline being experimented with.} % Add a caption to the figure
\label{fig:base-prompt}
\end{figure*}


\begin{figure*}
\small
\begin{tcolorbox}[
    % breakable,
    title=Prompt for Self-Refine Critic,
    colback=white,
    colframe=Black,
    arc=0pt,        % Remove rounded corners
    outer arc=0pt,  % Remove outer rounded corners (important for some styles)    
    % breakable,
]
\texttt{[ Customize goal text here: ]} \\
\texttt{MemeCap:} You will be given a meme along with its caption, and a candidate response that describes what meme poster is trying to convey. \\
\texttt{NewYorker:} You will be given an image along with its caption, and a candidate response that explains why the caption is funny for the given image. \\
\texttt{YesBut:} You will be given an image and a candidate response that describes why the image is funny or satirical. \\
\\
Your task is to criticize the candidate response based on the following evaluation criteria: \\
- Correctness: Does the explanation directly address why the caption is funny, considering both the image and its caption? \\
- Soundness: Does the explanation provide a meaningful and well-reasoned interpretation of the humor? \\
- Completeness: Does the explanation address all relevant aspects of the caption and image (e.g., visual details, text) that contribute to the humor? \\
- Faithfulness: Is the explanation factually consistent with the details in the image and caption? \\
- Clarity: Is the explanation clear, concise, and free from unnecessary ambiguity? \\
 \\
Proceed to criticize the candidate response ideally using less than 5 sentences:\\
\\
\text{[}Caption\text{]}: \texttt{[ caption ]}\\
\\
\text{[}Candidate Response\text{]}: \\
 \texttt{\text{[} Candidate Response \text{]}}\\
\\
\text{[}Output\text{]}: \\
\end{tcolorbox}
\caption{A prompt used in SR critic model.} % Add a caption to the figure
\label{fig:critic-prompt}
\end{figure*}

% \begin{figure*}[t]
%   \includegraphics[width=\linewidth]{figures/error-analysis.pdf} \hfill
%   \vspace{-20pt}
%   \caption {Examples of negative impact from implications from Phi (top) and GPT4o (bottom).}
%   \label{fig:error-analysis}
% \end{figure*}



\end{document}
\endinput
%%
%% End of file `sample-sigconf-authordraft.tex'.
