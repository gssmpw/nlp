%%
%% This is file `sample-sigconf-authordraft.tex',
%% generated with the docstrip utility.
%%
%% The original source files were:
%%
%% samples.dtx  (with options: `all,proceedings,bibtex,authordraft')
%% 
%% IMPORTANT NOTICE:
%% 
%% For the copyright see the source file.
%% 
%% Any modified versions of this file must be renamed
%% with new filenames distinct from sample-sigconf-authordraft.tex.
%% 
%% For distribution of the original source see the terms
%% for copying and modification in the file samples.dtx.
%% 
%% This generated file may be distributed as long as the
%% original source files, as listed above, are part of the
%% same distribution. (The sources need not necessarily be
%% in the same archive or directory.)
%%
%%
%% Commands for TeXCount
%TC:macro \cite [option:text,text]
%TC:macro \citep [option:text,text]
%TC:macro \citet [option:text,text]
%TC:envir table 0 1
%TC:envir table* 0 1
%TC:envir tabular [ignore] word
%TC:envir displaymath 0 word
%TC:envir math 0 word
%TC:envir comment 0 0
%%
%%
%% The first command in your LaTeX source must be the \documentclass
%% command.
%%
%% For submission and review of your manuscript please change the
%% command to \documentclass[manuscript, screen, review]{acmart}.
%%
%% When submitting camera ready or to TAPS, please change the command
%% to \documentclass[sigconf]{acmart} or whichever template is required
%% for your publication.
%%
%%
% \documentclass[sigconf,authordraft]{acmart}
% \documentclass[sigconf,review,anonymous]{acmart}
% \documentclass[manuscript,review,anonymous]{acmart}
\documentclass[sigconf]{acmart}

%%
%% \BibTeX command to typeset BibTeX logo in the docs
\AtBeginDocument{%
  \providecommand\BibTeX{{%
    Bib\TeX}}}

%% Rights management information.  This information is sent to you
%% when you complete the rights form.  These commands have SAMPLE
%% values in them; it is your responsibility as an author to replace
%% the commands and values with those provided to you when you
%% complete the rights form.
\setcopyright{acmlicensed}
\copyrightyear{2025}
\acmYear{2025}
\acmDOI{XXXXXXX.XXXXXXX}

%% These commands are for a PROCEEDINGS abstract or paper.
\acmConference[IUI'25]{30th Annual ACM Conference on Intelligent User Interfaces}{March 24--27,
  2025}{Cagliari, Italy}
%%
%%  Uncomment \acmBooktitle if the title of the proceedings is different
%%  from ``Proceedings of ...''!
%%
%%\acmBooktitle{Woodstock '18: ACM Symposium on Neural Gaze Detection,
%%  June 03--05, 2018, Woodstock, NY}
\acmISBN{978-1-4503-XXXX-X/18/06}


%%
%% Submission ID.
%% Use this when submitting an article to a sponsored event. You'll
%% receive a unique submission ID from the organizers
%% of the event, and this ID should be used as the parameter to this command.
%%\acmSubmissionID{123-A56-BU3}

%%
%% For managing citations, it is recommended to use bibliography
%% files in BibTeX format.
%%
%% You can then either use BibTeX with the ACM-Reference-Format style,
%% or BibLaTeX with the acmnumeric or acmauthoryear sytles, that include
%% support for advanced citation of software artefact from the
%% biblatex-software package, also separately available on CTAN.
%%
%% Look at the sample-*-biblatex.tex files for templates showcasing
%% the biblatex styles.
%%

%%
%% The majority of ACM publications use numbered citations and
%% references.  The command \citestyle{authoryear} switches to the
%% "author year" style.
%%
%% If you are preparing content for an event
%% sponsored by ACM SIGGRAPH, you must use the "author year" style of
%% citations and references.
%% Uncommenting
%% the next command will enable that style.
%%\citestyle{acmauthoryear}

\usepackage{listings}
\usepackage{color} % Needed for colors

\usepackage{array}
\usepackage{colortbl}
\usepackage{xcolor}
\usepackage{booktabs}
\usepackage{tabularx}

% \usepackage[margin=1in]{geometry}
% \usepackage{forest}
% \usepackage{tikz}

\usepackage{siunitx} 
\usepackage{verbatim}

% \usepackage{tcolorbox}
\usepackage{courier}  % For better monospace font
\usepackage{setspace}

% Define colors
\definecolor{lightgray}{RGB}{245,245,245}
\definecolor{orange}{RGB}{240,148,0}
\definecolor{codeblue}{RGB}{100,149,237}
\definecolor{codered}{RGB}{205,92,92}




% \usepackage{xcolor}
% \usepackage{soul} % For highlighting

% % Define custom colors
% \definecolor{lightgray}{rgb}{0.95, 0.95, 0.95}
% \definecolor{darkred}{RGB}{172,41,0}

% Define a light gray background color

% Define a very light blue-gray background color
\definecolor{lightbg}{RGB}{248,250,252}


\definecolor{codegreen}{rgb}{0,0.6,0}
\definecolor{codegray}{rgb}{0.5,0.5,0.5}
\definecolor{codepurple}{rgb}{0.58,0,0.82}





\lstdefinestyle{mystyle}{
    backgroundcolor=\color{backcolour},
    commentstyle=\color{codegreen},
    keywordstyle=\color{magenta},
    numberstyle=\tiny\color{codegray},
    stringstyle=\color{codepurple},
    basicstyle=\ttfamily\footnotesize,
    breakatwhitespace=false,
    breaklines=true,
    captionpos=b,
    keepspaces=true,
    numbers=left,
    numbersep=5pt,
    showspaces=false,
    showstringspaces=false,
    showtabs=false,
    tabsize=2
}


\lstset{style=mystyle}



%%
%% end of the preamble, start of the body of the document source.
\begin{document}

%%
%% The "title" command has an optional parameter,
%% allowing the author to define a "short title" to be used in page headers.
% \title{{\tool}: Annotating Text-to-SQL Data via Mixed-initIative Query Alignment}
% \title{{\tool}: Interactive Text-to-SQL Data Annotation via Query Alignment}

\title{Text-to-SQL Domain Adaptation via Human-LLM Collaborative Data Annotation}


%
% The "author" command and its associated commands are used to define
% the authors and their affiliations.
% Of note is the shared affiliation of the first two authors, and the
% "authornote" and "authornotemark" commands
% used to denote shared contribution to the research.

\author{Yuan Tian}
\authornote{This work was done during the first author's internship at Adobe.}
\email{tian211@purdue.edu}
% \orcid{1234-5678-9012}
\affiliation{%
  \institution{Purdue University}
  \city{West Lafayette}
  \state{Indiana}
  \country{USA}
}

\author{Daniel Lee}
\email{dlee1@adobe.com}
\affiliation{%
  \institution{Adobe Inc.}
  \city{San Jose}
  \state{California}
  \country{USA}
}

\author{Fei Wu}
\email{feiw@adobe.com}
\affiliation{%
  \institution{Adobe Inc.}
  \city{Seattle}
  \state{Washington}
  \country{USA}
}

\author{Tung Mai}
\email{tumai@adobe.com}
\affiliation{%
  \institution{Adobe Inc.}
  \city{San Jose}
  \state{California}
  \country{USA}
}

\author{Kun Qian}
\email{kunq@adobe.com}
\affiliation{%
  \institution{Adobe Inc.}
  \city{Seattle}
  \state{Washington}
  \country{USA}
}

\author{Siddhartha Sahai}
\email{siddharthas@adobe.com}
\affiliation{%
  \institution{Adobe Inc.}
  \city{Seattle}
  \state{Washington}
  \country{USA}
}

\author{Tianyi Zhang}
\email{tianyi@purdue.edu}
% \orcid{1234-5678-9012}
\affiliation{%
  \institution{Purdue University}
  \city{West Lafayette}
  \state{Indiana}
  \country{USA}
}


\author{Yunyao Li}
\email{yunyaol@adobe.com}
\affiliation{%
  \institution{Adobe Inc.}
  \city{San Jose}
  \state{California}
  \country{USA}
}










%%
%% By default, the full list of authors will be used in the page
%% headers. Often, this list is too long, and will overlap
%% other information printed in the page headers. This command allows
%% the author to define a more concise list
%% of authors' names for this purpose.
% \renewcommand{\shortauthors}{Trovato et al.}



\newcommand{\tool}{\textsc{SQLsynth}}
\newcommand{\todo}[1]{\textcolor{red}{#1}}
\newcommand{\circled}[1]{{\large \textcircled{\footnotesize #1}}}

% \newcommand{\edit}[1]{\hl{#1}}
% \newcommand{\edit}[1]{\textcolor{red}{#1}}
\newcommand{\edit}[1]{#1}

\begin{abstract}
Out-of-distribution (OOD) detection and OOD generalization are widely studied in Deep Neural Networks (DNNs), yet their relationship remains poorly understood. We empirically show that the degree of Neural Collapse (NC) in a network layer is inversely related with these objectives: stronger NC improves OOD detection but degrades generalization, while weaker NC enhances generalization at the cost of detection. This trade-off suggests that a single feature space cannot simultaneously achieve both tasks. To address this, we develop a theoretical framework linking NC to OOD detection and generalization. We show that entropy regularization mitigates NC to improve generalization, while a fixed Simplex Equiangular Tight Frame (ETF) projector enforces NC for better detection. Based on these insights, we propose a method to control NC at different DNN layers. In experiments, our method excels at both tasks across OOD datasets and DNN architectures. 

\begin{comment}   

Out-of-distribution (OOD) detection and OOD generalization are critical for deploying machine learning models in real-world scenarios. While substantial progress has been made in addressing these problems independently, few works have attempted to tackle them jointly. However, existing methods often rely on auxiliary OOD training data and primarily focus on covariate-shifted OOD data that share labels with in-distribution (ID) data. In contrast, we tackle the more realistic and challenging task of jointly detecting and generalizing to semantic OOD data with disjoint labels from the ID data, without auxiliary OOD training data.
Achieving both objectives simultaneously is inherently difficult due to a fundamental conflict — OOD generalization requires enhanced transferability, while OOD detection necessitates the inhibition of transfer.
To address this, we leverage insights from neural collapse (NC) — a phenomenon in deep networks where top-layer representations suppress feature variability and adopt a Simplex Equiangular Tight Frame (ETF) structure, impairing transferability. By controlling NC, we unify OOD detection and generalization: preventing NC enhances OOD transfer while inducing NC improves OOD detection.
Our proposed method excels at both tasks across various OOD datasets and architectures. 

\end{comment}


\end{abstract}

%%
%% The code below is generated by the tool at http://dl.acm.org/ccs.cfm.
%% Please copy and paste the code instead of the example below.
%%
\begin{CCSXML}
<ccs2012>
   <concept>
       <concept_id>10003120.10003121.10003129</concept_id>
       <concept_desc>Human-centered computing~Interactive systems and tools</concept_desc>
       <concept_significance>500</concept_significance>
       </concept>
   <concept>
       <concept_id>10010147.10010257</concept_id>
       <concept_desc>Computing methodologies~Machine learning</concept_desc>
       <concept_significance>500</concept_significance>
       </concept>
 </ccs2012>
\end{CCSXML}

\ccsdesc[500]{Human-centered computing~Interactive systems and tools}
\ccsdesc[500]{Computing methodologies~Machine learning}




%%
%% Keywords. The author(s) should pick words that accurately describe
%% the work being presented. Separate the keywords with commas.
\keywords{Natural Language Interface, Text-to-SQL, Databases, Domain Adaptation, Interactive Data Annotation, LLMs, PCFG}
%% A "teaser" image appears between the author and affiliation
%% information and the body of the document, and typically spans the
%% page.
\begin{teaserfigure}
  \centering
  \includegraphics[width=\textwidth]{figure/teaser.png}
  \caption{An overview of {\tool}: \textbf{(A)} Given a database schema, {\tool} populates a database and sample SQL queries for the database. \textbf{(B)} After a step-by-step analysis of each sampled SQL query, {\tool} translates the query into a natural language (NL) question. \textbf{(C)} The large language model (LLM) aligns the NL question with the SQL query through the step-by-step analysis. Users can hover over each step to check the corresponding span in both SQL and NL. \textbf{(D)} {\tool} detects potential errors in the NL and highlights them in red. Users can repair the NL by injecting missing steps or deleting redundant text. \textbf{(E)} Users monitor and visualize the composition of the annotated dataset, thereby controlling the annotation process.
    }
  \label{fig:teaser}
\end{teaserfigure}

% \received{20 February 2007}
% \received[revised]{12 March 2009}
% \received[accepted]{5 June 2009}

%%
%% This command processes the author and affiliation and title
%% information and builds the first part of the formatted document.
\maketitle

\section{Introduction}

% State of the world (robots for creative activites)
The term ``robot,'' originally signifying `forced labor,' has long been associated with labor and work. Robots have demonstrated their utility in various automated productive and social contexts, where the primary goals are improving productivity, safety, and fostering social interactions with humans~\cite{simoes2022designing, weidemann2021role, honig2018understanding}. However, an increasing number of cases feature using of robots in creative settings. Unlike productive contexts, where the focus is on efficiency and task completion~\cite{arents2022smart}, or social contexts, where communication and trust are prioritized~\cite{nam2020trust, saunderson2019robots}, creative environments prioritize artistic innovation and expression~\cite{hsueh2024counts}. This shift fundamentally alters the dynamics of human-robot interaction, redefining the roles and expectations for both humans and robots.

For instance, robots’ social behaviors are leveraged to support the generation and expression of creative ideas~\cite{hu2021exploring, sandoval2022human, alves2020creativity}, and programmable robotic movements and trajectories are employed to inspire artistic activities such as sketching~\cite{lin2020your}. These studies often engage participants from creative fields who possess limited prior experience with robotics, and are typically conducted in short-term, experimental settings. Consequently, the findings from these studies remain constrained since much can be learned from professional practitioners' experiences to inform system design such as digital fabrication~\cite{hirsch2023nothing}. There is a notable gap in research examining the long-term, active, and practical experience of integrating robotic systems into the creative processes. As a result, the deeper insights into how robots facilitate and shape creative processes, beyond simply augmenting human creativity, remain underexplored. In this study, we aim to better understand the impacts of robots on creative processes and outcomes.

As early as Leonardo da Vinci's 16th century ``Automaton,'' artists have explored the creative affordances of robotic systems~\cite{shanken2002cybernetics, pagliarini2009development, jeon2017robotic}. The artistic creation process typically encompasses various stages, including the exploration of materials and techniques, ongoing experimentation and iteration, and the continual refinement of the artists' insights into their creative subjects~\cite{lewis2023art, sturdee2022state}. Therefore, investigating the artistic process involving robots offers an opportunity to gain deeper insights into robots' creative potential. Robotic art, in particular, provides a compelling case for this exploration.

We define robotic art as artworks that utilize robotic or automated machines to create artistic experiences and tangible artifacts. One example is robotic installation art, in which robots are programmed to follow specific rules that embody the artist’s expression (\autoref{fig:teaser} (a)). Another example is responsive art, in which robots react to their environment, with behaviors that change over time or in response to spectators (\autoref{fig:teaser} (b)). Additionally, there are robotic creators, which possess a degree of agency, allowing them to collaborate with human artists and produce works that extend beyond mere replication of human-created art (\autoref{fig:teaser} (c) and (d)). As such, robotic art becomes a rich case for exploring human-machine interactions in creative contexts. Gaining a deeper understanding of how robots facilitate artistic expression can provide insights for designing computing systems to support creative activities~\cite{gomez2021robot}.

% Therefore, we did...
We draw on semi-structured, in-depth interviews with renowned professional robotic artists to investigate the use of robots in artistic practice. Specifically, our goal is to understand how artistic exploration of robotic systems challenges conventional assumptions about the functions of robots, such as their roles in automating repetitive tasks or serving human needs. We also explore the implications of robots in the artistic process and examine how creativity may emerge within robotic art. To address these interrelated inquiries, our study focuses on the practice of robotic art, posing the research question: \textit{How do robotic artists utilize robots in their artistic practice?} We approach this inquiry through the perspectives and experiences of robotic artists, who creatively design, modify, and repurpose robotic systems for artistic expression and exploration.

% The key findings are...
Our findings highlight the social, material, and temporal dimensions of artists' practices that shape their creativity and artistic outcomes. The creation of robotic art is largely a social process, as artists receive both explicit and implicit feedback through the audience's reactions and reception of their work. Simultaneously, the embodiment and malfunctions inherent to robotic systems drive artistic experimentation. The temporal processes of creation and exhibition, beyond just the final product, further enhance the creative value. Our empirical analysis presents how creativity emerges through the interplay of social, material, and temporal interactions among artists, robots, audiences, and the environment.

% The contributions of this work are...
We make two main contributions to HCI in this study. 
First, we elucidate the interactive mechanisms among key actors---human creators, machines, audiences, and environments---within the practice of robotic art, a topic that remains underexplored in HCI. Our findings reveal the significance of sociality (e.g., interactions between artists and audiences), materiality (e.g., the embodiment and malfunctions of robots), and temporality (e.g., the processes of creation and exhibition) in shaping creative values. We propose that these three facets are central to the creative process and facilitate the emergence of creativity in robotic art.
Second, drawing from the findings, we offer implications for \textit{socially informed}, \textit{material-attentive}, and \textit{process-oriented} creation with computing systems. We suggest leveraging these three aspects to enhance creativity and the creative experience. Specifically, we discuss the value of incorporating implicit audience feedback, designing with technical malfunctions, and focusing on the post-creation process to foster alternative creative experiences with machines~\cite{alter2010designing, juarez2022glitch}.



\paragraph{Uncertainty-based hallucination detection methods.}
Various approaches have been proposed to detect hallucinated content in LLMs generation.
Unlike other methods that require external knowledge sources for fact-checking~\citep{gou2024critic, chen-etal-2024-complex, min-etal-2023-factscore, huo2023retrieving}, uncertainty-based approaches are reference-free and rely only on LLM internal states or behaviors to determine hallucination~\citep{10.1145/3703155}. 
For instance, sampling-based approaches generate multiple responses and measure the diversity in meaning among them~\citep{fomicheva-etal-2020-unsupervised, kuhn2023semantic, lin2024generating}, while density-based approaches approximate the training data distribution and provide probabilities or unnormalized scores to assess how likely a generated response belongs to the distribution~\citep{yoo-etal-2022-detection, ren2023outofdistribution, vazhentsev-etal-2023-hybrid}.

In this paper, we focus on uncertainty quantification methods that rely on token-level likelihood or entropy~\citep{guerreiro-etal-2023-looking, malinin2021uncertainty}. 
Recent works have explored refining likelihood estimation by incorporating semantic relationships or reweighting token importance. For instance, Claim-Conditioned Probability (CCP)~\citep{fadeeva-etal-2024-fact} was introduced to recalculate likelihood according to semantical equivalence; while \citet{zhang-etal-2023-enhancing-uncertainty} and \citet{duan-etal-2024-shifting} adjust token weights to better convey meaning in uncertainty aggregation. \emph{Although these approaches leverage token-level information, they are typically evaluated at the sentence level, raising questions about their reliability}. To address this, we conduct a comprehensive analysis of entity-level hallucination detection for finer-grained performance insights.


\paragraph{Fine-grained hallucination detection benchmark.}

Most hallucination detection benchmarks are in sentence or paragraph level. For example, CoQA~\citep{reddy-etal-2019-coqa}, TriviaQA~\citep{joshi-etal-2017-triviaqa}, TruthfulQA~\citep{lin-etal-2022-truthfulqa}, and HaluEval~\citep{li-etal-2023-halueval}. These benchmarks classify each generated response as either hallucinated or correct. However, instance-level detection cannot pinpoint specific hallucinated content, which is crucial for correcting misinformation~\citep{cattan2024localizingfactualinconsistenciesattributable}. This limitation becomes particularly problematic in long-form text, where a single response often combines supported and unsupported information, making binary quality judgments inadequate~\citep{min-etal-2023-factscore}.

To address these challenges, recent works have advanced benchmarks for more granular hallucination detection. For example, \citet{min-etal-2023-factscore} introduced \textsc{FActScore}, which decomposes LLM-generated text into atomic facts---short sentences conveying a single piece of information---for more precise evaluation. In parallel, \citet{cattan2024localizingfactualinconsistenciesattributable} introduced \textsc{QASemConsistency}, decomposing LLM generated text with QA-SRL, a semantic formalism, to form simple QA pairs, where each QA pair represent one verifiable fact. \emph{However, these methods do not enable entity-level hallucination detection, as they lack explicit entity-level labeling (hallucinated or not) in the original generated text}.  
Beyond decomposition-based approaches, datasets like \textsc{HaDes}~\citep{liu-etal-2022-token} and CLIFF~\citep{cao-wang-2021-cliff} create token-level hallucinated content by perturbing human-written text, allowing token-level annotation on the same text. These perturbed hallucinated content, however, could be unrealistic, biased, and overly synthetic due to the limitations of models they used to perturb words. 
To bridge this gap, we create a new dataset with entity-level hallucination labels on the same LLMs generated text. This allows us to evaluate uncertainty-based hallucination detection approaches on a finer-grained level and analyze their reliability.





% 尽管HCI 研究开始关注mental health的homework的支持【】,但是艺术治疗里的homework对于HCI研究仍然是一个尚待理解和探索的新场景
%也尚未有HCI design cases探索如何设计能够较好支持艺术治疗homework的包含AI agents 的系统。
%因此,为了给我们接下来的设计探索收集inputs,我们组织了formative study。我们的主要目的有二:
    % 理解艺术治疗家庭作业的场景
    % 理解设计支持艺术治疗家庭作业的包含AI agent 的系统应该满足哪些需求,符合哪些quality(这里得到的结论应该是下一个阶段设计部分比较重要的交互或者界面或者功能特性,以及比较关键的设计rationales)

\section{Contextual Understanding}
Recent HCI research has pinpointed the significance of understanding therapy homework in mental health~\cite{Oewel_2024}, yet art therapy homework remains a unique and unaddressed domain. 
Therefore, we conducted a contextual study with a group of therapists to gain a concrete understanding of current art therapy homework practice and to identify common needs for technological support.
\begin{figure*}[tb]
  \centering
  \includegraphics[width=\linewidth]{images/1.jpg}
  \vspace{-4mm}
  \caption{Art therapy homework outcomes from the therapists' previous practice: (a): T4; (b)-(c): T5; (d)-(g): T3}
  \Description{This Figure showcases the outcomes of homework practices among art therapists. From left to right: (a) a client completing a homework task on a structured worksheet; (b) a depiction of a volcano represented by yellow patterns, with orange indicating imminent erupting lava; (c) an outline of a small figure containing a floral pattern in black and red; (d) a composition using text alongside red and blue floral designs; (e) a diary entry documented by a client; (f) a handcrafted green mountain created by a client; (g) a client-made black clay figurine placed on a patch of grass.}
  \label{fig:context1}
\end{figure*}

%second, to understand the needs and qualities of human-AI systems in supporting art therapy homework.
% formative study procedure
    % 找了谁
    % 怎么做的
        % 我们组织了一对一的疗愈师访谈,来理解艺术治疗家庭作业的当前practice,包扩(当前的practice,艺术治疗家庭作业是什么样一个形式,怎么布置的,做些什么,有哪些疗愈意义,治疗师想要通过家庭作业达到什么目标)
        % 其次,我们基于艺术治疗理论和相关的前期工作,以及访谈中新获得的理解,以准备了一个初步的demo和mockup来作为formative study的准备材料
            % 相关的理论和研究表明艺术治疗的家庭作业一般需要结合艺术创作与verbal反思两个元素,然而目前并未有将两者结合在一起的系统,为了和疗愈师共创式设计,我们构建了一个简单的家庭作业系统demo,它包含一个让用户通过绘制语义分割来生成图像的画板(类似的画板已被应用于艺术治疗practice,见DeepThink),以及一个可以理解用户在画板上绘制行动并提出问题鼓励用户近一步表达创作过程的AI agent。我们尽量保持系统的simplistic和open-ended以方便疗愈师参与到接下来的协同设计并能最大限度输出他们的经验。
            % 与此同时,我们构想了一个初步的疗愈师界面,目的是辅助疗愈师monitor和review来访者的家庭作业。我们制作了静态的mock up以便疗愈师在此基础上进一步协同创作发展设计。
        %我们用这个初步的demo和mockup组织疗愈师进行了两次的协同设计工作坊,流程:
            % 介绍了协同设计的目标(设计支持疗愈师和来访者的艺术治疗家庭作业的AI工具)
            % 我们展示了demo和mockup
            % 让治疗师进行了交互体验和自由讨论,疗愈师们在本地设备使用了我们的demo,体验了我们的疗愈师端mockup,并且分别进行了在线的讨论,表达了丰富的对于系统设计如何可以更好支持艺术治疗家庭作业的意见,以及交互体验方面的建议,然后我们对
\begin{table*}[tb]
\caption{Demographics of Participant Therapists: Experience refers to the number of years engaged in art therapy; The Number of Case refers to cases related to art therapy; The Number of Online Case refers to cases related to online art therapy}
\label{tab:expert}
\vspace{-3mm}
\small
\resizebox{\textwidth}{!}{
\begin{tabular}{ccccccccc}
\toprule
ID & Age & Gender & Experience & Education Level& Major & Region & The Number of Case&The Number of Online Case\\
\midrule
T1& 39& F& 6 & Master & Art Therapy & United States(Florida) &300+&65\\
T2& 41& F& 10 & Master & Art Therapy & Italy(Puglia) &200+&12\\
T3& 49& F& 8 & Master & Art Therapy & China(Guangdong) &350+&85\\
T4& 37& F& 5 & PhD & Cognitive Psychology\&Art Therapy&China(Hongkong) &100+&52\\
T5& 24& F& 2 & Master & Art Therapy& China(Hangzhou) &100&45\\
\bottomrule
\end{tabular}
}
\Description{Table 1 presents the demographics of the participant therapists. Experience refers to the number of years they have been engaged in art therapy, and the Number of Cases indicates the number of art therapy cases they have handled. The five therapists are as follows: T1 is 39 years old, female, with 6 years of experience. She holds a Master’s degree in Art Therapy and practices in Florida, United States, having managed over 300 cases. T2 is 41 years old, female, with 10 years of experience. She has a Master’s degree in Counseling Psychology and works in Puglia, Italy, with more than 200 cases. T3 is 49 years old, female, with 8 years of experience. She holds a Master’s degree in Art Therapy and is based in Guangdong, China, having overseen over 350 cases. T4 is 37 years old, female, with 5 years of experience. She has a PhD in Cognitive Psychology and Art Therapy and practices in Hong Kong, China, having handled more than 100 cases. T5 is 24 years old, female, with 2 years of experience. She holds a Master’s degree in Art Therapy and is located in Hangzhou, China, with approximately 100 cases managed. The Number of Online Case refers to cases related to online art therapy
}
\end{table*}

\subsection{Procedure and Preparation}
Five art therapists (T1-T5; all self-identified females; aged 24-49) participated in this study. None of the therapists were members of the research team. T3 was a previous collaborator; the other therapists were recruited via T3's professional network, intended for a diverse group of practitioners from various geographical locations.
Their demographics and expertise are detailed in~\autoref{tab:expert}.
We first conducted 60-minute remote one-on-one interviews with each therapist to understand their current homework practice. This was followed by two 60-minute online focus groups with the therapists. The researchers, acting as facilitators, moderated the discussion on the common challenges for homework practice, aiming to identify needs and design opportunities.
In addition, we kept close collaboration with the therapists throughout the development phase and conducted informal follow-ups to gather inputs in formulating design features.
The one-on-one interviews and focus group sessions were screen-recorded and transcribed. We conducted open coding and affinity diagramming to identify emerging insights reported below.

%Initially, we conducted remote, one-on-one semi-structured interviews with each therapist to gather insights into their current practices and the challenges regarding art therapy homework. 
%\textcolor{blue}{Subsequently, we held two 60-minute online focus group discussions with these therapists. The researchers, acting as facilitators, guided the discussions using the challenges and needs identified in prior one-on-one semi-structured interviews to encourage broader conversations. The goal was to identify the common needs and challenges therapists face in their practice, and, secondly, to closely collaborate with them in co-designing the system's core features.}
%Subsequently, we held two online focus groups with these therapists to foster a broader discussion, aiming to identify common needs and challenges in their practices.
%We recruited five art therapists (T1-T5; 5 self-identified females; aged 24-49) whose demographics and expertise are detailed in~\autoref{tab:expert}. 
%We first conducted remote, one-on-one semi-structured interviews with the five therapists in order to understand the current individuals' practices and challenges of art therapy homework. 
%Further, we conducted two remote focus groups with the five therapists in order to promote their discussion about these individuals' practices and challenges and identify common ground.
%First, we showcased the demo and mock-up, enabling the therapists to experience them. Following this, we went through online discussions where they provided feedback on how AI agent system design could enhance support for art therapy homework and offered suggestions for enhancing the interactive experience.


%Further, in order to co-design with the therapists, we developed a demo and mock-up as preparatory materials for context study. 
%First, Existing literature on art therapy homework, along with insights from interviews, suggests that therapy homework could combine art-making with verbal expression, yet no existing system combines these elements. Thus, we developed a demo featuring a drawing tool for AI image generation through semantic segmentation(similar to cases used in art therapy practices, such as DeepThink~\cite{du2024deepthink}) and a conversation agent that understands users' actions and asks questions to encourage description of the creation. 
%Second, we envisioned an therapist interface aimed at assisting therapists in monitoring and reviewing clients' therapy homework. The simplistic and open-ended demo and the static mock-up allowed therapists to contribute their expertise in future co-design workshops. To design an AI agent systems to support therapists and clients with art therapy homework, 


% formative study results  
% understanding current art therapy homework practice
        %介绍visual arts和written的形式
            % 的确有art thearpy homework
            % 疗愈师说了家庭作业是什么样的形式(介绍常有的形式,一般是艺术创作,记日记,拍照,做手工,)
             % 家庭作业很重要,为什么重要,有什么功能,可以怎么样影响来访者:
                % 1
                % 2
        %定制化和数据review
            %介绍定制化需求对于治疗师很重要
                % 治疗师会结合她掌握的疗愈技术和艺术治疗方法来定制家庭作业
                % 治疗师也会根据上一节session灵活调整家庭作业
            % 介绍review的重要性
                %艺术疗愈师需要看到家庭作业,需要用到家庭作业:为什么需要看到,为什么需要用到,怎么用的
                %retrieve,依从性compliance,不知道有没有按时做,
\subsection{Contextual Understanding: Current Practice and Common Challenges}

Our therapists confirmed that art therapy homework plays a crucial role in helping them understand and collaborate with clients between sessions. They shared their current methods for assigning art therapy homework, which often involves multi-modal activities~(see \autoref{fig:context1}) combining visual arts (e.g., drawing, collage-making, photography and clay sculpting) with written or spoken documentation of emotions and experiences (e.g., journaling, social media posts, and audio recordings).
The therapists noted that integrating visual presentations with verbal expression is a common practice, as it helps clients document and articulate their experiences. For example, T4 combined art-making with audio recording to assist clients in expressing their current feelings: \qt{I asked the elderly [clients] to take photos and create collages at home and encouraged them to record audio to share their daily emotions}. The therapists believed that this combination encourages clients to more fully describe their artwork, explore subconscious thoughts behind the creative process, and gain new perspectives on their lives.
Aside from their approach of leveraging art therapy homework in current practice, the therapists also share their challenges regarding art therapy homework. From their shared experiences, three major sets of challenges emerged, which are summarized below:
% As revealed by the therapists, they invited their clients to complete multi-modal forms of art therapy homework, mainly combining visual arts~(e.g., drawing, collage-making, photography) with the written and spoken document of current emotions and experiences~(e.g., journaling, social media posting, and audio recording).
% Our therapists noted that integrating visual presentations with verbal expression is a common practice in art therapy homework, as it helps clients document and articulate their current experiences.
% For example, T4 integrated art-making with audio recording to help clients document their current experiences and feelings. 
% Our therapists believed that combining art-making with verbal expression encourages clients to express and describe their artwork more fully, explore subconscious thoughts behind art-making, and cultivate new perspectives on various aspects of their lives. Further, our therapists emphasized they need to customize homework assignments in art therapy and track their the homework outcomes, which could build an therapeutic collaboration between therapists and clients.

\subsubsection{\textbf{CH1}: Challenges in Homework Threshold without Therapist Guidance} 

Our therapists indicated that art-making-based therapy homework can pose a creative barrier for clients without therapist guidance~(\textbf{CH1-1}). T4 noted that this barrier could lead to stress, self-criticism, and fear of failure: \qt{If a client is self-critical, they may fear creating something `ugly', which can increase pressure and hinder the therapeutic process}. Consistent with prior studies~\cite{Tang2017,Harwood2007}, the therapists also confirmed that clients may lack confidence in completing homework or producing emotional responses without guidance, which can result in lower compliance.
Additionally, therapists expressed concerns that clients might struggle to interpret their artwork in a therapeutic way without support, reducing their motivation for deep reflection~(\textbf{CH1-2}). T1 observed that without proper guiding, it can be difficult for clients to make full use of the exercise: \qt{Last time, I assigned a homework about `your ideal future family', but [...] she just scribbled a bit without expressing any clear thoughts}. The therapists emphasized the importance of guiding clients in verbalizing their emotions alongside art-making. T5 mentioned that while visual art can help explore subconscious thoughts, verbalizing these feelings provides a cathartic outlet and helps clients externalize their emotions.

% Moreover, therapists were concerned that clients might struggle to interpret their artwork in a therapeutic way without guidance, leading to reduced motivation for deep reflection. T1 noted that without clear direction, creating a meaningful drawing that promotes reflection can be difficult for clients.
% Additionally, therapists confirmed the importance of guiding clients to properly verbalize their feelings alongside art-making. As T5 mentioned, visual art can serve as a channel for exploring and expressing subconscious thoughts, while verbalizing these feelings provides a cathartic outlet and helps clients externalize their emotions.

% First, our therapists indicated that art-making-based therapy homework might present a creative threshold for clients.
% For example, T4 explained that the homework is to ensure that the creative process remains therapeutic and accessible, with low threshold, so participants can avoid stress, self-criticism, and fear of failure.
% Second, our therapists was concerned that clients may struggle to interpret their artwork in a therapeutic direction without the guidance of a therapist, which lead to a lack of motivation to engage in deep reflection. 
% Also, T1 suggested that it could be challenging for clients to create a meaningful drawing that effectively promotes reflection without clear guidance.
% Therapists also confirmed that it is crucial to prompt the clients to verbalize their feelings in addition to the art-making. As mentioend by T5, the visual art-making could be a channel for clients to explore and express themsleves at the subconscious level, whereas, verbalizationg could help them externalize the subconscious thoughts and find themself a carthartic outlet.
% Prior studies have shown that clients may struggle with confidence in completing homework and producing emotional arousal without the therapist's guidance~\cite{Tang2017,Harwood2007}, leading to reduced homework compliance.

\subsubsection{\textbf{CH2}: Challenges in Customizing Therapy Homework} 

Our therapists demonstrated their practice of customizing homework assignments in art therapy. For instance, T2 and T5 mentioned tailoring homework tasks and specific instructions based on their practical experience and therapeutic techniques (e.g., cognitive-behavior therapy or mindfulness): \qt{If I suggest therapy homework that integrates mindfulness with art-making, I might ask the client to notice any changes in their breathing [during homework]}~(T4). T1 also adjusted homework tasks based on feedback from previous in-sessions.
However, the therapists noted that adapting structured instructions flexibly was difficult with current verbal or written formats, often leading to clients forgetting or abandoning their guidance or instructions~(\textbf{CH2-1}). Additionally, T3 and T4 observed that offering encouraging words and support during homework could boost motivation, but they found it challenging to provide personalized encouragement outside of in-session times~(\textbf{CH2-2}).

% Our therapists demonstrated their practice of customizing homework assignments in art therapy, e.g., T2 and T5 both mentioned that 
% they tended to tailor diverse homework assignments and specific instructions based on drawing from their own practical experience and therapeutic techniques~(e.g., CBT or mindfulness): \qt{If I suggest therapy homework that integrates mindfulness with art-making, I might suggest that he noticed any changes in his breathing while observing the artwork~(T4)}.
% Also, T1 flexibly adjusted the homework tasks based on feedback from the previous in-session.
% However, our therapists noted that adapting structured instructions flexibly was challenging using existing verbal or written descriptions.
% This often lead to clients forgetting or abandoning their therapy homework.
% Moreover, T3 and T4 noted that offering encouraging words and support during homework could enhance motivation for completion. However, they currently find it challenging to tailor this encouragement and care after in-sessions.

\subsubsection{\textbf{CH3}: Challenges in Tracking Therapy Homework History} 

The therapists confirmed that original homework data---such as the artworks, conversation records about clients' creative states, and details of the creative process---were essential for their assessments. They also encouraged clients to bring homework outcomes to the next session. For example, T1 and T3 prompted clients to share their current feelings and perspectives during one-on-one sessions, while T4 encouraged clients to engage in re-creation based on their homework.
However, therapists commonly expressed difficulty in tracking homework history, as they relied on clients to record and report their own progress~(\textbf{CH3-1}): \qt{The client drew [an artwork] two months ago. When you showed her the artwork, she often didn't remember what had happened at the time~(T3)}. Additionally, T1, T3, and T4 raised concerns that current practices might miss valuable data regarding clients' emotional or mental states at the time the homework was completed~(\textbf{CH3-2}).

% Our therapists confirmed that original homework data, including the artwork, conversation records about clients' current creative states, and the creative process, were all crucial for their assessment.
% Meanwhile, the homework outcomes was encouraged to be brought to the next in-sessions, e.g., T1 and T3 encouraged clients to share their current feelings and perspectives on people and things during the one-on-one sessions. 
% Also, T4 encouraged clients to engage in re-creation activities during the in-sessions, building upon their homework outcomes.
% However, our therapists noted that they found difficult to track the homework history, as they relied on clients to record and report their own progress.
% Also, T1, T3, and T4 raised concerns that current practices might be missing valuable homework data regarding clients' homework assignments, specifically related to the client's status at the time the homework was completed.




   % challenges of current practice
        % 【但是】创作门槛高,便利性。(找话可以支持对应)
        % 【但是】:来访者缺少指导,没有引导,很难知道是否真的发生了反思(找话可以支持对应)
        % 【但是】:难以追踪,难以记录 (找话可以支持对应报告)
%\subsubsection{\textbf{Current challenges}}

%\textbf{D1: Supporting Therapy Homework by Integrating Verbal Expression with Art-making.} Our therapists suggested that therapy homework should be supported through combining art-making with verbal expression.They emphasized the value of integrating art-making and verbal expression in AI-infused art therapy. Likewise, T1 indicated that it not only enabled clients to gain a deeper understanding of their own artwork but also supported their process of self-expression. Further, the therapists envisioned that AI has the potential to further ask in-depth and structured questions based on artwork, thereby eliciting deeper reflections from clients. \textbf{D2: Supporting Customization of Therapy Homework via Agents.} T5 envisioned that conversational agents as \qt{homework assistants} that can guide clients to further explore some deeper self-reflections.Further, T2 suggested that AI agents have the advantage of conveying more caring and supportive messages from therapists to clients. Finally, our therapists noted that they needed to set homework topics and specific instructions in a therapist interface according to their own practice principles.\textbf{D3: Supporting Homework History Gathering and Summarization via AI agents.} The therapists further proposed that AI has the potential to assist in summarizing descriptions of clients' creations and capturing their emotions or experiences. T4 emphasized that AI agents should function as a summary tool rather than providing sophisticated interpretation. For example, T3 suggested that AI could identify and summarize recurring images in clients' artwork. This summarization can facilitate more in-depth discussions during one-on-one sessions.



\section{Implementation}

% \todo{You need to get rid of sentences about schema customization and repurpose your writing to schema comprehension. I don't think the schema editor is a novel IUI contribution. An undegrad can build such an editor using modern UI frameworks like Material UI. You may feel that every feature you have developed in the tool needs to be elaborated in the paper. It's not the case. You need to highlight the features and interface design choices that are relevant to the scientific problem solved by this work and downplay other features even though these other features are look visually nice. Then you will get a coherent scientific story. A research paper is not to sell an engineering piece. It's about presenting a general idea/design/method to solve a scientific problem and the findings on how well this idea/design/method works in an evaluation setting.}
% {\tool} comprises multiple UI components, corresponding to a step in text-to-SQL dataset annotation and address certain user needs, including schema comprehension, database population, SQL query and NL question generation, error detection and repair, and dataset diversity analysis.

% \begin{enumerate}
%     \item \textit{\textbf{Schema Editor}} (Fig.~\ref{fig:page1}) enables users to understand and customize a database schema.
%     \item \textit{\textbf{Database Synthesizer}} (Fig.~\ref{fig:page2}) allows users to swiftly populate a database based on the provided schema.
%     \item \textit{\textbf{Query Annotator}} (Figs.~\ref{fig:page3_A} and \ref{fig:page3_B}), the core component of {\tool}, facilitates efficient text-to-SQL dataset annotation with minimized bias, errors, and effort.
%     \item \textit{\textbf{Dataset Analyzer}} (Fig.~\ref{fig:page4}) provides dynamic visualization of dataset properties, allowing users to monitor and refine the annotation process.
% \end{enumerate}

% In this section, we describe the implementation for each component in detail.

{\tool} comprises multiple UI components, each corresponding to a step in text-to-SQL dataset annotation and addressing specific user needs mentioned in Section~\ref{sec:user_needs}. 
We describe each component in this section. 
Additional implementation details, such as algorithms and prompt design, are discussed in Appendix~\ref{app:pcfg} and Appendix~\ref{app:prompt}.

% \subsection{Schema Editor}
\subsection{Schema Visualization}

% Text-to-SQL dataset synthesis is intrinsically tied to a specific database schema. The initial step in creating these datasets depends on the customization of the database schema. However, our interview indicates that customizing database schemas is a non-trivial task without an effective interface, especially when the schema gets complex.
% To bridge this gap, {\tool} includes \textit{schema editor} at the first step.

\begin{figure*}[ht]
  \centering
  \includegraphics[width=\textwidth]{figure/page1.png}
  \caption{The user interface for schema visualization. \edit{Each node represents a database table, while each cell represents a column in the table. The blue cell marked with ``PK'' represents the primary key. The dashed gray edge represents a foreign key reference relationship between two tables. Users can (a) add a new table, (b) add a new column, (c) add a reference relationship, (d) define the data type for a column, (e) add a description for columns and tables, and (f) remove, upload, or download the database schema.}}
  \label{fig:page1}
\end{figure*}




Annotating text-to-SQL datasets requires annotators to understand the corresponding database schema. However, practical database schemas can be complex and challenging to comprehend. To facilitate user comprehension of schema, {\tool} enables users to visualize the schema in an interactive graph, as shown in Figure~\ref{fig:page1}.
Users can upload the schema by either dragging a schema file onto the canvas or using the upload button. Each table is visualized as a box, with columns listed as rows within it.
The primary keys are colored blue and marked with ``PK''. 
Reference relationships between columns in different tables are rendered as dashed lines, with a flow animation indicating the reference direction.
To inspect details, users can hover over columns and tables to view data types or entity descriptions. The interface allows for dragging tables, zooming in and out, and panning across the view.
 
The graph is editable, allowing users to update the schema as needed. To add new tables, users can click the ``ADD TABLE'' button in the top-left corner (Fig.~\ref{fig:page1} \circled{a}). Hovering over a table allows users to add new columns or designate primary keys (Fig.~\ref{fig:page1} \circled{b}).
Users can specify the data type for each column and directly link two columns to establish foreign key relationships (Fig.~\ref{fig:page1} \circled{c}). To remove columns, users can click the trash icon that appears on hover. 
Tables and reference links can be removed using the backspace key on the keyboard.

Given that practical database entities often use abbreviations, clear documentation can help LLMs better interpret the schema and provide more accurate annotation suggestions in subsequent steps. {\tool} encourages users to add descriptions to tables and columns (Fig.~\ref{fig:page1} \circled{e}).
For better schema management, users can quickly remove the entire schema, as well as download or upload it as a JSON file (Fig.~\ref{fig:page1} \circled{f}).

% Users can specify the data type of each column. Users can directly link two columns to specify the foreign key relationship (Fig.~\ref{fig:page1} \circled{c}).
% To remove columns, users can click the red trash icon that appears on hover. Tables and reference links can be deleted using the keyboard's delete key.
% Since practical database entities are often named using abbreviations, maintaining clear documentation can help LLMs interpret the schema and provide more accurate annotation suggestions in subsequent steps. {\tool} encourages users to add descriptions to tables and columns before annotation (Fig.~\ref{fig:page1} \circled{e}).
% Users can quickly remove the entire schema, and download or upload the schema as a JSON file (Fig.~\ref{fig:page1} \circled{f}).

% To facilitate better schema management, users can clear the current schema, and download or upload the schema as a JSON file (Fig.~\ref{fig:page1} \circled{f}).

% This page facilitates complex database schema comprehension and customization. 
% Once users finalize the schema, they can proceed to the second page to synthesize database records based on this customized schema.

\begin{figure*}[ht]
  \centering
  \includegraphics[width=0.85\textwidth]{figure/page2.png}
  \caption{The user interface for database population. \edit{Users can (a) populate the database with a specified number of records, (b) switch table views, and (c) upload or download synthesized records.}}
  \label{fig:page2}
\end{figure*}

\subsection{Database Population}

% \todo{It is not clear what the rules are, how many data types your synthesizer supports, and where the values for each data type is sampled from. For example, if a column requires social security numbers in a specific format, how do you synthesize these social security numbers in the right format? In the example below, student ID often has a fixed number of digits. How do you inform the synthesizer to generate student IDs with such a constraint?}
% A SQL query often refers to concrete values in the database.
% For example, ``\texttt{\textcolor[RGB]{172,41,0}{SELECT} student.id \textcolor[RGB]{172,41,0}{FROM} student \textcolor[RGB]{172,41,0}{WHERE} student.name = "Bob"}'' refers the value ``\textit{Bob}'' in database.
% A SQL query often refers to concrete values in the database but there are often no records in the database to annotate.
% To obtain reliable SQL queries, {\tool} allows users to instantly populate the database with numerous diverse values (Fig.~\ref{fig:page2}). 
% This database will be used for retrieving values and validating the correctness of the annotated SQL queries.
% This component ensures that annotated SQL queries align accurately with both the schema and the concrete data values within the database.
SQL queries often reference specific values in the database. 
However, there are often no existing records in the database to reference during annotation. 
To address this limitation, {\tool} enables users to instantly create a sandbox database populated with diverse values (Fig.~\ref{fig:page2}). This sandbox database serves two purposes: (1) it provides a source for retrieving values, and (2) it allows for executing the annotated SQL queries to validate their correctness.


% Users can instantly populate a specified number of diverse records with a single click
To populate the database, users can specify the desired number of records and create them with a single click (Fig.~\ref{fig:page2} \circled{a}). 
{\tool} employs a rule-based method to randomly synthesize records based on data types, which currently supports eight data types, including \texttt{text}, \texttt{boolean}, \texttt{int}, \texttt{timestamp}, \texttt{float}, \texttt{double}, \texttt{decimal}, and \texttt{enum}.
We design random generation rules for each data type. 
For example, in Figure~\ref{fig:page2}, ``apt\_id'' is a text field, and {\tool} generates values by taking ``\textit{apt\_id}'' as the prefix and append a random UUID to it;
``\textit{status\_date}'' is a timestamp field, generating values like ``\textit{2022-05-01T06:04:32}'';
``\textit{is\_available}'' is a boolean field, so it only yields either ``\textit{True}'' or ``\textit{False}'' in the records. 


When generating random values, we also consider primary and foreign key relationships to be the constraints.
For example, ``\textit{View\_Unit\_Status.apt\_id}'' is a foreign key referencing another column, ``\textit{Apartments.apt\_id}'', so {\tool} reuses existing values from the referenced column.
The probability of generating repetitive records can be easily adjusted by users in a configuration file, with a default probability of $0.3$.
Users can navigate between different tables via a drop-down menu (Fig.~\ref{fig:page2} \circled{b}). 
Users can save current database records or upload existing ones in JSON format (Fig.~\ref{fig:page2} \circled{c}).

% While our populated data does not follow real world distribution, text-to-SQL models are not sensitive to theses values.
% These values are placeholders used for data annotation and used to validate the correctness of annotated SQL queries.
% Text-to-SQL models are trained to learn the SQL structures and not sensistive to these values.

While these rule-based synthetic values may not reflect real-world distributions, this does not affect the annotation process. 
These values primarily act as placeholders for annotators to ensure that the annotated SQL queries can be executed with the desired behavior. 
Furthermore, values are used for reference purposes and do not alter the query's underlying structure or meaning. Models are not directly trained on these values. 
% Thus, the synthetic values will not compromise utility of annotated text-to-SQL data.


% While our rule-based approach for value generation may differ from real-world databases, we find it suitable for annotating text-to-SQL datasets. Unlike entity-linking tasks that require meaningful database values, text-to-SQL tasks focus primarily on understanding the structure and execution of SQL queries, independent of specific content within the database. With adequate diverse records, the execution results are meaningful and can validate query correctness.

{\tool} distinguishes itself from existing methods that only incorporate dummy values in SQL queries without providing an executable environment.
All SQL queries created by {\tool} are associated with their execution results for users to better understand the query behaviors.






% \subsection{Query Annotator}

% \begin{figure*}[ht]
%   \centering
%   \includegraphics[width=\textwidth]{figure/page3_core.png}
%   \caption{Query Annotator (Part I, Page 3)}
%   \label{fig:page3_A}
% \end{figure*}

% With the prepared database containing numerous records under the provided schema, users can now create text-to-SQL datasets. 
% \textit{Query Annotator} includes the core functionality of {\tool}. We describe each feature within this page in detail.


\begin{figure*}[ht]
  \centering
  \includegraphics[width=\textwidth]{figure/page3_core.png}
  \caption{The user interface for data generation, error detection, and repair. \edit{Users can (a) generate a suggested SQL query, (b) check the query result, (c) read the step-by-step explanation in natural language, (d) generate the corresponding suggested NL question, (e) check similar gold data, (f) hover on each step to highlight the corresponding SQL component, NL question chunk, and sub-question, (g) build alignments among SQL, question, and steps, (i) identify and remove redundant text in the question, (j) update the question by emphasizing a certain step, (h) identify a misaligned step, and (k) collect annotated data.}}
  \label{fig:page3_A}
\end{figure*}

\subsection{\textbf{SQL Query Generation}}
\label{sec:sql_sampling}
Creating unbiased SQL queries is challenging, particularly when dealing with a new and complex database schema. 
To address this challenge, {\tool} provides a suggested SQL query (Fig.~\ref{fig:page3_A} \circled{a}) that is randomly sampled using SQL grammar and values in the populated database.
Specifically, {\tool} utilizes a pre-defined probabilistic context-free grammar (PCFG) tailored for SQL queries. This PCFG can be easily modified in a configuration file, as exemplified in Appendix~\ref{app:pcfg}.
While users can configure the grammar manually, {\tool} also supports automatically learning keyword probability distributions from an imported dataset.
Users can directly edit the suggested SQL query to meet specific needs and check the execution result via the ``EXECUTE'' button (Fig.~\ref{fig:page3_A} \circled{b}). 
% We provide more algorithmic details in the Appendix.

Compared to using LLMs to generate SQL queries directly, our PCFG-based approach offers more fine-grained control over query diversity and correctness. It mitigates issues such as bias or hallucination introduced by LLMs. This is the rationale behind our decision to first generate the SQL query and then translate it into natural language (NL).
An alternative approach could be generating the NL question first and then generating the SQL query. However, this method has limitations compared to ours. 
First, generating a large amount of diverse NL questions from scratch is challenging. The lack of fixed syntax in NL diminishes the control over data diversity. 
\edit{In contrast, our approach ensures data diversity by directly controlling SQL patterns.}
Second, generating the SQL query from NL essentially implies solving the text-to-SQL task. In this scenario, we can only generate SQL queries by models, which may hallucinate and introduce generation errors in SQL queries~\cite{ning_empirical_2023, tiis_sql}.
\edit{In contrast, the SQL queries generated by our approach are guaranteed to be syntactically correct.}

\subsection{\textbf{Natural Language Question Generation}}
Based on the SQL query, {\tool} provides a suggested NL question by translating the SQL query using GPT-4 Turbo\footnote{All the LLMs mentioned in this paper refer to GPT-4 Turbo (\url{https://platform.openai.com/docs/models/gpt-4-turbo-and-gpt-4}).} (Fig.~\ref{fig:page3_A} \circled{d}).
For more accurate translation, {\tool} employs retrieval-augmented generation  (RAG)~\cite{rag1, rag2}. 
It retrieves similar examples from a text-to-SQL data pool, which collects previously annotated data and 1,500 real-world text-to-SQL pairs. 
\edit{Unlike commonly used retrievers in RAG, such as dense retriever~\cite{dense_retrieve} and BM25~\cite{bm25}, we develop an AST-based retriever tailored for SQL queries.}
\edit{Specifically, {\tool} calculates similarity scores between SQL queries by measuring the tree edit distance between their abstract syntax trees, retrieving the top five examples with scores above $0.5$.}  
Using these examples, {\tool} performs few-shot learning to translate the annotation SQL query into an NL question.
\edit{Figure~\ref{fig:prompt_question} shows the prompt.}
Furthermore, users can view the top similar examples by clicking the ``SIMILAR EXAMPLES'' button (Fig.~\ref{fig:page3_A} \circled{e}). These real-world examples also help users better assess the quality of the LLM-suggested NL question.
The NL question is editable, allowing users to make any necessary adjustments.



\subsection{Error Detection \& Repair}

\subsubsection{SQL Step-by-step Explanation in NL}
To enhance user comprehension of SQL queries and detect potential errors, {\tool} explains the SQL query step by step in NL (Fig.~\ref{fig:page3_A} \circled{c}).
We reuse the rule-based explanation generation approach from STEPS~\cite{STEPS}, which parses the SQL query and translates each part of the query to an NL description based on templates.
{\tool} enhances this approach using an LLM in two ways. 
First, if a SQL query cannot be fully covered by the translation rules, {\tool} prompts the LLM to generate the step-by-step explanation via few-shot learning from rule-based explanation examples. 
Second, {\tool} prompts the LLM to paraphrase the generated explanations based on the schema and documentation for better fluency and naturalness. \edit{Figure~\ref{fig:prompt_explanation} shows the prompt.} To further improve the readability, {\tool} identifies and highlights columns, tables, and values in different colors in the explanation. 
Furthermore, for each step, {\tool} renders a corresponding sub-question on the left as users hover the mouse over this step (Fig.~\ref{fig:page3_A} \circled{f}). 
% To generate the sub-question, {\tool} synthesizes a simple SQL query that only involves this step, which the LLM then translates to the question. 
The sub-question is translated from a simple SQL query that only involves this step.

\subsubsection{Visual Correspondence among SQL query, NL question, and Step-by-step Explanation}
The step-by-step explanation serves as a bridge between the SQL query and the NL question. When users click the ``CHECK ALIGNMENT'' button (Fig.~\ref{fig:page3_A} \circled{g}), {\tool} creates a triple-linkage among these elements.
First, since the step-by-step explanation is grammar-based, there is a one-to-one mapping between SQL components and explanation steps.
Second, {\tool} employs an LLM to align the step-by-step explanation with the NL question. For each explanation step, the LLM pinpoints related substrings in the NL question, maintaining a one-to-many mapping.
When users hover over an explanation step (Fig.~\ref{fig:page3_A} \circled{f}), {\tool} highlights the corresponding SQL component and related question substrings in yellow.
This triple-linkage helps users mentally connect the SQL query and the suggested NL question, enhancing user understanding of the data and aiding in the detection of potential errors.

\subsubsection{Misalignment Detection \& Correction}

While {\tool} guarantees the syntactic correctness of SQL queries sampled by PCFG, the NL question generated by the LLM can include errors or ambiguity.
{\tool} proposes a novel interactive error detection and correction strategy by aligning the NL question with the SQL query through the step-by-step explanation.
\edit{Motivated by research~\cite{multi_agent_collaboration} showing that multi-agent collaboration enhances generation accuracy, {\tool} accomplishes this task through a two-step prompting.}
\edit{We include our prompt design in Figures~\ref{fig:prompt_alignment_analysis} and~\ref{fig:prompt_alignment_map}, with further details discussed in Appendix~\ref{app:prompt}.}
If any substring in the NL question fails to map to any step in the explanation, the substring will be highlighted in red (Fig.~\ref{fig:page3_A} \circled{i}), suggesting that this text may be irrelevant to this SQL query.
Users can focus on the red text and consider removing them.
Similarly, if a certain explanation step does not map to any partial text in the NL question, this step will be highlighted in red (Fig.~\ref{fig:page3_A} \circled{h}), indicating the content mentioned in this step may be missing in the NL question.
In this case, users can update the NL question by clicking the ``INJECT'' button on the corresponding step (Fig.~\ref{fig:page3_A} \circled{j}). Then, the LLM is prompted to update the current NL question by amplifying the content mentioned in this step.
\edit{Figure~\ref{fig:prompt_inject} shows the prompt.}


\subsubsection{Confidence Scoring}
To help users better assess the quality of annotated data, {\tool} offers a post-annotation analysis via the ``POST-ANNOTATION ANALYSIS'' button (Fig.~\ref{fig:page3_B} \circled{a}). Recent research has demonstrated that LLMs can determine semantic equivalence between SQL queries~\cite{llm_sql_equivalence} and generate accurate confidence scores through self-reflection~\cite{calibration_and_correctness, tian2023justaskcalibrationstrategies}.
Based on these findings, {\tool} prompts the LLM to provide a final report and score indicating the quality and correctness of the data.  \edit{The prompt used in this step is shown in Figure~\ref{fig:prompt_equivalence}.}
The score is averaged after multiple rounds of analysis to ensure stability. 
This score serves as a confidence level, directing users to focus more on checking data with lower scores, as these data may contain potential issues.


Based on the analysis provided by {\tool}, users can choose to accept or reject the data (Fig.~\ref{fig:page3_B} \circled{b}). Accepted data is collected in the right panel (Fig.~\ref{fig:page3_A} \circled{k}), where users can review and download the dataset at any time.


\begin{figure*}[ht]
  \centering
  \includegraphics[width=\textwidth]{figure/page3_2.png}
  \caption{The user interface for post-synthesis analysis \& automated annotation. \edit{Users can (a) generate an analysis report and scoring for annotating the current data pair, (b) accept or reject the current data pair, and (c) start automated data annotation without human intervention.}}
  \label{fig:page3_B}
\end{figure*}


\subsection{Automated Dataset Annotation}
While {\tool} enables users to annotate text-to-SQL datasets in an interactive manner, {\tool} also supports fully automated data annotation without humans in the loop (Fig.~\ref{fig:page3_B} \circled{c}). This is useful when users need a large amount of data without a perfect dataset quality (e.g., for fine-tuning LLMs).
Users can specify how many queries to synthesize and start by one click.
All generated data will be automatically collected on the right (Fig.~\ref{fig:page3_A} \circled{k}).

\subsection{Dataset Diversity Analysis}
% \todo{The description in this subsection is too handwavy. You should draw connection to the goal of generating diverse datasets and helping users to assess the diversity of the generated data}


To ensure diversity and eliminate potential biases in the annotated dataset, {\tool} allows users to monitor dataset composition and property distributions.
Users can upload their dataset via drag-and-drop (Figure~\ref{fig:page4} \circled{a}). {\tool} then renders various property distributions in pie charts, bar charts, or line charts, in terms of SQL structure, keyword, clause number, column usage, etc. (Figure~\ref{fig:page4} \circled{b}).
For example, users can monitor the number of referenced values in a bar chart.
If users find that SQL queries with a sufficient number of referenced values are underrepresented in the current dataset, they can adjust the PCFG probabilities to generate SQL queries with more values. And they can selectively accept only those queries that contain an adequate number of values.
In addition to ensuring diversity, this UI component generally improves human control during collaboration with the LLM, enabling users to better manage annotation pace and focus.


\begin{figure*}[ht]
  \centering
  \includegraphics[width=\textwidth]{figure/page4.png}
  \caption{The user interface for dataset diversity analysis. \edit{Users can (a) upload existing annotated text-to-SQL datasets and (b) monitor various property distributions.}}
  \label{fig:page4}
\end{figure*}


\section{Usage Scenario}

Bob is a data scientist at a rapidly growing technology company. 
Now, his task is to create a high-quality text-to-SQL dataset for training and evaluating the natural language (NL) interface of the company's recently updated database system.
Bob faces several challenges that make this task particularly daunting.
First, the company has just completed a major update to its database schema, introducing new tables and relationships to accommodate its expanding business needs. 
This update makes previous datasets obsolete and incompatible with the current schema. 
Consequently, it is impossible to accurately evaluate the performance of the NL interface based on the updated database.
Adding to the complexity, the schema now becomes highly complex, with numerous tables and reference relationships. Manually updating previous datasets to reflect these changes would be impractical. Bob realizes that he needs a solution that can handle this complexity efficiently and accurately.
Furthermore, Bob needs to create diverse, unbiased SQL queries and their corresponding NL questions at scale. Doing this manually would be prohibitively time-consuming and challenging, especially given the complexity of the new schema.
Recognizing these challenges, Bob decides to use the newly developed text-to-SQL data annotation tool, {\tool}, to streamline his workflow and ensure the creation of a controllable, high-quality dataset.


\textbf{Schema Comprehension.}
Bob begins by uploading a JSON file containing the company's updated database schema to {\tool}. As the schema loads onto the drag-and-drop canvas, Bob is immediately impressed by how {\tool} transforms the complex JSON structure into an intuitive visual representation. Tables appear as clearly defined boxes with columns listed inside, while relationships between tables are displayed as animated dashed lines.
The visual layout allows Bob to quickly grasp the overall structure of the database, saving him hours of time that would have been spent mentally parsing the JSON file.
Using the intuitive interface, Bob makes necessary adjustments. He double-clicks to edit table names, drags lines to establish reference relationships, and documents the meaning of an abbreviated column name. The ability to zoom in and out further helps Bob navigate the complex structure. 
As he works, Bob realizes the significant improvement in efficiency compared to editing the schema through the original schema definition file directly. What might have taken hours of painstaking work is now being accomplished in minutes, with much greater accuracy and confidence. Finally, Bob downloads the updated schema. He feels confident that this new well-documented schema will serve as a valuable foundation for future projects.

\textbf{Database Population.}
For more convenient data annotation, Bob populates the database with synthetic records with {\tool}. He specifies a need for 1,000 employee records. Upon clicking the SYNTHESIZE button, {\tool} instantly creates these records. Bob reviews the generated data and notices that the synthetic employee names look diverse. He proceeds to generate records for the other tables. Satisfied with the data generation, Bob downloads the database for future use and moves on to the next step.


\textbf{Data Annotation.}
Given the database, Bob is ready to annotate text-to-SQL data by creating SQL queries and their corresponding NL questions.
% He navigates to the \textit{Query Synthesizer} page to begin this process.
% ## Generating SQL Queries
Bob finds manually creating a SQL query from scratch challenging, so he decides to generate a random SQL query by {\tool}:

\begin{center}
\texttt{\textcolor[RGB]{172,41,0}{SELECT} Employees.name} \texttt{\textcolor[RGB]{172,41,0}{FROM} Employees} \\
\texttt{\textcolor[RGB]{172,41,0}{WHERE} Employees.department\_id = 5 \textcolor[RGB]{172,41,0}{AND} Employees.salary > 50000}
\end{center}



% ## Step-by-step explanation generation
Bob finds the SQL query reasonable. To confirm his understanding, Bob clicks the ANALYZE SQL button, and {\tool} shows a step-by-step analysis:

\begin{enumerate}
    \item \texttt{\textcolor[RGB]{172,41,0}{FROM} Employees} $\rightarrow$ \textit{Which data source should we care about?} 
          \\ \colorbox[rgb]{0.95,0.95,0.95}{In employees} 
        
    
    \item \texttt{\textcolor[RGB]{172,41,0}{WHERE} Employees.department\_id = 5} $\rightarrow$ \textit{Which department are employees from?} 
          \\ \colorbox[rgb]{0.95,0.95,0.95}{Filter employees from department 5} 
          
    
    \item \texttt{\textcolor[RGB]{172,41,0}{AND} Employees.salary > 50000} $\rightarrow$ \textit{What salary range do we care about?} 
          \\ \colorbox[rgb]{0.95,0.95,0.95}{Keep employees with salary exceeding \$50,000} 
          
    
    \item \texttt{\textcolor[RGB]{172,41,0}{SELECT} Employees.name} $\rightarrow$ \textit{What information should be returned?} 
          \\ \colorbox[rgb]{0.95,0.95,0.95}{Return the names of employees} 
        
\end{enumerate}

As Bob hovers over each step, a corresponding sub-question is rendered in the tooltip and the corresponding SQL component is highlighted.
{\tool} then generates a suggested NL question for this query:
\begin{center}
``\textit{\textbf{Who are the employees in the marketing department with a salary higher than \$50,000 and have been with the company for over 5 years?}}''
\end{center}
However, Bob notices that this question does not perfectly match the SQL query and decides to use the alignment feature to refine it.
Bob clicks the CHECK ALIGNMENT button, eager to see how well the generated question matches the SQL query.
He is immediately drawn to a phrase in the question highlighted in red: ``\textit{marketing department}'', suggesting there is no corresponding element in the SQL query.
Bob realizes this information is irrelevant and needs to be removed.
To better understand the quality of this suggested query, Bob hovers over the step-by-step explanation. To his surprise, {\tool} further visually corresponds each explanation step to sub-strings of the NL question through simultaneous highlighting.
He notices one explanation step, ``\textit{Filter employees from department 5}'', is highlighted in red.
This visual cue tells Bob that this step is not reflected in the current question.

Bob decides to address these issues one by one. 
First, he removes the irrelevant information by deleting the red-highlighted phrase ``\textit{marketing}'' and the unrelated condition ``\textit{and have been with the company for over 5 years}'' from the question.
Next, he turns his attention to the missing information about the department. He hovers over the red-highlighted explanation step, ``{\em Filter employees from department 5}'', and an INJECT button appears. Bob clicks this button and the current NL question is updated by incorporating this step.
The question now becomes:
\begin{center}
\textit{\textbf{Who are the employees in Department 5 with a salary higher than \$50,000?}}
\end{center}
Excited to see the results of his edits, Bob clicks the CHECK ALIGNMENT button again. 
This time, Bob notices that there is no red highlight in either the explanation or the NL question.
% Each step successfully maps to a SQL component and relevant sub-strings in the NL question.
As a final check, Bob hovers over the explanation steps. He watches with satisfaction as each step successfully maps to a SQL component and sub-strings in the NL question.


\begin{table*}[htbp]
\centering
\small
\begin{tabularx}{\textwidth}{>{\raggedright\arraybackslash}p{0.33\textwidth}>{\raggedright\arraybackslash}p{0.33\textwidth}>{\raggedright\arraybackslash}X}
\toprule
\textbf{Explanation Step} & \textbf{SQL Query Component} & \textbf{Question Sub-string} \\
\midrule
(1) In employees & \texttt{\textcolor{brown}{FROM} Employees} & \textit{... the employees ...} \\
(2) Filter employees from department 5 & \texttt{\textcolor{brown}{WHERE} Employees.department\_id = 5} & \textit{... in department 5 ...} \\
(3) Keep employees with salary exceeding \$50,000 & \texttt{\textcolor{brown}{AND} Employees.salary > 50000} & \textit{... with a salary higher than \$50,000...} \\
(4) Return the names of employees & \texttt{\textcolor{brown}{SELECT} Employees.name} & \textit{Who are the employees ...} \\
\bottomrule
\end{tabularx}
\label{tab:mapping}
\end{table*}


% ## Final Review and Acceptance
According to the visual alignment, Bob is confident that the NL question matches the SQL query. He further validates it by clicking the POST-SYNTHESIS button. 
{\tool} then reports the equivalence analysis in a short paragraph, along with a high confidence score of 98. Pleased with the result, Bob accepts this text-to-SQL instance and collects it to the right panel. He appreciates how the interactive alignment feature and intuitive triple-linkage visualization help him efficiently identify and correct misalignments with high confidence in the data's quality.

As Bob progresses, he periodically uses {\tool} to analyze the dataset composition to ensure he creates a diverse and balanced dataset. He notices that queries involving the newly added tables and relationships are underrepresented, so he adjusts the query generation parameters to increase their frequency.
By the end of the day, Bob creates a substantial, high-quality text-to-SQL dataset that accurately reflects the company's updated database schema. This new dataset will be invaluable for both training and evaluating their natural language interface, something that was not possible before due to the lack of relevant evaluation data.
Bob feels a sense of accomplishment. He successfully updates and documents a complex schema that would have been extremely time-consuming and error-prone to modify manually. He creates a dataset specific to the company's current database schema, including new tables, columns, and relationships. More importantly, the dataset provides a strong evaluation benchmark for the updated schema, allowing the team to accurately evaluate the performance of their NL interface.
The interactive nature of the tool allows Bob to leverage his domain knowledge while benefiting from automated generation and analysis features. He appreciates how the tool transforms a typically tedious and challenging process into an efficient and engaging one, ultimately contributing to the improvement of the company's data interaction capabilities.
\section{User Study}
To investigate the usability and effectiveness of {\tool}, we conducted a within-subjects user study with 12 participants. The study compared {\tool} with manual annotation and the use of a conversational AI assistant. 



\subsection{Participants}
We recruited 12 participants (4 female, 8 male) from Adobe. They worked in different roles including Machine Learning Engineers, Research Scientists, Data Scientists, and Product Managers.
Their works were directly or indirectly related to querying data in the database.
All of them had either Master's or PhD degrees.
Participants self-rated their proficiency in SQL (\textit{3 Beginner}, \textit{3 Basic}, 4 \textit{Intermediate}, \textit{2 Advanced}) and usage frequency of LLMs (6 \textit{Yearly}, 2 \textit{Monthly}, 2 \textit{Weekly}, 3 \textit{Daily}).

\subsection{Tasks}

% \noindent \textbf{Task 1: Text-to-SQL Creation.} 
We randomly sampled 9 schemas on the widely used text-to-SQL benchmark, Spider~\cite{spider}. 
% This pool includes 3 simple schemas, 3 medium schemas, and 3 complex schemas, categorized based on the number of entities and references in each schema.
We provided these schemas in JSON format, whose syntax was comprehensible to all participants.
Based on the schema, participants were asked to annotate text-to-SQL data while optimizing both the data quantity and quality.


% \noindent \textbf{Task 2: Schema Customization.} To assess schema customization performance, we created a pool of schema editing tasks. For each sampled schema, we manually created 30 tasks requiring edits over the existing schema, e.g., "\textit{Add a new column 'Founded' (date) to the 'airlines' table.}"
% Participants were expected to complete these tasks sequentially, as some tasks depended on the completion of previous ones. 
% We maintained a consistent distribution of task types (e.g., the number of "add column" tasks) across different schemas. 
% \todo{Does this task only require editing the schema or does it also require the regeneration of the text-to-SQL data? If the former, I don't find it relevant to the goal of this work.}


\subsection{Comparison Baselines}
To the best of our knowledge, no text-to-SQL data annotation tools were readily available for comparison at the time of the user study. Thus, we compared {\tool} with two commonly applied scenarios for text-to-SQL dataset annotation in the industry, manual annotating and using an AI assistant. 

\noindent \textbf{Manual.} We asked participants to manually review and customize the schema, create SQL queries, and write corresponding NL questions. They recorded the results in an Excel sheet. 

%\vspace{1.5mm}
\noindent \textbf{AI Assistant.} We gave participants access to using a state-of-the-art conversational AI assistant, ChatGPT (GPT-4). 
For example, participants could directly upload the entire schema file in their conversation with ChatGPT and request the generation of sufficient text-to-SQL data.
We did not impose any restrictions on how participants should use ChatGPT.

\subsection{Protocol}
Each study began with a demographic survey and study introduction. Participants then watched a 4-minute tutorial video of {\tool} and spent 3 minutes practicing to get familiar with it. Meanwhile, we collected quality feedback from users.

After participants were familiar with {\tool}, they proceeded to annotate in the assigned condition (i.e., Manual, AI assistant, {\tool}). Each task consisted of three 5-minute sessions, one for each condition. 
We randomized the order of assigned conditions to mitigate learning effects.
For each session, participants were provided with a database schema and asked to annotate as many text-to-SQL datasets as possible. We asked participants to focus on not only the quantity but also the quality of annotated data.

After each session, participants completed a post-task survey to rate their experience with the assigned condition. The survey included the System Usability Score (SUS)~\cite{sus} and NASA Task Load Index (TLX)~\cite{NASA-TLX} questionnaires, using 7-point Likert scales to assess their perceptions. At the end of the study, participants filled out a final survey sharing their experiences, opinions, and thoughts. The entire study took approximately 70 minutes.


\section{Results}

This section describes the results of our user study. 
We first present the analysis of user annotation performance in different conditions. We measure annotation speed and annotation quality.
Then, we report user perception of different conditions, e.g., their confidence level of annotated data and the perceived cognitive load. 

\subsection{Annotation Speed}
%A faster speed suggests a less expensive annotation.

\begin{table}[htb]
    \centering
    \caption{Number of Annotated Data (5 minutes)}
    \vspace{-2.5mm}
    \resizebox{0.8\linewidth}{!}{%
    \begin{tabular}{lcc}
        \toprule
              & \textbf{Completed Annotation} & \textbf{SD} \\
        \midrule
        Manual & 2.00 & 0.91 \\
        AI Assistant & 3.75 & 2.09 \\
        {\tool} & \textbf{8.75} & 2.74 \\
        \bottomrule
    \end{tabular}
    }
    \label{tab:annotation_speed}
\end{table}


Since each session has a fixed annotation time period, we use the average number of annotated tasks to represent the annotation speed. 
We compare the number of tasks completed across three conditions: Manual, AI assistant, and {\tool}. Table~\ref{tab:annotation_speed} presents the average annotation count completed within 5 minutes of the task session for each condition.
When using {\tool}, participants annotated the most tasks ($Mean = 8.75, SD = 2.74$), followed by using the AI assistant ($Mean = 3.75, SD = 2.09$), and manually annotating ($Mean = 2.00, SD = 0.91$). The ANOVA test shows that the mean differences are statistically significant ($p$-value = 1.96e-8).

The substantial improvement in task achievement with {\tool} suggests that {\tool} could enhance productivity in real-world applications.
Compared to ChatGPT, {\tool} provides a better explanation for participant engagement and validation.
P3 said, ``\textit{Even though ChatGPT can give me plenty of data, I need to manually check its output. I feel that connecting its output SQL to the schema is challenging in the limited time.}''
Furthermore, {\tool} serves as an interactive interface, allowing users to iteratively refine the data.
Both P6 and P9 appreciated this utility.
P9 said, ``\textit{It's easier and quicker to test and iterate on these [data] using the tool}'',


\subsection{Annotation Quality}
We define the quality of a text-to-SQL dataset into (1) \textbf{Correctness} (whether there is a syntax error, and whether the NL question matches the SQL query), (2) \textbf{Naturalness} (whether the NL question is natural enough as a human daily question), and (2) \textbf{Diversity} (whether the dataset has comprehensive coverage of different entities and query types, without any biases).

\subsubsection{\textbf{Correctness}}

To evaluate the correctness of participants' annotations, we manually review all data collected during the user study. We evaluate two types of errors in the annotated data. 
First, we look for SQL syntax errors or the misuse of entities and references in the database schema. 
We identify this type of error by executing the SQL query in a sandbox database that is adequately populated from the schema.
The error leads to execution failures or the return of an empty result.
Second, we evaluate the equivalence between the SQL query and the NL question. While the SQL might be syntactically correct on the schema, it may not accurately represent the intent of the NL question. In this case, we manually evaluate their equivalence.
Table~\ref{tab:correctness_annotation} shows the two types of error rates and the overall accuracy for each condition.

We observe different reasons for errors in manual annotation compared to using the AI assistant. During manual annotation, participants who were less proficient in SQL often made grammatical mistakes (29.24\%). However, since they tended to write simple SQL queries, the equivalence error was less (5.15\%).
The AI assistant, ChatGPT, rarely introduces syntax errors. However, it tends to generate more complex SQL queries (e.g., multiple JOINs) than manual annotation. Despite participant refinement, these queries often fail to match the complex schema due to hallucination, leading to SQL execution errors (18.73\%). Moreover, these complex queries pose greater challenges in maintaining equivalence with the NL question, resulting in the highest error rate of equivalence ((8.34\%).

Compared to manual annotation and using the AI assistant, using {\tool} achieves the highest accuracy (95.56\%). For SQL errors, queries sampled by {\tool} are guaranteed to be syntactically correct. Regarding NL-SQL equivalence, {\tool} aligns SQL and NL through step-by-step analysis, achieving the best equivalence after human refinement. However, we observe cases (4.44\%) where they are not fully equivalent, suggesting room for further improvement in NL generation accuracy.



\begin{table}[htb]
    \centering
    \caption{Correctness of Annotated Data}
    \vspace{-2.5mm}
    \resizebox{1\linewidth}{!}{%
    \begin{tabular}{lccc}
        \toprule
              & \textbf{SQL Error} & \textbf{Equivalence Error} & \textbf{Accuracy} \\
        \midrule
        Manual  & 29.24\%  & 5.15\% & 65.61\% \\
        AI Assistant   & 18.73\%   & 8.34\% & 72.93\% \\
        {\tool} & \textbf{0}   & \textbf{4.44\%} & \textbf{95.56\%} \\
        \bottomrule
    \end{tabular}
    }
    
    \label{tab:correctness_annotation}
\end{table}



\subsubsection{\textbf{Naturalness}}
In addition to correctness, the naturalness of the NL question is crucial for the quality of text-to-SQL data.
While an NL question may accurately match its SQL query, it might be verbose. In the real world, people tend to ask concise questions that follow certain natural language patterns. To evaluate naturalness, we first calculate the \textit{Flesch-Kincaid Readability Score}~\cite{flesch_score}, an automatic metric measuring text readability on a scale from 0 to 100.
To better assess naturalness, we further manually rate all annotated questions from 1 to 7 after masking the conditions.\footnote{{\tool} references synthetic values in a sandbox database, which can be easily identified by the human raters. We replace all these values with commonly used values for fair evaluation.}

\begin{table}[htb]
    \centering
        \centering
        \caption{Flesch-Kincaid Readability Score of Annotated Questions (0-100).}
        \vspace{-2.5mm}
        \resizebox{0.69\linewidth}{!}{%
        \begin{tabular}{lcc}
            \toprule
                  & \textbf{Flesch-Kincaid Score} & \textbf{SD} \\
            \midrule
            Manual & 76.94 & 13.15 \\
            AI Assistant & 56.14 & 18.62 \\
            {\tool} & 72.32 & 16.54 \\
            \bottomrule
        \end{tabular}
        }
        \label{tab:Flesch-Kincaid}
\end{table}

\begin{table}[htb]
    \hspace{0.06\textwidth}
        \centering
        \caption{Manual Rating of Naturalness of Annotated Questions (0-7).}
        \vspace{-2.5mm}
        \resizebox{0.65\linewidth}{!}{%
        \begin{tabular}{lcc}
            \toprule
                  & \textbf{Human Rating} & \textbf{SD} \\
            \midrule
            Manual & 6.25 & 0.52 \\
            AI Assistant & 6.02 & 0.58 \\
            {\tool} & 6.19 & 0.44 \\
            \bottomrule
        \end{tabular}
        }
        \label{tab:natural_rating}
\end{table}


\begin{table*}[htb]
    \centering
    \caption{SQL Query Component Diversity Analysis in Annotated Datasets.}
    \vspace{-2.5mm}
    \resizebox{0.75\linewidth}{!}{%
    \begin{tabular}{l cc|cc|cc|cc}
        \toprule
        & \multicolumn{2}{c}{\textbf{Clause}} & \multicolumn{2}{c}{\textbf{Table}} & \multicolumn{2}{c}{\textbf{Column}} & \multicolumn{2}{c}{\textbf{Value}} \\
        \cmidrule(lr){2-9}
        \textbf{Method} & Diversity & Mean & Diversity & Mean & Diversity & Mean & Diversity & Mean \\
        \midrule
        Manual       & 0.48 & 2.40 & 0.41 & 1.29 & 0.17 & 1.09 & 0.44 & 0.33 \\
        AI Assistant & 0.64 & 5.10 & 0.81 & 5.75 & \textbf{0.64} & 3.73 & 0.62 & 1.25 \\
        {\tool}      & \textbf{0.69} & 3.18 & \textbf{0.83} & 2.54 & 0.59 & 1.71 & \textbf{0.66} & 0.85 \\
        \bottomrule
    \end{tabular}
    }
    \label{tab:diversity}
\end{table*}


\begin{figure*}[htb]
  \centering
\includegraphics[width=0.85\textwidth]{figure/task1_cognitive.png}
  \caption{NASA Task Load Index Ratings of Text-to-SQL Data Annotation}
  \label{fig:cognitive1}
\end{figure*}

\begin{figure*}[htb]
  \centering
  \includegraphics[width=0.92\textwidth]{figure/sus1.png}
  \caption{SUS Scores of Text-to-SQL Data Annotation}
  \label{fig:sus1}
\end{figure*}



As shown in Table~\ref{tab:Flesch-Kincaid} and Table~\ref{tab:natural_rating}, the \textit{Flesch-Kincaid Readability Score} is consistent with human ratings. While manual annotation achieves the highest score (Flesch-Kincaid Score: 76.94, Human Rating: 6.25) as expected, the NL questions annotated through {\tool} achieve a comparable score (Flesch-Kincaid Score: 73.32, Human Rating: 6.19). We observe that using ChatGPT achieves the worst naturalness in questions (Flesch-Kincaid Score: 56.14, Human Rating: 6.02). Notably, ChatGPT-generated questions often include SQL keywords or follow SQL query patterns (e.g., \textit{Return all student names that are grouped by grades.}). 
Participants often accept ChatGPT's generation without modifications. While {\tool} is also built upon ChatGPT, it employs step-by-step analysis and in-context learning from real-world questions, thereby better incorporating real question patterns into the generated questions. 
Moreover, {\tool} offers a better interface and provides helpful suggestions for refining questions. Participants report that they are more willing to polish the LLM-generated data when using {\tool}.
P8 said, ``\textit{I really enjoy the UI in this tool. It suggests how to polish the data. And I really enjoy playing with the alignment feature to see what I can do with existing data.}''



\subsubsection{\textbf{Diversity}}
To evaluate the diversity and potential biases in the annotated dataset, we analyze the composition of participant-annotated data across four dimensions, including the number of clauses, columns, tables, and values involved. We measure the diversity of each dimension using Simpson's Diversity Index~\cite{simpson_diversity}, which is used to quantify the level of heterogeneity of a certain property.





Table~\ref{tab:diversity} shows the diversity and mean values for each method and dataset property. {\tool} demonstrates better diversity in the generated SQL for most dimensions, except for columns generated by the AI assistant. 
This is because ChatGPT tended to join excessive tables (mean = 5.75) in \texttt{FROM} clauses and include excessive columns (mean = 3.73) in  \texttt{SELECT} clauses.
In contrast, {\tool} learned the distribution from real-world datasets, resulting in a more reasonable distribution. For manual annotation, participants typically wrote simple SQL queries. For instance, they rarely used JOIN clauses, leading to low diversity.





\subsection{User Cognitive Load \& Usability Rating}

% Reducing cognitive load during data annotation is crucial since it directly make the annotation process cheaper.
% Figure~\ref{fig:cognitive1} presents participants’ ratings on the five cognitive load factors from the NASA TLX questionnaire~\cite{NASA-TLX}. The ANOVA test demonstrates that the mean differences are all statistically significant ($p$-value=6.7e-5, 6e-6, 1.9e-5, 1.5e-5, 1.3e-4 respectively).

Reducing cognitive load during data annotation is crucial, since it directly makes this process more cost-effective. Figure~\ref{fig:cognitive1} illustrates participants' ratings on the five cognitive load factors from the NASA TLX questionnaire. The ANOVA test reveals statistically significant ($\alpha = 0.001$) differences in means for all factors: Mental Demand ($p = 6.7 \times 10^{-5}$), Temporal Demand ($p = 6.0 \times 10^{-6}$), Performance ($p = 1.9 \times 10^{-5}$), Effort ($p = 1.5 \times 10^{-5}$), and Frustration ($p = 1.3 \times 10^{-4}$).



The result demonstrates that {\tool} can significantly reduce users' cognitive load compared to manual annotation and using ChatGPT.
P1 said, ``\textit{Generating NL2SQL data felt effortless with the help of the tool.}''.
We believe this is achieved by a smoother collaboration between the huamn and the LLM.
P1 also comprehensively discussed how {\tool} reduces users' cognitive load,
``\textit{The system allows me to generate a variety of SQLs without any cognitive efforts. This was great because I didn't have to think about SQL syntax, different queries one might ask about the dataset. I also like that the system can generate corresponding NL questions for each SQL. While the generated NL question wasn't always accurate, the system already provided me something which I could iterate on. This is almost like someone gives you a draft that you can just revise vs.~gives you an empty doc for you to start from scratch in writing.}''





% \todo{discuss the comparison with ChatGPT}

Figure~\ref{fig:sus1} displays SUS scores reported by participants, showing consistently positive feedback across all dimensions. Notably, no participants disagreed with any dimension. While a few participants expressed neutral opinions about SQL queries and NL questions suggested by {\tool}, most viewed them positively. All participants expressed high confidence in the quality and quantity of data annotated using {\tool}.
\vspace{-10pt}
\section{Limitation \& Discussion}\label{sec:discussion}

%\sssec{High system cost}.
%The high computational and storage cost associated with many features significantly limits their practicality. Features such as \textit{Activation Map} and \textit{Joint Token Probabilities}, while effective in certain scenarios, require substantial resources due to their high theoretical complexity. This can lead to increased inference latency, making them unsuitable for real-time or resource-constrained applications.

\sssec{Limited transferability}.
Based on the transferability experiments and insights from the RAG analysis, it is evident that the internal features of LLMs differ greatly across different scenarios. This lack of consistency results in poor cross-dataset generalizability. For example, features trained on CNNDM fail to perform well on HaluEval and vice versa, highlighting the dataset-specific nature of these features. This presents a significant challenge in developing universally applicable detection systems.

\sssec{Potential explainability}.
Despite the limitations, the proposed approach offers a promising avenue for explainability. Attention-based features, such as \textit{Lookback Ratio}, and logit-based features, such as \textit{Joint Token Probabilities}, provide interpretable insights into the model's reasoning process. These features allow researchers to better understand why certain outputs are classified as hallucinated, thereby enhancing trust in the system and opening up new possibilities for debugging and model refinement.
\section{Conclusion}


In this paper, we studied how Bayesian mechanism design can be adapted to address the challenges posed by hallucination-prone predictions generated by modern machine learning models. By introducing a novel Bayesian framework, we modeled these imperfect signals and rigorously characterized the structure of optimal mechanisms, extending classical results like those of \citet{myerson1981optimal} to settings where posterior distributions lack continuous densities. Our findings provide new insights into how sellers can navigate uncertainty and optimize revenue in environments shaped by unreliable predictions.

Our framework has three main implications.  First, it bridges the gap between traditional auction theory and contemporary machine learning applications, offering a pathway to integrate uncertain predictive signals into practical mechanism design. Second, our comparative analysis with an alternative model, the value-with-noise model, underscores the sensitivity of optimal mechanisms to the underlying assumptions about signal generation, thereby encouraging careful model selection in real-world implementations. Finally, in contrast with the now classical formulation in the algorithm with prediction literature which assumes that advice are either correct or adversarially chosen, our Bayesian framework captures the fact that when the prediction of a machine learning model is wrong, it is in fact ``randomly'' wrong: we believe that exploring this paradigm for other problem classes could design algorithms which are not tailored towards worst-case analyses. 

Despite these contributions, several exciting questions remain. A critical open question lies in analyzing non-direct mechanisms, where signals are not directly disclosed to buyers and strategic interactions become significantly more complex. Understanding the revenue implications (if any) and computational challenges in such settings would greatly add to the value of our framework. Additionally, our results assume that the hallucination probability is known to the seller; relaxing this assumption to consider uncertainty in hallucination probabilities could further align the model with real-world applications. 




%%
%% The acknowledgments section is defined using the "acks" environment
%% (and NOT an unnumbered section). This ensures the proper
%% identification of the section in the article metadata, and the
%% consistent spelling of the heading.
\begin{acks}
We thank the anonymous reviewers for their helpful and detailed feedback, as well as the time and care they dedicated to reviewing our work. We also express our gratitude to all the participants in the interview and user study for their valuable comments. This work was supported by Adobe during the first author's internship.
\end{acks}

%%
%% The next two lines define the bibliography style to be used, and
%% the bibliography file.
\bibliographystyle{ACM-Reference-Format}
\bibliography{reference}


\clearpage
\begin{appendices}

\section{Production Fault Trace}
\label{appendix:production-fault-trace}
The production fault trace was collected from an 8-GPU node pretrain cluster with 2880 GPUs over a period of 160 days. The trace includes details such as fault start time, fault end time, and the ID of the faulty node. \figref{fig:simulation:trace:timetrace} and \figref{fig:simulation:trace:cdf} provide a macro-level overview of the production fault trace. On average, the ratio of faulty 8-GPU nodes at any given time is $3.83\%$, with a p99 value of $7.22\%$.

\begin{figure}[h!t]
    \centering
    \begin{subfigure}[b]{0.23\textwidth}
        \centering
        \includegraphics[width=\textwidth]{figs/evaluation/fault_server_ratio.pdf}
        \caption{Fault Node Ratio Trace.}
        \label{fig:simulation:trace:timetrace}
    \end{subfigure}
    \hspace{2pt}
    \begin{subfigure}[b]{0.23\textwidth}
        \centering
        \includegraphics[width=\textwidth]{figs/evaluation/fault_server_cdf.pdf}
        \caption{Cumulative Distribution.}
        \label{fig:simulation:trace:cdf}
    \end{subfigure}
    \vspace{-2ex}
    \caption{Fault node trace in the production AI DC.}
    \label{fig:simulation:trace}
\end{figure}

Since most of failure events are GPU faults, we normalized the trace of 8-GPU nodes to generate 4-GPU nodes trace. Assuming that the fault rates of GPUs are i.i.d. with a fault probability of $p$ for each GPU, and considering that a node is deemed faulty if any GPU within it fails, the fault rate of an 8-GPU node is calculated as:  

\vspace{-1em}
$$
P_{fault}(8\text{-GPU}) = 1 - (1-p)^8 = 3.83\%.
$$  

From this, we derive $p = 0.49\%$. The fault rate for a 4-GPU node is then:  
$$
P_{fault}(4\text{-GPU}) = 1 - (1-p)^4 = 1.93\%.
$$  

The fault event of 4-GPU node is generate with Bayesian Equation, as:


\begin{align*}\label{eq:convert-trace}
& P_{fault}( \text{4-GPU} \mid  \text{8-GPU})\\ 
    &=\frac{P_{fault}(\text{8-GPU} \mid \text{4-GPU}) P_{fault}(\text{4-GPU})}{P_{fault}(\text{8-GPU})} \\ 
    & =  \frac{1 \times 1.93\%}{3.83\%} = 50.39\% \\
\end{align*}

Thus, whenever a fault occurs in an 8-GPU node in the original trace, each of the two corresponding 4-GPU nodes at the same location has a $50.39\%$ probability of fault. This method is used to convert the traces.

As node faults are i.i.d., the simulator linearly maps the fault trace to different network architectures.

\section{GPT-MoE Architecture}
\label{appendix:gpt-moe}
This model is a mixture-of-experts (MoE) model with the following configuration:

\para{Model Configuration:}
\begin{itemize}
    \item \textbf{Number of Layers:} 192
    \item \textbf{Inner Layer Dimension:} 49152
    \item \textbf{Embedding Dimension:} 12288
    \item \textbf{Hidden Dimension:} 12288
    \item \textbf{Vocabulary Size:} 64000
    \item \textbf{Number of Attention Heads:} 128
    \item \textbf{Maximum Sequence Length:} 2048
    \item \textbf{Number of Experts:} 8
    \item \textbf{MoE Layer Ratio:} 0.5
    \item \textbf{Top-K Experts:} 2
\end{itemize}

\para{Runtime Configuration:}
\begin{itemize}
    \item \textbf{Virtual Pipeline Parallelism:} 3
    \item \textbf{Micro Batch Size:} 1
    \item \textbf{Global Batch Size:} 1536
    \item \textbf{Max Sequence Length:} 2048
\end{itemize}




\section{Theoretical analysis of wasted GPU ratio for \sys}
\label{appendix:ft-anay}

The count of backup lines as $2K - 2$ will significantly influence the fault tolerance of \sys. We use the expectation of waste ratio caused by GPU failure and fragmentation problem to evaluate this design, the result is shown in \tabref{table:design:1.5ratio}.

For one single working server in the middle of line, the count of breakpoints $B$ on its two sides has the expectation as:

\vspace{-1em}
\begin{equation*}
E_B(\eta = 1,middle) = 2(P_s^K + P_s^{2K})
\end{equation*}

Where $P_s$ is the fail probability of GPU server, and $\eta$ is count of servers. The expectation of breakpoints count is:

Once the distance between one server and the tail of line is $\alpha < K$, it will connect to all servers between itself and the last one, so there will be no breakpoints on this side, and the expectation of breakpoints count is less than servers in the middle of line.
Then, for any server in the line topology:

\vspace{-1em}
$$
E_B(\eta = 1) \leq E_B(\eta = 1,middle) 
$$

When the distance between two servers is $\beta \geq K$, the breakpoints among them can be calculated as independent.
Once the distance $\beta < K$, as all servers in this range are connected to these two servers, there will be no breakpoints between them. So, the expectation is less than two independent servers. Then,



\vspace{-1em}
\begin{align*}
E_B(\eta =& 2) < E_B(\eta = 2, \beta \geq K) =  2E(\eta = 1)   \\ 
 E_B(\eta =& N_s) \leq N_s E_B(\eta = 1) 
\end{align*}

For a LLM job which require a ring communication size (TP .etc) as $N_t$, \sys   will cut the whole line topology into several sub lines with the length of $N_t/R$.
Once \sys is cutting a new sub line from the remaining servers in the line, 
all $N_t$ GPU will be wasted when one break point exist in the middle of this sub line required, shown in \fig{fig:subline-waste}. 
Then the expectation for waste GPU caused by one single break point is:

\vspace{-1em}
$$
E_W(B=1) = N_t R\cdot (1 - (N_t/R)^{-1} ) = R(N_t -R)
$$

\begin{figure}[h!t]
    \centering
    \includegraphics[width=0.8\linewidth]{figs/design/intra-topo/break-topo.drawio.pdf}
    \caption{Break point can cause server waste compare to ideal situation.}
    \vspace{-1em}
    \label{fig:subline-waste}
\end{figure}

As the influence between two break points only reduce the expectation of wasted GPUs, we can have this for $X$ break points:

\vspace{-1em}
\begin{equation*}
E_W(B = X) \leq XE_W(B=1) = XR(N_t-R)
\end{equation*}

So the expectation of wasted GPU for a servers cluster with $N_s$ GPU servers is:

\vspace{-1em}
\begin{align*}
E_W(\eta = N_s) &\leq \sum P(B=X ,\eta = N_s) \cdot X\cdot  E_W(B=1)\\
&= E_B(\eta = N_s)\cdot E_W(B=1)\\
&\leq  \lim_{P_s\rightarrow 0}2N_s\cdot R \cdot (N_t-R)P_s^K
\end{align*}



The final expectation of GPUs waste ratio is \eqref{eq:design:ratio}:

\begin{equation}
E_{WR}(\eta = N_s) = \frac{E_W(\eta = N_s)}{N_g} \leq 2(N_t-R)(P_s)^K
\label{eq:design:ratio}
\end{equation}

In our trace for a 160 days long pre-train job on 10K-GPU, the p99 failure rate for 8-card machines is 7\%. If a TP32 jobs is running on \sys, we can get the upper bond for waste ratio expectation for various configuration in \tabref{table:design:1.5ratio}.

\begin{table}[h!t]
\centering
\begin{tabular}{cccc}
    \toprule
        & $K=2$&$K=3$&$K=4$\\
    \midrule
     R=4& $7.35\%$ & $0.26\%$ & $9.00\times 10^{-4}$ \\
     R=8& $27.4\%$ & $1.92\%$ & $0.13\%$ \\
     \bottomrule
\end{tabular}
\caption{Upper bond for waste ratio expectation of GPU, where GPU failure rate is 0.875\% and X is 32}
\vspace{-2em}
\label{table:design:1.5ratio}
\end{table}

As shown in the table, for 4 GPU server ($R=4$) 3 bundles ($K = 3$) design, the additional waste of GPU is less than 0.26\%, while the waste ratio for $R=8,K=4$ is less than 0.13\%. This is sufficient for production clusters. 

\section{Orchestration For Fat-Tree}
\label{appendix:orch-algo}
In this section, we introduce the orchestration algorithm under Fat-Tree DCN in detail.

\para{Notations}
\label{appendix:orch-algo:notation}
To ensure rigorous mathematical reasoning, we introduce the following notations:

\begin{itemize}
    \item {
        $n$: number of nodes in the data-center.
    }
    \item {
        $K$: \docs{} bundle (see \S\ref{section:design:topology}).
    }
    \item {
        $S_{all}$: ordered set, represents all nodes numbered from 1 according to their physical connection order in DCN fabric. $|S_{all}|=n$.
    }
    \item {
        $S$: ordered subset, represents nodes, $\forall u \in S, u \in S_{all}$. Adjacent elements in $S$ are also adjacent from the perspective of the \SYS{} topology. 
    }
    \item{
        $E$: The set of edges across $S$, should be equal to $\{ (S_i, S_j) \mid 1 \leq i < j \leq n, j - i \leq K \} $, representing the connections between nodes, including both primary and backup links, and $O(|E|) = O(K|S|)$.
    }
    \item {
        $InfHBD=<S,E>$: the topology of \SYS{} as an undirected graph.
    }
    \item {
        $F$: faulty nodes.
    }
    \item {
        $HealthyHBD=<H,HE>$: healthy node subgraph where the set of healthy nodes $H = S - F$ and the edge set $HE = \{ (u, v) \mid u \in H \text{ and } v \in H \text{ and } (u, v) \in E \}$.
    }
    \item{
        $t$: TP size, number of GPUs in one TP Group.
    }
    \item{
        $r$: GPU ranks per node.
    }
    \item{
        $m=t/r$: number of nodes in a TP group.
    }
    % \item{
    %     $k$: number of rails in rail-optimized network.
    % }
    \item{
        $s$: job scale, number of GPUs required for the job.
    }
    \item{
        $d$: Aggregation-Switches Domain size. Number of nodes under coverage of one group of Aggregation-Switches.
    }
    \item{
        $n_{constrains}$: number of applied constraints in binary-search-based orchestration algorithm.
    }
    \item{
        $p$: number of nodes under each ToR.
    }
    \item{
        $l$: shortest sub-line length under fat-tree orchestration.
    }
    \item{
        $n_{maxsubline}=\lfloor \frac{nd}{p} \rfloor$: max number of sub-lines.
    }
    \item{
        $G_{deploy}=<S_{deploy},E_{deploy}>$: deployed topology. After applying the deployment strategy, the topology from the perspective of \SYS{} is described as follows: $S_{\text{deploy}}$ is an ordered set where adjacent elements correspond to adjacent nodes in \SYS{}, and $E_{\text{deploy}}$ represents the connections between nodes.
    }
    
\end{itemize}


% For the \SYS{} the orchestration algorithm in ideal conditions is relatively straightforward. The detailed steps of the algorithm are outlined in \algref{alg:orchestration-ideal}.

% Assume that the \SYS{}(with \docs{} direction $K$) is represented as an undirected graph $ \text{InfHBD} = \langle S, E \rangle $, where the ordered set of nodes $ S $ represents nodes. Adjacent elements in $S$ are also adjacent from the perspective of the \SYS{} topology. The set of edges $E$ should be equal to $\{ (S_i, S_j) \mid 1 \leq i < j \leq n, j - i \leq K \} $, representing the connections between nodes, including both primary and backup links, and $O(|E|) = O(K|S|)$. The set of faulty nodes is denoted as $ F \subseteq S $.

% The algorithm proceeds as follows:

% \begin{enumerate}
%     \item {\textbf{Extract the Healthy Node Subgraph:} First, extract the subgraph $\text{HealthyHBD} = \langle H, HE \rangle$ where the set of healthy nodes $H = S - F$ and the edge set $HE = \{ (u, v) \mid u \in H \text{ and } v \in H \text{ and } (u, v) \in E \}$. See \algref{alg:orchestration-ideal}.
%     }
%     \item {\textbf{Identify Connected Components:} Next, identify all connected components in the graph $\text{HealthyHBD}$. Faulty nodes may cause disconnections in the \SYS{} fabric, splitting the original cluster into multiple sub-HBDs. These sub-HBDs are the connected components, and TP Groups cannot span across these disconnected sub-HBDs. We use a simple Depth-First Search (DFS) algorithm here. See \algref{alg:dfs}.}
%     \item {\textbf{Generate Placement Scheme:} Given the excellent physical properties of the \SYS{}, TP Groups can be arranged sequentially within each connected component to generate placement scheme maximizing GPU utilization. See \algref{alg:orchestration-ideal}.
%     }
% \end{enumerate}

% Since each of the three steps involves traversing the entire graph's edges and nodes only once, 
The orchestration algorithm (\algref{alg:orchestration-ideal}) without considering DCN has the overall time complexity $3\cdot O(|H| + |HE|) = O(|S| + |E|) = O((K+1)|S|) = O(|S|)$.

% \begin{algorithm}[!h]
% \small
% \caption{Connected-Component-DFS}
% \label{alg:dfs}
% \SetAlgoNlRelativeSize{-1}
% \SetAlgoNlRelativeSize{1}
%  \KwIn{ $node$, $HealthyHBD$, $visited$}
%  \KwOut{ $component$}

%  Initialize $stack = [node]$ \;
%  Initialize $component = []$\;

% \While{ stack is not empty}
% {
%      $current = stack.pop()$\;
%     \If{$current$ not in $visited$}
%     {
%          Add $current$ to $visited$\;
%          Add $current$ to $component$\;
%         \For{ each neighbor in $HealthyHBD.neighbors(current)$}
%         {
%              $stack.push(neighbor)$\;
%         }
%     }
% }
        
% \KwRet{$component$}
% \end{algorithm}

\begin{algorithm}[!h]
\small
\caption{Orchestration-DCN-Free}
\label{alg:orchestration-ideal}
\SetAlgoNlRelativeSize{-1}
\SetAlgoNlRelativeSize{1}
\KwIn{$\text{InfHBD}=\langle S, E \rangle$, $F$, $m$}
\KwOut{ Placement scheme maximizing GPU utilization}

 Initialize $H = S - F$\;
 Initialize $HE = \{ (u, v) \mid u \in H \text{ and } v \in H \text{ and } (u, v) \in E \}$\;
 Create subgraph $HealthyHBD = \langle H, HE \rangle$\;
 Initialize $component\_list = []$\;
 Initialize $visited = \{\}$\;
 Initialize $placement\_scheme= \{\}$\;

\For{ each node $s$ in $H$}
{
    \uIf{ $s$ not in $visited$}
    {
         $component = Connected-Component-DFS(s, HealthyHBD, visited)$\;
         Add $component.sortedinHBD()$ to $component\_list$\;
    }
}
\For{ each $component$ in $component\_list$}
{
    \While{ $component.size()\geq m$}
    {
         Add $component.pop(m)$ to $placement\_scheme$\;
    }
}
        
 \KwRet{$placement\_scheme$}
 \end{algorithm}
 
% \subsection{Algorithms under Rail-Optimized Network}
% \label{appendix:orch-algo:rail-optimized}

% This subsection provides a detailed description of the orchestration algorithm for Rail-Optimized network.  

% The rail-optimized network topology is specifically designed for highly regular machine learning workload traffic patterns, making it a commonly used and effective architecture. As illustrated in \fig{fig:rail-topo}, Rail Switch $i$ connects to GPU $i$ in node, dividing the network into multiple rails. Let $r$ denote the GPU ranks per node, and $k$ the number of rails. In traditional rail-optimized networks, $k = r$, and a typical training strategy involves running TP $r$ within the single-node HBD, while DP operates between HBDs. Since in DP, GPUs only communicate with GPUs of the same rank in different TP groups, in other words, DP traffic is confined to the rail itself. Therefore, the Rail-Optimized topology perfectly meets this requirement.

% % \begin{figure}[!h]
% %     \centering
% %     \includegraphics[width=\linewidth]{figs/design/Orchestration/rail-optimized.drawio.pdf}
% %     \caption{Rail-Optimized Network: GPU ranks per node $r=4$, Number of rails $k=8$, Aggregation-Switches Domain size $d$, Number of Aggregation-Switches Domain $nd$, Node IDs from 1 to $nd\cdot d$. }
% %     \label{fig:rail-topo}
% % \end{figure}

% \para{Orchestration Constraints. }To minimize the cross-rail traffic which can lead to congestion and latency, the rail-optimized network introduces two key constraints for orchestration algorithms:


% \begin{itemize}
%     \item {
%         \textbf{Aggregation-Switches Domain Coverage Constraint. }
%         The coverage domian of a group of Aggregation-Switches is limited, meaning that TP groups spanning across Aggregation-Switches domains would result in cross-rail traffic, which should be avoided as much as possible.
%     }
%     \item {
%         \textbf{Node Rail State Constraint. }When$ k = r$, this constraint does not apply, as there is no cross-rail traffic.However, as HBDs extend beyond single nodes and the need for larger DP scales due to the expansion of LLM scale, scenarios with $k = p \cdot r$ may arise. This results in $p$ different node states within the data center, with each state occupying $r$ rails, and inter-state communication leads to cross-rail traffic. The specific form of this constraint depends on the deployment strategy.
%     }
% \end{itemize}

% \para{Deployment Strategy. }If the \SYS{} connections continue to follow the physical layout of nodes on the DCN Fabric, avoiding cross-rail traffic would require each TP Group to have an equal number of nodes from each state, making the algorithm to maximize GPU Utilization NP-Complete (see Appendix.\ref{appendix:np-hard-orchestration}). However, by altering the physical connection sequence of \SYS{}, this NP-Complete problem can be reduced to polynomial time. As shown in \fig{fig:parallel-line}, nodes of each state are arranged into $p$ parallel sub-lines, which are then connected end-to-end to form a single line. By restricting DP to operate within sub-lines, all DP traffic remains within the rails, effectively reducing the $k = p * r$ scenario to $k = r$. 

% % \begin{figure}[!h]
% %     \centering
% %     \includegraphics[width=\linewidth]{figs/design/Orchestration/parallel-line.drawio.pdf}
% %     \caption{The deployment strategy example with $p=4$ and Aggregation-Switches Domain size $K=8$. Node IDs from 1 to n are arranged according to their connection order in the DCN Fabric.}
% %     \label{fig:parallel-line}
% % \end{figure}

% \para{The binary search-based Orchestration algorithm.} Based on the above-mentioned constraints and the deployment strategy, we developed an orchestration algorithm that maximizes the number of constraints satisfied while meeting the job scale requirements. This is achieved using a binary search approach with the number of satisfied constraints as the variable. Both types of constraints essentially involve splitting the Line into sub-lines. Therefore, controlling the number of constraints translates to managing the number of sub-lines: fewer sub-lines mean longer sub-lines, leading to higher GPU Utilization. Since the Ideal orchestration algorithm with complexity $O(n)$ can be applied within sub-lines.

% \algref{alg:orchestration-fat-tree} is the main binary-search-based orchestration algorithm. It begins by generating the topology from the perspective of \SYS{} based on the hardware deployment strategy (\algref{alg:deployment-strategy}). Using the number of satisfied constraints as a variable, the algorithm performs a binary search to identify the placement scheme that maximizes the number of satisfied constraints while meeting the job scale requirements.  

% \algref{alg:placement-rail-optimized} calculates the placement scheme for a given number of constraints. It divides the topology into multiple ideal sub-lines and applies the ideal-case orchestration algorithm (\algref{alg:orchestration-ideal}) to each sub-line.  

% Since the time complexity of \algref{alg:orchestration-ideal} is $O(|S|)$, the complexity of \algref{alg:placement-rail-optimized} is 

% \begin{align*}
% &\sum_{i=1}^{n_{constraints}} O(|S_{subline}|) \\
% &= O(\sum_{i=1}^{n_{constraints}} |S_{subline}|) \\
% &= O(|S_{all}|) = O(n)
% \end{align*}

% Thus, the overall time complexity of \algref{alg:orchestration-rail-optimized} is $O(n \log n)$.

\begin{algorithm}[!h]
\small
\caption{Deployment-Strategy}
\label{alg:deployment-strategy}
\SetAlgoNlRelativeSize{-1}
\SetAlgoNlRelativeSize{1}
 \KwIn{Node ordered set $S$, \docs{} direction $K$, parallel factor $p$}
 \KwOut{Deployment topology $G_{deploy}=<S_{deploy},E_{deploy}>$}
 Initialize ordered set $S_{deploy}=[]$\;
 Initialize $l=\lfloor \frac{|S|}{p}\rfloor$\;
\For{$i$ in $0...p-1$}
{
    \For{$j$ in $0...l-1$}{
         Add $i+j\cdot p$ to $S_{deploy}$\;}
}
 Create $E_{deploy}=\{(S_{deploy}^i,S_{deploy}^j)|1\leq i\le j\leq |S_{deploy}|, j-i\leq K \}$\;
 \KwRet{$G_{deploy}=<S_{deploy},E_{deploy}>$}
\end{algorithm}


% \begin{algorithm}[!h]
% \small
% \caption{Placement-Rail-Optimized}
% \label{alg:placement-rail-optimized}
% \SetAlgoNlRelativeSize{-1}
% \SetAlgoNlRelativeSize{1}
%  \KwIn{Deployment topology $G_{deploy}=<S_{deploy},E_{deploy}>$, Number of applied constraints $n_{constraints}$, Faulty node $F$, Sub-line length $l$, Number of node in one TP group $m$}
%  \KwOut{Placement scheme}
%  Initialize $placement\_scheme=\{\}$\;
% \For{$i$ in $1..n_{constraints}$}
% {
%      $S_{subline}=S_{deploy}.pop(l)$\;
%      $E_{subline}=\{(u,v)\mid u\in S_{subline} \text{ and } v\in S_{subline} \text{ and } (u,v)\in E_{subline}\}$\;
%      $F_{subline}=F\cap S_{subline}$\;
%      $placement\_scheme=placement\_scheme\cup \text{Orchestration-Ideal}(<S_{subline},E_{subline}>, F_{subline}, m)$\;
% }
%  $E_{res}=\{(u,v)\mid u \in S_{deploy} \text{ and } v \in S_{deploy} \text{ and } (u,v) \in E_{deploy}\}$\;
%  $F_{res}=F\cap S_{deploy}$\;
%  $placement\_scheme=placement\_scheme\cup \text{Orchestration-Ideal}(<S_{deploy},E_{res}>, F_{res},m)$\;
%  \KwRet{$placement\_scheme$}
% \end{algorithm}


% \begin{algorithm}[!h]
% \small
% \caption{Orchestration-Rail-Optimized}
% \label{alg:orchestration-rail-optimized}
% \SetAlgoNlRelativeSize{-1}
% \SetAlgoNlRelativeSize{1}
%  \KwIn{Node ordered set $S$ (from 1 to n in DCN Fabric), GPU ranks per node $r$, Number of rails $k$, Faulty set $F$, TP size $t$, Job scale $s$ (number of GPUs required for the job), Aggregation-Switches Domain size $d$, \docs{} directions $K$.}
%  \KwOut{Placement scheme that satisfies job scale and minimizes cross-rail traffic.}
%  Initialize $p=k/r$, $m=t/r$, $n=|S|$, $l=\lfloor \frac{d}{p}\rfloor$\;
%  Create graph $G_{deploy}=<S_{deploy},E_{deploy}>=\text{Deployment-Strategy}(S,K,p)$\;
%  Initialize $high=\lfloor\frac{nd}{p}\rfloor$\;
%  Initialize $low=0$\;
%  Initialize $placement\_scheme=\{\}$\;
% \While{ $low \leq$ high}
% {
%      $mid=\lfloor \frac{low+high}{2} \rfloor$\;
%      $placement\_scheme=\text{Placement-Rail-Optimized}(G_{deploy},mid,F,l,m)$\;
%     \eIf {$|placement\_scheme|\cdot m\cdot r\ge s$}
%     {
%          $low=mid+1$\;
%     }
%     {
%          $high=mid-1$\;
%     }
% }
    
% \eIf{$|placement\_scheme|\cdot m\cdot r\ge s$}
% {
%   \KwRet {$placement\_scheme$}
% }
% {
%     \KwRet {None}
% }
% \end{algorithm}
  

Fat-Tree topology is another common data center topology. A typical training strategy for this topology aims to maximize the bandwidth utilization under ToR (Top of Rack) Switches. Using Meta's two-stage clos topology\cite{sigcomm2024meta} as a reference, it can be observed that there is an attempt to run CP under ToR.

\para{Deployment Strategy:} Assuming there are $p$ nodes under each ToR, nodes with the same index under each ToR are deployed along the same parallel sub-line, and the $p$ sub-lines are connected end-to-end, as shown in \fig{fig:fat-tree-topo}. The training strategy involves running CP $p$ across the sub-lines and running TP within them.

\para{Orchestration Constraints. }To maximize the utilization of ToR bandwidth and minimize cross-ToR traffic, the fat-tree topology introduces two constraints:

\begin{packeditemize}
    \item {
        \textbf{Aggregation-Switches Domain Constraint: }The coverage domian of a group of Aggregation Switches is limited, meaning that TP groups spanning across Aggregation Switches domains would result in cross-rail traffic, which should be avoided as much as possible.
    }
    \item {
        \textbf{TP Group Alignment Constraint: } A CP Group consists of TP Groups across parallel sub-lines. To keep CP traffic within the ToR, the TP Groups must be aligned. If a node fails under one ToR, all nodes under that ToR are considered failed, expanding the failure radius by a factor of $p$. 
    }
\end{packeditemize}

\para{Binary-Search-Based Orchestration Algorithm.} Based on the constraints and deployment strategy, we develop a binary search orchestration algorithm (see \algref{alg:orchestration-fat-tree}) that adjusts the number of satisfied constraints. The binary search first relaxes the TP Group alignment constraints within the Aggregation-Switches Domain and then relaxes the TP Group crossing constraints between Aggregation-Switch domains (see \algref{alg:placement-fat-tree}). This process is monotonic.


% \begin{figure}[!h]
%     \centering
%     \includegraphics[width=\linewidth]{figs/design/Orchestration/meta-topo.drawio.pdf}
%     \caption{Orchestration example for Fat-Tree Topology under single Aggregation-Switches Domain with $p=2$. Green indicates active node, red indicates faulty node and yellow indicates idle nodes}
%     \label{fig:meta-topo}
% \end{figure}


The time complexity of \algref{alg:orchestration-ideal} is $O(|S|)$, and the complexity of \algref{alg:placement-fat-tree} is 

$$\sum_{i=1}^{n_{subline}} O(|S_{subline}|) = O(\sum_{i=1}^{n_{subline}} |S_{subline}|) = O(|S_{all}|) = O(n)$$  

Thus, the overall time complexity of \algref{alg:orchestration-fat-tree} is $O(n \log n)$.

\begin{algorithm}[!h]
\small
\caption{Placement-Fat-Tree}
\label{alg:placement-fat-tree}
\SetAlgoNlRelativeSize{-1}
\SetAlgoNlRelativeSize{1}
 \KwIn{$G_{deploy}=<S_{deploy},E_{deploy}>$, $n_{constraints}$, $F$, $l$, $m$, $n_{maxsubline}$, $d$, $p$}
 \KwOut{Placement scheme}
 Initialize $placement\_scheme=\{\}$\;
 Initialize $n_{align}=max(0,n_{constraints}-n_{maxsubline})$, $n_{subline}=min(n_{maxsubline},n_{constraints})$\;
 
\For{$i$ in $0..n_{align}-1$}
{
    \For{$j$ in $1..d$}
    {
        $sid=i*d+j$\;
        \If{$sid \in F$}
        {
            $F\cup \{\lfloor \frac{sid-1}{p}\rfloor\cdot p+1..(\lfloor \frac{sid-1}{p}\rfloor+1)\cdot p \}$\;
        }
    }
}
\For{$i$ in $1..n_{subline}$}
{
     $S_{subline}=S_{deploy}.pop(l)$\;
     $E_{subline}=\{(u,v)\mid u\in S_{subline} \text{ and } v\in S_{subline} \text{ and } (u,v)\in E_{subline}\}$\;
     $F_{subline}=F\cap S_{subline}$\;
     $placement\_scheme=placement\_scheme\cup \text{Orchestration-Ideal}(<S_{subline},E_{subline}>, F_{subline}, m)$\;
}
 $E_{res}=\{(u,v)\mid u \in S_{deploy} \text{ and } v \in S_{deploy} \text{ and } (u,v) \in E_{deploy}\}$\;
 $F_{res}=F\cap S_{deploy}$\;
 $placement\_scheme=placement\_scheme\cup \text{Orchestration-Ideal}(<S_{deploy},E_{res}>, F_{res},m)$\;
 \KwRet{$placement\_scheme$}
\end{algorithm}

\begin{algorithm}[!h]
\small
\caption{Orchestration-Fat-Tree}
\label{alg:orchestration-fat-tree}
\SetAlgoNlRelativeSize{-1}
\SetAlgoNlRelativeSize{1}
 \KwIn{$S$, $r$, $p$, $F$, $t$, $s$, $d$, $K$.}
 \KwOut{Placement scheme that satisfies job scale and minimizes cross-rail traffic.}
 Initialize $m=t/r$, $n=|S|$, $l=\lfloor\frac{d}{p}\rfloor$\, $n_{domain}=\lfloor\frac{n}{d}\rfloor$, $n_{maxsubline}=\lfloor\frac{nd}{p}\rfloor$\;
 Create graph $G_{deploy}=<S_{deploy},E_{deploy}>=\text{Deployment-Strategy}(S,K,p)$\;
 Initialize $high=n_{domain}+n_{maxsubline}$\;
 Initialize $low=0$\;
 Initialize $placement\_scheme=\{\}$\;
\While{ $low \leq$ high}
{
     $mid=\lfloor \frac{low+high}{2} \rfloor$\;
     $placement\_scheme=\text{Placement-Fat-Tree}(G_{deploy},mid,F,l,m,n_{maxsubline},d,p)$\;
    \eIf {$|placement\_scheme|\cdot m\cdot r\ge s$}
    {
         $low=mid+1$\;
    }
    {
         $high=mid-1$\;
    }
}
    
\eIf{$|placement\_scheme|\cdot m\cdot r\ge s$}
{
    \KwRet {$placement\_scheme$}
}
{
    \KwRet {None}
}
\end{algorithm}





\section{Additional Simulation Results for Fault Resilience}
\label{appendix:wasted-GPUs-ratio}
This section presents additional simulation results related to \S\ref{sec:simulation:fault}. \figref{fig:simulation:wasted-trace} shows the variation of the GPU waste ratio over time under the production fault trace. \figref{fig:simulation:waste-cdf:gr4:supple} presents the CDF data for the GPU waste ratio. \figref{fig:simulation:model:wasted-gr4} illustrates the waste GPU ratio for different HBD architectures under various node failure rates, including the results for TP-8 to TP-64. \figref{fig:simulation:breakdown-duration-supple} shows the proportion of job-fault waiting time relative to total time for different job scales. All the aforementioned experiments include results for TP-8, TP-16, TP-32, and TP-64 configurations.








\begin{figure*}[h!t]
    \centering
    \begin{subfigure}[b]{0.23\linewidth}
        \centering
        \includegraphics[width=\linewidth]{figs/evaluation/fault_trace_based/frag_trace_tp8_gr4.pdf}
        \caption{TP-8.}
        \label{fig:simulation:wasted-trace:tp8-4gpu}
    \end{subfigure}
    \hspace{2pt}
    \begin{subfigure}[b]{0.23\linewidth}
        \centering
        \includegraphics[width=\linewidth]{figs/evaluation/fault_trace_based/frag_trace_tp16_gr4.pdf}
        \caption{TP-16.}
        \label{fig:simulation:wasted-trace:tp16-4gpu}
    \end{subfigure}
    \hspace{2pt}
    \begin{subfigure}[b]{0.23\linewidth}
        \centering
        \includegraphics[width=\linewidth]{figs/evaluation/fault_trace_based/frag_trace_tp32_gr4.pdf}
        \caption{TP-32.}
        \label{fig:simulation:wasted-trace:tp32-4gpu}
    \end{subfigure}
    \hspace{2pt}
    \begin{subfigure}[b]{0.23\linewidth}
        \centering
        \includegraphics[width=\linewidth]{figs/evaluation/fault_trace_based/frag_trace_tp64_gr4.pdf}
        \caption{TP-64.}
        \label{fig:simulation:wasted-trace:tp64-4gpu}
    \end{subfigure}

    \vspace{-1ex}
    \caption{GPU waste ratio over production fault trace, 4 GPU node.}
    \label{fig:simulation:wasted-trace}
\end{figure*}


\begin{figure*}[h!t]
    \centering
    \begin{subfigure}[b]{0.23\linewidth}
        \centering
        \includegraphics[width=\linewidth]{figs/evaluation/fault_trace_based/cdf_trace_waste_tp8_gr4.pdf}
        \caption{TP-8.}
        \label{fig:simulation:waste-cdf:tp8-gr4}
    \end{subfigure}
    \hspace{2pt}
    \begin{subfigure}[b]{0.23\linewidth}
        \centering
        \includegraphics[width=\linewidth]{figs/evaluation/fault_trace_based/cdf_trace_waste_tp16_gr4.pdf}
        \caption{TP-16.}
        \label{fig:simulation:waste-cdf:tp16-gr4}
    \end{subfigure}
    \hspace{2pt}
    \begin{subfigure}[b]{0.23\linewidth}
        \centering
        \includegraphics[width=\linewidth]{figs/evaluation/fault_trace_based/cdf_trace_waste_tp32_gr4.pdf}
        \caption{TP-32.}
        \label{fig:simulation:waste-cdf:tp32-gr4}
    \end{subfigure}
    \hspace{2pt}
    \begin{subfigure}[b]{0.23\linewidth}
        \centering
        \includegraphics[width=\linewidth]{figs/evaluation/fault_trace_based/cdf_trace_waste_tp64_gr4.pdf}
        \caption{TP-64.}
        \label{fig:simulation:waste-cdf:tp64-gr4}
    \end{subfigure}
    \vspace{-1ex}
    \caption{CDF of GPU waste ratio over production fault trace, 4 GPU node.}
    \label{fig:simulation:waste-cdf:gr4:supple}
\end{figure*}


\begin{figure*}[h!t]
    \centering
    \begin{subfigure}[b]{0.23\linewidth}
        \centering
        \includegraphics[width=\linewidth]{figs/evaluation/fault_model_based/frag_ratio_tp8_gr4.pdf}
        \caption{TP-8.}
        \label{fig:simulation:model:wasted:tp8}
    \end{subfigure}
    \hspace{2pt}
    \begin{subfigure}[b]{0.23\linewidth}
        \centering
        \includegraphics[width=\linewidth]{figs/evaluation/fault_model_based/frag_ratio_tp16_gr4.pdf}
        \caption{TP-16.}
        \label{fig:simulation:model:wasted:tp16}
    \end{subfigure}
    \hspace{2pt}
    \begin{subfigure}[b]{0.23\linewidth}
        \centering
        \includegraphics[width=\linewidth]{figs/evaluation/fault_model_based/frag_ratio_tp32_gr4.pdf}
        \caption{TP-32.}
        \label{fig:simulation:model:wasted:tp32}
    \end{subfigure}
    \hspace{2pt}
    \begin{subfigure}[b]{0.23\linewidth}
        \centering
        \includegraphics[width=\linewidth]{figs/evaluation/fault_model_based/frag_ratio_tp64_gr4.pdf}
        \caption{TP-64.}
        \label{fig:simulation:model:wasted:tp64}
    \end{subfigure}
    \vspace{-1ex}
    \caption{GPU wastes ratio with different GPU fault ratio, 4-GPU node.}
    \label{fig:simulation:model:wasted-gr4}
\end{figure*}



\begin{figure*}[h!t]
    \centering
    \begin{subfigure}[b]{0.23\linewidth}
        \centering
        \includegraphics[width=\linewidth]{figs/evaluation/fault_trace_based/breakdown_ratio_tp8_gr4.pdf}
        \caption{TP-8.}
        \label{fig:simulation:breakdown-duration:tp8-4gpu}
    \end{subfigure}
    \hspace{2pt}
    \begin{subfigure}[b]{0.23\linewidth}
        \centering
        \includegraphics[width=\linewidth]{figs/evaluation/fault_trace_based/breakdown_ratio_tp16_gr4.pdf}
        \caption{TP-16.}
        \label{fig:simulation:breakdown-duration:tp16-4gpu}
    \end{subfigure}
    \hspace{2pt}
    \begin{subfigure}[b]{0.23\linewidth}
        \centering
        \includegraphics[width=\linewidth]{figs/evaluation/fault_trace_based/breakdown_ratio_tp32_gr4.pdf}
        \caption{TP-32.}
        \label{fig:simulation:breakdown-duration:tp32-4gpu}
    \end{subfigure}
    \hspace{2pt}
    \begin{subfigure}[b]{0.23\linewidth}
        \centering
        \includegraphics[width=\linewidth]{figs/evaluation/fault_trace_based/breakdown_ratio_tp64_gr4.pdf}
        \caption{TP-64.}
        \label{fig:simulation:breakdown-duration:tp64-4gpu}
    \end{subfigure}
    \vspace{-1ex}
    \caption{Job fault-waiting duration with different levels of job-scale, 4 GPU node}
    \label{fig:simulation:breakdown-duration-supple}
\end{figure*}





\vspace{-12em}
\section{Detailed Cost and power consumption Analysis}
\label{appendix:cost}
In this section, \tabref{tab:eval:components} provides a detailed description of the quantity, cost, bandwidth, and power consumption of the interconnect components in various network architectures, including Google TPUv4~\cite{isca2023tpu}, NVIDIA GB200 NVL series~\cite{nvl72}, Alibaba HPN\cite{sigcomm2024hpn}, and \sys{}.


\begin{table*}[h!t] \small
    \centering
    \begin{tabular}{lllll}
    \toprule
    
    \textbf{Component} & \textbf{Quantity} & \textbf{Unit Cost (\$)}  & \textbf{Unit Bandwidth (GBps)} & \textbf{Unit Power (W)} \\

    \midrule
    \multicolumn{5}{c}{\textbf{Google TPUv4\cite{isca2023tpu} with 4096 GPU, bandwidth 300GBps/GPU}} \\
    
    \midrule
    OCS\cite{sigcomm2023lightwave} & 48 & 80000 & 6400 & 108 \\
    DAC Cable\cite{400G_DAC} & 5120 & 63.60 & 50 & 0.1 \\
    Optical Module\cite{400G_OPTICAL_MODULE} & 6144 & 360 & 50 & 12  \\
    Fiber\cite{FIBER}& 6144 & 6.80 & 50 & 0 \\
    
    \midrule
    \multicolumn{5}{c}{\textbf{NVIDIA GB200 NVL-36\cite{SEMIANALYSIS_GB200} with 36 GPU, bandwidth 900GBps/GPU}}\\
    \midrule
    NVLink Switch\cite{SEMIANALYSIS_Power} & 9 & 28000 & 3600 & 275 \\
    DAC Cable\cite{200G_DAC} & 2592 & 35.60 & 25 & 0.1 \\
    
    \midrule
    \multicolumn{5}{c}{\textbf{NVIDIA GB200 NVL-72\cite{nvl72}\cite{SEMIANALYSIS_GB200} with 72 GPU, bandwidth 900GBps/GPU}}\\
    \midrule
    NVLink Switch\cite{SEMIANALYSIS_Power} & 18 & 28000 & 3600 & 275 \\
    DAC Cable\cite{200G_DAC} & 5184 & 35.60 & 25 & 0.1 \\
    \midrule
    \multicolumn{5}{c}{\textbf{NVIDIA GB200 NVL-36x2\cite{SEMIANALYSIS_GB200} with 72 GPU, bandwidth 900GBps/GPU}}\\
    \midrule
    NVLink Switch\cite{SEMIANALYSIS_Power} & 36 & 28000 & 3600 &  275\\
    DAC Cable\cite{200G_DAC} & 6480 & 35.60 & 25 & 0.1 \\
    ACC Cable\cite{SEMIANALYSIS_Power} & 162 & 320 & 200 & 2.5 \\

    \midrule
    \multicolumn{5}{c}{\textbf{NVIDIA GB200 NVL-576\cite{SEMIANALYSIS_GB200} with 576 GPU, bandwidth 900GBps/GPU}}\\
    \midrule
    NVLink Switch\cite{SEMIANALYSIS_Power} & 432 & 28000 & 3600 & 275 \\
    DAC Cable\cite{200G_DAC} & 41472 & 35.60 & 25 & 0.1 \\
    Optical Module\cite{OSFPXD} & 4608 & 850 & 200 & 25 \\
    Fiber\cite{FIBER} & 4608 & 6.80 & 200 & 0 \\

    \midrule
    \multicolumn{5}{c}{\textbf{Alibaba HPN\cite{sigcomm2024hpn} with 16320 GPU, bandwidth 50GBps/GPU}}\\
    \midrule
    EPS\cite{51.2T_EPS} & 360 & 14960 & 6400 & 3145 \\
    DAC Cable\cite{200G_DAC} & 32640 & 35.60 & 25 & 0.1\\
    Optical Module\cite{400G_OPTICAL_MODULE} & 28800 & 360 & 50 & 12 \\
    Fiber\cite{FIBER} & 14400 & 6.80 & 50 & 0 \\

    \midrule
    \multicolumn{5}{c}{\textbf{\SYS{}($K=2$)  with 4 GPU, bandwidth 800GBps/GPU}}\\
    \midrule
    DAC Cable\cite{1.6T_DAC}& 4 & 199.60 & 200 & 0.1\\
    dOCS Module & 16 & 600 & 100 & 12 \\
    Fiber\cite{FIBER} & 16 & 6.80 & 100 & 0 \\

    \midrule
    \multicolumn{5}{c}{\textbf{\SYS{}($K=3$)  with 4 GPU, bandwidth 800GBps/GPU}}\\
    \midrule
    DAC Cable\cite{1.6T_DAC} & 2 & 199.60 & 200 & 0.1\\
    dOCS Module & 24 & 600 & 100 & 12 \\
    Fiber\cite{FIBER} & 24 & 6.80 & 100 & 0 \\
    \bottomrule
    \end{tabular}
    \caption{Interconnect cost and power consumption of components used in different network architectures.}
    \label{tab:eval:components}
\end{table*}


\end{appendices}







\end{document}
\endinput
%%
%% End of file `sample-sigconf-authordraft.tex'.
