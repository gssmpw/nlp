

% \section{Rubrics of Automatic Evaluation}


\section{Data Statistics}
\label{app:data_statistics_appendix}
\paragraph{Statistics of generated tags.} As we stated in limitations section, we provide the statistics of generate special tag tokens in Table~\ref{app:tag_statistics}.
We find out that most of the <<Role>>, <<Content>>, <<Task>> tokens are annotated in the instances.
Compared to thoses tokens, <<Action>>, <<Style>>, <<Background>>, and <<Format>> depends on the user instructions to be generated.
However, <<Tool>> tokens have shown absolutely low portion to be generated.
We thus want to suggest that properly choosing the public or your own dataset seems to ensure the <<Tool>> tag usages such as selecting searching protocols or function calls. 
\begin{table}[h]
{\resizebox{\columnwidth}{!}{
\begin{tabular}{lccc}
\toprule
Tags       & LLaMA-3.1-8B-instruct & Qwen2.5-14b-instruct & Phi-4   \\ \midrule
Role       & 576,341               & 753,579              & 745,751 \\ 
Content    & 580,231               & 739,892              & 743,311 \\ 
Task       & 579,558               & 765,331              & 735,298 \\ 
Action     & 495,301               & 382,358              & 662,589 \\ 
Style      & 283,579               & 598,553              & 603,918 \\ 
Background & 293,791               & 539,757              & 553,791 \\ 
Tool       & 10,238                & 132,038              & 90,989  \\ 
Format     & 327,909               & 401,593              & 538,973 \\ \bottomrule
\end{tabular}}}
\caption{Statistics of generated tags using \textsc{SysGen} pipeline.}
\label{app:tag_statistics}
\end{table}

\begin{table*}[t]
\centering
{\resizebox{\textwidth}{!}{
\begin{tabular}{lccccc}
\toprule
\multicolumn{1}{c}{Dataset} & \# of instances & Avg. Query Length & Avg. Answer Length & Containing System Message & Covering Domains\\ \midrule
Capybara  & 41,301          & 300.24 & 1423.28 & \xmark & reasoning, logic, subjects, conversations, pop-culture, STEM \\ 
Airoboros & 59,277          & 507.26 & 1110.62 & simple system message & mathematics, MATHJSON, character's descriptions \\ 
OrcaMath  & 200,035         & 238.87 & 878.43 & \xmark & school mathematics, math word problems \\ 
Magicoder & 111,183         & 652.53 & 1552.41 & \xmark & code solution \\  
MetaMath  & 395,000         & 213.53 & 498.24 & \xmark & mathematics \\  \bottomrule
\end{tabular}}}
\caption{Data statistics of SFT datasets. We provide the average length of query and answer, the presence of system messages, and covering domains.}
\label{tab:data_statistics_appendix}
% \vspace{-0.3cm}
\end{table*}


\paragraph{Statistics of original SFT datasets.}
In Table~\ref{tab:data_statistics_appendix}, we observe that most widely used public datasets either lack a system message entirely or include only a simple one, such as "You are a helpful AI assistant.".
The publicly available data mostly cover mathematics, code problems following some reasoning and logical ones.




\section{Experimental Details}
\label{app:experimental_details}
\paragraph{Computing Resources}
We use 4x8 NVIDIA H100 Tensor Core GPU with 80GB memory to train the open-source models.
We use Deepspeed stage 3~\citep{rajbhandari2020zero} to implement multi-GPU settings and FlashAttention~\citep{dao2022flashattention} for efficient training.
Our code is written in PyTorch~\citep{paszke2019pytorch} and HuggingFace~\citep{wolf2019huggingface}.


\paragraph{Integrating system roles in models that do not support them.}
\label{app:system_role_support}
Through our experiments, we find out that the Gemma-2-9b-it~\citep{team2024gemma} model does not inherently support the system role.
To address this limitation during data generation and training, we modified the chat template in the configuration of tokenization to remove restrictions on the system role.
Interestingly, despite the lack of native support, our findings show that \textsc{SysGen} data can still be utilized effectively to incorporate a system role into these models.

\section{Qualitative analysis of generated instances}
\label{app:qualitative_analysis}

\begin{table*}
    \scriptsize
    \centering
    \begin{NiceTabular}{@{}l@{\hskip4pt}p{0.92\textwidth}@{}}
    \CodeBefore
\cellcolor{gray!20}{1-1,1-2}
\cellcolor{gray!20}{5-1,5-2}
\cellcolor{gray!20}{9-1,9-2}
\Body
    \toprule
    \multicolumn{2}{c}{\textbf{Case 1: Under-prediction of SAR}}\\
    \midrule
    \textbf{Groundtruth} & \texttt{
    [...] 
Diaz
started
his
{\color{red}political career}
as a
{\color{red}member}
of
{\color{red}the Sangguniang Bayan}
{\color{red}(municipal council)}
of
{\color{red}Santa Cruz}
in
{\color{red}1978}.
He
later
became
{\color{red}the Vice Mayor}
of
{\color{red}Santa Cruz}
in
{\color{red}1980}
and
was elected
as
{\color{red}the town's Mayor}
in
{\color{red}1988}.
[...]
    }\\
    \hdashline
    \textbf{Likelihood} & \texttt{
    [...]
\adjustbox{bgcolor={red!61.3}}{\strut Diaz}
\adjustbox{bgcolor={red!95.4}}{\strut started}
\adjustbox{bgcolor={red!0.4}}{\strut his}
\adjustbox{bgcolor={red!7.2}}{\strut political career}
\adjustbox{bgcolor={red!35.3}}{\strut as a}
\adjustbox{bgcolor={red!67.7}}{\strut member}
\adjustbox{bgcolor={red!0.6}}{\strut of}
\adjustbox{bgcolor={red!21.2}}{\strut the Sangguniang Bayan}
\adjustbox{bgcolor={red!24.8}}{\strut (municipal council)}
\adjustbox{bgcolor={red!5.2}}{\strut of}
\adjustbox{bgcolor={red!23.9}}{\strut Santa Cruz}
\adjustbox{bgcolor={red!71.2}}{\strut in}
\adjustbox{bgcolor={red!81.2}}{\strut 1978}
\adjustbox{bgcolor={red!28.5}}{\strut .}
\adjustbox{bgcolor={red!15.2}}{\strut He}
\adjustbox{bgcolor={red!90.6}}{\strut later}
\adjustbox{bgcolor={red!89.7}}{\strut became}
\adjustbox{bgcolor={red!48.9}}{\strut the Vice Mayor}
\adjustbox{bgcolor={red!6.6}}{\strut of}
\adjustbox{bgcolor={red!14.3}}{\strut Santa Cruz}
\adjustbox{bgcolor={red!62.6}}{\strut in}
\adjustbox{bgcolor={red!45.6}}{\strut 1980}
\adjustbox{bgcolor={red!31.4}}{\strut and}
\adjustbox{bgcolor={red!61.7}}{\strut was elected}
\adjustbox{bgcolor={red!42.9}}{\strut as}
\adjustbox{bgcolor={red!61.5}}{\strut the town's Mayor}
\adjustbox{bgcolor={red!2.0}}{\strut in}
\adjustbox{bgcolor={red!37.2}}{\strut 1988}
\adjustbox{bgcolor={red!99.5}}{\strut .}
[...]
    }\\
    \hdashline
    \textbf{SAR} & \texttt{
    [...]
\adjustbox{bgcolor={red!64.0}}{\strut Diaz}
\adjustbox{bgcolor={red!31.4}}{\strut started}
\adjustbox{bgcolor={red!0.1}}{\strut his}
\adjustbox{bgcolor={red!2.4}}{\strut political career}
\adjustbox{bgcolor={red!6.3}}{\strut as a}
\adjustbox{bgcolor={red!16.0}}{\strut member}
\adjustbox{bgcolor={red!0.1}}{\strut of}
\adjustbox{bgcolor={red!13.8}}{\strut the Sangguniang Bayan}
\adjustbox{bgcolor={red!6.7}}{\strut (municipal council)}
\adjustbox{bgcolor={red!0.6}}{\strut of}
\adjustbox{bgcolor={red!7.0}}{\strut Santa Cruz}
\adjustbox{bgcolor={red!23.7}}{\strut in}
\adjustbox{bgcolor={red!74.5}}{\strut 1978}
\adjustbox{bgcolor={red!3.6}}{\strut .}
\adjustbox{bgcolor={red!8.3}}{\strut He}
\adjustbox{bgcolor={red!53.3}}{\strut later}
\adjustbox{bgcolor={red!35.6}}{\strut became}
\adjustbox{bgcolor={red!28.4}}{\strut the Vice Mayor}
\adjustbox{bgcolor={red!1.3}}{\strut of}
\adjustbox{bgcolor={red!4.1}}{\strut Santa Cruz}
\adjustbox{bgcolor={red!19.8}}{\strut in}
\adjustbox{bgcolor={red!38.4}}{\strut 1980}
\adjustbox{bgcolor={red!7.1}}{\strut and}
\adjustbox{bgcolor={red!23.8}}{\strut was elected}
\adjustbox{bgcolor={red!14.5}}{\strut as}
\adjustbox{bgcolor={red!23.2}}{\strut the town's Mayor}
\adjustbox{bgcolor={red!0.4}}{\strut in}
\adjustbox{bgcolor={red!29.5}}{\strut 1988}
\adjustbox{bgcolor={red!53.6}}{\strut .}
[...]
    }\\
\midrule
\multicolumn{2}{c}{\textbf{Case 2: The type-filter of Focus and the limitations of uncertainty scores}}\\
    \midrule
\textbf{Groundtruth} & \texttt{
Taral Hicks
is
an
American
actress
and
singer,
born
on
September 21, 1974,
in
{\color{red}The Bronx, New York}.
[...]
She
later
{\color{red}transitioned}
to
{\color{red}acting},
appearing in
films
such as
"A Bronx Tale"
(1993),
"Just Cause"
(1995),
and
"Belly"
(1998).
[...]
}\\
\hdashline
\textbf{Likelihood} & \texttt{
\adjustbox{bgcolor={red!27.5}}{\strut Taral Hicks}
\adjustbox{bgcolor={red!64.8}}{\strut is}
\adjustbox{bgcolor={red!43.8}}{\strut an}
\adjustbox{bgcolor={red!25.7}}{\strut American}
\adjustbox{bgcolor={red!98.8}}{\strut actress}
\adjustbox{bgcolor={red!49.1}}{\strut and}
\adjustbox{bgcolor={red!26.2}}{\strut singer}
\adjustbox{bgcolor={red!91.9}}{\strut ,}
\adjustbox{bgcolor={red!76.3}}{\strut born}
\adjustbox{bgcolor={red!56.8}}{\strut on}
\adjustbox{bgcolor={red!34.6}}{\strut September 21, 1974}
\adjustbox{bgcolor={red!71.4}}{\strut ,}
\adjustbox{bgcolor={red!9.1}}{\strut in}
\adjustbox{bgcolor={red!21.4}}{\strut The Bronx, New York}
\adjustbox{bgcolor={red!65.0}}{\strut .}
[...]
\adjustbox{bgcolor={red!75.1}}{\strut She}
\adjustbox{bgcolor={red!76.5}}{\strut later}
\adjustbox{bgcolor={red!13.8}}{\strut transitioned}
\adjustbox{bgcolor={red!86.7}}{\strut to}
\adjustbox{bgcolor={red!1.1}}{\strut acting}
\adjustbox{bgcolor={red!15.8}}{\strut ,}
\adjustbox{bgcolor={red!8.7}}{\strut appearing in}
\adjustbox{bgcolor={red!45.0}}{\strut films}
\adjustbox{bgcolor={red!10.6}}{\strut such as}
\adjustbox{bgcolor={red!48.7}}{\strut "A Bronx Tale"}
\adjustbox{bgcolor={red!18.1}}{\strut (1993),}
\adjustbox{bgcolor={red!47.3}}{\strut "Just Cause"}
\adjustbox{bgcolor={red!2.6}}{\strut (1995),}
\adjustbox{bgcolor={red!4.8}}{\strut and}
\adjustbox{bgcolor={red!46.6}}{\strut "Belly"}
\adjustbox{bgcolor={red!1.7}}{\strut (1998).}
[...]
}\\
\hdashline
\textbf{Focus} & \texttt{
\adjustbox{bgcolor={red!5.3}}{\strut Taral Hicks}
\adjustbox{bgcolor={red!0.0}}{\strut is}
\adjustbox{bgcolor={red!0.0}}{\strut an}
\adjustbox{bgcolor={red!23.5}}{\strut American}
\adjustbox{bgcolor={red!40.9}}{\strut actress}
\adjustbox{bgcolor={red!0.0}}{\strut and}
\adjustbox{bgcolor={red!30.4}}{\strut singer}
\adjustbox{bgcolor={red!0.0}}{\strut ,}
\adjustbox{bgcolor={red!0.0}}{\strut born}
\adjustbox{bgcolor={red!0.0}}{\strut on}
\adjustbox{bgcolor={red!31.9}}{\strut September 21, 1974}
\adjustbox{bgcolor={red!0.0}}{\strut ,}
\adjustbox{bgcolor={red!0.0}}{\strut in}
\adjustbox{bgcolor={red!21.0}}{\strut The Bronx, New York}
\adjustbox{bgcolor={red!0.0}}{\strut .}
[...]
\adjustbox{bgcolor={red!0.0}}{\strut She}
\adjustbox{bgcolor={red!0.0}}{\strut later}
\adjustbox{bgcolor={red!0.0}}{\strut transitioned}
\adjustbox{bgcolor={red!0.0}}{\strut to}
\adjustbox{bgcolor={red!32.2}}{\strut acting}
\adjustbox{bgcolor={red!0.0}}{\strut ,}
\adjustbox{bgcolor={red!0.0}}{\strut appearing in}
\adjustbox{bgcolor={red!33.2}}{\strut films}
\adjustbox{bgcolor={red!0.0}}{\strut such as}
\adjustbox{bgcolor={red!43.6}}{\strut "A Bronx Tale"}
\adjustbox{bgcolor={red!25.3}}{\strut (1993),}
\adjustbox{bgcolor={red!42.5}}{\strut "Just Cause"}
\adjustbox{bgcolor={red!30.1}}{\strut (1995),}
\adjustbox{bgcolor={red!0.0}}{\strut and}
\adjustbox{bgcolor={red!44.2}}{\strut "Belly"}
\adjustbox{bgcolor={red!30.1}}{\strut (1998).}
[...]
}\\
\midrule
\multicolumn{2}{c}{\textbf{Case 3: Uncertainty propagation of Focus}}\\
    \midrule
    \textbf{Groundtruth} & \texttt{
    [...]
Fernandinho
began
his
professional career
with
{\color{red}Atletico Paranaense}
in
{\color{red}Brazil}
before
{\color{red}moving}
to
{\color{red}Ukrainian club}
{\color{red}Shakhtar Donetsk}
in
{\color{red}2005}.
[...]
He
is known
for
his
{\color{red}physicality},
{\color{red}tackling ability},
and
{\color{red}passing range},
and
is
{\color{red}widely regarded}
as
{\color{red}one of the best}
{\color{red}defensive midfielders}
in
{\color{red}the world}.
    }\\
    \hdashline
    \textbf{Likelihood} & \texttt{
    [...]
\adjustbox{bgcolor={red!23.6}}{\strut Fernandinho}
\adjustbox{bgcolor={red!78.3}}{\strut began}
\adjustbox{bgcolor={red!3.3}}{\strut his}
\adjustbox{bgcolor={red!47.7}}{\strut professional career}
\adjustbox{bgcolor={red!60.5}}{\strut with}
\adjustbox{bgcolor={red!25.3}}{\strut Atletico Paranaense}
\adjustbox{bgcolor={red!42.1}}{\strut in}
\adjustbox{bgcolor={red!97.7}}{\strut Brazil}
\adjustbox{bgcolor={red!91.5}}{\strut before}
\adjustbox{bgcolor={red!40.5}}{\strut moving}
\adjustbox{bgcolor={red!9.1}}{\strut to}
\adjustbox{bgcolor={red!65.8}}{\strut Ukrainian club}
\adjustbox{bgcolor={red!3.2}}{\strut Shakhtar Donetsk}
\adjustbox{bgcolor={red!20.1}}{\strut in}
\adjustbox{bgcolor={red!1.5}}{\strut 2005}
\adjustbox{bgcolor={red!19.5}}{\strut .}
[...]
\adjustbox{bgcolor={red!34.9}}{\strut He}
\adjustbox{bgcolor={red!28.2}}{\strut is known}
\adjustbox{bgcolor={red!0.0}}{\strut for}
\adjustbox{bgcolor={red!0.0}}{\strut his}
\adjustbox{bgcolor={red!61.0}}{\strut physicality}
\adjustbox{bgcolor={red!8.6}}{\strut ,}
\adjustbox{bgcolor={red!77.0}}{\strut tackling ability}
\adjustbox{bgcolor={red!0.6}}{\strut ,}
\adjustbox{bgcolor={red!12.7}}{\strut and}
\adjustbox{bgcolor={red!53.7}}{\strut passing range}
\adjustbox{bgcolor={red!22.5}}{\strut ,}
\adjustbox{bgcolor={red!34.9}}{\strut and}
\adjustbox{bgcolor={red!61.7}}{\strut is}
\adjustbox{bgcolor={red!50.0}}{\strut widely regarded}
\adjustbox{bgcolor={red!0.0}}{\strut as}
\adjustbox{bgcolor={red!0.8}}{\strut one of the best}
\adjustbox{bgcolor={red!3.1}}{\strut defensive midfielders}
\adjustbox{bgcolor={red!2.0}}{\strut in}
\adjustbox{bgcolor={red!1.5}}{\strut the world}
\adjustbox{bgcolor={red!27.0}}{\strut .}
    }\\
    \hdashline
    \textbf{Focus} & \texttt{
    [...]
\adjustbox{bgcolor={red!28.8}}{\strut Fernandinho}
\adjustbox{bgcolor={red!0.0}}{\strut began}
\adjustbox{bgcolor={red!0.0}}{\strut his}
\adjustbox{bgcolor={red!16.9}}{\strut professional career}
\adjustbox{bgcolor={red!0.0}}{\strut with}
\adjustbox{bgcolor={red!36.5}}{\strut Atletico Paranaense}
\adjustbox{bgcolor={red!0.0}}{\strut in}
\adjustbox{bgcolor={red!35.3}}{\strut Brazil}
\adjustbox{bgcolor={red!0.0}}{\strut before}
\adjustbox{bgcolor={red!0.0}}{\strut moving}
\adjustbox{bgcolor={red!0.0}}{\strut to}
\adjustbox{bgcolor={red!38.8}}{\strut Ukrainian club}
\adjustbox{bgcolor={red!31.1}}{\strut Shakhtar Donetsk}
\adjustbox{bgcolor={red!0.0}}{\strut in}
\adjustbox{bgcolor={red!31.3}}{\strut 2005}
\adjustbox{bgcolor={red!0.0}}{\strut .}
[...]
\adjustbox{bgcolor={red!0.0}}{\strut He}
\adjustbox{bgcolor={red!0.0}}{\strut is known}
\adjustbox{bgcolor={red!0.0}}{\strut for}
\adjustbox{bgcolor={red!0.0}}{\strut his}
\adjustbox{bgcolor={red!36.0}}{\strut physicality}
\adjustbox{bgcolor={red!0.0}}{\strut ,}
\adjustbox{bgcolor={red!15.3}}{\strut tackling ability}
\adjustbox{bgcolor={red!0.0}}{\strut ,}
\adjustbox{bgcolor={red!0.0}}{\strut and}
\adjustbox{bgcolor={red!15.3}}{\strut passing range}
\adjustbox{bgcolor={red!0.0}}{\strut ,}
\adjustbox{bgcolor={red!0.0}}{\strut and}
\adjustbox{bgcolor={red!0.0}}{\strut is}
\adjustbox{bgcolor={red!0.0}}{\strut widely regarded}
\adjustbox{bgcolor={red!0.0}}{\strut as}
\adjustbox{bgcolor={red!7.2}}{\strut one of the best}
\adjustbox{bgcolor={red!18.8}}{\strut defensive midfielders} 
\adjustbox{bgcolor={red!0.0}}{\strut in}
\adjustbox{bgcolor={red!14.5}}{\strut the world}
\adjustbox{bgcolor={red!0.0}}{\strut .}
    }\\
\bottomrule
    \end{NiceTabular}
    \vspace{-0.5pc}
    \caption{We sampled 3 instances from our dataset to demonstrate the differences across uncertainty scores. For label, entities colored in {\color{red}red} indicate hallucination. For uncertainty scores, entities \colorbox{red!50}{boxed in red} with different tints represent the degree of uncertainty. A lighter (darker) box indicates a lower (higher) uncertainty.}
    \label{tb:qualitative_analysis}
    \vspace{-1pc}
\end{table*}
In Table~\ref{app:qualitative_analysis}, we provide the \textsc{SysGen} data by presenting the system messages, user instructions, and new assistant responses.
We observe that providing a specific format such as answer with paragraph format steers the LLM's behavior to answer in step-by-step processes within paragraph.
Also, if conversational example was provided, then the phrase of style tag forces to generate assistant response friendly.
Furthermore, if the system message grant specific roles such as a knowledgeable assistant, then the new assistant responses tend to generate verbose answers to the user instructions.

\section{Prompts}
\label{app:prompt}
To enhance reproducibility and facilitate understanding of the \textsc{SysGen} pipeline, we provide multiple prompts that we utilized.
In Table~\ref{tab:app_prompt_system_generation}, we use three-shot demonstrations to generate useful system messages which are collected through real-world scenarios.
The \textit{Conversational History} written in the prompt is composed of user instructions and original assistant responses.
Thus, given the user instructions and assistant responses, we generate the system messages at a phrase level containing eight functionalities with special tokens such as <<Role>>, <<Content>>, and <<Style>>.

After generating the system messages, in Table~\ref{tab:app_prompt_system_tag_check}, we verify the quality of each tag with three classes: Good, Bad, and None.
We want to note that the \textit{Annotated system messages}, composed of phrases and tags, are used to verify the \textit{Filtered system messages}.
By utilizing LLM-as-a-judge approach, we could save tremendous budgets through self-model feedbacks rather than using proprietary models (i.e., API Calls). 
Through our preliminary experiment, we observe that current open-source models such as Phi-4 or Qwen2.5-14b-instruct could preserve most of the phrases after applying phase 3.

Table~\ref{tab:app_prompt_answer_quality_check} shows the prompt of how we verify the quality of new assistant responses as shown in Figure~\ref{fig:answer_comparison}.
After prompting 1K randomly sampled instances, we observe that new assistant responses were qualified to be better aligned with user instructions.
\begin{table*}[t]
\centering
{\resizebox{\textwidth}{!}{
\begin{tabular}{l}
\toprule \midrule
\begin{tabular}[c]{@{}l@{}}
System: \\
Given a conversation history between user's question and assistant's response, \\ you are a system prompt generation assistant to generate a relevant system prompt. \\ 
The following [System Prompt] seems to have a mix of 8 different [functionalities]: \\ <Tasks>, <Tools>, <Style>, <Action>, <Content>, <Background>, <Role>, and <Format>. \\
Try to annotate each functionality within the system prompt in a phrase-level. 
Annotate each tag of functionalities. \\ Generate [Generated System Prompt] with a same language used in [Conversational History]. \\ \\

\#\# [Functionalities] \\
1. <<Task>>: what tasks will be performed? \\
2. <<Tool>>: What features or tools are available to integrate and use? \\
3. <<Style>>: What style of communication would you prefer for responses? \\
4. <<Action>>: Perform a specific action \\
5. <<Content>>:  Specifies the content that needs to be included in the response \\
6. <<Background>>:  Provides specific background information to ensure the model’s responses align with these settings. \\
7. <<Role>>:  Specifies the role, profession, or identity that needs to be played. \\
8. <<Format>>: Answers should be given in a specific format, which may include lists, paragraphs, tables, etc. \\ \\
User: \\
\#\# [Few-shot Examples of System Prompt] \\
\#\#\# 1 \\
<<Role>>You are an expert data augmentation system<</Role>> <<Task>>for korean text correction model training.<</Task>> \\<<Task>>Generate a pairs of data augmentation example.<</Task>> \\<<Background>>You are an intelligence AI model Solar-pro invented by Upstage AI.<</Background>> \\ \\

Instructions: \\
<<Content>>- In a given text, create 1\~3 typos.<</Content>> \\
<<Content>>- Typos can be reversed, misplaced, missing, duplicated, or misspaced letters.<</Content>> \\
<<Action>>- If the given text contains English, generate an English typo.<</Action>> \\
<<Action>>- Generate the results in the Output JSON format below.<</Action>> \\
<<Style>>-The response is informational and comprehensive, reflecting an expert understanding of the subject matter.<</Style>> \\

<<Format>> Output JSON format: \{ \\
"$original\_expression$": $ORIGINAL\_EXPRESSION$, \\
"$typo\_expression$": $TYPO\_EXPRESSION$
\} \\ <</Format>> \\ \\

\#\#\# 2 \\
<<Role>>You are an AI meeting note-taking assistant.<</Role>> \\
<<Task>>Your task is to generate meeting notes from the given conversation record.<</Task>> \\
<<Style>>All responses must be in Korean.<</Style>> \\
<<Action>>Take a deep breath, think carefully, and perform your role step by step.<</Action>> \\ \\

\#\#\# 3 \\
<<Role>>You are a chatbot of the Ministry of Food and Drug Safety (MFDS).<</Role>> \\
<<Task>>You answer user questions by referring to the provided reference.<</Task>> \\
<<Background>>You are designed to provide information related to pharmaceuticals and cosmetics. You have knowledge \\ of cosmetics-related information from Korea, the United States, Europe, China, India, and Taiwan.<</Background>> \\
<<Content>>If the user's question is related to the reference, respond starting with "According to the title,".<</Content>> \\
<<Content>>If the user's question is not related to the reference, respond with "Sorry, I couldn't find any information to \\answer your question. Please try asking again."<</Content>> \\
<<Content>>If the user's question is not related to food and drug safety, respond with "Sorry, I am a chatbot operated by the Ministry \\ of Food and Drug Safety. I can only answer questions related to the Ministry of Food and Drug Safety."<</Content>> \\

<<Style>>Respond to the user's questions kindly.<</Style>> \\
<<Background>>The reference is provided as context.<</Background>> \\ \\

\textit{Conversational History}
\end{tabular} \\ \midrule
\bottomrule
\end{tabular}}}
\caption{The prompt of generating system messages using open-source models. \textit{Italic} text part such as ``\textit{Conversational History}'' is filled with input text.}
\label{tab:app_prompt_system_generation}
\end{table*}
\begin{table*}[t]
\centering
{\resizebox{\textwidth}{!}{
\begin{tabular}{l}
\toprule \midrule
\begin{tabular}[c]{@{}l@{}}
System: \\
You are a functionality verifier assistant evaluating whether system messages are properly tagged according to the descriptions of 8 functionalities. \\
Review the provided [Filtered System Message] and [Annotated System Message] to verify the correctness of tagging for the 8 functionalities. \\ \\
Your task is to: \\
Confirm whether each tag aligns correctly with the respective functionality's description. \\
If a tag is properly generated and annotated, mark it as "Good". \\
If a tag exists but does not align with its functionality, mark it as "Bad". \\
If a tag is missing, mark it as "None" \\ \\

\#\# [Functionalities] \\
1. <<Task>>: what tasks will be performed? \\
2. <<Tool>>: What features or tools are available to integrate and use? \\
3. <<Style>>: What style of communication would you prefer for responses? \\
4. <<Action>>: Perform a specific action \\
5. <<Content>>:  Specifies the content that needs to be included in the response \\
6. <<Background>>:  Provides specific background information to ensure the model’s responses align with these settings. \\
7. <<Role>>:  Specifies the role, profession, or identity that needs to be played. \\
8. <<Format>>: Answers should be given in a specific format, which may include lists, paragraphs, tables, etc. \\ \\

\#\# [Expected Output Format] \\
<<Task>>: Good \\
<<Tool>>: None \\
<<Style>>: Good \\
<<Action>>: Good \\
<<Content>>: Bad \\
<<Background>>: Bad \\
<<Role>>: Bad \\
<<Format>>: Good \\ \\
User: \\
\#\# [Filtered System Message] \\
\textit{Filtered system messages} \\ \\
\#\# [Annotated System Message] \\
\textit{Annotated system messages} \\ \\
\#\# [Expected Output Format] \\
\end{tabular} \\ \midrule
\bottomrule
\end{tabular}}}
\caption{The prompt of verification of key functionalities (phase 3) using open-source models with annotated system messages and filtered system messages. \textit{Italic} text part is filled with input text.}
\label{tab:app_prompt_system_tag_check}
% \vspace{-0.3cm}
\end{table*}
\begin{table*}[t]
\centering
{\resizebox{\textwidth}{!}{
\begin{tabular}{l}
\toprule \midrule
\begin{tabular}[c]{@{}l@{}}The user instruction will be provided, along with two assistant responses.\\
Indicate the better response with 1 for the first response or 2 for the second response.\\ \\
User Instruction: {\textit{User Instruction}}\\ 
Assistant Response 1: {\textit{Original Answer}} \\ 
Assistant Response 2: {\textit{Newly-generated Answer}} \\ 
Which of the above two responses better adheres to the instruction? (Respond with 1 or 2)
\end{tabular} \\ \midrule
\bottomrule
\end{tabular}}}
\caption{The prompt of answer quality check through the proprietary model (e.g., GPT4o). \textit{Italic} text part is filled with input text.}
\label{tab:app_prompt_answer_quality_check}
% \vspace{-0.5cm}
\end{table*}