
\section{Analysis}
\subsection{What makes \textsc{SysGen} pipeline useful?}
\label{ana:sysgen_useful}
\begin{table}[t]
\centering
{\resizebox{\columnwidth}{!}{
\begin{tabular}{lcc}
\toprule
\multicolumn{1}{c}{\textbf{Models}}           & \begin{tabular}[c]{@{}c@{}}\textbf{Multifacet}\\ \textbf{(Average)}\end{tabular} & \begin{tabular}[c]{@{}c@{}}\textbf{Unseen Benchmarks}\\ \textbf{(Average)}\end{tabular} \\ \midrule
\textit{No System Message} \\ \midrule
LLaMA-3.1-8B-instruct  & 3.98 & 50.85 \\ 
Phi-4 & 4.26 & 66.33 \\ 
\midrule
\textit{Common System Message} \\ \midrule
LLaMA-3.1-8B-instruct  & 3.89 & 51.23 \\ 
Phi-4 & 4.23 & 66.52 \\
\midrule
\textit{\textsc{SysGen} without A'} \\ \midrule
LLaMA-3.1-8B-instruct  & 4.09 & 51.89 \\ 
Phi-4 & 4.38 & 66.12 \\ 
\midrule
\textit{\textsc{SysGen}} \\ \midrule
LLaMA-3.1-8B-instruct  & 4.21 & 54.02 \\ 
Phi-4 & 4.54 & 68.08 \\
\bottomrule
\end{tabular}}}
\caption{Ablation studies of using system message and assistant's response. Using a common system message or generated system message does not provide insightful difference. Newly-generated answer and its corresponding system message can increase system abilities with lower decrease in unseen benchmarks.}
\label{tab:compare_system_message}
\vspace{-0.3cm}
\end{table}
To assess the impact of system messages generated by \textsc{SysGen} during training, we conduct ablation studies on four different model variations:
\begin{itemize}
    \item No System Message: The original SFT dataset which does not contain the system message. 
    \item Common System Message: An $\mathcal{S}\mathcal{Q}\mathcal{A}$ triplet where the common system message is inserted such as "You are a helpful AI assistant".
    \item \textsc{SysGen} without $\mathcal{A'}$: An $\mathcal{S}\mathcal{Q}\mathcal{A}$ triplet that includes only a system message generated by our \textsc{SysGen} pipeline.
    \item \textsc{SysGen}: An $\mathcal{S}\mathcal{Q}\mathcal{A'}$ triplet where both the \textsc{SysGen}-generated system message and the newly-generated answer are incorporated.
\end{itemize}
We measure the effectiveness of these models by analyzing score variations on the Multifacet and unseen benchmarks in Table~\ref{tab:compare_system_message}.

Training with data that includes common system messages does not result in a significant performance difference compared to training without system messages.
This led us to question: \emph{"Would it be sufficient to include only the most suitable system messages?"}.
To explore this, we train models using data that contains only system messages generated by \textsc{SysGen} pipeline.
As a result, we observe an improvement in Multifacet performance for both models, while the scores on the unseen benchmark remained similar.
Furthermore, when both system messages and assistant responses generated by \textsc{SysGen} are used for fine-tuning, we observe performance improvements in both Multifacet evaluation and unseen benchmarks.

\subsection{System message vs. User instruction}
\begin{table}[t]
\centering
{\resizebox{\columnwidth}{!}{
\begin{tabular}{lc}
\toprule
\multicolumn{1}{c}{\textbf{Models}}           & \begin{tabular}[c]{@{}c@{}}\textbf{Multifacet Average}\\ \textbf{(Use system role → Use user role)}\end{tabular} \\ \midrule
\textit{Open-source Models} \\ \midrule
Solar-10.7B-instruct   & 3.19 → 2.98 \\ 
LLaMA-3.1-8B-instruct  & 4.12 → 4.09 \\ 
Qwen2.5-14b-instruct   & 4.26 → 4.13 \\ 
Phi-4 & 4.41 → 4.26 \\ 
\midrule
\multicolumn{2}{l}{\textit{Open-source Models} (with \textbf{\textsc{SysGen}})} \\ \midrule
LLaMA-3.1-8B-instruct  & 4.21 → 4.13 \\ 
Qwen2.5-14B-instruct   & 4.28 → 4.16 \\ 
Phi-4 & 4.54 → 4.38 \\
\midrule
\multicolumn{2}{l}{\textit{Open-source Models} $+$ KD  (with \textbf{\textsc{SysGen}})} \\ \midrule
Solar-10.7b-instruct   & 3.76 → 3.64 \\ 
\bottomrule
\end{tabular}}}
\caption{There is a tendency for the score to decrease when the system message is reflected in the user instruction. The more a model is trained on system messages, the better it is to place them in the system role. KD indicates the knowledge distillation.}
\label{tab:no_system_message}
% \vspace{-0.5cm}
\end{table}
A key question arises that \emph{what happens if we add a message intended for the system role at the beginning of the user instruction? Could it serve as a replacement for the system role?}
To explore this, we conduct an experiment on a Multifacet benchmark.
Specifically, we included messages that should typically be in the system role within the user instruction during inference.

As shown in Table~\ref{tab:no_system_message}, we observe that open-source models tend to experience score degradation when system role messages are incorporated into the user instruction.
This trend suggests that adding such content can make the query itself more ambiguous to answer.
Furthermore, even in models trained with our \textsc{SysGen}, this trend persists similarly to the previous work~\citep{lee2024aligning}.
Despite additional fine-tuning on system roles, scores still remain low when system messages are reflected in the user instruction.
This highlights the importance of properly placing these messages in the system role to maintain performance.
% This indicates that properly placing these messages in the system role remains crucial for maintaining performance.

\subsection{New assistant responses align to the system messages}
\begin{figure}[t]
\centering
\includegraphics[width=\columnwidth]{figure/aligment_message_response.pdf}
\caption{
The GPT4o LLM-as-a-judge results of measuring the alignment between generated system messages and new assistant responses. We use 20 samples for each data source which sums up to 100 samples in total per models.
}
\label{fig:alignment_message_response}
\vspace{-0.3cm}
\end{figure}
In Table~\ref{tab:statistics_generated_answer}, we presented that the new assistant responses exhibit similar n-gram matching, high semantic similarities, and verbosity.
Therefore, it is necessary to verify whether the generated assistant responses aligned with the system messages.
Figure~\ref{fig:alignment_message_response} illustrates the GPT-4o results using LLM-as-a-judge approach.
Through the three \textsc{SysGen} data generated by Phi-4, LLaMA, and Qwen models, we determined that all of the assistant responses are highly aligned with the system messages.
Overall, the experiments and analyses reveal that our \textsc{SysGen} data were generated to effectively respond to various user instructions as system messages.
In addition, we observed that the assistant responses align with the system messages and are capable of generating better aligned responses compared the original assistant responses.
