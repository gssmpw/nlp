% This must be in the first 5 lines to tell arXiv to use pdfLaTeX, which is strongly recommended.
\pdfoutput=1
% In particular, the hyperref package requires pdfLaTeX in order to break URLs across lines.

\documentclass[11pt]{article}

% Change "review" to "final" to generate the final (sometimes called camera-ready) version.
% Change to "preprint" to generate a non-anonymous version with page numbers.
\usepackage[preprint]{acl}

% Standard package includes
\usepackage{times}
\usepackage{latexsym}

% For proper rendering and hyphenation of words containing Latin characters (including in bib files)
\usepackage[T1]{fontenc}
% For Vietnamese characters
% \usepackage[T5]{fontenc}
% See https://www.latex-project.org/help/documentation/encguide.pdf for other character sets

% This assumes your files are encoded as UTF8
\usepackage[utf8]{inputenc}
\usepackage{microtype}
\usepackage{inconsolata}
\usepackage{graphicx}
\usepackage{booktabs}       % professional-quality tables
\usepackage{microtype}      % microtypography
% Standard package includes
\usepackage{url}            % simple URL typesetting
\usepackage{amsfonts}       % blackboard math symbols
\usepackage{nicefrac}       % compact symbols for 1/2, etc.
\usepackage{amsmath}
\usepackage{graphicx}
\usepackage{bm, upgreek}
\usepackage{amssymb}
\usepackage{array}
\usepackage{multirow}
\usepackage{mathtools}
\usepackage{arydshln}
\usepackage{tabularx}
\usepackage{color}
\usepackage{soul}
\usepackage{caption}
\usepackage{xcolor}
\usepackage{comment}
\usepackage{adjustbox}
\usepackage{hyperref}
\usepackage{cleveref}
\usepackage[ruled,vlined]{algorithm2e}
\usepackage{makecell}
\usepackage{diagbox}
\usepackage{subfigure}
\usepackage{subcaption}
\usepackage{xcolor, soul}
\usepackage{wrapfig}
% \usepackage[numbers]{natbib}
% \captionsetup[sub]{font=normalsize,labelfont={bf,sf}}
\usepackage{pifont}% http://ctan.org/pkg/pifont
\newcommand{\cmark}{\ding{51}}%
\newcommand{\xmark}{\ding{55}}%
\definecolor{bluecolor}{HTML}{0000FF}
\definecolor{greencolor}{HTML}{8CD0A4}
\definecolor{yellowcolor}{HTML}{F9D17C}
\definecolor{redcolor}{HTML}{FF0000}

\newcommand{\cyan}[1]{\textcolor{cyan}{#1}}
\newcommand{\red}[1]{\textcolor{red}{#1}}
\newcommand{\blue}[1]{\textcolor{blue}{#1}}
\newcommand{\yellow}[1]{\textcolor{yellow}{#1}}
\definecolor{black}{rgb}{0,0,0}
\newcommand{\black}[1]{\textcolor{black}{#1}}
\newcommand{\dandelion}[1]{\textcolor{dandelion}{#1}}

\title{System Message Generation for User Preferences \\ using Open-Source Models}

\author{Minbyul Jeong\footnotemark[1] \\
  \\ \And
  Jungho Cho \\
  \\ \And
  Minsoo Khang \\
  Upstage AI \\ \And
  Dawoon Jung \\
  \\ \And
  Teakgyu Hong \\
}

% \vspace{-2.0cm}
\begin{document}
\maketitle
\footnotetext[1]{Corresponding authors.}

\begin{abstract}
System messages play a crucial role in interactions with large language models (LLMs), often serving as prompts to initiate conversations.
Through system messages, users can assign specific roles, perform intended tasks, incorporate background information, specify various output formats and communication styles.
Despite such versatility, publicly available data are often lack system messages and subject to strict license constraints in the industry field.
Manual labeling of publicly available data with system messages that align with user instructions demands significant resources.
In view of such challenges, our work introduces \textbf{\textsc{SysGen}}, a pipeline for generating system messages with better aligned assistant responses from the supervised fine-tuning dataset without system messages.
Training on \textsc{SysGen} data has demonstrated substantial improvements in the alignment of model responses with system messages and user instructions, as demonstrated across various open-source models on the Multifacet benchmark, while maintaining minimal impact on other unseen benchmarks such as Open LLM Leaderboard 2.
Our qualitative analysis highlights the importance of diverse system messages to ensure better adaptability across different contexts.
\end{abstract}

% \vspace{0.3cm}


\section{Introduction}
\IEEEPARstart{I}{n} recent years, flourishing of Artificial Intelligence Generated Content (AIGC) has sparked significant advancements in modalities such as text, image, audio, and even video. 
Among these, AI-Generated Image (AGI) has garnered considerable interest from both researchers and the public.
Plenty of remarkable AGI models and online services, such as StableDiffusion\footnote{\url{https://stability.ai/}}, Midjourney\footnote{\url{https://www.midjourney.com/}}, and FLUX\footnote{\url{https://blackforestlabs.ai/}}, offer users an excellent creative experience.
However, users often remain critical of the quality of the AGI due to image distortions or mismatches with user intentions.
Consequently, methods for assessing the quality of AGI are becoming increasingly crucial to help improve the generative capabilities of these models.

Unlike Natural Scene Image (NSI) quality assessment, which focuses primarily on perception aspects such as sharpness, color, and brightness, AI-Generated Image Quality Assessment (AGIQA) encompasses additional aspects like correspondence and authenticity. 
Since AGI is generated on the basis of user text prompts, it may fail to capture key user intentions, resulting in misalignment with the prompt.
Furthermore, authenticity refers to how closely the generated image resembles real-world artworks, as AGI can sometimes exhibit logical inconsistencies.
While traditional IQA models may effectively evaluate perceptual quality, they are often less capable of adequately assessing aspects such as correspondence and authenticity.

\begin{figure}\label{fig:radar}
    \centering
    \includegraphics[width=1.0\linewidth]{figures/radar_plot.pdf}
    \caption{A comparison on quality, correspondence, and authenticity aspects of AIGCIQA2023~\cite{wang2023aigciqa2023} dataset illustrates the superior performance of our method.}
\end{figure}

Several methods have been proposed specifically for the AGIQA task, including metrics designed to evaluate the authenticity and diversity of generated images~\cite{gulrajani2017improved,heusel2017gans}. 
Nevertheless, these methods tend to compare and evaluate grouped images rather than single instances, which limits their utility for single image assessment.
Beginning with AGIQA-1k~\cite{zhang2023perceptual}, a series of AGIQA databases have been introduced, including AGIQA-3k~\cite{li2023agiqa}, AIGCIQA-20k~\cite{li2024aigiqa}, etc.
Concurrently, there has been a surge in research utilizing deep learning methods~\cite{zhou2024adaptive,peng2024aigc,yu2024sf}, which have significantly benefited from pre-trained models such as CLIP~\cite{radford2021learning}. 
These approaches enhance the analysis by leveraging the correlations between images and their descriptive texts.
While these models are effective in capturing general text-image alignments, they may not effectively detect subtle inconsistencies or mismatches between the generated image content and the detailed nuances of the textual description.
Moreover, as these models are pre-trained on large-scale datasets for broad tasks, they might not fully exploit the textual information pertinent to the specific context of AGIQA without task-specific fine-tuning.
To overcome these limitations, methods that leverage Multimodal Large Language Models (MLLMs)~\cite{wang2024large,wang2024understanding} have been proposed.
These methods aim to fully exploit the synergies of image captioning and textual analysis for AGIQA.
Although they benefit from advanced prompt understanding, instruction following, and generation capabilities, they often do not utilize MLLMs as encoders capable of producing a sequence of logits that integrate both image and text context.

In conclusion, the field of AI-Generated Image Quality Assessment (AGIQA) continues to face significant challenges: 
(1) Developing comprehensive methods to assess AGIs from multiple dimensions, including quality, correspondence, and authenticity; 
(2) Enhancing assessment techniques to more accurately reflect human perception and the nuanced intentions embedded within prompts; 
(3) Optimizing the use of Multimodal Large Language Models (MLLMs) to fully exploit their multimodal encoding capabilities.

To address these challenges, we propose a novel method M3-AGIQA (\textbf{M}ultimodal, \textbf{M}ulti-Round, \textbf{M}ulti-Aspect AI-Generated Image Quality Assessment) which leverages MLLMs as both image and text encoders. 
This approach incorporates an additional network to align human perception and intentions, aiming to enhance assessment accuracy. 
Specially, we distill the rich image captioning capability from online MLLMs into a local MLLM through Low-Rank Adaption (LoRA) fine-tuning, and train this model with human-labeled data. The key contributions of this paper are as follows:
\begin{itemize}
    \item We propose a novel AGIQA method that distills multi-aspect image captioning capabilities to enable comprehensive evaluation. Specifically, we use an online MLLM service to generate aspect-specific image descriptions and fine-tune a local MLLM with these descriptions in a structured two-round conversational format.
    \item We investigate the encoding potential of MLLMs to better align with human perceptual judgments and intentions, uncovering previously underestimated capabilities of MLLMs in the AGIQA domain. To leverage sequential information, we append an xLSTM feature extractor and a regression head to the encoding output.
    \item Extensive experiments across multiple datasets demonstrate that our method achieves superior performance, setting a new state-of-the-art (SOTA) benchmark in AGIQA.
\end{itemize}

In this work, we present related works in Sec.~\ref{sec:related}, followed by the details of our M3-AGIQA method in Sec.~\ref{sec:method}. Sec.~\ref{sec:exp} outlines our experimental design and presents the results. Sec.~\ref{sec:limit},~\ref{sec:ethics} and~\ref{sec:conclusion} discuss the limitations, ethical concerns, future directions and conclusions of our study.
% % \section{Preliminaries}

% \subsection{History Taking for Doctor and Patient}

% In our paper, 

% \subsection{Medical Conversational Agent}

% \subsection{Automatic Evaluation of Long-form response in multi-turn dialogue}



% \begin{align}
%     P_{t+1} &= R(D_t, H_t) \\
%     D_{t+1} &= R(P_t, H_t)   
% \end{align}
     

\section{Related Works}
% \paragraph{Human preference learning.}

\paragraph{System message: utilization and evaluation.}
A system message is a unique component of LLMs to initiate a conversation with them.
It is utilized by many proprietary models (e.g.,  ChatGPT~\citep{openai2023b} and Claude~\citep{anthropic2024}) as well as open-source models (e.g.,  Mistral~\citep{alkhamissi2024investigating}, LLaMA~\citep{meta2024introducing}, Qwen~\citep{yang2025qwen2}, and DeepSeek~\citep{guo2025deepseek}).
The system messages serve the purpose of steering the LLM's generation behavior and are widely used for various functions, including imprinting the model's identity, recording the knowledge cut-off date of the training data, and providing guidelines for various tool usages~\citep{openai2024function, cohere2024, prompthub2025}.
Additionally, the system messages are used to guide the model in generating safe and harmless responses~\citep{touvron2023llama, lu2024sofa, wallace2404instruction}.

Despite the usefulness of system messages, there is a significant lack of data that includes system messages reflecting diverse user instructions without license constraints.
Furthermore, manually labeling such data requires substantial human resources and even among publicly available datasets, it is challenging to obtain data that includes various system messages~\citep{lin2024baichuan, xu2024magpie}.
\citet{lee2024aligning} provide data augmentation which reflects hierarchical dimensions of system role data with multiple aspects of evaluation benchmark.
Furthermore, \citet{qin2024sysbench} provide multi-turn benchmark to evaluate system message alignment.
In line of these works, our \textsc{SysGen} pipeline  ensures high-quality system messages and assistant responses by supplementing data using only open-source models without licensing concerns.
Furthermore, it demonstrates that data augmentation is possible on existing SFT datasets without requiring extensive human labeling efforts.


\begin{figure*}[]
\centering
\includegraphics[width=\textwidth]{figure/data_construction.pdf}
% \vspace{-0.3cm}
\caption{
Overall \textbf{\textsc{SysGen}} data construction pipeline. Our pipeline consists of four phases:
% (Phase 1 - System Message Generation) We first collect supervised fine-tuning (SFT) datasets which does not contain system message. Then, we use open-source models to generate system message given question and answer pair. Our system message is composed of phrases with manually selected eight key functionalities tags.
(Phase 1) We gather SFT datasets which do not contain system messages and use open-source models to generate system messages with manually selected eight key fuctionality tags. 
% (Phase 2) We apply a filtering process to remove cases where tag tokens are generated incorrectly. To ensure consistency in system messages, we reorganize the tags in a predefined order.
(Phase 2) We then remove incorrectly generated tag tokens and reorganize tags with phrases in a predefined order for consistency.
% (Phase 3) To verify the correctness of phrases generated for each tag, we introduce a self-checking LLM-as-a-judge mechanism. Phrases that are empty, overly specific, or unnatural are classified as Bad and subsequently removed along with their corresponding tags.
(Phase 3) We use a LLM-as-a-judge approach with self-model feedback to filter out empty, overly specific, and unnatural phrases.
% (Phase 4) We refine the generated system message by removing tags to create a more natural sentence. Based on this refined system message, we generate a new answer along with the user instruction.
(Phase 4) We finally remove tags to create natural system messages and generate new responses along with the user instructions.
}
\label{fig:data_construction}
\vspace{-0.3cm}
\end{figure*}
\section{\textsc{SysGen}: Pipeline of System and Assistant Response Generation}
Our \textbf{\textsc{SysGen}} pipeline consists of four phases: (1) generating system messages with eight key functionalities (Sec~\ref{sysgen:system_generation}), (2) filtering mis-specified system tags and reorganizing them (Sec~\ref{sysgen:filtering}), (3) verifying the key functionalities on a phrase level (Sec~\ref{sysgen:verification}), (4) generating the new assistant responses using the refined system messages and original user instructions (Sec~\ref{sysgen:answer_generation}).
Figure~\ref{fig:data_construction} depicts the overall architecture of the \textsc{SysGen} pipeline. 

\subsection{Phase 1: System Message Generation}
\label{sysgen:system_generation}
% The primary goal of our \textbf{\textsc{SysGen}} pipeline is to generate the system messages that are not included in the original SFT dataset.
The primary goal of our \textbf{\textsc{SysGen}} pipeline is to enhance existing SFT datasets by adding system messages that were not originally included.
% During the process of developing and releasing LLMs, license constraints inevitably arise, making it difficult to utilize most publicly available data.
As the system messages can steer the LLM's behaviors, we focus on these messages during the development and release of the models.
%The system messages can steer the LLM's behaviors, so we focus on these messages during the development and release of the models.
However, license constraints and substantial resource requirements of manually labeling the system messages inevitably arise, making it difficult to utilize most publicly available datasets.
Thus, we aim to generate system messages by leveraging open-source models and data without any license issues.
% We focus on system messages because they can control the LLM's behavior, set roles, provide additional background contexts, and maintain consistent responses.

\paragraph{Phrase level Annotation to System Messages}
% eight key features 나열, 하는 역할 제공 및 어떻게 annotate하고 이를 verify하는지
We manually classify eight functionalities that are widely used in the system messages referring to previous works~\citep{openai2024function, cohere2024, alkhamissi2024investigating,lee2024aligning}: 
(1) Specifies the role, profession, or identity that needs to be played (Role);
(2) Specifies the content that needs to be included in the response such as an identity of the company (Content);
(3) Identifies what to perform (Task);
(4) Specifies the behavior to perform (Action);
(5) Prefers the style of communication for responses (Style);
(6) Provides additional information to be served as an assistant (Background);
(7) Provides built-in methods to use (Tool); 
(8) Preference of what output should look like (Format).


As shown in Figure~\ref{fig:data_construction} (top left), all functionalities are annotated at a phrase level with pre-/post-fix tags. 
Given a pair of user instructions $\mathcal{Q}$ and assistant responses $\mathcal{A}$, we generate a system message $\mathcal{S}$ using the open-source LLMs $\mathcal{M}$ with a prompt $\mathcal{P}$ that includes few-shot demonstrations:
\begin{equation}
    \mathcal{M}(\mathcal{S}|\mathcal{P},\mathcal{Q},\mathcal{A})
\end{equation}
We provide details about the few-shot demonstrations in the Appendix~\ref{app:prompt}.


\subsection{Phase 2: Filtering Process}
\label{sysgen:filtering}
After generating the system messages, we filter out the abnormal system messages for consistent text format.
In Figure~\ref{fig:data_construction} (top right), we first identify and remove mis-tagged phrases.
% For example, if the start token is set as <<Task>> but the end token is set as <<Format>>, then we cannot guarantee the phrase between these tokens corresponds to <<Task>> or <<Format>>.
For example, we can guarantee the correctness of the phrases between these tokens only if the start and end tokens are the same (e.g., <<Task>>).
% For example, if the start and end tokens are the same (e.g., <<Task>>), then we can guarantee the correctness of the phrases between these tokens.
In addition, we remove invalid tags such as <<Example>> or <<System>>, which may be generated in phase 1.
To ensure a consistent structure of system messages, we reorder the tags and phrases in manually defined order.

\begin{table}[t]
{\resizebox{1.0\columnwidth}{!}{
\begin{tabular}{lccccccc}
\toprule
\multicolumn{1}{c}{\multirow{2}{*}{Models}} & \multicolumn{3}{c}{Words Composition}                 & \multirow{2}{*}{BERTScore} & \multirow{2}{*}{BLEURT} & \multirow{2}{*}{GLEU} & \multirow{2}{*}{Len.} \\ \cmidrule{2-4}
\multicolumn{1}{c}{} & \multicolumn{1}{c}{R1} & \multicolumn{1}{c}{R2} & \multicolumn{1}{c}{RL} & & & \\ \midrule
LLaMA-3.1-8B-instruct  & 33.3 & 15.6 & 23.1 & 81.3 & 33.6 & 28.2 & 1.35\\ 
Qwen2.5-14b-instruct  & 44.9 & 23.2 & 30.7 & 85.9 & 39.9 & 39.2 & 1.55\\ 
Phi-4 & 51.9 & 32.3 & 41.1 & 86.1 & 40.1 & 37.2 & 1.89 \\ 
\bottomrule
\end{tabular}}}
\caption{A statistic that measures the words composition (Rouge-1,-2, and -L), semantic similarity (BERTScore and BLEURT), fluency (GLEU), and average context length of the newly-generated answer compared to average context length of the original answer.}
\label{tab:statistics_generated_answer}
\vspace{-0.3cm}
\end{table}

\subsection{Phase 3: Verification of Eight Key Functionalities}
\label{sysgen:verification}
In this phase, we verify whether each generated phrase is appropriate for its assigned tag. 
Using the LLM-as-a-judge~\citep{zheng2023judging} approach with self-model feedback, we assign one of three labels for each tag: \textit{Good} if the tagging is appropriate, \textit{Bad} if the tagging is inappropriate, and \textit{None} if the tag or phrases are missing.
Phrases labeled as \textit{Bad} or \textit{None} are then removed from the system message to ensure accuracy and consistency.
We observe that most of the data instances (up to 99\%) are preserved after applying phase 3.
% In Figure~\ref{}, we provide the filtered statistics of the phase 2 \& 3.


\subsection{Phase 4: Assistant Response Generation}
\label{sysgen:answer_generation}
% remove annotated tags for naturality
After filtering and verifying the generated system messages, they can be used alongside existing QA pairs.
However, we hypothesize that if there is any potential misalignment between the human curated QA and model-generated system messages, a follow-up data alignment phase is necessary.
% However, our experimental results reveal that training LLMs on existing QA pairs with generated system messages leads to little to no improvement in benchmarks evaluating system message alignment such as Multifacet~\citep{lee2024aligning}.
% The performance degradation has also been observed in unseen benchmarks such as Open LLM Leaderboard 2~\citep{myrzakhan2024open}.
% Therefore, we hypothesize that we have to generate new assistant responses $\mathcal{A'}$ based on a refined system messages $\mathcal{S}$ and the user instructions $\mathcal{Q}$, ensuring better alignment with the given instructions.
Therefore, we generate new assistant responses $\mathcal{A'}$ based on a refined system messages $\mathcal{S}$ and the user instructions $\mathcal{Q}$, ensuring better alignment with the given instructions.
%Therefore, it is necessary to generate new assistant responses $\mathcal{A'}$ based on a refined system message $\mathcal{S}$ and the user's original instruction $\mathcal{Q}$.

% answer generation with system message and original query
To achieve this, we first remove the annotated tags from the system messages to guarantee that the refined messages seem natural.
We provide a detailed example in Figure~\ref{fig:data_construction} (bottom right).
Then, we use the open-source LLMs $\mathcal{M}$ employed in phase 1 to generate new responses $\mathcal{A'}$.
\begin{equation}
    \mathcal{M}(\mathcal{A'}|\mathcal{S},\mathcal{Q})
\end{equation}
% how answer could be diversified and preserved compared to original answer
In Table~\ref{tab:statistics_generated_answer}, the new responses preserve similar content with high n-gram matching compared to the original responses, but have shown diversified formats with high semanticity and verbosity.
We provide the cases in Appendix~\ref{app:qualitative_analysis}.

\begin{figure}[t]
\centering
\includegraphics[width=\columnwidth]{figure/answer_quality_check_2.pdf}
\caption{
A statistic that verifies whether the newly-generated answer is more suitable for the user query than the original answer.
It records the probability that GPT-4o would respond with the newly-generated answer being better than the original answer (the probability should ideally exceed 50\%).
}
\label{fig:answer_comparison}
% \vspace{-0.3cm}
\end{figure}

% LLM-as-a-judge to verify new answer is preferred
We also use LLM-as-a-judge with GPT-4o to analyze that the new responses $\mathcal{A'}$ are better aligned to the user instructions than the original responses $\mathcal{A}$.
Figure~\ref{fig:answer_comparison} illustrates the proportion of cases where the new responses are judged to be better aligned than the original responses when given the user instructions.
For simpler evaluation, we evaluated 1K randomly sampled instances from the generated datasets.
Overall, our findings suggest that generating responses based on the system messages lead to better alignment with user instructions.
% Additionally, our experiments show that models trained with both the system messages and the new responses achieve quantifiable performance improvements, highlighting the importance of generating new responses.
\section{Experimental Settings}

\begin{table}[h]
\centering
\caption{Training Dataset Statistics.}
\label{tab:data_statistics} 
\resizebox{\linewidth}{!}{
% \begin{small}
\begin{tabular}{@{}l|ccc|ccc|c@{}}
\toprule
\multirow{2}{*}{\textbf{Data Name}} & \multicolumn{3}{c|}{\textbf{Multi-Hop}} & \multicolumn{3}{c|}{\textbf{Single-Hop}} & \multirow{2}{*}{\textbf{Total}} \\
                                    & 2WikiMultiHopQA  & HotpotQA  & Musique  & Natural Questions  & PopQA   & TriviaQA  &                                 \\ \midrule
\textbf{Raw Data Size}              & 167,454          & 90,447    & 19,938   & 79,169             & 12,868  & 78,785    & 448,661                         \\
\textbf{Our Train Data Size}        & 6,000            & 6,000     & 6,000    & 6,000              & 6,000   & 6,000     & 36,000                          \\
\textbf{Sampling ratio}             & 3.5\%            & 6.6\%     & 30.1\%   & 7.5\%              & 46.6\%  & 7.6\%     & 8.02\%                          \\ \bottomrule
\end{tabular}
}
% \end{small}
\end{table}
\section{Experiments}
\label{sec:experiments}
The experiments are designed to address two key research questions.
First, \textbf{RQ1} evaluates whether the average $L_2$-norm of the counterfactual perturbation vectors ($\overline{||\perturb||}$) decreases as the model overfits the data, thereby providing further empirical validation for our hypothesis.
Second, \textbf{RQ2} evaluates the ability of the proposed counterfactual regularized loss, as defined in (\ref{eq:regularized_loss2}), to mitigate overfitting when compared to existing regularization techniques.

% The experiments are designed to address three key research questions. First, \textbf{RQ1} investigates whether the mean perturbation vector norm decreases as the model overfits the data, aiming to further validate our intuition. Second, \textbf{RQ2} explores whether the mean perturbation vector norm can be effectively leveraged as a regularization term during training, offering insights into its potential role in mitigating overfitting. Finally, \textbf{RQ3} examines whether our counterfactual regularizer enables the model to achieve superior performance compared to existing regularization methods, thus highlighting its practical advantage.

\subsection{Experimental Setup}
\textbf{\textit{Datasets, Models, and Tasks.}}
The experiments are conducted on three datasets: \textit{Water Potability}~\cite{kadiwal2020waterpotability}, \textit{Phomene}~\cite{phomene}, and \textit{CIFAR-10}~\cite{krizhevsky2009learning}. For \textit{Water Potability} and \textit{Phomene}, we randomly select $80\%$ of the samples for the training set, and the remaining $20\%$ for the test set, \textit{CIFAR-10} comes already split. Furthermore, we consider the following models: Logistic Regression, Multi-Layer Perceptron (MLP) with 100 and 30 neurons on each hidden layer, and PreactResNet-18~\cite{he2016cvecvv} as a Convolutional Neural Network (CNN) architecture.
We focus on binary classification tasks and leave the extension to multiclass scenarios for future work. However, for datasets that are inherently multiclass, we transform the problem into a binary classification task by selecting two classes, aligning with our assumption.

\smallskip
\noindent\textbf{\textit{Evaluation Measures.}} To characterize the degree of overfitting, we use the test loss, as it serves as a reliable indicator of the model's generalization capability to unseen data. Additionally, we evaluate the predictive performance of each model using the test accuracy.

\smallskip
\noindent\textbf{\textit{Baselines.}} We compare CF-Reg with the following regularization techniques: L1 (``Lasso''), L2 (``Ridge''), and Dropout.

\smallskip
\noindent\textbf{\textit{Configurations.}}
For each model, we adopt specific configurations as follows.
\begin{itemize}
\item \textit{Logistic Regression:} To induce overfitting in the model, we artificially increase the dimensionality of the data beyond the number of training samples by applying a polynomial feature expansion. This approach ensures that the model has enough capacity to overfit the training data, allowing us to analyze the impact of our counterfactual regularizer. The degree of the polynomial is chosen as the smallest degree that makes the number of features greater than the number of data.
\item \textit{Neural Networks (MLP and CNN):} To take advantage of the closed-form solution for computing the optimal perturbation vector as defined in (\ref{eq:opt-delta}), we use a local linear approximation of the neural network models. Hence, given an instance $\inst_i$, we consider the (optimal) counterfactual not with respect to $\model$ but with respect to:
\begin{equation}
\label{eq:taylor}
    \model^{lin}(\inst) = \model(\inst_i) + \nabla_{\inst}\model(\inst_i)(\inst - \inst_i),
\end{equation}
where $\model^{lin}$ represents the first-order Taylor approximation of $\model$ at $\inst_i$.
Note that this step is unnecessary for Logistic Regression, as it is inherently a linear model.
\end{itemize}

\smallskip
\noindent \textbf{\textit{Implementation Details.}} We run all experiments on a machine equipped with an AMD Ryzen 9 7900 12-Core Processor and an NVIDIA GeForce RTX 4090 GPU. Our implementation is based on the PyTorch Lightning framework. We use stochastic gradient descent as the optimizer with a learning rate of $\eta = 0.001$ and no weight decay. We use a batch size of $128$. The training and test steps are conducted for $6000$ epochs on the \textit{Water Potability} and \textit{Phoneme} datasets, while for the \textit{CIFAR-10} dataset, they are performed for $200$ epochs.
Finally, the contribution $w_i^{\varepsilon}$ of each training point $\inst_i$ is uniformly set as $w_i^{\varepsilon} = 1~\forall i\in \{1,\ldots,m\}$.

The source code implementation for our experiments is available at the following GitHub repository: \url{https://anonymous.4open.science/r/COCE-80B4/README.md} 

\subsection{RQ1: Counterfactual Perturbation vs. Overfitting}
To address \textbf{RQ1}, we analyze the relationship between the test loss and the average $L_2$-norm of the counterfactual perturbation vectors ($\overline{||\perturb||}$) over training epochs.

In particular, Figure~\ref{fig:delta_loss_epochs} depicts the evolution of $\overline{||\perturb||}$ alongside the test loss for an MLP trained \textit{without} regularization on the \textit{Water Potability} dataset. 
\begin{figure}[ht]
    \centering
    \includegraphics[width=0.85\linewidth]{img/delta_loss_epochs.png}
    \caption{The average counterfactual perturbation vector $\overline{||\perturb||}$ (left $y$-axis) and the cross-entropy test loss (right $y$-axis) over training epochs ($x$-axis) for an MLP trained on the \textit{Water Potability} dataset \textit{without} regularization.}
    \label{fig:delta_loss_epochs}
\end{figure}

The plot shows a clear trend as the model starts to overfit the data (evidenced by an increase in test loss). 
Notably, $\overline{||\perturb||}$ begins to decrease, which aligns with the hypothesis that the average distance to the optimal counterfactual example gets smaller as the model's decision boundary becomes increasingly adherent to the training data.

It is worth noting that this trend is heavily influenced by the choice of the counterfactual generator model. In particular, the relationship between $\overline{||\perturb||}$ and the degree of overfitting may become even more pronounced when leveraging more accurate counterfactual generators. However, these models often come at the cost of higher computational complexity, and their exploration is left to future work.

Nonetheless, we expect that $\overline{||\perturb||}$ will eventually stabilize at a plateau, as the average $L_2$-norm of the optimal counterfactual perturbations cannot vanish to zero.

% Additionally, the choice of employing the score-based counterfactual explanation framework to generate counterfactuals was driven to promote computational efficiency.

% Future enhancements to the framework may involve adopting models capable of generating more precise counterfactuals. While such approaches may yield to performance improvements, they are likely to come at the cost of increased computational complexity.


\subsection{RQ2: Counterfactual Regularization Performance}
To answer \textbf{RQ2}, we evaluate the effectiveness of the proposed counterfactual regularization (CF-Reg) by comparing its performance against existing baselines: unregularized training loss (No-Reg), L1 regularization (L1-Reg), L2 regularization (L2-Reg), and Dropout.
Specifically, for each model and dataset combination, Table~\ref{tab:regularization_comparison} presents the mean value and standard deviation of test accuracy achieved by each method across 5 random initialization. 

The table illustrates that our regularization technique consistently delivers better results than existing methods across all evaluated scenarios, except for one case -- i.e., Logistic Regression on the \textit{Phomene} dataset. 
However, this setting exhibits an unusual pattern, as the highest model accuracy is achieved without any regularization. Even in this case, CF-Reg still surpasses other regularization baselines.

From the results above, we derive the following key insights. First, CF-Reg proves to be effective across various model types, ranging from simple linear models (Logistic Regression) to deep architectures like MLPs and CNNs, and across diverse datasets, including both tabular and image data. 
Second, CF-Reg's strong performance on the \textit{Water} dataset with Logistic Regression suggests that its benefits may be more pronounced when applied to simpler models. However, the unexpected outcome on the \textit{Phoneme} dataset calls for further investigation into this phenomenon.


\begin{table*}[h!]
    \centering
    \caption{Mean value and standard deviation of test accuracy across 5 random initializations for different model, dataset, and regularization method. The best results are highlighted in \textbf{bold}.}
    \label{tab:regularization_comparison}
    \begin{tabular}{|c|c|c|c|c|c|c|}
        \hline
        \textbf{Model} & \textbf{Dataset} & \textbf{No-Reg} & \textbf{L1-Reg} & \textbf{L2-Reg} & \textbf{Dropout} & \textbf{CF-Reg (ours)} \\ \hline
        Logistic Regression   & \textit{Water}   & $0.6595 \pm 0.0038$   & $0.6729 \pm 0.0056$   & $0.6756 \pm 0.0046$  & N/A    & $\mathbf{0.6918 \pm 0.0036}$                     \\ \hline
        MLP   & \textit{Water}   & $0.6756 \pm 0.0042$   & $0.6790 \pm 0.0058$   & $0.6790 \pm 0.0023$  & $0.6750 \pm 0.0036$    & $\mathbf{0.6802 \pm 0.0046}$                    \\ \hline
%        MLP   & \textit{Adult}   & $0.8404 \pm 0.0010$   & $\mathbf{0.8495 \pm 0.0007}$   & $0.8489 \pm 0.0014$  & $\mathbf{0.8495 \pm 0.0016}$     & $0.8449 \pm 0.0019$                    \\ \hline
        Logistic Regression   & \textit{Phomene}   & $\mathbf{0.8148 \pm 0.0020}$   & $0.8041 \pm 0.0028$   & $0.7835 \pm 0.0176$  & N/A    & $0.8098 \pm 0.0055$                     \\ \hline
        MLP   & \textit{Phomene}   & $0.8677 \pm 0.0033$   & $0.8374 \pm 0.0080$   & $0.8673 \pm 0.0045$  & $0.8672 \pm 0.0042$     & $\mathbf{0.8718 \pm 0.0040}$                    \\ \hline
        CNN   & \textit{CIFAR-10} & $0.6670 \pm 0.0233$   & $0.6229 \pm 0.0850$   & $0.7348 \pm 0.0365$   & N/A    & $\mathbf{0.7427 \pm 0.0571}$                     \\ \hline
    \end{tabular}
\end{table*}

\begin{table*}[htb!]
    \centering
    \caption{Hyperparameter configurations utilized for the generation of Table \ref{tab:regularization_comparison}. For our regularization the hyperparameters are reported as $\mathbf{\alpha/\beta}$.}
    \label{tab:performance_parameters}
    \begin{tabular}{|c|c|c|c|c|c|c|}
        \hline
        \textbf{Model} & \textbf{Dataset} & \textbf{No-Reg} & \textbf{L1-Reg} & \textbf{L2-Reg} & \textbf{Dropout} & \textbf{CF-Reg (ours)} \\ \hline
        Logistic Regression   & \textit{Water}   & N/A   & $0.0093$   & $0.6927$  & N/A    & $0.3791/1.0355$                     \\ \hline
        MLP   & \textit{Water}   & N/A   & $0.0007$   & $0.0022$  & $0.0002$    & $0.2567/1.9775$                    \\ \hline
        Logistic Regression   &
        \textit{Phomene}   & N/A   & $0.0097$   & $0.7979$  & N/A    & $0.0571/1.8516$                     \\ \hline
        MLP   & \textit{Phomene}   & N/A   & $0.0007$   & $4.24\cdot10^{-5}$  & $0.0015$    & $0.0516/2.2700$                    \\ \hline
       % MLP   & \textit{Adult}   & N/A   & $0.0018$   & $0.0018$  & $0.0601$     & $0.0764/2.2068$                    \\ \hline
        CNN   & \textit{CIFAR-10} & N/A   & $0.0050$   & $0.0864$ & N/A    & $0.3018/
        2.1502$                     \\ \hline
    \end{tabular}
\end{table*}

\begin{table*}[htb!]
    \centering
    \caption{Mean value and standard deviation of training time across 5 different runs. The reported time (in seconds) corresponds to the generation of each entry in Table \ref{tab:regularization_comparison}. Times are }
    \label{tab:times}
    \begin{tabular}{|c|c|c|c|c|c|c|}
        \hline
        \textbf{Model} & \textbf{Dataset} & \textbf{No-Reg} & \textbf{L1-Reg} & \textbf{L2-Reg} & \textbf{Dropout} & \textbf{CF-Reg (ours)} \\ \hline
        Logistic Regression   & \textit{Water}   & $222.98 \pm 1.07$   & $239.94 \pm 2.59$   & $241.60 \pm 1.88$  & N/A    & $251.50 \pm 1.93$                     \\ \hline
        MLP   & \textit{Water}   & $225.71 \pm 3.85$   & $250.13 \pm 4.44$   & $255.78 \pm 2.38$  & $237.83 \pm 3.45$    & $266.48 \pm 3.46$                    \\ \hline
        Logistic Regression   & \textit{Phomene}   & $266.39 \pm 0.82$ & $367.52 \pm 6.85$   & $361.69 \pm 4.04$  & N/A   & $310.48 \pm 0.76$                    \\ \hline
        MLP   &
        \textit{Phomene} & $335.62 \pm 1.77$   & $390.86 \pm 2.11$   & $393.96 \pm 1.95$ & $363.51 \pm 5.07$    & $403.14 \pm 1.92$                     \\ \hline
       % MLP   & \textit{Adult}   & N/A   & $0.0018$   & $0.0018$  & $0.0601$     & $0.0764/2.2068$                    \\ \hline
        CNN   & \textit{CIFAR-10} & $370.09 \pm 0.18$   & $395.71 \pm 0.55$   & $401.38 \pm 0.16$ & N/A    & $1287.8 \pm 0.26$                     \\ \hline
    \end{tabular}
\end{table*}

\subsection{Feasibility of our Method}
A crucial requirement for any regularization technique is that it should impose minimal impact on the overall training process.
In this respect, CF-Reg introduces an overhead that depends on the time required to find the optimal counterfactual example for each training instance. 
As such, the more sophisticated the counterfactual generator model probed during training the higher would be the time required. However, a more advanced counterfactual generator might provide a more effective regularization. We discuss this trade-off in more details in Section~\ref{sec:discussion}.

Table~\ref{tab:times} presents the average training time ($\pm$ standard deviation) for each model and dataset combination listed in Table~\ref{tab:regularization_comparison}.
We can observe that the higher accuracy achieved by CF-Reg using the score-based counterfactual generator comes with only minimal overhead. However, when applied to deep neural networks with many hidden layers, such as \textit{PreactResNet-18}, the forward derivative computation required for the linearization of the network introduces a more noticeable computational cost, explaining the longer training times in the table.

\subsection{Hyperparameter Sensitivity Analysis}
The proposed counterfactual regularization technique relies on two key hyperparameters: $\alpha$ and $\beta$. The former is intrinsic to the loss formulation defined in (\ref{eq:cf-train}), while the latter is closely tied to the choice of the score-based counterfactual explanation method used.

Figure~\ref{fig:test_alpha_beta} illustrates how the test accuracy of an MLP trained on the \textit{Water Potability} dataset changes for different combinations of $\alpha$ and $\beta$.

\begin{figure}[ht]
    \centering
    \includegraphics[width=0.85\linewidth]{img/test_acc_alpha_beta.png}
    \caption{The test accuracy of an MLP trained on the \textit{Water Potability} dataset, evaluated while varying the weight of our counterfactual regularizer ($\alpha$) for different values of $\beta$.}
    \label{fig:test_alpha_beta}
\end{figure}

We observe that, for a fixed $\beta$, increasing the weight of our counterfactual regularizer ($\alpha$) can slightly improve test accuracy until a sudden drop is noticed for $\alpha > 0.1$.
This behavior was expected, as the impact of our penalty, like any regularization term, can be disruptive if not properly controlled.

Moreover, this finding further demonstrates that our regularization method, CF-Reg, is inherently data-driven. Therefore, it requires specific fine-tuning based on the combination of the model and dataset at hand.

\subsection{Training Dataset}
In Table~\ref{tab:data_statistics}, we provide the remaining instances after processing each phase of our generated datasets.
We target datasets with three conditions: (1) widely used as SFT datasets; (2) do not contain the system messages; (3) diverse domains are covered. 
We enumerate the selected datasets as follows: 
(1) Capybara~\citep{daniele2023amplify-instruct}, which focuses on information diversity across a wide range of domains.
(2) Airoboros~\citep{airoboros3.1} is composed of multi-step instructions with a diverse structured format.
(3) Orcamath~\citep{mitra2024orcamath} aims to provide various mathematical problem solving.
(4) MetamathQA~\citep{yu2023metamath} is an augmented version of several math instructions.
(5) Magicoder~\citep{luo2023wizardcoder} dataset provides various code generation problems.
We provide detailed statistics in Appendix~\ref{app:data_statistics_appendix}.




\subsection{Evaluation Benchmarks}
% Our primary goal is to ensure that the system messages generated using our \textsc{SysGen} piepline lead to less performance degradation on existing benchmarks while assessing how effectively they function in system-related benchmarks.
% To achieve this, we evaluate performance on  Multifacet~\citep{lee2024aligning}, which requires both a system message and a user instruction to generate the model response.
We evaluate performance on  Multifacet~\citep{lee2024aligning}, which requires both the system messages and the user instructions to generate the assistant responses.
For the source data, the Multifacet benchmark is constructed of approximately 921 samples by incorporating AlpacaEval~\citep{dubois2024length}, FLASK~\citep{ye2023flask}, MT-bench~\citep{bai2024mt}, Koala~\citep{koala_blogpost_2023}, and Self-Instruct~\citep{wang2022self}.
The authors of~\citet{lee2024aligning} set the multiple aspects of evaluating each response with four dimensions: style, background information, harmlessness, and informativeness.
We follow these evaluation settings in our experiments.

Additionally, we aim to investigate the impact of the \textsc{SysGen} data on unseen benchmarks by leveraging the Open LLM Leaderboard 2~\citep{myrzakhan2024open} as a test set.
The test set is composed of MMLU~\citep{hendrycks2020measuring}, MMLU-pro~\citep{wang2024mmlu}, Arc-challenge~\citep{clark2018think}, GPQA~\citep{rein2023gpqa}, HellaSwag~\citep{zellers2019hellaswag}, IFEVAL~\citep{zhou2023instruction}, MATHQA~\citep{amini-etal-2019-mathqa}, and BBH~\citep{suzgun2023challenging}.
We use the publicly available lm-evaluation harness~\citep{eval-harness} as an evaluation tool for a fair comparison.



\subsection{Open-source Models}
Our baseline models are composed of instruction-tuned open-source models and trained with supervised fine-tuning datasets without system messages.
We select and utilize one from each widely used open-source model family: 
(1) Solar-10.7B-instruct~\citep{kim-etal-2024-solar} (2) Gemma-2-9B-instruct~\citep{team2024gemma} (3) LLaMA-3.1-8B-instruct~\citep{meta2024introducing} (4) Qwen2.5-14B-instruct~\citep{yang2025qwen2}, and (5) Phi-4~\citep{abdin2024phi}.

\section{Experiments}
\begin{table}[t]
\centering
{\resizebox{\columnwidth}{!}{
\begin{tabular}{lccccccc}
\toprule
\multicolumn{1}{c}{\multirow{2}{*}{\textbf{Model}}} & \multirow{2}{*}{\begin{tabular}[c]{@{}c@{}}\textbf{Parameter}\\ \textbf{Scale}\end{tabular}}  & \multicolumn{5}{c}{\textbf{Multifacet}} & \multirow{2}{*}{\textbf{Average}} \\ \cmidrule{3-7}  
\multicolumn{1}{c}{} &  & \multicolumn{1}{l}{\textbf{AE}} & \multicolumn{1}{l}{\textbf{FL}} & \multicolumn{1}{l}{\textbf{Ko}} & \multicolumn{1}{l}{\textbf{MT}} & \multicolumn{1}{l}{\textbf{SI}} & \\ \midrule
\textit{Open-Source Models} \\ \midrule
Solar-10.7B-instruct & 10.7B & 3.30 & 3.31 & 3.09 & 3.19 & 3.08 & 3.19  \\  
Gemma-2-9b-it & 9B & 4.10 & 3.80 & 4.26 & 4.15 & 3.92 & 4.05  \\ 
\midrule
\multicolumn{8}{l}{\textit{Open-source Models} $+$ \textit{KD (Fine-tuning on \textbf{\textsc{SysGen}} dataset)}} \\
\midrule 
Solar-10.7B-instruct & 10.7B & 3.97 & 3.73 & 3.64 & 3.98 & 3.52 & 3.76 (+0.57) \\ 
Gemma-2-9b-it & 9B & 4.40 & 4.04 & 4.30 & 4.23 & 4.18 & 4.23 (+0.18) \\ 
\bottomrule
\end{tabular}}}
\caption{
We conduct a knowledge distillation (KD) experiments leveraging data generated by \textsc{SysGen} pipeline using Phi-4.}
\label{tab:knowledge_distillation}
\vspace{-0.3cm}
\end{table}
\begin{table*}[t]
\centering
{\resizebox{\textwidth}{!}{
\begin{tabular}{lcccccccccc}
\toprule
\multicolumn{1}{c}{\multirow{2}{*}{\textbf{Model}}} & \multirow{2}{*}{\begin{tabular}[c]{@{}c@{}}\textbf{Parameter}\\ \textbf{Scale}\end{tabular}}  & \multicolumn{8}{c}{\textbf{Unseen Benchmarks}} & \multirow{2}{*}{\textbf{Average}} \\  \cmidrule{3-10} 
\multicolumn{1}{c}{} &  & \multicolumn{1}{l}{\textbf{MMLU}} & \multicolumn{1}{l}{\textbf{MMLU-Pro}} & \multicolumn{1}{l}{\textbf{ARC-c}} & \multicolumn{1}{l}{\textbf{GPQA}} & \multicolumn{1}{l}{\textbf{HellaSwag}} & \multicolumn{1}{l}{\textbf{IFEVAL}} & \textbf{MATHQA} & \textbf{BBH} & \\ \midrule
\multicolumn{11}{l}{\textit{Open-Source Models}} \\ \midrule 
Solar-10.7B-instruct & 10.7B  &  63.28 & 30.20 & 63.99 & 30.36 & 86.35 & 38.59 &  36.38 & 37.28 & 48.31 \\
Gemma-2-9b-it & 9B & 73.27 & 32.78 & 67.89 & 31.05 & 81.92 & 74.78 & 38.87 & 41.98 & 55.31 \\ 
LLaMA-3.1-8B-instruct & 8B & 67.95 & 40.87 & 54.95 & 34.60 & 79.18 & 50.71 & 39.53 & 70.85 & 54.83 \\ 
% Mixtral-8x22B-instruct & 8x22B & 75.62 &  52.63 & 67.83 & 36.83 & 87.68 & 60.43 & 50.08 & 83.03 & \\   
Qwen2.5-14B-instruct & 14B &  79.73  & 51.22 & 67.39 & 45.51 & 82.31 & 79.83 & 42.12 & 78.25 & 65.79 \\ 
Phi-4 & 14B & 84.56 & 70.12  & 68.26 & 55.93 & 84.42 & 62.98 & 48.87 & 79.87 & 69.37 \\  
\midrule
\multicolumn{11}{l}{\textit{Open-Source Models (Fine-tuning on original SFT Dataset)}} \\ \midrule
Solar-10.7B-instruct & 10.7B & 62.38 & 29.12 & 58.87 & 29.17 & 81.58 & 31.27 & 37.21 & 32.85 & 45.30 (-3.01) \\ 
Gemma-2-9b-it & 9B & 71.85 & 31.67  & 62.57 & 30.51 & 77.54 & 69.25 & 39.12 & 37.25 & 52.47 (-2.84) \\ 
LLaMA-3.1-8B-instruct & 8B & 65.34 &  36.85 &  
54.18 & 33.93 & 77.98 & 35.64 & 40.03 & 62.83 & 50.85 (-3.98)  \\ 
% Mixtral-8x22B-instruct & 8x22B&  &   & & & & & &  & \\ 
Qwen2.5-14B-instruct & 14B & 75.87  & 49.85  & 66.89 & 43.98 & 80.99 & 62.57 & 43.28 & 71.17 & 61.82 (-3.97) \\ 
Phi-4 & 14B & 80.27 & 66.58  & 66.27 & 52.89 & 83.39 & 55.83 & 49.98 & 75.49 & 66.33 (-6.04) \\ 
\midrule
\multicolumn{11}{l}{\textit{Open-Source Models (Fine-tuning on \textbf{\textsc{SysGen}} dataset)}} \\ \midrule 
LLaMA-3.1-8B-instruct & 8B & 66.89 & 39.77 & 54.55 & 34.21 & 78.89 & 46.75 & 42.11 & 68.98 & 54.02 (-0.81) \\ 
% Gemma-2-9B-instruct & 9B &  &   & & & & & &  & \\ 
% Mixtral-8x22B-instruct & 8x22B&  &   & & & & & &  & \\ 
% Solar-10.7B-instruct & 10.7B & 63.28 & 30.20 & 63.99 & 30.36 & 86.35 & 38.59 &  36.38 & 37.28 & \\ 
Qwen2.5-14B-instruct & 14B & 78.92 & 43.38 & 66.82 & 44.46 & 80.98 & 74.59 & 43.23 & 76.28 & 63.58 (-2.20) \\ 
Phi-4 & 14B & 83.27 & 68.77  & 67.89 & 55.18 & 84.31 & 57.87 & 50.23 & 77.12 & 68.08 (-1.29) \\ 
\midrule
\multicolumn{11}{l}{\textit{Open-source Models} $+$ \textit{Knowledge Distillation (Fine-tuning on \textbf{\textsc{SysGen}} dataset))}} \\
\midrule 
Solar-10.7B-instruct & 10.7B & 59.98  & 29.26  & 62.81 & 30.25 & 85.91 & 34.58 & 38.25 & 35.97 & 47.12 (-1.19) \\ 
Gemma-2-9b-it & 9B & 72.19 & 31.56 & 66.75 & 30.89 & 81.53 & 71.37 & 40.27 & 40.38 & 54.37 (-0.94) \\ 
\bottomrule
\end{tabular}}}
\caption{We utilize the Open LLM Leaderboard 2 score as the unseen benchmark. This reveals the key finding that adding system messages to existing SFT datasets does not lead to significant performance degradation.}
\label{tab:unseen_experiments}
\end{table*}
The primary goal of \textsc{SysGen} pipeline is to enhance the utilization of the \emph{system role} while minimizing performance degradation on unseen benchmarks, thereby improving the effectiveness of supervised fine-tuning (SFT).
To validate this, we evaluate how well the models trained on \textsc{SysGen} data generate appropriate assistant responses given both the system messages and user instructions, using the Multifacet~\citep{lee2024aligning} dataset.
For models that cannot generate data independently, we apply knowledge distillation to assess their effectiveness.
Additionally, we leverage the widely used Open LLM Leaderboard 2~\citep{myrzakhan2024open} as an unseen benchmark to determine whether our approach can be effectively integrated into existing SFT workflows.


\paragraph{\textsc{SysGen} provides better system message and assistant response to align with user instructions.}
Given the system messages and user instructions, the assistant's response is evaluated across four dimensions: style, background knowledge, harmlessness, and informativeness.
Each of these four aspects is scored on a scale of 1 to 5 using a rubric, and the average score is presented as the final score for the given instruction.
As shown in Table~\ref{tab:main_experiments}, recent open-source models achieve comparable scores to the proprietary models, indicating that open-source models have already undergone training related to system roles~\citep{meta2024introducing, yang2024qwen2, abdin2024phi}.

When trained on \textsc{SysGen} data, both LLaMA (4.12 → 4.21) and Phi (4.41 → 4.54) show score improvements.
Among the four dimensions, LLaMA exhibits score increases in style (4.15 → 4.32) and harmlessness (4.23 → 4.29).
Similarly, Phi shows the improvements in style (4.42 → 4.61) and informativeness (4.37 → 4.49).
As a result, even open-source models that have already been trained on system roles demonstrate their positive effects on style, informativeness, and harmlessness.



\paragraph{Knowledge distillation through \textsc{SysGen} data.}
If an open-source model does not support the system roles, it may not generate the system messages properly using \textsc{SysGen} pipeline. 
However, the effectiveness of knowledge distillation, using data generated by another open-source model without the limitation, remains uncertain.
To explore this, we train Gemma~\citep{team2024gemma} and Solar~\citep{kim-etal-2024-solar} using data generated by Phi-4~\citep{abdin2024phi}.
We use the Phi-4 data because it preserves most of the data and provides high quality  assistant responses as shown in Table~\ref{tab:statistics_generated_answer} and \ref{tab:data_statistics}.

As shown in Table~\ref{tab:knowledge_distillation}, even for models that do not inherently support system roles, modifying the chat template to incorporate system role and training on knowledge distilled dataset leads to an improvement in Multifacet performance, as observed in Gemma (4.05 → 4.23).
We describe the details in the Appendix~\ref{app:system_role_support}.
Additionally, for the Solar model, which had not been trained on system roles, we observe a dramatic performance improvement (3.19 → 3.76).\footnote{We speculate that Solar model did not properly learn the system role because its initial Multifacet score was low.}
This demonstrates that the data generated by the \textsc{SysGen} pipeline effectively supports the system roles.


\paragraph{\textsc{SysGen} data minimizes the performance degradation in unseen benchmarks.}
When incorporating system messages that were not present in the original SFT datasets and modifying the corresponding assistant responses, it is crucial to ensure that the model’s existing performance should not degrade.
For example, one key consideration in post-training is maintaining the model's original performance.
To assess this, we observed performance difference in unseen benchmark after applying supervised fine-tuning.
As shown in Table~\ref{tab:unseen_experiments}, we use the Open LLM Leaderboard 2 dataset as an unseen benchmark, with performance categorized into four groups:
\begin{itemize}
    \item Performance of existing open-source models (row 1-6)
    \item Performance of fine-tuning with open-source models using SFT datasets (row 7-12)
    \item Performance of fine-tuning with \textsc{SysGen} data (row 13-16)
    \item Performance after applying knowledge distillation using Phi-4 \textsc{SysGen} data (row 17-19)
\end{itemize}
The average performance degradation reflects the scores missing from each open-source model's original performance (row 1-6).

When fine-tuning with independently generated data using \textsc{SysGen}, the performance degradation is significantly lower than fine-tuning with the original SFT datasets selected under the same conditions.
Additionally, even for models that cannot generate data independently (e.g., those that do not support system roles), knowledge distillation helps mitigate performance drops considerably.










\section{Analysis}
\subsection{What makes \textsc{SysGen} pipeline useful?}
\label{ana:sysgen_useful}
\begin{table}[t]
\centering
{\resizebox{\columnwidth}{!}{
\begin{tabular}{lcc}
\toprule
\multicolumn{1}{c}{\textbf{Models}}           & \begin{tabular}[c]{@{}c@{}}\textbf{Multifacet}\\ \textbf{(Average)}\end{tabular} & \begin{tabular}[c]{@{}c@{}}\textbf{Unseen Benchmarks}\\ \textbf{(Average)}\end{tabular} \\ \midrule
\textit{No System Message} \\ \midrule
LLaMA-3.1-8B-instruct  & 3.98 & 50.85 \\ 
Phi-4 & 4.26 & 66.33 \\ 
\midrule
\textit{Common System Message} \\ \midrule
LLaMA-3.1-8B-instruct  & 3.89 & 51.23 \\ 
Phi-4 & 4.23 & 66.52 \\
\midrule
\textit{\textsc{SysGen} without A'} \\ \midrule
LLaMA-3.1-8B-instruct  & 4.09 & 51.89 \\ 
Phi-4 & 4.38 & 66.12 \\ 
\midrule
\textit{\textsc{SysGen}} \\ \midrule
LLaMA-3.1-8B-instruct  & 4.21 & 54.02 \\ 
Phi-4 & 4.54 & 68.08 \\
\bottomrule
\end{tabular}}}
\caption{Ablation studies of using system message and assistant's response. Using a common system message or generated system message does not provide insightful difference. Newly-generated answer and its corresponding system message can increase system abilities with lower decrease in unseen benchmarks.}
\label{tab:compare_system_message}
\vspace{-0.3cm}
\end{table}
To assess the impact of system messages generated by \textsc{SysGen} during training, we conduct ablation studies on four different model variations:
\begin{itemize}
    \item No System Message: The original SFT dataset which does not contain the system message. 
    \item Common System Message: An $\mathcal{S}\mathcal{Q}\mathcal{A}$ triplet where the common system message is inserted such as "You are a helpful AI assistant".
    \item \textsc{SysGen} without $\mathcal{A'}$: An $\mathcal{S}\mathcal{Q}\mathcal{A}$ triplet that includes only a system message generated by our \textsc{SysGen} pipeline.
    \item \textsc{SysGen}: An $\mathcal{S}\mathcal{Q}\mathcal{A'}$ triplet where both the \textsc{SysGen}-generated system message and the newly-generated answer are incorporated.
\end{itemize}
We measure the effectiveness of these models by analyzing score variations on the Multifacet and unseen benchmarks in Table~\ref{tab:compare_system_message}.

Training with data that includes common system messages does not result in a significant performance difference compared to training without system messages.
This led us to question: \emph{"Would it be sufficient to include only the most suitable system messages?"}.
To explore this, we train models using data that contains only system messages generated by \textsc{SysGen} pipeline.
As a result, we observe an improvement in Multifacet performance for both models, while the scores on the unseen benchmark remained similar.
Furthermore, when both system messages and assistant responses generated by \textsc{SysGen} are used for fine-tuning, we observe performance improvements in both Multifacet evaluation and unseen benchmarks.

\subsection{System message vs. User instruction}
\begin{table}[t]
\centering
{\resizebox{\columnwidth}{!}{
\begin{tabular}{lc}
\toprule
\multicolumn{1}{c}{\textbf{Models}}           & \begin{tabular}[c]{@{}c@{}}\textbf{Multifacet Average}\\ \textbf{(Use system role → Use user role)}\end{tabular} \\ \midrule
\textit{Open-source Models} \\ \midrule
Solar-10.7B-instruct   & 3.19 → 2.98 \\ 
LLaMA-3.1-8B-instruct  & 4.12 → 4.09 \\ 
Qwen2.5-14b-instruct   & 4.26 → 4.13 \\ 
Phi-4 & 4.41 → 4.26 \\ 
\midrule
\multicolumn{2}{l}{\textit{Open-source Models} (with \textbf{\textsc{SysGen}})} \\ \midrule
LLaMA-3.1-8B-instruct  & 4.21 → 4.13 \\ 
Qwen2.5-14B-instruct   & 4.28 → 4.16 \\ 
Phi-4 & 4.54 → 4.38 \\
\midrule
\multicolumn{2}{l}{\textit{Open-source Models} $+$ KD  (with \textbf{\textsc{SysGen}})} \\ \midrule
Solar-10.7b-instruct   & 3.76 → 3.64 \\ 
\bottomrule
\end{tabular}}}
\caption{There is a tendency for the score to decrease when the system message is reflected in the user instruction. The more a model is trained on system messages, the better it is to place them in the system role. KD indicates the knowledge distillation.}
\label{tab:no_system_message}
% \vspace{-0.5cm}
\end{table}
A key question arises that \emph{what happens if we add a message intended for the system role at the beginning of the user instruction? Could it serve as a replacement for the system role?}
To explore this, we conduct an experiment on a Multifacet benchmark.
Specifically, we included messages that should typically be in the system role within the user instruction during inference.

As shown in Table~\ref{tab:no_system_message}, we observe that open-source models tend to experience score degradation when system role messages are incorporated into the user instruction.
This trend suggests that adding such content can make the query itself more ambiguous to answer.
Furthermore, even in models trained with our \textsc{SysGen}, this trend persists similarly to the previous work~\citep{lee2024aligning}.
Despite additional fine-tuning on system roles, scores still remain low when system messages are reflected in the user instruction.
This highlights the importance of properly placing these messages in the system role to maintain performance.
% This indicates that properly placing these messages in the system role remains crucial for maintaining performance.

\subsection{New assistant responses align to the system messages}
\begin{figure}[t]
\centering
\includegraphics[width=\columnwidth]{figure/aligment_message_response.pdf}
\caption{
The GPT4o LLM-as-a-judge results of measuring the alignment between generated system messages and new assistant responses. We use 20 samples for each data source which sums up to 100 samples in total per models.
}
\label{fig:alignment_message_response}
\vspace{-0.3cm}
\end{figure}
In Table~\ref{tab:statistics_generated_answer}, we presented that the new assistant responses exhibit similar n-gram matching, high semantic similarities, and verbosity.
Therefore, it is necessary to verify whether the generated assistant responses aligned with the system messages.
Figure~\ref{fig:alignment_message_response} illustrates the GPT-4o results using LLM-as-a-judge approach.
Through the three \textsc{SysGen} data generated by Phi-4, LLaMA, and Qwen models, we determined that all of the assistant responses are highly aligned with the system messages.
Overall, the experiments and analyses reveal that our \textsc{SysGen} data were generated to effectively respond to various user instructions as system messages.
In addition, we observed that the assistant responses align with the system messages and are capable of generating better aligned responses compared the original assistant responses.

\section{Conclusion}
This paper presents the first comprehensive study on the security risks of RACG systems, specifically the impact of vulnerable code in the knowledge base. Our experiments show that knowledge base poisoning significantly compromises the security of generated code, with up to 48\% of the code becoming vulnerable from a single poisoned sample in Scenario I with CodeLlama. We also identify that factors such as few-shot learning, programming language, and example-query similarity contribute to increased vulnerabilities. Our findings underline the need for improved knowledge base security and offer practical insights for reducing vulnerability propagation in LLM-generated code. This work pave the way for future works which aim at securing RACG systems and enhancing their reliability in software development.
\section*{Limitations}
While our \textsc{SysGen} pipeline demonstrates promising results in system messages alignment to the user instructions through Multifacet dataset.
However, our data construction pipeline only considers the single-turn conversation without handling multi-turn conversations~\citep{qin2024sysbench}.
We acknowledge that it is important for system messages to remain effective throughout multi-turn conversations, but our study focuses on evaluation and simple level of inference usage.

Additionally, our experimental results reveal that training with \textsc{SysGen} data shows minimal performance degradation on unseen benchmark, Open LLM Leaderboard 2 dataset.
However, we suspect that the observed performance drop may be due to the format of natural text that the SFT datasets we selected, rather than formats similar to multiple-choice questions commonly found in the unseen benchmark. 
Therefore, we are curious about how well the system messages could be generated in various formats such as True/False questions or Multiple Choice questions and prove its effectiveness.

Finally, in Table~\ref{app:tag_statistics}, we identify the special tokens of tags which are annotated to the publicly avaiable data.
The <<Tool>> tag has been absolutely shown small portion compared to other tags.
Our initial intention was to utilize the tag for generating data through search functionality or function calls. 
However, the selected public data deviated from this purpose, resulting in a very low proportion of the tag being generated.
Therefore, it would be beneficial to gather and generate data appropriately for each tag's intended use.

% \section*{Acknowledgments}


\bibliography{custom}

\appendix


% \section{Rubrics of Automatic Evaluation}


\section{Data Statistics}
\label{app:data_statistics_appendix}
\paragraph{Statistics of generated tags.} As we stated in limitations section, we provide the statistics of generate special tag tokens in Table~\ref{app:tag_statistics}.
We find out that most of the <<Role>>, <<Content>>, <<Task>> tokens are annotated in the instances.
Compared to thoses tokens, <<Action>>, <<Style>>, <<Background>>, and <<Format>> depends on the user instructions to be generated.
However, <<Tool>> tokens have shown absolutely low portion to be generated.
We thus want to suggest that properly choosing the public or your own dataset seems to ensure the <<Tool>> tag usages such as selecting searching protocols or function calls. 
\begin{table}[h]
{\resizebox{\columnwidth}{!}{
\begin{tabular}{lccc}
\toprule
Tags       & LLaMA-3.1-8B-instruct & Qwen2.5-14b-instruct & Phi-4   \\ \midrule
Role       & 576,341               & 753,579              & 745,751 \\ 
Content    & 580,231               & 739,892              & 743,311 \\ 
Task       & 579,558               & 765,331              & 735,298 \\ 
Action     & 495,301               & 382,358              & 662,589 \\ 
Style      & 283,579               & 598,553              & 603,918 \\ 
Background & 293,791               & 539,757              & 553,791 \\ 
Tool       & 10,238                & 132,038              & 90,989  \\ 
Format     & 327,909               & 401,593              & 538,973 \\ \bottomrule
\end{tabular}}}
\caption{Statistics of generated tags using \textsc{SysGen} pipeline.}
\label{app:tag_statistics}
\end{table}

\begin{table*}[t]
\centering
{\resizebox{\textwidth}{!}{
\begin{tabular}{lccccc}
\toprule
\multicolumn{1}{c}{Dataset} & \# of instances & Avg. Query Length & Avg. Answer Length & Containing System Message & Covering Domains\\ \midrule
Capybara  & 41,301          & 300.24 & 1423.28 & \xmark & reasoning, logic, subjects, conversations, pop-culture, STEM \\ 
Airoboros & 59,277          & 507.26 & 1110.62 & simple system message & mathematics, MATHJSON, character's descriptions \\ 
OrcaMath  & 200,035         & 238.87 & 878.43 & \xmark & school mathematics, math word problems \\ 
Magicoder & 111,183         & 652.53 & 1552.41 & \xmark & code solution \\  
MetaMath  & 395,000         & 213.53 & 498.24 & \xmark & mathematics \\  \bottomrule
\end{tabular}}}
\caption{Data statistics of SFT datasets. We provide the average length of query and answer, the presence of system messages, and covering domains.}
\label{tab:data_statistics_appendix}
% \vspace{-0.3cm}
\end{table*}


\paragraph{Statistics of original SFT datasets.}
In Table~\ref{tab:data_statistics_appendix}, we observe that most widely used public datasets either lack a system message entirely or include only a simple one, such as "You are a helpful AI assistant.".
The publicly available data mostly cover mathematics, code problems following some reasoning and logical ones.




\section{Experimental Details}
\label{app:experimental_details}
\paragraph{Computing Resources}
We use 4x8 NVIDIA H100 Tensor Core GPU with 80GB memory to train the open-source models.
We use Deepspeed stage 3~\citep{rajbhandari2020zero} to implement multi-GPU settings and FlashAttention~\citep{dao2022flashattention} for efficient training.
Our code is written in PyTorch~\citep{paszke2019pytorch} and HuggingFace~\citep{wolf2019huggingface}.


\paragraph{Integrating system roles in models that do not support them.}
\label{app:system_role_support}
Through our experiments, we find out that the Gemma-2-9b-it~\citep{team2024gemma} model does not inherently support the system role.
To address this limitation during data generation and training, we modified the chat template in the configuration of tokenization to remove restrictions on the system role.
Interestingly, despite the lack of native support, our findings show that \textsc{SysGen} data can still be utilized effectively to incorporate a system role into these models.

\section{Qualitative analysis of generated instances}
\label{app:qualitative_analysis}

\begin{table*}
    \scriptsize
    \centering
    \begin{NiceTabular}{@{}l@{\hskip4pt}p{0.92\textwidth}@{}}
    \CodeBefore
\cellcolor{gray!20}{1-1,1-2}
\cellcolor{gray!20}{5-1,5-2}
\cellcolor{gray!20}{9-1,9-2}
\Body
    \toprule
    \multicolumn{2}{c}{\textbf{Case 1: Under-prediction of SAR}}\\
    \midrule
    \textbf{Groundtruth} & \texttt{
    [...] 
Diaz
started
his
{\color{red}political career}
as a
{\color{red}member}
of
{\color{red}the Sangguniang Bayan}
{\color{red}(municipal council)}
of
{\color{red}Santa Cruz}
in
{\color{red}1978}.
He
later
became
{\color{red}the Vice Mayor}
of
{\color{red}Santa Cruz}
in
{\color{red}1980}
and
was elected
as
{\color{red}the town's Mayor}
in
{\color{red}1988}.
[...]
    }\\
    \hdashline
    \textbf{Likelihood} & \texttt{
    [...]
\adjustbox{bgcolor={red!61.3}}{\strut Diaz}
\adjustbox{bgcolor={red!95.4}}{\strut started}
\adjustbox{bgcolor={red!0.4}}{\strut his}
\adjustbox{bgcolor={red!7.2}}{\strut political career}
\adjustbox{bgcolor={red!35.3}}{\strut as a}
\adjustbox{bgcolor={red!67.7}}{\strut member}
\adjustbox{bgcolor={red!0.6}}{\strut of}
\adjustbox{bgcolor={red!21.2}}{\strut the Sangguniang Bayan}
\adjustbox{bgcolor={red!24.8}}{\strut (municipal council)}
\adjustbox{bgcolor={red!5.2}}{\strut of}
\adjustbox{bgcolor={red!23.9}}{\strut Santa Cruz}
\adjustbox{bgcolor={red!71.2}}{\strut in}
\adjustbox{bgcolor={red!81.2}}{\strut 1978}
\adjustbox{bgcolor={red!28.5}}{\strut .}
\adjustbox{bgcolor={red!15.2}}{\strut He}
\adjustbox{bgcolor={red!90.6}}{\strut later}
\adjustbox{bgcolor={red!89.7}}{\strut became}
\adjustbox{bgcolor={red!48.9}}{\strut the Vice Mayor}
\adjustbox{bgcolor={red!6.6}}{\strut of}
\adjustbox{bgcolor={red!14.3}}{\strut Santa Cruz}
\adjustbox{bgcolor={red!62.6}}{\strut in}
\adjustbox{bgcolor={red!45.6}}{\strut 1980}
\adjustbox{bgcolor={red!31.4}}{\strut and}
\adjustbox{bgcolor={red!61.7}}{\strut was elected}
\adjustbox{bgcolor={red!42.9}}{\strut as}
\adjustbox{bgcolor={red!61.5}}{\strut the town's Mayor}
\adjustbox{bgcolor={red!2.0}}{\strut in}
\adjustbox{bgcolor={red!37.2}}{\strut 1988}
\adjustbox{bgcolor={red!99.5}}{\strut .}
[...]
    }\\
    \hdashline
    \textbf{SAR} & \texttt{
    [...]
\adjustbox{bgcolor={red!64.0}}{\strut Diaz}
\adjustbox{bgcolor={red!31.4}}{\strut started}
\adjustbox{bgcolor={red!0.1}}{\strut his}
\adjustbox{bgcolor={red!2.4}}{\strut political career}
\adjustbox{bgcolor={red!6.3}}{\strut as a}
\adjustbox{bgcolor={red!16.0}}{\strut member}
\adjustbox{bgcolor={red!0.1}}{\strut of}
\adjustbox{bgcolor={red!13.8}}{\strut the Sangguniang Bayan}
\adjustbox{bgcolor={red!6.7}}{\strut (municipal council)}
\adjustbox{bgcolor={red!0.6}}{\strut of}
\adjustbox{bgcolor={red!7.0}}{\strut Santa Cruz}
\adjustbox{bgcolor={red!23.7}}{\strut in}
\adjustbox{bgcolor={red!74.5}}{\strut 1978}
\adjustbox{bgcolor={red!3.6}}{\strut .}
\adjustbox{bgcolor={red!8.3}}{\strut He}
\adjustbox{bgcolor={red!53.3}}{\strut later}
\adjustbox{bgcolor={red!35.6}}{\strut became}
\adjustbox{bgcolor={red!28.4}}{\strut the Vice Mayor}
\adjustbox{bgcolor={red!1.3}}{\strut of}
\adjustbox{bgcolor={red!4.1}}{\strut Santa Cruz}
\adjustbox{bgcolor={red!19.8}}{\strut in}
\adjustbox{bgcolor={red!38.4}}{\strut 1980}
\adjustbox{bgcolor={red!7.1}}{\strut and}
\adjustbox{bgcolor={red!23.8}}{\strut was elected}
\adjustbox{bgcolor={red!14.5}}{\strut as}
\adjustbox{bgcolor={red!23.2}}{\strut the town's Mayor}
\adjustbox{bgcolor={red!0.4}}{\strut in}
\adjustbox{bgcolor={red!29.5}}{\strut 1988}
\adjustbox{bgcolor={red!53.6}}{\strut .}
[...]
    }\\
\midrule
\multicolumn{2}{c}{\textbf{Case 2: The type-filter of Focus and the limitations of uncertainty scores}}\\
    \midrule
\textbf{Groundtruth} & \texttt{
Taral Hicks
is
an
American
actress
and
singer,
born
on
September 21, 1974,
in
{\color{red}The Bronx, New York}.
[...]
She
later
{\color{red}transitioned}
to
{\color{red}acting},
appearing in
films
such as
"A Bronx Tale"
(1993),
"Just Cause"
(1995),
and
"Belly"
(1998).
[...]
}\\
\hdashline
\textbf{Likelihood} & \texttt{
\adjustbox{bgcolor={red!27.5}}{\strut Taral Hicks}
\adjustbox{bgcolor={red!64.8}}{\strut is}
\adjustbox{bgcolor={red!43.8}}{\strut an}
\adjustbox{bgcolor={red!25.7}}{\strut American}
\adjustbox{bgcolor={red!98.8}}{\strut actress}
\adjustbox{bgcolor={red!49.1}}{\strut and}
\adjustbox{bgcolor={red!26.2}}{\strut singer}
\adjustbox{bgcolor={red!91.9}}{\strut ,}
\adjustbox{bgcolor={red!76.3}}{\strut born}
\adjustbox{bgcolor={red!56.8}}{\strut on}
\adjustbox{bgcolor={red!34.6}}{\strut September 21, 1974}
\adjustbox{bgcolor={red!71.4}}{\strut ,}
\adjustbox{bgcolor={red!9.1}}{\strut in}
\adjustbox{bgcolor={red!21.4}}{\strut The Bronx, New York}
\adjustbox{bgcolor={red!65.0}}{\strut .}
[...]
\adjustbox{bgcolor={red!75.1}}{\strut She}
\adjustbox{bgcolor={red!76.5}}{\strut later}
\adjustbox{bgcolor={red!13.8}}{\strut transitioned}
\adjustbox{bgcolor={red!86.7}}{\strut to}
\adjustbox{bgcolor={red!1.1}}{\strut acting}
\adjustbox{bgcolor={red!15.8}}{\strut ,}
\adjustbox{bgcolor={red!8.7}}{\strut appearing in}
\adjustbox{bgcolor={red!45.0}}{\strut films}
\adjustbox{bgcolor={red!10.6}}{\strut such as}
\adjustbox{bgcolor={red!48.7}}{\strut "A Bronx Tale"}
\adjustbox{bgcolor={red!18.1}}{\strut (1993),}
\adjustbox{bgcolor={red!47.3}}{\strut "Just Cause"}
\adjustbox{bgcolor={red!2.6}}{\strut (1995),}
\adjustbox{bgcolor={red!4.8}}{\strut and}
\adjustbox{bgcolor={red!46.6}}{\strut "Belly"}
\adjustbox{bgcolor={red!1.7}}{\strut (1998).}
[...]
}\\
\hdashline
\textbf{Focus} & \texttt{
\adjustbox{bgcolor={red!5.3}}{\strut Taral Hicks}
\adjustbox{bgcolor={red!0.0}}{\strut is}
\adjustbox{bgcolor={red!0.0}}{\strut an}
\adjustbox{bgcolor={red!23.5}}{\strut American}
\adjustbox{bgcolor={red!40.9}}{\strut actress}
\adjustbox{bgcolor={red!0.0}}{\strut and}
\adjustbox{bgcolor={red!30.4}}{\strut singer}
\adjustbox{bgcolor={red!0.0}}{\strut ,}
\adjustbox{bgcolor={red!0.0}}{\strut born}
\adjustbox{bgcolor={red!0.0}}{\strut on}
\adjustbox{bgcolor={red!31.9}}{\strut September 21, 1974}
\adjustbox{bgcolor={red!0.0}}{\strut ,}
\adjustbox{bgcolor={red!0.0}}{\strut in}
\adjustbox{bgcolor={red!21.0}}{\strut The Bronx, New York}
\adjustbox{bgcolor={red!0.0}}{\strut .}
[...]
\adjustbox{bgcolor={red!0.0}}{\strut She}
\adjustbox{bgcolor={red!0.0}}{\strut later}
\adjustbox{bgcolor={red!0.0}}{\strut transitioned}
\adjustbox{bgcolor={red!0.0}}{\strut to}
\adjustbox{bgcolor={red!32.2}}{\strut acting}
\adjustbox{bgcolor={red!0.0}}{\strut ,}
\adjustbox{bgcolor={red!0.0}}{\strut appearing in}
\adjustbox{bgcolor={red!33.2}}{\strut films}
\adjustbox{bgcolor={red!0.0}}{\strut such as}
\adjustbox{bgcolor={red!43.6}}{\strut "A Bronx Tale"}
\adjustbox{bgcolor={red!25.3}}{\strut (1993),}
\adjustbox{bgcolor={red!42.5}}{\strut "Just Cause"}
\adjustbox{bgcolor={red!30.1}}{\strut (1995),}
\adjustbox{bgcolor={red!0.0}}{\strut and}
\adjustbox{bgcolor={red!44.2}}{\strut "Belly"}
\adjustbox{bgcolor={red!30.1}}{\strut (1998).}
[...]
}\\
\midrule
\multicolumn{2}{c}{\textbf{Case 3: Uncertainty propagation of Focus}}\\
    \midrule
    \textbf{Groundtruth} & \texttt{
    [...]
Fernandinho
began
his
professional career
with
{\color{red}Atletico Paranaense}
in
{\color{red}Brazil}
before
{\color{red}moving}
to
{\color{red}Ukrainian club}
{\color{red}Shakhtar Donetsk}
in
{\color{red}2005}.
[...]
He
is known
for
his
{\color{red}physicality},
{\color{red}tackling ability},
and
{\color{red}passing range},
and
is
{\color{red}widely regarded}
as
{\color{red}one of the best}
{\color{red}defensive midfielders}
in
{\color{red}the world}.
    }\\
    \hdashline
    \textbf{Likelihood} & \texttt{
    [...]
\adjustbox{bgcolor={red!23.6}}{\strut Fernandinho}
\adjustbox{bgcolor={red!78.3}}{\strut began}
\adjustbox{bgcolor={red!3.3}}{\strut his}
\adjustbox{bgcolor={red!47.7}}{\strut professional career}
\adjustbox{bgcolor={red!60.5}}{\strut with}
\adjustbox{bgcolor={red!25.3}}{\strut Atletico Paranaense}
\adjustbox{bgcolor={red!42.1}}{\strut in}
\adjustbox{bgcolor={red!97.7}}{\strut Brazil}
\adjustbox{bgcolor={red!91.5}}{\strut before}
\adjustbox{bgcolor={red!40.5}}{\strut moving}
\adjustbox{bgcolor={red!9.1}}{\strut to}
\adjustbox{bgcolor={red!65.8}}{\strut Ukrainian club}
\adjustbox{bgcolor={red!3.2}}{\strut Shakhtar Donetsk}
\adjustbox{bgcolor={red!20.1}}{\strut in}
\adjustbox{bgcolor={red!1.5}}{\strut 2005}
\adjustbox{bgcolor={red!19.5}}{\strut .}
[...]
\adjustbox{bgcolor={red!34.9}}{\strut He}
\adjustbox{bgcolor={red!28.2}}{\strut is known}
\adjustbox{bgcolor={red!0.0}}{\strut for}
\adjustbox{bgcolor={red!0.0}}{\strut his}
\adjustbox{bgcolor={red!61.0}}{\strut physicality}
\adjustbox{bgcolor={red!8.6}}{\strut ,}
\adjustbox{bgcolor={red!77.0}}{\strut tackling ability}
\adjustbox{bgcolor={red!0.6}}{\strut ,}
\adjustbox{bgcolor={red!12.7}}{\strut and}
\adjustbox{bgcolor={red!53.7}}{\strut passing range}
\adjustbox{bgcolor={red!22.5}}{\strut ,}
\adjustbox{bgcolor={red!34.9}}{\strut and}
\adjustbox{bgcolor={red!61.7}}{\strut is}
\adjustbox{bgcolor={red!50.0}}{\strut widely regarded}
\adjustbox{bgcolor={red!0.0}}{\strut as}
\adjustbox{bgcolor={red!0.8}}{\strut one of the best}
\adjustbox{bgcolor={red!3.1}}{\strut defensive midfielders}
\adjustbox{bgcolor={red!2.0}}{\strut in}
\adjustbox{bgcolor={red!1.5}}{\strut the world}
\adjustbox{bgcolor={red!27.0}}{\strut .}
    }\\
    \hdashline
    \textbf{Focus} & \texttt{
    [...]
\adjustbox{bgcolor={red!28.8}}{\strut Fernandinho}
\adjustbox{bgcolor={red!0.0}}{\strut began}
\adjustbox{bgcolor={red!0.0}}{\strut his}
\adjustbox{bgcolor={red!16.9}}{\strut professional career}
\adjustbox{bgcolor={red!0.0}}{\strut with}
\adjustbox{bgcolor={red!36.5}}{\strut Atletico Paranaense}
\adjustbox{bgcolor={red!0.0}}{\strut in}
\adjustbox{bgcolor={red!35.3}}{\strut Brazil}
\adjustbox{bgcolor={red!0.0}}{\strut before}
\adjustbox{bgcolor={red!0.0}}{\strut moving}
\adjustbox{bgcolor={red!0.0}}{\strut to}
\adjustbox{bgcolor={red!38.8}}{\strut Ukrainian club}
\adjustbox{bgcolor={red!31.1}}{\strut Shakhtar Donetsk}
\adjustbox{bgcolor={red!0.0}}{\strut in}
\adjustbox{bgcolor={red!31.3}}{\strut 2005}
\adjustbox{bgcolor={red!0.0}}{\strut .}
[...]
\adjustbox{bgcolor={red!0.0}}{\strut He}
\adjustbox{bgcolor={red!0.0}}{\strut is known}
\adjustbox{bgcolor={red!0.0}}{\strut for}
\adjustbox{bgcolor={red!0.0}}{\strut his}
\adjustbox{bgcolor={red!36.0}}{\strut physicality}
\adjustbox{bgcolor={red!0.0}}{\strut ,}
\adjustbox{bgcolor={red!15.3}}{\strut tackling ability}
\adjustbox{bgcolor={red!0.0}}{\strut ,}
\adjustbox{bgcolor={red!0.0}}{\strut and}
\adjustbox{bgcolor={red!15.3}}{\strut passing range}
\adjustbox{bgcolor={red!0.0}}{\strut ,}
\adjustbox{bgcolor={red!0.0}}{\strut and}
\adjustbox{bgcolor={red!0.0}}{\strut is}
\adjustbox{bgcolor={red!0.0}}{\strut widely regarded}
\adjustbox{bgcolor={red!0.0}}{\strut as}
\adjustbox{bgcolor={red!7.2}}{\strut one of the best}
\adjustbox{bgcolor={red!18.8}}{\strut defensive midfielders} 
\adjustbox{bgcolor={red!0.0}}{\strut in}
\adjustbox{bgcolor={red!14.5}}{\strut the world}
\adjustbox{bgcolor={red!0.0}}{\strut .}
    }\\
\bottomrule
    \end{NiceTabular}
    \vspace{-0.5pc}
    \caption{We sampled 3 instances from our dataset to demonstrate the differences across uncertainty scores. For label, entities colored in {\color{red}red} indicate hallucination. For uncertainty scores, entities \colorbox{red!50}{boxed in red} with different tints represent the degree of uncertainty. A lighter (darker) box indicates a lower (higher) uncertainty.}
    \label{tb:qualitative_analysis}
    \vspace{-1pc}
\end{table*}
In Table~\ref{app:qualitative_analysis}, we provide the \textsc{SysGen} data by presenting the system messages, user instructions, and new assistant responses.
We observe that providing a specific format such as answer with paragraph format steers the LLM's behavior to answer in step-by-step processes within paragraph.
Also, if conversational example was provided, then the phrase of style tag forces to generate assistant response friendly.
Furthermore, if the system message grant specific roles such as a knowledgeable assistant, then the new assistant responses tend to generate verbose answers to the user instructions.

\section{Prompts}
\label{app:prompt}
To enhance reproducibility and facilitate understanding of the \textsc{SysGen} pipeline, we provide multiple prompts that we utilized.
In Table~\ref{tab:app_prompt_system_generation}, we use three-shot demonstrations to generate useful system messages which are collected through real-world scenarios.
The \textit{Conversational History} written in the prompt is composed of user instructions and original assistant responses.
Thus, given the user instructions and assistant responses, we generate the system messages at a phrase level containing eight functionalities with special tokens such as <<Role>>, <<Content>>, and <<Style>>.

After generating the system messages, in Table~\ref{tab:app_prompt_system_tag_check}, we verify the quality of each tag with three classes: Good, Bad, and None.
We want to note that the \textit{Annotated system messages}, composed of phrases and tags, are used to verify the \textit{Filtered system messages}.
By utilizing LLM-as-a-judge approach, we could save tremendous budgets through self-model feedbacks rather than using proprietary models (i.e., API Calls). 
Through our preliminary experiment, we observe that current open-source models such as Phi-4 or Qwen2.5-14b-instruct could preserve most of the phrases after applying phase 3.

Table~\ref{tab:app_prompt_answer_quality_check} shows the prompt of how we verify the quality of new assistant responses as shown in Figure~\ref{fig:answer_comparison}.
After prompting 1K randomly sampled instances, we observe that new assistant responses were qualified to be better aligned with user instructions.
\begin{table*}[t]
\centering
{\resizebox{\textwidth}{!}{
\begin{tabular}{l}
\toprule \midrule
\begin{tabular}[c]{@{}l@{}}
System: \\
Given a conversation history between user's question and assistant's response, \\ you are a system prompt generation assistant to generate a relevant system prompt. \\ 
The following [System Prompt] seems to have a mix of 8 different [functionalities]: \\ <Tasks>, <Tools>, <Style>, <Action>, <Content>, <Background>, <Role>, and <Format>. \\
Try to annotate each functionality within the system prompt in a phrase-level. 
Annotate each tag of functionalities. \\ Generate [Generated System Prompt] with a same language used in [Conversational History]. \\ \\

\#\# [Functionalities] \\
1. <<Task>>: what tasks will be performed? \\
2. <<Tool>>: What features or tools are available to integrate and use? \\
3. <<Style>>: What style of communication would you prefer for responses? \\
4. <<Action>>: Perform a specific action \\
5. <<Content>>:  Specifies the content that needs to be included in the response \\
6. <<Background>>:  Provides specific background information to ensure the model’s responses align with these settings. \\
7. <<Role>>:  Specifies the role, profession, or identity that needs to be played. \\
8. <<Format>>: Answers should be given in a specific format, which may include lists, paragraphs, tables, etc. \\ \\
User: \\
\#\# [Few-shot Examples of System Prompt] \\
\#\#\# 1 \\
<<Role>>You are an expert data augmentation system<</Role>> <<Task>>for korean text correction model training.<</Task>> \\<<Task>>Generate a pairs of data augmentation example.<</Task>> \\<<Background>>You are an intelligence AI model Solar-pro invented by Upstage AI.<</Background>> \\ \\

Instructions: \\
<<Content>>- In a given text, create 1\~3 typos.<</Content>> \\
<<Content>>- Typos can be reversed, misplaced, missing, duplicated, or misspaced letters.<</Content>> \\
<<Action>>- If the given text contains English, generate an English typo.<</Action>> \\
<<Action>>- Generate the results in the Output JSON format below.<</Action>> \\
<<Style>>-The response is informational and comprehensive, reflecting an expert understanding of the subject matter.<</Style>> \\

<<Format>> Output JSON format: \{ \\
"$original\_expression$": $ORIGINAL\_EXPRESSION$, \\
"$typo\_expression$": $TYPO\_EXPRESSION$
\} \\ <</Format>> \\ \\

\#\#\# 2 \\
<<Role>>You are an AI meeting note-taking assistant.<</Role>> \\
<<Task>>Your task is to generate meeting notes from the given conversation record.<</Task>> \\
<<Style>>All responses must be in Korean.<</Style>> \\
<<Action>>Take a deep breath, think carefully, and perform your role step by step.<</Action>> \\ \\

\#\#\# 3 \\
<<Role>>You are a chatbot of the Ministry of Food and Drug Safety (MFDS).<</Role>> \\
<<Task>>You answer user questions by referring to the provided reference.<</Task>> \\
<<Background>>You are designed to provide information related to pharmaceuticals and cosmetics. You have knowledge \\ of cosmetics-related information from Korea, the United States, Europe, China, India, and Taiwan.<</Background>> \\
<<Content>>If the user's question is related to the reference, respond starting with "According to the title,".<</Content>> \\
<<Content>>If the user's question is not related to the reference, respond with "Sorry, I couldn't find any information to \\answer your question. Please try asking again."<</Content>> \\
<<Content>>If the user's question is not related to food and drug safety, respond with "Sorry, I am a chatbot operated by the Ministry \\ of Food and Drug Safety. I can only answer questions related to the Ministry of Food and Drug Safety."<</Content>> \\

<<Style>>Respond to the user's questions kindly.<</Style>> \\
<<Background>>The reference is provided as context.<</Background>> \\ \\

\textit{Conversational History}
\end{tabular} \\ \midrule
\bottomrule
\end{tabular}}}
\caption{The prompt of generating system messages using open-source models. \textit{Italic} text part such as ``\textit{Conversational History}'' is filled with input text.}
\label{tab:app_prompt_system_generation}
\end{table*}
\begin{table*}[t]
\centering
{\resizebox{\textwidth}{!}{
\begin{tabular}{l}
\toprule \midrule
\begin{tabular}[c]{@{}l@{}}
System: \\
You are a functionality verifier assistant evaluating whether system messages are properly tagged according to the descriptions of 8 functionalities. \\
Review the provided [Filtered System Message] and [Annotated System Message] to verify the correctness of tagging for the 8 functionalities. \\ \\
Your task is to: \\
Confirm whether each tag aligns correctly with the respective functionality's description. \\
If a tag is properly generated and annotated, mark it as "Good". \\
If a tag exists but does not align with its functionality, mark it as "Bad". \\
If a tag is missing, mark it as "None" \\ \\

\#\# [Functionalities] \\
1. <<Task>>: what tasks will be performed? \\
2. <<Tool>>: What features or tools are available to integrate and use? \\
3. <<Style>>: What style of communication would you prefer for responses? \\
4. <<Action>>: Perform a specific action \\
5. <<Content>>:  Specifies the content that needs to be included in the response \\
6. <<Background>>:  Provides specific background information to ensure the model’s responses align with these settings. \\
7. <<Role>>:  Specifies the role, profession, or identity that needs to be played. \\
8. <<Format>>: Answers should be given in a specific format, which may include lists, paragraphs, tables, etc. \\ \\

\#\# [Expected Output Format] \\
<<Task>>: Good \\
<<Tool>>: None \\
<<Style>>: Good \\
<<Action>>: Good \\
<<Content>>: Bad \\
<<Background>>: Bad \\
<<Role>>: Bad \\
<<Format>>: Good \\ \\
User: \\
\#\# [Filtered System Message] \\
\textit{Filtered system messages} \\ \\
\#\# [Annotated System Message] \\
\textit{Annotated system messages} \\ \\
\#\# [Expected Output Format] \\
\end{tabular} \\ \midrule
\bottomrule
\end{tabular}}}
\caption{The prompt of verification of key functionalities (phase 3) using open-source models with annotated system messages and filtered system messages. \textit{Italic} text part is filled with input text.}
\label{tab:app_prompt_system_tag_check}
% \vspace{-0.3cm}
\end{table*}
\begin{table*}[t]
\centering
{\resizebox{\textwidth}{!}{
\begin{tabular}{l}
\toprule \midrule
\begin{tabular}[c]{@{}l@{}}The user instruction will be provided, along with two assistant responses.\\
Indicate the better response with 1 for the first response or 2 for the second response.\\ \\
User Instruction: {\textit{User Instruction}}\\ 
Assistant Response 1: {\textit{Original Answer}} \\ 
Assistant Response 2: {\textit{Newly-generated Answer}} \\ 
Which of the above two responses better adheres to the instruction? (Respond with 1 or 2)
\end{tabular} \\ \midrule
\bottomrule
\end{tabular}}}
\caption{The prompt of answer quality check through the proprietary model (e.g., GPT4o). \textit{Italic} text part is filled with input text.}
\label{tab:app_prompt_answer_quality_check}
% \vspace{-0.5cm}
\end{table*}

\end{document}
