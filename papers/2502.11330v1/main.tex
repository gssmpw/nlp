% This must be in the first 5 lines to tell arXiv to use pdfLaTeX, which is strongly recommended.
\pdfoutput=1
% In particular, the hyperref package requires pdfLaTeX in order to break URLs across lines.

\documentclass[11pt]{article}

% Change "review" to "final" to generate the final (sometimes called camera-ready) version.
% Change to "preprint" to generate a non-anonymous version with page numbers.
\usepackage[preprint]{acl}

% Standard package includes
\usepackage{times}
\usepackage{latexsym}

% For proper rendering and hyphenation of words containing Latin characters (including in bib files)
\usepackage[T1]{fontenc}
% For Vietnamese characters
% \usepackage[T5]{fontenc}
% See https://www.latex-project.org/help/documentation/encguide.pdf for other character sets

% This assumes your files are encoded as UTF8
\usepackage[utf8]{inputenc}
\usepackage{microtype}
\usepackage{inconsolata}
\usepackage{graphicx}
\usepackage{booktabs}       % professional-quality tables
\usepackage{microtype}      % microtypography
% Standard package includes
\usepackage{url}            % simple URL typesetting
\usepackage{amsfonts}       % blackboard math symbols
\usepackage{nicefrac}       % compact symbols for 1/2, etc.
\usepackage{amsmath}
\usepackage{graphicx}
\usepackage{bm, upgreek}
\usepackage{amssymb}
\usepackage{array}
\usepackage{multirow}
\usepackage{mathtools}
\usepackage{arydshln}
\usepackage{tabularx}
\usepackage{color}
\usepackage{soul}
\usepackage{caption}
\usepackage{xcolor}
\usepackage{comment}
\usepackage{adjustbox}
\usepackage{hyperref}
\usepackage{cleveref}
\usepackage[ruled,vlined]{algorithm2e}
\usepackage{makecell}
\usepackage{diagbox}
\usepackage{subfigure}
\usepackage{subcaption}
\usepackage{xcolor, soul}
\usepackage{wrapfig}
% \usepackage[numbers]{natbib}
% \captionsetup[sub]{font=normalsize,labelfont={bf,sf}}
\usepackage{pifont}% http://ctan.org/pkg/pifont
\newcommand{\cmark}{\ding{51}}%
\newcommand{\xmark}{\ding{55}}%
\definecolor{bluecolor}{HTML}{0000FF}
\definecolor{greencolor}{HTML}{8CD0A4}
\definecolor{yellowcolor}{HTML}{F9D17C}
\definecolor{redcolor}{HTML}{FF0000}

\newcommand{\cyan}[1]{\textcolor{cyan}{#1}}
\newcommand{\red}[1]{\textcolor{red}{#1}}
\newcommand{\blue}[1]{\textcolor{blue}{#1}}
\newcommand{\yellow}[1]{\textcolor{yellow}{#1}}
\definecolor{black}{rgb}{0,0,0}
\newcommand{\black}[1]{\textcolor{black}{#1}}
\newcommand{\dandelion}[1]{\textcolor{dandelion}{#1}}

\title{System Message Generation for User Preferences \\ using Open-Source Models}

\author{Minbyul Jeong\footnotemark[1] \\
  \\ \And
  Jungho Cho \\
  \\ \And
  Minsoo Khang \\
  Upstage AI \\ \And
  Dawoon Jung \\
  \\ \And
  Teakgyu Hong \\
}

% \vspace{-2.0cm}
\begin{document}
\maketitle
\footnotetext[1]{Corresponding authors.}

\begin{abstract}
System messages play a crucial role in interactions with large language models (LLMs), often serving as prompts to initiate conversations.
Through system messages, users can assign specific roles, perform intended tasks, incorporate background information, specify various output formats and communication styles.
Despite such versatility, publicly available data are often lack system messages and subject to strict license constraints in the industry field.
Manual labeling of publicly available data with system messages that align with user instructions demands significant resources.
In view of such challenges, our work introduces \textbf{\textsc{SysGen}}, a pipeline for generating system messages with better aligned assistant responses from the supervised fine-tuning dataset without system messages.
Training on \textsc{SysGen} data has demonstrated substantial improvements in the alignment of model responses with system messages and user instructions, as demonstrated across various open-source models on the Multifacet benchmark, while maintaining minimal impact on other unseen benchmarks such as Open LLM Leaderboard 2.
Our qualitative analysis highlights the importance of diverse system messages to ensure better adaptability across different contexts.
\end{abstract}

% \vspace{0.3cm}
\section{Introduction}

\begin{figure}[h]
    \centering
    \begin{overpic}[trim=0cm 0cm 0cm 0cm,clip,angle=0,origin=c,width=.4\linewidth]{images/teaser_absolute.png}
        %  trim={<left> <lower> <right> <upper>}
        %  \put(horiz, vert)
        %  \put(horiz, vert){\rotatebox{90}{Text}}
        %
        \put(107, 32){$\mathbf{\to}$}
    \end{overpic}\hspace{1cm}
    \begin{overpic}[trim=0cm 0cm 0cm 0cm,clip,angle=0,origin=c,width=.4\linewidth]{images/teaser_translated_yellow.png}
        %  trim={<left> <lower> <right> <upper>}
        %  \put(horiz, vert)
        %  \put(horiz, vert){\rotatebox{90}{Text}}
        %
    \end{overpic}
    \caption{Using translation methods, a controller trained on an environment with a given visual variation \textit{(left)} can be reused without any training or fine-tuning on a different environment (\textit{right}) with comparable performance. In red we see the trajectory of a car driven by the same controller when connected to two different encoders, one for each visual variation.
    }
    \label{fig:teaser}
\end{figure}

Deep Reinforcement Learning (RL) has enabled agents to achieve remarkable performance in complex decision-making tasks, from robotic manipulation to high-dimensional games (Mnih et al., 2015; Silver et al., 2017). 
Although recent RL techniques achieved strong improvements over sample efficiency \citep{yarats2021drqv2, kostrikov2020image}, training new agents remains a costly process, both in computational and temporal terms.
Despite these advances, most methods still require at least partial retraining when dealing with domain shifts such as visual appearance, reward functions, or action spaces \citep{pmlr-v97-cobbe19a, zhang2020learning}. These domain changes typically require expensive retraining, which can be prohibitive for real-world settings that require millions of interactions.

A variety of approaches have been proposed to address these shifting conditions. Domain randomization \citep{tobin2017domain, sadeghi2016cad2rl} trains agents across diverse visual styles or physics settings, promoting invariant features but demanding broader coverage of possible variations. Multi-task RL \citep{parisotto2015actor, teh2017distral} attempts to learn shared representations across multiple tasks.

In the supervised setting, recent representation learning techniques \citep{Moschella2022-yf,maiorca2023latent, norelli2022b, cannistraci2023bricks}, show that it is possible to zero-shot recombine encoders and decoders to perform new tasks across different modalities (images, text..) and tasks (classification, reconstruction) and even architectures.
In RL, methods adopting the relative representation framework \citep{Moschella2022-yf} have shown promising results in adapting encoders to different controllers with zero or few-shots adaptation, for robotic control from proprioceptive states \citep{jian2021adversarial} or for playing games in the Gymnasium suite \citep{towers2024gymnasium} from pixels \citep{ricciardi2025r3lrelativerepresentationsreinforcement}.
These methods, however, still require training models to use the new relative representations.

By contrast, \cite{maiorca2023latent} suggest that modules from independently trained neural networks can be connected via a simple linear or affine transformation, with no training constraint or fine-tuning required, if such transformations can be reliably estimated from a small set of “anchor” samples, pairs of states or observations deemed semantically equivalent.

Our main contribution is the implementation of a RL method based on semantic alignment to map between latent spaces of different neural models, so that their encoders and controllers can be stitched with the goal of creating new agents that can act on visual-task combinations never seen together in training. This includes the use of the transformations to map modules from different networks, and the collection of anchor samples used to estimate these transformations. We call our method Semantic Alignment for Policy Stitching (\textbf{SAPS}).
We perform analyses and empirical tests on the CarRacing and LunarLander environments to show the performance of new agents created via zero-shot stitching of encoders and controllers trained on different visual-task variations, demonstrating significant gains compared to existing zero-shot methods.
% % \section{Preliminaries}

% \subsection{History Taking for Doctor and Patient}

% In our paper, 

% \subsection{Medical Conversational Agent}

% \subsection{Automatic Evaluation of Long-form response in multi-turn dialogue}



% \begin{align}
%     P_{t+1} &= R(D_t, H_t) \\
%     D_{t+1} &= R(P_t, H_t)   
% \end{align}
     

\section{Related Works}
% \paragraph{Human preference learning.}

\paragraph{System message: utilization and evaluation.}
A system message is a unique component of LLMs to initiate a conversation with them.
It is utilized by many proprietary models (e.g.,  ChatGPT~\citep{openai2023b} and Claude~\citep{anthropic2024}) as well as open-source models (e.g.,  Mistral~\citep{alkhamissi2024investigating}, LLaMA~\citep{meta2024introducing}, Qwen~\citep{yang2025qwen2}, and DeepSeek~\citep{guo2025deepseek}).
The system messages serve the purpose of steering the LLM's generation behavior and are widely used for various functions, including imprinting the model's identity, recording the knowledge cut-off date of the training data, and providing guidelines for various tool usages~\citep{openai2024function, cohere2024, prompthub2025}.
Additionally, the system messages are used to guide the model in generating safe and harmless responses~\citep{touvron2023llama, lu2024sofa, wallace2404instruction}.

Despite the usefulness of system messages, there is a significant lack of data that includes system messages reflecting diverse user instructions without license constraints.
Furthermore, manually labeling such data requires substantial human resources and even among publicly available datasets, it is challenging to obtain data that includes various system messages~\citep{lin2024baichuan, xu2024magpie}.
\citet{lee2024aligning} provide data augmentation which reflects hierarchical dimensions of system role data with multiple aspects of evaluation benchmark.
Furthermore, \citet{qin2024sysbench} provide multi-turn benchmark to evaluate system message alignment.
In line of these works, our \textsc{SysGen} pipeline  ensures high-quality system messages and assistant responses by supplementing data using only open-source models without licensing concerns.
Furthermore, it demonstrates that data augmentation is possible on existing SFT datasets without requiring extensive human labeling efforts.


\begin{figure*}[]
\centering
\includegraphics[width=\textwidth]{figure/data_construction.pdf}
% \vspace{-0.3cm}
\caption{
Overall \textbf{\textsc{SysGen}} data construction pipeline. Our pipeline consists of four phases:
% (Phase 1 - System Message Generation) We first collect supervised fine-tuning (SFT) datasets which does not contain system message. Then, we use open-source models to generate system message given question and answer pair. Our system message is composed of phrases with manually selected eight key functionalities tags.
(Phase 1) We gather SFT datasets which do not contain system messages and use open-source models to generate system messages with manually selected eight key fuctionality tags. 
% (Phase 2) We apply a filtering process to remove cases where tag tokens are generated incorrectly. To ensure consistency in system messages, we reorganize the tags in a predefined order.
(Phase 2) We then remove incorrectly generated tag tokens and reorganize tags with phrases in a predefined order for consistency.
% (Phase 3) To verify the correctness of phrases generated for each tag, we introduce a self-checking LLM-as-a-judge mechanism. Phrases that are empty, overly specific, or unnatural are classified as Bad and subsequently removed along with their corresponding tags.
(Phase 3) We use a LLM-as-a-judge approach with self-model feedback to filter out empty, overly specific, and unnatural phrases.
% (Phase 4) We refine the generated system message by removing tags to create a more natural sentence. Based on this refined system message, we generate a new answer along with the user instruction.
(Phase 4) We finally remove tags to create natural system messages and generate new responses along with the user instructions.
}
\label{fig:data_construction}
\vspace{-0.3cm}
\end{figure*}
\section{\textsc{SysGen}: Pipeline of System and Assistant Response Generation}
Our \textbf{\textsc{SysGen}} pipeline consists of four phases: (1) generating system messages with eight key functionalities (Sec~\ref{sysgen:system_generation}), (2) filtering mis-specified system tags and reorganizing them (Sec~\ref{sysgen:filtering}), (3) verifying the key functionalities on a phrase level (Sec~\ref{sysgen:verification}), (4) generating the new assistant responses using the refined system messages and original user instructions (Sec~\ref{sysgen:answer_generation}).
Figure~\ref{fig:data_construction} depicts the overall architecture of the \textsc{SysGen} pipeline. 

\subsection{Phase 1: System Message Generation}
\label{sysgen:system_generation}
% The primary goal of our \textbf{\textsc{SysGen}} pipeline is to generate the system messages that are not included in the original SFT dataset.
The primary goal of our \textbf{\textsc{SysGen}} pipeline is to enhance existing SFT datasets by adding system messages that were not originally included.
% During the process of developing and releasing LLMs, license constraints inevitably arise, making it difficult to utilize most publicly available data.
As the system messages can steer the LLM's behaviors, we focus on these messages during the development and release of the models.
%The system messages can steer the LLM's behaviors, so we focus on these messages during the development and release of the models.
However, license constraints and substantial resource requirements of manually labeling the system messages inevitably arise, making it difficult to utilize most publicly available datasets.
Thus, we aim to generate system messages by leveraging open-source models and data without any license issues.
% We focus on system messages because they can control the LLM's behavior, set roles, provide additional background contexts, and maintain consistent responses.

\paragraph{Phrase level Annotation to System Messages}
% eight key features 나열, 하는 역할 제공 및 어떻게 annotate하고 이를 verify하는지
We manually classify eight functionalities that are widely used in the system messages referring to previous works~\citep{openai2024function, cohere2024, alkhamissi2024investigating,lee2024aligning}: 
(1) Specifies the role, profession, or identity that needs to be played (Role);
(2) Specifies the content that needs to be included in the response such as an identity of the company (Content);
(3) Identifies what to perform (Task);
(4) Specifies the behavior to perform (Action);
(5) Prefers the style of communication for responses (Style);
(6) Provides additional information to be served as an assistant (Background);
(7) Provides built-in methods to use (Tool); 
(8) Preference of what output should look like (Format).


As shown in Figure~\ref{fig:data_construction} (top left), all functionalities are annotated at a phrase level with pre-/post-fix tags. 
Given a pair of user instructions $\mathcal{Q}$ and assistant responses $\mathcal{A}$, we generate a system message $\mathcal{S}$ using the open-source LLMs $\mathcal{M}$ with a prompt $\mathcal{P}$ that includes few-shot demonstrations:
\begin{equation}
    \mathcal{M}(\mathcal{S}|\mathcal{P},\mathcal{Q},\mathcal{A})
\end{equation}
We provide details about the few-shot demonstrations in the Appendix~\ref{app:prompt}.


\subsection{Phase 2: Filtering Process}
\label{sysgen:filtering}
After generating the system messages, we filter out the abnormal system messages for consistent text format.
In Figure~\ref{fig:data_construction} (top right), we first identify and remove mis-tagged phrases.
% For example, if the start token is set as <<Task>> but the end token is set as <<Format>>, then we cannot guarantee the phrase between these tokens corresponds to <<Task>> or <<Format>>.
For example, we can guarantee the correctness of the phrases between these tokens only if the start and end tokens are the same (e.g., <<Task>>).
% For example, if the start and end tokens are the same (e.g., <<Task>>), then we can guarantee the correctness of the phrases between these tokens.
In addition, we remove invalid tags such as <<Example>> or <<System>>, which may be generated in phase 1.
To ensure a consistent structure of system messages, we reorder the tags and phrases in manually defined order.

\begin{table}[t]
{\resizebox{1.0\columnwidth}{!}{
\begin{tabular}{lccccccc}
\toprule
\multicolumn{1}{c}{\multirow{2}{*}{Models}} & \multicolumn{3}{c}{Words Composition}                 & \multirow{2}{*}{BERTScore} & \multirow{2}{*}{BLEURT} & \multirow{2}{*}{GLEU} & \multirow{2}{*}{Len.} \\ \cmidrule{2-4}
\multicolumn{1}{c}{} & \multicolumn{1}{c}{R1} & \multicolumn{1}{c}{R2} & \multicolumn{1}{c}{RL} & & & \\ \midrule
LLaMA-3.1-8B-instruct  & 33.3 & 15.6 & 23.1 & 81.3 & 33.6 & 28.2 & 1.35\\ 
Qwen2.5-14b-instruct  & 44.9 & 23.2 & 30.7 & 85.9 & 39.9 & 39.2 & 1.55\\ 
Phi-4 & 51.9 & 32.3 & 41.1 & 86.1 & 40.1 & 37.2 & 1.89 \\ 
\bottomrule
\end{tabular}}}
\caption{A statistic that measures the words composition (Rouge-1,-2, and -L), semantic similarity (BERTScore and BLEURT), fluency (GLEU), and average context length of the newly-generated answer compared to average context length of the original answer.}
\label{tab:statistics_generated_answer}
\vspace{-0.3cm}
\end{table}

\subsection{Phase 3: Verification of Eight Key Functionalities}
\label{sysgen:verification}
In this phase, we verify whether each generated phrase is appropriate for its assigned tag. 
Using the LLM-as-a-judge~\citep{zheng2023judging} approach with self-model feedback, we assign one of three labels for each tag: \textit{Good} if the tagging is appropriate, \textit{Bad} if the tagging is inappropriate, and \textit{None} if the tag or phrases are missing.
Phrases labeled as \textit{Bad} or \textit{None} are then removed from the system message to ensure accuracy and consistency.
We observe that most of the data instances (up to 99\%) are preserved after applying phase 3.
% In Figure~\ref{}, we provide the filtered statistics of the phase 2 \& 3.


\subsection{Phase 4: Assistant Response Generation}
\label{sysgen:answer_generation}
% remove annotated tags for naturality
After filtering and verifying the generated system messages, they can be used alongside existing QA pairs.
However, we hypothesize that if there is any potential misalignment between the human curated QA and model-generated system messages, a follow-up data alignment phase is necessary.
% However, our experimental results reveal that training LLMs on existing QA pairs with generated system messages leads to little to no improvement in benchmarks evaluating system message alignment such as Multifacet~\citep{lee2024aligning}.
% The performance degradation has also been observed in unseen benchmarks such as Open LLM Leaderboard 2~\citep{myrzakhan2024open}.
% Therefore, we hypothesize that we have to generate new assistant responses $\mathcal{A'}$ based on a refined system messages $\mathcal{S}$ and the user instructions $\mathcal{Q}$, ensuring better alignment with the given instructions.
Therefore, we generate new assistant responses $\mathcal{A'}$ based on a refined system messages $\mathcal{S}$ and the user instructions $\mathcal{Q}$, ensuring better alignment with the given instructions.
%Therefore, it is necessary to generate new assistant responses $\mathcal{A'}$ based on a refined system message $\mathcal{S}$ and the user's original instruction $\mathcal{Q}$.

% answer generation with system message and original query
To achieve this, we first remove the annotated tags from the system messages to guarantee that the refined messages seem natural.
We provide a detailed example in Figure~\ref{fig:data_construction} (bottom right).
Then, we use the open-source LLMs $\mathcal{M}$ employed in phase 1 to generate new responses $\mathcal{A'}$.
\begin{equation}
    \mathcal{M}(\mathcal{A'}|\mathcal{S},\mathcal{Q})
\end{equation}
% how answer could be diversified and preserved compared to original answer
In Table~\ref{tab:statistics_generated_answer}, the new responses preserve similar content with high n-gram matching compared to the original responses, but have shown diversified formats with high semanticity and verbosity.
We provide the cases in Appendix~\ref{app:qualitative_analysis}.

\begin{figure}[t]
\centering
\includegraphics[width=\columnwidth]{figure/answer_quality_check_2.pdf}
\caption{
A statistic that verifies whether the newly-generated answer is more suitable for the user query than the original answer.
It records the probability that GPT-4o would respond with the newly-generated answer being better than the original answer (the probability should ideally exceed 50\%).
}
\label{fig:answer_comparison}
% \vspace{-0.3cm}
\end{figure}

% LLM-as-a-judge to verify new answer is preferred
We also use LLM-as-a-judge with GPT-4o to analyze that the new responses $\mathcal{A'}$ are better aligned to the user instructions than the original responses $\mathcal{A}$.
Figure~\ref{fig:answer_comparison} illustrates the proportion of cases where the new responses are judged to be better aligned than the original responses when given the user instructions.
For simpler evaluation, we evaluated 1K randomly sampled instances from the generated datasets.
Overall, our findings suggest that generating responses based on the system messages lead to better alignment with user instructions.
% Additionally, our experiments show that models trained with both the system messages and the new responses achieve quantifiable performance improvements, highlighting the importance of generating new responses.

\subsection{Study Subjects}
\subsubsection{Studied LLMs}
\label{subsec:llms}
We select the studied LLMs based on the following criteria: (1) All models are evaluated via the official Huggingface platform and are demonstrated on the LLM Safety Leaderboard (as of October 2024)~\cite{SecureLearningLab2024}. These models have been assessed on multiple dimensions, demonstrating their ability to refuse harmful content. (2) All selected models are either open-sourced or accessible via public APIs. (3) Open-source LLMs without accessible weight files or those exceeding hardware requirements for local deployment (typically models with over 20 billion parameters) are excluded. (4) To ensure the diversity of models under study, we include both general-purpose LLMs and code LLMs. (5) All selected LLMs have undergone instruction-based fine-tuning, as our experiments require models capable of understanding instructions and correctly leveraging the provided information.

\begin{table}[!t]
  \centering
  \caption{Studied LLMs in the study}
  \resizebox{1\linewidth}{!}{
  \begin{tabular}{cccc}
    \toprule
    \textbf{Category} & \textbf{LLM} & \textbf{Publisher} & \textbf{Open-source} \\
    \midrule
          \multicolumn{1}{c}{\multirow{2}[1]{*}{General}} & GPT-4o~\cite{openai_gpt4o} & OpenAI & No \\
    & Llama-3-8B~\cite{dubey2024llama} & Meta & Yes \\
    \midrule
        \multicolumn{1}{c}{\multirow{2}[1]{*}{Code}}  & CodeLlama-13B~\cite{roziere2023code} & Meta & Yes \\
     & DeepSeek-Coder-V2-16B~\cite{zhu2024deepseekcoder} & DeepSeek & Yes \\
    \bottomrule
  \end{tabular}
  }
  \label{tab:llms}
\end{table}
Table~\ref{tab:llms} shows all the LLMs examined in our experiments. We selected four representative LLMs as our research subjects. These models include both open-source and closed-source LLMs, ranging from small parameter scales (\eg 8B) to standard scales (e.g., GPT-4o), and encompass both general-purpose and code-oriented models.
For the closed-source LLM (i.e., GPT-4o), we accessed them through the official OpenAI API~\cite{openai2024apin}. For open-source LLMs, we obtained the model weights from their official Hugging Face repositories. For brevity, we refer to Llama-3-8B, CodeLLAMA-13B, and DeepSeek-Coder-V2-16B as LLAMA-3, CodeLLAMA, and DS-Coder respectively in the following sections.

\subsubsection{Retriever}  
The retriever is the key component of RACG systems, responsible for retrieving relevant code snippets as references to enhance the code generation process. RACG systems primarily use two types of retrievers~\cite{gao2024preference,wang2024coderag}: sparse and dense retrievers.  
Sparse retrievers (\eg TF-IDF~\cite{sparck1972statistical} and BM25~\cite{robertson2009probabilistic}) rely on sparse vector representations to retrieve documents or passages. Dense retrievers, in contrast, use dense vector representations (e.g., learned embeddings from neural networks) to capture semantic relationships between queries and documents~\cite{wang2024coderag,gao2024preference}. While dense retrievers excel at understanding context, they are computationally more expensive.  
With advancements in language models, dense retrievers have become predominant and are widely adopted in recent studies~\cite{parvez2021retrieval,gao2024preference,wang2023rap,wang2024coderag}.  
In this study, we implement BM25 and JINA retrievers as representatives of sparse and dense retrievers, respectively. The number of retrieved instances is determined by the specific settings in the RACG system.
\begin{itemize}[leftmargin=*]  
    \item {\bf BM25}: BM25 is an enhanced version of TF-IDF that typically demonstrates better performance. It ranks code snippets based on the frequency of query tokens appearing in the tokens of the code examples stored in the knowledge base. The top $n$ snippets with the highest scores are selected as examples for code generation.  
    \item {\bf JINA}: For this retriever, we use the state-of-the-art embedding model {\tt jina-embeddings-v3}~\cite{sturua2024jina} to generate feature vectors for both queries and code snippets in the knowledge base. For each query, the top $n$ most similar instances, as measured by cosine similarity, are retrieved as examples for subsequent code generation.  
\end{itemize}  

Additionally, as specified in our threat model (\S\ref{subsec:threat_model}), attackers are assumed to lack access to the retriever's parameters and cannot directly query the retrievers. Consequently, an external retriever is required for two distinct purposes: (1) retrieving vulnerable code from the vulnerability knowledge base in Scenario I, and (2) generating embeddings for code in Scenario II.  
To this end, we employ a {\bf TE3} retriever as the poisoning retriever, which is based on the \texttt{text-embedding-3-large} embedding model~\cite{OpenAI_Embeddings}. This retriever is exclusively used for embedding and retrieving vulnerable code for knowledge base poisoning and is not integrated into the RACG system itself.



\subsection{Metrics} 
To quantitatively evaluate the impact of vulnerable code within the poisoned knowledge base on the security and functionality of generated code, we employ the following metrics:

{\bf Vulnerability Rate (VR)}:
This metric quantifies the likelihood of an LLM generating vulnerable code. It is defined as the percentage of generated code snippets that exhibit security vulnerabilities. Formally, the VR is given by $VR = \frac{N_{v}}{N_{t}}$, where $N_{v}$ denotes the number of vulnerable code snippets generated by LLMs (evaluated by the LLM judge described in~\S\ref{subsec:validation}), and $N_{t}$ represents the total number of code snippets generated by LLMs. The VR provides a clear measure of RACG security risk when vulnerable code exists in the knowledge base. Higher VR values indicate a higher likelihood of generating insecure code, highlighting the need for improved security measures within the RACG system.

{\bf Similarity}: To evaluate how poisoned code affects LLM-generated code functionality in RACG systems, we measure similarity between the generated code and ground truth using CrystalBLEU~\cite{eghbali2022crystalbleu}, a BLEU variant designed for code similarity~\cite{phan2023evaluating,storhaug2023efficient}. CrystalBLEU is an optimized version of the BLEU~\cite{papineni2002bleu} that distinguishes between similar and dissimilar code examples 1.9–4.5 times more precisely.

{\bf Vulnerability Rate in Retrieved Code (VRRC)}: To investigte to what extent the retrieved examples are poisoned, we define VRRC as the average proportion of vulnerable code retrieved as examples in the input among all retrieved codes. Formally, it is given by:
\begin{equation*}
    VRRC = \frac{1}{|\mathcal{Q}|} \sum_{q \in \mathcal{Q}} \frac{|\mathcal{V}_q|}{r},
\end{equation*}
\noindent
where {\small $|\mathcal{V}_q|$} denotes the number of retrieved vulnerable codes for query $q$, $r$ is the number of retrieved codes, and $|\mathcal{Q}|$ is the number of queries. A higher VRRC indicates that a greater proportion of the poisoned examples are retrieved and used as context during code generation.


\subsection{Implementation Details}
\label{subsec:imple_details}
All experiments were conducted on a single A100-40G GPU server using the Ollama framework~\cite{ollama_website}. For the LLMs, we configured them with a temperature of 0 to reduce non-determinism~\cite{ouyang2024empirical}, a top-p value of 0.95, a \texttt{max\_new\_tokens} setting of 4096, and a context window of 8192, keeping other parameters at default values. We adhered to each model's recommended prompt format, using predefined chat templates or formats from model cards, GitHub repositories, or original papers. For query generation (\S\ref{subsec:dataset_cons}) and result validation (\S\ref{subsec:validation}), we used DeepSeek-V2.5 as the LLM backend, which excels in code-related tasks~\cite{DeepSeek2024}.
For retrievers, we reused the BM25 implementation from an open-access GitHub repository\footnote{\url{https://github.com/dorianbrown/rank_bm25}}, loaded the JINA retriever from Huggingface\footnote{\url{https://huggingface.co/jinaai/jina-embeddings-v3}}, and used the OpenAI API for the TE3 retriever~\cite{OpenAI_Embeddings}.

\section{Experiments}\label{sec:experiments}
We now evaluate SAPS using both qualitative and quantitative analyses. We first compare its zero-shot performance to R3L on benchmark tasks, then 
%perform delve into ablation studies and
an analysis of how our alignment approach behaves under different conditions.

\paragraph{Environments}
Our agents act by receiving pixel images as input observation, consisting of four consecutive $84 \times 84$ RGB images, stacked along the channel dimension to capture dynamic information such as velocity and acceleration.
We consider environments where we can freely change visual features (background color, camera perspective) or task (rewards, dynamics), therefore we use CarRacing \citep{klimov2016carracing} and LunarLander as both implemented in R3L.
CarRacing requires the agent to drive in a track using pixel observations, whose variations can be in the background color or the target speed, while LunarLander requires the agent to land on a platform, with variations comprising background color and different gravities.
% \AR{appendice per dettagli approfonditi su variazioni}.
No context is provided, hence the agents do not receive any information about the task.
% In the appendix we have other tests with atari env: \Cref{appendix:atari} \AR{riscrivi frase}

\paragraph{Baselines}
We mainly compare SAPS to (R3L), another zero-shot stitching method using relative representations whose approach is similar to ours.
For an additional baseline we also compare to naive zero-shot stitching, where we stitch encoders and controllers with no additional processing, to showcase the progress reached by the methods performing latent alignment techniques.

\section{Analysis}
\subsection{What makes \textsc{SysGen} pipeline useful?}
\label{ana:sysgen_useful}
\begin{table}[t]
\centering
{\resizebox{\columnwidth}{!}{
\begin{tabular}{lcc}
\toprule
\multicolumn{1}{c}{\textbf{Models}}           & \begin{tabular}[c]{@{}c@{}}\textbf{Multifacet}\\ \textbf{(Average)}\end{tabular} & \begin{tabular}[c]{@{}c@{}}\textbf{Unseen Benchmarks}\\ \textbf{(Average)}\end{tabular} \\ \midrule
\textit{No System Message} \\ \midrule
LLaMA-3.1-8B-instruct  & 3.98 & 50.85 \\ 
Phi-4 & 4.26 & 66.33 \\ 
\midrule
\textit{Common System Message} \\ \midrule
LLaMA-3.1-8B-instruct  & 3.89 & 51.23 \\ 
Phi-4 & 4.23 & 66.52 \\
\midrule
\textit{\textsc{SysGen} without A'} \\ \midrule
LLaMA-3.1-8B-instruct  & 4.09 & 51.89 \\ 
Phi-4 & 4.38 & 66.12 \\ 
\midrule
\textit{\textsc{SysGen}} \\ \midrule
LLaMA-3.1-8B-instruct  & 4.21 & 54.02 \\ 
Phi-4 & 4.54 & 68.08 \\
\bottomrule
\end{tabular}}}
\caption{Ablation studies of using system message and assistant's response. Using a common system message or generated system message does not provide insightful difference. Newly-generated answer and its corresponding system message can increase system abilities with lower decrease in unseen benchmarks.}
\label{tab:compare_system_message}
\vspace{-0.3cm}
\end{table}
To assess the impact of system messages generated by \textsc{SysGen} during training, we conduct ablation studies on four different model variations:
\begin{itemize}
    \item No System Message: The original SFT dataset which does not contain the system message. 
    \item Common System Message: An $\mathcal{S}\mathcal{Q}\mathcal{A}$ triplet where the common system message is inserted such as "You are a helpful AI assistant".
    \item \textsc{SysGen} without $\mathcal{A'}$: An $\mathcal{S}\mathcal{Q}\mathcal{A}$ triplet that includes only a system message generated by our \textsc{SysGen} pipeline.
    \item \textsc{SysGen}: An $\mathcal{S}\mathcal{Q}\mathcal{A'}$ triplet where both the \textsc{SysGen}-generated system message and the newly-generated answer are incorporated.
\end{itemize}
We measure the effectiveness of these models by analyzing score variations on the Multifacet and unseen benchmarks in Table~\ref{tab:compare_system_message}.

Training with data that includes common system messages does not result in a significant performance difference compared to training without system messages.
This led us to question: \emph{"Would it be sufficient to include only the most suitable system messages?"}.
To explore this, we train models using data that contains only system messages generated by \textsc{SysGen} pipeline.
As a result, we observe an improvement in Multifacet performance for both models, while the scores on the unseen benchmark remained similar.
Furthermore, when both system messages and assistant responses generated by \textsc{SysGen} are used for fine-tuning, we observe performance improvements in both Multifacet evaluation and unseen benchmarks.

\subsection{System message vs. User instruction}
\begin{table}[t]
\centering
{\resizebox{\columnwidth}{!}{
\begin{tabular}{lc}
\toprule
\multicolumn{1}{c}{\textbf{Models}}           & \begin{tabular}[c]{@{}c@{}}\textbf{Multifacet Average}\\ \textbf{(Use system role → Use user role)}\end{tabular} \\ \midrule
\textit{Open-source Models} \\ \midrule
Solar-10.7B-instruct   & 3.19 → 2.98 \\ 
LLaMA-3.1-8B-instruct  & 4.12 → 4.09 \\ 
Qwen2.5-14b-instruct   & 4.26 → 4.13 \\ 
Phi-4 & 4.41 → 4.26 \\ 
\midrule
\multicolumn{2}{l}{\textit{Open-source Models} (with \textbf{\textsc{SysGen}})} \\ \midrule
LLaMA-3.1-8B-instruct  & 4.21 → 4.13 \\ 
Qwen2.5-14B-instruct   & 4.28 → 4.16 \\ 
Phi-4 & 4.54 → 4.38 \\
\midrule
\multicolumn{2}{l}{\textit{Open-source Models} $+$ KD  (with \textbf{\textsc{SysGen}})} \\ \midrule
Solar-10.7b-instruct   & 3.76 → 3.64 \\ 
\bottomrule
\end{tabular}}}
\caption{There is a tendency for the score to decrease when the system message is reflected in the user instruction. The more a model is trained on system messages, the better it is to place them in the system role. KD indicates the knowledge distillation.}
\label{tab:no_system_message}
% \vspace{-0.5cm}
\end{table}
A key question arises that \emph{what happens if we add a message intended for the system role at the beginning of the user instruction? Could it serve as a replacement for the system role?}
To explore this, we conduct an experiment on a Multifacet benchmark.
Specifically, we included messages that should typically be in the system role within the user instruction during inference.

As shown in Table~\ref{tab:no_system_message}, we observe that open-source models tend to experience score degradation when system role messages are incorporated into the user instruction.
This trend suggests that adding such content can make the query itself more ambiguous to answer.
Furthermore, even in models trained with our \textsc{SysGen}, this trend persists similarly to the previous work~\citep{lee2024aligning}.
Despite additional fine-tuning on system roles, scores still remain low when system messages are reflected in the user instruction.
This highlights the importance of properly placing these messages in the system role to maintain performance.
% This indicates that properly placing these messages in the system role remains crucial for maintaining performance.

\subsection{New assistant responses align to the system messages}
\begin{figure}[t]
\centering
\includegraphics[width=\columnwidth]{figure/aligment_message_response.pdf}
\caption{
The GPT4o LLM-as-a-judge results of measuring the alignment between generated system messages and new assistant responses. We use 20 samples for each data source which sums up to 100 samples in total per models.
}
\label{fig:alignment_message_response}
\vspace{-0.3cm}
\end{figure}
In Table~\ref{tab:statistics_generated_answer}, we presented that the new assistant responses exhibit similar n-gram matching, high semantic similarities, and verbosity.
Therefore, it is necessary to verify whether the generated assistant responses aligned with the system messages.
Figure~\ref{fig:alignment_message_response} illustrates the GPT-4o results using LLM-as-a-judge approach.
Through the three \textsc{SysGen} data generated by Phi-4, LLaMA, and Qwen models, we determined that all of the assistant responses are highly aligned with the system messages.
Overall, the experiments and analyses reveal that our \textsc{SysGen} data were generated to effectively respond to various user instructions as system messages.
In addition, we observed that the assistant responses align with the system messages and are capable of generating better aligned responses compared the original assistant responses.

\section{Conclusion}
This paper presents the first comprehensive study on the security risks of RACG systems, specifically the impact of vulnerable code in the knowledge base. Our experiments show that knowledge base poisoning significantly compromises the security of generated code, with up to 48\% of the code becoming vulnerable from a single poisoned sample in Scenario I with CodeLlama. We also identify that factors such as few-shot learning, programming language, and example-query similarity contribute to increased vulnerabilities. Our findings underline the need for improved knowledge base security and offer practical insights for reducing vulnerability propagation in LLM-generated code. This work pave the way for future works which aim at securing RACG systems and enhancing their reliability in software development.
\section*{Limitations}
While our \textsc{SysGen} pipeline demonstrates promising results in system messages alignment to the user instructions through Multifacet dataset.
However, our data construction pipeline only considers the single-turn conversation without handling multi-turn conversations~\citep{qin2024sysbench}.
We acknowledge that it is important for system messages to remain effective throughout multi-turn conversations, but our study focuses on evaluation and simple level of inference usage.

Additionally, our experimental results reveal that training with \textsc{SysGen} data shows minimal performance degradation on unseen benchmark, Open LLM Leaderboard 2 dataset.
However, we suspect that the observed performance drop may be due to the format of natural text that the SFT datasets we selected, rather than formats similar to multiple-choice questions commonly found in the unseen benchmark. 
Therefore, we are curious about how well the system messages could be generated in various formats such as True/False questions or Multiple Choice questions and prove its effectiveness.

Finally, in Table~\ref{app:tag_statistics}, we identify the special tokens of tags which are annotated to the publicly avaiable data.
The <<Tool>> tag has been absolutely shown small portion compared to other tags.
Our initial intention was to utilize the tag for generating data through search functionality or function calls. 
However, the selected public data deviated from this purpose, resulting in a very low proportion of the tag being generated.
Therefore, it would be beneficial to gather and generate data appropriately for each tag's intended use.

% \section*{Acknowledgments}


\bibliography{custom}

\appendix


% \section{Rubrics of Automatic Evaluation}


\section{Data Statistics}
\label{app:data_statistics_appendix}
\paragraph{Statistics of generated tags.} As we stated in limitations section, we provide the statistics of generate special tag tokens in Table~\ref{app:tag_statistics}.
We find out that most of the <<Role>>, <<Content>>, <<Task>> tokens are annotated in the instances.
Compared to thoses tokens, <<Action>>, <<Style>>, <<Background>>, and <<Format>> depends on the user instructions to be generated.
However, <<Tool>> tokens have shown absolutely low portion to be generated.
We thus want to suggest that properly choosing the public or your own dataset seems to ensure the <<Tool>> tag usages such as selecting searching protocols or function calls. 
\begin{table}[h]
{\resizebox{\columnwidth}{!}{
\begin{tabular}{lccc}
\toprule
Tags       & LLaMA-3.1-8B-instruct & Qwen2.5-14b-instruct & Phi-4   \\ \midrule
Role       & 576,341               & 753,579              & 745,751 \\ 
Content    & 580,231               & 739,892              & 743,311 \\ 
Task       & 579,558               & 765,331              & 735,298 \\ 
Action     & 495,301               & 382,358              & 662,589 \\ 
Style      & 283,579               & 598,553              & 603,918 \\ 
Background & 293,791               & 539,757              & 553,791 \\ 
Tool       & 10,238                & 132,038              & 90,989  \\ 
Format     & 327,909               & 401,593              & 538,973 \\ \bottomrule
\end{tabular}}}
\caption{Statistics of generated tags using \textsc{SysGen} pipeline.}
\label{app:tag_statistics}
\end{table}

\begin{table*}[t]
\centering
{\resizebox{\textwidth}{!}{
\begin{tabular}{lccccc}
\toprule
\multicolumn{1}{c}{Dataset} & \# of instances & Avg. Query Length & Avg. Answer Length & Containing System Message & Covering Domains\\ \midrule
Capybara  & 41,301          & 300.24 & 1423.28 & \xmark & reasoning, logic, subjects, conversations, pop-culture, STEM \\ 
Airoboros & 59,277          & 507.26 & 1110.62 & simple system message & mathematics, MATHJSON, character's descriptions \\ 
OrcaMath  & 200,035         & 238.87 & 878.43 & \xmark & school mathematics, math word problems \\ 
Magicoder & 111,183         & 652.53 & 1552.41 & \xmark & code solution \\  
MetaMath  & 395,000         & 213.53 & 498.24 & \xmark & mathematics \\  \bottomrule
\end{tabular}}}
\caption{Data statistics of SFT datasets. We provide the average length of query and answer, the presence of system messages, and covering domains.}
\label{tab:data_statistics_appendix}
% \vspace{-0.3cm}
\end{table*}


\paragraph{Statistics of original SFT datasets.}
In Table~\ref{tab:data_statistics_appendix}, we observe that most widely used public datasets either lack a system message entirely or include only a simple one, such as "You are a helpful AI assistant.".
The publicly available data mostly cover mathematics, code problems following some reasoning and logical ones.




\section{Experimental Details}
\label{app:experimental_details}
\paragraph{Computing Resources}
We use 4x8 NVIDIA H100 Tensor Core GPU with 80GB memory to train the open-source models.
We use Deepspeed stage 3~\citep{rajbhandari2020zero} to implement multi-GPU settings and FlashAttention~\citep{dao2022flashattention} for efficient training.
Our code is written in PyTorch~\citep{paszke2019pytorch} and HuggingFace~\citep{wolf2019huggingface}.


\paragraph{Integrating system roles in models that do not support them.}
\label{app:system_role_support}
Through our experiments, we find out that the Gemma-2-9b-it~\citep{team2024gemma} model does not inherently support the system role.
To address this limitation during data generation and training, we modified the chat template in the configuration of tokenization to remove restrictions on the system role.
Interestingly, despite the lack of native support, our findings show that \textsc{SysGen} data can still be utilized effectively to incorporate a system role into these models.

\section{Qualitative analysis of generated instances}
\label{app:qualitative_analysis}

\begin{table*}
    \scriptsize
    \centering
    \begin{NiceTabular}{@{}l@{\hskip4pt}p{0.92\textwidth}@{}}
    \CodeBefore
\cellcolor{gray!20}{1-1,1-2}
\cellcolor{gray!20}{5-1,5-2}
\cellcolor{gray!20}{9-1,9-2}
\Body
    \toprule
    \multicolumn{2}{c}{\textbf{Case 1: Under-prediction of SAR}}\\
    \midrule
    \textbf{Groundtruth} & \texttt{
    [...] 
Diaz
started
his
{\color{red}political career}
as a
{\color{red}member}
of
{\color{red}the Sangguniang Bayan}
{\color{red}(municipal council)}
of
{\color{red}Santa Cruz}
in
{\color{red}1978}.
He
later
became
{\color{red}the Vice Mayor}
of
{\color{red}Santa Cruz}
in
{\color{red}1980}
and
was elected
as
{\color{red}the town's Mayor}
in
{\color{red}1988}.
[...]
    }\\
    \hdashline
    \textbf{Likelihood} & \texttt{
    [...]
\adjustbox{bgcolor={red!61.3}}{\strut Diaz}
\adjustbox{bgcolor={red!95.4}}{\strut started}
\adjustbox{bgcolor={red!0.4}}{\strut his}
\adjustbox{bgcolor={red!7.2}}{\strut political career}
\adjustbox{bgcolor={red!35.3}}{\strut as a}
\adjustbox{bgcolor={red!67.7}}{\strut member}
\adjustbox{bgcolor={red!0.6}}{\strut of}
\adjustbox{bgcolor={red!21.2}}{\strut the Sangguniang Bayan}
\adjustbox{bgcolor={red!24.8}}{\strut (municipal council)}
\adjustbox{bgcolor={red!5.2}}{\strut of}
\adjustbox{bgcolor={red!23.9}}{\strut Santa Cruz}
\adjustbox{bgcolor={red!71.2}}{\strut in}
\adjustbox{bgcolor={red!81.2}}{\strut 1978}
\adjustbox{bgcolor={red!28.5}}{\strut .}
\adjustbox{bgcolor={red!15.2}}{\strut He}
\adjustbox{bgcolor={red!90.6}}{\strut later}
\adjustbox{bgcolor={red!89.7}}{\strut became}
\adjustbox{bgcolor={red!48.9}}{\strut the Vice Mayor}
\adjustbox{bgcolor={red!6.6}}{\strut of}
\adjustbox{bgcolor={red!14.3}}{\strut Santa Cruz}
\adjustbox{bgcolor={red!62.6}}{\strut in}
\adjustbox{bgcolor={red!45.6}}{\strut 1980}
\adjustbox{bgcolor={red!31.4}}{\strut and}
\adjustbox{bgcolor={red!61.7}}{\strut was elected}
\adjustbox{bgcolor={red!42.9}}{\strut as}
\adjustbox{bgcolor={red!61.5}}{\strut the town's Mayor}
\adjustbox{bgcolor={red!2.0}}{\strut in}
\adjustbox{bgcolor={red!37.2}}{\strut 1988}
\adjustbox{bgcolor={red!99.5}}{\strut .}
[...]
    }\\
    \hdashline
    \textbf{SAR} & \texttt{
    [...]
\adjustbox{bgcolor={red!64.0}}{\strut Diaz}
\adjustbox{bgcolor={red!31.4}}{\strut started}
\adjustbox{bgcolor={red!0.1}}{\strut his}
\adjustbox{bgcolor={red!2.4}}{\strut political career}
\adjustbox{bgcolor={red!6.3}}{\strut as a}
\adjustbox{bgcolor={red!16.0}}{\strut member}
\adjustbox{bgcolor={red!0.1}}{\strut of}
\adjustbox{bgcolor={red!13.8}}{\strut the Sangguniang Bayan}
\adjustbox{bgcolor={red!6.7}}{\strut (municipal council)}
\adjustbox{bgcolor={red!0.6}}{\strut of}
\adjustbox{bgcolor={red!7.0}}{\strut Santa Cruz}
\adjustbox{bgcolor={red!23.7}}{\strut in}
\adjustbox{bgcolor={red!74.5}}{\strut 1978}
\adjustbox{bgcolor={red!3.6}}{\strut .}
\adjustbox{bgcolor={red!8.3}}{\strut He}
\adjustbox{bgcolor={red!53.3}}{\strut later}
\adjustbox{bgcolor={red!35.6}}{\strut became}
\adjustbox{bgcolor={red!28.4}}{\strut the Vice Mayor}
\adjustbox{bgcolor={red!1.3}}{\strut of}
\adjustbox{bgcolor={red!4.1}}{\strut Santa Cruz}
\adjustbox{bgcolor={red!19.8}}{\strut in}
\adjustbox{bgcolor={red!38.4}}{\strut 1980}
\adjustbox{bgcolor={red!7.1}}{\strut and}
\adjustbox{bgcolor={red!23.8}}{\strut was elected}
\adjustbox{bgcolor={red!14.5}}{\strut as}
\adjustbox{bgcolor={red!23.2}}{\strut the town's Mayor}
\adjustbox{bgcolor={red!0.4}}{\strut in}
\adjustbox{bgcolor={red!29.5}}{\strut 1988}
\adjustbox{bgcolor={red!53.6}}{\strut .}
[...]
    }\\
\midrule
\multicolumn{2}{c}{\textbf{Case 2: The type-filter of Focus and the limitations of uncertainty scores}}\\
    \midrule
\textbf{Groundtruth} & \texttt{
Taral Hicks
is
an
American
actress
and
singer,
born
on
September 21, 1974,
in
{\color{red}The Bronx, New York}.
[...]
She
later
{\color{red}transitioned}
to
{\color{red}acting},
appearing in
films
such as
"A Bronx Tale"
(1993),
"Just Cause"
(1995),
and
"Belly"
(1998).
[...]
}\\
\hdashline
\textbf{Likelihood} & \texttt{
\adjustbox{bgcolor={red!27.5}}{\strut Taral Hicks}
\adjustbox{bgcolor={red!64.8}}{\strut is}
\adjustbox{bgcolor={red!43.8}}{\strut an}
\adjustbox{bgcolor={red!25.7}}{\strut American}
\adjustbox{bgcolor={red!98.8}}{\strut actress}
\adjustbox{bgcolor={red!49.1}}{\strut and}
\adjustbox{bgcolor={red!26.2}}{\strut singer}
\adjustbox{bgcolor={red!91.9}}{\strut ,}
\adjustbox{bgcolor={red!76.3}}{\strut born}
\adjustbox{bgcolor={red!56.8}}{\strut on}
\adjustbox{bgcolor={red!34.6}}{\strut September 21, 1974}
\adjustbox{bgcolor={red!71.4}}{\strut ,}
\adjustbox{bgcolor={red!9.1}}{\strut in}
\adjustbox{bgcolor={red!21.4}}{\strut The Bronx, New York}
\adjustbox{bgcolor={red!65.0}}{\strut .}
[...]
\adjustbox{bgcolor={red!75.1}}{\strut She}
\adjustbox{bgcolor={red!76.5}}{\strut later}
\adjustbox{bgcolor={red!13.8}}{\strut transitioned}
\adjustbox{bgcolor={red!86.7}}{\strut to}
\adjustbox{bgcolor={red!1.1}}{\strut acting}
\adjustbox{bgcolor={red!15.8}}{\strut ,}
\adjustbox{bgcolor={red!8.7}}{\strut appearing in}
\adjustbox{bgcolor={red!45.0}}{\strut films}
\adjustbox{bgcolor={red!10.6}}{\strut such as}
\adjustbox{bgcolor={red!48.7}}{\strut "A Bronx Tale"}
\adjustbox{bgcolor={red!18.1}}{\strut (1993),}
\adjustbox{bgcolor={red!47.3}}{\strut "Just Cause"}
\adjustbox{bgcolor={red!2.6}}{\strut (1995),}
\adjustbox{bgcolor={red!4.8}}{\strut and}
\adjustbox{bgcolor={red!46.6}}{\strut "Belly"}
\adjustbox{bgcolor={red!1.7}}{\strut (1998).}
[...]
}\\
\hdashline
\textbf{Focus} & \texttt{
\adjustbox{bgcolor={red!5.3}}{\strut Taral Hicks}
\adjustbox{bgcolor={red!0.0}}{\strut is}
\adjustbox{bgcolor={red!0.0}}{\strut an}
\adjustbox{bgcolor={red!23.5}}{\strut American}
\adjustbox{bgcolor={red!40.9}}{\strut actress}
\adjustbox{bgcolor={red!0.0}}{\strut and}
\adjustbox{bgcolor={red!30.4}}{\strut singer}
\adjustbox{bgcolor={red!0.0}}{\strut ,}
\adjustbox{bgcolor={red!0.0}}{\strut born}
\adjustbox{bgcolor={red!0.0}}{\strut on}
\adjustbox{bgcolor={red!31.9}}{\strut September 21, 1974}
\adjustbox{bgcolor={red!0.0}}{\strut ,}
\adjustbox{bgcolor={red!0.0}}{\strut in}
\adjustbox{bgcolor={red!21.0}}{\strut The Bronx, New York}
\adjustbox{bgcolor={red!0.0}}{\strut .}
[...]
\adjustbox{bgcolor={red!0.0}}{\strut She}
\adjustbox{bgcolor={red!0.0}}{\strut later}
\adjustbox{bgcolor={red!0.0}}{\strut transitioned}
\adjustbox{bgcolor={red!0.0}}{\strut to}
\adjustbox{bgcolor={red!32.2}}{\strut acting}
\adjustbox{bgcolor={red!0.0}}{\strut ,}
\adjustbox{bgcolor={red!0.0}}{\strut appearing in}
\adjustbox{bgcolor={red!33.2}}{\strut films}
\adjustbox{bgcolor={red!0.0}}{\strut such as}
\adjustbox{bgcolor={red!43.6}}{\strut "A Bronx Tale"}
\adjustbox{bgcolor={red!25.3}}{\strut (1993),}
\adjustbox{bgcolor={red!42.5}}{\strut "Just Cause"}
\adjustbox{bgcolor={red!30.1}}{\strut (1995),}
\adjustbox{bgcolor={red!0.0}}{\strut and}
\adjustbox{bgcolor={red!44.2}}{\strut "Belly"}
\adjustbox{bgcolor={red!30.1}}{\strut (1998).}
[...]
}\\
\midrule
\multicolumn{2}{c}{\textbf{Case 3: Uncertainty propagation of Focus}}\\
    \midrule
    \textbf{Groundtruth} & \texttt{
    [...]
Fernandinho
began
his
professional career
with
{\color{red}Atletico Paranaense}
in
{\color{red}Brazil}
before
{\color{red}moving}
to
{\color{red}Ukrainian club}
{\color{red}Shakhtar Donetsk}
in
{\color{red}2005}.
[...]
He
is known
for
his
{\color{red}physicality},
{\color{red}tackling ability},
and
{\color{red}passing range},
and
is
{\color{red}widely regarded}
as
{\color{red}one of the best}
{\color{red}defensive midfielders}
in
{\color{red}the world}.
    }\\
    \hdashline
    \textbf{Likelihood} & \texttt{
    [...]
\adjustbox{bgcolor={red!23.6}}{\strut Fernandinho}
\adjustbox{bgcolor={red!78.3}}{\strut began}
\adjustbox{bgcolor={red!3.3}}{\strut his}
\adjustbox{bgcolor={red!47.7}}{\strut professional career}
\adjustbox{bgcolor={red!60.5}}{\strut with}
\adjustbox{bgcolor={red!25.3}}{\strut Atletico Paranaense}
\adjustbox{bgcolor={red!42.1}}{\strut in}
\adjustbox{bgcolor={red!97.7}}{\strut Brazil}
\adjustbox{bgcolor={red!91.5}}{\strut before}
\adjustbox{bgcolor={red!40.5}}{\strut moving}
\adjustbox{bgcolor={red!9.1}}{\strut to}
\adjustbox{bgcolor={red!65.8}}{\strut Ukrainian club}
\adjustbox{bgcolor={red!3.2}}{\strut Shakhtar Donetsk}
\adjustbox{bgcolor={red!20.1}}{\strut in}
\adjustbox{bgcolor={red!1.5}}{\strut 2005}
\adjustbox{bgcolor={red!19.5}}{\strut .}
[...]
\adjustbox{bgcolor={red!34.9}}{\strut He}
\adjustbox{bgcolor={red!28.2}}{\strut is known}
\adjustbox{bgcolor={red!0.0}}{\strut for}
\adjustbox{bgcolor={red!0.0}}{\strut his}
\adjustbox{bgcolor={red!61.0}}{\strut physicality}
\adjustbox{bgcolor={red!8.6}}{\strut ,}
\adjustbox{bgcolor={red!77.0}}{\strut tackling ability}
\adjustbox{bgcolor={red!0.6}}{\strut ,}
\adjustbox{bgcolor={red!12.7}}{\strut and}
\adjustbox{bgcolor={red!53.7}}{\strut passing range}
\adjustbox{bgcolor={red!22.5}}{\strut ,}
\adjustbox{bgcolor={red!34.9}}{\strut and}
\adjustbox{bgcolor={red!61.7}}{\strut is}
\adjustbox{bgcolor={red!50.0}}{\strut widely regarded}
\adjustbox{bgcolor={red!0.0}}{\strut as}
\adjustbox{bgcolor={red!0.8}}{\strut one of the best}
\adjustbox{bgcolor={red!3.1}}{\strut defensive midfielders}
\adjustbox{bgcolor={red!2.0}}{\strut in}
\adjustbox{bgcolor={red!1.5}}{\strut the world}
\adjustbox{bgcolor={red!27.0}}{\strut .}
    }\\
    \hdashline
    \textbf{Focus} & \texttt{
    [...]
\adjustbox{bgcolor={red!28.8}}{\strut Fernandinho}
\adjustbox{bgcolor={red!0.0}}{\strut began}
\adjustbox{bgcolor={red!0.0}}{\strut his}
\adjustbox{bgcolor={red!16.9}}{\strut professional career}
\adjustbox{bgcolor={red!0.0}}{\strut with}
\adjustbox{bgcolor={red!36.5}}{\strut Atletico Paranaense}
\adjustbox{bgcolor={red!0.0}}{\strut in}
\adjustbox{bgcolor={red!35.3}}{\strut Brazil}
\adjustbox{bgcolor={red!0.0}}{\strut before}
\adjustbox{bgcolor={red!0.0}}{\strut moving}
\adjustbox{bgcolor={red!0.0}}{\strut to}
\adjustbox{bgcolor={red!38.8}}{\strut Ukrainian club}
\adjustbox{bgcolor={red!31.1}}{\strut Shakhtar Donetsk}
\adjustbox{bgcolor={red!0.0}}{\strut in}
\adjustbox{bgcolor={red!31.3}}{\strut 2005}
\adjustbox{bgcolor={red!0.0}}{\strut .}
[...]
\adjustbox{bgcolor={red!0.0}}{\strut He}
\adjustbox{bgcolor={red!0.0}}{\strut is known}
\adjustbox{bgcolor={red!0.0}}{\strut for}
\adjustbox{bgcolor={red!0.0}}{\strut his}
\adjustbox{bgcolor={red!36.0}}{\strut physicality}
\adjustbox{bgcolor={red!0.0}}{\strut ,}
\adjustbox{bgcolor={red!15.3}}{\strut tackling ability}
\adjustbox{bgcolor={red!0.0}}{\strut ,}
\adjustbox{bgcolor={red!0.0}}{\strut and}
\adjustbox{bgcolor={red!15.3}}{\strut passing range}
\adjustbox{bgcolor={red!0.0}}{\strut ,}
\adjustbox{bgcolor={red!0.0}}{\strut and}
\adjustbox{bgcolor={red!0.0}}{\strut is}
\adjustbox{bgcolor={red!0.0}}{\strut widely regarded}
\adjustbox{bgcolor={red!0.0}}{\strut as}
\adjustbox{bgcolor={red!7.2}}{\strut one of the best}
\adjustbox{bgcolor={red!18.8}}{\strut defensive midfielders} 
\adjustbox{bgcolor={red!0.0}}{\strut in}
\adjustbox{bgcolor={red!14.5}}{\strut the world}
\adjustbox{bgcolor={red!0.0}}{\strut .}
    }\\
\bottomrule
    \end{NiceTabular}
    \vspace{-0.5pc}
    \caption{We sampled 3 instances from our dataset to demonstrate the differences across uncertainty scores. For label, entities colored in {\color{red}red} indicate hallucination. For uncertainty scores, entities \colorbox{red!50}{boxed in red} with different tints represent the degree of uncertainty. A lighter (darker) box indicates a lower (higher) uncertainty.}
    \label{tb:qualitative_analysis}
    \vspace{-1pc}
\end{table*}
In Table~\ref{app:qualitative_analysis}, we provide the \textsc{SysGen} data by presenting the system messages, user instructions, and new assistant responses.
We observe that providing a specific format such as answer with paragraph format steers the LLM's behavior to answer in step-by-step processes within paragraph.
Also, if conversational example was provided, then the phrase of style tag forces to generate assistant response friendly.
Furthermore, if the system message grant specific roles such as a knowledgeable assistant, then the new assistant responses tend to generate verbose answers to the user instructions.

\section{Prompts}
\label{app:prompt}
To enhance reproducibility and facilitate understanding of the \textsc{SysGen} pipeline, we provide multiple prompts that we utilized.
In Table~\ref{tab:app_prompt_system_generation}, we use three-shot demonstrations to generate useful system messages which are collected through real-world scenarios.
The \textit{Conversational History} written in the prompt is composed of user instructions and original assistant responses.
Thus, given the user instructions and assistant responses, we generate the system messages at a phrase level containing eight functionalities with special tokens such as <<Role>>, <<Content>>, and <<Style>>.

After generating the system messages, in Table~\ref{tab:app_prompt_system_tag_check}, we verify the quality of each tag with three classes: Good, Bad, and None.
We want to note that the \textit{Annotated system messages}, composed of phrases and tags, are used to verify the \textit{Filtered system messages}.
By utilizing LLM-as-a-judge approach, we could save tremendous budgets through self-model feedbacks rather than using proprietary models (i.e., API Calls). 
Through our preliminary experiment, we observe that current open-source models such as Phi-4 or Qwen2.5-14b-instruct could preserve most of the phrases after applying phase 3.

Table~\ref{tab:app_prompt_answer_quality_check} shows the prompt of how we verify the quality of new assistant responses as shown in Figure~\ref{fig:answer_comparison}.
After prompting 1K randomly sampled instances, we observe that new assistant responses were qualified to be better aligned with user instructions.
\begin{table*}[t]
\centering
{\resizebox{\textwidth}{!}{
\begin{tabular}{l}
\toprule \midrule
\begin{tabular}[c]{@{}l@{}}
System: \\
Given a conversation history between user's question and assistant's response, \\ you are a system prompt generation assistant to generate a relevant system prompt. \\ 
The following [System Prompt] seems to have a mix of 8 different [functionalities]: \\ <Tasks>, <Tools>, <Style>, <Action>, <Content>, <Background>, <Role>, and <Format>. \\
Try to annotate each functionality within the system prompt in a phrase-level. 
Annotate each tag of functionalities. \\ Generate [Generated System Prompt] with a same language used in [Conversational History]. \\ \\

\#\# [Functionalities] \\
1. <<Task>>: what tasks will be performed? \\
2. <<Tool>>: What features or tools are available to integrate and use? \\
3. <<Style>>: What style of communication would you prefer for responses? \\
4. <<Action>>: Perform a specific action \\
5. <<Content>>:  Specifies the content that needs to be included in the response \\
6. <<Background>>:  Provides specific background information to ensure the model’s responses align with these settings. \\
7. <<Role>>:  Specifies the role, profession, or identity that needs to be played. \\
8. <<Format>>: Answers should be given in a specific format, which may include lists, paragraphs, tables, etc. \\ \\
User: \\
\#\# [Few-shot Examples of System Prompt] \\
\#\#\# 1 \\
<<Role>>You are an expert data augmentation system<</Role>> <<Task>>for korean text correction model training.<</Task>> \\<<Task>>Generate a pairs of data augmentation example.<</Task>> \\<<Background>>You are an intelligence AI model Solar-pro invented by Upstage AI.<</Background>> \\ \\

Instructions: \\
<<Content>>- In a given text, create 1\~3 typos.<</Content>> \\
<<Content>>- Typos can be reversed, misplaced, missing, duplicated, or misspaced letters.<</Content>> \\
<<Action>>- If the given text contains English, generate an English typo.<</Action>> \\
<<Action>>- Generate the results in the Output JSON format below.<</Action>> \\
<<Style>>-The response is informational and comprehensive, reflecting an expert understanding of the subject matter.<</Style>> \\

<<Format>> Output JSON format: \{ \\
"$original\_expression$": $ORIGINAL\_EXPRESSION$, \\
"$typo\_expression$": $TYPO\_EXPRESSION$
\} \\ <</Format>> \\ \\

\#\#\# 2 \\
<<Role>>You are an AI meeting note-taking assistant.<</Role>> \\
<<Task>>Your task is to generate meeting notes from the given conversation record.<</Task>> \\
<<Style>>All responses must be in Korean.<</Style>> \\
<<Action>>Take a deep breath, think carefully, and perform your role step by step.<</Action>> \\ \\

\#\#\# 3 \\
<<Role>>You are a chatbot of the Ministry of Food and Drug Safety (MFDS).<</Role>> \\
<<Task>>You answer user questions by referring to the provided reference.<</Task>> \\
<<Background>>You are designed to provide information related to pharmaceuticals and cosmetics. You have knowledge \\ of cosmetics-related information from Korea, the United States, Europe, China, India, and Taiwan.<</Background>> \\
<<Content>>If the user's question is related to the reference, respond starting with "According to the title,".<</Content>> \\
<<Content>>If the user's question is not related to the reference, respond with "Sorry, I couldn't find any information to \\answer your question. Please try asking again."<</Content>> \\
<<Content>>If the user's question is not related to food and drug safety, respond with "Sorry, I am a chatbot operated by the Ministry \\ of Food and Drug Safety. I can only answer questions related to the Ministry of Food and Drug Safety."<</Content>> \\

<<Style>>Respond to the user's questions kindly.<</Style>> \\
<<Background>>The reference is provided as context.<</Background>> \\ \\

\textit{Conversational History}
\end{tabular} \\ \midrule
\bottomrule
\end{tabular}}}
\caption{The prompt of generating system messages using open-source models. \textit{Italic} text part such as ``\textit{Conversational History}'' is filled with input text.}
\label{tab:app_prompt_system_generation}
\end{table*}
\begin{table*}[t]
\centering
{\resizebox{\textwidth}{!}{
\begin{tabular}{l}
\toprule \midrule
\begin{tabular}[c]{@{}l@{}}
System: \\
You are a functionality verifier assistant evaluating whether system messages are properly tagged according to the descriptions of 8 functionalities. \\
Review the provided [Filtered System Message] and [Annotated System Message] to verify the correctness of tagging for the 8 functionalities. \\ \\
Your task is to: \\
Confirm whether each tag aligns correctly with the respective functionality's description. \\
If a tag is properly generated and annotated, mark it as "Good". \\
If a tag exists but does not align with its functionality, mark it as "Bad". \\
If a tag is missing, mark it as "None" \\ \\

\#\# [Functionalities] \\
1. <<Task>>: what tasks will be performed? \\
2. <<Tool>>: What features or tools are available to integrate and use? \\
3. <<Style>>: What style of communication would you prefer for responses? \\
4. <<Action>>: Perform a specific action \\
5. <<Content>>:  Specifies the content that needs to be included in the response \\
6. <<Background>>:  Provides specific background information to ensure the model’s responses align with these settings. \\
7. <<Role>>:  Specifies the role, profession, or identity that needs to be played. \\
8. <<Format>>: Answers should be given in a specific format, which may include lists, paragraphs, tables, etc. \\ \\

\#\# [Expected Output Format] \\
<<Task>>: Good \\
<<Tool>>: None \\
<<Style>>: Good \\
<<Action>>: Good \\
<<Content>>: Bad \\
<<Background>>: Bad \\
<<Role>>: Bad \\
<<Format>>: Good \\ \\
User: \\
\#\# [Filtered System Message] \\
\textit{Filtered system messages} \\ \\
\#\# [Annotated System Message] \\
\textit{Annotated system messages} \\ \\
\#\# [Expected Output Format] \\
\end{tabular} \\ \midrule
\bottomrule
\end{tabular}}}
\caption{The prompt of verification of key functionalities (phase 3) using open-source models with annotated system messages and filtered system messages. \textit{Italic} text part is filled with input text.}
\label{tab:app_prompt_system_tag_check}
% \vspace{-0.3cm}
\end{table*}
\begin{table*}[t]
\centering
{\resizebox{\textwidth}{!}{
\begin{tabular}{l}
\toprule \midrule
\begin{tabular}[c]{@{}l@{}}The user instruction will be provided, along with two assistant responses.\\
Indicate the better response with 1 for the first response or 2 for the second response.\\ \\
User Instruction: {\textit{User Instruction}}\\ 
Assistant Response 1: {\textit{Original Answer}} \\ 
Assistant Response 2: {\textit{Newly-generated Answer}} \\ 
Which of the above two responses better adheres to the instruction? (Respond with 1 or 2)
\end{tabular} \\ \midrule
\bottomrule
\end{tabular}}}
\caption{The prompt of answer quality check through the proprietary model (e.g., GPT4o). \textit{Italic} text part is filled with input text.}
\label{tab:app_prompt_answer_quality_check}
% \vspace{-0.5cm}
\end{table*}

\end{document}
