\begin{figure*}[]
\centering
\includegraphics[width=\textwidth]{figure/data_construction.pdf}
% \vspace{-0.3cm}
\caption{
Overall \textbf{\textsc{SysGen}} data construction pipeline. Our pipeline consists of four phases:
% (Phase 1 - System Message Generation) We first collect supervised fine-tuning (SFT) datasets which does not contain system message. Then, we use open-source models to generate system message given question and answer pair. Our system message is composed of phrases with manually selected eight key functionalities tags.
(Phase 1) We gather SFT datasets which do not contain system messages and use open-source models to generate system messages with manually selected eight key fuctionality tags. 
% (Phase 2) We apply a filtering process to remove cases where tag tokens are generated incorrectly. To ensure consistency in system messages, we reorganize the tags in a predefined order.
(Phase 2) We then remove incorrectly generated tag tokens and reorganize tags with phrases in a predefined order for consistency.
% (Phase 3) To verify the correctness of phrases generated for each tag, we introduce a self-checking LLM-as-a-judge mechanism. Phrases that are empty, overly specific, or unnatural are classified as Bad and subsequently removed along with their corresponding tags.
(Phase 3) We use a LLM-as-a-judge approach with self-model feedback to filter out empty, overly specific, and unnatural phrases.
% (Phase 4) We refine the generated system message by removing tags to create a more natural sentence. Based on this refined system message, we generate a new answer along with the user instruction.
(Phase 4) We finally remove tags to create natural system messages and generate new responses along with the user instructions.
}
\label{fig:data_construction}
\vspace{-0.3cm}
\end{figure*}