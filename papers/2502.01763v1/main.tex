\documentclass[11pt]{article}
\setlength{\parskip}{0.5\baselineskip}%
\setlength{\parindent}{0pt}

\usepackage{Shorthands_icml}
\usepackage[margin=1in]{geometry}
\usepackage{amsthm, amsmath}


\renewcommand{\epsilon}{\varepsilon}


\def\ddefloop#1{\ifx\ddefloop#1\else\ddef{#1}\expandafter\ddefloop\fi}
\def\ddef#1{\expandafter\def\csname bb#1\endcsname{\ensuremath{\mathbb{#1}}}}
\ddefloop ABCDEFGHIJKLMNOPQRSTUVWXYZ\ddefloop

\def\ddefloop#1{\ifx\ddefloop#1\else\ddef{#1}\expandafter\ddefloop\fi}
\def\ddef#1{\expandafter\def\csname frak#1\endcsname{\ensuremath{\mathfrak{#1}}}}
\ddefloop ABCDEFGHIJKLMNOPQRSTUVWXYZ\ddefloop

\def\ddefloop#1{\ifx\ddefloop#1\else\ddef{#1}\expandafter\ddefloop\fi}
\def\ddef#1{\expandafter\def\csname fr#1\endcsname{\ensuremath{\mathfrak{#1}}}}
\ddefloop ABCDEFGHIJKLMNOPQRSTUVWXYZ\ddefloop


\def\ddefloop#1{\ifx\ddefloop#1\else\ddef{#1}\expandafter\ddefloop\fi}
\def\ddef#1{\expandafter\def\csname eul#1\endcsname{\ensuremath{\EuScript{#1}}}}
\ddefloop ABCDEFGHIJKLMNOPQRSTUVWXYZ\ddefloop


\def\ddefloop#1{\ifx\ddefloop#1\else\ddef{#1}\expandafter\ddefloop\fi}
\def\ddef#1{\expandafter\def\csname scr#1\endcsname{\ensuremath{\mathscr{#1}}}}
\ddefloop ABCDEFGHIJKLMNOPQRSTUVWXYZ\ddefloop

\def\ddefloop#1{\ifx\ddefloop#1\else\ddef{#1}\expandafter\ddefloop\fi}
\def\ddef#1{\expandafter\def\csname b#1\endcsname{\ensuremath{\mathbf{#1}}}}
\ddefloop ABCDEGHIJKLMNOPQRSTUVWXYZ\ddefloop


\def\ddefloop#1{\ifx\ddefloop#1\else\ddef{#1}\expandafter\ddefloop\fi}
\def\ddef#1{\expandafter\def\csname bhat#1\endcsname{\ensuremath{\hat{\mathbf{#1}}}}}
\ddefloop ABCDEFGHIJKLMNOPQRSTUVWXYZ\ddefloop

\def\ddefloop#1{\ifx\ddefloop#1\else\ddef{#1}\expandafter\ddefloop\fi}
\def\ddef#1{\expandafter\def\csname btil#1\endcsname{\ensuremath{\tilde{\mathbf{#1}}}}}
\ddefloop ABCDEFGHIJKLMNOPQRSTUVWXYZ\ddefloop

\def\ddefloop#1{\ifx\ddefloop#1\else\ddef{#1}\expandafter\ddefloop\fi}
\def\ddef#1{\expandafter\def\csname bst#1\endcsname{\ensuremath{\mathbf{#1}^\star}}}
\ddefloop ABCDEFGHIJKLMNOPQRSTUVWXYZ\ddefloop

\def\ddefloop#1{\ifx\ddefloop#1\else\ddef{#1}\expandafter\ddefloop\fi}
\def\ddef#1{\expandafter\def\csname bst#1\endcsname{\ensuremath{\mathbf{#1}^\star}}}
\ddefloop abcdeghijklmnopqrstuvwxyz\ddefloop

\def\ddefloop#1{\ifx\ddefloop#1\else\ddef{#1}\expandafter\ddefloop\fi}
\def\ddef#1{\expandafter\def\csname bhat#1\endcsname{\ensuremath{\hat{\mathbf{#1}}}}}
\ddefloop abcdefghijklmnopqrstuvwxyz\ddefloop

\def\ddefloop#1{\ifx\ddefloop#1\else\ddef{#1}\expandafter\ddefloop\fi}
\def\ddef#1{\expandafter\def\csname b#1\endcsname{\ensuremath{\mathbf{#1}}}}
\ddefloop abcdeghijklnopqrstuvwxyz\ddefloop

\def\ddefloop#1{\ifx\ddefloop#1\else\ddef{#1}\expandafter\ddefloop\fi}
\def\ddef#1{\expandafter\def\csname barb#1\endcsname{\ensuremath{\bar{\mathbf{#1}}}}}
\ddefloop abcdefghijklmnopqrstuvwxyz\ddefloop

\def\ddef#1{\expandafter\def\csname c#1\endcsname{\ensuremath{\mathcal{#1}}}}
\ddefloop ABCDEFGHIJKLMNOPQRSTUVWXYZ\ddefloop
\def\ddef#1{\expandafter\def\csname h#1\endcsname{\ensuremath{\widehat{#1}}}}
\ddefloop ABCDEFGHIJKLMNOPQRSTUVWXYZ\ddefloop
\def\ddef#1{\expandafter\def\csname hc#1\endcsname{\ensuremath{\widehat{\mathcal{#1}}}}}
\ddefloop ABCDEFGHIJKLMNOPQRSTUVWXYZ\ddefloop
\def\ddef#1{\expandafter\def\csname t#1\endcsname{\ensuremath{\widetilde{#1}}}}
\ddefloop ABCDEFGHIJKLMNOPQRSTUVWXYZ\ddefloop
\def\ddef#1{\expandafter\def\csname tc#1\endcsname{\ensuremath{\widetilde{\mathcal{#1}}}}}
\ddefloop ABCDEFGHIJKLMNOPQRSTUVWXYZ\ddefloop




\title{On The Concurrence of Layer-wise Preconditioning Methods\\ and Provable Feature Learning} 
\author{
Thomas T.\ Zhang\footnotemark[1]\quad
Behrad Moniri\footnotemark[1]\quad
Ansh Nagwekar\quad
Faraz Rahman\quad
Anton Xue\\[0.2cm]
Hamed Hassani\quad
Nikolai Matni\\[0.4cm]
{\normalsize University of Pennsylvania}
}
\date{}

\usepackage{natbib}

\theoremstyle{plain}
\newtheorem{theorem}{Theorem}[section]
\newtheorem{proposition}[theorem]{Proposition}
\newtheorem{lemma}[theorem]{Lemma}
\newtheorem{corollary}[theorem]{Corollary}
\theoremstyle{definition}
\newtheorem{definition}[theorem]{Definition}
\newtheorem{assumption}[theorem]{Assumption}
\theoremstyle{remark}
\newtheorem{remark}[theorem]{Remark}

\newif\ifshort
\shorttrue
\usepackage[hang,flushmargin]{footmisc}

\begin{document}
\maketitle
\renewcommand{\thefootnote}{\fnsymbol{footnote}} %
\footnotetext[1]{Equal Contribution. Correspondence to: \texttt{\{ttz2, bemoniri\}@seas.upenn.edu}\vspace{0.05cm}}
\renewcommand{\thefootnote}{\arabic{footnote}} %

\let\cite\citep

\begin{abstract}
Layer-wise preconditioning methods are a family of memory-efficient optimization algorithms that introduce preconditioners per axis of each layer's weight tensors. These methods have seen a recent resurgence, demonstrating impressive performance relative to entry-wise (``diagonal'') preconditioning methods such as Adam(W) on a wide range of neural network optimization tasks. Complementary to their practical performance, we demonstrate that layer-wise preconditioning methods are provably necessary from a statistical perspective.
To showcase this, we consider two prototypical models, \textit{linear representation learning} and \textit{single-index learning}, which are widely used to study how typical algorithms efficiently learn useful \textit{features} to enable generalization.
In these problems, we show SGD is a suboptimal feature learner when extending beyond ideal isotropic inputs $\bfx \sim \mathsf{N}(\mathbf{0}, \mathbf{I})$ and well-conditioned settings typically assumed in prior work.
We demonstrate theoretically and numerically that this suboptimality is fundamental, and that layer-wise preconditioning emerges naturally as the solution. We further show that standard tools like Adam preconditioning and batch-norm only mildly mitigate these issues, supporting the unique benefits of layer-wise preconditioning.
\end{abstract}



\vspace{-0.2cm}
\section{Introduction}
\vspace{-0.1cm}

    


Well-designed optimization algorithms have been an enabler to the staggering growth and success of machine learning. For the broader ML community, the \Adam \citep{kingma2014adam} optimizer is likely the go-to scalable and performant choice for most tasks. However, despite its popularity in practice, it has been notoriously challenging to understand \Adam-like optimizers theoretically, especially from a \emph{statistical} (e.g.\ generalization) perspective.\footnote{ To be contrasted with an ``optimization'' perspective, e.g.\ guarantees of convergence to a critical point of  the \emph{training} objective.} In fact, there exist many theoretical settings where \Adam and similar methods underperform in convergence or generalization relative to, e.g., well-tuned \SGD (see e.g.\ \citet{wilson2017marginal, keskar2017improving, reddi2018convergence, gupta2021adam, xie2022adaptive, dereich2024non}), further complicating a principled understanding the role of \Adam-like optimizers in deep learning.
Given these challenges, is there an alternative algorithmic paradigm that is comparable to the \Adam family in practice that is also well-motivated from a statistical learning perspective?
Encouragingly, in a recent large-scale deep learning optimization competition, AlgoPerf \citep{mlcommons2024algoperf}, \Adam and its variants were outperformed in various ``hold-out error per unit-compute''\footnote{See \citet[Section 4.2]{dahl2023benchmarking} for details.} metrics by a method known as \Shampoo \citep{gupta2018shampoo}, a member of a layer-wise ``Kronecker-Factored'' family of preconditioners, formally described in \Cref{sec:K-FAC-approx}, contrasted with ``diagonal'' preconditioning methods like \Adam. 

Notable members of the Kronecker-Factored preconditioning family include \Shampoo and \KFAC \citep{martens2015optimizing}, as well as their many variants and descendants. These algorithms are motivated from an approximation-theoretic perspective, aiming to approximate some \textit{ideal} curvature matrix (e.g.\ the Hessian or Fisher Information) in a way that mitigates the computational and memory challenges associated with second-order algorithms such as Newton's Method (\NM) or Natural Gradient Descent (\NGD). However, towards establishing the benefit of these preconditioners, an approximation viewpoint is bottlenecked by our limited understanding of how the \textit{idealized} second-order methods perform on neural-network learning tasks, even disregarding the computational considerations.
It in fact remains unclear whether these second-order methods are inherently superior to the approximations designed to emulate them. For example, recent work has shown that, surprisingly, \KFAC generally \emph{outperforms} its ideal counterpart \NGD in convergence rate and generalization error on typical deep learning tasks \citep{benzing2022gradient}. Thus, a key question remains:
\begin{quote}
\centering
\textit{How do we explain the performance of Kronecker-Factored preconditioned optimizers?}
\end{quote}
In a seemingly distant area, the learning theory community has been interested in studying the solutions learned by 
abstractions of ``typical'' deep learning set-ups, where the overall goal is to theoretically demonstrate how neural networks \textit{learn} features from data to perform better than classical ``fixed features" methods, e.g.\ kernel machines.
Much of this line of work focuses on analyzing the performance of \SGD on simplified models of deep learning (see e.g.\ \citet{collins2021exploiting, damian2022neural,ba2022high,barak2022hidden,abbe2023sgd,dandi2023learning,berthier2024learning,nichanitransformers, collins2024provable}). Almost invariably, certain innocuous-looking assumptions are made, such as isotropic covariates $\bfx \sim \normal(\mathbf{0}, \bI)$. Under these conditions, \SGD has been shown to exhibit desirable generalization properties. However, some works deviate from these assumptions in specific settings \cite{amari2020does, zhang2023meta}, and suggest that \SGD can exhibit severely suboptimal generalization. 
Thus, toward extending our understanding of feature learning, it seems beneficial to consider a broader family of optimizers. This raises the following question:
\begin{quote}
    \centering
    \textit{What is a \emph{practical} family of optimization algorithms that overcomes the deficiencies of \SGD for standard feature learning tasks?}
\end{quote}
We answer the above two questions by focusing on two prototypical problems used to theoretically study feature learning: \textit{linear representation learning} and \textit{single-index learning}. In both problems, we show that \SGD is clearly suboptimal outside ideal settings, such as when the ubiquitous isotropic data $\normal(\mathbf{0}, \mathbf{I})$ assumption is violated. By inspecting the root cause behind these suboptimalities, we show that Kronecker-Factored preconditioners arise naturally as a \emph{first-principles} solution to these issues. We provide novel non-approximation-theoretic motivations for this class of algorithms, while establishing new and improved learning-theoretic guarantees. We hope that this serves as strong evidence of an untapped synergy between deep learning optimization and feature learning theory.
\paragraph{Contributions.} Here we discuss the main contributions of the paper.
\begin{itemize}[left=0pt]
\vspace{-.7em}

    \item We study the linear representation learning problem under general anisotropic $\bfx \sim \normal(\mathbf{0}, \bSigma_\bfx)$ covariates and show that the convergence of \SGD can be drastically slow, even under mild anisotropy. Also, the convergence rate suffers an undesirable dependence on the ``conditioning'' of the instance even for ideal step-sizes. We arrive at a variant of \KFAC as the natural solution to these deficiencies of \SGD, giving rise to the first \emph{condition-number-free convergence rate} for the problem (Section~\ref{sec:lin_rep}).
    \vspace{0.2cm}
    \item Next, we consider the problem of learning a single-index model using a two-layer neural network in the high-dimensional proportional limit. We show that for anisotropic covariates $\bfx \sim \normal(\mathbf{0}, \bSigma_\bfx)$, \SGD fails to learn useful features, whereas it is known that it learns suitable features in the isotropic setting. Furthermore, we show that \KFAC is a natural fix to \SGD, greatly enhancing the learned features in anisotropic settings (Section~\ref{sec:single_index}).

    \item Lastly, we carefully numerically verify our theoretical predictions. Notably, we confirm the findings in \citet{benzing2022gradient} that full second-order methods heavily underperform \KFAC in convergence rate and stability. We also show standard tools like \Adam-like preconditioning and batch-norm \cite{ioffe2015batch} do not fix the issues we identify, even for our simple models, and may even \emph{hurt} generalization in the latter's case.

    
\end{itemize}

\vspace{-0.3cm}

In addition to the works discussed earlier, we provide extensive related work and background in \Cref{sec:related_work}.
\vspace{-0.4cm}
\paragraph{Notation.} We denote vector quantities by \textbf{bold} lower-case, and matrix quantities by \textbf{bold} upper-case. We use $\odot$ to denote element-wise (Hadamard) product, $\otimes$ for Kronecker product, and $\VEC(\cdot)$ the \emph{column-major} vectorization operator. Positive (semi-)definite matrices are denoted by $\bQ \succ\mem(\succeq)\; \bzero$, and the corresponding partial order $\bP \preceq \bQ \implies \bQ - \bP \succeq \bzero$. We use $\opnorm{\cdot}$, $\norm{\cdot}_F$ to denote the operator (spectral) and Frobenius norms, and $\kappa(\bA) = \smax(\bA)/\smin(\bA)$ denote the condition number. We use $\Ex[f(\bfx)]$ to denote the expectation of $f(\bfx)$, and $\prob[A(\bfx)]$ to denote the probability of event $A(\bfx)$. Given a batch $\scurly{\bfx_i}_{i=1}^n$, we denote the \emph{empirical} expectation $\hatEx[f(\bfx)] = \frac{1}{n}\sum_{i=1}^n f(\bfx_i)$. Given an indexed set of vectors, we use the upper case to denote the (row-wise) stacked matrix, e.g.\ $\bX \triangleq \bmat{\bfx_1 & \cdots & \bfx_n}^\top \in \R^{n \times \dx}$. We reserve $\bSigma$ ($\hatSigma[]$) for (sample) covariance matrices, e.g.\ $\bSigma_\bfx = \Ex[\bfx \bfx^\top]$, $\hatSigma = \hatEx[\bfx \bfx^\top] = \frac{1}{n} \bX^\top \bX$. We use $\lesssim, \gtrsim, \approx$ to omit universal numerical constants, and standard asymptotic notation $o(\cdot), \calO(\cdot), \Omega(\cdot), \Theta(\cdot)$. Lastly, we use the index shorthand $[n] = \scurly{1,\dots,n}$, and subscript $\,_+$ to denote the ``next iterate'', e.g.\ $\bG_+ = \mathrm{Next}(\bG)$.


\vspace{-0.3cm}
\section{Kronecker-Factored Approximation}
\label{sec:K-FAC-approx}

One of the longest-standing research efforts in optimization literature is dedicated to understanding the role of (local) curvature toward accelerating convergence rates of optimization methods. An example is Newton's method, where the curvature matrix (Hessian) serves as a preconditioner of the gradient, enabling one-shot convergence in quadratic optimization, in which gradient descent enjoys at best a linear convergence rate dictated by the conditioning of the problem.
However, for high-dimensional variables, computing and storing the full curvature matrix is often infeasible. Thus enter Quasi-Newton and (preconditioned) Conjugate Gradient methods, where the goal is to reap the benefits of curvature under  computational or structural specifications, such as \{block-diagonal, low-rank, sparsity, etc.\} constraints (e.g.\ \BFGS family \citep{goldfarb1970family, liu1989limited, nocedal1999numerical}), or accessing the curvature matrix only through matrix-vector products (see e.g.\ \citet{pearlmutter1994fast,schraudolph2002fast,martens2010deep}). 

Nevertheless, the use of these methods for neural network optimization introduces new considerations. Consider an $L$-layer fully-connected neural network (omitting biases) $$f_{\btheta}(\bfx) \triangleq \bW_L \sigma (\bW_{L-1} \cdots \sigma(\bW_1 \bfx) \cdots ),$$
where $\bW_\ell \in \R^{d \times d},\; \ell \in [L]$ and $\btheta \in \R^{L d^2}$ is the concatenation of $\btheta_\ell \triangleq \VEC(\bW_\ell)$, $\ell \in [L]$.
Firstly, establishing convergence of \SGD \citep{arora2019implicit}, \NGD \citep{zhang2019fast}, or Gauss-Newton \citep{cai2019gram} (or their corresponding gradient flows) to global minima of the \emph{training} objective is non-trivial, as optimization over $\btheta$ is non-convex. Moreover, these results do not directly characterize the structure of the resulting features learned by the algorithms.
Secondly, on the practical front, full preconditioners on $\btheta$ require memory $\calO(L^2 d^4)$, which grows prohibitively with depth and width. Block-diagonal approximations (where one curvature block $\bM_{\ell} \in \R^{d^2 \times d^2}$ corresponds to a layer $\btheta_\ell$) still require $\calO(L d^4)$. Thus, entry-wise preconditioning as in \Adam, with footprint $\calO(L d^2) \approx \dim(\btheta)$, is usually considered the only scalable class of preconditioners.


However, a distinct notion of ``Kronecker-Factored'' preconditioning emerged approximately concurrently with \Adam, with representative examples such as \KFAC and \Shampoo. As its name suggests, since full block-diagonal approximations are too expensive, a Kronecker-Factored approximation is made instead, where $\bM_\ell^{-1} \nabla_{\btheta_\ell} \calL(\btheta) = (\bQ_\ell \otimes \bP_\ell)^{-1}  \nabla_{\btheta_\ell} \calL(\btheta)$, $\bP_\ell, \bQ_\ell \succeq \bzero$. Using properties of the Kronecker product (see \Cref{lem:kron_properties}), this has the convenient interpretation of pre- and post-multiplying the weights in their \textit{matrix form}:
\begin{align}\label{eq:kf_precond}
    (\bQ_\ell \otimes \bP_\ell)^{-1}  \nabla_{\btheta_\ell} \calL(\btheta) \iff \bP_\ell^{-1}  \nabla_{\bW_\ell} \calL(\btheta) \bQ_\ell^{-1}.
\end{align}
As such, the memory requirement of Kronecker-Factored layer-wise preconditioning is $\calO(Ld^2)$, matching that of entry-wise preconditioning.
The notion of curvature differs from case to case, e.g., for \KFAC, this is the Fisher Information matrix corresponding to the distribution parameterized by $f_{\btheta}(\bfx)$, whereas for \Shampoo this is the full-matrix Adagrad preconditioner, in turn closely related to the Gauss-Newton matrix.\footnote{This is itself a positive-definite approximation of the Hessian.} We provide some sample derivations and background in \Cref{sec:KF_derivations}. However, as aforementioned, an approximation viewpoint falls short of explaining the practical performance of Kronecker-Factored methods, 
as they typically converge \textit{faster} than their corresponding second-order method \citep{benzing2022gradient} on deep learning tasks.
This motivates understanding the unique benefits of layer-wise preconditioning methods from first principles, which brings us to the following section.













\section{Feature Learning via Kronecker-Factored Preconditioning}

We present two prototypical models of feature learning, \textit{linear representation learning} and \textit{single-index learning}, and demonstrate how typical guarantees for the features learned by \SGD break down outside of idealized settings. We then show how to rectify these issues by deriving a modified algorithm from first principles, and demonstrate that both cases in fact coincide with a particular Kronecker-factored preconditioning method. We now set-up the model architecture and algorithm primitive considered in both problems. We consider two-layer feedforward neural network predictors:
\begin{align}\label{eq:two_layer_net}
    \fFG(\bfx) = \bF \sigma(\bG \bfx),
\end{align}
where $\bF \in \R^{\dy \times \dhid}$, $\bG \in \R^{\dhid \times \dx}$ denote the weight matrices and $\sigma(\cdot)$ is a predetermined activation function. For scalar outputs $\dy = 1$, we use $\ffG(\bfx) = \bff^\top \sigma(\bG \bfx)$. For our purposes, we omit the bias vectors from both layers. We further denote the intermediate covariate pre- and post-activation $\bfh \triangleq \bG \bfx$, $\bfz \triangleq \sigma(\bG \bfx)$. We consider a standard mean-squared-error (MSE) regression objective and its (batch) empirical counterpart:
\begin{align}\label{eq:regression_loss}
    \calL(\bF, \bG) \triangleq \frac{1}{2} \Ex_{(\bfx, \bfy)}\mem\brac{\norm{\bfy - \fFG(\bfx)}^2}, \quad
    \hatL(\bF, \bG) \triangleq \frac{1}{2} \hatEx \mem\brac{\norm{\bfy - \fFG(\bfx)}^2}.
\end{align}
Given a batch of inputs $\curly{\bfx_i}_{i=1}^n$, we define the left and right preconditioners of the two layers (recall equation~\eqref{eq:kf_precond}):
\begin{align}\label{eq:precond}
    \begin{split}
        &\bQ_\sbG = \hatSigma[\bfx] =  \hatEx[\bfx \bfx^\top], \quad \bQ_\sbF = \hatSigma[\bfz] = \hatEx[\bfz\bfz^\top],\\[0.2cm]
        &\bP_\sbG = \hatEx\brac{\paren{\pder{\fFG}{\bfh}}^\top \pder{\fFG}{\bfh}} (\text{or } \bI_{\dhid}), \quad \bP_{\sbF} = \bI_{\dy}.
    \end{split}
\end{align}
We introduce the flexibility of $\bP_{\sbG} = \bI_{\dhid}$ for when $\bP_{\sbG}$ does not play a significant role; notably, this recovers certain Kronecker-Factored preconditioners that avoid extra backwards passes (see \Cref{sec:KF_derivations}).
We consider a stylized alternating descent primitive, where we iteratively perform
\begin{align}\label{eq:KFAC_update}
    \begin{split}
        \bG_+ &= \bG - \eta_{\sbG} \bP_\sbG^{-1}\, \nabla_{\sbG} \hatL(\bfF, \bfG)\, (\bQ_\sbG + \lambda_\sbG \bI_{\dx})^{-1} \\[0.2cm]
        \bF_+ &= \bF - \eta_\sbF \bP_\sbF^{-1}\, \nabla_\sbF \hatL(\bfF, \bfG_+)\, \bQ_{\sbF}^{-1},
    \end{split}
\end{align}
where $\eta_\sbG, \eta_\sbF > 0$ are layer-wise learning rates, and $\lambda_\sbG \in \R$ is a regularization parameter.
In line with most prior work, we consider an alternating scheme for analytical convenience.
We also assume that $\bG_+, \bF_+$ are computed on \emph{independent batches} of data, equivalent to sample-splitting strategies in prior work (\citet{collins2021exploiting,zhang2023meta,ba2022high,moniri_atheory2023} etc).


The preconditioners \eqref{eq:precond} and update \eqref{eq:KFAC_update} bear a striking resemblance to \KFAC (cf.\ \Cref{sec:KF_derivations}). In fact, the preconditioners align exactly with \KFAC if we view $\LFG$ as a negative log-likelihood of a conditionally Gaussian model with fixed variance: $\widehat \bfy(\bfx) \sim \normal(\fFG(\bfx), \sigma^2 \bI)$. This is in some sense a coincidence (and a testament to the prescience of \KFAC's design): rather than deriving the above preconditioners via approximating the Fisher Information matrix, we will show shortly how they arise as a natural adjustment to \SGD in our featured problems. We note that Kronecker-Factored preconditioning methods often involve further moving parts such as damping exponents $\bP^{-\rho}$, additional ridge parameters on various preconditioners, and momentum. Though of great importance in practice, they are beyond the scope of this paper,\footnote{We refer the interested reader to \citet{ishikawa2023parameterization} for discussion of these settings in the ``maximal-update parameterization'' framework \cite{yang2021tensor}.} and we only feature parameters that play a role in our analysis.

A convenient observation that unifies various stylized algorithms on two-layer networks is that the $\bF$-update in \eqref{eq:KFAC_update} can be interpreted as an exponential moving average (EMA) over least-squares estimators conditioned on $\bfz$.
\begin{lemma}\label{lem:KFAC_is_EMA}
    Given $\bG$, define the least-squares estimator:
    \begin{align*}
        \Fls &\triangleq \argmin_{\widehat{\bF}} \frac{1}{2}\hatEx\big[\norm{\bfy - \widehat{\bF} \underbrace{\sigma(\bG \bfx)}_{\bfz}}^2\big]= \bY^\top \bZ \, (\bZ^\top \bZ)^{-1}
    \end{align*}
    Given $\eta_\sbF \in (0, 1]$, then the $\bF$-update in \eqref{eq:KFAC_update} can be re-written as an EMA of $\Fls$; i.e., $$\bF_+ = (1 - \eta_\sbF) \bF + \eta_\sbF \Fls.$$
\end{lemma}
In particular, many prior works (e.g.\ \citet{collins2021exploiting, nayer2022fast, thekumparampil2021sample, zhang2023meta}) consider an alternating ``minimization-descent'' approach, where out of analytical convenience $\bF$ is updated by performing least-squares regression holding the hidden layer $\bG$ fixed. In light of \Cref{lem:KFAC_is_EMA}, this corresponds to the case where $\eta_\sbF = 1$.

\subsection{Linear Representation Learning}\label{sec:lin_rep}

Assume we have data generated by the following process
\begin{align}\label{eq:linrep_datagen}
    \bfx_i \iidsim \normal(0, \bSigma_\bfx), \quad \bfy_i = \Fstar \Gstar \bfx_i + \bveps_i,
\end{align}
where $\bSigma_\bfx$ is the input covariance, $\Fstar \in \R^{\dy \times k}$, $\Gstar \in \R^{k \times \dx}$ are (unknown) rank-$k$ matrices, and $\bveps_i \iidsim \normal(0, \Sigmaeps)$ is additive label noise independent of all other randomness. We consider Gaussian data throughout the paper for conciseness; all results in this section can be extended to subgaussian $\bfx, \bveps$ via standard tools, affecting only the universal constants (see \Cref{sec:aux_results}). Let us define $\sigmaeps^2 \triangleq \lmax(\Sigmaeps)$. Accordingly, our predictor model is a two-layer feed-forward linear network \eqref{eq:two_layer_net} with $\bF \in \R^{\dy \times k}$, $\bG \in \R^{k \times \dx}$.

The goal of linear representation learning is to learn the low-dimensional feature space that $\Gstar$ maps to, which is equivalent to determining its row-space $\rowsp(\Gstar)$. Recovering $\Fstar, \Gstar$ is an ill-posed problem, as for any invertible $\bL \in \R^{k \times k}$, the matrices $\Fstar \bL$, $\bL^{-1} \Gstar$ remain optimal. Therefore, we measure recovery of $\rowsp(\Gstar)$ via a \textit{subspace distance}.
\begin{definition}[Subspace Distance {\citep{stewart1990matrix}}]\label{def:subspace_dist}
    Let $\bG, \Gstar \in \R^{k \times \dx}$ be matrices whose rows are orthonormal.
    Let $\Gperp \in \R^{\dx \times \dx}$ be the projection matrix onto $\rowsp(\Gstar)^\perp$. Define the distance between the subspaces spanned by the rows of $\bG$ and $\Gstar$ by
    \begin{align}\label{eq: subspace distance}
        \dist(\bG, \Gstar) &\triangleq \norm{\bG \Gperp}_{\rm op}
    \end{align}
\end{definition}
The subspace distance quantitatively captures the alignment between two subspaces, ranging between $0$ (occurring iff $\rowsp(\Gstar) = \rowsp(\bG)$) and $1$ (occurring iff $\rowsp(\Gstar) \perp \rowsp(\bG)$).
We further make the following non-degeneracy assumptions.
\begin{assumption}\label{assump:full_rank_orth}
    We assume $\Gstar$ is row-orthonormal, and $\Fstar$ is full-rank, $\rank(\Fstar) = k \leq \dy$. This is without loss of generality: if $k > \dy$, then recovering a $k$-dimensional row-space from $\bfy_i$ is underdetermined. If $\rank(\Fstar) = k' < k$, then it suffices to consider $\Gstar \in \R^{k' \times \dx}$.
\end{assumption}
The linear representation learning problem has often been studied in the context of multi-task learning \citep{du2020few, tripuraneni2020theory, collins2021exploiting, thekumparampil2021sample, zhang2023meta}.

\begin{remark}[Multi-task Learning]\label{rem:multitask}
    Multi-task learning considers data generated as $\yit = \Ftstar \Gstar \xit + \epsit$ for distinct tasks $t = 1,\dots, T$, with the same goal of recovering the \textit{shared} representation $\Gstar$. Our algorithm and guarantees naturally extend here, see \Cref{sec:extension_multi_task} for full details. In particular, by embedding $\Fstar = \bmat{\Ftstar[1]^\top & \cdots & \Ftstar[T]^\top}^\top$, \Cref{assump:full_rank_orth} is equivalent to the ``task-diversity'' conditions in the above works: $\rank(\Fstar) = \rank\Big(\sum_{t=1}^T \Ftstar^\top \Ftstar \Big) = k$.
\end{remark}
We maintain the ``single-task'' setting in this section for concise bookkeeping while preserving the essential features of the representation learning problem.
Various algorithms have been proposed toward provably recovering the representation $\Gstar$. A prominent example is an alternating minimization-\SGD scheme \citep{collins2021exploiting, vaswani2024efficient}.
In the cited works, a local convergence result\footnote{By local convergence here we mean $\dist(\bG, \Gstar)$ is sufficiently (but non-vanishingly) small.} is established for isotropic data $\bSigma_\bfx = \bI_{\dx}$. In \citet{zhang2023meta}, it is shown that using \SGD  can drastically slow convergence even under mild anisotropy; their proposed algorithmic adjustment equates to applying the right-preconditioner $\bQ_\sbG = \hatSigma$. However, their local convergence result suffers a dependence on the condition number of $\bF_\star$, slowing the linear convergence rate for ill-conditioned $\bF_\star$.
Let us now specify the algorithm template used in this section, that also encompasses the above work:
\begin{align}\label{eq:linrep_update}
    \overline\bG_+ \text{ via \eqref{eq:KFAC_update}},\; \bG_+ = \mathrm{Ortho}(\overline\bG_+),\; \bF_+ \text{ via \eqref{eq:KFAC_update}}.
\end{align}
Notably, we row-orthonormalize the representation after each update. Besides ease of analysis, we have observed this numerically mitigates the elements of $\bG$ from blow-up when running variants of \SGD. The alternating min-\SGD algorithms in \citet{collins2021exploiting, vaswani2024efficient} are equivalent to iterating \eqref{eq:linrep_update} setting $\bP_\sbG = \bQ_\sbG = \bI$, $\eta_\sbF = 1$ in \eqref{eq:precond}, whereas \citet{zhang2023meta} use $\bP_\sbG = \bI$, $\bQ_\sbG = \hatSigma$, $\eta_\sbF = 1$. 
Let us now write out the full-batch gradient update.

\paragraph{Full-Batch \SGD.}
Given a fresh batch of data $\xybatch$, and current weights $(\bF, \bG)$, we have the representation gradient and corresponding \SGD step:
\begin{align}
    \nabla_{\sbG} \hatL(\bfF, \bfG) &= \frac{1}{n}\bfF^\top \mem\paren{\bfF \bG \bX^\top \bX - \bY^\top \bX} \nonumber \\[0.2cm]
    \bG_+ &= \bG - \eta_\sbG \nabla_{\sbG} \hatL(\bfF, \bfG). \label{eq:linrep_grad}
\end{align}
When $\bfx$ is isotropic $\bSigma_\bfx = \bI_{\dx}$, the key observation is that by multiplying both sides of \eqref{eq:linrep_grad} by $\Gperp$, recalling $\bY^\top  = \Fstar \Gstar \bX^\top + \Eps^\top$, we have
\begin{align*}
    \overline\bG_+\Gperp \mem&= \mem\mem\paren{\mem\bG - \eta_\sbG \bfF^\top\mem\paren{(\bfF \bG - \Fstar \Gstar) \hatSigma - \frac{1}{n}\Eps^\top \bX}\mem}\mem\Gperp \\[0.2cm]
    &\approx (\bI_{k} - \eta_\sbG \bfF^\top \bfF) \bG \Gperp  + \frac{\eta_\sbG}{n} \bF^\top \Eps^\top \bX \Gperp,
\end{align*}
where the approximate equality hinges on covariance concentration $\hatSigma \approx \bI_{\dx}$ and $\Gstar\Gperp = \bzero$. Therefore, in the isotropic setting, for sufficiently large $n \gtrsim \dx$, and appropriately chosen $\eta_\sbG \approx \frac{1}{\lmax(\Fstar^\top \Fstar)}$, then (omitting many details) we have the one-step contraction \citep{collins2021exploiting, vaswani2024efficient}:
\begin{align}
        \hspace{-0.3cm}\dist(\bG_+, \Gstar) &\lesssim \paren{1 - \frac{\lmin(\Fstar^\top \Fstar)}{\lmax(\Fstar^\top \Fstar)}} \dist(\bG, \Gstar)+ \calO\paren{\sigmaeps \sqrt{\dx/n}}, \label{eq:SGD_onestep}
\end{align}
where $\calO(\cdot)$ here hides different problem parameters depending on the analysis.
Therefore, in low-noise/large-batch settings, this demonstrates \SGD  on the representation $\bG$ converges geometrically to $\Gstar$ (in subspace distance). However, there are clear suboptimalities to \SGD. Firstly, the above analysis critically relies on $\Sigmax = \bI_{\dx}$ such that $\Gstar \hatSigma \Gperp \approx \bzero$. As aforementioned, this is demonstrated to be crucial in \citet{zhang2023meta} for $\dist(\bG, \Gstar)$ to converge using \SGD.
Secondly, the convergence of \SGD is bottlenecked by the conditioning of $\Fstar$. In fact, we show the dependence on $\Fstar$ in the contraction rate bound \eqref{eq:SGD_onestep} cannot be improved in general, even under the most benign assumptions.
Following \citet{collins2021exploiting, vaswani2024efficient}, we define $\bG_T = \SGD(\bG_0; \eta_\sbG, T)$ as the output of alternating min-\SGD, i.e.\ iterating \eqref{eq:linrep_update} setting $\bP_\sbG = \bQ_\sbG = \bI$, $\eta_\sbF = 1$ in \eqref{eq:precond}, for $T$ steps with fixed step-size $\eta_\sbG$ starting from $\bG_0$. 
\begin{restatable}{proposition}{LinRepLowerBound}\label{prop:linrep_SGD_lower_bound}
    Let $\Sigmax = \bI_{\dx}$, $n = \infty$. Choose any $\dx > k,\;\dy \geq k \geq 2$. Let the learner be given knowledge of $\Fstar, \Gstar$ and $\dist(\bG_0, \Gstar)$. However, assume the learner must fix $\eta_\sbG > 0$ before observing $\bG_0$. Then, there exists $\Fstar \in \R^{\dy \times k}$, $\Gstar, \bG_0 \in \R^{k \times \dx}$, such that
    $\bG_T = \SGD(\bG_0; \eta_\sbG, T)$ satisfies:
    \begin{align*}
        \dist(\bG_T, \Gstar) &\geq \paren{1 - 4 \frac{\lmin(\Fstar^\top \Fstar)}{\lmax(\Fstar^\top \Fstar)}}^T \dist(\bG_0, \Gstar).
    \end{align*}
\end{restatable}

The proof can be found in \Cref{sec:SGD_lower_bound}.
Since we set $\Sigmax = \bI$, the lower bound also holds for the algorithm in \citet{zhang2023meta}. We remark departing from a worst-case analysis to a generic performance lower bound, e.g.\ random initialization or varying step-sizes, is a nuanced topic even for the simple case of convex quadratics; see e.g.\ \citet{bach2024scaling, altschuler2023acceleration}.
In light of \Cref{prop:linrep_SGD_lower_bound} and \eqref{eq:linrep_grad}, a sensible alteration might be to \emph{pre- and post- multiply} $\nabla_{\sbG} \hatL(\bfF, \bfG)$ by $(\bF^\top \bF)^{-1}$ and $\hatSigma^{-1}$. These observations bring us to the proposed recipe in \eqref{eq:KFAC_update}.

\paragraph{Stylized \KFAC.}

By analyzing the shortcomings of the \SGD  update, we arrive at the proposed representation update:
\begin{align*}
    \overline\bG_+ &= \bG - \eta_\sbG (\bF^\top \bF)^{-1}  \nabla_{\sbG} \hatL(\bfF, \bfG) \,\hatSigma^{-1}.
\end{align*}
We can verify from \eqref{eq:precond} and \eqref{eq:KFAC_update} that $\bP_\sbG = \bF^\top \bF$ and $\bQ_\sbG = \hatSigma$.
Thus, we have recovered a stylized variant of \KFAC  as previewed. Our main result in this section is a local convergence guarantee.
\fussy
\begin{restatable}{theorem}{LinRepMainThm}\label{thm:linrep_kfac_guarantee}
    Consider running \eqref{eq:linrep_update} with $\lambda_\sbG = 0$, $\eta_\sbG \in [0,1]$, and $\eta_\sbF = 1$. Define $\overline\sigma^2 \triangleq \frac{\sigmaeps^2}{\smin(\Fstar)^2\lmin(\Sigmax)}$. As long as $\dist(\bG, \Gstar) \leq \frac{0.01}{\kappa(\Sigmax)\kappa(\Fstar)}$ and $n \gtrsim  \max\scurly{1,\overline\sigma^2} \paren{\dx + \log(1/\delta)}$, we have with probability $\geq 1 - \delta$:
    \begin{align*}
        \dist(\bG_+, \Gstar) &\leq (1 - 0.9\eta_\sbG) \dist(\bG, \Gstar) + \calO(1)\, \eta_\sbG \overline\sigma\sqrt{\frac{\dx+\log(1/\delta)}{n}}.
    \end{align*}
\end{restatable}
\sloppy
Crucially, the contraction factor is condition-number-free, subverting the lower bound in \Cref{prop:linrep_SGD_lower_bound} for sufficiently ill-conditioned $\Fstar$. Therefore, setting $\eta_\sbG$ near $1$ ensures a universal constant contraction rate.
Curiously, our proposed stylized \KFAC \eqref{eq:linrep_update} aligns with an alternating ``min-min'' scheme \cite{jain2013low, thekumparampil2021sample}, where $\bF,\bG$ are alternately updated via solving the convex quadratic least-squares problem, by setting $\eta_\sbF = \eta_\sbG = 1$. However, our experiments (see \Cref{fig:lr_sweep}) demonstrate $\eta_\sbG = 1$ is generally suboptimal, highlighting the flexibility of viewing \KFAC as a descent method.

\subsubsection{Transfer Learning}\label{sec:lin_rep_BN}
The upshot of representation learning is the ability to \emph{transfer} (e.g.\ fine-tune) to a distinct, but related, task by only retraining $\bF$ \citep{du2020few, kumar2022fine}. Assume we now have target data generated by:
\begin{align}\label{eq:linrep_target_datagen}
    \xtest[i] \iidsim \normal(0, \Sigmatest),\quad\; \ytest[i] = \Fstartest \Gstar \xtest[i] + \bveps_i,
\end{align}
where $\bveps_i \iidsim \normal(\bzero, \Sigmaeps)$, $\Fstartest \in \R^{\dy \times k}$. Notably, $\Gstar$ is shared with the ``training'' distribution \eqref{eq:linrep_datagen}. Given $\widehat\bG$ (e.g.\ by running \eqref{eq:linrep_update} on training task), we consider fitting the last layer $\bF$ given a batch of $\ntest$ data from the target task \eqref{eq:linrep_target_datagen}.
\begin{restatable}{lemma}{LinRepTransfer}\label{lem:linrep_transfer}
    Let $\Flstest = \argmin_{\widehat \bF} \; \hatExtest[\norm{\ytest - \widehat \bF \ztest}_2^2]$, $\ztest \triangleq \hatG \xtest$ be the optimal $\bF$ on the batch of $\ntest$ target data \eqref{eq:linrep_target_datagen} given $\hatG$. Defining $\nu = \dist(\hatG, \Gstar)$, given $\ntest \gtrsim k + \log(1/\delta)$, we have with probability $\geq 1-\delta$:
    \begin{align*}
        &\Ltest(\Flstest, \widehat \bG) \triangleq \Ex\brac{\norm{\ytest - \Fstartest\Gstar\xtest}_2^2} 
        \lesssim \norm{\Fstartest}^2_F\lmax(\Sigmatest) \nu^2  + \frac{\sigmaeps^2(\dy k + \log(1/\delta))}{\ntest}.
    \end{align*}
\end{restatable}

As hoped, the MSE of the fine-tuned predictor decomposes into a bias term scaling with the quality of $\hatG$, and a noise term scaling with $\dim(\bF)/\ntest$. We comment the required data is $\approx k$ rather than $\approx \dx$ resulting from doing regression from scratch \cite{wainwright2019high}. Additionally, the noise term scales with $\dim(\bF)=\dy k$ rather than $\dy\dx$ of the full predictor space.
The transfer learning set-up \eqref{eq:linrep_target_datagen} also reveals why data normalization (e.g.\ whitening, batch-norm \cite{ioffe2015batch}) can be counterproductive. To illustrate this, consider perfectly whitening the training covariates $\bfv = \Sigmax^{-1/2}\bfx$. By this change of variables, the ground-truth predictor changes $\bfy \approx \Fstar\Gstar \bfx = \Fstar\Gstar\Sigmax^{1/2}\bfv$. This is unproblematic so far---in fact, since the covariates $\bfv$ are isotropic, \SGD now may converge. However, instead of $\rowsp(\Gstar)$, the representation now converges to $\rowsp(\Gstar\Sigmax^{1/2})$. Deploying on the target task, since $\Sigmax \neq \Sigmatest$, we have $\rowsp(\hatG) \approx \rowsp(\Gstar\Sigmax^{1/2}) \neq \rowsp(\Gstar(\Sigmatest)^{1/2})$. In other words, in return for stabilizing optimization, normalizing the data destroys the shared structure of the predictor model! We illustrate this effect in \Cref{fig:batchnorm_subpace_dist}.


\subsection{Single Index Learning}\label{sec:single_index}

Assume that we observe $n$ i.i.d. samples generated according to the following single-index model:
\begin{align}
    \label{eq:data_gen}
    \bfx_i \overset{\mathrm{i.i.d.}}{\sim} \normal(0, \bSigma_\bfx),  \quad y_i = \sigma_\star( \vbeta_\star^\top \bfx_i) + \varepsilon_i
\end{align}
where $\bSigma_\bfx \in \R^{\dx \times \dx}$ is the input covariance, $\sigma_\star:\R\to\R$ is the teacher activation function, $\vbeta_\star \in \R^{\dx}$ is the (unknown) target direction, and $\ep_i$ is an additive noise $\ep_i \overset{\mathrm{i.i.d.}}{\sim} \normal(0, \sigma_\ep^2)$ independent of all other sources of randomness. 
We also make the following common assumption on $\vbeta_\star$ (\citet{dicker2016ridge,dobriban2018high,tripuraneni2021covariate,moniri_atheory2023,moniri2024asymptotics}, etc.) that ensures the covariates $\bfx_i$ alone do not carry any information about the target direction.
\begin{assumption}
    \label{assumption:random-effect}
    The  vector $\vbeta_\star$ is drawn from $\vbeta_\star \sim \normal(\mathbf{0}, \dx^{-1} \bI_{\dx})$ independent of other sources of randomness.
\end{assumption}

In this section, we study the problem of fitting a two-layer feedforward neural network $f_{\bff, \bfG}$ for prediction of unseen data points drawn independently from \eqref{eq:data_gen} at test time. 
When $\bG$ is kept at a random initialization and $\bff$ is trained using ridge regression, the model coincides with a random features model \cite{RahimiRecht,montanari2019generalization,hu2022universality} and has repeatedly used as a toy model to study and explain various aspects of practical neural networks (see  \citet{lin2021causes,adlam2020understanding,tripuraneni2020theory,hassani2022curse,bombari23robustness,disagreement,bombari2023stability,bombari2024privacy}, etc.).

When the covariates are isotropic $\bSigma_{\bfx} = \bI_{\dx}$, it is shown that a single step of full-batch \SGD update on $\bfG$ can drastically improve the performance of the model over random features as a result of \textit{feature learning} by aligning the top right-singular-vector of the updated representation layer $\bfG$ with the direction $\vbeta_\star$ \citep{damian2022neural,ba2022high,moniri_atheory2023,cuiasymptotics,dandi2023learning,dandi2024random,dandibenefits}. In this section, we assume that the covariates are anisotropic and show that in this case, the one-step full batch \SGD is suboptimal and can learn an ill-correlated direction even when the sample size $n$ is large. We then demonstrate that the \KFAC update with the preconditioners from \eqref{eq:precond} is in fact the natural fix to the full batch \SGD. 

\paragraph{Full-Batch \SGD.} Following the prior work, at initialization, we set $ \bff = \dhid^{-1/2} \bff_0$ with $\bff_0 \sim \normal(0, \dhid^{-1}\bI_{\dx})$, and  $\bfG= \bfG_0$ with i.i.d. $\normal(0, \dx^{-1})$ entries. We update $\bfG$ with one step of full batch \SGD  with step size $\eta_{\sbG} = \eta \sqrt{\dhid}$; i.e.,
\begin{align*}
    \bfG_{\SGD } \triangleq \bfG_0  - \eta \sqrt{\dhid} \; \nabla_\bfG \hatL(\bff_0, \bfG_0).
\end{align*}
In the following theorem, we provide an approximation of the updated first layer $\bfG_{\SGD }$, which is a generalization of \citep[
Proposition 2.1]{ba2022high} for $\bSigma_\bfx \neq \bI_{\dx}$.
\begin{restatable}{theorem}{OneStepSGD}
    \label{thm:rank1}
    Assume that the activation function $\sigma$ is $\calO(1)$-Lipschitz and that Assumption~\ref{assumption:random-effect} holds. In the limit where $n, \dx, \dhid$ tend to infinity proportionally, 
    the matrix $\bfG_{\textup{\SGD }}$, with probability $1 - o(1)$, satisfies
    \begin{align*}
        \left\|\bfG_0 + {\alpha \eta}\, \bff_0 \betaSGD^\top - \bfG_{\textup{\SGD }}\right\|_{\rm op} \to 0,
    \end{align*}
    in which $\alpha = \Ex_{z}[\sigma'(z)]$ with $z \sim \normal(0,\dx^{-1}\trace(\bSigma_\bfx))$, and the vector $\betaSGD$ is given by $\betaSGD = n^{-1} \bX^\top \vy$.
\end{restatable}
This theorem shows that one step of full batch \SGD  update approximately adds a rank-one component $\alpha\, \eta \, \bff_0 \betaSGD^\top$ to the initialized weights $\bfG_0$. Thus, the pre-activation features for a given input $\bfx \in \R^{\dx}$ after the update are given by
\begin{align*}
    \bfh_{\SGD } = \bG_{\SGD}\bfx \approx \bfG_0 \bfx + \alpha\, \eta \left(\betaSGD^\top \bfx\right) \, \bff_0 \in \R^{\dhid}
\end{align*}
where the first and second term correspond to the \textit{random feature}, and the \textit{learned feature} respectively. 
To better understand the learned feature component, note that defining $c_{\star, 1} = \Ex_{z\sim\normal(0, \dx^{-1}\trace(\bSigma_\bfx))} [\sigma_{\star}'(z)]$,  the target function $\sigma_\star(\vbeta_\star^\top \bfx)$ can be decomposed as
\begin{align*}
    \sigma_\star( \vbeta_\star^\top \bfx)  = c_{\star, 1}\vbeta_\star^\top \bfx + \sigma_{\star, \perp}(\vbeta_\star^\top \bfx)
\end{align*}
satisfying $\Ex_\bfx\left[ c_{\star, 1} (\vbeta_\star^\top \bfx)\, \sigma_{\star, \perp}(\vbeta_\star^\top \bfx)\right] = 0$. Therefore, when $c_{\star, 1} \neq 0$, the target function has a \textit{linear part}. Full batch \SGD is estimating the direction of $\vbeta_\star$ using this linear part with the estimator $\betaSGD = \bfX^\top\bfy/n$. However, the natural choice for this task is in fact ridge regression $\hat\vbeta_\lambda = (\hatSigma + \lambda \, \bI_{\dx})^{-1} \bX^\top \vy/n$, and $\betaSGD$ is missing the prefactor $(\hatSigma + \lambda \bI_{\dx})^{-1}$. In the isotropic case $\bSigma_{\bfx} = \bI_{\dx}$, we expect 
$\hatSigma \approx \bI_{\dx}$ 
when $n \gg \dx$. Thus, in this case the estimator $\betaSGD$ is roughly equivalent to the ridge estimator and can recover the direction $\vbeta_\star$. However, in the anisotopic case, $\betaSGD$ is biased even when $n \gg \dx$. To make these intuitions rigorous, we characterize in the following proposition the correlation between the learned direction $\betaSGD$ and the true direction $\vbeta_\star$.

\begin{restatable}{lemma}{TildeAlignment}
    \label{lemma:beta_tilde_alignment}
    Under the assumptions of Theorem~\ref{thm:rank1}, the correlation between $\vbeta_\star$ and $\betaSGD$ satisfies
    \begin{align*}
        \left|\frac{\vbeta_\star^\top \betaSGD}{\|\betaSGD\|_2  \|\vbeta_\star\|_2}  - \frac{\frac{c_{\star,1}}{\dx}\trace(\bSigma_\bfx) }{\sqrt{\frac{ c_{\star}^2 + \sigma_\ep^2}{n}\trace(\bSigma_\bfx) +\frac{c_{\star,1}^2}{\dx}    \trace(\bSigma_\bfx^2) }}\right| \to 0
    \end{align*}
    with probability $1 - o(1)$, in which $c_{\star, 1} = \Ex_{z} [\sigma_{\star}'(z)]$ and $c_{\star}^2 = \Ex_z[\sigma_{\star}^2(z)] $ with $z \sim \normal(0,\dx^{-1}\trace(\bSigma_\bfx))$.
\end{restatable}
This lemma shows that the correlation is increasing in the strength of the linear component $c_{\star, 1}$ while keeping the signal strength $c_{\star}$ fixed. Also, based on this lemma, when $n \gg \dx$, the correlation is given by ${\dx^{-1} \trace(\bSigma_\bfx)}/{\sqrt{\dx^{-1}\trace(\bSigma_{\bfx}^2)}}$, which is equal to one \emph{if and only if} $\bSigma_{\bfx} = \sigma^2\, \bI_{\dx}$ for some $\sigma \in \R$. This means these are the only covariance matrices for which applying one step of full batch \SGD  update learns the correct direction of $\vbeta_\star$. 

\paragraph{Stylized \KFAC.} This time, we update $\bfG$ using the stylized \KFAC  update from \eqref{eq:KFAC_update} with the regularized $\bP_\sbG$. We use the same initialization as full-batch \SGD. The updated representation layer in this case is given by 
\begin{align*}
    \bfG_{\KFAC } \triangleq \bfG_0  - \eta \sqrt{\dhid} \; \nabla_\bfG \hatL(\bff_0, \bfG_0)\; (\bQ_{\sbG}+\lambda_\sbG \bI_{\dx})^{-1}.
\end{align*}
The preconditioning factor $(\bQ_{\sbG}+\lambda_\sbG \bI_{\dx})^{-1}$ with $\bQ_{\sbG} = \hatSigma$ is precisely the factor required so that the direction learned by the one-step update to match the ridge regression estimator with ridge parameter $\lambda_\sbG$ as shown in the following immediate corollary of Theorem~\ref{thm:rank1}.
\begin{corollary}
    \label{corr:rank1}
    Under the same set of assumptions as Theorem~\ref{thm:rank1},
    the matrix $\bfG_{\textup{\KFAC }}$, satisfies
    \begin{align*}
        \left\|\bfG_0 + {\alpha \eta}\, \bff_0  \betaKFAC^\top - \bfG_{\textup{\KFAC }}\right\|_{\rm op} \to 0
    \end{align*}
      with probability $1 - o(1)$, where $\alpha$ is defined in Theorem~\ref{thm:rank1}, and 
    $\betaKFAC =  (\bQ_{\sbG}+\lambda_\sbG \bI_{\dx})^{-1}\bX^\top \vy/n$.
\end{corollary}
Because $\betaKFAC$ is equivalent to ridge regression, we expect it to align well with $\vbeta_\star$ even for anisotropic $\bSigma_{\bfx}$, given a proper choice of $\lambda_{\sbG}$.  The following lemma formally characterizes the correlation between $\betaKFAC$ and $\vbeta_\star$ for any $\lambda_{\sbG}\in \R$.

\begin{restatable}{lemma}{HatAlignment}
    \label{lemma:beta_hat_alignment}
    Under the assumptions of Theorem~\ref{thm:rank1}, the correlation between $\vbeta_\star$ and $\betaKFAC$ satisfies
    \begin{align*}
        \left|\frac{\betaKFAC^\top \, \vbeta_\star}{\|\betaKFAC\|_2  \|\vbeta_\star\|_2}  - \frac{c_{\star,1} \Psi_1}{\sqrt{c_{\star,1}^2 \Psi_2 + \frac{\dx}{n} (c_{\star, >1}^2 + \sigma_\ep^2)\Psi_3}}\right| \to 0
    \end{align*}
     with probability $1 - o(1)$, where  $c^2_{\star, 1} = \Ex_{z}^2 [\sigma_{\star}'(z)]$, $c_{\star, >1}^2 = \Ex_{z}[\sigma_{\star, \perp}^2(z)]$ with $z \sim \normal(0, \dx^{-1}\trace(\bSigma_\bfx))$, and  $\Psi_1, \Psi_2, \Psi_3$ are defined in \eqref{eq:psi_def} and depend on $\bSigma_\bfx$, ${\dx}/{n},$ and $\lambda_{\sbG}$. In particular, as $\lambda_{\sbG} \to 0$ and $\dx/n \to 0$, we have
     \begin{align*}
          \frac{\betaKFAC^\top \, \vbeta_\star}{\|\betaKFAC\|_2  \|\vbeta_\star\|_2} \to 1.
     \end{align*}
\end{restatable}

This lemma shows that when $n \gg \dx$, and $\lambda_\sbG\to 0$,  the one-step stylized \KFAC  update---unlike the one-step full-batch \SGD ---perfectly 
recovers the target direction \(\vbeta_\star\), fixing the issue with full batch \SGD with anisotropic covariances.


\begin{remark}  It is well-known that, given features that align with $\vbeta_\star$, applying least-squares on $\bZ=\sigma(\bG_{\textup{\KFAC}}\bX)$, which from \Cref{lem:KFAC_is_EMA} is equivalent to the \KFAC $\bff$-update with $\eta_{\bff} = 1$, leverages the feature to obtain a solution with good generalization. See \Cref{sec:single_generalize} for more details.
    
\end{remark}


\section{Experiments}
\label{sec:experiments}
The experiments are designed to address two key research questions.
First, \textbf{RQ1} evaluates whether the average $L_2$-norm of the counterfactual perturbation vectors ($\overline{||\perturb||}$) decreases as the model overfits the data, thereby providing further empirical validation for our hypothesis.
Second, \textbf{RQ2} evaluates the ability of the proposed counterfactual regularized loss, as defined in (\ref{eq:regularized_loss2}), to mitigate overfitting when compared to existing regularization techniques.

% The experiments are designed to address three key research questions. First, \textbf{RQ1} investigates whether the mean perturbation vector norm decreases as the model overfits the data, aiming to further validate our intuition. Second, \textbf{RQ2} explores whether the mean perturbation vector norm can be effectively leveraged as a regularization term during training, offering insights into its potential role in mitigating overfitting. Finally, \textbf{RQ3} examines whether our counterfactual regularizer enables the model to achieve superior performance compared to existing regularization methods, thus highlighting its practical advantage.

\subsection{Experimental Setup}
\textbf{\textit{Datasets, Models, and Tasks.}}
The experiments are conducted on three datasets: \textit{Water Potability}~\cite{kadiwal2020waterpotability}, \textit{Phomene}~\cite{phomene}, and \textit{CIFAR-10}~\cite{krizhevsky2009learning}. For \textit{Water Potability} and \textit{Phomene}, we randomly select $80\%$ of the samples for the training set, and the remaining $20\%$ for the test set, \textit{CIFAR-10} comes already split. Furthermore, we consider the following models: Logistic Regression, Multi-Layer Perceptron (MLP) with 100 and 30 neurons on each hidden layer, and PreactResNet-18~\cite{he2016cvecvv} as a Convolutional Neural Network (CNN) architecture.
We focus on binary classification tasks and leave the extension to multiclass scenarios for future work. However, for datasets that are inherently multiclass, we transform the problem into a binary classification task by selecting two classes, aligning with our assumption.

\smallskip
\noindent\textbf{\textit{Evaluation Measures.}} To characterize the degree of overfitting, we use the test loss, as it serves as a reliable indicator of the model's generalization capability to unseen data. Additionally, we evaluate the predictive performance of each model using the test accuracy.

\smallskip
\noindent\textbf{\textit{Baselines.}} We compare CF-Reg with the following regularization techniques: L1 (``Lasso''), L2 (``Ridge''), and Dropout.

\smallskip
\noindent\textbf{\textit{Configurations.}}
For each model, we adopt specific configurations as follows.
\begin{itemize}
\item \textit{Logistic Regression:} To induce overfitting in the model, we artificially increase the dimensionality of the data beyond the number of training samples by applying a polynomial feature expansion. This approach ensures that the model has enough capacity to overfit the training data, allowing us to analyze the impact of our counterfactual regularizer. The degree of the polynomial is chosen as the smallest degree that makes the number of features greater than the number of data.
\item \textit{Neural Networks (MLP and CNN):} To take advantage of the closed-form solution for computing the optimal perturbation vector as defined in (\ref{eq:opt-delta}), we use a local linear approximation of the neural network models. Hence, given an instance $\inst_i$, we consider the (optimal) counterfactual not with respect to $\model$ but with respect to:
\begin{equation}
\label{eq:taylor}
    \model^{lin}(\inst) = \model(\inst_i) + \nabla_{\inst}\model(\inst_i)(\inst - \inst_i),
\end{equation}
where $\model^{lin}$ represents the first-order Taylor approximation of $\model$ at $\inst_i$.
Note that this step is unnecessary for Logistic Regression, as it is inherently a linear model.
\end{itemize}

\smallskip
\noindent \textbf{\textit{Implementation Details.}} We run all experiments on a machine equipped with an AMD Ryzen 9 7900 12-Core Processor and an NVIDIA GeForce RTX 4090 GPU. Our implementation is based on the PyTorch Lightning framework. We use stochastic gradient descent as the optimizer with a learning rate of $\eta = 0.001$ and no weight decay. We use a batch size of $128$. The training and test steps are conducted for $6000$ epochs on the \textit{Water Potability} and \textit{Phoneme} datasets, while for the \textit{CIFAR-10} dataset, they are performed for $200$ epochs.
Finally, the contribution $w_i^{\varepsilon}$ of each training point $\inst_i$ is uniformly set as $w_i^{\varepsilon} = 1~\forall i\in \{1,\ldots,m\}$.

The source code implementation for our experiments is available at the following GitHub repository: \url{https://anonymous.4open.science/r/COCE-80B4/README.md} 

\subsection{RQ1: Counterfactual Perturbation vs. Overfitting}
To address \textbf{RQ1}, we analyze the relationship between the test loss and the average $L_2$-norm of the counterfactual perturbation vectors ($\overline{||\perturb||}$) over training epochs.

In particular, Figure~\ref{fig:delta_loss_epochs} depicts the evolution of $\overline{||\perturb||}$ alongside the test loss for an MLP trained \textit{without} regularization on the \textit{Water Potability} dataset. 
\begin{figure}[ht]
    \centering
    \includegraphics[width=0.85\linewidth]{img/delta_loss_epochs.png}
    \caption{The average counterfactual perturbation vector $\overline{||\perturb||}$ (left $y$-axis) and the cross-entropy test loss (right $y$-axis) over training epochs ($x$-axis) for an MLP trained on the \textit{Water Potability} dataset \textit{without} regularization.}
    \label{fig:delta_loss_epochs}
\end{figure}

The plot shows a clear trend as the model starts to overfit the data (evidenced by an increase in test loss). 
Notably, $\overline{||\perturb||}$ begins to decrease, which aligns with the hypothesis that the average distance to the optimal counterfactual example gets smaller as the model's decision boundary becomes increasingly adherent to the training data.

It is worth noting that this trend is heavily influenced by the choice of the counterfactual generator model. In particular, the relationship between $\overline{||\perturb||}$ and the degree of overfitting may become even more pronounced when leveraging more accurate counterfactual generators. However, these models often come at the cost of higher computational complexity, and their exploration is left to future work.

Nonetheless, we expect that $\overline{||\perturb||}$ will eventually stabilize at a plateau, as the average $L_2$-norm of the optimal counterfactual perturbations cannot vanish to zero.

% Additionally, the choice of employing the score-based counterfactual explanation framework to generate counterfactuals was driven to promote computational efficiency.

% Future enhancements to the framework may involve adopting models capable of generating more precise counterfactuals. While such approaches may yield to performance improvements, they are likely to come at the cost of increased computational complexity.


\subsection{RQ2: Counterfactual Regularization Performance}
To answer \textbf{RQ2}, we evaluate the effectiveness of the proposed counterfactual regularization (CF-Reg) by comparing its performance against existing baselines: unregularized training loss (No-Reg), L1 regularization (L1-Reg), L2 regularization (L2-Reg), and Dropout.
Specifically, for each model and dataset combination, Table~\ref{tab:regularization_comparison} presents the mean value and standard deviation of test accuracy achieved by each method across 5 random initialization. 

The table illustrates that our regularization technique consistently delivers better results than existing methods across all evaluated scenarios, except for one case -- i.e., Logistic Regression on the \textit{Phomene} dataset. 
However, this setting exhibits an unusual pattern, as the highest model accuracy is achieved without any regularization. Even in this case, CF-Reg still surpasses other regularization baselines.

From the results above, we derive the following key insights. First, CF-Reg proves to be effective across various model types, ranging from simple linear models (Logistic Regression) to deep architectures like MLPs and CNNs, and across diverse datasets, including both tabular and image data. 
Second, CF-Reg's strong performance on the \textit{Water} dataset with Logistic Regression suggests that its benefits may be more pronounced when applied to simpler models. However, the unexpected outcome on the \textit{Phoneme} dataset calls for further investigation into this phenomenon.


\begin{table*}[h!]
    \centering
    \caption{Mean value and standard deviation of test accuracy across 5 random initializations for different model, dataset, and regularization method. The best results are highlighted in \textbf{bold}.}
    \label{tab:regularization_comparison}
    \begin{tabular}{|c|c|c|c|c|c|c|}
        \hline
        \textbf{Model} & \textbf{Dataset} & \textbf{No-Reg} & \textbf{L1-Reg} & \textbf{L2-Reg} & \textbf{Dropout} & \textbf{CF-Reg (ours)} \\ \hline
        Logistic Regression   & \textit{Water}   & $0.6595 \pm 0.0038$   & $0.6729 \pm 0.0056$   & $0.6756 \pm 0.0046$  & N/A    & $\mathbf{0.6918 \pm 0.0036}$                     \\ \hline
        MLP   & \textit{Water}   & $0.6756 \pm 0.0042$   & $0.6790 \pm 0.0058$   & $0.6790 \pm 0.0023$  & $0.6750 \pm 0.0036$    & $\mathbf{0.6802 \pm 0.0046}$                    \\ \hline
%        MLP   & \textit{Adult}   & $0.8404 \pm 0.0010$   & $\mathbf{0.8495 \pm 0.0007}$   & $0.8489 \pm 0.0014$  & $\mathbf{0.8495 \pm 0.0016}$     & $0.8449 \pm 0.0019$                    \\ \hline
        Logistic Regression   & \textit{Phomene}   & $\mathbf{0.8148 \pm 0.0020}$   & $0.8041 \pm 0.0028$   & $0.7835 \pm 0.0176$  & N/A    & $0.8098 \pm 0.0055$                     \\ \hline
        MLP   & \textit{Phomene}   & $0.8677 \pm 0.0033$   & $0.8374 \pm 0.0080$   & $0.8673 \pm 0.0045$  & $0.8672 \pm 0.0042$     & $\mathbf{0.8718 \pm 0.0040}$                    \\ \hline
        CNN   & \textit{CIFAR-10} & $0.6670 \pm 0.0233$   & $0.6229 \pm 0.0850$   & $0.7348 \pm 0.0365$   & N/A    & $\mathbf{0.7427 \pm 0.0571}$                     \\ \hline
    \end{tabular}
\end{table*}

\begin{table*}[htb!]
    \centering
    \caption{Hyperparameter configurations utilized for the generation of Table \ref{tab:regularization_comparison}. For our regularization the hyperparameters are reported as $\mathbf{\alpha/\beta}$.}
    \label{tab:performance_parameters}
    \begin{tabular}{|c|c|c|c|c|c|c|}
        \hline
        \textbf{Model} & \textbf{Dataset} & \textbf{No-Reg} & \textbf{L1-Reg} & \textbf{L2-Reg} & \textbf{Dropout} & \textbf{CF-Reg (ours)} \\ \hline
        Logistic Regression   & \textit{Water}   & N/A   & $0.0093$   & $0.6927$  & N/A    & $0.3791/1.0355$                     \\ \hline
        MLP   & \textit{Water}   & N/A   & $0.0007$   & $0.0022$  & $0.0002$    & $0.2567/1.9775$                    \\ \hline
        Logistic Regression   &
        \textit{Phomene}   & N/A   & $0.0097$   & $0.7979$  & N/A    & $0.0571/1.8516$                     \\ \hline
        MLP   & \textit{Phomene}   & N/A   & $0.0007$   & $4.24\cdot10^{-5}$  & $0.0015$    & $0.0516/2.2700$                    \\ \hline
       % MLP   & \textit{Adult}   & N/A   & $0.0018$   & $0.0018$  & $0.0601$     & $0.0764/2.2068$                    \\ \hline
        CNN   & \textit{CIFAR-10} & N/A   & $0.0050$   & $0.0864$ & N/A    & $0.3018/
        2.1502$                     \\ \hline
    \end{tabular}
\end{table*}

\begin{table*}[htb!]
    \centering
    \caption{Mean value and standard deviation of training time across 5 different runs. The reported time (in seconds) corresponds to the generation of each entry in Table \ref{tab:regularization_comparison}. Times are }
    \label{tab:times}
    \begin{tabular}{|c|c|c|c|c|c|c|}
        \hline
        \textbf{Model} & \textbf{Dataset} & \textbf{No-Reg} & \textbf{L1-Reg} & \textbf{L2-Reg} & \textbf{Dropout} & \textbf{CF-Reg (ours)} \\ \hline
        Logistic Regression   & \textit{Water}   & $222.98 \pm 1.07$   & $239.94 \pm 2.59$   & $241.60 \pm 1.88$  & N/A    & $251.50 \pm 1.93$                     \\ \hline
        MLP   & \textit{Water}   & $225.71 \pm 3.85$   & $250.13 \pm 4.44$   & $255.78 \pm 2.38$  & $237.83 \pm 3.45$    & $266.48 \pm 3.46$                    \\ \hline
        Logistic Regression   & \textit{Phomene}   & $266.39 \pm 0.82$ & $367.52 \pm 6.85$   & $361.69 \pm 4.04$  & N/A   & $310.48 \pm 0.76$                    \\ \hline
        MLP   &
        \textit{Phomene} & $335.62 \pm 1.77$   & $390.86 \pm 2.11$   & $393.96 \pm 1.95$ & $363.51 \pm 5.07$    & $403.14 \pm 1.92$                     \\ \hline
       % MLP   & \textit{Adult}   & N/A   & $0.0018$   & $0.0018$  & $0.0601$     & $0.0764/2.2068$                    \\ \hline
        CNN   & \textit{CIFAR-10} & $370.09 \pm 0.18$   & $395.71 \pm 0.55$   & $401.38 \pm 0.16$ & N/A    & $1287.8 \pm 0.26$                     \\ \hline
    \end{tabular}
\end{table*}

\subsection{Feasibility of our Method}
A crucial requirement for any regularization technique is that it should impose minimal impact on the overall training process.
In this respect, CF-Reg introduces an overhead that depends on the time required to find the optimal counterfactual example for each training instance. 
As such, the more sophisticated the counterfactual generator model probed during training the higher would be the time required. However, a more advanced counterfactual generator might provide a more effective regularization. We discuss this trade-off in more details in Section~\ref{sec:discussion}.

Table~\ref{tab:times} presents the average training time ($\pm$ standard deviation) for each model and dataset combination listed in Table~\ref{tab:regularization_comparison}.
We can observe that the higher accuracy achieved by CF-Reg using the score-based counterfactual generator comes with only minimal overhead. However, when applied to deep neural networks with many hidden layers, such as \textit{PreactResNet-18}, the forward derivative computation required for the linearization of the network introduces a more noticeable computational cost, explaining the longer training times in the table.

\subsection{Hyperparameter Sensitivity Analysis}
The proposed counterfactual regularization technique relies on two key hyperparameters: $\alpha$ and $\beta$. The former is intrinsic to the loss formulation defined in (\ref{eq:cf-train}), while the latter is closely tied to the choice of the score-based counterfactual explanation method used.

Figure~\ref{fig:test_alpha_beta} illustrates how the test accuracy of an MLP trained on the \textit{Water Potability} dataset changes for different combinations of $\alpha$ and $\beta$.

\begin{figure}[ht]
    \centering
    \includegraphics[width=0.85\linewidth]{img/test_acc_alpha_beta.png}
    \caption{The test accuracy of an MLP trained on the \textit{Water Potability} dataset, evaluated while varying the weight of our counterfactual regularizer ($\alpha$) for different values of $\beta$.}
    \label{fig:test_alpha_beta}
\end{figure}

We observe that, for a fixed $\beta$, increasing the weight of our counterfactual regularizer ($\alpha$) can slightly improve test accuracy until a sudden drop is noticed for $\alpha > 0.1$.
This behavior was expected, as the impact of our penalty, like any regularization term, can be disruptive if not properly controlled.

Moreover, this finding further demonstrates that our regularization method, CF-Reg, is inherently data-driven. Therefore, it requires specific fine-tuning based on the combination of the model and dataset at hand.

\section{Conclusion}
In this work, we propose a simple yet effective approach, called SMILE, for graph few-shot learning with fewer tasks. Specifically, we introduce a novel dual-level mixup strategy, including within-task and across-task mixup, for enriching the diversity of nodes within each task and the diversity of tasks. Also, we incorporate the degree-based prior information to learn expressive node embeddings. Theoretically, we prove that SMILE effectively enhances the model's generalization performance. Empirically, we conduct extensive experiments on multiple benchmarks and the results suggest that SMILE significantly outperforms other baselines, including both in-domain and cross-domain few-shot settings.

\section{Acknowledgments}
Thomas Zhang and Behrad Moniri gratefully acknowledge gifts from AWS AI to Penn Engineering's ASSET Center for Trustworthy AI. The work of Behrad Moniri and Hamed Hassani is supported by The Institute for Learning-enabled Optimization at Scale (TILOS), under award number NSF-CCF-2112665, and the NSF CAREER award CIF-1943064. Thomas Zhang and Nikolai Matni are supported in part by NSF Award SLES-2331880, NSF CAREER award ECCS-2045834, NSF EECS-2231349, and AFOSR Award FA9550-24-1-0102.

{\small
\bibliography{refs}
\bibliographystyle{icml2025/icml2025}
}

\clearpage
\appendix
\onecolumn
In this section, we provide an extended version of Tab.~\ref{tab:continual-learning-taxonomy}, focusing \textit{only} on the \textit{most recent literature}, and showing how our work is uniquely positioned in the landscape of model editing and continual learning, the two key related branches to our work.

\begin{table}[t]
\centering
\caption{Extended taxonomy of incremental Learning Approaches, showing some seminal work (top) and more recent literature (split into editing and continual learning).}
\resizebox{\textwidth}{!}{%
\begin{tabular}{@{}l p{3cm} c c c c c c@{}}
\toprule
\textbf{Examples} & \textbf{\makecell{Incremental \\ Type}} & \textbf{\makecell{Memory \\ Usage}} & \textbf{\makecell{Task \\ Awareness}} & \textbf{\makecell{Weight \\ Plasticity}} & \textbf{\makecell{Architecture}} & \textbf{\makecell{Conflict \\ Detection}} & \textbf{\makecell{Update \\ Mechanism}} \\ \midrule
% Classic/Foundational Approaches
iCaRL~\citep{rebuffi2017icarl} & Class-incremental & Replay & Task-Agnostic & Fixed & Fixed & No & Rehearsal \\
EWC~\citep{kirkpatrick2017overcoming} & Task-incremental & None & Task-Aware & Selective & Fixed & No & Regularization \\
Progressive Nets~\citep{rusu2016progressive} & Task-incremental & None & Task-Aware & Fixed & Expanding & No & New Subnetworks \\
DEN~\citep{yoon2017lifelong} & Task-incremental & None & Task-Aware & Selective & Expanding & No & Selective Expansion \\
GEM~\citep{lopez2017gradient} & Task-incremental & Replay & Task-Aware & Constrained & Fixed & No & Constrained Optimization \\
ROME~\citep{DeCao2021} & Fact-incremental & None & Fact-Aware & Localized & Fixed & No & Rank-One Update \\
OWM~\citep{zeng2019continual} & Task-incremental & None & Task-Aware & Orthogonal & Fixed & No & Orthogonal Projection \\
PackNet~\citep{mallya2018packnet} & Task-incremental & None & Task-Aware & Selective & Fixed & No & Weight Masking \\
HAT~\citep{serra2018overcoming} & Task-incremental & None & Task-Aware & Selective & Fixed & No & Attention Masking \\ \midrule
% Recent Knowledge Editing Approaches
MALMEN~\citep{tan2023massive} & Fact-incremental & None & Fact-Aware & Localized & Fixed & No & Parameter Shift Aggregation \\
EditAnalysis~\citep{li2023unveiling} & Fact-incremental & None & Fact-Aware & Analysis & Fixed & No & Consistency Analysis \\
D4S~\citep{huang2024reasons} & Fact-incremental & O(1) & Fact-Aware & Regulated & Fixed & No & Layer-Norm Control \\ \midrule
% Recent Continual Learning Approaches
Global Prototypes~\citep{baicontinual} & Task/Class-incremental & None & Task-Agnostic & Selective & Fixed & No & Global Prototype Alignment \\
NTE~\citep{benjamin2024continual} & Task-incremental & None & Task-Agnostic & Selective & Fixed & No & Bayesian Ensemble \\
UPGD~\citep{elsayed2024addressing} & Task-incremental & None & Task-Agnostic & Selective & Fixed & No & Utility-Gated Updates \\
~\citep{hiratani2024disentangling} & Task-incremental & None & Task-Aware & Selective & Fixed & No & Fisher Information \\
CLAP~\citep{jha2024clap4clip} & Class-incremental & None & Task-Aware & Selective & Fixed & No & Probabilistic Adaptation \\
VQ-Prompt~\citep{jiao2024vector} & Class-incremental & None & Task-Agnostic & Fixed & Fixed & No & Discrete Prompt Selection \\
IsCiL~\citep{lee2024incremental} & Task-incremental & None & Task-Aware & Selective & Fixed & No & Skill-based Adaptation \\
BGS~\citep{leecontinual} & Task/Domain/Class-incremental & Replay & Task-Aware & Selective & Fixed & Yes & Bias-Aware Update \\
SLM~\citep{peng2024scalable} & Task-incremental & None & Auto-detected & Selective & Fixed & No & Vector Space Retrieval \\
Train-Attention~\citep{seo2024train} & Knowledge-incremental & None & Task-Agnostic & Selective & Fixed & No & Token-Weighted Update \\
Refresh Learning~\citep{wang2024unified} & Task/Class-incremental & Optional & Task-Aware & Selective & Fixed & No & Unlearn-Relearn \\
RAIL~\citep{xu2024advancing} & Cross-domain-incremental & None & Task-Agnostic & Selective & Fixed & No & Regression-based Update \\
SAFE~\citep{zhao2024safeslowfastparameterefficient} & Class-incremental & None & Task-Agnostic & Selective & Fixed & No & Dual Parameter-Efficient Tuning \\ \midrule
% Your Paper
\textbf{This paper} & Fact-incremental & None & Conflict-Aware & Selective & Fixed & Yes & Neuron-Specific Update \\ \bottomrule
\end{tabular}%
}
\label{tab:continual-learning-taxonomy-extended}
\end{table}


\subsection{Continual learning}
Continual Learning (CL) methods enable models to learn new tasks without catastrophically forgetting previously mastered ones \citep{kirkpatrick2017overcoming}. These approaches fall into three main families: memory-based methods using exemplar buffers \citep{rebuffi2017icarl}, knowledge distillation techniques that transfer information across model versions \citep{lopez2017gradient}, and regularization-based methods that constrain weight updates \citep{kirkpatrick2017overcoming}. To ease the understanding of this landscape, we build a taxonomy that characterizes approaches by their incremental type (task, class, or fact-based), memory requirements, update mechanisms, and architectural constraints (Tab.~\ref{tab:continual-learning-taxonomy}). This taxonomy reveals how our work is different from existing continual learning attempts: while existing methods focus on preserving knowledge across distinct tasks, none explicitly address the detection and handling of conflicting information - a key capability in human cognition that our work empirically investigates.

One of the closest old approaches is deep mind's EWC~\citep{kirkpatrick2017overcoming}, a method designed to mitigate catastrophic forgetting in neural networks trained sequentially on distinct tasks. The core idea is to protect the most important weights (or neurons) for previously learned tasks during the training of new tasks. EWC identifies these important weights by calculating the Fisher Information Matrix during or after the training of a task, which estimates how sensitive each weight is to the task’s performance. Weights that significantly impact the output for a given task are marked as important. A quadratic penalty is then applied during future learning, constraining these weights to remain close to their values from the previous task. This ensures that knowledge from earlier tasks is preserved while still allowing the model to adapt to new tasks. However, EWC is \textbf{less suitable for LLMs}, which \textbf{do not have clearly defined tasks} when it comes to knowledge ingestion (probably different for other types of skills). EWC's effectiveness relies on distinct task boundaries and the ability to compute task-specific importance for weights, which is feasible in scenarios with well-defined tasks, such as classification or reinforcement learning. In LLMs, where learning spans a wide range of topics and linguistic structures without clear task delineation, it’s challenging to apply EWC's task-based strategy. The model would struggle to assign specific neurons or weights to individual tasks or concepts, making it difficult to protect task-specific knowledge without hindering the model’s overall generalization ability across a diverse dataset.

We cite in the remainder more recent literature that we project onto our taxonomy.

\citet{baicontinual} introduce a novel approach to continual learning that leverages global prototypes to mitigate catastrophic forgetting in neural networks. Their key insight is that maintaining stable connections between task-specific representations and pre-learned, general-purpose token embeddings (which serve as global prototypes) can significantly reduce forgetting without requiring explicit replay mechanisms. Through empirical validation on both task-incremental and class-incremental NLP scenarios, they demonstrate that models preserving strong connections to these global prototypes exhibit enhanced stability. While their work shares our goal of preserving knowledge during updates, it differs fundamentally in its approach and granularity: where they focus on task-level knowledge preservation through architectural mechanisms, our work addresses the more specific challenge of managing contradictory factual updates through cognitive-inspired conflict detection. Their finding that stable reference points aid knowledge retention is conceptually relevant to our work, though our results suggest that such architectural approaches alone may be insufficient when handling explicitly contradictory information, where more sophisticated cognitive mechanisms become necessary.

\citet{benjamin2024continual} proposed an elegant theoretical framework that interprets neural networks as Bayesian ensembles of classifiers. Their key insight is that a neural network with N parameters can be viewed as a weighted ensemble of N classifiers in the lazy regime, where the classifiers remain fixed throughout learning. This interpretation reveals that a properly designed posterior update rule, resembling SGD without momentum, can enable continual learning without forgetting - notably, they prove that momentum actually exacerbates forgetting. While their work focuses on preserving all knowledge in task-incremental learning, our paper specifically examines cases where knowledge needs to be deliberately updated or overridden. Their key contribution is showing that catastrophic forgetting is linked to the transition from lazy to rich regimes in neural networks, providing both a theoretical explanation for why larger models are more robust to forgetting and a biologically-inspired mechanism for knowledge preservation that perhaps complements our cognitive-based approach.

\citet{elsayed2024addressing} propose UPGD (Utility-based Perturbed Gradient Descent), a novel approach targeting both catastrophic forgetting and loss of plasticity in streaming learning scenarios. Their method protects useful network units while maintaining plasticity in less-used ones through utility-gated gradient updates and perturbations. Unlike previous approaches requiring task boundaries or memory buffers, UPGD operates in a challenging streaming setting with continuous non-stationarity. Using their newly introduced direct plasticity metric, they demonstrate UPGD's ability to maintain performance levels that surpass or match existing methods. This work complements our investigation by providing evidence that selective neuronal updates based on utility metrics can effectively balance stability and plasticity, though in a task-learning rather than knowledge-updating context.

\citet{hiratani2024disentangling} analyze how task similarity affects continual learning through a novel theoretical framework combining teacher-student models with latent structure. Their key insight is that high input feature similarity coupled with low readout similarity leads to catastrophic outcomes in both knowledge transfer and retention, even when tasks are positively correlated. They demonstrate that weight regularization in the Fisher information metric robustly helps retention regardless of task similarity, while common approaches like activity gating improve retention at the cost of transfer performance. Their theoretical predictions are validated on permuted MNIST tasks with latent variables.

\citet{jha2024clap4clip} propose a probabilistic approach to continual learning for vision-language models, specifically focusing on CLIP adaptation. Their method, CLAP, introduces visual-guided attention and task-specific probabilistic adapters to model the distribution of text features, while leveraging CLIP's pre-trained knowledge for initialization and regularization. This work demonstrates that probabilistic modeling can significantly reduce catastrophic forgetting in class-incremental learning scenarios, achieving state-of-the-art performance across multiple benchmarks.

\citet{jiao2024vector} propose VQ-Prompt, a novel prompt-based continual learning framework that addresses class-incremental learning with pretrained vision transformers. Their key innovation is incorporating vector quantization into prompt selection, enabling end-to-end optimization of discrete prompts with task loss while maintaining effective knowledge abstraction. This contrasts with our cognitive-dissonance aware approach, as they focus on task adaptation through prompt engineering rather than explicit conflict detection. Their empirical results on ImageNet-R and CIFAR-100 demonstrate superior performance compared to existing prompt-based methods, suggesting the effectiveness of discrete knowledge representation in continual learning.

\cite{lee2024incremental} propose IsCiL, a framework for continual imitation learning that uses retrievable skills and adapter-based architecture to enable efficient knowledge sharing across tasks. Unlike traditional approaches that isolate task-specific parameters, IsCiL introduces a prototype-based skill retrieval mechanism that allows selective reuse of previously learned skills for new tasks. While focused primarily on motor skills rather than resolving knowledge contradictions, their empirical results show that this selective adaptation approach significantly improves sample efficiency and reduces catastrophic forgetting compared to other adapter-based methods, particularly in scenarios with incomplete demonstrations.

\citet{leecontinual} present a systematic empirical investigation of how dataset bias affects continual learning. Through carefully designed experiments across task-incremental, domain-incremental, and class-incremental scenarios, they reveal that bias transfers both forward and backward between tasks. Their analysis shows that CL methods focusing on stability tend to preserve and propagate biases from previous tasks, while emphasis on plasticity allows new biases to contaminate previous knowledge. Based on these insights, they propose BGS (Balanced Greedy Sampling), a method that mitigates bias transfer by maintaining a balanced exemplar memory and retraining the classification head. Note that here, we used ``Replay'' for Memory Usage in the table since their best performing method (BGS) uses an exemplar memory, but they also evaluate methods without memory. 

\citet{peng2024scalable} proposed a continual learning approach that automates task selection through vector space retrieval, eliminating the need for explicit task IDs, experience replay, or optimization constraints. Their method, Scalable Language Model (SLM), combines Joint Adaptive Re-parameterization with dynamic knowledge retrieval to automatically identify relevant parameters for each input, enabling task-agnostic updates. While achieving state-of-the-art results across diverse tasks and model scales (BERT, T5, LLaMA-2), their key contribution is demonstrating that automatic task identification and parameter selection can enable continual learning without requiring explicit task boundaries or memory buffers.

\citet{seo2024train} presented Train-Attention, an interesting meta-learning approach for continual knowledge learning (CKL) in LLMs that predicts and applies weights to tokens \textit{based on their usefulness for future tasks}. Unlike previous approaches that uniformly update all parameters, their method enables\textit{ targeted knowledge updates by learning which tokens are most important} to focus on. Through experiments on LAMA-CKL and TemporalWiki benchmarks, they show that selective token-weighted learning significantly reduces catastrophic forgetting while improving learning speed. The work somewhat complements our cognitive-inspired approach, and demonstrates the benefits of selective attention, but it does not explicitly address the handling of contradictory information.

\citet{wang2024unified} proposed a unified framework for continual learning that reveals common mathematical structures across seemingly distinct approaches (regularization-based, Bayesian-based, and memory-replay). Building on this unification, they introduce ``refresh learning" - a plug-in mechanism that first unlearns current data before relearning it, inspired by the beneficial role of forgetting in human cognition. Their work primarily focuses on task-incremental and class-incremental scenarios, demonstrating improved accuracy across CIFAR and Tiny-ImageNet benchmarks. While their approach differs from our fact-level knowledge updates in LLMs, their findings about selective forgetting complement our observations about cognitive-inspired update mechanisms. Their theoretical analysis showing that refresh learning improves the flatness of the loss landscape offers an interesting perspective on how controlled forgetting might benefit knowledge retention in neural networks.

\citet{xu2024advancing} propose a cross-domain task-agnostic incremental learning framework (X-TAIL) for vision-language models, focusing on the challenge of preserving both incrementally learned knowledge and zero-shot abilities. Their approach, RAIL, uses recursive ridge regression with non-linear projections to adapt to new domains without catastrophic forgetting. Unlike previous work requiring domain identity hints or reference datasets, RAIL can classify images across both seen and unseen domains without domain hints, demonstrating superior performance in both discriminative ability and knowledge preservation. While their work advances the technical aspects of continual learning, it differs from our cognitive-inspired investigation as it doesn't address the fundamental challenge of detecting and resolving conflicting knowledge, instead focusing on domain adaptation without explicit conflict awareness.

\citet{zhao2024safeslowfastparameterefficient} propose a class-incremental learning framework for pre-trained vision models that balances stability and plasticity through two complementary parameter-efficient tuning mechanisms. Their SAFE approach first inherits generalizability from pre-trained models via a ``slow learner" that captures transferable knowledge in the first session, then maintains plasticity through a ``fast learner" that continuously adapts to new classes while resisting catastrophic forgetting. While focused on vision tasks rather than language models, their dual-speed learning strategy presents interesting parallels to our cognitive-inspired approach – particularly in how both works identify the importance of selective plasticity and the distinction between stable (``stubborn") and adaptable (``plastic") parameters. However, SAFE doesn't address the fundamental challenge of detecting and handling contradictory information that we identify as crucial for true cognitive-inspired learning.

\textit{Unlike the above work, our goal is to understand the fundamental cognitive mechanisms underlying the continuous knowledge updates in LLMs, particularly focusing on how models can detect and react to contradictory information. Rather than proposing a new continual learning method, we provide crucial insights into how different types of knowledge updates affect model behavior and stability.}

\subsection{Knowledge editing}
Next, a big portion of recent literature has focused on understanding and modifying the internal knowledge of Large Language Models (LLMs), post-training. Such knowledge editing aims to alter specific facts or associations within the model without the need for full retraining. 

\citet{Geva2020} were among the first to show that transformer Feed-Forward Network (FFN) layers act as unnormalized key-value stores encoding relational knowledge inside LLMs. This observation was later confirmed and complemented by others \citep{Meng2022,Dai2021} before being leveraged by subsequent work to master the editing of internal memories.
\citet{Meng2022} introduced ROME (Rank-One Model Editing), a method that uses causal tracing to empirically locate the layers essential to encoding a given association. They then modify these modules by applying small rank-one changes. To identify the relevant modules, they run the network multiple times, introducing corruptions to the input sequence to disturb the inference, and then restore individual states from the original non-corrupted pass. But this work an others worked only on single edits, and were often evaluated one edit at a time, starting each time from a fresh pre-trained model. The same authors later developed MEMIT, which follows the same causal tracing principle but with the goal of scaling up to 10,000 edits in bulk\citep{Meng2022a}.
Similarly, \citet{Dai2021} leveraged the identification of knowledge neurons to perform ``knowledge surgery" – editing factual knowledge within Transformers without the need for additional fine-tuning.
\citet{Zhu2020} approached the knowledge modification task as a constrained optimization problem. Their work found that constrained layer-wise fine-tuning emerges as an effective method for modifying the knowledge that Transformers learn, suggesting a different pathway for knowledge editing inside LLMs.
\citet{DeCao2021} proposed \textsc{KnowledgeEditor}, which achieved knowledge editing by training a hyper-network with constrained optimization to modify specific facts without fine-tuning or changing the overall stored knowledge. The method was demonstrated on smaller models like BERT for fact-checking and BART for question answering, achieving consistent changes in predictions across different formulations of queries.

\citet{li2023unveiling}  empirically investigate the pitfalls of knowledge editing in LLMs, revealing two critical issues: logical inconsistencies between multiple edits (like contradictory relationship updates) and knowledge distortion (where edits irreversibly damage the model's knowledge structure). Through carefully designed benchmarks \textsc{ConflictEdit} and \textsc{RoundEdit}, they demonstrate that current editing methods struggle with these challenges, particularly when handling reverse relationships or composite logical rules. While their work focuses on identifying limitations in maintaining logical consistency across edits, our paper takes a complementary cognitive-inspired perspective by addressing how models handle contradictions with their existing knowledge base. Their findings about knowledge distortion align with and reinforce our observations about the catastrophic nature of updates that modify existing knowledge.

Similarly, \citet{huang2024reasons} empirically investigate causes of performance degradation during knowledge editing in LLMs. They show degradation correlates with editing target complexity and L1-norm growth in edited layers. Their proposed Dump for Sequence (D4S) method regulates layer norm growth using O(1) space complexity, enabling multiple effective updates while minimizing model degradation. Their work provides valuable insights into the mechanisms of model degradation during knowledge editing, but it does not specifically address the distinction between contradictory and non-contradictory updates, as we do in this paper.

\citet{tan2023massive} propose MALMEN, a scalable hypernetwork approach for editing Large Language Models by aggregating parameter shifts using a least-squares formulation. While previous editing methods like MEND~\citep{mitchell2022fast} could handle only a few facts simultaneously, MALMEN can efficiently edit thousands of facts while maintaining comparable performance. Their key innovation lies in separating the computation between the hypernetwork and LM, enabling arbitrary batch sizes and reducing memory requirements. Their empirical results show that MALMEN can edit hundreds of times more facts than MEND while maintaining similar performance levels, though they note that the method still struggles with generalizing to rephrasing not seen during training. Like other editing approaches, MALMEN focuses on the mechanics of (by design conflicting) updates.

\textit{Unlike all the work above, our goal in this work is not to edit knowledge, but to understand the fundamental mechanisms and phenomena that govern how LLMs integrate new information with existing knowledge. By taking a cognitive-inspired approach focused on dissonance awareness and adaptive plasticity, we reveal critical insights about the nature of knowledge representation and updating in these models.}


% https://proceedings.neurips.cc/paper/2021/hash/2a10665525774fa2501c2c8c4985ce61-Abstract.html
% https://proceedings.neurips.cc/paper_files/paper/2022/hash/80133d0f6eccaace15508f91e3c5a93c-Abstract-Conference.html
% https://arxiv.org/abs/1903.04476
% https://proceedings.mlr.press/v199/abbasi22a
% https://openaccess.thecvf.com/content/CVPR2022/html/Yan_Learning_Bayesian_Sparse_Networks_With_Full_Experience_Replay_for_Continual_CVPR_2022_paper.html
% https://proceedings.neurips.cc/paper/2020/hash/258be18e31c8188555c2ff05b4d542c3-Abstract.html
% https://proceedings.mlr.press/v234/yildirim24a.html
% https://www.sciencedirect.com/science/article/pii/S0925231221001545
% https://openaccess.thecvf.com/content/ICCV2023/html/Jin_Growing_a_Brain_with_Sparsity-Inducing_Generation_for_Continual_Learning_ICCV_2023_paper.html
% https://proceedings.neurips.cc/paper/2021/hash/f1e709e6aef16ba2f0cd6c7e4f52b9b6-Abstract.html
% https://arxiv.org/abs/2401.05667
% https://openaccess.thecvf.com/content_CVPR_2019/html/Ostapenko_Learning_to_Remember_A_Synaptic_Plasticity_Driven_Framework_for_Continual_CVPR_2019_paper.html
% https://arxiv.org/abs/2206.09117
% https://arxiv.org/abs/2403.08763
% https://arxiv.org/abs/1803.03635
% https://arxiv.org/abs/2406.05955
% https://arxiv.org/abs/2402.01089
% https://arxiv.org/abs/2202.12002
% https://arxiv.org/abs/2404.16789
% https://arxiv.org/abs/2406.17245
% https://arxiv.org/abs/2205.12393
% https://openaccess.thecvf.com/content/CVPR2024/html/Roy_Convolutional_Prompting_meets_Language_Models_for_Continual_Learning_CVPR_2024_paper.html
% https://arxiv.org/abs/2110.03215
% https://arxiv.org/abs/2405.18653
% https://arxiv.org/abs/2012.15283
% https://arxiv.org/abs/2205.09357
% https://arxiv.org/abs/2301.12314
% https://aclanthology.org/2020.coling-main.318/
% https://arxiv.org/abs/2401.03129
% https://arxiv.org/abs/2302.03241
% https://arxiv.org/abs/2309.14763
% https://arxiv.org/abs/2406.06962
% https://arxiv.org/abs/2406.01392
% https://arxiv.org/abs/2404.00790
% https://arxiv.org/abs/2406.18708
% https://arxiv.org/abs/2406.01375
% https://www.cell.com/trends/cognitive-sciences/fulltext/S1364-66132030219-9
% https://openreview.net/forum?id=DimPeeCxKO
% https://arxiv.org/pdf/1612.00796


% New List: ongoing check of ICLR'24 and Nips'24:
% https://openreview.net/pdf?id=8SDsff42Lj
% https://arxiv.org/pdf/2408.17394
% https://arxiv.org/pdf/2410.20444
% https://cdn.iiit.ac.in/cdn/precog.iiit.ac.in/pubs/NeurIPS-RanDumb.pdf
% https://arxiv.org/pdf/2407.16920v1
% https://arxiv.org/pdf/2405.20236
% https://arxiv.org/pdf/2410.22658
% https://arxiv.org/pdf/2406.18868
% https://arxiv.org/pdf/2403.19137
% https://arxiv.org/pdf/2411.02175
% https://arxiv.org/pdf/2410.23843
% https://arxiv.org/pdf/2404.07470
% https://arxiv.org/pdf/2404.00781
% https://arxiv.org/pdf/2403.13249
% https://arxiv.org/pdf/2303.11863
% https://arxiv.org/pdf/2311.04661
% https://arxiv.org/pdf/2310.02129




% Interesting work:
% https://arxiv.org/pdf/2403.19137 
%The paper~\citep{jha2024clap4clip} tackles class-incremental continual learning with CLIP, where new groups of classes (tasks) arrive sequentially and the model must learn to classify them without forgetting previous classes. For example, in CIFAR100, they split 100 classes into 10 tasks of 10 classes each. Building on previous work that showed CLIP needs task-specific finetuning (either through learned prompts like CoOp or feature adapters like CLIP-Adapter), they identify a key problem: during continual learning, the adapted text features increasingly deviate from CLIP's frozen image features (measured by increasing angular distance), hurting performance. Their solution has three main components: (1) a visual-guided attention module that helps maintain alignment between visual and text features during adaptation, (2) task-specific probabilistic adapters that learn distributions over text features rather than point estimates, and (3) using CLIP's pre-trained language knowledge to initialize and regularize these adapters. They show this approach reduces cross-modal deviation and outperforms previous methods across multiple datasets. While simpler solutions might be possible (like directly enforcing alignment or transforming frozen features), the paper takes a more complex probabilistic approach that enables additional benefits like uncertainty estimation.

%https://nips.cc/virtual/2024/poster/94505
% Super interesting work, one of the closest to ours: shows that 

%https://arxiv.org/pdf/2404.00781
% Our work closely relates to the recent UPGD (Utility-based Perturbed Gradient Descent) approach [https://arxiv.org/pdf/2404.00781], which similarly investigates plasticity/stability mechanisms for continual learning in deep learning more generally. UPGD proposes a utility-based approach that protects important weights while allowing updates to less useful ones, combining gradient updates with perturbations based on a carefully approximated utility measure. While both works share similar motivations and empirical findings about the importance of selective plasticity, our work specifically focuses on cognitive dissonance awareness and the particular challenges of conflicting updates in language models. Our findings complement theirs by demonstrating that while non-conflicting updates can be handled relatively well with various strategies (including simple fine-tuning), conflicting updates pose a unique challenge that requires explicit conflict detection and resolution mechanisms. Additionally, while UPGD uses a theoretically-grounded utility approximation to guide updates, we show that even simple gradient-based historical tracking can effectively identify appropriate regions for targeted updates in practice.

% \begin{comment}
% \subsection{Raw list of continual learning}
% % does not compile wtf
% \begin{itemize}
%     \item \href{https://proceedings.neurips.cc/paper/2021/hash/2a10665525774fa2501c2c8c4985ce61-Abstract.html}{https://proceedings.neurips.cc/paper/2021/hash/2a10665525774fa2501c2c8c4985ce61-Abstract.html}
%     \item \href{https://proceedings.neurips.cc/paper_files/paper/2022/hash/80133d0f6eccaace15508f91e3c5a93c-Abstract-Conference.html}{https://proceedings.neurips.cc/paper_files/paper/2022/hash/80133d0f6eccaace15508f91e3c5a93c-Abstract-Conference.html}
%     \item \href{https://arxiv.org/abs/1903.04476}{https://arxiv.org/abs/1903.04476}
%     \item \href{https://proceedings.mlr.press/v199/abbasi22a}{https://proceedings.mlr.press/v199/abbasi22a}
%     \item \href{https://openaccess.thecvf.com/content/CVPR2022/html/Yan_Learning_Bayesian_Sparse_Networks_With_Full_Experience_Replay_for_Continual_CVPR_2022_paper.html}{https://openaccess.thecvf.com/content/CVPR2022/html/Yan_Learning_Bayesian_Sparse_Networks_With_Full_Experience_Replay_for_Continual_CVPR_2022_paper.html}
%     \item \href{https://proceedings.neurips.cc/paper/2020/hash/258be18e31c8188555c2ff05b4d542c3-Abstract.html}{https://proceedings.neurips.cc/paper/2020/hash/258be18e31c8188555c2ff05b4d542c3-Abstract.html}
%     \item \href{https://proceedings.mlr.press/v234/yildirim24a.html}{https://proceedings.mlr.press/v234/yildirim24a.html}
%     \item \href{https://www.sciencedirect.com/science/article/pii/S0925231221001545}{https://www.sciencedirect.com/science/article/pii/S0925231221001545}
%     \item \href{https://openaccess.thecvf.com/content/ICCV2023/html/Jin_Growing_a_Brain_with_Sparsity-Inducing_Generation_for_Continual_Learning_ICCV_2023_paper.html}{https://openaccess.thecvf.com/content/ICCV2023/html/Jin_Growing_a_Brain_with_Sparsity-Inducing_Generation_for_Continual_Learning_ICCV_2023_paper.html}
%     \item \href{https://proceedings.neurips.cc/paper/2021/hash/f1e709e6aef16ba2f0cd6c7e4f52b9b6-Abstract.html}{https://proceedings.neurips.cc/paper/2021/hash/f1e709e6aef16ba2f0cd6c7e4f52b9b6-Abstract.html}
%     \item \href{https://arxiv.org/abs/2401.05667}{https://arxiv.org/abs/2401.05667}
%     \item \href{https://openaccess.thecvf.com/content_CVPR_2019/html/Ostapenko_Learning_to_Remember_A_Synaptic_Plasticity_Driven_Framework_for_Continual_CVPR_2019_paper.html}{https://openaccess.thecvf.com/content_CVPR_2019/html/Ostapenko_Learning_to_Remember_A_Synaptic_Plasticity_Driven_Framework_for_Continual_CVPR_2019_paper.html}
%     \item \href{https://arxiv.org/abs/2206.09117}{https://arxiv.org/abs/2206.09117}
%     \item \href{https://arxiv.org/abs/2403.08763}{https://arxiv.org/abs/2403.08763}
%     \item \href{https://arxiv.org/abs/1803.03635}{https://arxiv.org/abs/1803.03635}
%     \item \href{https://arxiv.org/abs/2406.05955}{https://arxiv.org/abs/2406.05955}
%     \item \href{https://arxiv.org/abs/2402.01089}{https://arxiv.org/abs/2402.01089}
%     \item \href{https://arxiv.org/abs/2202.12002}{https://arxiv.org/abs/2202.12002}
%     \item \href{https://arxiv.org/abs/2404.16789}{https://arxiv.org/abs/2404.16789}
%     \item \href{https://arxiv.org/abs/2406.17245}{https://arxiv.org/abs/2406.17245}
%     \item \href{https://arxiv.org/abs/2205.12393}{https://arxiv.org/abs/2205.12393}
%     \item \href{https://openaccess.thecvf.com/content/CVPR2024/html/Roy_Convolutional_Prompting_meets_Language_Models_for_Continual_Learning_CVPR_2024_paper.html}{https://openaccess.thecvf.com/content/CVPR2024/html/Roy_Convolutional_Prompting_meets_Language_Models_for_Continual_Learning_CVPR_2024_paper.html}
%     \item \href{https://arxiv.org/abs/2110.03215}{https://arxiv.org/abs/2110.03215}
%     \item \href{https://arxiv.org/abs/2405.18653}{https://arxiv.org/abs/2405.18653}
%     \item \href{https://arxiv.org/abs/2012.15283}{https://arxiv.org/abs/2012.15283}
%     \item \href{https://arxiv.org/abs/2205.09357}{https://arxiv.org/abs/2205.09357}
%     \item \href{https://arxiv.org/abs/2301.12314}{https://arxiv.org/abs/2301.12314}
%     \item \href{https://aclanthology.org/2020.coling-main.318/}{https://aclanthology.org/2020.coling-main.318/}
%     \item \href{https://arxiv.org/abs/2401.03129}{https://arxiv.org/abs/2401.03129}
%     \item \href{https://arxiv.org/abs/2302.03241}{https://arxiv.org/abs/2302.03241}
%     \item \href{https://arxiv.org/abs/2309.14763}{https://arxiv.org/abs/2309.14763}
%     \item \href{https://arxiv.org/abs/2406.06962}{https://arxiv.org/abs/2406.06962}
%     \item \href{https://arxiv.org/abs/2406.01392}{https://arxiv.org/abs/2406.01392}
%     \item \href{https://arxiv.org/abs/2404.00790}{https://arxiv.org/abs/2404.00790}
%     \item \href{https://arxiv.org/abs/2406.18708}{https://arxiv.org/abs/2406.18708}
%     \item \href{https://arxiv.org/abs/2406.01375}{https://arxiv.org/abs/2406.01375}
%     \item \href{https://www.cell.com/trends/cognitive-sciences/fulltext/S1364-66132030219-9}{https://www.cell.com/trends/cognitive-sciences/fulltext/S1364-66132030219-9}
%     \item \href{https://openreview.net/forum?id=DimPeeCxKO}{https://openreview.net/forum?id=DimPeeCxKO}
%     \item \href{https://arxiv.org/pdf/1612.00796}{Elastic weight consolidation deepmind 2017} **Elastic Weight Consolidation (EWC)** is a method designed to mitigate catastrophic forgetting in neural networks trained sequentially on distinct tasks. The core idea is to protect the most important weights (or neurons) for previously learned tasks during the training of new tasks. EWC identifies these important weights by calculating the **Fisher Information Matrix** during or after the training of a task, which estimates how sensitive each weight is to the task’s performance. Weights that significantly impact the output for a given task are marked as important. A **quadratic penalty** is then applied during future learning, constraining these weights to remain close to their values from the previous task. This ensures that knowledge from earlier tasks is preserved while still allowing the model to adapt to new tasks.

% However, **EWC is less suitable for large language models (LLMs)**, which are typically trained on massive datasets that do not have clearly defined tasks. EWC's effectiveness relies on distinct task boundaries and the ability to compute task-specific importance for weights, which is feasible in scenarios with well-defined tasks, such as classification or reinforcement learning. In LLMs, where learning spans a wide range of topics and linguistic structures without clear task delineation, it’s challenging to apply EWC's task-based strategy. The model would struggle to assign specific neurons or weights to individual tasks or concepts, making it difficult to protect task-specific knowledge without hindering the model’s overall generalization ability across a diverse dataset.
% \end{itemize}

% % https://proceedings.neurips.cc/paper/2021/hash/2a10665525774fa2501c2c8c4985ce61-Abstract.html
% % https://proceedings.neurips.cc/paper_files/paper/2022/hash/80133d0f6eccaace15508f91e3c5a93c-Abstract-Conference.html
% % https://arxiv.org/abs/1903.04476
% % https://proceedings.mlr.press/v199/abbasi22a
% % https://openaccess.thecvf.com/content/CVPR2022/html/Yan_Learning_Bayesian_Sparse_Networks_With_Full_Experience_Replay_for_Continual_CVPR_2022_paper.html
% % https://proceedings.neurips.cc/paper/2020/hash/258be18e31c8188555c2ff05b4d542c3-Abstract.html
% % https://proceedings.mlr.press/v234/yildirim24a.html
% % https://www.sciencedirect.com/science/article/pii/S0925231221001545
% % https://openaccess.thecvf.com/content/ICCV2023/html/Jin_Growing_a_Brain_with_Sparsity-Inducing_Generation_for_Continual_Learning_ICCV_2023_paper.html
% % https://proceedings.neurips.cc/paper/2021/hash/f1e709e6aef16ba2f0cd6c7e4f52b9b6-Abstract.html
% % https://arxiv.org/abs/2401.05667
% % https://openaccess.thecvf.com/content_CVPR_2019/html/Ostapenko_Learning_to_Remember_A_Synaptic_Plasticity_Driven_Framework_for_Continual_CVPR_2019_paper.html
% % https://arxiv.org/abs/2206.09117
% % https://arxiv.org/abs/2403.08763
% % https://arxiv.org/abs/1803.03635
% % https://arxiv.org/abs/2406.05955
% % https://arxiv.org/abs/2402.01089
% % https://arxiv.org/abs/2202.12002
% % https://arxiv.org/abs/2404.16789
% % https://arxiv.org/abs/2406.17245
% % https://arxiv.org/abs/2205.12393
% % https://openaccess.thecvf.com/content/CVPR2024/html/Roy_Convolutional_Prompting_meets_Language_Models_for_Continual_Learning_CVPR_2024_paper.html
% % https://arxiv.org/abs/2110.03215
% % https://arxiv.org/abs/2405.18653
% % https://arxiv.org/abs/2012.15283
% % https://arxiv.org/abs/2205.09357
% % https://arxiv.org/abs/2301.12314
% % https://aclanthology.org/2020.coling-main.318/
% % https://arxiv.org/abs/2401.03129
% % https://arxiv.org/abs/2302.03241
% % https://arxiv.org/abs/2309.14763
% % https://arxiv.org/abs/2406.06962
% % https://arxiv.org/abs/2406.01392
% % https://arxiv.org/abs/2404.00790
% % https://arxiv.org/abs/2406.18708
% % https://arxiv.org/abs/2406.01375
% % https://www.cell.com/trends/cognitive-sciences/fulltext/S1364-66132030219-9
% % https://openreview.net/forum?id=DimPeeCxKO
% % https://arxiv.org/pdf/1612.00796


% \subsection{Raw list of continual learning}
% % does not compile wtf
% \begin{itemize}
%     \item \href{https://proceedings.neurips.cc/paper/2021/hash/2a10665525774fa2501c2c8c4985ce61-Abstract.html}{https://proceedings.neurips.cc/paper/2021/hash/2a10665525774fa2501c2c8c4985ce61-Abstract.html}
%     \item \href{https://proceedings.neurips.cc/paper_files/paper/2022/hash/80133d0f6eccaace15508f91e3c5a93c-Abstract-Conference.html}{https://proceedings.neurips.cc/paper_files/paper/2022/hash/80133d0f6eccaace15508f91e3c5a93c-Abstract-Conference.html}
%     \item \href{https://arxiv.org/abs/1903.04476}{https://arxiv.org/abs/1903.04476}
%     \item \href{https://proceedings.mlr.press/v199/abbasi22a}{https://proceedings.mlr.press/v199/abbasi22a}
%     \item \href{https://openaccess.thecvf.com/content/CVPR2022/html/Yan_Learning_Bayesian_Sparse_Networks_With_Full_Experience_Replay_for_Continual_CVPR_2022_paper.html}{https://openaccess.thecvf.com/content/CVPR2022/html/Yan_Learning_Bayesian_Sparse_Networks_With_Full_Experience_Replay_for_Continual_CVPR_2022_paper.html}
%     \item \href{https://proceedings.neurips.cc/paper/2020/hash/258be18e31c8188555c2ff05b4d542c3-Abstract.html}{https://proceedings.neurips.cc/paper/2020/hash/258be18e31c8188555c2ff05b4d542c3-Abstract.html}
%     \item \href{https://proceedings.mlr.press/v234/yildirim24a.html}{https://proceedings.mlr.press/v234/yildirim24a.html}
%     \item \href{https://www.sciencedirect.com/science/article/pii/S0925231221001545}{https://www.sciencedirect.com/science/article/pii/S0925231221001545}
%     \item \href{https://openaccess.thecvf.com/content/ICCV2023/html/Jin_Growing_a_Brain_with_Sparsity-Inducing_Generation_for_Continual_Learning_ICCV_2023_paper.html}{https://openaccess.thecvf.com/content/ICCV2023/html/Jin_Growing_a_Brain_with_Sparsity-Inducing_Generation_for_Continual_Learning_ICCV_2023_paper.html}
%     \item \href{https://proceedings.neurips.cc/paper/2021/hash/f1e709e6aef16ba2f0cd6c7e4f52b9b6-Abstract.html}{https://proceedings.neurips.cc/paper/2021/hash/f1e709e6aef16ba2f0cd6c7e4f52b9b6-Abstract.html}
%     \item \href{https://arxiv.org/abs/2401.05667}{https://arxiv.org/abs/2401.05667}
%     \item \href{https://openaccess.thecvf.com/content_CVPR_2019/html/Ostapenko_Learning_to_Remember_A_Synaptic_Plasticity_Driven_Framework_for_Continual_CVPR_2019_paper.html}{https://openaccess.thecvf.com/content_CVPR_2019/html/Ostapenko_Learning_to_Remember_A_Synaptic_Plasticity_Driven_Framework_for_Continual_CVPR_2019_paper.html}
%     \item \href{https://arxiv.org/abs/2206.09117}{https://arxiv.org/abs/2206.09117}
%     \item \href{https://arxiv.org/abs/2403.08763}{https://arxiv.org/abs/2403.08763}
%     \item \href{https://arxiv.org/abs/1803.03635}{https://arxiv.org/abs/1803.03635}
%     \item \href{https://arxiv.org/abs/2406.05955}{https://arxiv.org/abs/2406.05955}
%     \item \href{https://arxiv.org/abs/2402.01089}{https://arxiv.org/abs/2402.01089}
%     \item \href{https://arxiv.org/abs/2202.12002}{https://arxiv.org/abs/2202.12002}
%     \item \href{https://arxiv.org/abs/2404.16789}{https://arxiv.org/abs/2404.16789}
%     \item \href{https://arxiv.org/abs/2406.17245}{https://arxiv.org/abs/2406.17245}
%     \item \href{https://arxiv.org/abs/2205.12393}{https://arxiv.org/abs/2205.12393}
%     \item \href{https://openaccess.thecvf.com/content/CVPR2024/html/Roy_Convolutional_Prompting_meets_Language_Models_for_Continual_Learning_CVPR_2024_paper.html}{https://openaccess.thecvf.com/content/CVPR2024/html/Roy_Convolutional_Prompting_meets_Language_Models_for_Continual_Learning_CVPR_2024_paper.html}
%     \item \href{https://arxiv.org/abs/2110.03215}{https://arxiv.org/abs/2110.03215}
%     \item \href{https://arxiv.org/abs/2405.18653}{https://arxiv.org/abs/2405.18653}
%     \item \href{https://arxiv.org/abs/2012.15283}{https://arxiv.org/abs/2012.15283}
%     \item \href{https://arxiv.org/abs/2205.09357}{https://arxiv.org/abs/2205.09357}
%     \item \href{https://arxiv.org/abs/2301.12314}{https://arxiv.org/abs/2301.12314}
%     \item \href{https://aclanthology.org/2020.coling-main.318/}{https://aclanthology.org/2020.coling-main.318/}
%     \item \href{https://arxiv.org/abs/2401.03129}{https://arxiv.org/abs/2401.03129}
%     \item \href{https://arxiv.org/abs/2302.03241}{https://arxiv.org/abs/2302.03241}
%     \item \href{https://arxiv.org/abs/2309.14763}{https://arxiv.org/abs/2309.14763}
%     \item \href{https://arxiv.org/abs/2406.06962}{https://arxiv.org/abs/2406.06962}
%     \item \href{https://arxiv.org/abs/2406.01392}{https://arxiv.org/abs/2406.01392}
%     \item \href{https://arxiv.org/abs/2404.00790}{https://arxiv.org/abs/2404.00790}
%     \item \href{https://arxiv.org/abs/2406.18708}{https://arxiv.org/abs/2406.18708}
%     \item \href{https://arxiv.org/abs/2406.01375}{https://arxiv.org/abs/2406.01375}
%     \item \href{https://www.cell.com/trends/cognitive-sciences/fulltext/S1364-66132030219-9}{https://www.cell.com/trends/cognitive-sciences/fulltext/S1364-66132030219-9}
%     \item \href{https://openreview.net/forum?id=DimPeeCxKO}{https://openreview.net/forum?id=DimPeeCxKO}
% \end{itemize}

% \end{comment}

\section{Proofs and Additional Details for \Cref{sec:lin_rep}}

\subsection{Convergence Rate Lower Bound of \SGD}\label{sec:SGD_lower_bound}

Our goal is to establish the following lower bound construction.
\LinRepLowerBound*

\begin{proof}[Proof of \Cref{prop:linrep_SGD_lower_bound}]
    We prove the lower bound by construction. First, we write out the one-step \SGD update given step size $\eta_\sbG$.
    \begin{align*}
        \overline\bG_+ &= \bG - \eta_\sbG  \nabla_{\sbG} \hatL(\bfF, \bfG) \\
        &= \bG - \frac{1}{n}\bfF^\top \mem\paren{\bfF \bG \bX^\top \bX - \bY^\top \bX} \\
        &= \bG - \bfF^\top \mem\paren{\bfF\bG - \Fstar \Gstar }. \tag{$\bY^\top  = \Fstar \Gstar \bX^\top + \Eps^\top$, $n = \infty$, $\bSigma_\bfx = \bI$} \\
        \bG_+ &= \mathrm{Ortho}(\overline\bG_+).
    \end{align*}
    We recall $\bF$ is given by the $\bF$-update in \eqref{eq:KFAC_update} with $\eta_\sbF = 1$, which by \Cref{lem:KFAC_is_EMA} is equivalent to setting $\bF$ to the least-squares solution conditional on $\bG$:
    \begin{align*}
        \bF &= \bY^\top \bZ \, (\bZ^\top \bZ)^{-1} = \Fstar\Gstar \bSigma_\bfx \bG^\top \paren{\bG \bSigma_\bfx \bG^\top}^{-1} \tag{$\bfz = \bG \bfx$, $n = \infty$} \\
        &= \Fstar\Gstar \bG^\top \paren{\bG \bG^\top}^{-1} = \Fstar\Gstar \bG^\top. \tag{$\bG$ row-orthonormal}
    \end{align*}
    Therefore, plugging in $\bF$ into the $\SGD$ update yields:
    \begin{align*}
        \overline\bG_+ &= \bG - \eta_\sbG \bfF^\top \mem\paren{\bfF\bG - \Fstar \Gstar}= \bG - \eta_\sbG \bG \Gstar^\top \Fstar^\top (\Fstar\Gstar \bG^\top \bG - \Fstar \Gstar)
    \end{align*}
    Before proceeding, let us present the construction of $\Fstar, \Gstar$. We focus on the case $\dx = 3$, $k = 2$, as it will be clear the construction is trivially embedded to arbitrary $\dx > k \geq 2$. We observe that $\Fstar$ only appears in the \SGD update in the form $\Fstar^\top \Fstar \in \R^{k \times k}$, thus $\dy \geq k$ can be set arbitrarily as long as $\Fstar^\top \Fstar$ satisfies our specifications. Set $\Fstar, \Gstar$ such that
    \begin{align*}
        \Fstar^\top \Fstar &= \bmat{1 - \lambda & 0 \\ 0 & \lambda}, \lambda \in (0,1/2], \quad \Gstar = \bmat{1 & 0 & 0 \\ 0 & 1 & 0}.
    \end{align*}
    Accordingly, the initial representation $\bG_0$ (which the learner is not initially given) will have form
    \begin{align*}
        \bG_0 &= \bmat{1 & 0 & 0\\ 0 & \sqrt{1 - \varepsilon_0^2} & \varepsilon_0 }, \text{ or } \bmat{\sqrt{1 - \varepsilon_0^2} & 0 &\varepsilon_0 \\ 0 & 1 & 0}.
    \end{align*}
    We prove all results with the first form of $\bG_0$, as all results will hold for the second with the only change swapping $\lambda, 1 - \lambda$. It is clear that we may extend to arbitrary $\dx > k \geq 2$ by setting:
    \begin{align*}
        \Fstar^\top \Fstar &= \bmat{
        (1-\lambda) \bI_{k-1} & \bzero \\ \bzero & \lambda
        }, \quad \Gstar = \bmat{\bI_{k} & \bzero_{\dx-k}}, \quad \bG_0 = \bmat{\bI_{k-1} & \bzero & \cdots &  \\
        0 & \sqrt{1 - \varepsilon_0^2} & \varepsilon_0 & \bzero }.
    \end{align*}
    Returning to the $\dx = 3, k = 2$ case, we first prove the following invariance result.
    \begin{lemma}\label{lem:LB_facts}
        Given $\bG_0 = \bmat{1 & 0 & 0\\ 0 & \sqrt{1 - \varepsilon_0^2} & \varepsilon_0 }$, then for any $t \geq 0$, $\bG_t = \bmat{1 & 0 & 0 \\ 0 & c_1 & c_2}$ for some $c_1^2 + c_2^2 = 1$. 
        Furthermore, we have $\dist(\bG_t, \Gstar) = \abs{c_2}$.
    \end{lemma}
    \begin{proof}[Proof of \Cref{lem:LB_facts}]
    This follows by induction. The base case follows by definition of $\bG_0$. Now given $\bG_t = \bmat{1 & 0 & 0 \\ 0 & c_1 & c_2}$ for some $c_1,c_2$, we observe that
    \begin{align*}
        \bG_t \Gstar^\top = \bmat{1 & 0 \\ 0 & c_1 }, \quad
        \Fstar^\top \Fstar = \bmat{1 - \lambda & 0 \\ 0 & \lambda}.
    \end{align*}
    Notably, we may write
    \begin{align}
    \begin{split}\label{eq:LB_grad_update}
        \overline\bG_{t+1} &= \bG_t - \eta_\sbG  \bG_t \Gstar^\top \Fstar^\top (\Fstar\Gstar \bG_t^\top \bG_t - \Fstar \Gstar) \\
        &= \paren{\bI_{k} - \eta_\sbG  \bG_t \Gstar^\top \Fstar^\top \Fstar\Gstar \bG_t^\top } \bG_t + \eta_\sbG  \bG_t \Gstar^\top \Fstar^\top \Fstar \Gstar \\
        &= \paren{\bI_k - \eta_\sbG \bmat{1-\lambda & 0 \\ 0 & c_1^2 \lambda  }} \bG_t + \eta_\sbG \bmat{1-\lambda & 0 \\ 0 & c_1 \lambda  } \Gstar \\
        &= \bmat{1 & 0 & 0 \\
        0 & c_1 (1 + \eta_\sbG c_2^2 \lambda) & c_2 (1 - \eta_\sbG c_1^2 \lambda)} \\
        \bG_{t+1} &= \mathrm{Ortho}(\overline\bG_{t+1}).
    \end{split}
    \end{align}
    Therefore, $\bG_{t+1}$ shares the same support as $\bG_t$, and by the orthonormalization step, the squared entries of the second row of $\bG_{t+1}$ equal $1$, completing the induction step.

    To prove the second claim, we see that $\Gperp = \bmat{\bzero_2 & \\ & 1}$, and since $\bG_t$ is by assumption row-orthonormal, we have
    \begin{align*}
        \dist(\bG_t, \Gstar) &= \norm{\bG_t \Gperp} = \opnorm{\bmat{0 & 0 & c_2}^\top} = \abs{c_2},
    \end{align*}
    completing the proof.
    \end{proof}

With these facts in hand, we prove the following stability limit of the step-size, and the consequences for the contraction rate.
\begin{lemma}\label{lem:LB_max_lr}
    If $\eta_\sbG \geq \frac{4}{1-\lambda}$, then for any given $\dist(\bG_0, \Gstar)$ we may find $\bG_0$ such that $\limsup_t \dist(\bG_t, \Gstar) \geq \frac{1}{2}$.
\end{lemma}
\begin{proof}[Proof of \Cref{lem:LB_max_lr}]
    By assumption $\lambda \leq 1/2$ and thus $\lambda \leq 1- \lambda$. Evaluating \Cref{lem:LB_facts} instead on $\bG_0 = \bmat{\sqrt{1 - \varepsilon_0^2} & 0 &\varepsilon_0 \\ 0 & 1 & 0}$, writing out \eqref{eq:LB_grad_update} yields symmetrically:
\begin{align*}
    \dist(\bG_t, \Gstar) &= \abs{c_2} \\
    \overline\bG_{t+1} &= \bmat{
        c_1 (1 + \eta_\sbG c_2^2 (1-\lambda) ) & 0 & c_2 (1 - \eta_\sbG c_1^2 (1-\lambda) ) \\
        0 & 1 & 0} \\
    \bG_{t+1} &= \mathrm{Ortho}(\overline\bG_{t+1}).
\end{align*}
We first observe that regardless of $\eta_\sbG$, the norm of the first row $\bG_{t+1}$ is always greater than $1$ pre-orthonormalization. Let us define $\omega = \eta_\sbG(1-\lambda)$. Then, the squared-norm of the first row satisfies:
\begin{align*}
    \paren{c_1 (1+ \omega c_2^2)}^2 + \paren{c_2 (1 - \omega c_1^2)}^2 &= 1 + \omega^2 c_1^2 c_2^2. 
\end{align*}
Therefore, the norm is strictly bounded away from $1$ when $\omega > 0$ and either $c_1,c_2\neq 0$ by the constraint $c_1^2 + c_2^2 = 1$. Importantly, this implies that regardless of the step-size taken, the resulting first-row norm of $\bG_+$ must exceed $1$ prior to orthonormalization. Given this property, we observe that for $\omega \geq 1/c_1^2$, we have:
\begin{align*}
    \frac{\abs{1-\omega c_1^2}}{\abs{1+\omega c_2^2}} = \frac{\omega c_1^2 - 1}{1 + \omega c_2^2}.
\end{align*}
When this ratio is greater than $1$, we are guaranteed that the first-row coefficients $c_1', c_2'$ of $\bG_{t+1}$ \textit{post}-orthonormalization satisfy $c_2'/c_1' > c_2/c_1$, and recall from \Cref{lem:LB_facts} $\dist(\bG_{t+1}, \Gstar) = c_2'$, and thus $\dist(\bG_{t+1}, \Gstar) > \dist(\bG_t, \Gstar)$. Rearranging the above ratio, this is equivalent to the condition $\omega = \eta_\sbG (1-\lambda) \geq \frac{2}{c_1^2 - c_2^2}$, $\omega = \eta_\sbG (1-\lambda) \geq 1/c_1^2$. Setting $c_1^2 = 3/4, c_2^2 = 1/4$, this implies for $\eta_\sbG \geq \frac{4}{1-\lambda}$, the moment $\dist(\bG_t, \Gstar) \leq c_2 = 1/2$, then we are guaranteed $\dist(\bG_{t+1}, \Gstar) > \dist(\bG_t, \Gstar)$, and thus $\limsup_{t \to \infty} \dist(\bG_t, \Gstar) \geq \frac{1}{2}$, irregardless of $\dist(\bG_0, \Gstar)$.
\end{proof}

Now, to finish the construction of the lower bound, \Cref{lem:LB_max_lr} establishes that $\eta_\sbG \leq \frac{4}{1-\lambda}$ is necessary for convergence (though not sufficient!). This implies that when we plug back in $\bG_0 = \bmat{1 & 0 & 0\\ 0 & \sqrt{1 - \varepsilon_0^2} & \varepsilon_0 }$, we have $\bG_{t+1}$:
\begin{align*}
    \overline\bG_{t+1} &= \bmat{1 & 0 & 0 \\
        0 & c_1 (1 + \eta_\sbG c_2^2 \lambda) & c_2 (1 - \eta_\sbG c_1^2 \lambda)} \\
    \bG_{t+1} &= \mathrm{Ortho}(\overline\bG_{t+1}) = \bmat{1 & 0 & 0 \\
        0 & c_1' & c_2'}.
\end{align*}
We have trivially $1 - \eta_\sbG c_1^2 \lambda \geq 1 - \eta_\sbG \lambda$. Therefore, for $\lambda \leq 1/5$ such that $\frac{\lambda}{1-\lambda}\leq 1/4$, we have $1 - \eta_\sbG \lambda \geq 1 - \frac{4\lambda}{1-\lambda} \geq 0$.
As shown in the proof of \Cref{lem:LB_max_lr}, the norm of the second row pre-orthonormalization is strictly greater than $1$, and thus:
\begin{align*}
    \dist(\bG_{t+1}, \Gstar) = c_2' &\geq c_2 (1 - \eta_\sbG\lambda c_1^2) \geq (1 - \eta_\sbG \lambda) \dist(\bG_t, \Gstar)\geq \paren{1 - 4\frac{\lambda}{1-\lambda}}\dist(\bG_t, \Gstar).
\end{align*}
Applying this recursively to $\bG_0$ yields the desired lower bound.
\end{proof}



\subsection{Proof of \Cref{thm:linrep_kfac_guarantee}}

Recall that running an iteration of stylized \KFAC \eqref{eq:KFAC_update} with $\lambda_\sbF, \lambda_\sbG = 0$, $\eta_\sbF = 1$ yields:
\begin{align}
\begin{split}\label{eq:expanding_KFAC_update}
    \overline\bG_+ &= \bG - \eta_{\sbG} \bP_\sbG^{-1}\, \nabla_{\sbG} \hatL(\bfF, \bfG)\, (\bQ_\sbG + \lambda_\sbG \bI_{\dx})^{-1} \\
    &= \bG - \eta_\sbG (\bF^\top \bF)^{-1}\bfF^\top (\bfF \bG \hatSigma - \Fstar \Gstar \hatSigma - \frac{1}{n}\Eps^\top \bX ) \hatSigma^{-1} \\
    &= \bG - \eta_\sbG (\bF^\top \bF)^{-1}\bfF^\top (\bfF \bG - \Fstar \Gstar) + (\bF^\top \bF)^{-1}\bF^\top\Eps^\top \bX (\bX^\top \bX)^{-1},
\end{split} 
\end{align}
where the matrix $\bF$ is given by
\begin{align*}
        \bF &= \bF_{\rm{prev}} - \eta_\sbF \bP_\sbF^{-1}\, \nabla_\sbF \hatL(\bF_{\rm{prev}}, \bfG)\, (\bQ_{\sbF}+\lambda_\sbF \bI_{\dhid})^{-1} = \bY^\top \bZ (\bZ^\top \bZ)^{-1} \\
        &= \Fstar \Gstar \bX^\top \bZ (\bZ^\top \bZ)^{-1} + \Eps^\top \bZ (\bZ^\top \bZ)^{-1},
\end{align*}
recalling that $\bfz \triangleq \bG \bfx$.
Focusing on the representation update, we have
\begin{align*}
    \overline\bG_+ \Gperp &= (1 - \eta_\sbG) \bG \Gperp + \eta_\sbG (\bF^\top \bF)^{-1}\bfF^\top \Eps^\top \bX (\bX^\top \bX)^{-1}.
\end{align*}
Therefore, to prove a one-step contraction of $\rowsp(\bG_+)$ toward $\rowsp(\bG_\star)$, we require two main components:
\begin{itemize}
    \item Bounding the noise term $\eta_\sbG (\bF^\top \bF)^{-1}\bfF^\top \Eps^\top \bX^\top (\bX^\top \bX)^{-1}$.
    \item Bounding the orthonormalization factor; the subspace distance measures distance between two orthonormalized bases (a.k.a.\ elements of the Stiefel manifold \citep{absil2008optimization}), while a step of \SGD or \KFAC does not inherently conform to the Stiefel manifold, and thus the ``off-manifold'' shift must be considered when computing $\dist(\bG_+, \Gstar)$. This amounts to bounding the ``$\bR$''-factor of the $\bQ\bR$-decomposition \citep{tref2022numerical} of $\overline\bG_+$.
\end{itemize}
Thanks to the left-preconditioning by $(\bF^\top \bF)^{-1}$, the contraction factor is essentially determined by $(1- \eta_\sbG)$; however, the second point about the ``off-manifold'' shift is what prevents us from setting $\eta_\sbG = 1$.

\subsection*{Bounding the noise term} 

We start by observing $\Eps^\top \bX(\bX^\top \bX)^{-1} = \sum_{i=1}^n \bveps_i \bfx_i^\top \paren{\sum_{i=1}^n \bfx_i \bfx_i^\top}^{-1}$ (and thus $(\bF^\top \bF)^{-1}\bfF^\top \Eps^\top \bX (\bX^\top \bX)^{-1}$) is a least-squares error-like term, and thus can be bounded by standard self-normalized martingale arguments. In particular, defining $\Fbar \triangleq (\bF^\top \bF)^{-1}\bF^\top$ we may decompose
\begin{align*}
    \opnorm{\Fbar \Eps^\top \bX (\bX^\top \bX)^{-1}} &\leq \opnorm{\Fbar \Eps^\top \bX (\bX^\top \bX)^{-1/2}} \lmin(\bX^\top \bX)^{-1/2},
\end{align*}
where the first factor is the aforementioned self-normalized martingale (see e.g.\ \citet{abbasi2011regret, ziemann2023tutorial}), and the second can be bounded by standard covariance lower-tail bounds. Toward bounding the first factor, we invoke a high-probability self-normalized bound:
\begin{lemma}[cf.\ {\citet[Theorem 4.1]{ziemann2023tutorial}}]\label{lem:yasin_SNM}
Let $\scurly{\bfv_i, \bfw_i}_{i \geq 1}$ be a $\R^{d_v} \times \R^{d_w}$-valued process
and $\scurly{\calF_i}_{i \geq 1}$ be a filtration such that
$\scurly{\bfv_i}_{i \geq 1}$ is adapted to $\scurly{\calF_i}_{i \geq 1}$,
$\scurly{\bfw_i}_{i \geq 1}$ is adapted to $\scurly{\calF_i}_{i \geq 2}$, and 
$\scurly{\bfw_i}_{i \geq 1}$ is a $\sigma^2$-subgaussian martingale difference sequence\footnote{See \Cref{appdx:subgaussian} for discussion of formalism. It suffices for our purposes to consider $\bfw \iidsim \normal(\bzero, \bSigma_\bfw)$.}.
Fix (non-random) positive-definite matrix $\bQ$.
For $k \geq 1$, define $\hatSigma[k] \triangleq \sum_{i=1}^{k} \bfv_i \bfv_i^\top$.
Then, given any fixed $n \in \N_+$, with probability at least $1-\delta$:
\begin{align}
    \opnorm{\sum_{i=1}^n \bfw_i \bfv_i^\top \paren{\bQ + \hatSigma[n]}^{-1/2} }^2 \leq 4 \sigma^2  \log\paren{\frac{\det\paren{\bQ + \hatSigma[n]}}{\det(\bQ)}} + 13 d_w \sigma^2 + 8\sigma^2 \log(1/\delta).
\end{align}
\end{lemma}
Instantiating this for Gaussian $\bfw_i \iidsim \normal(\bzero, \bSigma_\bfw)$, $\bfv_i \iidsim \normal(\bzero, \bSigma_\bfv)$, we may set $\bQ \approx \bSigma_\bfv$ to yield:
\begin{lemma}\label{lem:SNM_bound}
    Consider the quantities defined in \Cref{lem:yasin_SNM} and assume $\bfw_i \iidsim \normal(\bzero, \bSigma_\bfw)$, $\bfv_i \iidsim \normal(\bzero, \bSigma_\bfv)$, defining $\sigma_\bfw^2 \triangleq \lmax(\bSigma_\bfw)$, $\sigma_\bfv^2 \triangleq \lmax(\bSigma_\bfv)$. Then, as long as $n \gtrsim \frac{18.27}{c^2} \paren{d_v + \log(1/\delta)}$, with probability at least $1 - \delta$:
    \begin{align*}
        &\opnorm{\sum_{i=1}^n \bfw_i \bfv_i^\top \paren{\hatSigma[n]}^{-1/2} }^2 \leq 8 d_v \log\paren{\frac{1+c}{1-c}} \sigma_\bfw^2  + 26d_w \sigma_\bfw^2 + 16\sigma_\bfw^2 \log(1/\delta) \\
        &\lmin(\hatSigma[n]) \geq (1-c) \lmin(\bSigma_\bfv).
    \end{align*}
\end{lemma}
\begin{proof}[Proof of \Cref{lem:SNM_bound}]
We observe that if $\hatSigma[n] \succeq \bQ$, then
\[
2 \hatSigma[n] \succeq \bQ + \hatSigma[n] \implies \paren{\hatSigma[n]}^{-1} \preceq 2\paren{\bQ + \hatSigma[n]}^{-1}.
\]
This implies
\begin{align}\label{eq:SNM_event}
    &\ones\mem\curly{\hatSigma[n] \succeq \bQ} \bnorm{\sum_{i=1}^n \bfw_i \bfv_i^\top \paren{\hatSigma[n]}^{-1/2} }^2
    \leq 2 \ones\mem\curly{\hatSigma[n] \succeq \bQ}  \bnorm{\sum_{i=1}^n \bfw_i \bfv_i^\top \paren{\bQ + \hatSigma[n]}^{-1/2} }^2.
\end{align}
Let us consider the event:
\[
(1-c) \bSigma_\bfv \preceq \hatSigma[n] \preceq (1+c) \bSigma_\bfv,
\]
which by \Cref{lem:gauss_cov_conc} occurs with probability at least $1 - \delta$ as long as $n \gtrsim   \frac{18.27}{c^2}(d_v + \log(1/\delta))$. This immediately establishes the latter desired inequality. Setting $\bQ = (1-c) n \bSigma_\bfv$ and conditioning on the above event, we observe that by definition $\hatSigma[n] \succeq \bQ$, and
\begin{align*}
    \log \left(\frac{\det\paren{\bQ + \hatSigma[n]}}{\det(\bQ)}\right) &= \log \det\paren{I_{d_v} + \hatSigma[n]\paren{\bQ}^{-1}} \\
    &\leq \log \det\paren{\paren{1 + \frac{1+c}{1-c}}I_{d_v} }  \\
    &\leq d_v \log\paren{\frac{1+c}{1-c}}.
\end{align*}
Plugging this into \Cref{lem:yasin_SNM}, applied to the RHS of \eqref{eq:SNM_event}, we get our desired result.
\end{proof}
Therefore, instantiating $\bfw \to \Fbar \bveps$, $\bfv \to \bfx$, $(\bfx_i, \bveps_i)_{i \geq 1}$ is a $\R^{\dx} \times \R^k$-valued process. Furthermore, since we assumed out of convenience that $\bF, \bG_+$ are computed on independent batches of data, we have that $\Fbar \bveps \sim \normal(\bzero, \Fbar \bSigma_{\bveps} \Fbar^{\top})$. In order to complete the noise term bound, it suffices to provide a uniform bound on $\norm{\Fbar} = 1/\smin(\bF)$ in terms of $\Fstar$.
\begin{lemma}\label{lem:F_close_to_Fstar}
    Assume the following conditions hold:
    \begin{align*}
        &n \gtrsim \max\curly{k + \log(1/\delta),\; \sigmaeps^2 \frac{\dy + k + \log(1/\delta)}{\smin(\Fstar)^2\lmin(\Sigmax)}} \\
        &\dist(\bG, \Gstar) \leq \frac{2}{5} \kappa(\Fstar)^{-1} \kappa(\Sigmax)^{-1},
    \end{align*}
    then with probability at least $1 - \delta$, we have $\norm{\Fbar} = 1/\smin(\bF) \leq 2 \smin(\Fstar)^{-1}$.
\end{lemma}
\begin{proof}[Proof of \Cref{lem:F_close_to_Fstar}]
    Recall we may write $\bF$ as
    \begin{align}
        \bF &= \Fstar \Gstar \bX^\top \bZ (\bZ^\top \bZ)^{-1} + \Eps^\top \bZ (\bZ^\top \bZ)^{-1} \nonumber \\
        &= \Fstar \Gstar \bG^\top + \Fstar \Gstar (\bI_{\dx} - \bG^\top \bG) \bX^\top \bZ (\bZ^\top \bZ)^{-1} + \Eps^\top \bZ (\bZ^\top \bZ)^{-1}\label{eq:linrep_Fls_expanded}
    \end{align}
    By Weyl's inequality for singular values \cite{horn2012matrix}, we have
    \begin{align*}
        \smin(\bF) &\geq \smin(\Fstar \Gstar \bG^\top) - \smax\paren{\Fstar \Gstar (\bI_{\dx} - \bG^\top \bG) \bX^\top \bZ (\bZ^\top \bZ)^{-1} + \Eps^\top \bZ (\bZ^\top \bZ)^{-1}}
    \end{align*}
    Since $\Gstar \bG^\top$ is an orthogonal matrix, the first term is equal to $\smin(\Fstar)$. On the other hand, applying triangle inequality on the second term, for $n \gtrsim k + \log(1/\delta)$ we have:
    \begin{align*}
        \opnorm{\Fstar \Gstar (\bI_{\dx} - \bG^\top \bG) \bX^\top \bZ (\bZ^\top \bZ)^{-1}} &\leq \opnorm{\Fstar}\opnorm{\Gstar (\bI_{\dx} - \bG^\top \bG)}\opnorm{\bX^\top \bZ (\bZ^\top \bZ)^{-1}} \\
        &\leq \opnorm{\Fstar} \dist(\bG, \Gstar) \paren{\frac{5}{4} \opnorm{\Sigmax} \lmin(\bG \Sigmax \bG^\top)} \\
        &\leq \frac{5}{4}\opnorm{\Fstar} \dist(\bG, \Gstar) \kappa(\Sigmax),
    \end{align*}
    where we used covariance concentration for the second inequality \Cref{lem:gauss_cov_conc} and the trivial bound $\lmin(\bA \bSigma \bA^\top) \geq \lmin(\bSigma)$ for the last inequality. In turn, we may bound:
    \begin{align*}
        \opnorm{\Eps^\top \bZ (\bZ^\top \bZ)^{-1}} &\leq \opnorm{\Eps^\top \bZ (\bZ^\top \bZ)^{-1/2}} \lmin(\bZ^\top \bZ)^{-1/2} \\
        &\lesssim \opnorm{\Eps^\top \bZ (\bZ^\top \bZ)^{-1/2}} n^{-1/2} \lmin(\Sigmax)^{-1/2} \tag{\Cref{lem:gauss_cov_conc}}\\
        &\lesssim \sigmaeps  \sqrt{\frac{\dy + k + \log(1/\delta)}{\lmin(\Sigmax) n}}.
    \end{align*}
    Therefore, setting $\dist(\bG, \Gstar) \leq \frac{2}{5} \kappa(\Fstar)^{-1} \kappa(\Sigmax)^{-1}$, and $n \gtrsim \sigmaeps^2 \frac{\dy + k + \log(1/\delta)}{\smin(\Fstar)^2\lmin(\Sigmax)}$, we have $\smin(\bF) \geq \frac{1}{2}\smin(\Fstar)$, which leads to our desired bound on $\norm{\Fbar}$.
\end{proof}

With a bound on $\opnorm{\Fbar}$, bounding the noise term is a straightforward application of \Cref{lem:SNM_bound}.
\begin{proposition}[\KFAC noise term bound]\label{prop:KFAC_noise_bound}
    Let the conditions in \Cref{lem:F_close_to_Fstar} hold. In addition, assume $n \gtrsim \dx + \log(1/\delta)$. Then, with probability at least $1-\delta$:
    \begin{align*}
        \opnorm{\Fbar \Eps^\top \bX (\bX^\top \bX)^{-1}}
        &\lesssim \sigmaeps  \sqrt{\frac{\dx + k + \log(1/\delta)}{\smin(\Fstar)^2\lmin(\Sigmax) n}}.
    \end{align*}
\end{proposition}


\begin{proof}[Proof of \Cref{prop:KFAC_noise_bound}]
    Condition on the event of \Cref{lem:F_close_to_Fstar}. Then, assuming $n \gtrsim \dx + \log(1/\delta)$, we may apply covariance concentration (\Cref{lem:gauss_cov_conc}) on $\hatSigma$ and \Cref{lem:SNM_bound} to bound the noise term by:
    \begin{align*}
        \opnorm{\Fbar \Eps^\top \bX (\bX^\top \bX)^{-1}} &\leq \opnorm{\Fbar \Eps^\top \bX (\bX^\top \bX)^{-1/2}} \lmin(\bX^\top \bX)^{-1/2} \\
        &\leq \opnorm{\Fbar}\opnorm{\Eps^\top \bX (\bX^\top \bX)^{-1/2}} n^{-1/2} \lmin(\Sigmax)^{-1/2} \tag{\Cref{lem:gauss_cov_conc}}\\
        &\lesssim \sigmaeps  \sqrt{\frac{\dx + k + \log(1/\delta)}{\smin(\Fstar)^2\lmin(\Sigmax) n}}. \tag{\Cref{lem:SNM_bound}},
    \end{align*}
    which completes the proof.
\end{proof}
This completes the bound on the noise term. We proceed to the orthonormalization factor.

\subsection*{Bounding the orthonormalization factor}

Toward bounding the orthonormalization factor from \eqref{eq:linrep_update}. Defining $\overline\bG_+$ as the updated representation pre-orthonormalization, we write $\overline\bG_+ = \bR \bG_+$,
where $\bG_+$ is the orthonormalized representation and $\bR \in \R^{k \times k}$ is the corresponding orthonormalization factor. Therefore, defining the shorthand $\Sym(\bA) = \bA + \bA^\top$, we have
\begin{align*}
    \bR\bR^\top &= \bR \bG_+ (\bR \bG_+)^\top \\
    &= \paren{\bG - \eta_\sbG (\bF^\top \bF)^{-1}\bfF^\top (\bfF \bG - \Fstar \Gstar) + (\bF^\top \bF)^{-1}\bF^\top \Eps^\top \bX (\bX^\top \bX)^{-1}} \paren{\cdots}^\top \tag{from \eqref{eq:expanding_KFAC_update}}\\
    &\succ \bI_{k} - \eta_\sbG \Sym\Big(\underbrace{\Fbar(\bF\bG - \Fstar \Gstar)\bG^\top}_{\triangleq \Gamma_1}\Big) + \eta_\sbG \Sym\Big(\underbrace{\Fbar \Eps^\top \bX (\bX^\top \bX)^{-1} \bG^\top}_{\triangleq \Gamma_2}\Big) \\
    &\qquad \quad - \eta_\sbG^2 \Sym\paren{\Fbar(\bF\bG - \Fstar \Gstar)\paren{\Fbar \Eps^\top \bX (\bX^\top \bX)^{-1}}^\top},
\end{align*}
where the strictly inequality comes from discarding the positive-definite ``diagonal'' terms of the expansion. Therefore, by Weyl's inequality for symmetric matrices \cite{horn2012matrix}, we have:
\begin{align*}
    \lmin(\bR \bR^\top) &\geq 1 - 2\eta_\sbG \paren{\opnorm{\Gamma_1} + \opnorm{\Gamma_2} + \eta_\sbG\opnorm{\Gamma_1}\opnorm{\Gamma_2}}.
\end{align*}
Toward bounding $\opnorm{\Gamma_1}$, let the conditions of \Cref{lem:F_close_to_Fstar} hold. Then,
\begin{align*}
    \opnorm{\Gamma_1} &= \opnorm{\Fbar(\bF\bG - \Fstar \Gstar)\bG^\top} \\
    &= \opnorm{(\bF^\top \bF)^{-1} \bF^\top \bF \bG \bG^\top - (\bF^\top \bF)^{-1} \bF^\top\Fstar \Gstar \bG^\top} \\
    &= \opnorm{\Fbar \paren{\Fstar \Gstar (\bI_{\dx} - \bG^\top \bG) \bX^\top \bZ (\bZ^\top \bZ)^{-1} + \Eps^\top \bZ (\bZ^\top \bZ)^{-1}} } \tag{from \eqref{eq:linrep_Fls_expanded}} \\
    &\leq \frac{5}{4}\smin(\bF) \opnorm{\Fstar} \dist(\bG, \Gstar) \kappa(\Sigmax) + \smin(\bF) \sigmaeps^2 \sqrt{\frac{k + \log(1/\delta)}{\lmin(\Sigmax) n}} \\
    &\leq \underbrace{\frac{5}{2}\kappa(\Fstar) \dist(\bG, \Gstar) \kappa(\Sigmax)}_{\triangleq \gamma_1} + \sigmaeps^2 \sqrt{\frac{k + \log(1/\delta)}{\smin(\Fstar)^2\lmin(\Sigmax) n}}. \tag{\Cref{lem:F_close_to_Fstar}}
\end{align*}
Similarly, letting the conditions of \Cref{prop:KFAC_noise_bound} hold, we have
\begin{align*}
    \opnorm{\Gamma_2} &= \opnorm{\Fbar \Eps^\top \bX (\bX^\top \bX)^{-1} \bG^\top} \\
    &\leq \sigmaeps  \sqrt{\frac{\dx + k + \log(1/\delta)}{\smin(\Fstar)^2\lmin(\Sigmax) n}} \tag{\Cref{prop:KFAC_noise_bound}} \\
    &\triangleq \gamma_2.
\end{align*}
We observe that $\gamma_2$ will always dominate the second term of the bound on $\opnorm{\Gamma_1}$, and therefore:
\begin{align*}
    \lmin(\bR \bR^\top) &\geq 1 - 2\eta_\sbG(\gamma_1 + 2\gamma_2 + \eta_\sbG(\gamma_1 + \gamma_2)\gamma_2).
\end{align*}
Therefore, we have the following bound on the orthonormalization factor.
\begin{proposition}\label{prop:linrep_orth_bound}
Let the following conditions hold:
    \begin{align*}
        &n \gtrsim \max\curly{\dx + \log(1/\delta),\; \frac{\sigmaeps^2}{\gamma_2^2} \frac{\dx + \log(1/\delta)}{\smin(\Fstar)^2\lmin(\Sigmax)}} \\
        &\dist(\bG, \Gstar) \leq \frac{2}{5\gamma_1} \kappa(\Fstar)^{-1} \kappa(\Sigmax)^{-1}.
    \end{align*}
    Then, with probability at least $1 - \delta$, we have the following bound on the orthonormalization factor:
    \begin{align*}
        \smin(\bR) &\geq \sqrt{1 - 2\eta_\sbG(\gamma_1 + 2\gamma_2 + \eta_\sbG(\gamma_1 + \gamma_2)\gamma_2)}.
    \end{align*}
\end{proposition}
The constants $\gamma_1, \gamma_2$ will be instantiated to control the deflation of the contraction factor $1-\eta_\sbG \implies 1-c\eta_\sbG$ due to the orthonormalization factor.

\subsection*{Completing the bound}

We are almost ready to complete the proof. By instantiating the noise bound \Cref{prop:KFAC_noise_bound} and the orthonormalization factor bound \Cref{prop:linrep_orth_bound}, we have:
\begin{align*}
    \opnorm{\bG_+ \Gperp} &= \opnorm{\bR^{-1}\paren{(1 - \eta_\sbG) \bG \Gperp + \eta_\sbG (\bF^\top \bF)^{-1}\bfF^\top \Eps^\top \bX (\bX^\top \bX)^{-1}}} \\
    &\leq \frac{1-\eta_\sbG}{\smin(\bR)} \opnorm{\bG \Gperp} + \eta_\sbG \opnorm{(\bF^\top \bF)^{-1}\bfF^\top \Eps^\top \bX (\bX^\top \bX)^{-1}} \\
    &\leq \frac{1-\eta_\sbG}{\sqrt{1 - 2\eta_\sbG(\gamma_1 + 2\gamma_2 + \eta_\sbG(\gamma_1 + \gamma_2)\gamma_2)}}\opnorm{\bG \Gperp} + \eta_\sbG \sigmaeps  \sqrt{\frac{\dx + k + \log(1/\delta)}{\smin(\Fstar)^2\lmin(\Sigmax) n}}.
\end{align*}
To understand the effective deflation of the convergence rate, we prove the following numerical helper lemma.
\begin{lemma}\label{lem: avoiding square root rate}
    Given $c, d \in (0,1)$ and $\varepsilon \in (0,1/2)$, if $\varepsilon \geq c$, then the following holds:
    \begin{align*}
        \frac{1 - d}{\sqrt{1 - c d}} < 1 - (1 - \varepsilon) d.
    \end{align*}
    Additionally, as long as $\varepsilon \leq 1-\frac{1-\sqrt{1-d}}{d}$, then $1 - (1-\varepsilon) d \leq \sqrt{1 - d}$.
\end{lemma}

\textit{Proof of \Cref{lem: avoiding square root rate}:} squaring both sides of the desired inequality and re-arranging some terms, we arrive at
\begin{align*}
    c  &\leq \frac{1}{d}\paren{1 - \frac{(1- d)^2}{(1 - (1-\varepsilon)d)^2}} \\
    &= \frac{1}{d}\paren{1 - \frac{1- d}{\underbrace{1 - (1-\varepsilon)d}_{< 1}}} \underbrace{\paren{1 + \frac{1- d}{1 - (1-\varepsilon)d}}}_{> 1}.
\end{align*}
To certify the above inequality, it suffices to lower-bound the RHS. Since $c, d \in (0,1)$, the last factor is at least $1$, such that we have
\begin{align*}
    \frac{1}{d}\paren{1 - \frac{1- d}{1 - (1-\varepsilon)d}} \paren{1 + \frac{1- d}{1 - (1-\varepsilon)d}} &>  \frac{1}{d}\paren{1 - \frac{1- d}{1 - (1-\varepsilon)d}} \\
    &= \frac{1}{d}\frac{(1 - \varepsilon) d}{1 - (1-\varepsilon)d} \\
    &> \varepsilon .
\end{align*}
Therefore, $c \leq \varepsilon $ is sufficient for certifying the desired inequality. The latter claim follows by squaring and rearranging terms to yield the quadratic inequality:
\begin{align*}
    (1-\varepsilon)^2  d - 2(1-\varepsilon) + 1 \leq 0,
\end{align*}
Setting $\lambda := 1-\varepsilon$, the solution interval is $\lambda \in \paren{\frac{1-\sqrt{1- d}}{ d}, \frac{1 + \sqrt{1- d}}{ d}}$. The upper limit is redundant as it exceeds 1 and $\varepsilon \in (0,1)$, leaving the lower limit as the condition on $\varepsilon$ proposed in the lemma.

Plugging in $\eta_\sbG = d \in (0,1]$ and $2(\gamma_1 + 2\gamma_2 + \eta_\sbG(\gamma_1 + \gamma_2)\gamma_2) \leq (\gamma_1 + 2\gamma_2 + (\gamma_1 + \gamma_2)\gamma_2 = c$, we try candidate values $\gamma_1 =1/40$, $\gamma_2 = 1/100$ and set $\varepsilon=c$ to get:
\begin{align*}
    \frac{1-\eta_\sbG}{\sqrt{1 - 2\eta_\sbG(\gamma_1 + 2\gamma_2 + \eta_\sbG(\gamma_1 + \gamma_2)\gamma_2)}} &< (1 - 0.9\eta_\sbG).
\end{align*}
Plugging in our candidate values of $\gamma_1, \gamma_2$ into the burn-in conditions of \Cref{prop:linrep_orth_bound} finishes the proof of \Cref{thm:linrep_kfac_guarantee}.


$\qedhere$


\LinRepMainThm*





\subsection{Multi-Task and Transfer Learning}\label{sec:extension_multi_task}

We first discuss how the ideas in our ``single-task'' setting directly translate to multi-task learning. For example, taking our proposed algorithm template in \eqref{eq:linrep_update}, an immediate idea is, given the current task heads and shared representation $(\scurly{\Ft}, \bG)$, to form task-specific preconditioners formed locally on each task's batch data:
\begin{align*}
    \bP_\sbF^{(t)} &= \hatEx^{(t)}[\bfz \bfz^\top],\; \bP_\sbG^{(t)} = \Ft^\top \Ft, \; \bQ_\sbG^{(t)} = \hatEx^{(t)}[\bfx \bfx^\top],
\end{align*}
and perform a local update on $\Ft, \bG$ before a central agent averages the resulting updated $\bG$:
\begin{align*}
    \Ft[+] &= \Ft - \eta_\sbF \nabla_\sbF \hatL(\Ft, \Gt)\, {\bQ^{(t)}_{\sbF}}^{-1}\\
    \Gt[+] &= \bG - \eta_{\sbG} {\bP^{(t)}_\sbG}^{-1}\, \nabla_{\sbG} \hatL^{(t)}(\Ft[+], \bG)\, {\bQ^{(t)}_\sbG}^{-1}, \quad t \in [T] \\
    \bG_+ &= \frac{1}{T}\sumT \Gt[+].
\end{align*}
However, this presumes $\Ft$ are invertible, i.e.\ the task-specific dimension $\dy > k$. As opposed to the single-task setting, where as stated in \Cref{rem:multitask} we are really viewing $\bF$ as the concatentation of of $\Ft$ to make recovering the representation a well-posed problem, in multi-task settings $\dy$ may often be small, e.g.\ $\dy = 1$ \cite{tripuraneni2021provable, du2020few, collins2021exploiting, thekumparampil2021sample}. Therefore, (pseudo)-inverting away $(\Ft^\top \Ft)^{\dagger}$ may be highly suboptimal. However, we observe that writing out the representation gradient \eqref{eq:linrep_grad}, \emph{as long as we invert away ${\bQ^{(t)}_\sbG}^{-1}$ first}, then we have:
\begin{align*}
    \Gt[+] &= \bG - \eta_{\sbG} \nabla_{\sbG} \hatL^{(t)}(\Ft[+], \bG){\bQ^{(t)}_\sbG}^{-1} \\
    \bG_+ \Gperp &= \frac{1}{T}\sumT \Gt[+] \Gperp \\
    &= \paren{\bI_k - \eta_\sbG \frac{1}{T}\sumT \Ft^\top \Ft} \bG \Gperp + \text{ (task-averaged) noise term}.
\end{align*}
Since by assumption $\frac{1}{T}\sumT \Ft^\top \Ft$ is full-rank (otherwise recovering the rank $k$ representation is impossible), then suggestively, we may instead invert away the \textit{task-averaged} preconditioner $\bP_{\sbG} = \frac{1}{T}\sumT \Ft^\top \Ft$ on the task-averaged $\Gt$ descent direction before taking a representation step $\bG_+$. To summarize, we propose the following two-stage preconditioning:
\begin{align}\label{eq:multitask_KFAC}
    \Ft[+] &= \Ft - \eta_\sbF \nabla_\sbF \hatL(\Ft, \Gt)\, {\bQ^{(t)}_{\sbF}}^{-1} \\
    \Dt &= \nabla_{\sbG} \hatL^{(t)}(\Ft[+], \bG){\bQ^{(t)}_\sbG}^{-1}, \quad t \in [T] \\
    \bG_+ &= \bG - \eta_\sbG \bP_\sbG^{-1} \paren{\frac{1}{T}\sumT \Dt} \\
    \text{such that }\bG_+ \Gperp &= \paren{1 - \eta_\sbG} \bG\Gperp + \text{ (task-averaged) noise term}.
\end{align}
The exact same tools used in the proof of \Cref{thm:linrep_kfac_guarantee} apply here, with the requirement of a few additional standard tools to study the ``task-averaged'' noise term(s). As an example, we refer to \citet{zhang2023meta} for some candidates. However, we note the qualitative behavior is unchanged. As such, since we are using $n$ data points per each of $T$ tasks to update the gradient, the scaling of the noise term goes from $\calO(\sigmaeps \sqrt{\dx/n})$ in our bounds to $\calO(\sigmaeps \sqrt{\dx/nT})$. 

We remark that in the multi-task setting, where each task may have differing covariances and task-heads $\Ftstar$, the equivalence of our stylized \KFAC variant and the alternating min-min algorithm proposed in \citet{jain2013low, thekumparampil2021sample} breaks down. In particular, the alternating min-min algorithm no longer in general admits $\bG$ iterates that can be expressed as a product of matrices as in \eqref{eq:KFAC_update} or \eqref{eq:multitask_KFAC}, and rather can only be stated in vectorized space $\VEC(\bG)$. This means that whereas \eqref{eq:multitask_KFAC} can be solved as $T$ parallel small matrix multiplication problems, the alternating min-min algorithm nominally requires operating in the vectorized-space $\dx k$.


\subsection*{Transfer Learning}

We first prove the proposed fine-tuning generalization bound.

\LinRepTransfer*

\begin{proof}[Proof of \Cref{lem:linrep_transfer}]
    We observe that we may write:
    \begin{align*}
        \Ltest(\Flstest, \widehat \bG) &= \Ex\brac{\norm{\ytest - \Flstest\hatG\xtest}_2^2} \\
        &= \Ex\brac{\norm{(\Flstest \hatG - \Fstartest \Gstar)\xtest}_2^2} \\
        &= \bnorm{(\Flstest \hatG - \Fstartest \Gstar) \Sigmatest^{1/2}}_F^2.
    \end{align*}
    Now writing out the definition of $\Flstest$, defining $\ztest = \hatG \xtest$, we have
    \begin{align*}
        \Flstest &= \argmin_{\widehat \bF} \; \hatExtest[\norm{\ytest - \widehat \bF \hatG \xtest}_2^2] \\
        &= \Ytest^\top \Ztest (\Ztest^\top \Ztest)^{-1} \\
        &= \Fstartest \Gstar \Xtest^\top \Ztest(\Ztest^\top \Ztest)^{-1} + \Eps^\top \Ztest (\Ztest^\top \Ztest)^{-1} \\
        &= \Fstartest \Gstar \hatG^\top + \Fstartest \Gstar \hatGperp \Xtest^\top \Ztest(\Ztest^\top \Ztest)^{-1} + \Eps^\top \Ztest (\Ztest^\top \Ztest)^{-1},
    \end{align*}
    where $\hatGperp = \bI_{\dx} - \hatG^\top \hatG$ is the projection matrix onto the rowspace of $\hatG$, using the fact that $\hatG$ is row-orthonormal \eqref{eq:linrep_update}.
    Therefore, plugging in the last line into error expression, we have
    \begin{align*}
        \Ltest(\Flstest, \widehat \bG) &= \bnorm{(\Flstest \hatG - \Fstartest \Gstar) \Sigmatest^{1/2}}_F^2 \\
        &\leq 2\bnorm{\paren{\Fstartest \Gstar \hatG^\top \hatG + \Fstartest \Gstar \hatGperp \Xtest^\top \Ztest(\Ztest^\top \Ztest)^{-1}\hatG - \Fstartest \Gstar} \Sigmatest^{1/2}}_F^2 \\
        &\qquad \quad + 2\bnorm{\Eps^\top \Ztest (\Ztest^\top \Ztest)^{-1} \hatG \Sigmatest^{1/2}}_F^2 \tag{$(a+b)^2 \leq 2a^2 + 2b^2$}.
    \end{align*}
    Focusing on the first term, we have:
    \begin{align*}
        &\Fstartest \Gstar \hatG^\top \hatG + \Fstartest \Gstar \hatGperp \Xtest^\top \Ztest(\Ztest^\top \Ztest)^{-1}\hatG - \Fstartest \Gstar  \\
        =\; &\Fstartest \Gstar \hatGperp  \paren{\Xtest^\top \Ztest(\Ztest^\top \Ztest)^{-1}\hatG - \bI_{\dx}}.
    \end{align*}
    By a covariance concentration argument \Cref{lem:gauss_cov_conc}, since $\Xtest^\top \Ztest$ and $\Ztest^\top \Ztest$ are rank-$k$ matrices, as long as $\ntest \gtrsim k + \log(1/\delta)$, we have with probability at least $1 - \delta$:
    \begin{align*}
        \Xtest^\top \Ztest \approx \ntest \Sigmatest \hatG^\top, \quad (\Ztest^\top \Ztest)^{-1} \approx \ntest \hatG \Sigmatest \hatG^\top,
    \end{align*}
    and thus
    \begin{align*}
        &\norm{\Fstartest \Gstar \hatGperp  \paren{\Xtest^\top \Ztest(\Ztest^\top \Ztest)^{-1}\hatG - \bI_{\dx}} \Sigmatest^{1/2}}_F \\
        \approx\;& \norm{\Fstartest \Gstar \hatGperp  \Sigmatest^{1/2} \paren{ \Sigmatest^{1/2} \hatG^\top (\hatG \Sigmatest \hatG^\top)^{-1}\hatG \Sigmatest^{1/2} - \bI_{\dx} }}_F\\
        \lesssim\;& \norm{\Fstartest}_F \opnorm{\Gstar \hatGperp} \opnorm{\Sigmatest^{1/2}} \opnorm{ \Sigmatest^{1/2} \hatG^\top (\hatG \Sigmatest \hatG^\top)^{-1}\hatG \Sigmatest^{1/2} - \bI_{\dx} }  \\
        \leq\;&\norm{\Fstartest}_F\; \dist(\hatG, \Gstar) \lmax(\Sigmatest)^{1/2},
    \end{align*}
    where in the last line we applied the definition $\dist(\hatG, \Gstar) = \opnorm{\Gstar \hatGperp} = \opnorm{\hatG \Gperp}$, and the fact that the matrix $\calP \triangleq \Sigmatest^{1/2} \hatG^\top (\hatG \Sigmatest \hatG^\top)^{-1}\hatG \Sigmatest^{1/2}$ can be verified to be a projection matrix $\calP^2 = \calP$, $\calP^\top = \calP$, such that $\calP - \bI = \calP^\perp$ is also an orthogonal projection and $\opnorm{\calP^\perp} = 1$. Now, we analyze the noise term:
    \begin{align*}
        \Eps^\top \Ztest (\Ztest^\top \Ztest)^{-1} \hatG \Sigmatest^{1/2} \approx \Eps^\top \Ztest (\Ztest^\top \Ztest)^{-1/2} (\ntest)^{-1/2} \paren{ \hatG \Sigmatest \hatG^\top}^{-1/2} \hatG \Sigmatest^{1/2},
    \end{align*}
    where we observed $\Ztest^\top \Ztest = n \bG \hatSigma \bG^\top$ and applied covariance concentration.
    Now, defining the (compact) SVD of $\hatG \Sigmatest^{1/2} = \bU_\bfz \bD_\bfz \bV_\bfz^\top$, we find
    \begin{align*}
        \norm{\Eps^\top \Ztest (\Ztest^\top \Ztest)^{-1} \hatG \Sigmatest^{1/2}}_F &\lesssim \frac{1}{\sqrt{\ntest}} \norm{\Eps^\top \Ztest (\Ztest^\top \Ztest)^{-1/2} \bU_\bfz \bV_\bfz^\top}_F \\
        &\lesssim \frac{1}{\sqrt{\ntest}} \norm{\Eps^\top \Ztest (\Ztest^\top \Ztest)^{-1/2} }_F \\
        &\lesssim \frac{1}{\sqrt{\ntest}} \sigmaeps \sqrt{\dy k + \log(1/\delta)},
    \end{align*}
    for $\ntest \gtrsim k + \log(1/\delta)$. The last line comes from the Frobenius norm variants of \Cref{lem:yasin_SNM} and \Cref{lem:SNM_bound} (see \citet[Theorem 4.1]{ziemann2023tutorial} or \citet[Lemma A.3]{zhang2023meta} for details). Putting the two bounds together yields the desired result.
\end{proof}


\section{Proofs and Additional Details for \Cref{sec:single_index}}

\subsection{Proof of Theorem~\ref{thm:rank1}}
\label{pfthm:rank1}
\OneStepSGD*
\begin{proof}
    To prove this theorem, we first note that
    \begin{align*}
    \nabla_\bfG \hatL\,(\bff_0,\bfG_0) = -\frac{1}{n} \sum_{i= 1}^{n}\left(y_i - \frac{1}{\sqrt{\dhid}}\bff_{0}^\top \sigma\left(\bfG_{0} \bfx_i\right)\right)\left(\frac{1}{\sqrt{\dhid}}\bff_{0} \odot \sigma'\left(\bfG_{0}\bfx_i\right)\right)\bfx_i^\top.
    \end{align*}
    Adopting the matrix notation
    \ $\bfX = [\bfx_1 | \dots | \bfx_n]^\top \in \R^{n \times \dx}$ and $\vy = [y_1, \dots, y_n]^\top \in \R^n$, we can write
    \begin{align}
        \label{eq:gradient}
        \nabla_\bfG \hatL\,(\bff_0,\bfG_0) = -\frac{1}{n}\left[\left(\dhid^{-1/2}\bff_0\vy^\top - \dhid^{-1}\bff_0\bff_0^\top \sigma(\bfG_{0} \bX^\top)\right)\odot \sigma'(\bfG_{0}\bX^\top)\right]\bX.
    \end{align}
    Let $ {z \sim \normal(0, \dx^{-1}\trace(\bSigma_\bfx))}$ and define $\alpha = \Ex_z \left[\sigma'(z)\right]$, and $\sigma_\perp: \R \to \R$ as $\sigma_\perp(z) =  \sigma(z) - \alpha z $. This function satisfies $\Ex_z \left[\sigma_\perp'(z)\right] = 0$. With this, we decompose the gradient into three components as $\nabla_\bfG \hatL\,(\bff_0,\bfG_0) = \bT_1 + \bT_2 + \bT_3$ with
    \begin{align*}
        &\bT_1 =  - \alpha\, \dhid^{-1/2} \; \bff_0 \left(\frac{\bX^\top\vy}{n}\right)^\top,\quad \bT_2 = -n^{-1} \dhid^{-1/2} \left[\bff_0 \,\vy^\top \odot \sigma_\perp'(\bfG_0 \bX^\top)\right]\bX,\\[0.2cm]
        &\bT_3 = n^{-1} \dhid^{-1} \,\Big[\left(\bff_0 \bff_0^\top \sigma(\bfG_0 \bX^\top)\right) \odot \sigma'(\bfG_0 \bX^\top)\Big]\bX.
    \end{align*}
    We will analyze each of these components separately.
    
    \begin{itemize}
        \item \textbf{Term 1:} For this term, using the facts that $\|\bff_0\|_2 = \calO(1)$ and $\|\bX^\top \vy/n\|_2 = \calO(1)$, we have
        \begin{align*}
            \left\|\bT_1\right\|_{\rm op} = \calO\left({ \dhid^{-1/2}} \right).
        \end{align*}
        \item \textbf{Term 2:} To analyze this term, note that
        \begin{align*}
            \bff_0\, \vy^\top \odot \sigma_\perp'(\bfG_0 \bX^\top) = \diag(\bff_0)\,  \sigma_\perp'(\bfG_0 \bX^\top)\,\diag(\vy),
        \end{align*}
        which gives
        \begin{align*}
            \left\|\bff_0\, \vy^\top \odot \sigma_\perp'(\bfG_0 \bX^\top)\right\|_{\rm op} &\leq \|\bff_0\|_{\rm \infty}\|\vy\|_{\rm \infty} \|\sigma_\perp'(\bfG_0 \bX^\top)\|_{\rm op}. %
        \end{align*}
        Using basic concentration arguments, we have $\|\bff_0\|_{\rm \infty} = \tilde \calO(\dhid^{-1/2})$, and $\|\vy\|_{\rm \infty} = \tilde \calO(1)$, with probability $1 - o(1)$. By construction of $\sigma_\perp(\cdot)$, the matrix $\sigma_\perp'(\bfG_0 \bX^\top)$ has mean zero entries, thus using \citep[Theorem 5.44]{vershynin2010introduction}, we have $\|\sigma_\perp'(\bfG_0 \bX^\top)\|_{\rm op} = \tilde{O}(\dhid^{1/2} + n^{1/2})$ with probability $1 - o(1)$ Thus, the norm of $\bT_2$ can be upper bounded as
        \begin{align*}
            \|\bT_2\|_{\rm op} = \tilde \calO\left(\frac{1}{ \dhid}\left(1 + \sqrt\frac{\dhid}{n}\right)\left(1 + \sqrt\frac{\dx}{n}\right)\right) = \tilde{\calO}\left(\dhid^{-1}\right).
        \end{align*}
        
        \item \textbf{Term 3:} Similar to the second term, note that
        \begin{align*}
            \left(\bff_0\, \bff_0^\top \sigma(\bfG_0 \bX^\top)\right) \odot \sigma'(\bfG_0 \bX^\top) = \diag(\bff_0)\,  \sigma'(\bfG_0 \bX^\top) \, \diag\left(\bff_0^\top \sigma(\bfG_0 \bX^\top)\right).
        \end{align*}
        Thus, the norm of the third term can be upper bounded as
            \begin{align*}
                \|\bT_3\|_{\rm op}&=\left\|\frac{1}{n \, \dhid}  \Big[\left(\bff_0 \bff_0^\top \sigma(\bfG_0 \bX^\top)\right) \odot \sigma'(\bfG_0 \bX^\top)\Big]\bX\right\|_{\rm op}\\ &\leq n^{-1} \dhid^{-1}{\|\bX\|_{\rm op}\|\bff_0\|_{\rm \infty}\|\bff_0^\top \sigma(\bfG_0 \bX^\top)\|_{\rm \infty} \left\|\sigma'(\bfG_0 \bX^\top)\right\|_{\rm op}}.
            \end{align*}
            To analyze the right hand side, note that assuming that $\sigma$ is $\calO(1)$-Lipschitz, the entries of $\sigma'(\bfG_0 \bX^\top)$ are bounded by the Lipschitz constant, and we have $\|\sigma'(\bfG_0 \bX^\top)\|_{\rm op} = \calO(\sqrt{n \,\dhid})$. Also, using a simple orderwise analysis we have $\|\bff_0^\top \sigma(\bfG_0 \bX^\top)\|_{\rm \infty} = \tilde \calO(1)$, which gives
            \begin{align*}
                \|\bT_3\|_{\rm op} = \tilde \calO\left(\frac{1}{ \dhid}\left(1 + \sqrt\frac{\dx}{n}\right)\right) = \tilde{\calO}(\dhid^{-1}).
            \end{align*}
    \end{itemize}
    To wrap up, note that $\dhid^{1/2} \|\bT_1\|_{\rm op} = \calO(1)$, whereas $\dhid^{1/2} \|\bT_2\|_{\rm op}$ and $\dhid^{1/2} \|\bT_3\|_{\rm op} = o(1)$. Thus, with probability $1 - o(1)$ we have
    \begin{align*}
        \bG_{\texttt{SGD}} = \bG_0 + \eta\,\dhid^{1/2}\, \nabla_\bfG \hatL\,(\bff_0,\bfG_0) = \bG_0 + \alpha \eta\, \bff_0 \left(n^{-1} \bfX^\top \vy\right)^\top + \mathbf{\Delta}
    \end{align*}
    with $\|\mathbf{\Delta}\|_{\rm op} = o(1)$, finishing the proof.
\end{proof}
\subsection{Proof of Lemma~\ref{lemma:beta_tilde_alignment}}
\label{pflemma:beta_tilde_alignment}
\TildeAlignment*
\begin{proof}
    Recall that $\betaSGD = \frac{1}{n} \bX^\top \vy$ and $\vy = \sigma_\star(\bX\vbeta_\star) + \boldsymbol{\ep}$ where $\boldsymbol{\ep} = [\ep_1, \dots, \ep_n]^\top$. Therefore, with probability $1 - o(1)$ we have
    \begin{align*}
        \betaSGD^\top\vbeta_\star = \frac{1}{n} (\bX\vbeta_\star)^\top \vy = \frac{1}{n} (\bX\vbeta_\star)^\top \sigma_\star(\bX\vbeta_\star) + o(1)
    \end{align*}
    where we have used the fact that $\boldsymbol{\ep}$ is mean zero. Thus, using the weak law of large numbers, 
    \begin{align}
    \label{eq:inner}
        \betaSGD^\top\vbeta_\star \to \Ex_{z} \left[z\sigma_\star(z)\right] = \dx^{-1}\trace(\bSigma_\bfx) \,\Ex_{z} \left[\sigma_\star'(z)\right] = c_{\star,1} \dx^{-1}\trace(\bSigma_\bfx)
    \end{align}
    in probability, where $z \sim \normal(0, \dx^{-1}\trace(\bSigma_\bfx))$. Similarly, $\|\betaSGD\|_2^2$ can be written as
    \begin{align*}
        \|\betaSGD\|_2^2 &= n^{-2} \vy^\top \bX \bX^\top \vy = n^{-2} \boldsymbol{\ep}^\top\bX \bX^\top\boldsymbol{\ep}+ n^{-2} \sigma_\star(\bX\vbeta_\star)^\top\bX \bX^\top\sigma_\star(\bX\vbeta_\star) + o(1).
    \end{align*}
    We will analyze each of the two remaining term separately.  For the first term, recall that $\boldsymbol{\ep}$ is independent of $\bX$. Using the Hanson-Wright inequality (Theorem~\ref{thm:hanson-wright}) we have
    \begin{align*}
        n^{-2} \boldsymbol{\ep}^\top\bX \bX^\top\boldsymbol{\ep} =   \sigma_\ep^2 n^{-1} \trace(\bX\bX^\top/n) + o(1) = \sigma_\ep^2 n^{-1}  \trace(\bSigma_\bfx) + o(1).
    \end{align*}
    For the second term, note that $\bX\vbeta_\star$ is a vector with i.i.d. elements $\bfx_i^\top \vbeta_\star$, each of them distributed according to $\normal(0, \vbeta_\star^\top \bSigma_\bfx\vbeta_\star)$. Let $z$ be a random variable distributed as $z \sim \normal(0, \vbeta_\star^\top \bSigma_\bfx\vbeta_\star) $. We decompose the function $\sigma_\star$ into a linear and a nonlinear part as
    \begin{align}
        \label{eq:expansions}
        \sigma_\star(z) = c_{\star,1} z + \sigma_{\star, \perp}(z).
    \end{align}
    This decomposition satisfies
    \begin{align*}
        &\Ex_z [\sigma_{\star, \perp}(z)] = \Ex_z  [\sigma_\star(z)] = 0\\
        &\Ex_z\,[ z\, \sigma_{\star, \perp} (z) ] = \Ex_z\, z\,\sigma_\star(z) - c_{\star,1}\,\Ex_z \, [z^2] =  \Ex_z\, [z\,\sigma_\star(z)] - c_{\star,1}\, \vbeta_\star^\top \bSigma_\bfx \vbeta_\star = 0,
    \end{align*}
    where the last equality is due to Stein's lemma (Lemma~\ref{lemma:stein's}). This shows that the random variables $z$ and $\sigma_{\star, \perp}(z)$ are uncorrelated. With this, we have
    \begin{align}
        \label{eq:sum1}
        n^{-2} \sigma_\star(&\bX\vbeta_\star)^\top\bX \bX^\top\sigma_\star(\bX\vbeta_\star) = n^{-2} (c_{\star,1} \bX\vbeta_\star + \sigma_{\star,\perp}(\bX\vbeta_\star))^\top\bX \bX^\top(c_{\star,1} \bX\vbeta_\star + \sigma_{\star,\perp}(\bX\vbeta_\star))\nonumber\\[0.2cm]
        &=  c_{\star,1}^2 n^{-2} \vbeta_\star^\top (\bX^\top \bX)^2 \vbeta_\star + 2  c_{\star,1} n^{-2} (\bX\vbeta_\star)^\top \bX\bX^\top \sigma_{\star,\perp}(\bX\vbeta_\star) +  n^{-2} \sigma_{\star,\perp}(\bX\vbeta_\star)^\top \bX\bX^\top \sigma_{\star,\perp}(\bX\vbeta_\star).
    \end{align}
    For the first term in this sum, by assumption~\ref{assumption:random-effect} and the Hanson-Wright inequality (Theorem~\ref{thm:hanson-wright}), we can write 
    \begin{align*}
        c_{\star,1}^2  n^{-2} \vbeta_\star^\top (\bX^\top \bX)^2 \vbeta_\star = c_{\star,1}^2\,\dx^{-1} \trace(n^{-2}(\bX^\top\bX)^2) + o(1) = c_{\star,1}^2 \dx^{-1} \trace(\bSigma_\bfx^2) + c_{\star,1}^2n^{-1} \dx^{-1} \trace^2(\bSigma_\bfx),
    \end{align*}
    where in the last we plugged in the second Wishart moment. For the second term in  \eqref{eq:sum1}, although by construction $\bX\vbeta_\star$ and $\sigma_{\star, \perp}(\bX\vbeta_\star)$ are uncorrelated, the vector $\bX\vbeta_\star$ and the matrix $\bX\bX^\top$ are dependent, which complicates the analysis. To resolve this issue, we define $\tilde \bX = \bX - \bX \vbeta_\star\vbeta_\star^\top$ which satisfies $\bX\vbeta_\star \independent \tilde\bX$ and write
    \begin{align*}
        \bX\bX^\top =  \tilde\bX\tilde\bX^\top + \tilde\bX\vbeta_\star\vbeta_\star^\top \bX^\top + \bX\vbeta_\star\vbeta_\star^\top\tilde\bX^\top + \bX\vbeta_\star\vbeta_\star^\top\bX^\top.
    \end{align*}
    Thus, the second term in \eqref{eq:sum1} can be written as
    \begin{align*}
        n^{-2} (\bX\vbeta_\star)^\top &\bX\bX^\top \sigma_{\star,\perp}(\bX\vbeta_\star) = n^{-2} (\bX\vbeta_\star)^\top \left[\tilde\bX\tilde\bX^\top + \tilde\bX\vbeta_\star\vbeta_\star^\top \bX^\top + \bX\vbeta_\star\vbeta_\star^\top\tilde\bX^\top + \bX\vbeta_\star\vbeta_\star^\top\bX^\top\right] \sigma_{\star,\perp}(\bX\vbeta_\star)\\
        &=\underbrace{n^{-2} (\bX\vbeta_\star)^\top \tilde\bX\tilde\bX^\top \sigma_{\star,\perp}(\bX\vbeta_\star)}_{T_1}\\ &\hspace{2cm}+  \underbrace{n^{-2} (\bX\vbeta_\star)^\top \left[ \tilde\bX\vbeta_\star\vbeta_\star^\top \bX^\top + \bX\vbeta_\star\vbeta_\star^\top\tilde\bX^\top + \bX\vbeta_\star\vbeta_\star^\top\bX^\top\right] \sigma_{\star,\perp}(\bX\vbeta_\star)}_{T_2}.
    \end{align*}
    The term $T_1$ can be shown to be $o(1)$ by using the Hanson-Wright inequality (Theorem~\ref{thm:hanson-wright}) and noting that $\tilde\bX\tilde\bX^\top$ is independent of $\bX\vbeta_\star$, and also the fact that  $\bX\vbeta_\star$ and $\sigma_{\star,\perp}(\bX\vbeta_\star)$ are orthogonal, by construction. Similarly, $T_2$ can also be shown to be $o(1)$ using a similar argument. For this, we also use that fact that by construction we have $\tilde\bX\vbeta_\star \independent \bX\vbeta_\star$ which alongside $\Ex \; [\tilde\bX\vbeta_\star] = \mathbf{0}_n$ proves that $n^{-1} (\bX\vbeta_\star)^\top \tilde\bX\vbeta_\star = o(1)$ and $n^{-1} (\tilde\bX\vbeta_\star)^\top \sigma_{\star,\perp}(\tilde\bX\vbeta_\star) = o(1)$. Hence, 
    \begin{align*}
        2  c_{\star,1}n^{-2} (\bX\vbeta_\star)^\top \bX\bX^\top \sigma_{\star,\perp}(\bX\vbeta_\star) \to 0.
    \end{align*}
    For the third term in \eqref{eq:sum1}, we can use a similar argument and replace $\bX$ with $\tilde\bX$ to show that
    \begin{align*}
        n^{-2} \sigma_{\star,\perp}(\bX\vbeta_\star)^\top \bX\bX^\top \sigma_{\star,\perp}(\bX\vbeta_\star) \to \Ex_z [\sigma_{\star, \perp}(z)]^2 \, n^{-1} \trace(\bSigma_\bfx).
    \end{align*}
    Putting everything together, we have
    \begin{align*}
    \|\betaSGD\|_2^2 &= c_{\star,1}^2 \dx^{-1} \trace(\bSigma_\bfx^2) +  \sigma_\ep^2 \,n^{-1}\, \trace(\bSigma_\bfx) + \Ex_z [\sigma_{\star, \perp}(z)]^2 \, n^{-1} \trace(\bSigma_\bfx) + c_{\star,1}^2n^{-1} \dx^{-1} \trace^2(\bSigma_\bfx)+ o(1)\\[0.2cm]
    &= c_{\star,1}^2 \dx^{-1} \trace(\bSigma_\bfx^2) +  n^{-1} \trace(\bSigma_\bfx) \left(\sigma_\ep^2 +  \Ex_z [\sigma_{\star, \perp}(z)]^2+c_{\star,1}^2 \dx^{-1}\trace(\bSigma_{\bfx})\right) + o(1).
    \end{align*}
    Note that given the decomposition \eqref{eq:expansions}, we have
    \begin{align*}
        \Ex_z \left[\sigma_\star^2(z)\right] = \Ex_z \left[\sigma_{\star,\perp}^2(z)\right] + c_{\star,1}^2 \dx^{-1} \trace(\bSigma_{\bfx})
    \end{align*}
    given the orthogonality of the linear and nonlinear terms. Hence,
    \begin{align*}
        \|\betaSGD\|_2^2 &= c_{\star,1}^2 \dx^{-1} \trace(\bSigma_\bfx^2) +  n^{-1} \trace(\bSigma_\bfx) \left(\sigma_\ep^2 +  c_{\star}^2\right) + o(1),
    \end{align*}
    which alongside \eqref{eq:inner} proves the lemma.
\end{proof}



\subsection{Proof of Lemma \ref{lemma:beta_hat_alignment}}
\label{pf_lemma:beta_hat_alignment}
\HatAlignment*
\begin{proof}
    
    Recall that $\bQ_\sbG = n^{-1} \bX^\top \bX$ and let $\bR = (\bQ_\sbG + \lambda_\sbG \bI_{\dx})^{-1}$. The inner product of $\vbeta_\star$ and $\betaKFAC$ is given by
    \begin{align*}
        \vbeta_\star^\top\betaKFAC = n^{-1} \vbeta_\star^\top\bR\bX^\top \sigma_\star(\bX\vbeta_\star) + o(1),
    \end{align*}
    where we have used the fact that $\boldsymbol{\ep}$ is mean zero and is independent of all other randomness in the problem. Defining $\bar\bR =\left( \bX \bX^\top/n + \lambda_\sbG \bI_n\right)^{-1}$, we can use the push-through identity to rewrite the inner product as
    \begin{align*}
        \vbeta_\star^\top\betaKFAC = n^{-1}(\bX\vbeta_\star)^\top\bar\bR\,\sigma_\star(\bX\vbeta_\star) + o(1).
    \end{align*}
    Note that $\bX\vbeta_\star$ is a vector with i.i.d. elements $\bfx_i^\top \vbeta_\star$, each of them distributed according to $\normal(0, \vbeta_\star^\top \bSigma_\bfx\vbeta_\star)$.  Using the same decomposition for $\sigma_\star$ as the one used in the proof of Lemma~\ref{lemma:beta_tilde_alignment} in \eqref{eq:expansions}, we have
    \begin{align}
    \label{eq:inner_hat}
    \vbeta_\star^\top\betaKFAC &= \frac{1}{n} (\bX\vbeta_\star)^\top \bar\bR\left(c_{\star,1} \bX \vbeta_\star + \sigma_{\star, \perp}(\bX\vbeta_\star)\right) + o(1)\nonumber\\[0.2cm]
    &= c_{\star,1} \dx^{-1} \trace\left(\bX^\top (\bX\bX^\top + \lambda_\sbG n \bI_n)^{-1}\bX\right) + n^{-1} (\bX\vbeta_\star)^\top \bar\bR\, \sigma_{\star, \perp}(\bX\vbeta_\star)+ o(1),
    \end{align}
    where for the first term we have used Assumption~\ref{assumption:random-effect} and the Hanson-Wright inequality (Theorem~\ref{thm:hanson-wright}). To analyze the second term, note that although by construction $\bX\vbeta_\star$ and $\sigma_{\star, \perp}(\bX\vbeta_\star)$ are uncorrelated, the vectors $\bX\vbeta_\star$ and $\bar\bR$ are dependent, which complicates the analysis. To resolve this issue, we use the same trick used in the proof of Lemma~\ref{lemma:beta_tilde_alignment} and set $\tilde \bX = \bX - \bX \vbeta_\star\vbeta_\star^\top$ which satisfies $\bX\vbeta_\star \independent \tilde\bX$, and use it to write
    \begin{align*}
        \bar\bR^{-1} &= n^{-1}  \bX \bX^\top + \lambda_\sbG \bI_n = n^{-1} \left(\tilde\bX + \bX \vbeta_\star \vbeta_\star^\top\right)\left(\tilde\bX + \bX \vbeta_\star \vbeta_\star^\top\right)^\top + \lambda_\sbG \bI_n\\[0.2cm]
        &= \left(n^{-1} \tilde\bX \tilde\bX^\top + \lambda_\sbG \bI_n\right) + n^{-1} (\bX\vbeta_\star)(\bX\vbeta_\star)^\top + n^{-1}(\tilde \bX\vbeta_\star)(\bX\vbeta_\star)^\top + n^{-1}( \bX\vbeta_\star)(\tilde \bX\vbeta_\star)^\top.
    \end{align*}
    Defining $\tilde\bR =\left( \tilde\bX \tilde\bX^\top/n + \lambda_\sbG \bI_n\right)^{-1} \in \R^{n \times n}$,  $\bV = [\,n^{-1/2}\bX\vbeta_\star\, |\, n^{-1/2} \tilde\bX \vbeta_\star] \in \R^{n \times 2}$, and 
    \begin{align*}
        \bD = \begin{bmatrix}
            1 & 1 \\[0.2cm]
            1 & 0
        \end{bmatrix},
    \end{align*}
    we have $\bar\bR = \tilde\bR + \bV \bD \bV^\top$. Using Woodbury matrix identity (Theorem~\ref{thm:woodbury}), $\bar\bR$ is given by
    \begin{align}
        \label{eq:woodbury}
        \bar\bR = \tilde\bR - \tilde\bR\,  \bV (\bD^{-1} + \bV^\top \tilde\bR\bV)^{-1}  \bV^\top\,\tilde\bR
    \end{align}
    and plugging this expression into the second term in \eqref{eq:inner_hat} gives
    \begin{align*}
         n^{-1} (\bX\vbeta_\star)^\top &\bar\bR\, \sigma_{\star, \perp}(\bX\vbeta_\star) = n^{-1} (\bX\vbeta_\star)^\top\left(\tilde\bR - \tilde\bR\,  \bV (\bD^{-1} + \bV^\top \tilde\bR\bV)^{-1}  \bV^\top\,\tilde\bR\right)\,\sigma_{\star, \perp}(\bX\vbeta_{\star}) + o(1)\\[0.2cm]
        &= n^{-1}(\bX\vbeta_\star)^\top \tilde\bR\; \sigma_{\star, \perp}(\bX\vbeta_{\star})  - n^{-1} (\bX\vbeta_\star)^\top\tilde\bR\left(\bV (\bD^{-1} + \bV^\top \tilde\bR\bV)^{-1}  \bV^\top\right)\,\tilde\bR\,\sigma_{\star, \perp}(\bX\vbeta_{\star}) + o(1).
    \end{align*}
    The first term can be shown to be $o(1)$ in probability by using  the fact that $\tilde\bR$ is independent of $\bX\vbeta_\star$ and the orthogonality of $\bX\vbeta_\star$ and $\sigma_{\star, \perp}(\bX\vbeta_\star)$. To analyze the second term, first note that the elements of the matrix $\boldsymbol{\Omega} = (\bD^{-1} + \bV^\top \tilde\bR\bV)^{-1} \in \R^{2 \times 2}$ can all be shown to be $\calO(1)$ by a simple norm argument. Moreover,
    \begin{align*}
       n^{-1} &(\bX\vbeta_\star)^\top\tilde\bR \bV (\bD^{-1} + \bV^\top \tilde\bR\bV)^{-1}  \bV^\top\,\tilde\bR\,\sigma_{\star, \perp}(\bX\vbeta_{\star}) \\[0.2cm]
       &= n^{-2} (\bX\vbeta_\star)^\top\tilde\bR \Big(\Omega_{11}(\bX\vbeta_\star)(\bX\vbeta_\star)^\top + \Omega_{12}(\bX\vbeta_\star)(\tilde\bX\vbeta_\star)^\top\\ &\hspace{5cm}+ \Omega_{21}(\tilde\bX\vbeta_\star)(\bX\vbeta_\star)^\top + \Omega_{22}(\tilde\bX\vbeta_\star)(\tilde\bX\vbeta_\star)^\top\Big)\tilde\bR\,\sigma_{\star, \perp}(\bX\vbeta_{\star})
    \end{align*}
    where $\Omega_{ij}$ are the elements of the matrix $\mathbf{\Omega}$. We  analyze each term in this sum separately and show that all of them are $o(1)$.
    \begin{itemize}
        \item \textbf{First Term.} Using a simple norm argument, $n^{-1} (\bX\vbeta_\star) \tilde\bR(\bX\vbeta_\star) = \calO(1)$. Also, by construction of $\sigma_{\star, \perp}$, we have
        \begin{align*}
            n^{-1}(\bX\vbeta_\star)^\top \tilde\bR\sigma_{\star, \perp}(\bX\vbeta_\star) \to 0.
        \end{align*}
        Thus, the whole term is $o(1)$.
        \item\textbf{Second Term.} Similar to the first term, we have $n^{-1} (\bX\vbeta_\star) \tilde\bR(\bX\vbeta_\star) = \calO(1)$. Also, $n^{-1}(\tilde\bX\vbeta_\star)^\top \tilde\bR\,\sigma_{\star, \perp}(\bX\vbeta_\star) \to 0$ in probability, using the weak law of large numbers by noting that $\sigma_{\star, \perp}(\bX\vbeta_\star)$ is independent of $n^{-1}(\tilde\bX\vbeta_\star)^\top \tilde\bR$ by construction and that it has mean zero. Hence, the whole term is $o(1)$.
        \item\textbf{Third Term.} By construction, the vector $\bX\vbeta_\star$ is independent of $\tilde\bR (\tilde\bX\vbeta_\star)$ and has mean zero, which gives $n^{-1}(\bX\vbeta_\star)^\top\tilde\bR(\tilde\bX\vbeta_\star) \to 0$. Also, using a simple norm argument, we have $n^{-1}(\bX\vbeta_\star)^\top \tilde\bR\sigma_{\star, \perp}(\bX\vbeta_\star) = \calO(1)$ which proves that the third term is also $o(1)$.
        \item \textbf{Fourth Term.} This term can be shown to be $o(1)$ with an argument very similar to the argument for the third term.
    \end{itemize}
    
    Putting these all together and using \eqref{eq:inner_hat}, we have
    \begin{align}
        \label{eq:inner_hat_final}
        \vbeta_\star^\top \betaKFAC = c_{\star,1} \dx^{-1} \trace\left((\bX^\top\bX/n)\,\bR\right) + o(1).
    \end{align}
    
    Next we move to the analysis of the squared $\ell_2$-norm of the vector $\betaKFAC$. By decomposing the function $\sigma_\star$ into a linear and an orthogonal nonlinear component similar to the one used for the analysis of the inner product term above, we write
    \begin{align*}
        \|\betaKFAC\|_2^2 &= n^{-2}\, \vy^\top \bX \bR^2 \bX^\top \vy
           = n^{-2}\boldsymbol{\ep}^\top \bX \bR^2 \bX^\top\boldsymbol{\ep}+ c_{\star,1}^2 n^{-2} \vbeta_\star^\top \bX^\top  \bX \bR^2 \bX^\top\bX\vbeta_\star\\[0.2cm] &\hspace{2cm} +n^{-2} \sigma_{\star, \perp}(\bX\vbeta_\star)^\top \bX \bR^2 \bX^\top\sigma_{\star, \perp}(\bX\vbeta_\star) + 2\, c_{\star, 1} \, n^{-2} \vbeta_\star^\top \bX^\top  \bX \bR^2 \bX^\top\sigma_{\star,\perp}(\bX\vbeta_\star).
    \end{align*}
    We will analyze each of these terms separately.
    \begin{itemize}
        \item \textbf{First Term.} Recalling that $\boldsymbol{\ep} \sim \normal(0, \sigma_\ep^2 \bI_n)$ independent of all randomness in the problem, using the Hanson-Wright inequality (Theorem~\ref{thm:hanson-wright}) we have
        \begin{align*}
            n^{-2}\boldsymbol{\ep}^\top \bX \bR^2 \bX^\top\boldsymbol{\ep} = \sigma_\ep^2 n^{-1} \trace( (\bX^\top\bX/n) \bR^2 ) + o(1).
        \end{align*}
    
        \item \textbf{Second Term.} Using Assumption~\ref{assumption:random-effect}, and by the Hanson-Wright inequality we have
        \begin{align*}
             n^{-2}c_{\star,1}^2\, \vbeta_\star^\top \bX^\top \bX\bR^2\bX^\top\bX\vbeta_\star =  c_{\star, 1}^2  \dx^{-1}   \trace \left[(\bX^\top\bX/n)^2\bR^2\right] + o(1)
        \end{align*}
        
        \item \textbf{Third Term.} Note that $\bar\bR$ and $ \bX\vbeta_\star$ are dependent. Note that $\bX\bR^2\bX^\top = \bX\bX^\top \bar\bR = n\bar\bR - \lambda_\sbG n \bar\bR^2$. Thus, an almost identical argument to the argument used above for the analysis of $\vbeta_\star^\top \betaKFAC$ using $\tilde\bX = \bX - \bX\vbeta_\star\vbeta_\star^\top$ gives
        \begin{align*}
        n^{-2} \sigma_{\star, \perp}(\bX\vbeta_\star)^\top \bX \bR^2 \bX^\top\sigma_{\star, \perp}(\bX\vbeta_\star)  
        &= \Ex_z[\sigma^2_{\star,\perp}(z)] \cdot n^{-1} \trace\left((\bX^\top\bX/n)\,\bR^2\right) + o(1)
        \end{align*}
        
        \item \textbf{Fourth Term.} This term can readily be shown to be $o(1)$ in the analysis of $\vbeta_\star^\top \betaKFAC$; i.e.,
        \begin{align*}
        2\, c_{\star, 1} \, n^{-2} \vbeta_\star^\top \bX^\top  \bX \bR^2 \bX^\top\sigma_{\star,\perp}(\bX\vbeta_\star) = o(1).
        \end{align*}
    \end{itemize}
    Putting everything together, we find
    \begin{align}
        \label{eq:norm_beta_hat}
        \|\betaKFAC\|_2^2 = c_{\star, 1}^2  \dx^{-1}   \trace \left[(\bX^\top\bX/n)^2\,\bR^2\right] + (\sigma_\ep^2 + \Ex_z[\sigma^2_{\star,\perp}(z)]) \; n^{-1} \trace\left((\bX^\top\bX/n)\,\bR^2\right) 
    \end{align}
    Now, given \eqref{eq:inner_hat_final} and \eqref{eq:norm_beta_hat}, defining $c_{\star,>1} = \Ex_z[\sigma^2_{\star,\perp}(z)]$, we have
    \begin{align}
        \label{eq:hat_corr_prelim}
        \frac{\vbeta_\star^\top\betaKFAC}{\|\betaKFAC\| \|\vbeta_\star\|}= \frac{c_{\star,1} \dx^{-1} \trace\left((\bX^\top\bX/n)\,\bR\right) }{\sqrt{c_{\star, 1}^2  \dx^{-1}   \trace \left[(\bX^\top\bX/n)^2\,\bR^2\right] + (\sigma_\ep^2 + c^2_{\star,>1}) \; n^{-1} \trace\left((\bX^\top\bX/n)\,\bR^2\right) }}.
    \end{align}
    Thus, noting that if $\dx/n\to 0$ and $\lambda_\sbG \to 0$, we have $\bR \to \bSigma_\bfx^{-1}$, we find
    \begin{align*}
        \lim_{\lambda\to 0} \lim_{\dx/n \to \infty}\frac{\vbeta_\star^\top\betaKFAC}{\|\betaKFAC\| \|\vbeta_\star\|} = 1,
    \end{align*}
    proving the second part of the lemma. For the first part, we define $m(z):\R \to \R$ as the limiting Stieltjes transform of the the empirical eigenvalue distribution of $n^{-1} \bX^\top\bX$; i.e.,
    \begin{align}
    \label{eq:stieltjes}
        m(z) =  \lim_{\dx, n\to \infty} \dx^{-1} \trace\left[\left(\bX^\top\bX/n - z \,\bI_{\dx}\right)^{-1}\right]
    \end{align}
    where the limit is taken under the assumption that $\dx/n \to \phi>0$. For a general covariance matrix $\bSigma_{\bfX}$, $m(z)$ does not have a closed form except for very special cases; however, it it can be
 efficiently computed. See Section~\ref{sec:stieltjes} for more details. The derivative of the function $m$ is given by
    \begin{align*}
        m'(z) = - \lim_{\dx, n \to \infty} \dx^{-1} \trace\left[\left(\bX^\top\bX/n - z \,\bI_{\dx}\right)^{-2}\right].
    \end{align*}
    We can write all the traces appearing in \eqref{eq:hat_corr_prelim} in terms of the function $m$ and its derivative:
    \begin{align*}
        &\dx^{-1} \trace\left((\bX^\top\bX/n)\,\bR\right)\\ &\hspace{1cm}= \dx^{-1} \trace\left((\bX^\top\bX/n + \lambda_{\bG}\bI_{\dx}-\lambda_{\bG}\bI_{\dx})\,\bR\right) = \dx^{-1} \trace\left(\bI_{\dx} -\lambda_{\bG}\bR\right) = 1 - \lambda_\sbG \, m(-\lambda_\sbG),\\[0.5cm]
        &\dx^{-1} \trace\left((\bX^\top\bX/n)^2\,\bR^2\right)\\ &\hspace{1cm} =  \dx^{-1} \trace\left((\bX^\top\bX/n + \lambda_{\bG}\bI_{\dx}-\lambda_{\bG}\bI_{\dx})^2\,\bR^2\right)= 1 - \lambda_\sbG^2\, m'(-\lambda_\sbG) - 2\lambda_\sbG\, m(-\lambda_\sbG), \\[0.5cm]
        &n^{-1} \trace\left((\bX^\top\bX/n)\,\bR^2\right)\\ &\hspace{1cm} = n^{-1} \trace\left(\bR  - \lambda_\sbG\bR^2\right) = \phi\, m(-\lambda_\sbG) + \phi\,\lambda_\sbG\, m'(-\lambda_\sbG).
    \end{align*}
    With these, the correlation is given by
    \begin{align*}
        \frac{\betaKFAC^\top\vbeta_\star}{\|\betaKFAC\|_2  \|\vbeta_\star\|_2} = \frac{c_{\star,1} [\,1 - \lambda_\sbG\,m(-\lambda_\sbG)\,]}{\sqrt{c_{\star,1}^2[\,1 - \lambda_\sbG^2\, m'(-\lambda_\sbG) - 2\lambda_\sbG\, m(-\lambda_\sbG)\,] + \phi(c_{\star,>1}^2 + \sigma_\ep^2)[\,m(-\lambda_\sbG) + \lambda_\sbG m'(-\lambda_\sbG)\,]}},
    \end{align*}
    which defining
    \begin{align}
        \label{eq:psi_def}
        &\Psi_1 = 1 - \lambda_\sbG\,m(-\lambda_\sbG)\nonumber\\[0.1cm]
        &\Psi_2 = 1 - \lambda_\sbG^2\, m'(-\lambda_\sbG) - 2\lambda_\sbG\, m(-\lambda_\sbG)\nonumber\\[0.1cm]
        &\Psi_3 = m(-\lambda_\sbG) + \lambda_\sbG m'(-\lambda_\sbG)
    \end{align}
    concludes the proof.
\end{proof}
\subsection{From Feature Learning to Generalization}
\label{sec:single_generalize}
In Section~\ref{sec:single_index}, we showed that after one step of \SGD and \KFAC, the first layer weights will become approximately equal to 
\begin{align}
    \label{eq:one_update}
    \widehat\bfG_{a} \approx \widehat\bfG_0 + \alpha \eta\, \bff_0 \hat\vbeta_{a}^\top, \quad a \in \{\SGD, \KFAC\}.
\end{align}
Given \Cref{lemma:beta_tilde_alignment} and \Cref{lemma:beta_hat_alignment}, we argued that compared to \SGD, the weights obtained by the \KFAC algorithm are more aligned to the true direction $\vbeta_\star$. Given a nontrivial alignment between the weights and the target direction, the second layer $\bff$ can be trained using least squares (or based on \Cref{lem:KFAC_is_EMA}, equivalently using one step of the \KFAC update on $\bff$ with $\eta_{\bff} = 1$) with $\Theta(d)$ samples to achieve good generalization performance (See e.g., \citet[Theorem 11]{ba2022high} and \citet[Section 3.4]{dandibenefits}). The existence of nontrivial alignment of the learned weights and the true direction in a single index model is often called \textit{weak recovery} and has been subject to extensive investigation (see e.g., \citet{arous2021online,dandibenefits,troiani2024fundamental,arnaboldi2024repetita}, etc.). 

To see this, consider the feature matrix $\bZ_a \in \R^{n \times \dhid}$ as $\bZ_{a} = \sigma(\bX\widehat\bG_a^\top)$, where the activation function is applied element-wise. Based on equation~\eqref{eq:one_update}, this matrix can be written as
\begin{align*}
    \bZ_{a} \approx \sigma\left(\bX \bG_0^\top + \alpha \eta \, (\bX\hat\vbeta_a) \bff_0^\top\right).
\end{align*}
This is an example of a random matrix in which a nonlinear function is applied element-wise to a random component plus a rank-one signal component which has been studied in the literature \citep{guionnet2023spectral,moniri_atheory2023,moniri2024signal}. In particular, by Taylor expanding the activation function, the feature matrix $\bZ_a$ can be written as
\begin{align*}
    \bZ_{a} &\approx \sigma(\bX\bfG_{0}^\top) +  \sum_{k = 1}^{\ell} \frac{\alpha^k\eta^k}{k!} \left(\sigma^{(k)} (\bX\bfG_{0}^\top)\right) \odot \left((\bX\hat\vbeta_a)^{\circ k}\,\bff_0^{\circ k \top}\right) + \mathcal{E}_\ell,
\end{align*}
where $\circ$ denotes element-wise power and $\mathcal{E}_\ell$ is the reminder term. Let $\eta = n^{\alpha}$ for some $\alpha \in [0, 0.5)$. Given $\alpha$, the integer $\ell$ is chosen to be large enough so that the operator norm of $\mathcal{E}_\ell$ is $o(\dx^{1/2})$ and the reminder term is negligible compared to $\sigma(\bX\bfG_{0}^\top)$. By a simple concentration argument, the matrix $\sigma^{(k)} (\bX\bfG_{0}^\top)$ can be replaced with its mean $\Ex\left(\sigma^{(k)} (\bX\bfG_{0}^\top)\right) = \mu \mathbf{1}\mathbf{1}^\top$ to get
\begin{align*}
    \bZ_{a} &\approx \sigma(\bX\bfG_{0}^\top) +  \sum_{k = 1}^{\ell} \frac{\alpha^k\eta^k \mu}{k!}(\bX\hat\vbeta_a)^{\circ k}\,\bff_0^{\circ k \top}.
\end{align*}
The first term $\sigma(\bX\bfG_{0}^\top)$ is the feature matrix of a random feature model and based on the Gaussian Equivalence Theorem (GET) (see e.g., \citet{goldt2022gaussian,hu2022universality,dandi2023learning,moniri_atheory2023}), we can 
linearize it; i.e., we can replace it with $\alpha \bX\bfG_{0}^\top + \bN$ where $\bN$ is a properly scaled independent Gaussian noise. The vectors $(\bX\hat\vbeta_a)^{\circ k}$ are nonlinear functions of the covariates with different degrees. The least squares estimator $\hat\bff_a$ is then fit on the features $\bfZ_a$ in a way that $\widehat\calL(\hat\bff_a, \widehat\bfG_a)$ is minimized; i.e.
\begin{align}
    \label{eq:train}
    \vy \approx \sigma(\bX\bfG_{0}^\top)\hat\bff_a  +  \sum_{k = 1}^{\ell} \frac{\alpha^k\eta^k \mu \,(\bff_0^{\circ k \top}\hat\bff_a)}{k!}(\bX\hat\vbeta_a)^{\circ k}.
\end{align}
Based on the GET, the random feature component can only learn linear functions with sample complexity of learning $n = \Theta(\dhid) = \Theta(\dx)$. When $\eta$ is large enough and $\vbeta_a$ is aligned to $\vbeta_\star$, with the finite dimensional correction to the random features model, the model can also represent nonlinear functions $(\vx^\top \hat\vbeta_a)^k$ of degree $k \leq \ell$ by matching the coefficients ${\alpha^k\eta^k \mu \,(\bff_0^{\circ k \top}\hat\bff_a)}/{k!}$ with the Taylor coefficients of the teacher function $\sigma_\star(\vx^\top \hat\vbeta_\star)$.

Although we have provided a complete proof sketch for providing generalization guarantees given weight alignment, a complete analysis require tedious computations and is beyond the scope of this work as we mainly focus on feature learning properties of different optimization algorithms.









\section{Additional Information on Kronecker-Factored Preconditioners}\label{sec:KF_derivations}

Here, we provide some additional background information regarding key Kronecker-Factored preconditioning methods, including their derivation and relations to various methods in the literature. We recall the running example of a fully-connected net omitting biases, introducing layer-wise dimensionality and a final non-linear layer (e.g.\ softmax) for completeness:
\begin{equation}\label{eq:fcnn}
    f_{\btheta}(\bfx) = \phi(\bW_L \sigma (\bW_{L-1} \cdots \sigma(\bW_1 \bfx) \cdots )), \quad \bW_\ell \in \R^{d_\ell \times d_{\ell - 1}}, d_0 = \dx.
\end{equation}
As before, we define $\btheta$ as the concatenation of $\btheta_\ell = \VEC(\bW_\ell)$, $\ell \in [L]$.
We define an expected loss induced by the neural network $\calL(\btheta) = \Ex_{(\bfx, \bfy)}[\ell(f_{\btheta}(\bfx), \bfy)]$, and its batch counterpart $\hatL(\btheta)$. Here, we define the family of Kronecker-Factored preconditioned optimizers as those that update weights in the following fashion:
\begin{align*}
    {\bW_\ell}_+ &= \bW_\ell - \eta\;\bP_\ell^{-1} \nabla_{\bW_\ell} \hatL(\btheta) \bQ_\ell^{-1},\quad \ell \in [L],
\end{align*}
where $\bP_\ell \in \R^{d_\ell \times d_\ell}$, $\bQ_\ell \in \R^{d_{\ell - 1} \times d_{\ell - 1}}$, $\ell \in [L]$ are square matrices.
For simplicity, we ignore moving parts such as momentum, damping exponents, adaptive learning rate schedules/regularization etc. We now demonstrate the basic principles and derivation of certain notable members of these preconditioning methods on the feedforward network \eqref{eq:fcnn}.

\subsection{Kronecker-Factored Approximate Curvature \KFAC}

As described in the main paper, \KFAC \cite{martens2015optimizing} is at its core an approximation to natural gradient descent. Given that we are approximating \NGD, a crucial presumption on $f_{\btheta}(\bfx)$ and $\calL(\btheta)$ is that the network output $f_{\btheta}(\bfx)$ parameterizes a conditional distribution $p(\bfy|\bfx; \btheta)$, and $\calL(\btheta) \propto \Ex_{(\bfx, \bfy)}[-\log p(\bfy|\bfx; \btheta)]$ is the corresponding negative log-likelihood. As such, \KFAC is technically only applicable to settings where such an interpretation exists. However, this notably subsumes cases $\calL(\btheta) = \Ex_{(\bfx, \bfy)}[\ell(f_{\btheta}(\bfx), \bfy)]$, where $\ell(\cdot)$ is a strictly convex function in $f_{\btheta}(\bfx)$, as this admits an interpretation as $f_{\btheta}(\bfx)$ parameterizing an exponential family distribution. In particular, the square-loss regression case $\ell(\hat\bfy, \bfy) = \norm{\hat\bfy - \bfy}^2$ corresponds to a conditionally-Gaussian predictive distribution with fixed variance $\hat\bfy(\bfx) \sim \normal(f_{\btheta}(\bfx), \sigma^2 \bI)$, and if $\phi(\cdot)$ is a softmax layer and $\ell(\hat\bfy, \bfy) = \texttt{CrossEnt}(\hat\bfy, \bfy)$, the multi-class classification case corresponds to a conditionally-multinomial predictive distribution.

Defining $\bfh_\ell = \bW_\ell \bfz_{\ell-1}$, $\bfz_\ell = \sigma(\bfh)$, $\bfz_0 = \bfx$, the Fisher Information of the predictive distribution $p(\bfy|\bfx; \btheta)$ at $\btheta$ can be expressed in block form:
\begin{align*}
    \mathbf{FI}(\btheta) &\triangleq \Ex_{\bfx}\brac{\pder{p(\bfy|\bfx; \btheta)}{\btheta}\paren{\pder{p(\bfy|\bfx; \btheta)}{\btheta}}^\top} \tag{recall $\btheta$ is $\VEC$-ed parameters} \\
    &= \bmat{
    \Ex_{\bfx} \brac{\pder{p(\bfy|\bfx; \btheta)}{\btheta_1}\paren{\pder{p(\bfy|\bfx; \btheta)}{\btheta_1}}^\top } & \cdots & \Ex_{\bfx}\brac{\pder{p(\bfy|\bfx; \btheta)}{\btheta_1}\paren{\pder{p(\bfy|\bfx; \btheta)}{\btheta_L}}^\top } \\
    \vdots & \ddots & \vdots \\
    \Ex_{\bfx} \brac{\pder{p(\bfy|\bfx; \btheta)}{\btheta_L}\paren{\pder{p(\bfy|\bfx; \btheta)}{\btheta_1}}^\top } & \cdots & \Ex_{\bfx} \brac{\pder{p(\bfy|\bfx; \btheta)}{\btheta_L}\paren{\pder{p(\bfy|\bfx; \btheta)}{\btheta_L}}^\top }
    }
\end{align*}
Looking at the $(i,j)$th block, we have
\begin{align*}
    \Ex_{\bfx} \brac{\pder{p(\bfy|\bfx; \btheta)}{\btheta_i}\paren{\pder{p(\bfy|\bfx; \btheta)}{\btheta_j}}^\top } &= \Ex_{\bfx} \brac{\VEC\paren{\pder{p(\bfy|\bfx; \btheta)}{\bW_i}}\VEC\paren{\pder{p(\bfy|\bfx; \btheta)}{\bW_j}}^\top } \\
    &= \Ex_{\bfx} \brac{(\bfz_{i-1} \otimes \bfg_i )(\bfz_{j-1} \otimes \bfg_j)^\top } \tag{$\bfg_\ell \triangleq - \pder{p(\bfy|\bfx; \btheta)}{\bfh_\ell}$} \\[0.2cm]
    &= \Ex_{\bfx}\brac{(\bfz_{i-1} \bfz_{j-1}^\top) \otimes (\bfg_{i} \bfg_j^\top)} \tag{\Cref{lem:kron_properties}, item 2},
\end{align*}
where the second line comes from writing out the backpropagation formula. \KFAC makes two key approximations:
\begin{enumerate}
    \item The matrix $\mathbf{FI}(\btheta)^{-1}$ is approximated by a block-diagonal, and hence so is $\mathbf{FI}(\btheta)$. We note the original formulation of \KFAC in \citet{martens2015optimizing} also supports a tridiagonal inverse approximation.
    \item The vectors $\bfz_{\ell-1}$ and $\bfg_\ell$ are independent for all $\ell \in [L]$, such that
    \begin{align*}
        \Ex_{\bfx}\brac{(\bfz_{\ell-1}\, \bfz_{\ell-1}^\top) \otimes (\bfg_{\ell} \,\bfg_\ell^\top)} &= \Ex[\bfz_{\ell - 1} \,\bfz_{\ell - 1}^\top] \otimes \Ex[\bfg_{\ell}\, \bfg_\ell^\top].
    \end{align*}
\end{enumerate}
Now replacing the true expectation with the empirical estimate, and defining $\bP_\ell = \hatEx[\bfg_{\ell}\, \bfg_\ell^\top]$, $\bQ_\ell = \hatEx[\bfz_{\ell - 1}\, \bfz_{\ell - 1}^\top]$ completes the Kronecker-Factored approximation to the Fisher Information. It is clear to see from the derivation that, as we previewed in the introduction and expressed emphatically in \citet{martens2015optimizing}, this approximation is \textit{never} expected to be tight. 


\subsection*{Some related preconditioners}
Having introduced \KFAC, we introduce some related preconditioners. Notably, it has been noted that computing $\bfg_\ell$ requires a backwards gradient computation, whereas $\bfz_\ell$ only requires a forward pass. In particular, various works have recovered the \emph{right} preconditioner $\bQ_\ell$ of \KFAC via various notions of ``local'' (layer-wise) losses. Notably, these alternative views allow \KFAC-like preconditioning to extend beyond the negative-log-likelihood interpretation.
\begin{itemize}
    \item \texttt{LocoProp}, square-loss case \cite{amid2022locoprop}:
    \begin{align*}
        \text{Update rule}: {\bW_{\ell}}_+ &= \argmin_{\bW} \frac{1}{2}\hatEx\left[\norm{\bW \bfz_{\ell-1} - \bfh_\ell}^2\right] + \frac{1}{2\eta}\norm{\bW - \bW_\ell}_F^2 \\
        &= \bW_\ell - \eta \nabla_{\bW_\ell} \hatL(\btheta)\paren{\bI_{d_{\ell-1}} + \eta \,\hatEx[\bfz_{\ell-1} \bfz_{\ell-1}^\top]}^{-1}.
    \end{align*}
    As noted in \citet{amid2022locoprop}, this update is also closely related to \texttt{ProxProp} \cite{frerix2018proximal}.

    \item \texttt{FOOF} \cite{benzing2022gradient}:
    \begin{align*}
        \text{Update rule}: \Delta \bW_\ell &= \argmin_{\Delta \bW}\;\hatEx \left[\norm{\Delta \bW \bfz_{\ell-1} - \eta\, \bfg_\ell}^2\right] + \frac{\lambda}{2} \norm{\Delta \bW}_F^2 && \left(\bfg_\ell = \pder{\ell(f_{\btheta}(\bfx), \bfy)}{\bfh_\ell}\right) \\[0.2cm]
        &= \eta \nabla_{\bW_\ell} \hatL(\btheta) \paren{\hatEx[\bfz_{\ell-1} \bfz_{\ell-1}^\top] + \lambda\, \bI_{d_{\ell-1}}}^{-1}, \\[0.2cm]
        {\bW_{\ell}}_+ &= \bW_\ell - \Delta \bW_\ell.
    \end{align*}
\end{itemize}
Interestingly, we note that these right-preconditioner-only variants subsume the \texttt{DFW} algorithm for two-layer linear representation learning proposed in \citet{zhang2023meta}; thus we may see the guarantee therein as support of the above algorithms from a feature learning perspective, albeit weaker than \Cref{thm:linrep_kfac_guarantee}.


\subsection{\Shampoo}

\Shampoo is designed to be a Kronecker-Factored approximation of the full \AdaGrad preconditioner, which we recall is the running sum of the outer-product of loss gradients. Turning off the \AdaGrad accumulator and instead considering the empirical batch estimate $\hatEx[\nabla_{\btheta}\, \ell(f_{\btheta}(\bfx), \bfy) \;\nabla_{\btheta} \, \ell(f_{\btheta}(\bfx), \bfy)^\top]$, the curvature matrix being estimated can also be viewed as the Gauss-Newton matrix $\Ex_{(\bfx, \bfy)}[\nabla_{\btheta}\, \ell(f_{\btheta}(\bfx), \bfy) \;\nabla_{\btheta}\, \ell(f_{\btheta}(\bfx), \bfy)^\top]$. As documented in various works (see e.g.\ \citet{martens2020new}), the (generalized) Gauss-Newton matrix in many cases is related or equal to the Fisher Information, establishing a link between the target curvatures of \KFAC and \Shampoo.

However, the \Shampoo preconditioners differ from \KFAC's. Let us define the $\bfz_\ell, \bfh_\ell$ as before, and $\bfg_\ell = \pder{\ell(f_{\btheta}(\bfx), \bfy)}{\bfh_\ell}$. Then, the \Shampoo preconditioners are given by
\begin{align*}
    \bP_\ell &= \hatEx\brac{\bfg_\ell \bfz_{\ell-1}^\top (\bfg_\ell \bfz_{\ell-1}^\top)^\top}^{1/4}, \quad \bQ_\ell = \hatEx\brac{\bfz_{\ell-1} \bfg_\ell^\top (\bfz_{\ell-1} \bfg_\ell^\top)^\top}^{1/4}.
\end{align*}
Notably, \Shampoo takes the fourth root in the preconditioners, as its target is the \AdaGrad preconditioner which is (modulo scaling) the square-root of the empirical Gauss-Newton matrix--analogous to the square-root of the second moment in \Adam. Whether the target curvature should be the square-root or not of the Gauss-Newton matrix is the topic of recent discussion \cite{morwani2024new, lin2024can}.

\subsection{Kronecker-Factored Preconditioners and the Modular Norm}\label{sec:layerwise_modular_norms}

The ``modular norm'' \cite{large2024scalable, bernstein2024modular, bernstein2024old} is a recently introduced notion that provides a general recipe for producing different optimization algorithms that act layer-wise. By specifying different norms customized for different kinds of layers (e.g.\ feed-forward, residual, convolutional etc.), one in principle has the flexibility to customize an optimizer to handle the different kinds of curvature induced by different parameter spaces. Given a choice of norm on the weight tensor $\bW_\ell$, the descent direction is returned by \textit{steepest descent} with respect to that norm. To introduce steepest descent, we require a few definitions (cf.\ \citet{bernstein2024modular}):
\begin{definition}[Dual norms, steepest direction]
    Given a norm $\norm{\cdot}$ defined over a finite-dimensional real vector space $\calV$. The dual norm $\norm{\cdot}_\dagger$ is defined by
    \begin{align*}
        \norm{\bfv}_\dagger = \max_{\norm{\bfu} = 1} \ip{\bfu,\bfv}.
    \end{align*}
    With $\bfg \in \calV$ and a ``sharpness'' parameter $\eta > 0$, the steepest direction(s) are given by the following variational representation:
    \begin{align*}
        \argmin_{\bfd} \brac{\ip{\bfg, \bfd} + \frac{1}{2\eta} \norm{\bfd}^2} = -\eta\norm{\bfg}_\dagger \cdot \argmax_{\norm{\bfu}=1} \;\ip{\bfg, \bfu}.
    \end{align*}
\end{definition}
Here we focus on finite-dimensional normed spaces, but note that these concepts extend \emph{mutatis mutandis} to general Banach spaces. The aforementioned works derive various standard optimizers by choosing different norms, including \emph{induced matrix norms} $\norm{\bW_\ell}_{\alpha \to \beta} = \max_{\bfx} \frac{\norm{\bW_\ell \bfx}_\beta}{\norm{\bfx}_\alpha}$, applied to a given layer's weight space, for example \cite{bernstein2024old}:
\begin{itemize}
    \item \SGD: induced by Frobenius (Euclidean) norm $\norm{\cdot} = \norm{\cdot}_F$. Note the Frobenius norm is \emph{not} an induced matrix norm.
    
    \item Sign-descent (``ideal'' \Adam with EMA on moments turned off): induced by $\norm{\cdot} = \norm{\cdot}_{\ell_1 \to \ell_\infty}$.

    \item \Shampoo (``ideal'' variant with moment accumulator turned off): induced by $\norm{\cdot} = \norm{\cdot}_{\ell_2 \to \ell_2} =  \opnorm{\cdot}$.
    
\end{itemize}

Therefore, in light of this characterization, a natural question to ask is \emph{what norm induces a given Kronecker-Factored preconditioner} (which includes \Shampoo). We provide a simple derivation that determines the norm.
\begin{proposition}[Kronecker-Factored matrix norm]\label{prop:KF_norm}
    Recall the fully-connected network \eqref{eq:fcnn}. Given preconditioners $\scurly{(\bP_\ell, \bQ_\ell)}_{\ell = 1}^{L}$, where $\bP_\ell \in \R^{d_\ell \times d_\ell}$, $\bQ_\ell \in \R^{d_{\ell - 1} \times d_{\ell - 1}}$, $\ell \in [L]$ are invertible square matrices. Then, the layer-wise Kronecker-Factored update:
    \begin{align*}
        {\bW_{\ell}}_+ = \bW_{\ell} - \eta\, \bP_\ell^{-1} \nabla_{\bW_\ell} \calL(\btheta)\; \bQ_\ell^{-1}, \quad \ell \in [L]
    \end{align*}
    is equivalent to layer-wise steepest descent with norm $\norm{\bM_\ell} \triangleq \norm{\bP_\ell^\top \bM_\ell \bQ_\ell^\top}_F$:
    \begin{align*}
        \argmin_{\bM} \brac{\ip{\nabla_{\bW_\ell} \calL(\btheta), \bM} + \frac{1}{2\eta} \norm{\bM}^2} = - \eta\, \bP_\ell^{-1} \nabla_{\bW_\ell} \calL(\btheta)\; \bQ_\ell^{-1}.
    \end{align*}
\end{proposition}
\begin{proof}[Proof of \Cref{prop:KF_norm}]
    It is straightforward to verify $\norm{\bM} \triangleq \norm{\bP^\top \bM \bQ^\top}_F$ for invertible $\bP, \bQ$ satisfies the axioms of a norm. It remains to verify the steepest descent direction:
    \begin{align*}
        \argmin_{\bM} \brac{\ip{\nabla_{\bW_\ell} \calL(\btheta), \bM} + \frac{1}{2\eta} \norm{\bM}^2} &= -\eta\,\norm{\nabla_{\bW_\ell} \calL(\btheta)}_\dagger \cdot \argmax_{\norm{\bM}=1} \;\ip{\nabla_{\bW_\ell} \calL(\btheta), \bM} = \bP_\ell^{-1} \nabla_{\bW_\ell} \calL(\btheta)\; \bQ_\ell^{-1}.
    \end{align*}
    We start by writing:
    \begin{align*}
        \norm{\nabla_{\bW_\ell} \calL(\btheta)}_\dagger &\triangleq \max_{\norm{\bM} = 1} \;\ip{\nabla_{\bW_\ell} \calL(\btheta), \bM} \\[0.1cm]
        &= \max_{\norm{\bP_\ell^\top \bM \bQ_\ell^\top}_F = 1} \trace(\bM^\top \nabla_{\bW_\ell} \calL(\btheta)) \\[0.1cm]
        &= \max_{\norm{\bD}_F = 1} \trace(\bQ_\ell^{-1} \bD^\top \bP_\ell^{-1}  \nabla_{\bW_\ell} \calL(\btheta)) \tag{$\bP_\ell^\top \bM \bQ_\ell^\top \to \bD$} \\[0.1cm]
        &= \norm{\bP_\ell^{-1} \nabla_{\bW_\ell} \calL(\btheta)\; \bQ_\ell^{-1}}_F. \tag{trace cyclic property, $\norm{\cdot}_F$ is self-dual}
    \end{align*}
    Similarly, it is straightforward to verify that the maximizing matrix is:
    \begin{align*}
        \argmax_{\norm{\bM}=1} \;\ip{\nabla_{\bW_\ell} \calL(\btheta), \bM} &= \frac{\bP_\ell^{-1} \nabla_{\bW_\ell} \calL(\btheta)\; \bQ_\ell^{-1}}{\norm{\bP_\ell^{-1} \nabla_{\bW_\ell} \calL(\btheta)\; \bQ_\ell^{-1}}_F},
    \end{align*}
    such that plugging it into the steepest descent expression yields:
    \begin{align*}
        \argmin_{\bM} \brac{\ip{\nabla_{\bW_\ell} \calL(\btheta), \bM} + \frac{1}{2\eta} \norm{\bM}^2} &= - \eta\; \norm{\bP_\ell^{-1} \nabla_{\bW_\ell} \calL(\btheta)\; \bQ_\ell^{-1}}_F  \cdot \frac{\bP_\ell^{-1} \nabla_{\bW_\ell} \calL(\btheta)\; \bQ_\ell^{-1}}{\norm{\bP_\ell^{-1} \nabla_{\bW_\ell} \calL(\btheta)\; \bQ_\ell^{-1}}_F} \\
        &= - \eta\; \bP_\ell^{-1} \nabla_{\bW_\ell} \calL(\btheta)\; \bQ_\ell^{-1},
    \end{align*}
    as required.
\end{proof}
We remark that for complex-valued matrices, the above holds without modification for the Hermitian transpose $\bA^{\mathrm H}$.
Notably, the layer-wise norm corresponding to Kronecker-Factored preconditioning is not an induced matrix norm, though modified optimizers can certainly be derived via induced-norm variants, such as a ``Mahalonobis-to-Mahalonobis'' induced norm:
\begin{align*}
    \norm{\bM}_{\bQ^{-1} \to \bP} &\triangleq \max_{\bfx} \frac{\sqrt{(\bM\bfx)^\top \bP(\bM \bfx)}}{\sqrt{\bfx^\top \bQ^{-1} \bfx}} \tag{$\bP, \bQ \succ \bzero$} \\[0.1cm]
    &= \max_{\norm{\bfx} = 1} \norm{\bP^{1/2} \bM \bQ^{1/2}}\\[0.1cm]
    &= \opnorm{\bP^{1/2} \bM \bQ^{1/2}}.
\end{align*}

\section{Auxiliary Results}\label{sec:aux_results}

\subsection{Properties of Kronecker Product}
Recall the definition of the Kronecker Product: given $\bA \in \R^{m \times n}$, $\bB \in \R^{p \times q}$
\begin{align*}
    \bA \otimes \bB &= \bmat{A_{11} \bB & \cdots & A_{1n} \bB \\ 
    \vdots & \ddots & \vdots \\
    A_{m 1} \bB & \cdots & A_{mn} \bB} \in \R^{mp \times nq}.
\end{align*}
Complementarily, the vectorization operator $\VEC(\bA)$ is defined by stacking the columns of $\bA$ on top of each other (i.e.\ column-major order)
\begin{align*}
    \VEC(\bA) &= \bmat{A_{11} & \cdots & A_{m1} & \cdots & A_{1n} & \cdots & A_{mn}}^\top \in \R^{mn}.
\end{align*}
We now introduce some fundamental facts about the Kronecker Product.
\begin{lemma}[Kronecker-Product Properties]\label{lem:kron_properties}
    The following properties hold:
    \begin{enumerate}
        \item $(\bA \otimes \bB)^{-1} = \bA^{-1} \otimes \bB^{-1}$. Holds for Moore-Penrose pseudoinverse $\;^{\dagger}$ as well.
        \item For size-compliant $\bA, \bB, \bC, \bD$, we have $(\bA \otimes \bB) (\bC \otimes \bD) = (\bA \bC) \otimes (\bB \otimes \bD)$.
        \item $\VEC(\bA\bX\bB) = (\bB^\top \otimes \bA)\VEC(\bX)$.
    \end{enumerate}
\end{lemma}


\subsection{Covariance Concentration}


We often use the following Gaussian covariance concentration result.
\begin{lemma}[Gaussian covariance concentration]\label{lem:gauss_cov_conc}
    Let $\bfx_i \iidsim \normal(\bzero, \bSigma_\bfx)$ for $i = 1,\dots,n$, where $\bfx_i \in \R^d$. Defining the empirical covariance matrix $\hatSigma \triangleq \frac{1}{n}\sum_{i=1}^n \bfx_i \bfx_i^\top$, as long as $n \geq \frac{18.27}{c^2} (d + \log(1/\delta))$, we have with probability at least $1 - \delta$,
    \begin{align*}
        (1-c) \bSigma_\bfx \preceq \hatSigma \preceq (1+c) \bSigma_\bfx.
    \end{align*}
\end{lemma}

\begin{proof}[Proof of \Cref{lem:gauss_cov_conc}]
    The result follows essentially from combining a by-now standard concentration inequality for Gaussian quadratic forms and a covering number argument. To be precise, we observe that
    \begin{align*}
        \opnorm{\hatSigma[\bfx] - \bSigma_\bfx} \leq c\norm{\bSigma_\bfx} \implies (1-c)\bSigma_\bfx \preceq \hatSigma[\bfx] \preceq (1+c)\bSigma_\bfx.
    \end{align*}
    Therefore, it suffices to establish a concentration bound on $\norm{\hatSigma[\bfx] - \bSigma_\bfx}$ and invert for $c\norm{\bSigma_\bfx}$. To do so, we recall a standard covering argument (see e.g.\ \citet[Chapter 4]{vershynin2018high}) yields: given an $\varepsilon$-covering of $\bbS^{d-1}$, $\calN \triangleq \calN(\bbS^{d-1}, \norm{\cdot}_2, \varepsilon)$, the operator norm of a symmetric matrix $\bSigma$ is bounded by
    \begin{align*}
        \norm{\bSigma} &\leq \frac{1}{1 - 2\varepsilon} \max_{\bfu \in \calN} \bfu^\top \bSigma \bfu,
    \end{align*}
    where the corresponding covering number is bounded by:
    \begin{align*}
        \abs{\calN(\bbS^{d-1}, \norm{\cdot}_2, \varepsilon)} &\leq \paren{1 + \frac{2}{\varepsilon}}^d.
    \end{align*}
    As such it suffices to provide a concentration bound on $\bfu^\top \bSigma \bfu$ for each $\bfu \in \calN$ and then union-bound. Toward establishing this, we first state the Gaussian quadratic form concentration bound due to \citet{hsu2012random}, which is in turn an instantiation of a chi-squared concentration bound from \citet{laurent2000adaptive}.
    \begin{proposition}[Prop.\ 1 in \cite{hsu2012random}]\label{prop:hsu_quad_form}
        Let $\bA \in \R^{m \times d}$ be a fixed matrix. Let $\bfg \sim \normal(\bzero, \bI_d)$ be a mean-zero, isotropic Gaussian random vector. For any $\delta \in (0,1)$, we have
        \begin{align*}
            \sfP[\norm{\bA \bfg}^2 > \trace(\bA^\top \bA) + 2\sqrt{\trace((\bA^\top \bA)^2) \log(1/\delta)} + 2\opnorm{\bA^\top \bA} \log(1/\delta)] \leq \delta.
        \end{align*}
    \end{proposition}
    Now, given $\bfu \in \bbS^{d-1}$, setting $\bA = \bfu^\top \bSigma^{1/2}$ such that $\bfu^\top \bSigma^{1/2} \bfg \overset{d}{=} \bfu^\top \bfx$, instantiating \Cref{prop:hsu_quad_form} yields:
    \begin{align*}
        \sfP\brac{\bfu^\top \hatSigma \bfu > \bfu^\top \bSigma_\bfx \bfu + 2\bfu^\top \bSigma_\bfx \bfu \sqrt{\frac{\log(1/\delta)}{n}} + 2\bfu^\top \bSigma_\bfx \bfu \frac{\log(1/\delta)}{n}} \leq \delta.
    \end{align*}
    Put another way, this says with probability at least $1 - \delta$:
    \begin{align*}
        \bfu^\top(\hatSigma-\bSigma) \bfu &\leq 2\bfu^\top \bSigma_\bfx \bfu \paren{\sqrt{\frac{\log(1/\delta)}{n}} + \frac{\log(1/\delta)}{n}}.
    \end{align*}
    Taking a union bound over $\bfu \in \calN$, we get with probability at least $1 - \delta$:
    \begin{align*}
        \max_{\bfu \in \calN} \bfu^\top(\hatSigma-\bSigma) \bfu &\leq \max_{\bfu \in \calN} 2\bfu^\top \bSigma_\bfx \bfu \paren{\sqrt{\frac{\log(\abs{\calN}/\delta)}{n}} + \frac{\log(\abs{\calN}/\delta)}{n}} \\
        &\leq 2 \opnorm{\bSigma_\bfx} \paren{\sqrt{\frac{d\log\paren{1 + \frac{2}{\varepsilon}}+\log(1/\delta)}{n}} + \frac{d\log\paren{1 + \frac{2}{\varepsilon}} + \log(1/\delta)}{n}} \\
        &\leq 4 \sqrt{\log\paren{1 + \frac{2}{\varepsilon}}} \opnorm{\bSigma_\bfx} \sqrt{\frac{d + \log(1/\delta)}{n}},
    \end{align*}    
    as long as $n \geq d\log\paren{1 + \frac{2}{\varepsilon}}+\log(1/\delta)$.
    Chaining together inequalities, this yields with probability at least $1 - \delta$ under the same condition on $n$:
    \begin{align*}
        \opnorm{\hatSigma - \bSigma_\bfx} &\leq \opnorm{\bSigma_\bfx}\frac{2}{1 - \varepsilon}\sqrt{\log\paren{1 + \frac{2}{\varepsilon}}} \sqrt{\frac{d + \log(1/\delta)}{n}}.
    \end{align*}
    Minimizing the RHS for $\varepsilon \approx 0.0605$ yields the result.
\end{proof}


\subsection{Extensions to subgaussianity}\label{appdx:subgaussian}

As previewed, many results can be extended from the Gaussian setting to subgaussian random vectors.
\begin{definition}\label{def:subgaussian}
    A (scalar) random variable $X$ is \emph{subgaussian} with variance proxy $\sigma^2$ if the following holds on its moment-generating function:
    \begin{align*}
        \Ex[\exp(\lambda X)] &\leq \exp\paren{\frac{\lambda^2 \sigma^2}{2}}.
    \end{align*}
    A mean-zero random vector $\bfx \in \R^d$, $\Ex[\bfx] = \bzero$, is \emph{subgaussian} with variance proxy $\sigma^2$ if every linear projection is a $\sigma^2$-subgaussian random variable:
    \begin{align*}
        \Ex[\exp(\lambda \bfv^\top \bfx)] &\leq \exp\paren{\frac{\lambda^2 \norm{\bfv}^2 \sigma^2}{2}}, \quad \text{for all }\bfv \in \R^d.
    \end{align*}

\end{definition}
With this in hand, we may introduce the subgaussian variant of covariance concentration and the Hanson-Wright inequality.

\subsection{Subgaussian Covariance Concentration}

We state the subgaussian variant of \Cref{lem:gauss_cov_conc}, whose proof is structurally the same, replacing the $\chi^2$ random variables with a generic subexponential (c.f.\ \citet[Chapter 2]{vershynin2018high}) random variable, and using a generic Bernstein's inequality rather than the specific $\chi^2$ concentration inequality. The result is qualitatively identical, sacrificing tight/explicit universal numerical constants. The result is relatively standard, and can be found in e.g., \citet[Chapter 5]{vershynin2018high} or \citet[Lemma A.6]{du2020few}.
\begin{lemma}[Subgaussian covariance concentration]
    Let $\bfx_i$ be i.i.d.\ zero-mean $\sigma^2$-subgaussian random vectors for $i = 1,\dots,n$, where $\bfx_i \in \R^d$, and $\Ex[\bfx \bfx^\top ] = \bSigma_\bfx$. Defining the empirical covariance matrix $\hatSigma \triangleq \frac{1}{n}\sum_{i=1}^n \bfx_i \bfx_i^\top$, there exists a universal constant $C_1>0 $ such that with probability at least $1 - 2\delta$:
    \begin{align*}
        \opnorm{\hatSigma - \bSigma_\bfx} &\leq C \sigma^2 \opnorm{\bSigma_\bfx} \paren{\sqrt{\frac{d + \log(1/\delta)}{n}} + \frac{d + \log(1/\delta)}{n}}.
    \end{align*}
    Therefore, as long as $n \geq C_2 \frac{\sigma^2}{c^2} (d + \log(1/\delta))$, we have with probability at least $1 - \delta$,
    \begin{align*}
        (1-c) \bSigma_\bfx \preceq \hatSigma \preceq (1+c) \bSigma_\bfx.
    \end{align*}
\end{lemma}

\subsection{Hanson-Wright Inequality}
We often use the following theorem  to prove the concentration inequality for quadratic forms. A modern proof of this theorem can be found in \citet{rudelson2013hanson}.

\begin{theorem}[Hanson-Wright Inequality \citep{hanson1971bound}]
    \label{thm:hanson-wright}
    Let $\bfx=\left(X_1, \ldots, X_n\right) \in \mathbb{R}^d$ be a random vector with independent sub-gaussian components $X_i$ with $\Ex X_i=0$. Let $\bD$ be an $n \times n$ matrix. Then, for every $t \geq 0$, we have
    $$
    \mathbb{P}\left\{\left|\bfx^{\top} \bD\, \bfx-\Ex\left[ \bfx^{\top} \bD\, \bfx\right]\right|>t\right\} \leq 2 \exp \left[-c \min \left(\frac{t^2}{\|\bD\|_{F}^2}, \frac{t}{\|\bD\|_{\rm op}}\right)\right],
    $$
    where $c$ is a constant that depends only on the subgaussian constants of $X_i$.
\end{theorem}
\subsection{Stein's Lemma}
We use the following simple lemma which is an application of integration by parts for Gaussian integrals.
\begin{lemma}[Stein's Lemma]
    \label{lemma:stein's}
    Let $X$ be a random variables drawn from $\normal(\mu, \sigma^2)$ and $g:\R \to \R$ be a differentiable function. We have $\Ex\left[\,g(X) (X - \mu)\,\right] = \sigma^2\, \Ex \left[\,g'(X)\right]$.    
\end{lemma}

\subsection{Woodbury Matrix Identity}
In the proofs, we use the following elementary identity which states that the inverse of a rank-$k$ correction of a matrix is equal to a rank-$k$ correction to the inverse of the original matrix. 
\begin{theorem}[Woodbury Matrix Identity \citep{woodbury1950inverting}]
    \label{thm:woodbury}
    Let $\bA \in \R^{n \times n}, \bC \in \R^{k \times k}, \bU \in \R^{n \times k}$, and $\bV\in \R^{k \times n}$. The following matrix identity holds:
    \begin{align*}
        (\bA + \bU\bC\bV)^{-1} = \bA^{-1} - \bA^{-1} \bU (\bC^{-1} + \bV \bA^{-1}\bU)^{-1} \bV \bA^{-1},
    \end{align*}
    assuming that the inverse matrices in the expression exist.
\end{theorem}

\subsection{Stieltjes Transform of Empirical Eigenvalue Distribution}
\label{sec:stieltjes}
For a distribution $\mu$ over $\R$, its Stieltjes transform is defined as 
\begin{align*}
    m_\mu(z) = \int\frac{d\mu(x)}{x - z}.
\end{align*}

Let $H_d$ be the (discrete) empirical eigenvalue distribution of $\bSigma_{\bfx} \in \R^{d\times d}$ and let $F_d$ be the (discrete) empirical eigenvalue distribution of the sample covariance matrix $\widehat\bSigma \in \R^{d\times d}$. Consider the proportional limit where $d, n \to \infty$ with $d/n\to \phi>0$. Suppose that the eigenvalue distribution $H_d$ converges to a limit population spectral distribution $H_{\bSigma_{\bfx}}$; i.e., $H_d \Rightarrow H_{\bSigma_{\bfx}}$ in distribution. Given the definition of $m(z)$ from equation~\eqref{eq:stieltjes}, we have $m(z) = m_{F}(z)$. The following theorem characterizes $m_F$ in terms of $H_{\bSigma_{\bfx}}$.

\begin{theorem}[Silverstein Equation \citep{silverstein1995analysis}]
    Let $\nu_F(z) = \phi(m_F(z) + 1/z) - 1/z$. The function $\nu_F$ is the solution of the following fixed-point equation:
    \begin{align*}
        -\frac{1}{\nu_F(z)}=z-\phi \int \frac{t\, }{1+t \,\nu_F(z)} d H_{\bSigma_{\bfx}}(t).
    \end{align*}
\end{theorem}

Thus, using this theorem, given $H_{\bSigma_{\bfx}}$, we can numerically compute $\nu_F$  (and hence, $m_F$) using fixed-point iteration. For example, for $\bSigma_{\bfx}^{(\varepsilon)}$ from equation~\eqref{eq:sigma2}, we have $F = 1/2\, \delta_{1-\varepsilon} + 1/2\, \delta_{1+\varepsilon}$.

\section{Additional Numerical Results and Details}\label{sec:additional_numerics}

\subsection{Details of the Experiment Setups}
\label{sec:details_exp_setups}
In the experiments, we generate $\bfF_0 \in \R^{\dy \times k}$ with i.i.d. $\normal(0,1)$ entries. Then, for each task $\msf{s}$ we randomly draw a matrix $\bfB_\msf{s} \in \R^{\dy \times \dy}$ and set $\bfF_\star^{\msf{s}}  = \exp\left(0.005  (\bfB_{\msf{s}} - \bfB_{\msf{s}}^\top)\right) \bfF_0$,
where $\exp(\cdot)$ is the matrix exponential. The shared representation matrix $\bfG_\star \in \R^{k \times \dx}$ is generated by sampling uniformly from the space of row-orthonormal matrices in $\R^{k \times \dx}$.

We consider two settings for the covariance matrices $\bSigma_{\bfx, {\msf{s}}}$;  the \textit{low-anisotropic}, and the \textit{high-anisotropic} settings. In the low-anisotropic setting, we define $\bE =  5\, \bI_{\dx} + \bN$ where $\bN \in \R^{\dx \times \dx}$ has i.i.d. $\normal(0,1)$ entries, and set $\bSigma_{\bfx,\msf{s}} = 0.5\,(\bE + \bE^\top)$. For the high-anisotropic setting, we first sample uniformly a rotation matrix $\bO \in \R^{\dx \times \dx}$ and set $\bSigma_{\bfx, ,\msf{s}} = \bO \bD \bO^\top$ where $\bD = \diag(\texttt{logspace}(0,5,\dx))$. In the experiments for the main paper, we always consider the high-anisotropic setting.

In the following experiments, in addition to the data generation process in equation~\eqref{eq:bern_data} used in the experiment in the main paper, we also consider a Gaussian data setup where samples for task $\msf{s}$ are generated according to
\begin{align}
    \label{eq:bern_data_gaussian}
    &\bfy_i ^{\msf{s}} = \mbf{F}_\star^{\msf{s}} \mbf{G}_\star \bfx_i ^{\msf{s}} + \bveps_i ^{\msf{s}}, \quad  \bfx_{i}^{\msf{s}} \sim \normal(\mathbf{0}, \bSigma_{\bfx, \msf{s}}),\quad \bveps_i ^{\msf{s}} \iidsim \normal(0, \sigma_{\ep,\msf{s}} \bfI_{\dy}), \quad \msf{s} \in \{\msf{test}, \msf{train}\}.
\end{align}

\subsection{Additional Experiments}

\subsubsection{Effect of Batch Normalization}
We used the same experiment setting described in Section \ref{sec:numerical_validation} to generate the plots in Figure \ref{fig:appendix_headtohead_plots}. Explicitly, we use data dimension \(\dx = 100\), task dimension \(\dy = 15\), and representation dimension \(k = 8\). We use the same learning rate $10^{-2}$ for each optimizer except for \NGD, in which we used $10^{-4}$. The batch size is \(1024\). In Figure~\ref{fig:batchnorm_subpace_dist} we considered the Uniform data \eqref{eq:bern_data} with high anisotropy. Here, we consider the other three setting: Uniform data \eqref{eq:bern_data} with low anisotropy, Gaussian data \eqref{eq:bern_data_gaussian} with low anisotropy, and Gaussian data \eqref{eq:bern_data_gaussian} with high anisotropy.


\begin{figure*}[ht]
\centering
\begin{minipage}{0.32\textwidth}
    \centering
    \includegraphics[width=\linewidth]{figs/combined_metrics_2000iters_10exps_bern_lowanistropy_100dx_15dy_8repsize_1024bsz.pdf}
\end{minipage}%
\begin{minipage}{0.32\textwidth}
    \centering
    \includegraphics[width=\linewidth]{figs/combined_metrics_2000iters_10exps_gauss_lowanistropy_100dx_15dy_8repsize_1024bsz.pdf}
\end{minipage}%
\begin{minipage}{0.32\textwidth}
    \centering
    \includegraphics[width=\linewidth]{figs/combined_metrics_2000iters_10exps_gauss_highanistropy_100dx_15dy_8repsize_1024bsz.pdf}
\end{minipage}%
\caption{
The effect of batch normalization (on \texttt{AMGD}) vs. \KFAC in our experiment settings
\textbf{(Left)} Uniform with low anisotropy.
\textbf{(Middle)} Gaussian with low anisotropy.
\textbf{(Right)} Gaussian with high anisotropy.
}
\label{fig:batchnorm_full}
\end{figure*}


As discussed in the main paper \Cref{sec:lin_rep_BN}, we expect \texttt{AMGD} with batch-norm to converge in training loss but to perform poorly with respect to the subspace distance from the optimal in settings in the case with high anistropy \textbf{(Right)}. However, in the experiment settings with low anisotropy \textbf{(Left and Center)}, we expect reasonable performance from this algorithm because $\rowsp(\bfG_\star \bSigma_{\bfx})$ is close to the target $\rowsp(\bfG_\star)$.


\ifshort
  \subsubsection{Learning Rate Sweep}
  \label{sec:sweep}
    We further test the performance of each learning algorithm at different learning rates from \(10^{-6}, 10^{-5.5}, \ldots, 10^{-0.5}, 10^{0}\), with results shown in Figure~\ref{fig:lr_sweep}, where we plot the subspace distance at \(1000\) iterations for different algorithms.
    If the algorithm encounters numerical instability, then we report the subspace distance as the maximal value of \(1.0\). We observe that \KFAC and \DFW  coverage to a solution with small subspace distance to the true representation for a wide range of step sizes, whereas the set of suitable learning rates for other algorithms is much narrower. Furthermore, we observe the poor performance of various algorithms in \Cref{fig:headtohead} and \Cref{fig:appendix_headtohead_plots} is not due to specific choice of learning rate.
    
    
    \begin{figure}[ht]
    \centering
    \includegraphics[width=0.4\linewidth]{figs/lr_sweep_1000iters_10exps_bern_highanistropy_100dx_15dy_8repsize_1024bsz.pdf}
    \caption{The subspace distance of representations learned by different algorithms after $1000$ iterations and the true representation as a function of learning rate.}
    \label{fig:lr_sweep}
    \end{figure}
\else
\fi

\subsubsection{Head-to-Head Experiments}
We again consider the same experimental setting used for Figure~\ref{fig:headtohead}. In particular, we use data dimension \(\dx = 100\), task dimension \(\dy = 15\), and representation dimension \(k = 8\). We use the same learning rate $10^{-2}$ for each optimizer except for \NGD optimizer, in which we used $10^{-4}$. The batch size is \(1024\). In Figure~\ref{fig:headtohead} we considered the Uniform data \eqref{eq:bern_data} with high anisotropy. Here, we consider the other three setting: Uniform data \eqref{eq:bern_data} with low anisotropy, Gaussian data \eqref{eq:bern_data_gaussian} with low anisotropy, and Gaussian data \eqref{eq:bern_data_gaussian} with high anisotropy.  We plot the training loss, subspace distance to the ground truth shared representation, and the transfer loss obtained by different algorithms. See \Cref{fig:appendix_headtohead_plots}. We observe that in all three settings, various algorithms converge in training loss. In the case with high anisotropy (second row), methods other than \KFAC do not converge to the optimal representation in subspace distance and transfer loss. However, in the low anisotropy settings (first and third rows), the performance of other algorithms also improve, but are notably still suboptimal relative to \KFAC, confirming the theoretical results showing that anisotropy is a root cause behind the sub-optimality of prior algorithms and analysis.
\begin{figure*}
\centering
\begin{minipage}{0.32\textwidth}
    \centering
    \includegraphics[width=\linewidth]{figs/headtohead_figs/headtohead_train_loss_2000iters_10exps_gauss_lowanistropy_100dx_15dy_8repsize_1024bsz.pdf}
\end{minipage}%
\begin{minipage}{0.32\textwidth}
    \centering
    \includegraphics[width=\linewidth]{figs/headtohead_figs/headtohead_subspace_dist_2000iters_10exps_gauss_lowanistropy_100dx_15dy_8repsize_1024bsz.pdf}
\end{minipage}%
\begin{minipage}{0.32\textwidth}
    \centering
    \includegraphics[width=\linewidth]{figs/headtohead_figs/headtohead_transfer_loss_2000iters_10exps_gauss_lowanistropy_100dx_15dy_8repsize_1024bsz.pdf}
\end{minipage}
\caption{Gaussian with low anisotropy}

\begin{minipage}{0.32\textwidth}
    \centering
    \includegraphics[width=\linewidth]{figs/headtohead_figs/headtohead_train_loss_2000iters_10exps_gauss_highanistropy_100dx_15dy_8repsize_1024bsz.pdf}
\end{minipage}%
\begin{minipage}{0.32\textwidth}
    \centering
    \includegraphics[width=\linewidth]{figs/headtohead_figs/headtohead_subspace_dist_2000iters_10exps_gauss_highanistropy_100dx_15dy_8repsize_1024bsz.pdf}
\end{minipage}%
\begin{minipage}{0.32\textwidth}
    \centering
    \includegraphics[width=\linewidth]{figs/headtohead_figs/headtohead_transfer_loss_2000iters_10exps_gauss_highanistropy_100dx_15dy_8repsize_1024bsz.pdf}
\end{minipage}
\caption{Gaussian with high anisotropy}

\begin{minipage}{0.32\textwidth}
    \centering
    \includegraphics[width=\linewidth]{figs/headtohead_figs/headtohead_train_loss_2000iters_10exps_bern_lowanistropy_100dx_15dy_8repsize_1024bsz.pdf}
\end{minipage}%
\begin{minipage}{0.32\textwidth}
    \centering
    \includegraphics[width=\linewidth]{figs/headtohead_figs/headtohead_subspace_dist_2000iters_10exps_bern_lowanistropy_100dx_15dy_8repsize_1024bsz.pdf}
\end{minipage}%
\begin{minipage}{0.32\textwidth}
    \centering
    \includegraphics[width=\linewidth]{figs/headtohead_figs/headtohead_transfer_loss_2000iters_10exps_bern_lowanistropy_100dx_15dy_8repsize_1024bsz.pdf}
\end{minipage}
\caption{Bernoulli with low anisotropy}

\caption{From \textbf{left} to \textbf{right}: the training loss, subspace distance, and transfer loss induced by various algorithms on a linear representation learning task.}
\label{fig:appendix_headtohead_plots}

\end{figure*}



\end{document}
