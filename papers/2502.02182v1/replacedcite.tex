\section{Related work}
The prediction of individual cell states from single microscopy images has found considerable interest in the literature. For instance____ and ____ showed that deep learning-based models allow to classify \emph{discrete} cell cycle states such as the G1, S, or G2 phase from static multichannel images of cells. In particular, ____ explored the use of brightfield and phase-imaging data for cell cycle classification without fluorescence labeling. However, accurately annotating discrete cell cycle stages highly depends on the imaging modality with nuclear stains like DAPI or Hoechst providing clear features, while phase or brightfield imaging require substantial more manual annotation expertise. 
Beyond static images, several studies have investigated how the \emph{temporal information} of images sequences can be leveraged to capture dynamic cell behaviors____. 
However, most approaches have so far been focused on well-defined cell phases such as mitosis, where large morphological changes facilitate annotation and prediction____. A notable exception is ____ which applies dynamic time warping (DTW) to features from an unsupervised autoencoder, enabling to temporally align cell trajectories and to continuously predict cycle states across interphase. 
\emph{Sequence models} such as recurrent neural networks and transformers____ have been shown to be effective for modeling temporal data____, with \emph{state-space models} such as Mamba____ having recently gained considerable interest due to their training and inference efficiency. 
As the cell cycle is a continuous causal process, sequence models appear to be a natural fit for our task. However, studies that address this question systematically remain missing, with the closest work being____ which uses recurrent neural networks for cell cycle prediction, yet for the case of a small number of discrete cell states.