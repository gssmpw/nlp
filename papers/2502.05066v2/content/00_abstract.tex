\begin{abstract}
State-of-the-art visual generation models, such as Diffusion Models (DMs) and Vision Auto-Regressive Models (VARs), produce highly realistic images. While prior work has successfully mitigated Not Safe For Work (NSFW) content in the visual domain, we identify a novel threat: the generation of NSFW text embedded within images. This includes offensive language, such as insults, racial slurs, and sexually explicit terms, posing significant risks to users.
We show that all state-of-the-art DMs (e.g., SD3, Flux, DeepFloyd IF) and VARs (e.g., Infinity) are vulnerable to this issue. Through extensive experiments, we demonstrate that existing mitigation techniques, effective for visual content, fail to prevent harmful text generation while substantially degrading benign text generation.
As an initial step toward addressing this threat, we explore safety fine-tuning of the text encoder underlying major DM architectures using a customized dataset. Thereby, we suppress NSFW generation while preserving overall image and text generation quality.
Finally, to advance research in this area, we introduce ToxicBench, an open-source benchmark for evaluating NSFW text generation in images. ToxicBench provides a curated dataset of harmful prompts, new metrics, and an evaluation pipeline assessing both NSFW-ness and generation quality. Our benchmark aims to guide future efforts in mitigating NSFW text generation in text-to-image models and is available at \href{https://github.com/sprintml/ToxicBench}{https://github.com/sprintml/ToxicBench}.


\end{abstract}
