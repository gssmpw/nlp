\section{Related Work}
% \label{sec: related work}

% TODO:
% - Consolidate more references here 
% - Improve the writings in this section
% \subsection{Gene-Gene Interaction Prediction Model} 
% \label{subsec: predictive model reference}
% A critical component of an active learning system is a data-driven model capable of predicting unknown gene-gene interactions along with associated uncertainties. Based on the predictions and associated uncertainties, the next batch of experiments is determined and the results of these experiments are fed back to further improve the predictive performance of the model. For predicting gene-gene interactions, multiple models have been proposed in the literature. A popular class of techniques relies on the low-rank structure of the underlying interaction matrix and employs matrix factorization techniques to predict missing values \cite{gordon2020quantitative}. With recent progress in applying deep learning models to related problems in drug discovery
% \cite{zhu2021prediction, pham2021deep, lotfollahi2019scgen, hetzel2022predicting}, various models have been proposed for RNA sequence perturbation predictions, including a graph-based model GEARS \cite{roohani2024predicting} and a multilevel deep learning method \cite{bai2024attentionpert}. %Though both \cite{roohani2024predicting} and \cite{bai2024attentionpert} utilizes graph neural networks, our representation model differ in: 1) Our model is built to utilize hetergeneous knowled

% Various deep learning methods have been proposed to identify epistatic interactions between Single Nucleotide Polymorphism (SNPs) observed to be associated with traits in Genome-Wide Association Study (GWAS) experiments \cite{cui2022gene,gracca2024distributed}. In these studies, the fundamental unit of interaction is a single SNP, which may have pleiotropic effects over the lifecycle of an organism. Hence, it cannot generally be determined whether the effect observed to be associated with SNP interactions is mediated through the action of one or multiple genes. Moreover, non-perturbative, observational data collection, as is typically used in GWAS experiments, results in relatively low power to detect higher-order SNP interactions to statistical significance. 

% \subsection{Computational Methods for Experimental Design} 
% \label{subsec: related work computational}

% One natural approach to the double knockdown experimental design problem is based on matrix factorization \cite{karimi2012exploiting}  where the interaction matrix is viewed as a reward matrix. This is a standard methodology for inferring missing observations in relational data sets (e.g., recommender systems) and has been used previously in the study of gene-to-gene interactions \cite{vzitnik2015data, norman2019exploring}. %However, due to budget constraints, an approach such as matrix factorization that only utilizes the observed experimental data is not guaranteed to efficiently explore the design space, and additional information is needed to reduce the complexity of the problem. 
% Based on GEARS \cite{roohani2024predicting}, \cite{huang2024sequential} show that incorporating additional information such as Protein structure, Protein-protein interaction network (PPI) into the predictive model can significantly improve the overall performance of the active learning framework for the experimental design of a single gene perturbation. More recently, there has also been work that utilizes a large language model to design experiments \cite{huang2024crispr, roohani2024biodiscoveryagent}. Although the initial results show benefit, these works are limited to small-scale experiment design due to API limitations and volatility of the language models' responses in the sequential setting.  

%Deep Active Learning (DeepAL) (\cite{ren2021survey}) is an emerging field in machine learning that aims to combine the representation power of deep learning and sample-efficiency of active learning to address various real world applications including drug discovery (\cite{Bailey_2023}), image classification \cite{ranganathan2017deep}, medical image analysis \cite{smailagic2018medal} and link predictions on knowledge graphs \cite{ostapuk2019activelink}, where either labeled data are limited or the sample selection itself is of interest. 

% Biological systems are often represented as networks of interactions between fundamental molecular entities. Among these, gene regulatory networks (GRNs) are critical as they map the regulatory interactions between transcription factors (proteins) and genes, which control cellular function \cite{alon2007network, levine2005gene}. However, despite significant advances, the precise architecture of GRNs remains largely unresolved \cite{babu2004structure}. To address these gaps, experimental techniques that perturb key components of these networks are employed to generate data and validate hypotheses \cite{segal2003module}. Active learning methods based on quantifying uncertainty of gene interactions has been deployed in this setting to extend our knowledge about GRNs more effectively \cite{ud2016optimal, sverchkov2017review}.  

%While the framework is broadly applicable, we focus on the host gene knockdowns experiment design problem and evaluate our framework on a dataset that measures HIV replication in HeLa cells with esiRNA knockdown of all possible pairs of 356 human genes involved in HIV pathogenesis \cite{gordon2020quantitative}. 
%the proposed framework is broadly applicable to any experimental design problems with additional graph information accessible and the configuration space is a Cartesian product of two sets of control variables (e.g., all possible gene-to-gene pairs).