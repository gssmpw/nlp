\documentclass[unnumsec,webpdf,contemporary,large]{oup-authoring-template}
% \documentclass[preprint,12pt,authoryear]{elsarticle}
% \documentclass[a4paper]{article}
% \documentclass[a4paper]{report}
%\usepackage{minitoc}

\usepackage[utf8]{inputenc}
\usepackage[T1]{fontenc}
\usepackage{textcomp}
\usepackage[english]{babel}
\usepackage{amsmath, amssymb,amsthm}
\usepackage{hyperref}
\usepackage{graphicx}
\usepackage{caption}
\usepackage{subcaption}
\usepackage{multirow}
\usepackage{threeparttable} 
\usepackage{footnote}
\makesavenoteenv{table}
% \usepackage{algorithm}
\usepackage[linesnumbered,ruled,vlined]{algorithm2e}
% \usepackage{algpseudocode}

% figure support
\usepackage{import}
\usepackage{xifthen}
\pdfminorversion=7
\usepackage{transparent}
\usepackage{tikz}
\usetikzlibrary{fit,positioning}
\usepackage{etoolbox}
% Note: make sure all figures are available offine 
\graphicspath{ {./images/} }

\onecolumn


\setlength{\parindent}{0 in}

\pdfsuppresswarningpagegroup=1

\newcommand{\ones}{\mathbf 1}
\newcommand{\reals}{{\mbox{\bf R}}}
\newcommand{\integers}{{\mbox{\bf Z}}}
\newcommand{\symm}{{\mbox{\bf S}}}  % symmetric matrices

\newcommand{\nullspace}{{\mathcal N}}
\newcommand{\range}{{\mathcal R}}
\newcommand{\Rank}{\mathop{\bf Rank}}
%\newcommand{\Tr}{\mathop{\bf Tr}}
\newcommand{\diag}{\mathop{\bf diag}}
\newcommand{\card}{\mathop{\bf card}}
\newcommand{\rank}{\mathop{\bf rank}}
\newcommand{\conv}{\mathop{\bf conv}}
\newcommand{\prox}{\mathbf{prox}}

\newcommand{\Expect}{\mathop{\bf E{}}}
\newcommand{\var}{\mathop{\bf var{}}}
\newcommand{\Prob}{\mathop{\bf Prob}}
\newcommand{\Co}{{\mathop {\bf Co}}} % convex hull
\newcommand{\dist}{\mathop{\bf dist{}}}
%\newcommand{\argmin}{\mathop{\rm argmin}}
%\newcommand{\argmax}{\mathop{\rm argmax}}
\newcommand{\epi}{\mathop{\bf epi}} % epigraph
\newcommand{\Vol}{\mathop{\bf vol}}
\newcommand{\dom}{\mathop{\bf dom}} % domain
\newcommand{\intr}{\mathop{\bf int}}
%\newcommand{\sign}{\mathop{\bf sign}}

\newcommand{\cf}{{\it cf.}}
\newcommand{\eg}{{\it e.g.}}
\newcommand{\ie}{{\it i.e.}}
\newcommand{\etc}{{\it etc.}}

\newcommand{\todo}{{\bf TODO}}

\newcommand{\bone}{\boldsymbol{1}}
\newcommand{\balpha}{\boldsymbol{\alpha}}
\newcommand{\bbeta}{\boldsymbol{\beta}}
\newcommand{\bdelta}{\boldsymbol{\delta}}
\newcommand{\bepsilon}{\boldsymbol{\epsilon}}
\newcommand{\blambda}{\boldsymbol{\lambda}}
\newcommand{\bomega}{\boldsymbol{\omega}}
\newcommand{\bpi}{\boldsymbol{\pi}}
\newcommand{\bnu}{\boldsymbol{\nu}}
\newcommand{\bphi}{\boldsymbol{\phi}}
\newcommand{\bvphi}{\boldsymbol{\varphi}}
\newcommand{\bpsi}{\boldsymbol{\psi}}
\newcommand{\bsigma}{\boldsymbol{\sigma}}
\newcommand{\btheta}{\boldsymbol{\theta}}
\newcommand{\bzeta}{\boldsymbol{\zeta}}
\newcommand{\bxi}{\boldsymbol{\xi}}
\newcommand{\ba}{\boldsymbol{a}}
\newcommand{\bb}{\boldsymbol{b}}
\newcommand{\bc}{\boldsymbol{c}}
\newcommand{\bd}{\boldsymbol{d}}
\newcommand{\be}{\boldsymbol{e}}
\newcommand{\boldf}{\boldsymbol{f}}
\newcommand{\bg}{\boldsymbol{g}}
\newcommand{\bh}{\boldsymbol{h}}
\newcommand{\bi}{\boldsymbol{i}}
\newcommand{\bj}{\boldsymbol{j}}
\newcommand{\bk}{\boldsymbol{k}}
\newcommand{\bell}{\boldsymbol{\ell}}
\newcommand{\bp}{\boldsymbol{p}}
\newcommand{\br}{\boldsymbol{r}}
\newcommand{\bs}{\boldsymbol{s}}
\newcommand{\bt}{\boldsymbol{t}}
\newcommand{\bu}{\boldsymbol{u}}
\newcommand{\bv}{\boldsymbol{v}}
\newcommand{\bw}{\boldsymbol{w}}
\newcommand{\bx}{{\boldsymbol{x}}}
\newcommand{\by}{\boldsymbol{y}}
\newcommand{\bz}{\boldsymbol{z}}
\newcommand{\bA}{\boldsymbol{A}}
\newcommand{\bB}{\boldsymbol{B}}
\newcommand{\bC}{\boldsymbol{C}}
\newcommand{\bD}{\boldsymbol{D}}
\newcommand{\bE}{\boldsymbol{E}}
\newcommand{\bF}{\boldsymbol{F}}
\newcommand{\bG}{\boldsymbol{G}}
\newcommand{\bH}{\boldsymbol{H}}
\newcommand{\bI}{\boldsymbol{I}}
\newcommand{\bJ}{\boldsymbol{J}}
\newcommand{\bL}{\boldsymbol{L}}
\newcommand{\bM}{\boldsymbol{M}}
\newcommand{\bP}{\boldsymbol{P}}
\newcommand{\bQ}{\boldsymbol{Q}}
\newcommand{\bR}{\boldsymbol{R}}
\newcommand{\bS}{\boldsymbol{S}}
\newcommand{\bT}{\boldsymbol{T}}
\newcommand{\bU}{\boldsymbol{U}}
\newcommand{\bV}{\boldsymbol{V}}
\newcommand{\bW}{\boldsymbol{W}}
\newcommand{\bX}{\boldsymbol{X}}
\newcommand{\bY}{\boldsymbol{Y}}
\newcommand{\bZ}{\boldsymbol{Z}}

% new theorems
% \newtheorem{theorem}{Theorem}
%\newtheorem*{proof}{Proof}

% usepackages
\usepackage{amsmath}
\usepackage{amsfonts}
\usepackage{textcomp} % for \textlangle and \textrangle macros
\newcommand{\qdist}[1]{\ifmmode\langle#1\rangle\else\textlangle#1\textrangle\fi}
\usepackage{xcolor}
\usepackage{algorithm} % for algorithms
\usepackage{algpseudocode} % for pseudocode
\usepackage{comment} % for large comments
\usepackage{bbm}
\usepackage{dsfont}
\usepackage{subfigure}
\usepackage{bm}
\usepackage{booktabs} % For better table lines
\usepackage{array} % For better column formatting
%\usepackage{appendix}
%\usepackage[english]{babel}
%\usepackage{amsthm}
\usepackage{graphicx} % for graphs






\begin{document}
\journaltitle{ISMB 2025 Proceeding}
\DOI{DOI}
\copyrightyear{2025}
\pubyear{2015}
\access{}
\appnotes{Paper}

\firstpage{1}

\title[Deep Active Learning for Host Targeted Therapeutics (Supplementary)]{Supplementary: Deep Active Learning based Experimental Design to Uncover Synergistic Genetic Interactions for Host Targeted Therapeutics}


\author[1,$\ast$]{Haonan Zhu}
\author[1]{Mary Silva}
\author[1]{Jose Cadena} 
\author[1]{Braden Soper}
\author[2,3]{Michał Lisicki}
\author[4]{Braian Peetoom}
\author[4]{Sergio E. Baranzini} 
\author[1]{Shankar Sundaram}
\author[1]{Priyadip Ray}
\author[1]{Jeff Drocco}
% Author affiliation
\authormark{Haonan Zhu et al.}

\address[1]{\orgname{Lawrence Livermore National Laboratory}, \orgaddress{\street{7000 East Ave}, \postcode{94550}, \state{CA}, \country{USA}}}
\address[2]{ \orgname{University of Guelph}, \orgaddress{\street{50 Stone Rd E}, \postcode{N1G 2W1}, \state{ON}, \country{Canada}}}
\address[3]{  \orgname{Vector Institute}, \orgaddress{\street{108 College St W1140}, \postcode{M5G 0C6}, \state{ON}, \country{Canada}}}
\address[4]{  \orgname{University of California San Francisco}, \orgaddress{\street{675 Nelson Rising Lane}, \postcode{94158}, \state{CA}, \country{USA}}}

\corresp[$\ast$]{Corresponding author. \href{email:zhu18@llnl.gov}{zhu18@llnl.gov}}

\begin{abstract}
\mbox{}
\end{abstract}

\maketitle

\section{Additional Experimental Results}
\label{sec: additional experimental results}

In this section, we provide additional experimental results on the HIV dataset \cite{gordon2020quantitative}. 

\subsection{Results for Batch Size=200}
\label{subsec: batch size 200}

In this subsection, we provide additional results where we run all algorithms for $33$ rounds (including the initial round of random selection), and at each round each algorithm selects $200$ gene pairs for observation,

\begin{figure}[h]
    \centering
    \begin{subfigure}{0.45\columnwidth}\hspace*{-1em}
        \centering
		\includegraphics[width=\textwidth]{images/Coverage_400_batchsize=200.pdf}
		\caption{Coverage@400} 
    \end{subfigure}
    \begin{subfigure}{0.45\columnwidth}
        \centering
		\includegraphics[width=\textwidth]{images/MAE_batch_size=200.pdf}
		\caption{Mean Absolute Errors on Unseen Data}
    \end{subfigure}  
    \caption{Comparison among the different acquisition strategies, and results are summarized over $20$ replicates. Optimism with $10\%$ quantiles achieves the best performance in terms of coverage at terminal phase. Maximum variance strategy performs the best in learning a generalizable model that predicts viral-replicates on unseen data due to an emphasis on exploration.}
    \label{fig: acquisition comparisons batsize 200}
\end{figure}

\begin{figure}[h]
    \centering
    \begin{subfigure}{0.45\columnwidth}\hspace*{-1em}
        \centering
		\includegraphics[width=\textwidth]{images/Coverage_400_ablations_batchsize=200.pdf}
		\caption{Coverage@400} 
    \end{subfigure}
    \begin{subfigure}{0.45\columnwidth}
        \centering
		\includegraphics[width=\textwidth]{images/MASE_batch_size=200_ablations.pdf}
		\caption{Mean Absolute Errors on Unseen Data}
    \end{subfigure}  
    \caption{Abalation studies to validate our proposed approach, and results are summarized over $20$ replicates.}
    \label{fig: ablation studies batsize 200}
\end{figure}

\subsection{Results for Batch Size=800}
\label{subsec: batch size 800}

In this subsection, we provide additional results where we run all algorithms for $9$ rounds (including the initial round of random selection), and at each round each algorithm selects $800$ gene pairs for observation,

\begin{figure}[h]
    \centering
    \begin{subfigure}{0.45\columnwidth}\hspace*{-1em}
        \centering
		\includegraphics[width=\textwidth]{images/Coverage_400_batchsize=800.pdf}
		\caption{Coverage@400} 
    \end{subfigure}
    \begin{subfigure}{0.45\columnwidth}
        \centering
		\includegraphics[width=\textwidth]{images/MAE_batch_size=800.pdf}
		\caption{Mean Absolute Errors on Unseen Data}
    \end{subfigure}  
    \caption{Comparison among the different acquisition strategies, and results are summarized over $20$ replicates. Optimism with $10\%$ quantiles achieves the best performance in terms of coverage at terminal phase. Maximum variance strategy performs the best in learning a generalizable model that predicts viral-replicates on unseen data due to an emphasis on exploration.}
    \label{fig: acquisition comparisons batsize 800}
\end{figure}

\begin{figure}[h]
    \centering
    \begin{subfigure}{0.45\columnwidth}\hspace*{-1em}
        \centering
		\includegraphics[width=\textwidth]{images/Coverage_400_ablations_batchsize=800.pdf}
		\caption{Coverage@400} 
    \end{subfigure}
    \begin{subfigure}{0.45\columnwidth}
        \centering
		\includegraphics[width=\textwidth]{images/MASE_batch_size=800_ablations.pdf}
		\caption{Mean Absolute Errors on Unseen Data}
    \end{subfigure}  
    \caption{Abalation studies to validate our proposed approach, and results are summarized over $20$ replicates.}
    \label{fig: ablation studies batsize 800}
\end{figure}


\bibliographystyle{plain}
% \bibliographystyle{elsarticle-harv} 
\bibliography{main}


\end{document}
