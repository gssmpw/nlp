\documentclass[numsec,webpdf,contemporary,large]{oup-authoring-template}
% \documentclass[preprint,12pt,authoryear]{elsarticle}
% \documentclass[a4paper]{article}
% \documentclass[a4paper]{report}
%\usepackage{minitoc}

\usepackage[utf8]{inputenc}
\usepackage[T1]{fontenc}
\usepackage{textcomp}
\usepackage[english]{babel}
\usepackage{amsmath, amssymb,amsthm}
\usepackage{hyperref}
\usepackage{graphicx}
\usepackage{caption}
\usepackage{subcaption}
\usepackage{multirow}
\usepackage{threeparttable} 
\usepackage{footnote}
\makesavenoteenv{table}
% \usepackage{algorithm}
\usepackage[linesnumbered,ruled,vlined]{algorithm2e}
% \usepackage{algpseudocode}

% figure support
\usepackage{import}
\usepackage{xifthen}
\pdfminorversion=7
\usepackage{transparent}
\usepackage{tikz}
\usetikzlibrary{fit,positioning}
% Note: make sure all figures are available offine 
\graphicspath{ {./images/} }


\newcommand{\MS}[1]{ {\color{magenta} (MS: #1)}}
\newcommand{\JC}[1]{ {\color{blue} (JC: #1)}}
\newcommand{\HZ}[1]{ {\color{cyan} (HZ: #1)}}

\setlength{\parindent}{0 in}

\pdfsuppresswarningpagegroup=1

\usepackage{bbm}
\usepackage{graphicx}
\usepackage{amsmath,amssymb,amsthm,amsfonts}

\usepackage{paralist}
\usepackage{bm}
\usepackage{xspace}
\usepackage{url}
\usepackage{prettyref}
\usepackage{boxedminipage}
\usepackage{wrapfig}
\usepackage{ifthen}
\usepackage{color}
\usepackage{xspace}

\newcommand{\ii}{{\sc Indicator-Instance}\xspace}
\newcommand{\midd}{{\sf mid}}


\usepackage{amsmath,amsthm,amsfonts,amssymb}
\usepackage{mathtools}
\usepackage{graphicx}


% \usepackage{fullpage}

\usepackage{nicefrac}

\newtheorem{inftheorem}{Informal Theorem}
\newtheorem{claim}{Claim}
\newtheorem*{definition*}{Definition}
\newtheorem{example}{Example}

\DeclareMathOperator*{\argmax}{arg\,max}
\DeclareMathOperator*{\argmin}{arg\,min}
\usepackage{subcaption}

\newtheorem{problem}{Problem}
\usepackage[utf8]{inputenc}
\newcommand{\rank}{\mathsf{rank}}
\newcommand{\tr}{\mathsf{Tr}}
\newcommand{\tv}{\mathsf{TV}}
\newcommand{\opt}{\mathsf{OPT}}
\newcommand{\rr}{\textsc{R}\space}
\newcommand{\alg}{\textsf{Alg}\space}
\newcommand{\sd}{\textsf{sd}_\lambda}
\newcommand{\lblq}{\mathfrak{lq} (X_1)}
\newcommand{\diag}{\textsf{diag}}
\newcommand{\sign}{\textsf{sgn}}
\newcommand{\BC}{\texttt{BC} }
\newcommand{\MM}{\texttt{MM} }
\newcommand{\Nexp}{N_{\mathrm{exp}}}
\newcommand{\Nrep}{N_{\mathrm{replay}}}
\newcommand{\Drep}{D_{\mathrm{replay}}}
\newcommand{\Nsim}{N_{\mathrm{sim}}}
\newcommand{\piBC}{\pi^{\texttt{BC}}}
\newcommand{\piRE}{\pi^{\texttt{RE}}}
\newcommand{\piEMM}{\pi^{\texttt{MM}}}
\newcommand{\mmd}{\texttt{Mimic-MD} }
\newcommand{\RE}{\texttt{RE} }
\newcommand{\dem}{\pi^E}
\newcommand{\Rlint}{\mathcal{R}_{\mathrm{lin,t}}}
\newcommand{\Rlipt}{\mathcal{R}_{\mathrm{lip,t}}}
\newcommand{\Rlin}{\mathcal{R}_{\mathrm{lin}}}
\newcommand{\Rlip}{\mathcal{R}_{\mathrm{lip}}}
\newcommand{\Rmax}{R_{\mathrm{max}}}
\newcommand{\Rall}{\mathcal{R}_{\mathrm{all}}}
\newcommand{\Rdet}{\mathcal{R}_{\mathrm{det}}}
\newcommand{\Fmax}{F_{\mathrm{max}}}
\newcommand{\Nmax}{\mathcal{N}_{\mathrm{max}}}
\newcommand{\piref}{\pi^{\mathrm{ref}}}
\newcommand{\green}{\text{\color{green!75!black} green}\;}
\newcommand{\thetaBC}{\widehat{\theta}^{\textsf{BC}}}
\newcommand{\ent}{\mathcal{E}_{\Theta,n,\delta}}
\newcommand{\eNt}{\mathcal{E}_{\Theta_t,\Nexp,\delta}}
\newcommand{\eNtH}{\mathcal{E}_{\Theta_t,\Nexp,\delta/H}}

\newcommand{\eref}[1]{(\ref{#1})}
\newcommand{\sref}[1]{Sec. \ref{#1}}
\newcommand{\dr}{\widehat{d}_{\mathrm{replay}}}
\newcommand{\figref}[1]{Fig. \ref{#1}}

\usepackage{xcolor}
\definecolor{expert}{HTML}{008000}
\definecolor{error}{HTML}{f96565}
\newcommand{\GKS}[1]{{\textcolor{violet}{\textbf{GKS: #1}}}}
\newcommand{\Q}[1]{{\textcolor{red}{\textbf{Question #1}}}}
\newcommand{\ZSW}[1]{{\textcolor{orange}{\textbf{ZSW: #1}}}}
\newcommand{\JAB}[1]{{\textcolor{teal}{\textbf{JAB: #1}}}}
\newcommand{\jab}[1]{{\textcolor{teal}{\textbf{JAB: #1}}}}
\newcommand{\SAN}[1]{{\textcolor{blue}{\textbf{SC: #1}}}}
\newcommand{\scnote}[1]{\SAN{#1}}
\newcommand{\norm}[1]{\left\lVert #1 \right\rVert}

\usepackage{color-edits}
\addauthor{sw}{blue}

\usepackage{thmtools}
\usepackage{thm-restate}

\usepackage{tikz}
\usetikzlibrary{arrows,calc} 
\newcommand{\tikzAngleOfLine}{\tikz@AngleOfLine}
\def\tikz@AngleOfLine(#1)(#2)#3{%
\pgfmathanglebetweenpoints{%
\pgfpointanchor{#1}{center}}{%
\pgfpointanchor{#2}{center}}
\pgfmathsetmacro{#3}{\pgfmathresult}%
}

\declaretheoremstyle[
    headfont=\normalfont\bfseries, 
    bodyfont = \normalfont\itshape]{mystyle} 
\declaretheorem[name=Theorem,style=mystyle,numberwithin=section]{thm}

% \usepackage{algorithm}
% \usepackage{algorithmic}
\usepackage[linesnumbered,algoruled,boxed,lined,noend]{algorithm2e}

\usepackage{listings}
\usepackage{amsmath}
\usepackage{amsthm}
\usepackage{tikz}
\usepackage{caption}
\usepackage{mdwmath}
\usepackage{multirow}
\usepackage{mdwtab}
\usepackage{eqparbox}
\usepackage{multicol}
\usepackage{amsfonts}
\usepackage{tikz}
\usepackage{multirow,bigstrut,threeparttable}
\usepackage{amsthm}
\usepackage{bbm}
\usepackage{epstopdf}
\usepackage{mdwmath}
\usepackage{mdwtab}
\usepackage{eqparbox}
\usetikzlibrary{topaths,calc}
\usepackage{latexsym}
\usepackage{cite}
\usepackage{amssymb}
\usepackage{bm}
\usepackage{amssymb}
\usepackage{graphicx}
\usepackage{mathrsfs}
\usepackage{epsfig}
\usepackage{psfrag}
\usepackage{setspace}
\usepackage[%dvips,
            CJKbookmarks=true,
            bookmarksnumbered=true,
            bookmarksopen=true,
%						bookmarks=false,
            colorlinks=true,
            citecolor=red,
            linkcolor=blue,
            anchorcolor=red,
            urlcolor=blue
            ]{hyperref}
%\usepackage{algorithm}
\usepackage[linesnumbered,algoruled,boxed,lined]{algorithm2e}
\usepackage{algpseudocode}
\usepackage{stfloats}
\RequirePackage[numbers]{natbib}

\usepackage{comment}
\usepackage{mathtools}
\usepackage{blkarray}
\usepackage{multirow,bigdelim,dcolumn,booktabs}

\usepackage{xparse}
\usepackage{tikz}
\usetikzlibrary{calc}
\usetikzlibrary{decorations.pathreplacing,matrix,positioning}

\usepackage[T1]{fontenc}
\usepackage[utf8]{inputenc}
\usepackage{mathtools}
\usepackage{blkarray, bigstrut}
\usepackage{gauss}

\newenvironment{mygmatrix}{\def\mathstrut{\vphantom{\big(}}\gmatrix}{\endgmatrix}

\newcommand{\tikzmark}[1]{\tikz[overlay,remember picture] \node (#1) {};}

%% Adapted form https://tex.stackexchange.com/questions/206898/braces-for-cases-in-tabular-environment/207704#207704
\newcommand*{\BraceAmplitude}{0.4em}%
\newcommand*{\VerticalOffset}{0.5ex}%  
\newcommand*{\HorizontalOffset}{0.0em}% 
\newcommand*{\blocktextwid}{3.0cm}%
\NewDocumentCommand{\InsertLeftBrace}{%
	O{} % #1 = draw options
	O{\HorizontalOffset,\VerticalOffset} % #2 = optional brace shift options
	O{\blocktextwid} % #3 = optional text width
	m   % #4 = top tikzmark
	m   % #5 = bottom tikzmark
	m   % #6 = node text
}{%
	\begin{tikzpicture}[overlay,remember picture]
	\coordinate (Brace Top)    at ($(#4.north) + (#2)$);
	\coordinate (Brace Bottom) at ($(#5.south) + (#2)$);
	\draw [decoration={brace, amplitude=\BraceAmplitude}, decorate, thick, draw=black, #1]
	(Brace Bottom) -- (Brace Top) 
	node [pos=0.5, anchor=east, align=left, text width=#3, color=black, xshift=\BraceAmplitude] {#6};
	\end{tikzpicture}%
}%
\NewDocumentCommand{\InsertRightBrace}{%
	O{} % #1 = draw options
	O{\HorizontalOffset,\VerticalOffset} % #2 = optional brace shift options
	O{\blocktextwid} % #3 = optional text width
	m   % #4 = top tikzmark
	m   % #5 = bottom tikzmark
	m   % #6 = node text
}{%
	\begin{tikzpicture}[overlay,remember picture]
	\coordinate (Brace Top)    at ($(#4.north) + (#2)$);
	\coordinate (Brace Bottom) at ($(#5.south) + (#2)$);
	\draw [decoration={brace, amplitude=\BraceAmplitude}, decorate, thick, draw=black, #1]
	(Brace Top) -- (Brace Bottom) 
	node [pos=0.5, anchor=west, align=left, text width=#3, color=black, xshift=\BraceAmplitude] {#6};
	\end{tikzpicture}%
}%
\NewDocumentCommand{\InsertTopBrace}{%
	O{} % #1 = draw options
	O{\HorizontalOffset,\VerticalOffset} % #2 = optional brace shift options
	O{\blocktextwid} % #3 = optional text width
	m   % #4 = top tikzmark
	m   % #5 = bottom tikzmark
	m   % #6 = node text
}{%
	\begin{tikzpicture}[overlay,remember picture]
	\coordinate (Brace Top)    at ($(#4.west) + (#2)$);
	\coordinate (Brace Bottom) at ($(#5.east) + (#2)$);
	\draw [decoration={brace, amplitude=\BraceAmplitude}, decorate, thick, draw=black, #1]
	(Brace Top) -- (Brace Bottom) 
	node [pos=0.5, anchor=south, align=left, text width=#3, color=black, xshift=\BraceAmplitude] {#6};
	\end{tikzpicture}%
}%

\usetikzlibrary{patterns}

\definecolor{cof}{RGB}{219,144,71}
\definecolor{pur}{RGB}{186,146,162}
\definecolor{greeo}{RGB}{91,173,69}
\definecolor{greet}{RGB}{52,111,72}

% provide arXiv number if available:
% \arxiv{cs.IT/1502.00326}

% put your definitions there:

%\newtheorem{remark}{Remark} \def\remref#1{Remark~\ref{#1}}
%\newtheorem{conjecture}{Conjecture} \def\remref#1{Remark~\ref{#1}}
%\newtheorem{example}{Example}

%\theorembodyfont{\itshape}
%\newtheorem{theorem}{Theorem}
%\newtheorem{proposition}{Proposition}
%\newtheorem{lemma}{Lemma} \def\lemref#1{Lemma~\ref{#1}}
%\newtheorem{corollary}{Corollary}


%\theorembodyfont{\rmfamily}
%\newtheorem{definition}{Definition}
%\numberwithin{equation}{section}
% \theoremstyle{plain}
% \newtheorem{theorem}{Theorem}
% \newtheorem{Example}{Example}
% \newtheorem{lemma}{Lemma}
% \newtheorem{remark}{Remark}
% \newtheorem{corollary}{Corollary}
% \newtheorem{definition}{Definition}
% \newtheorem{conjecture}{Conjecture}
% \newtheorem{question}{Question}
% \newtheorem*{induction}{Induction Hypothesis}
% \newtheorem*{folklore}{Folklore}
% \newtheorem{assumption}{Assumption}

\def \by {\bar{y}}
\def \bx {\bar{x}}
\def \bh {\bar{h}}
\def \bz {\bar{z}}
\def \cF {\mathcal{F}}
\def \bP {\mathbb{P}}
\def \bE {\mathbb{E}}
\def \bR {\mathbb{R}}
\def \bF {\mathbb{F}}
\def \cG {\mathcal{G}}
\def \cM {\mathcal{M}}
\def \cB {\mathcal{B}}
\def \cN {\mathcal{N}}
\def \var {\mathsf{Var}}
\def\1{\mathbbm{1}}
\def \FF {\mathbb{F}}


\newenvironment{keywords}
{\bgroup\leftskip 20pt\rightskip 20pt \small\noindent{\bfseries
Keywords:} \ignorespaces}%
{\par\egroup\vskip 0.25ex}
\newlength\aftertitskip     \newlength\beforetitskip
\newlength\interauthorskip  \newlength\aftermaketitskip















%%%%%%%%%%%%%%%%%%%%%%%%%%%% by Wu %%%%%%%%%%%%%%%%%%%%%%%%%%%%
\usepackage{xspace}

\newcommand{\Lip}{\mathrm{Lip}}
\newcommand{\stepa}[1]{\overset{\rm (a)}{#1}}
\newcommand{\stepb}[1]{\overset{\rm (b)}{#1}}
\newcommand{\stepc}[1]{\overset{\rm (c)}{#1}}
\newcommand{\stepd}[1]{\overset{\rm (d)}{#1}}
\newcommand{\stepe}[1]{\overset{\rm (e)}{#1}}
\newcommand{\stepf}[1]{\overset{\rm (f)}{#1}}


\newcommand{\floor}[1]{{\left\lfloor {#1} \right \rfloor}}
\newcommand{\ceil}[1]{{\left\lceil {#1} \right \rceil}}

\newcommand{\blambda}{\bar{\lambda}}
\newcommand{\reals}{\mathbb{R}}
\newcommand{\naturals}{\mathbb{N}}
\newcommand{\integers}{\mathbb{Z}}
\newcommand{\Expect}{\mathbb{E}}
\newcommand{\expect}[1]{\mathbb{E}\left[#1\right]}
\newcommand{\Prob}{\mathbb{P}}
\newcommand{\prob}[1]{\mathbb{P}\left[#1\right]}
\newcommand{\pprob}[1]{\mathbb{P}[#1]}
\newcommand{\intd}{{\rm d}}
\newcommand{\TV}{{\sf TV}}
\newcommand{\LC}{{\sf LC}}
\newcommand{\PW}{{\sf PW}}
\newcommand{\htheta}{\hat{\theta}}
\newcommand{\eexp}{{\rm e}}
\newcommand{\expects}[2]{\mathbb{E}_{#2}\left[ #1 \right]}
\newcommand{\diff}{{\rm d}}
\newcommand{\eg}{e.g.\xspace}
\newcommand{\ie}{i.e.\xspace}
\newcommand{\iid}{i.i.d.\xspace}
\newcommand{\fracp}[2]{\frac{\partial #1}{\partial #2}}
\newcommand{\fracpk}[3]{\frac{\partial^{#3} #1}{\partial #2^{#3}}}
\newcommand{\fracd}[2]{\frac{\diff #1}{\diff #2}}
\newcommand{\fracdk}[3]{\frac{\diff^{#3} #1}{\diff #2^{#3}}}
\newcommand{\renyi}{R\'enyi\xspace}
\newcommand{\lpnorm}[1]{\left\|{#1} \right\|_{p}}
\newcommand{\linf}[1]{\left\|{#1} \right\|_{\infty}}
\newcommand{\lnorm}[2]{\left\|{#1} \right\|_{{#2}}}
\newcommand{\Lploc}[1]{L^{#1}_{\rm loc}}
\newcommand{\hellinger}{d_{\rm H}}
\newcommand{\Fnorm}[1]{\lnorm{#1}{\rm F}}
%% parenthesis
\newcommand{\pth}[1]{\left( #1 \right)}
\newcommand{\qth}[1]{\left[ #1 \right]}
\newcommand{\sth}[1]{\left\{ #1 \right\}}
\newcommand{\bpth}[1]{\Bigg( #1 \Bigg)}
\newcommand{\bqth}[1]{\Bigg[ #1 \Bigg]}
\newcommand{\bsth}[1]{\Bigg\{ #1 \Bigg\}}
\newcommand{\xxx}{\textbf{xxx}\xspace}
\newcommand{\toprob}{{\xrightarrow{\Prob}}}
\newcommand{\tolp}[1]{{\xrightarrow{L^{#1}}}}
\newcommand{\toas}{{\xrightarrow{{\rm a.s.}}}}
\newcommand{\toae}{{\xrightarrow{{\rm a.e.}}}}
\newcommand{\todistr}{{\xrightarrow{{\rm D}}}}
\newcommand{\eqdistr}{{\stackrel{\rm D}{=}}}
\newcommand{\iiddistr}{{\stackrel{\text{\iid}}{\sim}}}
%\newcommand{\var}{\mathsf{var}}
\newcommand\indep{\protect\mathpalette{\protect\independenT}{\perp}}
\def\independenT#1#2{\mathrel{\rlap{$#1#2$}\mkern2mu{#1#2}}}
\newcommand{\Bern}{\text{Bern}}
\newcommand{\Poi}{\mathsf{Poi}}
\newcommand{\iprod}[2]{\left \langle #1, #2 \right\rangle}
\newcommand{\Iprod}[2]{\langle #1, #2 \rangle}
\newcommand{\indc}[1]{{\mathbf{1}_{\left\{{#1}\right\}}}}
\newcommand{\Indc}{\mathbf{1}}
\newcommand{\regoff}[1]{\textsf{Reg}_{\mathcal{F}}^{\text{off}} (#1)}
\newcommand{\regon}[1]{\textsf{Reg}_{\mathcal{F}}^{\text{on}} (#1)}

\definecolor{myblue}{rgb}{.8, .8, 1}
\definecolor{mathblue}{rgb}{0.2472, 0.24, 0.6} % mathematica's Color[1, 1--3]
\definecolor{mathred}{rgb}{0.6, 0.24, 0.442893}
\definecolor{mathyellow}{rgb}{0.6, 0.547014, 0.24}


\newcommand{\red}{\color{red}}
\newcommand{\blue}{\color{blue}}
\newcommand{\nb}[1]{{\sf\blue[#1]}}
\newcommand{\nbr}[1]{{\sf\red[#1]}}

\newcommand{\tmu}{{\tilde{\mu}}}
\newcommand{\tf}{{\tilde{f}}}
\newcommand{\tp}{\tilde{p}}
\newcommand{\tilh}{{\tilde{h}}}
\newcommand{\tu}{{\tilde{u}}}
\newcommand{\tx}{{\tilde{x}}}
\newcommand{\ty}{{\tilde{y}}}
\newcommand{\tz}{{\tilde{z}}}
\newcommand{\tA}{{\tilde{A}}}
\newcommand{\tB}{{\tilde{B}}}
\newcommand{\tC}{{\tilde{C}}}
\newcommand{\tD}{{\tilde{D}}}
\newcommand{\tE}{{\tilde{E}}}
\newcommand{\tF}{{\tilde{F}}}
\newcommand{\tG}{{\tilde{G}}}
\newcommand{\tH}{{\tilde{H}}}
\newcommand{\tI}{{\tilde{I}}}
\newcommand{\tJ}{{\tilde{J}}}
\newcommand{\tK}{{\tilde{K}}}
\newcommand{\tL}{{\tilde{L}}}
\newcommand{\tM}{{\tilde{M}}}
\newcommand{\tN}{{\tilde{N}}}
\newcommand{\tO}{{\tilde{O}}}
\newcommand{\tP}{{\tilde{P}}}
\newcommand{\tQ}{{\tilde{Q}}}
\newcommand{\tR}{{\tilde{R}}}
\newcommand{\tS}{{\tilde{S}}}
\newcommand{\tT}{{\tilde{T}}}
\newcommand{\tU}{{\tilde{U}}}
\newcommand{\tV}{{\tilde{V}}}
\newcommand{\tW}{{\tilde{W}}}
\newcommand{\tX}{{\tilde{X}}}
\newcommand{\tY}{{\tilde{Y}}}
\newcommand{\tZ}{{\tilde{Z}}}

\newcommand{\sfa}{{\mathsf{a}}}
\newcommand{\sfb}{{\mathsf{b}}}
\newcommand{\sfc}{{\mathsf{c}}}
\newcommand{\sfd}{{\mathsf{d}}}
\newcommand{\sfe}{{\mathsf{e}}}
\newcommand{\sff}{{\mathsf{f}}}
\newcommand{\sfg}{{\mathsf{g}}}
\newcommand{\sfh}{{\mathsf{h}}}
\newcommand{\sfi}{{\mathsf{i}}}
\newcommand{\sfj}{{\mathsf{j}}}
\newcommand{\sfk}{{\mathsf{k}}}
\newcommand{\sfl}{{\mathsf{l}}}
\newcommand{\sfm}{{\mathsf{m}}}
\newcommand{\sfn}{{\mathsf{n}}}
\newcommand{\sfo}{{\mathsf{o}}}
\newcommand{\sfp}{{\mathsf{p}}}
\newcommand{\sfq}{{\mathsf{q}}}
\newcommand{\sfr}{{\mathsf{r}}}
\newcommand{\sfs}{{\mathsf{s}}}
\newcommand{\sft}{{\mathsf{t}}}
\newcommand{\sfu}{{\mathsf{u}}}
\newcommand{\sfv}{{\mathsf{v}}}
\newcommand{\sfw}{{\mathsf{w}}}
\newcommand{\sfx}{{\mathsf{x}}}
\newcommand{\sfy}{{\mathsf{y}}}
\newcommand{\sfz}{{\mathsf{z}}}
\newcommand{\sfA}{{\mathsf{A}}}
\newcommand{\sfB}{{\mathsf{B}}}
\newcommand{\sfC}{{\mathsf{C}}}
\newcommand{\sfD}{{\mathsf{D}}}
\newcommand{\sfE}{{\mathsf{E}}}
\newcommand{\sfF}{{\mathsf{F}}}
\newcommand{\sfG}{{\mathsf{G}}}
\newcommand{\sfH}{{\mathsf{H}}}
\newcommand{\sfI}{{\mathsf{I}}}
\newcommand{\sfJ}{{\mathsf{J}}}
\newcommand{\sfK}{{\mathsf{K}}}
\newcommand{\sfL}{{\mathsf{L}}}
\newcommand{\sfM}{{\mathsf{M}}}
\newcommand{\sfN}{{\mathsf{N}}}
\newcommand{\sfO}{{\mathsf{O}}}
\newcommand{\sfP}{{\mathsf{P}}}
\newcommand{\sfQ}{{\mathsf{Q}}}
\newcommand{\sfR}{{\mathsf{R}}}
\newcommand{\sfS}{{\mathsf{S}}}
\newcommand{\sfT}{{\mathsf{T}}}
\newcommand{\sfU}{{\mathsf{U}}}
\newcommand{\sfV}{{\mathsf{V}}}
\newcommand{\sfW}{{\mathsf{W}}}
\newcommand{\sfX}{{\mathsf{X}}}
\newcommand{\sfY}{{\mathsf{Y}}}
\newcommand{\sfZ}{{\mathsf{Z}}}


\newcommand{\calA}{{\mathcal{A}}}
\newcommand{\calB}{{\mathcal{B}}}
\newcommand{\calC}{{\mathcal{C}}}
\newcommand{\calD}{{\mathcal{D}}}
\newcommand{\calE}{{\mathcal{E}}}
\newcommand{\calF}{{\mathcal{F}}}
\newcommand{\calG}{{\mathcal{G}}}
\newcommand{\calH}{{\mathcal{H}}}
\newcommand{\calI}{{\mathcal{I}}}
\newcommand{\calJ}{{\mathcal{J}}}
\newcommand{\calK}{{\mathcal{K}}}
\newcommand{\calL}{{\mathcal{L}}}
\newcommand{\calM}{{\mathcal{M}}}
\newcommand{\calN}{{\mathcal{N}}}
\newcommand{\calO}{{\mathcal{O}}}
\newcommand{\calP}{{\mathcal{P}}}
\newcommand{\calQ}{{\mathcal{Q}}}
\newcommand{\calR}{{\mathcal{R}}}
\newcommand{\calS}{{\mathcal{S}}}
\newcommand{\calT}{{\mathcal{T}}}
\newcommand{\calU}{{\mathcal{U}}}
\newcommand{\calV}{{\mathcal{V}}}
\newcommand{\calW}{{\mathcal{W}}}
\newcommand{\calX}{{\mathcal{X}}}
\newcommand{\calY}{{\mathcal{Y}}}
\newcommand{\calZ}{{\mathcal{Z}}}

\newcommand{\bara}{{\bar{a}}}
\newcommand{\barb}{{\bar{b}}}
\newcommand{\barc}{{\bar{c}}}
\newcommand{\bard}{{\bar{d}}}
\newcommand{\bare}{{\bar{e}}}
\newcommand{\barf}{{\bar{f}}}
\newcommand{\barg}{{\bar{g}}}
\newcommand{\barh}{{\bar{h}}}
\newcommand{\bari}{{\bar{i}}}
\newcommand{\barj}{{\bar{j}}}
\newcommand{\bark}{{\bar{k}}}
\newcommand{\barl}{{\bar{l}}}
\newcommand{\barm}{{\bar{m}}}
\newcommand{\barn}{{\bar{n}}}
\newcommand{\baro}{{\bar{o}}}
\newcommand{\barp}{{\bar{p}}}
\newcommand{\barq}{{\bar{q}}}
\newcommand{\barr}{{\bar{r}}}
\newcommand{\bars}{{\bar{s}}}
\newcommand{\bart}{{\bar{t}}}
\newcommand{\baru}{{\bar{u}}}
\newcommand{\barv}{{\bar{v}}}
\newcommand{\barw}{{\bar{w}}}
\newcommand{\barx}{{\bar{x}}}
\newcommand{\bary}{{\bar{y}}}
\newcommand{\barz}{{\bar{z}}}
\newcommand{\barA}{{\bar{A}}}
\newcommand{\barB}{{\bar{B}}}
\newcommand{\barC}{{\bar{C}}}
\newcommand{\barD}{{\bar{D}}}
\newcommand{\barE}{{\bar{E}}}
\newcommand{\barF}{{\bar{F}}}
\newcommand{\barG}{{\bar{G}}}
\newcommand{\barH}{{\bar{H}}}
\newcommand{\barI}{{\bar{I}}}
\newcommand{\barJ}{{\bar{J}}}
\newcommand{\barK}{{\bar{K}}}
\newcommand{\barL}{{\bar{L}}}
\newcommand{\barM}{{\bar{M}}}
\newcommand{\barN}{{\bar{N}}}
\newcommand{\barO}{{\bar{O}}}
\newcommand{\barP}{{\bar{P}}}
\newcommand{\barQ}{{\bar{Q}}}
\newcommand{\barR}{{\bar{R}}}
\newcommand{\barS}{{\bar{S}}}
\newcommand{\barT}{{\bar{T}}}
\newcommand{\barU}{{\bar{U}}}
\newcommand{\barV}{{\bar{V}}}
\newcommand{\barW}{{\bar{W}}}
\newcommand{\barX}{{\bar{X}}}
\newcommand{\barY}{{\bar{Y}}}
\newcommand{\barZ}{{\bar{Z}}}

\newcommand{\hX}{\hat{X}}
\newcommand{\Ent}{\mathsf{Ent}}
\newcommand{\awarm}{{A_{\text{warm}}}}
\newcommand{\thetaLS}{{\widehat{\theta}^{\text{\rm LS}}}}

\newcommand{\jiao}[1]{\langle{#1}\rangle}
\newcommand{\gaht}{\textsc{GoodActionHypTest}\;}
\newcommand{\iaht}{\textsc{InitialActionHypTest}\;}
\newcommand{\true}{\textsf{True}\;}
\newcommand{\false}{\textsf{False}\;}

% \usepackage[capitalize,noabbrev]{cleveref}
% \crefname{lemma}{Lemma}{Lemmas}
% \Crefname{lemma}{Lemma}{Lemmas}
% \crefname{thm}{Theorem}{Theorems}
% \Crefname{thm}{Theorem}{Theorems}
% \Crefname{assumption}{Assumption}{Assumptions}
% \Crefname{inftheorem}{Informal Theorem}{Informal Theorems}
% \crefformat{equation}{(#2#1#3)}

% % if you use cleveref..
% \usepackage[capitalize,noabbrev]{cleveref}
% \crefname{lemma}{Lemma}{Lemmas}
% \crefname{proposition}{Proposition}{Propositions}
% \crefname{remark}{Remark}{Remarks}
% \crefname{corollary}{Corollary}{Corollaries}
% \crefname{definition}{Definition}{Definitions}
% \crefname{conjecture}{Conjecture}{Conjectures}
% \crefname{figure}{Fig.}{Figures}


\begin{document}
\journaltitle{GLBIO 2025}
\DOI{DOI}
\copyrightyear{2025}
\pubyear{2015}
\access{}
\appnotes{Paper}

\firstpage{1}

%\subtitle{Subject Section}

\title[Deep Active Learning for Host Targeted Therapeutics]{Deep Active Learning based Experimental Design to Uncover Synergistic Genetic Interactions for Host Targeted Therapeutics}


\author[1,$\ast$]{Haonan Zhu}
\author[1]{Mary Silva}
\author[1]{Jose Cadena} 
\author[1]{Braden Soper}
\author[2,3]{Michał Lisicki}
\author[4]{Braian Peetoom}
\author[4]{Sergio E. Baranzini} 
\author[1]{Shivshankar Sundaram}
\author[1]{Priyadip Ray}
\author[1]{Jeff Drocco}
%% Author affiliation
% \affiliation[1]{organization={Lawrence Livermore National Laboratory}, %Department and Organization
%             addressline={7000 East Ave}, 
%             city={Livermore},
%             postcode={94550}, 
%             state={CA},
%             country={USA}}
% \affiliation[2]{organization={University of Guelph}, %Department and Organization
%             addressline={School of Engineering}, 
%             city={Guelph},
%             postcode={N1G 2W1}, 
%             state={Ontario},
%             country={Canada}}
% \affiliation[3]{organization={University of California San Francisco}, %Department and Organization
%             addressline={}, 
%             city={San Francisco},
%             postcode={94158}, 
%             state={CA},
%             country={USA}}
% \affiliation[∗\ast]{Corresponding Authors: zhu18@llnl.gov}  


\authormark{Haonan Zhu et al.}

\address[1]{\orgname{Lawrence Livermore National Laboratory}, \orgaddress{\street{7000 East Ave}, \postcode{94550}, \state{CA}, \country{USA}}}
\address[2]{ \orgname{University of Guelph}, \orgaddress{\street{50 Stone Rd E}, \postcode{N1G 2W1}, \state{ON}, \country{Canada}}}
\address[3]{  \orgname{Vector Institute}, \orgaddress{\street{108 College St W1140}, \postcode{M5G 0C6}, \state{ON}, \country{Canada}}}
\address[4]{  \orgname{University of California San Francisco}, \orgaddress{\street{675 Nelson Rising Lane}, \postcode{94158}, \state{CA}, \country{USA}}}

\corresp[$\ast$]{Corresponding author. \href{email:zhu18@llnl.gov}{zhu18@llnl.gov}}

% \received{Date}{0}{Year}
% \revised{Date}{0}{Year}
% \accepted{Date}{0}{Year}

%\editor{Associate Editor: Name}

%\abstract{
%\textbf{Motivation:} .\\
%\textbf{Results:} .\\
%\textbf{Availability:} .\\
%\textbf{Contact:} \href{name@email.com}{name@email.com}\\
%\textbf{Supplementary information:} Supplementary data are available at \textit{Journal Name}
%online.}

%Due to factors such as replication and mutation rates, combination therapies are necessary to inhibit viral replication sufficiently to prevent escape through mutations. 

\abstract{Recent technological advances have introduced new high-throughput methods for studying host-virus interactions, but testing synergistic interactions between host gene pairs during infection remains relatively slow and labor intensive. Identification of multiple gene knockdowns that effectively inhibit viral replication requires a search over the combinatorial space of all possible target gene pairs and is infeasible via brute-force experiments. Although active learning methods for sequential experimental design have shown promise, existing approaches have generally been restricted to single-gene knockdowns or small-scale double knockdown datasets. In this study, we present an integrated Deep Active Learning (DeepAL) framework that incorporates information from a biological knowledge graph (SPOKE, the Scalable Precision Medicine Open Knowledge Engine) to efficiently search the configuration space of a large dataset of all pairwise knockdowns of 356 human genes in HIV infection. Through graph representation learning, the framework is able to generate task-specific representations of genes while also balancing the exploration-exploitation trade-off to pinpoint highly effective double-knockdown pairs.  We additionally present an ensemble method for uncertainty quantification and an interpretation of the gene pairs selected by our algorithm via pathway analysis. To our knowledge, this is the first work to show promising results on double-gene knockdown experimental data of appreciable scale (356 by 356 matrix).}
\keywords{Deep Active Learning, Sequential Experimental Design, Combination Therapies, Gene-Gene Interactions, Knockdown Experiments, Representation Learning, Ensemble Method, Graph Learning}

\maketitle
            

%%Graphical abstract
% \begin{graphicalabstract}
% %\includegraphics{grabs}
% \end{graphicalabstract}

%% Keywords
% \begin{keyword}
%% keywords here, in the form: keyword \sep keyword
% Deep Active Learning \sep Sequential Experimental Design \sep Combination Therapies \sep Gene-Gene Interactions \sep Knockdown Experiments \sep Representation Learning \sep Ensemble Method \sep Graph Learning

%% PACS codes here, in the form: \PACS code \sep code

%% MSC codes here, in the form: \MSC code \sep code
%% or \MSC[2008] code \sep code (2000 is the default)

% \end{keyword}
% \end{frontmatter}


\section{Introduction}
\label{sec: intro}

% TODO: need edit from experts on the subject and provide additional references
Understanding the various pathways essential for viral replication is vital for creating effective antiviral treatments. Due to factors such as replication and mutation rates, many viruses quickly develop resistance to drugs targeting a single site, making them ineffective over time. In these situations, combination therapies are necessary to inhibit viral replication sufficiently to prevent escape through mutations \cite{tang2012hiv}. Although host cellular targets are not subject to the same selective pressures as viral genome components, targeting host factors crucial for viral replication may still benefit from a combination strategy. Viruses have developed intricate interactions with host cells to support their life cycle, and studies have shown that dual knockdowns of host genes can sometimes synergistically inhibit viral growth \cite{gordon2020quantitative}. Moreover, targeting multiple host factors may expand the effectiveness of a therapeutic across different viral pathogens \cite{richman2016antiviral}. 

Recent technological advances have introduced new high-throughput methods for studying host-virus interactions \cite{puschnik2017crispr}, but testing synergistic interactions between gene pairs during viral infections remains relatively slow and labor-intensive. Identification of promising gene-gene knockdowns requires a search in the combinatorial space of all possible target gene pairs and is infeasible via brute-force experimentation \cite{norman2019exploring, replogle2022mapping}. A potential solution is sequential experimental design guided by data-driven models to balance the trade-off between successful identification of relevant gene pairs and experimental cost \cite{sverchkov2017review}. Various approaches, including active learning (e.g., bandits \cite{pacchianoneural}) and traditional experimental design \cite{lyle2023discobax},  have been previously studied in the context of single gene knockdowns. See \cite{mehrjou2021genedisco} for an overview of the methods along with benchmark datasets for single genetic interventions. More recently, there has been related work in representation learning to discover pairwise genetic interactions from embeddings of single gene knockdowns \cite{jain2024active,jain2024automated}, and design of perturbation screens on RNA-seq data by incorporating prior knowledge from multiple data sources \cite{huang2024sequential}. However, the goal of \cite{jain2024active,jain2024automated} is to discover gene pairs that violate the additive assumption (i.e., the effect of a double gene knockdown is the sum of two single gene knockdowns) in double-knockdown experiments, and require measurements from microscopy images while incorporating no existing knowledge graph information to assist in the discovery; To predict the effects of unseen gene knockdown, \cite{huang2024sequential} operate on RNA-seq data, which have the major drawback that we can only observe perturbations that survive long enough to be measured. There has also been work that utilizes a large language model to design experiments \cite{huang2024crispr, roohani2024biodiscoveryagent}. While the initial results are promising, the number of relevant gene pairs that can be identified by the bio-agent is strictly limited due to API constraints and the tendency of large language models to generate hallucinations (i.e., plausible but factually incorrect or nonsensical information) \cite{xu2024hallucination}. In summary, although many proposed methods discuss extensions for double-knockdown experimental design, the experimental results from the current literature are limited to single-knockdown data \cite{huang2024sequential}, small-scale double-knockdown data (e.g., a search space of $50$ by $50$ matrix in \cite{jain2024active}, only $160$ out of $100576$ possible gene pairs are queried in \cite{roohani2024biodiscoveryagent}). 

%By contrast, experiments in which gene knockdown techniques are used to reduce or inhibit the expression of selected genes in cultured cells open the possibility of testing specific pairs of genes for synergistic effects. This approach was taken by experimenters seeking to predict morphological features and synthetic-lethal-type gene interactions in endothelial cells resulting from all pairwise combinations of 50 distinct gene knockdowns \cite{jain2024active}\cite{jain2024automated,}. Their approach involves using active learning to efficiently select experiments most likely to identify epistatic interactions. Active learning-based approaches have also been applied to predict various phenotypic outcomes from genome-wide CRISPR screens \cite{mehrjou2021genedisco} and, when incorporating prior information about genes and gene products, have been used to predict the effect of gene knockdowns in Perturb-seq experiments \cite{huang2024sequential}.


We present an integrated deep active learning (DeepAL) framework that incorporates existing knowledge graph information to efficiently search the configuration space while balancing the exploration-exploitation trade-off through the active learning loop that utilizes an ensemble method for uncertainty quantification \cite{abdar2021review,lakshminarayanan2017simple}. Specifically, we leverage the Scalable Precision Medicine Open Knowledge Engine (SPOKE) \cite{morris2023scalable}. By using graph representation learning \cite{schlichtkrull2018modeling}, our framework is able to provide interpretable results on the set of gene pairs recommended by the framework. Most notably, this work differs from the previous graph learning-based approach ITERPER \cite{huang2024sequential} in the following manners, 1)  our GNN model is built for heterogeneous knowledge graphs such as SPOKE that includes different edge types \footnote{Gene regulatory interactions, gene-protein encoding, protein-protein interactions, and gene-biological processes} associations whereas the primary graphs used in ITERPER include the gene co-expression graph (a homogeneous graph constructed through thresholding the Pearson correlations) and gene-ontology graph (a bipartite graph between genes and a pathway term constructed through thresholding computed Jaccard indices) and 2) our work leverages ensemble method for uncertainty quantification instead of fitting a Gaussian-noise model, which has been shown to better capture the model uncertainty of deep learning models \cite{lakshminarayanan2017simple}. Our main contribution is the development of a unified active learning framework that includes a heterogeneous graph-based representation model that utilizes the existing knowledge graph and an ensemble method for uncertainty quantification. To our knowledge, this is the first work to show promising results on double genes knockdown experimental data of appreciable scale (356 by 356 matrix) \cite{gordon2020quantitative}. 

The rest of the paper is organized as follows: section \ref{sec: notation and data} introduces the notation and data (SPOKE and HIV dataset) that we use to evaluate our methods, section \ref{sec: methods} presents the proposed ensemble DeepAL framework, section \ref{sec: experiments} provides our experimental results on the HIV dataset, and section \ref{sec: conclusion} summarizes the findings of the article and discusses future directions. 

% TODO: the following sentence need revision for related work 
% In contrast to previous work using active learning methods for uncovering biological
% networks \cite{ud2016optimal,sverchkov2017review} our approach differs in two main aspects: 1) the goal of our framework is to identify the best gene-pairs whose combined knockdown will result in lowest viral replications, while previous work focuses on exploring unobserved edges with greatest uncertainty, 2) unlike previous work, our framework is designed to incorporate external knowledge graph information to assist the sequential decision problem.  

% This results in a sequential experimental design problem where the configuration space is all the possible combinations of target genes pairs, and the goal is to identify the gene pair that results in the lowest viral replication when we are only allowed to query small portion of the configuration space due to limited experimental budget. As we are trying to choose pairs of genes for future knockdown experiments based on current observed experimental results

% % TODO: need motivation why knowledge graph might be a better alternative than other sources of informations 
% SPOKE (Scalable Precision Medicine Oriented Knowledge Engine) is a graphical network of biomedical databases that can bring the extra information needed for our sequential experimetn design problem, and it has been used for various biological applications such as drug discovery \cite{patel2020machine}, embedding protein-protein interactions \cite{liu2023graph}, predictive modeling for patient outcomes in electronic healthcare records \cite{tang2024leveraging}. 
%

% \section{Related Work}
% \label{sec: related work}

% TODO:
% - Consolidate more references here 
% - Improve the writings in this section
% \subsection{Gene-Gene Interaction Prediction Model} 
% \label{subsec: predictive model reference}
% A critical component of an active learning system is a data-driven model capable of predicting unknown gene-gene interactions along with associated uncertainties. Based on the predictions and associated uncertainties, the next batch of experiments is determined and the results of these experiments are fed back to further improve the predictive performance of the model. For predicting gene-gene interactions, multiple models have been proposed in the literature. A popular class of techniques relies on the low-rank structure of the underlying interaction matrix and employs matrix factorization techniques to predict missing values \cite{gordon2020quantitative}. With recent progress in applying deep learning models to related problems in drug discovery
% \cite{zhu2021prediction, pham2021deep, lotfollahi2019scgen, hetzel2022predicting}, various models have been proposed for RNA sequence perturbation predictions, including a graph-based model GEARS \cite{roohani2024predicting} and a multilevel deep learning method \cite{bai2024attentionpert}. %Though both \cite{roohani2024predicting} and \cite{bai2024attentionpert} utilizes graph neural networks, our representation model differ in: 1) Our model is built to utilize hetergeneous knowled

% Various deep learning methods have been proposed to identify epistatic interactions between Single Nucleotide Polymorphism (SNPs) observed to be associated with traits in Genome-Wide Association Study (GWAS) experiments \cite{cui2022gene,gracca2024distributed}. In these studies, the fundamental unit of interaction is a single SNP, which may have pleiotropic effects over the lifecycle of an organism. Hence, it cannot generally be determined whether the effect observed to be associated with SNP interactions is mediated through the action of one or multiple genes. Moreover, non-perturbative, observational data collection, as is typically used in GWAS experiments, results in relatively low power to detect higher-order SNP interactions to statistical significance. 

% \subsection{Computational Methods for Experimental Design} 
% \label{subsec: related work computational}

% One natural approach to the double knockdown experimental design problem is based on matrix factorization \cite{karimi2012exploiting}  where the interaction matrix is viewed as a reward matrix. This is a standard methodology for inferring missing observations in relational data sets (e.g., recommender systems) and has been used previously in the study of gene-to-gene interactions \cite{vzitnik2015data, norman2019exploring}. %However, due to budget constraints, an approach such as matrix factorization that only utilizes the observed experimental data is not guaranteed to efficiently explore the design space, and additional information is needed to reduce the complexity of the problem. 
% Based on GEARS \cite{roohani2024predicting}, \cite{huang2024sequential} show that incorporating additional information such as Protein structure, Protein-protein interaction network (PPI) into the predictive model can significantly improve the overall performance of the active learning framework for the experimental design of a single gene perturbation. More recently, there has also been work that utilizes a large language model to design experiments \cite{huang2024crispr, roohani2024biodiscoveryagent}. Although the initial results show benefit, these works are limited to small-scale experiment design due to API limitations and volatility of the language models' responses in the sequential setting.  

%Deep Active Learning (DeepAL) (\cite{ren2021survey}) is an emerging field in machine learning that aims to combine the representation power of deep learning and sample-efficiency of active learning to address various real world applications including drug discovery (\cite{Bailey_2023}), image classification \cite{ranganathan2017deep}, medical image analysis \cite{smailagic2018medal} and link predictions on knowledge graphs \cite{ostapuk2019activelink}, where either labeled data are limited or the sample selection itself is of interest. 

% Biological systems are often represented as networks of interactions between fundamental molecular entities. Among these, gene regulatory networks (GRNs) are critical as they map the regulatory interactions between transcription factors (proteins) and genes, which control cellular function \cite{alon2007network, levine2005gene}. However, despite significant advances, the precise architecture of GRNs remains largely unresolved \cite{babu2004structure}. To address these gaps, experimental techniques that perturb key components of these networks are employed to generate data and validate hypotheses \cite{segal2003module}. Active learning methods based on quantifying uncertainty of gene interactions has been deployed in this setting to extend our knowledge about GRNs more effectively \cite{ud2016optimal, sverchkov2017review}.  

%While the framework is broadly applicable, we focus on the host gene knockdowns experiment design problem and evaluate our framework on a dataset that measures HIV replication in HeLa cells with esiRNA knockdown of all possible pairs of 356 human genes involved in HIV pathogenesis \cite{gordon2020quantitative}. 
%the proposed framework is broadly applicable to any experimental design problems with additional graph information accessible and the configuration space is a Cartesian product of two sets of control variables (e.g., all possible gene-to-gene pairs).

\section{Notations and Data}
\label{sec: notation and data}


\subsection{Notations}
\label{subsec: notations}
We use bold uppercase letters for matrices, bold lowercase letters for vectors, and lowercase letters for scalars. The Hadamard (element-wise) product of vectors $\mb a$ and $\mb b$ is denoted by $\mb a \circ \mb b$, and $\diag$ denotes the function mapping a vector to a diagonal matrix with the components of the vector as its diagonal entries. We denote the Huber loss function \cite{huber1992robust} by:

\begin{equation}
\label{eqn: huber loss}
\text{Huber}\left( a \right) = \begin{cases}
    \frac{1}{2} a^2 & \text{if } |a| \leq 1, \\
    |a| - \frac{1}{2} & \text{if } |a| > 1,
\end{cases} 
\end{equation}

The activation function $\text{Softplus}$ is defined as:

\begin{equation}
\label{eqn: softplus}
\text{Softplus}\left( a \right) = \log(1+\exp(a))
\end{equation}

We denote the undirected knowledge graph as $\mc G=\left( \mc V, \mc E, \mc R \right)$, where $\mc V$ is the set of vertices, $\mc E$ is the set of edges and $\mc R$ is the set of relationships. Each node $v_i \in \mc V$ is associated with a label, $label(v) \in L$ that includes labeled edges. The edge labeled $\left(v_i, r, v_j \right) \in  \mc E$, where $i=1, \ldots, \abs{V}$ indexes the nodes, and $r \in \mc R$. Given a node $v_{i}$, the set of neighbor indices of node $v_i$ in relation $r \in \mc R$ is denoted as $\mc N_{i}^{r}$. The learned embedding for a given node $v_i \in \mc V$ is denoted as $\mb x_{i} \in \bb R^{d}$ where $d$ is the pre-assigned embedding dimension. The set of target genes (for example, HIV related genes in our experiments) is denoted as $H:=\set{v_1, \ldots, v_p} \subseteq \mc V$. The pairs of genes of interest are denoted as $S:= H \times H \subseteq \mc E$, and the matrix of viral loads (target) is denoted as $\mb Y \in \bb R^{p \times p}$, where $y_{ij}$ corresponds to the value of the viral loads when we down-regulate the gene $v_i$ and the gene $v_j$ of $H$.  

% We denote the undirected knowledge graph as $\mc G=\left( \mc V, \mc E, \mc R \right)$, with nodes (genes, biological processes, proteins etc.,) $v_i \in \mc V$ and labeled edges (relations) $\left(v_i, r, v_j \right) \in  \mc E$, where $i=1, \ldots, \abs{V}$ index the nodes and $r \in \mc R$ is a relation type. Given a node $v_{i} \in \mc V$, the set of neighbor indices of node $v_i$ under relation  $r \in \mc R$ is denoted as $\mc N_{i}^{r}$. The learned embedding for a given node $v_i \in \mc V$ is denoted as $\mb x_{v_i} \in \bb R^{d}$ where $d$ is the pre-assigned embedding dimension. The set of target gene-pairs of interests is denoted as $S \subseteq \abs{V} \times \abs{V} $, and the viral-loads (target) matrix is denoted as $\mb Y \in \bb R^{p \times p}$, where $p$ is the number of genes of interests, and $Y_{ij}$ corresponds to viral loads value when we down-regulate gene $i$ and gene $j$.  


\subsection{HIV Dataset}
\label{subsec: HIV dataset}

Data on the genetic interaction of HIV from \cite{gordon2020quantitative} focus on understanding the genetic interactions that influence HIV infection. It includes quantitative measurements of how different genetic perturbations, primarily through CRISPR-based methods, interact to affect viral loads of HIV. In total, the interaction matrix consists of 356 genes prioritized for their established or potential roles in HIV viral replication, as identified through prior studies on host-pathogen interactions and HIV-related protein interactions.
%carefully selected genes that are believed to play a significant role in HIV viral replication.

\section{Methods}
\label{sec: methods}

Our deep active learning framework incorporates information from the knowledge graph to efficiently explore the space of gene pairs. The framework has two phases: 1) initial self-supervised training of $M$-separate representation model(s) with different random initializations to learn the embeddings of genes that summarize the local topology of the knowledge graph,\footnote{$M=1$ for the base model}  2) an active learning loop where sequentially the representation model(s) and regression model(s) are further optimized to fit the current observed entries of the viral-loads matrix followed by the recommendations of the next batch of gene-pairs to uncover. The overall framework is summarized in Fig. \ref{fig: deepAL flow chart} and Algo. \ref{alg: DeepAL}. In Section \ref{subsec: model} we summarize the different components of the base model (the representation model and the regression model), in Section \ref{subsec: ensemble} we discuss uncertainty quantification via ensembles and in Section \ref{subsec: acquisition strategy} we discuss the acquisition strategy that is developed to balance the exploration-exploitation trade-off.

\begin{figure}[t]
    \centering
    \includegraphics[width=0.9\columnwidth]{images/flowchart.pdf}
    \caption{Flowchart of the proposed deep active learning framework. }
    \label{fig: deepAL flow chart}
\end{figure}

\begin{algorithm}[http]
\caption{DeepAL}
\label{alg: DeepAL}
\SetKwInOut{Require}{Require}
\SetKwFor{For}{for}{do}{end for}
\SetKwInput{KwData}{Require}
\SetKwInput{KwResult}{return}
\DontPrintSemicolon

\Require{knowledge graph \(\mc G=\left( \mc V, \mc E, \mc R \right)\), set of target genes $H$ with $p=\abs{H}$ and collection of gene-pairs of interests \(S \)}
\Require{viral-loads matrix \(\mb Y \in \bb R^{p \times p}\) to query}
\Require{number of rounds of experiments \(T\), number of gene-pairs to be selected per round \(N\)}
\Require{R-GCN models \(f_{\mb \Theta}^{(m)}: G \to \bb R^{\abs{V} \times d},\) where \(\mb \Theta\) denotes all the learnable parameters from the GNN model,  DistMulti weight matrix \(\mb R^{(m)} \in \bb R^{d \times d}\), bilinear regression weight matrices \(\mb A^{(m)} \in \bb R^{d \times d}\) and bias \(b^{(m)} \in \bb R\) , where \(m=1, \ldots, M\) denotes the distinct models with same architecture}
%TODO: think off better notation for acuqisition strategy
\Require{Acquisition Strategy, $\mb \Pi: S': \to S' \subseteq  S$ (see section \ref{subsec: acquisition strategy} for list of strategies considered)}

Train \(f_{\mb \Theta}^{(m)}\) and \(\mb R^{(m)} \) using self-supervised learning (see section \ref{subsec: model}) for each \(m=1, \ldots, M\)

Initialize $S'$ by randomly drawing \(N\) pairs of genes from \(S\)  

\For{\( t = 1, \ldots, T \) \tcp*{Active Learning Loop}}{

    \For{\(m=1, \ldots, M \) \tcp*{Train each model on \(S'\)}}{

    \( \min_{\mb \Theta^{(m)}, \mb A^{(m)}, b^{(m)}} \sum_{(i,j) \in S'} \text{Huber}(\hat{y}_{ij}^{(m)}-y_{ij})\), %where \(\hat{y}_{ij}^{(m)} \leftarrow \text{Softplus} \left(\mb x_{v_i}^\top \mb A^{(m)} \mb x_{v_j}\right) + b^{(m)} \quad \forall (i,j) \in S \setminus S', \forall m=1, \ldots, M \) %\tcp*{Forward Prediction on unobserved entries}
    }
    
    \(S_{new} \leftarrow \mb \Pi(S')\) %\tcp*{Samples chosen based on greedy scheme}
    
    \( S' \leftarrow S' \cup S_{new} \)\;
}
\KwResult{observed set of gene-pairs \(S' \subseteq S \).}
\end{algorithm}

\subsection{SPOKE Knowledge Graph}
\label{subsec: spoke}
The SPOKE knowledge graph \cite{morris2023scalable} consists of more than 20 thousand human gene nodes encoding protein from Entrez Gene, more than 200 million human proteins, a combined 18 thousand nodes of biological processes, molecular function, and cellular components derived from the Gene Ontology database. Additionally, the graph includes over 1 million gene relationships in which the knockdown or knockdown of one gene, achieved by short hairpin RNA or CRISPR, results in the up-regulation or down-regulation of another gene, as indicated by consensus transcriptional profiles.

To focus on our modeling objectives, we derive a subgraph based on a random walk sampling procedure using the target gene nodes in our viral load matrix $\mb Y$ as the starting nodes in the random walk. This is effective for excluding nodes that are not closely related to the target genes, and similar ideas have been proposed to learn a meaningful latent representation of social networks \cite{perozzi2014deepwalk}. Given the subgraph $\mc G=\left( \mc V', \mc E', \mc R' \right)$ of interests from SPOKE, where $V'$ contains all the nodes belonging to genes, biological processes, proteins. For each target gene $v \in H$, perform $w=5$ independent random walks of length $s=5$, i.e., for each walk, we repeatedly select a node from neighbors of the current node with equal probability up to $s=5$ times. This results in a final set of nodes $\mc V$ that lies within the $5$-th order neighborhood of the target genes $H$ and a subgraph $\mc G=\left( \mc V, \mc E, \mc R \right)$ where all edges and relationships are restricted to the set of nodes $\mc V$.


\subsection{Base Model}
\label{subsec: model}


The proposed DeepAL framework contains three major components: 1) a representation model based on Relational Graph Convolutional Networks (R-GCNs) \cite{schlichtkrull2018modeling} that maps knowledge graph ($\mc G$) to node embeddings $\{\mb x_{i}\}_{v_{i}\in \mc V} $, 2) an edge prediction model based on DistMulti\cite{yang2014embedding} that maps a given pair of embeddings $\left( \mb x_{i},\mb x_{j} \right) $ to a binary prediction of whether the two nodes are related by relationship $r \in \mc R$, and 3) a bilinear regression model that maps a pair of node embeddings to the target variable of interest in active learning loop. The additional hyperparameter configurations are summarized in Table \ref{tab:hyperparameters}.  

\subsubsection{R-GCN}
\label{subsubsec: R-GCN}

The main propagation model for the forward-pass of a node (i.e., $v_{i}$) is:
\begin{equation}
\label{eqn: R-conv}
               \mb h_{i}^{\left(l+1\right)} = \sigma\left( \sum_{r \in \mc R} \sum_{j \in \mc N_{i}^{r}} \frac{1}{c_{i,r}} \mb W_{r}^{\left( l \right) } \mb h_{j}^{\left( l \right) } \right) + \mb W_{0}^{\left( l \right) }\mb h_{i}^{\left( l \right) } ,
\end{equation}
where $l=1,\ldots, L$ indexes the layer number and $L$ is the total number of convolution layers. $\mb h_i^l \in \bb R^{d_h}$ denotes embedding of node $i$ at the $l$'th layer, where $d_h$ is the dimension of the hidden layers, $c_{i,r}$ is a normalization constant (e.g., $c_{i,r}=\abs{N_{i}^{r}}$, the number of neighbors of node $v_{i}$). Each graph convolution operation aggregates feature vectors of each node's neighbors. In this work, we use the standard basis vectors for the initial embeddings (i.e., $\mb h_i^0=\mb e_i \in \bb R^{\abs{V}}$ for all $v_i \in \mc V$). 


\subsubsection{DistMulti: Edge Prediction Model for Initialization} 
\label{subsubsec: DistMulti}

Every relation $r$ is associated with a diagonal matrix $\mb R_{r} \in \bb R^{d\times d}$, and for every pair of nodes $\left( v_{i}, v_{j} \right) $ is scored as:
\begin{equation}
\label{eqn: distmulti}
f\left( v_{i}, r, v_{j} \right)= \sigma\left( \mb x_{i}^{\top} \mb R_{r} \mb x_{j}  \right).
\end{equation}
The self-supervised learning task is performed based on negative sampling \cite{yang2014embedding, trouillon2016complex, schlichtkrull2018modeling} where for each observed relationship (that is, positive edge), we sample one false edge (i.e., negative edge), and both the representation model and DistMulti are jointly trained to classify these edges. 

\subsubsection{Bilinear Regression}
\label{subsubsec: bilinear}
For every pair of target nodes $\left( v_{i}, v_{j} \right) \in S$, the predicted  viral-loads are given by:
\begin{equation}
\label{eqn: bilinear reg}
\hat{y}_{ij}= \text{Softplus}\left(\mb x_{i}^{\top} \mb A \mb x_{j}\right)+b,
\end{equation}
where $\mb A$  is a trainable symmetric weight matrix and $b$ is trainable scalar for bias correction. This is a generalization of DistMulti, where the matrix is no longer restricted to be diagonal, and additional non-linearity is added to increase the expressiveness of the model. This has been studied extensively in the bilinear bandit literature \cite{jun2019bilinear,rizk2021best}. %The choice of $\text{SoftPlus}$ as  is due to the positivity of the observed log-varial 

\begin{table}[h!]
\centering
\resizebox{0.95\columnwidth}{!}{
    \begin{tabular}{|c|c|c|c|c|}
    \hline
    \multirow{2}{*}{Model} & \multicolumn{4}{c|}{Hyperparameters} \\ \cline{2-5} 
                                & Normalization & Non-Linearlity $\sigma$ &  Regularization & Dimensional Parameters \\ \hline
    \multirow{3}{*}{R-GCN}     & \multirow{3}{*}{DiffGroupNorm \cite{zhou2020towards}}         & \multirow{3}{*}{ReLu}           & \multirow{3}{*}{Weight Decay}   & L=$3$ \footnote{This balances the trade-off between expressiveness of the model and the over-smoothing phenomenon where all nodes became indistinguishable in the embedding space \cite{rusch2023survey}.}          \\ \cline{5-5} 
                                &           &           &           & $d_h=64$         \\ \cline{5-5} 
                                &           &          &          & $d=50$        \\ \hline
    DistMulti    & N/A       & Sigmoid         & N/A  & N/A \\ \cline{2-5} \hline
    \multirow{2}{*}{Bilinear}     & \multirow{2}{*}{N/A}        &  \multirow{2}{*}{Softplus}        & Dropout \cite{srivastava2014dropout},        & \multirow{2}{*}{N/A}         \\
    &           &          &         Weight Decay &        \\ \hline
    \end{tabular}
    }
\caption{Summary of hyperparameter setups for different components of the framework.}
\label{tab:hyperparameters}
\end{table}

% \paragraph{Resetting Bilinear Weight Matrix.} As summarized in Algo. \ref{alg: DeepAL} at the end of each round, we re-initialize the bilinear weight matrix $\mb W$. This is inspired by work from meta learning \cite{frati2023reset} where a significant improvement is observed over other approaches such as asymmetric training (i.e., different learning rates for the representation model and regression model) \cite{javed2019meta,beaulieu2020learning} and expensive second order methods \cite{finn2017model}. This can be seen as a generalization of the standard drop-out regularization \cite{srivastava2014dropout}, where by forcing the bi-linear weights to be re-initialized at each rounds, the representation model is forced to learn features that can adaptive quickly to different weights \cite{frati2023reset} and prevent over-fitting to the seen data. 

\subsection{Ensemble Method For Additional Uncertainty Quantification}
\label{subsec: ensemble}

Uncertainty is a critical part of active learning to balance the exploration-exploitation trade-off, and the ensemble method has been used with success for neural network-based models in supervised learning settings \cite{lakshminarayanan2017simple, abdar2021review}. 

Given the $M$-trained models, we have $M$-predictions ($\hat{Y}^{(m)}_{ij}, m=1, \ldots, M$, Eqn. \ref{eqn: bilinear reg}) for every pair of nodes $(v_i, v_j)$ that have not been uncovered. These predictions capture the model uncertainty. We can further compute estimators for standard deviations and quantiles.

\subsection{Acquisition Strategies}
\label{subsec: acquisition strategy}

We examine multiple acquisition strategies for both the base model and the ensemble model, and they are summarized in Table \ref{tab: acquisition strategies}. 

\begin{table*}[ht]
\scriptsize
\setlength{\tabcolsep}{4pt} 
\centering
\begin{tabular}{|p{3cm}|p{3.8cm}|p{5.3cm}|}
\hline 
\textbf{Acquisitions} & Base Model $M=1$ & Ensemble $M=20$ \\
\hline
Greedy &  $\arg\min_{(i,j) \in S'} \hat{y}_{ij}$ &  $\arg\min_{(i,j) \in S'} \text{median}_{m} \hat{y}_{ij}^{(m)}$ \\
\hline
Badge \footnotemark & $\arg\max_{(i,j) \in S'} \norm{\mb x_{i} \circ \mb x_{j}}{}$ &  $\arg\max_{(i,j) \in S'} \text{median}_m \norm{\mb x^{(m)}_{i} \circ \mb x_{j}}{}$\\
\hline
Optimism & N/A &  $\arg\min_{(i,j) \in S'} \text{quantile}_{m} \hat{y}_{ij}^{(m)}$\\
\hline
Maximum Variance &  N/A & $\arg\max_{(i,j) \in S'} \text{std}_{m} \hat{y}_{ij}^{(m)}$\\
\hline
\end{tabular}
\footnotetext{First proposed in \cite{ash2019deep} to use norm of the gradient from the last layer of the neural network for uncertainty quantification. This reduces to T(race)-optimality in optimal experimental design \cite{pukelsheim2006optimal} when there is no-hidden layers.} 
\caption{Comparison of acquisition strategies in the base model setting and ensemble setting. Both optimism and maximum variance requires uncertainty quantification hence unattainable in the base model setting.}
\label{tab: acquisition strategies}
\end{table*}

% For the base model, we have
% \begin{enumerate}
%     \item Greedy.
%     \item BADGE, this is first proposed in \cite{ash2019deep} to use norm of the gradient from the last layer as additional uncertainty quantification. This is closely related to T(race)-optimality in optimal experimental design \cite{pukelsheim2006optimal}.  
%     \item Random Sampling: this is a reference method to benchmark the performance. 
% \end{enumerate}

% For ensemble model, we have following additional strategies 
% \begin{enumerate}
%     \item Optimism, and we investigate variations in the confidence range in our experiment section. 
%     \item Maximum Variance.
% \end{enumerate}

\section{Experiments}
\label{sec: experiments}

In this section, we evaluate the performance of our methods on the HIV dataset described in Section \ref{subsec: HIV dataset}, and our main result shows that by incorporating information from the knowledge graph and utilizing uncertainty quantification information through the ensemble approach, the proposed ensemble DeepAL framework uncovers 92\% of the top $400$ gene-pairs after observing less than $6.3\%$ of the entire matrix. 

\subsection{Performance Evaluation}
\label{subsec: performance evaluation}
To mimic real-world experimental needs, we run all algorithms for $17$ rounds (including the initial round of random selection), and at each round each algorithm uncovers $400$ gene pairs for observation, where the batch size of $400$  is chosen to approximately balance typical computational and experimental capabilities while allowing sufficient exploration of the interaction matrix. We evaluated the algorithms by two different evaluation metrics: 1) Coverage@400 defined as the fraction of top-$400$ gene pairs uncovered by each round; 2) MAE (Mean of Absolute Error), defined as the sum of absolute errors between the predicted values on the unseen data given the current model and the ground truth values at each round, divided by the total number of samples. The former evaluates how well the active learning framework can uncover the most relevant gene pairs, while the latter evaluates how well a predictive model was learned along the sequential process. 

% \label{subsec: acquisition comparison}
\subsection{Comparison Among the Acquisition Strategies} 
\label{subsec: acquisition comparison}
We evaluate the list of acquisition functions detailed in Table. \ref{tab: acquisition strategies}, and the results are summarized in Fig. \ref{fig: acquisition comparisons}. Optimism with  $10\%$ quantiles achieves the best performance in terms of coverage at the end of the active sensing process, while the greedy solution offers competitive performance, especially in the early rounds. Badge and maximum variance strategies are suboptimal for coverage, but superior in learning a generalizable model that can predict viral-replicates on unseen data due to an emphasis on sampling gene-pairs where the predictive uncertainty is large (thus encouraging information gain over coverage). 

\begin{figure*}[t]
    \centering
    \begin{subfigure}{0.9\columnwidth}\hspace*{-1em}
        \centering
		\includegraphics[width=\textwidth]{images/Coverage_400_batchsize=400.pdf}
		\caption{Coverage@400} 
    \end{subfigure}
    \begin{subfigure}{0.9\columnwidth}
        \centering
		\includegraphics[width=\textwidth]{images/MAE_batch_size=400.pdf}
		\caption{Mean Absolute Errors on Unseen Data}
    \end{subfigure}  
    \caption{Comparison among the different acquisition strategies, and results are summarized over $20$ replicates. Optimism with $10\%$ quantiles achieves the best performance in terms of coverage at terminal phase, and uncovers 92\% of the top $400$ gene-pairs while only $<6.3\%$ of the entire matrix is observed. Maximum variance strategy performs the best in learning a generalizable model that predicts viral-replicates on unseen data due to an emphasis on exploration.}
    \label{fig: acquisition comparisons}
\end{figure*}

% \label{subsec: Ablation Studies}
\subsection{Ablation Studies}
\label{subsec: abaltion studies}
To demonstrate the significance of the various components of our model, we perform the following ablation studies:
\begin{enumerate}
   \item DeepAL: base model. 
   \item DeepAL-Ensemble: ensemble of base models with $M=20$. 
   \item DeepAL-Ensemble-Randominit: ensemble of the base models but all the weights are initialized randomly instead of performing self-supervised training through negative sampling. 
   \item Ensemble of Models with Fixed Features (FF-Ensemble): this is the case where the embeddings are frozen after initial training on the knowledge graph, and the comparison aims to examine the benefits of fine-tuning the embeddings on the observed viral-load data. 
   \item Ensemble of Models with Unconstrained Features (UF-Ensemble): this is the case where the embeddings are treated as free parameters to be optimized and has been studied extensively to understand behaviors of neural networks \cite{mixon2022neural, zhu2021geometric}. The comparison aims to examine the benefits of incorporating knowledge graph information into the model. 
\end{enumerate}
\noindent In Fig. \ref{fig: ablation studies} we compare our best performing acquisition strategy (optimism with $10\%$-quantile) for all ensemble based approaches listed above and a greedy acquisition strategy for the base model. The comparison between the base model and ensemble approaches show that ensemble methods are able to provide  meaningful uncertainty quantification that benefit coverage in the long-run. The comparison between DeepAL-Ensemble versus UF-Ensemble shows that the graph embeddings of the models extracted from SPOKE offers a more effective representation of the genes for downstream tasks; The comparison between the DeepAL-ensemble and DeepAL-Ensemble with random initialization shows that there is meaningful information in the SPOKE graph that can accelerate the search for best genes in the early rounds when only a small amount of experimental data has been observed; The comparison between DeepAL-Ensemble and FF-Ensemble shows that there is significant benefit in fine-tuning embeddings based on observed experimental data. 

\begin{figure*}[htbp]
    \centering
    \begin{subfigure}{0.9\columnwidth}\hspace*{-1em}
        \centering
		\includegraphics[width=\textwidth]{images/Coverage_400_ablations.pdf}
		\caption{Coverage@400} 
    \end{subfigure}
    \begin{subfigure}{0.9\columnwidth}
        \centering
		\includegraphics[width=\textwidth]{images/MASE_batch_size=400_ablations.pdf}
		\caption{Mean Absolute Errors on Unseen Data}
    \end{subfigure}  
    \caption{Abalation studies to validate our proposed approach, and results are summarized over $20$ replicates. The comparison between the base model and ensemble approaches shows that ensemble method is able to provides a meaningful uncertainty quantification that benefits coverage in the long-term; The comparison between DeepAL-Ensemble versus UF-Ensemble shows that the graph representation of the models from SPOKE offers a more effective representation of the genes for downstream tasks; The comparison between the DeepAL-ensemble and DeepAL-Ensemble with random initialization shows that there is meaningful information in the SPOKE graph that can benefits the search for best genes in the early rounds, but the benefits diminish rapidly as more experimental data is collected; The comparison between DeepAL-Ensemble and FF-Ensemble shows that there is significant benefits to fine-tuning the embeddings as more data is being collected.}
    \label{fig: ablation studies}
\end{figure*}

\subsection{Pathway Analysis}
\label{subsec: bio interpretations}

To provide a biologically intuitive representation of the gene selection process during active learning, we performed a pathway enrichment analysis, with the results summarized in Fig. \ref{fig: pathway}. This analysis involved identifying the gene pairs most frequently selected in $20$ independent runs and grouping the top pairs for each round. Using GOATOOLS \cite{klopfenstein2018goatools} for gene enrichment, each gene group was tested against all coding genes and overrepresented terms were plotted. We compare the results of optimism (best coverage performance) and maximum variance (best mean absolute error performance). Overall, both strategies retrieve a similar number of biological processes (BPs); however, the maximum variance strategy includes significantly more BPs that appear in only one round (32 compared to 25 for the optimism strategy). This difference can be attributed to the broader focus of the maximum-variance strategy on the entire matrix.

Among the biological processes present in both strategies, those highlighted in orange are of particular importance: almost all are related to translation. These processes are observed across more than four rounds in both strategies, underscoring the ability of both strategies to select genes that directly decrease viral replication by halting translational or pre-translational processes, such as mRNA elongation by RNA polymerase II. This is significant in HIV infection as Tat is known to enhance recruitment and activity of RNA polymerase II \cite{reeder2015hiv}.  Among the 30\% of BPs that differ between the strategies, DNA-related processes are explored more extensively by the maximum variance strategy, which identifies three times as many DNA-related BPs compared to the optimism strategy. For RNA- and transcription-related processes, the maximum-variance strategy notably identifies the "positive regulation of mRNA binding" BP in nine rounds. This BP is associated with processes that activate or increase the frequency, rate, or extent of mRNA binding. In contrast, the optimism strategy identifies the "positive regulation of DNA-templated transcription, elongation" BP over five rounds, a process that enhances the frequency of transcription elongation. Both BPs include genes whose silencing may decrease viral replication.

In conclusion, both strategies successfully identify genes that can decrease viral replication, although with differences in focus and approach.


% \begin{figure}
%     \centering
%     \begin{subfigure}{0.48\columnwidth}\hspace*{-1em}
%         \centering
% 		\includegraphics[width=\textwidth]{images/Unpruned.pdf}
% 		\caption{Without gene-pairs pre-filtering} 
%     \end{subfigure}
%     \begin{subfigure}{0.48\columnwidth}
%         \centering
% 		\includegraphics[width=\textwidth]{images/Pruned.pdf}
% 		\caption{With gene-pairs pre-filtering}
%     \end{subfigure}  
%     \caption{Fraction of top 50 gene-pairs uncovered across different rounds of experiments, and results are summarized over $100$ replicates. By incorporate information from the knowledge graph, we see significant improvement of the proposed DeepAL framework: uncovered 98\% of the top 50 gene-pairs by rounds 8 (<6\% of the entire matrix observed) and uncovered 98\% of the top 50 gene-pairs by rounds 7 (<5\% of the entire matrix observed) if we pruned the search space to \%25 using the diagonal entries. }
%     \label{fig: top50 plot}
% \end{figure}

\begin{figure*}[htbp]
    \centering
    \begin{subfigure}{0.9\columnwidth}\hspace*{-1em}
        \centering
        \includegraphics[width=\textwidth]{images/archive/max_variance_20_ensembles.pdf}
        \caption{Pathway analysis based on maximum variance.}
    \end{subfigure}
    \begin{subfigure}{0.9\columnwidth}
        \centering
        \includegraphics[width=\textwidth]{images/archive/max_optimism_20_ensembles.pdf}
        \caption{Pathway analysis based on optimism.}
    \end{subfigure}  
    \caption{Pathway enrichment analysis on the gene-pairs selected by DeepAL. The left panel shows the biological process terms selected optimizing by maximum variance, and the right panel shows the terms enriched when optimizing by maximum optimism. The $x$-axis represents the rounds, while the $y$-axis shows the total number of Gene Ontology (GO) biological process terms that are most frequently selected by the algorithm.}
    \label{fig: pathway}
\end{figure*}

\section{Conclusion}
\label{sec: conclusion}

In this work, an integrated Deep Active Learning (DeepAL) framework is proposed to incorporate existing heterogeneous knowledge graph information for efficient double-gene knockdown experiment design for host-targeted therapeutics. This unified framework utilizes the expressiveness of graph representation learning, the uncertainty quantification of the ensemble method, and the sample efficiency of the active learning loop to show promising results on double-gene knockdown experimental data of appreciable scale. In addition, the framework can provide meaningful representations of genes, leading to interpretable results on the set of gene pairs recommended by the framework. 

There are several directions for future work. One direction is to integrate pre-trained large language models into the framework for initial recommendation steps, since these pre-trained models have demonstrated the ability to provide valuable information for a warm start; the second direction is to incorporate additional information from the knowledge graph into the model (e.g., node features) to better differentiate the gene nodes; a third direction is to develop an efficient Graph Learning inference method through neighbor subsampling which enables us to utilize the full knowledge graph into our model instead of a subgraph.    

\section{Competing interests}
No competing interest is declared.

\section{Author contributions statement}

H.Z., M.S., J.C.P and M.L. contributed to the implementation of the framework, H.Z., P.R., and B.S. contributed to the development of the methodology of the active learning framework. B.P. and S.B. contributed to the biological interpretation of the findings. J.D. contributed to the process of obtaining and preprocessing of the HIV dataset. All authors collaboratively wrote and reviewed the manuscript.

\section{Acknowledgments}
This work was performed under the auspices of the U.S. Department of Energy by Lawrence Livermore National Laboratory under Contract DE-AC52-07NA27344(LLNL-JRNL-872100). Funding was provided by the Lawrence Livermore National Laboratory Directed Research and Development project 23-SI-005.


\bibliographystyle{plain}
% \bibliographystyle{elsarticle-harv} 
\bibliography{main}

\begin{biography}{{\color{black!20}\rule{77pt}{77pt}}}
{\author{Haonan Zhu}
received the Ph.D. degree in electrical and computer engineering from the University of Michigan in 2023. He is currently a Postdoctoral
Researcher with the Lawrence Livermore National Laboratory.} 
\end{biography}

\begin{biography}{{\color{black!20}\rule{77pt}{77pt}}}
{\author{Mary Silva} received an M.S. in Statistics from University of California, Santa Cruz. She is currently a Data Scientist in the Biomolecular Design and Development Group at Lawrence Livermore National Laboratory.} 
\end{biography}

\begin{biography}{{\color{black!20}\rule{77pt}{77pt}}}
{\author{Jose Cadena} received the Ph.D. degree in computer science from the Department of Computer Science at Virginia Tech. He is currently a Research Staff Member with the Computational Engineering Division at Lawrence Livermore National Laboratory.} 
\end{biography}


\begin{biography}{{\color{black!20}\rule{77pt}{77pt}}}
{\author{Braden Soper} received the Ph.D. degree in Applied Mathematics and Statistics from the University of California, Santa Cruz. He is currently Data Scientist at Lawrence Livermore National Laboratory.} 
\end{biography}



\begin{biography}{{\color{black!20}\rule{77pt}{77pt}}}
{\author{Michał Lisicki} is completing his Ph.D. at the University of Guelph and the Vector Institute in Ontario, Canada, focusing on sequential decision making. He contributed to the present work during his internship at Lawrence Livermore National Laboratory.} 
\end{biography}

\begin{biography}{{\color{black!20}\rule{77pt}{77pt}}}
{\author{Braian Peetoom} earned an M.D. from Universidad Nacional de La Plata in 2022. He is currently a Postdoctoral Researcher at Baranzini Lab in UCSF.} 
\end{biography}

\begin{biography}{{\color{black!20}\rule{77pt}{77pt}}}
{\author{Sergio E Baranzini} received his Ph.D. degree in molecular human genetics from the University of Buenos Aires in 1997. He is currently a distinguished Professor of Neurology at the University of California San Francisco.} 
\end{biography}

\begin{biography}{{\color{black!20}\rule{77pt}{77pt}}}
{\author{Priyadip Ray} received the Ph.D. degree in electrical
engineering from Syracuse University in 2009. He is
currently a Staff Scientist with the Machine Learning Group at Lawrence Livermore National Laboratory.}
\end{biography}

\begin{biography}{{\color{black!20}\rule{77pt}{77pt}}}
{\author{Jeff Drocco} completed his Ph.D. in physics at Princeton University in 2011. He is the Deputy Group Leader for Genomics in the Biosciences and Biotechnology Division at Lawrence Livermore National Laboratory.}
\end{biography}


\end{document}


% TODO: use BMF information from braden below
%In probabilistic matrix factorization \cite{I’ll add these later}, large sparsely observed matrices are modeled as the product of two matrices with additive observation noise. The factored matrices are often assumed to be of much lower rank than the dimension of the observed data matrix. Given observed entries of the data matrix, statistical inference methods like maximum likelihood estimation can be used to learn the factors that best fit the observed data, thus allowing predictions of unobserved entries in the data matrix. Bayesian matrix factorization \cite{I’ll add these later}  involves placing prior distributions on the unknown factors and other model parameters, then leveraging Bayesian inference methods such as Markov chain Monte Carlo (MCMC) methods to approximate posterior distributions of all model parameters. Posterior prediction distributions can then be approximated for all unobserved entries in the data matrix. 
 
Probabilistic matrix factorization is a standard methodology for inferring missing observations in relation data sets, such as in recommender systems. This is a natural approach for genetic interaction data, as we are trying to choose pairs of genes for future knockdown experiments base on very few observed experimental results. In this way we can treat the gene-gene prediction problem as a matrix factorization problem.  In this study we leveraged a Bayesian matrix factorization model so that we can leverage the posterior predictive distributions in the acquisition functions. For example, we can choose gene pairs that that have the lowest posterior predictive mean viral load (a greedy purely approach), or we can choose gene pairs with the greatest posterior predictive variance in viral load (a purely explorative approach). We can also leverage acquisition functions that fall in between these two extremes like Thompson sampling or optimism in the face of uncertainty, e.g. UCB-style algorithms. 
 
The specific hierarchical Bayesian model we use is as follows.


\begin{algorithm}[H]
\label{alg: DeepAL}
\caption{DeepAL}
\SetKwInOut{Require}{Require}
\SetKwFor{For}{for}{do}{end for}
\SetKwInput{KwData}{Require}
\SetKwInput{KwResult}{return}
\DontPrintSemicolon

\Require{knowledge graph \(\mc G=\left( \mc V, \mc E, \mc R \right)\), set of target genes $H$ with $p=\abs{H}$ and collection of gene-pairs of interests \(S \)}
\Require{viral-loads matrix \(\mb Y \in \bb R^{p \times p}\) to query}
\Require{number of rounds of experiments \(T\), number of gene-pairs to be selected per round \(N\)}
\Require{R-GCN model \(f_{\mb \Theta}: G \to \bb R^{\abs{V} \times d}\), DistMulti weight matrix \(\mb R \in \bb R^{d \times d}\), bilinear regression weight matrix \(\mb A \in \bb R^{d \times d}\) and bias \(b \in \bb R\) }
Train \(f_{\mb \Theta}\) and \(\mb R \) using self-supervised learning (see section \ref{subsec: model})  

Initialize $S'$ by randomly drawing \(N\) pairs of genes from \(S\)  

\For{\( t = 1, \ldots, T \) \tcp*{Active Learning Loop}}{

    \(N_{g} \leftarrow \left\lfloor \frac{t N}{T} \right\rfloor \) \tcp*{Number of Greedy Samples}
    \(N_{e} \leftarrow N-N_e\) \tcp*{Number of Exploring Samples}
    Train models on S': \( \min_{\mb \Theta, \mb A, b} \sum_{(i,j) \in S'} \text{Huber}(\hat{Y}_{ij}-Y_{ij})\)  
    
    \(\hat{Y}_{ij} \leftarrow \mb x_{v_i}^\top \mb A \mb x_{v_j} + b\quad \forall (i,j) \in S \setminus S' \) \tcp*{Forward Prediction on unobserved entries}
    \(S_g \leftarrow \set{\left(i,j\right) \in S \setminus S' \mid N_g \text{ smallest values from }\hat{Y}_{ij}}\) \tcp*{Samples chosen based on greedy scheme}
    \(S_e \leftarrow \set{\left(i,j\right) \in S \setminus (S'\cup S_g) \mid N_e \text{ largest values from} \norm{\mb x_{v_i}}{}\norm{\mb x_{v_j}}{}}\) \tcp*{Samples chosen based on exploring scheme} 
    \( S' \leftarrow S' \cup S_g \cup S_e \)\;
    Reset \(\mb A\) \tcp*{Reset the bilinear weight matrix}
}
\KwResult{observed set of gene-pairs \(S' \subseteq S \).}
\end{algorithm}

Our acquisition strategy for the active learning loop is based on a linear annealing schedule that transitions from exploration phase to exploitation phase (see line 5-6 from Algo. \ref{alg: DeepAL}). The exploitation scheme is based on greedy scheme, and the exploration scheme is based on T (Trace)-optimal experiment design \cite{pukelsheim2006optimal}.

\paragraph{$T$(Trace)-optimal Experimental Design} Under linear regression setting where samples of the data matrix are denoted as $\{\mathbf x_i\}_{i=1}^n$, the empirical fisher information (up to a normalization constant) is given by: 
\begin{equation}
\label{eqn: fisher information linear regression}
\mathcal{I} := \sum_{i=1}^n \mathbf{x}_i \mathbf{x}_i^\top.
\end{equation}
The $T$-optimal experimental design aim to maximize the trace of the fisher information matrix, and in the case of linear regression it is equivalent of maximize the frobenius norm of the data matrix. This criteria is computationally efficient since it does not require any matrix inversions nor eigen decompositions, and have shown success in deep active learning setting \cite{ash2019deep, holzmuller2023framework}. See chapter 9 of \cite{pukelsheim2006optimal} for a detailed discussion and its connection to other optimal experiment design criteria.  

%There are two different scenarios where in one case we perform a pre-run screening that we use results from single knock-down experiments to reduce the search space to 25\%. The result is summarized in Fig. \ref{fig: top50 plot}. By incorporate information from the knowledge graph, we see significant improvement of the proposed DeepAL framework: uncovered 98\% of the top 50 gene-pairs by rounds 8 (<6\% of the entire matrix observed). % and uncovered 98\% of the top 50 gene-pairs by rounds 7 (<5\% of the entire matrix observed) if we pruned the search space to \%25 using the diagonal entries. 