Despite the challenges, the field of medical VLMs is ripe with opportunities for innovation. By leveraging large-scale multimodal datasets, improving cross-modal generalization, and prioritizing interpretability and ethical considerations, researchers can develop models that are not only more accurate but also more aligned with clinical needs. These advancements have the potential to transform medical workflows, enhance diagnostic accuracy, and improve patient outcomes, paving the way for a new era of AI-driven healthcare.

\subsection{Scaling and diversifying datasets}
 Scaling and diversifying datasets could address current limitations in data quality and task specificity. For example, Medtrinity-25M’s integration of radiology, pathology, and EHR data provides a template for building comprehensive multimodal datasets that bridge imaging, text, and clinical context \cite{xie2024medtrinity25mlargescalemultimodaldataset}. Similarly, synthetic data generation techniques, as proposed in VividMed, could augment smaller benchmarks like VQA-RAD or SLAKE, enhancing their utility for training robust models \cite{luo2024vividmedvisionlanguagemodel}. Collaborative efforts to expand datasets such as PMC-VQA (1.5M samples) and OmniMedVQA (128k samples) could also improve coverage of rare diseases and underrepresented modalities \cite{Zhang2024,hu2024omnimedvqanewlargescalecomprehensive}.

\subsection{Cross-modal generalization}
Benchmarks like GMAI-MMBench, which evaluate 18 tasks across 38 modalities, highlight the need for models capable of integrating diverse data types \cite{chen2024gmaimmbenchcomprehensivemultimodalevaluation}. Techniques such as contrastive learning, employed in ConVIRT and MedCLIP, could be extended to align features from radiology, pathology, and EHRs, enabling VLMs to perform tasks like correlating imaging findings with lab results \cite{zhang2022contrastivelearningmedicalvisual,wang2020vdbertunifiedvisiondialog}. Additionally, frameworks like LLaMA-Adapter-V2, which use parameter-efficient tuning to adapt models to new modalities, could facilitate rapid deployment across clinical settings without requiring retraining from scratch \cite{gao2023llamaadapterv2parameterefficientvisual}.

\subsection{Interpretability and clinical relevance}
Advancing interpretability and clinical relevance will require rethinking evaluation metrics. For instance, benchmarks such as RadBench and BESTMVQA could incorporate criteria for explainability, such as attention map accuracy or alignment with clinical guidelines \cite{kuo2024radbenchevaluatinglargelanguage,hong2023bestmvqabenchmarkevaluationmedical}. Models like VividMed, which localize anatomical regions during diagnosis, demonstrate how visual grounding mechanisms can make outputs more transparent \cite{luo2024vividmedvisionlanguagemodel}. Similarly, integrating expert feedback loops, as seen in Flamingo-CXR’s human evaluation framework, could refine model outputs to better match clinician expectations \cite{Tanno2024}.

\subsection{Ethical and computational opportunities}
Finally, addressing ethical and computational challenges will be essential for real-world deployment. Federated learning approaches, as explored in BiomedGPT, could enable training on distributed datasets like Medtrinity-25M without compromising patient privacy \cite{Zhang2024}. Lightweight architectures such as DeepSeek-VL (1.3B parameters) and LLaMA-Adapter-V2, which reduce computational costs through token compression and parameter-efficient tuning, offer pathways to democratize access to VLMs in resource-constrained settings \cite{lu2024deepseekvlrealworldvisionlanguageunderstanding,gao2023llamaadapterv2parameterefficientvisual}. By aligning technical advancements with clinical needs—such as seamless EHR integration and real-time decision support—future VLMs could transform medical workflows while adhering to ethical and regulatory standards.
