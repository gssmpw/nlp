\section{RELATED WORK}
Diffusion Models: Diffusion models operate on the fundamental principle of simulating the gradual degradation (by adding noise) and subsequent restoration (through denoising) of data. Due to their realistic and diverse generation capabilities, diffusion models have undergone rapid advancements recently. This progress has led to the development of various notable models, including DDPM ____ and DDIM ____. Peebles et al. ____ deviated from the traditional UNet-based diffusion models by proposing a Transformer-based diffusion model, DIT, which achieved improved generation results. Unlike the DIT, which retains the encoder architecture, our model maintains an encoder-decoder structure and modifies the decoder to better integrate the guiding conditions.

Diffusion Policy: DP ____ involves applying diffusion models to robot control, generating executable action trajectories for robots. This method can learn multiple behavior patterns from demonstration data, offering better generalization and achieving widespread adoption in the robotics field. The STMDP model ____ applies diffusion policy to spiking neural networks, constructing a robot control model based on the Spiking Transformer. The OCTO model ____ develops a novel Transformer-based universal robot policy model and incorporates a lightweight action head to implement diffusion policy, effectively converting model outputs into actions. The Robotics Diffusion Transformer (RDT) ____ introduces a physically interpretable unified action space that standardizes action representations across various robots while preserving the physical meanings of the original actions. In terms of model architecture, RDT builds a diffusion foundational model based on the Transformer and incorporating a branch structure from DIT. Compared to OCTO, which employs a lightweight action head to realize diffusion policy, Hou et al. ____ developed a large-scale diffusion policy model based on DIT for universal robot policy learning. Our work focuses on exploring the efficient structure of Transformers suitable for the diffusion policy.

%Our work focuses on proposing a Transformer model suitable for the diffusion policy, rather than constructing a large-scale diffusion model.

\begin{figure*}[htbp]
	\centering
	\includegraphics[width=0.9\textwidth]{f3.pdf} %
	\caption{Tasks used for the experiment. }
	\label{fig3}
\end{figure*}