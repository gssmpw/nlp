\begin{figure}
    \centering
\begin{tikzpicture}[scale=0.6,decoration={
    markings,
    mark=at position 0.50 with {\arrow{>}}}]

%%%%%%%%%%%%%%%%%%%%%%%%%%%%%%

\begin{scope}[shift={(0,0)}]
%\draw[thin,yellow] (0,0) grid (4,4);

\draw[thick,dashed] (0.5,4) -- (2.5,4);

\draw[thick,dashed] (0.5,0) -- (2.5,0);

\node at (1.5,4.5) {$-p$};

\node at (1.5,-0.5) {$p$};


\draw[thick,postaction={decorate}] (1.5,4) -- (1.5,2);

\draw[thick,fill] (1.65,2) arc (0:360:1.5mm);


\draw[thick,postaction={decorate}] (1.5,0) -- (1.5,2);

\node at (3.5,2) {$=$};

\end{scope}

%%%%%%%%%%%%%%%%%%%%%%%%%%%%%%

\begin{scope}[shift={(4.5,0)}]
%\draw[thin,yellow] (0,0) grid (3,4);

\draw[thick,dashed] (0,4) -- (3,4);

\draw[thick,dashed] (0,0) -- (3,0);

\node at (0.5,4.5) {$-p$};

\node at (0.5,-0.5) {$p$};

\node at (2.5,4.5) {$0$};

\node at (2.25,2.55) {$0$};


\draw[thick,postaction={decorate}] (0.5,4) -- (0.5,2);

\draw[thick,fill] (0.65,2) arc (0:360:1.5mm);

\draw[thick,postaction={decorate}] (0.5,2) .. controls (0.75,2) and (2.25,2.5) .. (2.5,4);

\draw[thick,postaction={decorate}] (0.5,0) -- (0.5,2);

\end{scope}

%%%%%%%%%%%%%%%%%%%%%%%%%%%%%%
 

\begin{scope}[shift={(12,0)}]
%\draw[thin,yellow] (0,0) grid (4,4);

\draw[thick,dashed] (0,4) -- (4,4);

\draw[thick,dashed] (0,0) -- (4,0);

\node at (0.5,-0.5) {$p$};

\node at (3.5,-0.5) {$-p$};


\draw[thick,postaction={decorate}] (0.5,0) -- (2,2.25);

\draw[thick,postaction={decorate}] (3.5,0) -- (2,2.25);

\draw[thick,postaction={decorate}] (2,2.25) -- (2,4);


\node at (2,4.5) {$0$};

\node at (2.5,3) {$0$};

\node at (5,2) {$=$}; 
\end{scope}

%%%%%%%%%%%%%%%%%%%%%%%%%%%%%%

\begin{scope}[shift={(19,0)}]
%\draw[thin,yellow] (0,0) grid (4,4);

\draw[thick,dashed] (0,4) -- (4,4);

\draw[thick,dashed] (0,0) -- (4,0);

\draw[thick,fill] (2.15,2.25) arc (0:360:1.5mm);

\node at (0.5,-0.5) {$p$};

\node at (3.5,-0.5) {$-p$};

\draw[thick,postaction={decorate}] (0.5,0) -- (2,2.25);

\draw[thick,postaction={decorate}] (3.5,0) -- (2,2.25);


\end{scope}

%%%%%%%%%%%%%%%%%%%%%%%%%%%%%%
 





\end{tikzpicture}
    \caption{We introduce a dot to represent reversal of orientation. Left: we can erase the $0$-line. The 0-line can go in or out, to the top or bottom boundary.
    Right: we have $\langle p, -p\rangle =0$.}
    \label{fig7_004}
\end{figure}