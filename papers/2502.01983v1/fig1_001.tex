\begin{figure}
    \centering
\begin{tikzpicture}[scale=0.6,decoration={
    markings,
    mark=at position 0.65 with {\arrow{>}}}]
\begin{scope}[shift={(0,0)}]
%\draw[thin,yellow] (0,0) grid (5,9);
%\draw[thin,gray] (5,0) grid (10,9);
%\draw[thin,yellow] (10,0) grid (15,9);
%\draw[thin,gray] (15,0) grid (20,9);
%\draw[thin,yellow] (20,0) grid (25,9);

\draw[thick,dashed] (0,9) -- (25.5,9);
\draw[thick,dashed] (0,0) -- (25.5,0);

\draw[thick,postaction={decorate}] (2,0) .. controls (2.1,1) and (2.4,1.5) .. (3,2.5);

\draw[thick,postaction={decorate}] (4,0) .. controls (3.9,1) and (3.6,1.5) .. (3,2.5);

\draw[thick,postaction={decorate}] (3,2.5) .. controls (3.1,2.75) and (3.4,3.0) .. (4,3.5);

\draw[thick,postaction={decorate}] (8,0) .. controls (7,2.5) and (5.7,3) .. (4,3.5);

\draw[thick,postaction={decorate}] (4,3.5) -- (4,4.5);

\draw[thick] (4,4.5) -- (4,5);

\draw[thick,postaction={decorate}] (6,0) .. controls (6.5,2.25) and (8,2.5) .. (9,2.5);

\draw[thick,postaction={decorate}] (10,0) .. controls (9.9,1) and (9.6,1.5) .. (9,2.5);

\draw[thick,postaction={decorate}] (9,2.5) -- (9,4.75);

\draw[line width=0.55mm,dotted,red!100!green!36!blue!100] (6.25,2.60) ellipse (5 and 2);

\draw[line width=0.63mm, densely dotted,blue] (6.25,3.42) ellipse (6 and 3.18);

\draw[thick,postaction={decorate}] (4,5) .. controls (3.9,5.1) and (3.5,5.2) .. (3,6.5);

\draw[thick,postaction={decorate}] (4,5) .. controls (4.1,5.00) and (4.5,5.10) .. (5,5.40);

\draw[thick,postaction={decorate}] (3,6.5) .. controls (2.8,7.5) and (2.3,6.75) .. (2,9);

\draw[thick,postaction={decorate}] (9,4.75) .. controls (7,5) and (5,5.5) .. (4.5,6.5);

\draw[thick,postaction={decorate}] (5,5.40) .. controls (5.2,5.6) and (6.5,6) .. (8,9);

\draw[thick,postaction={decorate}] (4.5,6.5) .. controls (4.6,7.25) and (4.7,7.25) .. (6,9);

\draw[thick,postaction={decorate}] (9,4.75) .. controls (11,6.5) and (10,7.5) .. (7.45,8);

\draw[thick,postaction={decorate}] (9.8,7) .. controls (10.5,7.2) and (11,7.5) .. (12,9);

\draw[line width=0.50mm,red] (4,9) decorate [decoration={snake,amplitude=0.20mm}] {.. controls (4.25,6) and (9,8.5) .. (12,5.5)};

\draw[line width=0.50mm,red] (12,5.5) decorate [decoration={snake,amplitude=0.20mm}] {.. controls (13,4.75) and (13.5,4) .. (14,0)};

\draw[line width=0.50mm,red] (3.65,8.1) -- (4.15,8.25);
\node at (3.6,8.55) {$c_2$};

\draw[line width=0.50mm,red] (13.45,1.25) -- (13.95,1.25);
\node at (13.4,1.75) {$c_1$};

\draw[thick,postaction={decorate}] (10.75,7.5) .. controls (13.5,7) and (21,6) .. (22,0);

\draw[line width=0.50mm,red] (12,5.5) decorate [decoration={snake,amplitude=0.20mm}] {.. controls (12.5,6) and (13,6.5) .. (14,9)};

\draw[line width=0.50mm,red] (13.1,8) -- (13.6,8);
\node at (14.65,8.1) {$c_2^{-1}c_1$};

\draw[thick,postaction={decorate}] (16,0) .. controls (16.5,2) and (19.5,2) .. (20,0);

\draw[line width=0.50mm,red] (18,0) decorate [decoration={snake,amplitude=0.20mm}] {.. controls (18.5,2) and (19,7) .. (20,9)};

\draw[line width=0.50mm,red] (19,7.5) -- (19.5,7.5);
\node at (20,7.5) {$c_3$};

\draw[thick,<-] (17,3.5) arc (0:360:1);
\node at (15,4.5) {$p_j$};

\begin{scope}[shift={(-0.25,0)}]
\draw[thick,<-] (24,6) arc (0:180:1.5);
\draw[thick,<-] (21,6) arc (180:360:1.5);

\draw[thick,postaction={decorate}] (22.75,4.52) arc (-110:-260:1.5);
\node at (20.85,4.9) {$p_k$};
\node at (22.35,5.75) {$p_{\ell}$};
\end{scope}
\node at (24.40,4.60) {$p_k+p_{\ell}$};

\node at (2,-0.75) {$p_1$};
\node at (4,-0.75) {$p_2$};
\node at (6,-0.75) {$\ldots$};
\node at (8,-0.75) {$\ldots$};
\node at (10,-0.75) {$p_n$};

\node at (16,-0.75) {$p_{n+1}$};
\node at (20,-0.75) {$\ldots$};
\node at (22,-0.75) {$p_m$};
\node at (24,-0.75) {$\ldots$};

\draw[thick,->] (1,-2) .. controls (0.5,0) and (0.5,1.5) .. (2.25,2.5);
\node at (2,-2.5) {probability for};
\node at (2.25,-3.5) {classical physics};
\node at (6.25,-3.5) {$\displaystyle{\sum_{i=1}^n} \: p_i=1$};

\draw[thick,->] (1.5,10) .. controls (0,8) and (0,6) .. (2,4.5);
\node at (1.60,10.5) {probability};

\end{scope}
\end{tikzpicture}
    \caption{The diagram shows a morphism (cobordism). Small purple ellipse encloses a part of the morphism that admits an interpretation in earlier categorical approaches to entropy. Larger blue ellipse encloses a part of the diagram that can be interpreted in classical probability. Parts of the diagram that include wavy (red) multiplicative lines, U-turns of additive lines and additive lines pointing down or closing upon themselves make sense in a larger category, but not in the familiar classical probability, since rescaling is involved and one could remove the renormalization to $1$. Multiplicative lines scale labels of additive lines.}
    \label{fig1_001}
\end{figure}