%%%%%%%% ICML 2025 EXAMPLE LATEX SUBMISSION FILE %%%%%%%%%%%%%%%%%

\documentclass{article}

% Recommended, but optional, packages for figures and better typesetting:
\usepackage{microtype}
\usepackage{graphicx}
\usepackage{subfigure}
\usepackage{csquotes}
% \usepackage[nolist]{lineno}
\usepackage{booktabs} % for
\usepackage{wrapfig}
\usepackage[utf8]{inputenc}
\usepackage{tikz}
\usepackage{array}
\usepackage{listings}
\usepackage{float}
\usepackage{bbm}
\usepackage{fbox}
\usepackage{amsmath}
\usepackage{amsfonts}
\usepackage{amssymb}
\usepackage{mdframed}
\usepackage{xcolor}
\usepackage{soul}
\usepackage{pifont}
\usepackage{rotating}
\usepackage{framed}
%professional tables

% hyperref makes hyperlinks in the resulting PDF.
% If your build breaks (sometimes temporarily if a hyperlink spans a page)
% please comment out the following usepackage line and replace
% \usepackage{icml2025} with \usepackage[nohyperref]{icml2025} above.
\usepackage[backref=page]{hyperref}

% Attempt to make hyperref and algorithmic work together better:
\newcommand{\theHalgorithm}{\arabic{algorithm}}

% Use the following line for the initial blind version submitted for review:
% \usepackage{icml2025}

% If accepted, instead use the following line for the camera-ready submission:
\usepackage[accepted]{icml2025}
% For theorems and such
\usepackage{amsmath}
\usepackage{amssymb}
\usepackage{mathtools}
\usepackage{amsthm}

% if you use cleveref..
\usepackage[capitalize,noabbrev]{cleveref}

%%%%%%%%%%%%%%%%%%%%%%%%%%%%%%%%
% THEOREMS
%%%%%%%%%%%%%%%%%%%%%%%%%%%%%%%%
\theoremstyle{plain}
\newtheorem{theorem}{Theorem}[section]
\newtheorem{proposition}[theorem]{Proposition}
\newtheorem{lemma}[theorem]{Lemma}
\newtheorem{corollary}[theorem]{Corollary}
\theoremstyle{definition}
\newtheorem{definition}[theorem]{Definition}
\newtheorem{assumption}[theorem]{Assumption}
\theoremstyle{remark}
\newtheorem{remark}[theorem]{Remark}

% Todonotes is useful during development; simply uncomment the next line
%    and comment out the line below the next line to turn off comments
%\usepackage[disable,textsize=tiny]{todonotes}
\usepackage[textsize=tiny]{todonotes}


% The \icmltitle you define below is probably too long as a header.
% Therefore, a short form for the running title is supplied here:
\icmltitlerunning{Agentic LLM Unlearning}
\DeclareUnicodeCharacter{2212}{-}

\begin{document}
\twocolumn[
\icmltitle{\texttt{ALU}: Agentic LLM Unlearning}

% It is OKAY to include author information, even for blind
% submissions: the style file will automatically remove it for you
% unless you've provided the [accepted] option to the icml2025
% package.

% List of affiliations: The first argument should be a (short)
% identifier you will use later to specify author affiliations
% Academic affiliations should list Department, University, City, Region, Country
% Industry affiliations should list Company, City, Region, Country

% You can specify symbols, otherwise they are numbered in order.
% Ideally, you should not use this facility. Affiliations will be numbered
% in order of appearance and this is the preferred way.
\icmlsetsymbol{equal}{*}

\begin{icmlauthorlist}
\icmlauthor{Debdeep Sanyal}{yyy}
\icmlauthor{Murari Mandal}{yyy}
% \icmlauthor{Firstname3 Lastname3}{comp}
% \icmlauthor{Firstname4 Lastname4}{sch}
% \icmlauthor{Firstname5 Lastname5}{yyy}
% \icmlauthor{Firstname6 Lastname6}{sch,yyy,comp}
% \icmlauthor{Firstname7 Lastname7}{comp}
%\icmlauthor{}{sch}
% \icmlauthor{Firstname8 Lastname8}{sch}
% \icmlauthor{Firstname8 Lastname8}{yyy,comp}
%\icmlauthor{}{sch}
%\icmlauthor{}{sch}
\end{icmlauthorlist}

\icmlaffiliation{yyy}{RespAI Lab, KIIT Bhubaneswar, India}
% \icmlaffiliation{comp}{Company Name, Location, Country}
% \icmlaffiliation{sch}{School of ZZZ, Institute of WWW, Location, Country}

\icmlcorrespondingauthor{Murari Mandal}{murari.nus@gmail.com}
% \icmlcorrespondingauthor{Firstname2 Lastname2}{first2.last2@www.uk}

% You may provide any keywords that you
% find helpful for describing your paper; these are used to populate
% the "keywords" metadata in the PDF but will not be shown in the document
\icmlkeywords{Machine Learning, ICML}

\vskip 0.3in
]

% this must go after the closing bracket ] following \twocolumn[ ...

% This command actually creates the footnote in the first column
% listing the affiliations and the copyright notice.
% The command takes one argument, which is text to display at the start of the footnote.
% The \icmlEqualContribution command is standard text for equal contribution.
% Remove it (just {}) if you do not need this facility.

\printAffiliationsAndNotice{}  % leave blank if no need to mention equal contribution

% \printAffiliationsAndNotice{\icmlEqualContribution} % otherwise use the standard text.
%\title{Formatting Instructions for ICLR 2025 \\ Conference Submissions}

% Authors must not appear in the submitted version. They should be hidden
% as long as the \iclrfinalcopy macro remains commented out below.
% Non-anonymous submissions will be rejected without review.



% The \author macro works with any number of authors. There are two commands
% used to separate the names and addresses of multiple authors: \And and \AND.
%
% Using \And between authors leaves it to \LaTeX{} to determine where to break
% the lines. Using \AND forces a linebreak at that point. So, if \LaTeX{}
% puts 3 of 4 authors names on the first line, and the last on the second
% line, try using \AND instead of \And before the third author name.

\newcommand{\fix}{\marginpar{FIX}}
\newcommand{\new}{\marginpar{NEW}}

\newcommand{\STAB}[1]{\begin{tabular}{@{}c@{}}#1\end{tabular}}

%\usepackage[pagebackref,breaklinks,colorlinks]{hyperref}
%\usepackage{url}

% added


%\section*{Metadata}
\label{sec:metadata}

    \newcommand{\ourshorttitle}{
Spectral-factorized Positive-definite Curvature Learning for NN Training
}
\newcommand{\ourtitle}{
Spectral-factorized Positive-definite Curvature Learning for NN Training
}
\icmltitlerunning{\ourshorttitle}

\twocolumn[

\icmltitle{\ourtitle}

%
%
%
%

%
%
%
%

%
%
%
%

\begin{icmlauthorlist}
\icmlauthor{Wu Lin}{vector}
\icmlauthor{Felix Dangel}{vector}
\icmlauthor{Runa Eschenhagen}{cambridge}
\icmlauthor{Juhan Bae}{vector,ut}
\icmlauthor{Richard E. Turner}{cambridge}
\icmlauthor{
Roger B. Grosse
}{vector,ut}
\end{icmlauthorlist}

%
\icmlaffiliation{vector}{Vector Institute, Canada}
\icmlaffiliation{cambridge}{Cambridge University, United Kingdom}
\icmlaffiliation{ut}{University of Toronto, Canada}

\icmlcorrespondingauthor{Wu Lin}{yorker.lin@gmail.com \vspace{-0.2cm}}

%
%
%
\icmlkeywords{Natural Gradient Descent, Second Order Method, Optimization, Deep Learning}

\vskip 0.3in
]

%

%
%
%
%
%

\printAffiliationsAndNotice{}  %
%

%
%
%
%


%\maketitle
\begin{abstract}


The choice of representation for geographic location significantly impacts the accuracy of models for a broad range of geospatial tasks, including fine-grained species classification, population density estimation, and biome classification. Recent works like SatCLIP and GeoCLIP learn such representations by contrastively aligning geolocation with co-located images. While these methods work exceptionally well, in this paper, we posit that the current training strategies fail to fully capture the important visual features. We provide an information theoretic perspective on why the resulting embeddings from these methods discard crucial visual information that is important for many downstream tasks. To solve this problem, we propose a novel retrieval-augmented strategy called RANGE. We build our method on the intuition that the visual features of a location can be estimated by combining the visual features from multiple similar-looking locations. We evaluate our method across a wide variety of tasks. Our results show that RANGE outperforms the existing state-of-the-art models with significant margins in most tasks. We show gains of up to 13.1\% on classification tasks and 0.145 $R^2$ on regression tasks. All our code and models will be made available at: \href{https://github.com/mvrl/RANGE}{https://github.com/mvrl/RANGE}.

\end{abstract}


\section{Introduction}

Video generation has garnered significant attention owing to its transformative potential across a wide range of applications, such media content creation~\citep{polyak2024movie}, advertising~\citep{zhang2024virbo,bacher2021advert}, video games~\citep{yang2024playable,valevski2024diffusion, oasis2024}, and world model simulators~\citep{ha2018world, videoworldsimulators2024, agarwal2025cosmos}. Benefiting from advanced generative algorithms~\citep{goodfellow2014generative, ho2020denoising, liu2023flow, lipman2023flow}, scalable model architectures~\citep{vaswani2017attention, peebles2023scalable}, vast amounts of internet-sourced data~\citep{chen2024panda, nan2024openvid, ju2024miradata}, and ongoing expansion of computing capabilities~\citep{nvidia2022h100, nvidia2023dgxgh200, nvidia2024h200nvl}, remarkable advancements have been achieved in the field of video generation~\citep{ho2022video, ho2022imagen, singer2023makeavideo, blattmann2023align, videoworldsimulators2024, kuaishou2024klingai, yang2024cogvideox, jin2024pyramidal, polyak2024movie, kong2024hunyuanvideo, ji2024prompt}.


In this work, we present \textbf{\ours}, a family of rectified flow~\citep{lipman2023flow, liu2023flow} transformer models designed for joint image and video generation, establishing a pathway toward industry-grade performance. This report centers on four key components: data curation, model architecture design, flow formulation, and training infrastructure optimization—each rigorously refined to meet the demands of high-quality, large-scale video generation.


\begin{figure}[ht]
    \centering
    \begin{subfigure}[b]{0.82\linewidth}
        \centering
        \includegraphics[width=\linewidth]{figures/t2i_1024.pdf}
        \caption{Text-to-Image Samples}\label{fig:main-demo-t2i}
    \end{subfigure}
    \vfill
    \begin{subfigure}[b]{0.82\linewidth}
        \centering
        \includegraphics[width=\linewidth]{figures/t2v_samples.pdf}
        \caption{Text-to-Video Samples}\label{fig:main-demo-t2v}
    \end{subfigure}
\caption{\textbf{Generated samples from \ours.} Key components are highlighted in \textcolor{red}{\textbf{RED}}.}\label{fig:main-demo}
\end{figure}


First, we present a comprehensive data processing pipeline designed to construct large-scale, high-quality image and video-text datasets. The pipeline integrates multiple advanced techniques, including video and image filtering based on aesthetic scores, OCR-driven content analysis, and subjective evaluations, to ensure exceptional visual and contextual quality. Furthermore, we employ multimodal large language models~(MLLMs)~\citep{yuan2025tarsier2} to generate dense and contextually aligned captions, which are subsequently refined using an additional large language model~(LLM)~\citep{yang2024qwen2} to enhance their accuracy, fluency, and descriptive richness. As a result, we have curated a robust training dataset comprising approximately 36M video-text pairs and 160M image-text pairs, which are proven sufficient for training industry-level generative models.

Secondly, we take a pioneering step by applying rectified flow formulation~\citep{lipman2023flow} for joint image and video generation, implemented through the \ours model family, which comprises Transformer architectures with 2B and 8B parameters. At its core, the \ours framework employs a 3D joint image-video variational autoencoder (VAE) to compress image and video inputs into a shared latent space, facilitating unified representation. This shared latent space is coupled with a full-attention~\citep{vaswani2017attention} mechanism, enabling seamless joint training of image and video. This architecture delivers high-quality, coherent outputs across both images and videos, establishing a unified framework for visual generation tasks.


Furthermore, to support the training of \ours at scale, we have developed a robust infrastructure tailored for large-scale model training. Our approach incorporates advanced parallelism strategies~\citep{jacobs2023deepspeed, pytorch_fsdp} to manage memory efficiently during long-context training. Additionally, we employ ByteCheckpoint~\citep{wan2024bytecheckpoint} for high-performance checkpointing and integrate fault-tolerant mechanisms from MegaScale~\citep{jiang2024megascale} to ensure stability and scalability across large GPU clusters. These optimizations enable \ours to handle the computational and data challenges of generative modeling with exceptional efficiency and reliability.


We evaluate \ours on both text-to-image and text-to-video benchmarks to highlight its competitive advantages. For text-to-image generation, \ours-T2I demonstrates strong performance across multiple benchmarks, including T2I-CompBench~\citep{huang2023t2i-compbench}, GenEval~\citep{ghosh2024geneval}, and DPG-Bench~\citep{hu2024ella_dbgbench}, excelling in both visual quality and text-image alignment. In text-to-video benchmarks, \ours-T2V achieves state-of-the-art performance on the UCF-101~\citep{ucf101} zero-shot generation task. Additionally, \ours-T2V attains an impressive score of \textbf{84.85} on VBench~\citep{huang2024vbench}, securing the top position on the leaderboard (as of 2025-01-25) and surpassing several leading commercial text-to-video models. Qualitative results, illustrated in \Cref{fig:main-demo}, further demonstrate the superior quality of the generated media samples. These findings underscore \ours's effectiveness in multi-modal generation and its potential as a high-performing solution for both research and commercial applications.
\section{Related Work}
\label{sec:related-works}
\subsection{Novel View Synthesis}
Novel view synthesis is a foundational task in the computer vision and graphics, which aims to generate unseen views of a scene from a given set of images.
% Many methods have been designed to solve this problem by posing it as 3D geometry based rendering, where point clouds~\cite{point_differentiable,point_nfs}, mesh~\cite{worldsheet,FVS,SVS}, planes~\cite{automatci_photo_pop_up,tour_into_the_picture} and multi-plane images~\cite{MINE,single_view_mpi,stereo_magnification}, \etal
Numerous methods have been developed to address this problem by approaching it as 3D geometry-based rendering, such as using meshes~\cite{worldsheet,FVS,SVS}, MPI~\cite{MINE,single_view_mpi,stereo_magnification}, point clouds~\cite{point_differentiable,point_nfs}, etc.
% planes~\cite{automatci_photo_pop_up,tour_into_the_picture}, 


\begin{figure*}[!t]
    \centering
    \includegraphics[width=1.0\linewidth]{figures/overview-v7.png}
    %\caption{\textbf{Overview.} Given a set of images, our method obtains both camera intrinsics and extrinsics, as well as a 3DGS model. First, we obtain the initial camera parameters, global track points from image correspondences and monodepth with reprojection loss. Then we incorporate the global track information and select Gaussian kernels associated with track points. We jointly optimize the parameters $K$, $T_{cw}$, 3DGS through multi-view geometric consistency $L_{t2d}$, $L_{t3d}$, $L_{scale}$ and photometric consistency $L_1$, $L_{D-SSIM}$.}
    \caption{\textbf{Overview.} Given a set of images, our method obtains both camera intrinsics and extrinsics, as well as a 3DGS model. During the initialization, we extract the global tracks, and initialize camera parameters and Gaussians from image correspondences and monodepth with reprojection loss. We determine Gaussian kernels with recovered 3D track points, and then jointly optimize the parameters $K$, $T_{cw}$, 3DGS through the proposed global track constraints (i.e., $L_{t2d}$, $L_{t3d}$, and $L_{scale}$) and original photometric losses (i.e., $L_1$ and $L_{D-SSIM}$).}
    \label{fig:overview}
\end{figure*}

Recently, Neural Radiance Fields (NeRF)~\cite{2020NeRF} provide a novel solution to this problem by representing scenes as implicit radiance fields using neural networks, achieving photo-realistic rendering quality. Although having some works in improving efficiency~\cite{instant_nerf2022, lin2022enerf}, the time-consuming training and rendering still limit its practicality.
Alternatively, 3D Gaussian Splatting (3DGS)~\cite{3DGS2023} models the scene as explicit Gaussian kernels, with differentiable splatting for rendering. Its improved real-time rendering performance, lower storage and efficiency, quickly attract more attentions.
% Different from NeRF-based methods which need MLPs to model the scene and huge computational cost for rendering, 3DGS has stronger real-time performance, higher storage and computational efficiency, benefits from its explicit representation and gradient backpropagation.

\subsection{Optimizing Camera Poses in NeRFs and 3DGS}
Although NeRF and 3DGS can provide impressive scene representation, these methods all need accurate camera parameters (both intrinsic and extrinsic) as additional inputs, which are mostly obtained by COLMAP~\cite{colmap2016}.
% This strong reliance on COLMAP significantly limits their use in real-world applications, so optimizing the camera parameters during the scene training becomes crucial.
When the prior is inaccurate or unknown, accurately estimating camera parameters and scene representations becomes crucial.

% In early works, only photometric constraints are used for scene training and camera pose estimation. 
% iNeRF~\cite{iNerf2021} optimizes the camera poses based on a pre-trained NeRF model.
% NeRFmm~\cite{wang2021nerfmm} introduce a joint optimization process, which estimates the camera poses and trains NeRF model jointly.
% BARF~\cite{barf2021} and GARF~\cite{2022GARF} provide new positional encoding strategy to handle with the gradient inconsistency issue of positional embedding and yield promising results.
% However, they achieve satisfactory optimization results when only the pose initialization is quite closed to the ground-truth, as the photometric constrains can only improve the quality of camera estimation within a small range.
% Later, more prior information of geometry and correspondence, \ie monocular depth and feature matching, are introduced into joint optimisation to enhance the capability of camera poses estimation.
% SC-NeRF~\cite{SCNeRF2021} minimizes a projected ray distance loss based on correspondence of adjacent frames.
% NoPe-NeRF~\cite{bian2022nopenerf} chooses monocular depth maps as geometric priors, and defines undistorted depth loss and relative pose constraints for joint optimization.
In earlier studies, scene training and camera pose estimation relied solely on photometric constraints. iNeRF~\cite{iNerf2021} refines the camera poses using a pre-trained NeRF model. NeRFmm~\cite{wang2021nerfmm} introduces a joint optimization approach that simultaneously estimates camera poses and trains the NeRF model. BARF~\cite{barf2021} and GARF~\cite{2022GARF} propose a new positional encoding strategy to address the gradient inconsistency issues in positional embedding, achieving promising results. However, these methods only yield satisfactory optimization when the initial pose is very close to the ground truth, as photometric constraints alone can only enhance camera estimation quality within a limited range. Subsequently, 
% additional prior information on geometry and correspondence, such as monocular depth and feature matching, has been incorporated into joint optimization to improve the accuracy of camera pose estimation. 
SC-NeRF~\cite{SCNeRF2021} minimizes a projected ray distance loss based on correspondence between adjacent frames. NoPe-NeRF~\cite{bian2022nopenerf} utilizes monocular depth maps as geometric priors and defines undistorted depth loss and relative pose constraints.

% With regard to 3D Gaussian Splatting, CF-3DGS~\cite{CF-3DGS-2024} also leverages mono-depth information to constrain the optimization of local 3DGS for relative pose estimation and later learn a global 3DGS progressively in a sequential manner.
% InstantSplat~\cite{fan2024instantsplat} focus on sparse view scenes, first use DUSt3R~\cite{dust3r2024cvpr} to generate a set of densely covered and pixel-aligned points for 3D Gaussian initialization, then introduce a parallel grid partitioning strategy in joint optimization to speed up.
% % Jiang et al.~\cite{Jiang_2024sig} proposed to build the scene continuously and progressively, to next unregistered frame, they use registration and adjustment to adjust the previous registered camera poses and align unregistered monocular depths, later refine the joint model by matching detected correspondences in screen-space coordinates.
% \gjh{Jiang et al.~\cite{Jiang_2024sig} also implemented an incremental approach for reconstructing camera poses and scenes. Initially, they perform feature matching between the current image and the image rendered by a differentiable surface renderer. They then construct matching point errors, depth errors, and photometric errors to achieve the registration and adjustment of the current image. Finally, based on the depth map, the pixels of the current image are projected as new 3D Gaussians. However, this method still exhibits limitations when dealing with complex scenes and unordered images.}
% % CG-3DGS~\cite{sun2024correspondenceguidedsfmfree3dgaussian} follows CF-3DGS, first construct a coarse point cloud from mono-depth maps to train a 3DGS model, then progressively estimate camera poses based on this pre-trained model by constraining the correspondences between rendering view and ground-truth.
% \gjh{Similarly, CG-3DGS~\cite{sun2024correspondenceguidedsfmfree3dgaussian} first utilizes monocular depth estimation and the camera parameters from the first frame to initialize a set of 3D Gaussians. It then progressively estimates camera poses based on this pre-trained model by constraining the correspondences between the rendered views and the ground truth.}
% % Free-SurGS~\cite{freesurgs2024} matches the projection flow derived from 3D Gaussians with optical flow to estimate the poses, to compensate for the limitations of photometric loss.
% \gjh{Free-SurGS~\cite{freesurgs2024} introduces the first SfM-free 3DGS approach for surgical scene reconstruction. Due to the challenges posed by weak textures and photometric inconsistencies in surgical scenes, Free-SurGS achieves pose estimation by minimizing the flow loss between the projection flow and the optical flow. Subsequently, it keeps the camera pose fixed and optimizes the scene representation by minimizing the photometric loss, depth loss and flow loss.}
% \gjh{However, most current works assume camera intrinsics are known and primarily focus on optimizing camera poses. Additionally, these methods typically rely on sequentially ordered image inputs and incrementally optimize camera parameters and scene representation. This inevitably leads to drift errors, preventing the achievement of globally consistent results. Our work aims to address these issues.}

Regarding 3D Gaussian Splatting, CF-3DGS~\cite{CF-3DGS-2024} utilizes mono-depth information to refine the optimization of local 3DGS for relative pose estimation and subsequently learns a global 3DGS in a sequential manner. InstantSplat~\cite{fan2024instantsplat} targets sparse view scenes, initially employing DUSt3R~\cite{dust3r2024cvpr} to create a densely covered, pixel-aligned point set for initializing 3D Gaussian models, and then implements a parallel grid partitioning strategy to accelerate joint optimization. Jiang \etal~\cite{Jiang_2024sig} develops an incremental method for reconstructing camera poses and scenes, but it struggles with complex scenes and unordered images. 
% Similarly, CG-3DGS~\cite{sun2024correspondenceguidedsfmfree3dgaussian} progressively estimates camera poses using a pre-trained model by aligning the correspondences between rendered views and actual scenes. Free-SurGS~\cite{freesurgs2024} pioneers an SfM-free 3DGS method for reconstructing surgical scenes, overcoming challenges such as weak textures and photometric inconsistencies by minimizing the discrepancy between projection flow and optical flow.
%\pb{SF-3DGS-HT~\cite{ji2024sfmfree3dgaussiansplatting} introduced VFI into training as additional photometric constraints. They separated the whole scene into several local 3DGS models and then merged them hierarchically, which leads to a significant improvement on simple and dense view scenes.}
HT-3DGS~\cite{ji2024sfmfree3dgaussiansplatting} interpolates frames for training and splits the scene into local clips, using a hierarchical strategy to build 3DGS model. It works well for simple scenes, but fails with dramatic motions due to unstable interpolation and low efficiency.
% {While effective for simple scenes, it struggles with dramatic motion due to unstable view interpolation and suffers from low computational efficiency.}

However, most existing methods generally depend on sequentially ordered image inputs and incrementally optimize camera parameters and 3DGS, which often leads to drift errors and hinders achieving globally consistent results. Our work seeks to overcome these limitations.

% \section{Preliminaries}
% \textbf{Agentic LLM workflow.} Similar to how humans are more efficient in a well-coordinated group, language models also benefit from having a team of peer agents which contribute to the task and make the overall workflow more efficient. Some common practices involve decomposing the task into multiple subtasks, which are assigned to one or more agents for completion. While this makes the pipeline more involved, it vastly enhances the framework's efficiency. Creating agents specifically prompted to be efficient at that subtask is analogous to having multiple \emph{expert agents} who work in harmony, unlike a single generalized agent for the entire workflow. Moreover, LLM agents can also act as reviewer to process or evaluate responses and actions from other agents~\cite{zhuge2024agentasajudgeevaluateagentsagents}. This alleviates human evaluation in certain scenarios requiring significant time and compute. Agents can also be employed for guardrailing, preventing adversarial attacks on the framework in attempts to extract sensitive information. Section \ref{sec:5.3} highlights the efficacy of multi-agent frameworks against jailbreaking techniques.

% \textbf{LLM Unlearning.} Given a set $S = \{s_1, s_2, \cdots, s_N\}$ of $N$ unlearning targets and a user query $x \in \mathcal{X}$, the principle of an unlearning framework is to ensure that the unlearned model $\pi_{\theta_{\text{ul}}}$ generates responses $y$ which maximize unlearning efficacy and response utility. Hence, an ideal response must answer the user query effectively while obscuring references to the unlearning targets. We formalize this objective as follows 
% \begin{equation*}
% \begin{split}
% \pi^* = \underset{\pi_{\theta_{\text{ul}}}}{\operatorname{argmin}} \Biggl[ & \underbrace{\mathcal{D}_{\text{KL}}(\pi_{\theta_{\text{ul}}}(\cdot|x) || \pi_{\theta}(\cdot|x))}_{\text{Utility preservation}} \\
% & + \lambda \underbrace{\mathbb{E}_{y \sim \pi_{\theta_{\text{ul}}}(\cdot|x)} \left[\mathbbm{1}_{\{\exists s \in S : s \in y\}} \right]}_{\text{Unlearning penalty}} \Biggr]
% \end{split}
% \end{equation*}
% Here, $\mathcal{D}_{\text{KL}}(\cdots)$ measures the Kullback-Leibler divergence between the unlearned model  $\pi_{\theta_{\text{ul}}}$ and the original (non-unlearned) model $\pi_{\theta}$. Minimizing the KL-divergence between the two distributions allows the response from $\pi_{\theta_{\text{ul}}}$ to retain the utility of the response from $\pi_{\theta}$. $\lambda \ge 0$ is a hyperparameter that balances the utility and the unlearning strictness, increasing which will encourage the model to emphasize rigorous unlearning at the cost of response utility.

% $\mathcal{X}$ can contain prompts which are engineered to extract sensitive information from $\pi_{\theta_{\text{ul}}}$ \cite{zou2023universaltransferableadversarialattacks}, and we do not make assumptions about the intention of the user as done in \citet{thaker2024guardrail}. \citet{liu2024revisitingwhosharrypotter} points out that current post hoc unlearning methods are brittle to state-of-the-art adversarial attacks \cite{lynch2024eight, anil2024many}, preventing them from being deployed in practical settings. Section \ref{sec:5.4} highlights the robustness of \texttt{ALU} under adversarial attacks. 

Effective human-robot cooperation in CoNav-Maze hinges on efficient communication. Maximizing the human’s information gain enables more precise guidance, which in turn accelerates task completion. Yet for the robot, the challenge is not only \emph{what} to communicate but also \emph{when}, as it must balance gathering information for the human with pursuing immediate goals when confident in its navigation.

To achieve this, we introduce \emph{Information Gain Monte Carlo Tree Search} (IG-MCTS), which optimizes both task-relevant objectives and the transmission of the most informative communication. IG-MCTS comprises three key components:
\textbf{(1)} A data-driven human perception model that tracks how implicit (movement) and explicit (image) information updates the human’s understanding of the maze layout.
\textbf{(2)} Reward augmentation to integrate multiple objectives effectively leveraging on the learned perception model.
\textbf{(3)} An uncertainty-aware MCTS that accounts for unobserved maze regions and human perception stochasticity.
% \begin{enumerate}[leftmargin=*]
%     \item A data-driven human perception model that tracks how implicit (movement) and explicit (image transmission) information updates the human’s understanding of the maze layout.
%     \item Reward augmentation to integrate multiple objectives effectively leveraging on the learned perception model.
%     \item An uncertainty-aware MCTS that accounts for unobserved maze regions and human perception stochasticity.
% \end{enumerate}

\subsection{Human Perception Dynamics}
% IG-MCTS seeks to optimize the expected novel information gained by the human through the robot’s actions, including both movement and communication. Achieving this requires a model of how the human acquires task-relevant information from the robot.

% \subsubsection{Perception MDP}
\label{sec:perception_mdp}
As the robot navigates the maze and transmits images, humans update their understanding of the environment. Based on the robot's path, they may infer that previously assumed blocked locations are traversable or detect discrepancies between the transmitted image and their map.  

To formally capture this process, we model the evolution of human perception as another Markov Decision Process, referred to as the \emph{Perception MDP}. The state space $\mathcal{X}$ represents all possible maze maps. The action space $\mathcal{S}^+ \times \mathcal{O}$ consists of the robot's trajectory between two image transmissions $\tau \in \mathcal{S}^+$ and an image $o \in \mathcal{O}$. The unknown transition function $F: (x, (\tau, o)) \rightarrow x'$ defines the human perception dynamics, which we aim to learn.

\subsubsection{Crowd-Sourced Transition Dataset}
To collect data, we designed a mapping task in the CoNav-Maze environment. Participants were tasked to edit their maps to match the true environment. A button triggers the robot's autonomous movements, after which it captures an image from a random angle.
In this mapping task, the robot, aware of both the true environment and the human’s map, visits predefined target locations and prioritizes areas with mislabeled grid cells on the human’s map.
% We assume that the robot has full knowledge of both the actual environment and the human’s current map. Leveraging this knowledge, the robot autonomously navigates to all predefined target locations. It then randomly selects subsequent goals to reach, prioritizing grid locations that remain mislabeled on the human’s map. This ensures that the robot’s actions are strategically focused on providing useful information to improve map accuracy.

We then recruited over $50$ annotators through Prolific~\cite{palan2018prolific} for the mapping task. Each annotator labeled three randomly generated mazes. They were allowed to proceed to the next maze once the robot had reached all four goal locations. However, they could spend additional time refining their map before moving on. To incentivize accuracy, annotators receive a performance-based bonus based on the final accuracy of their annotated map.


\subsubsection{Fully-Convolutional Dynamics Model}
\label{sec:nhpm}

We propose a Neural Human Perception Model (NHPM), a fully convolutional neural network (FCNN), to predict the human perception transition probabilities modeled in \Cref{sec:perception_mdp}. We denote the model as $F_\theta$ where $\theta$ represents the trainable weights. Such design echoes recent studies of model-based reinforcement learning~\cite{hansen2022temporal}, where the agent first learns the environment dynamics, potentially from image observations~\cite{hafner2019learning,watter2015embed}.

\begin{figure}[t]
    \centering
    \includegraphics[width=0.9\linewidth]{figures/ICML_25_CNN.pdf}
    \caption{Neural Human Perception Model (NHPM). \textbf{Left:} The human's current perception, the robot's trajectory since the last transmission, and the captured environment grids are individually processed into 2D masks. \textbf{Right:} A fully convolutional neural network predicts two masks: one for the probability of the human adding a wall to their map and another for removing a wall.}
    \label{fig:nhpm}
    \vskip -0.1in
\end{figure}

As illustrated in \Cref{fig:nhpm}, our model takes as input the human’s current perception, the robot’s path, and the image captured by the robot, all of which are transformed into a unified 2D representation. These inputs are concatenated along the channel dimension and fed into the CNN, which outputs a two-channel image: one predicting the probability of human adding a new wall and the other predicting the probability of removing a wall.

% Our approach builds on world model learning, where neural networks predict state transitions or environmental updates based on agent actions and observations. By leveraging the local feature extraction capabilities of CNNs, our model effectively captures spatial relationships and interprets local changes within the grid maze environment. Similar to prior work in localization and mapping, the CNN architecture is well-suited for processing spatially structured data and aligning the robot’s observations with human map updates.

To enhance robustness and generalization, we apply data augmentation techniques, including random rotation and flipping of the 2D inputs during training. These transformations are particularly beneficial in the grid maze environment, which is invariant to orientation changes.

\subsection{Perception-Aware Reward Augmentation}
The robot optimizes its actions over a planning horizon \( H \) by solving the following optimization problem:
\begin{subequations}
    \begin{align}
        \max_{a_{0:H-1}} \;
        & \mathop{\mathbb{E}}_{T, F} \left[ \sum_{t=0}^{H-1} \gamma^t \left(\underbrace{R_{\mathrm{task}}(\tau_{t+1}, \zeta)}_{\text{(1) Task reward}} + \underbrace{\|x_{t+1}-x_t\|_1}_{\text{(2) Info reward}}\right)\right] \label{obj}\\ 
        \subjectto \quad
        &x_{t+1} = F(x_t, (\tau_t, a_t)), \quad a_t\in\Ocal \label{const:perception_update}\\ 
        &\tau_{t+1} = \tau_t \oplus T(s_t, a_t), \quad a_t\in \Ucal\label{const:history_update}
    \end{align}
\end{subequations} 

The objective in~\eqref{obj} maximizes the expected cumulative reward over \( T \) and \( F \), reflecting the uncertainty in both physical transitions and human perception dynamics. The reward function consists of two components: 
(1) The \emph{task reward} incentivizes efficient navigation. The specific formulation for the task in this work is outlined in \Cref{appendix:task_reward}.
(2) The \emph{information reward} quantifies the change in the human’s perception due to robot actions, computed as the \( L_1 \)-norm distance between consecutive perception states.  

The constraint in~\eqref{const:history_update} ensures that for movement actions, the trajectory history \( \tau_t \) expands with new states based on the robot’s chosen actions, where \( s_t \) is the most recent state in \( \tau_t \), and \( \oplus \) represents sequence concatenation. 
In constraint~\eqref{const:perception_update}, the robot leverages the learned human perception dynamics \( F \) to estimate the evolution of the human’s understanding of the environment from perception state $x_t$ to $x_{t+1}$ based on the observed trajectory \( \tau_t \) and transmitted image \( a_t\in\Ocal \). 
% justify from a cognitive science perspective
% Cognitive science research has shown that humans read in a way to maximize the information gained from each word, aligning with the efficient coding principle, which prioritizes minimizing perceptual errors and extracting relevant features under limited processing capacity~\cite{kangassalo2020information}. Drawing on this principle, we hypothesize that humans similarly prioritize task-relevant information in multimodal settings. To accommodate this cognitive pattern, our robot policy selects and communicates high information-gain observations to human operators, akin to summarizing key insights from a lengthy article.
% % While the brain naturally seeks to gain information, the brain employs various strategies to manage information overload, including filtering~\cite{quiroga2004reducing}, limiting/working memory, and prioritizing information~\cite{arnold2023dealing}.
% In this context of our setup, we optimize the selection of camera angles to maximize the human operator's information gain about the environment. 

\subsection{Information Gain Monte Carlo Tree Search (IG-MCTS)}
IG-MCTS follows the four stages of Monte Carlo tree search: \emph{selection}, \emph{expansion}, \emph{rollout}, and \emph{backpropagation}, but extends it by incorporating uncertainty in both environment dynamics and human perception. We introduce uncertainty-aware simulations in the \emph{expansion} and \emph{rollout} phases and adjust \emph{backpropagation} with a value update rule that accounts for transition feasibility.

\subsubsection{Uncertainty-Aware Simulation}
As detailed in \Cref{algo:IG_MCTS}, both the \emph{expansion} and \emph{rollout} phases involve forward simulation of robot actions. Each tree node $v$ contains the state $(\tau, x)$, representing the robot's state history and current human perception. We handle the two action types differently as follows:
\begin{itemize}
    \item A movement action $u$ follows the environment dynamics $T$ as defined in \Cref{sec:problem}. Notably, the maze layout is observable up to distance $r$ from the robot's visited grids, while unexplored areas assume a $50\%$ chance of walls. In \emph{expansion}, the resulting search node $v'$ of this uncertain transition is assigned a feasibility value $\delta = 0.5$. In \emph{rollout}, the transition could fail and the robot remains in the same grid.
    
    \item The state transition for a communication step $o$ is governed by the learned stochastic human perception model $F_\theta$ as defined in \Cref{sec:nhpm}. Since transition probabilities are known, we compute the expected information reward $\bar{R_\mathrm{info}}$ directly:
    \begin{align*}
        \bar{R_\mathrm{info}}(\tau_t, x_t, o_t) &= \mathbb{E}_{x_{t+1}}\|x_{t+1}-x_t\|_1 \\
        &= \|p_\mathrm{add}\|_1 + \|p_\mathrm{remove}\|_1,
    \end{align*}
    where $(p_\mathrm{add}, p_\mathrm{remove}) \gets F_\theta(\tau_t, x_t, o_t)$ are the estimated probabilities of adding or removing walls from the map. 
    Directly computing the expected return at a node avoids the high number of visitations required to obtain an accurate value estimate.
\end{itemize}

% We denote a node in the search tree as $v$, where $s(v)$, $r(v)$, and $\delta(v)$ represent the state, reward, and transition feasibility at $v$, respectively. The visit count of $v$ is denoted as $N(v)$, while $Q(v)$ represents its total accumulated return. The set of child nodes of $v$ is denoted by $\mathbb{C}(v)$.

% The goal of each search is to plan a sequence for the robot until it reaches a goal or transmits a new image to the human. We initialize the search tree with the current human guidance $\zeta$, and the robot's approximation of human perception $x_0$. Each search node consists consists of the state information required by our reward augmentation: $(\tau, x)$. A node is terminal if it is the resulting state of a communication step, or if the robot reaches a goal location. 

% A rollout from the expanded node simulates future transitions until reaching a terminal state or a predefined depth $H$. Actions are selected randomly from the available action set $\mathcal{A}(s)$. If an action's feasibility is uncertain due to the environment's unknown structure, the transition occurs with probability $\delta(s, a)$. When a random number draw deems the transition infeasible, the state remains unchanged. On the other hand, for communication steps, we don't resolve the uncertainty but instead compute the expected information gain reward: \philip{TODO: adjust notation}
% \begin{equation}
%     \mathbb{E}\left[R_\mathrm{info}(\tau, x')\right] = \sum \mathrm{NPM(\tau, o)}.
% \end{equation}

\subsubsection{Feasibility-Adjusted Backpropagation}
During backpropagation, the rewards obtained from the simulation phase are propagated back through the tree, updating the total value $Q(v)$ and the visitation count $N(v)$ for all nodes along the path to the root. Due to uncertainty in unexplored environment dynamics, the rollout return depends on the feasibility of the transition from the child node. Given a sample return \(q'_{\mathrm{sample}}\) at child node \(v'\), the parent node's return is:
\begin{equation}
    q_{\mathrm{sample}} = r + \gamma \left[ \delta' q'_{\mathrm{sample}} + (1 - \delta') \frac{Q(v)}{N(v)} \right],
\end{equation}
where $\delta'$ represents the probability of a successful transition. The term \((1 - \delta')\) accounts for failed transitions, relying instead on the current value estimate.

% By incorporating uncertainty-aware rollouts and backpropagation, our approach enables more robust decision-making in scenarios where the environment dynamics is unknown and avoids simulation of the stochastic human perception dynamics.

% \begin{table}[!t]
% \centering
% \scalebox{0.68}{
%     \begin{tabular}{ll cccc}
%       \toprule
%       & \multicolumn{4}{c}{\textbf{Intellipro Dataset}}\\
%       & \multicolumn{2}{c}{Rank Resume} & \multicolumn{2}{c}{Rank Job} \\
%       \cmidrule(lr){2-3} \cmidrule(lr){4-5} 
%       \textbf{Method}
%       &  Recall@100 & nDCG@100 & Recall@10 & nDCG@10 \\
%       \midrule
%       \confitold{}
%       & 71.28 &34.79 &76.50 &52.57 
%       \\
%       \cmidrule{2-5}
%       \confitsimple{}
%     & 82.53 &48.17
%        & 85.58 &64.91
     
%        \\
%        +\RunnerUpMiningShort{}
%     &85.43 &50.99 &91.38 &71.34 
%       \\
%       +\HyReShort
%         &- & -
%        &-&-\\
       
%       \bottomrule

%     \end{tabular}
%   }
% \caption{Ablation studies using Jina-v2-base as the encoder. ``\confitsimple{}'' refers using a simplified encoder architecture. \framework{} trains \confitsimple{} with \RunnerUpMiningShort{} and \HyReShort{}.}
% \label{tbl:ablation}
% \end{table}
\begin{table*}[!t]
\centering
\scalebox{0.75}{
    \begin{tabular}{l cccc cccc}
      \toprule
      & \multicolumn{4}{c}{\textbf{Recruiting Dataset}}
      & \multicolumn{4}{c}{\textbf{AliYun Dataset}}\\
      & \multicolumn{2}{c}{Rank Resume} & \multicolumn{2}{c}{Rank Job} 
      & \multicolumn{2}{c}{Rank Resume} & \multicolumn{2}{c}{Rank Job}\\
      \cmidrule(lr){2-3} \cmidrule(lr){4-5} 
      \cmidrule(lr){6-7} \cmidrule(lr){8-9} 
      \textbf{Method}
      & Recall@100 & nDCG@100 & Recall@10 & nDCG@10
      & Recall@100 & nDCG@100 & Recall@10 & nDCG@10\\
      \midrule
      \confitold{}
      & 71.28 & 34.79 & 76.50 & 52.57 
      & 87.81 & 65.06 & 72.39 & 56.12
      \\
      \cmidrule{2-9}
      \confitsimple{}
      & 82.53 & 48.17 & 85.58 & 64.91
      & 94.90&78.40 & 78.70& 65.45
       \\
      +\HyReShort{}
       &85.28 & 49.50
       &90.25 & 70.22
       & 96.62&81.99 & \textbf{81.16}& 67.63
       \\
      +\RunnerUpMiningShort{}
       % & 85.14& 49.82
       % &90.75&72.51
       & \textbf{86.13}&\textbf{51.90} & \textbf{94.25}&\textbf{73.32}
       & \textbf{97.07}&\textbf{83.11} & 80.49& \textbf{68.02}
       \\
   %     +\RunnerUpMiningShort{}
   %    & 85.43 & 50.99 & 91.38 & 71.34 
   %    & 96.24 & 82.95 & 80.12 & 66.96
   %    \\
   %    +\HyReShort{} old
   %     &85.28 & 49.50
   %     &90.25 & 70.22
   %     & 96.62&81.99 & 81.16& 67.63
   %     \\
   % +\HyReShort{} 
   %     % & 85.14& 49.82
   %     % &90.75&72.51
   %     & 86.83&51.77 &92.00 &72.04
   %     & 97.07&83.11 & 80.49& 68.02
   %     \\
      \bottomrule

    \end{tabular}
  }
\caption{\framework{} ablation studies. ``\confitsimple{}'' refers using a simplified encoder architecture. \framework{} trains \confitsimple{} with \RunnerUpMiningShort{} and \HyReShort{}. We use Jina-v2-base as the encoder due to its better performance.
}
\label{tbl:ablation}
\end{table*}

\section{Results}
\label{sec:results}

In this section, we present detailed results demonstrating \emph{CellFlow}'s state-of-the-art performance in cellular morphology prediction under perturbations, outperforming existing methods across multiple datasets and evaluation metrics.

\subsection{Datasets}

Our experiments were conducted using three cell imaging perturbation datasets: BBBC021 (chemical perturbation)~\cite{caie2010high}, RxRx1 (genetic perturbation)~\cite{sypetkowski2023rxrx1}, and the JUMP dataset (combined perturbation)~\cite{chandrasekaran2023jump}. We followed the preprocessing protocol from IMPA~\cite{palma2023predicting}, which involves correcting illumination, cropping images centered on nuclei to a resolution of 96×96, and filtering out low-quality images. The resulting datasets include 98K, 171K, and 424K images with 3, 5, and 6 channels, respectively, from 26, 1,042, and 747 perturbation types. Examples of these images are provided in Figure~\ref{fig:comparison}. Details of datasets are provided in \S\ref{sec:data}.

\subsection{Experimental Setup}

\textbf{Evaluation metrics.} We evaluate methods using two types of metrics: (1) FID and KID, which measure image distribution similarity via Fréchet and kernel-based distances, computed on 5K generated images for BBBC021 and 100 randomly selected perturbation classes for RxRx1 and JUMP; we report both overall scores across all samples and conditional scores per perturbation class. (2) Mode of Action (MoA) classification accuracy, which assesses biological fidelity by using a trained classifier to predict a drug’s effect from perturbed images and comparing it to its known MoA from the literature.

\textbf{Baselines.} We compare our approach against two baselines, PhenDiff~\cite{bourou2024phendiff} and IMPA~\cite{palma2023predicting}, the only two baselines that incorporate control images into their model design --- a crucial setup for distinguishing true perturbation effects from artifacts such as batch effects. PhenDiff uses diffusion models to first map control images to noise and then transform the noise into target images. In contrast, IMPA employs GANs with an AdaIN layer to transfer the style of control images to target images, specifically designed for paired image-to-image mappings. Our method uses flow matching, which is tailored for distribution-to-distribution mapping, providing a more suitable solution for our problem. We reproduce these baselines with official codes.

\textbf{Training details.} \emph{CellFlow} employs a UNet-based velocity field with a four-stage design. Perturbations are encoded following IMPA~\cite{palma2023predicting}. Training is conducted for 100 epochs on 4 A100 GPUs. Details are in \S\ref{sec:experimental}.


\begin{table*}[ht!]
\centering
\footnotesize
\renewcommand{\arraystretch}{1.2}
\begin{tabular}{p{2.15cm} p{14.05cm}}
\toprule
 & \includegraphics[height=2.6cm]{figure/qualitative_example.pdf}\\
\midrule
Original \phantom{00000} Description & Bromazepam binds to the GABA receptor GABA-A, causing a conformational change and \textbf{increasing inhibitory effects of GABA}. Other neurotransmitters are not influenced. One of the \textbf{benzodiazepines} that is \textbf{used in the treatment of anxiety disorders.} \\
\midrule
User & Explain the features of the given molecule that are most distinct from those of other molecules. \\
\midrule
Mol-LLaMA (Ours) & The given molecule is a member of the \textcolor{tabblue}{benzodiazepine class}, which is characterized by a fusion of a benzene ring and a seven-membered diazepine ring. This structure is crucial for its biological activity, particularly its role as \textcolor{tabblue}{a sedative and hypnotic agent}. ...$<$omitted$>$...\\

& The given molecule is also characterized by a ketone group within the diazepine ring, which is a common feature in benzodiazepines. \textcolor{tabblue}{This functional group is important for the interaction with the GABA-A receptor, enhancing the inhibitory effects of GABA}. ...$<$omitted$>$...\\

\midrule
GPT-4o & 1. **Quinazoline Core**: The molecule contains a \textcolor{tabred}{quinazoline core}, ...$<$omitted$>$... \\
 & Overall, the combination of a \textcolor{tabred}{quinazoline} structure with unique substitution patterns, including bromine halogenation and the presence of a pyridinyl group, ...$<$omitted$>$... . \\
\midrule
LLaMo & The molecule has a \textcolor{tabblue}{benzodiazepine} structure with a bromo substituent at the 7-position and \textcolor{tabred}{a methyl group at the 1-position}. It is distinct from other molecules in that it contains a bromine atom, a nitrogen atom, and a methyl group, which are not present in the other molecules. \\
\midrule
3D-MoLM & ...$<$omitted$>$... It belongs to the class of compounds called \textcolor{tabred}{quinazolines}, which are characterized by a bicyclic structure consisting of a benzene ring fused to a pyrimidine ring. ...$<$omitted$>$...\\
& The molecule's structure suggests potential applications in medicinal chemistry, as quinazolines have been found to possess various biological activities, including \textcolor{tabred}{antitumor, antimicrobial, and anti-inflammatory properties.} \\
\midrule
Mol-Instructions & The molecule is a \textcolor{tabred}{quinoxaline derivative}. \\
\bottomrule
\end{tabular}
\vspace{-0.1in}
\caption{\small Case study to compare molecular understanding and reasoning ability. Mol-LLaMA accurately understands the molecular features, answering a correct molecular taxonomy and providing its distinct properties that are relevant to the given molecule.}
\label{tab:qualitative}
\vspace{-0.1in}
\end{table*}

\subsection{Main Results}

\textbf{\emph{CellFlow} generates highly realistic cell images.}  
\emph{CellFlow} outperforms existing methods in capturing cellular morphology across all datasets (Table~\ref{tab:results}a), achieving overall FID scores of 18.7, 33.0, and 9.0 on BBBC021, RxRx1, and JUMP, respectively --- improving FID by 21\%–45\% compared to previous methods. These gains in both FID and KID metrics confirm that \emph{CellFlow} produces significantly more realistic cell images than prior approaches.

\textbf{\emph{CellFlow} accurately captures perturbation-specific morphological changes.}  
As shown in Table~\ref{tab:results}a, \emph{CellFlow} achieves conditional FID scores of 56.8 (a 26\% improvement), 163.5, and 84.4 (a 16\% improvement) on BBBC021, RxRx1, and JUMP, respectively. These scores are computed by measuring the distribution distance for each specific perturbation and averaging across all perturbations.   
Table~\ref{tab:results}b further highlights \emph{CellFlow}’s performance on six representative chemical and three genetic perturbations. For chemical perturbations, \emph{CellFlow} reduces FID scores by 14–55\% compared to prior methods.
The smaller improvement (5–12\% improvements) on RxRx1 is likely due to the limited number of images per perturbation type.

\textbf{\emph{CellFlow} preserves biological fidelity across perturbation conditions.} 
Table~\ref{tab:ablation}a presents mode of action (MoA) classification accuracy on the BBBC021 dataset using generated cell images. MoA describes how a drug affects cellular function and can be inferred from morphology. To assess this, we train an image classifier on real perturbed images and test it on generated ones. \emph{CellFlow} achieves 71.1\% MoA accuracy, closely matching real images (72.4\%) and significantly surpassing other methods (best: 63.7\%), demonstrating its ability to maintain biological fidelity across perturbations. Qualitative comparisons in Figure~\ref{fig:comparison} further highlight \emph{CellFlow}’s accuracy in capturing key biological effects. For example, demecolcine produces smaller, fragmented nuclei, which other methods fail to reproduce accurately.

\textbf{\emph{CellFlow} generalizes to out-of-distribution (OOD) perturbations.}  
On BBBC021, \emph{CellFlow} demonstrates strong generalization to novel chemical perturbations never seen during training (Table~\ref{tab:ablation}b). It achieves 6\% and 28\% improvements in overall and conditional FID over the best baseline. This OOD generalization is critical for biological research, enabling the exploration of previously untested interventions and the design of new drugs.

\textbf{Ablations highlight the importance of each component in \emph{CellFlow}.}  
Table~\ref{tab:ablation}c shows that removing conditional information, classifier-free guidance, or noise augmentation significantly degrades performance, leading to higher FID scores. These underscore the critical role of each component in enabling \emph{CellFlow}’s state-of-the-art performance.  

\begin{figure*}[!tb]
    \centering
     \includegraphics[width=\linewidth]{imgs/interpolation.pdf}
     \vspace{-2em}
    \caption{
    \textbf{\emph{CellFlow} enables new capabilities.} 
\textit{(a.1) Batch effect calibration.}  
\emph{CellFlow} initializes with control images, enabling batch-specific predictions. Comparing predictions from different batches highlights actual perturbation effects (smaller cell size) while filtering out spurious batch effects (cell density variations).  
\textit{(a.2) Interpolation trajectory.}  
\emph{CellFlow}'s learned velocity field supports interpolation between cell states, which might provide insights into the dynamic cell trajectory. 
\textit{(b) Diffusion model comparison.}  
Unlike flow matching, diffusion models that start from noise cannot calibrate batch effects or support interpolation.  
\textit{(c) Reverse trajectory.}  
\emph{CellFlow}'s reversible velocity field can predict prior cell states from perturbed images, offering potential applications such as restoring damaged cells.
    }
    \label{fig:interpolation}
    \vspace{-1em}
\end{figure*}

\subsection{New Capabilities}

\textbf{\emph{CellFlow} addresses batch effects and reveals true perturbation effects.}  
\emph{CellFlow}’s distribution-to-distribution approach effectively addresses batch effects, a significant challenge in biological experimental data collection. As shown in Figure~\ref{fig:interpolation}a, when conditioned on two distinct control images with varying cell densities from different batches, \emph{CellFlow} consistently generates the expected perturbation effect (cell shrinkage due to mevinolin) while recapitulating batch-specific artifacts, revealing the true perturbation effect. Table~\ref{tab:ablation}d further quantifies the importance of conditioning on the same batch. By comparing generated images conditioned on control images from the same or different batches against the target perturbation images, we find that same-batch conditioning reduces overall and conditional FID by 21\%. This highlights the importance of modeling control images to more accurately capture true perturbation effects—an aspect often overlooked by prior approaches, such as diffusion models that initialize from noise (Figure~\ref{fig:interpolation}b).

\textbf{\emph{CellFlow} has the potential to model cellular morphological change trajectories.}
Cell trajectories could offer valuable information about perturbation mechanisms, but capturing them with current imaging technologies remains challenging due to their destructive nature. Since \emph{CellFlow} continuously transforms the source distribution into the target distribution, it can generate smooth interpolation paths between initial and final predicted cell states, producing video-like sequences of cellular transformation based on given source images (Figure~\ref{fig:interpolation}a). This suggests a possible approach for simulating morphological trajectories during perturbation response, which diffusion methods cannot achieve (Figure~\ref{fig:interpolation}b). Additionally, the reversible distribution transformation learned through flow matching enables \emph{CellFlow} to model backward cell state reversion (Figure~\ref{fig:interpolation}c), which could be useful for studying recovery dynamics and predicting potential treatment outcomes.

\section{Conclusions}\label{sec-conclusions}
We present Interaction-aware Conformal Prediction (ICP) to explicitly address the mutual influence between robot and humans in crowd navigation problems. We achieve interaction awareness by proposing an iterative process of robot motion planning based on human motion uncertainty and conformal prediction of the human motion dependent on the robot motion plan. Our crowd navigation simulation experiments show ICP strikes a good balance of performance among navigation efficiency, social awareness, and uncertainty quantification compared to previous works. ICP generalizes well to navigation tasks across different crowd densities, and its fast runtime and manageable memory usage indicates potential for real-world applications.

In future work, we will address infeasible robot planning solutions with adaptive failure probability and conduct real-world crowd navigation experiments to evaluate the effectiveness of ICP. As ICP is a task-agnostic algorithm, we would like to explore its applications in manipulation settings, such as collaborative manufacturing.

\begin{credits}
\subsubsection{\ackname} This work was supported by the National Science Foundation under Grant No. 2143435 and by the National Science Foundation under Grant CCF 2236484.
\end{credits}
%%%%%%%%% REFERENCES
{
\bibliography{main_bib}
\bibliographystyle{icml2025}
}

% --- supplementary material
\begin{figure*}[h]
    \centering
    \includegraphics[width=0.95\textwidth]{figures/ood_objs_part1.pdf}
    \caption{The out-of-distribution objects used in our experiments (part 1).}\label{fig:ood_objs_part1}
\end{figure*}

\section{Appendix}
\label{sec:appendix}

\subsection{Evaluation Metrics}
For real robot task, we record the percentage of trials where the robot successfully completes the assigned task. This is a fundamental metric for any robot experiment. Multiple trials are conducted for the evaluation.


\subsection{Implementation Details}
All experiments were conducted on eight NVIDIA A800 GPUs using the Adam optimizer with a constant learning rate of 2e-5 and a global batch size of 128.  Training proceeded for 50,000 steps, with the final checkpoint selected based on validation performance. Unless otherwise stated, the ratio of robot data to visual-text data was 10:1.  Empirically, we observed that increasing the proportion of robot data significantly degraded manipulation performance.  Our base model is DiVLA\cite{wen2024diffusionvla} with a Qwen2-VL-2B backbone~\cite{wang2024qwen2}. As our focus is developing a co-training method for novel object generalization, we retained the original model architecture.


\subsection{Example of Objects Used in Experiments}
We provide a comprehensive list of \text{out-of-distrbution} objects and names that we used for training and evaluation, which are shown in Figure~\ref{fig:ood_objs_part1} and Figure~\ref{fig:ood_objs_part2}.



\begin{figure*}[t]
    \centering
    \includegraphics[width=0.95\textwidth]{figures/ood_objs_part2.pdf}
    \caption{The out-of-distribution objects used in our experiments (part 2).}\label{fig:ood_objs_part2}
\end{figure*}
%%%%%%%%% REFERENCES
%{
%    \small
%    \bibliographystyle{aaai23}
%    \bibliography{macros,main}
%}
\end{document}
