\section{Related Work}
% Works examining uncertainty quantification in medicine can be divided into two groups. The first group of articles examines the we call \textit{post-facto} uncertainty. In these works, the authors calculate the degree of model uncertainty after each prediction, and the resulting value is analyzed. In the medical field, these approaches are also used to assess uncertainty. Thus, in the work~\cite{10.1145/3368555.3384457}, the authors, using ensembles of models~\cite{lakshminarayanan2017simple} and various Bayesian neural networks~\cite{DBLP:journals/corr/FortunatoBV17}, studied the uncertainty of models in the tasks of in-patient mortality prediction and diagnosis prediction at discharge. The findings demonstrated that significant diversity in predictions and decisions tailored to individual patients may signal the potential fragility of a model's decision. This type of decision presents a chance to recognize supplementary data, aiming to mitigate the extent of uncertainty associated with the model.

  Uncertainty quantification methods have evolved to include various techniques to improve model reliability. For example, softmax response~\cite{selective-classification}, one of the simplest UQ methods, estimates the confidence of a model by the softmax output, assuming high probability indicates high certainty. Additionally, a group of Monte Carlo Dropout approaches~\cite{pmlr-v48-gal16, kampffmeyer2016semantic, gal2017deep} addresses this problem by applying dropout during inference, generating multiple predictions from the same model and estimating uncertainty from their variability. Another group of methods, density-based methods~\cite{lee2018simple, yoo-etal-2022-detection, Mukhoti2022ddu, Kotelevskii2022nuq}, focuses on analyzing the distribution of points in a model feature space using the Gaussian distribution. Moreover, a recent state-of-the-art method named Hybrid Uncertainty Quantification (HUQ; \cite{vazhentsev-etal-2023-hybrid}) integrates different strategies, such as combining density-based methods with the neural softmax response, to capture both model and data uncertainties comprehensively.

  While the described methods are well-established in uncertainty quantification, their application in medicine, particularly in electronic health records (EHRs), remains limited. However, there is a growing trend toward using UQ to enhance decision-making processes. For example, the work~\cite{10.1145/3368555.3384457} explored the uncertainty of models in the tasks of in-patient mortality prediction and diagnosis prediction at discharge. This paper considered using ensembles of models~\cite{lakshminarayanan2017simple} and various Bayesian neural networks~\cite{DBLP:journals/corr/FortunatoBV17}.   After each prediction, they calculated the degree of model uncertainty and analyzed the resulting value. The findings demonstrated that significant diversity in predictions and decisions tailored to individual patients may signal the potential fragility of a model's decision. This type of decision presents a chance to use supplementary data to mitigate the extent of uncertainty associated with the model.
  
  However, it is important not only to estimate the model's uncertainty for subsequent analysis and interpretation but also to influence the model's decision in real time. For example, in dynamic environments like emergency rooms or intensive care units, real-time inference ensures that decisions are based on current data, improving patient safety and treatment effectiveness. To solve this problem, a special approach mentioned earlier, namely \textit{selective prediction}, is used. This method relies on the model deciding to classify a sample based on confidence in its prediction. Namely, in~\cite{heo2018uncertaintyaware}, the authors used stochastic attention to obtain the ``I don't know'' prediction to solve several problems, such as mortality prediction, cardiac condition, and surgery recovery on physiological signals and vital patient information. Making this decision reduced the number of false positives/negatives and demonstrated the feasibility of using UQ for selective prediction in the medical field.
  
  In~\cite{qiu2019modeling}, when solving the problem of in-hospital mortality using the Bayesian deep learning approach on patient monitor records, the authors demonstrated that when dealing with patients who present lower uncertainty in their health conditions, the final performance metric significantly improves compared to the outcomes observed in patients with higher uncertainty. This result suggests that the model is more effective and accurate when the uncertainty value is small. These findings highlight the impact of patient uncertainty on model performance and underscore the critical role of selective prediction in the medical domain, where careful decision-making can lead to more reliable results and better overall patient outcomes.

  The authors of~\cite{li2020deep} proposed an algorithm leveraging deep kernel learning~\cite{pmlr-v51-wilson16} and Bayesian neural networks to tackle the task of detecting heart failure, diabetes, and depression using medical codes and medication information sequences. Their architecture demonstrated improved final metrics by discarding samples with high uncertainty. Moreover, it revealed the greater significance of uncertainty in smaller classes when addressing data imbalances.

  In the case of selective prediction, the prediction outcome can not only be filtered based on uncertainty assessment but also subjected to an additional classification process. For example, in~\cite{ASHFAQ2023104019}, the authors estimated model uncertainty in diagnostic predictions by demographics and clinical details using the evidential deep learning approach~\cite{sensoy2018evidential}. When the confidence level of prediction is low, the model triggers a secondary classification step by searching for the nearest neighbors within the ICD code space. This approach ensures that cases with higher uncertainty should be handled more cautiously, enhancing the overall reliability of the prediction system.

  The described works demonstrate the importance and relevance of studying the uncertainty quantification when working with selective prediction on such types of EHR data as patient monitor records~\cite{heo2018uncertaintyaware, qiu2019modeling}, visit history~\cite{ASHFAQ2023104019} and ICD code sequences~\cite{li2020deep}. However, selective prediction studies specifically on medical texts are also needed since this approach has great potential to work and obtain high-quality results in the medical field. Textual data contains context, details, and nuances that are difficult to express through codes and numeric parameters only. This information provides a more complete picture of the patient's condition, including symptoms and anamnesis description that may not be reflected in described EHR data. Integrating text information improves the accuracy of diagnostic models and prognoses, facilitating a more personalized approach to medical decisions.

  Thus, in~\cite{PELUSO2024104576}, the authors applied various selective prediction techniques to enhance the efficiency of cancer registries by automating the extraction of disease-related information from electronic pathology reports at the stages of diagnosis and surgery. They explored multiple approaches to evaluate uncertainty, including fixed confidence score, delta difference score, entropy ratio confidence score, and Bayes beta confidence score. The findings showed that these selective prediction methods successfully achieved the desired level of accuracy in a trade-off analysis, which aimed to reduce the rejection rate.

  % The studies described above demonstrate the significance and relevance of uncertainty quantification in selective prediction tasks within EHR data, especially in the context of medical texts. These works highlight the crucial role of uncertainty quantification in improving classification performance when dealing with medical data. However, there is a growing need to apply state-of-the-art UQ approaches that extend to broader important medical tasks, such as ICD code assignment or mortality prediction based on text containing medical information. These applications pose significant challenges due to the inherent uncertainty arising from the vast number of codes and the variability of medical language used in clinical documentation.

  % Our work addresses this gap by systematically comparing existing state-of-the-art uncertainty quantification methods for medical texts using public benchmarks and proprietary outpatient visits dataset. We aim to assess the effectiveness of these methods in handling the complexities of medical text classification, including tasks like ICD code assignment and mortality predictions.