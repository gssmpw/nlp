\begin{table}[t!]
\centering
\begin{tabular}{lllr}
\toprule
\textbf{H-VLM} & \textbf{IoU} & \textbf{Prompting} & \textbf{Rec.} \\ \cmidrule(lr){1-4}
%\multirow{2}{*}{\makecell{\textbf{Easy \\MSCOCO}}} & \multirow{4}{*}{LLaVA} & \multirow{2}{*}{API~(CLIP)}& 86.23 & 84.21  & 89.19 & 83.27 & 86.63 \\
\multirow{4}{*}{CLIP} &\multirow{2}{*}{$\geqq$ 10} & w/o prpt. & 92.60\\
&& API~(CLIP)& \ensuremath{\blacktriangle}93.16 \\
&\multirow{2}{*}{< 10} & w/o prpt. & 97.46\\
&& API~(CLIP)& \ensuremath{\triangledown}97.37 \\
 \cmidrule(lr){1-4}
\multirow{4}{*}{LLaVA} & \multirow{2}{*}{$\geqq$ 10} & w/o prpt. & 93.64\\
&& API~(LLaVA) & \ensuremath{\blacktriangle}93.65  \\
&\multirow{2}{*}{< 10} & w/o prpt. & 96.82\\
&& API~(LLaVA) & \ensuremath{\triangledown}95.02  \\
 \bottomrule
\end{tabular}
\caption{POPE results on objects present in cases where the output of API - Seg. is correct.}
\label{table6}
\end{table}

\begin{figure}[t]
    \centering
    \includegraphics[width=0.47\textwidth]{images/size4-crop.pdf}
    \caption{Image Size vs. POPE Results.}
    \label{size}
\end{figure}

In conclusion, an analysis of success and failure cases in attention-driven visual prompting for object hallucination yielded the following insights:

\begin{enumerate}
\item Background information is essential for VLMs to accurately determine object presence.
\item Cutoff API Prompting improves performance by both highlighting relevant objects and preserving background context.
\item API Prompting is more effective when visual attention aligns well with the target objects.
\item Smaller objects rely more heavily on background information.
\end{enumerate}

Visual prompting that partially masks an image, even if limited to the background, proves ineffective in mitigating object hallucination. Conversely, emphasizing the target object without full occlusion contributes to its reduction. Future visual prompting should focus on enhancing visibility rather than obscuring image elements.