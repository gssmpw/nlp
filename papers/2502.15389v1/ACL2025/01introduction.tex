Vision-Language Models (VLMs), represented by CLIP~\cite{clip} and LLaVA~\cite{llava}, have advanced rapidly and garnered attention. A Vision-Language Model is equipped with an image encoder and a text encoder, allowing it to process both image and text information simultaneously. With these encoders, multimodal tasks such as visual question answering~\cite{vqa}, image captioning~\cite{caption}, and image retrieval~\cite{retrieval} have been realized, and applications have expanded into fields like autonomous driving~\cite{autonomous} and medical diagnosis~\cite{medical}. However, as these applications expand, the issue of hallucination has become increasingly apparent.

Hallucination refers to inconsistencies between the input image and the output text~\cite{hallucination}. Among them, Hallucinations of an object's presence, attributes, and relationships are defined as object, attribution, and relation hallucinations, respectively. POPE~\cite{pope} has been proposed and widely used as a method to evaluate the object hallucination. POPE works by asking questions about objects that are either present or absent in an image and verifying whether the responses are correct~\cite{pope}.

Prompting in VLMs has been proposed as methods to reduce hallucination~\cite{prompting}. In particular, visual prompting, which emphasizes target objects by adding marks has been shown to suppress hallucination~\cite{visualprompting}. Among these techniques, a method of visual prompting known as Attention Prompting on Image (API) has been introduced; it is automatic way of visual prompting, which demonstrated improved accuracy in VQA and object hallucination~\cite{api}. 

\begin{figure}[t]
    \centering
    \includegraphics[width=0.45\textwidth]{images/apis-crop.pdf}
    \caption{API Prompting Process: Without Cutoff~\cite{api} vs. With Cutoff (Proposed).}
    \label{api}
\end{figure}

However, in the work reported by~\citet{api}, experiments mostly focused on VQA, leaving a deeper analysis of its effect on object hallucination unexamined. This study analyzes success and failure cases of API prompting to explore effective visual prompting for reducing object hallucination. Specifically, this work examines the relationship between attention-target object alignment and API prompting accuracy and evaluates whether removing the background reduces object hallucination as a validation of cases where attention was focused solely on the target object.