\begin{table}[t!]
\centering
\begin{tabular}{llr}
\toprule
\textbf{API - Seg.} & \textbf{Prompting} & \textbf{Rec.} \\ \cmidrule(lr){1-3}
%\multirow{2}{*}{\makecell{\textbf{Easy \\MSCOCO}}} & \multirow{4}{*}{LLaVA} & \multirow{2}{*}{API~(CLIP)}& 86.23 & 84.21  & 89.19 & 83.27 & 86.63 \\
\multirowcell{5}{Correct\\(72\%)} & w/o prpt. & 94.74\\
& API~(CLIP)& \ensuremath{\blacktriangle}95.02 \\
& API~(CLIP) w Cutoff & \ensuremath{\blacktriangle}97.08 \\
 & API~(LLaVA) & \ensuremath{\triangledown}94.23  \\
 & API~(LLaVA) w Cutoff & \ensuremath{\blacktriangle}96.64
 \\\cmidrule(lr){1-3}
\multirowcell{5}{Incorrect\\(28\%)} & w/o prpt. & 75.05\\
& API~(CLIP) & \ensuremath{\triangledown}73.75  \\
& API~(CLIP) w Cutoff & \ensuremath{\blacktriangle}80.61  \\
 & API~(LLaVA) & \ensuremath{\triangledown}72.59  \\
 & API~(LLaVA) w Cutoff & \ensuremath{\blacktriangle}80.51 \\
 \bottomrule
\end{tabular}
\caption{POPE results categorized by correct and incorrect API - Seg. outputs.}
\label{table5}
\end{table}

The overall better performance of API Prompting with Cutoff supports the claim that masking parts of the image is not an effective visual prompting method for reducing object hallucination. This conclusion is further reinforced by the significantly worse results of API - Seg. without Cutoff compared to the w/o prompting case.

This is likely due to the fact that background information surrounding the target object contributes to its identification. Given that API Prompting has shown particularly strong results in prior studies, especially in VQA, conventional API Prompting, which completely hides parts of the image, may be effective when the area requiring attention is already clear to the VLM. However, in the case of object hallucination, the VLM appears to identify objects in conjunction with surrounding information, making the complete removal of image content undesirable. Nevertheless, considering that API Prompting with Cutoff outperforms the w/o prompting case, emphasizing the target object appears to be effective in reducing hallucination.  

As Table~\ref{table6} demonstrated, in cases where responses were successful without background information, observing API Prompting results at an IoU threshold of 10 reveals that when IoU was greater than 10, meaning the target object and attention were well aligned, API Prompting was particularly effective. This indicates that when an answer can be derived without background information, API Prompting is more effective if the target object is emphasized.  

As shown in Figure~\ref{size}, API - Seg. without Cutoff tends to produce incorrect outputs when the target object is small. This suggests that background information plays a more crucial role for smaller objects.