
\az{Query syntax does not use binder notation.}
% Rules for MiniMod used in algorithm implementation examples
\begin{align*}
  \cUser{typeOfExpr}(\svar{s}, \svar{n}, \tyINT) \rTurnstile {} &
  \cEmp{}
  \tag{T-Num}
  \label{eq:t-num}
  \\
  %
  %
  %
  % [T-Add] typeOfExpr(s, Plus(e1, e2), INT()) :-
  %   typeOfExpr(s, e1, INT()),
  %   typeOfExpr(s, e2, INT()).
  \cUser{typeOfExpr}(\svar{s}, \svar{e_1} + \svar{e_2}, \tyINT) \rTurnstile {} &
    \cUser{typeOfExpr}(\svar{s},\svar{e_1}, \tyINT)
    \cConj\ 
    \cUser{typeOfExpr}(\svar{s},\svar{e_2}, \tyINT)
  \tag{T-Add}
  \label{eq:t-add}
  \\
  %
  %
  %
  % [T-Var] typeOfExpr(s, Var(x), T) :- {p}
  %   query var
  %     filter P*I? and { x' :- x' == x }
  %     in s |-> [(p, (x, T))].
  \cUser{typeOfExpr}(\svar{s}, \svar{x}, \svar{T}) \rTurnstile {} &
    \cQuery[\lblOrdLexVAR]{\svar{s}}{
      \reclos{\lblLEX}\reopt{\lblIMPORT}\lblVAR
    }{
      \hoPred{isVar}{\svar{x}}
    }{
      \set{\tuple{\_, \svar{x} : \svar{T}}}
    }
    %
  \tag{T-Var}
  \label{eq:t-var}
  \\[0.5em]
  \cUser{typeOfExpr}(\svar{s}, \svar{a}.\svar{x}, \svar{T}) \rTurnstile {} &
    \cExists{\svar{s_m}}\,
    \cUser{scopeOfMod}(\svar{s}, \svar{a}, \svar{s_m}) \cConj\\
  & \cQuery[]{\svar{s_m}}{
      \lblVAR
    }{
      \hoPred{isVar}{\svar{x}}
    }{
      \set{\tuple{\_, \svar{x} : \svar{T}}}
    }
    %
  \tag{T-QRef}
  \label{eq:t-qref}
  \\[0.5em]
  % memberOk(s, FieldDecl(name, expr)) :- {T}
  %   typeOfExpr(s, expr) == T,
  %   declareField(s, name, T).
  \cUser{memberOk}(\svar{s}, \text{\MiniMod{var}}\ \svar{x}\ \text{\MiniMod{=}}\ \svar{e}) \rTurnstile {} & \cExists {\svar{T}\, \svar{s_x}}
    \cUser{typeOfExpr}(\svar{s}, \svar{e}, \svar{T})
    \cConj {}\\ &
    \cNew {\svar{s_x}} \mapsto \svar{x} : \svar{T}
    \cConj {}
    \svar{s} \cEdge[\lblVAR] \svar{s_x}
  \tag{M-Var}
  \label{eq:m-var}
  \\[0.5em]
  % [T-Mod]
  % modOk(s, ModDecl(modname, imports, members)) :- {s_mod}
  %   new s_mod,
  %   s_mod -LEX-> s,
  %   declareMod(s, modname, s_mod),
  %   importsOrIncludesOk(s_mod, imports),
  %   membersOk(s_mod, members).
  \cUser{modOk}(\svar{s},
    \text{\MiniMod{mod}}\ \svar{a}\text{\MiniMod{\ \{}}\ \overline{\svar{imp}}\ \overline{\svar{mem}\vphantom{imp}} \text{\MiniMod{\ \}}})
  \rTurnstile {} & \cExists{\svar{s_m}}
    \cNew \svar{s_m} \mapsto a
    \cConj {}
    \svar{s_m} \cEdge[\lblLEX] s
    \cConj {}
    \svar{s} \cEdge[\lblMOD] \svar{s_m}
    \cConj {}\\ &
    \cUser{importsOk}(\svar{s_m}, \overline{\svar{imp}})
    \cConj
    \cUser{membersOk}(\svar{s_m}, \overline{\svar{mem}})
  \tag{M-Mod}
  \label{eq:m-mod}
  \\[0.5em]
  %
  % [T-QVar] typeOfExpr(s, QVar(m, x), T) :- {s_m}
  %   scopeOfMod(s, m) == s_m,
  %   query var 
  %      filter e and { x' :- x' == x } 
  %      in s_m |-> [(_ (_, T))]. 
  % \cUser{typeOfExpr}(\svar{s}, \emb{\svar{m}.\svar{x}}, \svar{T}) \rTurnstile {} &
  %   \cExists{\svar{s_m}}
  %   \cUser{scopeOfMod}(\svar{s}, \svar{m}, \svar{s_m})
  %   \cConj {} \\
  %   &
  %   \cQuery{\svar{s_m}}{
  %     \lblVAR
  %   }{
  %     \hoPred{isVar}{\svar{x}}
  %   }{
  %     \set{\tuple{\_, \svar{x} : \svar{T}}}
  %   }
  %   %
  % \tag{T-QVar}
  % \label{eq:t-qvar}
  % \\
  % [F-Fun] typeOfExpr(s, Fun(x, Tx, e), FUN(Tx, T)) :- {s_f}
  %   new s_f, s_f -LEX-> s,
  %   !var[x, Tx],
  %   typeOfExpr(s_f, e, T).
  % \hspace{-0.875em} \cUser{typeOfExpr}(s, \emb{\lambda(\svar{x}{:}\, \svar{T_x}). \svar{e}}, \svar{T_x} \mathop{\to} \svar{T}) \rTurnstile {} &
  %   \cExists{\svar{s_f}\, \svar{s}_x}
  %       \cNew \svar{s_f} \mapsto \svar{s_f}
  %       \cConj {}\
  %       \svar{s_f} \cEdge[\lblLEX] s
  %       \cConj {}\\
  %   &\cNew \svar{s}_x \mapsto (\svar{x}{:}\, \svar{T_x})
  %       \cConj {}\
  %       \svar{s_f} \cEdge[\lblVAR] \svar{s}_x
  %       \cConj {}\\
  %   &\cUser{typeOfExpr}(\svar{s_f}, {\svar{e}}, \svar{T})
  %   \tag{T-Lam}
  %   \label{eq:t-fun} % backward compatibility
  % \\
  %
  %
  % [D-ImportOk] importOk(s, StarImport(m)) :- {sm}
  %   scopeOfMod(s, m, sm),
  %   s -I-> sm.
  \cUser{importOk}(\svar{s}, \text{\MiniMod{import}}\ \svar{a}\text{\MiniMod{::*}}) \rTurnstile {} &
    \cExists{\svar{s_m}}
    \cUser{scopeOfMod}(\svar{s}, \svar{a}, \svar{s_m})
    \cConj {}\
    \svar{s} \cEdge[\lblIMP] \svar{s_m}
    %
  \tag{D-ImportOk}
  \label{eq:d-importok}
  \\[0.5em]
  %
  %
  %
  % [S-Mod] scopeOfMod(s, Ref(x), s_m) :-
  %   query mod 
  %     filter P* and { x' :- x' == x } 
  %     in s |-> [(_, (_, s_m))]
  \cUser{scopeOfMod}(\svar{s}, \svar{x}, \svar{s_m}) \rTurnstile {} &
    \cExists{\svar{p}}
    \cQuery[\lblOrdLexMOD]{\svar{s}}{
      \reclos{\lblLEX}\lblMOD
    }{
      \hoPred{isMod}{\svar{x}}
    }{
      \set{\tuple{\svar{p}, \svar{x}}}
    }
    \cConj
    \svar{s_m} \cEq \mathsf{tgt}(p)
    %
  \tag{S-Mod}
  \label{eq:s-mod}
  \\[0.5em]
  %
  %
  %
  % [S-QMod] scopeOfMod(s, QRef(a, x), s_m) :- {s_p}
  %   scopeOfMod(s, a, s_p),
  %   query mod 
  %     filter e and { x' :- x' == x } 
  %     in s_p |-> [(_, (_, s_m))]
  \cUser{scopeOfMod}(\svar{s}, \svar{a}.\svar{x}, \svar{s_m}) \rTurnstile {} &
    \cExists{\svar{p}\, \svar{s_m'}}\,
    \cUser{scopeOfMod}(\svar{s}, \svar{a}, \svar{s_m'})
    \cConj {} \\
    &
    \cQuery{\svar{s_m'}}{
      \lblMOD
    }{
      \hoPred{isMod}{\svar{x}}
    }{
      \set{\tuple{\svar{p}, \svar{x}}}
    }
    \cConj
    \svar{s_m} \cEq \mathsf{tgt}(p)
    %
  \tag{S-QMod}
  \label{eq:s-qmod}
  \\[0.5em]
  %
  %\\\noindent\rule{\textwidth}{0.4pt}\\
  %
  \hoPred{isVar}{\svar{x}} \triangleq {} &
    \lambda \svar{x'}.\: \, \exists \svar{T}.\: (\svar{x}{:}\, \svar{T}) \cEq \svar{x'}
  \\
  %
  \hoPred{isMod}{\svar{x}} \triangleq {} &
    \lambda \svar{x'}.\: \svar{x} \cEq \svar{x'}
\end{align*}
