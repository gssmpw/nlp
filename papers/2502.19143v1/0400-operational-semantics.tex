\section{Operational Semantics}%
\label{sec:operational-semantics}

The previous section illustrated our approach to synthesize concrete references using an extension of the Statix solver.
This section presents an operational semantics that defines that extension.


\subsection{Syntax of Statix}%
\label{subsec:statix-syntax}

The syntax of Statix terms and constraints, defined in~\cref{fig:statix-syntax}, follows \citet{RouvoetAPKV20} and has a separation-logic-inspired flavor,
as the declarative semantics of Statix constraints is defined using separation logic.
We refer to \citeauthor{RouvoetAPKV20} for the details of this declarative semantics, and focus on the operational semantics instead in~\cref{subsec:statix-operational-semantics} and ~\cref{subsec:refsyn-operational-semantics}.

The syntax uses these distinct enumerable sets of symbols:
$\mathit{TermConstructor}$ for term constructor symbols,
$\mathit{Var}$ for term variables,
$\mathit{SetVar}$ for set variables,
$\mathit{PredSymbol}$ for names of predicates (such as $\mathsf{typeOfExpr}$ in~\cref{fig:alg-minimod-rules}),
$\mathit{Scope}$ for scope graph node identifiers,
$\mathit{Label}$ for scope graph edge labels,
$\mathit{RegEx}$ is the set of regular expressions over words comprised of label symbols, and
$\mathit{PartialOrd}$ is the set of partial orders on label symbols.

In the $\mathit{Constraint}$ syntax, $\cEmp{}$ is the trivially satisfiable constraint, akin to a $\cTrue$ constraint in a traditional logic.
Conversely, $\cFalse$ is never satisfiable.
$C_1 \cConj{} C_2$ is a \emph{separating conjunction}, where the declarative and operational semantics of Statix guarantees $C_1$ and $C_2$ construct separate scope graph fragments.
However, the reader can approximately think of $C_1 \cConj{} C_2$ as traditional logic conjunction.
The constraint $\cExists{x} C$ asserts the existence of some term named $x$, which may be referenced and constrained by $C$.
$\mathsf{single}(t, \underline{t})$ asserts that set term $\underline{t}$ is a singleton set whose element is equal to $t$, while $\cForall{x}{\underline{t}}{C}$ asserts that $C$ holds for all its elements $x$.
$\cNew t_1 \mapsto t_2$ asserts that the scope graph contains a scope identified by $t_1$ and with associated data $t_2$, whereas $t_1 \cEdge[l] t_2$ asserts that the scope identified by $t_1$ is connected via an $l$-labeled edge to the scope identified by $t_2$.

\begin{figure}[t]
  \begin{gather*}
  \begin{array}{c}
    \begin{array}{rcl}
      f &\in &\mathit{TermConstructor}
      \\
      x &\in &\mathit{Var}
      \\
      z &\in &\mathit{SetVar}
      \\
      P &\in &\mathit{PredSymbol}
    \end{array}
    \quad
    \begin{array}{rcl}
      s &\in &\mathit{Scope}
      \\
      l &\in &\mathit{Label}
      \\
      r &\in &\mathit{RegEx}
      \\
      o &\in &\mathit{PartialOrd}
    \end{array}
  \end{array}
  \\
  \begin{array}{rcl}
    \mathit{Term}\ni t
    & ::= & x \mid f(t^*) \mid l \mid s
    \\[0.5em]
    \mathit{SetTerm}\ni \underline{t}
    & ::= & z \mid \zeta
    \\[0.5em]
    \mathit{SetLit}\ni \zeta
    & ::= & \emptyset \mid \{t\}
            \mid \zeta \sqcup \zeta
    \\[0.5em]
    \mathit{Constraint}\ni C
    & ::= & \cEmp{} \mid \cFalse{}
            \mid C \cConj C
            \mid \cExists{x} C
            \mid \cSingle{t}{\underline{t}}
            \mid \cForall{x}{\underline{t}}{C}
    \\
    &\mid & \cNew t \mapsto t
            \mid t \cEdge[l] t
            \mid \cQuery[\orderSyntax]{t}{r}{\lambda x.\: E}{z}.\: C
            \mid E
            \mid P(t^*)
    \\[0.5em]
    \mathit{EqConstraint}\ni E
    & ::= & t \cEq t
            \mid \cDataOf{t}{t}
            \mid E \cConj E
            \mid \cExists{x} E
  \end{array}
\end{gather*}
  \caption{
    Syntax of Statix terms and constraints.
  }%
  \label{fig:statix-syntax}
\end{figure}

The syntax of queries is $\cQuery[\orderSyntax]{t}{r}{\lambda x.\: E}{z}.\: C$.
Here, the term $t$ represents a source scope term; $r$ represents a regular expression that determines reachability; $o$ is a partial order that determines visibility; $\lambda x.\: E$ is a \emph{data-well-formedness constraint} which characterizes whether a target scope and its associated data matches the query; $z$ is a set variable that will be bound in $C$ to the result of the query.
The syntax used by~\citeauthor{RouvoetAPKV20} provides a separate constraint for applying the visibility ordering $o$.
We include this ordering as a part of the query, following the syntax used by the implementation of Statix found in the Spoofax Language Workbench~\cite{KatsV10a}.\footnote{\url{https://spoofax.dev/}}

Also in contrast to~\citeauthor{RouvoetAPKV20}, we distinguish equality constraints (ranged over by $E$) from plain constraints ($C$).
This way, data well-formedness predicates of queries ($\lambda x.\: E$) use constraints that can only inspect terms and data, but cannot extend the scope graph.
Another difference from~\citeauthor{RouvoetAPKV20} is that we define the semantics of predicate constraints.
The constraint $P(t^*)$ represents an invocation of a user-specified predicate, such as those from \cref{fig:alg-minimod-rules}.
The rules we discuss next are parameterized by a specification $\mathbb{S}$ comprising rules of the form $P(t^*)$.

In the syntax of constraints and terms in \cref{fig:statix-syntax} and throughout the paper, we use $\overline{\plhdr}$ notation to represent (possibly empty) sequences, and $\plhdr^*$ notation to represent their syntax.
For example, $P(t^*)$ represents the syntax of a predicate symbol followed by a parenthesized sequence of terms.
We use $x;y$ for sequences that can be freely reordered (\eg, $x;y \approx y;x$) and $x \cdot y$ for sequences that cannot (\eg, $x \cdot y \not\approx y \cdot x$).
We overload notation and use $x;y$ and $x \cdot y$ to represent sequences both when $x$ is an element and when $x$ is a sequence, and similarly for $y$.



\subsection{Operational Semantics of Statix with Hole State Tracking}%
\label{subsec:statix-operational-semantics}

\begin{figure}[t]
  \begin{gather*}
  \begin{array}{rcl}
    h & \in & \mathit{Hole}
  \end{array}
  \\
  \begin{array}{rcl}
    \mathit{Configuration}\ni \kappa & ::= & \opsRSState
      { \mathcal{G} }
      { C^\ast }
      { U \in (\mathit{Var} \rightharpoonup \mathit{Hole}) }
      { H \in (\mathit{Hole} \rightharpoonup (s^\ast \times t )) }
    \\[0.5em]
    \mathit{ScopeGraph}\ni \mathcal{G} & ::= & \langle
      S \subseteq \mathit{Scope},\ 
      R \subseteq ( \mathit{Scope} \times \mathit{Label} \times \mathit{Scope} ),\ 
      \rho \in (\mathit{Scope} \rightharpoonup \mathit{Term})
      \rangle
  \end{array}
\end{gather*}

  \caption{
    Syntax of Statix configurations and scope graphs.
  }%
  \label{fig:statix-config-syntax}
\end{figure}

The operational semantics in~\cref{fig:statix-operational-semantics} also follows~ \citet{RouvoetAPKV20}.
The transition relation uses the configurations whose syntax is given in \cref{fig:statix-config-syntax}.
A configuration $\opsRSState{\mathcal{G}}{\overline{C}}{U}{H}$ comprises:

\begin{itemize} 
  \item $\mathcal{G}$: the currently constructed scope graph.
  \item $\overline{C}$: the current set of constraints.
  \item $U \in (\mathit{Var} \rightharpoonup \mathit{Hole})$: associates unification variables with holes.
  This lets us determine to which hole a given constraint might relate.
  \item $H \in (\mathit{Hole} \rightharpoonup (s^* \times t))$: maps each hole in the program to its state, consisting of a list of traversed scopes $s^*$ and the term $t$ constructed so far.
\end{itemize}

\noindent
A main difference from~\citeauthor{RouvoetAPKV20} is that we extended the configuration to track the state of reference synthesis \emph{holes} via the entities $U$ and $H$.
While the rules in~\cref{fig:statix-operational-semantics} never access them, $U$ and $H$ are explicitly propagated by these rules such that substitutions resulting from unification get applied to them.
We return to the role of $U$ and $H$ in \cref{subsec:refsyn-operational-semantics}.

\begin{figure}[p]
  \centering
  \subcaptionbox*{
    % Operational semantics of constraints in Statix.
    % \label{fig:statix-operational-semantics-a}
  }[\textwidth]{
    % Statix operational semantics
    \framebox[\columnwidth]{$\mathbb{S} \vdash \kappa \rightarrow \kappa'$ \hfill Configuration $\kappa$ steps to $\kappa'$ using specification $\mathbb{S}$}
\begin{mathpar}
  \inferrule[Op-Conj\label{eq:op-conj}]{}{
    \mathbb{S} \vdash \opsRSState
      { \mathcal{G} }
      { (C_1 \cConj C_2) ; \overline{C} }
      { U }
      { H }
    \rightarrow
    \opsRSState
      { \mathcal{G} }
      { C_1 ; C_2 ; \overline{C} }
      { U }
      { H }
  }
  %
  \and
  % 
  \inferrule[Op-Emp\label{eq:op-emp}]{}{
    \mathbb{S} \vdash \opsRSState
      { \mathcal{G} }
      { \cEmp{} ; \overline{C} }
      { U }
      { H }
    \rightarrow
    \opsRSState
      { \mathcal{G} }
      { \overline{C} }
      { U }
      { H }
  }
  %
  \and
  %
  \inferrule*[Left=Op-Exists\label{eq:op-exists}]{
    y\: \text{is fresh}
  }{
    \mathbb{S} \vdash \opsRSState
      { \mathcal{G} }
      { \lparen \cExists{x} C \rparen ; \overline{C} }
      { U }
      { H }
    \rightarrow
    \opsRSState
      { \mathcal{G} }
      { C \lbrack y/x \rbrack ; \overline{C} }
      { U }
      { H }
  }
  %
  \\
  %
  \inferrule*[Left=Op-Eq\label{eq:op-eq-true}]{
    \mathsf{mgu} \lparen t_1, t_2 \rparen = \theta
  }{
    \mathbb{S} \vdash \opsRSState
      { \mathcal{G} }
      { \lparen t_1 \cEq t_2 \rparen ; \overline{C} }
      { U }
      { H }
    \rightarrow
    \opsRSState
      { \mathcal{G} }
      { \overline{C} }
      { U }
      { H }
      \theta
  }
  %
  \and
  %
  \inferrule[Op-Singleton\label{eq:op-singleton}]{}{
    \mathbb{S} \vdash \opsRSState
      { \mathcal{G} }
      { \cSingle{t}{\{ t' \}} ; \overline{C} }
      { U }
      { H }
    \rightarrow
    \opsRSState
      { \mathcal{G} }
      { \lparen t \cEq t' \rparen ; \overline{C} }
      { U }
      { H }
  }
  %
  \and
  %
  \inferrule[Op-Forall\label{eq:op-forall}]{}{
    \mathbb{S} \vdash \opsRSState
      { \mathcal{G} }
      { \lparen \cForall{x}{\zeta}{C} \rparen ; \overline{C} }
      { U }
      { H }
    \rightarrow
    \opsRSState
      { \mathcal{G} }
      { \{ C \lbrack t/x \rbrack \mid t \in \zeta \} ; \overline{C} }
      { U }
      { H }
  }
  %
  \and
  %
  \inferrule*[left=Op-New-Scope\label{eq:op-new-scope}]{
    s \notin S
  }{
    \mathbb{S} \vdash \opsRSState
      { \langle S, R, \rho \rangle }
      { \lparen \cNew x \mapsto t \rparen ; \overline{C} }
      { U }
      { H }
    \rightarrow
    \opsRSState
      { \langle s ; S, R, \rho \lbrack s \mapsto t \rbrack \rangle }
      { \overline{C} }
      { U }
      { H }
      \lbrack s/x \rbrack
  }
  %
  \and
  %
  \inferrule[Op-New-Edge\label{eq:op-new-edge}]{}{
    \mathbb{S} \vdash \opsRSState
      { \langle S, R, \rho \rangle }
      { \lparen s_1 \cEdge[\ell] s_2 \rparen ; \overline{C} }
      { U }
      { H }
    \rightarrow
    \opsRSState
      { \langle S, \lparen s_1, \ell, s_2 \rparen ; E, \rho \rangle }
      { \overline{C} }
      { U }
      { H }
  }
  %
  \and
  %
  \inferrule*[Left=Op-Data\label{eq:op-data}]{
    \rho_{\mathcal{G}}(s) = t_2
  }{
    \mathbb{S} \vdash \opsRSState
      { \mathcal{G} }
      { \mathsf{dataOf}(s, t_1) ; \overline{C} }
      { U }
      { H }
    \rightarrow
    \opsRSState
      { \mathcal{G} }
      { (t_1 \cEq t_2) ; \overline{C} }
      { U }
      { H }
  }
  %
  \and
  %
  \inferrule*[Left=Op-Query\label{eq:op-query}]{
    A = \mathsf{Ans}( \mathcal{G}, \qBase{\orderSyntax}{s}{r}{\lambda x.\: E} )
    \\
    \mathsf{guard}(\mathcal{G}, (\cQuery[\orderSyntax]{s}{r}{\lambda x.\: E}{z}.\: C), \overline{C})
  }{
    \mathbb{S} \vdash \opsRSState
      { \mathcal{G} }
      { \lparen \cQuery[\orderSyntax]{s}{r}{\lambda x.\: E}{z}.\: C \rparen ; \overline{C} }
      { U }
      { H }
    \rightarrow
    \opsRSState
      { \mathcal{G} }
      { C \lbrack A/z \rbrack ; \overline{C} }
      { U }
      { H }
  }
  %
  \and
  %}
  \inferrule*[Left=Op-Pred\label{eq:op-pred}]{
    \exists! (P(\overline{t_2}) \rTurnstile C) \in \mathbb{S}.\: \exists\theta.\:
      \mathsf{mgu}(\overline{t_1}, \overline{t_2}) = \theta
  }{
    \mathbb{S} \vdash \opsRSState
      { \mathcal{G} }
      { P(\overline{t_1}) ; \overline{C} }
      { U }
      { H }
    \rightarrow
    \opsRSState
      { \mathcal{G} }
      { C; \overline{C} }
      { U }
      { H }
      \theta
  }
\end{mathpar}

  }
  % \bigskip
  \subcaptionbox*{
    % Operational semantics of equality constraints in Statix.
    % \label{fig:statix-operational-semantics-b}
  }[\textwidth]{
    % Statix operational semantics equality constraints
    \framebox[\columnwidth]{$\mathcal{G} \vdash E \rightsquigarrow \theta$ \hfill Equality constraint $E$ produces a unifier $\theta$}
\begin{mathpar}
  \inferrule[E-Eq\label{eq:e-eq}]{
    \mathsf{mgu} \lparen t_1, t_2 \rparen = \theta
  }{
    \mathcal{G} \vdash t_1 \cEq t_2
    \rightsquigarrow
    \theta
  }
  %
  \and
  %
  \inferrule[E-DataOf\label{eq:e-dataof}]{
    \mathsf{mgu} \lparen \rho_{\mathcal{G}}(s), t \rparen = \theta
  }{
    \mathcal{G} \vdash \cDataOf{s}{t}
    \rightsquigarrow
    \theta
  }
  %
  \and
  %
  \inferrule[E-Conj\label{eq:e-conj}]{
    \mathcal{G} \vdash E_1 \rightsquigarrow \theta_1
    \\\\
    \mathcal{G} \vdash E_2 \rightsquigarrow \theta_2
  }{
    \mathcal{G} \vdash E_1 \cConj E_2
    \rightsquigarrow
    \theta_1\theta_2
  }
  %
  \and
  %
  \inferrule[E-Exists\label{eq:e-exists}]{
    y\text{ is fresh }
    \\\\
    \mathcal{G} \vdash E[y/x] \rightsquigarrow \theta
  }{
    \mathcal{G} \vdash \cExists{x} E
    \rightsquigarrow
    \theta
  }
\end{mathpar}
  }%
  \vspace{-1.5\baselineskip}
  \caption{Operational semantics of constraints in Statix}%
  \label{fig:statix-operational-semantics}
\end{figure}


\paragraph{Predicate Constraints}
Rule \textsc{Op-Pred} defines the semantics of \emph{predicate expansion}.
To support this rule, the transition judgment in \cref{fig:statix-operational-semantics} is parameterized by a specification $\mathbb{S}$.
This specification is given by a set of predicate rules where each rule has the shape $P(\overline{t}) \rTurnstile{} C$.
Each rule is closed (\ie, $FV(P(\overline{t}) \rTurnstile{} C) = \emptyset$), and we assume that the domain of every predicate rule is disjoint from all other rules; \ie:
%
\[
    \forall P\ \overline{t_1}\ \overline{t_2}\ C_1\ C_2.
    \ \left( P(\overline{t_1}) \rTurnstile{} C_1 \right) \in \mathbb{S} \wedge
     \left( P(\overline{t_2}) \rTurnstile{} C_2 \right) \in \mathbb{S} \wedge
     (\exists \theta.\: \mathsf{mgu}(\overline{t_1}, \overline{t_2}) = \theta) \Rightarrow
     \overline{t_1} \equiv \overline{t_2} \wedge C_1 \equiv C_2
\]%

\noindent
We also assume that premises that access rules in a specification $\mathbb{S}$ (\eg, $(P(\overline{t}) \rTurnstile{} C) \in \mathbb{S}$) are automatically $\alpha$-renamed to avoid variable capture.
Rule \textsc{Op-Pred} thus expands a predicate only when there exists a \emph{unique} rule $P(\overline{t_2}) \rTurnstile C$ in specification $\mathbb{S}$ whose head matches the predicate constraint $P(\overline{t_1})$.
In case the predicate constraint matches multiple rules in $\mathbb{S}$, rule \textsc{Op-Pred} does not apply.
For example, if $P$ is a predicate symbol, $f$ and $g$ are constructors, $x$, $y$ are variables, and we have a specification with rules $P(f(x)) \rTurnstile C_1$ and $P(g(x)) \rTurnstile C_2$,
then the \textsc{Op-Pred} rule does not apply to the predicate constraint $P(y)$ because $y$ can be instantiated to both $f(x)$ and $g(x)$.
%; hence both rules might apply.

The substitution yielded by \textsf{mgu} in the premise of \textsc{Op-Pred} is applied to the entire configuration after unfolding a predicate.
Here and in the rest of the paper, $\mathsf{mgu}$ is the partial function computing the most general unifier (\ie, a substitution).
We use $\bot$ to denote failure in partial functions.
We use $\theta,\theta_1,\theta_{result},\ldots$ to range over substitutions of variables by terms ($\mathit{Var} \rightharpoonup \mathit{Term}$) or set variables by set terms ($\mathit{SetVar} \rightharpoonup \mathit{SetTerm}$).
The type of substitution will be clear from the context.
The substitution functions for constraints, terms, and scope graphs are standard and elided for brevity, except for the reference entity $U$ which we describe in~\cref{subsec:refsyn-operational-semantics}.

\paragraph{Logic Constraints}
The other rules in \cref{fig:statix-operational-semantics} are directly adapted from \citet{RouvoetAPKV20}.
Rule \textsc{Op-Conj} splits a separating conjunction constraint into two constraints.
\textsc{Op-Emp} dispatches the vacuously satisfiable constraint $\cEmp{}$.
\textsc{Op-Exists} unpacks an existentially quantified constraint by choosing a fresh variable name, which may get unified using, for example, \textsc{Op-Eq}.


\paragraph{Set Constraints}
Queries yield \emph{sets} of results, so the rules in \cref{fig:statix-operational-semantics} include two dedicated constraints for matching on sets.
The semantics of $\cSingle{t_1}{\underline{t_2}}$ is given in rule \textsc{Op-Singleton} which asserts that $t_2$ must be a singleton set $\{ t' \}$, such that $t_1$ unifies with $t'$.
The semantics of constraint $\cForall{x}{\underline{t}}{C}$ is given in rule \textsc{Op-Forall}.
The rule asserts that $\underline{t}$ must be some set literal $\zeta$; \ie, a union of singleton sets.
The rule expands $\cForall{x}{\zeta}{C}$ into as many constraints as $\zeta$ has singleton sets, in each case substituting $x$ for the singleton set inhabitant.


\paragraph{Scope Graph Constraints}
The rules \textsc{Op-New-Scope} and \textsc{Op-New-Edge} create new scopes and edges in the scope graph, respectively.
Rule \textsc{Op-Data} asserts that a constraint $\cDataOf{t_1}{t_2}$ can be solved when $t_1$ is a scope $s$, and $t_2$ unifies with the term associated with scope $s$.
Rule \textsc{Op-Query} has two premises.
The function $\mathsf{Ans}$ returns the set of all paths that match the query parameters (see~\cref{subsec:scope-graphs}).
The $\mathsf{guard}$ predicate ensures that the query is \emph{guarded} in the sense that the constraints $C; \overline{C}$ do not add a new edge to the scope graph that would cause the query to yield a different answer.
Both $\mathsf{Ans}$ and $\mathsf{guard}$ are discussed in detail by \citet[\S3.1 and \S{}5.3]{RouvoetAPKV20}.

\vspace{1em}
\subsection{Operational Semantics of Reference Synthesis}%
\label{subsec:refsyn-operational-semantics}

The usual operational semantics of Statix in \cref{fig:statix-operational-semantics} is conservative about solving query and predicate constraints.
As discussed before, \textsc{Op-Pred} solves a predicate constraint only when there is exactly one possible expansion, otherwise it is \emph{stuck}.
Similarly, \textsc{Op-Query} only solves a query constraint when the source scope of the query is ground (\ie, it is a scope rather than a variable), and the $\mathsf{guard}$ premise holds.

As we can observe from the example discussed in~\cref{sec:reference-synthesis-by-example}, speculatively solving predicate and query constraints allows the Statix solver to \emph{infer} what syntax of valid references to substitute for each hole of a program.
To admit such inference, the rules in~\cref{fig:refsyn-operational-semantics} let us \emph{speculatively} expand \emph{predicate} and \emph{query constraints} in states that would otherwise be stuck.
We achieve this by introducing two new relations: $\rightarrowtail$ performs a speculative expansion step, and $\twoheadrightarrowtail$ relates sets of potentially speculatively expanded configurations.


\paragraph{The $\twoheadrightarrowtail$ Relation}
As long as the regular Statix constraint solving rules can make progress, the \textsc{Op-Solve} rule applies.
Once the solver gets stuck, \textsc{Op-Expand} applies.
The set comprehension in the bottom premise of the rule lets us speculatively solve stuck query constraints and expand predicates.
Configurations for which neither \textsc{Op-Solve} nor \textsc{Op-Expand} apply are truly stuck, and will be pruned by the set comprehension premise of \textsc{Op-Expand}.


\paragraph{Speculative Predicate Expansion}
The rule \textsc{Op-Expand-Pred} augments the plain Statix constraint solving rules from \cref{fig:statix-operational-semantics} to support selecting an \emph{arbitrary} rule for expanding a predicate constraint.


\paragraph{Speculative Query Expansion}
The rule \textsc{Op-Expand-Query} augments the plain Statix constraint solving rules from \cref{fig:statix-operational-semantics} with support for solving a stuck query constraint,
by synthesizing a path from a (possibly unknown) source scope to the current target scope of the related hole.
The rule assumes that we are resolving a reference given by a composite path, and attempts to ``prepend'' a step to the composite path.
Intuitively, if we think of composite paths as (qualified) references, this corresponds to attempting to prepend a qualifier.


\textsc{Op-Expand-Query} uses the $U$ and $H$ components of the state, to determine which hole it is expanding.
For each free variable in the program $U$ tracks to which hole it is related.
For this reason, its substitution function $U[t/x]$ has a guard that checks that each free variable in $t$ are either not related to a hole, or are related to the same hole as $x$.
As shown previously in~\cref{fig:statix-config-syntax}, the state of a hole $H(h)$ is given by a pair $(\overline{s}, t)$.
Here, $t$ is a term representing the inferred syntax for the reference, while $\overline{s}$ represents a non-empty sequence of \emph{query-connected scopes} that form its composite path.



\begin{definition}[Query-Connected Scopes]
  For a given scope graph $\mathcal{G}$, two scopes $s_1$ and $s_2$ in $\mathcal{G}$ are \emph{query-connected} by a query \smash{$q = \qBase{\orderSyntax}{s_1}{r}{\lambda x.\: E}$}, which we denote \smash{$s_1 \xxrightarrow{q} s_2$}, when there exists an $p$  such that \smash{$p \in \mathsf{Ans}(\mathcal{G}, \qBase{o}{s_1}{r}{\lambda x.\: E})$} where either $s_2 = \mathsf{tgt}(p)$ or $s_2 \in \rho_{\mathcal{G}}(\mathsf{tgt}(p))$.
  \label{def:query-connected-scopes}
\end{definition}

This uses the notation $s_2 \in \rho_{\mathcal{G}}(\mathsf{tgt}(p))$ to mean that $s_2$ is a syntactic sub-term of the data associated with $\mathsf{tgt}(p)$.
Using the definition we can define composite paths, which model references.

\begin{definition}[Composite Path]
  \label{def:composite-path}
  For a given scope graph $\mathcal{G}$, a sequence of scopes $s_0\ldots s_n$, and a sequence of queries $q_i \in \overline{q}$, a composite path in the scope graph is given by a series of query-connected scopes; \ie:
  \smash{$s_0 \xxrightarrow{q_1} s_1 \cdots \xxrightarrow{q_n} s_n$}
\end{definition}

The first premise of rule \textsc{Op-Expand-Query} asserts that the data well-formedness predicate ($\lambda y. \mkern4mu E$) has a variable occurring in it which is relevant for a hole $h$.
The head scope $s_t$ of the composite path component in the state of $h$ represents the scope that we need to connect to, in order to prepend a step to a query-connected path.
The remaining premises assert that we choose a source scope $s'$ and a target scope $s''$ that is connected to $s_t$, such the query resolves from~$s'$ to~$s''$.


\paragraph{Accepting States}
The $\mathsf{Accept}(\overline{C}, H)$ premise of \textsc{Op-Expand} holds iff (1)~$\overline{C}$ is empty (all constraints are solved), and (2) the state of each hole in $H$ has a composite path component of length $> 1$ (we have constructed a composite path for each hole).


\begin{figure}[t]
  % Refsyn operational semantics
  \framebox[\columnwidth]{$\mathbb{S} \vdash \overline{\kappa} \twoheadrightarrowtail \overline{\kappa'}$ \hfill Configurations $\overline{\kappa}$ step to $\overline{\kappa'}$ using specification $\mathbb{S}$}
\begin{mathpar}
  \inferrule[Op-Solve\label{eq:op-solve}]{
    \mathbb{S} \vdash \kappa \rightarrow \kappa'
  }{
    \mathbb{S} \vdash \kappa ; \overline{\kappa}
    \twoheadrightarrowtail
    \kappa' ; \overline{\kappa}
  }
  %
  \and
  % %
  % \inferrule[\label{eq:op-solve-false}Op-Solve-Fail]{
  %   \kappa =
  %     \opsRSState
  %       { \mathcal{G} }
  %       { \{ \cFalse \} }
  %       { H }
  % }{
  %   \mathbb{S} \vdash \kappa ; \overline{\kappa}
  %   \rightsquigarrow
  %   \overline{\kappa}
  % }
  % %
  % \and
  %
  \inferrule*[left=Op-Expand\label{eq:op-expand}]{
    \mathbb{S} \vdash \opsRSState{\mathcal{G}}{\overline{C}}{U}{H} \not\to
    \qquad
    \neg \mathsf{Accept}(\overline{C}, H)
    \\
    \overline{\kappa'} = \left\{\,
        %\addvspace{0.2em}
        \opsRSState{\mathcal{G}}{\overline{C}}{U}{H'} \theta 
      \ \middle|\  
        \mathbb{S} \vdash \opsRSState{\mathcal{G}}{\overline{C}}{U}{H} \rightarrowtail \theta, H'\, \right\}
  }{
    \mathbb{S} \vdash \opsRSState{\mathcal{G}}{\overline{C}}{U}{H} ; \overline{\kappa}
    \twoheadrightarrowtail
    \overline{\kappa'} ; \overline{\kappa}
  }
\end{mathpar}
\framebox[\columnwidth]{$\mathbb{S} \vdash \kappa \rightarrowtail \theta, H$ \hfill For $\kappa$, synthesize substitution $\theta$ and hole state $H$, using specification $\mathbb{S}$}
\makeatletter % allow us to mention @-commands
\def\arcr{\@arraycr}
\makeatother
\begin{mathpar}
  \inferrule*[Left=Op-Expand-Pred\label{eq:op-expand-pred}]{
    % \set{h} = \set{ h \mid x \in FV(\overline{t_1}), h = U(x) }
    % \\
    % h \in \mathsf{dom}(H)
    % \\
    (P(\overline{t_2}) \leftarrow C) \in \mathbb{S}
    \\
    \mathsf{mgu}(\overline{t_1}, \overline{t_2}) = \theta
  }{
    \mathbb{S} \vdash \opsRSState
      { \mathcal{G} }
      { P(\overline{t_1}) ; \overline{C} }
      { U }
      { H }
    \rightarrowtail
      \theta
      , H
  }
  % 
  \and
  % 
  \inferrule[Op-Expand-Query\label{eq:op-expand-query}]{
    h \in \set{ h \mid x \in FV(\lambda y.\: E), h = U(x) }
    \qquad
    H(h) = \big(s_t \cdot \overline{s'_t}, t'\big)
    \\
    s', s'' \in S_{\mathcal{G}}
    \\
    \mathsf{mgu}(t, s') = \theta_1
    \\
    (s_t = s'' \vee s_t \in \rho_{\mathcal{G}}(s''))
    \\
    \mathcal{G} \vdash E[s''/y]\theta_1 \rightsquigarrow \theta_2
    \\
    \exists p \in \mathsf{Ans}( \mathcal{G}\theta_1\theta_2, \qBase{\orderSyntax}{s'}{r}{\lambda y.\: E\theta_1\theta_2} ).\: \mathsf{tgt}( p ) = s''
    \\
    \mathsf{guard}(\mathcal{G}\theta_1\theta_2, (\cQuery[\orderSyntax]{s'}{r}{\lambda x.\: E}{z}.\: C)\theta_1\theta_2, \overline{C})
  }{
    \mathbb{S} \vdash \opsRSState
      { \mathcal{G} }
      { \cQuery[\orderSyntax]{t}{r}{\lambda y.\: E}{z}.\: C ; \overline{C} }
      { U }
      { H }
    \rightarrowtail
    \theta_1 \theta_2
    , H\big[ h \mapsto \big(s' \cdot s_t \cdot \overline{s'_t}, t' \big) \big]
  }
\end{mathpar}
\framebox[\columnwidth]{Auxiliary functions}
\begin{align*}
  U[t/x] &=
    \begin{cases}
      U & \text{if $x \not\in \mathsf{dom}(U)$}
      \\
      U[ \{ y \mapsto U(x) \mid y \in FV(t) \} ]
        & {\arraycolsep=0pt\def\arraystretch{1}
          \begin{array}[t]{l}
          \text{if $x \in \mathsf{dom}(U)$ and}\\
          \quad\text{$\forall y \in FV(t), h'.\ U(y) = h' \Rightarrow h' = U(x)$}
          \end{array}}
      \\
      \bot 
        & \text{otherwise}
    \end{cases}
\end{align*}

  %
  \captionof{figure}{
    Operational semantics of reference synthesis.
  }%
  \label{fig:refsyn-operational-semantics}
\end{figure}

\subsection{Building the Synthesize Function}%
\label{subsec:building-the-synthesizer}

Now, we have all the pieces to build the $\mathsf{synthesize}$ function:
\[
  \inferrule{
    D = \set{ x \mapsto d \mid \rhole[4]{d} \in \mathcal{P}_{\text{locked}}, x \text{ fresh} }
    \\
    U = \set{ x \mapsto h \mid x \in \mathsf{dom}(D), h \text{ fresh} }
    \\
    \theta_1 = \set{ \rhole[4]{d} \mapsto x \mid D(x) = d }
    \\
    \mathcal{P}_0 = \mathcal{P}_{\text{locked}}\theta_1
    %(\mathcal{P}_0, X) =\mathcal{P}_{\text{locked}}[x / \rhole{d} | X(x) = (h, d)]
    \\
    \opsRSState
      {\emptyset}
      {P_0(\mathcal{P}_0)}
      {U}
      {\emptyset}
    \rightarrow^\ast
    \opsRSState
      {\SG_0}
      {\overline{C}}
      {U}
      {\emptyset}
    \\
    H = \set{ h \mapsto (s_d, x) \mid U(x) = h, D(x) \in \rho_{\mathcal{G}_0}(s_d) }
    \\
    \opsRSState
      {\SG_0}
      {\overline{C}}
      {U}
      {H}
    \twoheadrightarrowtail^\ast
    \overline{\kappa}
    \\
    \opsRSState
      {\SG}
      {\overline{C}'}
      {U'}
      {H'}
    \in
    \overline{\kappa}
    \\
    \mathsf{Accept} \big(\overline{C}', H' \big)
    \\
    \theta_{result} = \set{ x \mapsto t \mid H'(U(x) ) = (s_t, t) }
  }{
    \mathsf{synthesize}(\mathcal{P}_{\text{locked}}) = \mathcal{P}_0\theta_{result}
  }
\]

\noindent
We first create a fresh unification variable and hole for each locked reference in the program ($D$ and $U$, respectively),
and replace each locked reference by a fresh unification variable in the program ($\theta_1$).
Then, we solve the initial constraint ($P_0)$ on the program with holes.
From the partial scope graph in the result state, we initialize the hole state $H$ for each hole.
Then, we synthesize the references, and extract an accepted state.
From this state, we build a substitution $\theta_{result}$ that substitutes each hole variable with the synthesized reference term.

\pagebreak[4]  %% Suggested page break

\subsection{Properties}%
\label{subsec:operational-semantics-discussion}

In this section, we discuss some properties of our operational semantics: soundness, completeness, and confluence; and we discuss liveness.

\paragraph{Soundness}
%%%%%%%%%%%%%%
% Well-typed %
%%%%%%%%%%%%%%
First, we consider the \emph{soundness} of our approach.
Soundness consists of two components:
(1) the resulting solutions are well-typed; and
(2) each solution corresponds to a composite path in the scope graph to the target declaration.
%
\begin{theorem}[Soundness 1]
  \label{thm:soundness-1}
  Programs with synthesized references are well-typed.
\end{theorem}
%
%%%%%%%%%%%%%%%%%%%%%
% Correct reference %
%%%%%%%%%%%%%%%%%%%%%
\begin{theorem}[Soundness 2]
  \label{thm:soundness-2}
  Every synthesized reference corresponds to a composite path (\cref{def:composite-path}) to the target it was initially locked to.
\end{theorem}%
\iftoggle{extended}{%
  \Cref{thm:soundness-1,thm:soundness-2} have formal definitions and proofs in~\cref{subsec:well-typed-solutions,subsec:composite-paths}, respectively.
}{%
  Formal definitions and proofs of~\cref{thm:soundness-1,thm:soundness-2} can be found in the extended version of this paper~\cite{Pelsmaeker_OOPSLA25_Extended}, appendices A.2 and A.3, respectively.
}%

\paragraph{Completeness}
Ideally, we would conjecture \emph{completeness} as well.
However, completeness is hard to define, as it must rely on a generic (language-independent) definition of a reference.
For the purpose of this paper, we consider a well-formed reference in scope $s_0$ to declaration $d$ to be a composite path (\cref{def:composite-path}) through scope graph $\mathcal{G}$, from $s_0$ to $s_d$ where $d \in \rho_{\mathcal{G}}(s_d)$ (\ie, $s_d$ is associated with declaration $d$).
However, this definition is an over-approximation, as queries could be connected `by accident'.
For example, a Java expression \texttt{a.m(b)} where \id{a} is an instance of the class in which this expression occurs could be considered a reference \id{a.b} by our definition, as the target of the query for \id{a} would match the source of the query of \id{b}.
Thus, our definition is not suitable to state a completeness theorem.
In~\cref{subsec:results} we show experimental evidence that our approach is practically complete.



%%%%%%%%%%%%%%
% Confluence %
%%%%%%%%%%%%%%
\paragraph{Confluence}
We also consider the property of \emph{confluence}: if two different expansions are possible, eventually, the final state sequence will be equivalent.

\begin{theorem}[Confluence]
  If\ $\overline{\kappa} \twoheadrightarrowtail \overline{\kappa}_1$~and~$\overline{\kappa} \twoheadrightarrowtail \overline{\kappa}_2$, %
  then $\exists \overline{\kappa'}.\: \overline{\kappa}_1 \twoheadrightarrowtail^\ast \overline{\kappa'} \land \overline{\kappa}_2 \twoheadrightarrowtail^\ast \overline{\kappa'} $
\end{theorem}

\begin{proof}
  This is a proof by case analysis on the expanded state:
  \begin{itemize}
    \item If different states were expanded in $\overline{\kappa}_1$ and $\overline{\kappa}_2$, the step made to obtain $\overline{\kappa}_1$ can be applied on~$\overline{\kappa}_2$ as well, and vice versa.
            This yields equivalent states again.
    \item If the same state was expanded in $\overline{\kappa}_1$ and $\overline{\kappa}_2$, either one of the following is true:
    \begin{itemize}
      \item An \textsc{Op-Solve}-step was made in both cases.
        In this case, confluence holds by virtue of~$\rightarrow$~being confluent~\citep[Theorem~4.5]{RouvoetAPKV20}.
      \item An \textsc{Op-Expand}-step was made in both cases.
        As this rule ranges over all possible expansions, $\overline{\kappa}_1 = \overline{\kappa}_2$, so confluence trivially holds.
    \end{itemize}
    The same state cannot be expanded with both \textsc{Op-Solve} and \textsc{Op-Expand}, as the first premise of \textsc{Op-Expand} requires the state to be stuck (\ie, \textsc{Op-Solve} does not apply).
  \end{itemize}
  \vspace{-1.2\baselineskip}
\end{proof}
\noindent
This confluence result is especially important for our next section, as we exploit this property to design heuristics that reduce the huge search space of possible expansions.



%%%%%%%%%%%%
% Liveness %
%%%%%%%%%%%%

\paragraph{Liveness}
Finally, albeit not a property of the operational semantics, we discuss \emph{liveness} here as well.
Liveness consists of two components:
(1) the synthesis finds each solution in finite time, and
(2) when no (new) solution is available, the synthesis terminates.
Property (1) can be guaranteed by scheduling the expansion steps `fairly'; \ie in a breadth-first manner.
Property (2) requires special care for predicates that are (potentially) infinitely expanding without yielding a solution.
We return to this in~\cref{subsec:recursive-qualifiers}.
