\section{Scope Graphs and Statix}%
\label{sec:scope-graphs-and-statix}

This paper presents a \emph{language-parametric} approach to synthesizing concrete references.
To this end, we build on existing work: (1) \emph{scope graphs} as a language-parametric model of name binding,
and (2) \emph{Statix} as a uniform representation of typing rules.

In this section we first describe what scope graphs are (\cref{subsec:scope-graphs}), and how they let us resolve references via graph search (\cref{subsec:scope-graph-queries}).
Then we provide a high-level introduction to the Statix language (\cref{subsec:statix-rules}) and its constraint solver (\cref{subsec:statix-constraint-solver}).
% AZ: Rouvoet et al. do not discuss typing rules, which are kind-of essential here.
% The
%   closely follows % AZ: sounds to me like the narrative structure is the same, which IMHO is not really true/not really the point
% summarizes
%~\citet{RouvoetAPKV20}, and readers familiar with this work may wish to skip to \cref{sec:reference-synthesis-by-example} directly.


\subsection{Scope Graphs}%
\label{subsec:scope-graphs}
%%%%%%%%%%%%%%%%%%%%%%%%%%%
The example program in~\cref{fig:scope-graph-example-a} contains two declarations named~$\id{x}$, namely $\posid{x}{1}$ on line 1 and $\posid{x}{2}$ on line 3, and a named reference $\id{x}$ on line 7.\footnote{
  We use \emph{subscript indices} to distinguish different \emph{occurrences} of a particular name.
  For example, $\posid{x}{2}$ uniquely identifies one of the declarations named $\id{x}$.
}
The question is: does~$\id{x}$ refer to declaration~$\posid{x}{1}$ or~$\posid{x}{2}$?
Either can be true, depending on the semantics of the programming language.

Scope graphs~\cite{NeronTVW15,RouvoetAPKV20,AntwerpenPRV18,AntwerpenNTVW16,ZwaanA23} offer a uniform model for name resolution that supports sophisticated name binding patterns in programming languages.
As their name suggests, scope graphs model the scoping structure of programs as graphs.
Such graphs let us model both nested and recursive scoping structures,
and name resolution policies as graph search queries.

To illustrate this, consider the program and its scope graph in~\cref{fig:scope-graph-example}.
The nodes in the graph represent scopes: $s_0$ represents the \emph{global scope}, while~$s_{\id{A}}$ and~$s_{\id{B}}$ represent the scopes of modules~$\id{A}$ and~$\id{B}$, respectively.
Scopes $s_{\id{x}1}$, $s_{\id{x}2}$, and $s_{\id{y}}$ represent named declarations.
A scope $s$ may have data~$d$ associated with it, written as $s \mapsto d$, such as the name of the module that they correspond to or the name and type of their declaration.
As we shall see later, associating scopes with names lets us define \emph{name resolution queries} that resolve module names to their corresponding scopes.

Edges between scopes represent \emph{reachability} relations.
Queries can follow these edges to reach other scopes and declarations.
The two module scopes are reachable from the global scope via $\lblMOD$ edges, representing the fact that $\id{A}$ and $\id{B}$ are modules declared in the global scope $s_0$.
The module scopes $s_{\id{A}}$ and $s_{\id{B}}$ are also lexical children of the global scope, so each is connected via a $\lblLEX$ edge to $s_0$.
$\lblVAR$ edges connect scopes to declared names.
Due to the wildcard \MiniMod{import A::*} on line 6 (which imports all members of the module \id{A}) module scope $s_{\id{B}}$ is connected to module scope $s_{\id{A}}$ via an $\lblIMPORT$ edge, making all declarations in module \id{A} reachable from module \id{B}.

\begin{figure}
  \subcaptionbox{
    Program.
    \label{fig:scope-graph-example-a}
  }[0.25\textwidth]{
    \begin{lstlisting}[
  language=MiniMod
]
  var `\posidc{x}{1}` = 42
  mod `\id{A}` {
    var `\posidc{x}{2}` = 0
  }
  mod `\id{B}` {
    import `\id{A}`::*
    var `\id{y}` = `\id{x}`
  }
\end{lstlisting}

  }%
  \hfill
  \subcaptionbox{
    Scope graph for the program in~\cref{fig:scope-graph-example-a}.
    \label{fig:scope-graph-example-b}
  }[0.74\textwidth]{
    \centering
\begin{tikzpicture}[
  scopegraph,
  node distance = 2.5em and 3.5em
]
  % global scope
  \node[scope] (s0) {$s_0$};
  % .. variable x_1
  \node[scope, below = of s0] (sx1) {$s_{\id{x}1} \mapsto \posid{x}{1} : \tyINT$};
  \draw (s0) edge[lbl=$\lblVAR$] (sx1);

  % module A
  \node[scope, left = of s0] (sA) {$s_{\id{A}} \mapsto \id{A}$};
  \draw (s0) edge[lbl=$\lblMOD$] (sA);
  \draw (sA.south east) edge[lbl=$\lblLEX$, bend right=20] (s0.south west);
  % .. variable x_2
  \node[scope, below = of sA] (sAx2) {$s_{\id{x}2} \mapsto \posid{x}{2} : \tyINT$};
  \draw (sA) edge[lbl=$\lblVAR$] (sAx2);

  % module B
  \node[scope, right = of s0] (sB) {$s_{\id{B}} \mapsto \id{B}$};
  \draw (s0) edge[lbl=$\lblMOD$] (sB);
  \draw (sB.south west) edge[lbl=$\lblLEX$, bend left=20] (s0.south east);
  % .. variable y
  \node[scope, below = of sB] (sBy) {$s_{\id{y}} \mapsto \id{y} : \tyINT$};
  \draw (sB) edge[lbl=$\lblVAR$] (sBy);

  % .. import
  \draw (sB.north) edge[bend right, lbl=$\lblIMPORT$] (sA.north);

  % query
  \begin{scope}[color=colorblind-bright-3]
    % query node
    \node[ref, right = of sB] (qx) {
      $\qBase{\lblOrdLexVAR}{
        s_\id{B}
      }{
        \reclos{\lblLEX}\reopt{\lblIMPORT}\lblVAR
      }{
        \hoPred{isVar}{\id{x}}
      }$};
    \draw (qx) edge[ref] (sB);

    % s_B - X -> s_y
    % \draw (sB) edge[ref, bend right, strike through=0.30] (sBy);

    % s_B - - -> s_0
    %\draw (sB.north west) edge[ref, -, bend right=20, strike through=0.70] (s0.north east);
    % s_0 - - -> s_x1
    %\draw (s0) edge[ref, bend right, strike through=0.30] (sx1);
    
    % s_0 - X -> s_A
    % \draw (sA.north east) edge[ref, -, bend left, strike through=0.70] (s0.north west)
    % s_A - - -> s_x3
    \draw (sA) edge[ref, bend right] (sAx2);
    \node (sAx2-green) [scope, fit=(sAx2), inner sep=0, line width=1.0pt] {}; % Green thick line
    \node (sAx2-green) [scope, color=black, fit=(sAx2), inner sep=0] {};      % Bring black line back

    % s_B - - -> sA
    \draw (sB) edge[ref, bend right=25] (sA);

  \end{scope}
\end{tikzpicture}
  }%
  \caption{
    An example LM~\cite{NeronTVW15} program and its scope graph, where boxes and arrows represent scopes and reachability relations between scopes, respectively.
    The {\color{colorblind-bright-3}dashed box} represents a query and the {\color{colorblind-bright-3}dashed arrows} its resolution path to $\posid{x}{2}$, the second occurrence of a declaration named $\id{x}$.
  }%
  \label{fig:scope-graph-example}
\end{figure}





\subsection{Scope Graph Queries}%
\label{subsec:scope-graph-queries}

%%%%%%%%%%%%%%%%%%%%%%%%%%%%%%%
We define name resolution as \emph{queries} in scope graphs.
Resolving a query entails finding all paths from this source scope to matching declarations.
To explain the syntax of a name resolution query, we take the query shown in the {\color{colorblind-bright-3}dashed box} on the right of the graph in \cref{fig:scope-graph-example-b}:

\begin{equation*}
  \qBase{
    \lblOrdLexVAR
  }{
    s_{\id{B}}
  }{
    \reclos{\lblLEX}\reopt{\lblIMPORT}\lblVAR
  }{
    \hoPred{isVar}{\id{x}}
  }
\end{equation*}

\noindent
Here, $s_{\id{B}}$ is the initial scope of the graph search, and $\hoPred{isVar}{\id{x}}$ is a filter that ensures only declarations with name \id{x} are selected.
The regular expression $\reclos{\lblLEX}\reopt{\lblIMPORT}\lblVAR$ is a \emph{reachability policy} declaring which declarations are reachable; \ie, those declarations we can reach by following a sequence of labeled edges that match the regular expression.
The path ordering $\lblOrdLexVAR$ is a \emph{visibility policy} used to disambiguate which reachable names are visible, \ie, to model shadowing.
For example, both $s_{\id{x}1}$ and $s_{\id{x}2}$ are reachable in~\cref{fig:scope-graph-example-b}.
However, the order prefers $\lblIMPORT$ edges over $\lblLEX$ edges, so the only valid path through the graph is the path to $s_{\id{x}2}$.


\subsection{Statix Rules and Constraints}%
\label{subsec:statix-rules}
%%%%%%%%%%%%%%%%%%%%%%%%%%%
In classical typing rules, terms are typed relative to one or more \emph{typing contexts}~\cite{Pierce2002}, or typed via \emph{symbol tables}~\cite{au72} or \emph{class tables}~\cite{IgarashiPW01}.
Following existing work~\cite{AntwerpenPRV18,PoulsenRTKV18,RouvoetAPKV20}, we can define typing rules in a similar style, but with terms typed relative to one or more scopes in a scope graph instead.
The constraint language Statix~\cite{AntwerpenPRV18,RouvoetAPKV20,AntwerpenV21} lets us declare such inference rules using a syntax inspired by logic programming.
Type system specifications written in Statix have a declarative interpretation, specifying a class of well-typed programs.
Alternatively, specifications can be used operationally to type check programs by constructing a scope graph and resolving references by traversing it.
This subsection highlights the main features of Statix rules and constraints.
For a more detailed breakdown of the syntax, we refer to the discussion in~\cref{subsec:statix-syntax} and the work of~\citet{RouvoetAPKV20}.

\begin{figure}[t]
  
\az{Query syntax does not use binder notation.}
% Rules for MiniMod used in algorithm implementation examples
\begin{align*}
  \cUser{typeOfExpr}(\svar{s}, \svar{n}, \tyINT) \rTurnstile {} &
  \cEmp{}
  \tag{T-Num}
  \label{eq:t-num}
  \\
  %
  %
  %
  % [T-Add] typeOfExpr(s, Plus(e1, e2), INT()) :-
  %   typeOfExpr(s, e1, INT()),
  %   typeOfExpr(s, e2, INT()).
  \cUser{typeOfExpr}(\svar{s}, \svar{e_1} + \svar{e_2}, \tyINT) \rTurnstile {} &
    \cUser{typeOfExpr}(\svar{s},\svar{e_1}, \tyINT)
    \cConj\ 
    \cUser{typeOfExpr}(\svar{s},\svar{e_2}, \tyINT)
  \tag{T-Add}
  \label{eq:t-add}
  \\
  %
  %
  %
  % [T-Var] typeOfExpr(s, Var(x), T) :- {p}
  %   query var
  %     filter P*I? and { x' :- x' == x }
  %     in s |-> [(p, (x, T))].
  \cUser{typeOfExpr}(\svar{s}, \svar{x}, \svar{T}) \rTurnstile {} &
    \cQuery[\lblOrdLexVAR]{\svar{s}}{
      \reclos{\lblLEX}\reopt{\lblIMPORT}\lblVAR
    }{
      \hoPred{isVar}{\svar{x}}
    }{
      \set{\tuple{\_, \svar{x} : \svar{T}}}
    }
    %
  \tag{T-Var}
  \label{eq:t-var}
  \\[0.5em]
  \cUser{typeOfExpr}(\svar{s}, \svar{a}.\svar{x}, \svar{T}) \rTurnstile {} &
    \cExists{\svar{s_m}}\,
    \cUser{scopeOfMod}(\svar{s}, \svar{a}, \svar{s_m}) \cConj\\
  & \cQuery[]{\svar{s_m}}{
      \lblVAR
    }{
      \hoPred{isVar}{\svar{x}}
    }{
      \set{\tuple{\_, \svar{x} : \svar{T}}}
    }
    %
  \tag{T-QRef}
  \label{eq:t-qref}
  \\[0.5em]
  % memberOk(s, FieldDecl(name, expr)) :- {T}
  %   typeOfExpr(s, expr) == T,
  %   declareField(s, name, T).
  \cUser{memberOk}(\svar{s}, \text{\MiniMod{var}}\ \svar{x}\ \text{\MiniMod{=}}\ \svar{e}) \rTurnstile {} & \cExists {\svar{T}\, \svar{s_x}}
    \cUser{typeOfExpr}(\svar{s}, \svar{e}, \svar{T})
    \cConj {}\\ &
    \cNew {\svar{s_x}} \mapsto \svar{x} : \svar{T}
    \cConj {}
    \svar{s} \cEdge[\lblVAR] \svar{s_x}
  \tag{M-Var}
  \label{eq:m-var}
  \\[0.5em]
  % [T-Mod]
  % modOk(s, ModDecl(modname, imports, members)) :- {s_mod}
  %   new s_mod,
  %   s_mod -LEX-> s,
  %   declareMod(s, modname, s_mod),
  %   importsOrIncludesOk(s_mod, imports),
  %   membersOk(s_mod, members).
  \cUser{modOk}(\svar{s},
    \text{\MiniMod{mod}}\ \svar{a}\text{\MiniMod{\ \{}}\ \overline{\svar{imp}}\ \overline{\svar{mem}\vphantom{imp}} \text{\MiniMod{\ \}}})
  \rTurnstile {} & \cExists{\svar{s_m}}
    \cNew \svar{s_m} \mapsto a
    \cConj {}
    \svar{s_m} \cEdge[\lblLEX] s
    \cConj {}
    \svar{s} \cEdge[\lblMOD] \svar{s_m}
    \cConj {}\\ &
    \cUser{importsOk}(\svar{s_m}, \overline{\svar{imp}})
    \cConj
    \cUser{membersOk}(\svar{s_m}, \overline{\svar{mem}})
  \tag{M-Mod}
  \label{eq:m-mod}
  \\[0.5em]
  %
  % [T-QVar] typeOfExpr(s, QVar(m, x), T) :- {s_m}
  %   scopeOfMod(s, m) == s_m,
  %   query var 
  %      filter e and { x' :- x' == x } 
  %      in s_m |-> [(_ (_, T))]. 
  % \cUser{typeOfExpr}(\svar{s}, \emb{\svar{m}.\svar{x}}, \svar{T}) \rTurnstile {} &
  %   \cExists{\svar{s_m}}
  %   \cUser{scopeOfMod}(\svar{s}, \svar{m}, \svar{s_m})
  %   \cConj {} \\
  %   &
  %   \cQuery{\svar{s_m}}{
  %     \lblVAR
  %   }{
  %     \hoPred{isVar}{\svar{x}}
  %   }{
  %     \set{\tuple{\_, \svar{x} : \svar{T}}}
  %   }
  %   %
  % \tag{T-QVar}
  % \label{eq:t-qvar}
  % \\
  % [F-Fun] typeOfExpr(s, Fun(x, Tx, e), FUN(Tx, T)) :- {s_f}
  %   new s_f, s_f -LEX-> s,
  %   !var[x, Tx],
  %   typeOfExpr(s_f, e, T).
  % \hspace{-0.875em} \cUser{typeOfExpr}(s, \emb{\lambda(\svar{x}{:}\, \svar{T_x}). \svar{e}}, \svar{T_x} \mathop{\to} \svar{T}) \rTurnstile {} &
  %   \cExists{\svar{s_f}\, \svar{s}_x}
  %       \cNew \svar{s_f} \mapsto \svar{s_f}
  %       \cConj {}\
  %       \svar{s_f} \cEdge[\lblLEX] s
  %       \cConj {}\\
  %   &\cNew \svar{s}_x \mapsto (\svar{x}{:}\, \svar{T_x})
  %       \cConj {}\
  %       \svar{s_f} \cEdge[\lblVAR] \svar{s}_x
  %       \cConj {}\\
  %   &\cUser{typeOfExpr}(\svar{s_f}, {\svar{e}}, \svar{T})
  %   \tag{T-Lam}
  %   \label{eq:t-fun} % backward compatibility
  % \\
  %
  %
  % [D-ImportOk] importOk(s, StarImport(m)) :- {sm}
  %   scopeOfMod(s, m, sm),
  %   s -I-> sm.
  \cUser{importOk}(\svar{s}, \text{\MiniMod{import}}\ \svar{a}\text{\MiniMod{::*}}) \rTurnstile {} &
    \cExists{\svar{s_m}}
    \cUser{scopeOfMod}(\svar{s}, \svar{a}, \svar{s_m})
    \cConj {}\
    \svar{s} \cEdge[\lblIMP] \svar{s_m}
    %
  \tag{D-ImportOk}
  \label{eq:d-importok}
  \\[0.5em]
  %
  %
  %
  % [S-Mod] scopeOfMod(s, Ref(x), s_m) :-
  %   query mod 
  %     filter P* and { x' :- x' == x } 
  %     in s |-> [(_, (_, s_m))]
  \cUser{scopeOfMod}(\svar{s}, \svar{x}, \svar{s_m}) \rTurnstile {} &
    \cExists{\svar{p}}
    \cQuery[\lblOrdLexMOD]{\svar{s}}{
      \reclos{\lblLEX}\lblMOD
    }{
      \hoPred{isMod}{\svar{x}}
    }{
      \set{\tuple{\svar{p}, \svar{x}}}
    }
    \cConj
    \svar{s_m} \cEq \mathsf{tgt}(p)
    %
  \tag{S-Mod}
  \label{eq:s-mod}
  \\[0.5em]
  %
  %
  %
  % [S-QMod] scopeOfMod(s, QRef(a, x), s_m) :- {s_p}
  %   scopeOfMod(s, a, s_p),
  %   query mod 
  %     filter e and { x' :- x' == x } 
  %     in s_p |-> [(_, (_, s_m))]
  \cUser{scopeOfMod}(\svar{s}, \svar{a}.\svar{x}, \svar{s_m}) \rTurnstile {} &
    \cExists{\svar{p}\, \svar{s_m'}}\,
    \cUser{scopeOfMod}(\svar{s}, \svar{a}, \svar{s_m'})
    \cConj {} \\
    &
    \cQuery{\svar{s_m'}}{
      \lblMOD
    }{
      \hoPred{isMod}{\svar{x}}
    }{
      \set{\tuple{\svar{p}, \svar{x}}}
    }
    \cConj
    \svar{s_m} \cEq \mathsf{tgt}(p)
    %
  \tag{S-QMod}
  \label{eq:s-qmod}
  \\[0.5em]
  %
  %\\\noindent\rule{\textwidth}{0.4pt}\\
  %
  \hoPred{isVar}{\svar{x}} \triangleq {} &
    \lambda \svar{x'}.\: \, \exists \svar{T}.\: (\svar{x}{:}\, \svar{T}) \cEq \svar{x'}
  \\
  %
  \hoPred{isMod}{\svar{x}} \triangleq {} &
    \lambda \svar{x'}.\: \svar{x} \cEq \svar{x'}
\end{align*}

  \caption{
    A subset of the typing rules of LM, a toy language from~\cite{NeronTVW15} used for the examples in this paper.
  }%
  \label{fig:alg-minimod-rules}%
  \vspace{-0.5\baselineskip} % Latex inserts empty line here
\end{figure}

The rules in~\cref{fig:alg-minimod-rules} show a representative subset of the Statix rules we derived for LM, a toy language from~\cite{NeronTVW15} used throughout this paper.
The figure declares rules for five different typing relations: $\cUser{typeOfExpr}$, $\cUser{memberOk}$, $\cUser{modOk}$, $\cUser{importOk}$, and $\cUser{scopeOfMod}$.
Each rule has a conclusion on the left of an arrow ($\leftarrow$), and a premise given by one or more constraints on the right.
For example, the \rref{eq:t-add} states that the expression $\svar{e_1} + \svar{e_2}$ has type $\tyINT$ in scope $\svar{s}$, if both $\svar{e_1}$ and $\svar{e_2}$ have type $\tyINT$ under the same scope $s$.
\Rref{eq:t-qref}~is a more complex example, where a qualified module access expression $a.x$ has type $T$ when the $a$ resolves to a module
(asserted by the predicate constraint $\cUser{scopeOfMod}(\svar{s}, \svar{a}, \svar{s_m})$),
and $x$ resolves from that module to a declaration of type $T$
(asserted by the query constraint \smash{$\qBase{}{\svar{s_m}}{\lblVAR}{\hoPred{isVar}{\svar{x}}}$}).

The rules in~\cref{fig:alg-minimod-rules} do not mention the underlying scope graph explicitly.
Instead, premises of rules assert requirements on the scope graph structure, such as the existence of scopes with associated data ($\cNew{\svar{s_x}} \mapsto \svar{x} : \svar{T}$) and edges ($\svar{s} \cEdge[\lblVAR] \svar{s_x}$) and the ability to resolve query constraints.
A program is well-typed when a \emph{minimal} scope graph exists that satisfies each such assertion and query.
Minimality implies that the scope graph only has the scopes and edges asserted by the rules of a program: no extraneous edges or scopes.
There exists a solver for Statix constraints that computes this minimal scope graph, which we discuss in the next section.



\subsection{Statix Constraint Solver}%
\label{subsec:statix-constraint-solver}
%%%%%%%%%%%%%%%%%%%%%%%%%%%%%%%%%%%%%%%
Following~\citet{RouvoetAPKV20}, the operational semantics of Statix is given by
%an executable implementation of
a constraint solver that soundly constructs and queries scope graphs, and uses unification to solve equality constraints.
The reference synthesis approach we illustrate in \cref{sec:reference-synthesis-by-example} is sound by construction because it builds on this operational semantics.
We defer a deeper discussion of the operational semantics to \cref{sec:operational-semantics}.

The Statix solver will solve as many constraints as possible, yielding either a state with no unsolved constraints (\ie, the program type-checks), a state that derives $\mathsf{false}$ (\ie, the program does not type-check), or a \emph{stuck} state, where the solver does not have enough information to solve the remaining constraints.
There are two reasons why constraints get stuck:
either (1) it is not sufficiently instantiated,
or (2) it is a query constraint which is not yet guaranteed to yield a \emph{stable answer}.
For (1), the solver will only expand a predicate such as $\cUser{typeOfExpr}(x, y, z)$ once $x$,~$y$, and $z$ are sufficiently instantiated such that only a single rule matches.
Similarly, it will only run and solve a query constraint once its \emph{source scope} (\eg, $s$ in \smash{$\qBase{\lblOrdLexVAR}{s}{\reclos{\lblLEX}\reopt{\lblIMPORT}\lblVAR}{\hoPred{isVar}{\svar{x}}}$}) and \emph{data well-formedness predicate} (\eg, $\hoPred{isVar}{\svar{x}}$) are ground.
In case (2), a query gets stuck when it needs to run but another unsolved constraint might add a scope graph edge that could invalidate the query.
%Following~\citet{RouvoetAPKV20},
The Statix solver implements guards that detect these cases~\citep{RouvoetAPKV20}.

Our $\mathsf{synthesize}$ function runs the Statix solver on a program with holes, where each hole is represented by a free unification variable that maps to a target scope representing the hole's intended target declaration.
The unification variables cause the Statix solver to get stuck on predicate and query constraints directly related to the holes.
Once a stuck state is reached, our reference synthesis approach extends the usual operational semantics of Statix with the ability to use the typing rules of a language to refine the holes of the program, and use the Statix solver to verify the solution.
Once the term of a hole becomes ground and all constraints in the state have been solved, we have successfully synthesized a concrete reference.

We will illustrate how the Statix solver, scope graph, and typing rules are used in our reference synthesis algorithm in \hyperref[sec:reference-synthesis-by-example]{the next section}, and discuss the operational semantics of Statix and our extension in more detail in~\cref{sec:operational-semantics}.

