\section{Reference Synthesis by Example}%
\label{sec:reference-synthesis-by-example}

As the name suggests, \emph{reference synthesis} is used to synthesize a concrete (qualified) \emph{reference} to a given declaration.
References in many languages take the shape $x_1.x_2. \cdots .x_n$, modulo syntax.
Here, the first name $x_1$ is resolved from the place in the program where the reference occurs, and subsequent names $x_i$ are resolved relative to wherever the previous qualifiers $x_1. \cdots .x_{i-1}$ led.
The final name $x_n$ leads to the target declaration.

This informal definition of a reference encompasses many syntactic constructs that we intuitively recognize as (qualified) references across languages, for example \Java|Person.this.name| in Java, \lstinline[language=C]|std::option::Option| in Rust, and \lstinline[language=Cobol]|ID| \lstinline[language=Cobol]|IN| \lstinline[language=Cobol]|CUSTOMER| \lstinline[language=Cobol]|IN| \lstinline[language=Cobol]|LAST-TRANSACTION| in Cobol. %\footnote{Even a domain name such as \texttt{en.wikipedia.org} has the shape of a qualified reference, although it is not part of a programming language.}
On the other hand, according to our definition, syntax like \Java|List<String>| in Java does not constitute a reference: it is a parameterized type, akin to how a method call \Java|foo(x, y)| would also not be considered a reference.
We give a more precise definition of a reference in~\cref{sec:operational-semantics}.

%\vspace{0.1em}%
\[
  \color{gray!75!black}
  \mathcal{P} =
  X\overline{[r]} \xrightarrow{\textsf{lock}}
  X\overline{[\rhole[5]{d}]} \xrightarrow{\textsf{transform}}
  {\color{black}
  X'\overline{[\rhole[5]{d}]} \xrightarrow{\textsf{unlock}}                 
  X'\overline{[r']}
  }
  = \mathcal{P}'
\]%
%\vspace{0.1em}

\noindent
The above diagram reiterates program transformation with locked references.
Initially, we have a program $\mathcal{P}$, which can be represented as a context\footnote{We can regard a `context' as a zipper-like structure over an AST.} $X$ with references $\overline{r}$ occurring in it.
First, we $\mathsf{lock}$ relevant references.
That is, we replace concrete references with locked references that durably remember which declaration they point to, regardless of where the locked reference occurs in the program (\smash{$X\overline{[\rhole[5]{d}]}$}).
Then we $\mathsf{transform}$ the program, possibly moving code around (\smash{$X'\overline{[\rhole[5]{d}]}$}).
Finally, we $\mathsf{unlock}$ the locked references: synthesizing their concrete references ($\overline{r'}$) and plugging them into the program to yield the concrete transformed program (\smash{$\mathcal{P}' = X'\overline{[r']}$}).

In this section, we use a simple program with a locked reference, shown in~\cref{fig:minimod-locked-reference-example}, to illustrate the semantics of our $\mathsf{synthesize}$ function.
The idea is to model locked references as holes given by unification variables, and strategically apply typing rules to infer a substitution for each hole.
The strategy used to apply typing rules must guarantee that inferred substitutions correspond to references that resolve as intended.

%% Small example program with a locked reference
\begin{figure}[t]
  \centering
  \subcaptionbox{
    Program with locked reference.
    \label{fig:minimod-locked-reference-example-a}
  }[0.33\textwidth]{
    \begin{lstlisting}[
  language=MiniMod
]
mod `\id{A}` {
  var `\id{x}` = `\rholec[5]{\id{y}}`
  var `\id{y}` = 1
}
\end{lstlisting}
\hspace{1em}
  }%
  \hfill
  \subcaptionbox{
    Program with hole.
    \label{fig:minimod-locked-reference-example-b}
  }[0.22\textwidth]{
    \begin{lstlisting}[
  language=MiniMod
]
mod `\id{A}` {
  var `\id{x}` = `\UV{E}`
  var `\id{y}` = 1
}
\end{lstlisting}
\hspace{1em}
  }%
  \hfill
  \subcaptionbox{
    Scope graph.
    \label{fig:minimod-locked-reference-example-c}
  }{
    \begin{minipage}[r]{0.06\textwidth}
      $\mathcal{G} =$
    \end{minipage}%
    \begin{minipage}[l]{0.28\textwidth}
      \vspace{1em}

\begin{tikzpicture}[scopegraph,node distance=2em and 2.5em]
  \node[scope] (s_0) {$s_0$};
  \node[scope,right=2.75em of s_0] (s_A) {$s_{\id{A}} \mapsto \id{A}$};
  \node[scope,below left=2.5em and -1em of s_A] (s_x) {$s_{\id{x}} \mapsto \id{x} : \UV{T}$};
  \node[scope,below right=2.5em and -1em of s_A] (s_y) {$s_{\id{y}} \mapsto \id{y} : \tyINT$};

  \draw (s_0) edge[lbl=$\lblMOD$] (s_A);
  \draw (s_A) edge[lbl=$\lblVAR$] (s_x);
  \draw (s_A) edge[lbl=$\lblVAR$] (s_y);
  \draw (s_A) edge[lbl=$\lblLEX$, bend right] (s_0);
\end{tikzpicture}%
    \end{minipage}%
  }
  %
  \caption{Small example program and its scope graph\hspace{0.02\textwidth}}%
  \label{fig:minimod-locked-reference-example}
\end{figure}

\subsection{Initial Constraint Solving}
Consider the example in~\cref{fig:minimod-locked-reference-example-a}, where for the locked reference $\rhole[4]{\id{y}}$ we want to synthesize a concrete reference that must resolve to variable $\id{y}$'s declaration scope in the underlying scope graph.
Valid concrete references that our approach could yield include $\id{y}$ and $\id{A.y}$.

The first step of our approach is to replace each locked reference in the program by a hole, represented by a fresh unification variable, shown in~\cref{fig:minimod-locked-reference-example-b}.
Then we use the original language's static semantic rules and run the Statix solver on the input program.
The presence of holes causes the solver to yield a stuck state, where the solver neither has enough information to solve all constraints nor can derive $\mathsf{false}$, due to the free unification variables.

In our example, the Statix solver will recursively expand predicates and solve scope graph constraints to yield a solver state, shown below, with the inferred scope graph $\mathcal{G}$ (see~\cref{fig:minimod-locked-reference-example-c}) and a single constraint that is stuck because Statix cannot infer which rule to apply to expand the predicate.
Additionally, in the state we record that the unification variable $\UV{E}$ is associated with hole~$h$, and that hole~$h$ should become a concrete reference that resolves to the scope $s_{\id{y}}$, which is the scope associated with the locked reference's target declaration $\id{y}$ in $\mathcal{G}$.

\begin{figure}[H]
  \begin{HugeAngles}
    \begin{minipage}[c]{0.03\textwidth}
      \hyperref[fig:minimod-locked-reference-example-c]{$\mathcal{G}$}
    \end{minipage}
    \vline
    \begin{minipage}[c]{0.49\textwidth}
      \begin{gather*}
        \cUser{typeOfExpr}(s_{\id{A}}, \hl{$\UV{E}$}, \UV{T})
      \end{gather*}
    \end{minipage}
    \vline
    \begin{minipage}[c]{0.12\textwidth}
      \begin{gather*}
        \ \hl{$\UV{E} \mapsto h$}
      \end{gather*}
    \end{minipage}
    \vline
    \begin{minipage}[c]{0.26\textwidth}
      \begin{gather*}
        \hl{$ h \mapsto (s_{\id{y}}, \UV{E})$}
      \end{gather*}
    \end{minipage}
  \end{HugeAngles}
\end{figure}

\noindent
The state has the form \smash{$\opsRSState{\mathcal{G}}{\overline{C}}{U}{H}$}.
Following \citet{RouvoetAPKV20}, the solver state is given by the (partially constructed) scope graph $\mathcal{G}$ and the set of yet unsolved constraints $\overline{C}$.
To support reference synthesis, we have augmented the solver state with $U$ and $H$.
$U$ is a partial function that maps free unification variables to hole identifiers where applicable.
$H$ maps each hole identifier in a program to a pair $(s_t, t)$ of the current target scope $s_t$ and the term $t$ synthesized for the hole so far.
Note that $s_t$ changes as we synthesize qualifiers for the concrete reference.


\subsection{Forking States}
By default, the Statix solver only expands a constraint when there is exactly one possible expansion, and otherwise the constraint gets stuck.
To support reference synthesis we augment the solver to allow solver states to be \emph{forked}.
This way we can obtain the solver state for each possible expansion of a constraint.
This way we can speculatively apply each possible expansion of a constraint, obtaining a solver state.
% This way we can speculatively apply all possible expansions of a constraint and obtain solver states for each possible expansion.
Forked solver states that fail are discarded, but those that can be successfully solved represent programs for which we have synthesized a valid reference.

For the stuck state discussed above, we can speculatively expand the stuck $\cUser{typeOfExpr}$ predicate constraint and fork the state for each of the $\cUser{typeOfExpr}$ rules shown in~\cref{fig:alg-minimod-rules}, yielding different states.
However, some of those rules will not lead to well-formed references, and therefore we only expand to rules that could yield a reference.
For the simple LM language shown in \cref{fig:alg-minimod-rules}, the relevant rules are \ref{eq:t-var} and \ref{eq:t-qref}.
We fork the solver state, and discuss how each of these two rules leads to a synthesized reference.


\subsection{Expanding Query Constraints}
Applying the \rref{eq:t-var} and solving the constraints as far as possible yields the first forked solver state with one stuck query constraint:

\begin{figure}[H]
  \begin{HugeAngles}
    \begin{minipage}[c]{0.03\textwidth}
      \hyperref[fig:minimod-locked-reference-example-c]{$\mathcal{G}$}
    \end{minipage}
    \vline
    \begin{minipage}[c]{0.49\textwidth}
      \begin{gather*}
        \hl{$
          \vphantom{()}  % To make \hl higlight the line correctly
          \cQuery[\lblOrdLexVAR]{\svar{s}_{\id{A}}}{
            \reclos{\lblLEX}\reopt{\lblIMPORT}\lblVAR
          }{
            \hoPred{isVar}{\UV{x}}
          }{
            \set{\tuple{\_, \UV{x} : \UV{T}}}
          }
      $}
      \end{gather*}
    \end{minipage}
    \vline
    \begin{minipage}[c]{0.12\textwidth}
      \begin{gather*}
        \ \UV{x} \mapsto h
      \end{gather*}
    \end{minipage}
    \vline
    \begin{minipage}[c]{0.26\textwidth}
      \begin{gather*}
        \ h \mapsto \big( s_{\id{y}}, \hl{$\UV{x}$} \big)
      \end{gather*}
    \end{minipage}
  \end{HugeAngles}
\end{figure}

\noindent
As the free variable $\UV{x}$ in the stuck constraint is related to hole $h$, we can infer that the query is also related to hole $h$: attempting to solve the constraint could be fruitful for synthesizing a reference to the target scope $s_{\id{y}}$.
Therefore, reference synthesis searches for valid scope graph paths from $s_{\id{A}}$ to the intended target the scope $s_{\id{y}}$, while respecting the reachability regex and visibility ordering of the query.
There is exactly one such path, namely the one-step path traversing the $\lblVAR$ edge from $s_{\id{A}}$ to $s_{\id{y}}$.
This implies that $\UV{x} = \id{y}$ and $\UV{T} = \tyINT$.
This solves all constraints and results in the mapping $h \mapsto (s_{\id{y}}, \id{y})$.
As all constraints have been solved and the term for the hole is ground, the unqualified reference $\id{y}$ is returned as a solution.



\subsection{Qualified Reference}
In the second forked solver state, applying the \rref{eq:t-qref} and solving constraints yields the following solver state with a stuck predicate constraint and a stuck query constraint, where the term for the hole is representing a possible \emph{qualified} reference $\UV{a}\id{.}\UV{x}$:

\begin{figure}[H]
  \begin{HugeAngles}
    \begin{minipage}[c]{0.03\textwidth}
      \hyperref[fig:minimod-locked-reference-example-c]{$\mathcal{G}$}
    \end{minipage}
    \vline
    \begin{minipage}[c]{0.49\textwidth}
      \begin{gather*}
        \hl{$
          \cUser{scopeOfMod}(\svar{s}_{\id{A}}, \UV{a}, \UV{s_m})
        $}
        \\
        \hl{$
          \vphantom{()}  % To make \hl higlight the line correctly
          \cQuery{\UV{s_m}}{
            \lblVAR
          }{
            \hoPred{isVar}{\UV{x}}
          }{
            \set{\tuple{\_, \UV{x} : \UV{T}}}
          }
      $}
      \end{gather*}
    \end{minipage}
    \vline
    \begin{minipage}[c]{0.12\textwidth}
      \begin{gather*}
        \ \UV{a} \mapsto h
        \\
        \ \UV{x} \mapsto h
      \end{gather*}
    \end{minipage}
    \vline
    \begin{minipage}[c]{0.26\textwidth}
      \begin{gather*}
        \ h \mapsto \big( s_{\id{y}}, \hl{$\UV{a}\id{.}\UV{x}$} \big)
      \end{gather*}
    \end{minipage}
  \end{HugeAngles}
\end{figure}

\noindent
Next, we expand the $\cUser{scopeOfMod}$ predicate.
One of the possible expansions (using~\rref{eq:s-mod}) yields the following state:

\begin{figure}[H]
  \begin{HugeAngles}
    \begin{minipage}[c]{0.03\textwidth}
      \hyperref[fig:minimod-locked-reference-example-c]{$\mathcal{G}$}
    \end{minipage}
    \vline
    \begin{minipage}[c]{0.465\textwidth}
      \begin{gather}
        \hl{$
          \vphantom{()}  % To make \hl higlight the line correctly
          \cQuery[\lblOrdLexMOD]{\svar{s}_{\id{A}}}{
            \reclos{\lblLEX}\lblMOD
          }{
            \hoPred{isMod}{\UV{a}}
          }{
            \set{\tuple{\UV{p}, \UV{a}}}
          }
        $}
        \\ 
        \hl{$
          \UV{s_m} \cEq \mathsf{tgt}(\UV{p})
        $}
        \\
        \cQuery{\UV{s_m}}{
          \lblVAR
        }{
          \hoPred{isVar}{\UV{x}}
        }{
          \set{\tuple{\_, \UV{x} : \UV{T}}}
        }
      \end{gather}
    \end{minipage}%
    \quad  %% Space between gather env and vline
    \vline
    \begin{minipage}[c]{0.12\textwidth}
      \begin{gather*}
        \ \UV{a} \mapsto h
        \\
        \ \UV{x} \mapsto h
      \end{gather*}
    \end{minipage}
    \vline
    \begin{minipage}[c]{0.26\textwidth}
      \begin{gather*}
        h \mapsto \big( s_{\id{y}}, \UV{a}\id{.}\UV{x} \big)
      \end{gather*}
    \end{minipage}
  \end{HugeAngles}
\end{figure}

\noindent
The solver state has two stuck query constraints: the module query (1) and the variable query~(3).
Both query constraints have unification variables that relate to the hole $h$, so we cannot know which of these query should resolve to the target scope $s_{\id{y}}$.
Therefore, we fork the solver state again: one branch where we attempt to expand query (1) and one where we attempt to expand query (3).
Only the fork that attempts to expand the variable query (3) to the target $s_{\id{y}}$ will succeed,
so in this example we continue with that branch.

Because of the mapping~$\UV{x} \mapsto h$, query (3) may be relevant for resolving to the target scope~$s_{\id{y}}$.
Thus, we inspect the scope graph and search for well-formed paths to~$s_{\id{y}}$.
Since the query in question has the unification variable~$\UV{s_m}$ as its source, we need to look for paths with \emph{any} possible source.
For the graph~\cref{fig:minimod-locked-reference-example-c} in the solver state, only the one-step path from $s_{\id{A}}$ to $s_{\id{y}}$ matches the regular expression of the query.
Hence, the query is resolved by substituting $s_{\id{A}}$ for the scope variable $\UV{s_m}$.
Next, we make $s_{\id{A}}$ the new target for the hole $h$, since we assume that the source scope was not ground because the query forms part of a qualified reference; \ie, a sequence of paths.
Eventually, we solve the constraint by substituting $\UV{s_m} \mapsto s_{\id{A}}$ and $\UV{x} \mapsto \id{y}$.


\begin{figure}[H]
  \begin{HugeAngles}
    \begin{minipage}[c]{0.03\textwidth}
      \hyperref[fig:minimod-locked-reference-example-c]{$\mathcal{G}$}
    \end{minipage}
    \vline
    \begin{minipage}[c]{0.49\textwidth}
      \begin{gather*}
        \cQuery[\lblOrdLexMOD]{\svar{s}_{\id{A}}}{
          \reclos{\lblLEX}\lblMOD
        }{
          \hoPred{isMod}{\UV{a}}
        }{
          \set{\tuple{\UV{p}, \UV{a}}}
        }
        \\ 
        \hl{$s_{\id{A}}$} \cEq \mathsf{tgt}(\UV{p})
      \end{gather*}
    \end{minipage}
    \vline
    \begin{minipage}[c]{0.12\textwidth}
      \begin{gather*}
        \ \UV{a} \mapsto h
      \end{gather*}
    \end{minipage}
    \vline
    \begin{minipage}[c]{0.26\textwidth}
      \begin{gather*}
        \ h \mapsto \big( \hl{$s_{\id{A}}$} \cdot s_{\id{y}}, \UV{a}\id{.}\mkern-4mu\hl{$\mkern-4mu\id{y}\mkern-4mu$} \big)
      \end{gather*}
    \end{minipage}
  \end{HugeAngles}
\end{figure}

\noindent
In addition to refining the hole term to $\UV{a}\id{.}\id{y}$, the target scope was also refined to $s_{\id{A}}$.
The new problem to be solved is to find the qualifier that resolves to $s_{\id{A}}$.
Using the same principles as illustrated above, the remaining stuck query can be expanded.
This will solve the remaining constraints and make the hole term ground, yielding $\id{A.y}$ as the solution.


\begin{figure}[H]
  \begin{HugeAngles}
    \begin{minipage}[c]{0.03\textwidth}
      \hyperref[fig:minimod-locked-reference-example-c]{$\mathcal{G}$}
    \end{minipage}
    \vline
    \begin{minipage}[c]{0.49\textwidth}
      \centering
      $\emptyset$
    \end{minipage}
    \vline
    \begin{minipage}[c]{0.12\textwidth}
      \centering
      $\emptyset$
    \end{minipage}
    \vline
    \begin{minipage}[c]{0.26\textwidth}
      \begin{gather*}
        \ h \mapsto \big( \hl{$s_{\id{A}}$} \cdot s_{\id{A}} \cdot s_{\id{y}}, \hl{$\mkern-4mu\id{A}\mkern-4mu$}\mkern-4mu\id{.}\id{y} \big)
      \end{gather*}
    \end{minipage}
  \end{HugeAngles}
\end{figure}

\noindent
Following to our definition of a reference, the solution $\id{A.y}$ is a composite path in the scope graph from the source scope $s_{\id{A}}$,
{\color{colorblind-bright-3}$s_{\id{A}}$} \tikz[scopegraph]{\draw (0,0) edge[ref, dashdotted, color=colorblind-bright-3] (0.5,0);} {\color{colorblind-bright-3}$s_0$} \tikz[scopegraph]{\draw (0,0) edge[ref, dashdotted, color=colorblind-bright-3] (0.5,0);} {\color{colorblind-bright-3}$s_{\id{A}}$} to the scope of qualifier $\id{A}$,
followed by {\color{colorblind-bright-1}$s_{\id{A}}$} \tikz[scopegraph]{\draw (0,0) edge[ref, color=colorblind-bright-1] (0.5,0);} {\color{colorblind-bright-1}$s_{\id{y}}$} to the scope of the intended target declaration $\id{y}$, as shown here:


\begin{figure}[H]
  \vspace{-0.5\baselineskip} % Latex inserts empty line here
  \begin{tikzpicture}[scopegraph,node distance=2em and 2em]
    \node[scope] (s_0) {$s_0$};
    \node[scope,right= 2.5em of s_0] (s_A) {$s_{\id{A}} \mapsto \id{A}$};
    \node[scope,above right= -0.5em and 2.5em of s_A] (s_x) {$s_{\id{x}} \mapsto \id{x} : \UV{T}$};
    \node[scope,below right= -0.5em and 2.5em of s_A] (s_y) {$s_{\id{y}} \mapsto \id{y} : \tyINT$};

    \draw (s_0) edge[lbl=$\lblMOD$] (s_A);
    \draw (s_A) edge[lbl=$\lblVAR$] (s_x.west);
    \draw (s_A) edge[lbl=$\lblVAR$] (s_y.west);
    \draw (s_A) edge[lbl=$\lblLEX$, bend right] (s_0);

    \draw (s_A) edge[ref, dashdotted, bend right=50,color=colorblind-bright-3] (s_0);
    \draw (s_0) edge[ref, dashdotted, bend right,color=colorblind-bright-3] (s_A);
    \draw (s_A) edge[ref, bend right,color=colorblind-bright-1] (s_y);
  \end{tikzpicture}
  \vspace{-0.5\baselineskip} % Latex inserts empty line here
\end{figure}

\noindent
The \hyperref[sec:operational-semantics]{next section} formally defines the approach illustrated above.
In~\cref{sec:heuristics} we describe the heuristics we apply to make the approach usable in practice.
We evaluate our $\mathsf{synthesize}$ function on test programs with locked references in~\cref{sec:evaluation}.
