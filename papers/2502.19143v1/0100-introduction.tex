\pagebreak[4]  % Suggested page break

\section{Introduction}%
\label{sec:introduction}

%% Intro quote
\begin{displayquote}[{\citet[\S 3]{EkmanSV08}}][{.}]
  \textins*{P}reserving bindings is at the heart of any refactoring\\ that moves, creates, or duplicates code
\end{displayquote}

%% Refactorings should preserve behavior, or at least preserve name bindings.
\noindent
As software projects evolve, their code is frequently refactored to improve their structure and maintainability.
Refactoring often involves copying or moving code from one code unit (such as a class, module, or trait) to another, in a way that preserves the program's behavior.
A crucial aspect of behavior-preserving transformations is name binding preservation, to ensure references in refactored code resolve to the same distinct declarations as before.
While behavior preservation also needs control- and data flow analysis, name binding preservation can be achieved using only the static semantic analysis of the program.
However, due to the sophisticated name binding features found in many modern programming languages, preserving the name resolution semantics of code across transformations is generally challenging.

%% Binding preservation is tricky, illustrated with a renaming example in Java.
To illustrate the complexity of reasoning about advanced name binding features, consider the Java program shown in~\cref{fig:java-rename-example-before}.
If we rename the field \Java|x| (line~2) to \Java|y|, Java's static semantics would cause the reference to \Java|y| on line 7 (in method \Java|foo|) to resolve to the newly renamed field \Java|y| on line~2, rather than the intended declaration of \Java|y| on line 5.
This undesired change would alter the name binding structure of the program.
To prevent this, the reference to \Java|y| on line 7 should be \emph{qualified} as \Java|Outer.this.y|, as shown in the refactored example in \cref{fig:java-rename-example-after}.

%% Many refactorings require name binding preservation, and are therefore provided as automated refactorings in IDEs.
Transformations that require name binding preservation are common across many refactorings, such as those from Fowler's catalog~\cite{Fowler99}.
Yet, manually refactoring code is time-consuming and error-prone.
Consequently, many modern Integrated Development Environments (IDEs) provide automated refactorings such as \textsc{Rename}, \textsc{Inline/Extract Method}, and \textsc{Pull Up/Push Down}~\cite{Fowler99}, which attempt to automatically \emph{requalify} references to maintain the program's binding structure.

%% IDEs have refactoring implementations that are unsound. This shows how difficult this is.
However, even popular IDEs for mainstream languages struggle to implement sound refactorings.
For example, \citet{EkmanSV08} identify several bugs in Eclipse~3.4 where automated refactorings inadvertently altered the program's binding structure.
These errors arise from the difficulty of accurately determining which references need to be fixed and computing the correct requalifications.
Not only references in the modified code, but references \emph{throughout the entire code base} may require requalification.
Ensuring both \emph{soundness} (preserve name bindings) and \emph{completeness} (finding all possible requalifications) is particularly difficult.

%% The difficulty of name binding preservation stems from the complexity of the language. We need a sound and principled way to do name binding preservation across languages and transformations.
\Citeauthor{EkmanSV08} conclude that these challenges are ``not related to the core ingredients of the implemented refactoring, \textins*{but} inherent to the complexity of name binding rules in mainstream languages.''
As a result, existing research on the sound requalification of references is often language-specific, focusing on mainstream languages like Java~\cite{SchaferTST12}.
Implementing sound automated refactorings for other languages, like Domain-Specific Languages (DSLs) with small language developer teams, can require a prohibitively high effort.
As such, a more principled and language-parametric approach to guarantee name binding preservation is needed.

%% Java: renaming example
\begin{figure}[t]
  \centering
\begin{subfigure}[t]{0.48\columnwidth}
  \begin{lstlisting}[language=Java]
    class Base {
      int x = 1;
    }
    class Outer {
      int y = 2;
      class Inner extends Base {
        int foo() { return y; }
      }
    }
  \end{lstlisting}%
  \caption{Before renaming.}%
  \label{fig:java-rename-example-before}
\end{subfigure}%
\hspace{1em}%
\begin{subfigure}[t]{0.48\columnwidth}
  \begin{lstlisting}[
      linebackgroundcolor={
        \ifnum
          \the\value{lstnumber}=2\color{codebghl}
        \else \ifnum
          \the\value{lstnumber}=7\color{codebghl}
        \fi \fi
      },
      language=Java
    ]
    class Base {
      int y = 1; // x renamed to y
    }
    class Outer {
      int y = 2;
      class Inner extends Base {
        int foo() { return Outer.this.y; }
      }
    }
  \end{lstlisting}%
  \caption{After renaming.}%
  \label{fig:java-rename-example-after}
\end{subfigure}%
  \caption{
    \textsc{Rename} refactoring of a small Java program, where renaming the field \Java|x| to \Java|y| on line 2 requires the reference \Java|x| on line 7 to be appropriately qualified.
  }%
  \label{fig:java-rename-example}
\end{figure}

%% Java: renaming example, intermediate steps
\begin{figure}[t]
  \centering
\begin{subfigure}[t]{0.48\columnwidth}
  \begin{lstlisting}[
    linebackgroundcolor={
      \ifnum
        \the\value{lstnumber}=7\color{codebghl}
      \fi
    },
    language=Java
  ]
  class Base {
    int `\posidc{x}{1}` = 1;
  }
  class Outer {
    int `\posidc{y}{2}` = 2;
    class Inner extends Base {
      int foo() { return `\rholec[5]{\posid{y}{2}}`; }
    }
  }
  \end{lstlisting}%
  \caption{After locking references, before renaming.}%
  \label{fig:java-rename-example-intermediate-steps-before}
\end{subfigure}%
\hspace{1em}%
\begin{subfigure}[t]{0.48\columnwidth}
  \begin{lstlisting}[
    linebackgroundcolor={
      \ifnum
        \the\value{lstnumber}=2\color{codebghl}
      \fi
    },
    language=Java
  ]
  class Base {
    int `\posidc{y}{1}` = 1; // x renamed to y
  }
  class Outer {
    int `\posidc{y}{2}` = 2;
    class Inner extends Base {
      int foo() { return `\rholec[5]{\posid{y}{2}}`; }
    }
  }
  \end{lstlisting}%
  \caption{After renaming, before unlocking references.}%
  \label{fig:java-rename-example-intermediate-steps-after}
\end{subfigure}%
  \caption{
    Intermediate steps for performing the \textsc{Rename} refactoring from~\cref{fig:java-rename-example} using locked references.
    After locking the relevant reference $\id{y}$ to declaration $\posid{y}{2}$ (\hyperref[fig:java-rename-example-intermediate-steps-before]{a}) and performing the transformation (\hyperref[fig:java-rename-example-intermediate-steps-after]{b}), our approach would synthesize a solution for the locked reference and obtain~\cref{fig:java-rename-example-after}.
  }%
  \label{fig:java-rename-example-intermediate-steps}
\end{figure}

%% What are locked references and how are they part of a refactoring pipeline?
\subsection{Locked References}%
\label{subsec:language-parametric-locked-references}

%% We need a refactoring building block: locked references.
\Citet{EkmanSV08} observe that many bugs in automated refactorings could ``be avoided if a set of carefully crafted building blocks were available to refactoring developers.''
One such building block is \emph{locked references}%
\footnote{
  Terminology introduced by~\citet{SchaferTST12}.
  Also referred to as ``bound names''~\citep{ecoop09refactoring}, ``locked names''~\citep{SchaferMOOPSLA2010}, and ``locked bindings''~\citep{SchaferTST12}.
  We use ``locked references'' throughout this paper.
},
proposed by Sch\"afer et al.\@ in previous work~\cite{SchaferEM08,SchaferMOOPSLA2010}.
A locked reference is an abstract reference that continues to refer to the same unique declaration even if code is moved or the declaration is renamed.
This ensures that transformations cannot cause such a reference to accidentally capture a different declaration.

The following diagram summarizes program transformation with locked references:

\[
  \mathcal{P}
  \xrightarrow{\textsf{lock}}
  \mathcal{P}^{}_{\text{locked}}
  \xrightarrow{\textsf{transform}}
  \mathcal{P}'_{\text{locked}}
  \xrightarrow{\textsf{unlock}}
  \mathcal{P}'
\]%
\vspace{0em}

%% Illustrate lock, transform, unlock pipeline for locked references using the example
\noindent
Before refactoring, we first `$\mathsf{lock}$' each relevant concrete reference by replacing it with a locked reference pointing to the original declaration.
In~\cref{fig:java-rename-example-intermediate-steps-before} we replace the concrete reference $\id{y}$ (line~7) with a locked reference $\rhole[5]{\posid{y}{2}}$ to the declaration $\posid{y}{2}$ on line~5.\footnote{
  Our syntax for locked references $\rhole[5]{d}$ is inspired by the syntax \citet{OmarVHAH17} use for holes.
} (We use subscript indices to distinguish different occurrences of the same name, but the indices are not part of the syntax.)

Next, we `$\mathsf{transform}$' the program as required for the refactoring, renaming declarations and moving code.
In~\cref{fig:java-rename-example-intermediate-steps-after}, the declaration on line 2 is renamed to $\id{y}$.
Finally, we `$\mathsf{unlock}$' each locked reference in the program by replacing it with a \emph{synthesized} concrete reference that unambiguously resolves to the intended declaration.
In this case, unlocking replaces the locked reference with \inlineJava|Outer.this.y|, maintaining the name binding semantics of the program (see~\cref{fig:java-rename-example-after}).

Every step in this pipeline gives rise to challenges, but in this paper we focus on the key challenge of synthesizing concrete references when unlocking locked references.
The program should remain well-typed and synthesized references should resolve to their intended declarations.
Separating name binding preservation from the transformation guarantees that refactorings preserve name bindings, and also makes it easier to implement refactorings.

There are numerous potential applications of reference synthesis.
In the line of work by~\citet{SchaferEM08,SchaferTST12}, it can be applied to implement sound (editor) refactorings.
Furthermore, it provides a powerful transformation tool for implementing sound transformations of DSL programs or performing large-scale codebase transformations aiming to improve the overall code quality.
However, one can also envision \emph{user-extensible refactoring tools} (such as presented by Li and Thompson~\citep{LiT12-19}) or \emph{transformation languages} (such as IntelliJ's structural search and replace) that need to preserve name bindings.
Finally, it could be used to build an editor service that suggests fixes for type errors (\eg, \textsc{quick fix} in Eclipse).



\subsection{Language-Parametric Reference Synthesis}%
\label{subsec:language-parametric-reference-synthesis}

Following \citet{SchaferEM08,SchaferTST12}, a reference synthesis function can be thought of as the right inverse of a reference resolution function.
That is, if $\mathit{QRef}$ is the set of qualified references, $\mathit{Decl}$ is the set of uniquely identified declarations, and the function $\mathsf{resolve}_p : \mathit{QRef} \rightharpoonup \mathit{Decl}$ resolves a reference at some location $p$ in the program, then locked reference synthesis should be a function $\mathsf{synthesize}_p : \mathit{Decl} \rightharpoonup \mathit{QRef}$ such that $\mathsf{resolve}_p(\mathsf{synthesize}_p(d)) = d$ for any $p$.\footnote{\Citeauthor{SchaferEM08} use ``lookup'' instead of ``resolve'' and ``access'' instead of ``synthesize'', but the idea is the same.}
The $\mathsf{lock}$ function uses $\mathsf{resolve}$ to obtain the target declaration when replacing a concrete reference with a locked reference, and conversely, $\mathsf{unlock}$ uses the $\mathsf{synthesize}$ function to replace the locked reference with a concrete reference.

The reference synthesis pipeline shown above is conceptually \emph{language-parametric}.
However, as discussed before, implementing correct reference synthesizers manually is error-prone and time-consuming.
In this paper, we present a \emph{language-parametric} approach to derive the $\mathsf{synthesize}$ function automatically from declarative type system specifications, letting language designers generically synthesize valid concrete references for programs with multiple locked references.
The goal of reference synthesis is to find for each locked reference in a program a valid concrete reference that resolves to the intended declaration.

We derive the $\mathsf{synthesize}$ function from \emph{only} a declarative specification of the language's static semantics, specified using the \emph{Statix} specification language~\cite{AntwerpenPRV18,RouvoetAPKV20}.
Statix allows syntax-directed typing rules to be specified, uses \emph{scope graphs} as a declarative model of static name binding and name resolution~\cite{NeronTVW15}, and generates executable type checkers by interpreting specifications as constraint programs.
In our implementation of $\mathsf{synthesize}$, we reinterpret the \emph{name resolution queries} from the specification to infer syntax for concrete references that resolve to a particular target declaration, guaranteeing name binding preservation.
We reuse the Statix solver (see~\cref{sec:scope-graphs-and-statix}) to validate that the syntax we infer is \emph{sound} with respect to the typing rules and represents a reference that resolves to the intended declaration.

This paper makes the following technical contributions:

\begin{itemize}

  \item
    We present a language-parametric implementation of the $\mathsf{synthesize}$ function (see~\cref{sec:operational-semantics} and~\cref{sec:heuristics}).
    This function automatically synthesizes concrete references that resolve to the specified declaration, and that are sound with respect to a type system specification written in Statix.

  \item 
    We evaluate our implementation on \allTotalTests{} test programs of Java, ChocoPy, and Featherweight Generic Java (\cref{sec:evaluation}).
    Our results demonstrate that our approach applies to mainstream languages with complex name binding semantics without modifying their typing rules.

\end{itemize}

\noindent
We first (\cref{sec:scope-graphs-and-statix}) introduce scope graphs and Statix.
Next, in~\cref{sec:reference-synthesis-by-example} we illustrate our reference synthesis algorithm.
Then, in~\cref{sec:operational-semantics} we give an operational semantics of our $\mathsf{synthesize}$ function's implementation, and the heuristics we apply in~\cref{sec:heuristics}.
We evaluate our implementation in~\cref{sec:evaluation}, and discuss related work in~\cref{sec:related-work}.
We conclude in~\cref{sec:conclusion}.

