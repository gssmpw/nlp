\section{Experimental Evaluation}%
\label{sec:evaluation}

% Goal of the evaluation
In evaluating reference synthesis, we focus on three key criteria: (1) ensuring that synthesized references are valid according to the static semantic specification of the language and resolve to the intended declaration (\emph{soundness}), (2) verify the ability to find a valid reference if it exists (\emph{completeness}), and (3) assess the efficiency of the synthesis process (\emph{performance}).
In this section we discuss how we evaluated these aspects of reference synthesis.

%% TODO: [Camera-ready version]
% % Example evaluation test file
% \begin{wrapfigure}{r}{5.5cm}
%   %% moveClass/conflictingNames_before
\begin{lstlisting}[
  language=Java,
]
test;
[`\posidc{p2}{1}`] {
  [Test]
    package p2;
    public class `\posidc{Test}{2}` {
      public static void `\posidc{f}{3}`() {}
    }
  [First]
    package p2;
    public class `\posidc{First}{4}` {}
}
[p1] {
  [Second]
    package p1;

    import `\rholec[6]{\posidc{p2}{1}}`.*;

    class `\posidc{Second}{5}` {
      {
        new `\rholec[9]{\posidc{First}{6}}`();
        `\rholec[6]{\posidc{f}{3}}`();
      }
    }
  [First]
    package p1;
    public class `\posidc{First}{6}`
      extends `\rholec[9]{\posidc{First}{4}}` {}
}
\end{lstlisting}%
%   \caption{
%     One of the Java test cases used to evaluate our approach.
%     The test case is a single file but comprises multiple independent Java files that are combined using special syntax that our Java parser and semantic analysis can understand.
%   }%
%   \label{fig:java-testcase-example}
% \end{wrapfigure}

% Chosen languages and test sets
\pagebreak[4]
\subsection{Languages}
We evaluated our $\mathsf{synthesize}$ function on three larger programming languages: Java, ChocoPy, and Featherweight Generic Java (FGJ).

\begin{enumerate}
  \item \emph{Java} is a mainstream language with sophisticated name binding features.
  We evaluated our approach using an existing Statix specification of Java 8 from the artifact~\cite{AntwerpenV21-artifact} associated with the work of~\citet{AntwerpenV21}.
  This specification of Java unfortunately does not support generics or method references.
  We derived test cases from Java files used to validate the implementation of refactorings in the JetBrains IntelliJ IDE.\footnote{https://github.com/JetBrains/intellij-community/tree/idea/233.14808.21/java/java-tests/testData/refactoring}
  
  \item \emph{ChocoPy}~\cite{PhadyeSH19-SPLASHE} is a statically-typed dialect of Python.
  We used an existing Statix specification and ChocoPy files for our evaluation.
  
  \item \emph{Featherweight Generic Java}~\cite{IgarashiPW01} (FGJ) is a functional Java core language with full generics support.
  We used the Statix specification from the artifact~\cite{AntwerpenPRV18-artifact} of~\citet{AntwerpenPRV18}.
\end{enumerate}


%% Test selection
\begin{figure}[t]
  \captionbox{
    Test selection.
    \label{fig:test-selection}
  }[0.55\textwidth]{
    \footnotesize
    \begin{tabular}{|l|r|r|r|}
      \hline
                            &                     Java &                     ChocoPy &                    FGJ \\
      \hline
      \hline
      Included              & \javaTotalTests{} (62\%) & \chocopyTotalTests{} (64\%) & \fgjTotalTests{} (94\%) \\
      \hline
      Negative              &                0   (0\%) &                  55  (18\%) &               0   (0\%) \\
      Java 8 incompat.      &             1076  (26\%) &                 N/A         &             N/A         \\
      Spec incompat.        &              197   (5\%) &                  10   (3\%) &               0   (0\%) \\
      No references         &              304   (7\%) &                  47  (15\%) &               3   (6\%) \\
      \hline
      Total                 & \javaOriginalTestSet{} (100\%) & \chocopyOriginalTestSet{} (100\%) & \fgjOriginalTestSet{} (100\%) \\
      \hline
    \end{tabular}%
  }%
  \hfill
  \captionbox{
    Test outcomes.
    \label{fig:test-outcomes}
  }[0.45\textwidth]{
    \footnotesize
    %%%%%%%%%%%%%%%%%%%%%%%%%%%%%%%%%%%%%%%%%%%%%%%%%%%%%%%%%%%%%%%%%%%%%%%%%%%%%%
% This file is generated using `python3 dataviz.py` in `evaluation/dataviz/`
% Based on this result set: 20240929T173103
%%%%%%%%%%%%%%%%%%%%%%%%%%%%%%%%%%%%%%%%%%%%%%%%%%%%%%%%%%%%%%%%%%%%%%%%%%%%%%
\begin{tabular}{|l|r|r|r|}
  \hline
  &  Java  &  ChocoPy  &  FGJ  \\
  \hline
  \hline
  Success & 2382 (94\%) & 155 (79\%) & 38 (78\%) \\
  Timeout & 146 (6\%) & 41 (21\%) & 11 (22\%) \\
  Failure & 0 (0\%) & 0 (0\%) & 0 (0\%) \\
  \hline
  Total & 2528 (100\%) & 196 (100\%) & 49 (100\%) \\
  \hline
\end{tabular}
  }%
\end{figure}

% Method of the evaluation
\subsection{Method}
For each language we used the existing Statix semantic specifications \emph{without modification}, as our $\mathsf{synthesize}$ function works directly with these specifications.
We collected a set of test cases: single-file programs where we locked variable names, type names, and qualified member names that occur in it.
As a result, many test cases contain multiple locked references.
Each locked reference only has the target declaration as a parameter.
The original syntax of the reference is erased.
We excluded negative test cases (tests that validated that the specification or implementation gives an error on incorrect programs), those that are incompatible with our Statix specification or Java 8 (in the case of Java tests), and those that had no references to lock.
The resulting selection is shown in~\cref{fig:test-selection}.
Our reference synthesis algorithm was then applied to propose concrete references until the original reference in the program was recovered.
We set a timeout of 60~seconds per test case and ran the evaluation on a MacBook~Pro~2019 with a 2.4~GHz Intel Core~i9 processor and 64~GB memory.




% Main takeaways
\subsection{Results}%
\label{subsec:results}

A summary of the results of the evaluation is shown in~\cref{fig:test-outcomes}.
The results confirm the \emph{soundness} of our reference synthesis algorithm.
Substituting the synthesized references back into the original programs only produced well-typed programs (\cref{thm:soundness-1}).

Our evaluation demonstrates a high level of \emph{completeness}, as it successfully synthesized each reference encountered in our test cases, including non-trivial references such as Java's \Java|A.super.x| and references with ambiguous qualifiers~\cite[\S{6.5.2}]{JLS8}.
Nevertheless, formally proving \emph{completeness} of the algorithm is challenging (\cref{subsec:operational-semantics-discussion}).
%
% \begin{figure}[t]
%   \begin{center}
%     \includegraphics[width=\textwidth]{fig/0600-timeperhole-violinplot-log.pdf}
%   \end{center}%
%   \caption{
%     Logarithmic plot of the time spent synthesizing references per hole in each of \allSuccessfulTests{} successful test cases.
%     The {\color{red}dashed line} marks 1 second.
%   }%
%   \label{fig:reference-synthesis-time-plot}
% \end{figure}
%

\noindent
\begin{wrapfigure}{r}{7cm}
  \begin{center}
    \includegraphics[width=7cm]{fig/0600-timeperhole-violinplot-log.pdf}
  \end{center}%
  \caption{
    Logarithmic plot of the time spent synthesizing references per hole in each of \allSuccessfulTests{} successful test cases.
    The {\color{red}dashed line} marks 1 second.
  }%
  \label{fig:reference-synthesis-time-plot}
\end{wrapfigure}
%
While \emph{performance} was not the main focus of our approach, we also measured the time taken to synthesize each reference, shown as a violin plot in~\cref{fig:reference-synthesis-time-plot}.
The plot illustrates the distribution of the synthesis time per hole as a density curve, where for all three languages most results lie between 10 and 200 milliseconds.
A small box plot is depicted inside each density curve, highlighting the median, range of the central 50\% results, and range of the remaining data.
For the majority of locked references the algorithm proposed a solution within one second.
In around 7\% of the test cases, reference synthesis failed to find a solution for all locked references within the 60-second timeout.
This typically occurred when references were strongly interdependent, as reference synthesis will exhaustively explore all combinations of solutions for the dependent and the dependency (see~\cref{subsec:isolating-holes}).


% Threats to validity
\subsection{Threats to Validity}
To mitigate bias in the references we lock, or the alternatives we try, we make two conservative assumptions that are not generally true for a refactoring:
(1) \emph{all} references in the program must be locked; and
(2) the original reference syntax is unknown.
These assumptions make our test cases more challenging and may result in a worse performance than it would be in practical scenarios.

Instead of locking all references in the program, a typical refactoring would only need to lock a subset of those references, namely those that could get changed due to the program transformation.
%For example, before moving a function, the refactoring should lock references from the body of the moved function, references \emph{to} the function itself, but also references to other functions with the same name that might be shadowed due to the transformation.
Consequently, the likelihood of interdependent locked references would be lower.
Furthermore, most existing references are likely to remain valid despite the transformation.
Therefore, reference synthesis could prioritize verifying that the existing reference syntax still resolves to the intended declaration before synthesizing a new concrete reference.
As a result, only a handful of references need to be synthesized.

% In contrast, the test cases we use for the evaluation forget the prior syntax of each reference, and lock all references within the program, which can cause more hole dependencies than typical.
% As the test cases are doing more work than this, their timings and timeouts are higher than a typical refactoring would encounter.

Another potential threat to validity is that our approach is parameterized by the language's static semantics written in Statix.
As shown by~\citet{RouvoetAPKV20, AntwerpenPRV18, ZwaanP24-0}, Statix can express many interesting name binding and type system concepts, but it is yet unclear what language concepts Statix cannot express.
The features of Statix that we heavily rely on are the solver interface, the presence of predicate constraints and query constraints, and conservative scheduling.
Given that we do not expect significant changes in any of these, we expect that our reference synthesis algorithm will work without fundamental modifications being required by possible future extensions to Statix.




