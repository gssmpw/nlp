\begin{proof}
    We will prove this theorem by transforming the synthesis trace for $\mathcal{P}$ in a valid solver trace for $\mathcal{P}'$.
    Observe that each valid synthesis can be considered a trace with the following shape:
    \[
        \kappa_1 = \opsRSState
        {\emptyset}
        {P_0(\mathcal{P})}
        {U_1}
        {H_1}
        \rightarrow^{\bullet}
        \kappa_2
        \rightarrowtail
        \kappa_2'
        \twoheadrightarrowtail^{\bullet}
        \kappa_n = \opsRSState
        {\SG_n}
        {\emptyset}
        {U_n}
        {H_n}
    \]
    where $\kappa_2' = \kappa_2\theta_2$, given that $\theta_2$ is the substitution that was computed in the expand step $\kappa_2 \rightarrowtail \kappa_2'$.
    By~\cref{lem:embedding-equality}, we can create a new trace as follows:
    \[
        \kappa_2 \oplus {\color{red} \llbracket \theta_2 \rrbracket}
        \rightarrow^{\bullet}
        \kappa_2'
        \twoheadrightarrowtail^{\bullet}
        \kappa_n
    \]
    By repeating this on all expand steps, we can create the following sequence of traces:
    \begin{alignat*}{1}
        \kappa_1 &\rightarrow^{\bullet} \kappa_2 \\
        \kappa_2 \oplus \llbracket \theta_2 \rrbracket &\rightarrow^{\bullet} \kappa_3 \\
        &\ldots \\
        \kappa_{n-1} \oplus \llbracket \theta_{n-1} \rrbracket &\rightarrow^{\bullet} \kappa_n
    \end{alignat*}
    These traces are all valid solver traces (i.e., each step is valid according to one of the \textsc{Op-*} rules).

    Next, we use $\theta_{\text{result}}$ from $\mathsf{synthesize}$ to eliminate all intermediate equalities $\theta_2 \ldots \theta_{n-1}$.
    Because $\theta_{\text{result}}$ is derived from $U_n$ and $H_n$, $\theta_{\text{result}}$ is redundant in $\kappa_n$ (\ie, $\kappa_n \theta_{\text{result}} = \kappa_n$).
    When prepending the embedding of this unifier to the last solver trace, by~\cref{lem:strengthening-eq}, we construct the following trace:%
    \footnote{
        Note that the variables in $\theta_{\text{result}}$ are free in the original constraint.
        For that reason, assuming the state tracks non-free variables (see the remark at~\cref{lem:strengthening-eq}), we can freely move around this constraint.
    }
    \[
        \kappa_{n-1} \oplus \llbracket \theta_{n-1} \rrbracket \oplus
            {{\color{red} \llbracket \theta_{\text{result}} \rrbracket}}
        \rightarrow^{\bullet}
        \kappa_n
    \]
    Using~\cref{lem:weakening} (removing $\llbracket \theta_{n-1} \rrbracket$), Statix' confluence, and the fact that all variables in $\theta_{\text{result}}$ are free in the original constraint, we infer that $\kappa_{n-1} \oplus
    {{\llbracket \theta_{\text{result}} \rrbracket}}$ either steps to $\kappa_4$, or gets stuck.
    By applying the same reasoning to the previous traces, we can conclude that $\kappa_1 \oplus {\llbracket \theta_{\text{result}} \rrbracket}$ either steps to $\kappa_4$, or gets stuck.

    Using confluence and~\cref{lem:embedding-equality}, together with the fact that ${\llbracket \theta_{\text{result}} \rrbracket}$ can be solved first, we infer the following trace exists:
    \[
        \kappa_1 \oplus {\llbracket \theta_{\text{result}} \rrbracket}
        \rightarrow^{\ast}
        \kappa_1' = \opsRSState
        {\emptyset}
        {\overline{C}_1\theta_{\text{result}}}
        {U_1'}
        {H_1'}
        \rightarrow^{\bullet}
        \kappa_n'
    \]
    where $\kappa_n' = \kappa_n$ or some stuck state.
    Note that $\overline{C}_1\theta_{\text{result}} = {P_0(\mathcal{P}')}$,
    and hence the intermediate state~$\kappa_1'$ is exactly the initial state for solving $P_0(\mathcal{P}')$.

    Finally, by our assumption that the initial constraint cannot get stuck on a ground program, $\kappa_n'$ cannot be stuck.
    Hence $\kappa_n' = \kappa_n$, which is a successful state.
    For that reason, we conclude that $\mathcal{P}'$ is well-typed.

\end{proof}