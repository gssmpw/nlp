
%%% Ensure each section starts on its own page (for debugging the paper)
% \usepackage{etoolbox}
% \preto{\section}{\clearpageafterfirst}
% \newcommand{\clearpageafterfirst}{%
%   \gdef\clearpageafterfirst{\clearpage}%
% }


%%% Figures
\usepackage{adjustbox}
\usepackage{caption}
\usepackage{subcaption}
\usepackage{graphicx}
\usepackage{wrapfig}

%%% Tables
\usepackage{tabularx}

%%% References
\usepackage{cleveref}
\newcommand{\rref}[1]{rule~\ref{#1}}
\newcommand{\Rref}[1]{Rule~\ref{#1}}

\crefformat{section}{\S#2#1#3}
\crefmultiformat{section}{\S\S#2#1#3}{ and~#2#1#3}{, #2#1#3}{, and~#2#1#3}

\crefformat{subsection}{\S#2#1#3}
\crefmultiformat{subsection}{\S\S#2#1#3}{ and~#2#1#3}{, #2#1#3}{, and~#2#1#3}

\crefformat{figure}{Fig.\@~#2#1#3}
\crefmultiformat{figure}{Figs.\@~#2#1#3}{ and~#2#1#3}{, #2#1#3}{, and~#2#1#3}

%% Prevent page breaks in the middle of bibliography entries (https://tex.stackexchange.com/a/92690/108670)
\patchcmd{\thebibliography}{\clubpenalty4000}{\clubpenalty10000}{}{}
\patchcmd{\thebibliography}{\widowpenalty4000}{\clubpenalty10000}{}{}

%%% Quotes
\usepackage{csquotes}

%%% Notation
%\usepackage{amsmath}  % Already included in the acmart package
\usepackage{mathtools}
%\usepackage[warnings-off={mathtools-colon,mathtools-overbracket}]{unicode-math}  % Hide warnings about unicode-math overriding mathtools commands
%\usepackage{newcomputermodern} % Loads unicode-math, for the ⦇ and ⦈ symbols
\usepackage{stmaryrd} % For the ⦇ and ⦈ symbols

\usepackage{mathpartir} % For \inferrule
\mprset{vskip=0.2ex} % E.g. Op-Expand has toching premises if reduced.

%% Proof Sketch environment
\newenvironment{proofsketch}{%
\renewcommand{\proofname}{Proof Sketch}\proof
}{\endproof}

% Acronyms
\usepackage[nolist,nohyperlinks]{acronym}
\acrodef{API}{Application Programming Interface}  \acused{API}

% A two-headed right arrow
\newcommand{\xtwoheadrightarrow}[2][]{%
  \xrightarrow[#1]{#2}\mathrel{\mkern-14mu}\rightarrow{}
}

\newcommand{\xxrightarrow}[1]{
  \xrightarrow{\raisebox{-1.35pt}[0pt][0pt]{$\scriptstyle #1$}}
}

\newcommand{\twoheadrightarrowtail}{
  \rightarrowtail\mathrel{\mkern-14mu}\rightarrow
}

% Double square brackets \lBrack and \rBrack
\DeclarePairedDelimiter\Brackets{\lBrack}{\rBrack}

% A short right arrow
\newcommand{\veryshortrightarrow}[1][3pt]{\mathrel{%
  \hbox{\rule[\dimexpr\fontdimen22\textfont2-.2pt\relax]{#1}{.4pt}}%
  \mkern-4mu\hbox{\usefont{U}{lasy}{m}{n}\symbol{41}}}
}

% Compact cdots
\DeclareRobustCommand
  \ccdots{\mathinner{\cdotp\mkern-2mu\cdotp\mkern-2mu\cdotp}}


%%% Macros
\usepackage{xspace}
\newcommand*{\eg}{e.g.\@\xspace}
\newcommand*{\ie}{i.e.\@\xspace}
\newcommand*{\Eg}{E.g.\@\xspace}
\newcommand*{\Ie}{I.e.\@\xspace}
\newcommand{\code}[1]{\texttt{#1}}
\newcommand{\CSharp}{C\#}



%%% Code
\usepackage{calc}
\usepackage{relsize}
\renewcommand\RSsmallest{4.5pt}  % Removes warnings: "package relsize: Font size 4.61pt is too small. Using 5.0pt instead."
\newcommand{\posc}[1]{_{\makebox[\widthof{\texttt{x}}]{%
  \smaller[3]%
  \StrLen{#1}[\mystringlen]%
  \ifthenelse{\mystringlen < 2}{}{\hspace{0.4em}}%
  \ensuremath{#1}%
}}\ifmmode\else\@\fi}
\newcommand{\id}[1]{\texttt{#1}}
\newcommand{\posid}[2]{\id{#1}_{#2}}
\newcommand{\posidc}[2]{\id{#1}\ensuremath{\posc{#2}}} % enforce fixed width of one character
\newcommand{\qhole}[1]{\ensuremath{\mathcolor{violet}{{\scalebox{0.9}{$\llparenthesis$}}} #1 \mathcolor{violet}{{\scalebox{1}{$\rrparenthesis$}}}}}
% Make a monospace aligned invisible box. Usage in code: `\qholec[charwidth]{body}`
\newcommand{\qholec}[2][1]{\makebox[#1\fontcharwd\font`x]{\qhole{#2}}} % box with #1 times monospace character width
% Usage: `\qholecx[charwidth]{body}{subscript}'
\newcommand{\qholecx}[4][1]{\qholec[#1]{#2}\ensuremath{^{#3}_{#4}}}
% Hole for a reference in running text (has a normal font). Usage: `\rhole[charwidth]{body}'
\newcommand{\rhole}[2][1]{\qholec[#1]{\small{\rightarrow}#2}}
% Hole for a reference in a code listing (has a slightly smaller font). Usage: `\rholec[charwidth]{body}'
\newcommand{\rholec}[2][1]{\qholec[#1]{\tiny{\rightarrow}\scriptsize{#2}}}
%\the\numexpr #1 + 3 \relax
\newcommand{\with}{\mapsto}
\newcommand{\ty}[1]{\mathbf{#1}}
\newcommand{\tyINT}{\ty{int}}
\newcommand{\tyFLOAT}{\ty{float}}
\newcommand{\tyCLS}[1]{\ty{cls}(#1)}
\newcommand{\tyREC}[1]{\ty{rec}(#1)}
\newcommand{\tyFUN}[2]{#1 \to #2} % Function with one argument

\newcommand{\UV}[1]{\text{?$#1$}}     % Unification variable


%%% Code Listings
\usepackage{listings}
% \usepackage{listings-rust}
\usepackage{lstautogobble}

% Ugly workaround from: https://tex.stackexchange.com/questions/451532/recent-issues-with-lstlinebgrd-package-with-listings-after-the-latters-updates
\makeatletter
\let\old@lstKV@SwitchCases\lstKV@SwitchCases
\def\lstKV@SwitchCases#1#2#3{}
\makeatother

\usepackage{lstlinebgrd}

\makeatletter
\let\lstKV@SwitchCases\old@lstKV@SwitchCases

\lst@Key{numbers}{none}{%
    \def\lst@PlaceNumber{\lst@linebgrd}%
    \lstKV@SwitchCases{#1}%
    {none:\\%
     left:\def\lst@PlaceNumber{\llap{\normalfont
                \lst@numberstyle{\thelstnumber}\kern\lst@numbersep}\lst@linebgrd}\\%
     right:\def\lst@PlaceNumber{\rlap{\normalfont
                \kern\linewidth \kern\lst@numbersep
                \lst@numberstyle{\thelstnumber}}\lst@linebgrd}%
    }{\PackageError{Listings}{Numbers #1 unknown}\@ehc}}
\makeatother
% end ugly workaround




%%% Colors
\usepackage{tolcolors}
% From: https://tex.stackexchange.com/a/261480/108670
\makeatletter
\def\mathcolor#1#{\@mathcolor{#1}}
\def\@mathcolor#1#2#3{%
  \protect\leavevmode
  \begingroup
    \color#1{#2}#3%
  \endgroup
}
\makeatother

\definecolor{codekeywords}{rgb}{0.2,0.2,0.5}
\definecolor{codegreen}{rgb}{0.0,0.4,0.2}
\definecolor{codegray}{rgb}{0.5,0.5,0.5}
\definecolor{codepurple}{rgb}{0.58,0,0.82}
\colorlet{codebghl}{lightgray!70!white}

\def\colorM{olive}
\def\colorA{olive}
\def\colorR{green!60!black}
\def\colorx{olive!60!black}

\usepackage[most]{tcolorbox}
\tcbset{
  on line, 
  boxsep=3pt, left=0pt,right=0pt,top=0pt,bottom=0pt,
  colframe=white,colback=codebghl,  
  highlight math style={enhanced}
}
% \newcommand{\hl}[1]{%
%   \begingroup
%   \setlength{\fboxsep}{0pt}%  
%   \colorbox{gray!25}{#1}%
%   \endgroup
% }
% \hl{} but with padding
% \DeclareRobustCommand{\hl}[1]{\colorbox{gray!25}{#1}}
\DeclareRobustCommand{\hl}[1]{\tcbox{#1}}


%%% Huge Angled Brackets
\NewEnviron{HugeAngles}{%
  \vspace{-0.2\baselineskip} % Latex inserts a skip here, remove it
  \tikz[baseline=0pt]
    \draw[line width=1pt] 
     node[append after command={
      (\tikzlastnode.north west) -- ($(\tikzlastnode.west)+.03*(\tikzlastnode.west)!1!90:(\tikzlastnode.north west)$) -- (\tikzlastnode.south west)
      (\tikzlastnode.north east) -- ($(\tikzlastnode.east)+.03*(\tikzlastnode.east)!1!270:(\tikzlastnode.north east)$) -- (\tikzlastnode.south east)
      }] { \BODY };
 \vspace{-0.2\baselineskip} % Latex inserts a skip here, remove it
}%
{}


%% Reset equation counter every time
\preto\equation{\setcounter{equation}{0}}
\makeatletter
\pretocmd\start@gather{\setcounter{equation}{0}}{}{}
% \pretocmd\start@align{\setcounter{equation}{0}}{}{}
% \pretocmd\start@multline{\setcounter{equation}{0}}{}{}
\makeatother


%%% Placeholder symbol
\newcommand{\plhdr}{\mathord{\color{black!33}\bullet}}%

%%% Listings
\lstdefinestyle{mystyle}{
    backgroundcolor=\color{white},
    commentstyle=\color{codegreen},
    keywordstyle=\bfseries\color{codekeywords},
    numberstyle=\tiny\color{codegray},
    stringstyle=\color{codepurple},
    basicstyle=%
      \ttfamily%
      \mdseries%
      \lst@ifdisplaystyle\scriptsize\fi,  % Conditional \scriptsize, if it is not inline
    % breakatwhitespace=false,
    % breaklines=true,
    % captionpos=b,
    % keepspaces=true,
    numbers=left,      % we regularly point to line numbers, so make those resolvable
    xleftmargin=2em,   % prevent line numbers overflowing margins
    % numbersep=5pt,
    % showspaces=false,
    % showstringspaces=false,
    % showtabs=false,
    % tabsize=2
}

\lstdefinelanguage{MiniMod}{
    keywords={mod, include, import, val, var, lam, null},
    sensitive=true,           % keywords are case-sensitive
    comment=[l]{//},          % single-line comments
    comment=[s]{/*}{*/},      % multi-line comments
    string=[b]"               % strings
}%
\newcommand{\MiniMod}{\lstinline[language=MiniMod]}
\newcommand{\Java}{\lstinline[language=Java]}
\newcommand{\inlineJava}{\lstinline[language=Java]}

\lstset{
  style=mystyle,
  language=Java,
  autogobble=true,
  escapeinside={`}{`},
  mathescape
}

% \lstdefinelanguage{Kotlin}{
%   comment=[l]{//},
%   commentstyle={\color{gray}\ttfamily},
%   emph={filter, first, firstOrNull, forEach, lazy, map, mapNotNull, println},
%   emphstyle={\color{OrangeRed}},
%   identifierstyle=\color{black},
%   keywords={!in, !is, abstract, actual, annotation, as, as?, break, by, catch, class, companion, const, constructor, continue, crossinline, data, delegate, do, dynamic, else, enum, expect, external, false, field, file, final, finally, for, fun, get, if, import, in, infix, init, inline, inner, interface, internal, is, lateinit, noinline, null, object, open, operator, out, override, package, param, private, property, protected, public, receiveris, reified, return, return@, sealed, set, setparam, super, suspend, tailrec, this, throw, true, try, typealias, typeof, val, var, vararg, when, where, while},
%   keywordstyle={\color{NavyBlue}\bfseries},
%   morecomment=[s]{/*}{*/},
%   morestring=[b]",
%   morestring=[s]{"""*}{*"""},
%   ndkeywords={@Deprecated, @JvmField, @JvmName, @JvmOverloads, @JvmStatic, @JvmSynthetic, Array, Byte, Double, Float, Int, Integer, Iterable, Long, Runnable, Short, String, Any, Unit, Nothing},
%   ndkeywordstyle={\color{BurntOrange}\bfseries},
%   sensitive=true,
%   stringstyle={\color{ForestGreen}\ttfamily},
% }



%%% Pseudo Code Algorithms

% \usepackage{algorithm}
\usepackage[noend]{algpseudocode} % also loads package: algorithmicx
\algrenewcommand\algorithmicindent{1.0em}%


\newcommand{\LoopBreak}{\textbf{break}}
% Line comment (left-aligned)
\algnewcommand{\LeftComment}[1]{\(\triangleright\) #1}
% Long line comment
\makeatletter
\newlength{\trianglerightwidth}
\settowidth{\trianglerightwidth}{$\triangleright$~}
\algnewcommand{\LineComment}[1]{\Statex \hskip\ALG@thistlm $\triangleright$ #1}
\algnewcommand{\LineCommentCont}[1]{\Statex \hskip\ALG@thistlm%
  \parbox[t]{\dimexpr\linewidth-\ALG@thistlm}{\hangindent=\trianglerightwidth \hangafter=1 \strut$\triangleright$ #1\strut}}
\makeatother

% Defines \Indent and \EndIndent for use in pseudo code
\algdef{SE}[SUBALG]{Indent}{EndIndent}{}{\algorithmicend\ }%
\algtext*{Indent}
\algtext*{EndIndent}

% Continue in for-loops
\newcommand{\algorithmiccontinue}{\textbf{continue}}
\newcommand{\Continue}{\State \algorithmiccontinue}

% Accept
\newcommand{\algorithmicaccept}{\textbf{accept}}
\newcommand{\Accept}{\State \algorithmicaccept}

% Where
\algnewcommand{\Where}{\textbf{where}}

% function -> fun
\algrenewcommand\algorithmicfunction{\textbf{fun}}

% FnName{#1}
\algnewcommand{\FnName}[1]{\textsc{#1}}
% FnType{#1}
\algnewcommand{\FnType}[1]{\textsc{#1}}

\makeatletter
\algnewcommand{\SAlgText}[1]{%
  \parbox[t]{\dimexpr\linewidth-\ALG@thistlm}{\strut#1\strut}}
\makeatother

\makeatletter
\algnewcommand{\AlgText}[1]{\Statex \hskip\ALG@thistlm%
  \parbox[t]{\dimexpr\linewidth-\ALG@thistlm}{\strut#1\strut}}
\makeatother

% <+
\newcommand{\lc}{\mathrel{<\!\!\!+}}
% <+>
\newcommand{\fork}{\mathrel{<\!\!\!+\!\!\!>}}

% Struct block
\algdef{SE}[STRUCT]{Struct}{EndStruct}[1]{\textbf{struct} \textsc{#1}}{\textbf{end struct}}

% Loop strategy block
% \algnewcommand\algorithmicsloop{\textbf{loop}}
% \algblockdefx[SLOOP]{SLoop}{EndSLoop}[1]
%   {\algorithmicsloop\ \textsc{#1}}
%   {\algorithmicend}
% \makeatletter
% \ifthenelse{\equal{\ALG@noend}{t}}{\algtext*{EndSLoop}}{}% Don't print if 'noend' is set
% \makeatother

% While strategy block
\algblockdefx[SWHILE]{SWhile}{EndSWhile}[1]
  {\algorithmicwhile\ \textsc{#1}}
  {\algorithmicend}
\makeatletter
\ifthenelse{\equal{\ALG@noend}{t}}{\algtext*{EndSWhile}}{}% Don't print if 'noend' is set
\makeatother

% Repeat strategy block
\algblockdefx[SREPEAT]{SRepeat}{EndSRepeat}[1]
  {\algorithmicrepeat\ \textsc{#1}}
  {\algorithmicend}
\makeatletter
\ifthenelse{\equal{\ALG@noend}{t}}{\algtext*{EndSRepeat}}{}% Don't print if 'noend' is set
\makeatother



%%% Statix

%\usepackage{tensor}

% terms
\newcommand{\svar}[1]{\mathit{#1}}                                % syntax var
\newcommand{\uvar}[1]{{{\scriptstyle ?}\mkern-1mu\mathit{#1}}}    % unification var
\newcommand{\ovar}[1]{\texttt{#1}}                                % object language var
\newcommand{\appl}[1]{\mathsf{#1}}                                % constructor applications
\newcommand{\tuple}[1]{(#1)}
\newcommand{\set}[1]{\left\{\mkern1mu #1 \mkern1mu\right\}}
\newcommand{\emb}[1]{\smash{\raisebox{0.65ex}{$\scriptstyle \ulcorner$}}\mkern-4mu\mathtt{\mathbf{#1}}\mkern-4mu{\scriptstyle \raisebox{0.65ex}{$\scriptstyle \urcorner$}}}

\newcommand{\qBase}[4]{
  #2 \xvisible[#1]{#3} #4}

\newcommand{\orderSyntax}{\raisebox{1.5pt}[0pt][0pt]{\ensuremath{\scriptscriptstyle o}}}

% constraints
\newcommand{\cTrue}{\mathsf{true}}
%\newcommand{\cForall}[2]{\forall#1\ \mathsf{in}\ #2.\:}
\newcommand{\cExists}[1]{\exists#1.\:}
\newcommand{\cConj}{\mathrel{*}}
\newcommand{\cEq}{\mathbin{\scriptstyle\smash{\stackrel{?}{=}}}}    % equality constraint
\newcommand{\cNEq}{\mathbin{\scriptstyle\smash{\stackrel{?}{\neq}}}}% inequality constraint
\newcommand{\cUser}[1]{\mathsf{#1}}                                 % user-defined constraint
\newcommand{\cNew}{\nabla}
\newcommand{\cEdge}[1][]{\scopeedget[#1]}
\newcommand{\cQuery}[5][]{\smash{
  %\mathsf{query} \
    \qBase{#1}{#2}{#3}{#4} \mathrel{\mapsto} #5
}}
\newcommand{\cQueryBase}[4][]{\smash{
  %\mathsf{query} \
    \qBase{#1}{#2}{#3}{#4}
}}
\newcommand{\cEmp}{\mathsf{emp}}
\newcommand{\cFalse}{\mathsf{false}}
\newcommand{\cSingle}[2]{\mathsf{single}(#1, #2)}
\newcommand{\cForall}[3]{\forall#1\: \mathsf{in}\: #2.\: #3}
\newcommand{\cDataOf}[2]{\mathsf{dataOf}(#1, #2)}

% terms
\newcommand{\tHole}[1]{\llparenthesis #1 \rrparenthesis}

% queries
\newcommand{\reeps}{\ensuremath{\varepsilon}}
\newcommand{\reempty}{\ensuremath{\emptyset}}
\newcommand{\reclos}[1]{\ensuremath{#1^\ast}}
\newcommand{\reopt}[1]{\ensuremath{#1{}^?}}

\newcommand{\lblOrdLexDEF}{\lblDEF < \lblIMPORT < \lblLEX}
\newcommand{\lblOrdLexVAR}{\lblVAR < \lblIMPORT < \lblLEX}
\newcommand{\lblOrdLexMOD}{\lblMOD < \lblLEX}

% rules
\newcommand{\rTurnstile}{\, \leftarrow\, } %ugh, \mathrel does not work when using alignment symbols
\newcommand{\rName}[1]{\operatorname{\textsc{#1}}}            % rule name
\newcommand{\hoPred}[2]{\mathsf{#1}_{#2}}                     % predicate


% solver
\newcommand{\StxState}{Z}
\newcommand{\Delays}{\mathcal{D}}
\newcommand{\StateSG}[1][\StxState]{\SG_{#1}}
\newcommand{\StateDelays}[1][\StxState]{\Delays_{#1}}
\newcommand{\solve}{\mathsf{solve}}
\newcommand{\stBigAnd}{{\raisebox{-.4ex}{\scalebox{2}{$\ast$}}}}

% operational semantics
\newcommand{\opsState}[2]{
  \big\langle\, #1\: \big|\: #2\, \big\rangle
}
\newcommand{\opsRSState}[4]{
  \big\langle\, #1\: \big|\: #2\: \big|\: #3\: \big|\: #4\, \big\rangle
}
\newcommand{\opsRSStateQ}[5]{
  \big\langle\, #1\: \big|\: #2\: \big|\: #3\: \big|\: #4\: \big|\: #5\, \big\rangle
}

% For FontAwesome fixed-width icons (e.g., lock icon)
\usepackage{fontawesome5}


%%% Scope Graphs
\usepackage{scopegraphs}
\newcommand{\SG}{\mathcal{G}}

\newcommand{\lbl}[1]{\mathsf{#1}}

\newcommand{\lblLEX}{\lbl{LEX}}
\newcommand{\lblIMP}{\lbl{IMP}}

\newcommand{\lblMOD}{\lbl{MOD}}
\newcommand{\lblCLS}{\lbl{CLS}}
\newcommand{\lblMTH}{\lbl{MTH}}
\newcommand{\lblFUN}{\lbl{FUN}}

\newcommand{\lblDEF}{\lbl{DEF}}
\newcommand{\lblVAR}{\lbl{VAR}}
\newcommand{\lblFLD}{\lbl{FLD}}
\newcommand{\lblSFLD}{\lbl{S\_FLD}}

% Aliases:
\newcommand{\lblIMPORT}{\lbl{IMP}}
\newcommand{\lblFIELD}{\lbl{FLD}}
\newcommand{\lblSTATICFIELD}{\lbl{S\_FLD}}

%% Evaluation Results
%%%%%%%%%%%%%%%%%%%%%%%%%%%%%%%%%%%%%%%%%%%%%%%%%%%%%%%%%%%%%%%%%%%%%%%%%%%%%%
% This file is generated using `python3 dataviz.py` in `evaluation/dataviz/`
% Based on this result set: 20240929T173103
%%%%%%%%%%%%%%%%%%%%%%%%%%%%%%%%%%%%%%%%%%%%%%%%%%%%%%%%%%%%%%%%%%%%%%%%%%%%%%
\newcommand{\javaTotalTests}{2528}
\newcommand{\javaSuccessfulTests}{2382}
\newcommand{\javaTimedoutTests}{146}
\newcommand{\javaFailingTests}{0}
\newcommand{\chocopyTotalTests}{196}
\newcommand{\chocopySuccessfulTests}{155}
\newcommand{\chocopyTimedoutTests}{41}
\newcommand{\chocopyFailingTests}{0}
\newcommand{\fgjTotalTests}{49}
\newcommand{\fgjSuccessfulTests}{38}
\newcommand{\fgjTimedoutTests}{11}
\newcommand{\fgjFailingTests}{0}
\newcommand{\allTotalTests}{2773}
\newcommand{\allSuccessfulTests}{2575}
\newcommand{\allTimedoutTests}{198}
\newcommand{\allFailingTests}{0}

\newcommand{\javaOriginalTestSet}{4105}
\newcommand{\chocopyOriginalTestSet}{308}
\newcommand{\fgjOriginalTestSet}{52}
\newcommand{\allOriginalTestSet}{4465}


%%% Administrative
\usepackage{ournotes}

\newcommand{\sep}{\mathrel{|}}

% Language
\hyphenation{Mini-Mod}



%%% Environments
\theoremstyle{acmplain}
\newtheorem*{theorem*}{Theorem}

\theoremstyle{remark}
\newtheorem*{remark}{Remark}