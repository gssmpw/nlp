\section{Related Work}
%
\label{sec:related-work}

% \az{\url{https://doi.org/10.1145/2328876.2328881}: our work could be a backbone of these kinds of tools. Very much in line with the MasCot vision}

Refactoring as a discipline and subject goes back to the pioneering works by Griswold____ and Opdyke____.
Since then, refactoring has become a well-established field that has grown exponentially.
Steimann____ notes that the growth has been so large that a recent attempt to update the Mens~and~Tourw\`{e}'s survey____ was abandoned as there were too many works to be considered.
Consequently, we focus our discussion on prior work most relevant to reference synthesis.

A distinguished line of work on implementing refactorings is Thompson et al.'s work on refactoring tools for Haskell____ and Erlang____, which provides support for scripting a general-purpose code transformations with possibilities to specify pre- and post-conditions that ensure the transformation preserves certain properties explicitly.
This support is realized via name-binding APIs implemented from scratch for each individual language.

% \subsection{Reference Synthesis in Previous Work}
% \label{sec:related-reference-synthesis}

As discussed in the introduction, our work builds upon previous work by____, who introduced the idea of tracking name-binding dependencies by ``locking'' references to declarations and then synthesizing concrete references.
Sch\"{a}fer~and~de~Moor applied this concept to synthesizing references for Java but deemed their reference synthesis algorithm to be beyond the scope of their paper~\cite[\S{2}]{SchaferMOOPSLA2010}.
Our reference synthesis algorithm is applicable to any language whose typing rules are defined using Statix____.

% Our algorithm currently only provides an API for querying name-binding information.
% To guarantee binding preservation, Sch\"{a}fer et al.\@ also support querying and checking control flow and data flow properties.
% The question of how to define and query control and data flow language parametrically is an interesting one that we would like to explore in future work.

% Sch\"{a}fer also discusses other aspects of implementing refactorings, such as using a more restricted version of the object language as input for the refactoring, yet producing a richer version of the object language as output.
% The restricted input language reduces the number of (corner) cases to deal with, while the richer output language allows more complex concepts and refactorings to be expressed easily.
% At both ends, it would be necessary to translate the program: first from the object language to its restricted input version, and finally from its richer output version to the object language.
% We scoped our work to leave out these other details of refactorings that have little to do with fixing the references.

In other closely related previous work, de~Jonge and Visser____ describe a language-generic API for name-binding preserving refactorings.
Their approach is inspired by the work of Ekman et~al.____ on JastAdd____, which they generalize by giving operations for querying name-binding information and requalifying names.
They demonstrate that this approach could be applied to multiple languages, including Stratego and a subset of Java____.
The main difference between their generic API and our work is that de~Jonge~and~Visser require the requalification function to be explicitly provided.
Our work provides this function.
% However, implementing type checkers/compilers that provide the name-binding information required by this API is challenging.
% For example, Sch\"{a}fer~and~de~Moor ascribe bugs found in previous versions of automated refactorings for Eclipse to the ``inherent complexity of name-binding rules in mainstream languages''~\cite[§1]{SchaferMOOPSLA2010}.
% Scope graphs____ provide a declarative model of name binding that was designed to address this inherent complexity.
% Our work provides an API that leverages this model and design.
% As a result, we were able to reuse the existing Statix name-binding specification for Java due to van Antwerpen~and~Visser____ to automatically derive an API for querying the complex name-binding structure of Java programs.

% \az{We should clearly explain the delta over this paper, as, based on the title, it solves exactly the same problem.}

Tip et al.____ demonstrate that the problem of checking that a refactoring preserves well-typedness in a language like Java can be expressed as a \emph{type constraint problem}____.
Their type constraint framework supports a wide range set of refactorings for a large subset of Java.
% This framework is based on a constraint language that includes Java-specific constraints for querying name-binding information and doing name resolution.
% These constraints allow them to define a wide range of refactorings that retain references in Java programs.
While their work focuses on Java-specific constraints, we address the broader problem of guaranteeing that references resolve to the same declaration for \emph{any} programming language whose semantics of name resolution is defined using scope graphs, with Java serving as one of the case studies.

Steimann____ generalizes the work of Tip et al.\@ to also consider binding preservation and provides a more general and language-independent foundation for constraint-based refactoring.
Although this foundation could in principle support refactorings in terms of name-binding constraints that would guarantee binding preservation in any language, the question of how to provide a language-parametric semantics for such constraints is left open.
Our language-parametric algorithm might provide an answer to this question.

An important aspect of refactorings that move code is to avoid accidental name capture.
% For example, if a reference to a particular declaration is moved to a scope where a \emph{different} declaration with the same name is reachable, the reference may be \emph{captured} such that the refactored program has a different meaning.
% Preserving name binding subsumes the name capture problem, since binding preservation implies that names resolve to the same declarations as before.
% However, the
% Some techniques for solving name capture differ from those used in reference synthesis.
A common approach to avoiding name capture is \emph{renaming} (following, \eg, the Barendregt convention____).
While our reference synthesis algorithm currently focusses on synthesizing (qualified) references, it does not produce suggestions where name capture could be avoided by renaming declarations.
The language-parametric Name-Fix algorithm due to Erdweg~et~al.____ provides an interesting solution to this problem.

% \subsection{Meta-tooling for Specifying and Implementing Refactorings}
% \label{sec:related-other-transformation}

% The language-parametric synthesis algorithm presented in this paper has been implemented in the Spoofax Language Workbench____.
% To implement transformations in Spoofax we used Stratego____, a domain-specific language for code transformation.
% There exist several other domain-specific languages for code transformation, such as TXL____, DMS____, ASF+SDF____, and Rascal____.
% While these languages provide similarly declarative support for object language code transformation as Stratego, none of these other languages integrate with the Statix solver____ which provides the name-binding information that is key to our algorithm and approach.

Despite serving a very different purpose, ____ define a language-parametric code completion algorithm that also relies on the Statix specification.
Like our approach, they insert a free unification variable in a placeholder position, and partially type-check the program.
By inspecting the stuck constraints, they find and propose type-sound suggestions for code completion.
Their work, together with ours, shows that having a declarative but executable specification of the static semantics is essential to deriving sound, language-parametric editor services.



% Cubix by Koppel et al.____ represents a different approach to language-parametric transformation from what we have considered in this paper.
% The idea is to define transformations that work on abstract syntax tree nodes of some type $A$ that are independent from particular object language syntax tree nodes, but where syntax tree nodes of $A$ can be mapped to/from object language syntax tree nodes.
% By selectively mapping object language syntax tree nodes to nodes of type $A$, we can apply generic transformations that work on AST nodes of type $A$, and subsequently project the transformed program back into object language syntax.
% Thus Cubix uses a different notion of ``language parametricity'' from us.