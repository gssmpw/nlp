\section{Related Work}
\vspace{-1mm}
\label{sec:related}


% Limited energy availability in AGRs impacts performance and utilization. 
Research focused on managing limited energy availability on AGR can be can be grouped into the following categories:
% two main strategies: (1) \textit{energy budgeting} through allocation with constraints, and (2) developing \textit{energy-efficient solutions} to optimize available energy use.

\noindent{\textbf{Energy Budgeting.}} Several studies propose task and charging scheduling strategies for AGRs by budgeting energy for specific tasks or periods____. These solutions often lack control over energy utilization, necessitating replanning based on updated energy availability. Tasks such as food delivery, which cannot be easily rescheduled once started. Additionally, certain tasks may lead to inefficient energy use, such as operating at maximum computing frequency unnecessarily. This highlights the need for more effective energy management strategies.

\noindent{\textbf{Energy-Efficient Solutions.}} Studies  on energy-efficient solutions focus on minimize  locomotion and computation energy consumption____. However, these approaches often prioritize energy efficiency over maximizing the energy budget for improved performance.

These studies often work independently, limiting their potential by not combining both approaches. \textit{To the best of our knowledge, no study focuses on maximizing the energy budget utilization for scheduled tasks to enhance performance while guaranteeing reactiveness by design (through reaction distance).} \looseness -1



\vspace{-1mm}