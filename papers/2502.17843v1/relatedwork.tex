\section{Related Work}
The evolution of Autonomous Vehicle Detection (AVD) has been significantly influenced by deep learning. Initially, Convolutional Neural Networks (CNNs) were the go-to approach due to their effectiveness in handling image data. One of the pioneering models, AlexNet \cite{krizhevsky2012imagenet}, demonstrated the power of deep learning in image classification, which soon found applications in AVD. This was followed by models like Faster R-CNN \cite{ren2015faster}, which introduced region proposal networks, enhancing both speed and accuracy.

Despite these advancements, CNN-based methods struggled in complex environments such as treacherous roads. Issues like slow inference times and an inability to capture the global context effectively led to the exploration of more sophisticated architectures. Single-stage detectors, such as YOLO \cite{redmon2016you}, attempted to address these problems by predicting bounding boxes and class probabilities in a single evaluation. However, these models often faltered in complex scenes and small object detection \cite{liu2016ssd}.

To overcome these challenges, the focus shifted to Transformer-based models, initially popularized in natural language processing by \textit{Vaswani et al}. \cite{vaswani2017attention}. The DETR (DEtection TRansformer) model \cite{Carion2020DETR} emerged as a groundbreaking approach for image tasks. Unlike traditional CNNs, DETR utilizes a transformer encoder-decoder architecture to predict object sets directly, capturing the global context of an image. This method eliminates the need for hand-crafted components like anchor boxes and non-maximum suppression, simplifying the detection process and improving accuracy, particularly in cluttered scenes \cite{zhu2020deformable}.

The application of DETR in AVD is still in its nascent stage. Most studies, such as those by \textit{Zhu et al.} \cite{zhu2020deformable} and Carion et al. \cite{Carion2020DETR}, have focused on general object detection and segmentation. However, recent research indicates that integrating DETR into AVD can leverage its robust set prediction capabilities and global context understanding \cite{gao2021road}. Enhancements like Collaborative Hybrid Assignments Training (Co-DETR) show promise in improving detection accuracy and efficiency in challenging driving environments, highlighting the potential of transformer-based models in AVD.

    \begin{figure}[!htb]
        \centering
        \includegraphics[width=\linewidth]{dataset2.png}
        \caption{Class Distribution of BadODD Dataset}
        \label{fig:class-dist}
    \end{figure}
    

%------------------------------------------------------------------------