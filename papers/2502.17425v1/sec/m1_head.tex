\usepackage{bm}

\DeclareMathOperator*{\argmin}{\arg\min}
\newtheorem{proposition}{Proposition}
\newtheorem{theorem}{Theorem}[section]
\newtheorem{lemma}[theorem]{Lemma}
\newtheorem{remark}{Remark}
\newtheorem{subproposition}{Part}[proposition]

\renewcommand{\theproposition}{\arabic{proposition}}
\renewcommand{\thesubproposition}{\arabic{subproposition}}

\newcommand{\sign}{\text{sign}}
\newcommand{\detach}{\text{detach}}
\newcommand{\softmax}{\text{softmax}}
\newcommand{\sg}{\text{sg}}

\newcommand{\mystar}{\raisebox{0.5ex}{\tiny *}}

\usepackage{subcaption}

\usepackage{multirow} 
\usepackage{makecell}

% \usepackage[table]{xcolor}
\usepackage{tikz}

% Define dark green color
\definecolor{myred}{rgb}{1.0, 0.62, 0.61}
\definecolor{mygreen}{RGB}{8, 147, 146}
\definecolor{myorange}{RGB}{233, 159, 105}
\definecolor{mypink}{RGB}{207, 89, 126}

\newcommand{\cb}[1]{
    \hspace{-0.5em}
  \begin{tikzpicture}[baseline=-0.35em]
  \pgfmathsetmacro{\opacitylevel}{#1/100}
    \node[circle, fill=myred, fill opacity=\opacitylevel, inner sep=0.2em] at (0,0) {};
  \end{tikzpicture}
  \hspace{-0.6em}
}

\usepackage{array} % 用于自定义表格列格式

\usepackage{setspace}
\usepackage{tocloft}

\usepackage{amssymb} % for checkmarks and xmarks
\usepackage{pifont}
% Define checkmark and xmark
\definecolor{fullgreen}{rgb}{0.502, 0.788, 0.643}
\definecolor{fullred}{rgb}{0.800, 0.447, 0.541}
\newcommand{\cmark}{\textcolor{fullgreen}{\textbf{\checkmark}}}
\newcommand{\xmark}{\textcolor{fullred}{\textbf{\ding{55}}}}
\definecolor{lightgreen}{RGB}{225, 239, 217}   
\definecolor{lightblue}{RGB}{203, 220, 235}   
\definecolor{fullgray}{RGB}{219, 223, 234}   
\definecolor{fullpurple}{RGB}{205, 193, 255}
\definecolor{darkred}{RGB}{204, 114, 138}
\definecolor{darkpurple}{RGB}{171, 151, 255}
\definecolor{darkgray}{RGB}{114, 114, 114}
% \newcommand{\whiteline}{\textcolor{white}{$|$}}

\newcommand{\rsta}{\colorbox{fullgray}{\strut st}$\,$}
\newcommand{\rstb}{\colorbox{lightblue}{\strut $x_{\text{min}}$}$\,$}
\newcommand{\rstc}{\colorbox{lightblue}{\strut $y_{\text{min}}$}$\,$}
\newcommand{\rstd}{\colorbox{lightblue}{\strut $x_{\text{max}}$}$\,$}
\newcommand{\rste}{\colorbox{lightblue}{\strut $y_{\text{max}}$}$\,$}
\newcommand{\rstf}{\colorbox{fullgray}{\strut  ed}}

\newcommand{\dftb}{\colorbox{lightblue}{\strut  DINO\_Ctrl}$\,$}

\usepackage{colortbl}

\usepackage{soul}
\setul{0.1ex}{0.3ex}
\usepackage{enumitem}
% \makeatletter
% \renewcommand
% \tableofcontents
% {
%   % \begin{center}
%   % \begingroup
%   % \Large
%   % \textbf{Contents}
%   % \endgroup
%   % \\~\\
%   \begin{spacing}{1.1}
%   \large
%   \@starttoc{toc}%
%   \end{spacing}
%   % \end{center}
% }
% \makeatother

% 调整目录中编号和标题之间的距离
\setlength{\cftsecnumwidth}{3em} % 章节编号的宽度
\setlength{\cftsubsecnumwidth}{3em} % 小节编号的宽度

% 调整目录字体
\renewcommand{\cftsecfont}{\large} % 目录中章节标题的字体大小
\renewcommand{\cftsecpagefont}{\large} % 目录中章节页码的字体大小

% 调整目录行间距
\renewcommand{\cftparskip}{0.4em} % 行间距
\setlength{\cftbeforesecskip}{1em} % 章节前的间距
\setlength{\cftaftertoctitleskip}{0em} % 目录标题后的间距

% 删除目录标题
\renewcommand{\contentsname}{}

% \usepackage[accsupp]{axessibility}

\usepackage{listings}
\lstset{
    % basicstyle=\ttfamily\small,   % 字体为打字机字体,小号
    backgroundcolor=\color{fullgreen!10}, % 背景色为浅灰色
    frame=single,                 % 显示边框
    breaklines=true,              % 自动换行
    numbers=none                  % 不显示行号(可设置为 left/right 显示行号)
}
\usepackage{tcolorbox}