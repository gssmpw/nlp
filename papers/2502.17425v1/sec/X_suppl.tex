
\clearpage

\onecolumn
\renewcommand{\thefigure}{S\arabic{figure}}
\renewcommand{\thetable}{S\arabic{table}}
\renewcommand{\thesection}{S\arabic{section}}
\setcounter{figure}{0}
\setcounter{table}{0}
\setcounter{section}{0}
{
    \centering
    \Large
    \textbf{Introducing Visual Perception Token into Multimodal Large Language Model\\\textit{- Supplementary Material -}}\\
    \vspace{1.0em}
}

\section{Implement Details}
\subsection{Training Details}
Our training process consists of two phases: alignment and finetuning. 
The alignment stage aligns the additional vision features with the LLM embeddings. If the original vision encoder is used for re-encoding, the alignment stage is omitted. We use the same image-text pair data for the LLaVA 1.5 alignment, and only use the additional vision branch as the LLM's input. During training, all components except the projector are frozen. In this phase, we train the model for 1 epoch with a learning rate of 2e-3 and a batch size of 128.
The second finetuning stage allows the model to learn to output the correct Region Selection Tokens and to transmit information through the Vision Re-Encoding Tokens. We finetune the model using our constructed  dataset, as well as remaining samples from the LLaVA 1.5 finetuning dataset that were not included in our dataset. In this stage, all components except the original visual encoder and the additional vision encoder are unfrozen. In this phase, we train the model for 1 epoch with a learning rate of 2e-5 and a batch size of 256. For both the first and the second phase, we use AdamW optimizer. The experiments are deployed on 8 A100 GPU. The total training time is about 20 hours. For the 7B model, the rank of the LoRA is set to 512.

\subsection{Evaluation Prompt}
Following established practices~\cite{mmvet,llava15}, we used GPT-4o (2024-08-06) to evaluate the alignment between the model's responses and the ground truth for each question. We use the evaluation prompt in \cite{shao2024visual}.
\begin{tcolorbox}[parbox=false,colback=fullgreen!10, colframe=fullgreen!50, title=Evaluation Prompt, coltitle=black]
You are responsible for proofreading the answers, you need to give a score to the model's answer by referring to the standard answer, based on the given question. The full score is 1 point and the minimum score is 0 points. Please output the score in the form 'score: $<$score$>$'. The evaluation criteria require that the closer the model's answer is to the standard answer, the higher the score.

\noindent Question:  $<$question$>$

\noindent Ground Truth:  $<$ground truth$>$

\noindent Answer:  $<$answer$>$
\end{tcolorbox}

\subsection{Template of the Training Data Examples}
Here, we show the format of our training examples. 
The training example for the Region Selection Token is essentially the same as the samples used in \cite{shao2024visual}, except that the method for representing regions has changed from bounding boxes to region tokens.  
The training example for the Vision Re-Encoding Token is almost identical to the data in the original LLava~\cite{llava} fine-tuning dataset, with the only difference being the insertion of an additional round of dialogue between the original question and answer. This added dialogue includes the Vision Re-Encoding Token.  
\begin{tcolorbox}[parbox=false,colback=fullgreen!10, colframe=fullgreen!50, title=Template of Training Example for Region Selection Token, coltitle=black]
\textbf{User}: $<$image$>$ $<$question$>$ Please identify the region that can help you answer the question better, and then answer the question.

\noindent\textbf{Assistant}: $<$Region\_Selection\_Start$>$ $<$x\_min$>$ $<$y\_min$>$ $<$x\_max$>$ $<$y\_max$>$ $<$Region\_Selection\_End$>$.

\noindent\textbf{User}: $<$image$>$

\noindent\textbf{Assistant}: $<$ground truth$>$
\end{tcolorbox}
\begin{tcolorbox}[parbox=false,colback=fullgreen!10, colframe=fullgreen!50, title=Template of Training Example for Vision Re-Encoding Token, coltitle=black]
\textbf{User}: $<$image$>$ $<$question$>$ Please require additional perception features, and then answer the question.

\noindent\textbf{Assistant}: $<$Re-Encoding\_Start$>$  $<$Re-Encoding\_Control$>$ $<$Re-Encoding\_End$>$.

\noindent\textbf{User}: $<$image$>$

\noindent\textbf{Assistant}: $<$ground truth$>$
\end{tcolorbox}

The training for the free-choice experiment differs from other experiments only in the sample template. For the free-choice experiment, we removed the additional prompt from the questions. The training sample template is as follows.

\begin{tcolorbox}[parbox=false,colback=fullgreen!10, colframe=fullgreen!50, title=Template of Training Example for Region Selection Token (Free Choice), coltitle=black]
\textbf{User}: $<$image$>$ $<$question$>$

\noindent\textbf{Assistant}: $<$Region\_Selection\_Start$>$ $<$x\_min$>$ $<$y\_min$>$ $<$x\_max$>$ $<$y\_max$>$ $<$Region\_Selection\_End$>$.

\noindent\textbf{User}: $<$image$>$

\noindent\textbf{Assistant}: $<$ground truth$>$
\end{tcolorbox}
\begin{tcolorbox}[parbox=false,colback=fullgreen!10, colframe=fullgreen!50, title=Template of Training Example for Vision Re-Encoding Token (Free Choice), coltitle=black]
\textbf{User}: $<$image$>$ $<$question$>$

\noindent\textbf{Assistant}: $<$Re-Encoding\_Start$>$  $<$Re-Encoding\_Control$>$ $<$Re-Encoding\_End$>$.

\noindent\textbf{User}: $<$image$>$

\noindent\textbf{Assistant}: $<$ground truth$>$
\end{tcolorbox}

\section{Supplementary Experiments}
We conducted experiments on the MME~\cite{mme} and MM-Bench~\cite{mmb} benchmarks without using the Visual Perception Token, allowing the model to generate answers directly. This assessed the impact of our fine-tuning on general benchmarks. Results in \cref{tab:benchmark} show that our model does not cause degeneration and even improves performance on these benchmarks. 
Frontier language models demonstrate a remarkable mismatch between their problem-solving capabilities and poor out-of-box verification capabilities.
These limitations have largely been attributed to the inability of current language models to self-diagnose hallucinations or enforce rigour \citep{zhang_how_2023,orgad_llms_2024,snyder_early_2024,kamoi_evaluating_2024, tyen_llms_2024, DBLP:conf/iclr/0009CMZYSZ24}.
However, our findings that models can be directed to accurately perform verifications at scale suggest that these out-of-box limitations can be addressed with standard methods like instruction tuning.
We compiled a set of challenging reasoning problems and candidate solutions to provide a benchmark for these deficits.

Each entry in this benchmark consists of a question, a correct candidate response, and an incorrect candidate response, and is manually curated from the residuals of our sampling-based search experiments (Section~\ref{section:pipeline}).
An example entry from this benchmark can be found below (see Appendix~\ref{app:examplebenchmark} for more).

\vspace{0.4cm}
\begin{tcolorbox}[title=Question from LiveBench Reasoning (Web-of-Lies Puzzle), breakable]
In this question, assume each person either always tells the truth or always lies. The person at the campground thinks their friend is lying. Mateo is at the aquarium. The person at the restaurant says the person at the hotel lies. Farid is at the movie theater. The person at the movie theater says the person at the campground lies. Ryan is at the shopping mall. The person at the cafe says the person at the campground lies. The person at the observatory says the person at the museum lies. The person at the museum says the person at the restaurant tells the truth. The person at the campground says the person at the airport tells the truth. The person at the campground tells the truth. Luna is at the restaurant. The person at the movie theater saw a firetruck. Isabella is at the observatory. The person at the amusement park says the person at the museum tells the truth. The person at the restaurant says the person at the shopping mall tells the truth. Liam is at the beach. Ayaan is at the airport. The person at the restaurant says the person at the beach tells the truth. Maya is at the amusement park. The person at the restaurant says the person at the aquarium lies. The person at the restaurant says the person at the movie theater lies. The person at the library says the person at the campground tells the truth. Elowen is at the hotel. Priya is at the library. The person at the gym says the person at the restaurant tells the truth. Kehinde is at the cafe. Anika is at the museum. Quan is at the campground. Soren is at the gym. Does the person at the museum tell the truth? Does the person at the restaurant tell the truth? Does the person at the movie theater tell the truth? Think step by step, and then put your answer in **bold** as a list of three words, yes or no (for example, **yes, no, yes**). If you don't know, guess.
\end{tcolorbox}
\definecolor{mydarkgreen}{rgb}{0.0, 0.5, 0.0} %


\begin{paracol}{2}

\switchcolumn[0]
\begin{tcolorbox}[breakable, colframe=mydarkgreen!70, colback=mydarkgreen!10, title=Correct response]
Here's how we can break down this logic puzzle:

1. **Quan is at the campground and tells the truth.** This is a given fact.

...

4. **The person at the restaurant (Luna) says the person at the hotel (Elowen) lies.**  We don't know yet if Luna is lying or telling the truth.

...

10. **Since Luna tells the truth about Farid, Luna *always* tells the truth.**  This unlocks several other pieces of information.

...

13. **The person at the cafe (Kehinde) says the person at the campground lies.** Since Quan tells the truth, Kehinde is lying.

Therefore:
* **Anika (museum):** Tells the truth.
* **Luna (restaurant):** Tells the truth.
* **Farid (movie theater):** Lies.

**yes, yes, no**
\end{tcolorbox}

\switchcolumn
\begin{tcolorbox}[breakable, colframe=red!70, colback=red!10, title=Incorrect Response Due to Hallucination]
Here's how we can solve this logic puzzle:

1. **Quan is at the campground and tells the truth.** This is given.

2. **The person at the airport (Ayaan) tells the truth.** Quan (campground) says the person at the airport tells the truth, and Quan tells the truth.

...

4. **Luna (restaurant) lies.**  The person at the restaurant says the person at the hotel (Elowen) lies. \textbf{\color{red} Later, the person at the restaurant says the person at the campground (Quan) lies.} Since Quan tells the truth, Luna (restaurant) must be lying.

...

Therefore:
* **Anika (museum): lies**
* **Luna (restaurant): lies**
* **Farid (movie theater): lies**

**Answer: no, no, no** 
\end{tcolorbox}
\end{paracol}

\noindent\begin{minipage}{\textwidth}
\captionof{figure}{Example of an entry in our verification benchmark. The question is sourced from the LiveBench Reasoning benchmark, and the two responses are generated by Gemini v1.5 Pro. The green response has the correct final answer; the red response has the wrong final answer due to hallucinating a non-existent clause.}
\label{fig:example}
\vspace{0.4cm}
\end{minipage}

\noindent
On each entry, our benchmark studies verification accuracy on two tasks:
\begin{enumerate}
    \item \textbf{Scoring task.} When given only the question and one of the responses, is the model able to discern the correctness of the response?
    \item \textbf{Comparison task.} When provided the whole tuple with the correctness labels of the responses masked and a guarantee that at least one response is correct, is the model able to discern which response is correct and which is incorrect?
\end{enumerate}

\noindent
The scoring task is also evaluated over a separate set of (question, response) pairs where the response reaches the correct final answer by coincidence but contains fatal errors and should be labeled by a reasonable verifier as being incorrect; an example can be found in Appendix~\ref{app:examplebenchmark}.
In the scoring task, models are provided only with the task description; in the comparison task, models are provided only with the task description and a suggestion to identify disagreements between responses in its reasoning.

Table~\ref{tab:benchmark} lists the baseline performances of current commercial model offerings on this benchmark.
Gemini v1.5 Pro is omitted from the benchmark as the entries in the benchmark are curated from the residuals of Gemini v1.5 Pro.
The prompts used in Table~\ref{tab:benchmark} are provided in Appendix~\ref{app:benchmarkprompts}.

As we previously observed, and has been noted in prior works \citep{tyen_llms_2024, kamoi_evaluating_2024}, verification errors are typically due to low recall.
Even the easier comparison task, models perform only marginally better---and often worse---than random chance.
In many cases, Consistency@5 performs worse than one-shot inference because Consistency simply averages out noise from an output distribution, meaning that a model biased towards producing an incorrect answer will do so with higher probability under Consistency.
Addressing these deficits in verification capabilities---which we see as low-hanging fruit for post-training---would enable not only better sampling-based search, but also other downstream applications of verification including reinforcement learning \citep[e.g.][]{o1-preview,deepseekai2025deepseekr1incentivizingreasoningcapability}, data flywheeling \citep[e.g.,][]{welleck_generating_2022}, and end-user experience (see Section~\ref{sec:related} for further discussion).


\begin{table}[htbp]
\centering
\begin{tabular}{llcccccc}
\toprule
\textbf{Model} & \textbf{Metric} & \multicolumn{3}{c}{\textbf{Scoring Accuracy}} & \multicolumn{1}{c}{\textbf{Comparison Accuracy}} \\
\cmidrule(lr){3-5} \cmidrule(lr){6-6}
 &  & \textbf{Correct} & \textbf{Wrong} & \textbf{Flawed} &  \\
\midrule
\multirow{2}{*}{GPT-4o} & Pass@1    & 76.5\%  & 31.0\% & 22.2\% & 43.2\%\\
 & Consistency@5 & 77.4\% & 30.0\% & 11.1\% & 35.4\% \\
\midrule
\multirow{2}{*}{Claude 3.5 Sonnet} & Pass@1 & 89.6\% & 22.5\% & 33.3\% & 56.1\% \\
 & Consistency@5 & 90.3\% & 17.5\% & 33.3\% & 61.2\% \\
\midrule
\multirow{2}{*}{o1-preview} & Pass@1 & 100\% & 68.8\% & 80.0\% & 84.5\% \\
 & Consistency@5 & 100\% & 79.4\% & 88.8\% & 92\% \\
\midrule
\multirow{2}{*}{Gemini 2.0 Flash} & Pass@1 & 73.5\% & 44.5\% & 60\% & 58\%  \\
 & Consistency@5 & 77.4\% & 42.5\% & 66.6\% & 58.7\% \\
\midrule
\multirow{2}{*}{Gemini 2.0 Thinking Flash} & Pass@1 & 75.4\% & 56.5\% & 53.3\%  & 80\%  \\
 & Consistency@5 & 77.4\%  & 55\% & 55.5\%  & 89.1\% \\
\midrule
\multicolumn{2}{c}{Random guessing}  & 80\% & 20\% & 20\% & 50\% \\
\bottomrule
\end{tabular}
\caption{Accuracy rates of commercial language models on our verification benchmark. For the task of response scoring (Scoring Accuracy), accuracy rates are broken down for entries that require identifying a correct response as being correct (Correct), entries that require identifying a wrong response as being wrong (Wrong), and entries that require identifying a wrong response that coincidentally reaches the correct answer as being wrong (Flawed).
GPT-4o and Claude 3.5 Sonnet only perform marginally better than random guessing across all tasks. o1-Preview performs better, but still fails to identify 20-30\% of wrong responses.
}
\label{tab:benchmark}
\end{table}


To verify the advantage of the Region Selection Token over direct BBox prediction, we compared the predicted regions with ground truth using IoU and Intersection over Ground Truth (IoGT), defined as:  
$$(\text{IoGT} = \frac{\text{Area of } (GT \cap \text{Pred})}{\text{Area of } GT}).$$ Results in \cref{tab:iou} show that Region Selection Token significantly outperforms direct BBox prediction in accuracy.  
\begin{table}[t]
\caption{Comparison of the more evaluation metrics.}
% \vspace{-2mm}
\centering

\small
\setlength{\tabcolsep}{5pt}

\begin{tabular}{l|ccc|ccc}
\toprule

\multirow{2}{*}{Mehthods} & \multicolumn{3}{c|}{Pr@0.6(RefCOCO)} & \multicolumn{3}{c}{Pr@0.8(RefCOCO)} \\

& val & testA & testB & val & testA & testB \\ \midrule
MaPPER\cite{liu2024mapper} & 82.23 & 86.03 & 76.11 & 66.62 & 72.63 &57.50
\\
SwimVG & \textbf{85.26} & \textbf{87.33} & \textbf{80.61}  & \textbf{68.86} & \textbf{72.83} & \textbf{63.04}
\\
% \hline
% MaPPER\cite{liu2024mapper} & 81.02 & 82.72 & 78.35 & \textbf{85.55} & 86.79  & 80.28
% \\
% SwimVG & 81.02 & 82.72 & 78.35 &\textbf{88.29}  & 90.37 & 84.89
% \\
\bottomrule
\end{tabular}


\vspace{-2mm}
\label{Table:iou}
\end{table}

% \begin{table*}[ht]
\centering
\scalebox{0.95}{
\begin{tabular}{l|ccc|ccc|cc}
\tablestyle{1.02pt}{1.15}
\multirow{2}{*}{\textbf{Method}} & \multicolumn{3}{c}{\textbf{refCOCO}  }                                                   & \multicolumn{3}{c}{\textbf{refCOCO+}  }                                                  & \multicolumn{2}{c}{\textbf{refCOCOg}   }                    \\
                        & \multicolumn{1}{c}{val} & \multicolumn{1}{c}{testA} & \multicolumn{1}{c}{testB} & \multicolumn{1}{c}{val} & \multicolumn{1}{c}{testA} & \multicolumn{1}{c}{testB} & \multicolumn{1}{c}{val} & \multicolumn{1}{c}{test} \\
                        \shline

\textcolor{lightgray}{GLaMM*~\cite{hanoona2023GLaMM}} & \textcolor{lightgray}{77.5} & \textcolor{lightgray}{79.2}    & \textcolor{lightgray}{74.9}    & \textcolor{lightgray}{71.3} & \textcolor{lightgray}{74.7}    & \textcolor{lightgray}{61.5}    & \textcolor{lightgray}{71.3} & \textcolor{lightgray}{71.9}     \\
PixelLM~\cite{ren2024pixellm}                 & 73.0                      & 76.5                      & 68.2                      & 66.3                    & 71.7                      & 58.3                      & 69.3                    & 70.5                     \\

LISA-7B~\cite{lai2024lisa}                 & 74.1                    & 76.5                      & 71.1                      & 62.4                    & 67.4                      & 56.5                      & 66.4                    & 68.5                     \\
PanCaper$^{+}$                    & 74.5                    & 76.7                      & 69.9                      & 69.9                    & 73.4                      & 59.5                      & 69.8                    & 70.6                     \\

PanCaper$^{+}$ +  COCONut-PanCap    & 76.2                    & 77.1                      & 72.3                      & 70.5                    & 73.9                      & 60.1                      & 72.1                    & 71.6                    
\end{tabular}
}
\vspace{-5pt} 
\caption{\textbf{Benchmark Results on Referring Segmentation.} * denotes reproduced results. It is noted that GLaMM uses extra data from the GranD dataset for pretraining. $^{+}$ denotes our PanCaper model is adapted for referring segmentation task.}
% \vspace{5pt} 
\label{tab:refcocco}
\end{table*}




\section{Further Examples}
Here we present additional examples obtained using the visual perception token. \cref{fig:moreexample:g3,fig:moreexample:g4} include the responses generated with the Vision Re-Encoding Token. \cref{fig:moreexample:g1,fig:moreexample:g2} present the responses generated with the Region Selection Token, with the regions selected by the Region Selection Token highlighted in the images.
\begin{figure*}[h]
  \centering
    \begin{minipage}[t]{0.45\textwidth}
        \centering
        \vspace{0pt}
        \includegraphics[width=\textwidth]{fig/examples/moreexamples16.pdf}
    \end{minipage}
    \hfill
    \begin{minipage}[t]{0.45\textwidth}
        \centering
        \vspace{0pt}
        \includegraphics[width=\textwidth]{fig/examples/moreexamples14.pdf}
    \end{minipage}
   \caption{This set of images demonstrates how the DINO Feature Token assists MLLMs in identifying specific objects within images. These objects are often difficult for MLLMs to recognize directly due to their small size or interference from surrounding objects.}
   \label{fig:moreexample:g3}
\end{figure*}
\textcolor{white}{empty}
\vspace{0pt}
\begin{figure*}[h]
\vspace{0pt}
  \centering
    \begin{minipage}[t]{0.45\textwidth}
        \centering
        \vspace{0pt}
        \includegraphics[width=\textwidth]{fig/examples/moreexamples11.pdf}
    \end{minipage}
    \hfill
    \begin{minipage}[t]{0.45\textwidth}
        \centering
        \vspace{0pt}
        \includegraphics[width=\textwidth]{fig/examples/moreexamples12.pdf}
    \end{minipage}
    \\
    \begin{minipage}[t]{0.45\textwidth}
        \centering
        \vspace{0pt}
        \includegraphics[width=\textwidth]{fig/examples/moreexamples13.pdf}
    \end{minipage}
    \hfill
    % 图片 4
    \begin{minipage}[t]{0.45\textwidth}
        \centering
        \vspace{0pt}
        \includegraphics[width=\textwidth]{fig/examples/moreexamples15.pdf}
    \end{minipage}
   \caption{This set of images illustrates how the DINO Feature Token assists MLLMs in counting the number of objects in an image. Counting has long been a significant limitation for MLLMs. By leveraging the DINO Feature, the DINO Feature Token enables precise localization of individual objects within the image, thereby improving the counting capability of MLLMs.}
   \label{fig:moreexample:g4}
\end{figure*}
\textcolor{white}{empty}
\begin{figure*}[h]
  \centering
    \begin{minipage}[t]{0.45\textwidth}
        \centering
        \vspace{0pt}
        \includegraphics[width=\textwidth]{fig/examples/moreexamples1.pdf}
    \end{minipage}
    \hfill
    \begin{minipage}[t]{0.45\textwidth}
        \centering
        \vspace{0pt}
        \includegraphics[width=\textwidth]{fig/examples/moreexamples4.pdf}
    \end{minipage}
    \\
    \begin{minipage}[t]{0.45\textwidth}
        \centering
        \vspace{0pt}
        \includegraphics[width=\textwidth]{fig/examples/moreexamples3.pdf}
    \end{minipage}
    \hfill
    % 图片 4
    \begin{minipage}[t]{0.45\textwidth}
        \centering
        \vspace{0pt}
        \includegraphics[width=\textwidth]{fig/examples/moreexamples2.pdf}
    \end{minipage}
   \caption{This group of examples shows how the Region Selection Token aids MLLMs in understanding textual information within images by correctly identifying the corresponding regions. The image inputs primarily consist of large but structured documents, such as tables, forms, or letters.}
   \label{fig:moreexample:g1}
\end{figure*}
\textcolor{white}{empty}
\begin{figure*}[h]
  \centering
    \begin{minipage}[t]{0.45\textwidth}
        \centering
        \vspace{0pt}
        \includegraphics[width=\textwidth]{fig/examples/moreexamples6.pdf}
    \end{minipage}
    \hfill
    \begin{minipage}[t]{0.45\textwidth}
        \centering
        \vspace{0pt}
        \includegraphics[width=\textwidth]{fig/examples/moreexamples7.pdf}
    \end{minipage}
    \\
    \begin{minipage}[t]{0.45\textwidth}
        \centering
        \vspace{0pt}
        \includegraphics[width=\textwidth]{fig/examples/moreexamples8.pdf}
    \end{minipage}
    \hfill
    % 图片 4
    \begin{minipage}[t]{0.45\textwidth}
        \centering
        \vspace{0pt}
        \includegraphics[width=\textwidth]{fig/examples/moreexamples9.pdf}
    \end{minipage}
   \caption{This set of images illustrates how the Region Selection Token enables MLLMs to comprehend textual information within real-world scenes by accurately identifying the corresponding regions. The image inputs consist of real-world scenarios, such as signboards and trademarks, where the text occupies only a small portion of the overall scene and is highly susceptible to interference from the surrounding context.}
   \label{fig:moreexample:g2}
\end{figure*}
\textcolor{white}{empty}

\section{Additional Related Works}
\subsection{Reasoning Token} 
In Large Language Model (LLM), there are tokens, similar to Visual Perception Token, which are designed to control the generation process of LLM. These token are termed reasoning tokens or planning token and have recently been introduced in OpenAI's o1 model~\cite{openai_reasoning_guide} and other LLMs.
For example, to enhances models' reasoning capabilities, reasoning tokens were explicitly integrated into OpenAI's o1 models to segment prompts into smaller, manageable parts, exploring multiple response strategies before generating the final output~\cite{openai_reasoning_guide}.  Similar methods aim to incorporate CoT reasoning into language models through planning tokens or distillation techniques. For example, a hierarchical generation framework using planning tokens has been proposed, embedding high-level plans at each reasoning stage with minimal parameter increase~\cite{wang2024guiding}. Moreover, techniques like Rephrase and Respond have been distilled back into models, improving efficiency and accuracy in reasoning, as demonstrated in \cite{distill2to1}.

Our work focuses on MLLMs, where we design visual perception tokens to enhance the visual perception capabilities of MLLMs, not for LLM.
% In contrast, previous approaches in language models have primarily focused on improving the reasoning abilities of LLMs. 
Moreover, our exploration goes beyond LLM reasoning tokens. Unlike these tokens, which merely trigger specific actions and lack the ability to convey detailed instructions or rich information, we focus on designing tokens capable of transmitting nuanced control information for fine-grained visual perception.

\section{Discussion}
\textbf{Adaptability of Visual Perception Token.}
The design of the visual perception token depends on the specific visual perception method. In this paper, we use Crop and the addition of vision features as examples to introduce two types of visual perception tokens. However, our approach can be extended to other visual prompting techniques or visual encoder models, and even to LLM-agent or LLM-tool systems beyond vision.


