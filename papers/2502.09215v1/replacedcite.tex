\section{Related Work}
\label{sec:related}
% 0.5 pages
%\textcolor{blue}{Daniela -- to shorten}

Our work expands on Harders and Inclezan's ____ notions of behavior modes w.r.t. norm-compliance. Another work on norm-aware agents is that by by Meyer and Inclezan ____ who created the \apia architecture for norm-aware {\em intentional} agents. \apia agents operate with {\em activities} instead of simple plans, by building upon the AIA architecture by ____. %An activity is a pair $\langle g, \alpha \rangle$ where $g$ is a goal and $\alpha$ is a sequence of actions or sub-activities that achieves the goal. 
\apia agents can reason about agent intentions, %, such as serendipitous cases where an agent's goal is achieved by someone else's actions, thus enabling the agent to accomplish a goal early. However, the \apia architecture 
but does not allow the agent's controller to easily set and change behavior modes. %preferences about what to prioritize (e.g., norm-compliance versus plan length), which are fundamental for defining and simulating different behavior modes.
____ introduced an ASP framework for reasoning and planning with norms for autonomous agents. The agent actions in their framework have an associated duration and can incur penalties, while policies have an expiration deadline. On the other hand, their framework does not model different behavior modes and changes between behavior modes, which is the focus of this paper. Other existing approaches on norm-aware agents focus solely on compliant behavior (e.g., ____), while we were interested in studying a range of behavior modes on a spectrum for norm-abiding to non-compliant to enable the simulation of human behavior as well.
%
In our work, we assume that changes between behavior modes are justified in certain situations, such as emergency rescue operations, and this should be modeled and simulated. The question of emergency situations in relation to norms was previous studied by Alves and Fernández ____, but only in the context of access control policies. In contrast, the use \aopl for norm specification in our architecture allows us to express not only access control policies (i.e., authorizations), but also obligations, both strict and defeasible, and preferences between policy statements.
%
In terms of defining behavior modes via priorities between different metrics, our work indicates some connections to Son and Pontelli's $\mathscr{PP}$ for specifying basic preferences ____. It is not clear though whether maximizations of percentage metrics, which occur in our description of behavior modes, can be achieved within the $\mathscr{PP}$ framework.

%Another aspect relevant to our work is answer set planning. A survey paper by Son et al. (____) summarizes the state-of-the-art in this domain and the different special avenues of explored research: classical planning, conformant planning, conditional planning, and planning with preferences. The closest area of research to ours is that of planning with preferences. 

%Craven et al. (____) discussed issues that may arise from the analysis of a policy. Some of these  are relevant to planning, specifically {\em modality conflicts} which occur when there are seemingly contradictory statements in the obligation and authorization policy, for instance when an agent is obligated to perform an action that it is not permitted to execute. Craven et al. employ Event Calculus ____ for the description of an agent's changing environment. In our work we use ASP and leverage existing research in the ASP community on representing and reasoning about action and change, policy compliance, planning, and autonomous agents.