\documentclass{article}
% margin
\usepackage[margin=1.5in, top=1in, bottom=1in]{geometry}

% packages
\usepackage{float}
\usepackage{amsmath}
\usepackage{amsfonts}
\usepackage{xcolor}
\usepackage{booktabs}
\usepackage{multirow}
\usepackage{array,tabularx}
\usepackage{booktabs}
\usepackage{authblk}
\usepackage{graphicx}
\usepackage{subfig}
\usepackage{algorithm}
\usepackage{algorithmic}
\usepackage{tcolorbox}
\usepackage[numbers]{natbib}
\setcitestyle{square,aysep=t,yysep={,}}
\usepackage{fancyvrb}
\DefineVerbatimEnvironment{verbatim}{Verbatim}{xleftmargin=.18in}

%%
%% \BibTeX command to typeset BibTeX logo in the docs
\AtBeginDocument{%
	\providecommand\BibTeX{{%
			Bib\TeX}}}



\begin{document}
	
	%%
	%% The "title" command has an optional parameter,
	%% allowing the author to define a "short title" to be used in page headers.
	\title{Advanced Digital Simulation for Financial Market Dynamics: A Case of Commodity Futures}
	
	\author[1]{Cheng Wang}
	\author[1]{Chuwen Wang}
	\author[1]{Shirong Zeng}
	\author[1]{Changjun Jiang}
	\affil[1]{\centering School of Computer Science and Technology, Tongji University, China  
		\protect\newline
		\texttt{\{cwang, 1950060, 2051857, cjjiang\}@tongji.edu.cn}
	}
	
	\maketitle
	
	\begin{abstract}
		After decades of evolution, the financial system has increasingly deviated from an idealized framework based on precise theorems. It necessitates accurate projections of complex market dynamics and human behavioral patterns. With the development of data science and machine intelligence, researchers are trying to digitalize and automate market prediction. However, existing methodologies struggle to represent the diversity of individuals and are regardless of the domino effects of interactions on market dynamics, leading to the poor performance facing abnormal market conditions where non-quantitative information dominates the market. To alleviate these disadvantages requires the introduction of knowledge about how non-quantitative information, like news and policy, affects market dynamics. This study investigates overcoming these challenges through rehearsing potential market trends based on the financial large language model agents whose behaviors are aligned with their cognition and analyses in markets. We propose a \textbf{hierarchical knowledge architecture} for financial large language model agents, integrating fine-tuned language models and specialized generators optimized for trading scenarios. For financial market, we develop an advanced \textbf{interactive behavioral simulation system} that enables users to configure agents and automate market simulations. In this work, we take commodity futures as an example to research the effectiveness of our methodologies. Our real-world case simulation succeeds in rehearsing abnormal market dynamics under geopolitical events and reaches an average accuracy of 3.4\% across various points in time after the event on predicting futures price. Under normal market conditions, with corresponding news, our simulator also exhibits lower mean square error than series deep learning models and large language models in predicting three-day futures price of specific commodities. All experimental results demonstrate our method effectively leverages diverse information to simulate behaviors and their impact on market dynamics through systematic interaction.
	\end{abstract}
	
	\section{Main}
	
	The proliferation of financial derivatives in commodity markets, including forward contracts, futures, and options, has been primarily driven by the necessity for price risk mitigation. While these instruments enable investors to profit through finance, they have transformed commodity trading markets into complex human systems \cite{battiston2016complexity, wen2019complexity, ghent2019complexity}. Due to its zero-sum properties, commodity futures represent a relatively straightforward segment within the financial system. This paper employs it as a case study to research the inherent challenges in financial system analysis techniques. 
	
	In financial systems, investors' various interpretations of market news and information manifest in their behaviors \cite{chang2015information}, potentially introducing systemic risks through complex interactions \cite{shiller1990market, li2023information}, especially under abnormal market conditions. Although numerous methodologies have been proposed to analyze these systems over recent decades \cite{arinaminpathy2012size, axtell2022agent, zawadowski2013entangled, bou2020forecasting}, the majority fail to incorporate non-quantitative information and its impact on specific human behavioral patterns. This information frequently drives market volatility, constituting the primary source of significant financial losses. Consequently, systems capable of processing such information could provide substantial value in commodity trading applications.
	
	Contemporary market trend prediction methodologies can be categorized into two primary approaches. Machine learning methods, characterized by pattern recognition from historical data, demonstrate performance limitations and lack interpretability, failing to account for dynamic distributions and complex interactions in real-world markets \cite{henrique2019literature, Bollerslev1986Generalized, Kim2000Genetic, Huang2009Hybrid, Li2024FinReport}. Simulation methods, featuring agent-based action generation through observational learning and predefined models, offer enhanced interpretability in financial market trend projection \cite{Zhan2021Towards, Wang2023Platform, Andrea2022Learning}. Machine learning methods' dependence on historical data constrains their efficacy to conventional scenarios, rendering them ineffective when confronting unprecedented events. Due to the absence of very-long-range regularity within chaotic systems \cite{lorenz1963deterministic, may1976simple}, simulation methods are more promising among the two categories.
	
	However, the agents utilized in existing simulation methods are not authentic enough. Current rule-based or machine-learning-based agents are either too rigid or too rational to simulate the flexible and irrational human behaviors observed in response to diverse non-quantitative information in real-world markets \cite{wang2024behavioral}. The challenge of designing agents capable of providing diverse feedback on market information corresponding to their identity has persisted for a long period \cite{matthews2021evolution}. Recent advances in open-domain generative agents and associated interactive simulation techniques present potential solutions to this challenge.
	
	The fundamental prerequisite for authentic simulation in real-world complex systems is the agent's possession of comprehensive knowledge, without which performance deteriorates when confronted with open-domain input out of agent's input space \cite{huang2020challenges}. Large language models (LLMs), which have become an effective research tool in various fields \cite{NMI1, NMI2, NMI3}, address this prerequisite through their accumulation of commonsense knowledge during pre-training, demonstrating capabilities in open-domain response generation and reasoning in textual format \cite{brown2020language}. By leveraging this advanced functionality, daily life simulation in a small town utilizing these LLM agents has been realized \cite{park2023generative}. However, market simulation presents additional requirements: agents should possess varying levels of financial knowledge \cite{bellofatto2018subjective} and exhibit behaviorally consistent irrationality observed in real-world market participants \cite{becker1962irrational}.
	
	In this work, we propose a hierarchical knowledge architecture for financial LLM agents that emulates the knowledge structure of human cognition following ``model tower'' architecture proposed in 2024  \cite{wang2024behavioral}. This architecture facilitates the integration of domain-specific financial knowledge and enables generation of nuanced market behaviors. Within the financial market context, we implement a financial expert language model and specialized generator to refine agents' reasoning and actions through financial knowledge integration. Moreover, each agent is endowed with personality and individual knowledge through the system configuration of its LLM. Utilizing these architecturally enhanced agents, we develop an advanced digital simulation system enabling flexible environment configuration, financial reasoning, behavior generation, and agent interaction. Given the features that taking agents' mental factors into account and scrutinizing interactions \cite{wang2024behavioral}, we name it interactive behavioral simulation system. It effectively transforms complex real-world interactions into manageable digital simulations. The agents operate according to their defined characteristics and interact based on environmental parameters, including news events, investor sentiment, market information, and related factors. Simulation results provide empirically grounded insights for market prospect under specified conditions, offering analytical and decision-making support.
	
	We also examine the behavioral alignment between LLM agents and human market participants through an evaluation methodology grounded in behavioral economics theory. According to the prospect theory \cite{kahneman1979prospect}, people asymmetrically perceive losses as more significant than equivalent gains, a bias that consistently leads to irrational decision-making in financial markets \cite{odean1998investors}.

	\section{Results}
	
	\subsection{Overview of Hierarchical knowledge Architecture and Interactive Behavioral Simulation System}
	
		To enhance agent performance in financial contexts beyond commonsense reasoning, domain-specific financial knowledge integration is essential for accurate market environment interpretation and response generation. Drawing from cognitive science research on human knowledge structures, we propose a novel framework: financial LLM agents implemented through a hierarchical knowledge architecture comprising commonsense knowledge, professional expertise, and individualized knowledge representations.
		
		As illustrated in Figure.~\ref{fig:overview} part A, this architecture is implemented by augmenting the foundation model with domain-specific models that have undergone financial data fine-tuning and personalization through specific cognitive profiles. When processing environmental observations, an agent initially conducts analysis utilizing commonsense reasoning aligned with its configured personal knowledge configurations. To enhance the analytical precision through domain expertise, the agent engages in an iterative consultation process with an expert language model, which generates specialized financial insights. Following multiple iterations of refined reasoning, the action generation phase employs a specialized generator, derived from real-world market transaction data, to transform the agent's textual behavioral tendencies into exact quantitative trading action. This methodology, as opposed to direct quantitative output generation by the LLM, ensures enhanced alignment with real-world investor behavior patterns while mitigating potential token-based biases inherent in language models during numerical content generation.
		
		We have developed an interactive behavioral simulation system  (Figure.~\ref{fig:overview} part B) which organizes our agents to rehearse market transactions. The system architecture enables diverse initial market parameters including price dynamics, trading protocols, and investor characteristics to be configured through natural language interfaces. Through multiple iterations of chronological simulation under specified conditions, the system generates and maintains comprehensive behavioral data that serve as empirical foundations for market risk management strategies. This simulation framework facilitates the systematic evaluation of market dynamics under varying conditions while maintaining computational tractability and behavioral authenticity.
	
	\begin{figure}[!t]
		\centering
		\includegraphics[width=1.0\linewidth]{pic/Overview_NMI.pdf}
		\caption{\textbf{Overview of Hierarchical knowledge Architecture and Interactive Behavioral Simulation System.} \textbf{Part A,} Our \textbf{financial large language model (LLM) agent} utilizes hierarchical knowledge architecture to supplement financial knowledge in reasoning and action generation. Reasoning knowledge is obtained through the textual conversation or consultation with the expert language model. Action knowledge is obtained through the generation process of a specialized generator trained on action datasets. Personal knowledge is obtained through textual system configuration of LLMs. \textbf{Part B,} Interactive behavioral simulation system, organizing agents to conduct simulation, is separated into four parts. Initialization module, simulation module, and output module constitute the simulator. Outside the simulator, human actions processed by a series of interfaces are the input of simulator and the postprocess of simulator output.}
		\label{fig:overview}
	\end{figure}
	
	\subsection{Simulation of The Tsingshan Nickel Incident}
	
	We choose an influential incident as the experimental simulation scenario. The Tsingshan nickel incident of March 2022 was a significant market disruption in the global nickel futures market. It was triggered by a combination of geopolitical tensions, supply concerns, and a large short position held by Tsingshan Group, a major Chinese nickel producer. As nickel price on the London Metal Exchange (LME) surged from around 29,000 dollars to over 100,000 dollars per tonne in just two days, LME suspended trading and canceled trades, leading to market chaos and liquidity issues. Glencore, a major commodities trader, reportedly played a key role in the price surge by aggressively buying nickel contracts, effectively squeezing Tsingshan's short position. The rapid price increase triggered margin calls and forced liquidations, exacerbating the price spike and potentially exposing Tsingshan to billions in losses. LME's unprecedented decision to cancel trades and suspend the market was an attempt to prevent a cascade of defaults. The event exposed vulnerabilities in the futures market, raised questions about market manipulation and risk management, and had far-reaching consequences for the nickel industry and commodity trading.
	
	Following our investigation, we categorize the agents participating in the futures simulation into the following distinct classes: \textbf{Tsingshan Group}, manages nickel futures by leveraging production capacity and market information to maximize profits. Their strategy involves selling at peak prices during surges and subsequently lowering prices through increased supply to maintain long-term market dominance. \textbf{Glencore Company}, utilizes its substantial capital reserves to influence commodities price, aiming to capitalize on market opportunities to ensure Glencore's dominance in the commodities market. \textbf{Institutional Investor}, with significant capital and market expertise, focuses on providing stable returns through detailed analysis and strategic execution. \textbf{Nickel Buyer}, engages in nickel trading on LME with the objective of securing futures contracts at stable, competitive prices. It ensures a consistent nickel supply for industrial production while avoiding purchases at abnormally high prices during market spikes. \textbf{Aggressive Investor}, pursues high-risk, high-reward opportunities, exploiting market volatility for short-term gains. \textbf{Conservative Investor}, emphasizes long-term holding and gradual position building to achieve stable, reliable returns. Prioritizes patience and risk management, avoiding frequent trades and basing decisions on comprehensive market analysis and risk assessments. To optimize computational resources, we represent large quantities of investors as aggregated agent groups embodying these characteristics. The simulation comprises ten rounds, representing the period from February 2022 to March 2022.
	
	\begin{figure}[!t]
		\centering
		\captionsetup[subfloat]{font=small} % 设置子图 caption 字号为 small
		\subfloat[Nickel futures price trend across\\ \qquad simulation rounds]{
			\includegraphics[width=0.45\textwidth]{pic/price_trend_29.pdf}
			\label{fig:price_trend}
		}
		\subfloat[Order price trends and ranges across\\ \qquad simulation rounds]{
		\includegraphics[width=0.45\textwidth]{pic/order_trend_29.pdf}
		\label{fig:order_trend}
		}
		\quad
		\centering
		\subfloat[Comparison of price trends in the actual market and the simulated market]{
		\includegraphics[width=0.7\textwidth]{pic/trend_comparison.pdf}
		\label{fig:trend_comparison}
		}
		
		\caption{(a) illustrates the Nickel futures price trajectory at the beginning of each round within three times of simulation under the same configurations. The settlement price at the conclusion of each round is computed as the weighted average of deal prices. (b) depicts the highest and lowest prices of both bid prices and ask prices in each round as ranges. The trends of weighted average order prices are represented by separate lines. (c) illustrates the comparison of the simulated nickel futures prices from three times of simulations with the actual nickel futures prices. For the lack of stream price data of LME nickel, its price is converted from SHFE (Shanghai Futures Exchange) nickel price which is closely related to LME nickel price. Glencore began going long on Mar. 7th, 2022, and the market was suspended at 03:10, Mar. 8th, 2022. Each round of simulated trading represents four hours from round four when Glencore agent begins going long to round ten when Tsingshan faces large forced liquidation in  simulations.}
		
		\label{fig:macro}
	\end{figure}
	
	We conduct three times of simulation with the same initial settings and a LLM temperature parameter of 1. Figure \ref{fig:macro}.(a) illustrates a consistent upward trend in Nickel futures prices throughout the simulated period, with an accelerated rate of increase following the onset of geopolitical tensions. Simulation 1 and Simulation 2 have very closed trajectory, while Simulation 3 differs in the Round 3 and keeps parallel incremental trend. Take Simulation 1 as example (also in the following results), this trend is mirrored in the order prices, as evidenced in Figure \ref{fig:macro}.(b). Notably, while the lowest selling price consistently exceeds the lowest bid price in the order book, the highest bid price surpasses the highest ask price in Rounds 5, 6, 8, and 9. When compared to the actual price trajectory of LME Nickel futures from February 2022 to March 2022 described in news (pre-modification prices are unavailable), our simulation exhibits remarkably similar characteristics. These include a sharp increase preceding the announcement of geopolitical tensions and a subsequent price surge. The bidding behavior of our agents adheres to established market rules. With the exception of periods dominated by Glencore's long position, the highest bid price consistently exceeds the highest ask price. During market upward characterized by supply shortages, the average order prices of bid prices and ask prices converge, with buyer prices marginally higher. Figure \ref{fig:macro}.(c) presents a comparative analysis between our three simulation iterations and the actual LME nickel price trajectory. The simulated prices exhibit statistically significant correlation with actual data. At critical junctures in periods four and ten, the simulated price closely approximates the actual market price. These macroscopic alignments demonstrate the credibility of our simulated price trajectory.
	
	\begin{figure}[!t]
		\centering
		\captionsetup[subfloat]{font=small} % 设置子图 caption 字号为 small
		\subfloat[Trading behavior index for\\ \qquad conservative and aggressive agents]{
			\includegraphics[width=0.45\textwidth]{pic/agg_com.pdf}
			\label{fig:agg_con}
		}
		\subfloat[Buyer agent's order volume relative to\\ \qquad maximum affordable volume]{
			\includegraphics[width=0.45\textwidth]{pic/buyer_order.pdf}
			\label{fig:buyer}
		}
		\centering
		\quad
		\subfloat[Glencore's completed contracts]{
		\includegraphics[width=0.45\textwidth]{pic/GLC_long.pdf}
		\label{fig:GLC_long}
		}
		\subfloat[Tsingshan's cumulative forced \\ \qquad liquidation volume]{
		\includegraphics[width=0.45\textwidth]{pic/TS_liquidation.pdf}
		\label{fig:TS_liquidation}
		}
		\caption{(a) displays the \textit{trading behavior index} for each round, representing the ratio of \textbf{actual} futures contracts volume by an agent (conservative or aggressive) at its average order price to the \textbf{maximum} affordable volume, thus characterizing its trading pattern. (b) illustrates the ratio of the buyer agent's buy orders to its maximum affordable volume.(c) presents the volume and weighted-average price of Glencore's completed contracts as nickel buyer in each round from Round 4 to 10, calculated after each round's settlement phase. (d) illustrates Tsingshan's cumulative forced liquidation volume from Round 4 to 10, calculated after each round's settlement phase.}
		\label{fig:micro}
	\end{figure}
	
	Figure \ref{fig:micro}.(a) reveals a marked disparity in trading behavior between aggressive and conservative agents. The trading behavior index of aggressive agents is always greater than that of conservative agents, maintaining an index above 0.75 throughout the simulation. In contrast, conservative agents' index even drops below 0.6 in Round 5.
	
	As depicted in Figure \ref{fig:micro}.(b), the buyer agent demonstrates a gradual increase in Nickel acquisitions during the initial four rounds. However, a notable shift occurs in Round 5, where the agent opts to give up purchases entirely. This is followed by a resumption of high-volume purchases from Round 6 onwards.
	
	Aggressive investors consistently allocate a larger proportion of their capital to market transactions compared to conservative investors. In response to influential events that contradict prevailing market trends (as observed in the fifth round), conservative agents significantly reduce their trading volume. The Nickel buyer agent gradually increases its purchase orders in response to the Nickel shortage during the initial four rounds. However, in Round 5, influenced by news of Tsingshan's intention to short the market, it suspends Nickel purchases, anticipating potential price inflation. Following Glencore's long position, the buyer resumes large-scale Nickel futures purchases to mitigate raw material costs. These behavioral patterns demonstrate strong alignment between our simulated agents and real-world investors in futures markets.
	
	Figure \ref{fig:micro}.(c) illustrates Glencore's trading behavior, characterized by a significant increase in buy orders starting from Round 4, coinciding with the emergence of geopolitical tensions. The executed order volume peaks in Round 5, exceeding 40,000 tons, and maintains a level of about 8,000 tons per round thereafter. Concurrently, the average deal price exhibits a rapid ascent, aligning with the overall market price trajectory for Nickel futures.
	
	Figure \ref{fig:micro}.(d) delineates the growth in Tsingshan Group's cumulative forced liquidation volume. A marked surge is observed following the settlement of Round 5, with steady increments in subsequent rounds until Round 9. The liquidation volume plateaus at approximately 2.6 billion dollars by the final round.
	
	The results reveal that Glencore capitalizes on the emergence of geopolitical tensions by initiating a substantial volume of buy orders in response to bullish news. When rumors of Tsingshan's short position circulate in the fifth round, Glencore anticipates a flood of low-priced sell orders from Tsingshan. Consequently, Glencore places a large number of high-priced buy orders, resulting in a significant volume of completed transactions in this round. Subsequently, Glencore leverages its capital advantage to continue driving up market prices through sustained high-priced buy orders. This strategy forces continuous liquidation of Tsingshan's sell orders due to escalating market prices until the ninth round. By this point, all of Tsingshan's previous sell orders are margin-deficient, despite their earlier abandonment of the short position. This simulated sequence of events closely mirrors the real-world forced liquidation process experienced by the Tsingshan group in LME.
	
	\subsection{Ablation Study of Expert Language Model and Specialized Generator}
	
	To evaluate the efficacy of professional knowledge components, we conduct ablation studies on two critical modules of the agent architecture, the expert language model and the specialized generator. In the expert language model ablation, we remove expert advice on market news and trading strategies from the agents. For the specialized generator ablation, we modify the agents' action generation from a tendency-action pattern to a direct action pattern where agents are required to specify precise order quantities and prices for input into the market engine, rather than inputting a tendency into the specialized generator. All other system and profile configurations remain consistent with the baseline experiment.
	
	\begin{figure}[!t]
		\centering
		\captionsetup[subfloat]{font=small} % 设置子图 caption 字号为 small
		\subfloat[W/O Expert LM]{
			\includegraphics[width=0.3\textwidth]{pic/ab_01_orders.pdf}
			\label{fig:ab_1}
		}
		\subfloat[W/O Specialized Generator]{
			\includegraphics[width=0.3\textwidth]{pic/ab_02_orders.pdf}
			\label{fig:ab_2}
		}
		\subfloat[W/O Both Components]{
			\includegraphics[width=0.3\textwidth]{pic/ab_03_orders.pdf}
			\label{fig:ab_3}
		}
		\caption{Price ranges and weighted average prices of bid prices and ask prices across simulation rounds in ablation experiments: (a) Expert Language Model removed, (b) Specialized Generator removed, and (c) Both components removed. The graphs depict the highest and lowest order prices for each round.}
		\label{fig:ab}
	\end{figure}
	
	Figure \ref{fig:ab} presents the results of our ablation studies, illustrating the impact of removing key components from our agent architecture on order pricing dynamics.
	
	In Figure \ref{fig:ab}.(a), we observe the effects of ablating the expert language model. The weighted average prices of bid prices and ask prices demonstrate a consistent upward trajectory across rounds. However, a notable divergence from the baseline simulation is the persistent and substantial gap between the weighted average prices of buy and sell orders. 
	
	Figure \ref{fig:ab}.(b) depicts the results of the ablation experiment on the specialized generator. In this scenario, the average bid prices exhibit relatively stable growth over the rounds. In contrast, the ask prices display volatility and inconsistency. A striking observation is the irregularity in order price ranges, with some rounds showing uniform pricing across all orders. Furthermore, the final round's order price of approximately 35,000 dollars per tonne markedly differs from the baseline simulation's 70,000 dollars per tonne.
	
	Figure \ref{fig:ab}.(c) illustrates the outcomes of simulations using only the basic LLM agent, with both the expert language model and specialized generator removed. This configuration results in highly volatile and inconsistent order pricing for both buy and sell orders. Among the various metrics analyzed, only the general price change trend bears resemblance to the baseline simulation. 
	
	The ablation study results reveal a significant disparity in the weighted average price gaps between sell and buy orders when the expert language model is removed from the simulation. Analysis of the agents' inference records indicates that this phenomenon stems from the absence of optimal bidding strategies in trade request orders without expert advice. For instance, when faced with excessively high buying prices, agents failed to place appropriately higher-priced selling orders in subsequent rounds, which would typically yield greater profits. The results also demonstrate that the absence of the specialized generator leads to a marked reduction and homogenization of the agents' order price ranges. This can be due to the token-by-token generation mode of large language models, where the highest probability candidate word is preferentially selected. When tasked with direct numerical data generation, this characteristic introduces significant bias. In the context of submitted orders, agents instructed to provide specific order prices consistently gravitate towards a uniform price point, typically 5\% above the current market price. Consequently, many rounds exhibit minimal order price ranges, with some instances of complete price uniformity. This homogeneity severely compromises the fidelity of market price simulation and undermines the overall rationality of the simulation. Therefore, the expert language model and specialized generator module both prove to be essential components of agent for market simulation.
	
	\subsection{Futures Simulation under Normal Market Condition}
	
	\begin{figure}[!t]
		\centering
		\includegraphics[width=1.0\linewidth]{pic/combined_chart.pdf}
		\caption{The performance of various prediction methods in forecasting the next three days' settlement prices for six futures contracts is evaluated based on the MSE of predicted returns. For the first three futures contracts, additional news information was incorporated into the predictions using LLMs, i.e. Deepseek and our proposed method.}
		\label{fig:priceforecasting}
	\end{figure}
	
	In addition to reproducing historical events and assisting in risk management, our simulation method exhibits robust generalization capabilities. Specifically, it can simulate trading scenarios for targeted futures contracts by leveraging external information, with the simulated settlement price for each trading day serving as a predictive estimate of the actual settlement price in real-world markets. To evaluate the generalization performance of our simulation method in price prediction, we conduct experiments on six distinct futures contracts.
	
	Specifically, we select six futures contracts, containing energy, chemicals, finance, and agricultural products, for our experiments: SC2501 (crude oil futures from the Shanghai International Energy Exchange, with delivery in January 2025), SF2503 (soybean futures from the Chicago Mercantile Exchange Group, with delivery in March 2025), TA501 (terephthalic acid futures from the Zhengzhou Commodity Exchange, with delivery in January 2025), CH2503 (corn futures from the Chicago Mercantile Exchange Group, with delivery in March 2025), IH2412 (Shanghai Stock Exchange 50 Index futures from the China Financial Futures Exchange, with delivery in December 2024), and CGC2502 (gold futures from the New York Mercantile Exchange, with delivery in February 2025). Historical daily trading data for these contracts are collected from their respective exchanges. Our simulation method was initialized using this historical data, with each trading round representing a single trading day. The settlement price for each round is recorded as the predicted value, and when available, news information is incorporated as an external input during the simulation process.
	
	To assess the performance of our method, we compared it against three widely recognized open-source time-series models: chronos-t5-large \cite{ansari2024chronos}, timesfm-2.0-500m \cite{dasdecoder}, and morai-1.1-R-large \cite{woo2024unified}. These models predict future settlement prices based on historical settlement prices (past 128 trading day) of the futures contracts. Additionally, we include DeepSeek, the LLM integrated into our simulation framework. DeepSeek predicts future settlement prices using settlement prices from the five days preceding the prediction date, along with monthly and weekly price return rate and news data from the current and previous days (when available).
	
	We employ the Mean Squared Error (MSE) of the return rate as the evaluation metric. Let $s_0$ represent the futures settlement price on the last day of historical data, and let $s_i, i\in[1,n]$ denote the actual settlement price of the futures contract on day $i$ in the future, where n is the number of prediction days. The true return rate is $y_i=\frac{s_i-s_0}{s_0}$. Similarly, let $\hat{s}_i, i\in{1,2,3}$ denote the predicted settlement price on day $i$. The predicted return rate is $\hat{y}_i=\frac{\hat{s}_i-s_0}{s_0}$. The return rate MSE is defined as:$L = \frac{1}{n} \sum_{i=1}^{n} (y_i - \hat{y}_i)^2$. In our experiments, $n=3$.
	
	The experimental results are shown in Figure. \ref{fig:priceforecasting}. For SC2501, TA501, and IH2412, news information is incorporated into the simulation. For SC2501 and TA501, our method consistently outperform the baseline models over a three-day horizon, indicating that the simulation framework effectively captures the market impact of news and translates it into more accurate price predictions. However, for IH2412, even with the inclusion of news information, our method's performance is suboptimal, ranking nearly the lowest among the compared approaches. We hypothesize that this discrepancy arises from the complex market dynamics of stock index futures, which are influenced not only by external information but also by policy changes and macroeconomic conditions—factors that are not adequately expressed in limited news information.
	
	In scenarios where news information is unavailable (GCG2502, CH2503, SF2503), our method performs comparably poorly relative to the baseline models. This suggests that our model relies heavily on news data to predict market fluctuations and make informed decisions. Without such information, the model struggles to capture market volatility, underscoring the critical role of news data in our simulation framework. Furthermore, we observe that DeepSeek's predictive performance is significantly degraded in the absence of news information, highlighting its dependence on external data for accurate market modeling. However, when news information is included, DeepSeek demonstrat substantial performance improvements, emphasizing the importance of external data in enhancing its market understanding and predictive capabilities.
	
	The results highlight both the strengths and limitations of our approach. Our simulation method effectively leverages external information, such as news and market price, to generate accurate price predictions for certain futures contracts, particularly in sectors like energy and chemicals. However, its performance deteriorates significantly in the absence of news information or in markets with more complex dynamics, such as stock index futures. These findings suggest that our method is not universally applicable and requires further refinement to improve its adaptability and robustness across diverse market scenarios.
	
	\subsection{Behavioral Alignment Validation through Prospect Theory Experiments}
	
	Inspired by \textit{Prospect Theory} \cite{kahneman1979prospect}, which offers an explanation for human irrationality in monetary decision-making, we validate the behavioral alignment between agents and human.
	
	We replicate and adapt prospect theory experiments conducted on human for LLM agents. We utilize a questionnaire that closely mirrors the structure employed in the original human studies, comprising single-choice questions that present different risk-reward scenarios. For example, an agent might be asked to choose between a guaranteed gain of 450 dollars and a 50\% chance of gaining 1000 dollars, or between a certain loss of 450 dollars and a 50\% chance of losing 1000 dollars.
	
	In contrast to the research conducted by \citet{liu2024large}, which evaluated LLMs' decision-making without personas and demonstrated their tendency to assume human behavior is more rational than it actually is, we aim to investigate whether LLM agents with diverse personas exhibit irrational behavior similar to human. Specifically, the LLM agents in our evaluation are configured with distinct personas (Supplementary Note 1: Agent Types and Personas) that reflect the diversity of human cognitive and emotional profiles.
	
	We conduct prospect theory questionnaire-based experiments. Commercial model ERNIE-Bot \cite{wang2021ernie}, DeepSeek-V2-0628 \cite{deepseek-v2}, and open-source model Qwen1.5-32B \cite{qwen} are selected as our LLM models empowering agents. By comparing the agents' choices to those of human subjects from the statistical experimental results reported by \citet{kahneman1979prospect}, we can determine whether the statistical distributions of choices made by LLM agents align with those of human participants.
	
	Each agent completes a questionnaire comprising 17 questions about monetary gains or losses, mirroring the original experiments (Supplementary Note 2: Monetary Questions of Prospect Theory). To mitigate potential biases related to option order, we implement a systematic randomization of choice sequences for each agent.
	
	\begin{table}[t]
		\begin{tabular}{c|ccc|c|c}
			\toprule
			Question & ERNIE-Bot & DeepSeek-V2 & Qwen1.5-32B & Human & Question Prospect \\
			\midrule
			1 & \textbf{B} & \textbf{B} & \textbf{B} & B & Positive \\
			2 & \textbf{A} & \textbf{A} & \textbf{A} & A & Positive \\
			3 & \textbf{B} & \textbf{B} & \textbf{B} & B & Positive \\
			4 & \textbf{A} & B & B & A & Negative \\
			5 & \textbf{A} & \textbf{A} & \textbf{A} & A & Positive \\
			6 & A & \textbf{B} & \textbf{B} & B & Negative \\
			7 & \textbf{B} & \textbf{B} & \textbf{B} & B & Positive \\
			8 & \textbf{A} & B & \textbf{A} & A & Negative \\
			9 & \textbf{A} & \textbf{A} & B & A & Positive \\
			10 & \textbf{B} & \textbf{B} & A & B & Negative \\
			11 & \textbf{B} & \textbf{B} & \textbf{B} & B & Positive \\
			12 & \textbf{B} & \textbf{B} & \textbf{B} & B & Positive \\
			13 & B & B & B & A & Negative \\
			14 & \textbf{B} & \textbf{B} & A & B & Positive \\
			15 & \textbf{A} & B & B & A & Negative \\
			16 & B & B & B & A & Positive \\
			17 & \textbf{B} & \textbf{B} & \textbf{B} & B & Negative \\
			\bottomrule
		\end{tabular}
		\caption{The results of questionnaires completed by three types of LLM agents, compared with the original human experiment results reported by \citet{kahneman1979prospect}. Options A or B indicate \textbf{the majority choice} made by \textbf{agents or human} in the questionnaire. Bold options represent choices made by LLM agents that align with human choices. The `Question Prospect' column distinguishes between `Positive' (gaining money) and `Negative' (losing money) scenarios in monetary decision-making.}
		\label{tab:prospect_theory}
	\end{table}
	
	The results of comparison between LLM and human in Table.~\ref{tab:prospect_theory} show that agents powered by commercial models demonstrate alignment with human choices in 14 out of 17 questions for ERNIE-Bot, and 12 out of 17 for DeepSeek, while agents utilizing the open-source Qwen model align in 10 out of 17 questions. Categorizing questions by prospect type, ERNIE-Bot agents match human choices in 9 out of 10 positive prospect questions and 5 out of 7 negative prospect questions. DeepSeek agents achieve 8 out of 10 and 4 out of 7, while Qwen agents match 7 out of 10 and 3 out of 7 for positive and negative prospects, respectively. The discrepancies between LLM agents' and human choices are predominantly observed in negative prospect scenarios. Specifically, 2 out of 3 mismatches for ERNIE-Bot, 3 out of 5 for DeepSeek, and 4 out of 7 for Qwen occur in negative prospect questions.
	
	The statistics reveal that agents empowered by commercial models exhibit superior alignment with human decision-making compared to those utilizing open-source models. The highest matching rate achieved by an LLM agent is $14/17\approx82\%$, indicating a robust comprehension of character settings and behavior consistent with their knowledge. We think LLMs understand the behavioral biases of diverse individuals during pre-training, and profile or role-play prompt guides an agent to a specific bias through token-by-token generation process. Despite relative weaknesses in decision-making regarding negative prospects, LLM agents, particularly those powered by advanced commercial models, demonstrate a promising capacity to reproduce irrational behaviors observed in financial markets.
	
	\section{Discussion}
	
	From a theoretical perspective, the inherent complexity of financial markets, characterized by chaotic interactions among individuals, precludes the use of closed-form expressions or precise distributions to predict future states \cite{lorenz1963deterministic, may1976simple}. Consequently, there exists no fixed distribution of individual behaviors and states in complex financial markets. As evidenced by the price trends shown in Figure \ref{fig:price_trend}, Simulation 1 and Simulation 3, despite sharing same configurations, yield distinct final system states through the complex interactions of agent behaviors. This diversity of outcomes under same conditions is an unexpected phenomenon when concerning the predictable behaviors observed in simulations of simple organized systems (e.g., classical mechanics) or the fixed distributions characteristic of complex disorganized systems (e.g., statistical mechanics) \cite{weaver1948science}. \textbf{However, It is a natural consequence of chaotic systems, where even slight differences in some states can lead to dramatically divergent future trajectories.}
	
	Within our experiments, the temperature parameters of the LLM agents and specialized generator introduce randomness analogous to the variable choices made by human under repeated identical conditions \cite{holtzman2019curious}. It is this slight difference introduced by randomness that finally results in the disparate price increments observed in Round 3 across Simulation 1, 2, and 3. Despite these divergences, both simulation results remain reasonable and plausible within the context of a chaotic system. It implies that an increased volume of data, particularly data that diverges from existing historical patterns, facilitates a more comprehensive analysis of the system. Relying solely on outputs from deep learning models or other mathematical expressions may prove ineffective, as these approaches invariably produce actions conforming to distributions derived from limited market data inputs. The incorporation of open-domain environmental factors influencing individuals in financial markets (e.g., news, confidence, character) is crucial. Interactive behavioral simulation uniquely combines the advantages of existing techniques while addressing the shortcomings of models that neglect interaction. As such, it currently represents the most viable method for extending or generating novel data for market risk management.
	
	The potential utility of our interactive behavioral simulation system is evident in scenarios such as the aftermath of significant geopolitical events \cite{fernandez2008war, hossain2024impact}. Had this system been available to Tsingshan following the onset of geopolitical tensions, it could have provided valuable insights for market risk management, potentially enabling the company to abandon its short position and mitigate losses. More broadly, this system represents a powerful tool for informing critical decision-making processes in complex market environments.
	
	The experimental results substantiate the significant implications of paradigm integration methodology in addressing problems of organized complexity proposed by \citet{wang2024behavioral}. Agents based on hierarchical knowledge architectures successfully overcome the limitations inherent in the third paradigm (conventional simulation techniques), specifically the idealization and formalization constraints of simulation units. Simultaneously, advanced digital simulation technology mitigates the constraints of the fourth paradigm (data science), which typically relies on relatively static historical data distributions to infer future system dynamics. Experimental evidence demonstrates that interactive behavioral simulation, when applied to futures markets, not only achieves accurate historical event reproduction but also exhibits the capability to reasonably extrapolate the potential market impacts of unprecedented events. Given open-domain environmental information, this approach demonstrates superior predictive performance and interpretability compared to current mainstream fourth-paradigm methodologies. We believe that the interactive behavioral simulation system, which integrates elements from the second, third, and fourth paradigms, shows promising potential as an enhanced analytical tool applicable across various financial contexts and other complex systems.
	
	However, limitation remains. Scalability is a critical challenge not only for our system but for all LLM agent-based simulations. The computational resources and associated costs can escalate exponentially, potentially reaching prohibitive levels. For example, our current experimental setup requires approximately 22 hours to complete a single simulation iteration with just 10 agents serially. Recent research, such as the work by \cite{pan2024very}, has begun to address these scaling challenges. We believe that, with continued advancements in this domain, the scalability of interactive behavioral simulation systems will become increasingly feasible in the near future.
	
	\section{Methods}
	
	The methods articulated in this work represent a specialized deployment for behavioral simulation methodology addressing organizational complexity within complex systems, originally conceptualized by \citet{wang2024behavioral}, in financial domain contexts. Specifically, within the financial scenario, the proposed ``Model Tower'' architecture integrates a sophisticated approach that synthesizes financial expert language models with specialized generator designed to simulate financial market behaviors. Furthermore, predicated on the Financial LLM Agent architecture, the research elaborates a comprehensive communication protocol between the financial market simulator engine and our agents, thereby facilitating nuanced behavioral simulation within financial market environments.
	
	\subsection{Hierarchical Knowledge Architecture for Financial LLM Agents}
	
	The primary objective of enhancing agents is to better align their behavior with human decision-making in financial markets. Given LLM agents with diverse profiles ensure the tendency of irrational monetary behaviors, the remaining challenge lies in integrating financial knowledge, without which the agents would struggle to comprehend the simulation environment and user input effectively. We categorize the required financial knowledge into two distinct types: reasoning-related knowledge and action-related knowledge. In our hierarchical knowledge architecture, wherein an expert language model and a specialized generator augment the foundation LLM to incorporate these two types of knowledge respectively.
	
	The expert language model is trained or finetuned on domain-specific data, particularly financial analysis texts such as news articles, reports, earnings call transcripts, etc. This model serves to evaluate and refine the agent's initial reasoning on its observations. The expert model provides corrections or advice, which are then formatted as input for a second reasoning process. The degree to which agents accept this advice of financial knowledge varies according to their individual profiles, simulating real-world investors' varying levels of receptivity to market information and expert analysis. Following this second reasoning phase, the agent formulates its final evaluation of observation, expressed as an investment tendency.
	
	The specialized generator is a generative model designed to translate agents' investment tendencies into concrete actions, taking into account real-world action distributions. We posit that the distribution of actions under specific tendencies remains relatively stable, while the distribution of tendencies in response to open-domain news and specific market information is highly dynamic. LLMs typically exhibit abnormal biases in generating specific actions due to their token-by-token generation pattern \cite{shanahan2023role}. For instance, when instructed to simulate a dice roll, an LLM agent may produce a significantly higher frequency of `four' outcomes than the expected $\frac{1}{6}$ probability \cite{wang2024behavioral}. By equipping a specialized generator, similar to existing tool-using functions \cite{gao2023retrieval}, our agents can produce more authentic and accurate actions within the simulation environment. Specialized generator should be a generator trained on categorized investment action data set. Within this data set, action of different investment tendencies and investor styles should be separated. The specialized generator trained on this data set takes tendency and investor style as input and outputs orders, representing the mapping from investors' tendencies to specific actions. In comparison with existing generators that map historical price data to next orders, our specialized generator leverages well-processed market reasoning information provided by foundational model and expert language model, concentrating on the simpler mapping with less market dynamics.
	
	The financial LLM agents in our experimental framework are constructed using state-of-the-art pre-trained models. The foundational LLM employed is DeepSeek-V2-0628, a Chinese LLM developed by DeepSeek \cite{deepseek-v2}. For specialized financial analysis, we utilize CFGPT2-7B, a Chinese financial expert language model provided by \citet{li2024ra}. Due to the absence of trading data, we implement a specialized order generator that transforms trading tendencies into orders based on a normal distribution with corresponding mean and standard deviation extracted from historical transaction data through $k$-means clustering.
	
	\subsection{Interactive Behavioral Simulation System}
	
	The interactive behavioral simulation system integrates agents, a simulation engine, and user interfaces. Users primarily control the simulation system through a series of natural language interfaces to configure and participate in the simulation. The system also supports a fully automated mode where all participants are agents, allowing continuous simulation without user input intervals.
	
	The initialization module provides two primary interfaces: the engine configuration interface and the agent configuration interface. Through the engine configuration interface, users specify initial market conditions, trading rules, and simulation duration. The engine is then initialized based on these parameters, along with predefined rules that emulate real-world market mechanisms, such as deal execution and account settlement protocols. The agent configuration interface allows users to define agent profiles, account information, and agents' models (including model selection, temperature, and top\_p values for our agent implementation). Upon completion of these configurations, the agents synchronize with the engine settings, concluding the initialization phase.
	
	The simulation module executes iterative simulations in time steps, or `frames'. At the beginning of each frame, users have the option to input new environmental information, such as news or agent-specific information, in natural language format to influence the simulation dynamics. Subsequently, users can either actively participate by submitting transaction requests (in a format consistent with agent inputs) or passively observe agent interactions. For example, if the user is one of the characters in some simulation, he or she is allowed to directly input actions in this phase, enhancing simulation authenticity, instead of guiding his duplicated agent. Alternatively, users can design and deploy an agent as their proxy, enabling fully automated system operation. The engine processes all market behavior inputs—from both agents and users—as requests, updates the simulated market states, and generates interaction outputs. The simulation concludes after a predetermined number of rounds.
	
	The output module constructs a database to store data generated during the simulation process. Agent interactions are preserved as graph-structured data frames, while engine-processed records are stored in relational tables. These data sets constitute the primary outputs of the simulator and are presented through a specialized analysis interface. This interface facilitates manual market risk management analyses referring to the simulation outcomes.
	
	To elucidate the simulation procedure of our interactive behavioral simulation system, we present a detailed example. (Notations in Extended Table.~\ref{tab:notations}, Algorithm in Supplementary Note 3: )
	
	During the initialization phase, the user configures agents $\mathcal{A}$ , and concurrently, the engine $\mathcal{E}$ is initialized with configurations. The engine then synchronizes these data with the database and agents to ensure consistency. Finally, the action records set $\mathbf{A}$ is initialized as an empty set.
	
	The simulator executes $n$ rounds of simulation, where $n$ is user-defined. In each round $i$, the user first inputs market-influencing information $U_i$. Subsequently, each agent $a \in \mathcal{A}$ observes and analyzes $U_i$ along with current market data (e.g., prices, major holders) retrieved from the database via engine queries. To refine its market trend analysis, each agent calls an expert language model for advice. The trading phase commences after all agents complete their observations and analyses.
	
	The trading phase in each round comprises five consecutive turns. At the onset of this phase, the set of order requests $\mathbf{T}$ is initialized as empty. For each turn, every agent $a$ performs the following steps:
	\begin{enumerate}
		\item Retrieve its account information $m_a$ from the engine $\mathcal{E}$.
		\item Formulate a preliminary trading strategy based on its account state and its attitude towards the market.
		\item Consult the expert language model to refine its trading tendency.
		\item Utilize a specialized generator to transform the trading tendency into quantitative order requests $t$, which are then appended to the set $\mathbf{T}$.
	\end{enumerate}
	The engine $\mathcal{E}$ then matches requests and generates a set of successful deals $\mathbf{T}_{success}$. $\mathbf{T}$ is updated by removing matched requests. Both $\mathbf{T}$ and $\mathbf{T}_{success}$ are merged into the actions set $\mathbf{A}$ and stored in the database.
	
	Upon receiving matching results, each agent $a$ determines whether to withdraw unmatched requests. Withdrawn requests $\mathbf{T}_{w}$ are removed from $\mathbf{T}$. As the conclusion of the trading phase, the engine $\mathcal{E}$ executes settlement for all deals across the five turns and updates market and account states accordingly.
	
	At the termination of each round, analogous to the close of a trading day, each agent $a$ conducts reflection of its strategy and actions. This reflection serves as input for decision-making in subsequent round.

\begin{thebibliography}{99}		
	\bibitem{battiston2016complexity}
	Battiston, S., Farmer, J. D., Flache, A., Garlaschelli, D., Haldane, A. G., Heesterbeek, H., Hommes, C., Jaeger, C., May, R. \& Scheffer, M.
	\newblock Complexity theory and financial regulation.
	\newblock {\em Science} 351(6275), 818--819 (2016).
	
	\bibitem{wen2019complexity}
	Wen, C. \& Yang, J.
	\newblock Complexity evolution of chaotic financial systems based on fractional calculus.
	\newblock {\em Chaos, Solitons \& Fractals} 128, 242--251 (2019).
	
	\bibitem{ghent2019complexity}
	GHENT, A. C., TOROUS, W. N. \& VALKANOV, R. I. 
	\newblock Complexity in structured finance.
	\newblock {\em The Review of Economic Studies} 86(2), 694--722 (2019).
	
	\bibitem{chang2015information}
	Chang, Y. C. \& Cheng, H. W.
	\newblock Information environment and investor behavior.
	\newblock {\em Journal of Banking \& Finance} 59, 250--264 (2015).
	
	\bibitem{shiller1990market}
	Shiller, R. J.
	\newblock Market volatility and investor behavior.
	\newblock {\em The American Economic Review} 80(2), 58--62 (1990).
	
	\bibitem{li2023information}
	Li, J.,Zhou, Z. Q.,Zhang, Y. J.  \& Xiong, X.
	\newblock Information interaction among institutional investors and stock price crash risk based on multiplex networks.
	\newblock {\em International Review of Financial Analysis} 89, 102780 (2023).
	
	\bibitem{arinaminpathy2012size}
	Arinaminpathy, N., Kapadia, S. \& May, R. M.
	\newblock Size and complexity in model financial systems.
	\newblock {\em Proceedings of the National Academy of Sciences of the United States of America} 109(45), 18338--18343 (2012).
	
	\bibitem{axtell2022agent}
	Axtell, R. L. \& Farmer, J. D.
	\newblock Agent-based modeling in economics and finance: Past, present, and future.
	\newblock {\em Journal of Economic Literature}, 1--101 (2022).
	
	\bibitem{zawadowski2013entangled}
	Zawadowski, A.
	\newblock  Entangled financial systems.
	\newblock {\em The Review of Financial Studies} 26(5), 1291--1323 (2013).
	
	\bibitem{bou2020forecasting}
	Bou-Hamad, I. \& Jamali, I.
	\newblock Forecasting financial time-series using data mining models: A simulation study.
	\newblock {\em Research in International Business and Finance} 51, 101072 (2020).
	
	\bibitem{henrique2019literature}
	Henrique, B. M., Sobreiro, V. A. \& Kimura, H.
	\newblock Literature review: Machine learning techniques applied to financial market prediction.
	\newblock {\em Expert Systems with Applications} 124, 226--251 (2019).
	
	\bibitem{Bollerslev1986Generalized}
	Bollerslev, T.
	\newblock Generalized autoregressive conditional heteroskedasticity.
	\newblock {\em Journal of Econometrics} 31(3), 307--327 (1986).
	
	\bibitem{Kim2000Genetic}
	Kim K. J. \& Han, I.
	\newblock Genetic algorithms approach to feature discretization in artificial neural networks for the prediction of stock price index.
	\newblock {\em Expert Systems with Applications} 19(2), 125--132 (2000).
	
	\bibitem{Huang2009Hybrid}
	Huang, C. L. \& Tsai, C. Y.
	\newblock A hybrid sofm-svr with a filter-based feature selection for stock market forecasting.
	\newblock {\em Expert Systems with Applications} 36(2), 1529--1539 (2009).
	
	\bibitem{Li2024FinReport}
	Li, X. Y., Shen, X. J., Zeng, Y. W., Xing X. F. \& Xu, J.
	\newblock Finreport: Explainable stock earnings forecasting via news factor analyzing model.
	\newblock In {\em Companion Proceedings of the ACM Web Conference 2024}, WWW '24, page 319–327, New York, NY, USA, 2024. Association for Computing Machinery.
	
	\bibitem{Zhan2021Towards}
	Zhan, R., Christakopoulou, K., Le, Y.,  Ooi, J., Mladenov, M., Beutel, A., Boutilier, C., Chi, E. \& Chen, M.
	\newblock Towards content provider aware recommender systems: A simulation study on the interplay between user and provider utilities.
	\newblock In {\em Proceedings of the Web Conference 2021}, WWW '21, page 3872--3883, New York, NY, USA, 2021. Association for Computing Machinery.
	
	\bibitem{Wang2023Platform}
	Wang, X., Ma, G. Q. , Eden,  A., Li,  C., Trott,  A., Zheng, S. \& Parkes, D. 
	\newblock Platform behavior under market shocks: A simulation framework and reinforcement-learning based study.
	\newblock In {\em Proceedings of the ACM Web Conference 2023}, WWW '23, page 3592--3602, New York, NY, USA, 2023. Association for Computing Machinery.
	
	\bibitem{Andrea2022Learning}
	Coletta, A., Moulin, A., Vyetrenko, S. \& Balch, T.
	\newblock Learning to simulate realistic limit order book markets from data as a world agent.
	\newblock In {\em Proceedings of the Third ACM International Conference on AI in Finance}, ICAIF'22, page 428–436, New York, NY, USA, 2022. Association for Computing Machinery.
	
	\bibitem{lorenz1963deterministic}
	Lorenz, E. N.
	\newblock Deterministic nonperiodic flow.
	\newblock {\em Journal of Atmospheric Sciences} 20(2), 130--141 (1963).
	
	\bibitem{may1976simple}
	May, R. M.
	\newblock Simple mathematical models with very complicated dynamics.
	\newblock {\em Nature} 261(5560), 459--467 (1976).
	
	\bibitem[Wang, et~al.]{wang2024behavioral}
	Wang, C., Wang,  C. W., Zhao, Y., Zeng, S. R., Zhang, W., Ronghui Ning \& Jiang, C. J.
	\newblock Next-Generation simulation illuminates scientific problems of organised complexity
	\newblock Preprint at https://arxiv.org/abs/2401.09851 (2024).
	
	\bibitem{matthews2021evolution}
	Matthews, G., Hancock, P. A., Lin, J. C.,Panganiban, A. R., Reinerman-Jones, L. E., Szalma, J. L. \& Wohleber, R. W.
	\newblock Evolution and revolution: Personality research for the coming world of robots, artificial intelligence, and autonomous systems.
	\newblock {\em Personality and Individual Differences} 169, 109969 (2021).
	
	\bibitem{huang2020challenges}
	Huang, M., Zhu, X. Y. \& Gao, J. F.
	\newblock Challenges in building intelligent open-domain dialog systems.
	\newblock {\em ACM Transactions on Information Systems (TOIS)} 38(3), 1--32 (2020).
	
	\bibitem{NMI1}
	Deng, J., Gu, M., Zhang, P. et al. 
	\newblock Nanobody–antigen interaction prediction with ensemble deep learning and prompt-based protein language models. 
	\newblock {\em Nat. Mach. Intell.} 6, 1594–-1604 (2024).
	
	\bibitem{NMI2}
	Jablonka, K.M., Schwaller, P., Ortega-Guerrero, A. et al.
	\newblock Leveraging large language models for predictive chemistry.
	\newblock {\em Nat. Mach. Intell.} 6, 161–-169 (2024).
	
	\bibitem{NMI3}
	Schramowski, P., Turan, C., Andersen, N. et al.
	\newblock Large pre-trained language models contain human-like biases of what is right and wrong to do.
	\newblock {\em Nat. Mach. Intell.} 4, 258--268 (2022).
	
	\bibitem{brown2020language}
	Brown, T. B., Mann, B., Ryder, N. \& et al.
	\newblock Language models are few-shot learners.
	\newblock In {\em Proceedings of the 34th International Conference on Neural Information Processing Systems}, NIPS'20, 1--25, Red Hook, NY, USA, 2020. Curran Associates Inc.
	
	\bibitem[Kahneman, D. \& Tversky, A.]{kahneman1979prospect}
	Kahneman, D. \& Tversky, A.
	\newblock Prospect theory: An analysis of decision under risk.
	\newblock {\em Econometrica: Journal of the Econometric Society} 47(2), 363--391 (1979).
	
	\bibitem{park2023generative}
	Park, J. S., O'Brien, J., Cai, C. J., Morris, M. R., Liang, P. \& Bernstein, M. S.
	\newblock Generative agents: Interactive simulacra of human behavior.
	\newblock In {\em Proceedings of the 36th Annual ACM Symposium on User Interface Software and Technology}, UIST '23, New York, NY, USA, 2023. Association for Computing Machinery.
	
	\bibitem{becker1962irrational}
	Becker, G. S.
	\newblock Irrational behavior and economic theory.
	\newblock {\em Journal of Political Economy} 70(1), 1--13 (1962).
	
	\bibitem{odean1998investors}
	Odean, T.
	\newblock Are investors reluctant to realize their losses?
	\newblock {\em The Journal of Finance} 53(5), 1775--1798 (1998).
	
	\bibitem{bellofatto2018subjective}
	Bellofatto, A., D'Hondt, C. \& De~Winne, R.
	\newblock Subjective financial literacy and retail investors’ behavior.
	\newblock {\em Journal of Banking \& Finance} 92, 168--181 (2018).
	
	\bibitem{ansari2024chronos}
	Ansari, A. F., Stella, L., Turkmen, C., et~al.
	\newblock Chronos: Learning the language of time series.
	\newblock Preprint at https://arxiv.org/abs/2403.07815 (2024).
	
	\bibitem{dasdecoder}
	Das, A., Kong, W., Sen, R., et~al.
	\newblock A decoder-only foundation model for time-series forecasting.
	\newblock Preprint at https://arxiv.org/abs/2310.10688 (2023).
	
	\bibitem{woo2024unified}
	Woo, G., Liu, C., Kumar, A., et~al.
	\newblock Unified Training of Universal Time Series Forecasting Transformers
	\newblock Preprint at https://arxiv.org/abs/2402.02592 (2024).
	
	\bibitem[Liu, et al.]{liu2024large}
	Liu, R., Geng, J. Y., Peterson, J. C., Sucholutsky, I. \& Griffiths, T. L.
	\newblock Large language models assume people are more rational than we really are.
	\newblock Preprint at https://arxiv.org/abs/2406.17055 (2024).
	
	\bibitem{wang2021ernie}
	Wang, S., Sun, Y., Xiang, Y., et~al.
	\newblock Ernie 3.0 titan: Exploring larger-scale knowledge enhanced pre-training for language understanding and generation.
	\newblock Preprint at https://arxiv.org/abs/2112.12731 (2021).
	
	\bibitem{deepseek-v2}
	DeepSeek-AI, Liu, A., Feng, B. \& et al.
	\newblock Deepseek-v2: A strong, economical, and efficient mixture-of-experts language model.
	\newblock Preprint at https://arxiv.org/abs/2405.04434 (2024).
	
	\bibitem{qwen}
	Bai, J. Z., Bai, S., Chu, Y. F. \& et al.
	\newblock Qwen technical report.
	\newblock Perprint at https://arxiv.org/abs/2309.16609 (2023).
	
	\bibitem{weaver1948science}
	Weaver, W.
	\newblock Science and complexity.
	\newblock {\em American Scientist} 36(4), 536--544 (1948).
	
	\bibitem{holtzman2019curious}
	Holtzman, A., Buys, J., Du, L., Forbes, M. \& Choi, Y.
	\newblock The curious case of neural text degeneration.
	\newblock Perprint at https://arxiv.org/abs/1904.09751 (2019).
	
	\bibitem{fernandez2008war}
	Fernandez, V.
	\newblock The war on terror and its impact on the long-term volatility of financial markets.
	\newblock {\em International Review of Financial Analysis} 17(1), 1--26 (2008).
	
	\bibitem{hossain2024impact}
	Hossain, A. T., Masum, A., \& Saadi, S.
	\newblock The impact of geopolitical risks on foreign exchange markets: Evidence from the russia--ukraine war.
	\newblock {\em Finance Research Letters} 59, 104750 (2024).
	
	\bibitem{pan2024very}
	Pan, X. C., Gao, D. W., Xie, Y. X., Wei, Z. W., Li, Y. L., Ding, B. L., Wen, J. R. \& Zhou, J. R.
	\newblock Very large-scale multi-agent simulation in agentscope.
	\newblock Preprint at https://arxiv.org/abs/2407.17789 (2024).
	
	\bibitem{shanahan2023role}
	Shanahan, M., McDonell, K. \& Reynolds, L.
	\newblock Role play with large language models.
	\newblock {\em Nature} 623(7987), 493--498 (2023).
	
	\bibitem{gao2023retrieval}
	Gao, Y. F., Xiong, Y., Gao, X. Y., Jia, K. X., Pan, J. L., Bi, Y. X., Dai, Y., Sun, J. W. \& Wang, H. F.
	\newblock Retrieval-augmented generation for large language models: A survey.
	\newblock Perprint at https://arxiv.org/abs/2312.10997 (2023)
	
	\bibitem[Li, et al.]{li2024ra}
	Li, J. T., Lei, Y., Bian, Y. X., Cheng, D. W., Ding, Z. J. \& Jiang, C. J.
	\newblock Ra-cfgpt: Chinese financial assistant with retrieval-augmented large language model.
	\newblock {\em Frontiers of Computer Science} 18(5), 185350 (2024).
	
	
\end{thebibliography}

	\appendix
	
	\section{Notations of Simulation Procedure}
	
		\begin{table}[htbp]
			\caption{Notations of simulation procedure}
			\centering
			\begin{tabular}{cc}
					\toprule
					Notation & Description\\
					\midrule
					$\mathcal{A}\{a_i\}$ & The set of agents\\
					$\mathcal{E}$ & The engine\\
					$n$ & The number of rounds\\
					$\mathbf{A}$ & The set of actions and interactions in whole procedure\\
					$\mathbf{T}\{t_i\}$ & The set of order requests in a specific round\\
					$\mathbf{T}_{success}$ & The set of successfully matched transactions\\
					$\mathbf{T}_{w}$ & The set of order requests to be withdrawn\\
					$U_i$ & The information input by the user in round $i$\\
					\bottomrule
				\end{tabular}
			\label{tab:notations}
		\end{table}
	
\end{document}