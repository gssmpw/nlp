\documentclass[12pt]{article}
%\usepackage{algorithm}
%\usepackage{algpseudocode}
\usepackage{amsmath, amssymb, amsthm}
\usepackage{float, fullpage, graphicx, multirow,parskip, subcaption, setspace}
\usepackage{comment}
\usepackage{url}
\usepackage{enumitem}
\usepackage{tikz}
\usepackage{bm, bbm}
\usetikzlibrary{shapes.geometric, positioning}
\usetikzlibrary{quotes, angles}
\usepackage{rotating}
\usepackage{hyperref}
\usepackage{natbib}
\usepackage{placeins}
\usepackage{cleveref}
\usepackage{algorithm}
\usepackage{algorithmic}
%\RestyleAlgo{ruled}

\definecolor{SkyBlue}{RGB}{14, 118, 188}
\definecolor{BrightRed}{RGB}{223,82, 78}


\hypersetup{pdfborder = {0 0 0.5 [3 3]}, colorlinks = true, linkcolor = BrightRed, citecolor = SkyBlue}

\bibliographystyle{apalike}

\DeclareMathOperator{\tr}{tr}
\DeclareMathOperator{\sign}{sign}
\DeclareMathOperator{\diag}{diag}
\DeclareMathOperator{\vect}{vec}
\DeclareMathOperator{\GA}{Gamma}
\DeclareMathOperator{\BER}{Bernoulli}
\DeclareMathOperator{\BA}{Beta}
\DeclareMathOperator{\Pois}{Poisson}
\DeclareMathOperator{\Multinomial}{Multinomial}
%\DeclareMathOperator*{\argmax}{arg\,max}
%\DeclareMathOperator*{\argmin}{arg\,min}
\newtheorem{myTheorem}{Theorem}
\newtheorem{myLemma}{Lemma}
\def\R{\mathbb{R}}

\newcommand{\embd}{\bm{\Phi}}
\newcommand{\nonlinembdplain}{\bm{\Psi}}

\newcommand{\nonlinembd}[2]{\nonlinembdplain_{#1,#2}}

\newcommand{\lastembd}[2]{\embd_{#1,#2}}
\newcommand{\lastweight}{\bm{\beta}_{\text{-1}}}
\newcommand{\truereward}{r}
\newcommand{\preference}[1]{h_{#1}}
\newcommand{\sigmoid}{\sigma}
\newcommand{\pairidx}{i}
\newcommand{\pairidxalt}{j}
\newcommand{\totalpairs}{I}
\newcommand{\fisherinfo}{\mathcal{I}}

\newcommand{\prompt}{x}
\newcommand{\response}{y}
\newcommand{\chosenset}{\mathcal{C}}
\newcommand{\budget}{c}

\newcommand{\poolset}{\mathcal{P}}
\newcommand{\labeledset}{\mathcal{D}}
\newcommand{\alstepidx}{s}
\newcommand{\model}[1]{\mathcal{M}_{#1}}
\newcommand{\scoringrule}{\mathcal{S}}

\DeclareMathOperator{\argmax}{argmax}
\newcommand{\levelset}{\text{entropy}}
\newcommand{\dopt}{\text{dopt}}
\newcommand{\maxdiff}{\text{maxdiff}}
\newcommand{\coreset}{\text{coreset}}
\newcommand{\XtX}{\text{XtX}}
\newcommand{\batchbald}{\text{bBALD}}

%%%%%%%%%%%%%%%%%%%%%%%%%%%%%%%%
% THEOREMS
%%%%%%%%%%%%%%%%%%%%%%%%%%%%%%%%
\theoremstyle{plain}
\newtheorem{theorem}{Theorem}[section]
\newtheorem{proposition}[theorem]{Proposition}
\newtheorem{lemma}[theorem]{Lemma}
\newtheorem{corollary}[theorem]{Corollary}
\theoremstyle{definition}
\newtheorem{definition}[theorem]{Definition}
\newtheorem{assumption}[theorem]{Assumption}
\theoremstyle{remark}
\newtheorem{remark}[theorem]{Remark}


\newcommand\numberthis{\addtocounter{equation}{1}\tag{\theequation}}
\newcommand{\numbereqn}{\addtocounter{equation}{1}\tag{\theequation}} % use \numberthis to add number in align* mode
\newcommand{\skd}[1]{\textcolor{cyan}{\small [skd]: #1}}
\newcommand{\yunyi}[1]{{\textcolor{blue}{\small [ys]: #1}}}

\def\keywordname{{\bfseries \emph Keywords}}%
\def\keywords#1{\par\addvspace\medskipamount{\rightskip=0pt plus1cm
\def\and{\ifhmode\unskip\nobreak\fi\ $\cdot$
}\noindent\keywordname\enspace\ignorespaces#1\par}}

\onehalfspacing

% \title{Active Reward Modeling: Adaptive Preference Labeling for Large Language Model Alignment}
\title{Reviving The Classics: Active Reward Modeling in Large Language Model Alignment}
%\author{ Yunyi Shen\thanks{YS and HS contributed equally to this paper.}~~~ Hao Sun\footnotemark[1]~~~ Jean-Francois Ton \\
%MIT, Cambridge, ByteDance Research\\
%\texttt{yshen99@mit.edu, hs789@cam.ac.uk,  jeanfrancois@bytedance.com}
%}
\author{
Yunyi Shen\thanks{YS and HS contributed equally to this paper.}~~\thanks{Massachusetts Institute of Technology, \texttt{yshen99@mit.edu}}\and
Hao Sun\footnotemark[1]~~\thanks{University of Cambridge, \texttt{hs789@cam.ac.uk}}\and
Jean-Fran\c cois Ton\thanks{ByteDance Research, \texttt{jeanfrancois@bytedance.com}}
}

\begin{document}
\def\bY{\bm{Y}}
\def\by{\bm{y}} % vector of all observations



\def\bz{\bm{z}}
\def\bX{\bm{X}}
\def\bx{\bm{x}} % vector of single set of covariates

\def\R{\mathbb{R}}
\def\N{\mathcal{N}}
\def\P{\mathbb{P}}
\def\E{\mathbb{E}}

\def\Xcal{\mathcal{X}}





\maketitle

\begin{abstract}
\begin{abstract}
Retrieval-Augmented Generation (RAG) is often used with Large Language Models (LLMs) to infuse domain knowledge or user-specific information. In RAG, given a user query, a retriever extracts chunks of relevant text from a knowledge base. These chunks are sent to an LLM as part of the input prompt. Typically, any given chunk is repeatedly retrieved across user questions. However, currently, for every question, attention-layers in LLMs fully compute the key values (KVs) repeatedly for the input chunks, as state-of-the-art methods cannot reuse KV-caches when chunks appear at arbitrary locations with arbitrary contexts. Naive reuse leads to output quality degradation.  This leads to potentially redundant computations on expensive GPUs and increases latency. In this work, we propose \sys, a system for managing and reusing precomputed KVs corresponding to the text chunks (we call \textit{chunk-caches}) in RAG-based systems. We present how to identify \hl{\textit{chunk-caches} that are reusable}, how to efficiently perform a small fraction of recomputation to \textit{fix} the cache to maintain output quality, and how to efficiently store and evict \textit{chunk-caches} in the hardware for maximizing reuse while masking any overheads. With real production workloads as well as synthetic datasets, we show that \sys reduces redundant computation by \textbf{51\%} over SOTA prefix-caching and \textbf{75\%} over full recomputation.
\hl{Additionally, with continuous batching on a real production workload, we get a \textbf{1.6$\times$} speedup in throughput and a \textbf{2$\times$} reduction in end-to-end response latency over prefix-caching while maintaining quality, for both the \llama-3-8B and \llama-3-70B models. 
}
\end{abstract}





\end{abstract}

\section{Introduction}
\section{Introduction}
\label{sec:intro}

\begin{figure*}[tb]
    \centering
    \includegraphics[width=0.848\linewidth]{figs/circuitnn.pdf} 
    \caption{Illustration of differentiable CircuitNN. CircuitNN is designed based on differentiable NAND gates. After DAS is guided by PI and PO pairs of the truth table, CircuitNN can get the precise circuit architecture logic equivalent to the truth table.}
    \label{fig:circuitnn}
\end{figure*}

% 1. Describe the importance of logic synthesis
% 2. Existing Problems
% (a) Neural Architecture Search: Unstable, Predefined Setting, etc.
% (b) Circuit Generation: Probabilistic Model, Logic Equivalence

With the rapid advancement of technology, the scale of integrated circuits (ICs) has expanded exponentially. 
This expansion has introduced significant challenges in chip manufacturing, particularly concerning power and area metrics.
A primary objective in IC design is achieving the same circuit function with fewer transistors, thereby reducing power usage and area occupancy.

Logic synthesis~\cite{hachtel2005logicsynth}, a critical step in electronic design automation (EDA), transforms behavioral-level circuit designs into optimized gate-level circuits, ultimately yielding the final IC layout. 
The primary goal of logic synthesis is to identify the physical implementation with the fewest gates for a given circuit function. 
This task constitutes a challenging NP-hard combinatorial optimization problem. 
Current logic synthesis tools~\cite{brayton2010abc, wolf2013yosys} rely on human-designed heuristics, often leading to sub-optimal outcomes.

Differentiable architecture search (DAS) techniques~\cite{liu2018darts, chu2020darts} offer novel perspectives on addressing challenges in this problem.
Circuit functions can be represented through truth tables, which map binary inputs to their corresponding outputs. 
Truth tables provide a precise representation of input-output relationships, ensuring the design of functionally equivalent circuits.
Inspired by this, researchers~\cite{deepmind2024ai4sys, wang2024tnet} have begun exploring the application of DAS to synthesize circuits directly from truth tables.
Specifically, \citet{deepmind2024ai4sys} proposed CircuitNN, a framework that learns differentiable connection structures with logic gates, enabling the automatic generation of logic circuits from truth tables.
This approach significantly reduces the complexity of traditional circuit generation. 
Building on this, \citet{wang2024tnet} introduced T-Net, a triangle-shaped variant of CircuitNN, incorporating regularization techniques to enhance the efficiency of DAS.

Despite these advancements, several challenges remain. 
The computational complexity of DAS grows quadratically with the number of gates, posing scalability issues.
Although triangle-shaped architecture~\cite{wang2024tnet} partially mitigates this problem, redundancy persists. 
%Additionally, DAS is susceptible to converging to local optima, limiting the ability to search architectures that satisfy the given truth tables~\cite{liu2018darts}. 
%Furthermore, hyperparameters (network depth and layer width) require extensive searches, introducing complexity and prolonging the synthesis process. 
Additionally, DAS is susceptible to converging to local optima~\cite{liu2018darts} and hyperparameters (network depth and layer width) require extensive searches. 
The challenges arise from the vast search space in DAS. 
% Even with predefined settings for CircuitNN, finding a configuration that meets the truth table requires extensive trial and error during the DAS process. 
Intuitively, limiting the search space through predefined parameters (network depth, gates per layer, and connection probabilities) can significantly reduce the complexity.

Recent advances~\cite{openai2023gpt4, abramson2024alphafold3, esser2024sd3, li2024mar} in conditional generative models have demonstrated remarkable performance across language, vision, and graph generation tasks. 
Motivated by these developments, we propose a novel approach to circuit generation that generates preliminary circuit structures to guide DAS in generating refined circuits matching specified truth tables. 
Firstly, we introduce CircuitVQ, a tokenizer with a discrete codebook for circuit tokenization. 
Built upon our Circuit AutoEncoder framework~\cite{hou2022graphmae,li2023maskgae,wu2025mgvga}, CircuitVQ is trained through a circuit reconstruction task. 
Specifically, the CircuitVQ encoder encodes input circuits into discrete tokens using a learnable codebook, while the decoder reconstructs the circuit adjacency matrix based on these tokens.
Subsequently, the CircuitVQ encoder serves as a circuit tokenizer for CircuitAR pretraining, which employs a masked autoregressive modeling paradigm~\cite{chang2022maskgit, li2023mage}. 
In this process, the discrete codes function as supervision signals. 
After training, CircuitAR can generate discrete tokens progressively, which can be decoded into initial circuit structures by the decoder of the CircuitVQ. 
These prior insights can guide DAS in producing refined circuits that match the target truth tables precisely.

Our key contributions can be summarized as follows:
\begin{itemize}
\item We introduce CircuitVQ, a circuit tokenizer that facilitates graph autoregressive modeling for circuit generation, based on our Circuit AutoEncoder framework;
\item Develop CircuitAR, a model trained using masked autoregressive modeling, which generates initial circuit structures conditioned on given truth tables;
\item Propose a refinement framework that integrates differentiable architecture search to produce functionally equivalent circuits guided by target truth tables;
\item Comprehensive experiments demonstrating the scalability and capability emergence of our CircuitAR and the superior performance of the proposed circuit generation approach.
\end{itemize}

% Motivation
% (a) Diffusion (Vision, Graph), Autoregressive (Language, Vision)
% (b) Circuit Generation for Predefined Setting
% (c) Neural Architecture Search for Strict Logic Equivalence

% Contribution
% (a) Circuit Tokenizer (new transformer arch, training strategy)
% (b) CircuitAR (train and gen strategies, post-ar strategy)
% (c) Extensive Evaluation including BitD (Bit Distance) for Scalability


%\newpage
\section{Background and setup}
\iffalse
\begin{table*}[htbp]
\tiny
\begin{center}
\begin{tabular}{lccccccccccccc}\toprule
Model, ft setting & mem & \#param & ARC-c & ARC-e & BoolQ & HS & OBQA & PIQA & rte & SIQA & WG & Avg
%\\\cmidrule(lr){2-3}\cmidrule(lr){4-5} \cmidrule(lr){6-7} \cmidrule(lr){8-9}\cmidrule(lr){10-11} \cmidrule(lr){12-13} \cmidrule(lr){14-15} \cmidrule(lr){16-17} 
\\\cmidrule(lr){1-13}
Llama2(7B), LoRA, $r=64$ & 23.46GB & 159.9M(2.37\%) & \textbf{44.97} & 77.02 & 77.43 & \textbf{57.75} & 32.0 & \textbf{78.45} & 62.09 & \textbf{47.75} & 68.75 & 60.69\\
Llama2(7B), SPruFT, $r=128$ & \textbf{17.62GB} & 145.8M(2.16\%) & 43.60 & \textbf{77.26} & \textbf{77.77} & 57.47 & \textbf{32.6} & 78.07 & \textbf{64.98} & 46.67 & \textbf{69.30} & \textbf{60.86} \\\cmidrule(lr){2-13}
Llama2(7B), FA-LoRA, $r=64$ & 17.25GB & 92.8M(1.38\%) & 43.77 & \textbf{77.57} & 77.74 & \textbf{57.45} & 31.0 & 77.86 & \textbf{66.06} & \textbf{47.13} & 69.06 & 60.85\\
Llama2(7B), FA-SPruFT, $r=128$ & \textbf{15.21GB} & 78.6M(1.17\%) & \textbf{43.94} & 77.22 & \textbf{77.83} & 57.11 & \textbf{32.0} & \textbf{78.18} & 65.70 & 46.47 & \textbf{69.38} & \textbf{60.87}\\\midrule
Llama3(8B), LoRA, $r=64$ & 30.37GB & 167.8M(2.09\%) & \textbf{53.07} & \textbf{81.40} & \textbf{82.32} & \textbf{60.67} & 34.2 & \textbf{79.98} & 69.68 & \textbf{48.52} & \textbf{73.56} & \textbf{64.82}\\
Llama3(8B), SPruFT, $r=128$ & \textbf{24.49GB} & 159.4M(1.98\%) & 52.47 & 81.10 & 81.28 & 60.29 & \textbf{34.6} & 79.76 & \textbf{70.04} & 47.75 & 73.24 & 64.50 \\\cmidrule(lr){2-13}
Llama3(8B), FA-LoRA, $r=64$ & 24.55GB & 113.2M(1.41\%) & \textbf{52.47} & \textbf{81.36} & \textbf{82.23} & 60.17 & \textbf{35.0} & \textbf{79.76} & \textbf{70.04} & \textbf{48.31} & \textbf{73.56} & \textbf{64.77}\\
Llama3(8B), FA-SPruFT, $r=128$ & \textbf{22.41GB} & 92.3M(1.15\%) & 52.22 & 81.19 & 81.35 & \textbf{60.20} & 34.2 & 79.71 & 69.31 & 47.13 & 73.01 & 64.26 \\\bottomrule
\end{tabular}
%\vspace{-0.2cm}
\caption{Fine-tuning Llama on Alpaca dataset for 5 epochs and evaluating on 9 tasks from EleutherAI LM Harness. "mem" represents the memory usage, with further details provided in Appendix~\ref{apdx:measure}. \#param is the number of trainable parameters, where the difference of \#param between the two approaches depends on the architecture of Llama, as some layers have $d_{in} \neq d_{out}$. Note that 10 million trainable parameters only account for less than 0.15GB of memory requirement. FA indicates that we freeze attention layers, but not including MLP layers followed by attention blocks. HS, OBQA, and WG represent HellaSwag, OpenBookQA, and WinoGrande datasets. More details of datasets can be found in Appendix~\ref{apdx:data}. The ablation study for different $r$ and the comparison with other LoRA variants can be found in Appendix~\ref{apdx:ablation}. All reported results are accuracies on the corresponding tasks. \textbf{Bold} indicates the best results of two approaches on the same task.} \label{tab:llm} 
\end{center}
\end{table*}
\fi

\begin{table*}[htbp]
\tiny
\begin{center}
\begin{tabular}{lccccccccccccc}\toprule
Model, ft setting & mem & \#param & ARC-c & ARC-e & BoolQ & HS & OBQA & PIQA & rte & SIQA & WG & Avg
\\\cmidrule(lr){1-13}
Llama2(7B)\\ \cmidrule(lr){1-1} 
LoRA, $r=64$ & 23.46GB & 159.9M(2.37\%) & \textbf{44.97} & 77.02 & 77.43 & 57.75 & 32.0 & \textbf{78.45} & 62.09 & 47.75 & 68.75 & 60.69\\
VeRA, $r=64$ & 22.97GB & 1.374M(0.02\%) & 43.26 & 76.43 & 77.40 & 57.26 & 31.6 & 78.02 & 62.09 & 45.85 & 68.75 & 60.07\\
DoRA, $r=64$ & 44.85GB & 161.3M(2.39\%) & 44.71 & 77.02 & 77.55 & \textbf{57.79} & 32.4 & 78.29 & 61.73 & \textbf{47.90} & 68.98 & 60.71\\
RoSA, $r=32, d=1.2\%$ & 44.69GB & 157.7M(2.34\%) & 43.86 & \textbf{77.48} & \textbf{77.86} & 57.42 & 32.2 & 77.97 & 63.90 &  47.29 & 69.06 & 60.78\\
SPruFT, $r=128$ & \textbf{17.62GB} & 145.8M(2.16\%) & 43.60 & 77.26 & 77.77 & 57.47 & \textbf{32.6} & 78.07 & \textbf{64.98} & 46.67 & \textbf{69.30} & \textbf{60.86} %\\\cmidrule(lr){2-13}
%FA-LoRA, $r=64$ & 17.25GB & 92.8M(1.38\%) & 43.77 & \textbf{77.57} & 77.74 & \textbf{57.45} & 31.0 & 77.86 & 66.06 & \textbf{47.13} & 69.06 & 60.85\\
%FA-DoRA, $r=64$ & 30.61GB & 93.6M(1.39\%) & 43.94 & 77.44 & 77.49 & 57.44 & 31.0 & 77.86 & \textbf{66.43} & 46.98 & 69.14 & 60.86\\
%FA-RoSA, $r=32, d=1.2\%$ & 38.34GB & 98.3M(1.46\%) & \textbf{44.28} & 77.02 & 77.68 & 57.22 & 31.0 & 77.97 & 64.26 & 46.32 & 69.22 & 60.55\\
%FA-SPruFT, $r=128$ & \textbf{15.21GB} & 78.6M(1.17\%) & 43.94 & 77.22 & \textbf{77.83} & 57.11 & \textbf{32.0} & \textbf{78.18} & 65.70 & 46.47 & \textbf{69.38} & \textbf{60.87}
\\\midrule
Llama3(8B)\\ \cmidrule(lr){1-1} 
LoRA, $r=64$ & 30.37GB & 167.8M(2.09\%) & 53.07 & 81.40 & 82.32 & 60.67 & 34.2 & 79.98 & 69.68 & 48.52 & 73.56 & 64.82\\
VeRA, $r=64$ & 29.49GB & 1.391M(0.02\%) & 50.26 & 80.30 & 81.41 & 60.16 & 34.4 & 79.60 & 69.31 & 46.93 & 72.77 & 63.90\\
DoRA, $r=64$ & 51.45GB & 169.1M(2.11\%) & \textbf{53.33} & \textbf{81.57} & \textbf{82.45} & \textbf{60.71} & 34.2 & \textbf{80.09} & 69.31 & \textbf{48.67} & \textbf{73.64} & \textbf{64.88}\\
RoSA, $r=32, d=1.2\%$ & 48.40GB & 167.6M(2.09\%) & 51.28 & 81.27 & 81.80 & 60.18 & 34.4 & 79.87 & 69.31 & 47.95 & 73.16 & 64.36\\
SPruFT, $r=128$ & \textbf{24.49GB} & 159.4M(1.98\%) & 52.47 & 81.10 & 81.28 & 60.29 & \textbf{34.6} & 79.76 & \textbf{70.04} & 47.75 & 73.24 & 64.50 %\\\cmidrule(lr){2-13}
%FA-LoRA, $r=64$ & 24.55GB & 113.2M(1.41\%) & 52.47 & 81.36 & 82.23 & 60.17 & \textbf{35.0} & 79.76 & 70.04 & 48.31 & \textbf{73.56} & 64.77\\
%FA-DoRA, $r=64$ & 40.62GB & 114.3M(1.42\%) & \textbf{52.56} & \textbf{81.69} & \textbf{82.26} & \textbf{60.20} & 34.4 & \textbf{79.82} & \textbf{70.40} & \textbf{48.46} & 73.40 & \textbf{64.80}\\
%FA-RoSA, $r=32, d=1.2\%$ & 42.31GB & 124.3M(1.55\%) & 52.22 & 81.19 & 82.05 & 60.11 & 34.4 & 79.76 & 69.31 & 47.70 & 73.16 & 64.43\\
%FA-SPruFT, $r=128$ & \textbf{22.41GB} & 92.3M(1.15\%) & 52.22 & 81.19 & 81.35 & \textbf{60.20} & 34.2 & 79.71 & 69.31 & 47.13 & 73.01 & 64.26 
\\\bottomrule
\end{tabular}
%\vspace{-0.2cm}
\caption{Fine-tuning Llama on Alpaca dataset for 5 epochs and evaluating on 9 tasks from EleutherAI LM Harness. ``mem" represents the memory usage, with further details provided in Appendix~\ref{apdx:measure}. \#param is the number of trainable parameters, where the difference of \#param between the two approaches depends on the architecture of Llama, as some layers have $d_{in} \neq d_{out}$. %FA indicates that we freeze attention layers, but not including MLP layers followed by attention blocks. 
HS, OBQA, and WG represent HellaSwag, OpenBookQA, and WinoGrande datasets. %More details of datasets can be found in Appendix~\ref{apdx:data}. 
The ablation study for different $r$ can be found in Appendix~\ref{apdx:ranks}. All reported results are accuracies on the corresponding tasks. \textbf{Bold} indicates the best result on the same task. } \label{tab:llm} 
\end{center}
\end{table*}

\section{Experimental Setup}\label{sec:setup}

%(0.5 page)
%Why the chosen framework?
%Some prior approaches

%- parameter settings
%- uniform across layers vs greedy ... 
%- potential transformer-specific details

%Equations about what these methods do.. 

%(0.5 page)
%Which NN architectures are used, why?
%Number of parameters, layers, ...

%(Potential prior work on compression -- )

\subsection{Datasets} \label{subsec:dataset}
We use multiple datasets for different tasks. For image classification, we fine-tune models on the training split and evaluate it on the validation split of Tiny-ImageNet~\citep{tavanaei2020embedded}, CIFAR100~\citep{alex2009learning}, and Caltech101~\citep{li_andreeto_ranzato_perona_2022}. For text generation, we fine-tune LLMs on 256 samples from Stanford-Alpaca~\citep{alpaca} and assess zero-shot performance on nine EleutherAI LM Harness tasks~\citep{gao2021framework}. See Appendix~\ref{apdx:data} for details.

\subsection{Models and Baselines} \label{subsec:models}

We fine-tune full-precision Llama-2-7B and Llama-3-8B (float32) using our SPruFT, LoRA~\citep{hulora}, VeRA~\citep{kopiczko2024vera}, DoRA~\citep{liu2024dora}, and RoSA~\citep{nikdan2024rosa}. RoSA is chosen as the representative SFT method and is the only SFT due to the high memory demands of other SFT approaches, while full fine-tuning is excluded for the same reason. We freeze Llama’s classification layers and fine-tune only the linear layers in attention and MLP blocks.

Next, we evaluate importance metrics by fine-tuning Llamas and image models, including DeiT~\citep{touvron2021training}, ViT~\citep{dosovitskiy2020image}, ResNet101~\citep{he2016deep}, and ResNeXt101~\citep{xie2017aggregated} on CIFAR100, Caltech101, and Tiny-ImageNet. For image tasks, we set the fine-tuning ratio at 5\%, meaning the trainable parameters are a total of 5\% of the backbone plus classification layers.

\subsection{Training Details} \label{subsec:training}
Our fine-tuning framework is built on torch-pruning\footnote{Torch-pruning is not required, all their implementations are based on PyTorch.}~\citep{fang2023depgraph}, PyTorch~\citep{paszke2019pytorch}, PyTorch-Image-Models~\citep{rw2019timm}, and HuggingFace Transformers~\citep{wolf2020transformers}. Most experiments run on a single A100-80GB GPU, while DoRA and RoSA use an H100-96GB GPU. We use the Adam optimizer~\citep{KingBa15} and fine-tune all models for a fixed number of epochs without validation-based model selection.

%Structured pruning often considers parameter dependencies in importance evaluation~\citep{liu2021group, fang2023depgraph, ma2023llmpruner}. This becomes the following process in our work: first, searching for dependencies by tracing the computation graph of gradient; next, evaluating the importance of parameter groups; and finally, fine-tuning the parameters within those important groups collectively. For instance, if $\W^{a}_{\cdot j}$ and $\W^{b}_{i\cdot}$ are dependent, where $\W^{a}_{\cdot j}$ is the $j$-th column in parameter matrix (or the $j$-th input channels/features) of layer $a$ and $\W^{b}_{i\cdot}$ is the $i$-th row in parameter matrix (or the $i$-th output channels/features) of layer $b$, then $\W^{a}_{\cdot j}$ and $\W^{b}_{i\cdot}$ will be fine-tuned simultaneously while the corresponding $\M^{a}_{dep}$ for $\W^{a}_{\cdot j}$ becomes column selection matrix and $\W^a_s$ becomes $\W^a_{f,dep}\M^a_{dep}$. Consequently, fine-tuning $2.5\%$ output channels for layer $b$ will result in fine-tuning additional $2.5\%$ input channels in each dependent layer. Therefore, for the $5\%$ of desired fine-tuning ratio, the fine-tuning ratio with considering dependencies is set to $2.5\%$\footnote{In some complex models, considering dependencies results in slightly more than twice the number of trainable parameters. However, in most cases, the factor is 2.} for the approach that includes dependencies. More details for dependencies of NN can be found in Appendix~\ref{apdx:dep}. 

\textbf{Image models}: The learning rate is set to $10^{-4}$ with cosine annealing decay~\citep{loshchilov2017sgdr}, where the minimum learning rate is $10^{-9}$. All image models used in this study are pre-trained on ImageNet. 

\textbf{Llama}: For LoRA and DoRA, we set $\alpha = 16$, a dropout rate of $0.1$, and a learning rate of $10^{-4}$  with linear decay (
$0.01$ decay rate). For SPruFT, we control trainable parameters using rank instead of fine-tuning ratio for direct comparison. The learning rate is $2 \cdot 10^{-5}$ with the same decay settings. Linear decay is applied after a warmup over the first $3$\% of training steps. The maximum sequence length is $2048$, with truncation for longer inputs and padding for shorter ones.



\section{Designing of comparisons}
%!TEX root = ../main.tex

\subsection{Linear BT Regression.}
Consider a simplified case where the true reward function is linear with respect to some intermediate embedding, $\truereward(\lastembd{\pairidx}{1}) = \lastembd{\pairidx}{1}^\top \lastweight$, for weight vector $\lastweight$. We use $\embd$ instead of $\nonlinembdplain$ because the reward may not be linear with respect to the original embedding $\nonlinembdplain$ used in reward modeling, and we wish to avoid confusion. The subscript $-1$ in $\lastweight$ reflects how we will apply these results in practice: $\embd$ represents the output before the final linear layer, and $\lastweight$ corresponds to the weight of this last layer. For now, we assume that this linear feature $\embd$ is known to us. Note that there is no bias term because linear BT is identified only up to translation. 

Under this simplified setting the preference generating process of $\pairidx$th pair $\preference{\pairidx}$ can be simplified to 
\begin{equation}
    \preference{\pairidx} \sim \BER[\sigmoid[(\lastembd{\pairidx}{1}-\lastembd{\pairidx}{2})^\top\lastweight]]
    \label{eq:BT}
\end{equation}
It can be observed that this corresponds to a logistic regression, where the covariates are the difference $\lastembd{\pairidx}{1} - \lastembd{\pairidx}{2}$.

By applying the theory from generalized linear models, we know that the maximum likelihood estimate $\hat{\lastweight}$ is asymptotically Gaussian distributed, with mean $\lastweight$ and covariance matrix $\fisherinfo^{-1}$, where $\fisherinfo$ denotes the Fisher information (FI) matrix \citep[see e.g., ][Ch. 4.5.2]{shao2008mathematical}. For the linear Bradley-Terry model, the FI is
\begin{equation}
    \fisherinfo=\sum_{\pairidx=1}^\totalpairs (\lastembd{\pairidx}{1}-\lastembd{\pairidx}{2})(\lastembd{\pairidx}{1}-\lastembd{\pairidx}{2})^\top p_{\pairidx}(1-p_{\pairidx})
    \label{eq:FI}
\end{equation}
Where $p_{\pairidx} = \sigmoid[(\lastembd{\pairidx}{1} - \lastembd{\pairidx}{2})^\top \lastweight]$, it can be observed that $p_{\pairidx}(1 - p_{\pairidx})$ represents the variance of a Bernoulli random variable.

The Fisher information matrix can be interpreted as the metric tensor in a Riemannian manifold of distributions, where the distance between them is given by the symmetrized KL divergence \citep{costa2015fisher}. FI quantifies the amount of information in the dataset for estimating the parameters $\lastweight$. From a Bayesian perspective, the Bernstein-von Mises theorem \citep[][Ch. 10.2, Thm 10.1]{van2000asymptotic} states that $\fisherinfo^{-1}$ is also the asymptotic covariance matrix of the posterior distribution of $\lastweight$, assuming mild regularity conditions on the prior.

The FI can be viewed as a sum over all independent data points' contribution. For each data point, there are two terms multiplied together: the empirical covariance of embedding differences $(\lastembd{\pairidx}{1} - \lastembd{\pairidx}{2})(\lastembd{\pairidx}{1} - \lastembd{\pairidx}{2})^\top$, and $p_{\pairidx}(1 - p_{\pairidx})$, the variance of the comparison results. \citet{sun2024rethinking} suggested that improving the variance of comparisons can be interpreted as improving annotation quality which can also be seen from FI. 

To make the FI large \cref{eq:FI} an ideal comparison should exhibit both a large variance in the embedding difference (thus $(\lastembd{\pairidx}{1} - \lastembd{\pairidx}{2})(\lastembd{\pairidx}{1} - \lastembd{\pairidx}{2})^\top$ having large eigenvalues) and a high variance in the comparison outcomes (thus $p_{\pairidx}(1 - p_{\pairidx})$ large). This implies that the embedding space should be diverse, such that $\lastembd{\pairidx}{1} - \lastembd{\pairidx}{2}$ captures a wide range of differences, and each comparison should be informative—not too close to 0 or 1. The former encourages exploration within the embedding space, leading to a better regression model, while the latter ensures that comparisons are not trivial, improving sample efficiency. An everyday analogy for comparing non-obvious pairs would be that comparing a world champion to a newbie in chess offers little insight into the abilities of either player.

The FI plays a crucial role in the classical theory of experimental design, both in frequentist and Bayesian frameworks, as highlighted by the Bernstein-von Mises theorem. This leads to a family of design strategies known as alphabetical designs \citep{chaloner1995bayesian, pukelsheim2006optimal}. 

\textbf{(Bayesian) D-optimality \citep{chaloner1995bayesian}.}
%In classic statistics and experimental design literature, one strategy is the use of so-called alphabetical designs \citep{chaloner1995bayesian, pukelsheim2006optimal}, 
The alphabetical designs focus on the (co)variance of either estimating weights $\lastweight$ or making predictions under new embeddings, typically summarized through the covariance matrix. For example, the D-optimal design minimizes the determinant of the (asymptotic) covariance matrix of the last layer weights, $\lastweight$. Since $|\fisherinfo^{-1}| = 1 / |\fisherinfo|$, this is equivalent to maximizing the determinant of the FI.

The Bayesian variant of D-optimal involves having prior contribution, such as maximizing $|\fisherinfo + I/\sigma^2|$, where $I$ is the identity matrix, to avoid a determinant of zero. This corresponds to the inverse covariance matrix of the Laplace approximation of the posterior of $\lastweight$, assuming a normal prior with variance $\sigma^2$.

A plug-in estimator of $p_{\pairidx}$, $\hat{p}_{\pairidx}$, using the current best model, can be used to estimate the FI \citep{chaloner1995bayesian, pukelsheim2006optimal}. In this approach, the scoring rule is the determinant of the Fisher Information matrix.
\begin{equation}
    \scoringrule_{\dopt}(\chosenset) = \lvert \sum_{\pairidx\in \chosenset}  (\lastembd{\pairidx}{1}-\lastembd{\pairidx}{2})(\lastembd{\pairidx}{1}-\lastembd{\pairidx}{2})^\top \hat{p}_{\pairidx}(1-\hat{p}_{\pairidx})\rvert
\end{equation}
In experiments, we refer to this strategy as \texttt{D-opt}. Other forms of optimality also exist, each targeting different summaries of the Fisher Information (FI), such as A-optimality, which focuses on minimizing the trace of $\fisherinfo^{-1}$. When the prediction of a new, known embedding is the primary concern, G-optimality aims to minimize the variance of predictions on new embeddings. %In this case, the criterion can be expressed as $(\lastembd{\pairidx}{1} - \lastembd{\pairidx}{2})^\top \fisherinfo^{-1} (\lastembd{\pairidx}{1} - \lastembd{\pairidx}{2})$.

In this work, we suggest using D-optimality because it avoids the need to invert the FI, as required in A-optimality, and doesn't require specifying which samples to predict, as in G-optimality. For readers interested in further details, we refer to \citet{pukelsheim2006optimal} (Ch. 9).

The D-optimality strategy can be made a past-aware version by incorporating previously collected data. The asymptotic covariance of the full data-conditioned posterior is then $(\fisherinfo_{\text{past}} + \fisherinfo)^{-1}$, where $\fisherinfo_{\text{past}}$ is computed using prior data and \cref{eq:FI}. This approach relates to Bayesian methods like Bayesian active learning by disagreement (BALD) \citep{houlsby2011bayesian}, which minimizes posterior entropy. Since Gaussian entropy is proportional to the log-determinant of its covariance. In our experiments, we refer to this variant as \texttt{PA D-opt}.

Next, we review some other strategies that can be applied to BT models.

\textbf{Entropy sampling \citep{settles2009active, muldrew2024active}.}
This strategy aims to select samples about which the current model is most uncertain \citep{settles2009active}. In the context of binary preference modeling, this corresponds to choosing data whose predictions $\hat{p}_{\pairidx}$ are closest to 0.5, effectively exploring the level set of the reward. This is similar to a binary classification problem where the goal is to explore the decision boundary. This approach was also proposed by \citet{muldrew2024active} as maximizing predictive entropy. The scoring rule is then,
\begin{equation}
    \scoringrule_{\levelset}(\chosenset)=\sum_{\pairidx\in \chosenset} \left[-\hat{p}_{\pairidx}\log \hat{p}_{\pairidx} -(1-\hat{p}_{\pairidx})\log(1-\hat{p}_{\pairidx})\right]
\end{equation}
Since the entropy of a Bernoulli distribution reaches its maximum when $p = 0.5$, this approach is equivalent to selecting the top $\budget$ pairs where the predicted probability is closest to 0.5. In our experiments, we refer to this method as \texttt{Entropy}.

\textbf{Maximum difference \citep{muldrew2024active}.}
Contrasting with entropy sampling, this strategy focuses on comparing samples that the current reward model predicts to be the best and the worst, corresponding to probabilities close to 0 or 1. This approach was used by \citet{muldrew2024active} to measure model certainties. The scoring rule to be maximized can thus be interpreted as difference in estimated reward $|\hat{\truereward}_{\pairidx,1}-\hat{\truereward}_{\pairidx,2}|$.
\begin{equation}
    \scoringrule_{\maxdiff}(\chosenset)=\sum_{\pairidx\in \chosenset} |\hat{\truereward}_{\pairidx,1}-\hat{\truereward}_{\pairidx,2}|%\left[\hat{p}_{\pairidx}\log \hat{p}_{\pairidx} +(1-\hat{p}_{\pairidx})\log(1-\hat{p}_{\pairidx})\right]
\end{equation}
This strategy encourages exploration in the \textit{reward} space rather than the embedding space. It is sometimes used in active learning when the goal is to identify positive examples rather than the best classification \citep{settles2009active}. This justifies its use in reward modeling, where the goal is to obtain responses that yield better rewards in downstream tasks. In our experiments, we refer to this method as \texttt{Maxdiff}.

\textbf{Optimizing design matrix \citep{mukherjee2024optimal}.} 
This strategy focuses on finding the best collection of embeddings, or the design matrix in statistics terms $\lastembd{\pairidx}{1} - \lastembd{\pairidx}{2}$, without looking at model predictions. A common objective is to optimize the covariance matrix of the designs, $\Sigma = \sum_{\pairidx=1}^\totalpairs (\lastembd{\pairidx}{1} - \lastembd{\pairidx}{2})(\lastembd{\pairidx}{1} - \lastembd{\pairidx}{2})^\top$. One approach is to maximize the determinant of $\Sigma$, $|\Sigma|$, which encourages exploration over a large space of embedding differences. In fact, if we assume a linear regression model with additive Gaussian noise instead of logistic regression, this covariance matrix corresponds to the Fisher Information matrix of the regression coefficients, and this strategy aligns with the D-optimal design. The scoring rule is
\begin{equation}
    \scoringrule_{\XtX}(\chosenset) = \lvert \sum_{\pairidx\in \chosenset} (\lastembd{\pairidx}{1}-\lastembd{\pairidx}{2})(\lastembd{\pairidx}{1}-\lastembd{\pairidx}{2})^\top\rvert
\end{equation}
\citet{mukherjee2024optimal} used a similar strategy for a different type of preference data that is not purely binary. In our experiments, we refer to this method as \texttt{det(XtX)}, for the determinant of $X^\top X$.

\textbf{Coreset \citep{huggins2016coresets,munteanu2018coresets}.}
Instead of minimizing uncertainty in parameter estimations, the Coreset strategy aims to find a small subset of samples such that the trained model closely approximates the one trained on the full dataset, effectively transforming the problem into a sparse approximation task on weighting data points. The Coreset method for logistic regression has been studied recently by \citet{munteanu2018coresets} and \citet{huggins2016coresets} in both frequentist and Bayesian settings. In our experiment, we adopted the method of \citet{huggins2016coresets}. The scoring rule does not have a simple closed-form solution, so we refer interested readers to \citet{huggins2016coresets} and denote it as $\scoringrule_{\coreset}$. In our experiments, we refer to this method as \texttt{Coreset}.

\textbf{BALD and batchBALD \citep{houlsby2011bayesian, kirsch2019batchbald}.} When transitioning from frequentist to Bayesian framework, BALD \citep{houlsby2011bayesian} and BatchBALD \citep{kirsch2019batchbald} select data with high mutual information between the candidate batch's prediction and model parameters, making the data more informative. \citet{houlsby2011bayesian} showed that this approach maximizes expected posterior entropy reduction. This strategy applies to preference learning \citep{houlsby2011bayesian} but requires a Bayesian model. We denote the corresponding scoring rule as $\scoringrule_{\batchbald}$. In our experiments, we refer to this method as \texttt{BatchBald}. We used implementation in \texttt{batchbald\_redux} \citep{kirsch2019batchbald}.

This strategy relates to Bayesian D-optimality; when posterior entropy is tractable, it can be minimized directly instead of relying on approximations from \citet{houlsby2011bayesian}. If the posterior is Gaussian, entropy is proportional to the log-determinant of its covariance, leading to D-optimality.

\subsection{Gradient Approximation for Combinatorial Optimization.}
In some strategies, we select a data subset to maximize information criteria like the determinant of FI or the design matrix. These often lead to intractable combinatorial optimization problems. To address this, we use the sensitivity approach from the coreset and robustness literature \citep{huggins2016coresets, campbell2018bayesian, campbell2019automated}. When the information criteria are expressed as a nonlinear function over sum of data point contributions, i.e., $\scoringrule = f(\sum_{\pairidx} c_\pairidx)$, where each data point contributes $c_\pairidx$, we introduce weights $w_i$, allowing the score to be rewritten as $\scoringrule(\bm{w}) = f(\sum_{\pairidx} w_i c_\pairidx)$. For instance, the D-optimal score expresses the determinant of FI of a subset $\chosenset$ as a weighted sum.
\begin{equation}
%\small
    \scoringrule_{\dopt}(\bm{w}) = \lvert \sum_{\pairidx} w_\pairidx (\lastembd{\pairidx}{1}-\lastembd{\pairidx}{2})(\lastembd{\pairidx}{1}-\lastembd{\pairidx}{2})^\top \hat{p}_{\pairidx}(1-\hat{p}_{\pairidx})\rvert
\end{equation}
Each candidate pair is assigned a weight $w_i = 1_{i\in \chosenset}$. Selecting a subset $\chosenset$ to maximize $\scoringrule_{\dopt}$ is equivalent to finding a sparse 0-1 weight vector $\bm{w}$ that maximizes $\scoringrule_{\dopt}(\bm{w})$.

To approximate the optimization, we treat $\bm{w}$ as continuous and perform a Taylor expansion around $\bm{w} = \bm{1}$, the all 1 vector, i.e., all data points are included.
\begin{equation}
   \scoringrule(\bm w)\approx \scoringrule(\bm 1) - (\bm 1-\bm{w})^\top \nabla_{\bm w} \scoringrule(\bm w)|_{\bm w=\bm 1}
   \label{eq:taylorexpansion}
\end{equation}
The approximated optimization problem becomes
\begin{equation}
    \argmax_{\bm w}\scoringrule(\bm w)\approx \argmax_{\bm w} \bm{w}^\top \nabla_{\bm w} \scoringrule(\bm w)|_{\bm w=\bm 1}
\end{equation}
A sparse 0-1 valued vector $\bm w$ that optimizes the right-hand side of \cref{eq:taylorexpansion} can be obtained by selecting the data points with the largest gradient, $\nabla_{\bm w} \scoringrule(\bm w) \big|_{\bm w = \bm 1}$. A probabilistic approach, when all gradients are positive, involves sampling according to the weights given by $\nabla_{\bm w} \scoringrule(\bm w) \big|_{\bm w = \bm 1}$. 



\subsection{Handling nonlinear model using last layer features.}
For nonlinear reward models in \cref{eq:BT}, the dependencies on embeddings become more complex. Strategies like maximum difference and entropy sampling, which depend only on model predictions, remain unaffected by the architecture, while batchBALD is designed for (Bayesian) deep models. Feature-based methods like coreset or D-optimal need adaptation. A heuristic from the Bayesian last layer \citep{tran2019bayesian} and computer vision literature \citet{sener2017active} suggests using the last layer before the linear output as a feature, applying linear strategies to it.
\begin{equation}
    \truereward(\nonlinembdplain) = F_{\bm \theta}(\nonlinembdplain)^\top \lastweight
\end{equation}
For some nonlinear function $F_{\bm \theta}$ parameterized by $\bm\theta$, e.g., an MLP and $\embd := F_{\bm \theta}(\nonlinembdplain)$. We apply methods in linear settings with features $F_{\bm \theta}(\nonlinembdplain)$. We then train $\bm\theta$ and $\lastweight$ together once data are labeled. In particular, in \citet{sener2017active} the nonlinear function $F_{\bm \theta}$ is a CNN and they took a coreset approach. Here we apply this strategy to the coreset, optimal design matrix and D-optimal setting.  





\section{Illustrative Examples in Dimension Two}
\label{app:2Dilustration}
\textbf{Experiment Setups}
In this experiment, we provide a two-dimensional example of the comparisons made by each strategy. The ground truth reward was defined as the log probability of a mixture of two Gaussians, centered at $(-2.5, -2.5)$ and $(2.5, 2.5)$ with a variance of 0.25. Preference data was simulated using the BT model, and we attempted to learn the reward function with a 3-layer MLP with $16$ hidden units. For each round, $1000$ points were sampled from a standard normal distribution, and $200$ comparisons were selected using different strategies. $4$ rounds are shown in \cref{fig:what_were_compared}.%, with a zoomed-in version in \cref{fig:what_were_compared-big}.
\begin{figure}[htp]
    \centering
    %\vspace{-0.58cm}
    \includegraphics[width=1.0\linewidth]{Figs/2D_GM_illustration.pdf}\vspace{-0.35cm}
    \caption{\small Comparisons drawn by different strategies to learn a 2D bimodal reward function. The heat map showed the estimated functions. Red dots connected by lines are \textbf{selected pairs} and gray dots on the first column are candidate points to choose from.} %\vspace{-0.42cm}
    \label{fig:what_were_compared}
\end{figure}

\textbf{What were compared in dimension two?}
We observed that D-optimal selects diverse samples with many anchoring points, often comparing multiple points to a single one, spreading out the level set in the original space. Entropy sampling, similar to random sampling, focuses on points near reward values, effectively traversing the reward function's level set. Coreset also selects diverse comparisons, though not always among points with similar reward values. The best design matrix method behaves similarly to coreset, emphasizing diversity in comparisons. In contrast, the max difference method tends to compare extreme values with many others, promoting exploration but potentially yielding less informative comparisons. BatchBALD also selects diverse comparisons, though without a clear pattern. These observations suggest that most methods encourage exploration, entropy sampling prioritizes informative comparisons, and D-optimal seeks a balance between the two.
\section{Experiments: Planning outperforms Heuristics}
\label{sec:experiment}

We begin our empirical demonstrations by showcasing the effectiveness of our planning framework on both synthetic and real datasets. We focus on the simplest planning algorithm, 1-step lookaheads (Algorithm~\ref{alg:complete}), and show that even basic planning can hold great promise. 
We illustrate our framework using two uncertainty quantification modules---GPs and 
\ensembles/ \ensembleplus. 

Throughout this section, we focus on evaluating the mean squared error of 
a regression model $\model$,  and develop adaptive policies that minimize uncertainty on $g(f)$ defined in~\eqref{eqn:l2-g-f}.
When GPs provide a valid model of uncertainty, 
our experiments show that our planning framework significantly outperforms other baselines. 
We further demonstrate that our conceptual framework extends to deep learning-based uncertainty quantification methods such as  \ensembleplus while highlighting computational challenges that need to be resolved in order to scale our ideas. 
For simplicity, we assume a naive predictor, i.e., $\psi(\cdot) \equiv 0$. However, we emphasize that this problem is just as complex as if we were using a sophisticated model $\psi(.)$. The performance gap between the algorithms 
primarily depends
on the level  of uncertainty in our prior beliefs.

To evaluate the performance of our algorithm, we benchmark it against several baselines. 
%Active learning baselines use an acquisition function $\ac$ to select points that have the highest   function value: $X\opt_t \in \argmax_{X \in \xpoolj{t}} \ac({X})$ at every step $t$. These methods may also need an UQ module, which we simply use the same UQ module as in our algorithm, and it  outputs $V(X)$ that measures the the uncertainty of each point $X \in \xpoolj{t}$.
Our first set of baselines are from active learning~\citep{AggarwalKoGuHaPh14}:
\\ % \noindent\textbf{Active Learning Heuristics:} 
\textbf{(1)} 
\textsf{Uncertainty Sampling (Static):}  In this approach, we query the samples for which the model is least certain about. Specifically, we estimate the variance of the latent output $f(X)$ for each $X \in \xpool$ using the UQ module and select the top-$K$ points with the highest uncertainty. \\
\textbf{(2)} \textsf{Uncertainty Sampling (Sequential):} This is a greedy heuristic that sequentially selects the points with the highest uncertainty within a batch, while updating the posterior beliefs using pseudo labels from the current posterior state. Unlike \textsf{Uncertainty Sampling (Static)}, this method takes into account the information gained from each point within batch, and hence tries to diversify the selected points within a batch. 

 
We also compare our approach to the  \textbf{(3)} \textsf{Random Sampling}, which selects each batch uniformly at random from the pool. Additionally, we compare solving the planning problem using  \textsf{REINFORCE}-based policy gradients with   $\mathsf{Smoothed\text{-}Autodiff}$ policy gradients.\footnote{Our code repository is available at
  \url{https://github.com/namkoong-lab/adaptive-labeling}.}
%Detailed experimental setups are provided in Section \ref{sec:details-experiments}.

%We repeat all experiments with 10 random seeds.




\begin{figure}[t]
\centering
\begin{minipage}[b]{0.49\textwidth}
\centering
\includegraphics[width=\textwidth, height=5cm]{figures/original_scale/Var_of_l_2_loss.pdf}
\caption{(Synthetic data) Variance of mean squared loss evaluated through the posterior belief $\mu_t$ at each horizon $t$. This is the objective that policy gradient methods like \textsf{REINFORCE} and $\ouralgo$ optimizes. 1-step lookaheads are surprisingly effective even in long horizons.}
\label{fig:var-l2-sim}
\end{minipage}
\hfill
\begin{minipage}[b]{0.49\textwidth}
\centering \includegraphics[width=\textwidth, height=5cm]{figures/original_scale/Error_of_estimated_model_l_2_loss.pdf}
\caption{(Synthetic data) Error between MSE calculated based on collected data $\mc{D}^{0:T}$ vs. population oracle MSE over $\mc{D}_{\rm eval} \sim P_X$. Reducing uncertainty over posteriors directly leads to better OOD evaluations. 1-step lookaheads significantly outperform active learning heuristics in small horizons.}
\label{fig:mean-l2-sim}
\end{minipage}
%\caption{Simulated data for GPs}
%\label{fig:both_plots}
\end{figure}

\subsection{Planning with Gaussian processes}
\label{sec:experiment-plan-GP}
We now briefly describe the data generation process for the GP experiments,  deferring a more detailed discussion of the dataset generation to Section~\ref{sec:details-experiments}. 
We use both the synthetic data and the real data to test our methodology.
For the \emph{simulated data},  we construct a setting where the general population is distributed across \emph{51 non-overlapping clusters} while the initial labeled data $\dtrain$ just comes from one cluster. In contrast, both $\dpool \defeq (\xpool,\ypool),\deval \defeq (\xeval,\yeval)$ are generated   from all the clusters. 
We begin with a low-dimensional scenario, generating a one-dimensional regression setting using a GP. %Gaussian Process (GP).
Although the data-generating process is not known to the algorithms,  we assume that the GP hyperparameters are known to all the algorithms
to ensure fair comparisons. This can be viewed as a setting where our prior is well-specified, allowing us to isolate the effects
of different policy optimization approaches
 without any concerns about the misspecified priors. We select $10$ batches, each of size $K=5$ across $T = 10$ time horizons.

To examine the robustness of our method against the distributional assumptions made  in the simulated case, we then move to a real dataset where the correct prior is not known. We simulate selection bias from the eICU dataset~\citep{PollardJoRaCeMaBa18}, which contains real-world patient data with in-hospital mortality outcomes. 
We conduct a $k$-means clustering to generate 51 clusters and then select data from those clusters. We view this to be a credible replication of practice, as severe distribution shifts are common due to selection bias in clinical labels.  To convert the binary mortality labels into a regression setting, we train a  random forest classifier and fit a GP on predicted scores, which serves as the UQ module for all the algorithms. As before, the task is to select 10 batches, each consisting of 5 samples, across 10 time horizons.

 In Figures~\ref{fig:var-l2-sim} and~\ref{fig:mean-l2-sim}, we present results for the simulated data. 
Figure~\ref{fig:var-l2-sim} shows the variance of $\ell_2$ loss, and Figure~\ref{fig:mean-l2-sim} presents the error in the estimated $\ell_2$ loss using $\mu_t$ (relative to true $\ell_2$ loss, that is unknown to the algorithm). 
As we can see from these plots, our method one-step lookahead  gives substantial improvements  over active learning baselines and random sampling. In addition,
compared to the one-step lookahead planning approach using \textsf{REINFORCE}-based policy gradients, 
we observe that $\mathsf{Smoothed\text{-}Autodiff}$-based policy gradients provide significantly more robust performance over all horizons.

In Figures~\ref{fig:var-l2-real}~and~\ref{fig:mean-l2-real}, we observe similar findings on the eICU data. We see that planning policies (\textsf{REINFORCE} and $\mathsf{Smoothed\text{-}Autodiff}$) consistently outperform other heuristics by a large margin.  Active learning baselines perform poorly in these small-horizon batched problems and can sometimes be even worse than the random search baselines.  Overall, our results show the importance of careful planning in adaptive labeling for reliable model evaluation. 

We offer some intuition as to why one-step lookahead planning may outperform other heuristic algorithms. 
 First,  \textsf{Uncertainty sampling (Static)} while myopically selects the
 top-$K$ inputs with the highest uncertainty, it fails to consider 
the overlap in information content among the ``best” instances; see \citep{AggarwalKoGuHaPh14} for more details. 
In other words,  it might acquire points from the same region with high uncertainty while failing to induce diversity among the batch.
Although \textsf{Uncertainty Sampling (Sequential)} somewhat addresses the issue of information overlap, a significant drawback of 
this algorithm
is the disconnect between the objective we aim to optimize and the algorithm. For example, it might sample from a region with high uncertainty but very low density. 

\begin{figure}[t]
\centering
\begin{minipage}[b]{0.48\textwidth}
\centering
\includegraphics[width=\textwidth, height=5cm]{figures/original_scale/Var_of_l_2_loss_real.pdf}
\caption{(Real-world eICU data) Variance of mean squared loss evaluated through the posterior belief $\mu_t$ at each horizon $t$. Even 1-step lookaheads are extremely effective planners, and auto-differentiation-based pathwise policy gradients provide a reliable optimization algorithm based on low-variance gradient estimates.}
\label{fig:var-l2-real}
\end{minipage}
\hfill
\begin{minipage}[b]{0.48\textwidth}
\centering \includegraphics[width=\textwidth, height=5cm]{figures/original_scale/Error_of_estimated_model_l_2_loss_real.pdf}
\caption{(Real-world eICU data) Error between MSE calculated based on collected data $\mc{D}^{0:T}$ vs. population oracle MSE over $\mc{D}_{\rm eval} \sim P_X$. Reducing uncertainty over posteriors directly leads to better OOD evaluations. Our method significantly outperforms active learning-based heuristics, and random sampling.}
\label{fig:mean-l2-real}
\end{minipage}
%\caption{Real data for GPs}
\end{figure}
 
%\vspace{-1.5cm}
% \begin{wrapfigure}{r}{.32\columnwidth}
%   \vspace{-.5cm} 
%   \centering
% \includegraphics[scale=.29]{figures/Var of l2l_2 loss.pdf}
%   \vspace{-0.2cm}
%   \caption{Results of GP}
% \label{fig:var-l2-gp}
%   \vspace{-0.1cm}
% \end{wrapfigure}


% Attempts have been made  in the past to address these  drawbacks heuristically  (see \citep{AggarwalKoGuHaPh14}). We give a unified computational framework while approaching the problem in a more principled manner and solving it more optimally.




\subsection{Planning with  neural network-based uncertainty quantification methods ($\ensembleplus$)}


We now provide a proof-of-concept that shows the generalizability of our conceptual framework  to the deep learning-based UQ modules, specifically focusing on $\ensembleplus$ due to their previously observed superior performance~\citep{OsbandWenAsDwIbLuRo23}. Recall that implementing our framework with deep learning-based UQ modules  requires us to retrain the model across multiple possible random actions $\bm{a}(\theta)$ sampled from the current policy $\pi_\theta$.
This requires significant computational resources, in sharp contrast to the GPs where the posteriors are in closed form and can be readily updated and differentiated. 

Due to the computational constraints, we test $\ensembleplus$ on a toy setting to demonstrate the generalizability of our framework. We consider a setting where the general population consists of four clusters, while the initial labeled data only comes from one cluster. Again we generate data using GPs.  The task is to select a batch of 2 points in one horizon. We detail the $\ensembleplus$ architecture in Section \ref{sec:details-experiments}, and we assume prior uncertainty to be large (depends on the scaling of the prior generating functions). 
The results are summarized in the Table~\ref{tab:UQ_ensemble}.

% \begin{table}[H]
% \vspace{-10pt}
% \caption{Performance under \ensembleplus as UQ module}
%     \centering
%     \begin{tabular}{|m{3cm}|m{2.5cm}|m{2cm}|} 
%     \hline
%       Algorithm   & Variance of $\loss_2$ loss estimate & Error of $\loss_2$ loss estimate  \\ \hline Random Sampling 
%          & $1710.9 \pm 1352.1$ & $8.67\pm6.62$ 
%       \\ \hline \ouralgo & $1.30 \pm 0.68$ & $0.91\pm0.25$ \\ \hline
%     \end{tabular}
%     \label{tab:UQ_ensemble}
%     %\vspace{-10pt}
% \end{table}




\begin{table}[h]
\vspace{-10pt}
\caption{Performance under \ensembleplus as the UQ module}
\centering
\begin{tabular}{|l|l|l|}
\hline
Algorithm   & Variance of $\loss_2$ loss estimate & Error of $\loss_2$ loss estimate  \\
\hline
\textsf{Random sampling} & 7129.8 $\pm$ 1027.0 & 136.2 $\pm$ 8.28 \\ \hline
\textsf{Uncertainty sampling (Static)} & 10852 $\pm$ 0.0 & 162.156 $\pm$ 0.0 \\ \hline
\textsf{Uncertainty sampling (Sequential)} & 8585.5 $\pm$ 898.9 & 144 $\pm$ 6.93 \\ \hline
\textsf{REINFORCE} & 1697.1 $\pm$ 0.0 & 45.27 $\pm$ 0.0 \\ \hline
\ouralgo & 1697.1 $\pm$ 0.0 & 45.27 $\pm$ 0.0 \\ \hline
\end{tabular}
%\caption{Comparison of different algorithms based on variance   and   error in $\ell_2$ loss estimation with Ensemble $+$ as the UQ module. Our results demonstrate that {\ouralgo} and REINFORCE outperformthe other active learning based heuristics, confirming the benefits of our MDP formulation for the adaptive labeling problem, as also demonstrated in Section 4.\\
%\footnotesize{Experimental details: We use Gaussian Processes as our data generating process, GP parameters are the same as in Section D.3.  The task is to select a batch of 2 points along one horizon.The marginal distribution $p_X$ has 4 \textit{non-overlapping} clusters. Initial data comes from one cluster, while pool and evaluation points comes from all the clusters. We have $20$ initial labeled data points, $10$ pool points, and $252$ evaluation points.  Training procedures are similar to the one in Section D.3.} }
\label{tab:UQ_ensemble}
\end{table}



% We faced  issues in scaling up these experiments which will be our focus in the future. 





% \begin{itemize}
%     \item Posteriors should be consistent. Two dimensions: even with less training,  
%     \item the inference should be  fast enough
% \end{itemize}


% Potential research directions for uncertainty quantification

% In this section we consider a simple setting We consider a simpler setting and 


% For synthetic dataset generation, we use ...... For real datasets, we use ...... We compare our methodolgy to several baselines ()    This Section is structured as follows:
% \begin{itemize}
%     \item \textbf{GPs, square loss objective} (Section \ref{}): 
%     %the broad aim of the experiments  in this section is to isolate the performance of our methodology without any concerns for the inefficiencies induced due to a mis-specified prior or imperfect posterior inference. To accomplish this we generate synthetic datasets using GPs (detailed later). We use the well specified prior (GPs - with same hyperparameter setting) as our UQ module.   
%      As GPs provide differentaible posterior inference - any errors induced due to imperfect posterior updates are also isolated. We note that under this setting
%      \item In Section\ref{} we demonstrate why our methodology performs better than other baselines - by devising various synthetic experiments ()
%     \item  \textbf{UQ Benchmarking }(Section \ref{}): Before diving into the experiments using $\ensembleplus$ and ENNs,  we showcase our benchmarking experiments in Section \ref{}. We use real datasets We observe that ENNs perform better
%      \item \textbf{Ensemble $+$}, objective: recall, accuracy
%     \item \textbf{ENN}, objective: recall, accuracy
% \end{itemize}




% In Section {}, we test 
% \subsection{Experimental details}

% \begin{itemize}
%     \item UQ methodologies - GPs, ENNs
%     \item Objectives - Recall,  ATE
%     \item Datasets - ATE-synthetic datasets, Recall-synthetic, real datasets
%     \item Baselines - 
%     \begin{itemize}
%         \item Random sampling
%         \item Active learning - Uncertainty based sampling - In regression setting almost all of the 
%         \item Myopic greedy - Greedy Batch based sampling
%         \item Policy Gradient
%     \end{itemize}
    
% \end{itemize}

% \subsection{Experiments}
%     \begin{itemize}
%     \item GPs with square loss
%     \item Benchmarking ENN
%         \item ENNs with ATE
%         \item ENNs with Recall
%     \end{itemize}

% \subsection{Benefits over other algorithms - intuition and experiments}

%Active learning - Myopic greedy / Don't rely on the objective rather some entropy version.


%%% Local Variables:
%%% mode: latex
%%% TeX-master: "main"
%%% End:


\section{Discussion}
This work identifies signal collapse as a critical bottleneck in one-shot neural network pruning. Performance loss in pruned networks is due to \textbf{signal collapse} in addition to the removal of critical parameters. We propose \textbf{REFLOW} (\textbf{Re}storing \textbf{F}low of \textbf{Low}-variance signals), a simple yet effective method that mitigates signal collapse without computationally expensive weight updates. By focusing on signal preservation, REFLOW highlights the importance of mitigating signal collapse in sparse networks and enables magnitude pruning to match or surpass state-of-the-art one-shot pruning methods such as CHITA, CBS, and WF.

REFLOW consistently achieves state-of-the-art accuracy across diverse architectures, restoring ResNeXt-101 from under 4.1\% to 78.9\% top-1 accuracy at 80\% sparsity on ImageNet. Its lightweight design makes it a practical solution for both research and deployment, delivering high-quality sparse models without the overhead of traditional approaches. These findings challenge the traditional emphasis on weight selection strategies and underscore the critical role of signal propagation for achieving high-quality sparse networks in the context of one-shot pruning.




%\section*{Software and Data}

%If a paper is accepted, we strongly encourage the publication of software and data with the
%camera-ready version of the paper whenever appropriate. This can be
%done by including a URL in the camera-ready copy. However, \textbf{do not}
%include URLs that reveal your institution or identity in your
%submission for review. Instead, provide an anonymous URL or upload
%the material as ``Supplementary Material'' into the OpenReview reviewing
%system. Note that reviewers are not required to look at this material
%when writing their review.

% Acknowledgements should only appear in the accepted version.
%\section*{Acknowledgements}
%\clearpage
\section*{Impact Statement}
Our work advances the efficiency of aligning LLMs with human values by optimizing the way human preferences are queried. Since human feedback is costly and time-consuming, our approach can potentially reduce wasted effort on uninformative comparisons, maximizing the value of each annotation. By improving the efficiency of learning from human preferences, this research has the potential to accelerate the development of safer and more helpful AI systems.


\bibliography{references}
%\bibliographystyle{plainnat}


%%%%%%%%%%%%%%%%%%%%%%%%%%%%%%%%%%%%%%%%%%%%%%%%%%%%%%%%%%%%%%%%%%%%%%%%%%%%%%%
%%%%%%%%%%%%%%%%%%%%%%%%%%%%%%%%%%%%%%%%%%%%%%%%%%%%%%%%%%%%%%%%%%%%%%%%%%%%%%%
% APPENDIX
%%%%%%%%%%%%%%%%%%%%%%%%%%%%%%%%%%%%%%%%%%%%%%%%%%%%%%%%%%%%%%%%%%%%%%%%%%%%%%%
%%%%%%%%%%%%%%%%%%%%%%%%%%%%%%%%%%%%%%%%%%%%%%%%%%%%%%%%%%%%%%%%%%%%%%%%%%%%%%%
\newpage
\appendix
%\onecolumn

\clearpage
\section{Additional Experiment Results}
%\subsection{Two dimensional example}
\begin{figure}[htp]
    \centering
    \includegraphics[width=\linewidth]{Figs/2D_GM_illustration.pdf}
    \caption{\small Comparisons drawn by different strategies to better learn a 2D bimodal reward function. The heat map showed the current estimated function. Red crosses connected by black lines are compared pairs and gray dots on the left-most penal are candidate points to be compared.}
    \label{fig:what_were_compared-big}
\end{figure}

\newpage


\subsection{Comparing Annotation Efficiency on the \texttt{Helpful} Dataset}
\label{appdx:more_results_results_main}

\paragraph{In-Prompt Annotation} efficiency is provided in Figure~\ref{fig:results_main_helpful} (as supplementary of Figure~\ref{fig:results_main} in the main text).

\begin{figure}[h!]
    \centering
    \includegraphics[width=1.0\linewidth]{Figs/Main_xpromptFalse_hidden64_cand500_320_SPLIT_helpful.pdf}
    \caption{Comparing annotation efficiency of different methods. (\texttt{Helpful} Dataset, 3 Models, 8 Methods). First row: 1 - Spearman's Correlation (lower is better); second row: Best-of-N reward. Experiments are repeated with 5 seeds.}
    \label{fig:results_main_helpful}
\end{figure}

\paragraph{Cross-Prompt Annotation} efficiency is provided in Figure~\ref{fig:results_main_xprompt_helpful} (as supplementary of Figure~\ref{fig:results_main_xprompt} in the main text).

\begin{figure}[h!]
    \centering
    \includegraphics[width=1.0\linewidth]{Figs/Main_xpromptTrue_hidden64_cand500_320_SPLIT_helpful.pdf}
    \caption{\small Comparing annotation efficiency of different methods under the \textbf{Cross-Prompt} annotation setups. (\texttt{Helpful} Dataset, 3 Models, 8 Methods). First row: 1 - Spearman's Correlation (lower is better); second row: Best-of-N reward. Experiments are repeated with 5 seeds.}
    \label{fig:results_main_xprompt_helpful}
\end{figure}





\subsection{Annotation Batch Size}
\label{appdx:more_results_annotation_bs}
\paragraph{Results on All Models}
Due to the space limit of the main text, we deferred the experiment results with Gemma7B and the LLaMA3-8B model when studying the effect of different annotation batch sizes in the following Figures (Figure~\ref{fig:results_annotation_bs_gemma2b_zoom}, Figure~\ref{fig:results_annotation_bs_gemma7b_zoom}, Figure~\ref{fig:results_annotation_bs_llama38b_zoom}). 
To summarize the main takeaways --- we observe the same trend as we have observed with the Gemma2B model, the proposed methods achieve better performances in the small batch size setups (more online setups). The stability of small batch setups is in general higher than the large batch setups.


\begin{figure}[h!]
    \centering
    \includegraphics[width=0.8\linewidth]{Figs/CompareAnnoBatch_gemma2b_64_xpromptFalse_h=7.pdf}
    \caption{\small Investigating how annotation batch size choices affect learning performance of different methods. Model: Gemma 2B.}
    \label{fig:results_annotation_bs_gemma2b_zoom}
\end{figure}


\begin{figure}[h!]
    \centering
    \includegraphics[width=0.8\linewidth]{Figs/CompareAnnoBatch_gemma7b_64_xpromptFalse_h=7.pdf}
    \caption{\small Investigating how annotation batch size choices affect learning performance of different methods. Model: Gemma 7B.}
    \label{fig:results_annotation_bs_gemma7b_zoom}
\end{figure}

\begin{figure}[h!]
    \centering
    \includegraphics[width=0.8\linewidth]{Figs/CompareAnnoBatch_llama38b_64_xpromptFalse_h=7.pdf}
    \caption{\small Investigating how annotation batch size choices affect learning performance of different methods. Model: LLaMA3-8B.}
    \label{fig:results_annotation_bs_llama38b_zoom}
\end{figure}

\paragraph{Results with All Methods.}
In addition, we use the figures below (Figure~\ref{fig:results_annotation_bs_gemma2b_total}, Figure~\ref{fig:results_annotation_bs_gemma7b_total}, Figure~\ref{fig:results_annotation_bs_llama38b_total}) for a full analysis on the annotation batch size choices for all methods. For other methods, we do not observe a clear trend on the effect of increasing or decreasing annotation batch sizes.


\begin{figure}[h!]
    \centering
    \includegraphics[width=1.0\linewidth]{Figs/CompareAnnoBatch_gemma2b_64_xpromptFalse.pdf}
    \caption{\small Investigating how annotation batch size choices affect learning performance of different methods. Model: Gemma 2B.}
    \label{fig:results_annotation_bs_gemma2b_total}
\end{figure}


\begin{figure}[h!]
    \centering
    \includegraphics[width=1.0\linewidth]{Figs/CompareAnnoBatch_gemma7b_64_xpromptFalse.pdf}
    \caption{\small Investigating how annotation batch size choices affect learning performance of different methods. Model: Gemma 7B.}
    \label{fig:results_annotation_bs_gemma7b_total}
\end{figure}

\begin{figure}[h!]
    \centering
    \includegraphics[width=1.0\linewidth]{Figs/CompareAnnoBatch_llama38b_64_xpromptFalse.pdf}
    \caption{\small Investigating how annotation batch size choices affect learning performance of different methods. Model: LLaMA3 8B.}
    \label{fig:results_annotation_bs_llama38b_total}
\end{figure}

\newpage
\subsection{Compare Cross-Prompt Comparisons and In-Prompt Comparisons}
\label{appdx:more_results_xprompt-in-prompt_comparison}
In this section, we provide direct comparisons of learning efficiency when using cross-prompt annotations and in-prompt annotations. In most cases, annotating comparisons using cross-prompt comparison improves learning efficiency, and this can be observed across all methods. Specifically, with the entropy-based method, cross-prompt annotations bring a noticeable boost to learning efficiency and reward model performance.

\begin{figure}[h!]
    \centering
    \includegraphics[width=1.0\linewidth]{Figs/CompareXprompt_gemma2b_rnds80.pdf}
\caption{\small Cross-Prompt preference annotation improves overall annotation efficiency. Annotation batch size 500. Model: Gemma2B.}
    \label{fig:results_xprompt_inprompt_abs500_gemma2b}
\end{figure} 


\begin{figure}[h!]
    \centering
    \includegraphics[width=1.0\linewidth]{Figs/CompareXprompt_gemma7b_rnds80.pdf}
\caption{\small Cross-Prompt preference annotation improves overall annotation efficiency. Annotation batch size 500. Model: Gemma7B.}
    \label{fig:results_xprompt_inprompt_abs500_gemma7b}
\end{figure} 


\begin{figure}[h!]
    \centering
    \includegraphics[width=1.0\linewidth]{Figs/CompareXprompt_llama38b_rnds80.pdf}
    \caption{\small Cross-Prompt preference annotation improves overall annotation efficiency. Annotation batch size 500. Model: LLaMA3-8B.}
    \label{fig:results_xprompt_inprompt_abs500_llama38b}
\end{figure} 


% \begin{figure}[h!]
%     \centering
%     \includegraphics[width=1.0\linewidth]{Figs/CompareXprompt_gemma2b_rnds320.pdf}
% \caption{\small Cross-Prompt preference annotation improves overall annotation efficiency. Annotation batch size 125. Model: Gemma2B.}
%     \label{fig:results_xprompt_inprompt_abs125_gemma2b}
% \end{figure}\\


% \begin{figure}[h!]
%     \centering
%     \includegraphics[width=1.0\linewidth]{Figs/CompareXprompt_gemma7b_rnds320.pdf}
% \caption{\small Cross-Prompt preference annotation improves overall annotation efficiency. Annotation batch size 125. Model: Gemma7B.}
%     \label{fig:results_xprompt_inprompt_abs125_gemma7b}
% \end{figure}  \\


% \begin{figure}[h!]
%     \centering
%     \includegraphics[width=1.0\linewidth]{Figs/CompareXprompt_llama38b_rnds320.pdf}
%     \caption{\small Cross-Prompt preference annotation improves overall annotation efficiency. Annotation batch size 125. Model: LLaMA3-8B.}
%     \label{fig:results_xprompt_inprompt_abs125_llama38b}
% \end{figure} \\% 

\clearpage
\subsection{Hyper-Parameter Sensitivity Analysis}
\label{appdx:more_results_hyper_param_sensitivity}

\paragraph{Number of Candidate Numbers}
In the main text, our empirical pipeline starts by sampling $500$ candidates (\texttt{candidate number}) from the training prompts, and then randomly generates $20000$ pairs of comparisons using either in-prompt comparison or cross-prompt comparison. Then, we select \texttt{annotation batch size} number of comparisons to annotate. In this section, we evaluate the performance difference by using a larger \texttt{candidate number} $1000$. 

In experiments, we find those setups do not significantly change the performance of different methods. The performance of D-opt and Past-Aware D-opt are especially robust to those hyper-parameter choices.

% \begin{figure}[h!]
%     \centering
%     % \includegraphics[width=1.0\linewidth]{Figs/CompareCandidates_gemma2b_rnds40.pdf}
%     % \includegraphics[width=1.0\linewidth]{Figs/CompareCandidates_gemma2b_rnds80.pdf}
%     \includegraphics[width=1.0\linewidth]{Figs/CompareCandidates_gemma2b_rnds320.pdf}
%     \caption{\small Preference annotation with different \texttt{candidate number} choices. Annotation batch size 125. Model: Gemma2B.}
%     \label{fig:results_candidate_abs125_gemma2b}
% \end{figure}

\begin{figure}[h!]
    \centering
    \includegraphics[width=1.0\linewidth]{Figs/CompareCandidates_gemma2b_rnds80.pdf}
    \caption{\small Preference annotation with different \texttt{candidate number} choices. Annotation batch size 500. Model: Gemma2B.}
    \label{fig:results_candidate_abs500_gemma2b} 
\end{figure}
% \begin{figure}[h!]
%     \centering
%     \vspace{-1.3cm}
%     % \includegraphics[width=1.0\linewidth]{Figs/CompareCandidates_gemma2b_rnds40.pdf}
%     % \includegraphics[width=1.0\linewidth]{Figs/CompareCandidates_gemma2b_rnds80.pdf}
%     \includegraphics[width=1.0\linewidth]{Figs/CompareCandidates_gemma7b_rnds320.pdf}
%     \caption{\small Preference annotation with different \texttt{candidate number} choices. Annotation batch size 125. Model: Gemma2B.}
%     \label{fig:results_candidate_abs125_gemma7b} 
% \end{figure}

\begin{figure}[h!]
    \centering
    % \includegraphics[width=1.0\linewidth]{Figs/CompareCandidates_gemma2b_rnds40.pdf}
    % \includegraphics[width=1.0\linewidth]{Figs/CompareCandidates_gemma2b_rnds80.pdf}
    \includegraphics[width=1.0\linewidth]{Figs/CompareCandidates_gemma7b_rnds80.pdf}
    \caption{\small Preference annotation with different \texttt{candidate number} choices. Annotation batch size 500. Model: Gemma7B.}
    \label{fig:results_candidate_abs500_gemma7b} 
\end{figure}



% \begin{figure}[h!]
%     \centering
%     % \includegraphics[width=1.0\linewidth]{Figs/CompareCandidates_gemma2b_rnds40.pdf}
%     % \includegraphics[width=1.0\linewidth]{Figs/CompareCandidates_gemma2b_rnds80.pdf}
%     \includegraphics[width=1.0\linewidth]{Figs/CompareCandidates_llama38b_rnds320.pdf}
%     \caption{\small Preference annotation with different \texttt{candidate number} choices. Annotation batch size 125. Model: LLaMA3-8B.}
%     \label{fig:results_candidate_abs125_llama38b} 
% \end{figure}

\begin{figure}[h!]
    \centering
    % \includegraphics[width=1.0\linewidth]{Figs/CompareCandidates_gemma2b_rnds40.pdf}
    % \includegraphics[width=1.0\linewidth]{Figs/CompareCandidates_gemma2b_rnds80.pdf}
    \includegraphics[width=1.0\linewidth]{Figs/CompareCandidates_llama38b_rnds80.pdf}
    \caption{\small Preference annotation with different \texttt{candidate number} choices. Annotation batch size 500. Model: LLaMA3-8B.}
    \label{fig:results_candidate_abs500_llama38b} 
\end{figure} 


\newpage
\paragraph{Number of Hidden Units in 3-Layer MLPs}
In all main text experiments, we use 3-layer MLPs with $64$ \textbf{hidden units}. In this section, we evaluate the performance difference by using a larger \texttt{hidden unit} $128$. 


\begin{figure}[h!]
    \centering
    \includegraphics[width=1.0\linewidth]{Figs/CompareHidden_gemma2b_80_xpromptFalse.pdf}
    \caption{\small Experiments with different \texttt{hidden unit} choices. Annotation batch size 500. Model: Gemma 2B.}
    \label{fig:results_hidden_abs500_gemma2b} 
\end{figure} 


\begin{figure}[h!]
    \centering
    \includegraphics[width=1.0\linewidth]{Figs/CompareHidden_gemma7b_80_xpromptFalse.pdf}
    \caption{\small Experiments with different \texttt{hidden unit} choices. Annotation batch size 500. Model: Gemma 7B.}
    \label{fig:results_hidden_abs500_gemma7b} 
\end{figure} 

\begin{figure}[h!]
    \centering
    \includegraphics[width=1.0\linewidth]{Figs/CompareHidden_llama38b_80_xpromptFalse.pdf}
    \caption{\small Experiments with different \texttt{hidden unit} choices. Annotation batch size 500. Model: LLaMA3-8B.}
    \label{fig:results_hidden_abs500_llama38b} 
\end{figure} 




\clearpage
\section{Further discussions}
\subsection{Cross-Prompt Annotations.}
\label{app:futherdisc}
Cross-Prompt annotations were explored as a way to increase annotation quality by \citet{sun2024rethinking} and empirically verified in \citet{yin2024relative}. A natural question is whether this is possible in practice. If one is willing to assume there exists a scalar value reward function, and human comparisons are based on that function, then cross-prompt is possible because each prompt-response pairs are assigned a real value that are comparable to each other. A single-word change in the prompt without changing its meaning likely will not change what responses are helpful or harmful and make these pairs, even if cross-prompt, comparable. It is possible however, that the reward function is very rough in changing prompts making the reward function for one prompt not transferable to the other and hard to get a better response for one prompt using a reward function learned from other prompts. Even though, if one is willing to believe that prompts live in some lower dimensional manifold and the reward function acquires some regularity in that space, Cross-Prompt annotations might help better learn these dependencies. 


\end{document}