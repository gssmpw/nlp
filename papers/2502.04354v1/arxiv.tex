\documentclass[12pt]{article}
%\usepackage{algorithm}
%\usepackage{algpseudocode}
\usepackage{amsmath, amssymb, amsthm}
\usepackage{float, fullpage, graphicx, multirow,parskip, subcaption, setspace}
\usepackage{comment}
\usepackage{url}
\usepackage{enumitem}
\usepackage{tikz}
\usepackage{bm, bbm}
\usetikzlibrary{shapes.geometric, positioning}
\usetikzlibrary{quotes, angles}
\usepackage{rotating}
\usepackage{hyperref}
\usepackage{natbib}
\usepackage{placeins}
\usepackage{cleveref}
\usepackage{algorithm}
\usepackage{algorithmic}
%\RestyleAlgo{ruled}

\definecolor{SkyBlue}{RGB}{14, 118, 188}
\definecolor{BrightRed}{RGB}{223,82, 78}


\hypersetup{pdfborder = {0 0 0.5 [3 3]}, colorlinks = true, linkcolor = BrightRed, citecolor = SkyBlue}

\bibliographystyle{apalike}

\DeclareMathOperator{\tr}{tr}
\DeclareMathOperator{\sign}{sign}
\DeclareMathOperator{\diag}{diag}
\DeclareMathOperator{\vect}{vec}
\DeclareMathOperator{\GA}{Gamma}
\DeclareMathOperator{\BER}{Bernoulli}
\DeclareMathOperator{\BA}{Beta}
\DeclareMathOperator{\Pois}{Poisson}
\DeclareMathOperator{\Multinomial}{Multinomial}
%\DeclareMathOperator*{\argmax}{arg\,max}
%\DeclareMathOperator*{\argmin}{arg\,min}
\newtheorem{myTheorem}{Theorem}
\newtheorem{myLemma}{Lemma}
\def\R{\mathbb{R}}

\newcommand{\embd}{\bm{\Phi}}
\newcommand{\nonlinembdplain}{\bm{\Psi}}

\newcommand{\nonlinembd}[2]{\nonlinembdplain_{#1,#2}}

\newcommand{\lastembd}[2]{\embd_{#1,#2}}
\newcommand{\lastweight}{\bm{\beta}_{\text{-1}}}
\newcommand{\truereward}{r}
\newcommand{\preference}[1]{h_{#1}}
\newcommand{\sigmoid}{\sigma}
\newcommand{\pairidx}{i}
\newcommand{\pairidxalt}{j}
\newcommand{\totalpairs}{I}
\newcommand{\fisherinfo}{\mathcal{I}}

\newcommand{\prompt}{x}
\newcommand{\response}{y}
\newcommand{\chosenset}{\mathcal{C}}
\newcommand{\budget}{c}

\newcommand{\poolset}{\mathcal{P}}
\newcommand{\labeledset}{\mathcal{D}}
\newcommand{\alstepidx}{s}
\newcommand{\model}[1]{\mathcal{M}_{#1}}
\newcommand{\scoringrule}{\mathcal{S}}

\DeclareMathOperator{\argmax}{argmax}
\newcommand{\levelset}{\text{entropy}}
\newcommand{\dopt}{\text{dopt}}
\newcommand{\maxdiff}{\text{maxdiff}}
\newcommand{\coreset}{\text{coreset}}
\newcommand{\XtX}{\text{XtX}}
\newcommand{\batchbald}{\text{bBALD}}

%%%%%%%%%%%%%%%%%%%%%%%%%%%%%%%%
% THEOREMS
%%%%%%%%%%%%%%%%%%%%%%%%%%%%%%%%
\theoremstyle{plain}
\newtheorem{theorem}{Theorem}[section]
\newtheorem{proposition}[theorem]{Proposition}
\newtheorem{lemma}[theorem]{Lemma}
\newtheorem{corollary}[theorem]{Corollary}
\theoremstyle{definition}
\newtheorem{definition}[theorem]{Definition}
\newtheorem{assumption}[theorem]{Assumption}
\theoremstyle{remark}
\newtheorem{remark}[theorem]{Remark}


\newcommand\numberthis{\addtocounter{equation}{1}\tag{\theequation}}
\newcommand{\numbereqn}{\addtocounter{equation}{1}\tag{\theequation}} % use \numberthis to add number in align* mode
\newcommand{\skd}[1]{\textcolor{cyan}{\small [skd]: #1}}
\newcommand{\yunyi}[1]{{\textcolor{blue}{\small [ys]: #1}}}

\def\keywordname{{\bfseries \emph Keywords}}%
\def\keywords#1{\par\addvspace\medskipamount{\rightskip=0pt plus1cm
\def\and{\ifhmode\unskip\nobreak\fi\ $\cdot$
}\noindent\keywordname\enspace\ignorespaces#1\par}}

\onehalfspacing

% \title{Active Reward Modeling: Adaptive Preference Labeling for Large Language Model Alignment}
\title{Reviving The Classics: Active Reward Modeling in Large Language Model Alignment}
%\author{ Yunyi Shen\thanks{YS and HS contributed equally to this paper.}~~~ Hao Sun\footnotemark[1]~~~ Jean-Francois Ton \\
%MIT, Cambridge, ByteDance Research\\
%\texttt{yshen99@mit.edu, hs789@cam.ac.uk,  jeanfrancois@bytedance.com}
%}
\author{
Yunyi Shen\thanks{YS and HS contributed equally to this paper.}~~\thanks{Massachusetts Institute of Technology, \texttt{yshen99@mit.edu}}\and
Hao Sun\footnotemark[1]~~\thanks{University of Cambridge, \texttt{hs789@cam.ac.uk}}\and
Jean-Fran\c cois Ton\thanks{ByteDance Research, \texttt{jeanfrancois@bytedance.com}}
}

\begin{document}
\def\bY{\bm{Y}}
\def\by{\bm{y}} % vector of all observations



\def\bz{\bm{z}}
\def\bX{\bm{X}}
\def\bx{\bm{x}} % vector of single set of covariates

\def\R{\mathbb{R}}
\def\N{\mathcal{N}}
\def\P{\mathbb{P}}
\def\E{\mathbb{E}}

\def\Xcal{\mathcal{X}}





\maketitle

\begin{abstract}
\begin{abstract}  
Test time scaling is currently one of the most active research areas that shows promise after training time scaling has reached its limits.
Deep-thinking (DT) models are a class of recurrent models that can perform easy-to-hard generalization by assigning more compute to harder test samples.
However, due to their inability to determine the complexity of a test sample, DT models have to use a large amount of computation for both easy and hard test samples.
Excessive test time computation is wasteful and can cause the ``overthinking'' problem where more test time computation leads to worse results.
In this paper, we introduce a test time training method for determining the optimal amount of computation needed for each sample during test time.
We also propose Conv-LiGRU, a novel recurrent architecture for efficient and robust visual reasoning. 
Extensive experiments demonstrate that Conv-LiGRU is more stable than DT, effectively mitigates the ``overthinking'' phenomenon, and achieves superior accuracy.
\end{abstract}  
\end{abstract}

\section{Introduction}
\section{Introduction}


\begin{figure}[t]
\centering
\includegraphics[width=0.6\columnwidth]{figures/evaluation_desiderata_V5.pdf}
\vspace{-0.5cm}
\caption{\systemName is a platform for conducting realistic evaluations of code LLMs, collecting human preferences of coding models with real users, real tasks, and in realistic environments, aimed at addressing the limitations of existing evaluations.
}
\label{fig:motivation}
\end{figure}

\begin{figure*}[t]
\centering
\includegraphics[width=\textwidth]{figures/system_design_v2.png}
\caption{We introduce \systemName, a VSCode extension to collect human preferences of code directly in a developer's IDE. \systemName enables developers to use code completions from various models. The system comprises a) the interface in the user's IDE which presents paired completions to users (left), b) a sampling strategy that picks model pairs to reduce latency (right, top), and c) a prompting scheme that allows diverse LLMs to perform code completions with high fidelity.
Users can select between the top completion (green box) using \texttt{tab} or the bottom completion (blue box) using \texttt{shift+tab}.}
\label{fig:overview}
\end{figure*}

As model capabilities improve, large language models (LLMs) are increasingly integrated into user environments and workflows.
For example, software developers code with AI in integrated developer environments (IDEs)~\citep{peng2023impact}, doctors rely on notes generated through ambient listening~\citep{oberst2024science}, and lawyers consider case evidence identified by electronic discovery systems~\citep{yang2024beyond}.
Increasing deployment of models in productivity tools demands evaluation that more closely reflects real-world circumstances~\citep{hutchinson2022evaluation, saxon2024benchmarks, kapoor2024ai}.
While newer benchmarks and live platforms incorporate human feedback to capture real-world usage, they almost exclusively focus on evaluating LLMs in chat conversations~\citep{zheng2023judging,dubois2023alpacafarm,chiang2024chatbot, kirk2024the}.
Model evaluation must move beyond chat-based interactions and into specialized user environments.



 

In this work, we focus on evaluating LLM-based coding assistants. 
Despite the popularity of these tools---millions of developers use Github Copilot~\citep{Copilot}---existing
evaluations of the coding capabilities of new models exhibit multiple limitations (Figure~\ref{fig:motivation}, bottom).
Traditional ML benchmarks evaluate LLM capabilities by measuring how well a model can complete static, interview-style coding tasks~\citep{chen2021evaluating,austin2021program,jain2024livecodebench, white2024livebench} and lack \emph{real users}. 
User studies recruit real users to evaluate the effectiveness of LLMs as coding assistants, but are often limited to simple programming tasks as opposed to \emph{real tasks}~\citep{vaithilingam2022expectation,ross2023programmer, mozannar2024realhumaneval}.
Recent efforts to collect human feedback such as Chatbot Arena~\citep{chiang2024chatbot} are still removed from a \emph{realistic environment}, resulting in users and data that deviate from typical software development processes.
We introduce \systemName to address these limitations (Figure~\ref{fig:motivation}, top), and we describe our three main contributions below.


\textbf{We deploy \systemName in-the-wild to collect human preferences on code.} 
\systemName is a Visual Studio Code extension, collecting preferences directly in a developer's IDE within their actual workflow (Figure~\ref{fig:overview}).
\systemName provides developers with code completions, akin to the type of support provided by Github Copilot~\citep{Copilot}. 
Over the past 3 months, \systemName has served over~\completions suggestions from 10 state-of-the-art LLMs, 
gathering \sampleCount~votes from \userCount~users.
To collect user preferences,
\systemName presents a novel interface that shows users paired code completions from two different LLMs, which are determined based on a sampling strategy that aims to 
mitigate latency while preserving coverage across model comparisons.
Additionally, we devise a prompting scheme that allows a diverse set of models to perform code completions with high fidelity.
See Section~\ref{sec:system} and Section~\ref{sec:deployment} for details about system design and deployment respectively.



\textbf{We construct a leaderboard of user preferences and find notable differences from existing static benchmarks and human preference leaderboards.}
In general, we observe that smaller models seem to overperform in static benchmarks compared to our leaderboard, while performance among larger models is mixed (Section~\ref{sec:leaderboard_calculation}).
We attribute these differences to the fact that \systemName is exposed to users and tasks that differ drastically from code evaluations in the past. 
Our data spans 103 programming languages and 24 natural languages as well as a variety of real-world applications and code structures, while static benchmarks tend to focus on a specific programming and natural language and task (e.g. coding competition problems).
Additionally, while all of \systemName interactions contain code contexts and the majority involve infilling tasks, a much smaller fraction of Chatbot Arena's coding tasks contain code context, with infilling tasks appearing even more rarely. 
We analyze our data in depth in Section~\ref{subsec:comparison}.



\textbf{We derive new insights into user preferences of code by analyzing \systemName's diverse and distinct data distribution.}
We compare user preferences across different stratifications of input data (e.g., common versus rare languages) and observe which affect observed preferences most (Section~\ref{sec:analysis}).
For example, while user preferences stay relatively consistent across various programming languages, they differ drastically between different task categories (e.g. frontend/backend versus algorithm design).
We also observe variations in user preference due to different features related to code structure 
(e.g., context length and completion patterns).
We open-source \systemName and release a curated subset of code contexts.
Altogether, our results highlight the necessity of model evaluation in realistic and domain-specific settings.






%\newpage
\section{Background and setup}
\newcommand{\tabincell}[2]{\begin{tabular}{@{}#1@{}}#2\end{tabular}}
\newcommand{\rowstyle}[1]{\gdef\currentrowstyle{#1}%
	#1\ignorespaces
}

\newcommand{\className}[1]{\textbf{\sf #1}}
\newcommand{\functionName}[1]{\textbf{\sf #1}}
\newcommand{\objectName}[1]{\textbf{\sf #1}}
\newcommand{\annotation}[1]{\textbf{#1}}
\newcommand{\todo}[1]{\textcolor{blue}{\textbf{[[TODO: #1]]}}}
\newcommand{\change}[1]{\textcolor{blue}{#1}}
\newcommand{\fetch}[1]{\textbf{\em #1}}
\newcommand{\phead}[1]{\vspace{1mm} \noindent {\bf #1}}
\newcommand{\wei}[1]{\textcolor{blue}{{\it [Wei says: #1]}}}
\newcommand{\peter}[1]{\textcolor{red}{{\it [Peter says: #1]}}}
\newcommand{\zhenhao}[1]{\textcolor{dkblue}{{\it [Zhenhao says: #1]}}}
\newcommand{\feng}[1]{\textcolor{magenta}{{\it [Feng says: #1]}}}
\newcommand{\jinqiu}[1]{\textcolor{red}{{\it [Jinqiu says: #1]}}}
\newcommand{\shouvick}[1]{\textcolor{violet(ryb)}{{\it [Shouvick says: #1]}}}
\newcommand{\pattern}[1]{\emph{#1}}
%\newcommand{\tool}{{{DectGUILag}}\xspace}
\newcommand{\tool}{{{GUIWatcher}}\xspace}


\newcommand{\guo}[1]{\textcolor{yellow}{{\it [Linqiang says: #1]}}}

\newcommand{\rqbox}[1]{\begin{tcolorbox}[left=4pt,right=4pt,top=4pt,bottom=4pt,colback=gray!5,colframe=gray!40!black,before skip=2pt,after skip=2pt]#1\end{tcolorbox}}


\section{Designing of comparisons}
%!TEX root = ../main.tex

\subsection{Linear BT Regression.}
Consider a simplified case where the true reward function is linear with respect to some intermediate embedding, $\truereward(\lastembd{\pairidx}{1}) = \lastembd{\pairidx}{1}^\top \lastweight$, for weight vector $\lastweight$. We use $\embd$ instead of $\nonlinembdplain$ because the reward may not be linear with respect to the original embedding $\nonlinembdplain$ used in reward modeling, and we wish to avoid confusion. The subscript $-1$ in $\lastweight$ reflects how we will apply these results in practice: $\embd$ represents the output before the final linear layer, and $\lastweight$ corresponds to the weight of this last layer. For now, we assume that this linear feature $\embd$ is known to us. Note that there is no bias term because linear BT is identified only up to translation. 

Under this simplified setting the preference generating process of $\pairidx$th pair $\preference{\pairidx}$ can be simplified to 
\begin{equation}
    \preference{\pairidx} \sim \BER[\sigmoid[(\lastembd{\pairidx}{1}-\lastembd{\pairidx}{2})^\top\lastweight]]
    \label{eq:BT}
\end{equation}
It can be observed that this corresponds to a logistic regression, where the covariates are the difference $\lastembd{\pairidx}{1} - \lastembd{\pairidx}{2}$.

By applying the theory from generalized linear models, we know that the maximum likelihood estimate $\hat{\lastweight}$ is asymptotically Gaussian distributed, with mean $\lastweight$ and covariance matrix $\fisherinfo^{-1}$, where $\fisherinfo$ denotes the Fisher information (FI) matrix \citep[see e.g., ][Ch. 4.5.2]{shao2008mathematical}. For the linear Bradley-Terry model, the FI is
\begin{equation}
    \fisherinfo=\sum_{\pairidx=1}^\totalpairs (\lastembd{\pairidx}{1}-\lastembd{\pairidx}{2})(\lastembd{\pairidx}{1}-\lastembd{\pairidx}{2})^\top p_{\pairidx}(1-p_{\pairidx})
    \label{eq:FI}
\end{equation}
Where $p_{\pairidx} = \sigmoid[(\lastembd{\pairidx}{1} - \lastembd{\pairidx}{2})^\top \lastweight]$, it can be observed that $p_{\pairidx}(1 - p_{\pairidx})$ represents the variance of a Bernoulli random variable.

The Fisher information matrix can be interpreted as the metric tensor in a Riemannian manifold of distributions, where the distance between them is given by the symmetrized KL divergence \citep{costa2015fisher}. FI quantifies the amount of information in the dataset for estimating the parameters $\lastweight$. From a Bayesian perspective, the Bernstein-von Mises theorem \citep[][Ch. 10.2, Thm 10.1]{van2000asymptotic} states that $\fisherinfo^{-1}$ is also the asymptotic covariance matrix of the posterior distribution of $\lastweight$, assuming mild regularity conditions on the prior.

The FI can be viewed as a sum over all independent data points' contribution. For each data point, there are two terms multiplied together: the empirical covariance of embedding differences $(\lastembd{\pairidx}{1} - \lastembd{\pairidx}{2})(\lastembd{\pairidx}{1} - \lastembd{\pairidx}{2})^\top$, and $p_{\pairidx}(1 - p_{\pairidx})$, the variance of the comparison results. \citet{sun2024rethinking} suggested that improving the variance of comparisons can be interpreted as improving annotation quality which can also be seen from FI. 

To make the FI large \cref{eq:FI} an ideal comparison should exhibit both a large variance in the embedding difference (thus $(\lastembd{\pairidx}{1} - \lastembd{\pairidx}{2})(\lastembd{\pairidx}{1} - \lastembd{\pairidx}{2})^\top$ having large eigenvalues) and a high variance in the comparison outcomes (thus $p_{\pairidx}(1 - p_{\pairidx})$ large). This implies that the embedding space should be diverse, such that $\lastembd{\pairidx}{1} - \lastembd{\pairidx}{2}$ captures a wide range of differences, and each comparison should be informative—not too close to 0 or 1. The former encourages exploration within the embedding space, leading to a better regression model, while the latter ensures that comparisons are not trivial, improving sample efficiency. An everyday analogy for comparing non-obvious pairs would be that comparing a world champion to a newbie in chess offers little insight into the abilities of either player.

The FI plays a crucial role in the classical theory of experimental design, both in frequentist and Bayesian frameworks, as highlighted by the Bernstein-von Mises theorem. This leads to a family of design strategies known as alphabetical designs \citep{chaloner1995bayesian, pukelsheim2006optimal}. 

\textbf{(Bayesian) D-optimality \citep{chaloner1995bayesian}.}
%In classic statistics and experimental design literature, one strategy is the use of so-called alphabetical designs \citep{chaloner1995bayesian, pukelsheim2006optimal}, 
The alphabetical designs focus on the (co)variance of either estimating weights $\lastweight$ or making predictions under new embeddings, typically summarized through the covariance matrix. For example, the D-optimal design minimizes the determinant of the (asymptotic) covariance matrix of the last layer weights, $\lastweight$. Since $|\fisherinfo^{-1}| = 1 / |\fisherinfo|$, this is equivalent to maximizing the determinant of the FI.

The Bayesian variant of D-optimal involves having prior contribution, such as maximizing $|\fisherinfo + I/\sigma^2|$, where $I$ is the identity matrix, to avoid a determinant of zero. This corresponds to the inverse covariance matrix of the Laplace approximation of the posterior of $\lastweight$, assuming a normal prior with variance $\sigma^2$.

A plug-in estimator of $p_{\pairidx}$, $\hat{p}_{\pairidx}$, using the current best model, can be used to estimate the FI \citep{chaloner1995bayesian, pukelsheim2006optimal}. In this approach, the scoring rule is the determinant of the Fisher Information matrix.
\begin{equation}
    \scoringrule_{\dopt}(\chosenset) = \lvert \sum_{\pairidx\in \chosenset}  (\lastembd{\pairidx}{1}-\lastembd{\pairidx}{2})(\lastembd{\pairidx}{1}-\lastembd{\pairidx}{2})^\top \hat{p}_{\pairidx}(1-\hat{p}_{\pairidx})\rvert
\end{equation}
In experiments, we refer to this strategy as \texttt{D-opt}. Other forms of optimality also exist, each targeting different summaries of the Fisher Information (FI), such as A-optimality, which focuses on minimizing the trace of $\fisherinfo^{-1}$. When the prediction of a new, known embedding is the primary concern, G-optimality aims to minimize the variance of predictions on new embeddings. %In this case, the criterion can be expressed as $(\lastembd{\pairidx}{1} - \lastembd{\pairidx}{2})^\top \fisherinfo^{-1} (\lastembd{\pairidx}{1} - \lastembd{\pairidx}{2})$.

In this work, we suggest using D-optimality because it avoids the need to invert the FI, as required in A-optimality, and doesn't require specifying which samples to predict, as in G-optimality. For readers interested in further details, we refer to \citet{pukelsheim2006optimal} (Ch. 9).

The D-optimality strategy can be made a past-aware version by incorporating previously collected data. The asymptotic covariance of the full data-conditioned posterior is then $(\fisherinfo_{\text{past}} + \fisherinfo)^{-1}$, where $\fisherinfo_{\text{past}}$ is computed using prior data and \cref{eq:FI}. This approach relates to Bayesian methods like Bayesian active learning by disagreement (BALD) \citep{houlsby2011bayesian}, which minimizes posterior entropy. Since Gaussian entropy is proportional to the log-determinant of its covariance. In our experiments, we refer to this variant as \texttt{PA D-opt}.

Next, we review some other strategies that can be applied to BT models.

\textbf{Entropy sampling \citep{settles2009active, muldrew2024active}.}
This strategy aims to select samples about which the current model is most uncertain \citep{settles2009active}. In the context of binary preference modeling, this corresponds to choosing data whose predictions $\hat{p}_{\pairidx}$ are closest to 0.5, effectively exploring the level set of the reward. This is similar to a binary classification problem where the goal is to explore the decision boundary. This approach was also proposed by \citet{muldrew2024active} as maximizing predictive entropy. The scoring rule is then,
\begin{equation}
    \scoringrule_{\levelset}(\chosenset)=\sum_{\pairidx\in \chosenset} \left[-\hat{p}_{\pairidx}\log \hat{p}_{\pairidx} -(1-\hat{p}_{\pairidx})\log(1-\hat{p}_{\pairidx})\right]
\end{equation}
Since the entropy of a Bernoulli distribution reaches its maximum when $p = 0.5$, this approach is equivalent to selecting the top $\budget$ pairs where the predicted probability is closest to 0.5. In our experiments, we refer to this method as \texttt{Entropy}.

\textbf{Maximum difference \citep{muldrew2024active}.}
Contrasting with entropy sampling, this strategy focuses on comparing samples that the current reward model predicts to be the best and the worst, corresponding to probabilities close to 0 or 1. This approach was used by \citet{muldrew2024active} to measure model certainties. The scoring rule to be maximized can thus be interpreted as difference in estimated reward $|\hat{\truereward}_{\pairidx,1}-\hat{\truereward}_{\pairidx,2}|$.
\begin{equation}
    \scoringrule_{\maxdiff}(\chosenset)=\sum_{\pairidx\in \chosenset} |\hat{\truereward}_{\pairidx,1}-\hat{\truereward}_{\pairidx,2}|%\left[\hat{p}_{\pairidx}\log \hat{p}_{\pairidx} +(1-\hat{p}_{\pairidx})\log(1-\hat{p}_{\pairidx})\right]
\end{equation}
This strategy encourages exploration in the \textit{reward} space rather than the embedding space. It is sometimes used in active learning when the goal is to identify positive examples rather than the best classification \citep{settles2009active}. This justifies its use in reward modeling, where the goal is to obtain responses that yield better rewards in downstream tasks. In our experiments, we refer to this method as \texttt{Maxdiff}.

\textbf{Optimizing design matrix \citep{mukherjee2024optimal}.} 
This strategy focuses on finding the best collection of embeddings, or the design matrix in statistics terms $\lastembd{\pairidx}{1} - \lastembd{\pairidx}{2}$, without looking at model predictions. A common objective is to optimize the covariance matrix of the designs, $\Sigma = \sum_{\pairidx=1}^\totalpairs (\lastembd{\pairidx}{1} - \lastembd{\pairidx}{2})(\lastembd{\pairidx}{1} - \lastembd{\pairidx}{2})^\top$. One approach is to maximize the determinant of $\Sigma$, $|\Sigma|$, which encourages exploration over a large space of embedding differences. In fact, if we assume a linear regression model with additive Gaussian noise instead of logistic regression, this covariance matrix corresponds to the Fisher Information matrix of the regression coefficients, and this strategy aligns with the D-optimal design. The scoring rule is
\begin{equation}
    \scoringrule_{\XtX}(\chosenset) = \lvert \sum_{\pairidx\in \chosenset} (\lastembd{\pairidx}{1}-\lastembd{\pairidx}{2})(\lastembd{\pairidx}{1}-\lastembd{\pairidx}{2})^\top\rvert
\end{equation}
\citet{mukherjee2024optimal} used a similar strategy for a different type of preference data that is not purely binary. In our experiments, we refer to this method as \texttt{det(XtX)}, for the determinant of $X^\top X$.

\textbf{Coreset \citep{huggins2016coresets,munteanu2018coresets}.}
Instead of minimizing uncertainty in parameter estimations, the Coreset strategy aims to find a small subset of samples such that the trained model closely approximates the one trained on the full dataset, effectively transforming the problem into a sparse approximation task on weighting data points. The Coreset method for logistic regression has been studied recently by \citet{munteanu2018coresets} and \citet{huggins2016coresets} in both frequentist and Bayesian settings. In our experiment, we adopted the method of \citet{huggins2016coresets}. The scoring rule does not have a simple closed-form solution, so we refer interested readers to \citet{huggins2016coresets} and denote it as $\scoringrule_{\coreset}$. In our experiments, we refer to this method as \texttt{Coreset}.

\textbf{BALD and batchBALD \citep{houlsby2011bayesian, kirsch2019batchbald}.} When transitioning from frequentist to Bayesian framework, BALD \citep{houlsby2011bayesian} and BatchBALD \citep{kirsch2019batchbald} select data with high mutual information between the candidate batch's prediction and model parameters, making the data more informative. \citet{houlsby2011bayesian} showed that this approach maximizes expected posterior entropy reduction. This strategy applies to preference learning \citep{houlsby2011bayesian} but requires a Bayesian model. We denote the corresponding scoring rule as $\scoringrule_{\batchbald}$. In our experiments, we refer to this method as \texttt{BatchBald}. We used implementation in \texttt{batchbald\_redux} \citep{kirsch2019batchbald}.

This strategy relates to Bayesian D-optimality; when posterior entropy is tractable, it can be minimized directly instead of relying on approximations from \citet{houlsby2011bayesian}. If the posterior is Gaussian, entropy is proportional to the log-determinant of its covariance, leading to D-optimality.

\subsection{Gradient Approximation for Combinatorial Optimization.}
In some strategies, we select a data subset to maximize information criteria like the determinant of FI or the design matrix. These often lead to intractable combinatorial optimization problems. To address this, we use the sensitivity approach from the coreset and robustness literature \citep{huggins2016coresets, campbell2018bayesian, campbell2019automated}. When the information criteria are expressed as a nonlinear function over sum of data point contributions, i.e., $\scoringrule = f(\sum_{\pairidx} c_\pairidx)$, where each data point contributes $c_\pairidx$, we introduce weights $w_i$, allowing the score to be rewritten as $\scoringrule(\bm{w}) = f(\sum_{\pairidx} w_i c_\pairidx)$. For instance, the D-optimal score expresses the determinant of FI of a subset $\chosenset$ as a weighted sum.
\begin{equation}
%\small
    \scoringrule_{\dopt}(\bm{w}) = \lvert \sum_{\pairidx} w_\pairidx (\lastembd{\pairidx}{1}-\lastembd{\pairidx}{2})(\lastembd{\pairidx}{1}-\lastembd{\pairidx}{2})^\top \hat{p}_{\pairidx}(1-\hat{p}_{\pairidx})\rvert
\end{equation}
Each candidate pair is assigned a weight $w_i = 1_{i\in \chosenset}$. Selecting a subset $\chosenset$ to maximize $\scoringrule_{\dopt}$ is equivalent to finding a sparse 0-1 weight vector $\bm{w}$ that maximizes $\scoringrule_{\dopt}(\bm{w})$.

To approximate the optimization, we treat $\bm{w}$ as continuous and perform a Taylor expansion around $\bm{w} = \bm{1}$, the all 1 vector, i.e., all data points are included.
\begin{equation}
   \scoringrule(\bm w)\approx \scoringrule(\bm 1) - (\bm 1-\bm{w})^\top \nabla_{\bm w} \scoringrule(\bm w)|_{\bm w=\bm 1}
   \label{eq:taylorexpansion}
\end{equation}
The approximated optimization problem becomes
\begin{equation}
    \argmax_{\bm w}\scoringrule(\bm w)\approx \argmax_{\bm w} \bm{w}^\top \nabla_{\bm w} \scoringrule(\bm w)|_{\bm w=\bm 1}
\end{equation}
A sparse 0-1 valued vector $\bm w$ that optimizes the right-hand side of \cref{eq:taylorexpansion} can be obtained by selecting the data points with the largest gradient, $\nabla_{\bm w} \scoringrule(\bm w) \big|_{\bm w = \bm 1}$. A probabilistic approach, when all gradients are positive, involves sampling according to the weights given by $\nabla_{\bm w} \scoringrule(\bm w) \big|_{\bm w = \bm 1}$. 



\subsection{Handling nonlinear model using last layer features.}
For nonlinear reward models in \cref{eq:BT}, the dependencies on embeddings become more complex. Strategies like maximum difference and entropy sampling, which depend only on model predictions, remain unaffected by the architecture, while batchBALD is designed for (Bayesian) deep models. Feature-based methods like coreset or D-optimal need adaptation. A heuristic from the Bayesian last layer \citep{tran2019bayesian} and computer vision literature \citet{sener2017active} suggests using the last layer before the linear output as a feature, applying linear strategies to it.
\begin{equation}
    \truereward(\nonlinembdplain) = F_{\bm \theta}(\nonlinembdplain)^\top \lastweight
\end{equation}
For some nonlinear function $F_{\bm \theta}$ parameterized by $\bm\theta$, e.g., an MLP and $\embd := F_{\bm \theta}(\nonlinembdplain)$. We apply methods in linear settings with features $F_{\bm \theta}(\nonlinembdplain)$. We then train $\bm\theta$ and $\lastweight$ together once data are labeled. In particular, in \citet{sener2017active} the nonlinear function $F_{\bm \theta}$ is a CNN and they took a coreset approach. Here we apply this strategy to the coreset, optimal design matrix and D-optimal setting.  





\section{Illustrative Examples in Dimension Two}
\label{app:2Dilustration}
\textbf{Experiment Setups}
In this experiment, we provide a two-dimensional example of the comparisons made by each strategy. The ground truth reward was defined as the log probability of a mixture of two Gaussians, centered at $(-2.5, -2.5)$ and $(2.5, 2.5)$ with a variance of 0.25. Preference data was simulated using the BT model, and we attempted to learn the reward function with a 3-layer MLP with $16$ hidden units. For each round, $1000$ points were sampled from a standard normal distribution, and $200$ comparisons were selected using different strategies. $4$ rounds are shown in \cref{fig:what_were_compared}.%, with a zoomed-in version in \cref{fig:what_were_compared-big}.
\begin{figure}[htp]
    \centering
    %\vspace{-0.58cm}
    \includegraphics[width=1.0\linewidth]{Figs/2D_GM_illustration.pdf}\vspace{-0.35cm}
    \caption{\small Comparisons drawn by different strategies to learn a 2D bimodal reward function. The heat map showed the estimated functions. Red dots connected by lines are \textbf{selected pairs} and gray dots on the first column are candidate points to choose from.} %\vspace{-0.42cm}
    \label{fig:what_were_compared}
\end{figure}

\textbf{What were compared in dimension two?}
We observed that D-optimal selects diverse samples with many anchoring points, often comparing multiple points to a single one, spreading out the level set in the original space. Entropy sampling, similar to random sampling, focuses on points near reward values, effectively traversing the reward function's level set. Coreset also selects diverse comparisons, though not always among points with similar reward values. The best design matrix method behaves similarly to coreset, emphasizing diversity in comparisons. In contrast, the max difference method tends to compare extreme values with many others, promoting exploration but potentially yielding less informative comparisons. BatchBALD also selects diverse comparisons, though without a clear pattern. These observations suggest that most methods encourage exploration, entropy sampling prioritizes informative comparisons, and D-optimal seeks a balance between the two.
\section{Experiments}
\label{sec:experiments}
The experiments are designed to address two key research questions.
First, \textbf{RQ1} evaluates whether the average $L_2$-norm of the counterfactual perturbation vectors ($\overline{||\perturb||}$) decreases as the model overfits the data, thereby providing further empirical validation for our hypothesis.
Second, \textbf{RQ2} evaluates the ability of the proposed counterfactual regularized loss, as defined in (\ref{eq:regularized_loss2}), to mitigate overfitting when compared to existing regularization techniques.

% The experiments are designed to address three key research questions. First, \textbf{RQ1} investigates whether the mean perturbation vector norm decreases as the model overfits the data, aiming to further validate our intuition. Second, \textbf{RQ2} explores whether the mean perturbation vector norm can be effectively leveraged as a regularization term during training, offering insights into its potential role in mitigating overfitting. Finally, \textbf{RQ3} examines whether our counterfactual regularizer enables the model to achieve superior performance compared to existing regularization methods, thus highlighting its practical advantage.

\subsection{Experimental Setup}
\textbf{\textit{Datasets, Models, and Tasks.}}
The experiments are conducted on three datasets: \textit{Water Potability}~\cite{kadiwal2020waterpotability}, \textit{Phomene}~\cite{phomene}, and \textit{CIFAR-10}~\cite{krizhevsky2009learning}. For \textit{Water Potability} and \textit{Phomene}, we randomly select $80\%$ of the samples for the training set, and the remaining $20\%$ for the test set, \textit{CIFAR-10} comes already split. Furthermore, we consider the following models: Logistic Regression, Multi-Layer Perceptron (MLP) with 100 and 30 neurons on each hidden layer, and PreactResNet-18~\cite{he2016cvecvv} as a Convolutional Neural Network (CNN) architecture.
We focus on binary classification tasks and leave the extension to multiclass scenarios for future work. However, for datasets that are inherently multiclass, we transform the problem into a binary classification task by selecting two classes, aligning with our assumption.

\smallskip
\noindent\textbf{\textit{Evaluation Measures.}} To characterize the degree of overfitting, we use the test loss, as it serves as a reliable indicator of the model's generalization capability to unseen data. Additionally, we evaluate the predictive performance of each model using the test accuracy.

\smallskip
\noindent\textbf{\textit{Baselines.}} We compare CF-Reg with the following regularization techniques: L1 (``Lasso''), L2 (``Ridge''), and Dropout.

\smallskip
\noindent\textbf{\textit{Configurations.}}
For each model, we adopt specific configurations as follows.
\begin{itemize}
\item \textit{Logistic Regression:} To induce overfitting in the model, we artificially increase the dimensionality of the data beyond the number of training samples by applying a polynomial feature expansion. This approach ensures that the model has enough capacity to overfit the training data, allowing us to analyze the impact of our counterfactual regularizer. The degree of the polynomial is chosen as the smallest degree that makes the number of features greater than the number of data.
\item \textit{Neural Networks (MLP and CNN):} To take advantage of the closed-form solution for computing the optimal perturbation vector as defined in (\ref{eq:opt-delta}), we use a local linear approximation of the neural network models. Hence, given an instance $\inst_i$, we consider the (optimal) counterfactual not with respect to $\model$ but with respect to:
\begin{equation}
\label{eq:taylor}
    \model^{lin}(\inst) = \model(\inst_i) + \nabla_{\inst}\model(\inst_i)(\inst - \inst_i),
\end{equation}
where $\model^{lin}$ represents the first-order Taylor approximation of $\model$ at $\inst_i$.
Note that this step is unnecessary for Logistic Regression, as it is inherently a linear model.
\end{itemize}

\smallskip
\noindent \textbf{\textit{Implementation Details.}} We run all experiments on a machine equipped with an AMD Ryzen 9 7900 12-Core Processor and an NVIDIA GeForce RTX 4090 GPU. Our implementation is based on the PyTorch Lightning framework. We use stochastic gradient descent as the optimizer with a learning rate of $\eta = 0.001$ and no weight decay. We use a batch size of $128$. The training and test steps are conducted for $6000$ epochs on the \textit{Water Potability} and \textit{Phoneme} datasets, while for the \textit{CIFAR-10} dataset, they are performed for $200$ epochs.
Finally, the contribution $w_i^{\varepsilon}$ of each training point $\inst_i$ is uniformly set as $w_i^{\varepsilon} = 1~\forall i\in \{1,\ldots,m\}$.

The source code implementation for our experiments is available at the following GitHub repository: \url{https://anonymous.4open.science/r/COCE-80B4/README.md} 

\subsection{RQ1: Counterfactual Perturbation vs. Overfitting}
To address \textbf{RQ1}, we analyze the relationship between the test loss and the average $L_2$-norm of the counterfactual perturbation vectors ($\overline{||\perturb||}$) over training epochs.

In particular, Figure~\ref{fig:delta_loss_epochs} depicts the evolution of $\overline{||\perturb||}$ alongside the test loss for an MLP trained \textit{without} regularization on the \textit{Water Potability} dataset. 
\begin{figure}[ht]
    \centering
    \includegraphics[width=0.85\linewidth]{img/delta_loss_epochs.png}
    \caption{The average counterfactual perturbation vector $\overline{||\perturb||}$ (left $y$-axis) and the cross-entropy test loss (right $y$-axis) over training epochs ($x$-axis) for an MLP trained on the \textit{Water Potability} dataset \textit{without} regularization.}
    \label{fig:delta_loss_epochs}
\end{figure}

The plot shows a clear trend as the model starts to overfit the data (evidenced by an increase in test loss). 
Notably, $\overline{||\perturb||}$ begins to decrease, which aligns with the hypothesis that the average distance to the optimal counterfactual example gets smaller as the model's decision boundary becomes increasingly adherent to the training data.

It is worth noting that this trend is heavily influenced by the choice of the counterfactual generator model. In particular, the relationship between $\overline{||\perturb||}$ and the degree of overfitting may become even more pronounced when leveraging more accurate counterfactual generators. However, these models often come at the cost of higher computational complexity, and their exploration is left to future work.

Nonetheless, we expect that $\overline{||\perturb||}$ will eventually stabilize at a plateau, as the average $L_2$-norm of the optimal counterfactual perturbations cannot vanish to zero.

% Additionally, the choice of employing the score-based counterfactual explanation framework to generate counterfactuals was driven to promote computational efficiency.

% Future enhancements to the framework may involve adopting models capable of generating more precise counterfactuals. While such approaches may yield to performance improvements, they are likely to come at the cost of increased computational complexity.


\subsection{RQ2: Counterfactual Regularization Performance}
To answer \textbf{RQ2}, we evaluate the effectiveness of the proposed counterfactual regularization (CF-Reg) by comparing its performance against existing baselines: unregularized training loss (No-Reg), L1 regularization (L1-Reg), L2 regularization (L2-Reg), and Dropout.
Specifically, for each model and dataset combination, Table~\ref{tab:regularization_comparison} presents the mean value and standard deviation of test accuracy achieved by each method across 5 random initialization. 

The table illustrates that our regularization technique consistently delivers better results than existing methods across all evaluated scenarios, except for one case -- i.e., Logistic Regression on the \textit{Phomene} dataset. 
However, this setting exhibits an unusual pattern, as the highest model accuracy is achieved without any regularization. Even in this case, CF-Reg still surpasses other regularization baselines.

From the results above, we derive the following key insights. First, CF-Reg proves to be effective across various model types, ranging from simple linear models (Logistic Regression) to deep architectures like MLPs and CNNs, and across diverse datasets, including both tabular and image data. 
Second, CF-Reg's strong performance on the \textit{Water} dataset with Logistic Regression suggests that its benefits may be more pronounced when applied to simpler models. However, the unexpected outcome on the \textit{Phoneme} dataset calls for further investigation into this phenomenon.


\begin{table*}[h!]
    \centering
    \caption{Mean value and standard deviation of test accuracy across 5 random initializations for different model, dataset, and regularization method. The best results are highlighted in \textbf{bold}.}
    \label{tab:regularization_comparison}
    \begin{tabular}{|c|c|c|c|c|c|c|}
        \hline
        \textbf{Model} & \textbf{Dataset} & \textbf{No-Reg} & \textbf{L1-Reg} & \textbf{L2-Reg} & \textbf{Dropout} & \textbf{CF-Reg (ours)} \\ \hline
        Logistic Regression   & \textit{Water}   & $0.6595 \pm 0.0038$   & $0.6729 \pm 0.0056$   & $0.6756 \pm 0.0046$  & N/A    & $\mathbf{0.6918 \pm 0.0036}$                     \\ \hline
        MLP   & \textit{Water}   & $0.6756 \pm 0.0042$   & $0.6790 \pm 0.0058$   & $0.6790 \pm 0.0023$  & $0.6750 \pm 0.0036$    & $\mathbf{0.6802 \pm 0.0046}$                    \\ \hline
%        MLP   & \textit{Adult}   & $0.8404 \pm 0.0010$   & $\mathbf{0.8495 \pm 0.0007}$   & $0.8489 \pm 0.0014$  & $\mathbf{0.8495 \pm 0.0016}$     & $0.8449 \pm 0.0019$                    \\ \hline
        Logistic Regression   & \textit{Phomene}   & $\mathbf{0.8148 \pm 0.0020}$   & $0.8041 \pm 0.0028$   & $0.7835 \pm 0.0176$  & N/A    & $0.8098 \pm 0.0055$                     \\ \hline
        MLP   & \textit{Phomene}   & $0.8677 \pm 0.0033$   & $0.8374 \pm 0.0080$   & $0.8673 \pm 0.0045$  & $0.8672 \pm 0.0042$     & $\mathbf{0.8718 \pm 0.0040}$                    \\ \hline
        CNN   & \textit{CIFAR-10} & $0.6670 \pm 0.0233$   & $0.6229 \pm 0.0850$   & $0.7348 \pm 0.0365$   & N/A    & $\mathbf{0.7427 \pm 0.0571}$                     \\ \hline
    \end{tabular}
\end{table*}

\begin{table*}[htb!]
    \centering
    \caption{Hyperparameter configurations utilized for the generation of Table \ref{tab:regularization_comparison}. For our regularization the hyperparameters are reported as $\mathbf{\alpha/\beta}$.}
    \label{tab:performance_parameters}
    \begin{tabular}{|c|c|c|c|c|c|c|}
        \hline
        \textbf{Model} & \textbf{Dataset} & \textbf{No-Reg} & \textbf{L1-Reg} & \textbf{L2-Reg} & \textbf{Dropout} & \textbf{CF-Reg (ours)} \\ \hline
        Logistic Regression   & \textit{Water}   & N/A   & $0.0093$   & $0.6927$  & N/A    & $0.3791/1.0355$                     \\ \hline
        MLP   & \textit{Water}   & N/A   & $0.0007$   & $0.0022$  & $0.0002$    & $0.2567/1.9775$                    \\ \hline
        Logistic Regression   &
        \textit{Phomene}   & N/A   & $0.0097$   & $0.7979$  & N/A    & $0.0571/1.8516$                     \\ \hline
        MLP   & \textit{Phomene}   & N/A   & $0.0007$   & $4.24\cdot10^{-5}$  & $0.0015$    & $0.0516/2.2700$                    \\ \hline
       % MLP   & \textit{Adult}   & N/A   & $0.0018$   & $0.0018$  & $0.0601$     & $0.0764/2.2068$                    \\ \hline
        CNN   & \textit{CIFAR-10} & N/A   & $0.0050$   & $0.0864$ & N/A    & $0.3018/
        2.1502$                     \\ \hline
    \end{tabular}
\end{table*}

\begin{table*}[htb!]
    \centering
    \caption{Mean value and standard deviation of training time across 5 different runs. The reported time (in seconds) corresponds to the generation of each entry in Table \ref{tab:regularization_comparison}. Times are }
    \label{tab:times}
    \begin{tabular}{|c|c|c|c|c|c|c|}
        \hline
        \textbf{Model} & \textbf{Dataset} & \textbf{No-Reg} & \textbf{L1-Reg} & \textbf{L2-Reg} & \textbf{Dropout} & \textbf{CF-Reg (ours)} \\ \hline
        Logistic Regression   & \textit{Water}   & $222.98 \pm 1.07$   & $239.94 \pm 2.59$   & $241.60 \pm 1.88$  & N/A    & $251.50 \pm 1.93$                     \\ \hline
        MLP   & \textit{Water}   & $225.71 \pm 3.85$   & $250.13 \pm 4.44$   & $255.78 \pm 2.38$  & $237.83 \pm 3.45$    & $266.48 \pm 3.46$                    \\ \hline
        Logistic Regression   & \textit{Phomene}   & $266.39 \pm 0.82$ & $367.52 \pm 6.85$   & $361.69 \pm 4.04$  & N/A   & $310.48 \pm 0.76$                    \\ \hline
        MLP   &
        \textit{Phomene} & $335.62 \pm 1.77$   & $390.86 \pm 2.11$   & $393.96 \pm 1.95$ & $363.51 \pm 5.07$    & $403.14 \pm 1.92$                     \\ \hline
       % MLP   & \textit{Adult}   & N/A   & $0.0018$   & $0.0018$  & $0.0601$     & $0.0764/2.2068$                    \\ \hline
        CNN   & \textit{CIFAR-10} & $370.09 \pm 0.18$   & $395.71 \pm 0.55$   & $401.38 \pm 0.16$ & N/A    & $1287.8 \pm 0.26$                     \\ \hline
    \end{tabular}
\end{table*}

\subsection{Feasibility of our Method}
A crucial requirement for any regularization technique is that it should impose minimal impact on the overall training process.
In this respect, CF-Reg introduces an overhead that depends on the time required to find the optimal counterfactual example for each training instance. 
As such, the more sophisticated the counterfactual generator model probed during training the higher would be the time required. However, a more advanced counterfactual generator might provide a more effective regularization. We discuss this trade-off in more details in Section~\ref{sec:discussion}.

Table~\ref{tab:times} presents the average training time ($\pm$ standard deviation) for each model and dataset combination listed in Table~\ref{tab:regularization_comparison}.
We can observe that the higher accuracy achieved by CF-Reg using the score-based counterfactual generator comes with only minimal overhead. However, when applied to deep neural networks with many hidden layers, such as \textit{PreactResNet-18}, the forward derivative computation required for the linearization of the network introduces a more noticeable computational cost, explaining the longer training times in the table.

\subsection{Hyperparameter Sensitivity Analysis}
The proposed counterfactual regularization technique relies on two key hyperparameters: $\alpha$ and $\beta$. The former is intrinsic to the loss formulation defined in (\ref{eq:cf-train}), while the latter is closely tied to the choice of the score-based counterfactual explanation method used.

Figure~\ref{fig:test_alpha_beta} illustrates how the test accuracy of an MLP trained on the \textit{Water Potability} dataset changes for different combinations of $\alpha$ and $\beta$.

\begin{figure}[ht]
    \centering
    \includegraphics[width=0.85\linewidth]{img/test_acc_alpha_beta.png}
    \caption{The test accuracy of an MLP trained on the \textit{Water Potability} dataset, evaluated while varying the weight of our counterfactual regularizer ($\alpha$) for different values of $\beta$.}
    \label{fig:test_alpha_beta}
\end{figure}

We observe that, for a fixed $\beta$, increasing the weight of our counterfactual regularizer ($\alpha$) can slightly improve test accuracy until a sudden drop is noticed for $\alpha > 0.1$.
This behavior was expected, as the impact of our penalty, like any regularization term, can be disruptive if not properly controlled.

Moreover, this finding further demonstrates that our regularization method, CF-Reg, is inherently data-driven. Therefore, it requires specific fine-tuning based on the combination of the model and dataset at hand.

\section{Discussion}
\section{Discussion of Assumptions}\label{sec:discussion}
In this paper, we have made several assumptions for the sake of clarity and simplicity. In this section, we discuss the rationale behind these assumptions, the extent to which these assumptions hold in practice, and the consequences for our protocol when these assumptions hold.

\subsection{Assumptions on the Demand}

There are two simplifying assumptions we make about the demand. First, we assume the demand at any time is relatively small compared to the channel capacities. Second, we take the demand to be constant over time. We elaborate upon both these points below.

\paragraph{Small demands} The assumption that demands are small relative to channel capacities is made precise in \eqref{eq:large_capacity_assumption}. This assumption simplifies two major aspects of our protocol. First, it largely removes congestion from consideration. In \eqref{eq:primal_problem}, there is no constraint ensuring that total flow in both directions stays below capacity--this is always met. Consequently, there is no Lagrange multiplier for congestion and no congestion pricing; only imbalance penalties apply. In contrast, protocols in \cite{sivaraman2020high, varma2021throughput, wang2024fence} include congestion fees due to explicit congestion constraints. Second, the bound \eqref{eq:large_capacity_assumption} ensures that as long as channels remain balanced, the network can always meet demand, no matter how the demand is routed. Since channels can rebalance when necessary, they never drop transactions. This allows prices and flows to adjust as per the equations in \eqref{eq:algorithm}, which makes it easier to prove the protocol's convergence guarantees. This also preserves the key property that a channel's price remains proportional to net money flow through it.

In practice, payment channel networks are used most often for micro-payments, for which on-chain transactions are prohibitively expensive; large transactions typically take place directly on the blockchain. For example, according to \cite{river2023lightning}, the average channel capacity is roughly $0.1$ BTC ($5,000$ BTC distributed over $50,000$ channels), while the average transaction amount is less than $0.0004$ BTC ($44.7k$ satoshis). Thus, the small demand assumption is not too unrealistic. Additionally, the occasional large transaction can be treated as a sequence of smaller transactions by breaking it into packets and executing each packet serially (as done by \cite{sivaraman2020high}).
Lastly, a good path discovery process that favors large capacity channels over small capacity ones can help ensure that the bound in \eqref{eq:large_capacity_assumption} holds.

\paragraph{Constant demands} 
In this work, we assume that any transacting pair of nodes have a steady transaction demand between them (see Section \ref{sec:transaction_requests}). Making this assumption is necessary to obtain the kind of guarantees that we have presented in this paper. Unless the demand is steady, it is unreasonable to expect that the flows converge to a steady value. Weaker assumptions on the demand lead to weaker guarantees. For example, with the more general setting of stochastic, but i.i.d. demand between any two nodes, \cite{varma2021throughput} shows that the channel queue lengths are bounded in expectation. If the demand can be arbitrary, then it is very hard to get any meaningful performance guarantees; \cite{wang2024fence} shows that even for a single bidirectional channel, the competitive ratio is infinite. Indeed, because a PCN is a decentralized system and decisions must be made based on local information alone, it is difficult for the network to find the optimal detailed balance flow at every time step with a time-varying demand.  With a steady demand, the network can discover the optimal flows in a reasonably short time, as our work shows.

We view the constant demand assumption as an approximation for a more general demand process that could be piece-wise constant, stochastic, or both (see simulations in Figure \ref{fig:five_nodes_variable_demand}).
We believe it should be possible to merge ideas from our work and \cite{varma2021throughput} to provide guarantees in a setting with random demands with arbitrary means. We leave this for future work. In addition, our work suggests that a reasonable method of handling stochastic demands is to queue the transaction requests \textit{at the source node} itself. This queuing action should be viewed in conjunction with flow-control. Indeed, a temporarily high unidirectional demand would raise prices for the sender, incentivizing the sender to stop sending the transactions. If the sender queues the transactions, they can send them later when prices drop. This form of queuing does not require any overhaul of the basic PCN infrastructure and is therefore simpler to implement than per-channel queues as suggested by \cite{sivaraman2020high} and \cite{varma2021throughput}.

\subsection{The Incentive of Channels}
The actions of the channels as prescribed by the DEBT control protocol can be summarized as follows. Channels adjust their prices in proportion to the net flow through them. They rebalance themselves whenever necessary and execute any transaction request that has been made of them. We discuss both these aspects below.

\paragraph{On Prices}
In this work, the exclusive role of channel prices is to ensure that the flows through each channel remains balanced. In practice, it would be important to include other components in a channel's price/fee as well: a congestion price  and an incentive price. The congestion price, as suggested by \cite{varma2021throughput}, would depend on the total flow of transactions through the channel, and would incentivize nodes to balance the load over different paths. The incentive price, which is commonly used in practice \cite{river2023lightning}, is necessary to provide channels with an incentive to serve as an intermediary for different channels. In practice, we expect both these components to be smaller than the imbalance price. Consequently, we expect the behavior of our protocol to be similar to our theoretical results even with these additional prices.

A key aspect of our protocol is that channel fees are allowed to be negative. Although the original Lightning network whitepaper \cite{poon2016bitcoin} suggests that negative channel prices may be a good solution to promote rebalancing, the idea of negative prices in not very popular in the literature. To our knowledge, the only prior work with this feature is \cite{varma2021throughput}. Indeed, in papers such as \cite{van2021merchant} and \cite{wang2024fence}, the price function is explicitly modified such that the channel price is never negative. The results of our paper show the benefits of negative prices. For one, in steady state, equal flows in both directions ensure that a channel doesn't loose any money (the other price components mentioned above ensure that the channel will only gain money). More importantly, negative prices are important to ensure that the protocol selectively stifles acyclic flows while allowing circulations to flow. Indeed, in the example of Section \ref{sec:flow_control_example}, the flows between nodes $A$ and $C$ are left on only because the large positive price over one channel is canceled by the corresponding negative price over the other channel, leading to a net zero price.

Lastly, observe that in the DEBT control protocol, the price charged by a channel does not depend on its capacity. This is a natural consequence of the price being the Lagrange multiplier for the net-zero flow constraint, which also does not depend on the channel capacity. In contrast, in many other works, the imbalance price is normalized by the channel capacity \cite{ren2018optimal, lin2020funds, wang2024fence}; this is shown to work well in practice. The rationale for such a price structure is explained well in \cite{wang2024fence}, where this fee is derived with the aim of always maintaining some balance (liquidity) at each end of every channel. This is a reasonable aim if a channel is to never rebalance itself; the experiments of the aforementioned papers are conducted in such a regime. In this work, however, we allow the channels to rebalance themselves a few times in order to settle on a detailed balance flow. This is because our focus is on the long-term steady state performance of the protocol. This difference in perspective also shows up in how the price depends on the channel imbalance. \cite{lin2020funds} and \cite{wang2024fence} advocate for strictly convex prices whereas this work and \cite{varma2021throughput} propose linear prices.

\paragraph{On Rebalancing} 
Recall that the DEBT control protocol ensures that the flows in the network converge to a detailed balance flow, which can be sustained perpetually without any rebalancing. However, during the transient phase (before convergence), channels may have to perform on-chain rebalancing a few times. Since rebalancing is an expensive operation, it is worthwhile discussing methods by which channels can reduce the extent of rebalancing. One option for the channels to reduce the extent of rebalancing is to increase their capacity; however, this comes at the cost of locking in more capital. Each channel can decide for itself the optimum amount of capital to lock in. Another option, which we discuss in Section \ref{sec:five_node}, is for channels to increase the rate $\gamma$ at which they adjust prices. 

Ultimately, whether or not it is beneficial for a channel to rebalance depends on the time-horizon under consideration. Our protocol is based on the assumption that the demand remains steady for a long period of time. If this is indeed the case, it would be worthwhile for a channel to rebalance itself as it can make up this cost through the incentive fees gained from the flow of transactions through it in steady state. If a channel chooses not to rebalance itself, however, there is a risk of being trapped in a deadlock, which is suboptimal for not only the nodes but also the channel.

\section{Conclusion}
This work presents DEBT control: a protocol for payment channel networks that uses source routing and flow control based on channel prices. The protocol is derived by posing a network utility maximization problem and analyzing its dual minimization. It is shown that under steady demands, the protocol guides the network to an optimal, sustainable point. Simulations show its robustness to demand variations. The work demonstrates that simple protocols with strong theoretical guarantees are possible for PCNs and we hope it inspires further theoretical research in this direction.

%\section*{Software and Data}

%If a paper is accepted, we strongly encourage the publication of software and data with the
%camera-ready version of the paper whenever appropriate. This can be
%done by including a URL in the camera-ready copy. However, \textbf{do not}
%include URLs that reveal your institution or identity in your
%submission for review. Instead, provide an anonymous URL or upload
%the material as ``Supplementary Material'' into the OpenReview reviewing
%system. Note that reviewers are not required to look at this material
%when writing their review.

% Acknowledgements should only appear in the accepted version.
%\section*{Acknowledgements}
%\clearpage
\section*{Impact Statement}
Our work advances the efficiency of aligning LLMs with human values by optimizing the way human preferences are queried. Since human feedback is costly and time-consuming, our approach can potentially reduce wasted effort on uninformative comparisons, maximizing the value of each annotation. By improving the efficiency of learning from human preferences, this research has the potential to accelerate the development of safer and more helpful AI systems.


\bibliography{references}
%\bibliographystyle{plainnat}


%%%%%%%%%%%%%%%%%%%%%%%%%%%%%%%%%%%%%%%%%%%%%%%%%%%%%%%%%%%%%%%%%%%%%%%%%%%%%%%
%%%%%%%%%%%%%%%%%%%%%%%%%%%%%%%%%%%%%%%%%%%%%%%%%%%%%%%%%%%%%%%%%%%%%%%%%%%%%%%
% APPENDIX
%%%%%%%%%%%%%%%%%%%%%%%%%%%%%%%%%%%%%%%%%%%%%%%%%%%%%%%%%%%%%%%%%%%%%%%%%%%%%%%
%%%%%%%%%%%%%%%%%%%%%%%%%%%%%%%%%%%%%%%%%%%%%%%%%%%%%%%%%%%%%%%%%%%%%%%%%%%%%%%
\newpage
\appendix
%\onecolumn

\clearpage
\section{Additional Experiment Results}
%\subsection{Two dimensional example}
\begin{figure}[htp]
    \centering
    \includegraphics[width=\linewidth]{Figs/2D_GM_illustration.pdf}
    \caption{\small Comparisons drawn by different strategies to better learn a 2D bimodal reward function. The heat map showed the current estimated function. Red crosses connected by black lines are compared pairs and gray dots on the left-most penal are candidate points to be compared.}
    \label{fig:what_were_compared-big}
\end{figure}

\newpage


\subsection{Comparing Annotation Efficiency on the \texttt{Helpful} Dataset}
\label{appdx:more_results_results_main}

\paragraph{In-Prompt Annotation} efficiency is provided in Figure~\ref{fig:results_main_helpful} (as supplementary of Figure~\ref{fig:results_main} in the main text).

\begin{figure}[h!]
    \centering
    \includegraphics[width=1.0\linewidth]{Figs/Main_xpromptFalse_hidden64_cand500_320_SPLIT_helpful.pdf}
    \caption{Comparing annotation efficiency of different methods. (\texttt{Helpful} Dataset, 3 Models, 8 Methods). First row: 1 - Spearman's Correlation (lower is better); second row: Best-of-N reward. Experiments are repeated with 5 seeds.}
    \label{fig:results_main_helpful}
\end{figure}

\paragraph{Cross-Prompt Annotation} efficiency is provided in Figure~\ref{fig:results_main_xprompt_helpful} (as supplementary of Figure~\ref{fig:results_main_xprompt} in the main text).

\begin{figure}[h!]
    \centering
    \includegraphics[width=1.0\linewidth]{Figs/Main_xpromptTrue_hidden64_cand500_320_SPLIT_helpful.pdf}
    \caption{\small Comparing annotation efficiency of different methods under the \textbf{Cross-Prompt} annotation setups. (\texttt{Helpful} Dataset, 3 Models, 8 Methods). First row: 1 - Spearman's Correlation (lower is better); second row: Best-of-N reward. Experiments are repeated with 5 seeds.}
    \label{fig:results_main_xprompt_helpful}
\end{figure}





\subsection{Annotation Batch Size}
\label{appdx:more_results_annotation_bs}
\paragraph{Results on All Models}
Due to the space limit of the main text, we deferred the experiment results with Gemma7B and the LLaMA3-8B model when studying the effect of different annotation batch sizes in the following Figures (Figure~\ref{fig:results_annotation_bs_gemma2b_zoom}, Figure~\ref{fig:results_annotation_bs_gemma7b_zoom}, Figure~\ref{fig:results_annotation_bs_llama38b_zoom}). 
To summarize the main takeaways --- we observe the same trend as we have observed with the Gemma2B model, the proposed methods achieve better performances in the small batch size setups (more online setups). The stability of small batch setups is in general higher than the large batch setups.


\begin{figure}[h!]
    \centering
    \includegraphics[width=0.8\linewidth]{Figs/CompareAnnoBatch_gemma2b_64_xpromptFalse_h=7.pdf}
    \caption{\small Investigating how annotation batch size choices affect learning performance of different methods. Model: Gemma 2B.}
    \label{fig:results_annotation_bs_gemma2b_zoom}
\end{figure}


\begin{figure}[h!]
    \centering
    \includegraphics[width=0.8\linewidth]{Figs/CompareAnnoBatch_gemma7b_64_xpromptFalse_h=7.pdf}
    \caption{\small Investigating how annotation batch size choices affect learning performance of different methods. Model: Gemma 7B.}
    \label{fig:results_annotation_bs_gemma7b_zoom}
\end{figure}

\begin{figure}[h!]
    \centering
    \includegraphics[width=0.8\linewidth]{Figs/CompareAnnoBatch_llama38b_64_xpromptFalse_h=7.pdf}
    \caption{\small Investigating how annotation batch size choices affect learning performance of different methods. Model: LLaMA3-8B.}
    \label{fig:results_annotation_bs_llama38b_zoom}
\end{figure}

\paragraph{Results with All Methods.}
In addition, we use the figures below (Figure~\ref{fig:results_annotation_bs_gemma2b_total}, Figure~\ref{fig:results_annotation_bs_gemma7b_total}, Figure~\ref{fig:results_annotation_bs_llama38b_total}) for a full analysis on the annotation batch size choices for all methods. For other methods, we do not observe a clear trend on the effect of increasing or decreasing annotation batch sizes.


\begin{figure}[h!]
    \centering
    \includegraphics[width=1.0\linewidth]{Figs/CompareAnnoBatch_gemma2b_64_xpromptFalse.pdf}
    \caption{\small Investigating how annotation batch size choices affect learning performance of different methods. Model: Gemma 2B.}
    \label{fig:results_annotation_bs_gemma2b_total}
\end{figure}


\begin{figure}[h!]
    \centering
    \includegraphics[width=1.0\linewidth]{Figs/CompareAnnoBatch_gemma7b_64_xpromptFalse.pdf}
    \caption{\small Investigating how annotation batch size choices affect learning performance of different methods. Model: Gemma 7B.}
    \label{fig:results_annotation_bs_gemma7b_total}
\end{figure}

\begin{figure}[h!]
    \centering
    \includegraphics[width=1.0\linewidth]{Figs/CompareAnnoBatch_llama38b_64_xpromptFalse.pdf}
    \caption{\small Investigating how annotation batch size choices affect learning performance of different methods. Model: LLaMA3 8B.}
    \label{fig:results_annotation_bs_llama38b_total}
\end{figure}

\newpage
\subsection{Compare Cross-Prompt Comparisons and In-Prompt Comparisons}
\label{appdx:more_results_xprompt-in-prompt_comparison}
In this section, we provide direct comparisons of learning efficiency when using cross-prompt annotations and in-prompt annotations. In most cases, annotating comparisons using cross-prompt comparison improves learning efficiency, and this can be observed across all methods. Specifically, with the entropy-based method, cross-prompt annotations bring a noticeable boost to learning efficiency and reward model performance.

\begin{figure}[h!]
    \centering
    \includegraphics[width=1.0\linewidth]{Figs/CompareXprompt_gemma2b_rnds80.pdf}
\caption{\small Cross-Prompt preference annotation improves overall annotation efficiency. Annotation batch size 500. Model: Gemma2B.}
    \label{fig:results_xprompt_inprompt_abs500_gemma2b}
\end{figure} 


\begin{figure}[h!]
    \centering
    \includegraphics[width=1.0\linewidth]{Figs/CompareXprompt_gemma7b_rnds80.pdf}
\caption{\small Cross-Prompt preference annotation improves overall annotation efficiency. Annotation batch size 500. Model: Gemma7B.}
    \label{fig:results_xprompt_inprompt_abs500_gemma7b}
\end{figure} 


\begin{figure}[h!]
    \centering
    \includegraphics[width=1.0\linewidth]{Figs/CompareXprompt_llama38b_rnds80.pdf}
    \caption{\small Cross-Prompt preference annotation improves overall annotation efficiency. Annotation batch size 500. Model: LLaMA3-8B.}
    \label{fig:results_xprompt_inprompt_abs500_llama38b}
\end{figure} 


% \begin{figure}[h!]
%     \centering
%     \includegraphics[width=1.0\linewidth]{Figs/CompareXprompt_gemma2b_rnds320.pdf}
% \caption{\small Cross-Prompt preference annotation improves overall annotation efficiency. Annotation batch size 125. Model: Gemma2B.}
%     \label{fig:results_xprompt_inprompt_abs125_gemma2b}
% \end{figure}\\


% \begin{figure}[h!]
%     \centering
%     \includegraphics[width=1.0\linewidth]{Figs/CompareXprompt_gemma7b_rnds320.pdf}
% \caption{\small Cross-Prompt preference annotation improves overall annotation efficiency. Annotation batch size 125. Model: Gemma7B.}
%     \label{fig:results_xprompt_inprompt_abs125_gemma7b}
% \end{figure}  \\


% \begin{figure}[h!]
%     \centering
%     \includegraphics[width=1.0\linewidth]{Figs/CompareXprompt_llama38b_rnds320.pdf}
%     \caption{\small Cross-Prompt preference annotation improves overall annotation efficiency. Annotation batch size 125. Model: LLaMA3-8B.}
%     \label{fig:results_xprompt_inprompt_abs125_llama38b}
% \end{figure} \\% 

\clearpage
\subsection{Hyper-Parameter Sensitivity Analysis}
\label{appdx:more_results_hyper_param_sensitivity}

\paragraph{Number of Candidate Numbers}
In the main text, our empirical pipeline starts by sampling $500$ candidates (\texttt{candidate number}) from the training prompts, and then randomly generates $20000$ pairs of comparisons using either in-prompt comparison or cross-prompt comparison. Then, we select \texttt{annotation batch size} number of comparisons to annotate. In this section, we evaluate the performance difference by using a larger \texttt{candidate number} $1000$. 

In experiments, we find those setups do not significantly change the performance of different methods. The performance of D-opt and Past-Aware D-opt are especially robust to those hyper-parameter choices.

% \begin{figure}[h!]
%     \centering
%     % \includegraphics[width=1.0\linewidth]{Figs/CompareCandidates_gemma2b_rnds40.pdf}
%     % \includegraphics[width=1.0\linewidth]{Figs/CompareCandidates_gemma2b_rnds80.pdf}
%     \includegraphics[width=1.0\linewidth]{Figs/CompareCandidates_gemma2b_rnds320.pdf}
%     \caption{\small Preference annotation with different \texttt{candidate number} choices. Annotation batch size 125. Model: Gemma2B.}
%     \label{fig:results_candidate_abs125_gemma2b}
% \end{figure}

\begin{figure}[h!]
    \centering
    \includegraphics[width=1.0\linewidth]{Figs/CompareCandidates_gemma2b_rnds80.pdf}
    \caption{\small Preference annotation with different \texttt{candidate number} choices. Annotation batch size 500. Model: Gemma2B.}
    \label{fig:results_candidate_abs500_gemma2b} 
\end{figure}
% \begin{figure}[h!]
%     \centering
%     \vspace{-1.3cm}
%     % \includegraphics[width=1.0\linewidth]{Figs/CompareCandidates_gemma2b_rnds40.pdf}
%     % \includegraphics[width=1.0\linewidth]{Figs/CompareCandidates_gemma2b_rnds80.pdf}
%     \includegraphics[width=1.0\linewidth]{Figs/CompareCandidates_gemma7b_rnds320.pdf}
%     \caption{\small Preference annotation with different \texttt{candidate number} choices. Annotation batch size 125. Model: Gemma2B.}
%     \label{fig:results_candidate_abs125_gemma7b} 
% \end{figure}

\begin{figure}[h!]
    \centering
    % \includegraphics[width=1.0\linewidth]{Figs/CompareCandidates_gemma2b_rnds40.pdf}
    % \includegraphics[width=1.0\linewidth]{Figs/CompareCandidates_gemma2b_rnds80.pdf}
    \includegraphics[width=1.0\linewidth]{Figs/CompareCandidates_gemma7b_rnds80.pdf}
    \caption{\small Preference annotation with different \texttt{candidate number} choices. Annotation batch size 500. Model: Gemma7B.}
    \label{fig:results_candidate_abs500_gemma7b} 
\end{figure}



% \begin{figure}[h!]
%     \centering
%     % \includegraphics[width=1.0\linewidth]{Figs/CompareCandidates_gemma2b_rnds40.pdf}
%     % \includegraphics[width=1.0\linewidth]{Figs/CompareCandidates_gemma2b_rnds80.pdf}
%     \includegraphics[width=1.0\linewidth]{Figs/CompareCandidates_llama38b_rnds320.pdf}
%     \caption{\small Preference annotation with different \texttt{candidate number} choices. Annotation batch size 125. Model: LLaMA3-8B.}
%     \label{fig:results_candidate_abs125_llama38b} 
% \end{figure}

\begin{figure}[h!]
    \centering
    % \includegraphics[width=1.0\linewidth]{Figs/CompareCandidates_gemma2b_rnds40.pdf}
    % \includegraphics[width=1.0\linewidth]{Figs/CompareCandidates_gemma2b_rnds80.pdf}
    \includegraphics[width=1.0\linewidth]{Figs/CompareCandidates_llama38b_rnds80.pdf}
    \caption{\small Preference annotation with different \texttt{candidate number} choices. Annotation batch size 500. Model: LLaMA3-8B.}
    \label{fig:results_candidate_abs500_llama38b} 
\end{figure} 


\newpage
\paragraph{Number of Hidden Units in 3-Layer MLPs}
In all main text experiments, we use 3-layer MLPs with $64$ \textbf{hidden units}. In this section, we evaluate the performance difference by using a larger \texttt{hidden unit} $128$. 


\begin{figure}[h!]
    \centering
    \includegraphics[width=1.0\linewidth]{Figs/CompareHidden_gemma2b_80_xpromptFalse.pdf}
    \caption{\small Experiments with different \texttt{hidden unit} choices. Annotation batch size 500. Model: Gemma 2B.}
    \label{fig:results_hidden_abs500_gemma2b} 
\end{figure} 


\begin{figure}[h!]
    \centering
    \includegraphics[width=1.0\linewidth]{Figs/CompareHidden_gemma7b_80_xpromptFalse.pdf}
    \caption{\small Experiments with different \texttt{hidden unit} choices. Annotation batch size 500. Model: Gemma 7B.}
    \label{fig:results_hidden_abs500_gemma7b} 
\end{figure} 

\begin{figure}[h!]
    \centering
    \includegraphics[width=1.0\linewidth]{Figs/CompareHidden_llama38b_80_xpromptFalse.pdf}
    \caption{\small Experiments with different \texttt{hidden unit} choices. Annotation batch size 500. Model: LLaMA3-8B.}
    \label{fig:results_hidden_abs500_llama38b} 
\end{figure} 




\clearpage
\section{Further discussions}
\subsection{Cross-Prompt Annotations.}
\label{app:futherdisc}
Cross-Prompt annotations were explored as a way to increase annotation quality by \citet{sun2024rethinking} and empirically verified in \citet{yin2024relative}. A natural question is whether this is possible in practice. If one is willing to assume there exists a scalar value reward function, and human comparisons are based on that function, then cross-prompt is possible because each prompt-response pairs are assigned a real value that are comparable to each other. A single-word change in the prompt without changing its meaning likely will not change what responses are helpful or harmful and make these pairs, even if cross-prompt, comparable. It is possible however, that the reward function is very rough in changing prompts making the reward function for one prompt not transferable to the other and hard to get a better response for one prompt using a reward function learned from other prompts. Even though, if one is willing to believe that prompts live in some lower dimensional manifold and the reward function acquires some regularity in that space, Cross-Prompt annotations might help better learn these dependencies. 


\end{document}