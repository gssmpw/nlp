\clearpage
\section{Further discussions}
\subsection{Cross-Prompt Annotations.}
\label{app:futherdisc}
Cross-Prompt annotations were explored as a way to increase annotation quality by \citet{sun2024rethinking} and empirically verified in \citet{yin2024relative}. A natural question is whether this is possible in practice. If one is willing to assume there exists a scalar value reward function, and human comparisons are based on that function, then cross-prompt is possible because each prompt-response pairs are assigned a real value that are comparable to each other. A single-word change in the prompt without changing its meaning likely will not change what responses are helpful or harmful and make these pairs, even if cross-prompt, comparable. It is possible however, that the reward function is very rough in changing prompts making the reward function for one prompt not transferable to the other and hard to get a better response for one prompt using a reward function learned from other prompts. Even though, if one is willing to believe that prompts live in some lower dimensional manifold and the reward function acquires some regularity in that space, Cross-Prompt annotations might help better learn these dependencies. 