\section{Major Results}
In this section, we present both theoretical and numerical results. We begin by introducing theorems that establish stability and instability conditions, followed by their application in stability verification. Next, we provide examples illustrating the alignment of our theorems with numerical simulations. We also discuss the insights gained from our proposed methods. 

\subsection{Stability \& instability conditions}

We present two theorems below. Theorem~\ref{thm_2} states one stability condition, derived using the Foster-Lyapunov criterion \cite{meyn2012markov}, while Theorem~\ref{thm_3} provides one instability condition, based on the transience property of Markov chains \cite{meyn2012markov}.
\begin{thm}
\label{thm_2}
The system \eqref{eq_sys_1}-\eqref{eq_sys_2} is stable if there exists a vector $\theta:=[\theta_{e_1},\theta_{e_2}]^{\mathrm{T}}\in[0, 1]^2$ and a negative scalar $\gamma<0$ such that
\begin{align}
    &\bar{D}- \sum_{e\in\{e_1,e_2\}} (1-\theta_{e})\mathbb{E}[q_{e}(x, p, C)]- \sum_{e\in\{e_1,e_2\}}\theta_e f_e(x) < \gamma, ~\forall x \in\{x\in\mathcal{X}|x_{e_0}=x_{e_0}^c\}, 
    \label{eq_thm2_1}
\end{align}
where $x_{e_0}^c$ is given by \eqref{eq_critical} and $\mathbb{E}[q_{e}(x, p, C)] := \int_{\mathcal{C}} q_{e}(x, p, c)) \Gamma_{x,p}(\mathrm{d}c)$.
\end{thm}

\begin{thm}
\label{thm_3}
The system \eqref{eq_sys_1}-\eqref{eq_sys_2} is unstable if there exists a vector $\theta:=[\theta_{e_1},\theta_{e_2}]^{\mathrm{T}}\in[0,1]^2$ and a non-negative scalar $\gamma\geq0$ such that
\begin{align}
    &\bar{D}- \sum_{e\in\{e_1,e_2\}} (1-\theta_{e})\mathbb{E}[q_{e}(x, p, C)] - \sum_{e\in\{e_1,e_2\}}\theta_{e} f_{e}(x) \geq \gamma, ~\forall x \in\{x\in\mathcal{X}|x_{e_0}=x_{e_0}^c\}. \label{eq_thm3_1}
\end{align}
\end{thm}

Theorem~\ref{thm_2} (resp. Theorem~\ref{thm_3}) essentially says that the network is stable (resp. unstable) if the weighted expected net flow is negative (resp. non-negative) over the traffic state space $\{x\in\mathcal{X}|x_{e_0}=x_{e_0}^c\}$. One can implement Theorem~\ref{thm_2} by solving the following Semi-Infinite Programming (SIP \cite{stein2012solve}):
\begin{equation*}
    (P_1)~ \min_{\theta,\gamma}~\gamma~\text{subject to}~\eqref{eq_thm2_1}.
\end{equation*}
If the optimal $\gamma^*$ is negative, the stability is concluded. Similarly, the instability verification requires solving the SIP: 
\begin{equation*}
    (P_2)~ \max_{\theta,\gamma}~\gamma~\text{subject to}~\eqref{eq_thm3_1}.
\end{equation*}
If the optimal $\gamma^*$ is non-negative, we say that the system \eqref{eq_sys_1}-\eqref{eq_sys_2} is unstable.

\subsection{Numerical examples}

The following presents settings in our numerical examples. First, the demand $D(t)$ is assumed to be uniformly distributed on $[D^{\min}, D^{\max}]$. Then, the sending and receiving flows are specified by $f_{e}(x) = \min\{v_{e}x_{e}, Q_{e}\}$ for $e\in\{e_0, e_1, e_2\}$ and $r_{e}(x) = \min\{R_{e}-w_{e}x_{e},  Q_{e}\}$ for $e\in\{e_1, e_2\}$, respectively. 
We consider a fixed routing ratio based on the link capacities $Q_{e_1}$ and $Q_{e_2}$, namely  $\alpha:=Q_{e_1}/(Q_{e_1}+Q_{e_2})$. The compliance rate $C_e(x,p)$ is uniformly distributed on  $[\max\{\bar{C}_e(x, p)-\epsilon_e, 0\}, \min\{\bar{C}_e(x, p)+\epsilon_e, 1\}]$, where $\bar{C}_e(x, p)$ is given by
\begin{equation*}
    \bar{C}_e(x, p) := \frac{1}{1+e^{\beta_{e}^0+\beta_{e}^1 x_{e_1} + \beta_{e}^2 x_{e_2} + \beta_{e}^3 p}}.    
\end{equation*}
The parameters are summarized in Table~\ref{tab_para}. Note that negative $\beta_{e_1}^0$ and positive $\beta_{e_2}^0$ indicate that drivers naturally prefer the corridor.
\begin{table}[htbp]
    \centering
    \caption{Parameter settings.}
    \small
    \begin{tabular}{ cc|cc|cc|cc|cc }
    \hline
    $v_{e_0}$ & 80 (km/h)   & $v_{e_1}$ & 100 (km/h)  & $v_{e_2}$ & 50 (km/h)  & $\beta_{e_1}^0$ & $-4$ & $\beta_{e_2}^0$ & $1$ \\ 
    $Q_{e_0}$ & 8000 (veh/h) & $Q_{e_1}$ & 4000 (veh/h) & $Q_{e_2}$ & 2000 (veh/h)  & $\beta_{e_1}^1$ & 0.01 & $\beta_{e_2}^1$ & $-0.02$ \\ 
     $D^{\min}$  & 4000 (veh/h)     & $R_{e_1}$ &  4800 (veh/h)    & $R_{e_2}$ & 2400 (veh/h) &  $\beta_{e_1}^2$ & $-0.02$ & $\beta_{e_2}^2$ & $0.03$ \\ 
     $D^{\max}$      & [5000, 8000] (veh/h)  & $w_{e_1}$ & 20 (km/h)   & $w_{e_2}$ & 10 (km/h)  & $\beta_{e_1}^3$ & $0.3$ & $\beta_{e_2}^3$ & $-0.6$ \\ 
           & & & & & & $\epsilon_{e_1}$ & 0.1 & $\epsilon_{e_2}$ & 0.1  \\
    \hline
    \end{tabular}
    \label{tab_para}
\end{table}

\subsubsection{Impacts of tolls and demands}
Figure~\ref{fig_demand_pricing}(a) shows the time-average traffic densities after $10^4$ steps and reveals the stability and instability regions. The white boundary is obtained from Theorem~\ref{thm_2}, while the red boundary is derived from Theorem~\ref{thm_3}. Therefore, we can conclude that the region to the left of the white boundary is stable, whereas the areas in the upper right and lower right corners are unstable. These findings are consistent with the numerical results. Note that there is a gap between the white and red boundaries, within which stability or instability cannot be determined. This is because we do not have sufficient and necessary stability conditions. In fact the gap is not a concern, as it can be narrowed by using more advanced Lyapunov or test functions, though at the expense of increased computational cost. 

The key findings from Figure~\ref{fig_demand_pricing} are summarized as follows. First, setting the toll $p$  either too low or too high can result in network instability. Second, in the case study, a toll of approximately 5 veh/\$ is identified as optimal for maximizing the lower bound of throughput.

\begin{figure}[htbp]
    \centering
    \begin{subfigure}{0.25\linewidth}
        \centering
        \includegraphics[width=\linewidth]{Images/case1.pdf}
    \caption{Stability regions.}
    \end{subfigure}
    \begin{subfigure}{0.25\linewidth}
        \centering
        \includegraphics[width=\linewidth]{Images/Throughput.pdf}
    \caption{Throughput bounds.}
    \end{subfigure}

    \caption{Impact analysis of tolls and expected demands.}
    \label{fig_demand_pricing}
\end{figure}



\subsubsection{Impacts of variances of compliance rates}
Figure~\ref{fig_variance} illustrates the impacts of compliance rate variances by keeping the same $\bar{C}_{e_2}(x,p)$ but  selecting different $\epsilon_{e_2}$. From the upper right corners of Figures~\ref{fig_variance}(a)-(c), we can see the instability regions enlarge as $\epsilon_{e_2}$ increases. This  demonstrates that uncertainties in compliance rates may bring negative impacts on traffic management. From the lower right corners of Figures~\ref{fig_variance}(a)-(c), it is interesting to observe that, for the same level of demand, increasing tolls can stabilize a previously unstable network as $\epsilon_{e_2}$ increases. This result is reasonable since more uncertainties indicate higher tolls to persuade drivers to choose link $e_2$. 

More importantly, our white and red boundaries in Figures~\ref{fig_variance}(a)-(c) capture those necessary changes. This demonstrates that our developed theorems offer practical yet powerful tools for evaluating traffic systems without the need for extensive simulations.

\begin{figure}[htbp]
    \centering
    \begin{subfigure}{0.25\linewidth}
        \centering
        \includegraphics[width=\linewidth]{Images/eps2_0.pdf}
    \caption{$\epsilon_{e_2}=0$.}
    \end{subfigure}
    \begin{subfigure}{0.25\linewidth}
        \centering
        \includegraphics[width=\linewidth]{Images/eps2_02.pdf}
    \caption{$\epsilon_{e_2}=0.2$.}
    \end{subfigure}
    \begin{subfigure}{0.25\linewidth}
        \centering
        \includegraphics[width=\linewidth]{Images/eps2_04.pdf}
    \caption{$\epsilon_{e_2}=0.4$.}
    \end{subfigure}
    \caption{Stability and instability regions under different $\epsilon_{e_2}$.}
    \label{fig_variance}
\end{figure}