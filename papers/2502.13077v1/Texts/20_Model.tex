\section{Problem Statement}
In this section, we first present our model formulation. Then we introduce the formal definitions of stability and throughput, which are closely related to our research problems. 

\subsection{Model formulation}

The considered network comprises links $e_0$, $e_1$ and $e_2$, as shown in Figure~\ref{fig_twolink}(b). Each link $e\in\{e_0,e_1,e_2\}$ is characterized by a length $l_e$ and a state of traffic density $x_e(t)$ at time $t$. Particularly, link $e_0$ serves a buffer receiving the upstream demand $D(t)\in\mathcal{D}$. We assume that $D(t)$ is governed by a stationary stochastic process with $\mathbb{E}[D(t)]=\Bar{D}$. Following the convention \cite{daganzo1995cell}, the buffer is assumed to have infinite storage, indicating $x_{e_0}(t)\in [0, \infty)$. Link $e_0$ is also associated with a bounded and non-decreasing sending flow $f_{e_0}:[0, \infty)\to[0, Q_{e_0}]$, where $Q_{e_0}$ denotes the capacity. We define its \emph{critical density} as 
\begin{equation}
    x_{e_0}^c:=\inf\{x|f_{e_0}(x)=Q_{e_0}\}, \label{eq_critical}
\end{equation}
which represents the lowest traffic density at which the sending flow $f_{e_0}$ attains its capacity. As for link $e\in\{e_1, e_2\}$, it only has finite storage such that $x_e\in[0, x_e^{\max}]$ where $x_e^{\max}$ is interpreted as the \emph{jam density}. In addition to the bounded and non-decreasing sending flow $f_{e}:[0, x_e^{\max}]\to[0, Q_e]$, link $e\in\{e_1, e_2\}$ also has a bounded and non-increasing receiving flow $r_{e}:[0, x_e^{\max}]\to[0, Q_e]$. For notational convenience, we let 
$\mathcal{X}:=[0,\infty)\times[0, x_{e_1}^{\max}]\times[0, x_{e_2}^{\max}]$ and $x:=[x_{e_0}, x_{e_1}, x_{e_2}]^{\mathrm{T}}\in\mathcal{X}$.

For traffic management strategies, we first consider a routing policy, denoted by $\alpha:\mathcal{X}\to[0,1]$, specifying the proportion of vehicles assigned onto link $e_1$. The routing policy $\alpha(x)$ is assumed to be non-increasing with respect to $x_{e_1}$ and non-decreasing with respect to $x_{e_2}$. Besides, a fixed toll $p\geq0$ is set on link $e_1$ for each passing vehicle. Then, we model drivers' response to the traffic management. Particularly, we denote by $C_e(x,p)$ the compliance rate regarding the routing instruction to link $e\in\{e_1,e_2\}$, which generally depends on the traffic state $x$ and the fee $p$. We consider that $C(x,p):=[C_{e_1}(x,p), C_{e_2}(x,p)]^{\mathrm{T}}$ is a random vector conditioned on $x$ and $p$, with a distribution $\Gamma_{x,p}$ supported on $\mathcal{C}\subseteq[0,1]^2$. We also assume i) that $\mathbb{E}[C_{e_1}(x,p)]$ is non-increasing with respect to $x_{e_1}$ and $p$, and non-decreasing with respect to $x_{e_2}$, and ii) that $\mathbb{E}[C_{e_2}(x,p)]$ is non-decreasing with respect to $x_{e_1}$ and $p$, and non-increasing with respect to $x_{e_2}$.

The conservation law yields the following dynamics:
\begin{subequations}
    \begin{align}
        x_{e_0}(t+1) =& x_{e_0}(t) + \frac{\delta_t}{l_{e_0}}\Big(D(t)-q_{e_1}(x,p,C)-q_{e_2}(x,p,C)\Big), \label{eq_sys_1} \\
        x_{e}(t+1) =& x_{e}(t) + \frac{\delta_t}{l_{e}}\Big(q_{e}(x,p,C)-f_{e}(x_e)\Big),~ e\in\{e_1, e_2\}, \label{eq_sys_2}
    \end{align}
\end{subequations}
where $\delta_t$ is the time step size, and the flow from link $e_0$ to link $e\in\{e_1, e_2\}$, denoted by $q_{e}(x, p, C)$, is given below. 
\begin{subequations}
    \begin{align}
            q_{e_1}(x, p, C) =& \min\Big\{\Big(\alpha(x)C_{e_1}(x,p) + (1-\alpha(x))(1-C_{e_2}(x,p))\Big)f_{e_0}(x_{e_0}), r_{e_1}(x_{e_1})\Big\}, \label{eq_q_1} \\
            q_{e_2}(x, p, C) =& \min\Big\{\Big(\alpha(x)(1-C_{e_1}(x,p)) + (1-\alpha(x))C_{e_2}(x,p)\Big)f_{e_0}(x_{e_0}), r_{e_2}(x_{e_2})\Big\}. \label{eq_q_2}
    \end{align}
\end{subequations}
Clearly, we obtain a nonlinear stochastic system \eqref{eq_sys_1}-\eqref{eq_sys_2} that is a Markov chain. 

Now we briefly discuss how the toll $p$ influences the inter-link flows $q_{e_1}(x, p, C)$ and $q_{e_2}(x, p, C)$. As mentioned in previous section, this paper considers that drivers may resist being redirected to local streets. When the toll $p$ is low, the compliance rate $C_{e_1}(x,p)$ is high but $C_{e_2}(x,p)$ could be low, consequently compromising the routing policy $\alpha$. However, extremely high tolls may render low $C_{e_1}(x,p)$ and high $C_{e_2}(x,p)$, which also nullifies traffic routing.  

\subsection{Stability and throughput}
The following gives the definition of stability considered in this paper. 
\begin{dfn}[Stability \& Instability] \label{dfn_sta}
A stochastic process $\{Y(t):t\geq0\}$ with a state space $\mathcal{Y}$ is \emph{stable} if there exists a scalar $Z<\infty$ such that for any initial condition $y\in\mathcal{Y}$
\begin{equation}\label{eq_bounded}
  \limsup_{t\to\infty}\frac{1}{t}\sum_{\tau=0}^t\mathbb {E}[|Y(\tau)||Y(0)=y] \le Z,
\end{equation}
where $|Y(\tau)|$ denotes 1-norm of $Y(\tau)$. The network is \emph{unstable} if there does not exist $Z<\infty$ such that \eqref{eq_bounded} holds for any initial condition $y\in\mathcal{Y}$.
\end{dfn}

The stability above is widely used in studying traffic control \cite{barman2023throughput}. It indicates that the time-average traffic density is bounded in the long term. Obviously, in practice one is more concerned about traffic performance within finite time (e.g. peak hours). Although Definition~\ref{dfn_sta} simplifies the analysis of real-world traffic systems, our later numerical examples illustrate that methods based on this definition are sufficient to produce insightful results for evaluating and designing management strategies. Moreover, this establishes a foundation for future research on refined finite-time stability \cite{hong2010finite}.

The \emph{throughput} $\Bar{D}^{\alpha,p}$ of the network, given the routing policy $\alpha$ and the toll $p$, is defined as the maximal expected demand that the network can accept while maintaining stability: 
\begin{equation*}
    \Bar{D}^{\alpha,p}:=\sup\Bar{D}\quad \text{subject to the system \eqref{eq_sys_1}-\eqref{eq_sys_2} is stable.} 
\end{equation*}

Our research problems are i) how to verify whether the system \eqref{eq_sys_1}-\eqref{eq_sys_2} under the routing and pricing policies satisfies the condition \eqref{eq_bounded}, and ii) how to select appropriate $p$ to maximize the throughput. 