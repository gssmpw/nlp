\section{Methodology}\label{sec:approach}


\subsection{Overview}\label{subsec:overview}
MonoEmbed follows, overall, the same structure as most decomposition approaches. Fig     \ref{fig:overview} showcases two main components: \textbf{Analysis} and \textbf{Inference}. The design and application of our approach is done in two phases.

The first phase revolves around training and preparing the LM of the \textbf{Analysis} component. It involves the creation of a triplet samples dataset based on Microservices applications, the selection between different Pre-Trained LMs and the fine-tuning of the model through a Contrastive Learning method.

The second phase showcases the usage of MonoEmbed on monolithic applications with the Fine-Tuned LM. In this phase, the \textbf{Analysis} component processes the monolith's source code using the LM, generating a feature matrix  \( (N{\times}M) \), where \( (N) \) is the number of classes and \( (M) \) is the embedding vector size. The \textbf{Inference} component then normalizes and partitions this data with a clustering algorithm to produce the decomposition, a set of microservices with distinct classes. For this research, we limit our input to source code at the class level granularity.



\begin{figure}[ht]
\centering
\includegraphics[width=\linewidth]{img/03/fig_overview.pdf}
\caption{An overview of the suggested decomposition approach.} \label{fig:overview}
\end{figure}


% \subsection{Selecting the Embedding Models}
\subsection{Model Selection}\label{subsec:mselection}





While various analysis methods could satisfy our definition of the analysis component, even including static analysis approaches, this research primarily focuses on utilizing Language Models to transform source code into meaningful feature vectors. A study of the usage of Pre-Trained Models \cite{niu2023codeembedreview} in Software Enginnering in 2023 showcases the large number of potential models, which has been growing significantly ever since. For this reason, the selection of an appropriate LM from the diverse array of options is complex, considering factors such as source code nature, desired representation granularity, and computational efficiency, all of which can impact decomposition task performance. To facilitate understanding and inform the selection process, we provide an overview of potential embedding models, categorizing them by types and characteristics. Table \ref{tab:ptmodels} summarizes the models used and their respective categories. It includes the following columns: \textit{Size} (approximate number of parameters), \textit{Context} (maximum input tokens), \textit{Code} (Pre-Trained on source code), and \textit{Modality} (acceptable input types for each model).




\begin{table}[]
\caption{Metadata of the potential Pre-Trained Language Models}
\label{tab:ptmodels}
\resizebox{\linewidth}{!}{%
\begin{tabular}{llllll}
\hline
Model Name                                              & Group           & Size & Context & Code & Modality    \\ \hline
Code2Vec \cite{alon2018code2vec}               & Static       & \textless{}1M & -    & Yes & AST paths \\
Code2Seq \cite{alon2019code2seq}               & RNN    & \textless{}1M & -    & Yes & AST paths \\
CuBERT \cite{kanade2020cubert}         & ET    & 340M & 2048        & Yes      & PL          \\
CodeBERT \cite{feng2020codebert}       & ET    & 125M & 512         & Yes      & NL-PL       \\
GraphCodeBERT \cite{guo2021graphcodebert}      & ET & 125M                           & 512  & Yes & DF-NL-PL  \\
UniXcoder \cite{guo2022unixcoder}              & ET & 125M                           & 1024 & Yes & DF-NL-PL  \\
CodeT5+ \cite{wang2023codet5p}         & EDT & 110M & 512         & Yes      & PL          \\
Meta Llama 3 \cite{aimeta2024llama3modelcard}     & DT & 8B                           & 8K  & No & NL        \\
SFR Mistral \cite{SFRAIResearch2024sfrmistral} & DT & 7B                             & 4096 & No  & NL        \\
E5 Mistral \cite{wang2024e5mistral}    & DT    & 7B   & 4096        & No       & NL          \\
CodeLlama \cite{rozière2024codellama}  & DT    & 7B   & 100K        & Yes      & NL          \\
DeepSeek Coder \cite{guo2024deepseekcoder}     & DT & 6.7B                           & 16K  & Yes & NL        \\
GritLM \cite{muennighoff2024gritlm}    & LME    & 7B   & 4096        & No       & Instruct-NL          \\
NVEmbed \cite{lee2024nvembed}          & LME             & 7B   & 4096        & No       & Instruct-NL \\
LLM2Vec \cite{parishad2024llm2vec}     & LME             & 8B   & 512         & No       & Instruct-NL \\
VoyageAI \cite{voyageai2024embeddings} & CEM   & -    & 16K         & No       & NL          \\
OpenAI \cite{openai2024embeddings}     & CEM   & -    & 8191        & No       & NL          \\
Cohere \cite{cohere2024embeddings}     & CEM   & -    & 512         & No       & NL          \\ \hline
\end{tabular}%
}
\end{table}





\subsubsection{Static and RNN based EMs}
\textit{Code2Vec} \cite{alon2018code2vec} is a static EM that was tailored for encoding code sample ASTs into static vector representations. It was extended by \textit{Code2Seq} \cite{alon2019code2seq}, a RNN-based EM, to incorporate input context.

\subsubsection{Encoder-only Transformers (ET)}
In our research, we considered several prominent ET models trained on software engineering problems and source code \cite{niu2023codeembedreview}. We selected \textit{CuBERT} \cite{kanade2020cubert} for its Java source code pairs modality. \textit{CodeBERT} \cite{feng2020codebert} and \textit{GraphCodeBERT} \cite{guo2021graphcodebert} were chosen for their unique multi-modal approach, incorporating Natural Language (NL), Programming Language (PL), and Data-Flow (DF) information. Additionally, we include \textit{UniXcoder} \cite{guo2022unixcoder}, which, although based on \textit{RoBERTa} \cite{liu2019roberta} like its predecessors, stands out as a unified model trained on both representation and generative tasks, offering enhanced potential for fine-tuning.
\subsubsection{Encoder-Decoder Transformers (EDT)}
\textit{CodeT5+} \cite{wang2023codet5p} is an Encoder-Decoder LLM that was trained on a large and diverse set of objectives so that it efficiently adapts to downstream tasks. We include its encoder in our evaluation.
\subsubsection{Decoder-only Transformers (DT)}
While DT LLMs are primarily designed for generative tasks, certain models \cite{wang2024e5mistral} have demonstrated strong performance in representation benchmarks, notably the Massive Text Embedding Benchmark (MTEB)\footnote{\label{footnote:mteb}\url{https://huggingface.co/spaces/mteb/leaderboard}} \cite{muennighoff2022mteb}. Our study encompasses a diverse range of DT models, as detailed in Table \ref{tab:ptmodels}, spanning from general-purpose models \cite{aimeta2024llama3modelcard} to those specialized for code-related tasks \cite{rozière2024codellama}.
\subsubsection{Language Model Embeddings (LME)}
We define LMEs as the DT LLMS that have been adapted for representation learning. We include state-of-the-art LMEs such as \textit{LLM2Vec} \cite{parishad2024llm2vec}, \textit{NV-Embed} \cite{lee2024nvembed} and \textit{GritLM} \cite{muennighoff2024gritlm}.
\subsubsection{Closed-source Embedding Models (CEM)}
CEMs refer to LLM based proprietary tools, like \textit{VoyageAI} \cite{voyageai2024embeddings}, designed to generate vector representations of text. They are accessed through APIs, enabling embedding generation without direct access to the underlying model. 

% \subsection{Fine-Tuning the Model}

\subsection{Fine-Tuning the Model}\label{subsec:finetune}




\begin{figure}
\centering
\includegraphics[width=\linewidth]{img/03/fig_cl_example.pdf}
\caption{An example of Contrastive Learning training.} \label{fig:clexample}
\end{figure}


The EM in our analysis is used to encode source code into vectors where classes from the same microservices have similar vectors, and those from different microservices are farther apart. Through the model selection step, we created a benchmark of Pre-Trained Models. Now, we propose a Contrastive Learning based fine-tuning approach to adapt these models and create embeddings suited for the decomposition task. Let's consider the example in Fig. \ref{fig:clexample}. Code samples \textit{A} and \textit{B} belong to the same microservice \textit{Car} while \textit{C} belongs to the microservice \textit{Person}. Due to the near-identical source code of \textit{A} and \textit{C}, Pre-Trained Model embeddings are closer together, with \textit{B} farther apart (Fig. \ref{fig:clexample} part 2). Using CL with a tailored objective function and dataset, the model should learn to prioritize the semantic similarity between \textit{A} and \textit{B} over the syntax similarity between \textit{A} and \textit{C}, as shown in part 3.

In particular, we employ Contrastive Learning with the triplet loss function \cite{florian2015triplet}. The input to this function is a triplet of Java classes: \textit{anchor}, \textit{positive}, and \textit{negative}. The \textit{anchor} and \textit{positive} belong to the same microservice while the \textit{negative} class belongs to a different microservice. Using \textit{hard negatives} in triplet CL has been shown to enhance the training \cite{florian2015triplet} where \textit{hard negatives} refer to samples that are similar to the \textit{anchor} but should not be regrouped together. This is why the \textit{negative}, in our case, must belong to the same application. In fact, source code samples from the same application often share structural and syntactical similarities. Thus, using them as \textit{hard negatives} encourages the model to focus on the semantics related to microservices and business logic instead. The triplet loss is defined as follows: 
\begin{equation}\label{eq:tripletloss}
L(a, p, n) = \max(0, \|a - p\|_2 - \|a - n\|_2 + \alpha)
\end{equation}
Where \(a\), \(p\) and \(n\) are the embeddings of the \textit{anchor}, \textit{positive} and \textit{negative} samples. \(\alpha\) is the margin between \textit{positive} and \textit{negative} pairs and \(\|\cdot\|_2\) denotes the Euclidean norm.





For the smaller models (e.g. UnixCoder), we train all of their weights. As for LLMs, we employ the Low-Rank Adaptation (LoRA) method \cite{edward2022lora} to significantly reduce the computational cost and memory requirements while keeping a competitive performance. LoRA has been utilized by recent state-of-the-art models \cite{parishad2024llm2vec,lee2024nvembed}. As for DT LLMs, we adopted the following instruction to enhance model performance \cite{muennighoff2024gritlm,lee2024nvembed,parishad2024llm2vec}:

\begin{center}
\resizebox{\columnwidth}{!}{
\texttt{Given the source code, retrieve the bounded contexts;}}
\end{center}


% \subsection{Creating the Datasets}
\subsection{Creating the Dataset}\label{subsec:dataset}







One of the biggest challenges in microservices decomposition research is the lack of data for supervised learning solutions \cite{oumoussa2024decompsurvey,Abgaz2023decompsurvey}. Monoliths rarely have a single definitive decomposition, making it difficult to rely on existing decompositions for model fine-tuning. However, the open-source ecosystem is abundant with Microservices solutions with various frameworks and diverse designs. We can leverage this fact to construct a dataset for training and evaluating the EMs which is only possible due to LLMs' generalization ability and Contrastive Learning self-supervised methods. The dataset creation process involves three steps:


\subsubsection{Application selection} focuses on choosing microservices repositories, primarily Java applications. Sources include a curated list \cite{rahman2019curatedset}, applications from decomposition research \cite{khaled2022hierdecomp,kalia2021mono2micro,desai2021cogcn,faustino2022stepwise}, and Github API queries. The latter includes only repositories with at least 10 Github stars, that have Java source files and that verify this regex:
\begin{center}
\resizebox{\columnwidth}{!}{\texttt{micro( |-)?services?( |-)(architecture|system|application)}}
\end{center}

\subsubsection{Repository Analysis} involves cloning and analyzing the selected repositories to filter out false positives. We define the sub-directory pattern \textit{"main/java"} as a microservice source root and remove projects with less than 2 microservices. For the remaining projects, we extract their classes and match their source code samples with the corresponding microservice.
\subsubsection{Triplets Sampling} creates the training dataset based on the extracted classes and microservices. Let  \( K \) be the maximum number of samples. We iteratively select random samples to create anchor, positive, and negative triplets. Anchor and positive classes are chosen from the same microservice within a randomly selected repository, while the negative class comes from a different service in the same repository. This process repeats  \( K \) times, after which duplicates are removed.

\subsection{Generating the Decompositions}\label{subsec:inference}
The \textbf{Inference} component in MonoEmbed is responsible for generating decompositions using the embeddings from the \textbf{Analysis} component. By relying on advanced LLMs and our fine-tuning process, the high-level and efficient embeddings will enable better performance while reducing the complexity of the inference. While more advanced approaches such as GNNs can be used for this component, the current implementation of this phase will focus on clustering algorithms. In fact, these algorithms often rely on distances and similarities between the feature vectors of their inputs. Which is why EMs that generate high quality embeddings are well suited with clustering algorithms. In addition, we employ a normalization step, using the z-score standardization method to scale the feature values within the embeddings matrix. This step helps mitigate the impact of variance between feature values.

The output of the \textbf{Inference} component is a vector of integer values which specifies the suggested microservices decomposition. As such, a decomposition is defined as $D = [M_1, M_2, ..., M_K]$. Each microservice in the decomposition is defined as a subset of classes  $M_i = \{c_j | j \in [0, N]\}$ and $|M_i| < N$ where $N$ is the number of classes.