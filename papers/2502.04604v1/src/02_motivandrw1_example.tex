\section{Motivating Example}\label{subsec:mexample}



A decomposition approach survey \cite{Abgaz2023decompsurvey} reveals that most monolithic application decomposition methods rely on static code analysis to extract component relationships, primarily between classes. Common approaches include generating Abstract Syntax Trees (AST) and call graphs, where nodes represent classes and edges represent invocations \cite{khaled2022hydecomp,desai2021cogcn}. The resulting adjacency matrix, referred to as \textit{ST-Calls}, represents the monolith's features. Semantic representations using Bag-of-Words (\textit{SM-BoW}) or Term Frequency - Inverse Document Frequency (\textit{SM-TFIDF}) vectors are commonly used as well.


\begin{figure*}
\centering
\includegraphics[width=0.9\linewidth]{img/03/fig_motivating_example.pdf}
\caption{A UMAP 2-dimensional projection of calls, interactions and bag-of-words representations.}
\label{fig:mexample}
\end{figure*}


Fig \ref{fig:mexample} illustrates a 2-D projection of three representations for the Spring PetClinic Microservices application \cite{microapps2024petclinic}, a microservices variant of a benchmark in decomposition research \cite{jin2021fosci,khaled2022hydecomp}. The figure shows classes as points, with colors and shapes indicating their respective microservices. The representations include ST-Calls, ST-Interaction (combining calls, inheritance, and variable references) \cite{khaled2022hierdecomp,khaled2022hydecomp}, and SM-BoW. The projection was generated using the Uniform Manifold Approximation and Projection (UMAP) algorithm \cite{mcinnes2020umapuniformmanifoldapproximation}. We can see that across all representations, most classes are scattered, making it difficult to discern clear microservice groupings. However, the \textit{ST-Interactions} representation captures relationships between some \textit{Customers} service classes. While the \textit{SM-BoW} representation shows groupings of some \textit{Vets}, \textit{Customers} and \textit{API-Gateway} microservice classes, they are too close to each other to differentiate clearly.
On the other hand, some groupings don't reflect actual microservices, as seen in the top-left cluster of \textit{SM-BoW}. Here, seven classes from different microservices cluster together. Upon closer inspection, these classes are service entry points with nearly identical code, differing only in names. It highlights a weakness of this semantic representation approach. All these representations have weaknesses that could hinder decomposition approaches. Their simple encodings alone are insufficient to create domain-relevant representations. Which is why we propose an approach based on Language Model representations.

