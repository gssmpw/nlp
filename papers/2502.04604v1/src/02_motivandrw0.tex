


% \subsection{Motivating example}
\section{Motivating Example}\label{subsec:mexample}



A decomposition approach survey \cite{Abgaz2023decompsurvey} reveals that most monolithic application decomposition methods rely on static code analysis to extract component relationships, primarily between classes. Common approaches include generating Abstract Syntax Trees (AST) and call graphs, where nodes represent classes and edges represent invocations \cite{khaled2022hydecomp,desai2021cogcn}. The resulting adjacency matrix, referred to as \textit{ST-Calls}, represents the monolith's features. Semantic representations using Bag-of-Words (\textit{SM-BoW}) or Term Frequency - Inverse Document Frequency (\textit{SM-TFIDF}) vectors are commonly used as well.


\begin{figure*}
\centering
\includegraphics[width=0.9\linewidth]{img/03/fig_motivating_example.pdf}
\caption{A UMAP 2-dimensional projection of calls, interactions and bag-of-words representations.}
\label{fig:mexample}
\end{figure*}


Fig \ref{fig:mexample} illustrates a 2-D projection of three representations for the Spring PetClinic Microservices application \cite{microapps2024petclinic}, a microservices variant of a benchmark in decomposition research \cite{jin2021fosci,khaled2022hydecomp}. The figure shows classes as points, with colors and shapes indicating their respective microservices. The representations include ST-Calls, ST-Interaction (combining calls, inheritance, and variable references) \cite{khaled2022hierdecomp,khaled2022hydecomp}, and SM-BoW. The projection was generated using the Uniform Manifold Approximation and Projection (UMAP) algorithm \cite{mcinnes2020umapuniformmanifoldapproximation}. We can see that across all representations, most classes are scattered, making it difficult to discern clear microservice groupings. However, the \textit{ST-Interactions} representation captures relationships between some \textit{Customers} service classes. While the \textit{SM-BoW} representation shows groupings of some \textit{Vets}, \textit{Customers} and \textit{API-Gateway} microservice classes, they are too close to each other to differentiate clearly.
On the other hand, some groupings don't reflect actual microservices, as seen in the top-left cluster of \textit{SM-BoW}. Here, seven classes from different microservices cluster together. Upon closer inspection, these classes are service entry points with nearly identical code, differing only in names. It highlights a weakness of this semantic representation approach. All these representations have weaknesses that could hinder decomposition approaches. Their simple encodings alone are insufficient to create domain-relevant representations. Which is why we propose an approach based on Language Model representations.



\section{Background and Related Work}

% \subsection{Background}
\subsection{Background}
\subsubsection{Embedding Models (EMs)} and in particular Neural-Network (NN) based Embedding Models like Word2Vec \cite{mikolov2013word2vec1} transform input sequences (e.g. word tokens, code samples) into continuous vector representations in a high-dimensional space. These embedding vectors capture semantic and syntactic relationships between the tokens, allowing similar concepts to have similar vector representations. 

\subsubsection{Language Models (LMs)} are probabilistic models that learn to predict the likelihood of a sequence of words. Modern LMs are based on NN architectures like Recurrent Neural Networks (RNN) \cite{alon2019code2seq} and Transformers \cite{devlin2019bert,liu2019roberta,radford2018gpt1}. Large Language Models (LLMs) often refers to Pre-Trained Neural LMs based on the Transformers \cite{vaswani2017attention} architecture and have more than 1 Billion parameters. While LMs and EMs are not the same, recent NN-based LMs such as BERT \cite{devlin2019bert} and GPT \cite{radford2018gpt1} can generate rich contextual representations and, therefore, can be utilized as EMs. For the rest of the paper, we will use both terms interchangeably.

\subsubsection{Transformers} are advanced NN architectures designed to handle sequential data using the self-attention mechanism \cite{vaswani2017attention}.  Transformers consist of an encoder-decoder structure, where the encoder processes the input sequence to generate contextual embeddings and the decoder uses these embeddings to produce the output sequence. While encoder-decoder transformers \cite{raffel2020t5} are versatile for a wide range of sequence-to-sequence tasks, encoder-only transformers, such as BERT \cite{devlin2019bert}, are more often used as EMs to generate efficient contextual embeddings. Decoder-only Transformers, such as the GPTs \cite{radford2018gpt1,brown2020gpt3}, have proven to be most successful on generative problems. However, recent efforts have been able to adapt the LLM decoder-only transformers into EMs \cite{springer2024llmrepetition,parishad2024llm2vec,muennighoff2024gritlm,SFRAIResearch2024sfrmistral} in order to take advantage of their capabilities.




\subsubsection{Contrastive Learning (CL)} is a self-supervised representation learning approach where an EM is trained to recognize similarities and differences between data samples. The core idea is to push similar or related samples (positive pairs) closer together in the embedding space, while pushing dissimilar samples (negative pairs) farther apart. 







\subsection{Related Work}
Most decomposition approaches rely on static analysis or dynamic analysis in order to extract microservices from a monolithic application. For example, TopicDec \cite{brito2021topicmodeling} relies on Topic Modeling methods and static analysis to extract domain and syntax level representation of the classes in a monolith. In a similar fashion, HierDec \cite{khaled2022hierdecomp} and MSExtractor \cite{khaled2022msextractor,saidani2019msextractor} apply static analysis and traditional Natural Language Processing pipelines to extract the structural and semantic relationships between the classes. On the other hand, approaches such as FoSCI \cite{jin2021fosci} and Mono2Micro \cite{kalia2021mono2micro} rely mainly on dynamic analysis in conjunction with clustering or genetic algorithms to recommend microservices. HyDec \cite{khaled2022hydecomp} attempts to combine both analysis methods and avoid their drawbacks. Other methods have used different inputs and representations such as MEM \cite{mazlami2017mem} which relied on the version history and software evolution \cite{benomar2015evolution} and the semi-automated methods \cite{daoud2023multimodeldec,li2019dataflowdec} like ServiceCutter \cite{gysel2016servicecutter} which utilized the design artifacts or dataflow diagrams. More recently, there has been an interest in incorporating the databases in the decomposition as seen by the approaches CHGNN \cite{mathai2022chgnn}, CARGO \cite{vikram2022cargo} and DataCentric \cite{yamina2022datacentric}. 




Several Neural Network-based approaches have been proposed for the decomposition problem. CoGCN \cite{desai2021cogcn} employs a Graph Neural Network (GNN) to learn class partitioning and detect outliers through dynamic analysis. Deeply \cite{yedida2023deeply} and CHGNN \cite{mathai2022chgnn} enhance this method with hyper-parameter tuning, a novel loss function and additional input representations while GDC-DVF \cite{qian2023gdcdvf} combines structural and business representations in its GNN-based approach. MicroMiner \cite{trabelsi2023microminer} offers a 3-step framework, utilizing CodeBERT \cite{feng2020codebert} embeddings and a classifier to categorize source code fragments. In contrast, Code2VecDec \cite{aldebagy2021code2vec} leverages the Code2Vec \cite{alon2018code2vec} embedding model to generate class vectors, which are then used for clustering. While GNN-based decomposition methods incorporate some representation learning, it's limited to their training applications and input analysis approaches. For example, CoGCN's encoder learns class embeddings but requires training for each application based on structural and dynamic analysis results. To our knowledge, only MicroMiner and Code2VecDec have utilized Transformers or static embedding models in decompositions. Our approach extends this by comparing the performance of various models, including state-of-the-art Large Language Models, and proposing a fine-tuning method to enhance their effectiveness. This strategy aims to deepen the use of such models in addressing the decomposition problem, going beyond the limitations of existing methods.