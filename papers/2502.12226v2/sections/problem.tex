\section{Problem}
\label{sec:problem}
\subsection{Preliminaries}
\noindent \textbf{Time Series Forecasting}
% \zhen{I'd suggest framing this as time series prediction, because that is the general problem the proposed rating framework could apply to, same as the causal diagram. It's just the experiments are focused on stock price, and why using stock price as the time series of interest needs to be motivated well as comments in abstract and introduction.}
% \kl{Change to time series prediction}
Let the time series be represented by \{x$_{t-n+1}$, x$_{t-n+2}$, ...., x$_{t}$, x$_{t+1}$, ..., x$_{t+d}$\}, where each x$_{t-n+i}$ represents a value in time series, 
% an adjusted closing stock price 
where $n$ is called the sliding window size and $d$ is the number of future values the model predicts. Let X$_{t}$ $=$ \{x$_{t-n+1}$, x$_{t-n+2}$, ...., x$_{t}$\}, and $\hat{Y_{t}}$ $=$ \{$\hat{x}$$_{t+1}$, $\hat{x}$$_{t+2}$, ...., $\hat{x}$$_{t+d}$\}, where $\hat{Y_{t}}=f(X_t$; $\theta$) for uni-modal FMTS, and in the case of multi-modal FMTS, $X_t$ includes both the numerical time series values and the corresponding time series line plots (images). The function $f$ represents pre-trained FMTS with parameter $\theta$ used in a zero-shot manner without task-specific fine-tuning that predicts the values for the next `d' timesteps based on the values at previous $n$ timesteps. Let $Y_{t}$ denote the true values for the next `d' timesteps. Let $S$ be the set of FMTS we want to rate. Let R$_{t}$ be the residual for the sliding window [t + 1, t + d] and is computed by ($\hat{Y_{t}}$ - $Y_{t}$) at each timestep.
% As the objective of our rating method is to consider the worst possible case and to communicate the worst possible behavior of the model to the end-user, let us consider the MAX(R$_{t}$) (maximum of residual). Let it be denoted by $R^{max}_{t}$. 
Our rating method aims to highlight the worst-case scenario for the model. Therefore, we consider the maximum residual, denoted as $R^{max}_{t}$.
\vspace{-1em}

\begin{figure}[!h]
    \centering
    \includegraphics[width=0.20\textwidth]{figs/Causal_General.png}
    \caption{Causal model $\mathcal{M}$ for FMTS. The validity of link `1' depends on the data distribution ($P|Z$), while the validity of the links `2' and `3' are tested in our experiments.}
   \label{fig:causal-model}
   \vspace{-2.2em}
\end{figure}

\begin{figure}[!h]
    \centering
    \includegraphics[width=0.28\textwidth]{figs/causal_variants.png}
    \caption{Variants of the causal diagram in Figure \ref{fig:causal-model} used to answer different research questions (RQs).}
   \label{fig:cms}
   \vspace{-1em}
\end{figure}

\noindent \textbf{Causal Model} The causal model \( \mathcal{M} \), is shown in Figure \ref{fig:causal-model}. Arrowheads indicate the causal direction from cause to effect. If \emph{Sensitive Attribute} ($Z$) is a common cause for both \emph{Perturbation} ($P$) and \emph{Residual} ($R^{max}_{t}$), it introduces a spurious correlation between $P$ and $R^{max}_{t}$, known as the confounding effect, making $Z$ the confounder. The path from \emph{Perturbation} to \emph{Residual} through the confounder is called a \textit{backdoor path} and is undesirable. Various backdoor adjustment techniques can remove the confounding effect \cite{xu2022neural, fang2024backdoor, liu2021preferences}. The deconfounded distribution, after adjustment, is represented as $(R^{max}_{t} | do(P))$. The `do(.)' operator in causal inference denotes an intervention to measure the causal effect of $P$ on the $R^{max}_{t}$. Solid red arrows with `?' in Figure \ref{fig:causal-model} denote the causal links tested in our experiments, while the dotted red arrow represents a potential causal link, depending on the distribution $(P | Z)$ across different values of $Z$.

\subsection{Problem Formulation}
We aim to answer the following research questions (RQs) (with causal diagrams in Fig~\ref{fig:cms}) through our causal analysis when different perturbations denoted by $P$ = \{0, 1, 2, 3\} (or simply $P0, P1, P2, P3$) are applied to the input given to the set of FMTS $S$:

\noindent{\bf RQ1: Does $Z$ affect $R^{max}_{t}$, even though $Z$ has no effect on $P$?} That is, if perturbations are independent of the sensitive attribute, can the attribute still affect the system outcome, leading to statistical bias (i.e., lack of fairness)? Causal analysis is unnecessary here due to no confounding effect.

% \noindent That is, if the applied perturbations do not depend on the value of the sensitive attribute, would the sensitive attribute still affect the system outcome leading to a statistical bias? In this case, causal analysis is not required to answer the question as there is no confounding effect. %Figure \ref{fig:h1} shows the corresponding causal diagram. 
% \zhen{why do we need P in the corresponding RQ figure? or we can just gray P out, and not have that dotted line between P and R. Sorry I found it confusing.}\kl{P can still effect R and it is not something we can neglect. As we are using the dashed line notation to represent the effect that may exist but won't be answered in the RQ, it is better to stick with that.}

\noindent {\bf RQ2: Does $Z$ affect the relationship between $P$ and $R^{max}_{t}$ when $Z$ has an effect on $P$?} That is, if the applied perturbations depend on the value of the sensitive attribute, would the sensitive attribute add a spurious (false) correlation between the perturbation and the outcome of a system leading to confounding bias?
% \zhen{is this essentially asking if Z a confounder when Z has an effect on P?}\kl{yes. As explained in this simple description below.}

% \noindent That is, if the applied perturbations depend on the value of the sensitive attribute, would the sensitive attribute add a spurious (false) correlation between the perturbation and the outcome of a system leading to confounding bias? %Figure \ref{fig:h2} shows the corresponding causal diagram. 
% \zhen{why are we drawing a link from P to R in the corresponding RQ figure?}\kl{I added a '?' as well now. When the confounder is present vs. absent, how does the relation between P and R change? Z to P link is known. If Z turns out to be the confounder (through PIE), the other causal links with ? will become valid}

\noindent {\bf RQ3: Does $P$ affect $R^{max}_{t}$ when $Z$ may have an effect on $R^{max}_{t}$?} That is, what is the impact of the perturbation on the outcome of a system when the sensitive attribute may still have an effect on the outcome of a system? 

% \noindent That is, what is the impact of the perturbation on the outcome of a system when the sensitive attribute may still have an effect on the outcome of a system? %Figure \ref{fig:h3} shows the corresponding causal diagram.

\noindent {\bf RQ4: Does $P$ affect the accuracy of $S$?} That is, do the perturbations affect the performance of the systems' accuracy? Causal analysis is not required to answer this question as we only need to compute appropriate accuracy metrics to assess how robust a system is against different perturbations.

% \zhen{is this more on average residual $R_t^{avg}$ than $R_t^{max}$? the question itself seems repeating itself, maybe just simply does P affect $R_t^{avg}$?}\kl{No, we are assessing the prediction accuracy through the accuracy metrics (in which we are not using $R_t^{avg}$ or $R_t^{max}$. I framed it differently now.}

% \noindent That is, do the perturbations affect the performance of the systems in terms of accuracy? Causal analysis is not required to answer this question as we only need to compute appropriate accuracy metrics to assess how robust a system is against different perturbations.

% \zhen{IMHO, it might be easier for the reader to start with simpler questions and move on to more complex ones. So RQ4 first, then RQ1, then RQ3, then RQ2?}

