%File: anonymous-submission-latex-2025.tex
\documentclass[letterpaper]{article} % DO NOT CHANGE THIS
\usepackage{aaai25}  % DO NOT CHANGE THIS
\usepackage{times}  % DO NOT CHANGE THIS
\usepackage{helvet}  % DO NOT CHANGE THIS
\usepackage{courier}  % DO NOT CHANGE THIS
\usepackage[hyphens]{url}  % DO NOT CHANGE THIS
\usepackage{graphicx} % DO NOT CHANGE THIS
\usepackage{tcolorbox}
\urlstyle{rm} % DO NOT CHANGE THIS
\def\UrlFont{\rm}  % DO NOT CHANGE THIS
\usepackage{natbib}  % DO NOT CHANGE THIS AND DO NOT ADD ANY OPTIONS TO IT
\usepackage{caption} % DO NOT CHANGE THIS AND DO NOT ADD ANY OPTIONS TO IT
\frenchspacing  % DO NOT CHANGE THIS
\setlength{\pdfpagewidth}{8.5in} % DO NOT CHANGE THIS
\setlength{\pdfpageheight}{11in} % DO NOT CHANGE THIS
%
% These are recommended to typeset algorithms but not required. See the subsubsection on algorithms. Remove them if you don't have algorithms in your paper.
% \usepackage{algorithm}
% \usepackage{algorithmic}

%
% These are are recommended to typeset listings but not required. See the subsubsection on listing. Remove this block if you don't have listings in your paper.
\usepackage{newfloat}
\usepackage{listings}
% Define colors
\definecolor{headercolor}{gray}{0.85}
\definecolor{boxcolor}{gray}{0.95}

% Setup for fancy headers and footers
% Define custom colors
\definecolor{lightgray}{gray}{0.95}
\definecolor{darkgray}{gray}{0.85}
\definecolor{highlight}{gray}{0.9}
\definecolor{questioncolor}{gray}{0.9} % Light gray for question
\definecolor{answercolor}{gray}{0.95}   % Slightly darker gray for answers

% \definecolor{elicolor}{}{}

\colorlet{Mycolor1}{red!30}
\usepackage{etoolbox}
\AtBeginEnvironment{tcolorbox}{\small}


\DeclareCaptionStyle{ruled}{labelfont=normalfont,labelsep=colon,strut=off} % DO NOT CHANGE THIS
\lstset{%
	basicstyle={\footnotesize\ttfamily},% footnotesize acceptable for monospace
	numbers=left,numberstyle=\footnotesize,xleftmargin=2em,% show line numbers, remove this entire line if you don't want the numbers.
	aboveskip=0pt,belowskip=0pt,%
	showstringspaces=false,tabsize=2,breaklines=true}
% \floatstyle{ruled}
% \newfloat{listing}{tb}{lst}{}
% \floatname{listing}{Listing}
%
% Keep the \pdfinfo as shown here. There's no need
% for you to add the /Title and /Author tags.
\pdfinfo{
/TemplateVersion (2025.1)
}

% DISALLOWED PACKAGES
% \usepackage{authblk} -- This package is specifically forbidden
% \usepackage{balance} -- This package is specifically forbidden
% \usepackage{color (if used in text)
% \usepackage{CJK} -- This package is specifically forbidden
% \usepackage{float} -- This package is specifically forbidden
% \usepackage{flushend} -- This package is specifically forbidden
% \usepackage{fontenc} -- This package is specifically forbidden
% \usepackage{fullpage} -- This package is specifically forbidden
% \usepackage{geometry} -- This package is specifically forbidden
% \usepackage{grffile} -- This package is specifically forbidden
% \usepackage{hyperref} -- This package is specifically forbidden
% \usepackage{navigator} -- This package is specifically forbidden
% (or any other package that embeds links such as navigator or hyperref)
% \indentfirst} -- This package is specifically forbidden
% \layout} -- This package is specifically forbidden
% \multicol} -- This package is specifically forbidden
% \nameref} -- This package is specifically forbidden
% \usepackage{savetrees} -- This package is specifically forbidden
% \usepackage{setspace} -- This package is specifically forbidden
% \usepackage{stfloats} -- This package is specifically forbidden
% \usepackage{tabu} -- This package is specifically forbidden
% \usepackage{titlesec} -- This package is specifically forbidden
% \usepackage{tocbibind} -- This package is specifically forbidden
% \usepackage{ulem} -- This package is specifically forbidden
% \usepackage{wrapfig} -- This package is specifically forbidden
% DISALLOWED COMMANDS
% \nocopyright -- Your paper will not be published if you use this command
% \addtolength -- This command may not be used
% \balance -- This command may not be used
% \baselinestretch -- Your paper will not be published if you use this command
% \clearpage -- No page breaks of any kind may be used for the final version of your paper
% \columnsep -- This command may not be used
% \newpage -- No page breaks of any kind may be used for the final version of your paper
% \pagebreak -- No page breaks of any kind may be used for the final version of your paperr
% \pagestyle -- This command may not be used
% \tiny -- This is not an acceptable font size.
% \vspace{- -- No negative value may be used in proximity of a caption, figure, table, section, subsection, subsubsection, or reference
% \vskip{- -- No negative value may be used to alter spacing above or below a caption, figure, table, section, subsection, subsubsection, or reference

\setcounter{secnumdepth}{2} %May be changed to 1 or 2 if section numbers are desired.


% \title{Rating \underline{F}oundation \underline{M}odels  Supporting \underline{T}ime-\underline{S}eries (FMTS) for Robustness through a Causal Lens}
% -- Alternative
\title{\textbf{On Creating a Causally Grounded Usable Rating Method for Assessing the Robustness of Foundation Models  Supporting Time Series}}
% -- Alternative
%\title{Effectiveness of Ratings in Assessing the Robustness of \underline{F}oundation \underline{M}odels  Supporting \underline{T}ime-\underline{S}eries (FMTS)}

\author {
    % Authors
    Kausik Lakkaraju\textsuperscript{\rm 1}, 
    Rachneet Kaur\textsuperscript{\rm 2},  
    Parisa Zehtabi\textsuperscript{\rm 3}, 
    Sunandita Patra\textsuperscript{\rm 2}, \\
    Siva Likitha Valluru\textsuperscript{\rm 1}, 
    Zhen Zeng\textsuperscript{\rm 2}, 
    Biplav Srivastava\textsuperscript{\rm 1}, 
    Marco Valtorta\textsuperscript{\rm 1}
}
\affiliations {
    \textsuperscript{\rm 1}University of South Carolina, USA\\
    \textsuperscript{\rm 2}J.P. Morgan AI Research, USA\\
    \textsuperscript{\rm 3}J.P. Morgan AI Research, UK
}

% \author{
%     %Authors
%     % All authors must be in the same font size and format.
%     Written by AAAI Press Staff\textsuperscript{\rm 1}\thanks{With help from the AAAI Publications Committee.}\\
%     AAAI Style Contributions by Pater Patel Schneider,
%     Sunil Issar,\\
%     J. Scott Penberthy,
%     George Ferguson,
%     Hans Guesgen,
%     Francisco Cruz\equalcontrib,
%     Marc Pujol-Gonzalez\equalcontrib
% }
% \affiliations{
%     %Afiliations
%     \textsuperscript{\rm 1}Association for the Advancement of Artificial Intelligence\\
%     % If you have multiple authors and multiple affiliations
%     % use superscripts in text and roman font to identify them.
%     % For example,

%     % Sunil Issar\textsuperscript{\rm 2}, 
%     % J. Scott Penberthy\textsuperscript{\rm 3}, 
%     % George Ferguson\textsuperscript{\rm 4},
%     % Hans Guesgen\textsuperscript{\rm 5}
%     % Note that the comma should be placed after the superscript

%     1101 Pennsylvania Ave, NW Suite 300\\
%     Washington, DC 20004 USA\\
%     % email address must be in roman text type, not monospace or sans serif
%     proceedings-questions@aaai.org
% %
% % See more examples next
% }


% REMOVE THIS: bibentry
% This is only needed to show inline citations in the guidelines document. You should not need it and can safely delete it.
\usepackage{bibentry}
% END REMOVE bibentry

% Packages we added.
\usepackage{amsmath}
\usepackage{algorithm2e}
\RestyleAlgo{ruled}
% \usepackage{pdfpages}
\usepackage{subcaption}
\usepackage{array}
\usepackage{multirow}
\usepackage{titling}
% \newcommand{\kl}[1]{{\color{red}~{\em Comment by Kausik: #1}}}
% \newcommand{\biplav}[1]{{\color{blue}~{\em Comment by Biplav: #1}}}
% \definecolor{sunanditacolor}{HTML}{FF5478} % Using hexadecimal
% \newcommand{\sunandita}[1]{{\color{sunanditacolor}~{\em Comment by Sunandita: #1}}}
% \newcommand{\parisa}[1]{{\color{magenta}~{\em Comment by Parisa: #1}}}
% \newcommand{\zhen}[1]{{\color{cyan}~{\em Comment by Zhen: #1}}}
% \newcommand{\rachneet}[1]{{\color{orange}~{\em Comment by Rachneet: #1}}}
% \newcommand{\marco}[1]{{\color{green}~{\em Comment by Marco: #1}}}
% \newcommand{\likitha}[1]{{\color{teal}~{\em Comment by Likitha: #1}}}



% \renewcommand{\kl}[1]{}
%  \renewcommand{\biplav}[1]{}
%  \renewcommand{\sunandita}[1]{}
%  \renewcommand{\parisa}[1]{}
%  \renewcommand{\zhen}[1]{}
%  \renewcommand{\rachneet}[1]{}
% \renewcommand{\likitha}[1]{}

\begin{document}

\maketitle

\begin{abstract}
%The emergence of Foundation Models (FMs) has brought advancements to a range of tasks, including complex ones such as time-series forecasting across various critical sectors such as healthcare and finance. However, like most AI models, FMs are susceptible to noise and perturbations in input data, which can lead to inaccuracies. In the finance sector, these can have significant impact, affecting various stakeholders such as investors, analysts and traders. This paper introduces a novel approach to evaluate the robustness of Time-series Foundation Models (TFMs) using causally grounded experimental setup.  We evaluate seven time-series forecasting models, ranging in architecture, size, and functionality - from general-purpose to time-series-specific across experimental settings that include four perturbation scenarios with real-world relevance and data from six leading stocks in three industries over a year. These models support different modalities, handling diverse data types. Through causal analysis, we aim to assess the reliability and accuracy of the TFMs, thereby supporting developers and end-users in making informed decisions. Additionally, we conduct a user study to determine the effectiveness of our rating approach in communicating the behavior of TFMs to the end-users, highlighting how well our approach practically applies to real-world scenarios. 

Foundation Models (FMs) have improved time series forecasting in various sectors, such as finance, but their vulnerability to input disturbances can hinder their adoption by stakeholders, such as, investors and analysts. To address this, we propose a causally grounded rating framework to study the robustness of Foundational Models for Time Series (FMTS) with respect to input perturbations. 
%\zhen{missing the explicit motivation for using financial stock time series} 
We evaluate our approach on the stock price prediction problem, a well studied problem with easily accessible public data, evaluating six state-of-the-art (some multi-modal) FMTS across six prominent stocks spanning three industries. 
The ratings proposed by our framework effectively 
% \zhen{do we have support for this claim of accuracy? because it aligns with user study?} 
assess the robustness of FMTS and also offer actionable insights 
% \zhen{make sure to summarize actionable insights in experiments overall conclusion} 
for model selection and deployment.
Within the scope of our study, we find that (1) multi-modal FMTS exhibit better robustness and accuracy compared to their uni-modal versions and,
%\zhen{is it that the models just perform equally bad across perturbations? I guess what a useful model would be is something that is accurate and robust at the same time, is there any model like that we would like to highlight and conclude here?}, 
(2) FMTS pre-trained on time series forecasting task exhibit better robustness and forecasting accuracy compared to general-purpose FMTS pre-trained across diverse settings. 
%\zhen{for next token prediction? ``diverse settings" might lead to a misunderstanding that the FMTS is pre-trained for diverse time series tasks. Or just say ``general-purpose FMTS".}.
Further, to validate our framework's usability, we conduct a user study showcasing FMTS prediction errors along with our computed ratings. The study confirmed that our ratings reduced the difficulty for users in comparing the robustness of different systems.
\end{abstract}
\documentclass[../main.tex]{subfiles}
\graphicspath{{../images/}}
\makeatletter
\def\input@path{{../images/}}
\makeatother
\begin{document}
\section{Introduction}
\begin{figure}
\centering
\begin{tikzpicture}
\node[inner sep=0pt] (ws) at (0, 0) {
\includegraphics[height=.4\textwidth, trim={10cm 0 10cm 0},clip]{world_space.png}};
\node[inner sep=0pt] (cs) at (6,0) {\includegraphics[height=.4\textwidth, trim={10cm 1cm 10cm 4cm},clip]{conf_space.png}};
\end{tikzpicture}
\vspace{-5pt}
\label{fig:pbrm_intro}
\caption{\textbf{Left}: Shows world space obstacles as grey spheres. Robots start and goal configuration is colored red and green, respectively. Configurations along the computed path are colored transparent blue. \textbf{Right:} Mapped world space scenario to configuration space. Obstacle region is the grey mesh. Red spheres are collision-free regions computed by the neural SCDF. The optimized shortest path in the convex corridor is the blue curve.}
\vspace{-25pt}
\end{figure}
Motion planning is the problem of finding a collision-free trajectory that connects a given start and goal configuration. The planning takes place in the configuration space of the robot. For single body robots, like mobile robots or drones, the configuration space and the world space are usually the same. This simplifies the planning, since explicit obstacle representations are available which enables geometrical tools like separating hyperplanes, smallest distance to obstacles etc., to be used when designing motion planning algorithms. For multi-body robots like manipulators, the situation is completely different. The world space obstacles are usually mapped to non-convex regions, and to make the problem even harder, the mapping is usually not known. Forming explicit representations of the obstacle region in the configuration space is usually too expensive or intractable. Despite all of this, sampling based planners are used with great success, which mainly is due to their use of implicit representations of the obstacle region. The basic idea is to construct a graph in the configuration space that covers and connects the collision-free region. From this graph, a path can be extracted that connects a given start and goal configuration. The approach is computationally expensive, since the graph is constructed with the smallest geometrical building block available, points, which represents a collision-check. Furthermore, the extracted paths from the graph are non-smooth and jagged due to the stochastic nature of the approach. This adds an additional post-processing step to the process, where the paths are shortcutted and smoothened, before the path can be used for tracking. Clearly a lot of time is invested to form this graph and produce smooth paths. Thus, if the obstacles start to move, then all of this work is done in no use, since all points that make up this graph need to be re-verified, which is simply too time consuming to be done in real time.
\\\\
In this work, we want to address the existing drawbacks of the sampling based planners. Our main contribution is an improved motion planner where each vertex in the graph covers a collision-free region in the form of a sphere instead of a point and where the edges are formed with neighboring intersecting spheres. This representation has the advantage of instead of returning piecewise linear paths, returning a sequence of overlapping spheres, i.e. a convex corridor, that connects a given start and goal configuration, illustrated in Figure \ref{fig:pbrm_intro}. This convex corridor allows us to use convex optimization to produce smooth trajectories, instead of computationally expensive post-processing methods. The representation further allows us to estimate the coverage of the collision-free space, which gives us awareness and feedback in the offline roadmap construction phase. Finally, our representation is simple to adapt to moving obstacles, simply requery for the new radii and recheck for intersections. 
\\\\
The spherical collision-free regions are formed using a signed distance function (SDF), which is a function that returns the smallest distance from an arbitrary point to the boundary of an obstacle. As the name implies, the distance is signed, thus if the point is inside the obstacle it is negative otherwise positive. If the distance is positive, a sphere with radius equal to the distance is guaranteed to cover a collision-free region. Using an SDF in motion planning is not new, but what is novel about our approach is that we express the distance in the configuration space instead of the world space and by doing so allows us to form these convex collision-free regions. We refer to the resulting SDF as a signed configuration distance function (SCDF). Computing an SCDF analytically is non-trivial, our approach is therefore to parameterize the SCDF with a deep neural network and learn the mapping by supervised learning. Our resulting neural SCDF can compute distances for different parameter values of obstacle shapes and we also show how multiple distances can be combined, thus making our approach flexible.
\section{Related work}
Motion planning algorithms can roughly be divided into three families, grid-based, sampling based and optimization based methods. Grid-based methods (GBM) discretize the planning space from which a graph is then compiled. A standard search method is A$^\star$ \citep{a_star}, which is classified as an \textit{informed} search method, since it employs a heuristic function to speed up the search. A$^\star$ guarantees to return an optimal path at the level of discretization used. GBMs usually discretize the planning space by a regular lattice and this limits the GBMs to problems with low dimensionality due to the curse of dimensionality. Thus, GBMs are usually limited to single-body robots where the degrees of freedom (DOF) are low. To overcome the inherent scaling problem with the GBMs, stochastic methods are usually used for multi-body robots. These methods are termed as sampling-based methods (SBM) and core members within this family are the rapidly-exploring random trees (RRT) \citep{rrt} and the probabilistic roadmap (PRM) \citep{prm}. RRT grows a tree from the start configuration and explores the collision-free region in a rapid way until it is able to connect to the goal region. RRT is usually improved by bi-directional planning \citep{rrt_connect}, i.e. an additional tree is grown from the goal configuration and the trees are tested for connection after any tree has been expanded. RRT is a single-query method, thus it searches for a path from scratch each time it is queried. Contrary to this, PRM is a multi-query method, which solves for multiple queries without starting from scratch. PRM does this by creating a roadmap (graph) that covers the collision-free space as an offline step. The graph is then used to solve for multiple queries. PRMs are used in cases where the environment does not change since the extra offline step is too computationally costly and needs to be re-done if the environment is changed. In our work, we address this inherent issue by using a different roadmap representation. Our vertices in the graph cover a collision-free region in the form of spheres and we form the edges by checking for intersecting spheres. If something in the environment changes, we recompute the spheres radii and recheck the intersections, without relying on collision detection. We use a trained neural network to compute the sphere radius, therefore querying for the radius can be done fast, hence our representation enables the PRM for dynamic environments.
\\\\
In the recent decades, optimization based methods (OBM) \citep{chomp, schulman, itomp, stomp} have been introduced as an alternative to SBM for multi-body robots. Like the SBM, the OBMs scale well to higher dimensional problems and produce smoother motion. It is common to use a SDF in the optimization since it is a smooth function, thus enabling gradient-based methods. However, the standard way of expressing the SDF is in world space. The distance therefore needs to be mapped to the configuration space by the forward kinematics. This mapping makes the optimization problem a non-linear program (NLP), which is computationally expensive to solve. Recently, a different approach has been proposed. In \cite{mp_gcs} motion planning is formulated as a convex optimization problem by using the graph of convex sets framework \citep{gcs}. The underlying idea is to decompose the collision-free space into intersecting convex sets from which a convex optimization problem is formulated. In cases where an explicit representation of the obstacles in the configuration space exists, like for single-body robots, creating collision-free convex regions can be done fast \citep{iris}. For multi-body robots, this is non-trivial. Existing work does this successfully \citep{iris_nlp, iris_c} by an optimization based approach, but the methods are still too time consuming to be used in the presence of moving obstacles. Our approach is instead to use deep learning to learn an SDF expressed in the configuration space. With this, we can query for shortest distances to the collision boundary, which allows us to expand spherical regions which are collision-free. Our approach is fast and therefore enables our suggested roadmap planner to be used in dynamic environments.
\\\\
Recent research has focused on learning collision detection \citep{fk_kernel_distance, diffco, graphdistnet} by predicting the signed distance between the robot links and the surrounding obstacles in the world space. The learned SDF is used in trajectory optimization but since the distance is expressed in the world space, the problem becomes an NLP and therefore takes a long time to solve. We take a novel approach and suggest to instead express the signed distance in the configuration space. This allows us to improve the PRM at the same time as it enables convex optimization for trajectory optimization, which runs faster and is more reliable than NLP solvers. In \cite{cspf} a learned signed distance function in the configuration space is proposed similar to our approach. However, their approach is restricted to point cloud representations, while we propose to represent the obstacles as parameterized geometric shapes, e.g. spheres. Furthermore, we also show how to use our learned SCDF to improve an existing roadmap planner.
\section{Problem formulation}
A robot is located in the world space, $\W \subset \R^3 $. The unique location of the robot is given by its configuration $\q \in \C$, where $\C$ is the configuration space. The set of points covered by the robots bodies at a certain configuration is expressed as $\B(\q) \subset \W$. The robot is surrounded by $\NrObst$ obstacles $\O = \bigcup_{i=1}^{\NrObst} \O_i$, where  $\O_i \subset \W$. The representation of the obstacle in the configuration space is the set $\C\O_i = \{\q \in \C \: |\: \B(\q) \cap \O_i \neq \emptyset \}$. The obstacle space is formed as $\Co = \bigcup_{i=1}^{\NrObst} \C \O_i$. The complement is referred to as the free space, $\Cf = \C \setminus \Co$. The path planning problem is a tuple, ($\Cf$, $\qStart$, $\qGoal$), where we want to connect a query pair, consisting of a start, $\qStart$, and goal configuration, $\qGoal$, with a geometric path, $\q(s): [0, 1] \mapsto \Cf$, such that $\q(0)=\qStart$ and $\q(1)=\qGoal$, or report correctly when such a path does not exist.
\end{document}

\section{Related Work}
\label{sec:related-work}
%We now contextualize our work with related literature so that our contributions are highlighted. We cover FMTS, perturbations in time-series, 
% robustness testing of FMs, 
%and rating of AI systems. 

\noindent \textbf{Foundation Models Supporting Time Series} 
The use of FMs for time series forecasting has advanced significantly. 
% \cite{lu2022frozen} first demonstrated that transformers pre-trained on text data (LLMs) can effectively solve sequence modeling tasks in other modalities, paving the way for leveraging language pre-trained transformers for time series analysis. Recent studies have focused on reprogramming LLMs for time series tasks through parameter-efficient fine-tuning and suitable tokenization strategies \cite{zhou2023one, gruver2024large, jin2023time, cao2023tempo, ekambaram2024tiny}. These methods have successfully adapted transformers to the unique challenges of time series forecasting. \cite{zhou2023one} and \cite{jin2023time} further illustrate the versatility and robustness of fine-tuned language pre-trained transformers for diverse time series tasks.
\cite{lu2022frozen} showed that transformers pre-trained on text data can solve sequence modeling tasks in other modalities, enabling their application to time series analysis. Recent studies have reprogrammed LLMs for time series tasks through parameter-efficient fine-tuning and tokenization strategies \cite{zhou2023one, gruver2024large, jin2023time, cao2023tempo, ekambaram2024tiny}. 
% These methods have successfully adapted transformers to the unique challenges of time series forecasting. 
\cite{zhou2023one} and \cite{jin2023time} further illustrate the versatility and robustness of fine-tuned language pre-trained transformers for diverse time series tasks.
% Several models have contributed to the advancement of time series forecasting. \cite{ansari2024chronos} and \cite{woo2024unified} have improved forecasting accuracy and model generalization.  
% % \cite{ansari2024chronos} and \cite{woo2024unified} have pushed the boundaries of forecasting accuracy and model generalization. 
% \cite{rasul2023lag} and \cite{das2023decoder} have explored new tokenization strategies and fine-tuning methods to improve model performance. Additionally, \cite{garza2023timegpt} and \cite{ekambaram2024tiny} have focused on creating lightweight and efficient models for real-time applications. \cite{talukder2024totem} stands out with its unique approach to integrating multiple temporal patterns, enhancing forecasting precision.
% FMs trained from scratch have achieved SOTA on time series tasks. Zero-shot forecasting, exemplified by \cite{gruver2024large}, showcases the ability of these models to make accurate predictions without domain-specific training. \cite{cao2023tempo} and \cite{goswami2024moment} have introduced approaches to enhance the performance and efficiency of time series models, leveraging transformer architectures to capture temporal dependencies more effectively. In our experiments, we select Gemini-V and Phi-3 as the GP models and Chronos and MOMENT as TS models due to their SOTA performance in their respective categories.
Several models have advanced time series forecasting. \cite{ansari2024chronos} and \cite{woo2024unified} have improved forecasting accuracy and model generalization, while
% \cite{ansari2024chronos} and \cite{woo2024unified} have pushed the boundaries of forecasting accuracy and model generalization. 
\cite{rasul2023lag} and \cite{das2023decoder} have explored new tokenization strategies and fine-tuning methods. \cite{garza2023timegpt} and \cite{ekambaram2024tiny} developed lightweight models for real-time applications, and \cite{talukder2024totem} integrated multiple temporal patterns to improve precision. FMs trained from scratch, like \cite{gruver2024large}, achieved SOTA in zero-shot forecasting, with \cite{cao2023tempo} and \cite{goswami2024moment} further improving model performance. 
%In our experiments, we select Gemini-V and Phi-3 as the GP models and Chronos and MOMENT as TS models due to their SOTA performance in their respective categories.
Please see Section~\ref{sec:exp_app} for the FMTS we selected due to their SOTA performance in their respective categories.

%The use of FMs for time series forecasting has seen significant advancements in recent years. \cite{lu2022frozen} first demonstrated that transformers pre-trained on text data (LLMs) can effectively solve sequence modeling tasks in other modalities. This work opened the door to leveraging language pre-trained transformers for time series analysis. Recent studies have built on this foundation, focusing on reprogramming LLMs for time series tasks through parameter-efficient fine-tuning and suitable tokenization strategies \cite{zhou2023one, gruver2024large, jin2023time, cao2023tempo, ekambaram2024tiny}. These methods have proven successful in adapting the powerful capabilities of transformers to the unique challenges of time series forecasting. OneFitsAll \cite{zhou2023one} and Time-LLM \cite{jin2023time} further illustrate how language pre-trained transformers can be fine-tuned for diverse time series tasks, demonstrating their versatility and robustness. 
% \zhen{reason why we didn't include these models in our study, weights not available? or other justification, to prevent that naturally raised question from readers.}\kl{Good point. We need to discuss. I added 2 sentences at the bottom but they are probably not very convincing.}
%Several other models have contributed to the advancement of time series forecasting. Chronos \cite{ansari2024chronos} and Moirai \cite{woo2024unified} have pushed the boundaries of forecasting accuracy and model generalization. Lag-llama \cite{rasul2023lag} and TimesFM \cite{das2023decoder} have explored new tokenization strategies and fine-tuning methods to improve model performance. Additionally, Time-GPT1 \cite{garza2023timegpt} and Tiny-Time Mixers \cite{ekambaram2024tiny} have focused on creating lightweight and efficient models suitable for real-time applications. TOTEM \cite{talukder2024totem} stands out with its unique approach to integrating multiple temporal patterns, further enhancing forecasting precision.
%Aside from reprogramming LLMs for time series, FMs trained from scratch have achieved SOTA on times series tasks. 
%Zero-shot forecasting, exemplified by \cite{gruver2024large}, showcases the ability of these models to make accurate predictions without domain-specific training.  TEMPO \cite{cao2023tempo} and MOMENT \cite{goswami2024moment} have introduced approaches to enhance the performance and efficiency of time series models, leveraging transformer architectures to capture temporal dependencies more effectively.
% \zhen{and these are on various time series tasks including time series forecasting?}
% \zhen{These are models that are specifically trained for time series forecasting, I'd suggest mentioning them first after the LLM reprogramming, and then expanding to the models that are trained across time series tasks instead. The flow of this subsection feels a bit odd as of now.} \kl{Done.}
%In our experiments, we select Gemini-V and Phi-3 as the GP models and Chronos and MOMENT as TS models due to their SOTA performance in their respective categories. 

%\vspace{-0.3em}
\noindent \textbf{Perturbations in Time Series Data} TS data is commonly stored in spreadsheets and databases, which are prone to changes due to acts of omission (e.g., negligence, data-entry errors) or commission (e.g., adversarial attacks, sabotage). Omission errors are most common \cite{spreadsheets-errors-risks-survey}. Tools like Microsoft Excel and Google Sheets are widely used for data collection and analysis, allowing end-user programming \cite{spreadsheets-future-workshop}. However, over 90\% of spreadsheets contain errors due to issues like incorrect formulae, leading to multi-billion dollar losses \cite{spreadsheet-qa-survey}.
%\cite{spreadsheet-qa-survey,spreadsheets-errors-risks-survey}.
Adversarial attacks are also increasing in data stores and AI models for tasks like forecasting.
% \cite{papernot2016transferability} introduced a black-box attack method using a substitute model to generate adversarial examples, demonstrating transferability across tasks. \cite{baluja2017adversarial} focused on white-box attacks using gradient information. 
\cite{karim2019adversarial} adapted these concepts to time series, exploring both black-box and white-box attacks. \cite{oregi2018adversarial} revealed the vulnerability of distance-based classifiers. \cite{rathore2020untargeted} examined various adversarial attacks on time series classifiers. TSFool \cite{li2022tsfool} introduced a multi-objective black-box attack to craft imperceptible adversarial time series to fool RNN classifiers.
%Time series (TS) data is widely stored and manipulated in spreadsheets and databases. These are also the tools which see considerable changes or perturbations due to acts of omission that are unintended (e.g., negligence, data-entry errors) or commission which are deliberate (e.g., adversarial attacks, sabotage). 
%Among these, changes due to omission are most common \cite{spreadsheets-errors-risks-survey}.
%For example, a spreadsheet, implemented in tools like Microsoft Excel and Google Sheets, is a common data collection and analysis environment that also allows end-user programming \cite{spreadsheets-future-workshop}. They are used widely at the workplace and are often a door opener to more advanced scientific tools. But gaining expertise in them needs practice since a large proportion of spreadsheets ($\succ$ 90\%) are known to have errors due to issues like incorrect formulae caused by improper understanding of behavior during routine operations like copy-paste and end-user programming, which have caused losses of multi-billion dollars \cite{spreadsheet-qa-survey,spreadsheets-errors-risks-survey}.
% \zhen{do we need to relate our perturbations to these attacks? otherwise, we must manage the readers' expectations on what types of perturbations we focus on other than adversarial attacks, and motivate it properly}
%\zhen{Play down this a bit, and emphasize and justify why we focus on the type of perturbations we consider in the paper, to mimic operational errors in practice apart from adversarial attacks, citing the 2024 and 1996 papers Biplav added.} 
%Furthermore, adversarial attacks are also increasing both in data stores and in AI models created to solve tasks like forecasting.
%Foundational work by ~\cite{papernot2016transferability} introduced a black-box attack method that involved training a substitute model to generate adversarial examples capable of misleading the target model, demonstrating the transferability property across similar tasks. In contrast, research by ~\cite{baluja2017adversarial} focused on white-box attacks, using gradient information and probabilistic outputs to craft adversarial examples. Researchers~\cite{karim2019adversarial} have adapted these concepts to the time series domain, exploring both black-box and white-box attacks on time series classification models. In addition, ~\cite{oregi2018adversarial} revealed the susceptibility of distance-based time-series classifiers to adversarial examples. ~\cite{rathore2020untargeted} examined untargeted, targeted, and universal adversarial attacks on time series classifiers, demonstrating the effectiveness of these attacks across various datasets. TSFool~\cite{li2022tsfool} introduced a multi-objective black-box attack to craft highly imperceptible adversarial time series to fool RNN classifiers.
%Adversarial attacks on time-series data are initially focused on time-series classification tasks, leveraging concepts adapted from adversarial attacks in other domains.
%explored adversarial sample crafting for time series classification using elastic similarity measures,  %These works collectively underscore the ongoing efforts to understand and mitigate the risks posed by adversarial attacks on time series classification models.
% More recently, research into adversarial attacks on time series forecasting models has revealed distinct challenges and novel attack strategies. One primary challenge is targeted attacks. While targeted adversarial attacks on time series classification aim to misclassify specific instances, achieving similar precision in time series forecasting is more complex due to the sequential nature of the data. Perturbations must be designed to influence specific aspects of the forecast (e.g., directional shifts or amplitude changes) without disrupting the overall temporal dependencies, making precise control more challenging~\cite{govindarajulu2023targeted}. Another challenge is attacks on multivariate forecasting. Adversarial attacks could exploit the inter-dependences between variables. ~\cite{liu2022robust} introduced sparse and indirect cross-time-series attacks in multivariate settings, which are more effective and realistic than direct attacks in univariate cases.
% \zhen{Biplav, could we make a quick comment here as well that we focus more on data error side in practice, other than attacks? and cite the paper that you mentioned on data errors? Otherwise this section of adversarial attacks feel a bit standalone to other sections}
%These challenges underscore the need for ongoing research to develop effective adversarial attack strategies and robust defense mechanisms tailored to the unique characteristics of time series forecasting models.
% -----


\noindent \textbf{Rating AI Systems} Several works have assessed and rated AI systems for trustworthiness from a third-party perspective without access to training data. \cite{srivastava2020rating} proposed a method to rate AI systems for bias, specifically targeting gender bias in machine translators \cite{srivastava2018towards}, and used visualizations to communicate these ratings \cite{bernagozzi2021vega}. They conducted user studies on trust perception through visualizations \cite{vega-userstudy-translatorbias}, but these lacked causal interpretation. \cite{kausik2024rating} introduced a causal analysis approach to rate bias in sentiment analysis systems, extending it to assess their impact when used with translators \cite{kausik2023the}. We extend their method to rate MM-TSFM for robustness against perturbations. Causal analysis offers advantages over statistical analysis by determining accountability, aligning with humanistic values, and quantifying the direct influence of various attributes on forecasting accuracy.


\subsection{Problem Formulation}

% We begin by formulating the problem of dynamic benchmarking for LLMs.
A dynamic benchmark is defined as  
$
\small
\mathcal{B}_{\text{dynamic}} = (\mathcal{D}, T(\cdot)), \quad 
\mathcal{D} = (\mathcal{X}, \mathcal{Y}, \mathcal{S}(\cdot))
$
where \( \mathcal{D} \) represents the static benchmark dataset. 
% consisting of input prompts \( \mathcal{X} \), expected outputs \( \mathcal{Y} \), and a scoring function \( \mathcal{S}(\cdot) \) that evaluates the quality of an LLM's outputs by comparing them against \( \mathcal{Y} \). 
The transformation function \( T(\cdot) \) modifies the data set during the benchmarking to avoid possible data contamination.
The dynamic dataset for the evaluation of an LLM can then be expressed as
$
\small
        \mathcal{D}_t = T_t(\mathcal{D}),  \quad
        \forall t \in \{1, \dots, N\}
$
where \( \mathcal{D}_t \) represents the evaluation data set at the timestamp \( t \), and \( N\) is the total timestamp number, which could be finite or infinite. % \ie $N= \infty$.
If the seed dataset $\mathcal{D}$ is empty, the dynamic benchmarking dataset will be created from scratch.


% \begin{figure}
%     \centering
%     \includegraphics[width=0.5\linewidth]{Move_teaser.pdf}
%     \caption{Comparison of different dynamic compute approaches. length of arrow indicates residual transformation per token while width indicates velocity of transformation.}
%     \label{fig:enter-label}
% \end{figure}

\section{Method}
\label{sec:method}
Residual connections play a crucial role in shaping token representations, yet their dynamics remain underexplored in the context of efficient decoding. In this work, we delve deeper into transformer residual dynamics and investigate how modulating residual transformation velocity can improve inference efficiency in token-level processing, optimizing both dense and sparse MoE transformers.


\subsection{Residual Dynamics and Motivation for Multi-rate Residuals} \label{sec:motivation}

To analyze how hidden representations evolve across different layers of a transformer architecture, it's crucial to consider the effect of residual connections. Each transformer decoder layer typically has residual connections across attention and MLP submodules. As the residual stream $h_i$ traverses from interval $E_j$ to $E_{j+1}$, it undergoes a residual transformation given by:  
% \begin{equation}
% \label{eq:slow_residual_transformation}
% H_{E_{j+1}} = H_{E_j} \prod_{i=E_j}^{E_{j+1}} \left( I + \mathcal{A}_i \right) \left( I + \mathcal{M}_i \right) \quad \text{where} \quad \mathcal{A}_i = f(c_i, h_{i}), \mathcal{M}_i = g(h_i)
% \end{equation}

\begin{equation} \label{eq:slow_residual_transformation}
h_{E_{j+1}} = h_{E_j} + \sum_{i=E_j}^{E_{j+1}-1} \left( \mathcal{A}_i(h_i) + \mathcal{M}_i(h_i + \mathcal{A}_i(h_i)) \right) \quad \text{where} \quad \mathcal{A}_i = f(c_i, h_{i}), \mathcal{M}_i = g(h_i). 
\end{equation}

Here, \( \mathcal{A}_i \) denotes the non-linear transformation introduced by the multi-head attention mechanism at layer \( i \), while \( \mathcal{M}_i \) corresponds to the non-linear transformation of the MLP block at the same layer. These transformations depend on the input residual stream \( h_i \) and, in the case of \( \mathcal{A}_i \), the previous contextual representation \( c_i \).\footnote{Normalization layers are typically applied in practice but are omitted here for simplicity of the argument.}


% For easy tokens, the magnitude and direction of this delta transformation become progressively smaller with each successive layer as shown in \cref{fig:delta_transformation}. Consequently, it is feasible to predict these tokens after only a few residual connections, whereas harder tokens necessitate more extensive processing through additional layers.

\begin{figure}[ht]
    \centering
    \begin{subfigure}{0.48\textwidth}
        \centering
        \includegraphics[width=\textwidth]{sections/figures/residual_change.pdf}
        \caption{}
        \label{fig:residual_change}
    \end{subfigure}%
    \hfill
    \begin{subfigure}{0.48\textwidth}
        \centering
        \includegraphics[width=\textwidth]{sections/figures/alignment_wrt_dedicated_model.pdf}
        \caption{}
    \label{fig:alignment_wrt_dedicated_model}
    \end{subfigure}
    \caption{(a) As residual streams propagate through the model, the directional shifts in the residuals become progressively smaller. (b) A dedicated model with $k$ layers achieves a faster rate of change in residual streams and higher alignment than base model leveraging early exit mechanisms at layer $k$.}
    \label{fig}
\end{figure}


To examine whether residual transformations can be accelerated across layers, we conducted experiments using a diverse set of prompts on a pre-trained Phi3 model~\cite{phi3_report}. As illustrated in \cref{fig:residual_change}, we measured the directional shift in residual states as \( 1 - \mathcal{C}(h_{i-1}, h_i) \), where \(\mathcal{C}\) denotes normalized cosine similarity. This shift is notably higher in the initial layers, gradually decreasing in subsequent layers. This behavior allows traditional early exit approaches to effectively accelerate decoding by enabling earlier exits for simpler tokens. However, these approaches typically rely on a distance-based approximation, where the full residual transformation of the model is approximated by the residual transformations of the initial layers. To gain deeper insights into the distance versus velocity aspects of residual transformation, we conducted a comparative study. Specifically, we trained an early exit head at layer $k$ of the Phi3 model, which consists of 32 layers, restricting the distance traveled by each token. To accelerate the residual transformation relative to number of layers, we trained a smaller model consisting of only $k$ layers, while keeping all other hyperparameters consistent. We then compared the next-token prediction accuracy of the early exit head of the base model with that of the smaller model. To ensure an equal number of trainable parameters, we inserted low-rank adapters into the smaller model and trained only these adapters, whereas, in the distance-based approach, we trained solely the early exit head. In addition, to accelerate the residual transformation in smaller model, we distilled the residual streams from the larger model by incorporating a distillation loss ~\cite{sanh2019distilbert} between the residual state at layer \(i\) of the smaller model and the residual state at layer \(4 \times i\) of the larger model. As shown in ~\cref{fig:alignment_wrt_dedicated_model} the smaller model demonstrates a significantly faster rate of change in residual streams, leading to higher next token prediction accuracy after $k$ layers compared to the base model that employs traditional early exit mechanisms after $k$ layers \cite{schuster2022confident, chen2023eellm, varshney-etal-2024-investigating}. This experimental setup, which modifies only the rate of change in residual streams while keeping other factors constant, suggests that dense transformers, trained with a fixed number of layers, may inherently possess a slow residual transformation bias.

This observation raises an intriguing question: if the rate of change in residual streams could be accelerated relative to the number of layers, is it possible to facilitate earlier alignment for a greater proportion of tokens? Earlier alignment would be beneficial to not only facilitate dynamic computation but also for generating speculative tokens efficiently with high acceptance rates in speculative decoding setups ~\cite{leviathan2023fast, chen2023accelerating}. 

%thereby enhancing the efficiency of early exiting? 
 % This bias likely constrains the effectiveness of early exiting, particularly for easier tokens. By addressing this limitation through accelerated residual transformations, we hypothesize that it is possible to substantially improve the efficiency and accuracy of early exit strategies in transformer models.

\subsection{Multi-Rate Residual Transformation} \label{m2r2_method}

To address the slow residual transformation bias described in ~\cref{sec:motivation}, we introduce \textit{accelerated residual streams} that operate at rate $R$ relative to original slow residual stream. We pair slow residual stream, $h$ with an accelerated residual stream, $p$, which has an intrinsic bias towards earlier alignment. Relative to ~\cref{eq:slow_residual_transformation}, accelerated residual transformation from interval $E_j$ to $E_{j+1}$ can be represented as: 

% \begin{equation}
% \label{eq:fast_residual_transformation}
% P_{E_{j+1}} = P_{E_j} \prod_{i=E_j}^{E_{j+1}} \left( I + \hat{\mathcal{A}_i} \right) \left( I + \hat{\mathcal{M}_i} \right) \quad \text{where} \quad \hat{\mathcal{A}_i} = \hat{f}(c_i, P_{i}), \hat{\mathcal{M}_i} = \hat{g}(P_{i})
% \end{equation}


\begin{equation} \label{eq:fast_residual_transformation}
p_{E_{j+1}} = p_{E_j} + \sum_{i=E_j}^{E_{j+1}-1} \left( \hat{\mathcal{A}_i}(p_i) + \hat{\mathcal{M}_i}(p_i + \hat{\mathcal{A}_i}(p_i)) \right) \quad \text{where} \quad \hat{\mathcal{A}_i} = \hat{f}(c_i, p_{i}), \hat{\mathcal{M}_i} = \hat{g}(h_i), 
\end{equation}



where $\hat{\mathcal{A}_i}$ and $\hat{\mathcal{M}_i}$ denote non-linear transformation added by layer $i$ to previous accelerated residual $p_{i}$. Similar to $\mathcal{A}_i$, non-linear transformation $\hat{\mathcal{A}_i}$ attends to same context $c_i$ but uses a different transformation $\hat{f}$ for accelerating $p_{E_j}$ relative to $h_{E_j}$. 

We integrate accelerated residual transformation directly into the base network using parallel accelerator adapters such that rank of accelerator adapters $R_p << d$ where $d$ denotes base model hidden dimension. This setup allows the slow residual stream $h_{E_j}$ to pass through the base model layers while the accelerated residual stream $p_{E_j}$ utilizes these parallel adapters as shown in ~\cref{fig:m2r2_main}. Both slow and accelerated residuals are processed in same forward pass via attention masking and incur negligible additional inference latency in memory bound decoding setups, while in compute bound decoding setups where FLOPs optimization is essential, accelerated residual stream utilizes a fraction of attention heads that of slow residual (see ~\cref{sec:flops_optimization}). Additionally, to maximize the utility of accelerated residual transformations without introducing dedicated KV caches, we propose a shared caching mechanism between the slow and accelerated streams which minimally impact alignment benefits of our approach while offering substantial memory savings (see ~\cref{fig:koala_alignment}). Specifically, the attention operation on the slow residuals \( \text{MHA}(h_t, h_{\leq t}, h_{\leq t}) \) is redefined for accelerated residuals as 
\[
\hat{\mathcal{A}} = MHA(p_t, h_{<t} \oplus p_t, h_{<t} \oplus p_t),
\]
where the accelerated residual at time-step $t$, \( p_t \) attends to the slow residual’s KV cache, facilitating the reuse of contextual information across both residual streams without incurring additional caching costs. Here, \(MHA(q, k, v) \) represents multi-head attention between query \( q \), key \( k \), and value \( v \).

\begin{figure}
    \centering
    \includegraphics[width=0.8\linewidth]{sections//figures/m2r2_main2.pdf}
    \caption{Multi-rate Residuals Framework: Slow residual stream of base model is accompanied by a faster stream that operates at a $2-(J+1)\times$ rate relative to the slow stream, undergoing transformations via accelerator adapters as detailed in \cref{m2r2_method}, where J denotes number of early exit intervals. Colors within the slow and fast residual streams indicate similarity, with matching colors representing the most closely aligned residual states. At the beginning of the forward pass and at each exit point, the accelerated residual state is initialized from the corresponding slow residual state to avoid gradient conflict during training (see ~\cref{sec:grad_conflict}). Early exiting decisions are informed by the Accelerated Residual Latent Attention (ARLA) mechanism, described in \cref{method_arla}, which evaluates residual dynamics across consecutive exit gates.}
    \label{fig:m2r2_main}
\end{figure}

% Furthermore. to maximize the benefits of fast residual transformations without using dedicated KV caches, we propose sharing the fast network’s cache with the slow network. Formally speaking, We modify attention operation on slow residuals $MHA(H_t, H_{<=t}, H_{<=t})$ as $MHA(P_{t}, H_{<t} \oplus P_t, H_{<t}  \oplus P_t)$ such that accelerated residuals attend to previous slow context KV cache, where $MHA(q,k,v)$ denotes multi head attention between query, $q$, key $k$ and value $v$.


\subsection{Enhanced Early Residual Alignment}
Early residual alignment is instrumental in optimizing early exiting, speculative decoding, and Mixture-of-Experts (MoE) inference mechanisms. In this section, we provide a detailed analysis of how accelerated residuals enhance these inference setups.

% By aligning the residual states of intermediate layers with the final output representations, the model can maintain high prediction accuracy even when computations are truncated at earlier layers. This enables more reliable early exiting, reducing the overall computational cost while preserving performance. Additionally, in speculative decoding, early residual alignment allows the model to make confident predictions using faster, partial computations, thereby accelerating inference without sacrificing output quality.


\subsubsection{Early Exiting} \label{method_early_exiting}

A prevalent strategy for enabling early exiting at an intermediate layer $E_{j}$ involves approximating the residual transformation between $E_{j}$ and the final layer $N-1$ using a linear, context independent mapping, $\mathcal{T}$, such that $H_{N-1} \approx \mathcal{T}(H_{E_{j}})$. This approximation has been extensively employed in conventional approaches ~\cite{schuster2022confident, chen2023eellm, varshney-etal-2024-investigating}, providing a computationally efficient means to project the output of deeper layers from intermediate states. Specifically, residual state of layer $N-1$ with this approximation can be expressed as:


% \begin{equation}
% \label{eq: vanila_ea_assumption}
% \Phi(H_{E_{j}}) \sim H_{E_{j}} \prod_{i=E_{j}}^{N}\left( I + \mathcal{A}_i \right) \left( I + \mathcal{M}_i \right) \quad \text{where} \quad \Phi \perp C
% \end{equation}

\begin{equation} \label{eq:early_exiting}
h_{E_j} + \sum_{i=E_j}^{N-1} \left( \mathcal{A}_i(h_i) + \mathcal{M}_i(h_i + \mathcal{A}_i(h_i)) \right) \sim \mathcal{T}(h_{E_{j}})  \quad \text{where} \quad \mathcal{T} \perp c. 
\end{equation}


Here, $\mathcal{A}_i$ and $\mathcal{M}_i$ represent the residual contributions of the multi-head attention and MLP layers, respectively, while $\mathcal{T}$ remains independent of $c$, the preceding context.

This approach is inherently limited by two major factors: first, the assumption of linearity between $h_{E_{j}}$ and $h_{N-1}$ may not hold uniformly for all tokens, particularly when $E_j \ll N$. Second, the linear transformation $\mathcal{T}$ disregards the influence of the context $c$ and fails to account for the latent representations of previous contextual states. In contrast, M2R2 accelerated residual states mitigate both of these challenges by approximating the slow residual transformation of all layers via a faster residual transformation of fewer layers as:
% \begin{equation}
% H_{E_j} \prod_{i=E_j}^{N}\left( I + \mathcal{A}_i \right) \left( I + \mathcal{M}_i \right) \sim P_{E_j} \prod_{i=E_j}^{E_j+1}\left( I + \hat{\mathcal{A}_i} \right) \left( I + \hat{\mathcal{M}_i} \right)
% \end{equation}


\begin{equation} \label{eq:m2r2_approximating_ea}
h_{E_j} + \sum_{i=E_j}^{N-1} \left( \mathcal{A}_i(h_i) + \mathcal{M}_i(h_i + \mathcal{A}_i(h_i)) \right) \sim p_{E_j} + \sum_{i=E_j}^{E_{j+1}-1} \left( \hat{\mathcal{A}_i}(p_i) + \hat{\mathcal{M}_i}(p_i + \hat{\mathcal{A}_i}(p_i)) \right), 
\end{equation}

% \begin{equation} \label{eq:fast_residual_transformation}
% p_{E_{j+1}} = p_{E_j} + \sum_{i=E_j}^{E_{j+1}-1} \left( \hat{\mathcal{A}_i}(p_i) + \hat{\mathcal{M}_i}(p_i + \hat{\mathcal{A}_i}(p_i)) \right) \quad \text{where} \quad \hat{\mathcal{A}_i} = \hat{f}(c_i, p_{i}), \hat{\mathcal{M}_i} = \hat{g}(h_i) 
% \end{equation}






where $p_{E_j}$ is initialized from the slow residual state $h_{E_j}$ at each early exit interval $E_j$ using an identity transformation (see ~\cref{fig:m2r2_main}). As shown in ~\cref{fig:m2r2_residual_sim}, accelerated residuals offer a smoother, more consistent shift in residual direction across layers, in contrast to the abrupt changes typically seen at early exit points in standard early exit methods. Moreover, the normalized cosine similarity between accelerated states at early exit intervals and final residual states is substantially higher compared to traditional early exit techniques, highlighting improved alignment with final layer representations. Traditional adaptive compute methods are constrained by two principal factors: the number of tokens eligible for early exit at intermediate layers and the precision of early exit decision. If residual streams fail to saturate early, the majority of tokens remain ineligible for exit, thereby diminishing potential speedups. Additionally, imprecise delineations between tokens suitable for early exit can lead to underthinking (premature exits that adversely affect accuracy) or overthinking (unnecessary processing that compromises efficiency) ~\cite{zhou2020self, dai2020dynamic}. Enhanced early alignment using ~\cref{eq:m2r2_approximating_ea} helps to address  first issue. To address the second issue we introduce Accelerated Residual Latent Attention, which dynamically assesses the saturation of the residual stream, allowing for a more precise differentiation between tokens that can exit early and those requiring further processing.

% This results in uniform change in residual direction    
% % We keep $\mathcal{A} = \hat{\mathcal{A}}$, while $\hat{\mathcal{M}}$ is accelerated by a factor of $2 - (N_{E}+1)X$ relative to the slower residual transformation $\mathcal{M}$, where $N_E$ represents number of early exiting intervals.
% Figure~\cref{fig:rate_change_comparison} illustrates the comparative rate of change between these transformation streams.



% fig:rate_change_comparison
% - grid plot x axis -> layer id (0, 8) , y axis -> layer id -> dark color cell for max similarity , lighter for lower 
% 
-------------------------------------------------------
Let's consider residual stream $h_i$ traverses through interval $E_j$ to $E_{j+1}$ and undergoes residual transformation given by 
\begin{equation}
h_{E_{j+1}} = h_{E_j} \prod_{i=E_j}^{E_{j+1}} \left( 1 + \delta_i \right)    
\end{equation}

where $\delta_i$ denotes non-linear transformation added by layer $i$. Each non-linear transformation of layer $i$ is a function of previous contextual representation, $c_i$ and input residual stream $h_i-1$ as
$\delta_i = f(c_i, h_{i-1})$ 

One way to exit early at exit $E_j+1$ is to assume that residual transformation from $E_j+1$ to final layer $N-1$ can be approximated by a linear function $\phi$ as $h_{N-1} \sim \Phi(h_{E_j+1})$ and most conventional approaches such as \todo{cite EA papers} use this approach. In other words, 

\begin{equation}
\Phi(h_{E_j+1} \sim h_{E_j+1} \prod_{i=E_j+1}^{N} \left( 1 + \delta_i \right)   
\end{equation}

This approach suffers from two primary issues, linearity assumption from $h_E_j+1$ to $H_N-1$ if often incorrect, particularly when $E_j << N$. More importantly, linear transformation $\Phi$ doesn't consider effect of context $C_i$. M2R2  effectively addresses these issues as accelerated residual stream at interval $E_j+1$ can be represented as 

\begin{equation}
r_{E_{j+1}} = r_{E_j} \prod_{i=E_j}^{E_{j+1}} \left( 1 + \gamma_i \right)    
\end{equation}

where $\gamma_i$ denotes non-linear transformation added by layer $i$ to previous accelerated residual $r_i-1$. Similar to $\delta_i$, non-linear transformation $\gamma_i$ considers context $C_i$ as 
$\gamma_i = g(c_i, r_{i-1})$. So in summary, slow residual transformation is approximated by accelerated residual as: 

\begin{equation}
h_{E_j} \prod_{i=E_j}^{N} \left( 1 + \delta_i \right) \sim h_{E_j} \prod_{i=E_j}^{E_j+1} \left( 1 + \gamma_i \right)
\end{equation}

It's worth noting that accelerated residual $r_i$ and slow residual $h_i$ are processed concurrently at layer $i$ by constructing proper attention mask such as attention of slow residual is represented as 

$MHA(H_it, H_{i<=t}, H_{i<=t}$ while attention of fast residual is computed as 

$MHA(r_it, H_{i<=t}, H_{i<=t}$ where $MHA(q,k,v$ denotes multi head attention between query, $q$, key $k$ and value $v$.


------------------------------------------------------------------

Vertical latent attention on accelerated residual is computed as 
$MHA(S_mt, S(Ej<=i<=m)t, S(Ej<=i<=m)t)$ where $Smt$ denotes query/key/value projection in latent domain at layer $m$ at time $t$. 
------------------------------------------------------------------

Gradient conflict Avoidance: 

Let's consider $w_j$ is a trainable parameter that belongs to a layer between $E_j$ and $E_j+1$. Consider early exit loss at gate $E_j+1$, $L_j+1$, gradient propagation of $w_j$ at another trainable parameter $w_j-n$ can be gives as 

$\sum_{k=E_j-n}^{E_j} \beta_k \frac{\partial L_{E_k}}{\partial w_k}$

where $\beta_j$ denotes backward transformation coefficient for weight $w_j$ to reach gate $E_j$. 
 
On the other hand, gradient propagation in proposed approach can be represented as 

\[
\frac{\partial L_{E_j}}{\partial w_j} = 
\begin{cases} 
\beta_j \frac{\partial L_{E_j}}{\partial w_j} & \text{if } E_j \leq w_j \leq E_{j+1} \\
0 & \text{otherwise}
\end{cases}
\]







% \begin{figure}[ht]
%     \centering
%     \includegraphics[width=0.8\textwidth, height=5cm]{rate_change_comparison.png}
%     \caption{Rate of change comparison between fast and slow residual streams.}
%     \label{fig:rate_change_comparison}
% \end{figure}

%vary k and and plot EA accuracy for larger and smaller models. 

% \begin{figure}[ht]
%     \centering
%     \includegraphics[width=0.5\textwidth,height=5cm]{sections/figures/alignment_comparison_dialogsum.pdf}
%     \caption{Alignment of exited tokens for different early exit layers using traditional early exiting heads, dedicated faster networks, and faster residuals.}
%     \label{fig:small_model_early_exiting}
% \end{figure}


\textbf{Accelerated Residual Latent Attention} \label{method_arla}

In the context of residual streams, we observe that the decision to exit at a given layer can be more effectively informed by analyzing the dynamics of residual stream transformations, instead of solely relying on a classification head applied at the early exit interval $E_j$. To capture the subtle dynamics of residual acceleration, we propose a \textit{Accelerated Residual Latent Attention} (ARLA) mechanism. This approach involves making the exit decision at gate $E_j$ by attending to the residuals spanning from gate $E_{j-1}$ to $E_j$, rather than considering only the residual at gate $E_j$. To minimize the computational overhead associated with exit decision-making, the attention mechanism operates within the latent domain as depicted in ~\cref{fig:arla_arch}. Formally, for each interval $[E_j, E_{j+1}]$, the accelerated residuals are projected into Query ($Q^s_{E_j}, \ldots, Q^s_{E_{j+1}}$), Key ($K^s_{E_j}, \ldots, K^s_{E_{j+1}}$), and Value ($V^s_{E_j}, \ldots, V^s_{E_{j+1}}$) vectors, with latent dimension $d^s$ for $Q^s$, $K^s$, and $V^s$ being significantly smaller than hidden dimension of $p$.\footnote{We use $d^s = 64$ for experiments described in ~\cref{sec:experiments}.} Notably, when the router is allowed to make exit decisions at gate $E_j$ based on residual change dynamics, we observe that the attention is not confined to the residual state at $E_j$ but is distributed across residual states from $E_{j-1}$ to $E_j$, %as illustrated in Figure~\ref{fig:vertical_latent_attention_dynamics}. 
This broader focus on residual dynamics significantly reduces decision ambiguity in early exits, as demonstrated in Figure~\ref{fig:roc_arla}, which contrasts routers based on the last hidden state, and the proposed ARLA router.

%show R -> S transformation. 
%show parameter and flop overhead as compared to adapter on last hidden state.

% \begin{figure}[ht]
%     \centering
%     \includegraphics[width=0.5\textwidth,height=5cm]{sections/figures/roc_arla.pdf}
%     \caption{ROC curves of early exit decision strategies: confidence-based methods (CALM/LITE), routers based on the accelerated hidden state, and latent attention routers.}
%     \label{fig:decision_making_comparison}
% \end{figure}

% \begin{figure}[ht]
%     \centering
%     \includegraphics[width=0.5\textwidth,height=5cm]{vertical_latent_attention.png}
%     \caption{Vertical latent attention mechanism for optimizing early exit decisions by considering residuals from gate \(M\) through \(M-1\).}
%     \label{fig:vertical_latent_attention}
% \end{figure}

\begin{figure}[ht]
    \centering
    \begin{subfigure}{0.52\textwidth}
        \centering
        \includegraphics[width=\textwidth, height = 4cm]{sections/figures/arla_arch.pdf}
        \caption{Accelerated Residual Latent Attention (ARLA): Accelerated residuals between early exit gates are projected into latent domain and attention over residual states within the interval is computed to capture residual dynamics and exit decision is made based on residual saturation.}
        \label{fig:arla_arch}
    \end{subfigure}%
    \hfill
    \begin{subfigure}{0.45\textwidth}
        \centering
        \includegraphics[width=\textwidth, height = 4.5cm]{sections/figures/vla_roc.pdf}
        \caption{ROC classification curves of early exit decision strategies using a linear router used on last residual state ~\cite{schuster2022confident, varshney-etal-2024-investigating, chen2023eellm}  and using ARLA approach that considers residual dynamics. }
        \label{fig:roc_arla}
    \end{subfigure}
    \caption{Effectiveness of ARLA in capturing residual dynamics for early exiting decisions.}


\end{figure}



% \begin{figure}[ht]
%     \centering
%     \includegraphics[width=1\textwidth,height=5cm]{sections/figures/arla.pdf}
%     \caption{fig that plots 32 rows 2 cols heatmap showing attention at each gate}
%     \label{fig:vertical_latent_attention_dynamics}
% \end{figure}

\subsubsection{Self Speculative Decoding} \label{method_self_speculative_decoding}

An alternative means to exploit the early alignment properties of our approach is through the use of accelerated residual states for speculative token sampling to accelerate autoregressive decoding. Speculative decoding aims to speed up memory-bound transformer inference by employing a lightweight draft model to predict candidate tokens, while verifying speculated tokens in parallel and advancing token generation by more than one token per full model invocation \cite{leviathan2023fast, chen2023accelerating, xia2023speculative, miao2023specinfer}. Despite its effectiveness in accelerating large language models (LLMs), speculative decoding introduces substantial complexity in both deployment and training. A separate draft model must be specifically trained and aligned with the target model for each application, which increases the training load and operational complexity ~\cite{chen2023accelerating}. Additionally, this approach is resource-inefficient, as it requires both the draft and target models to be simultaneously maintained in memory during inference \cite{leviathan2023fast, chen2023accelerating}. 

One strategy to address this inefficiency is to leverage the initial layers of the target model itself to generate speculative candidates, as depicted in ~\cite{Tang2024}. While this method reduces the autoregressive overhead associated with speculation, it suffers from suboptimal acceptance rates. This occurs because the linear transformation employed for translating hidden states from layer $k$ to the final layer $N$ is typically a poor approximation, as discussed in ~\cref{sec:motivation} and ~\cref{method_early_exiting}. Our approach resolves this limitation by utilizing accelerated residuals, which demonstrate higher fidelity to their slower counterparts. By utilizing accelerated residuals operating at a rate of $N/k$, where $k$ denotes the number of layers used for candidate speculation, we are able to efficiently generate speculative tokens for decoding.\footnote{We typically set $k = 4$ to balance the trade-off between autoregressive drafting overhead and acceptance rate, as discussed in~\cref{sec:experiments}.}
 This technique not only obviates the need for multiple models during inference but also improves the overall efficiency and effectiveness of speculative decoding.

\begin{figure}
    \centering    \includegraphics[width=1\linewidth]{sections/figures/m2r2_aot_loading.pdf}
    \caption{Ahead-of-Time Expert Loading: M2R2 accelerated residual stream predicts experts required for future layers, reducing reliance on on-demand lazy loading. Speculative pre-loading is efficiently overlapped with computation of multi-head attention (MHA) and MLP transformations. Only incorrectly speculated experts are loaded lazily, resulting in faster inference steps and improved computational efficiency. Here, H indicates LBM Host while D indicates HBM Device.}
    \label{fig:moe_expert_aot_loading}
\end{figure}


\subsubsection{Ahead of Time Expert Loading:} \label{method_aot_expert_loading}

Recent advancements in sparse Mixture-of-Experts (MoE) architectures ~\cite{shazeer2017outrageously, fedus2022switch, artetxe2019massively, lepikhin2020gshard, zoph2022designing} have introduced a paradigm shift in token generation by dynamically activating only a subset of experts per input, achieving superior efficiency in comparison to dense models, particularly under memory-bound constraints of autoregressive decoding \cite{fedus2022switch, zoph2022designing}. This sparse activation approach enables MoE-based language models to generate tokens more swiftly, leveraging the efficiency of selective expert usage and avoiding the overhead of full dense layer invocation. In dense transformer models, pre-loading layers is a common strategy to enhance throughput, as computations of current layer can be overlapped with pre-loading of next layer parameters ~\cite{narayanan2021efficient, shoeybi2020megatron}. However, MoE models face a unique challenge: expert selection occurs dynamically based on previous layer’s output, making it infeasible to preload next layer’s experts in parallel. This limitation results in inherent latency, as expert loading becomes a sequential, on-demand process ~\cite{lepikhin2020gshard, fedus2022switch}.

To address this inefficiency, our method introduces a mechanism with \textit{accelerated residuals}, which not only captures key characteristics of base slower residual states but also exhibit high cosine similarity with their final counterparts (as illustrated in \cref{fig:m2r2_residual_sim}). By employing accelerated residual streams, we can effectively predict the necessary experts for future layers well in advance of their actual invocation. Specifically, using a $2\times$ accelerated residual, the experts needed for layers $2i+2$ and $2i+3$ can be identified while still computing in layer $i$, thus overcoming the bottleneck of sequential, on-demand expert selection and mitigating latency in the decoding pipeline, as shown in \cref{fig:moe_expert_aot_loading}. Note that, we use fixed set of accelerator adapters for transforming accelerated residuals (as discussed in ~\cref{m2r2_method}) while slow residual is transformed via expert routing mechanism. 

Furthermore, our approach integrates a Least Recently Used (LRU) caching strategy, which enhances memory efficiency by replacing the least recently used experts with speculated experts that are anticipated to be needed in upcoming layers. This hybrid approach of preemptive expert loading with LRU caching yields substantial improvements over traditional on-demand loading or standalone caching strategies. By minimizing cache misses and efficiently managing memory, this approach addresses both compute and memory bottlenecks, leading to faster, more resource-efficient token generation in MoE architectures. A comprehensive evaluation of this strategy, in relation to state-of-the-art methods, is provided in \cref{experiments_aot}, and the compute and memory traces on an A100 GPU are detailed in \cref{fig:moe_aot_cuda_trace}.



% Recent advancements in sparse Mixture-of-Experts (MoE) architectures have introduced the concept of utilizing distinct computational paths for different tokens \cite{shazeer2017outrageously}. This approach, wherein only a subset of experts are activated per input, enables MoE-based language models to generate tokens more swiftly compared to their dense counterparts due to memory-bound nature of auto-regressive decoding. In dense models, pre-loading layers in advance is a common strategy to enhance computational efficiency. However, this technique is not applicable to MoE models, where expert selection occurs dynamically based on the outputs of previous layers, preventing parallel pre-fetching of experts.

% Our proposed method addresses this inefficiency. Accelerated residuals, which are highly similar to their slower counterparts (see \cref{fig:similarity}), can reliably predict the necessary experts ahead of time. For instance, by utilizing $2X$ accelerated residual stream, we can predict the experts needed for the layer $2i+1$ and $2i+3$ while carrying out computation in layer $i$. This enables us to commence expert loading significantly earlier, as illustrated in \cref{expert_loading}, effectively mitigating the delays observed with the naive on-demand expert loading. Additionally, our method benefits from incorporating a Least Recently Used (LRU) strategy, where speculated experts replace those that are least recently utilized, resulting in improved performance compared to using either strategy alone. For a comprehensive evaluation, refer to \cref{moe_trace}, which provides a CUDA compute and memory trace of our approach executed on <>.



% A naive solution involves using the residual state of the previous layer along with the gating function of the next layer to predict which experts need to be loaded, and initiating the expert loading process in parallel with the attention computation of the next layer. Yet, as shown in \cref{fig:MOE_attn_vs_loading_time}, the attention computation for medium to long contexts is considerably faster than the expert loading time, making this approach inefficient.




\subsection{Training} \label{method_training}
% This approach is feasible due to the absence of gradient conflicts, as discussed in \cref{sec:grad_conflict}.

To accelerate residual streams, we employ parallel accelerator adapters as described in \cref{m2r2_method}.  For the early exiting use-case outlined in \cref{method_early_exiting}, we define the training objective for these adapters using the following loss function, which combines cross-entropy loss at each exit $E_j$ with distillation loss at each layer $i$. Loss weights coefficients $\alpha_0$ and $\alpha_1$ are employed to balance contribution of corresponding losses.

\begin{align} \label{eq:mr_loss}
L_{\text{m2r2}} = \underbrace{-\alpha_0 \sum_{j=1}^{J} \sum_{t=1}^{T} \log p_{\theta} \left( \hat{y}_t^{E_j} \mid y_{<t}, x \right)}_{\text{cross-entropy loss}} 
+ \underbrace{\alpha_1\sum_{i=1}^{E_{J-1}} \sum_{t=1}^{T} \| \mathbf{p}_{t}^{i} - \mathbf{h}_{t}^{((i - E_{j(i)}) \cdot R_i) + E_{j(i)})} \|^2}_{\text{distillation loss}}.
\end{align}

where $\hat{y}_t^{E_j}$ denotes the predictions from the accelerated residual stream at layer $E_j$ and time step $t$, $y_t$ represents the corresponding ground truth tokens, and $x$ indicates previous context tokens. The distillation loss at each layer $i$ is computed by comparing accelerated residuals at layer $i$ with slow residuals at layer $(i - E_{j(i)}) \cdot R_i + E_{j(i)}$, where $R_i$ denotes the rate of accelerated residuals at layer $i$ while $E_{j(i)}$ represents the most recent gate layer index such that $E_{j(i)} <= i$. \( J \) represents the total number of early exit gates, N denotes number of hidden layers and $E_j$ denotes layer index corresponding to gate index $j$ and \( T \) denotes the sequence length. 

In dynamic compute settings, after training of accelerator adapters, we optimize the query, key, and value parameters governing the ARLA routers (see ~\cref{method_arla}) across all exits in parallel on binary cross entropy loss between predicted decision and ground truth exiting decision. The ground truth labels for the router are determined based on whether the application of the final logit head on $\hat{y}_t^{E_j}$ yields the correct next-token prediction. 


% The objective for this optimization is defined by the following loss function:


%TODO are equations required ? 
% \begin{equation} \label{eq:arla_loss_combined}\small
%     L_{\text{arla}} = -\frac{1}{N} \sum_{t=1}^{T} \left( \sum_{j=1}^{E_n} \left[ O_t^{E_j} \log(\hat{O}_t^{E_j}) + (1 - O_t^{E_j}) \log(1 - \hat{O}_t^{E_j}) \right] \right), \quad \text{where} \quad 
%     O_t^{E_j} = \begin{cases} 
%     1, & \text{if } L(\hat{y}_t^{E_j}) = y_t^{E_j} \\
%     0, & \text{otherwise}
%     \end{cases}
% \end{equation}

% where $\hat{O}_t^{E_j}$ represents the binary predicted logits produced by the vertical latent attention router, as described in \cref{sec:arla}, at gate $E_j$ and time step $t$, and $O_t^{E_j}$ denotes the corresponding ground truth labels. The ground truth labels for the router are determined based on whether the application of the logit head on $\hat{y}_t^{E_j}$ yields the correct next-token prediction. The parameters controlling vertical latent attention are trained concurrently to ensure consistency and efficient use of computational resources.

For self-speculative decoding, as described in \cref{method_self_speculative_decoding}, the training objective remains the same as \cref{eq:mr_loss}, but with the number of intervals set to $J = 1$ and the rate of residual transformation set to $R_n = N/k$, where the first $k$ layers generate speculative candidate tokens. In the context of Ahead-of-Time Expert Loading for Mixture-of-Experts (MoE) models (see \cref{method_aot_expert_loading}), setting the rate of residual transformation to $R_n = 2$ typically offers a good trade-off between the accuracy of expert speculation and AoT pre-loading of experts. 

% Thus, we set $J = 1$ and $E_1 = 16$.


~\subsection{FLOPs Optimization} \label{sec:flops_optimization}

Naively implemented, M2R2 incurs higher FLOP overhead compared to traditional speculative decoding and early exiting approaches such as ~\cite{medusa, schuster2022confident, Tang2024}. However, modern accelerators demonstrate compute bandwidth that exceeds memory access bandwidth by an order of magnitude or more~\cite{databricksLLMInference2023, jouppi2021ten}, meaning increased FLOPs do not necessarily translate to increased decoding latency. Nevertheless, to ensure fair comparison and efficiency in compute bound scenarios, we introduce targeted optimizations.

~\textbf{Attention FLOPs Optimization} For medium-to-long context lengths, attention computation dominates FLOPs in the self-attention layer, surpassing the contribution from MLP layers. Specifically, matrix multiplications involving queries, cached keys, and cached values scale with $l_{kv} * l_{q}$ where $l_{kv}$ denotes previous context length and $l_q$ denotes current query length. Since M2R2 pairs accelerated residuals with slow residuals, a naive implementation results in twice the FLOPs consumption compared to a standard attention layer. To address this, we limit the attention of accelerated residual stream to selectively attend to the top-k most relevant tokens, identified by the slow residual stream based on top attention coefficients\footnote{We set to k = 64 and attend to top 64 tokens as identified by the slow residual stream.}. This is possible since slow and accelerated residual streams are processed in same forward pass and accelerated streams have access to attention coefficients of slow stream. Note that, the faster residual stream still retains the flexibility to assign distinct attention coefficients to these tokens. Furthermore, we design the faster residual stream to employ only 8 attention heads, compared to the 32 heads used in the slow residual stream of the Phi-3 model, reducing query, key, value, and output projection FLOPs by a factor of 1/4. ~\cref{fig:m2r2_num_heads_ablation} indicates effect of using a slicker stream on alignment. As depicted, using $\hat{n}_h = 8$ offers a good trade-off between alignment and FLOPs overhead. 

~\textbf{MLP FLOPs Optimization} The accelerator adapters operating on the accelerated residual stream are intentionally designed with lower rank than their counterparts in the base model. This reduces FLOP overhead by a factor proportional to $hiddenSize / rank$. Additionally, since the faster residual stream uses only 8 attention heads (compared to 32 in the slow residual stream of Phi-3), the subsequent MLP layers process a smaller set of activations, further reducing FLOPs by another factor of 1/4.

These optimizations significantly reduce the FLOP overhead per speculative draft generation, as illustrated in ~\cref{fig:flops_optmization}. Notably, while traditional early-exiting speculative approaches such as DEED require propagating the full slow residual state through the initial layers, incurring substantial computational costs, M2R2 achieves efficient token generation via slimmer, low-rank faster residual streams. In contrast, Medusa introduces considerable FLOP overhead due to per-head computations scaling with $d^2+dv$\footnote{Here $d$ denotes hidden state dimension while $v$ denotes vocab size.}, whereas M2R2 employs low-rank layers for both MLP and language modeling heads, maintaining computational efficiency. All experiments involving the M2R2 approach, as detailed in ~\cref{sec:experiments}, are conducted using these FLOPs optimizations.









% \[
% O_t^{E_j} = 
% \begin{cases} 
% 1, & \text{if } L(\hat{y}_t^{E_j}) = y_t^{E_j} \\
% 0, & \text{otherwise}
% \end{cases}
% \]




%add distillation
% We train accelerator adapters described in \cref{m2r2_method} to accelerate residual streams on next token prediction all in parallel since there are no gradient conflict issues as described in \cref{sec:grad_conflict}.

% \begin{align} \label{eq:mr_loss}
% L_{mr} =  & -\sum_{j = 1}^{E_n} (\sum_{t=1}^{T}\log p_{\theta} (\hat{y}_t^{E_j} | \hat{y}_{<t}, x)) \nonumber
% \end{align}

% where $\hat{y_t^{E_j}}$ denotes predicted logits obtained from accelerated residual stream at gate $E_j$ and time-step $t$ while $y_t^{E_j}$ denotes corresponding truth tokens. 

% Upon training of adapters responsible for accelerating residual streams, we train query, key, value parameters responsible for vertical latent attention of all gates in parallel as

% \begin{equation} \label{eq:arla_loss}
%     L_{arla} = -\frac{1}{N} (\sum_{t=1}^{T}(1\sum_{j=1}^{E_n} \left[ O_t^{E_j} \log(\hat{O}_t^{E_j}) + (1 - o_t^{E_j}) \log(1 - \hat{o_t}_{E_j}) \right]))
% \end{equation}

% where $\hat{O_t^{E_j}}$ denotes binary predicted logits obtained from vertical latent attention router described in \cref{sec:arla} at gate $E_j$ and timestep $t$ while $O_t^{E_j}$ denotes corresponding truth label. Truth labels for router are obtained by computing whether logit head application on $\hat{y}_t^j$ results in true next token prediction. Formally speaking, 

% $O_t^{E_j} = 1 if L(\hat{y_t^{E_j}}) == y_t^{E_j} , 0 otherwise$. 

% Parameters responsible for vertical latent attention are also trained in parallel as well. 

%todo: training slow and fast residuals together and distillation can be two training mdoes. 
%Distillation can be an ablation. 




% Although transformer decoding is memory bound on most mainstream accelerators, there could be scenarios where flop savings are crucial. For instance, on on-device settings power consumption is directly correlated with flops per decoding step and reducing flops does help with overall energy consumption. Vanilla early exiting methods help with flop reduction but suffer from mismatch between training and inference due to early exited tokens. If token at decoding step $t$, $T_t$ exited at layer $E_i$, while token $T_{t+k}$ exits at layer $E_j$ such that $E_i < E_j$, hidden state $H_{t+k}l$ does not have corresponding hidden state $H_tl$ to attend to where $E_i < l <= E_j$. One solution that's often used in literature is to rely on last hidden state available, $H_t{E_j}$, however it tends to be sub-optimal and does affect generation quality \cite{ref}.  To alleviate this mismatch while reducing flops, we train router such that attention mask between token $T_{t+k}$ and token $T_{<t+k}$ is given by: 

% \begin{equation}
%     a_{T_{{t+k}{T_{<t+k}}} = 1 if  E_{T_{<t+k}} >= E{T_{t+k}}
%     else 0
% \end{equation}

% This attention mask enables router to account for exited tokens and get trained accordingly. Since attention mechanism during decoding remains exactly same as that during training, impact on generation quality tends to be minimal as noted in \cref{fig:gen_auality_with_and_without_recompute_attention_show_flops}.  Although MoD does not suffer from training and inference mismatch, we observe that it suffers from discountinuity between pre-training and super-vised fine-tuning resulting in sub-optimal perplexity. On the other hand, our method doesn't not require pre-training , doesn't suffer from discountinuity, and achieves much better perplexity in super-vised fine-tuning and instruction tuning setups as shown in \cref{fig:Mod_vs_m2r2_loss_curves}.






% Our techniques are directly applicable in such scenarios.    




%expert loading with cuda streams in experiments
\section{Experiments: Planning outperforms Heuristics}
\label{sec:experiment}

We begin our empirical demonstrations by showcasing the effectiveness of our planning framework on both synthetic and real datasets. We focus on the simplest planning algorithm, 1-step lookaheads (Algorithm~\ref{alg:complete}), and show that even basic planning can hold great promise. 
We illustrate our framework using two uncertainty quantification modules---GPs and 
\ensembles/ \ensembleplus. 

Throughout this section, we focus on evaluating the mean squared error of 
a regression model $\model$,  and develop adaptive policies that minimize uncertainty on $g(f)$ defined in~\eqref{eqn:l2-g-f}.
When GPs provide a valid model of uncertainty, 
our experiments show that our planning framework significantly outperforms other baselines. 
We further demonstrate that our conceptual framework extends to deep learning-based uncertainty quantification methods such as  \ensembleplus while highlighting computational challenges that need to be resolved in order to scale our ideas. 
For simplicity, we assume a naive predictor, i.e., $\psi(\cdot) \equiv 0$. However, we emphasize that this problem is just as complex as if we were using a sophisticated model $\psi(.)$. The performance gap between the algorithms 
primarily depends
on the level  of uncertainty in our prior beliefs.

To evaluate the performance of our algorithm, we benchmark it against several baselines. 
%Active learning baselines use an acquisition function $\ac$ to select points that have the highest   function value: $X\opt_t \in \argmax_{X \in \xpoolj{t}} \ac({X})$ at every step $t$. These methods may also need an UQ module, which we simply use the same UQ module as in our algorithm, and it  outputs $V(X)$ that measures the the uncertainty of each point $X \in \xpoolj{t}$.
Our first set of baselines are from active learning~\citep{AggarwalKoGuHaPh14}:
\\ % \noindent\textbf{Active Learning Heuristics:} 
\textbf{(1)} 
\textsf{Uncertainty Sampling (Static):}  In this approach, we query the samples for which the model is least certain about. Specifically, we estimate the variance of the latent output $f(X)$ for each $X \in \xpool$ using the UQ module and select the top-$K$ points with the highest uncertainty. \\
\textbf{(2)} \textsf{Uncertainty Sampling (Sequential):} This is a greedy heuristic that sequentially selects the points with the highest uncertainty within a batch, while updating the posterior beliefs using pseudo labels from the current posterior state. Unlike \textsf{Uncertainty Sampling (Static)}, this method takes into account the information gained from each point within batch, and hence tries to diversify the selected points within a batch. 

 
We also compare our approach to the  \textbf{(3)} \textsf{Random Sampling}, which selects each batch uniformly at random from the pool. Additionally, we compare solving the planning problem using  \textsf{REINFORCE}-based policy gradients with   $\mathsf{Smoothed\text{-}Autodiff}$ policy gradients.\footnote{Our code repository is available at
  \url{https://github.com/namkoong-lab/adaptive-labeling}.}
%Detailed experimental setups are provided in Section \ref{sec:details-experiments}.

%We repeat all experiments with 10 random seeds.




\begin{figure}[t]
\centering
\begin{minipage}[b]{0.49\textwidth}
\centering
\includegraphics[width=\textwidth, height=5cm]{figures/original_scale/Var_of_l_2_loss.pdf}
\caption{(Synthetic data) Variance of mean squared loss evaluated through the posterior belief $\mu_t$ at each horizon $t$. This is the objective that policy gradient methods like \textsf{REINFORCE} and $\ouralgo$ optimizes. 1-step lookaheads are surprisingly effective even in long horizons.}
\label{fig:var-l2-sim}
\end{minipage}
\hfill
\begin{minipage}[b]{0.49\textwidth}
\centering \includegraphics[width=\textwidth, height=5cm]{figures/original_scale/Error_of_estimated_model_l_2_loss.pdf}
\caption{(Synthetic data) Error between MSE calculated based on collected data $\mc{D}^{0:T}$ vs. population oracle MSE over $\mc{D}_{\rm eval} \sim P_X$. Reducing uncertainty over posteriors directly leads to better OOD evaluations. 1-step lookaheads significantly outperform active learning heuristics in small horizons.}
\label{fig:mean-l2-sim}
\end{minipage}
%\caption{Simulated data for GPs}
%\label{fig:both_plots}
\end{figure}

\subsection{Planning with Gaussian processes}
\label{sec:experiment-plan-GP}
We now briefly describe the data generation process for the GP experiments,  deferring a more detailed discussion of the dataset generation to Section~\ref{sec:details-experiments}. 
We use both the synthetic data and the real data to test our methodology.
For the \emph{simulated data},  we construct a setting where the general population is distributed across \emph{51 non-overlapping clusters} while the initial labeled data $\dtrain$ just comes from one cluster. In contrast, both $\dpool \defeq (\xpool,\ypool),\deval \defeq (\xeval,\yeval)$ are generated   from all the clusters. 
We begin with a low-dimensional scenario, generating a one-dimensional regression setting using a GP. %Gaussian Process (GP).
Although the data-generating process is not known to the algorithms,  we assume that the GP hyperparameters are known to all the algorithms
to ensure fair comparisons. This can be viewed as a setting where our prior is well-specified, allowing us to isolate the effects
of different policy optimization approaches
 without any concerns about the misspecified priors. We select $10$ batches, each of size $K=5$ across $T = 10$ time horizons.

To examine the robustness of our method against the distributional assumptions made  in the simulated case, we then move to a real dataset where the correct prior is not known. We simulate selection bias from the eICU dataset~\citep{PollardJoRaCeMaBa18}, which contains real-world patient data with in-hospital mortality outcomes. 
We conduct a $k$-means clustering to generate 51 clusters and then select data from those clusters. We view this to be a credible replication of practice, as severe distribution shifts are common due to selection bias in clinical labels.  To convert the binary mortality labels into a regression setting, we train a  random forest classifier and fit a GP on predicted scores, which serves as the UQ module for all the algorithms. As before, the task is to select 10 batches, each consisting of 5 samples, across 10 time horizons.

 In Figures~\ref{fig:var-l2-sim} and~\ref{fig:mean-l2-sim}, we present results for the simulated data. 
Figure~\ref{fig:var-l2-sim} shows the variance of $\ell_2$ loss, and Figure~\ref{fig:mean-l2-sim} presents the error in the estimated $\ell_2$ loss using $\mu_t$ (relative to true $\ell_2$ loss, that is unknown to the algorithm). 
As we can see from these plots, our method one-step lookahead  gives substantial improvements  over active learning baselines and random sampling. In addition,
compared to the one-step lookahead planning approach using \textsf{REINFORCE}-based policy gradients, 
we observe that $\mathsf{Smoothed\text{-}Autodiff}$-based policy gradients provide significantly more robust performance over all horizons.

In Figures~\ref{fig:var-l2-real}~and~\ref{fig:mean-l2-real}, we observe similar findings on the eICU data. We see that planning policies (\textsf{REINFORCE} and $\mathsf{Smoothed\text{-}Autodiff}$) consistently outperform other heuristics by a large margin.  Active learning baselines perform poorly in these small-horizon batched problems and can sometimes be even worse than the random search baselines.  Overall, our results show the importance of careful planning in adaptive labeling for reliable model evaluation. 

We offer some intuition as to why one-step lookahead planning may outperform other heuristic algorithms. 
 First,  \textsf{Uncertainty sampling (Static)} while myopically selects the
 top-$K$ inputs with the highest uncertainty, it fails to consider 
the overlap in information content among the ``best” instances; see \citep{AggarwalKoGuHaPh14} for more details. 
In other words,  it might acquire points from the same region with high uncertainty while failing to induce diversity among the batch.
Although \textsf{Uncertainty Sampling (Sequential)} somewhat addresses the issue of information overlap, a significant drawback of 
this algorithm
is the disconnect between the objective we aim to optimize and the algorithm. For example, it might sample from a region with high uncertainty but very low density. 

\begin{figure}[t]
\centering
\begin{minipage}[b]{0.48\textwidth}
\centering
\includegraphics[width=\textwidth, height=5cm]{figures/original_scale/Var_of_l_2_loss_real.pdf}
\caption{(Real-world eICU data) Variance of mean squared loss evaluated through the posterior belief $\mu_t$ at each horizon $t$. Even 1-step lookaheads are extremely effective planners, and auto-differentiation-based pathwise policy gradients provide a reliable optimization algorithm based on low-variance gradient estimates.}
\label{fig:var-l2-real}
\end{minipage}
\hfill
\begin{minipage}[b]{0.48\textwidth}
\centering \includegraphics[width=\textwidth, height=5cm]{figures/original_scale/Error_of_estimated_model_l_2_loss_real.pdf}
\caption{(Real-world eICU data) Error between MSE calculated based on collected data $\mc{D}^{0:T}$ vs. population oracle MSE over $\mc{D}_{\rm eval} \sim P_X$. Reducing uncertainty over posteriors directly leads to better OOD evaluations. Our method significantly outperforms active learning-based heuristics, and random sampling.}
\label{fig:mean-l2-real}
\end{minipage}
%\caption{Real data for GPs}
\end{figure}
 
%\vspace{-1.5cm}
% \begin{wrapfigure}{r}{.32\columnwidth}
%   \vspace{-.5cm} 
%   \centering
% \includegraphics[scale=.29]{figures/Var of l2l_2 loss.pdf}
%   \vspace{-0.2cm}
%   \caption{Results of GP}
% \label{fig:var-l2-gp}
%   \vspace{-0.1cm}
% \end{wrapfigure}


% Attempts have been made  in the past to address these  drawbacks heuristically  (see \citep{AggarwalKoGuHaPh14}). We give a unified computational framework while approaching the problem in a more principled manner and solving it more optimally.




\subsection{Planning with  neural network-based uncertainty quantification methods ($\ensembleplus$)}


We now provide a proof-of-concept that shows the generalizability of our conceptual framework  to the deep learning-based UQ modules, specifically focusing on $\ensembleplus$ due to their previously observed superior performance~\citep{OsbandWenAsDwIbLuRo23}. Recall that implementing our framework with deep learning-based UQ modules  requires us to retrain the model across multiple possible random actions $\bm{a}(\theta)$ sampled from the current policy $\pi_\theta$.
This requires significant computational resources, in sharp contrast to the GPs where the posteriors are in closed form and can be readily updated and differentiated. 

Due to the computational constraints, we test $\ensembleplus$ on a toy setting to demonstrate the generalizability of our framework. We consider a setting where the general population consists of four clusters, while the initial labeled data only comes from one cluster. Again we generate data using GPs.  The task is to select a batch of 2 points in one horizon. We detail the $\ensembleplus$ architecture in Section \ref{sec:details-experiments}, and we assume prior uncertainty to be large (depends on the scaling of the prior generating functions). 
The results are summarized in the Table~\ref{tab:UQ_ensemble}.

% \begin{table}[H]
% \vspace{-10pt}
% \caption{Performance under \ensembleplus as UQ module}
%     \centering
%     \begin{tabular}{|m{3cm}|m{2.5cm}|m{2cm}|} 
%     \hline
%       Algorithm   & Variance of $\loss_2$ loss estimate & Error of $\loss_2$ loss estimate  \\ \hline Random Sampling 
%          & $1710.9 \pm 1352.1$ & $8.67\pm6.62$ 
%       \\ \hline \ouralgo & $1.30 \pm 0.68$ & $0.91\pm0.25$ \\ \hline
%     \end{tabular}
%     \label{tab:UQ_ensemble}
%     %\vspace{-10pt}
% \end{table}




\begin{table}[h]
\vspace{-10pt}
\caption{Performance under \ensembleplus as the UQ module}
\centering
\begin{tabular}{|l|l|l|}
\hline
Algorithm   & Variance of $\loss_2$ loss estimate & Error of $\loss_2$ loss estimate  \\
\hline
\textsf{Random sampling} & 7129.8 $\pm$ 1027.0 & 136.2 $\pm$ 8.28 \\ \hline
\textsf{Uncertainty sampling (Static)} & 10852 $\pm$ 0.0 & 162.156 $\pm$ 0.0 \\ \hline
\textsf{Uncertainty sampling (Sequential)} & 8585.5 $\pm$ 898.9 & 144 $\pm$ 6.93 \\ \hline
\textsf{REINFORCE} & 1697.1 $\pm$ 0.0 & 45.27 $\pm$ 0.0 \\ \hline
\ouralgo & 1697.1 $\pm$ 0.0 & 45.27 $\pm$ 0.0 \\ \hline
\end{tabular}
%\caption{Comparison of different algorithms based on variance   and   error in $\ell_2$ loss estimation with Ensemble $+$ as the UQ module. Our results demonstrate that {\ouralgo} and REINFORCE outperformthe other active learning based heuristics, confirming the benefits of our MDP formulation for the adaptive labeling problem, as also demonstrated in Section 4.\\
%\footnotesize{Experimental details: We use Gaussian Processes as our data generating process, GP parameters are the same as in Section D.3.  The task is to select a batch of 2 points along one horizon.The marginal distribution $p_X$ has 4 \textit{non-overlapping} clusters. Initial data comes from one cluster, while pool and evaluation points comes from all the clusters. We have $20$ initial labeled data points, $10$ pool points, and $252$ evaluation points.  Training procedures are similar to the one in Section D.3.} }
\label{tab:UQ_ensemble}
\end{table}



% We faced  issues in scaling up these experiments which will be our focus in the future. 





% \begin{itemize}
%     \item Posteriors should be consistent. Two dimensions: even with less training,  
%     \item the inference should be  fast enough
% \end{itemize}


% Potential research directions for uncertainty quantification

% In this section we consider a simple setting We consider a simpler setting and 


% For synthetic dataset generation, we use ...... For real datasets, we use ...... We compare our methodolgy to several baselines ()    This Section is structured as follows:
% \begin{itemize}
%     \item \textbf{GPs, square loss objective} (Section \ref{}): 
%     %the broad aim of the experiments  in this section is to isolate the performance of our methodology without any concerns for the inefficiencies induced due to a mis-specified prior or imperfect posterior inference. To accomplish this we generate synthetic datasets using GPs (detailed later). We use the well specified prior (GPs - with same hyperparameter setting) as our UQ module.   
%      As GPs provide differentaible posterior inference - any errors induced due to imperfect posterior updates are also isolated. We note that under this setting
%      \item In Section\ref{} we demonstrate why our methodology performs better than other baselines - by devising various synthetic experiments ()
%     \item  \textbf{UQ Benchmarking }(Section \ref{}): Before diving into the experiments using $\ensembleplus$ and ENNs,  we showcase our benchmarking experiments in Section \ref{}. We use real datasets We observe that ENNs perform better
%      \item \textbf{Ensemble $+$}, objective: recall, accuracy
%     \item \textbf{ENN}, objective: recall, accuracy
% \end{itemize}




% In Section {}, we test 
% \subsection{Experimental details}

% \begin{itemize}
%     \item UQ methodologies - GPs, ENNs
%     \item Objectives - Recall,  ATE
%     \item Datasets - ATE-synthetic datasets, Recall-synthetic, real datasets
%     \item Baselines - 
%     \begin{itemize}
%         \item Random sampling
%         \item Active learning - Uncertainty based sampling - In regression setting almost all of the 
%         \item Myopic greedy - Greedy Batch based sampling
%         \item Policy Gradient
%     \end{itemize}
    
% \end{itemize}

% \subsection{Experiments}
%     \begin{itemize}
%     \item GPs with square loss
%     \item Benchmarking ENN
%         \item ENNs with ATE
%         \item ENNs with Recall
%     \end{itemize}

% \subsection{Benefits over other algorithms - intuition and experiments}

%Active learning - Myopic greedy / Don't rely on the objective rather some entropy version.


%%% Local Variables:
%%% mode: latex
%%% TeX-master: "main"
%%% End:

This work identifies signal collapse as a critical bottleneck in one-shot neural network pruning. Performance loss in pruned networks is due to \textbf{signal collapse} in addition to the removal of critical parameters. We propose \textbf{REFLOW} (\textbf{Re}storing \textbf{F}low of \textbf{Low}-variance signals), a simple yet effective method that mitigates signal collapse without computationally expensive weight updates. By focusing on signal preservation, REFLOW highlights the importance of mitigating signal collapse in sparse networks and enables magnitude pruning to match or surpass state-of-the-art one-shot pruning methods such as CHITA, CBS, and WF.

REFLOW consistently achieves state-of-the-art accuracy across diverse architectures, restoring ResNeXt-101 from under 4.1\% to 78.9\% top-1 accuracy at 80\% sparsity on ImageNet. Its lightweight design makes it a practical solution for both research and deployment, delivering high-quality sparse models without the overhead of traditional approaches. These findings challenge the traditional emphasis on weight selection strategies and underscore the critical role of signal propagation for achieving high-quality sparse networks in the context of one-shot pruning.



%\section*{Conclusion}
This paper aims to enhance our understanding of the computational complexity of computing various Shapley value variants. We found that for various ML models --- including decision trees, regression tree ensembles, weighted automata, and linear regression --- both local and global interventional and baseline SHAP can be computed in polynomial time under HMM modeled distributions. This extends popular algorithms, such as TreeSHAP, beyond their empirical distributional scope. We also establish strict complexity gaps between the various SHAP variants (baseline, interventional, and conditional) and prove the intractability of computing SHAP for tree ensembles and neural networks in simplified scenarios. Overall, we present SHAP as a versatile framework whose complexity depends on four key factors: \begin{inparaenum}[(i)] \item model type, \item SHAP variant, \item distribution modeling approach, \item and local vs. global explanations\end{inparaenum}. We believe this perspective provides deeper insight into the computational complexity of SHAP, paving the way for future work.




%We believe that our framework provides a more intricate understanding of SHAP computation complexity across different models, distributions, and variants, paving the way for further research.

Our work opens promising directions for future research. First, expanding our computational analysis to other SHAP-related metrics, such as asymmetric SHAP~\citep{frye20} and SAGE~\citep{covert2020understanding}, would be valuable. Additionally, we aim to explore more expressive distribution classes and relaxed assumptions beyond those in Section \ref{sec:tractable} while maintaining tractable SHAP computation. Finally, when exact computation is intractable (Section \ref{sec:intractable}), investigating the approximability of SHAP metrics through approximation and parameterized complexity theory~\citep{downey2012parameterized} is an important direction.

%Our work opens several promising avenues for future research on the computational properties of explainable AI methods, with a particular focus on SHAP. First, it would be interesting to broaden the computational analysis conducted in this work to include other popular SHAP-related metrics in the literature, such as asymmetric SHAP \cite{frye20} and SAGE \cite{covert2020understanding}. Also, in the future, we aim to explore more expressive distribution classes and relaxed distributional assumptions—extending beyond those examined in Section \ref{sec:tractable} —that still yield tractable SHAP computation. Finally, when exact computation proves intractable (Section \ref{sec:intractable}), it is worthwhile to theoretically investigate the question of the approximability of computing the SHAP metrics across various configurations, through the lens of approximation and parametrized complexity theory \cite{arora2009computational}.

%This paper aims to deepen our understanding of the computational complexity involved in obtaining different Shapley value variants. We found that for a variety of ML models, including decision trees, tree ensembles for regression, weighted automata, and linear regression models — computing both local and global interventional and baseline SHAP can be done in polynomial time when distributions are modeled by HMMs. This extends the distributional scope of popular algorithms like TreeSHAP, which is limited to empirical distributions. Additionally, we demonstrate a strict complexity gap between SHAP variants, showing that interventional and baseline SHAP can be strictly easier to compute than conditional SHAP. Despite these positive results, we uncovered intractability for various SHAP variants in neural networks and tree ensembles. Finally, we provided generalized complexity relations across SHAP variants. We believe that our framework offers a deeper understanding of the complexity involved in computing SHAP across various variants, models, distributions, as well as in both local and global computations, laying the groundwork for future research.


\noindent {\textbf {Acknowledgements.}} This paper was prepared for informational purposes in part by the Artificial Intelligence Research group of JPMorgan Chase \& Co and its affiliates (“J.P. Morgan”) and is not a product of the Research Department of J.P. Morgan.  J.P. Morgan makes no representation and warranty whatsoever and disclaims all liability, for the completeness, accuracy or reliability of the information contained herein.  This document is not intended as investment research or investment advice, or a recommendation, offer or solicitation for the purchase or sale of any security, financial instrument, financial product or service, or to be used in any way for evaluating the merits of participating in any transaction, and shall not constitute a solicitation under any jurisdiction or to any person, if such solicitation under such jurisdiction or to such person would be unlawful.
% \noindent {\textbf {Acknowledgements.}} This paper was prepared for informational purposes in part by the Artificial Intelligence Research group of JPMorgan Chase \& Co and its affiliates (“J.P. Morgan”) and is not a product of the Research Department of J.P. Morgan.  J.P. Morgan makes no representation and warranty whatsoever and disclaims all liability, for the completeness, accuracy or reliability of the information contained herein.  This document is not intended as investment research or investment advice, or a recommendation, offer or solicitation for the purchase or sale of any security, financial instrument, financial product or service, or to be used in any way for evaluating the merits of participating in any transaction, and shall not constitute a solicitation under any jurisdiction or to any person, if such solicitation under such jurisdiction or to such person would be unlawful.
\bibliography{references}

% ------
\clearpage
\appendix
This is the supplementary material for the paper titled:
{\em On Creating a Causally Grounded Usable Rating Method for Assessing the
Robustness of Foundation Models Supporting Time Series}, submitted to AAAI 2025.


In this supplementary material, we provide additional details. 
Section~\ref{sec:appendix-algo-details} gives the rating algorithms. Section~\ref{sec:appendix-related-work} provides additional related work on robustness testing of FMs. Section~\ref{sec:appendix-metrics} provides the detailed description of existing evaluation metrics used to rate the FMTS in our experiments.
Section~\ref{sec:appendix-high-res-figs} contains the higher resolution version of the figures presented in the main paper. 
Section~\ref{sec:appendix-experiments} provides additional experimental results. Section~\ref{sec:appendix-user-study} contains additional user study results containing all the hypotheses validated along with statistical test results, and conclusions.
Section~\ref{sec:appendix-study-q} contains the user study form that was circulated to collect responses from the users.
Section~\ref{sec:appendix-source-code} contains source code used to process the datasets downloaded from Yahoo! finance.
Section~\ref{sec:appendix-implementation-details} contains additional system implementation details such as hyperparameters chosen.
Section~\ref{sec:appx-reproduc} contains the reproducibility checklist.

% \biplav{Accurately mention sections with usage of labels.}
% the deta algorithms adapted from \cite{kausik2024rating} to rate FMTS for robustness. We also provide additional experimental results to help better understand our work in detail. 
% The material is organized as follows:
% \tableofcontents




%----
\section{Details of  Rating Algorithms}
\label{sec:appendix-algo-details}

Apart from Algorithm 2, the rest were adapted from \cite{kausik2024rating} to suit the FMTS forecasting setting. Here is how the rating method works.
\begin{enumerate}
    \item Algorithm 1 computes the weighted rejection score (WRS) which was defined in Appendix \ref{sec:appendix-metrics} in the main paper.
    \item Algorithm 2 computes the PIE \% based on Propensity Score Matching (PSM) which was defined in Section \ref{sec:metrics} in the main paper.
    \item Algorithm 3 creates a partial order of systems within each perturbation based on the raw scores computed. It will arrange the systems in ascending order w.r.t the raw score. The final partial order (PO) will be a dictionary of dictionaries. 
    \item Algorithm 4 computes the final ratings for systems within each perturbation based on the PO from previous algorithm. It splits the set of raw score values obtained within each perturbation into `L' parts where `L' is the rating level chosen by the user. Each of the systems is given a rating based on the compartment number in which its raw score belongs. The algorithm will return a dictionary with perturbations as keys and ratings provided to each system within the perturbation as the value.
\end{enumerate}



\SetKwComment{Comment}{/* }{ */}
\begin{algorithm}
\small
	\caption{\emph{WeightedRejectionScore}}
	
	\textbf{Purpose:} is used to calculate the weighted sum of the number of rejections of null-hypothesis for Dataset $d_j$ pertaining to a system $s$, Confidence Intervals (CI) $ci_k$ and Weights $w_k$.
	
	\textbf{Input:}\\
	 $d$, dataset corresponding to a specific perturbation.\\
	 $CI$, confidence intervals (95\%, 70\%, 60\%). \\
      $s$, system for which WRS is being computed.   \\
     $W$, weights corresponding to different CIs (1, 0.8, 0.6).  \\
    \textbf{Output:} \\
     $Z$, Sensitive attribute  \\ 
	$\psi$, weighted rejection score.
	    $ \psi \gets 0$  
		\For {each $ci_i, w_i \in CI, W$} {
                    // $z_m, z_n$ are classes of $Z$

                    
		            \For {each $z_m, z_n \in Z$} { 
		                $t, pval, dof \gets T-Test(z_m, z_n) $\; 
		                $t_{crit} \gets LookUp(ci_i, dof) $\;
		                \eIf{$t_{crit} > t$} {
		                    $\psi \gets \psi + 0 $\;
		                 }
		                    {$\psi \gets \psi + w_i $}
		               
		            }
    }
    \Return $\psi$
\end{algorithm} 


\begin{algorithm}
\small
	\caption{ \emph{ComputePIEScore}}
	
	\textbf{Purpose:}  is used to calculate the Deconfounding Impact Estimation using Propensity Score Matching (PIE).
	
	\textbf{Input:}\\
	$s$, a system belonging to the set of test systems, $S$.\\
	$D$, datasets pertaining to a perturbation (different distributions).\\
        $p$, A perturbation other than $p_0$ \\
        $p_0$, control perturbation (or no perturbation. \\

    \textbf{Output:} \\
	$\psi$, PIE score.
	

    $ \psi \gets 0$
    
    $PIE\_list \gets []$
    \hspace{0.5 in} //  To store the list of PIE \% of all the datasets.

	\For {each $d_j \in D$} {
    	    $ APE\_o \gets  E(R|P=p_m) - E(R|P=p_0) $\; 
    	    $ APE\_m \gets  E(R|do(P=p_m)) - E(R|do(P=p_0)) $\; 
    		$PIE\_list[j] \gets (APE\_m - APE\_o) * 100$\;
        }
        $\psi \gets MAX(PIE\_list)$\;
    \Return $\psi$
\end{algorithm}


\begin{algorithm}
\small
	\caption{ \emph{CreatePartialOrder}}
	
	\textbf{Purpose:} is used to create a partial order based on the computed weighted rejection score / the PIE \%.
		
	\textbf{Input:}\\
	$S$, Set of systems.\\
        $P$, Set of perturbations.\\
	% $D$, datasets pertaining to different dataset groups.\\
	% $CI$, confidence intervals (95\%, 70\%, 60\%).  \\
 %    $W$, weights corresponding to different CIs (1, 0.8, 0.6).  \\
    $F$, Flag that says whether the confounder is present (1) or not (0).  \\
    $D$, $CI$, $W$ (as defined in the previous algorithms). \\
    \textbf{Output:} \\
	$PO$, dictionary with a partial order for each perturbation.

 
	    $ PO \gets \{\}$\; 
            $SD \gets \{\}$;

            
	    \eIf {F == 0}{
		    \For {each $p_i \in P$}{ 
                    \For {each $s_j \in S$} {
		              $\psi \gets WeightedRejectionScore(p_i, s_j, D, CI, W)$\; 
		              $SD[s_j] \gets \psi$\;
 		    }
                $PO[p_i] \gets SORT(SD)$\;
                }
 		    }
 		    {\For {each $p_j \in P$} {
                    \For {each $s_i \in S$}{ 
		        $\psi \gets ComputePIEScore(s_i, p_j, p_0, D, CI, W)$\; 
		          $SD[s_j] \gets \psi$\;
 		    }
                $PO[p_i] \gets SORT(SD)$\;
                }
 		   }
	    \Return $PO$
\end{algorithm}  


\begin{algorithm}
\small
	\caption{\emph{AssignRating}}
	
	\textbf{Purpose:} \emph{AssignRating} assigns a rating to each of the SASs based on the partial order and the number of rating levels, $L$.
		
	\textbf{Input:}\\
	$S$, $D$, $CI$, $W$, $P$ (as defined in the previous algorithms).\\
	$L$, rating levels chosen by the user.\\

    \textbf{Output:} \\
	$R$, dictionary with perturbations as keys and ratings assigned to each system within each perturbation as the values.\\

    $R \gets \{\}$\;
    $PO \gets \text{CreatePartialOrder}(S,D,CI,W,G)$\;

    \For{$p_i \in P$}{
        $\psi \gets [PO[p_i].\text{values()}]$\;

        \eIf{$\text{len}(S) > 1$}{
            $G \gets \text{ArraySplit}(\psi, L)$\;
            \For{$k, i \in PO[p_i], \psi$}{
                \For{$g_j \in G$}{
                    \If{$i \in g_j$}{
                        $SD[k] \gets j$\;
                    }
                }
            }
        }{
            // Case of a single SAS in $S$\\
            \eIf{$\psi == 0$}{
                $SD[k] \gets 1$\;
            }{
                $SD[k] \gets L$\;
            }
        }
        $R[p_i] \gets SD$\;
    }
    \Return{$R$}\;
\end{algorithm}



\clearpage
\section{Additional Related Work}
\label{sec:appendix-related-work}
\subsection{Robustness Testing of Foundation Models}
\cite{zhang2022contrastive} examined group distribution shifts and evaluated FMs on image classification tasks with spurious confounders. In our work, we assess the robustness of FMs within time series forecasting by measuring their performance in the presence of two confounders across various perturbation settings and test dataset distributions.
\cite{zhang2023foundation} used foundation models as a surrogate oracle to measure the robustness of image classification models.  However, their test systems did not include any foundation models. \cite{xiao2024ritfis} introduces a framework, RITFIS, to assess the LLM-based software against natural language input. However, they did not consider any other modalities. \cite{schlarmann2023adversarial} showed that imperceivable attacks on images to change the caption output of multi-modal FMs can lead to broadcasting of fake information to honest users. They only evaluate robustness of OpenFlamingo model under different attacks but does not compare its performance with any other FMs. None of these works assess the effectiveness and usability of their robustness testing methods. We address this gap through a user study. 


% In our experiments, we consider perturbations in the input data  to TS models common inspired by omission errors and adversarial attacks, and dependent on both input modality. 
% \cite{chanda2022omission}
% \clearpage

\section{Evaluation Metrics}
\label{sec:appendix-metrics}
% In this section, we define our evaluation metrics. We evaluate the systems across two dimensions: Forecasting accuracy and Robustness.
In this section, we define our evaluation metrics: forecasting accuracy and robustness.
\subsubsection{Forecasting Accuracy Metrics}
We evaluate the systems' forecasting accuracy using established metrics commonly applied in time-series forecasting tasks \cite{makridakis2022m5}.

\noindent {\bf Symmetric mean absolute percentage error (SMAPE) } is defined as, 
    {\tiny
    \begin{equation} \label{eq:smape}
        SMAPE = \frac{1}{T}\sum_{t=1}^T\frac{|x_t - \hat{x}_t|}{(|x_t| +|\hat{x}_t|)/2}, 
    \end{equation}
    }
where $T= 20$ (i.e., the value of $d$) is the total number of observations in the predicted time series. 
% $x_t$ represents the actual observed values, and $\hat{x}_t$ the predicted values at each time step $t= 1, \ldots, 20$. 
SMAPE scores range from 0 to 2, with lower scores indicating more precise forecasts. 

% \noindent {\bf Mean absolute scaled error (MASE)} quantifies the mean absolute error of the forecasts relative to the mean absolute error of a naive one-step forecast calculated on the training data.
\noindent {\bf Mean absolute scaled error (MASE)} measures the mean absolute error of forecasts relative to that of a naive one-step forecast on the training data.
{\small
\begin{equation}\label{eq:mase}
    MASE=\frac{\frac{1}{T} \sum_{i=t+1}^{t+T}|x_{i} - \hat{x_{i}}|}{\frac{1}{t}\sum_{i=1}^{t}|x_{i} - x_{i-1}|},
\end{equation}
}
where in our case, $t = 80$, and $T = 100$.
Lower MASE values indicate better forecasts. 

\noindent {\bf Sign Accuracy} quantifies the average classification accuracy across all test samples, where a higher accuracy indicates more precise predictions. This metric classifies based on how the predicted forecasts align with the most recent observed values in the input time series.

\noindent {\bf Robustness Metrics}
%\subsubsection{Robustness Metrics}
We adapt WRS metric originally proposed in \cite{kausik2024rating} to answer RQ1. Additionally, we introduce two new metrics: APE and PIE \% (modified versions of ATE \cite{abdia2017propensity} and DIE \% \cite{kausik2024rating}) tailored to answering RQ2 and RQ3.

\noindent{\bf Weighted Rejection Score (WRS):} WRS, introduced in \cite{kausik2024rating}, measures statistical bias. First, Student's t-test \cite{student1908probable} compares max residual distributions $(R^{max}_{t} | Z)$ for different values of the protected attribute $Z$. We measure this between each pair of industries or companies, resulting in $^3C_2 = 3$ comparisons. 
% The computed t-value of each pair is compared with the critical t-value based on the confidence interval (CI) and degrees of freedom (DoF), to either reject or accept the null hypothesis. 
\cite{kausik2024rating} chose different confidence intervals (CI) [95\%, 75\%, 60\%] that have different critical values and if the computed t-value is less than the critical value, the null hypothesis is rejected. WRS is mathematically defined by the following equation:
\vspace{-0.5em}
\begin{equation}
\textbf{Weighted Rejection Score (WRS)} = \sum_{i\in CI} w_i*x_i,
\label{eq:wrs}
\vspace{-0.5em}
\end{equation}
\noindent where, $x_i$ is the variable set based on whether the null hypothesis is accepted (0) or rejected (1). $w_i$ is the weight that is multiplied by $x_i$ based on the CI. For example, if CI is 95\%, $x_1$ is multiplied by 1. The lower the CI, the lower the weight will be. WRS helps us answer RQ1 (see Section~\ref{sec:problem}). 


\clearpage
\section{Figures in a Higher Resolution from Main Paper}
\label{sec:appendix-high-res-figs}
\begin{figure}[!h]
    \centering
    \includegraphics[width=0.4\textwidth]{figs/Causal_General.png}
    \caption{Causal model $\mathcal{M}$ for FMTS. The validity of link `1' depends on the data distribution ($P|Z$), while the validity of the links `2' and `3' are tested in our experiments.}
   \label{fig:causal-model-supp}
   \vspace{-2.2em}
\end{figure}

\begin{figure}[!h]
    \centering
    \includegraphics[width=0.4\textwidth]{figs/causal_variants.png}
    \caption{Variants of the causal diagram in Figure \ref{fig:causal-model} used to answer different research questions (RQs).}
   \label{fig:cms-supp}
   \vspace{-1em}
\end{figure}


\begin{figure*}
 \centering
\includegraphics[width=\textwidth]{figs/FMTS_Workflow.png}
\caption{\textbf{Data to predictions}. Workflow for performing statistical and causal analysis to compute raw scores and assign final ratings to the test systems}
\label{fig:system-workflow-supp}
 \end{figure*}
 
 \begin{figure*}
 \centering
 \includegraphics[width=\textwidth]{figs/rating-workflow-v2.png}
 \caption{\textbf{Predictions to ratings}. Black arrows denote the unperturbed and red arrows indicate the perturbed paths. Dashed lines shows the multi-modal path. The perturbed parts of the plots are highlighted in red.}
 \label{fig:rating-workflow-supp}
 \end{figure*}

\begin{figure*}[!h]
\centering
\includegraphics[width=\textwidth]{figs/all_metrics_scatter_plot_v2.png}
\caption{Studying each metric with respect to impact of company and industry as confounders for all models and all perturbations. Plotted in double logarithmic scale, lower values indicate better robustness. Ratings generated by our method (with $L=3$) are shown on the top of each plot. The complete final order (with ratings) are shown in Table \ref{tab:ratings} in Appendix \ref{sec:appendix-experiments}.
}
\label{fig:confs-supp}
\end{figure*}

\begin{figure*}
    \centering
    \includegraphics[width=\textwidth]{figs/Radar_all_v1.png}
    \caption{Radar plots showing for all systems (a) mean forecasting accuracy with respect to all metrics, (b) forecasting accuracy under P2 (half-valued pertubation) Appendix \ref{sec:appendix-experiments}: Figs. \ref{fig:radar-rob}, 
 \ref{fig:radar-acc} show all perturbations; (c) mean robustness metrics for FMTS and $S_a$, and (d) robustness under P2. Each axis is normalized and inverted if needed so that outer ring implies better performance.}
    \label{fig:radar-agg-supp}
\end{figure*}

\begin{figure*}
    \centering
    \includegraphics[width=\textwidth]{figs/Radarchart_Modality_v2.png}
    \caption{Effect of the modalities for $S_g$ (left) and $S_p$ (right).}
    \label{fig:radar_modality-supp}
\end{figure*}

\begin{figure*}[t]
    \centering
    \includegraphics[width=\textwidth]{figs/Radarchart_Arch_v2.png}
    \caption{Role of architecture in forecasting accuracy and robustness. Performance is averaged across models within each category. See Table~\ref{tab:fms}.}
    \label{fig:radar-arch-supp}
\end{figure*}

\clearpage
\section{Additional Experimental Results}
\label{sec:appendix-experiments}
In this section, Table \ref{tab:ratings} shows the partial order and final order with respect to all the metrics defined in Section \ref{sec:metrics}. Table \ref{tab:h4} shows the forecasting accuracy values for all the systems. Table \ref{tab:cases-values} shows the research questions, and average values for all metrics across systems and perturbations (average values are referred to in the conclusions made for each RQ in Section \ref{sec:expts}). Figure \ref{fig:bar-acc} and \ref{fig:bar-rob} shows bar plots with all the robustness metric values and forecasting accuracy values. Figure \ref{fig:heatmaps} shows the heatmap for all metrics for all the models. 

Table \ref{tab:ratings} shows the partial order and final order with respect to all the metrics defined in Section \ref{sec:metrics}. Table \ref{tab:h4} shows the forecasting accuracy values for all the systems. Table \ref{tab:cases-values} shows the research questions, and average values for all metrics across systems and perturbations (average values are referred to in the conclusions made for each RQ in Section \ref{sec:expts}). Figure \ref{fig:bar-acc} and \ref{fig:bar-rob} shows bar plots with all the robustness metric values and forecasting accuracy values. Figure \ref{fig:heatmaps} shows the heatmap for all metrics for all the models. 






\begin{table*}[!ht]
\centering
{\tiny
    \begin{tabular}{|p{5em}|p{0.2em}|p{45em}|p{25em}|}
    \hline
          {\bf Forecasting Evaluation Dimensions} &
          {\bf P} &    
          {\bf Partial Order} &
          {\bf Complete Order} 
          \\ \hline 
          \multirow{3}{6em}{WRS$_I$$\downarrow$} &
          P0 & 
          \{$S_g$: 4.6, S$_m$: 4.6, $S_r$: 4.6, $S_c$: 5.9, $S_a$: 5.9, $S_p^{ni}$: 5.9, $S_g^{ni}$: 6.9, $S_p$: 6.9, $S_b$: 6.9\}   &
          \{$S_g$: 1, S$_m$: 1, $S_r$: 1, $S_c$: 2, $S_a$: 2, $S_p^{ni}$: 2, $S_g^{ni}$: 3, $S_p$: 3, $S_b$: 3\}
          \\ \cline{2-4}
          &
          P1 & 
          \{$S_a$: 2.6, $S_m$: 4.6, $S_g$: 4.6, $S_g^{ni}$: 4.6, $S_r$: 4.6, $S_p^{ni}$: 5.9, $S_p$: 6.9, $S_c$: 6.9, $S_b$: 6.9\}  &
          \{$S_a$: 1, $S_m$: 1, $S_g$: 1, $S_g^{ni}$: 1, $S_r$: 1, $S_p^{ni}$: 2, $S_p$: 3, $S_c$: 3, $S_b$: 3\}
          \\ \cline{2-4}
          &
          P2 & 
          \{$S_a$: 4.6, $S_g$: 4.6, $S_g^{ni}$: 4.6, $S_p^{ni}$: 4.6, $S_m$: 4.6,  $S_r$: 4.6, $S_c$: 6.9, $S_p$: 6.9, $S_b$: 6.9\}   &
          \{$S_a$: 1, $S_g$: 1, $S_g^{ni}$: 1, $S_p^{ni}$: 1, $S_m$: 1,  $S_r$: 1, $S_c$: 2, $S_p$: 2, $S_b$: 2\} 
          \\ \cline{2-4}
          &
          P3 & 
          \{$S_g$: 4.6, $S_g^{ni}$: 4.6, S$_m$: 4.6, $S_r$: 4.6, $S_c$: 4.6, $S_a$: 5.9, $S_p^{ni}$: 6.9, $S_p$: 6.9, $S_b$: 6.9\}   &
          \{$S_g$: 1, $S_g^{ni}$: 1, S$_m$: 1, $S_r$: 1, $S_c$: 1, $S_a$: 2, $S_p$: 3, $S_p^{ni}$: 3, $S_b$: 3\} 
          \\ \hline
          \multirow{3}{6em}{WRS$_C$$\downarrow$} &
          P0 & 
          \{$S_a$: 2.6, S$_g$: 4.6, $S_g^{ni}$: 4.6, $S_p^{ni}$: 4.6, $S_c$: 5.6, $S_p$: 6.9, $S_m$: 6.9, $S_r$: 6.9, $S_b$: 6.9\}    &
          \{$S_a$: 1, S$_g$: 1, $S_g^{ni}$: 1, $S_p^{ni}$: 1, $S_c$: 2, $S_p$: 3, $S_m$: 3, $S_r$: 3, $S_b$: 3\}
          \\ \cline{2-4}
          &
          P1 & 
          \{$S_a$: 0.6, $S_c$: 4.6, $S_p$: 5.9, $S_p^{ni}$: 5.9, $S_r$: 5.9, $S_g^{ni}$: 5.9, $S_g$: 6.9, $S_m$: 6.9, $S_b$: 6.9\}    &
          \{$S_a$: 1, $S_c$: 1, $S_p$: 2, $S_p^{ni}$: 2, $S_r$: 2, $S_g^{ni}$: 2, $S_g$: 3, $S_m$: 3, $S_b$: 3\}
          \\ \cline{2-4}
          &
          P2 & 
          \{$S_a$: 2.6, $S_c$: 4.6, $S_r$: 4.6, $S_p^{ni}$: 4.6, $S_p$: 5.2, $S_g$: 5.9, $S_g^{ni}$: 5.9, $S_m$: 6.9, $S_b$: 6.9\}   &
          \{$S_a$: 1, $S_c$: 1, $S_r$: 1, $S_p^{ni}$: 1, $S_p$: 2, $S_g$: 2, $S_g^{ni}$: 2, $S_m$: 3, $S_b$: 3\}
          \\ \cline{2-4}
          &
          P3 & 
          \{$S_g$: 4.6, $S_g^{ni}$: 4.6, $S_p^{ni}$: 4.6, S$_p$: 4.6, $S_c$: 4.6, $S_m$: 6.9, $S_a$: 6.9, $S_r$: 6.9, $S_b$: 6.9\}   &
          \{$S_g$: 1, $S_g^{ni}$: 1, $S_p^{ni}$: 1, S$_p$: 1, $S_c$: 1, $S_m$: 2, $S_a$: 2, $S_r$: 2, $S_b$: 2\}
          \\ \hline
          \multirow{3}{6em}{PIE$_I$ \%$\downarrow$} &
          P1 & 
          \{$S_g^{ni}$: 124.50, $S_g$: 600.31, $S_r$: 1041.01, $S_p^{ni}$: 1196, $S_m$: 1426.81, $S_c$: 1441.59, $S_p$: 1765.84, $S_a$: 2720.26, $S_b$: 3283.88\} &
          \{$S_g^{ni}$: 1, $S_g$: 1, $S_r$: 1, $S_p^{ni}$: 2, $S_m$: 2, $S_c$: 2, $S_p$: 2, $S_a$: 3, $S_b$: 3\}
          \\ \cline{2-4}
          &
          P2 & 
          \{$S_c$: 357.72, $S_g^{ni}$: 527.76, $S_g$: 597.54, $S_a$: 902.54, $S_m$: 1326.20, $S_r$: 1463.71, $S_p^{ni}$: 1653.53, $S_b$: 2174.39, $S_p$: 2295.68\}  &
          \{$S_c$: 1, $S_g^{ni}$: 1, $S_g$: 1, $S_a$: 2, $S_m$: 2, $S_r$: 2, $S_p^{ni}$: 3, $S_b$: 3, $S_p$: 3\}  
          \\\cline{2-4}
          &
          P3 & 
          \{$S_g$: 703.94, $S_g^{ni}$: 884.34, $S_c$: 911.53, $S_p^{ni}$: 972.95, $S_a$: 1195.04, $S_m$: 2998.25, $S_p$: 3208.04, $S_r$: 3560.94, $S_b$: 7489.48\}   &
          \{$S_g$: 1, $S_g^{ni}$: 1, $S_c$: 1, $S_p^{ni}$: 2, $S_a$: 2, $S_m$: 2, $S_p$: 3, $S_r$: 3, $S_b$: 3\}   
          \\\hline
          \multirow{3}{4em}{PIE$_C$ \%$\downarrow$} &
          P1 & 
          \{$S_c$: 515.91, $S_g$: 663.75, $S_g^{ni}$: 696.44, $S_a$: 982.38, $S_m$: 1028.48, $S_p^{ni}$: 1101.24, $S_p$: 1474.76, $S_r$: 4756.40, $S_b$: 6916.11\}  &
          \{$S_c$: 1, $S_g$: 1, $S_g^{ni}$: 1, $S_a$: 2, $S_m$: 2, $S_p^{ni}$: 2, $S_p$: 3, $S_r$: 3, $S_b$: 3\}
          \\ \cline{2-4}
          &
          P2 & 
          \{$S_g^{ni}$: 469.16, $S_c$: 576.18, $S_g$: 651.07, $S_m$: 1150.45, $S_a$: 1275.04, $S_p^{ni}$: 2238.21, $S_p$: 3257.35, $S_r$: 4274.38, $S_b$: 9474.61\}   &
          \{$S_g^{ni}$: 1, $S_c$: 1, $S_g$: 1, $S_m$: 2, $S_a$: 2, $S_p^{ni}$: 2, $S_p$: 3, $S_r$: 3, $S_b$: 3\} 
          \\ \cline{2-4}
          &
          P3 & 
          \{$S_g^{ni}$: 436.33, $S_g$: 513.47, $S_c$: 650.20, $S_m$: 866.61, $S_r$: 1305.78, $S_a$: 1716.68, $S_b$: 1846.56, $S_p^{ni}$: 2773.74, $S_p$: 4064.03\}  &
          \{$S_g^{ni}$: 1, $S_g$: 1, $S_c$: 1, $S_m$: 2, $S_r$: 2, $S_a$: 2, $S_b$: 3, $S_p^{ni}$: 3, $S_p$: 3\} 
          \\ \hline
          \multirow{3}{6em}{APE$_I$$\downarrow$} &
          P1 & 
          \{$S_g^{ni}$: 2.50, $S_g$: 11.75, $S_c$: 14.69, $S_p^{ni}$: 17.41, $S_m$: 19.94, $S_p$: 26.50, $S_r$: 48.80, $S_a$: 61.87, $S_b$: 101.31\}   &
          \{$S_g^{ni}$: 1, $S_g$: 1, $S_c$: 1, $S_p^{ni}$: 2, $S_m$: 2, $S_p$: 2, $S_r$: 3, $S_a$: 3, $S_b$: 3\} 
          \\ \cline{2-4}
          &
          P2 & 
          \{$S_g$: 3.72, $S_c$: 4.79, $S_g^{ni}$: 6.06, $S_a$: 11.32, $S_m$: 13.36, $S_p^{ni}$: 18.94, $S_p$: 26.02, $S_r$: 42.91, $S_b$: 101.20\}  &
          \{$S_g$: 1, $S_c$: 1, $S_g^{ni}$: 1, $S_a$: 2, $S_m$: 2, $S_p^{ni}$: 2, $S_p$: 3, $S_r$: 3, $S_b$: 3\} 
          \\ \cline{2-4}
          &
          P3 & 
          \{$S_a$: 7.87, $S_g$: 8.40, $S_g^{ni}$: 9.09, $S_c$: 9.50, $S_p^{ni}$: 16.73, $S_m$: 31.36, $S_r$: 36.59, $S_p$: 37.39, $S_b$: 99.72\} &
          \{$S_a$: 1, $S_g$: 1, $S_g^{ni}$: 1, $S_c$: 2, $S_p^{ni}$: 2,  $S_m$: 2, $S_r$: 3, $S_p$: 3, $S_b$: 3\}
          \\ \hline
          \multirow{3}{6em}{APE$_C$$\downarrow$} &
          P1 & 
          \{$S_b$: 0, $S_c$: 6.31, $S_g^{ni}$: 9.49, $S_g$: 10.41, $S_m$: 15.33, $S_r$: 15.36, $S_p^{ni}$: 15.57, $S_p$: 23.99, $S_a$: 59.80\}  &
          \{$S_b$: 1, $S_c$: 1, $S_g^{ni}$: 1, $S_g$: 2, $S_m$: 2, $S_r$: 2, $S_p^{ni}$: 3, $S_p$: 3, $S_a$: 3\} 
          \\ \cline{2-4}
          &
          P2 & 
          \{$S_b$: 0, $S_g^{ni}$: 5.31, $S_c$: 6.42, $S_g$: 8.69, $S_p$: 10.81, $S_m$: 13.92, $S_r$: 17.61, $S_a$: 21.39, $S_p^{ni}$: 27.63\}   &
          \{$S_b$: 1, $S_g^{ni}$: 1, $S_c$: 1, $S_g$: 2, $S_p$: 2, $S_m$: 2, $S_r$: 3, $S_a$: 3, $S_p^{ni}$: 3\} 
          \\ \cline{2-4}
          &
          P3 & 
          \{$S_b$: 0, $S_g^{ni}$: 6.48, $S_g$: 7.06, $S_a$: 7.42, $S_c$: 8.80, $S_m$: 10.87, $S_r$: 16.63, $S_p^{ni}$: 35.35, $S_p$: 46.50 \}    &
          \{$S_b$: 1, $S_g^{ni}$: 1, $S_g$: 1, $S_a$: 2, $S_c$: 2,$S_m$: 2, $S_r$: 3, $S_p^{ni}$: 3, $S_p$: 3 \} 
          \\ \hline
          \multirow{3}{6em}{SMAPE$\downarrow$} &
          P0 & 
          \{$S_a$: 0.040, $S_c$: 0.043, $S_g$: 0.049, $S_p^{ni}$: 0.079, $S_p$: 0.095, $S_g^{ni}$: 0.095, $S_m$: 0.097, $S_r$: 0.829, $S_b$: 1.276 \} &
          \{$S_a$: 1, $S_c$: 1, $S_g$: 1, $S_p^{ni}$: 2, $S_p$: 2, $S_g^{ni}$: 2, $S_m$: 2, $S_r$: 3, $S_b$: 3 \} 
          \\ \cline{2-4}
          &
          P1 & 
          \{$S_c$: 0.065, $S_g^{ni}$: 0.067, $S_g$: 0.072, $S_a$: 0.084, $S_m$: 0.100, $S_p$: 0.100, $S_p^{ni}$: 0.100, $S_r$: 0.830, $S_b$: 1.276 \} &
          \{$S_c$: 1, $S_g^{ni}$: 1, $S_g$: 1, $S_a$: 2, $S_m$: 2, $S_p$: 2, $S_p^{ni}$: 2, $S_r$: 3, $S_b$: 3 \}
          \\ \cline{2-4}
          &
          P2 & 
          \{$S_g$: 0.051, $S_c$: 0.053, $S_g^{ni}$: 0.060, $S_a$: 0.069, $S_p^{ni}$: 0.095, $S_m$: 0.098, $S_p$: 0.100, $S_r$: 0.830, $S_b$: 1.276 \}   &
          \{$S_g$: 1, $S_c$: 1, $S_g^{ni}$: 1, $S_a$: 2, $S_p^{ni}$: 2, $S_m$: 2, $S_p$: 3, $S_r$: 3, $S_b$: 3 \} 
          \\ \cline{2-4}
          &
          P3 & 
          \{$S_a$: 0.040, $S_c$: 0.043, $S_g$: 0.049, $S_g^{ni}$: 0.056, $S_p^{ni}$: 0.078, $S_p$: 0.092, $S_m$: 0.097, $S_r$: 0.830, $S_b$: 1.276 \}   &
          \{$S_a$: 1, $S_c$: 1, $S_g$: 1, $S_g^{ni}$: 2, $S_p^{ni}$: 2, $S_p$: 2, $S_m$: 3, $S_r$: 3, $S_b$: 3 \} 
          \\ \hline
          \multirow{4}{6em}{MASE$\downarrow$} &
          P0 & 
          \{$S_a$: 3.79, $S_c$: 4.18, $S_g$: 4.64, $S_p^{ni}$: 7.19, $S_p$: 8.91, $S_m$: 9.03, $S_g^{ni}$: 10.37, $S_r$: 86.45, $S_b$: 947.56 \}    &
          \{$S_a$: 1, $S_c$: 1, $S_g$: 1, $S_p^{ni}$: 2, $S_p$: 2, $S_m$: 2, $S_g^{ni}$: 3, $S_r$: 3, $S_b$: 3 \} 
          \\ \cline{2-4}
          &
          P1 & 
          \{$S_c$: 5.40, $S_g^{ni}$: 5.65, $S_g$: 6.13, $S_p^{ni}$: 8.87, $S_p$: 9.19, $S_m$: 9.32, $S_a$: 18.36, $S_r$: 86.99, $S_b$: 947.56 \}   & 
          \{$S_c$: 1, $S_g^{ni}$: 1, $S_g$: 1, $S_p^{ni}$: 2, $S_p$: 2, $S_m$: 2, $S_a$: 3, $S_r$: 3, $S_b$: 3 \} 
          \\ \cline{2-4}
          &
          P2 & 
          \{$S_g$: 4.74, $S_c$: 4.99, $S_g^{ni}$: 5.59, $S_a$: 8.24, $S_p^{ni}$: 8.49, $S_m$: 9.15, $S_p$: 9.32, $S_r$: 86.87, $S_b$: 947.56 \}    &
          \{$S_g$: 1, $S_c$: 1, $S_g^{ni}$: 1, $S_a$: 2, $S_p^{ni}$: 2, $S_m$: 2, $S_p$: 3, $S_r$: 3, $S_b$: 3 \}
          \\ \cline{2-4}
          &
          P3 & 
          \{$S_a$: 3.79, $S_c$: 4.10, $S_g$: 4.64, $S_g^{ni}$: 5.39, $S_p^{ni}$: 7.11, $S_p$: 8.68, $S_m$: 9.03,  $S_r$: 86.65, $S_b$: 947.56 \} &
          \{$S_a$: 1, $S_c$: 1, $S_g$: 1, $S_g^{ni}$: 2, $S_p^{ni}$: 2, $S_p$: 2, $S_m$: 3,  $S_r$: 3, $S_b$: 3 \} 
          \\ \hline
          \multirow{3}{6em}{Sign Accuracy \%$\uparrow$} &
          P0 & 
          \{$S_m$: 40.70, $S_p$: 45.09, $S_p^{ni}$: 47.67, $S_r$: 49.88, $S_g^{ni}$: 50.41, $S_g$: 52.08, $S_c$: 53.75, $S_a$: 60.08, $S_b$: 62.60 \} &  
          \{$S_m$: 1, $S_p$: 1, $S_p^{ni}$: 1, $S_r$: 2, $S_g^{ni}$: 2, $S_g$: 2, $S_c$: 3, $S_a$: 3, $S_b$: 3 \}
          \\ \cline{2-4}
          &
          P1 & 
          \{$S_m$: 41.19, $S_p$: 44.33, $S_p^{ni}$: 46.77, $S_r$: 49.62, $S_g$: 50.53, $S_c$: 52.09, $S_g^{ni}$: 53.93, $S_a$: 57.08, $S_b$: 62.60 \} &
          \{$S_m$: 1, $S_p$: 1, $S_p^{ni}$: 1, $S_r$: 2, $S_g$: 2, $S_c$: 2,  $S_g^{ni}$: 3, $S_a$: 3, $S_b$: 3 \}
          \\ \cline{2-4}
          &
          P2 & 
          \{$S_m$: 41.05, $S_p$: 44.02, $S_p^{ni}$: 47.67, $S_r$: 49.64, $S_g$: 49.75, $S_c$: 50.79, $S_g^{ni}$: 54.43, $S_a$: 57.13, $S_b$: 62.60 \}    &
          \{$S_m$: 1, $S_p$: 1, $S_p^{ni}$: 1, $S_r$: 2, $S_g$: 2, $S_c$: 2, $S_g^{ni}$: 3, $S_a$: 3, $S_b$: 3 \} 
          \\ \cline{2-4}
          &
          P3 & 
          \{$S_m$: 40.72, $S_p$: 44.26, $S_p^{ni}$: 47.50, $S_r$: 49.71, $S_g$: 51.34, $S_c$: 51.35, $S_g^{ni}$: 52.97, $S_a$: 59.98, $S_b$: 62.60 \}  &
          \{$S_m$: 1, $S_p$: 1, $S_p^{ni}$: 1, $S_r$: 2, $S_g$: 2, $S_c$: 2, $S_g^{ni}$: 3, $S_a$: 3, $S_b$: 3 \}
          \\ \hline
    \end{tabular}
    }
    \caption{Final raw scores and ratings based on different metrics computed. Higher ratings indicate higher bias for WRS and PIE \%, higher disruption for APE, greater inaccuracy for MASE and SMAPE, and higher accuracy for Sign Accuracy. For simplicity, we denoted the raw scores for accuracy metrics using just the mean value, but standard deviation was also considered for rating. The chosen rating level, L = 3. Overall, across all the settings, system $S_p$ exhibited statistical bias in 50 \% of cases, confounding bias in 100 \% of cases, and disruptive behavior in 50 \% of the cases based on APE values.}
    \label{tab:ratings}
\end{table*}



\begin{table*}
\centering
    \begin{tabular}{|p{5em}|p{1em}|p{3em}|p{3em}|p{3em}|p{3em}|p{3em}|p{3em}|p{3em}|p{3em}|p{3em}|}
    \hline
          {\bf Metric} &
          {\bf P} &    
          {\bf $S_{g}$} &
          {\bf $S_g^{ni}$} &
          {\bf $S_{p}$} &
          {\bf $S_p^{ni}$} &
          {\bf $S_{m}$} &
          {\bf $S_{c}$} &
          {\bf $S_a$} & 
          {\bf $S_b$} &
          {\bf $S_r$}
          \\ \hline 
          \multirow{6}{4em}{SMAPE$\downarrow$} & 
          P0 &
           0.049 $\pm$ 0.047  &
           0.095 $\pm$ 0.103  &
           0.095 $\pm$ 0.075 &
           0.079 $\pm$ 0.081 &
           0.097 $\pm$ 0.072 &
           0.043 $\pm$ 0.054 &
           \textbf{0.040 $\pm$ 0.037} &
          \multirow{6}{1.5em}{1.276 $\pm$ 0.663} &
          0.829 $\pm$ 0.638
          \\ \cline{2-9}
            \cline{11-11}
          % -------
            & 
          P1 &
          0.072 $\pm$ 0.123 &
          0.067 $\pm$ 0.178 &
          0.100 $\pm$ 0.125 &
          0.100 $\pm$ 0.143 &
          0.100 $\pm$ 0.076 &
          \textbf{0.065 $\pm$ 0.189} &
          0.084 $\pm$ 0.282 &
          &
          0.830 $\pm$ 0.639
          \\ \cline{2-9}
            \cline{11-11}
          % -------
            & 
          P2 &
          \textbf{0.051 $\pm$ 0.047} &
          0.060 $\pm$ 0.085 &
          0.100 $\pm$ 0.088 &
          0.095 $\pm$ 0.097 &
          0.098 $\pm$ 0.074 &
          0.053 $\pm$ 0.092 &
          0.069 $\pm$ 0.217 &
          &
          0.830 $\pm$ 0.639
          \\ \cline{2-9}
            \cline{11-11}
          % -------
            & 
          P3 &
          0.049 $\pm$ 0.045 &
          0.056 $\pm$ 0.052  &
          0.092 $\pm$ 0.074 &
          0.078 $\pm$ 0.078 &
          0.097 $\pm$ 0.071 &
          0.043 $\pm$ 0.048 &
          \textbf{0.040 $\pm$ 0.037} &
           &
          0.830 $\pm$ 0.640
          \\ \hline
          % -------

          \multirow{6}{4em}{MASE$\downarrow$} & 
          P0 &
          4.64 $\pm$ 4.62 &
          10.37 $\pm$ 13.63 &
          8.91 $\pm$ 7.01 &
          7.19 $\pm$ 6.94 &
          9.03 $\pm$ 6.91 &
          4.18 $\pm$ 7.75 &
          \textbf{3.79 $\pm$ 3.59} &
          \multirow{6}{1em}{947.56 $\pm$ 767.65} &
          86.45 $\pm$ 72.72
          \\ \cline{2-9}
            \cline{11-11}
          % -------
             & 
          P1 &
          6.13 $\pm$ 8.31 &
          5.65 $\pm$ 10.23 &
          9.19 $\pm$ 8.61 &
          8.87 $\pm$ 8.94 &
          9.32 $\pm$ 7.39 &
          \textbf{5.40 $\pm$ 12.45}&
          18.36 $\pm$ 168.82 &
          &
          86.99 $\pm$ 73.53 
          \\ \cline{2-9}
            \cline{11-11}
          % -------
             & 
          P2 &
          \textbf{4.74 $\pm$ 4.53} &
          5.59 $\pm$ 8.19 &
          9.32 $\pm$ 7.94 &
          8.49 $\pm$ 7.94 &
          9.15 $\pm$ 7.15 &
          4.99 $\pm$ 9.90 &
          8.24 $\pm$ 48.58 &
          &
          86.87 $\pm$ 73.32 
          \\ \cline{2-9}
            \cline{11-11}
             & 
          P3 &
          4.64 $\pm$ 4.42 &
          5.39 $\pm$ 5.27 &
          8.68 $\pm$ 7.18 &
          7.11 $\pm$ 6.75 &
          9.03 $\pm$ 6.90 &
          4.10 $\pm$ 6.33 &
          \textbf{3.79 $\pm$ 3.57} &
             &
          86.65 $\pm$ 73.11 
          \\ \hline
          % -------
          \multirow{4}{4em}{Sign Accuracy (\%)$\uparrow$}   & 
          P0 &
          52.08 &
          50.41 &
          45.09 &
          47.67 &
          40.70 &
          53.75 &
          \textbf{60.08} &
          \multirow{6}{1em}{62.60} &
          49.88
          \\ \cline{2-9}
            \cline{11-11}
             & 
          P1 &
          50.53 &
          53.93 &
          44.33 &
          46.77 &
          41.19 &
          52.09 &
          \textbf{57.08} &
          &
          49.62
          \\ \cline{2-9}
            \cline{11-11}
             & 
          P2 &
          49.75 &
          54.43 &
          44.02 &
          47.67 &
          41.05 &
          50.79 &
          \textbf{57.13} &
          &
          49.64
          \\ \cline{2-9}
            \cline{11-11}
              & 
          P3 &
          51.34 &
          52.97 &
          44.26 &
          47.50 &
          40.72 &
          51.35 &
          \textbf{59.98} &
           &
          49.71
          \\ \hline
    \end{tabular}
    %}
    \caption{Performance metrics for test systems across different perturbations. SMAPE and MASE scores are reported as mean $\pm$ standard deviation.}
    \label{tab:h4}
\end{table*}

\begin{table*}[ht]
\centering
   {\small
    \begin{tabular}{|p{8em}|p{8em}|p{3em}|p{12em}|p{9em}|p{8em}|}
    \hline
          {\bf Research Question} &    
          {\bf Causal Diagram} &
          {\bf Metrics Used} &
          {\bf Comparison across Systems} &
          {\bf Comparison across Perturbations} &
          {\bf Key Conclusions} \\ \hline 
          % 
          \textbf{RQ1:} Does $Z$ affect $R^{max}_{t}$, even though $Z$ has no effect on $P$? & 
          \begin{minipage}{.05\textwidth}
          \vspace{2.5mm}
          \centering
          \includegraphics[width=25mm, height=13mm]{figs/Causal_H1.png} 
          \end{minipage} &
          WRS &
          \{\textcolor{green}{$S_a$: 3.96}, $S_g$: 5.05, $S_g^{ni}$: 5.21, $S_r$: 5.34, $S_p^{ni}$: 5.38, $S_c$: 5.46, $S_m$: 5.75, \textcolor{red}{$S_p$: 6.28}, $S_b$: 6.9\} &
          \{\textcolor{green}{P2: 5.18}, P1: 5.2, P3: 5.35, \textcolor{red}{P0: 5.46}\} &
          \textbf{\textit{S} with low statistical bias}: $S_a$. 
          \textbf{\textit{S} with high statistical bias}: $S_p$.
          \textbf{\textit{P} that led to more statistical bias}: P0
          \textbf{Analysis with more discrepancy}: Inter-industry
          \\ \hline 
          % -------
          \textbf{RQ2:} Does $Z$ affect the relationship between $P$ and $R^{max}_{t}$ when $Z$ has an effect on $P$? & 
          \begin{minipage}{.05\textwidth}
          \vspace{2.5mm}
          \centering
          \includegraphics[width=25mm, height=13mm]{figs/Causal_H2.png}
          \end{minipage} &
          PIE \% &
          \{\textcolor{green}{$S_g^{ni}$: 523.09}, $S_g$: 621.68, $S_c$: 742.19, $S_a$: 1206.44, $S_m$: 1466.13, $S_p^{ni}$: 1655.94, \textcolor{red}{$S_p$: 2677.62}, $S_r$: 2733.7, $S_b$: 5197.51\} &
          \{\textcolor{green}{P1: 995.19}, P2: 1252.2, \textcolor{red}{P3: 1563.94}\} &
          \textbf{\textit{S} with low confounding bias}: $S_g^{ni}$. 
          \textbf{\textit{S} with high confounding bias}: $S_p$. 
          \textbf{\textit{P} that led to more confounding bias}: P3.  
          \textbf{Confounder that led to more bias}: \textit{Industry}
          \\ \hline 
          % -------
          \textbf{RQ3:} Does $P$ affect $R^{max}_{t}$ when $Z$ may have an effect on $R^{max}_{t}$? & 
          \begin{minipage}{.05\textwidth}
          \vspace{2.5mm}
          \centering
          \includegraphics[width=25mm, height=13mm]{figs/Causal_H3.png}
          \end{minipage} &
          APE &
          \{\textcolor{green}{$S_g^{ni}$: 6.49}, $S_g$: 8.34, $S_c$: 8.42, $S_m$: 17.46, $S_p^{ni}$: 21.94, $S_a$: 28.28, \textcolor{red}{$S_p$: 28.53}, $S_r$: 29.65, $S_b$: 50.37\} &
          \{\textcolor{green}{P2: 12.74}, P3: 17.34, \textcolor{red}{P1: 21.11}\} &
          \textbf{\textit{S} with low APE}: $S_g^{ni}$.
          \textbf{\textit{S} with high APE}: $S_p$.
          \textbf{\textit{P} with low APE}: P2.
          \textbf{\textit{P} with high APE}: P1.
          \textbf{Confounder that led to high APE}: \textit{Company}
          \\ \hline 
          % -------
          \textbf{RQ4:} Does $P$ affect the accuracy of $S$? & 
          This hypothesis does not necessitate a causal model for its evaluation. &
          SMAPE, MASE, Sign Accuracy &
          \textbf{SMAPE}: \{\textcolor{green}{$S_c$: 0.05}, $S_a$: 0.06, $S_g$: 0.06, $S_g^{ni}$: 0.07, $S_p^{ni}$: 0.09, $S_p$: 0.1, \textcolor{red}{$S_m$: 0.1}, $S_r$: 0.83, $S_b$: 1.28\}; 

          
          \textbf{MASE}: \{\textcolor{green}{$S_c$: 4.67}, $S_g$: 5.04, $S_g^{ni}$: 6.75, $S_p^{ni}$: 7.91, $S_a$: 8.54, $S_p$: 9.03, \textcolor{red}{$S_m$: 9.13}, $S_r$: 86.74, $S_b$: 947.56\};  

          
          \textbf{Sign Accuracy}: \{\textcolor{red}{$S_m$: 40.91}, $S_p$: 44.42, $S_p^{ni}$: 47.4, $S_r$: 49.71, $S_g$: 50.93, $S_c$: 51.99, $S_g^{ni}$: 52.94, \textcolor{green}{$S_a$: 58.57}, $S_b$: 62.6\}&
          \textbf{SMAPE}: \{\textcolor{green}{P3: 0.06}, P0: 0.07, P1: 0.08, \textcolor{red}{P2: 0.08}\}; 

          
          \textbf{MASE}: \{\textcolor{green}{P3: 6.11}, P0: 6.87, P2: 7.22, \textcolor{red}{P1: 8.99}\}; 

          
          \textbf{Sign Accuracy}: \{\textcolor{red}{P2: 49.26}, P1: 49.42, P3: 49.73, \textcolor{green}{P0: 49.97}\}&
          \textbf{\textit{S} with good performance}: $S_c$. 
          \textbf{\textit{S} with poor performance}: $S_m$. 
          \textbf{\textit{P} with high impact on performance}: P2.  
          \\ \hline 
    \end{tabular}
    }
    \caption{Summary of the research questions answered in the paper, causal diagram, metrics used in the experiment, average of the metric values compared across different systems, average computed across different perturbations, and the key conclusions drawn from the experiment. \textbf{Overall, multi-modal FMTS demonstrated greater robustness and forecasting accuracy compared to multi-modal FMTS. TS FMTS demonstrated greater robustness and forecasting accuracy compared to GP FMTS}. All the raw scores and ratings are shown in Table \ref{tab:ratings}.}
    \label{tab:cases-values}
\end{table*}


 \begin{figure*}[h]
  \centering
  \begin{subfigure}{\textwidth}
  \centering
  \includegraphics[width=.8\textwidth]{plots/smape.png} 
 \caption{SMAPE}
 \label{fig:bar-smape}
  \end{subfigure}
  \begin{subfigure}{\textwidth}
  \centering
  \includegraphics[width=.8\textwidth]{plots/mase.png}
 \caption{MASE}
 \label{fig:bar-mase}
  \end{subfigure}
  \begin{subfigure}{\textwidth}
  \centering
  \includegraphics[width=.8\textwidth]{plots/SignAcc.png}
  \caption{Sign Accuracy}
  \label{fig:bar-sign-acc}
  \end{subfigure}
  \caption{Bar plots showing the robustness metrics values across different systems and perturbations.}
 \label{fig:bar-acc}
\end{figure*}


 \begin{figure*}[h]
  \centering
  \begin{subfigure}{\textwidth}
  \centering
  \includegraphics[width=.8\textwidth]{plots/wrs.png} 
 \caption{WRS}
 \label{fig:bar-wrs}
  \end{subfigure}
  \begin{subfigure}{\textwidth}
  \centering
  \includegraphics[width=.8\textwidth]{plots/pie.png}
 \caption{PIE \% scores}
 \label{fig:bar-pie}
  \end{subfigure}
  \begin{subfigure}{\textwidth}
  \centering
  \includegraphics[width=.8\textwidth]{plots/ape.png}
  \caption{APE scores}
  \label{fig:bar-ape}
  \end{subfigure}
  \caption{Bar plots showing the robustness metrics values across different systems and perturbations.}
 \label{fig:bar-rob}
\end{figure*}

\begin{figure*}
 \centering
\includegraphics[width=1.1\textwidth]{plots/Radar_Rob.png}
\caption{Radar plots showing robustness metrics for all FMTS and $S_a$ under different perturbations. }
\label{fig:radar-rob}
\end{figure*}

\begin{figure*}[b]
     \centering
     \includegraphics[width=1.1\textwidth]{plots/Radar_Acc.png}
     \caption{Radar plots showing forecasting accuracy metrics for all systems under different perturbations.}
     \label{fig:radar-acc}
\end{figure*}



\begin{figure*}
    \centering
    \includegraphics[width=\linewidth]{figs/heatmap.png}
    \caption{Heatmap for all metrics for all models. Lighter shade indicates better performance.}
    \label{fig:heatmaps}
\end{figure*}

\clearpage
\section{Additional User Study Results}
\label{sec:appendix-user-study}
In this section, we present all the hypotheses, the results from the statistical tests conducted to validate these hypotheses, and the conclusions drawn from the results. 


\begin{table*}[h!]
\centering
\begin{tabular}{|p{2.5em}|p{2.5em}|p{2.5em}|p{2.5em}|c|c|c|c|c|c|c|c|}
\hline
\textbf{Metric} & 
\textbf{Q1} & 
\textbf{Q2} & 
\textbf{Q4} & 
\textbf{Q5} & 
\textbf{Q6} & 
\textbf{Q8} & 
\textbf{Q9} & 
\textbf{Q10} & 
\textbf{Q12} &
\textbf{Q13} &
\textbf{Q14} 
\\ \hline
$\mu$ & 
3.1923 & 
2.8077 & 
2.5385 & 
2.7692 & 
2.9231 & 
2.6923 & 
2.9231 & 
3.2308 & 
2.6538 & 
2.8077 &
3.0769 \\ \hline
$\sigma$ & 
1.2335 & 
1.3570 & 
1.3336 & 
1.1767 & 
1.3834 & 
1.0870 & 
1.2625 & 
1.4507 & 
1.1981 & 
1.3570 & 
1.4676 \\ \hline
t-statistic & 4.9287 & 3.0349 & 2.0588 & 3.3333 & 3.4023 & 3.2476 & 3.7282 & 4.3259 & 2.7828 & 3.0349 & 3.7417 \\ \hline
p-value & 0.0000$^*$ & 0.0028$^*$ & 0.0250$^*$ & 0.0013$^*$ & 0.0011$^*$ & 0.0017$^*$ & 0.0005$^*$ & 0.0001$^*$ & 0.0051$^*$ & 0.0028$^*$ & 0.0005$^*$ \\ \hline
\end{tabular}
\caption{Summary of one sample right-tailed t-test results: Comparison of sample means to the hypothesized mean of 2 with a sample size of 26. The right-tailed p-values indicate whether the sample means are significantly greater the hypothesized mean. $^*$ denotes that mean of responses for all the questions is greater than 2.}
\label{tab:user-study-sanity}
\end{table*}


\begin{table*}[!ht]
\centering
    \begin{tabular}{|p{6cm}|p{2cm}|p{2cm}|p{6cm}|}
    \hline
        \textbf{Hypothesis} & 
        \textbf{Test Performed}  &
        \textbf{Statistics} &
        \textbf{Conclusion}\\
        \hline
         There is a high positive correlation between users' fairness rankings and rankings generated by our rating method.  & 
         Spearman Rank Correlation &
         $\rho = 0.73$ &
         The fairness rankings generated by our rating method aligns well with users' rankings. 
         \\ \hline
         The mean of the responses for Q4 is less than or equal to the mean of the responses for Q6. & 
         Paired t-test &
         t-statistic: -1.18, p-val: 0.12 &
         Users found it easy to interpret the behavior of the systems from rankings compared to graphs and statistics with a confidence interval of 85 \%.
         \\ \hline
         There is a very high positive correlation between users' rankings and rankings generated by our rating method. & 
         Spearman Rank Correlation &
         $\rho$: 0.91 &
         The robustness rankings generated by our rating method aligns very well with users' rankings.
         \\ \hline
         The mean of the responses for Q8 is less than or equal to the mean of the responses for Q10. & 
         Paired t-test &
         t-statistic: -1.89, p-val: 0.03 &
         Users found it easy to interpret the behavior of the systems from rankings compared to graphs and statistics with a confidence interval of 95 \%.
         \\  \hline
         There is a weak positive correlation between users' rankings and rankings generated by our rating method. & 
         Spearman Rank Correlation &
         $\rho$: 0.14 &
         The robustness rankings generated by our rating method weakly aligns with users' rankings.
         \\ \hline
         The mean of the responses for Q12 is less than or equal to the mean of the responses for Q14. & 
         Paired t-test &
         t-statistic: -1.62, p-val: 0.06 &
         Users found it easy to interpret the behavior of the systems from rankings compared to graphs and statistics with a confidence interval of 90 \%.
         \\  \hline
    \end{tabular}
    \caption{Table with the hypotheses evaluated in the user study, statistical tests used to validate the hypotheses, results obtained, and conclusions drawn.}
    \label{tab:user-study-results}
\end{table*}



% \begin{table}[h!]
% {\tiny
%     \centering
%     \begin{tabular}{|m{4.8cm}|m{0.6cm}|m{0.5cm}|m{0.7cm}|}
%         \hline
%         \textbf{Question} & \textbf{$\bar{x}$} & \textbf{$t$} & \textbf{p-value} \\
%         \hline
%         Familiarity with time-series & 3.1818 & 4.1608 & 0.0002* \\
%         \hline
%         Familiarity with financial tasks & 2.5909 & 2.2004 & 0.0196* \\
%         \hline
%         Ease of interpreting behavior through graphs in fairness study & 2.4545 & 1.5554 & 0.0674 \\
%         \hline
%         Rating accuracy in fairness study & 2.7273 & 2.7478 & 0.0060* \\
%         \hline
%         Ease of interpreting behavior through ratings & 2.8636 & 2.8444 & 0.0049* \\
%         \hline
%         Ease of interpreting behavior through graphs in robustness study 1 & 2.5455 & 2.4208 & 0.0123* \\
%         \hline
%         Rating accuracy in robustness study 1 & 2.6818 & 2.6418 & 0.0076* \\
%         \hline
%         Ease of interpreting behavior through ratings & 3.0000 & 3.2404 & 0.0020* \\
%         \hline
%         Ease of interpreting behavior through graphs in robustness study 2 & 2.4545 & 1.8002 & 0.0431* \\
%         \hline
%         Rating accuracy in robustness study 2 & 2.6364 & 2.1877 & 0.0201* \\
%         \hline
%         Ease of interpreting behavior through ratings & 2.9091 & 2.8257 & 0.0051* \\
%         \hline
%     \end{tabular}
%     \caption{Table showing the questions, sample mean, t-statistic and p-value from t-test computed for user responses (on a scale of 1-5) to the study. The hypothesized mean for all questions is 2. Significant p-values (p $<$ 0.05) are marked with an asterisk (*)\kl{Add Q3, Q7, and Q11 as well.}\kl{Move to supplementary}\kl{Hypothesis, correlation value, implications.}\biplav{Update numbers. All are significant now.}}
%     \label{tab:user-study}
% }
% \end{table}


% --- 
% \clearpage
% \section{User Study Questionnaire}
% \label{sec:appendix-study-q}
% Below is the user study form (questionnaire) that was circulated to collect responses for the user study.
% \includepdf[pages=-]{UserStudyForm.pdf}

\clearpage
\section{Source Code for Data Processing}
\label{sec:appendix-source-code}
\begin{lstlisting}
# Convert data from Yahoo! finance to sliding window format.
def sliding_window(data, window_size, company):
    sequences = []
    for i in range(len(data) - window_size):
        seq = data[i:(i + window_size + 1)].tolist()
        sequences.append([company] + seq)
    return pd.DataFrame(sequences)

# Perturbations:
# Drop-to-zero: Every 80th stock price in the numerical data will be turned into zero.
def drop_to_zero(df, col):

  new_df = df.copy()
  new_df.loc[new_df.index % 80 == 0, col] = 0

  return new_df

# Value halved: Every 80th stock price in the numerical data will be halved.
def value_halved(df, col):

  new_df = df.copy()
  new_df.loc[new_df.index % 80 == 0, col] /= 2

  return new_df

# Missing values: Every 80th stock price in the numerical data will be 'NaN'.
def missing_values(df, col):

  new_df = df.copy()
  new_df.loc[new_df.index % 80 == 0, col] = float('nan')

  return new_df

# Code to generate time series line plots.
def plot_ts(input_path, output_path):
    data = pd.read_csv(input_path)

    companies = data.iloc[:, 0]
    time_steps = data.iloc[:, 1:]

    for i, company in enumerate(companies):
        plt.figure(figsize=(12, 6))
        plt.plot(time_steps.columns, time_steps.iloc[i], marker='o')
        plt.title(f'Time Series for {company}', fontsize=19)
        plt.xlabel('Time Steps', fontsize=17)
        plt.ylabel('Values', fontsize=17)
        plt.grid(True)
        x_ticks = time_steps.columns[::5]
        plt.xticks(x_ticks, rotation=45, fontsize=15)
        plt.yticks(fontsize=15)
        plt.tight_layout()
        plt.savefig(os.path.join(output_path, f'sample_{i+1}_time_series.png'))
        plt.close()
\end{lstlisting}


\clearpage
\section{Additional Implementation Details}
\label{sec:appendix-implementation-details}
All Forecasting Model Training Systems (FMTS) were executed on Colab notebooks utilizing the L4 GPU available through Colab Pro, which offers 22.5 GB of GPU RAM. Additional details regarding the models such as the inference times and other architectural details can be found in Section \ref{sec:systems}.

\subsubsection{Hyperparameters set}
\begin{itemize}
    \item \textbf{MOMENT}: head\_dropout: 0.1, weight\_decay: 0, freeze\_encoder: True, freeze\_embedder: True, freeze\_head: False
    \item \textbf{Phi-3}: \_attn\_implementation='eager', max\_new\_tokens: 300, temperature: 0.0, do\_sample: False
    \item \textbf{Gemini}: Temperature: 0. Rest of the parameters were default.
\end{itemize}

        
\section{Reproducibility Checklist}
\label{sec:appx-reproduc}
This paper:

\begin{enumerate}
    \item Includes a conceptual outline and/or pseudocode description of AI methods introduced 


    \textbf{Answer}: Yes (Appendix \ref{sec:appendix-algo-details}).

    \item Clearly delineates statements that are opinions, hypothesis, and speculation from objective facts and results 

    \textbf{Answer}: Yes (Section \ref{sec:expts})

    \item Provides well marked pedagogical references for less-familiar readers to gain background necessary to replicate the paper 

    \textbf{Answer}: Yes (Sections \ref{sec:introduction}, \ref{sec:related-work})
    
    
    Does this paper make theoretical contributions? (yes/no)

    \textbf{Answer}: No

% If yes, please complete the list below.

% All assumptions and restrictions are stated clearly and formally. (yes/partial/no)
% All novel claims are stated formally (e.g., in theorem statements). (yes/partial/no)
% Proofs of all novel claims are included. (yes/partial/no)
% Proof sketches or intuitions are given for complex and/or novel results. (yes/partial/no)
% Appropriate citations to theoretical tools used are given. (yes/partial/no)
% All theoretical claims are demonstrated empirically to hold. (yes/partial/no/NA)
% All experimental code used to eliminate or disprove claims is included. (yes/no/NA)

    \item Does this paper rely on one or more datasets? 

    \textbf{Answer}: Yes (Description in Section \ref{sec:exp_app}

    \item A motivation is given for why the experiments are conducted on the selected datasets 

    \textbf{Answer}: Yes (Sections \ref{sec:introduction}, \ref{sec:related-work})

    \item All novel datasets introduced in this paper are included in a data appendix. 

    \textbf{Answer}: NA (We used an existing dataset from Yahoo! Finance.)

    
    \item All novel datasets introduced in this paper will be made publicly available upon publication of the paper with a license that allows free usage for research purposes. 

    \textbf{Answer}: NA

    \item All datasets drawn from the existing literature (potentially including authors’ own previously published work) are accompanied by appropriate citations. 

    \textbf{Answer}: NA
    
    \item All datasets drawn from the existing literature (potentially including authors’ own previously published work) are publicly available. 

    \textbf{Answer}: Yes (Section \ref{sec:exp_app})

    
    \item All datasets that are not publicly available are described in detail, with explanation why publicly available alternatives are not scientifically satisfying. 

    \textbf{Answer}: NA 

    \item Does this paper include computational experiments? 

    \textbf{Answer}: Yes

    \item Any code required for pre-processing data is included in the appendix.

    \textbf{Answer}: Yes, code required to convert the data downloaded from Yahoo! finance to sliding window, apply perturbations, and generate time series plots are provided in Appendix \ref{sec:appendix-source-code}.
    
    \item All source code required for conducting and analyzing the experiments is included in a code appendix.

    \textbf{Answer}: Yes. You can find the source code and data here: \url{https://anonymous.4open.science/r/rating-fmts-1B30/README.md}

    \item All source code required for conducting and analyzing the experiments will be made publicly available upon publication of the paper with a license that allows free usage for research purposes. 

    \textbf{Answer}: Yes

    \item All source code implementing new methods have comments detailing the implementation, with references to the paper where each step comes from 

    \textbf{Answer}: Yes
    
    \item If an algorithm depends on randomness, then the method used for setting seeds is described in a way sufficient to allow replication of results. 

    \textbf{Answer}: Yes (provided in the source code).

    \item This paper specifies the computing infrastructure used for running experiments (hardware and software), including GPU/CPU models; amount of memory; operating system; names and versions of relevant software libraries and frameworks. 

    \textbf{Answer}: Yes (in Appendix \ref{sec:appendix-implementation-details})

    \item This paper formally describes evaluation metrics used and explains the motivation for choosing these metrics. 

    \textbf{Answer}: Yes (Section \ref{sec:metrics} and Appendix \ref{sec:appendix-metrics})
    
    \item This paper states the number of algorithm runs used to compute each reported result. 

    \textbf{Answer}: No.

    \item Analysis of experiments goes beyond single-dimensional summaries of performance (e.g., average; median) to include measures of variation, confidence, or other distributional information.

    \textbf{Answer}: Yes (Section \ref{sec:expts}, Appendix \ref{sec:appendix-experiments}, Section \ref{sec:userstudy}, and Appendix \ref{sec:appendix-user-study})

    \item The significance of any improvement or decrease in performance is judged using appropriate statistical tests (e.g., Wilcoxon signed-rank). 

    \textbf{Answer}: Yes (Section \ref{sec:expts}, Appendix \ref{sec:appendix-experiments}, Section \ref{sec:userstudy}, and Appendix \ref{sec:appendix-user-study}) 

    
    \item This paper lists all final (hyper-)parameters used for each model/algorithm in the paper’s experiments. (yes/partial/no/NA)

    \textbf{Answer}: Yes (Appendix Appendix \ref{sec:appendix-implementation-details}).

    
    \item This paper states the number and range of values tried per (hyper-) parameter during development of the paper, along with the criterion used for selecting the final parameter setting. (yes/partial/no/NA)

    \textbf{Answer}: NA

\end{enumerate}
\end{document}
