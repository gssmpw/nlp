%File: anonymous-submission-latex-2025.tex
\documentclass[letterpaper]{article} % DO NOT CHANGE THIS
\usepackage{aaai25}  % DO NOT CHANGE THIS
\usepackage{times}  % DO NOT CHANGE THIS
\usepackage{helvet}  % DO NOT CHANGE THIS
\usepackage{courier}  % DO NOT CHANGE THIS
\usepackage[hyphens]{url}  % DO NOT CHANGE THIS
\usepackage{graphicx} % DO NOT CHANGE THIS
\usepackage{tcolorbox}
\urlstyle{rm} % DO NOT CHANGE THIS
\def\UrlFont{\rm}  % DO NOT CHANGE THIS
\usepackage{natbib}  % DO NOT CHANGE THIS AND DO NOT ADD ANY OPTIONS TO IT
\usepackage{caption} % DO NOT CHANGE THIS AND DO NOT ADD ANY OPTIONS TO IT
\frenchspacing  % DO NOT CHANGE THIS
\setlength{\pdfpagewidth}{8.5in} % DO NOT CHANGE THIS
\setlength{\pdfpageheight}{11in} % DO NOT CHANGE THIS
%
% These are recommended to typeset algorithms but not required. See the subsubsection on algorithms. Remove them if you don't have algorithms in your paper.
% \usepackage{algorithm}
% \usepackage{algorithmic}

%
% These are are recommended to typeset listings but not required. See the subsubsection on listing. Remove this block if you don't have listings in your paper.
\usepackage{newfloat}
\usepackage{listings}
% Define colors
\definecolor{headercolor}{gray}{0.85}
\definecolor{boxcolor}{gray}{0.95}

% Setup for fancy headers and footers
% Define custom colors
\definecolor{lightgray}{gray}{0.95}
\definecolor{darkgray}{gray}{0.85}
\definecolor{highlight}{gray}{0.9}
\definecolor{questioncolor}{gray}{0.9} % Light gray for question
\definecolor{answercolor}{gray}{0.95}   % Slightly darker gray for answers

% \definecolor{elicolor}{}{}

\colorlet{Mycolor1}{red!30}
\usepackage{etoolbox}
\AtBeginEnvironment{tcolorbox}{\small}


\DeclareCaptionStyle{ruled}{labelfont=normalfont,labelsep=colon,strut=off} % DO NOT CHANGE THIS
\lstset{%
	basicstyle={\footnotesize\ttfamily},% footnotesize acceptable for monospace
	numbers=left,numberstyle=\footnotesize,xleftmargin=2em,% show line numbers, remove this entire line if you don't want the numbers.
	aboveskip=0pt,belowskip=0pt,%
	showstringspaces=false,tabsize=2,breaklines=true}
% \floatstyle{ruled}
% \newfloat{listing}{tb}{lst}{}
% \floatname{listing}{Listing}
%
% Keep the \pdfinfo as shown here. There's no need
% for you to add the /Title and /Author tags.
\pdfinfo{
/TemplateVersion (2025.1)
}

% DISALLOWED PACKAGES
% \usepackage{authblk} -- This package is specifically forbidden
% \usepackage{balance} -- This package is specifically forbidden
% \usepackage{color (if used in text)
% \usepackage{CJK} -- This package is specifically forbidden
% \usepackage{float} -- This package is specifically forbidden
% \usepackage{flushend} -- This package is specifically forbidden
% \usepackage{fontenc} -- This package is specifically forbidden
% \usepackage{fullpage} -- This package is specifically forbidden
% \usepackage{geometry} -- This package is specifically forbidden
% \usepackage{grffile} -- This package is specifically forbidden
% \usepackage{hyperref} -- This package is specifically forbidden
% \usepackage{navigator} -- This package is specifically forbidden
% (or any other package that embeds links such as navigator or hyperref)
% \indentfirst} -- This package is specifically forbidden
% \layout} -- This package is specifically forbidden
% \multicol} -- This package is specifically forbidden
% \nameref} -- This package is specifically forbidden
% \usepackage{savetrees} -- This package is specifically forbidden
% \usepackage{setspace} -- This package is specifically forbidden
% \usepackage{stfloats} -- This package is specifically forbidden
% \usepackage{tabu} -- This package is specifically forbidden
% \usepackage{titlesec} -- This package is specifically forbidden
% \usepackage{tocbibind} -- This package is specifically forbidden
% \usepackage{ulem} -- This package is specifically forbidden
% \usepackage{wrapfig} -- This package is specifically forbidden
% DISALLOWED COMMANDS
% \nocopyright -- Your paper will not be published if you use this command
% \addtolength -- This command may not be used
% \balance -- This command may not be used
% \baselinestretch -- Your paper will not be published if you use this command
% \clearpage -- No page breaks of any kind may be used for the final version of your paper
% \columnsep -- This command may not be used
% \newpage -- No page breaks of any kind may be used for the final version of your paper
% \pagebreak -- No page breaks of any kind may be used for the final version of your paperr
% \pagestyle -- This command may not be used
% \tiny -- This is not an acceptable font size.
% \vspace{- -- No negative value may be used in proximity of a caption, figure, table, section, subsection, subsubsection, or reference
% \vskip{- -- No negative value may be used to alter spacing above or below a caption, figure, table, section, subsection, subsubsection, or reference

\setcounter{secnumdepth}{2} %May be changed to 1 or 2 if section numbers are desired.


% \title{Rating \underline{F}oundation \underline{M}odels  Supporting \underline{T}ime-\underline{S}eries (FMTS) for Robustness through a Causal Lens}
% -- Alternative
\title{\textbf{On Creating a Causally Grounded Usable Rating Method for Assessing the Robustness of Foundation Models  Supporting Time Series}}
% -- Alternative
%\title{Effectiveness of Ratings in Assessing the Robustness of \underline{F}oundation \underline{M}odels  Supporting \underline{T}ime-\underline{S}eries (FMTS)}

\author {
    % Authors
    Kausik Lakkaraju\textsuperscript{\rm 1}, 
    Rachneet Kaur\textsuperscript{\rm 2},  
    Parisa Zehtabi\textsuperscript{\rm 3}, 
    Sunandita Patra\textsuperscript{\rm 2}, \\
    Siva Likitha Valluru\textsuperscript{\rm 1}, 
    Zhen Zeng\textsuperscript{\rm 2}, 
    Biplav Srivastava\textsuperscript{\rm 1}, 
    Marco Valtorta\textsuperscript{\rm 1}
}
\affiliations {
    \textsuperscript{\rm 1}University of South Carolina, USA\\
    \textsuperscript{\rm 2}J.P. Morgan AI Research, USA\\
    \textsuperscript{\rm 3}J.P. Morgan AI Research, UK
}

% \author{
%     %Authors
%     % All authors must be in the same font size and format.
%     Written by AAAI Press Staff\textsuperscript{\rm 1}\thanks{With help from the AAAI Publications Committee.}\\
%     AAAI Style Contributions by Pater Patel Schneider,
%     Sunil Issar,\\
%     J. Scott Penberthy,
%     George Ferguson,
%     Hans Guesgen,
%     Francisco Cruz\equalcontrib,
%     Marc Pujol-Gonzalez\equalcontrib
% }
% \affiliations{
%     %Afiliations
%     \textsuperscript{\rm 1}Association for the Advancement of Artificial Intelligence\\
%     % If you have multiple authors and multiple affiliations
%     % use superscripts in text and roman font to identify them.
%     % For example,

%     % Sunil Issar\textsuperscript{\rm 2}, 
%     % J. Scott Penberthy\textsuperscript{\rm 3}, 
%     % George Ferguson\textsuperscript{\rm 4},
%     % Hans Guesgen\textsuperscript{\rm 5}
%     % Note that the comma should be placed after the superscript

%     1101 Pennsylvania Ave, NW Suite 300\\
%     Washington, DC 20004 USA\\
%     % email address must be in roman text type, not monospace or sans serif
%     proceedings-questions@aaai.org
% %
% % See more examples next
% }


% REMOVE THIS: bibentry
% This is only needed to show inline citations in the guidelines document. You should not need it and can safely delete it.
\usepackage{bibentry}
% END REMOVE bibentry

% Packages we added.
\usepackage{amsmath}
\usepackage{algorithm2e}
\RestyleAlgo{ruled}
% \usepackage{pdfpages}
\usepackage{subcaption}
\usepackage{array}
\usepackage{multirow}
\usepackage{titling}
% \newcommand{\kl}[1]{{\color{red}~{\em Comment by Kausik: #1}}}
% \newcommand{\biplav}[1]{{\color{blue}~{\em Comment by Biplav: #1}}}
% \definecolor{sunanditacolor}{HTML}{FF5478} % Using hexadecimal
% \newcommand{\sunandita}[1]{{\color{sunanditacolor}~{\em Comment by Sunandita: #1}}}
% \newcommand{\parisa}[1]{{\color{magenta}~{\em Comment by Parisa: #1}}}
% \newcommand{\zhen}[1]{{\color{cyan}~{\em Comment by Zhen: #1}}}
% \newcommand{\rachneet}[1]{{\color{orange}~{\em Comment by Rachneet: #1}}}
% \newcommand{\marco}[1]{{\color{green}~{\em Comment by Marco: #1}}}
% \newcommand{\likitha}[1]{{\color{teal}~{\em Comment by Likitha: #1}}}



% \renewcommand{\kl}[1]{}
%  \renewcommand{\biplav}[1]{}
%  \renewcommand{\sunandita}[1]{}
%  \renewcommand{\parisa}[1]{}
%  \renewcommand{\zhen}[1]{}
%  \renewcommand{\rachneet}[1]{}
% \renewcommand{\likitha}[1]{}

\begin{document}

\maketitle

\begin{abstract}
%The emergence of Foundation Models (FMs) has brought advancements to a range of tasks, including complex ones such as time-series forecasting across various critical sectors such as healthcare and finance. However, like most AI models, FMs are susceptible to noise and perturbations in input data, which can lead to inaccuracies. In the finance sector, these can have significant impact, affecting various stakeholders such as investors, analysts and traders. This paper introduces a novel approach to evaluate the robustness of Time-series Foundation Models (TFMs) using causally grounded experimental setup.  We evaluate seven time-series forecasting models, ranging in architecture, size, and functionality - from general-purpose to time-series-specific across experimental settings that include four perturbation scenarios with real-world relevance and data from six leading stocks in three industries over a year. These models support different modalities, handling diverse data types. Through causal analysis, we aim to assess the reliability and accuracy of the TFMs, thereby supporting developers and end-users in making informed decisions. Additionally, we conduct a user study to determine the effectiveness of our rating approach in communicating the behavior of TFMs to the end-users, highlighting how well our approach practically applies to real-world scenarios. 

Foundation Models (FMs) have improved time series forecasting in various sectors, such as finance, but their vulnerability to input disturbances can hinder their adoption by stakeholders, such as, investors and analysts. To address this, we propose a causally grounded rating framework to study the robustness of Foundational Models for Time Series (FMTS) with respect to input perturbations. 
%\zhen{missing the explicit motivation for using financial stock time series} 
We evaluate our approach on the stock price prediction problem, a well studied problem with easily accessible public data, evaluating six state-of-the-art (some multi-modal) FMTS across six prominent stocks spanning three industries. 
The ratings proposed by our framework effectively 
% \zhen{do we have support for this claim of accuracy? because it aligns with user study?} 
assess the robustness of FMTS and also offer actionable insights 
% \zhen{make sure to summarize actionable insights in experiments overall conclusion} 
for model selection and deployment.
Within the scope of our study, we find that (1) multi-modal FMTS exhibit better robustness and accuracy compared to their uni-modal versions and,
%\zhen{is it that the models just perform equally bad across perturbations? I guess what a useful model would be is something that is accurate and robust at the same time, is there any model like that we would like to highlight and conclude here?}, 
(2) FMTS pre-trained on time series forecasting task exhibit better robustness and forecasting accuracy compared to general-purpose FMTS pre-trained across diverse settings. 
%\zhen{for next token prediction? ``diverse settings" might lead to a misunderstanding that the FMTS is pre-trained for diverse time series tasks. Or just say ``general-purpose FMTS".}.
Further, to validate our framework's usability, we conduct a user study showcasing FMTS prediction errors along with our computed ratings. The study confirmed that our ratings reduced the difficulty for users in comparing the robustness of different systems.
\end{abstract}
\section{Introduction}
\label{sec:introduction}
The business processes of organizations are experiencing ever-increasing complexity due to the large amount of data, high number of users, and high-tech devices involved \cite{martin2021pmopportunitieschallenges, beerepoot2023biggestbpmproblems}. This complexity may cause business processes to deviate from normal control flow due to unforeseen and disruptive anomalies \cite{adams2023proceddsriftdetection}. These control-flow anomalies manifest as unknown, skipped, and wrongly-ordered activities in the traces of event logs monitored from the execution of business processes \cite{ko2023adsystematicreview}. For the sake of clarity, let us consider an illustrative example of such anomalies. Figure \ref{FP_ANOMALIES} shows a so-called event log footprint, which captures the control flow relations of four activities of a hypothetical event log. In particular, this footprint captures the control-flow relations between activities \texttt{a}, \texttt{b}, \texttt{c} and \texttt{d}. These are the causal ($\rightarrow$) relation, concurrent ($\parallel$) relation, and other ($\#$) relations such as exclusivity or non-local dependency \cite{aalst2022pmhandbook}. In addition, on the right are six traces, of which five exhibit skipped, wrongly-ordered and unknown control-flow anomalies. For example, $\langle$\texttt{a b d}$\rangle$ has a skipped activity, which is \texttt{c}. Because of this skipped activity, the control-flow relation \texttt{b}$\,\#\,$\texttt{d} is violated, since \texttt{d} directly follows \texttt{b} in the anomalous trace.
\begin{figure}[!t]
\centering
\includegraphics[width=0.9\columnwidth]{images/FP_ANOMALIES.png}
\caption{An example event log footprint with six traces, of which five exhibit control-flow anomalies.}
\label{FP_ANOMALIES}
\end{figure}

\subsection{Control-flow anomaly detection}
Control-flow anomaly detection techniques aim to characterize the normal control flow from event logs and verify whether these deviations occur in new event logs \cite{ko2023adsystematicreview}. To develop control-flow anomaly detection techniques, \revision{process mining} has seen widespread adoption owing to process discovery and \revision{conformance checking}. On the one hand, process discovery is a set of algorithms that encode control-flow relations as a set of model elements and constraints according to a given modeling formalism \cite{aalst2022pmhandbook}; hereafter, we refer to the Petri net, a widespread modeling formalism. On the other hand, \revision{conformance checking} is an explainable set of algorithms that allows linking any deviations with the reference Petri net and providing the fitness measure, namely a measure of how much the Petri net fits the new event log \cite{aalst2022pmhandbook}. Many control-flow anomaly detection techniques based on \revision{conformance checking} (hereafter, \revision{conformance checking}-based techniques) use the fitness measure to determine whether an event log is anomalous \cite{bezerra2009pmad, bezerra2013adlogspais, myers2018icsadpm, pecchia2020applicationfailuresanalysispm}. 

The scientific literature also includes many \revision{conformance checking}-independent techniques for control-flow anomaly detection that combine specific types of trace encodings with machine/deep learning \cite{ko2023adsystematicreview, tavares2023pmtraceencoding}. Whereas these techniques are very effective, their explainability is challenging due to both the type of trace encoding employed and the machine/deep learning model used \cite{rawal2022trustworthyaiadvances,li2023explainablead}. Hence, in the following, we focus on the shortcomings of \revision{conformance checking}-based techniques to investigate whether it is possible to support the development of competitive control-flow anomaly detection techniques while maintaining the explainable nature of \revision{conformance checking}.
\begin{figure}[!t]
\centering
\includegraphics[width=\columnwidth]{images/HIGH_LEVEL_VIEW.png}
\caption{A high-level view of the proposed framework for combining \revision{process mining}-based feature extraction with dimensionality reduction for control-flow anomaly detection.}
\label{HIGH_LEVEL_VIEW}
\end{figure}

\subsection{Shortcomings of \revision{conformance checking}-based techniques}
Unfortunately, the detection effectiveness of \revision{conformance checking}-based techniques is affected by noisy data and low-quality Petri nets, which may be due to human errors in the modeling process or representational bias of process discovery algorithms \cite{bezerra2013adlogspais, pecchia2020applicationfailuresanalysispm, aalst2016pm}. Specifically, on the one hand, noisy data may introduce infrequent and deceptive control-flow relations that may result in inconsistent fitness measures, whereas, on the other hand, checking event logs against a low-quality Petri net could lead to an unreliable distribution of fitness measures. Nonetheless, such Petri nets can still be used as references to obtain insightful information for \revision{process mining}-based feature extraction, supporting the development of competitive and explainable \revision{conformance checking}-based techniques for control-flow anomaly detection despite the problems above. For example, a few works outline that token-based \revision{conformance checking} can be used for \revision{process mining}-based feature extraction to build tabular data and develop effective \revision{conformance checking}-based techniques for control-flow anomaly detection \cite{singh2022lapmsh, debenedictis2023dtadiiot}. However, to the best of our knowledge, the scientific literature lacks a structured proposal for \revision{process mining}-based feature extraction using the state-of-the-art \revision{conformance checking} variant, namely alignment-based \revision{conformance checking}.

\subsection{Contributions}
We propose a novel \revision{process mining}-based feature extraction approach with alignment-based \revision{conformance checking}. This variant aligns the deviating control flow with a reference Petri net; the resulting alignment can be inspected to extract additional statistics such as the number of times a given activity caused mismatches \cite{aalst2022pmhandbook}. We integrate this approach into a flexible and explainable framework for developing techniques for control-flow anomaly detection. The framework combines \revision{process mining}-based feature extraction and dimensionality reduction to handle high-dimensional feature sets, achieve detection effectiveness, and support explainability. Notably, in addition to our proposed \revision{process mining}-based feature extraction approach, the framework allows employing other approaches, enabling a fair comparison of multiple \revision{conformance checking}-based and \revision{conformance checking}-independent techniques for control-flow anomaly detection. Figure \ref{HIGH_LEVEL_VIEW} shows a high-level view of the framework. Business processes are monitored, and event logs obtained from the database of information systems. Subsequently, \revision{process mining}-based feature extraction is applied to these event logs and tabular data input to dimensionality reduction to identify control-flow anomalies. We apply several \revision{conformance checking}-based and \revision{conformance checking}-independent framework techniques to publicly available datasets, simulated data of a case study from railways, and real-world data of a case study from healthcare. We show that the framework techniques implementing our approach outperform the baseline \revision{conformance checking}-based techniques while maintaining the explainable nature of \revision{conformance checking}.

In summary, the contributions of this paper are as follows.
\begin{itemize}
    \item{
        A novel \revision{process mining}-based feature extraction approach to support the development of competitive and explainable \revision{conformance checking}-based techniques for control-flow anomaly detection.
    }
    \item{
        A flexible and explainable framework for developing techniques for control-flow anomaly detection using \revision{process mining}-based feature extraction and dimensionality reduction.
    }
    \item{
        Application to synthetic and real-world datasets of several \revision{conformance checking}-based and \revision{conformance checking}-independent framework techniques, evaluating their detection effectiveness and explainability.
    }
\end{itemize}

The rest of the paper is organized as follows.
\begin{itemize}
    \item Section \ref{sec:related_work} reviews the existing techniques for control-flow anomaly detection, categorizing them into \revision{conformance checking}-based and \revision{conformance checking}-independent techniques.
    \item Section \ref{sec:abccfe} provides the preliminaries of \revision{process mining} to establish the notation used throughout the paper, and delves into the details of the proposed \revision{process mining}-based feature extraction approach with alignment-based \revision{conformance checking}.
    \item Section \ref{sec:framework} describes the framework for developing \revision{conformance checking}-based and \revision{conformance checking}-independent techniques for control-flow anomaly detection that combine \revision{process mining}-based feature extraction and dimensionality reduction.
    \item Section \ref{sec:evaluation} presents the experiments conducted with multiple framework and baseline techniques using data from publicly available datasets and case studies.
    \item Section \ref{sec:conclusions} draws the conclusions and presents future work.
\end{itemize}

%\section{Related Work}
%\label{sec:related-work}

%\subsection{Background}

%Defect detection is critical to ensure the yield of integrated circuit manufacturing lines and reduce faults. Previous research has primarily focused on wafer map data, which engineers produce by marking faulty chips with different colors based on test results. The specific spatial distribution of defects on a wafer can provide insights into the causes, thereby helping to determine which stage of the manufacturing process is responsible for the issues. Although such research is relatively mature, the continual miniaturization of integrated circuits and the increasing complexity and density of chip components have made chip-level detection more challenging, leading to potential risks\cite{ma2023review}. Consequently, there is a need to combine this approach with magnified imaging of the wafer surface using scanning electron microscopes (SEMs) to detect, classify, and analyze specific microscopic defects, thus helping to identify the particular process steps where defects originate.

%Previously, wafer surface defect classification and detection were primarily conducted by experienced engineers. However, this method relies heavily on the engineers' expertise and involves significant time expenditure and subjectivity, lacking uniform standards. With the ongoing development of artificial intelligence, deep learning methods using multi-layer neural networks to extract and learn target features have proven highly effective for this task\cite{gao2022review}.

%In the task of defect classification, it is typical to use a model structure that initially extracts features through convolutional and pooling layers, followed by classification via fully connected layers. Researchers have recently developed numerous classification model structures tailored to specific problems. These models primarily focus on how to extract defect features effectively. For instance, Chen et al. presented a defect recognition and classification algorithm rooted in PCA and classification SVM\cite{chen2008defect}. Chang et al. utilized SVM, drawing on features like smoothness and texture intricacy, for classifying high-intensity defect images\cite{chang2013hybrid}. The classification of defect images requires the formulation of numerous classifiers tailored for myriad inspection steps and an Abundance of accurately labeled data, making data acquisition challenging. Cheon et al. proposed a single CNN model adept at feature extraction\cite{cheon2019convolutional}. They achieved a granular classification of wafer surface defects by recognizing misclassified images and employing a k-nearest neighbors (k-NN) classifier algorithm to gauge the aggregate squared distance between each image feature vector and its k-neighbors within the same category. However, when applied to new or unseen defects, such models necessitate retraining, incurring computational overheads. Moreover, with escalating CNN complexity, the computational demands surge.

%Segmentation of defects is necessary to locate defect positions and gather information such as the size of defects. Unlike classification networks, segmentation networks often use classic encoder-decoder structures such as UNet\cite{ronneberger2015u} and SegNet\cite{badrinarayanan2017segnet}, which focus on effectively leveraging both local and global feature information. Han Hui et al. proposed integrating a Region Proposal Network (RPN) with a UNet architecture to suggest defect areas before conducting defect segmentation \cite{han2020polycrystalline}. This approach enables the segmentation of various defects in wafers with only a limited set of roughly labeled images, enhancing the efficiency of training and application in environments where detailed annotations are scarce. Subhrajit Nag et al. introduced a new network structure, WaferSegClassNet, which extracts multi-scale local features in the encoder and performs classification and segmentation tasks in the decoder \cite{nag2022wafersegclassnet}. This model represents the first detection system capable of simultaneously classifying and segmenting surface defects on wafers. However, it relies on extensive data training and annotation for high accuracy and reliability. 

%Recently, Vic De Ridder et al. introduced a novel approach for defect segmentation using diffusion models\cite{de2023semi}. This approach treats the instance segmentation task as a denoising process from noise to a filter, utilizing diffusion models to predict and reconstruct instance masks for semiconductor defects. This method achieves high precision and improved defect classification and segmentation detection performance. However, the complex network structure and the computational process of the diffusion model require substantial computational resources. Moreover, the performance of this model heavily relies on high-quality and large amounts of training data. These issues make it less suitable for industrial applications. Additionally, the model has only been applied to detecting and segmenting a single type of defect(bridges) following a specific manufacturing process step, limiting its practical utility in diverse industrial scenarios.

%\subsection{Few-shot Anomaly Detection}
%Traditional anomaly detection techniques typically rely on extensive training data to train models for identifying and locating anomalies. However, these methods often face limitations in rapidly changing production environments and diverse anomaly types. Recent research has started exploring effective anomaly detection using few or zero samples to address these challenges.

%Huang et al. developed the anomaly detection method RegAD, based on image registration technology. This method pre-trains an object-agnostic registration network with various images to establish the normality of unseen objects. It identifies anomalies by aligning image features and has achieved promising results. Despite these advancements, implementing few-shot settings in anomaly detection remains an area ripe for further exploration. Recent studies show that pre-trained vision-language models such as CLIP and MiniGPT can significantly enhance performance in anomaly detection tasks.

%Dong et al. introduced the MaskCLIP framework, which employs masked self-distillation to enhance contrastive language-image pretraining\cite{zhou2022maskclip}. This approach strengthens the visual encoder's learning of local image patches and uses indirect language supervision to enhance semantic understanding. It significantly improves transferability and pretraining outcomes across various visual tasks, although it requires substantial computational resources.
%Jeong et al. crafted the WinCLIP framework by integrating state words and prompt templates to characterize normal and anomalous states more accurately\cite{Jeong_2023_CVPR}. This framework introduces a novel window-based technique for extracting and aggregating multi-scale spatial features, significantly boosting the anomaly detection performance of the pre-trained CLIP model.
%Subsequently, Li et al. have further contributed to the field by creating a new expansive multimodal model named Myriad\cite{li2023myriad}. This model, which incorporates a pre-trained Industrial Anomaly Detection (IAD) model to act as a vision expert, embeds anomaly images as tokens interpretable by the language model, thus providing both detailed descriptions and accurate anomaly detection capabilities.
%Recently, Chen et al. introduced CLIP-AD\cite{chen2023clip}, and Li et al. proposed PromptAD\cite{li2024promptad}, both employing language-guided, tiered dual-path model structures and feature manipulation strategies. These approaches effectively address issues encountered when directly calculating anomaly maps using the CLIP model, such as reversed predictions and highlighting irrelevant areas. Specifically, CLIP-AD optimizes the utilization of multi-layer features, corrects feature misalignment, and enhances model performance through additional linear layer fine-tuning. PromptAD connects normal prompts with anomaly suffixes to form anomaly prompts, enabling contrastive learning in a single-class setting.

%These studies extend the boundaries of traditional anomaly detection techniques and demonstrate how to effectively address rapidly changing and sample-scarce production environments through the synergy of few-shot learning and deep learning models. Building on this foundation, our research further explores wafer surface defect detection based on the CLIP model, especially focusing on achieving efficient and accurate anomaly detection in the highly specialized and variable semiconductor manufacturing process using a minimal amount of labeled data.

\section{Problem Formulation} \label{sec:probdef}

This section formally defines the problem of restoring a given pruned network with only using its original pretrained CNN in a way free of data and fine-tuning.



% Unlike many existing works utilize data for identifying unimportant filters as well as fine-tuning to this end, we cannot evaluate the filter importance by data-dependent values like activation maps (\textit{a.k.a.} channels) as our focus in this paper is not to use any training data. Thus, in our problem setting, we can only exploit the values of filters in the original network, and thereby have to make some changes in the remaining filters of the pruned network so that the network can return the output not too much different from the original one.

% No matter how much we carefully select unimportant filters to be pruned, some kinds of retraining process appears inevitable as done by the most existing works to this end. However, since our focus in this paper is not to use any training data, we cannot evaluate the importance of filters by data-dependent values like activation maps (\textit{a.k.a.} channels). 

% To this end, they not only use a careful criterion (\textit{e.g.}, L1-norm), but also fine-tune the network using the original data.
% Most of filter pruning methods try to select filters to be pruned prudently so that pruned network's output be similar to the original network's. To this end, they prune the unimportant filters and then fine-tune the pruned network with using the train data. 

% How can we restore the the pruned networks without any data? In other words, it implies that we cannot use any data-driven values(i.e., activation maps) and we can only exploit the values of original filters. In that case, the only thing we can do maybe changing the weights of remained filters appropriately not to amplify the difference between pruned and unpruned network's outputs through the information of original filters.

\begin{figure*}[t]
	\centering
    \subfigure[\label{fig:matrix:a}Pruning matrix]{\hspace{6mm}\includegraphics[width=0.35\columnwidth]{./figure/LBYL_figure_2_1.pdf}\hspace{6mm}} 
    \subfigure[\label{fig:matrix:b}Delivery matrix for LBYL]{\hspace{6mm}\includegraphics[width=0.35\columnwidth]{./figure/LBYL_figure_2_2.pdf}\hspace{6mm}}
    \subfigure[\label{fig:matrix:c}Delivery matrix for one-to-one]{\hspace{9mm}\includegraphics[width=0.35\columnwidth]{./figure/LBYL_figure_2_3.pdf}\hspace{9mm}} 
    \caption{Comparison between pruning matrix and delivery matrix, where the $4$-th and $6$-th filters are being pruned among $6$ original filters}
	\label{fig:matrix}
	\vspace{-2mm}
\end{figure*}



\subsection{Filter Pruning in a CNN}
Consider a given CNN to be pruned with $L$ layers, where each $\ell$-th layer starts with a convolution operation on its input channels, which are the output of the previous $(\ell-1)$-th layer $\mathbf{A}^{(\ell-1)}$, with the group of convolution filters $\mathbf{W}^{{(\ell)}}$ and thereby obtain the set of \textit{feature maps} $\mathbf{Z}^{(\ell)}$ as follows:
\begin{equation}
\boldsymbol{\mathbf{Z}}^{(\ell)} = {\mathbf{A}^{(\ell-1)} \circledast {\mathbf{W}}^{(\ell)}},
\nonumber
\end{equation}
where $\circledast$ represents the convolution operation. Then, this convolution process is normally followed by a batch normalization (BN) process and an activation function such as ReLU, and the $\ell$-th layer finally outputs an \textit{activation map} $\mathbf{A}^{(\ell)}$ to be sent to the $(\ell+1)$-th layer through this sequence of procedures as:
\begin{equation}
\mathbf{A}^{(\ell)} = \F(\N(\mathbf{Z}^{(\ell)})),
\nonumber
\end{equation}
where $\F(\cdot)$ is an activation function and $\N(\cdot)$ is a BN procedure.

Note that all of $\mathbf{W}^{(\ell)}$, $\mathbf{Z}^{(\ell)}$, and $\mathbf{A}^{(\ell)}$ are tensors such that: $\mathbf{W}^{(\ell)} \in \mathbb{R}^{m \times n \times k \times k}$ and $\mathbf{Z}^{(\ell)},\mathbf{A}^{(\ell)} \in \mathbb{R}^{m \times w \times h}$, where (1) $m$ is the number of filters, which also equals the number of output activation maps, (2) $n$ is the number of input activation maps resulting from the $(\ell-1)$-th layer, (3) $k \times k$ is the size of each filter, and (4) $w \times h$ is the size of each output channel for the $\ell$-th layer.

\smalltitle{Filter pruning as n-mode product}
When filter pruning is performed at the $\ell$-th layer, all three tensors above are consequently modified to their \textit{damaged} versions, namely $\mathbf{\Tilde{W}}^{(\ell)}$, $\mathbf{\Tilde{Z}}^{(\ell)}$, and $\mathbf{\Tilde{A}}^{(\ell)}$, respectively, in a way that: $\mathbf{\Tilde{W}}^{(\ell)} \in \mathbb{R}^{t \times n \times k \times k}$ and $\mathbf{\Tilde{Z}}^{(\ell)},\mathbf{\Tilde{A}}^{(\ell)} \in \mathbb{R}^{t \times w \times h}$, where $t$ is the number of remaining filters after pruning and therefore $t < m$. Mathematically, the tensor of remaining filters, \textit{i.e.}, $\mathbf{\Tilde{W}}^{(\ell)}$, is obtained by the \textit{$1$-mode product} \cite{DBLP:journals/siamrev/KoldaB09} of the tensor of the original filters $\mathbf{W}^{(\ell)}$ with a \textit{pruning matrix} $\boldsymbol{\S} \in \mathbb{R}^{m \times t}$ (see Figure \ref{fig:matrix:a})
as follows:
\begin{eqnarray}\begin{split}\label{eq:pruning}
\mathbf{\Tilde{W}}^{(\ell)} = {\mathbf{W}}^{(\ell)} \times_{1} {\boldsymbol{\S}}^{T},\text{where }\boldsymbol{\S}_{i,k} = 
  \begin{cases} 
   1~ \text{if } i = i'_k \\
   0~ \text{otherwise}
  \end{cases} \\
  \text{s.t. } i, i'_k \in [1, m] 
  \text{ and } k \in [1, t].
  \end{split}
\end{eqnarray}
  
By Eq. (\ref{eq:pruning}), each $i'_k$-th filter is not pruned and the other $(m-t)$ filters are completely removed from $\mathbf{W}^{(\ell)}$ to be $\mathbf{\Tilde{W}}^{(\ell)}$.

This reduction at the $\ell$-th layer causes another reduction for each filter of the $(\ell+1)$-th layer so that $\mathbf{W}^{(\ell+1)}$ is now modified to $\mathbf{\Tilde{W}}^{(\ell+1)} \in \mathbb{R}^{m' \times t \times k' \times k'}$, where $m'$ is the number of filters of size $k' \times k'$ in the $(\ell+1)$-th layer. Due to this series of information losses, the resulting feature map (\textit{i.e.}, $\mathbf{Z}^{(\ell+1)}$) would severely be damaged to be $\mathbf{\Tilde{Z}}^{(\ell+1)}$ as shown below:
\begin{equation}
{\mathbf{\Tilde{Z}}}^{{(\ell+1)}} = \mathbf{\Tilde{A}}^{(\ell)} \circledast {\mathbf{\Tilde{W}}}^{(\ell+1)}~~~\not\approx~~~\mathbf{Z}^{(\ell+1)}
\label{eq:eq}\nonumber
\end{equation}
The shape of $\mathbf{\Tilde{Z}}^{(\ell+1)}$ remains the same unless we also prune filters for the $(\ell+1)$-th layer. If we do so as well, the loss of information will be accumulated and further propagated to the next layers. Note that $\mathbf{\Tilde{W}}^{(\ell+1)}$ can also be represented by the \textit{$2$-mode product} \cite{DBLP:journals/siamrev/KoldaB09} of $\mathbf{W}^{(\ell+1)}$ with the transpose of the same matrix $\boldsymbol{\S}$ as:
\begin{equation} \label{eq:pruning2}
\mathbf{\Tilde{W}}^{(\ell+1)} = {\mathbf{W}}^{(\ell+1)} \times_{2} {\boldsymbol{\S}^T}
\end{equation}




\subsection{Problem of Restoring a Pruned Network without Data and Fine-Tuning}
As mentioned earlier, our goal is to restore a pruned and thus damaged CNN without using any data and re-training process, which implies the following two facts. First, we have to use a pruning criterion exploiting only the values of filters themselves such as L1-norm. In this sense, this paper does not focus on proposing a sophisticated pruning criterion but intends to recover a network somehow pruned by such a simple criterion. Secondly, since we cannot make appropriate changes in the remaining filters by fine-tuning, we should make the best use of the original network and identify how the information carried by a pruned filter can be delivered to the remaining filters.

% For brevity, we formulate our problem here with respect to a specific layer, say $\ell$, and then it can trivially be generalized for the entire network. 
\smalltitle{Delivery matrix}
In order to represent the information to be delivered to the preserved filters, let us first think of what the pruning matrix $\boldsymbol{\S}$ means. As defined in Eq. (\ref{eq:pruning}) and shown in Figure \ref{fig:matrix:a}, each row is either a zero vector (for filters being pruned) or a one-hot vector (for remaining filters), which is intended only to remove filters without delivering any information. Intuitively, we can transform this pruning matrix into a \textit{delivery matrix} that carries information for filters being pruned by replacing some meaningful values with some of the zero values therein. Once we find such an \textit{ideal} $\boldsymbol{\S^*}$, we can plug it into $\boldsymbol{\S}$ of Eq. (\ref{eq:pruning2}) to deliver missing information propagated from the $\ell$-th layer to the filters at the $(\ell+1)$-th layer, which will hopefully generate an approximation $\mathbf{\hat{Z}}^{(\ell+1)}$ close to the original feature map as follows:
\begin{equation} \label{eq:fmap_approx}
{\mathbf{\hat{Z}}}^{{(\ell+1)}} = {\mathbf{\Tilde{A}}^{(\ell)} \circledast ({\mathbf{W}}^{(\ell+1)} \times_{2} {\boldsymbol{\S^*}^T})}
~~~\approx~~~\mathbf{Z}^{(\ell+1)}
\end{equation}
Thus, using the delivery matrix $\boldsymbol{\mathcal{S^*}}$, the information loss caused by pruning at each layer is recovered at the feature map of the next layer.

\smalltitle{Problem statement}
Given a pretrained CNN, our problem aims to find the best delivery matrix $\boldsymbol{\mathcal{S^*}}$ for each layer without any data and training process such that the following \textit{reconstruction error} is minimized:
\begin{equation}
\sum\limits_{i = 1}^{m'}\|{{\mathbf{Z}}_{i}^{{(\ell+1)}}-{\hat{\mathbf{Z}}}_{i}^{{(\ell+1)}}}\|_1,
\label{eq:goal}
\end{equation}
where ${\mathbf{Z}}_i^{{(\ell+1)}}$ and ${\hat{\mathbf{Z}}}_i^{{(\ell+1)}}$ indicate the $i$-th original feature map and its corresponding approximation, respectively, out of $m'$ filters in the $(\ell+1)$-th layer. Note that what is challenging here is that we cannot obtain the activation maps in $\mathbf{A}^{(\ell)}$ and $\mathbf{\Tilde{A}}^{(\ell)}$ without data as they are data-dependent values.

% = \sum\limits_{i = 1}^{m'}\|{{\mathbf{Z}}_{i}^{{(\ell+1)}}-{\mathbf{\Tilde{A}}^{(\ell)} \circledast ({\mathbf{W}}^{(\ell+1)} \times_{2} {\boldsymbol{\mathcal{S^*}^T}})}}\|_{1}


% Our goal is finding the approximation matrix $\boldsymbol{\mathcal{S}}$ to minimize the reconstruction error between the pruned model and the original model without any data, and effectively deliver missing information for pruned filters using this approximation matrix


% $\testit{s}$,which can be represented as below.

% \begin{equation}
% \boldsymbol{\mathcal{S}} =  \underset{{\boldsymbol{\mathcal{S}}}}{\mathrm{argmin}} \sum\limits_{{i} = 1}^{m_{\ell+1}} \|{{\mathbf{Z}}_{i,:,:}^{{(\ell+1)}}-{\hat{\mathbf{Z}}}_{i,:,:}^{{(\ell+1)}}}\|_{1} 
% \label{eq:eq1}
% \end{equation}



% Let us first recall that the ultimate goal of network pruning is to make the output of a pruned network as close as possible to that of its original network. Unlike many existing pruning methods, our focus is not to use any training data at all for the entire pruning and recovery process, and this implies the following two facts. First, we cannot evaluate the filter importance by data-dependent values like activation values or gradients, but have to use a pruning criterion exploiting only the values of filters themselves such as L1-norm. Furthermore, instead of fine-tuning with data, the only thing we can do for the pruned network is to make appropriate changes in the remaining filters by identifying some relationships between pruned filters and the other preserved ones without any support from data. Based on this intuition, this section mathematically and generally defines the problem of restoring a pruned neural network in a manner free of data and fine-tuning.


% Thus, we make approximation matrix $\testit{s}$ $\in$ $\mathbb{R}^{m_{\ell} \times t_{\ell}}$ with relationship between the pruned filter and preserved filters in $\ell$-th layer and then apply it to the original filters in $(\ell+1)$-th layer to compensate for pruned feature maps $\boldsymbol{\hat{\mathbf{Z}}}^{{(\ell+1)}}$ as shown below.
% (\textit{i.e.}, Let $\hat{\mathbf{W}}^{(\ell+1)}$ be ${\mathbf{W}}^{(\ell+1)}$ $\times_2$ ${{\textit{s}}} $, where $\times_2$ is 2-mode matrix product) 

% \begin{equation}
% \mathbf{Z}^{(\ell+1)} = {\mathbf{A}}^{(\ell)} \circledast {\mathbf{W}}^{(\ell+1)}
% \approx {\hat{\mathbf{A}}^{(\ell)} \circledast ({\mathbf{W}}^{(\ell+1)} \times_{2} {{s}}) = {\hat{\mathbf{Z}}}^{{(\ell+1)}}}
% \label{eq:eq}\nonumber
% \end{equation}




% For a Convolutional Neural Network (CNN) with $L$ layers, we denote $\mathcal{A}{^{(\ell-1)}}$ $\in$ $\mathbb{R}^{n_{\ell -1 } \times h_{\ell -1} \times w_{\ell -1}}$ is activation maps at $\ell-1$-th layer, where $n_{\ell -1}$, $h_{\ell -1}$ and $w_{\ell -1}$ are the number of channels, height and width in activation maps, respectively. and we denote $\mathbf{W}^{{(\ell )}}$ $\in$  $\mathbb{R}^{m_{\ell} \times n_{\ell -1}\times k \times k}$ is covolution filters in $\ell$-th layer,where $m_{\ell}$, $n_{\ell-1}$ and $k$ are the number of filters, number of channels and kernel size, respectively. Trough the convolution operation using activation map $\mathcal{A}{^{(\ell-1)}}$ and convolution filter $\mathbf{W}^{{(\ell)}}$ in $\ell$-th layer, the feature maps $\boldsymbol{\mathbf{Z}}^{{(\ell)}}$ $\in$ $\mathbb{R}^{m_{\ell} \times h_{\ell+1} \times w_{\ell+1}}$ is computed as shown as below.


% \begin{equation}
% \boldsymbol{\mathbf{Z}}^{(\ell)} = {\mathcal{A}^{(\ell-1)} \circledast {\mathbf{W}}^{(\ell)}}
% \label{eq:eq1}\nonumber
% \end{equation}
% where $\circledast$ is convolution operation.

% and the feature maps passed through the BN and ReLU layer are activation maps $\mathcal{A}{^{(\ell)}}$ $\in$ $\mathbb{R}^{m_{{\ell}} \times h_{\ell+1} \times w_{\ell+1}} $ in $\ell$-th layer as shown as below.

% \begin{equation}
% \mathcal{A}^{(\ell)} = \mathcal{F}(\mathbf{Z}^{(\ell)} \circledast {\mathbf{W}}^{(\ell)})
% \label{eq:eq2}\nonumber
% \end{equation}
% where $\mathcal{F}$ is the function that implement batch normalization and non-linear activation(\textit{e.g.}, ReLU).

% \smalltitle{Filter Pruning}
% If the filter pruning is performed in $\ell$-th layer, the shape of original filters $\mathbf{W}^{{(\ell)}}$ $\in$ $\mathbb{R}^{m_{\ell} \times n_{\ell-1}\times k \times k}$ is modified to ${\hat {\mathbf{W}}^{(\ell)}}$ $\in$ $\mathbb{R}^{t_{\ell} \times n_{\ell-1}\times k \times k}$, where $t_{\ell}$ $<$ $m_{\ell}$ by pruning criterion. Therefore, the pruned activation maps ${\hat {\mathcal{A}}}{^{({\ell+1})}}$ $\in$ $\mathbb{R}^{t_{{\ell}} \times h_{{\ell+2}} \times w_{{\ell+2}}}$ in (${\ell+1}$)-th layer is computed as below.

% \begin{equation}
% \mathbf{\hat{A}}^{(l+1)} = \mathcal{F}({\mathbf{A}^{(\ell)} \circledast {\mathbf{\hat{W}}}^{(\ell+1)}})
% \label{eq:eq3}\nonumber
% \end{equation}

% Moreover, corresponding channels of each filters in ($\ell +1$)-th layer are sequentially removed. As a result, shape of original filters $\mathbf{W}^{{(\ell+1)}}$ $\in$ $\mathbb{R}^{m_{\ell+1} \times m_{\ell}\times k \times k}$ in ($\ell+1$)-th layer is changed to  ${\hat {\mathbf{W}}^{(\ell+1)}}$ $\in$ $\mathbb{R}^{m_{\ell+1} \times t_{\ell}\times k \times k}$. Although feature maps ${\hat{\mathbf{Z}}}^{{(\ell+1)}}$ $\in$ $\mathbb{R}^{m_{\ell+1} \times h_{\ell+2} \times w_{\ell+2}}$ in ($\ell+1$)-th layer after pruning have same shape with original feature maps ${\mathbf{Z}}^{{(\ell+1)}}$ $\in$ $\mathbb{R}^{m_{\ell+1} \times h_{\ell+2} \times w_{\ell+2}}$, the pruned feature maps $\boldsymbol{\hat{\mathbf{Z}}}^{{(\ell+1)}}$ are damaged.
\section{Method}\label{sec:method}
\begin{figure}
    \centering
    \includegraphics[width=0.85\textwidth]{imgs/heatmap_acc.pdf}
    \caption{\textbf{Visualization of the proposed periodic Bayesian flow with mean parameter $\mu$ and accumulated accuracy parameter $c$ which corresponds to the entropy/uncertainty}. For $x = 0.3, \beta(1) = 1000$ and $\alpha_i$ defined in \cref{appd:bfn_cir}, this figure plots three colored stochastic parameter trajectories for receiver mean parameter $m$ and accumulated accuracy parameter $c$, superimposed on a log-scale heatmap of the Bayesian flow distribution $p_F(m|x,\senderacc)$ and $p_F(c|x,\senderacc)$. Note the \emph{non-monotonicity} and \emph{non-additive} property of $c$ which could inform the network the entropy of the mean parameter $m$ as a condition and the \emph{periodicity} of $m$. %\jj{Shrink the figures to save space}\hanlin{Do we need to make this figure one-column?}
    }
    \label{fig:vmbf_vis}
    \vskip -0.1in
\end{figure}
% \begin{wrapfigure}{r}{0.5\textwidth}
%     \centering
%     \includegraphics[width=0.49\textwidth]{imgs/heatmap_acc.pdf}
%     \caption{\textbf{Visualization of hyper-torus Bayesian flow based on von Mises Distribution}. For $x = 0.3, \beta(1) = 1000$ and $\alpha_i$ defined in \cref{appd:bfn_cir}, this figure plots three colored stochastic parameter trajectories for receiver mean parameter $m$ and accumulated accuracy parameter $c$, superimposed on a log-scale heatmap of the Bayesian flow distribution $p_F(m|x,\senderacc)$ and $p_F(c|x,\senderacc)$. Note the \emph{non-monotonicity} and \emph{non-additive} property of $c$. \jj{Shrink the figures to save space}}
%     \label{fig:vmbf_vis}
%     \vspace{-30pt}
% \end{wrapfigure}


In this section, we explain the detailed design of CrysBFN tackling theoretical and practical challenges. First, we describe how to derive our new formulation of Bayesian Flow Networks over hyper-torus $\mathbb{T}^{D}$ from scratch. Next, we illustrate the two key differences between \modelname and the original form of BFN: $1)$ a meticulously designed novel base distribution with different Bayesian update rules; and $2)$ different properties over the accuracy scheduling resulted from the periodicity and the new Bayesian update rules. Then, we present in detail the overall framework of \modelname over each manifold of the crystal space (\textit{i.e.} fractional coordinates, lattice vectors, atom types) respecting \textit{periodic E(3) invariance}. 

% In this section, we first demonstrate how to build Bayesian flow on hyper-torus $\mathbb{T}^{D}$ by overcoming theoretical and practical problems to provide a low-noise parameter-space approach to fractional atom coordinate generation. Next, we present how \modelname models each manifold of crystal space respecting \textit{periodic E(3) invariance}. 

\subsection{Periodic Bayesian Flow on Hyper-torus \texorpdfstring{$\mathbb{T}^{D}$}{}} 
For generative modeling of fractional coordinates in crystal, we first construct a periodic Bayesian flow on \texorpdfstring{$\mathbb{T}^{D}$}{} by designing every component of the totally new Bayesian update process which we demonstrate to be distinct from the original Bayesian flow (please see \cref{fig:non_add}). 
 %:) 
 
 The fractional atom coordinate system \citep{jiao2023crystal} inherently distributes over a hyper-torus support $\mathbb{T}^{3\times N}$. Hence, the normal distribution support on $\R$ used in the original \citep{bfn} is not suitable for this scenario. 
% The key problem of generative modeling for crystal is the periodicity of Cartesian atom coordinates $\vX$ requiring:
% \begin{equation}\label{eq:periodcity}
% p(\vA,\vL,\vX)=p(\vA,\vL,\vX+\vec{LK}),\text{where}~\vec{K}=\vec{k}\vec{1}_{1\times N},\forall\vec{k}\in\mathbb{Z}^{3\times1}
% \end{equation}
% However, there does not exist such a distribution supporting on $\R$ to model such property because the integration of such distribution over $\R$ will not be finite and equal to 1. Therefore, the normal distribution used in \citet{bfn} can not meet this condition.

To tackle this problem, the circular distribution~\citep{mardia2009directional} over the finite interval $[-\pi,\pi)$ is a natural choice as the base distribution for deriving the BFN on $\mathbb{T}^D$. 
% one natural choice is to 
% we would like to consider the circular distribution over the finite interval as the base 
% we find that circular distributions \citep{mardia2009directional} defined on a finite interval with lengths of $2\pi$ can be used as the instantiation of input distribution for the BFN on $\mathbb{T}^D$.
Specifically, circular distributions enjoy desirable periodic properties: $1)$ the integration over any interval length of $2\pi$ equals 1; $2)$ the probability distribution function is periodic with period $2\pi$.  Sharing the same intrinsic with fractional coordinates, such periodic property of circular distribution makes it suitable for the instantiation of BFN's input distribution, in parameterizing the belief towards ground truth $\x$ on $\mathbb{T}^D$. 
% \yuxuan{this is very complicated from my perspective.} \hanlin{But this property is exactly beautiful and perfectly fit into the BFN.}

\textbf{von Mises Distribution and its Bayesian Update} We choose von Mises distribution \citep{mardia2009directional} from various circular distributions as the form of input distribution, based on the appealing conjugacy property required in the derivation of the BFN framework.
% to leverage the Bayesian conjugacy property of von Mises distribution which is required by the BFN framework. 
That is, the posterior of a von Mises distribution parameterized likelihood is still in the family of von Mises distributions. The probability density function of von Mises distribution with mean direction parameter $m$ and concentration parameter $c$ (describing the entropy/uncertainty of $m$) is defined as: 
\begin{equation}
f(x|m,c)=vM(x|m,c)=\frac{\exp(c\cos(x-m))}{2\pi I_0(c)}
\end{equation}
where $I_0(c)$ is zeroth order modified Bessel function of the first kind as the normalizing constant. Given the last univariate belief parameterized by von Mises distribution with parameter $\theta_{i-1}=\{m_{i-1},\ c_{i-1}\}$ and the sample $y$ from sender distribution with unknown data sample $x$ and known accuracy $\alpha$ describing the entropy/uncertainty of $y$,  Bayesian update for the receiver is deducted as:
\begin{equation}
 h(\{m_{i-1},c_{i-1}\},y,\alpha)=\{m_i,c_i \}, \text{where}
\end{equation}
\begin{equation}\label{eq:h_m}
m_i=\text{atan2}(\alpha\sin y+c_{i-1}\sin m_{i-1}, {\alpha\cos y+c_{i-1}\cos m_{i-1}})
\end{equation}
\begin{equation}\label{eq:h_c}
c_i =\sqrt{\alpha^2+c_{i-1}^2+2\alpha c_{i-1}\cos(y-m_{i-1})}
\end{equation}
The proof of the above equations can be found in \cref{apdx:bayesian_update_function}. The atan2 function refers to  2-argument arctangent. Independently conducting  Bayesian update for each dimension, we can obtain the Bayesian update distribution by marginalizing $\y$:
\begin{equation}
p_U(\vtheta'|\vtheta,\bold{x};\alpha)=\mathbb{E}_{p_S(\bold{y}|\bold{x};\alpha)}\delta(\vtheta'-h(\vtheta,\bold{y},\alpha))=\mathbb{E}_{vM(\bold{y}|\bold{x},\alpha)}\delta(\vtheta'-h(\vtheta,\bold{y},\alpha))
\end{equation} 
\begin{figure}
    \centering
    \vskip -0.15in
    \includegraphics[width=0.95\linewidth]{imgs/non_add.pdf}
    \caption{An intuitive illustration of non-additive accuracy Bayesian update on the torus. The lengths of arrows represent the uncertainty/entropy of the belief (\emph{e.g.}~$1/\sigma^2$ for Gaussian and $c$ for von Mises). The directions of the arrows represent the believed location (\emph{e.g.}~ $\mu$ for Gaussian and $m$ for von Mises).}
    \label{fig:non_add}
    \vskip -0.15in
\end{figure}
\textbf{Non-additive Accuracy} 
The additive accuracy is a nice property held with the Gaussian-formed sender distribution of the original BFN expressed as:
\begin{align}
\label{eq:standard_id}
    \update(\parsn{}'' \mid \parsn{}, \x; \alpha_a+\alpha_b) = \E_{\update(\parsn{}' \mid \parsn{}, \x; \alpha_a)} \update(\parsn{}'' \mid \parsn{}', \x; \alpha_b)
\end{align}
Such property is mainly derived based on the standard identity of Gaussian variable:
\begin{equation}
X \sim \mathcal{N}\left(\mu_X, \sigma_X^2\right), Y \sim \mathcal{N}\left(\mu_Y, \sigma_Y^2\right) \Longrightarrow X+Y \sim \mathcal{N}\left(\mu_X+\mu_Y, \sigma_X^2+\sigma_Y^2\right)
\end{equation}
The additive accuracy property makes it feasible to derive the Bayesian flow distribution $
p_F(\boldsymbol{\theta} \mid \mathbf{x} ; i)=p_U\left(\boldsymbol{\theta} \mid \boldsymbol{\theta}_0, \mathbf{x}, \sum_{k=1}^{i} \alpha_i \right)
$ for the simulation-free training of \cref{eq:loss_n}.
It should be noted that the standard identity in \cref{eq:standard_id} does not hold in the von Mises distribution. Hence there exists an important difference between the original Bayesian flow defined on Euclidean space and the Bayesian flow of circular data on $\mathbb{T}^D$ based on von Mises distribution. With prior $\btheta = \{\bold{0},\bold{0}\}$, we could formally represent the non-additive accuracy issue as:
% The additive accuracy property implies the fact that the "confidence" for the data sample after observing a series of the noisy samples with accuracy ${\alpha_1, \cdots, \alpha_i}$ could be  as the accuracy sum  which could be  
% Here we 
% Here we emphasize the specific property of BFN based on von Mises distribution.
% Note that 
% \begin{equation}
% \update(\parsn'' \mid \parsn, \x; \alpha_a+\alpha_b) \ne \E_{\update(\parsn' \mid \parsn, \x; \alpha_a)} \update(\parsn'' \mid \parsn', \x; \alpha_b)
% \end{equation}
% \oyyw{please check whether the below equation is better}
% \yuxuan{I fill somehow confusing on what is the update distribution with $\alpha$. }
% \begin{equation}
% \update(\parsn{}'' \mid \parsn{}, \x; \alpha_a+\alpha_b) \ne \E_{\update(\parsn{}' \mid \parsn{}, \x; \alpha_a)} \update(\parsn{}'' \mid \parsn{}', \x; \alpha_b)
% \end{equation}
% We give an intuitive visualization of such difference in \cref{fig:non_add}. The untenability of this property can materialize by considering the following case: with prior $\btheta = \{\bold{0},\bold{0}\}$, check the two-step Bayesian update distribution with $\alpha_a,\alpha_b$ and one-step Bayesian update with $\alpha=\alpha_a+\alpha_b$:
\begin{align}
\label{eq:nonadd}
     &\update(c'' \mid \parsn, \x; \alpha_a+\alpha_b)  = \delta(c-\alpha_a-\alpha_b)
     \ne  \mathbb{E}_{p_U(\parsn' \mid \parsn, \x; \alpha_a)}\update(c'' \mid \parsn', \x; \alpha_b) \nonumber \\&= \mathbb{E}_{vM(\bold{y}_b|\bold{x},\alpha_a)}\mathbb{E}_{vM(\bold{y}_a|\bold{x},\alpha_b)}\delta(c-||[\alpha_a \cos\y_a+\alpha_b\cos \y_b,\alpha_a \sin\y_a+\alpha_b\sin \y_b]^T||_2)
\end{align}
A more intuitive visualization could be found in \cref{fig:non_add}. This fundamental difference between periodic Bayesian flow and that of \citet{bfn} presents both theoretical and practical challenges, which we will explain and address in the following contents.

% This makes constructing Bayesian flow based on von Mises distribution intrinsically different from previous Bayesian flows (\citet{bfn}).

% Thus, we must reformulate the framework of Bayesian flow networks  accordingly. % and do necessary reformulations of BFN. 

% \yuxuan{overall I feel this part is complicated by using the language of update distribution. I would like to suggest simply use bayesian update, to provide intuitive explantion.}\hanlin{See the illustration in \cref{fig:non_add}}

% That introduces a cascade of problems, and we investigate the following issues: $(1)$ Accuracies between sender and receiver are not synchronized and need to be differentiated. $(2)$ There is no tractable Bayesian flow distribution for a one-step sample conditioned on a given time step $i$, and naively simulating the Bayesian flow results in computational overhead. $(3)$ It is difficult to control the entropy of the Bayesian flow. $(4)$ Accuracy is no longer a function of $t$ and becomes a distribution conditioned on $t$, which can be different across dimensions.
%\jj{Edited till here}

\textbf{Entropy Conditioning} As a common practice in generative models~\citep{ddpm,flowmatching,bfn}, timestep $t$ is widely used to distinguish among generation states by feeding the timestep information into the networks. However, this paper shows that for periodic Bayesian flow, the accumulated accuracy $\vc_i$ is more effective than time-based conditioning by informing the network about the entropy and certainty of the states $\parsnt{i}$. This stems from the intrinsic non-additive accuracy which makes the receiver's accumulated accuracy $c$ not bijective function of $t$, but a distribution conditioned on accumulated accuracies $\vc_i$ instead. Therefore, the entropy parameter $\vc$ is taken logarithm and fed into the network to describe the entropy of the input corrupted structure. We verify this consideration in \cref{sec:exp_ablation}. 
% \yuxuan{implement variant. traditionally, the timestep is widely used to distinguish the different states by putting the timestep embedding into the networks. citation of FM, diffusion, BFN. However, we find that conditioned on time in periodic flow could not provide extra benefits. To further boost the performance, we introduce a simple yet effective modification term entropy conditional. This is based on that the accumulated accuracy which represents the current uncertainty or entropy could be a better indicator to distinguish different states. + Describe how you do this. }



\textbf{Reformulations of BFN}. Recall the original update function with Gaussian sender distribution, after receiving noisy samples $\y_1,\y_2,\dots,\y_i$ with accuracies $\senderacc$, the accumulated accuracies of the receiver side could be analytically obtained by the additive property and it is consistent with the sender side.
% Since observing sample $\y$ with $\alpha_i$ can not result in exact accuracy increment $\alpha_i$ for receiver, the accuracies between sender and receiver are not synchronized which need to be differentiated. 
However, as previously mentioned, this does not apply to periodic Bayesian flow, and some of the notations in original BFN~\citep{bfn} need to be adjusted accordingly. We maintain the notations of sender side's one-step accuracy $\alpha$ and added accuracy $\beta$, and alter the notation of receiver's accuracy parameter as $c$, which is needed to be simulated by cascade of Bayesian updates. We emphasize that the receiver's accumulated accuracy $c$ is no longer a function of $t$ (differently from the Gaussian case), and it becomes a distribution conditioned on received accuracies $\senderacc$ from the sender. Therefore, we represent the Bayesian flow distribution of von Mises distribution as $p_F(\btheta|\x;\alpha_1,\alpha_2,\dots,\alpha_i)$. And the original simulation-free training with Bayesian flow distribution is no longer applicable in this scenario.
% Different from previous BFNs where the accumulated accuracy $\rho$ is not explicitly modeled, the accumulated accuracy parameter $c$ (visualized in \cref{fig:vmbf_vis}) needs to be explicitly modeled by feeding it to the network to avoid information loss.
% the randomaccuracy parameter $c$ (visualized in \cref{fig:vmbf_vis}) implies that there exists information in $c$ from the sender just like $m$, meaning that $c$ also should be fed into the network to avoid information loss. 
% We ablate this consideration in  \cref{sec:exp_ablation}. 

\textbf{Fast Sampling from Equivalent Bayesian Flow Distribution} Based on the above reformulations, the Bayesian flow distribution of von Mises distribution is reframed as: 
\begin{equation}\label{eq:flow_frac}
p_F(\btheta_i|\x;\alpha_1,\alpha_2,\dots,\alpha_i)=\E_{\update(\parsnt{1} \mid \parsnt{0}, \x ; \alphat{1})}\dots\E_{\update(\parsn_{i-1} \mid \parsnt{i-2}, \x; \alphat{i-1})} \update(\parsnt{i} | \parsnt{i-1},\x;\alphat{i} )
\end{equation}
Naively sampling from \cref{eq:flow_frac} requires slow auto-regressive iterated simulation, making training unaffordable. Noticing the mathematical properties of \cref{eq:h_m,eq:h_c}, we  transform \cref{eq:flow_frac} to the equivalent form:
\begin{equation}\label{eq:cirflow_equiv}
p_F(\vec{m}_i|\x;\alpha_1,\alpha_2,\dots,\alpha_i)=\E_{vM(\y_1|\x,\alpha_1)\dots vM(\y_i|\x,\alpha_i)} \delta(\vec{m}_i-\text{atan2}(\sum_{j=1}^i \alpha_j \cos \y_j,\sum_{j=1}^i \alpha_j \sin \y_j))
\end{equation}
\begin{equation}\label{eq:cirflow_equiv2}
p_F(\vec{c}_i|\x;\alpha_1,\alpha_2,\dots,\alpha_i)=\E_{vM(\y_1|\x,\alpha_1)\dots vM(\y_i|\x,\alpha_i)}  \delta(\vec{c}_i-||[\sum_{j=1}^i \alpha_j \cos \y_j,\sum_{j=1}^i \alpha_j \sin \y_j]^T||_2)
\end{equation}
which bypasses the computation of intermediate variables and allows pure tensor operations, with negligible computational overhead.
\begin{restatable}{proposition}{cirflowequiv}
The probability density function of Bayesian flow distribution defined by \cref{eq:cirflow_equiv,eq:cirflow_equiv2} is equivalent to the original definition in \cref{eq:flow_frac}. 
\end{restatable}
\textbf{Numerical Determination of Linear Entropy Sender Accuracy Schedule} ~Original BFN designs the accuracy schedule $\beta(t)$ to make the entropy of input distribution linearly decrease. As for crystal generation task, to ensure information coherence between modalities, we choose a sender accuracy schedule $\senderacc$ that makes the receiver's belief entropy $H(t_i)=H(p_I(\cdot|\vtheta_i))=H(p_I(\cdot|\vc_i))$ linearly decrease \emph{w.r.t.} time $t_i$, given the initial and final accuracy parameter $c(0)$ and $c(1)$. Due to the intractability of \cref{eq:vm_entropy}, we first use numerical binary search in $[0,c(1)]$ to determine the receiver's $c(t_i)$ for $i=1,\dots, n$ by solving the equation $H(c(t_i))=(1-t_i)H(c(0))+tH(c(1))$. Next, with $c(t_i)$, we conduct numerical binary search for each $\alpha_i$ in $[0,c(1)]$ by solving the equations $\E_{y\sim vM(x,\alpha_i)}[\sqrt{\alpha_i^2+c_{i-1}^2+2\alpha_i c_{i-1}\cos(y-m_{i-1})}]=c(t_i)$ from $i=1$ to $i=n$ for arbitrarily selected $x\in[-\pi,\pi)$.

After tackling all those issues, we have now arrived at a new BFN architecture for effectively modeling crystals. Such BFN can also be adapted to other type of data located in hyper-torus $\mathbb{T}^{D}$.

\subsection{Equivariant Bayesian Flow for Crystal}
With the above Bayesian flow designed for generative modeling of fractional coordinate $\vF$, we are able to build equivariant Bayesian flow for each modality of crystal. In this section, we first give an overview of the general training and sampling algorithm of \modelname (visualized in \cref{fig:framework}). Then, we describe the details of the Bayesian flow of every modality. The training and sampling algorithm can be found in \cref{alg:train} and \cref{alg:sampling}.

\textbf{Overview} Operating in the parameter space $\bthetaM=\{\bthetaA,\bthetaL,\bthetaF\}$, \modelname generates high-fidelity crystals through a joint BFN sampling process on the parameter of  atom type $\bthetaA$, lattice parameter $\vec{\theta}^L=\{\bmuL,\brhoL\}$, and the parameter of fractional coordinate matrix $\bthetaF=\{\bmF,\bcF\}$. We index the $n$-steps of the generation process in a discrete manner $i$, and denote the corresponding continuous notation $t_i=i/n$ from prior parameter $\thetaM_0$ to a considerably low variance parameter $\thetaM_n$ (\emph{i.e.} large $\vrho^L,\bmF$, and centered $\bthetaA$).

At training time, \modelname samples time $i\sim U\{1,n\}$ and $\bthetaM_{i-1}$ from the Bayesian flow distribution of each modality, serving as the input to the network. The network $\net$ outputs $\net(\parsnt{i-1}^\mathcal{M},t_{i-1})=\net(\parsnt{i-1}^A,\parsnt{i-1}^F,\parsnt{i-1}^L,t_{i-1})$ and conducts gradient descents on loss function \cref{eq:loss_n} for each modality. After proper training, the sender distribution $p_S$ can be approximated by the receiver distribution $p_R$. 

At inference time, from predefined $\thetaM_0$, we conduct transitions from $\thetaM_{i-1}$ to $\thetaM_{i}$ by: $(1)$ sampling $\y_i\sim p_R(\bold{y}|\thetaM_{i-1};t_i,\alpha_i)$ according to network prediction $\predM{i-1}$; and $(2)$ performing Bayesian update $h(\thetaM_{i-1},\y^\calM_{i-1},\alpha_i)$ for each dimension. 

% Alternatively, we complete this transition using the flow-back technique by sampling 
% $\thetaM_{i}$ from Bayesian flow distribution $\flow(\btheta^M_{i}|\predM{i-1};t_{i-1})$. 

% The training objective of $\net$ is to minimize the KL divergence between sender distribution and receiver distribution for every modality as defined in \cref{eq:loss_n} which is equivalent to optimizing the negative variational lower bound $\calL^{VLB}$ as discussed in \cref{sec:preliminaries}. 

%In the following part, we will present the Bayesian flow of each modality in detail.

\textbf{Bayesian Flow of Fractional Coordinate $\vF$}~The distribution of the prior parameter $\bthetaF_0$ is defined as:
\begin{equation}\label{eq:prior_frac}
    p(\bthetaF_0) \defeq \{vM(\vm_0^F|\vec{0}_{3\times N},\vec{0}_{3\times N}),\delta(\vc_0^F-\vec{0}_{3\times N})\} = \{U(\vec{0},\vec{1}),\delta(\vc_0^F-\vec{0}_{3\times N})\}
\end{equation}
Note that this prior distribution of $\vm_0^F$ is uniform over $[\vec{0},\vec{1})$, ensuring the periodic translation invariance property in \cref{De:pi}. The training objective is minimizing the KL divergence between sender and receiver distribution (deduction can be found in \cref{appd:cir_loss}): 
%\oyyw{replace $\vF$ with $\x$?} \hanlin{notations follow Preliminary?}
\begin{align}\label{loss_frac}
\calL_F = n \E_{i \sim \ui{n}, \flow(\parsn{}^F \mid \vF ; \senderacc)} \alpha_i\frac{I_1(\alpha_i)}{I_0(\alpha_i)}(1-\cos(\vF-\predF{i-1}))
\end{align}
where $I_0(x)$ and $I_1(x)$ are the zeroth and the first order of modified Bessel functions. The transition from $\bthetaF_{i-1}$ to $\bthetaF_{i}$ is the Bayesian update distribution based on network prediction:
\begin{equation}\label{eq:transi_frac}
    p(\btheta^F_{i}|\parsnt{i-1}^\calM)=\mathbb{E}_{vM(\bold{y}|\predF{i-1},\alpha_i)}\delta(\btheta^F_{i}-h(\btheta^F_{i-1},\bold{y},\alpha_i))
\end{equation}
\begin{restatable}{proposition}{fracinv}
With $\net_{F}$ as a periodic translation equivariant function namely $\net_F(\parsnt{}^A,w(\parsnt{}^F+\vt),\parsnt{}^L,t)=w(\net_F(\parsnt{}^A,\parsnt{}^F,\parsnt{}^L,t)+\vt), \forall\vt\in\R^3$, the marginal distribution of $p(\vF_n)$ defined by \cref{eq:prior_frac,eq:transi_frac} is periodic translation invariant. 
\end{restatable}
\textbf{Bayesian Flow of Lattice Parameter \texorpdfstring{$\boldsymbol{L}$}{}}   
Noting the lattice parameter $\bm{L}$ located in Euclidean space, we set prior as the parameter of a isotropic multivariate normal distribution $\btheta^L_0\defeq\{\vmu_0^L,\vrho_0^L\}=\{\bm{0}_{3\times3},\bm{1}_{3\times3}\}$
% \begin{equation}\label{eq:lattice_prior}
% \btheta^L_0\defeq\{\vmu_0^L,\vrho_0^L\}=\{\bm{0}_{3\times3},\bm{1}_{3\times3}\}
% \end{equation}
such that the prior distribution of the Markov process on $\vmu^L$ is the Dirac distribution $\delta(\vec{\mu_0}-\vec{0})$ and $\delta(\vec{\rho_0}-\vec{1})$, 
% \begin{equation}
%     p_I^L(\boldsymbol{L}|\btheta_0^L)=\mathcal{N}(\bm{L}|\bm{0},\bm{I})
% \end{equation}
which ensures O(3)-invariance of prior distribution of $\vL$. By Eq. 77 from \citet{bfn}, the Bayesian flow distribution of the lattice parameter $\bm{L}$ is: 
\begin{align}% =p_U(\bmuL|\btheta_0^L,\bm{L},\beta(t))
p_F^L(\bmuL|\bm{L};t) &=\mathcal{N}(\bmuL|\gamma(t)\bm{L},\gamma(t)(1-\gamma(t))\bm{I}) 
\end{align}
where $\gamma(t) = 1 - \sigma_1^{2t}$ and $\sigma_1$ is the predefined hyper-parameter controlling the variance of input distribution at $t=1$ under linear entropy accuracy schedule. The variance parameter $\vrho$ does not need to be modeled and fed to the network, since it is deterministic given the accuracy schedule. After sampling $\bmuL_i$ from $p_F^L$, the training objective is defined as minimizing KL divergence between sender and receiver distribution (based on Eq. 96 in \citet{bfn}):
\begin{align}
\mathcal{L}_{L} = \frac{n}{2}\left(1-\sigma_1^{2/n}\right)\E_{i \sim \ui{n}}\E_{\flow(\bmuL_{i-1} |\vL ; t_{i-1})}  \frac{\left\|\vL -\predL{i-1}\right\|^2}{\sigma_1^{2i/n}},\label{eq:lattice_loss}
\end{align}
where the prediction term $\predL{i-1}$ is the lattice parameter part of network output. After training, the generation process is defined as the Bayesian update distribution given network prediction:
\begin{equation}\label{eq:lattice_sampling}
    p(\bmuL_{i}|\parsnt{i-1}^\calM)=\update^L(\bmuL_{i}|\predL{i-1},\bmuL_{i-1};t_{i-1})
\end{equation}
    

% The final prediction of the lattice parameter is given by $\bmuL_n = \predL{n-1}$.
% \begin{equation}\label{eq:final_lattice}
%     \bmuL_n = \predL{n-1}
% \end{equation}

\begin{restatable}{proposition}{latticeinv}\label{prop:latticeinv}
With $\net_{L}$ as  O(3)-equivariant function namely $\net_L(\parsnt{}^A,\parsnt{}^F,\vQ\parsnt{}^L,t)=\vQ\net_L(\parsnt{}^A,\parsnt{}^F,\parsnt{}^L,t),\forall\vQ^T\vQ=\vI$, the marginal distribution of $p(\bmuL_n)$ defined by \cref{eq:lattice_sampling} is O(3)-invariant. 
\end{restatable}


\textbf{Bayesian Flow of Atom Types \texorpdfstring{$\boldsymbol{A}$}{}} 
Given that atom types are discrete random variables located in a simplex $\calS^K$, the prior parameter of $\boldsymbol{A}$ is the discrete uniform distribution over the vocabulary $\parsnt{0}^A \defeq \frac{1}{K}\vec{1}_{1\times N}$. 
% \begin{align}\label{eq:disc_input_prior}
% \parsnt{0}^A \defeq \frac{1}{K}\vec{1}_{1\times N}
% \end{align}
% \begin{align}
%     (\oh{j}{K})_k \defeq \delta_{j k}, \text{where }\oh{j}{K}\in \R^{K},\oh{\vA}{KD} \defeq \left(\oh{a_1}{K},\dots,\oh{a_N}{K}\right) \in \R^{K\times N}
% \end{align}
With the notation of the projection from the class index $j$ to the length $K$ one-hot vector $ (\oh{j}{K})_k \defeq \delta_{j k}, \text{where }\oh{j}{K}\in \R^{K},\oh{\vA}{KD} \defeq \left(\oh{a_1}{K},\dots,\oh{a_N}{K}\right) \in \R^{K\times N}$, the Bayesian flow distribution of atom types $\vA$ is derived in \citet{bfn}:
\begin{align}
\flow^{A}(\parsn^A \mid \vA; t) &= \E_{\N{\y \mid \beta^A(t)\left(K \oh{\vA}{K\times N} - \vec{1}_{K\times N}\right)}{\beta^A(t) K \vec{I}_{K\times N \times N}}} \delta\left(\parsn^A - \frac{e^{\y}\parsnt{0}^A}{\sum_{k=1}^K e^{\y_k}(\parsnt{0})_{k}^A}\right).
\end{align}
where $\beta^A(t)$ is the predefined accuracy schedule for atom types. Sampling $\btheta_i^A$ from $p_F^A$ as the training signal, the training objective is the $n$-step discrete-time loss for discrete variable \citep{bfn}: 
% \oyyw{can we simplify the next equation? Such as remove $K \times N, K \times N \times N$}
% \begin{align}
% &\calL_A = n\E_{i \sim U\{1,n\},\flow^A(\parsn^A \mid \vA ; t_{i-1}),\N{\y \mid \alphat{i}\left(K \oh{\vA}{KD} - \vec{1}_{K\times N}\right)}{\alphat{i} K \vec{I}_{K\times N \times N}}} \ln \N{\y \mid \alphat{i}\left(K \oh{\vA}{K\times N} - \vec{1}_{K\times N}\right)}{\alphat{i} K \vec{I}_{K\times N \times N}}\nonumber\\
% &\qquad\qquad\qquad-\sum_{d=1}^N \ln \left(\sum_{k=1}^K \out^{(d)}(k \mid \parsn^A; t_{i-1}) \N{\ydd{d} \mid \alphat{i}\left(K\oh{k}{K}- \vec{1}_{K\times N}\right)}{\alphat{i} K \vec{I}_{K\times N \times N}}\right)\label{discdisc_t_loss_exp}
% \end{align}
\begin{align}
&\calL_A = n\E_{i \sim U\{1,n\},\flow^A(\parsn^A \mid \vA ; t_{i-1}),\N{\y \mid \alphat{i}\left(K \oh{\vA}{KD} - \vec{1}\right)}{\alphat{i} K \vec{I}}} \ln \N{\y \mid \alphat{i}\left(K \oh{\vA}{K\times N} - \vec{1}\right)}{\alphat{i} K \vec{I}}\nonumber\\
&\qquad\qquad\qquad-\sum_{d=1}^N \ln \left(\sum_{k=1}^K \out^{(d)}(k \mid \parsn^A; t_{i-1}) \N{\ydd{d} \mid \alphat{i}\left(K\oh{k}{K}- \vec{1}\right)}{\alphat{i} K \vec{I}}\right)\label{discdisc_t_loss_exp}
\end{align}
where $\vec{I}\in \R^{K\times N \times N}$ and $\vec{1}\in\R^{K\times D}$. When sampling, the transition from $\bthetaA_{i-1}$ to $\bthetaA_{i}$ is derived as:
\begin{equation}
    p(\btheta^A_{i}|\parsnt{i-1}^\calM)=\update^A(\btheta^A_{i}|\btheta^A_{i-1},\predA{i-1};t_{i-1})
\end{equation}

The detailed training and sampling algorithm could be found in \cref{alg:train} and \cref{alg:sampling}.




\section{Experiments}
\label{sec:experiments}
The experiments are designed to address two key research questions.
First, \textbf{RQ1} evaluates whether the average $L_2$-norm of the counterfactual perturbation vectors ($\overline{||\perturb||}$) decreases as the model overfits the data, thereby providing further empirical validation for our hypothesis.
Second, \textbf{RQ2} evaluates the ability of the proposed counterfactual regularized loss, as defined in (\ref{eq:regularized_loss2}), to mitigate overfitting when compared to existing regularization techniques.

% The experiments are designed to address three key research questions. First, \textbf{RQ1} investigates whether the mean perturbation vector norm decreases as the model overfits the data, aiming to further validate our intuition. Second, \textbf{RQ2} explores whether the mean perturbation vector norm can be effectively leveraged as a regularization term during training, offering insights into its potential role in mitigating overfitting. Finally, \textbf{RQ3} examines whether our counterfactual regularizer enables the model to achieve superior performance compared to existing regularization methods, thus highlighting its practical advantage.

\subsection{Experimental Setup}
\textbf{\textit{Datasets, Models, and Tasks.}}
The experiments are conducted on three datasets: \textit{Water Potability}~\cite{kadiwal2020waterpotability}, \textit{Phomene}~\cite{phomene}, and \textit{CIFAR-10}~\cite{krizhevsky2009learning}. For \textit{Water Potability} and \textit{Phomene}, we randomly select $80\%$ of the samples for the training set, and the remaining $20\%$ for the test set, \textit{CIFAR-10} comes already split. Furthermore, we consider the following models: Logistic Regression, Multi-Layer Perceptron (MLP) with 100 and 30 neurons on each hidden layer, and PreactResNet-18~\cite{he2016cvecvv} as a Convolutional Neural Network (CNN) architecture.
We focus on binary classification tasks and leave the extension to multiclass scenarios for future work. However, for datasets that are inherently multiclass, we transform the problem into a binary classification task by selecting two classes, aligning with our assumption.

\smallskip
\noindent\textbf{\textit{Evaluation Measures.}} To characterize the degree of overfitting, we use the test loss, as it serves as a reliable indicator of the model's generalization capability to unseen data. Additionally, we evaluate the predictive performance of each model using the test accuracy.

\smallskip
\noindent\textbf{\textit{Baselines.}} We compare CF-Reg with the following regularization techniques: L1 (``Lasso''), L2 (``Ridge''), and Dropout.

\smallskip
\noindent\textbf{\textit{Configurations.}}
For each model, we adopt specific configurations as follows.
\begin{itemize}
\item \textit{Logistic Regression:} To induce overfitting in the model, we artificially increase the dimensionality of the data beyond the number of training samples by applying a polynomial feature expansion. This approach ensures that the model has enough capacity to overfit the training data, allowing us to analyze the impact of our counterfactual regularizer. The degree of the polynomial is chosen as the smallest degree that makes the number of features greater than the number of data.
\item \textit{Neural Networks (MLP and CNN):} To take advantage of the closed-form solution for computing the optimal perturbation vector as defined in (\ref{eq:opt-delta}), we use a local linear approximation of the neural network models. Hence, given an instance $\inst_i$, we consider the (optimal) counterfactual not with respect to $\model$ but with respect to:
\begin{equation}
\label{eq:taylor}
    \model^{lin}(\inst) = \model(\inst_i) + \nabla_{\inst}\model(\inst_i)(\inst - \inst_i),
\end{equation}
where $\model^{lin}$ represents the first-order Taylor approximation of $\model$ at $\inst_i$.
Note that this step is unnecessary for Logistic Regression, as it is inherently a linear model.
\end{itemize}

\smallskip
\noindent \textbf{\textit{Implementation Details.}} We run all experiments on a machine equipped with an AMD Ryzen 9 7900 12-Core Processor and an NVIDIA GeForce RTX 4090 GPU. Our implementation is based on the PyTorch Lightning framework. We use stochastic gradient descent as the optimizer with a learning rate of $\eta = 0.001$ and no weight decay. We use a batch size of $128$. The training and test steps are conducted for $6000$ epochs on the \textit{Water Potability} and \textit{Phoneme} datasets, while for the \textit{CIFAR-10} dataset, they are performed for $200$ epochs.
Finally, the contribution $w_i^{\varepsilon}$ of each training point $\inst_i$ is uniformly set as $w_i^{\varepsilon} = 1~\forall i\in \{1,\ldots,m\}$.

The source code implementation for our experiments is available at the following GitHub repository: \url{https://anonymous.4open.science/r/COCE-80B4/README.md} 

\subsection{RQ1: Counterfactual Perturbation vs. Overfitting}
To address \textbf{RQ1}, we analyze the relationship between the test loss and the average $L_2$-norm of the counterfactual perturbation vectors ($\overline{||\perturb||}$) over training epochs.

In particular, Figure~\ref{fig:delta_loss_epochs} depicts the evolution of $\overline{||\perturb||}$ alongside the test loss for an MLP trained \textit{without} regularization on the \textit{Water Potability} dataset. 
\begin{figure}[ht]
    \centering
    \includegraphics[width=0.85\linewidth]{img/delta_loss_epochs.png}
    \caption{The average counterfactual perturbation vector $\overline{||\perturb||}$ (left $y$-axis) and the cross-entropy test loss (right $y$-axis) over training epochs ($x$-axis) for an MLP trained on the \textit{Water Potability} dataset \textit{without} regularization.}
    \label{fig:delta_loss_epochs}
\end{figure}

The plot shows a clear trend as the model starts to overfit the data (evidenced by an increase in test loss). 
Notably, $\overline{||\perturb||}$ begins to decrease, which aligns with the hypothesis that the average distance to the optimal counterfactual example gets smaller as the model's decision boundary becomes increasingly adherent to the training data.

It is worth noting that this trend is heavily influenced by the choice of the counterfactual generator model. In particular, the relationship between $\overline{||\perturb||}$ and the degree of overfitting may become even more pronounced when leveraging more accurate counterfactual generators. However, these models often come at the cost of higher computational complexity, and their exploration is left to future work.

Nonetheless, we expect that $\overline{||\perturb||}$ will eventually stabilize at a plateau, as the average $L_2$-norm of the optimal counterfactual perturbations cannot vanish to zero.

% Additionally, the choice of employing the score-based counterfactual explanation framework to generate counterfactuals was driven to promote computational efficiency.

% Future enhancements to the framework may involve adopting models capable of generating more precise counterfactuals. While such approaches may yield to performance improvements, they are likely to come at the cost of increased computational complexity.


\subsection{RQ2: Counterfactual Regularization Performance}
To answer \textbf{RQ2}, we evaluate the effectiveness of the proposed counterfactual regularization (CF-Reg) by comparing its performance against existing baselines: unregularized training loss (No-Reg), L1 regularization (L1-Reg), L2 regularization (L2-Reg), and Dropout.
Specifically, for each model and dataset combination, Table~\ref{tab:regularization_comparison} presents the mean value and standard deviation of test accuracy achieved by each method across 5 random initialization. 

The table illustrates that our regularization technique consistently delivers better results than existing methods across all evaluated scenarios, except for one case -- i.e., Logistic Regression on the \textit{Phomene} dataset. 
However, this setting exhibits an unusual pattern, as the highest model accuracy is achieved without any regularization. Even in this case, CF-Reg still surpasses other regularization baselines.

From the results above, we derive the following key insights. First, CF-Reg proves to be effective across various model types, ranging from simple linear models (Logistic Regression) to deep architectures like MLPs and CNNs, and across diverse datasets, including both tabular and image data. 
Second, CF-Reg's strong performance on the \textit{Water} dataset with Logistic Regression suggests that its benefits may be more pronounced when applied to simpler models. However, the unexpected outcome on the \textit{Phoneme} dataset calls for further investigation into this phenomenon.


\begin{table*}[h!]
    \centering
    \caption{Mean value and standard deviation of test accuracy across 5 random initializations for different model, dataset, and regularization method. The best results are highlighted in \textbf{bold}.}
    \label{tab:regularization_comparison}
    \begin{tabular}{|c|c|c|c|c|c|c|}
        \hline
        \textbf{Model} & \textbf{Dataset} & \textbf{No-Reg} & \textbf{L1-Reg} & \textbf{L2-Reg} & \textbf{Dropout} & \textbf{CF-Reg (ours)} \\ \hline
        Logistic Regression   & \textit{Water}   & $0.6595 \pm 0.0038$   & $0.6729 \pm 0.0056$   & $0.6756 \pm 0.0046$  & N/A    & $\mathbf{0.6918 \pm 0.0036}$                     \\ \hline
        MLP   & \textit{Water}   & $0.6756 \pm 0.0042$   & $0.6790 \pm 0.0058$   & $0.6790 \pm 0.0023$  & $0.6750 \pm 0.0036$    & $\mathbf{0.6802 \pm 0.0046}$                    \\ \hline
%        MLP   & \textit{Adult}   & $0.8404 \pm 0.0010$   & $\mathbf{0.8495 \pm 0.0007}$   & $0.8489 \pm 0.0014$  & $\mathbf{0.8495 \pm 0.0016}$     & $0.8449 \pm 0.0019$                    \\ \hline
        Logistic Regression   & \textit{Phomene}   & $\mathbf{0.8148 \pm 0.0020}$   & $0.8041 \pm 0.0028$   & $0.7835 \pm 0.0176$  & N/A    & $0.8098 \pm 0.0055$                     \\ \hline
        MLP   & \textit{Phomene}   & $0.8677 \pm 0.0033$   & $0.8374 \pm 0.0080$   & $0.8673 \pm 0.0045$  & $0.8672 \pm 0.0042$     & $\mathbf{0.8718 \pm 0.0040}$                    \\ \hline
        CNN   & \textit{CIFAR-10} & $0.6670 \pm 0.0233$   & $0.6229 \pm 0.0850$   & $0.7348 \pm 0.0365$   & N/A    & $\mathbf{0.7427 \pm 0.0571}$                     \\ \hline
    \end{tabular}
\end{table*}

\begin{table*}[htb!]
    \centering
    \caption{Hyperparameter configurations utilized for the generation of Table \ref{tab:regularization_comparison}. For our regularization the hyperparameters are reported as $\mathbf{\alpha/\beta}$.}
    \label{tab:performance_parameters}
    \begin{tabular}{|c|c|c|c|c|c|c|}
        \hline
        \textbf{Model} & \textbf{Dataset} & \textbf{No-Reg} & \textbf{L1-Reg} & \textbf{L2-Reg} & \textbf{Dropout} & \textbf{CF-Reg (ours)} \\ \hline
        Logistic Regression   & \textit{Water}   & N/A   & $0.0093$   & $0.6927$  & N/A    & $0.3791/1.0355$                     \\ \hline
        MLP   & \textit{Water}   & N/A   & $0.0007$   & $0.0022$  & $0.0002$    & $0.2567/1.9775$                    \\ \hline
        Logistic Regression   &
        \textit{Phomene}   & N/A   & $0.0097$   & $0.7979$  & N/A    & $0.0571/1.8516$                     \\ \hline
        MLP   & \textit{Phomene}   & N/A   & $0.0007$   & $4.24\cdot10^{-5}$  & $0.0015$    & $0.0516/2.2700$                    \\ \hline
       % MLP   & \textit{Adult}   & N/A   & $0.0018$   & $0.0018$  & $0.0601$     & $0.0764/2.2068$                    \\ \hline
        CNN   & \textit{CIFAR-10} & N/A   & $0.0050$   & $0.0864$ & N/A    & $0.3018/
        2.1502$                     \\ \hline
    \end{tabular}
\end{table*}

\begin{table*}[htb!]
    \centering
    \caption{Mean value and standard deviation of training time across 5 different runs. The reported time (in seconds) corresponds to the generation of each entry in Table \ref{tab:regularization_comparison}. Times are }
    \label{tab:times}
    \begin{tabular}{|c|c|c|c|c|c|c|}
        \hline
        \textbf{Model} & \textbf{Dataset} & \textbf{No-Reg} & \textbf{L1-Reg} & \textbf{L2-Reg} & \textbf{Dropout} & \textbf{CF-Reg (ours)} \\ \hline
        Logistic Regression   & \textit{Water}   & $222.98 \pm 1.07$   & $239.94 \pm 2.59$   & $241.60 \pm 1.88$  & N/A    & $251.50 \pm 1.93$                     \\ \hline
        MLP   & \textit{Water}   & $225.71 \pm 3.85$   & $250.13 \pm 4.44$   & $255.78 \pm 2.38$  & $237.83 \pm 3.45$    & $266.48 \pm 3.46$                    \\ \hline
        Logistic Regression   & \textit{Phomene}   & $266.39 \pm 0.82$ & $367.52 \pm 6.85$   & $361.69 \pm 4.04$  & N/A   & $310.48 \pm 0.76$                    \\ \hline
        MLP   &
        \textit{Phomene} & $335.62 \pm 1.77$   & $390.86 \pm 2.11$   & $393.96 \pm 1.95$ & $363.51 \pm 5.07$    & $403.14 \pm 1.92$                     \\ \hline
       % MLP   & \textit{Adult}   & N/A   & $0.0018$   & $0.0018$  & $0.0601$     & $0.0764/2.2068$                    \\ \hline
        CNN   & \textit{CIFAR-10} & $370.09 \pm 0.18$   & $395.71 \pm 0.55$   & $401.38 \pm 0.16$ & N/A    & $1287.8 \pm 0.26$                     \\ \hline
    \end{tabular}
\end{table*}

\subsection{Feasibility of our Method}
A crucial requirement for any regularization technique is that it should impose minimal impact on the overall training process.
In this respect, CF-Reg introduces an overhead that depends on the time required to find the optimal counterfactual example for each training instance. 
As such, the more sophisticated the counterfactual generator model probed during training the higher would be the time required. However, a more advanced counterfactual generator might provide a more effective regularization. We discuss this trade-off in more details in Section~\ref{sec:discussion}.

Table~\ref{tab:times} presents the average training time ($\pm$ standard deviation) for each model and dataset combination listed in Table~\ref{tab:regularization_comparison}.
We can observe that the higher accuracy achieved by CF-Reg using the score-based counterfactual generator comes with only minimal overhead. However, when applied to deep neural networks with many hidden layers, such as \textit{PreactResNet-18}, the forward derivative computation required for the linearization of the network introduces a more noticeable computational cost, explaining the longer training times in the table.

\subsection{Hyperparameter Sensitivity Analysis}
The proposed counterfactual regularization technique relies on two key hyperparameters: $\alpha$ and $\beta$. The former is intrinsic to the loss formulation defined in (\ref{eq:cf-train}), while the latter is closely tied to the choice of the score-based counterfactual explanation method used.

Figure~\ref{fig:test_alpha_beta} illustrates how the test accuracy of an MLP trained on the \textit{Water Potability} dataset changes for different combinations of $\alpha$ and $\beta$.

\begin{figure}[ht]
    \centering
    \includegraphics[width=0.85\linewidth]{img/test_acc_alpha_beta.png}
    \caption{The test accuracy of an MLP trained on the \textit{Water Potability} dataset, evaluated while varying the weight of our counterfactual regularizer ($\alpha$) for different values of $\beta$.}
    \label{fig:test_alpha_beta}
\end{figure}

We observe that, for a fixed $\beta$, increasing the weight of our counterfactual regularizer ($\alpha$) can slightly improve test accuracy until a sudden drop is noticed for $\alpha > 0.1$.
This behavior was expected, as the impact of our penalty, like any regularization term, can be disruptive if not properly controlled.

Moreover, this finding further demonstrates that our regularization method, CF-Reg, is inherently data-driven. Therefore, it requires specific fine-tuning based on the combination of the model and dataset at hand.
\section{Discussion of Assumptions}\label{sec:discussion}
In this paper, we have made several assumptions for the sake of clarity and simplicity. In this section, we discuss the rationale behind these assumptions, the extent to which these assumptions hold in practice, and the consequences for our protocol when these assumptions hold.

\subsection{Assumptions on the Demand}

There are two simplifying assumptions we make about the demand. First, we assume the demand at any time is relatively small compared to the channel capacities. Second, we take the demand to be constant over time. We elaborate upon both these points below.

\paragraph{Small demands} The assumption that demands are small relative to channel capacities is made precise in \eqref{eq:large_capacity_assumption}. This assumption simplifies two major aspects of our protocol. First, it largely removes congestion from consideration. In \eqref{eq:primal_problem}, there is no constraint ensuring that total flow in both directions stays below capacity--this is always met. Consequently, there is no Lagrange multiplier for congestion and no congestion pricing; only imbalance penalties apply. In contrast, protocols in \cite{sivaraman2020high, varma2021throughput, wang2024fence} include congestion fees due to explicit congestion constraints. Second, the bound \eqref{eq:large_capacity_assumption} ensures that as long as channels remain balanced, the network can always meet demand, no matter how the demand is routed. Since channels can rebalance when necessary, they never drop transactions. This allows prices and flows to adjust as per the equations in \eqref{eq:algorithm}, which makes it easier to prove the protocol's convergence guarantees. This also preserves the key property that a channel's price remains proportional to net money flow through it.

In practice, payment channel networks are used most often for micro-payments, for which on-chain transactions are prohibitively expensive; large transactions typically take place directly on the blockchain. For example, according to \cite{river2023lightning}, the average channel capacity is roughly $0.1$ BTC ($5,000$ BTC distributed over $50,000$ channels), while the average transaction amount is less than $0.0004$ BTC ($44.7k$ satoshis). Thus, the small demand assumption is not too unrealistic. Additionally, the occasional large transaction can be treated as a sequence of smaller transactions by breaking it into packets and executing each packet serially (as done by \cite{sivaraman2020high}).
Lastly, a good path discovery process that favors large capacity channels over small capacity ones can help ensure that the bound in \eqref{eq:large_capacity_assumption} holds.

\paragraph{Constant demands} 
In this work, we assume that any transacting pair of nodes have a steady transaction demand between them (see Section \ref{sec:transaction_requests}). Making this assumption is necessary to obtain the kind of guarantees that we have presented in this paper. Unless the demand is steady, it is unreasonable to expect that the flows converge to a steady value. Weaker assumptions on the demand lead to weaker guarantees. For example, with the more general setting of stochastic, but i.i.d. demand between any two nodes, \cite{varma2021throughput} shows that the channel queue lengths are bounded in expectation. If the demand can be arbitrary, then it is very hard to get any meaningful performance guarantees; \cite{wang2024fence} shows that even for a single bidirectional channel, the competitive ratio is infinite. Indeed, because a PCN is a decentralized system and decisions must be made based on local information alone, it is difficult for the network to find the optimal detailed balance flow at every time step with a time-varying demand.  With a steady demand, the network can discover the optimal flows in a reasonably short time, as our work shows.

We view the constant demand assumption as an approximation for a more general demand process that could be piece-wise constant, stochastic, or both (see simulations in Figure \ref{fig:five_nodes_variable_demand}).
We believe it should be possible to merge ideas from our work and \cite{varma2021throughput} to provide guarantees in a setting with random demands with arbitrary means. We leave this for future work. In addition, our work suggests that a reasonable method of handling stochastic demands is to queue the transaction requests \textit{at the source node} itself. This queuing action should be viewed in conjunction with flow-control. Indeed, a temporarily high unidirectional demand would raise prices for the sender, incentivizing the sender to stop sending the transactions. If the sender queues the transactions, they can send them later when prices drop. This form of queuing does not require any overhaul of the basic PCN infrastructure and is therefore simpler to implement than per-channel queues as suggested by \cite{sivaraman2020high} and \cite{varma2021throughput}.

\subsection{The Incentive of Channels}
The actions of the channels as prescribed by the DEBT control protocol can be summarized as follows. Channels adjust their prices in proportion to the net flow through them. They rebalance themselves whenever necessary and execute any transaction request that has been made of them. We discuss both these aspects below.

\paragraph{On Prices}
In this work, the exclusive role of channel prices is to ensure that the flows through each channel remains balanced. In practice, it would be important to include other components in a channel's price/fee as well: a congestion price  and an incentive price. The congestion price, as suggested by \cite{varma2021throughput}, would depend on the total flow of transactions through the channel, and would incentivize nodes to balance the load over different paths. The incentive price, which is commonly used in practice \cite{river2023lightning}, is necessary to provide channels with an incentive to serve as an intermediary for different channels. In practice, we expect both these components to be smaller than the imbalance price. Consequently, we expect the behavior of our protocol to be similar to our theoretical results even with these additional prices.

A key aspect of our protocol is that channel fees are allowed to be negative. Although the original Lightning network whitepaper \cite{poon2016bitcoin} suggests that negative channel prices may be a good solution to promote rebalancing, the idea of negative prices in not very popular in the literature. To our knowledge, the only prior work with this feature is \cite{varma2021throughput}. Indeed, in papers such as \cite{van2021merchant} and \cite{wang2024fence}, the price function is explicitly modified such that the channel price is never negative. The results of our paper show the benefits of negative prices. For one, in steady state, equal flows in both directions ensure that a channel doesn't loose any money (the other price components mentioned above ensure that the channel will only gain money). More importantly, negative prices are important to ensure that the protocol selectively stifles acyclic flows while allowing circulations to flow. Indeed, in the example of Section \ref{sec:flow_control_example}, the flows between nodes $A$ and $C$ are left on only because the large positive price over one channel is canceled by the corresponding negative price over the other channel, leading to a net zero price.

Lastly, observe that in the DEBT control protocol, the price charged by a channel does not depend on its capacity. This is a natural consequence of the price being the Lagrange multiplier for the net-zero flow constraint, which also does not depend on the channel capacity. In contrast, in many other works, the imbalance price is normalized by the channel capacity \cite{ren2018optimal, lin2020funds, wang2024fence}; this is shown to work well in practice. The rationale for such a price structure is explained well in \cite{wang2024fence}, where this fee is derived with the aim of always maintaining some balance (liquidity) at each end of every channel. This is a reasonable aim if a channel is to never rebalance itself; the experiments of the aforementioned papers are conducted in such a regime. In this work, however, we allow the channels to rebalance themselves a few times in order to settle on a detailed balance flow. This is because our focus is on the long-term steady state performance of the protocol. This difference in perspective also shows up in how the price depends on the channel imbalance. \cite{lin2020funds} and \cite{wang2024fence} advocate for strictly convex prices whereas this work and \cite{varma2021throughput} propose linear prices.

\paragraph{On Rebalancing} 
Recall that the DEBT control protocol ensures that the flows in the network converge to a detailed balance flow, which can be sustained perpetually without any rebalancing. However, during the transient phase (before convergence), channels may have to perform on-chain rebalancing a few times. Since rebalancing is an expensive operation, it is worthwhile discussing methods by which channels can reduce the extent of rebalancing. One option for the channels to reduce the extent of rebalancing is to increase their capacity; however, this comes at the cost of locking in more capital. Each channel can decide for itself the optimum amount of capital to lock in. Another option, which we discuss in Section \ref{sec:five_node}, is for channels to increase the rate $\gamma$ at which they adjust prices. 

Ultimately, whether or not it is beneficial for a channel to rebalance depends on the time-horizon under consideration. Our protocol is based on the assumption that the demand remains steady for a long period of time. If this is indeed the case, it would be worthwhile for a channel to rebalance itself as it can make up this cost through the incentive fees gained from the flow of transactions through it in steady state. If a channel chooses not to rebalance itself, however, there is a risk of being trapped in a deadlock, which is suboptimal for not only the nodes but also the channel.

\section{Conclusion}
This work presents DEBT control: a protocol for payment channel networks that uses source routing and flow control based on channel prices. The protocol is derived by posing a network utility maximization problem and analyzing its dual minimization. It is shown that under steady demands, the protocol guides the network to an optimal, sustainable point. Simulations show its robustness to demand variations. The work demonstrates that simple protocols with strong theoretical guarantees are possible for PCNs and we hope it inspires further theoretical research in this direction.
%\section{Conclusion}
In this work, we propose a simple yet effective approach, called SMILE, for graph few-shot learning with fewer tasks. Specifically, we introduce a novel dual-level mixup strategy, including within-task and across-task mixup, for enriching the diversity of nodes within each task and the diversity of tasks. Also, we incorporate the degree-based prior information to learn expressive node embeddings. Theoretically, we prove that SMILE effectively enhances the model's generalization performance. Empirically, we conduct extensive experiments on multiple benchmarks and the results suggest that SMILE significantly outperforms other baselines, including both in-domain and cross-domain few-shot settings.


\noindent {\textbf {Acknowledgements.}} This paper was prepared for informational purposes in part by the Artificial Intelligence Research group of JPMorgan Chase \& Co and its affiliates (“J.P. Morgan”) and is not a product of the Research Department of J.P. Morgan.  J.P. Morgan makes no representation and warranty whatsoever and disclaims all liability, for the completeness, accuracy or reliability of the information contained herein.  This document is not intended as investment research or investment advice, or a recommendation, offer or solicitation for the purchase or sale of any security, financial instrument, financial product or service, or to be used in any way for evaluating the merits of participating in any transaction, and shall not constitute a solicitation under any jurisdiction or to any person, if such solicitation under such jurisdiction or to such person would be unlawful.
% \noindent {\textbf {Acknowledgements.}} This paper was prepared for informational purposes in part by the Artificial Intelligence Research group of JPMorgan Chase \& Co and its affiliates (“J.P. Morgan”) and is not a product of the Research Department of J.P. Morgan.  J.P. Morgan makes no representation and warranty whatsoever and disclaims all liability, for the completeness, accuracy or reliability of the information contained herein.  This document is not intended as investment research or investment advice, or a recommendation, offer or solicitation for the purchase or sale of any security, financial instrument, financial product or service, or to be used in any way for evaluating the merits of participating in any transaction, and shall not constitute a solicitation under any jurisdiction or to any person, if such solicitation under such jurisdiction or to such person would be unlawful.
\bibliography{references}

% ------
\clearpage
\appendix
This is the supplementary material for the paper titled:
{\em On Creating a Causally Grounded Usable Rating Method for Assessing the
Robustness of Foundation Models Supporting Time Series}, submitted to AAAI 2025.


In this supplementary material, we provide additional details. 
Section~\ref{sec:appendix-algo-details} gives the rating algorithms. Section~\ref{sec:appendix-related-work} provides additional related work on robustness testing of FMs. Section~\ref{sec:appendix-metrics} provides the detailed description of existing evaluation metrics used to rate the FMTS in our experiments.
Section~\ref{sec:appendix-high-res-figs} contains the higher resolution version of the figures presented in the main paper. 
Section~\ref{sec:appendix-experiments} provides additional experimental results. Section~\ref{sec:appendix-user-study} contains additional user study results containing all the hypotheses validated along with statistical test results, and conclusions.
Section~\ref{sec:appendix-study-q} contains the user study form that was circulated to collect responses from the users.
Section~\ref{sec:appendix-source-code} contains source code used to process the datasets downloaded from Yahoo! finance.
Section~\ref{sec:appendix-implementation-details} contains additional system implementation details such as hyperparameters chosen.
Section~\ref{sec:appx-reproduc} contains the reproducibility checklist.

% \biplav{Accurately mention sections with usage of labels.}
% the deta algorithms adapted from \cite{kausik2024rating} to rate FMTS for robustness. We also provide additional experimental results to help better understand our work in detail. 
% The material is organized as follows:
% \tableofcontents




%----
\section{Details of  Rating Algorithms}
\label{sec:appendix-algo-details}

Apart from Algorithm 2, the rest were adapted from \cite{kausik2024rating} to suit the FMTS forecasting setting. Here is how the rating method works.
\begin{enumerate}
    \item Algorithm 1 computes the weighted rejection score (WRS) which was defined in Appendix \ref{sec:appendix-metrics} in the main paper.
    \item Algorithm 2 computes the PIE \% based on Propensity Score Matching (PSM) which was defined in Section \ref{sec:metrics} in the main paper.
    \item Algorithm 3 creates a partial order of systems within each perturbation based on the raw scores computed. It will arrange the systems in ascending order w.r.t the raw score. The final partial order (PO) will be a dictionary of dictionaries. 
    \item Algorithm 4 computes the final ratings for systems within each perturbation based on the PO from previous algorithm. It splits the set of raw score values obtained within each perturbation into `L' parts where `L' is the rating level chosen by the user. Each of the systems is given a rating based on the compartment number in which its raw score belongs. The algorithm will return a dictionary with perturbations as keys and ratings provided to each system within the perturbation as the value.
\end{enumerate}



\SetKwComment{Comment}{/* }{ */}
\begin{algorithm}
\small
	\caption{\emph{WeightedRejectionScore}}
	
	\textbf{Purpose:} is used to calculate the weighted sum of the number of rejections of null-hypothesis for Dataset $d_j$ pertaining to a system $s$, Confidence Intervals (CI) $ci_k$ and Weights $w_k$.
	
	\textbf{Input:}\\
	 $d$, dataset corresponding to a specific perturbation.\\
	 $CI$, confidence intervals (95\%, 70\%, 60\%). \\
      $s$, system for which WRS is being computed.   \\
     $W$, weights corresponding to different CIs (1, 0.8, 0.6).  \\
    \textbf{Output:} \\
     $Z$, Sensitive attribute  \\ 
	$\psi$, weighted rejection score.
	    $ \psi \gets 0$  
		\For {each $ci_i, w_i \in CI, W$} {
                    // $z_m, z_n$ are classes of $Z$

                    
		            \For {each $z_m, z_n \in Z$} { 
		                $t, pval, dof \gets T-Test(z_m, z_n) $\; 
		                $t_{crit} \gets LookUp(ci_i, dof) $\;
		                \eIf{$t_{crit} > t$} {
		                    $\psi \gets \psi + 0 $\;
		                 }
		                    {$\psi \gets \psi + w_i $}
		               
		            }
    }
    \Return $\psi$
\end{algorithm} 


\begin{algorithm}
\small
	\caption{ \emph{ComputePIEScore}}
	
	\textbf{Purpose:}  is used to calculate the Deconfounding Impact Estimation using Propensity Score Matching (PIE).
	
	\textbf{Input:}\\
	$s$, a system belonging to the set of test systems, $S$.\\
	$D$, datasets pertaining to a perturbation (different distributions).\\
        $p$, A perturbation other than $p_0$ \\
        $p_0$, control perturbation (or no perturbation. \\

    \textbf{Output:} \\
	$\psi$, PIE score.
	

    $ \psi \gets 0$
    
    $PIE\_list \gets []$
    \hspace{0.5 in} //  To store the list of PIE \% of all the datasets.

	\For {each $d_j \in D$} {
    	    $ APE\_o \gets  E(R|P=p_m) - E(R|P=p_0) $\; 
    	    $ APE\_m \gets  E(R|do(P=p_m)) - E(R|do(P=p_0)) $\; 
    		$PIE\_list[j] \gets (APE\_m - APE\_o) * 100$\;
        }
        $\psi \gets MAX(PIE\_list)$\;
    \Return $\psi$
\end{algorithm}


\begin{algorithm}
\small
	\caption{ \emph{CreatePartialOrder}}
	
	\textbf{Purpose:} is used to create a partial order based on the computed weighted rejection score / the PIE \%.
		
	\textbf{Input:}\\
	$S$, Set of systems.\\
        $P$, Set of perturbations.\\
	% $D$, datasets pertaining to different dataset groups.\\
	% $CI$, confidence intervals (95\%, 70\%, 60\%).  \\
 %    $W$, weights corresponding to different CIs (1, 0.8, 0.6).  \\
    $F$, Flag that says whether the confounder is present (1) or not (0).  \\
    $D$, $CI$, $W$ (as defined in the previous algorithms). \\
    \textbf{Output:} \\
	$PO$, dictionary with a partial order for each perturbation.

 
	    $ PO \gets \{\}$\; 
            $SD \gets \{\}$;

            
	    \eIf {F == 0}{
		    \For {each $p_i \in P$}{ 
                    \For {each $s_j \in S$} {
		              $\psi \gets WeightedRejectionScore(p_i, s_j, D, CI, W)$\; 
		              $SD[s_j] \gets \psi$\;
 		    }
                $PO[p_i] \gets SORT(SD)$\;
                }
 		    }
 		    {\For {each $p_j \in P$} {
                    \For {each $s_i \in S$}{ 
		        $\psi \gets ComputePIEScore(s_i, p_j, p_0, D, CI, W)$\; 
		          $SD[s_j] \gets \psi$\;
 		    }
                $PO[p_i] \gets SORT(SD)$\;
                }
 		   }
	    \Return $PO$
\end{algorithm}  


\begin{algorithm}
\small
	\caption{\emph{AssignRating}}
	
	\textbf{Purpose:} \emph{AssignRating} assigns a rating to each of the SASs based on the partial order and the number of rating levels, $L$.
		
	\textbf{Input:}\\
	$S$, $D$, $CI$, $W$, $P$ (as defined in the previous algorithms).\\
	$L$, rating levels chosen by the user.\\

    \textbf{Output:} \\
	$R$, dictionary with perturbations as keys and ratings assigned to each system within each perturbation as the values.\\

    $R \gets \{\}$\;
    $PO \gets \text{CreatePartialOrder}(S,D,CI,W,G)$\;

    \For{$p_i \in P$}{
        $\psi \gets [PO[p_i].\text{values()}]$\;

        \eIf{$\text{len}(S) > 1$}{
            $G \gets \text{ArraySplit}(\psi, L)$\;
            \For{$k, i \in PO[p_i], \psi$}{
                \For{$g_j \in G$}{
                    \If{$i \in g_j$}{
                        $SD[k] \gets j$\;
                    }
                }
            }
        }{
            // Case of a single SAS in $S$\\
            \eIf{$\psi == 0$}{
                $SD[k] \gets 1$\;
            }{
                $SD[k] \gets L$\;
            }
        }
        $R[p_i] \gets SD$\;
    }
    \Return{$R$}\;
\end{algorithm}



\clearpage
\section{Additional Related Work}
\label{sec:appendix-related-work}
\subsection{Robustness Testing of Foundation Models}
\cite{zhang2022contrastive} examined group distribution shifts and evaluated FMs on image classification tasks with spurious confounders. In our work, we assess the robustness of FMs within time series forecasting by measuring their performance in the presence of two confounders across various perturbation settings and test dataset distributions.
\cite{zhang2023foundation} used foundation models as a surrogate oracle to measure the robustness of image classification models.  However, their test systems did not include any foundation models. \cite{xiao2024ritfis} introduces a framework, RITFIS, to assess the LLM-based software against natural language input. However, they did not consider any other modalities. \cite{schlarmann2023adversarial} showed that imperceivable attacks on images to change the caption output of multi-modal FMs can lead to broadcasting of fake information to honest users. They only evaluate robustness of OpenFlamingo model under different attacks but does not compare its performance with any other FMs. None of these works assess the effectiveness and usability of their robustness testing methods. We address this gap through a user study. 


% In our experiments, we consider perturbations in the input data  to TS models common inspired by omission errors and adversarial attacks, and dependent on both input modality. 
% \cite{chanda2022omission}
% \clearpage

\section{Evaluation Metrics}
\label{sec:appendix-metrics}
% In this section, we define our evaluation metrics. We evaluate the systems across two dimensions: Forecasting accuracy and Robustness.
In this section, we define our evaluation metrics: forecasting accuracy and robustness.
\subsubsection{Forecasting Accuracy Metrics}
We evaluate the systems' forecasting accuracy using established metrics commonly applied in time-series forecasting tasks \cite{makridakis2022m5}.

\noindent {\bf Symmetric mean absolute percentage error (SMAPE) } is defined as, 
    {\tiny
    \begin{equation} \label{eq:smape}
        SMAPE = \frac{1}{T}\sum_{t=1}^T\frac{|x_t - \hat{x}_t|}{(|x_t| +|\hat{x}_t|)/2}, 
    \end{equation}
    }
where $T= 20$ (i.e., the value of $d$) is the total number of observations in the predicted time series. 
% $x_t$ represents the actual observed values, and $\hat{x}_t$ the predicted values at each time step $t= 1, \ldots, 20$. 
SMAPE scores range from 0 to 2, with lower scores indicating more precise forecasts. 

% \noindent {\bf Mean absolute scaled error (MASE)} quantifies the mean absolute error of the forecasts relative to the mean absolute error of a naive one-step forecast calculated on the training data.
\noindent {\bf Mean absolute scaled error (MASE)} measures the mean absolute error of forecasts relative to that of a naive one-step forecast on the training data.
{\small
\begin{equation}\label{eq:mase}
    MASE=\frac{\frac{1}{T} \sum_{i=t+1}^{t+T}|x_{i} - \hat{x_{i}}|}{\frac{1}{t}\sum_{i=1}^{t}|x_{i} - x_{i-1}|},
\end{equation}
}
where in our case, $t = 80$, and $T = 100$.
Lower MASE values indicate better forecasts. 

\noindent {\bf Sign Accuracy} quantifies the average classification accuracy across all test samples, where a higher accuracy indicates more precise predictions. This metric classifies based on how the predicted forecasts align with the most recent observed values in the input time series.

\noindent {\bf Robustness Metrics}
%\subsubsection{Robustness Metrics}
We adapt WRS metric originally proposed in \cite{kausik2024rating} to answer RQ1. Additionally, we introduce two new metrics: APE and PIE \% (modified versions of ATE \cite{abdia2017propensity} and DIE \% \cite{kausik2024rating}) tailored to answering RQ2 and RQ3.

\noindent{\bf Weighted Rejection Score (WRS):} WRS, introduced in \cite{kausik2024rating}, measures statistical bias. First, Student's t-test \cite{student1908probable} compares max residual distributions $(R^{max}_{t} | Z)$ for different values of the protected attribute $Z$. We measure this between each pair of industries or companies, resulting in $^3C_2 = 3$ comparisons. 
% The computed t-value of each pair is compared with the critical t-value based on the confidence interval (CI) and degrees of freedom (DoF), to either reject or accept the null hypothesis. 
\cite{kausik2024rating} chose different confidence intervals (CI) [95\%, 75\%, 60\%] that have different critical values and if the computed t-value is less than the critical value, the null hypothesis is rejected. WRS is mathematically defined by the following equation:
\vspace{-0.5em}
\begin{equation}
\textbf{Weighted Rejection Score (WRS)} = \sum_{i\in CI} w_i*x_i,
\label{eq:wrs}
\vspace{-0.5em}
\end{equation}
\noindent where, $x_i$ is the variable set based on whether the null hypothesis is accepted (0) or rejected (1). $w_i$ is the weight that is multiplied by $x_i$ based on the CI. For example, if CI is 95\%, $x_1$ is multiplied by 1. The lower the CI, the lower the weight will be. WRS helps us answer RQ1 (see Section~\ref{sec:problem}). 


\clearpage
\section{Figures in a Higher Resolution from Main Paper}
\label{sec:appendix-high-res-figs}
\begin{figure}[!h]
    \centering
    \includegraphics[width=0.4\textwidth]{figs/Causal_General.png}
    \caption{Causal model $\mathcal{M}$ for FMTS. The validity of link `1' depends on the data distribution ($P|Z$), while the validity of the links `2' and `3' are tested in our experiments.}
   \label{fig:causal-model-supp}
   \vspace{-2.2em}
\end{figure}

\begin{figure}[!h]
    \centering
    \includegraphics[width=0.4\textwidth]{figs/causal_variants.png}
    \caption{Variants of the causal diagram in Figure \ref{fig:causal-model} used to answer different research questions (RQs).}
   \label{fig:cms-supp}
   \vspace{-1em}
\end{figure}


\begin{figure*}
 \centering
\includegraphics[width=\textwidth]{figs/FMTS_Workflow.png}
\caption{\textbf{Data to predictions}. Workflow for performing statistical and causal analysis to compute raw scores and assign final ratings to the test systems}
\label{fig:system-workflow-supp}
 \end{figure*}
 
 \begin{figure*}
 \centering
 \includegraphics[width=\textwidth]{figs/rating-workflow-v2.png}
 \caption{\textbf{Predictions to ratings}. Black arrows denote the unperturbed and red arrows indicate the perturbed paths. Dashed lines shows the multi-modal path. The perturbed parts of the plots are highlighted in red.}
 \label{fig:rating-workflow-supp}
 \end{figure*}

\begin{figure*}[!h]
\centering
\includegraphics[width=\textwidth]{figs/all_metrics_scatter_plot_v2.png}
\caption{Studying each metric with respect to impact of company and industry as confounders for all models and all perturbations. Plotted in double logarithmic scale, lower values indicate better robustness. Ratings generated by our method (with $L=3$) are shown on the top of each plot. The complete final order (with ratings) are shown in Table \ref{tab:ratings} in Appendix \ref{sec:appendix-experiments}.
}
\label{fig:confs-supp}
\end{figure*}

\begin{figure*}
    \centering
    \includegraphics[width=\textwidth]{figs/Radar_all_v1.png}
    \caption{Radar plots showing for all systems (a) mean forecasting accuracy with respect to all metrics, (b) forecasting accuracy under P2 (half-valued pertubation) Appendix \ref{sec:appendix-experiments}: Figs. \ref{fig:radar-rob}, 
 \ref{fig:radar-acc} show all perturbations; (c) mean robustness metrics for FMTS and $S_a$, and (d) robustness under P2. Each axis is normalized and inverted if needed so that outer ring implies better performance.}
    \label{fig:radar-agg-supp}
\end{figure*}

\begin{figure*}
    \centering
    \includegraphics[width=\textwidth]{figs/Radarchart_Modality_v2.png}
    \caption{Effect of the modalities for $S_g$ (left) and $S_p$ (right).}
    \label{fig:radar_modality-supp}
\end{figure*}

\begin{figure*}[t]
    \centering
    \includegraphics[width=\textwidth]{figs/Radarchart_Arch_v2.png}
    \caption{Role of architecture in forecasting accuracy and robustness. Performance is averaged across models within each category. See Table~\ref{tab:fms}.}
    \label{fig:radar-arch-supp}
\end{figure*}

\clearpage
\section{Additional Experimental Results}
\label{sec:appendix-experiments}
In this section, Table \ref{tab:ratings} shows the partial order and final order with respect to all the metrics defined in Section \ref{sec:metrics}. Table \ref{tab:h4} shows the forecasting accuracy values for all the systems. Table \ref{tab:cases-values} shows the research questions, and average values for all metrics across systems and perturbations (average values are referred to in the conclusions made for each RQ in Section \ref{sec:expts}). Figure \ref{fig:bar-acc} and \ref{fig:bar-rob} shows bar plots with all the robustness metric values and forecasting accuracy values. Figure \ref{fig:heatmaps} shows the heatmap for all metrics for all the models. 

Table \ref{tab:ratings} shows the partial order and final order with respect to all the metrics defined in Section \ref{sec:metrics}. Table \ref{tab:h4} shows the forecasting accuracy values for all the systems. Table \ref{tab:cases-values} shows the research questions, and average values for all metrics across systems and perturbations (average values are referred to in the conclusions made for each RQ in Section \ref{sec:expts}). Figure \ref{fig:bar-acc} and \ref{fig:bar-rob} shows bar plots with all the robustness metric values and forecasting accuracy values. Figure \ref{fig:heatmaps} shows the heatmap for all metrics for all the models. 






\begin{table*}[!ht]
\centering
{\tiny
    \begin{tabular}{|p{5em}|p{0.2em}|p{45em}|p{25em}|}
    \hline
          {\bf Forecasting Evaluation Dimensions} &
          {\bf P} &    
          {\bf Partial Order} &
          {\bf Complete Order} 
          \\ \hline 
          \multirow{3}{6em}{WRS$_I$$\downarrow$} &
          P0 & 
          \{$S_g$: 4.6, S$_m$: 4.6, $S_r$: 4.6, $S_c$: 5.9, $S_a$: 5.9, $S_p^{ni}$: 5.9, $S_g^{ni}$: 6.9, $S_p$: 6.9, $S_b$: 6.9\}   &
          \{$S_g$: 1, S$_m$: 1, $S_r$: 1, $S_c$: 2, $S_a$: 2, $S_p^{ni}$: 2, $S_g^{ni}$: 3, $S_p$: 3, $S_b$: 3\}
          \\ \cline{2-4}
          &
          P1 & 
          \{$S_a$: 2.6, $S_m$: 4.6, $S_g$: 4.6, $S_g^{ni}$: 4.6, $S_r$: 4.6, $S_p^{ni}$: 5.9, $S_p$: 6.9, $S_c$: 6.9, $S_b$: 6.9\}  &
          \{$S_a$: 1, $S_m$: 1, $S_g$: 1, $S_g^{ni}$: 1, $S_r$: 1, $S_p^{ni}$: 2, $S_p$: 3, $S_c$: 3, $S_b$: 3\}
          \\ \cline{2-4}
          &
          P2 & 
          \{$S_a$: 4.6, $S_g$: 4.6, $S_g^{ni}$: 4.6, $S_p^{ni}$: 4.6, $S_m$: 4.6,  $S_r$: 4.6, $S_c$: 6.9, $S_p$: 6.9, $S_b$: 6.9\}   &
          \{$S_a$: 1, $S_g$: 1, $S_g^{ni}$: 1, $S_p^{ni}$: 1, $S_m$: 1,  $S_r$: 1, $S_c$: 2, $S_p$: 2, $S_b$: 2\} 
          \\ \cline{2-4}
          &
          P3 & 
          \{$S_g$: 4.6, $S_g^{ni}$: 4.6, S$_m$: 4.6, $S_r$: 4.6, $S_c$: 4.6, $S_a$: 5.9, $S_p^{ni}$: 6.9, $S_p$: 6.9, $S_b$: 6.9\}   &
          \{$S_g$: 1, $S_g^{ni}$: 1, S$_m$: 1, $S_r$: 1, $S_c$: 1, $S_a$: 2, $S_p$: 3, $S_p^{ni}$: 3, $S_b$: 3\} 
          \\ \hline
          \multirow{3}{6em}{WRS$_C$$\downarrow$} &
          P0 & 
          \{$S_a$: 2.6, S$_g$: 4.6, $S_g^{ni}$: 4.6, $S_p^{ni}$: 4.6, $S_c$: 5.6, $S_p$: 6.9, $S_m$: 6.9, $S_r$: 6.9, $S_b$: 6.9\}    &
          \{$S_a$: 1, S$_g$: 1, $S_g^{ni}$: 1, $S_p^{ni}$: 1, $S_c$: 2, $S_p$: 3, $S_m$: 3, $S_r$: 3, $S_b$: 3\}
          \\ \cline{2-4}
          &
          P1 & 
          \{$S_a$: 0.6, $S_c$: 4.6, $S_p$: 5.9, $S_p^{ni}$: 5.9, $S_r$: 5.9, $S_g^{ni}$: 5.9, $S_g$: 6.9, $S_m$: 6.9, $S_b$: 6.9\}    &
          \{$S_a$: 1, $S_c$: 1, $S_p$: 2, $S_p^{ni}$: 2, $S_r$: 2, $S_g^{ni}$: 2, $S_g$: 3, $S_m$: 3, $S_b$: 3\}
          \\ \cline{2-4}
          &
          P2 & 
          \{$S_a$: 2.6, $S_c$: 4.6, $S_r$: 4.6, $S_p^{ni}$: 4.6, $S_p$: 5.2, $S_g$: 5.9, $S_g^{ni}$: 5.9, $S_m$: 6.9, $S_b$: 6.9\}   &
          \{$S_a$: 1, $S_c$: 1, $S_r$: 1, $S_p^{ni}$: 1, $S_p$: 2, $S_g$: 2, $S_g^{ni}$: 2, $S_m$: 3, $S_b$: 3\}
          \\ \cline{2-4}
          &
          P3 & 
          \{$S_g$: 4.6, $S_g^{ni}$: 4.6, $S_p^{ni}$: 4.6, S$_p$: 4.6, $S_c$: 4.6, $S_m$: 6.9, $S_a$: 6.9, $S_r$: 6.9, $S_b$: 6.9\}   &
          \{$S_g$: 1, $S_g^{ni}$: 1, $S_p^{ni}$: 1, S$_p$: 1, $S_c$: 1, $S_m$: 2, $S_a$: 2, $S_r$: 2, $S_b$: 2\}
          \\ \hline
          \multirow{3}{6em}{PIE$_I$ \%$\downarrow$} &
          P1 & 
          \{$S_g^{ni}$: 124.50, $S_g$: 600.31, $S_r$: 1041.01, $S_p^{ni}$: 1196, $S_m$: 1426.81, $S_c$: 1441.59, $S_p$: 1765.84, $S_a$: 2720.26, $S_b$: 3283.88\} &
          \{$S_g^{ni}$: 1, $S_g$: 1, $S_r$: 1, $S_p^{ni}$: 2, $S_m$: 2, $S_c$: 2, $S_p$: 2, $S_a$: 3, $S_b$: 3\}
          \\ \cline{2-4}
          &
          P2 & 
          \{$S_c$: 357.72, $S_g^{ni}$: 527.76, $S_g$: 597.54, $S_a$: 902.54, $S_m$: 1326.20, $S_r$: 1463.71, $S_p^{ni}$: 1653.53, $S_b$: 2174.39, $S_p$: 2295.68\}  &
          \{$S_c$: 1, $S_g^{ni}$: 1, $S_g$: 1, $S_a$: 2, $S_m$: 2, $S_r$: 2, $S_p^{ni}$: 3, $S_b$: 3, $S_p$: 3\}  
          \\\cline{2-4}
          &
          P3 & 
          \{$S_g$: 703.94, $S_g^{ni}$: 884.34, $S_c$: 911.53, $S_p^{ni}$: 972.95, $S_a$: 1195.04, $S_m$: 2998.25, $S_p$: 3208.04, $S_r$: 3560.94, $S_b$: 7489.48\}   &
          \{$S_g$: 1, $S_g^{ni}$: 1, $S_c$: 1, $S_p^{ni}$: 2, $S_a$: 2, $S_m$: 2, $S_p$: 3, $S_r$: 3, $S_b$: 3\}   
          \\\hline
          \multirow{3}{4em}{PIE$_C$ \%$\downarrow$} &
          P1 & 
          \{$S_c$: 515.91, $S_g$: 663.75, $S_g^{ni}$: 696.44, $S_a$: 982.38, $S_m$: 1028.48, $S_p^{ni}$: 1101.24, $S_p$: 1474.76, $S_r$: 4756.40, $S_b$: 6916.11\}  &
          \{$S_c$: 1, $S_g$: 1, $S_g^{ni}$: 1, $S_a$: 2, $S_m$: 2, $S_p^{ni}$: 2, $S_p$: 3, $S_r$: 3, $S_b$: 3\}
          \\ \cline{2-4}
          &
          P2 & 
          \{$S_g^{ni}$: 469.16, $S_c$: 576.18, $S_g$: 651.07, $S_m$: 1150.45, $S_a$: 1275.04, $S_p^{ni}$: 2238.21, $S_p$: 3257.35, $S_r$: 4274.38, $S_b$: 9474.61\}   &
          \{$S_g^{ni}$: 1, $S_c$: 1, $S_g$: 1, $S_m$: 2, $S_a$: 2, $S_p^{ni}$: 2, $S_p$: 3, $S_r$: 3, $S_b$: 3\} 
          \\ \cline{2-4}
          &
          P3 & 
          \{$S_g^{ni}$: 436.33, $S_g$: 513.47, $S_c$: 650.20, $S_m$: 866.61, $S_r$: 1305.78, $S_a$: 1716.68, $S_b$: 1846.56, $S_p^{ni}$: 2773.74, $S_p$: 4064.03\}  &
          \{$S_g^{ni}$: 1, $S_g$: 1, $S_c$: 1, $S_m$: 2, $S_r$: 2, $S_a$: 2, $S_b$: 3, $S_p^{ni}$: 3, $S_p$: 3\} 
          \\ \hline
          \multirow{3}{6em}{APE$_I$$\downarrow$} &
          P1 & 
          \{$S_g^{ni}$: 2.50, $S_g$: 11.75, $S_c$: 14.69, $S_p^{ni}$: 17.41, $S_m$: 19.94, $S_p$: 26.50, $S_r$: 48.80, $S_a$: 61.87, $S_b$: 101.31\}   &
          \{$S_g^{ni}$: 1, $S_g$: 1, $S_c$: 1, $S_p^{ni}$: 2, $S_m$: 2, $S_p$: 2, $S_r$: 3, $S_a$: 3, $S_b$: 3\} 
          \\ \cline{2-4}
          &
          P2 & 
          \{$S_g$: 3.72, $S_c$: 4.79, $S_g^{ni}$: 6.06, $S_a$: 11.32, $S_m$: 13.36, $S_p^{ni}$: 18.94, $S_p$: 26.02, $S_r$: 42.91, $S_b$: 101.20\}  &
          \{$S_g$: 1, $S_c$: 1, $S_g^{ni}$: 1, $S_a$: 2, $S_m$: 2, $S_p^{ni}$: 2, $S_p$: 3, $S_r$: 3, $S_b$: 3\} 
          \\ \cline{2-4}
          &
          P3 & 
          \{$S_a$: 7.87, $S_g$: 8.40, $S_g^{ni}$: 9.09, $S_c$: 9.50, $S_p^{ni}$: 16.73, $S_m$: 31.36, $S_r$: 36.59, $S_p$: 37.39, $S_b$: 99.72\} &
          \{$S_a$: 1, $S_g$: 1, $S_g^{ni}$: 1, $S_c$: 2, $S_p^{ni}$: 2,  $S_m$: 2, $S_r$: 3, $S_p$: 3, $S_b$: 3\}
          \\ \hline
          \multirow{3}{6em}{APE$_C$$\downarrow$} &
          P1 & 
          \{$S_b$: 0, $S_c$: 6.31, $S_g^{ni}$: 9.49, $S_g$: 10.41, $S_m$: 15.33, $S_r$: 15.36, $S_p^{ni}$: 15.57, $S_p$: 23.99, $S_a$: 59.80\}  &
          \{$S_b$: 1, $S_c$: 1, $S_g^{ni}$: 1, $S_g$: 2, $S_m$: 2, $S_r$: 2, $S_p^{ni}$: 3, $S_p$: 3, $S_a$: 3\} 
          \\ \cline{2-4}
          &
          P2 & 
          \{$S_b$: 0, $S_g^{ni}$: 5.31, $S_c$: 6.42, $S_g$: 8.69, $S_p$: 10.81, $S_m$: 13.92, $S_r$: 17.61, $S_a$: 21.39, $S_p^{ni}$: 27.63\}   &
          \{$S_b$: 1, $S_g^{ni}$: 1, $S_c$: 1, $S_g$: 2, $S_p$: 2, $S_m$: 2, $S_r$: 3, $S_a$: 3, $S_p^{ni}$: 3\} 
          \\ \cline{2-4}
          &
          P3 & 
          \{$S_b$: 0, $S_g^{ni}$: 6.48, $S_g$: 7.06, $S_a$: 7.42, $S_c$: 8.80, $S_m$: 10.87, $S_r$: 16.63, $S_p^{ni}$: 35.35, $S_p$: 46.50 \}    &
          \{$S_b$: 1, $S_g^{ni}$: 1, $S_g$: 1, $S_a$: 2, $S_c$: 2,$S_m$: 2, $S_r$: 3, $S_p^{ni}$: 3, $S_p$: 3 \} 
          \\ \hline
          \multirow{3}{6em}{SMAPE$\downarrow$} &
          P0 & 
          \{$S_a$: 0.040, $S_c$: 0.043, $S_g$: 0.049, $S_p^{ni}$: 0.079, $S_p$: 0.095, $S_g^{ni}$: 0.095, $S_m$: 0.097, $S_r$: 0.829, $S_b$: 1.276 \} &
          \{$S_a$: 1, $S_c$: 1, $S_g$: 1, $S_p^{ni}$: 2, $S_p$: 2, $S_g^{ni}$: 2, $S_m$: 2, $S_r$: 3, $S_b$: 3 \} 
          \\ \cline{2-4}
          &
          P1 & 
          \{$S_c$: 0.065, $S_g^{ni}$: 0.067, $S_g$: 0.072, $S_a$: 0.084, $S_m$: 0.100, $S_p$: 0.100, $S_p^{ni}$: 0.100, $S_r$: 0.830, $S_b$: 1.276 \} &
          \{$S_c$: 1, $S_g^{ni}$: 1, $S_g$: 1, $S_a$: 2, $S_m$: 2, $S_p$: 2, $S_p^{ni}$: 2, $S_r$: 3, $S_b$: 3 \}
          \\ \cline{2-4}
          &
          P2 & 
          \{$S_g$: 0.051, $S_c$: 0.053, $S_g^{ni}$: 0.060, $S_a$: 0.069, $S_p^{ni}$: 0.095, $S_m$: 0.098, $S_p$: 0.100, $S_r$: 0.830, $S_b$: 1.276 \}   &
          \{$S_g$: 1, $S_c$: 1, $S_g^{ni}$: 1, $S_a$: 2, $S_p^{ni}$: 2, $S_m$: 2, $S_p$: 3, $S_r$: 3, $S_b$: 3 \} 
          \\ \cline{2-4}
          &
          P3 & 
          \{$S_a$: 0.040, $S_c$: 0.043, $S_g$: 0.049, $S_g^{ni}$: 0.056, $S_p^{ni}$: 0.078, $S_p$: 0.092, $S_m$: 0.097, $S_r$: 0.830, $S_b$: 1.276 \}   &
          \{$S_a$: 1, $S_c$: 1, $S_g$: 1, $S_g^{ni}$: 2, $S_p^{ni}$: 2, $S_p$: 2, $S_m$: 3, $S_r$: 3, $S_b$: 3 \} 
          \\ \hline
          \multirow{4}{6em}{MASE$\downarrow$} &
          P0 & 
          \{$S_a$: 3.79, $S_c$: 4.18, $S_g$: 4.64, $S_p^{ni}$: 7.19, $S_p$: 8.91, $S_m$: 9.03, $S_g^{ni}$: 10.37, $S_r$: 86.45, $S_b$: 947.56 \}    &
          \{$S_a$: 1, $S_c$: 1, $S_g$: 1, $S_p^{ni}$: 2, $S_p$: 2, $S_m$: 2, $S_g^{ni}$: 3, $S_r$: 3, $S_b$: 3 \} 
          \\ \cline{2-4}
          &
          P1 & 
          \{$S_c$: 5.40, $S_g^{ni}$: 5.65, $S_g$: 6.13, $S_p^{ni}$: 8.87, $S_p$: 9.19, $S_m$: 9.32, $S_a$: 18.36, $S_r$: 86.99, $S_b$: 947.56 \}   & 
          \{$S_c$: 1, $S_g^{ni}$: 1, $S_g$: 1, $S_p^{ni}$: 2, $S_p$: 2, $S_m$: 2, $S_a$: 3, $S_r$: 3, $S_b$: 3 \} 
          \\ \cline{2-4}
          &
          P2 & 
          \{$S_g$: 4.74, $S_c$: 4.99, $S_g^{ni}$: 5.59, $S_a$: 8.24, $S_p^{ni}$: 8.49, $S_m$: 9.15, $S_p$: 9.32, $S_r$: 86.87, $S_b$: 947.56 \}    &
          \{$S_g$: 1, $S_c$: 1, $S_g^{ni}$: 1, $S_a$: 2, $S_p^{ni}$: 2, $S_m$: 2, $S_p$: 3, $S_r$: 3, $S_b$: 3 \}
          \\ \cline{2-4}
          &
          P3 & 
          \{$S_a$: 3.79, $S_c$: 4.10, $S_g$: 4.64, $S_g^{ni}$: 5.39, $S_p^{ni}$: 7.11, $S_p$: 8.68, $S_m$: 9.03,  $S_r$: 86.65, $S_b$: 947.56 \} &
          \{$S_a$: 1, $S_c$: 1, $S_g$: 1, $S_g^{ni}$: 2, $S_p^{ni}$: 2, $S_p$: 2, $S_m$: 3,  $S_r$: 3, $S_b$: 3 \} 
          \\ \hline
          \multirow{3}{6em}{Sign Accuracy \%$\uparrow$} &
          P0 & 
          \{$S_m$: 40.70, $S_p$: 45.09, $S_p^{ni}$: 47.67, $S_r$: 49.88, $S_g^{ni}$: 50.41, $S_g$: 52.08, $S_c$: 53.75, $S_a$: 60.08, $S_b$: 62.60 \} &  
          \{$S_m$: 1, $S_p$: 1, $S_p^{ni}$: 1, $S_r$: 2, $S_g^{ni}$: 2, $S_g$: 2, $S_c$: 3, $S_a$: 3, $S_b$: 3 \}
          \\ \cline{2-4}
          &
          P1 & 
          \{$S_m$: 41.19, $S_p$: 44.33, $S_p^{ni}$: 46.77, $S_r$: 49.62, $S_g$: 50.53, $S_c$: 52.09, $S_g^{ni}$: 53.93, $S_a$: 57.08, $S_b$: 62.60 \} &
          \{$S_m$: 1, $S_p$: 1, $S_p^{ni}$: 1, $S_r$: 2, $S_g$: 2, $S_c$: 2,  $S_g^{ni}$: 3, $S_a$: 3, $S_b$: 3 \}
          \\ \cline{2-4}
          &
          P2 & 
          \{$S_m$: 41.05, $S_p$: 44.02, $S_p^{ni}$: 47.67, $S_r$: 49.64, $S_g$: 49.75, $S_c$: 50.79, $S_g^{ni}$: 54.43, $S_a$: 57.13, $S_b$: 62.60 \}    &
          \{$S_m$: 1, $S_p$: 1, $S_p^{ni}$: 1, $S_r$: 2, $S_g$: 2, $S_c$: 2, $S_g^{ni}$: 3, $S_a$: 3, $S_b$: 3 \} 
          \\ \cline{2-4}
          &
          P3 & 
          \{$S_m$: 40.72, $S_p$: 44.26, $S_p^{ni}$: 47.50, $S_r$: 49.71, $S_g$: 51.34, $S_c$: 51.35, $S_g^{ni}$: 52.97, $S_a$: 59.98, $S_b$: 62.60 \}  &
          \{$S_m$: 1, $S_p$: 1, $S_p^{ni}$: 1, $S_r$: 2, $S_g$: 2, $S_c$: 2, $S_g^{ni}$: 3, $S_a$: 3, $S_b$: 3 \}
          \\ \hline
    \end{tabular}
    }
    \caption{Final raw scores and ratings based on different metrics computed. Higher ratings indicate higher bias for WRS and PIE \%, higher disruption for APE, greater inaccuracy for MASE and SMAPE, and higher accuracy for Sign Accuracy. For simplicity, we denoted the raw scores for accuracy metrics using just the mean value, but standard deviation was also considered for rating. The chosen rating level, L = 3. Overall, across all the settings, system $S_p$ exhibited statistical bias in 50 \% of cases, confounding bias in 100 \% of cases, and disruptive behavior in 50 \% of the cases based on APE values.}
    \label{tab:ratings}
\end{table*}



\begin{table*}
\centering
    \begin{tabular}{|p{5em}|p{1em}|p{3em}|p{3em}|p{3em}|p{3em}|p{3em}|p{3em}|p{3em}|p{3em}|p{3em}|}
    \hline
          {\bf Metric} &
          {\bf P} &    
          {\bf $S_{g}$} &
          {\bf $S_g^{ni}$} &
          {\bf $S_{p}$} &
          {\bf $S_p^{ni}$} &
          {\bf $S_{m}$} &
          {\bf $S_{c}$} &
          {\bf $S_a$} & 
          {\bf $S_b$} &
          {\bf $S_r$}
          \\ \hline 
          \multirow{6}{4em}{SMAPE$\downarrow$} & 
          P0 &
           0.049 $\pm$ 0.047  &
           0.095 $\pm$ 0.103  &
           0.095 $\pm$ 0.075 &
           0.079 $\pm$ 0.081 &
           0.097 $\pm$ 0.072 &
           0.043 $\pm$ 0.054 &
           \textbf{0.040 $\pm$ 0.037} &
          \multirow{6}{1.5em}{1.276 $\pm$ 0.663} &
          0.829 $\pm$ 0.638
          \\ \cline{2-9}
            \cline{11-11}
          % -------
            & 
          P1 &
          0.072 $\pm$ 0.123 &
          0.067 $\pm$ 0.178 &
          0.100 $\pm$ 0.125 &
          0.100 $\pm$ 0.143 &
          0.100 $\pm$ 0.076 &
          \textbf{0.065 $\pm$ 0.189} &
          0.084 $\pm$ 0.282 &
          &
          0.830 $\pm$ 0.639
          \\ \cline{2-9}
            \cline{11-11}
          % -------
            & 
          P2 &
          \textbf{0.051 $\pm$ 0.047} &
          0.060 $\pm$ 0.085 &
          0.100 $\pm$ 0.088 &
          0.095 $\pm$ 0.097 &
          0.098 $\pm$ 0.074 &
          0.053 $\pm$ 0.092 &
          0.069 $\pm$ 0.217 &
          &
          0.830 $\pm$ 0.639
          \\ \cline{2-9}
            \cline{11-11}
          % -------
            & 
          P3 &
          0.049 $\pm$ 0.045 &
          0.056 $\pm$ 0.052  &
          0.092 $\pm$ 0.074 &
          0.078 $\pm$ 0.078 &
          0.097 $\pm$ 0.071 &
          0.043 $\pm$ 0.048 &
          \textbf{0.040 $\pm$ 0.037} &
           &
          0.830 $\pm$ 0.640
          \\ \hline
          % -------

          \multirow{6}{4em}{MASE$\downarrow$} & 
          P0 &
          4.64 $\pm$ 4.62 &
          10.37 $\pm$ 13.63 &
          8.91 $\pm$ 7.01 &
          7.19 $\pm$ 6.94 &
          9.03 $\pm$ 6.91 &
          4.18 $\pm$ 7.75 &
          \textbf{3.79 $\pm$ 3.59} &
          \multirow{6}{1em}{947.56 $\pm$ 767.65} &
          86.45 $\pm$ 72.72
          \\ \cline{2-9}
            \cline{11-11}
          % -------
             & 
          P1 &
          6.13 $\pm$ 8.31 &
          5.65 $\pm$ 10.23 &
          9.19 $\pm$ 8.61 &
          8.87 $\pm$ 8.94 &
          9.32 $\pm$ 7.39 &
          \textbf{5.40 $\pm$ 12.45}&
          18.36 $\pm$ 168.82 &
          &
          86.99 $\pm$ 73.53 
          \\ \cline{2-9}
            \cline{11-11}
          % -------
             & 
          P2 &
          \textbf{4.74 $\pm$ 4.53} &
          5.59 $\pm$ 8.19 &
          9.32 $\pm$ 7.94 &
          8.49 $\pm$ 7.94 &
          9.15 $\pm$ 7.15 &
          4.99 $\pm$ 9.90 &
          8.24 $\pm$ 48.58 &
          &
          86.87 $\pm$ 73.32 
          \\ \cline{2-9}
            \cline{11-11}
             & 
          P3 &
          4.64 $\pm$ 4.42 &
          5.39 $\pm$ 5.27 &
          8.68 $\pm$ 7.18 &
          7.11 $\pm$ 6.75 &
          9.03 $\pm$ 6.90 &
          4.10 $\pm$ 6.33 &
          \textbf{3.79 $\pm$ 3.57} &
             &
          86.65 $\pm$ 73.11 
          \\ \hline
          % -------
          \multirow{4}{4em}{Sign Accuracy (\%)$\uparrow$}   & 
          P0 &
          52.08 &
          50.41 &
          45.09 &
          47.67 &
          40.70 &
          53.75 &
          \textbf{60.08} &
          \multirow{6}{1em}{62.60} &
          49.88
          \\ \cline{2-9}
            \cline{11-11}
             & 
          P1 &
          50.53 &
          53.93 &
          44.33 &
          46.77 &
          41.19 &
          52.09 &
          \textbf{57.08} &
          &
          49.62
          \\ \cline{2-9}
            \cline{11-11}
             & 
          P2 &
          49.75 &
          54.43 &
          44.02 &
          47.67 &
          41.05 &
          50.79 &
          \textbf{57.13} &
          &
          49.64
          \\ \cline{2-9}
            \cline{11-11}
              & 
          P3 &
          51.34 &
          52.97 &
          44.26 &
          47.50 &
          40.72 &
          51.35 &
          \textbf{59.98} &
           &
          49.71
          \\ \hline
    \end{tabular}
    %}
    \caption{Performance metrics for test systems across different perturbations. SMAPE and MASE scores are reported as mean $\pm$ standard deviation.}
    \label{tab:h4}
\end{table*}

\begin{table*}[ht]
\centering
   {\small
    \begin{tabular}{|p{8em}|p{8em}|p{3em}|p{12em}|p{9em}|p{8em}|}
    \hline
          {\bf Research Question} &    
          {\bf Causal Diagram} &
          {\bf Metrics Used} &
          {\bf Comparison across Systems} &
          {\bf Comparison across Perturbations} &
          {\bf Key Conclusions} \\ \hline 
          % 
          \textbf{RQ1:} Does $Z$ affect $R^{max}_{t}$, even though $Z$ has no effect on $P$? & 
          \begin{minipage}{.05\textwidth}
          \vspace{2.5mm}
          \centering
          \includegraphics[width=25mm, height=13mm]{figs/Causal_H1.png} 
          \end{minipage} &
          WRS &
          \{\textcolor{green}{$S_a$: 3.96}, $S_g$: 5.05, $S_g^{ni}$: 5.21, $S_r$: 5.34, $S_p^{ni}$: 5.38, $S_c$: 5.46, $S_m$: 5.75, \textcolor{red}{$S_p$: 6.28}, $S_b$: 6.9\} &
          \{\textcolor{green}{P2: 5.18}, P1: 5.2, P3: 5.35, \textcolor{red}{P0: 5.46}\} &
          \textbf{\textit{S} with low statistical bias}: $S_a$. 
          \textbf{\textit{S} with high statistical bias}: $S_p$.
          \textbf{\textit{P} that led to more statistical bias}: P0
          \textbf{Analysis with more discrepancy}: Inter-industry
          \\ \hline 
          % -------
          \textbf{RQ2:} Does $Z$ affect the relationship between $P$ and $R^{max}_{t}$ when $Z$ has an effect on $P$? & 
          \begin{minipage}{.05\textwidth}
          \vspace{2.5mm}
          \centering
          \includegraphics[width=25mm, height=13mm]{figs/Causal_H2.png}
          \end{minipage} &
          PIE \% &
          \{\textcolor{green}{$S_g^{ni}$: 523.09}, $S_g$: 621.68, $S_c$: 742.19, $S_a$: 1206.44, $S_m$: 1466.13, $S_p^{ni}$: 1655.94, \textcolor{red}{$S_p$: 2677.62}, $S_r$: 2733.7, $S_b$: 5197.51\} &
          \{\textcolor{green}{P1: 995.19}, P2: 1252.2, \textcolor{red}{P3: 1563.94}\} &
          \textbf{\textit{S} with low confounding bias}: $S_g^{ni}$. 
          \textbf{\textit{S} with high confounding bias}: $S_p$. 
          \textbf{\textit{P} that led to more confounding bias}: P3.  
          \textbf{Confounder that led to more bias}: \textit{Industry}
          \\ \hline 
          % -------
          \textbf{RQ3:} Does $P$ affect $R^{max}_{t}$ when $Z$ may have an effect on $R^{max}_{t}$? & 
          \begin{minipage}{.05\textwidth}
          \vspace{2.5mm}
          \centering
          \includegraphics[width=25mm, height=13mm]{figs/Causal_H3.png}
          \end{minipage} &
          APE &
          \{\textcolor{green}{$S_g^{ni}$: 6.49}, $S_g$: 8.34, $S_c$: 8.42, $S_m$: 17.46, $S_p^{ni}$: 21.94, $S_a$: 28.28, \textcolor{red}{$S_p$: 28.53}, $S_r$: 29.65, $S_b$: 50.37\} &
          \{\textcolor{green}{P2: 12.74}, P3: 17.34, \textcolor{red}{P1: 21.11}\} &
          \textbf{\textit{S} with low APE}: $S_g^{ni}$.
          \textbf{\textit{S} with high APE}: $S_p$.
          \textbf{\textit{P} with low APE}: P2.
          \textbf{\textit{P} with high APE}: P1.
          \textbf{Confounder that led to high APE}: \textit{Company}
          \\ \hline 
          % -------
          \textbf{RQ4:} Does $P$ affect the accuracy of $S$? & 
          This hypothesis does not necessitate a causal model for its evaluation. &
          SMAPE, MASE, Sign Accuracy &
          \textbf{SMAPE}: \{\textcolor{green}{$S_c$: 0.05}, $S_a$: 0.06, $S_g$: 0.06, $S_g^{ni}$: 0.07, $S_p^{ni}$: 0.09, $S_p$: 0.1, \textcolor{red}{$S_m$: 0.1}, $S_r$: 0.83, $S_b$: 1.28\}; 

          
          \textbf{MASE}: \{\textcolor{green}{$S_c$: 4.67}, $S_g$: 5.04, $S_g^{ni}$: 6.75, $S_p^{ni}$: 7.91, $S_a$: 8.54, $S_p$: 9.03, \textcolor{red}{$S_m$: 9.13}, $S_r$: 86.74, $S_b$: 947.56\};  

          
          \textbf{Sign Accuracy}: \{\textcolor{red}{$S_m$: 40.91}, $S_p$: 44.42, $S_p^{ni}$: 47.4, $S_r$: 49.71, $S_g$: 50.93, $S_c$: 51.99, $S_g^{ni}$: 52.94, \textcolor{green}{$S_a$: 58.57}, $S_b$: 62.6\}&
          \textbf{SMAPE}: \{\textcolor{green}{P3: 0.06}, P0: 0.07, P1: 0.08, \textcolor{red}{P2: 0.08}\}; 

          
          \textbf{MASE}: \{\textcolor{green}{P3: 6.11}, P0: 6.87, P2: 7.22, \textcolor{red}{P1: 8.99}\}; 

          
          \textbf{Sign Accuracy}: \{\textcolor{red}{P2: 49.26}, P1: 49.42, P3: 49.73, \textcolor{green}{P0: 49.97}\}&
          \textbf{\textit{S} with good performance}: $S_c$. 
          \textbf{\textit{S} with poor performance}: $S_m$. 
          \textbf{\textit{P} with high impact on performance}: P2.  
          \\ \hline 
    \end{tabular}
    }
    \caption{Summary of the research questions answered in the paper, causal diagram, metrics used in the experiment, average of the metric values compared across different systems, average computed across different perturbations, and the key conclusions drawn from the experiment. \textbf{Overall, multi-modal FMTS demonstrated greater robustness and forecasting accuracy compared to multi-modal FMTS. TS FMTS demonstrated greater robustness and forecasting accuracy compared to GP FMTS}. All the raw scores and ratings are shown in Table \ref{tab:ratings}.}
    \label{tab:cases-values}
\end{table*}


 \begin{figure*}[h]
  \centering
  \begin{subfigure}{\textwidth}
  \centering
  \includegraphics[width=.8\textwidth]{plots/smape.png} 
 \caption{SMAPE}
 \label{fig:bar-smape}
  \end{subfigure}
  \begin{subfigure}{\textwidth}
  \centering
  \includegraphics[width=.8\textwidth]{plots/mase.png}
 \caption{MASE}
 \label{fig:bar-mase}
  \end{subfigure}
  \begin{subfigure}{\textwidth}
  \centering
  \includegraphics[width=.8\textwidth]{plots/SignAcc.png}
  \caption{Sign Accuracy}
  \label{fig:bar-sign-acc}
  \end{subfigure}
  \caption{Bar plots showing the robustness metrics values across different systems and perturbations.}
 \label{fig:bar-acc}
\end{figure*}


 \begin{figure*}[h]
  \centering
  \begin{subfigure}{\textwidth}
  \centering
  \includegraphics[width=.8\textwidth]{plots/wrs.png} 
 \caption{WRS}
 \label{fig:bar-wrs}
  \end{subfigure}
  \begin{subfigure}{\textwidth}
  \centering
  \includegraphics[width=.8\textwidth]{plots/pie.png}
 \caption{PIE \% scores}
 \label{fig:bar-pie}
  \end{subfigure}
  \begin{subfigure}{\textwidth}
  \centering
  \includegraphics[width=.8\textwidth]{plots/ape.png}
  \caption{APE scores}
  \label{fig:bar-ape}
  \end{subfigure}
  \caption{Bar plots showing the robustness metrics values across different systems and perturbations.}
 \label{fig:bar-rob}
\end{figure*}

\begin{figure*}
 \centering
\includegraphics[width=1.1\textwidth]{plots/Radar_Rob.png}
\caption{Radar plots showing robustness metrics for all FMTS and $S_a$ under different perturbations. }
\label{fig:radar-rob}
\end{figure*}

\begin{figure*}[b]
     \centering
     \includegraphics[width=1.1\textwidth]{plots/Radar_Acc.png}
     \caption{Radar plots showing forecasting accuracy metrics for all systems under different perturbations.}
     \label{fig:radar-acc}
\end{figure*}



\begin{figure*}
    \centering
    \includegraphics[width=\linewidth]{figs/heatmap.png}
    \caption{Heatmap for all metrics for all models. Lighter shade indicates better performance.}
    \label{fig:heatmaps}
\end{figure*}

\clearpage
\section{Additional User Study Results}
\label{sec:appendix-user-study}
In this section, we present all the hypotheses, the results from the statistical tests conducted to validate these hypotheses, and the conclusions drawn from the results. 


\begin{table*}[h!]
\centering
\begin{tabular}{|p{2.5em}|p{2.5em}|p{2.5em}|p{2.5em}|c|c|c|c|c|c|c|c|}
\hline
\textbf{Metric} & 
\textbf{Q1} & 
\textbf{Q2} & 
\textbf{Q4} & 
\textbf{Q5} & 
\textbf{Q6} & 
\textbf{Q8} & 
\textbf{Q9} & 
\textbf{Q10} & 
\textbf{Q12} &
\textbf{Q13} &
\textbf{Q14} 
\\ \hline
$\mu$ & 
3.1923 & 
2.8077 & 
2.5385 & 
2.7692 & 
2.9231 & 
2.6923 & 
2.9231 & 
3.2308 & 
2.6538 & 
2.8077 &
3.0769 \\ \hline
$\sigma$ & 
1.2335 & 
1.3570 & 
1.3336 & 
1.1767 & 
1.3834 & 
1.0870 & 
1.2625 & 
1.4507 & 
1.1981 & 
1.3570 & 
1.4676 \\ \hline
t-statistic & 4.9287 & 3.0349 & 2.0588 & 3.3333 & 3.4023 & 3.2476 & 3.7282 & 4.3259 & 2.7828 & 3.0349 & 3.7417 \\ \hline
p-value & 0.0000$^*$ & 0.0028$^*$ & 0.0250$^*$ & 0.0013$^*$ & 0.0011$^*$ & 0.0017$^*$ & 0.0005$^*$ & 0.0001$^*$ & 0.0051$^*$ & 0.0028$^*$ & 0.0005$^*$ \\ \hline
\end{tabular}
\caption{Summary of one sample right-tailed t-test results: Comparison of sample means to the hypothesized mean of 2 with a sample size of 26. The right-tailed p-values indicate whether the sample means are significantly greater the hypothesized mean. $^*$ denotes that mean of responses for all the questions is greater than 2.}
\label{tab:user-study-sanity}
\end{table*}


\begin{table*}[!ht]
\centering
    \begin{tabular}{|p{6cm}|p{2cm}|p{2cm}|p{6cm}|}
    \hline
        \textbf{Hypothesis} & 
        \textbf{Test Performed}  &
        \textbf{Statistics} &
        \textbf{Conclusion}\\
        \hline
         There is a high positive correlation between users' fairness rankings and rankings generated by our rating method.  & 
         Spearman Rank Correlation &
         $\rho = 0.73$ &
         The fairness rankings generated by our rating method aligns well with users' rankings. 
         \\ \hline
         The mean of the responses for Q4 is less than or equal to the mean of the responses for Q6. & 
         Paired t-test &
         t-statistic: -1.18, p-val: 0.12 &
         Users found it easy to interpret the behavior of the systems from rankings compared to graphs and statistics with a confidence interval of 85 \%.
         \\ \hline
         There is a very high positive correlation between users' rankings and rankings generated by our rating method. & 
         Spearman Rank Correlation &
         $\rho$: 0.91 &
         The robustness rankings generated by our rating method aligns very well with users' rankings.
         \\ \hline
         The mean of the responses for Q8 is less than or equal to the mean of the responses for Q10. & 
         Paired t-test &
         t-statistic: -1.89, p-val: 0.03 &
         Users found it easy to interpret the behavior of the systems from rankings compared to graphs and statistics with a confidence interval of 95 \%.
         \\  \hline
         There is a weak positive correlation between users' rankings and rankings generated by our rating method. & 
         Spearman Rank Correlation &
         $\rho$: 0.14 &
         The robustness rankings generated by our rating method weakly aligns with users' rankings.
         \\ \hline
         The mean of the responses for Q12 is less than or equal to the mean of the responses for Q14. & 
         Paired t-test &
         t-statistic: -1.62, p-val: 0.06 &
         Users found it easy to interpret the behavior of the systems from rankings compared to graphs and statistics with a confidence interval of 90 \%.
         \\  \hline
    \end{tabular}
    \caption{Table with the hypotheses evaluated in the user study, statistical tests used to validate the hypotheses, results obtained, and conclusions drawn.}
    \label{tab:user-study-results}
\end{table*}



% \begin{table}[h!]
% {\tiny
%     \centering
%     \begin{tabular}{|m{4.8cm}|m{0.6cm}|m{0.5cm}|m{0.7cm}|}
%         \hline
%         \textbf{Question} & \textbf{$\bar{x}$} & \textbf{$t$} & \textbf{p-value} \\
%         \hline
%         Familiarity with time-series & 3.1818 & 4.1608 & 0.0002* \\
%         \hline
%         Familiarity with financial tasks & 2.5909 & 2.2004 & 0.0196* \\
%         \hline
%         Ease of interpreting behavior through graphs in fairness study & 2.4545 & 1.5554 & 0.0674 \\
%         \hline
%         Rating accuracy in fairness study & 2.7273 & 2.7478 & 0.0060* \\
%         \hline
%         Ease of interpreting behavior through ratings & 2.8636 & 2.8444 & 0.0049* \\
%         \hline
%         Ease of interpreting behavior through graphs in robustness study 1 & 2.5455 & 2.4208 & 0.0123* \\
%         \hline
%         Rating accuracy in robustness study 1 & 2.6818 & 2.6418 & 0.0076* \\
%         \hline
%         Ease of interpreting behavior through ratings & 3.0000 & 3.2404 & 0.0020* \\
%         \hline
%         Ease of interpreting behavior through graphs in robustness study 2 & 2.4545 & 1.8002 & 0.0431* \\
%         \hline
%         Rating accuracy in robustness study 2 & 2.6364 & 2.1877 & 0.0201* \\
%         \hline
%         Ease of interpreting behavior through ratings & 2.9091 & 2.8257 & 0.0051* \\
%         \hline
%     \end{tabular}
%     \caption{Table showing the questions, sample mean, t-statistic and p-value from t-test computed for user responses (on a scale of 1-5) to the study. The hypothesized mean for all questions is 2. Significant p-values (p $<$ 0.05) are marked with an asterisk (*)\kl{Add Q3, Q7, and Q11 as well.}\kl{Move to supplementary}\kl{Hypothesis, correlation value, implications.}\biplav{Update numbers. All are significant now.}}
%     \label{tab:user-study}
% }
% \end{table}


% --- 
% \clearpage
% \section{User Study Questionnaire}
% \label{sec:appendix-study-q}
% Below is the user study form (questionnaire) that was circulated to collect responses for the user study.
% \includepdf[pages=-]{UserStudyForm.pdf}

\clearpage
\section{Source Code for Data Processing}
\label{sec:appendix-source-code}
\begin{lstlisting}
# Convert data from Yahoo! finance to sliding window format.
def sliding_window(data, window_size, company):
    sequences = []
    for i in range(len(data) - window_size):
        seq = data[i:(i + window_size + 1)].tolist()
        sequences.append([company] + seq)
    return pd.DataFrame(sequences)

# Perturbations:
# Drop-to-zero: Every 80th stock price in the numerical data will be turned into zero.
def drop_to_zero(df, col):

  new_df = df.copy()
  new_df.loc[new_df.index % 80 == 0, col] = 0

  return new_df

# Value halved: Every 80th stock price in the numerical data will be halved.
def value_halved(df, col):

  new_df = df.copy()
  new_df.loc[new_df.index % 80 == 0, col] /= 2

  return new_df

# Missing values: Every 80th stock price in the numerical data will be 'NaN'.
def missing_values(df, col):

  new_df = df.copy()
  new_df.loc[new_df.index % 80 == 0, col] = float('nan')

  return new_df

# Code to generate time series line plots.
def plot_ts(input_path, output_path):
    data = pd.read_csv(input_path)

    companies = data.iloc[:, 0]
    time_steps = data.iloc[:, 1:]

    for i, company in enumerate(companies):
        plt.figure(figsize=(12, 6))
        plt.plot(time_steps.columns, time_steps.iloc[i], marker='o')
        plt.title(f'Time Series for {company}', fontsize=19)
        plt.xlabel('Time Steps', fontsize=17)
        plt.ylabel('Values', fontsize=17)
        plt.grid(True)
        x_ticks = time_steps.columns[::5]
        plt.xticks(x_ticks, rotation=45, fontsize=15)
        plt.yticks(fontsize=15)
        plt.tight_layout()
        plt.savefig(os.path.join(output_path, f'sample_{i+1}_time_series.png'))
        plt.close()
\end{lstlisting}


\clearpage
\section{Additional Implementation Details}
\label{sec:appendix-implementation-details}
All Forecasting Model Training Systems (FMTS) were executed on Colab notebooks utilizing the L4 GPU available through Colab Pro, which offers 22.5 GB of GPU RAM. Additional details regarding the models such as the inference times and other architectural details can be found in Section \ref{sec:systems}.

\subsubsection{Hyperparameters set}
\begin{itemize}
    \item \textbf{MOMENT}: head\_dropout: 0.1, weight\_decay: 0, freeze\_encoder: True, freeze\_embedder: True, freeze\_head: False
    \item \textbf{Phi-3}: \_attn\_implementation='eager', max\_new\_tokens: 300, temperature: 0.0, do\_sample: False
    \item \textbf{Gemini}: Temperature: 0. Rest of the parameters were default.
\end{itemize}

        
\section{Reproducibility Checklist}
\label{sec:appx-reproduc}
This paper:

\begin{enumerate}
    \item Includes a conceptual outline and/or pseudocode description of AI methods introduced 


    \textbf{Answer}: Yes (Appendix \ref{sec:appendix-algo-details}).

    \item Clearly delineates statements that are opinions, hypothesis, and speculation from objective facts and results 

    \textbf{Answer}: Yes (Section \ref{sec:expts})

    \item Provides well marked pedagogical references for less-familiar readers to gain background necessary to replicate the paper 

    \textbf{Answer}: Yes (Sections \ref{sec:introduction}, \ref{sec:related-work})
    
    
    Does this paper make theoretical contributions? (yes/no)

    \textbf{Answer}: No

% If yes, please complete the list below.

% All assumptions and restrictions are stated clearly and formally. (yes/partial/no)
% All novel claims are stated formally (e.g., in theorem statements). (yes/partial/no)
% Proofs of all novel claims are included. (yes/partial/no)
% Proof sketches or intuitions are given for complex and/or novel results. (yes/partial/no)
% Appropriate citations to theoretical tools used are given. (yes/partial/no)
% All theoretical claims are demonstrated empirically to hold. (yes/partial/no/NA)
% All experimental code used to eliminate or disprove claims is included. (yes/no/NA)

    \item Does this paper rely on one or more datasets? 

    \textbf{Answer}: Yes (Description in Section \ref{sec:exp_app}

    \item A motivation is given for why the experiments are conducted on the selected datasets 

    \textbf{Answer}: Yes (Sections \ref{sec:introduction}, \ref{sec:related-work})

    \item All novel datasets introduced in this paper are included in a data appendix. 

    \textbf{Answer}: NA (We used an existing dataset from Yahoo! Finance.)

    
    \item All novel datasets introduced in this paper will be made publicly available upon publication of the paper with a license that allows free usage for research purposes. 

    \textbf{Answer}: NA

    \item All datasets drawn from the existing literature (potentially including authors’ own previously published work) are accompanied by appropriate citations. 

    \textbf{Answer}: NA
    
    \item All datasets drawn from the existing literature (potentially including authors’ own previously published work) are publicly available. 

    \textbf{Answer}: Yes (Section \ref{sec:exp_app})

    
    \item All datasets that are not publicly available are described in detail, with explanation why publicly available alternatives are not scientifically satisfying. 

    \textbf{Answer}: NA 

    \item Does this paper include computational experiments? 

    \textbf{Answer}: Yes

    \item Any code required for pre-processing data is included in the appendix.

    \textbf{Answer}: Yes, code required to convert the data downloaded from Yahoo! finance to sliding window, apply perturbations, and generate time series plots are provided in Appendix \ref{sec:appendix-source-code}.
    
    \item All source code required for conducting and analyzing the experiments is included in a code appendix.

    \textbf{Answer}: Yes. You can find the source code and data here: \url{https://anonymous.4open.science/r/rating-fmts-1B30/README.md}

    \item All source code required for conducting and analyzing the experiments will be made publicly available upon publication of the paper with a license that allows free usage for research purposes. 

    \textbf{Answer}: Yes

    \item All source code implementing new methods have comments detailing the implementation, with references to the paper where each step comes from 

    \textbf{Answer}: Yes
    
    \item If an algorithm depends on randomness, then the method used for setting seeds is described in a way sufficient to allow replication of results. 

    \textbf{Answer}: Yes (provided in the source code).

    \item This paper specifies the computing infrastructure used for running experiments (hardware and software), including GPU/CPU models; amount of memory; operating system; names and versions of relevant software libraries and frameworks. 

    \textbf{Answer}: Yes (in Appendix \ref{sec:appendix-implementation-details})

    \item This paper formally describes evaluation metrics used and explains the motivation for choosing these metrics. 

    \textbf{Answer}: Yes (Section \ref{sec:metrics} and Appendix \ref{sec:appendix-metrics})
    
    \item This paper states the number of algorithm runs used to compute each reported result. 

    \textbf{Answer}: No.

    \item Analysis of experiments goes beyond single-dimensional summaries of performance (e.g., average; median) to include measures of variation, confidence, or other distributional information.

    \textbf{Answer}: Yes (Section \ref{sec:expts}, Appendix \ref{sec:appendix-experiments}, Section \ref{sec:userstudy}, and Appendix \ref{sec:appendix-user-study})

    \item The significance of any improvement or decrease in performance is judged using appropriate statistical tests (e.g., Wilcoxon signed-rank). 

    \textbf{Answer}: Yes (Section \ref{sec:expts}, Appendix \ref{sec:appendix-experiments}, Section \ref{sec:userstudy}, and Appendix \ref{sec:appendix-user-study}) 

    
    \item This paper lists all final (hyper-)parameters used for each model/algorithm in the paper’s experiments. (yes/partial/no/NA)

    \textbf{Answer}: Yes (Appendix Appendix \ref{sec:appendix-implementation-details}).

    
    \item This paper states the number and range of values tried per (hyper-) parameter during development of the paper, along with the criterion used for selecting the final parameter setting. (yes/partial/no/NA)

    \textbf{Answer}: NA

\end{enumerate}
\end{document}
