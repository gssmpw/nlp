

The Allen-Cahn equation describes the process of phase separation in multi-component alloy systems. We consider a space-time domain $\Omega \times [0, T] $ with 
$\Omega=(-1,1)$, $ T = 60 $. The equation on the phase field variable $ u : \Omega \times [0, T] \rightarrow \mathbb R $ is 
$$
    \frac{\partial u}{\partial t} = \eta^2 \Delta u + u - u^3,\quad \text{on} \quad \Omega \times [0, T],
$$
with $\eta =10^{-2} $, and with boundary conditions 
$
u(-1, t) = -1$ and $u(1, t) = 1
$ for $t\in [0,T]$, and initial condition 
$
u(x, 0) = \lambda x + (1-\lambda) \text{sin}(-1.5 \pi x)$ for $x\in \Omega$, where $\lambda$ is a uniform random variable with distribution   $U([0.5, 0.6])$ \cite{geelen2024}. 


The spatial domain is discretized into $D=512$ equispaced points. A time step size $\Delta t=0.1 $ is fixed to discretize the time interval $ [0, 60] $. For the training set, three values of $ \lambda $ are considered, $ \lambda \in \{0.5, 0.55, 0.60\} $. For the test set, $ 10 $ values of $ \lambda $ are uniformly sampled in $ [0.5, 0.6] $. We therefore have $1803$ training data and $6010$ test data, including the initial conditions. 
Table \ref{tab:allen_cahn_comparisons} 

We run our method CPN-LR with a target precision $\epsilon =10^{-3}$ and  a polynomial degree $p=3$, which results in a dimension $N = 7$ and a manifold dimension $ n= 2$ (selected by the algorithm).  We  compare  different  methods  in  Table  \ref{tab:allen_cahn_comparisons}  for  the same manifold dimension $n= 2$.  We again observe that CPN-LR outperforms SOTA methods by one order of magnitude. 

\begin{table}[h!]
\centering % Adjust line thickness

\begin{tabular}{|c|c|c|c|c|c|}
\hline
 Method & $p$  & $n$ & $N$ & $ \text{RE}_{\text{train}} $ & $ \text{RE}_{\text{test}} $ \\ 
 \hline
 Linear & / &  2  & / & $3.38 \times 10^{-2}$ & $3.35 \times 10^{-2}$ \\ 
 \hline
 Quadratic & 2 & 2 & 5 & $ 1.87 \times 10^{-2}$ & $ 1.84 \times 10^{-2}$ \\ 
 \hline
 Additive-AM & 3 & 2 & 7 & $ 3.42 \times 10^{-3}$ & $3.45 \times 10^{-3}$ \\
  \hline
 Sparse & 3 & 2 & 7 & $ 2.34 \times 10^{-3}$  & $ 2.19 \times 10^{-3} $ \\
 \hline
 Low-rank & 3 & 2 & 7 & $1.05 \times 10^{-3} $ & $ 1. \times 10^{-3}$ \\
 \hline
 CPN-LR $(\epsilon=10^{-3})$ & 3 & 2 & 7 & $ 3.90 \times 10^{-4} $ & $ 3.71 \times 10^{-4} $ \\ 
 \hline
\end{tabular}
\caption{(Allen-Cahn) Comparison of methods for the same manifold dimension $n=2$.}
\label{tab:allen_cahn_comparisons}
\end{table}

Figure \ref{fig:allen_cahn_graphs} illustrates the compositional networks for three coefficients. 

\begin{figure}[h]
    \centering
    \subfigure[$a_{3}$]{\includegraphics[width=0.3\textwidth]{New_figures/allen_cahn_a_3.png}} 
    \subfigure[$ a_{5} $]{\includegraphics[width=0.3\textwidth]{New_figures/allen_cahn_a_5.png}} 
    \subfigure[$ a_{6} $]{\includegraphics[width=0.3\textwidth]{New_figures/allen_cahn_a_6.png}} 
    \caption{(Allen-Cahn) Networks for different coefficients, using CPN-LR with $\epsilon=10^{-3}$ and $p=3$.}
    \label{fig:allen_cahn_graphs}
\end{figure}


Figure \ref{fig:allen_cahn_viz_solution} shows the predicted solutions for $\lambda = 0.55188$. It illustrates the capacity of CPN-LR to provide a very accurate approximation  with a manifold of dimension $2$. 

 \begin{figure}[H]
    \centering
    \subfigure[Exact solution]{\includegraphics[width=0.4\textwidth]{New_figures/Allen_cahn_exact.png}}
    \subfigure[CPN-LR]{\includegraphics[width=0.4\textwidth]{New_figures/Allen_cahn_cpn.png}} 
    \caption{(Allen-Cahn) Predictions of CPN-LR with $n=2$, $\lambda = 0.55188$.}
    \label{fig:allen_cahn_viz_solution}
\end{figure}

% \begin{table}[h!]
% \centering % Adjust line thickness
% \setlength{\tabcolsep}{10pt}      % Adjust cell padding
% \renewcommand{\arraystretch}{1.5} % Adjust row height   

% \begin{tabular}{|>{\centering\arraybackslash}m{3cm}|>
% {\centering\arraybackslash}m{3cm}|>
% {\centering\arraybackslash}m{3cm}|}
% \hline
% \textbf{Step} & \textbf{Indices of input coeffs.} & \textbf{Indices of learnt coeffs.} \\
% \hline
% 1 & 1 & / \\
% \hline
% 2 & 1, 2 & 3, 7 \\
% \hline
% 3 & 1, 2, \textbf 3 & 4, 5, 6 \\
% \hline
% \end{tabular}
% \caption{(Allen-Cahn) Learning procedure of CPN-S for $ p = 6 $ and $ \epsilon = 10^{-3}$}
% \label{tab:sparse_table}
% \end{table}
