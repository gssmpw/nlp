%
%% This is file `sample-acmsmall-conf.tex',
%% generated with the docstrip utility.
%%
%% The original source files were:
%%
%% samples.dtx  (with options: `all,proceedings,bibtex,acmsmall-conf')
%% 
%% IMPORTANT NOTICE:
%% 
%% For the copyright see the source file.
%% 
%% Any modified versions of this file must be renamed
%% with new filenames distinct from sample-acmsmall-conf.tex.
%% 
%% For distribution of the original source see the terms
%% for copying and modification in the file samples.dtx.
%% 
%% This generated file may be distributed as long as the
%% original source files, as listed above, are part of the
%% same distribution. (The sources need not necessarily be
%% in the same archive or directory.)
%%
%%
%% Commands for TeXCount
%TC:macro \cite [option:text,text]
%TC:macro \citep [option:text,text]
%TC:macro \citet [option:text,text]
%TC:envir table 0 1
%TC:envir table* 0 1
%TC:envir tabular [ignore] word
%TC:envir displaymath 0 word
%TC:envir math 0 word
%TC:envir comment 0 0
%%
%%
%% The first command in your LaTeX source must be the \documentclass
%% command.
%%
%% For submission and review of your manuscript please change the
%% command to \documentclass[manuscript, screen, review]{acmart}.
%%
%% When submitting camera ready or to TAPS, please change the command
%% to \documentclass[sigconf]{acmart} or whichever template is required
%% for your publication.
%%
%%
\documentclass[acmsmall,screen,nonacm]{acmart}

%%
%% \BibTeX command to typeset BibTeX logo in the docs
%\AtBeginDocument{%
%  \providecommand\BibTeX{{%
%    Bib\TeX}}}

%% Rights management information.  This information is sent to you
%% when you complete the rights form.  These commands have SAMPLE
%% values in them; it is your responsibility as an author to replace
%% the commands and values with those provided to you when you
%% complete the rights form.
\setcopyright{cc}
%\copyrightyear{2025}
%\acmYear{2025}
%\acmDOI{XXXXXXX.XXXXXXX}

%% These commands are for a PROCEEDINGS abstract or paper.
%\acmConference[FSE' 25]{The ACM International Conference on the Foundations of Software Engineering}{June 23--27,
%  2025}{Trondheim, Norway}

%\acmBooktitle{ }
%%
%%  Uncomment \acmBooktitle if the title of the proceedings is different
%%  from ``Proceedings of ...''!
%%
%%\acmBooktitle{Woodstock '18: ACM Symposium on Neural Gaze Detection,
%%  June 03--05, 2018, Woodstock, NY}
%\acmISBN{978-1-4503-XXXX-X/18/06}


%%
%% Submission ID.
%% Use this when submitting an article to a sponsored event. You'll
%% receive a unique submission ID from the organizers
%% of the event, and this ID should be used as the parameter to this command.
%%\acmSubmissionID{123-A56-BU3}

%%
%% For managing citations, it is recommended to use bibliography
%% files in BibTeX format.
%%
%% You can then either use BibTeX with the ACM-Reference-Format style,
%% or BibLaTeX with the acmnumeric or acmauthoryear sytles, that include
%% support for advanced citation of software artefact from the
%% biblatex-software package, also separately available on CTAN.
%%
%% Look at the sample-*-biblatex.tex files for templates showcasing
%% the biblatex styles.
%%
%\usepackage[table,xcdraw]{xcolor}
\usepackage{colortbl}
\DeclareRobustCommand{\hlcyan}[1]{{\sethlcolor{cyan}\hl{#1}}}
%\usepackage{listings}
%\usepackage{multirow}
%\usepackage{minted}
%\usepackage{changepage}
\usepackage{enumitem}
%\usepackage[table,xcdraw]{xcolor}

%%
%% The majority of ACM publications use numbered citations and
%% references.  The command \citestyle{authoryear} switches to the
%% "author year" style.
%%
%% If you are preparing content for an event
%% sponsored by ACM SIGGRAPH, you must use the "author year" style of
%% citations and references.
%% Uncommenting
%% the next command will enable that style.
%%\citestyle{acmauthoryear}


%%
%% end of the preamble, start of the body of the document source.
\begin{document}

%%
%% The "title" command has an optional parameter,
%% allowing the author to define a "short title" to be used in page headers.
\title{Causal Models in Requirement Specifications for Machine Learning: A vision}

%%
%% The "author" command and its associated commands are used to define
%% the authors and their affiliations.
%% Of note is the shared affiliation of the first two authors, and the
%% "authornote" and "authornotemark" commands
%% used to denote shared contribution to the research.
\author{Hans-Martin Heyn}
\email{hans-martin.heyn@gu.se}
\affiliation{%
  \institution{Chalmers University of Technology and University of Gothenburg}
  \city{Göteborg}
  \country{Sweden}
}

\author{Yufei Mao}
\author{Roland Weiß}
\email{{yufei.mao,rolandweiss}@siemens.com}
\affiliation{%
  \institution{Siemens AG}
  \city{München}
  \country{Germany}
}

\author{Eric Knauss}
\email{Eric.Knauss@cse.gu.se}
\affiliation{%
  \institution{Chalmers University of Technology and University of Gothenburg}
  \city{Göteborg}
  \country{Sweden}
}
%%
%% By default, the full list of authors will be used in the page
%% headers. Often, this list is too long, and will overlap
%% other information printed in the page headers. This command allows
%% the author to define a more concise list
%% of authors' names for this purpose.
\renewcommand{\shortauthors}{Heyn et al.}

%%
%% The abstract is a short summary of the work to be presented in the
%% article.
\begin{abstract}
Specifying data requirements for machine learning (ML) software systems remains a challenge in requirements engineering (RE). This vision paper explores causal modelling as an RE activity that allows the systematic integration of prior domain knowledge into the design of ML software systems. We propose a workflow to elicit low-level model and data requirements from high-level prior knowledge using causal models. The approach is demonstrated on an industrial fault detection system. This paper outlines future research needed to establish causal modelling as an RE practice.
\end{abstract}

%%
%% The code below is generated by the tool at http://dl.acm.org/ccs.cfm.
%% Please copy and paste the code instead of the example below.
%%
%%
%% Keywords. The author(s) should pick words that accurately describe
%% the work being presented. Separate the keywords with commas.
\keywords{AI Engineering, Causal Modelling, Data Requirements, Requirements Engineering}
%% A "teaser" image appears between the author and affiliation
%% information and the body of the document, and typically spans the
%% page.
%\received{20 February 2007}
%\received[revised]{12 March 2009}
%\received[accepted]{5 June 2009}

%%
%% This command processes the author and affiliation and title
%% information and builds the first part of the formatted document.
\maketitle

\section{Introduction}
Rahimi et al. called for more attention towards the ability of specifying software with machine learning (ML) components~\cite{Rahimi2019}. 
Many industrial applications require \emph{robustness} of ML models against changes in input data distribution~\cite{Borg2019}. 
A key reason for lacking robustness is the difficulty of specifying ML models, because ``if input and/or output data are high-dimensional, both defining preconditions and detailed function specifications are difficult"\cite{Kuwajima2020}. 
Robustness against context changes can only be tested if the expected operational context is explicitly defined, for instance through contextual requirements\cite{Knauss2014, Knauss2016}. 
However, assumptions about the operational context are often implicit in the design process~\cite{Mitchell2021}, such as in the selection of the training dataset.
Recent surveys on requirements engineering (RE) confirm that specifying training data for ML models remains an open challenge~\cite{pei2022, ahmad2023, franch2023}. 
Current RE techniques struggle to translate high-level functional and non-functional requirements into data requirements~\cite{alves2023, villamizar2024}. 
This leads to an \emph{underspecification} causing variability in implementation choices and a lack of robustness against context changes~\cite{Fantechi2018}.\par
A possible way to address underspecification is reasoning about expected causal relationships in the ML system’s operational context. 
Typically, ML cannot infer causality from data alone~\cite{Pearl2019}. 
An ML model learns a probabilistic representation from data that seems compatible in a training context, but its performance might deviate drastically in a different operational context as statistical correlations do not capture true causal mechanisms~\cite{damour2022}.
Addressing this challenge requires incorporating prior domain knowledge and causal reasoning into the design of ML systems.\par
This vision paper proposes causal modelling to communicate \emph{prior knowledge} about causal relations in the operational context. 
We argue that by formulating prior domain knowledge as causal models we can derive requirements towards data, as well as deduce rules for runtime verification. 
This will lead to causally motivated requirements specifications for software with ML.
\vspace{-0.3em}
\paragraph{Objective of this vision paper}
First, we outline our vision of integrating causal modelling as an RE activity for ML systems. 
Then, we illustrate its application in eliciting data requirements for an industrial prototype of an ML-based cooling fault-detection system for electric motors. 
Finally, we discuss a research agenda to explore the potential of causal modelling as an RE activity for ML systems. 

\section{Related Work}
The potential of using causal modelling as part of RE activities is not yet fully explored~\cite{giamattei2024}. Fischbach et al. proposed an NLP-based process to extract and structure causal relationships from natural language~\cite{fischbach2020, fischbach2021}. A tree recursive neural network (TRNN) model was trained to detect causality in natural language requirements using logical markers such as conjunctions and negations~\cite{jadallah2021}. They further developed an approach to converts extracted causal relationships into a DAG-like structure to automatically generate test cases~\cite{fischbach2023}.
Maier et al. proposed modelling cause-effect relationships as part of scenario-based testing for automotive system safety~\cite{maier2022}. Maier et al. also introduced the concept of ``CausalOps'', an industrial lifecycle framework for causal models~\cite{maier2024}.
Gren et Brentsson Svensson proposed Bayesian Data Analysis (BDA) to evaluate the outcome of experiments on the effect of obsolete requirements on software effort estimation~\cite{gren2021}. Similarly, Frattini et al. investigated the impact of requirements quality defects on domain modelling by using BDA and causal reasoning in a in a controlled experiment~\cite{frattini2025}. While the latter two studies do not use causal modelling as an explicit RE activity, these studies demonstrates the potential of applying causal reasoning to RE activities.

\section{Causal modelling as an RE activity}
%We propose causal modelling and specifically the use of graphical causal models as means to systematically capture high level requirements, prior knowledge about expected causal structures of the system, and assumptions about the operational context. 
In a typical ML development pipeline, causal modelling would be a step between problem definition and data collection as it allows to formalise domain knowledge, identify relevant variables, and refine data requirements by distinguishing causal relationships from mere correlations before collecting the training data.
Particularly, graphical causal models in the form of directed acyclic graph (DAG) allow to communicate  explicitly assumed directions of causality and assumptions about \emph{confounders}, i.e., situations in which a variable $Z$ is associated to two unrelated random variables $X_1$ and $X_2$ such that a \emph{spurious relationship} between $X_1$ and $X_2$ can be observed: $X_1 \, \leftarrow \, Z \, \rightarrow \, X_2$. \par
\par

\begin{figure}[!t]
    \centering
    \includegraphics[width=0.5\linewidth]{Workflow_CausalRE.pdf}
    \caption{A proposed workflow for Causal RE}
    \label{fig:Workflow_CausalRE}
\end{figure}

Figure~\ref{fig:Workflow_CausalRE} outlines a proposed workflow. The workflow bases on the principle of \emph{causal factorisation}~\cite{Scholkopf2021}:

\begin{equation} \label{eq:decompose}
        p(X_1,\ldots,X_n) = \prod_{i=1}^{n} p\left(X_i|\mathbf{PA}_i\right)
\end{equation}
\emph{Causal factorisation} implies that an observed joint distribution of interest can be decomposed into a product of conditional distributions, where each term corresponds to a causal mechanism. \par
\textbf{Step 1) Identify individual causal mechanisms:} The idea is to identify individual causal mechanisms based on high-level requirements, prior domain knowledge, and context assumptions.\par
\textbf{Step 2) Update causal graph:} Once a causal mechanism is identified, the relevant observable and latent variables are determined, and a causal graph is updated to include these variables along with the assumed directions of cause-and-effect relationships. \par

%\textbf{Step 3) Perform d-separation and extract requirements:}  



%Additionally, Step 3 provides \emph{independence criteria} based on global Markov properties:  
%If $X_1$ and $X_2$ are d-separated by $Z$, they are conditionally independent given $Z$ in the probability distribution, i.e., $X_1 \perp X_2 | Z$.  
%This offers \emph{testable criteria} to verify prior knowledge and assumptions encoded in the causal graph.  


\textbf{Step 3) Perform d-separation and extract requirements:} 
With the causal model, \emph{d-separation}\footnote{Due to space constraints, background on d-separation is omitted but can be found in~\cite{Pearl2019}.} allows to identify variables that are needed to block ``non-causal'' association paths.  
Taking the example from above, in $X_1 \leftarrow Z \rightarrow X_2$, there is a ``non-causal'' path between $X_1$ and $X_2$.  
If the ML model can condition on $Z$ (assuming $Z$ is observable), $X_1$ and $X_2$ become d-separated, closing the ``non-causal'' path.
This is an example of a resulting data requirement: 
$Z$ must be included in the training dataset to avoid learning a spurious correlation between $X_1$ and $X_2$.  
Additionally, Step 3 provides \emph{independence criteria} based on global Markov properties:  
If $X_1$ and $X_2$ are d-separated by $Z$, they are conditionally independent given $Z$, i.e., $X_1 \perp X_2 \, |\, Z$.  
This provides \emph{testable criteria} to verify prior knowledge and assumptions encoded in the causal graph.\par

\textbf{Step 4) Check consistency and observability:}
The graphical causal model must be checked for cyclic dependencies because a variable cannot be its own cause~\cite{Glymour2016}.  
Furthermore, variables needed to block ``non-causal'' paths must be observable.  
If this is not the case, the system must be adjusted to enable their observation or suitable instrument variables must be identified~\cite{angrist1996}.\par
The resulting causal graph becomes part of an ML specification because it encodes the assumed causal structures, prior knowledge, and operational context, from which data and model requirements, as well as testing criteria, are derived.  

\section{Demonstration on industrial prototype}
We demonstrate the use of causal modelling as an RE activity on an industrial prototype use case, specifically a system for detecting faults in the cooling system of electric motors.

\paragraph{Methodology:}
We held three workshops with two Siemens engineers and two academic researchers to explore using causal models for requirements specification in the second half of 2022.
The researchers introduced causal models with examples like \emph{temperature} $\leftarrow$ \emph{sunrise} $\rightarrow$ \emph{birds chirping} and explaining key concepts such as \emph{confounding}, \emph{colliders}, and \emph{d-separation} using for example the \emph{back-door criteria}.  
The company experts then presented the prototype system, and prior knowledge rules were formalised. 
Together, the team identified causal mechanisms and updated the causal model iteratively with each newly found causal mechanism. 
We then applied \emph{d-separation} to close non-causal paths which resulted in data and model requirements to ensure the ML model controls for potential confounding.  

\par
\paragraph{Description of demonstration case:}
The demonstration case, provided by Siemens, is a motor diagnostic application for monitoring electrical motors  using an attachable sensor device. Initially, the system detected cooling faults from vibrations caused by mechanical faults, such as a broken fan blade. The new device will use an ML model to detect faults based on multiple sensor inputs. The high-level functional requirement is:  

\begin{description}
    \item[FR-1:] \emph{GIVEN indoor operational environment WHEN the cooling system is abnormal THEN an alarm should be raised.}
\end{description}
The following prior knowledge of the company engineers was considered for identifying causal mechanisms:
\begin{description}
    \item[PK-1:] A fault in the cooling system can affect the magnetic flux by changing the temperature of the rotor material and thus affecting the electrical resistance.
    \item[PK-2:] Mechanical faults of the fan can reduce the available airflow.
    \item[PK-3:] Mechanical faults cause vibrations of the system.
    \item[PK-4:] Environmental temperature has an influence on the temperature signal because the sensor is mounted outside the motor. 
    \item[PK-5:] Unmeasured sensor disturbances exist in general.
\end{description}
\paragraph{Results:}
The resulting causal model for the motor diagnostic use case is shown in Figure~\ref{fig:motor_diagnostic_DAG}. 

\begin{figure}[!ht]
    \centering
    \includegraphics[width=0.6\linewidth]{motor_DAG.pdf}
    \caption{DAG for the motor diagnostic use case. Gray-background nodes are latent (unobservable) variables, while white-background nodes are observable at runtime.}
    \label{fig:motor_diagnostic_DAG}
\end{figure}

Explanations for the variables and their relations to the prior knowledge are provided in Table~\ref{tab:motorcond}. 

\begin{table}[!t]
\centering
\small
\caption{Variables for motor diagnostic use case.}
\label{tab:motorcond}
\begin{tabular}{l c l}
\textbf{Variable}    & \textbf{Related PKs} & \textbf{Definition}               \\ \midrule
Cooling Fault    & PK1, PK2   & Fan system status         \\ 
$Q$         & PK2       & Max. possible airflow           \\ 
Mech. Fault & PK2, PK3 & Mechanical fault of motor \\ 
$P_M$       & PK1         & Mechanical power          \\ 
$R_1$       & PK1         & Electrical (inner) losses   \\ 
$T_E$       & PK4         & Environmental temperature              \\ 
$U_X$       & PK5         & Unmeasured noises              \\ 
$T$ ($T_s$) &               & Surface temperature (measured) \\ 
$H$ ($H_s$) &               & Magnetic Flux (measured)               \\ 
$V$ ($V_s$) &               & Vibrations (measured)         \\ \bottomrule
\end{tabular}
\end{table}

The causal graph in Figure~\ref{fig:motor_diagnostic_DAG} includes three causal mechanisms between \emph{Cooling Fault} occurrence and \emph{Classification} whether or not a cooling fault has occurred: \par 
\vspace{0.2em}
\textbf{Temperature mechanism:} A cooling fault increases the motor's surface temperature $T$ (via the core temperature), measured by the temperature sensor $T_s$, which can be used to classify a cooling fault.

\textbf{Magnetic flux mechanism:} A cooling fault changes the inner resistance (via the core temperature), which affect the magnetic flux $H$. This is measured by the fluxmeter $H_s$ for classification.  

\textbf{Mechanical power mechanism:} A cooling fault changes the magnetic flux $H$, which affects the mechanical power $P_M$ and surface temperature $T$. The latter is measured by the sensor $T_s$ for classification. \par
\vspace{0.2em}
Two confounding paths were identified:\par
\vspace{0.2em}
\textbf{Mechanical fault confounding:} A mechanical fan blade fault can reduce the available airflow $Q$ causing a cooling fault and vibrations $V$, which are measured by a vibration sensor $V_s$ for classification.  

\textbf{Environmental temperature confounding:} A sudden change in environment temperature $T_E$ can temporarily limit cooling without indicating a fault and it affects the surface temperature $T$.\par

\paragraph{Data and model requirements:} 
We checked which variables must be observed and controlled for to close non-causal paths between cooling fault occurrence and classification of a cooling fault, which resulted in the requirements listed in Table~\ref{tab:motorrqs}. 
Vibration data $V_s$ alone is insufficient to detect cooling faults, as not all mechanical faults lead to a cooling fault (RQ-D1, RQ-M1). Instead, data on temperature and magnetic flux mechanisms should be included (RQ-M2).\footnote{In fact, vibration data may be unnecessary for detecting cooling faults unless it is desired to distinguish mechanical from non-mechanical causes.} An additional sensor should record the environmental temperature $T_E$ to control for confounding (RQ-D2). Sensor noise must also be represented in the training data (RQ-D3). 

\begin{table}[!t]
\centering
\small
\caption{Requirements derived from causal graph}
\label{tab:motorrqs}
\begin{tabular}{lp{7.1cm}}
\textbf{ID} & \textbf{Requirement}  \\ \midrule
RQ-D1 & Training data shall include cases where mechanical faults cause vibrations $V$ without leading to cooling faults. \\
RQ-D2 & The occurrence of cooling faults shall be conditioned on different environmental temperatures $T_E$ such that the model can learn the confounding influence of $T_E$. \\ 
RQ-D3 & Measurements shall include characteristic sensor noise. \\
RQ-M1 & Cooling faults shall not be classified based on vibration data $V_s$ alone. \\
RQ-M2 & The input layer shall accept temperature, magnetic flux, and vibration measurements. \\
\bottomrule
\end{tabular}
\end{table}


\paragraph{Testing and runtime checks:} The causal graph in Figure~\ref{fig:motor_diagnostic_DAG} implies a set of independence conditions:
\[
\begin{aligned}
    \textbf{ID1}:~&\text{Classification} \perp T_E \mid H_s, T_s, V_s \\
    \textbf{ID2}:~&H_s \perp T_E \mid \text{Cooling Fault} \\
    \textbf{ID3}:~&H_s \perp V_s \mid \text{Cooling Fault} \\
    \textbf{ID4}:~&T_s \perp V_s \mid \text{Cooling Fault}, T_E \\
    \textbf{ID5}:~&V_s \perp T_E
\end{aligned}
\]
As an example for a resulting test case, ID1 states that classification is independent of $T_E$ given $H_s$, $T_s$, and $V_s$. A test case could trigger faults at varying $T_E$ to verify that the detection probability remains unchanged. 
As an example for runtime monitoring, ID5 suggests $V_s$ and $T_E$ should be independent. A monitor could track their correlation and trigger an alarm if a threshold is exceeded which would indicate a shift in the assumed operational context (e.g., the probability of a mechanical fault could depend on the environmental temperature which would be a violation of the assumed causal models for this system).

%The causal model reveals the indirect influence from the cooling system to vibration and magnetic flux sensing and thus provides arguments for including both features in the cooling system condition classification (M-M1). 
%The collider $Cooling \rightarrow T \leftarrow T_E$ suggest the necessity of including both cooling system status and environmental temperature as variables in the dataset (M-D1).

\section{Discussion and research agenda}
In this vision paper we argue that causal modelling and its mathematical framework have significant potential as an RE activity for ML software system development by systematically integrating prior knowledge into the design. 
However, based on the experience in our demonstration use case, further research is needed before this vision becomes standard industry practice.\par
\textit{Causal models as complement to natural language requirements.} Causal graphs originates from mathematics. We must explore how they can complement current requirements specifications and how they must be adopted for RE. Terms like ``treatment'', ``confounder'', and ``collider'' are uncommon in RE and require interpretation.\par
\textit{Completeness criteria for variable selection.} A key challenge is knowing when a causal graphs includes ``enough'' prior knowledge. We need methods to identify the necessary set of variables that must be included for a given use case and which completeness criteria can guide the search for a sufficient causal representation.\par
\textit{Modularisation of ML software systems.} Isolating causal mechanisms can guide the modularisation of ML systems, i.e., dividing large monolithic ML models into smaller sub-models.\par
\textit{A common language between different stakeholders of ML software systems.} Causal models provide a unified way to communicate prior knowledge and assumptions. Research should explore how this can facilitate coordination between different groups such as data scientists, product experts, and software engineers.\par
\textit{Data requirements derived through causal reasoning.} Causal reasoning in RE helps identifying data requirements. 
Further research should assess to what degree data requirements derived from causal models can enhance ML robustness and reduce data needs compared to traditional RE methods.\par
\textit{Testing and runtime checks.} 
ML software system must align with expected (causal) behaviour. Causal graphs imply independence criteria that lead to \emph{testable implications} for the runtime behaviour. Research should explore how to translate these into testing strategies and monitors and how reliable such monitors are in practise. 

\paragraph*{Conclusion}
Causal reasoning offers a systematic way to integrate prior knowledge into RE for ML software systems. 
We outlined a vision and demonstrated a preliminary workflow to derive and argue for low level model and data requirements from high level prior knowledge using causal graphs. We discussed future research activities that are needed to turn this vision into industrial practise. 

%%
%% The acknowledgments section is defined using the "acks" environment
%% (and NOT an unnumbered section). This ensures the proper
%% identification of the section in the article metadata, and the
%% consistent spelling of the heading.
%\begin{acks}
%To Robert, for the bagels and explaining CMYK and color spaces.
%\end{acks}

%%
%% The next two lines define the bibliography style to be used, and
%% the bibliography file.
%\clearpage
\bibliographystyle{ACM-Reference-Format}
\bibliography{bib}


%%
%% If your work has an appendix, this is the place to put it.
%\appendix

\end{document}
\endinput
%%
%% End of file `sample-acmsmall-conf.tex'.
