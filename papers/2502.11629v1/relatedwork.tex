\section{Related Work}
The potential of using causal modelling as part of RE activities is not yet fully explored~\cite{giamattei2024}. Fischbach et al. proposed an NLP-based process to extract and structure causal relationships from natural language~\cite{fischbach2020, fischbach2021}. A tree recursive neural network (TRNN) model was trained to detect causality in natural language requirements using logical markers such as conjunctions and negations~\cite{jadallah2021}. They further developed an approach to converts extracted causal relationships into a DAG-like structure to automatically generate test cases~\cite{fischbach2023}.
Maier et al. proposed modelling cause-effect relationships as part of scenario-based testing for automotive system safety~\cite{maier2022}. Maier et al. also introduced the concept of ``CausalOps'', an industrial lifecycle framework for causal models~\cite{maier2024}.
Gren et Brentsson Svensson proposed Bayesian Data Analysis (BDA) to evaluate the outcome of experiments on the effect of obsolete requirements on software effort estimation~\cite{gren2021}. Similarly, Frattini et al. investigated the impact of requirements quality defects on domain modelling by using BDA and causal reasoning in a in a controlled experiment~\cite{frattini2025}. While the latter two studies do not use causal modelling as an explicit RE activity, these studies demonstrates the potential of applying causal reasoning to RE activities.