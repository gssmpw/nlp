\section{Related work}
\textbf{SAT dataset.}
SAT problems can be classified into different categories, mainly as random, crafted, and industrial \cite{alyahya2022structure}, while most come from either synthetic generators or existing datasets. Most generators such as CNFgen \cite{lauriaCNFgenGeneratorCrafted2017}, focus on random problems like random \textit{k}-SAT or combinatorial problems. The largest established datasets are SATLIB \cite{hoosSATLIBOnlineResource} and SAT Competitions (SATCOMP) \cite{InternationalSATCompetition}, while only SATCOMP contains industrial problems and is limited in size. To overcome this constraint, new generators have been proposed to generate pseudo-industrial instances, including hand-crafted generators Community Attachment \cite{giraldez-cruModularityBasedRandomSAT2015} and Popularity-Similarity  \cite{giraldez-cruLocalityRandomSAT2017}, or graph-generative models generators \cite{youG2SATLearningGenerate2019,li2024distribution,li2023hardsatgen,chen2023matching,garzon2022performance}. For GNN based SAT solving, \cite{selsam2018learning} proposed random generator SR(\textit{n}) and \cite{cameronPredictingPropositionalSatisfiability2020} generated uniform-random 3-SAT instances as datasets. G4SATBench \cite{liG4SATBenchBenchmarkingAdvancing2023a} produced a dataset including 7 types of synthetic generators with varying variable sizes.


\textbf{SAT structural properties.}
As traditional SAT solvers perform differently over random, crafted, and industrial instances, it is believed that different SAT domains have distinct underlying properties \cite{alyahyaStructureBooleanSatisfiability2023}, including problem-based properties and solver-based properties. Results in this filed have been wildy used in improving traditional solvers \cite{audemard2009predicting} and portfolio based solvers \cite{sonobe2016cbpenelope2016}. 
Problem-based properties are further divided to CNF-based, including phase-transition \cite{cheeseman1991really}, backdoor \cite{kilby2005backbones} and  backbones \cite{kilby2005backbones}, and graph-based, including scale-free \cite{ansoteguiStructureIndustrialSAT2009}, self-similar \cite{ansoteguiFractalDimensionSAT2014}, centrality \cite{katsirelos2012eigenvector}, small-worlds \cite{walsh1999search} and community structure \cite{ansoteguiCommunityStructureIndustrial2019}. Solver-based properties are related to SAT solvers such as: mergeability and resolvability \cite{zulkoski2018effect}. These measures are either proved mathematically, or analyzed through solver-related parameters such as solving time. However, all the works are experimented with traditional SAT solvers, and most work only focus on single property measure \cite{li2021hierarchical, zulkoski2018effect}. In this work, we select several related properties, and calculate their values on each of our selected domains. We also split our dataset to different values based on each structure, %while analyzing them against cross-domain generalisation result on GNN, 
and analyze the important or hard features for GNN to capture. %and traditional solver time.


\textbf{GNN for SAT solving.} GNN has been applied to SAT solving mainly as problem solvers, including standalone solvers, which are networks trained to classify problem satisfiability themselves and mainly encodes the formula as LCG graphs \cite{selsam2018learning,zhangBetterGeneralizationNeural2022,changPredictingPropositionalSatisfiability2022,shi2023satformer,cameronPredictingPropositionalSatisfiability2020,hartfordDeepModelsInteractions2018,duanAugmentCareContrastive2022,ozolinsGoalAwareNeuralSAT2022}, and hybrid solvers, which treat GNN as a guidance to traditional solvers by replacing their heuristics with network predictions. These solvers focuses on specific tasks and corresponding problems such as UNSAT core \cite{selsam2019guiding}, glue clauses \cite{hanEnhancingSATSolvers2020} or backbone variables \cite{wangNeuroBackImprovingCDCL2023} for CDCL solvers and initial assignment \cite{zhangNLocalSATBoostingLocal2020, liNSNetGeneralNeural2022} for SLS solvers. Although hybrid solvers generally achieve better results than standalone solvers, they acts as modifications of traditional solvers instead of discussing solvability of GNNs,  which is out of scope of our work.