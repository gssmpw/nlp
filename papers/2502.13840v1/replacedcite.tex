\section{Related Work}
\label{sec:dnm9}
IPS-based methods mitigate popularity bias in recommender systems through sample weighting.
We categorize related work by the type of loss function: point-wise or pair-wise.

\textbf{Point-wise.}
Rel-MF____ formulates interaction occurrence as a two-step process---exposure and interaction---and introduces an unbiased point-wise loss function.
DU____ refines this method by enhancing propensity score estimation.
CJMF____ introduces a joint learning framework to simultaneously model unbiased user-item relevance and propensity.
BISER____ addresses item bias via self-inverse propensity weighting and employs bilateral unbiased learning to unify two complementary models.
More recently, ReCRec____ improves bias correction by distinguishing unexposed from disliked items, enabling better reasoning over unclicked data.

\textbf{Pair-wise.}
UBPR____ builds on the two-step interaction assumption to introduce an unbiased pair-wise loss function.
UPL____ extends this approach with a variance-reduced learning method.
CPR____ redefines unbiasedness in ranking and leverages cross pairs to improve unbiased learning.
UpliftRec____ applies uplift modeling to dynamically optimize top-N recommendations, revealing latent user preferences while maximizing click-through rates.
Additionally, PU____ models feedback labels as a noisy proxy for exposure outcomes and integrates a theoretically noise-resistant loss function into propensity estimation.