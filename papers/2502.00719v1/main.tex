\documentclass[twocolumn]{article}
%\documentclass[twocolumn]{jsarticle}
\setlength{\columnsep}{10mm}

\usepackage{PRIMEarxiv}
%\usepackage{arxiv}

\usepackage[utf8]{inputenc} % allow utf-8 input
\usepackage[T1]{fontenc}    % use 8-bit T1 fonts
\usepackage{hyperref}       % hyperlinks
\usepackage{url}            % simple URL typesetting
\usepackage{booktabs}       % professional-quality tables
\usepackage{bm}
\usepackage{amsfonts}       % blackboard math symbols
\usepackage{amsthm}
\usepackage{amsmath}
\usepackage{amssymb}
\usepackage{mathtools}
%\usepackage{nicefrac}       % compact symbols for 1/2, etc.
\usepackage{microtype}      % micro typography
\usepackage{lipsum}
\usepackage{fancyhdr}       % header
\usepackage{graphicx}       % graphics
\graphicspath{{media/}}     % organize your images and other figures under media/ folder
\usepackage{comment}
\usepackage{here}
\usepackage{color}
\usepackage{multirow}
\usepackage{hhline}
\usepackage{makecell}

%\usepackage{algorithmic}
%\usepackage{algorithm}
\usepackage{tabularx}

% Support for easy cross-referencing
\usepackage[capitalize]{cleveref}
\crefname{section}{Sec.}{Secs.}
\Crefname{section}{Section}{Sections}
\Crefname{table}{Table}{Tables}
\crefname{table}{Tab.}{Tabs.}



\setlength{\columnwidth}{\dimexpr\linewidth-2\tabcolsep}


%Header
\pagestyle{fancy}
\thispagestyle{empty}
\rhead{ \textit{ }} 

% Theorem style
\newtheorem{definition}{Definition}
\newtheorem{proposition}{Proposition}
\newtheorem{remark}{Remark}
\newtheorem{example}{Example}

\newcommand{\argmax}{\mathop{\mathrm arg~max}\limits}
\newcommand{\argmin}{\mathop{\mathrm arg~min}\limits}
\newcommand\lx[1]{\textcolor{blue}{Linxin: #1}}
\newcommand{\bhline}[1]{\noalign{\hrule height #1}}

% Update your Headers here
\fancyhead[LO]{Kosuke Sakurai et al.}
% \fancyhead[RE]{Firstauthor and Secondauthor} % Firstauthor et al. if more than 2 - must use \documentclass[twoside]{article}

%% Title
\title{Vision and Language Reference Prompt into SAM \\ for Few-shot Segmentation}

\author{
  Kosuke Sakurai \\
  Waseda University \\
  \texttt{kosukesakurai@toki.waseda.jp} \\\
\And
  Ryotaro Shimizu \\
  Waseda University \\
  \texttt{shi3mizu8-r@fuji.waseda.jp} \\\
\And
  Masayuki Goto \\
  Waseda University \\
  \texttt{masagoto@waseda.jp} \\
}

\begin{document}
\twocolumn[
\maketitle

\vspace{-5mm}
\begin{abstract}
\vspace{3mm}
   Segment Anything Model (SAM) represents a large-scale segmentation model that enables powerful zero-shot capabilities with flexible prompts. While SAM can segment any object in zero-shot, it requires user-provided prompts for each target image and does not attach any label information to masks. Few-shot segmentation models addressed these issues by inputting annotated reference images as prompts to SAM and can segment specific objects in target images without user-provided prompts. Previous SAM-based few-shot segmentation models only use annotated reference images as prompts, resulting in limited accuracy due to a lack of reference information. In this paper, we propose a novel few-shot segmentation model, \textit{Vision and Language reference Prompt into SAM} (VLP-SAM), that utilizes the visual information of the reference images and the semantic information of the text labels by inputting not only images but also language as reference information. In particular, VLP-SAM is a simple and scalable structure with minimal learnable parameters, which inputs prompt embeddings with vision-language information into SAM using a multimodal vision-language model. To demonstrate the effectiveness of VLP-SAM, we conducted experiments on the PASCAL-$5^i$ and COCO-$20^i$ datasets, and achieved high performance in the few-shot segmentation task, outperforming the previous state-of-the-art model by a large margin ($6.3\%$ and $9.5\%$ in mIoU, respectively). Furthermore, VLP-SAM demonstrates its generality in unseen objects that are not included in the training data. Our code is available at \href{https://github.com/kosukesakurai1/VLP-SAM}{https://github.com/kosukesakurai1/VLP-SAM}.
\end{abstract}

\vspace{3mm}
% keywords can be removed
\vspace{8mm}
]

%--------------------------------------------------------------------------------------------------------------------------------
%---------------------------------------------------------------------

\section{Introduction}

\begin{figure*}[ht]
\centering
\includegraphics[width=0.9\linewidth]{VLP-SAM2.png}
\caption{Comparison of our proposed method VLP-SAM and three previous models. (a) SAM~\cite{SAM} is a category-agnostic interactive zero-shot segmentation model that inputs user-provided prompts for each target image. (b) PerSAM~\cite{PerSAM} is a training-free SAM-based few-shot segmentation model that inputs positive and negative points on the target image into SAM, without requiring user-provided prompts. (c) VRP-SAM~\cite{VRP-SAM} is a SAM-based few-shot segmentation model incorporating a VRP encoder to generate prompt embeddings for SAM from an annotated reference image. (d) Our proposed method, VLP-SAM, introduces a novel prompt encoder for SAM, called the VLP encoder, which accepts both reference images and text labels as input via a vision-language model.}
\label{fig:2}
\end{figure*}

In recent years, Segment Anything Model (SAM)~\cite{SAM} has been released as a foundational image segmentation model trained on a large dataset with billion mask labels. SAM has powerful zero-shot capabilities that can segment any object in a target image by taking user-provided prompts consisting of points, bounding boxes, or coarse masks. However, user-provided prompts require the user's comprehensive understanding of the target objects, and different customized prompts are needed for each target image. Furthermore, SAM outputs category-agnostic masks, limiting its usability in real-world applications.

Few-shot segmentation models~\cite{PerSAM, Mather, VRP-SAM, APSeg} addressed these issues by inputting a few annotated reference images as prompts to SAM and can segment specific objects in target images without user-provided prompts. Based on the pixel-level similarity between annotated reference images and target images, few-shot segmentation models segment target objects of the same category as the reference images. On the other hand, previous SAM-based few-shot segmentation models only use annotated reference images as prompts, resulting in limited accuracy due to the lack of reference information.

In this work, we propose a novel few-shot segmentation model, \textit{Vision and Language reference Prompt into SAM} (VLP-SAM), that inputs not only reference images but also text labels as prompts to SAM. By using text information, VLP-SAM can leverage not only the visual similarity between annotated reference images and target images but also the semantic similarity between text labels and target objects. 

In particular, VLP-SAM is a simple and scalable structure that trains a new SAM's prompt encoder with minimal learnable parameters that outputs prompt embedding from a target image, an annotated reference image, and a text label. First, VLP-SAM embeds images (a target image and a reference image) and language (a text label) in the same embedding space using a multimodal vision-language model (VLM). Instead of CLIP~\cite{CLIP}, which is trained only on CLS tokens, a model that has pixel-level similarity between images and texts (e.g., CLIP Surgery~\cite{CLIP-Surgery}) is used as the VLM. Then, a visual prototype that aggregates the features of the target object from the annotated reference image and a textual prototype that aggregates semantic information of the target object from the text embedding are concatenated to each embedding. Furthermore, following previous work~\cite{VRP-SAM}, a pseudo mask of the target image created from the annotated reference image and an attention mask of the target image created from the text label are added to improve accuracy. Finally, VLP-SAM outputs prompt embeddings for SAM from learnable queries via Transformer-based attention~\cite{Transformer} interacting with each embedding. The prompt embeddings created by the VLP-SAM that contain visual and semantic information about the target object are input to the SAM's mask decoder, allowing the VLP-SAM to segment target objects without user-provided prompts.

Our proposed method, VLP-SAM, is a highly accurate and general few-shot segmentation model that can segment any class not included in the training data by leveraging the benefits of large-scale models of SAM and VLM. Furthermore, VLP-SAM is a scalable SAM-based segmentation model that leverages a new prompt encoder that can input a variety of modalities, such as reference images and text, rather than traditional SAM prompts, such as points and bounding boxes. 

To demonstrate the effectiveness of VLP-SAM, we conducted experiments on the PASCAL-$5^i$~\cite{PASCAL} and COCO-$20^i$~\cite{COCO} datasets. Our experimental results show that VLP-SAM outperforms the previous state-of-the-art model~\cite{VRP-SAM} by a large margin ($6.3\%$ and $9.5\%$ in mIoU, respectively) in a few-shot segmentation task. Furthermore, VLP-SAM demonstrates its generality in unseen objects.

The main contributions are summarized as follows:
\begin{itemize}
    \item We propose a novel few-shot segmentation model, \textit{Vision and Language reference Prompt into SAM} (VLP-SAM), which inputs not only reference images but also text labels as prompts to SAM. Our method is a highly accurate model that can utilize the visual information of reference images and the semantic information of text labels.
    \item VLP-SAM introduces a novel SAM's prompt encoder using VLM with pixel-text matching, achieving a scalable SAM-based model with minimal learnable parameters.
    \item We demonstrate that VLP-SAM outperforms the previous state-of-the-art model by a large margin in unseen objects.
\end{itemize}
%---------------------------------------------------------------------



%---------------------------------------------------------------------
\section{Related Work}
\subsection{Segment Anything Model}
Segment Anything Model (SAM)~\cite{SAM} is a category-agnostic interactive segmentation model trained on a large-scale SA-1B dataset containing over 1 billion masks. SAM shows powerful zero-shot capabilities to new objects without additional training by taking user-provided prompts consisting of points, bounding boxes, or coarse masks. Due to its generality, SAM has been applied to a variety of research~\cite{HQSAM, MedSAM, FastSAM, Semantic-sam, Grounded-sam, Depth-anything, Samrs, survey-sam}. As depicted in \cref{fig:2}(a), SAM comprises three modules: an image encoder, a prompt encoder, and a mask decoder. The image encoder is a Vision Transformer~\cite{ViT} backbone to extract image embeddings. The prompt encoder generates prompt embeddings from geometric prompts such as points and boxes. The mask decoder is a Transformer-based decoder~\cite{Transformer} that outputs a class-agnostic mask from image embeddings and prompt embeddings. While SAM can segment any object in zero-shot, it requires user-provided prompts for each target image, meaning human interaction and knowledge of the target image are needed. Furthermore, SAM does not output the class labels for each mask, limiting its usability in real-world applications.

\subsection{Few-shot Segmentation Model with SAM}
Few-shot segmentation models~\cite{Panet, PGMA-Net, Seggpt, BAM, RefLDM, Segic, PI-CLIP, SINE, survey-few} aim to segment objects in target images belonging to the same category as annotated reference images. Specifically, SAM-based few-shot segmentation models~\cite{PerSAM, VRP-SAM, Mather, APSeg, SAMIC, Alignsam, PDM, NubbleDrop, PMC, Foreground-Covering} only take a few reference image-mask pairs as prompts, instead of user-provided prompts for each target image. This allows leveraging the richness of large-scale foundational model SAM while addressing SAM's weaknesses: user interaction and class-agnostic masks.

SAM-based few-shot segmentation models can be mainly classified into two types: training-free models~\cite{PerSAM, Mather, PDM, NubbleDrop, PMC} that input geometric prompts derived from annotated reference images into SAM's prompt encoder, and meta-learning models~\cite{VRP-SAM, APSeg, Foreground-Covering} that introduce the new SAM's prompt encoder and input prompt embeddings into SAM's mask decoder. Training-free models generate geometric prompts of target objects from pixel-level correlations between annotated reference images and target images without user interaction. For example, as shown in \cref{fig:2}(b), PerSAM~\cite{PerSAM} selects the most positive and negative points from pixel-level correlations to input into SAM as prompts, and segments target objects without training. However, the performance of training-free models heavily depends on the quality of pseudo masks generated from pixel-level correlations, leading to incorrect prompts and reduced accuracy. Additionally, training-free models limit their scalability because SAM only accepts points, boxes, and masks as prompts.

Meta-learning models generate prompt embeddings for SAM derived from annotated reference images instead of geometric prompts via a meta-learning procedure. As depicted in \cref{fig:2}(c), VRP-SAM~\cite{VRP-SAM} introduces a novel prompt encoder, the VRP encoder, which outputs prompt embeddings from annotated reference and target images. These prompt embeddings are then input into SAM's mask decoder, creating a scalable few-shot segmentation model capable of predicting unseen classes not included in the training data. In this study, as shown in \cref{fig:2}(d), we introduce a novel SAM's prompt encoder (VLP encoder) that leverages a vision-language model (VLM) to capture both visual and semantic information for higher segmentation accuracy.

\subsection{Multimodal Vision-Language Model}
Multimodal vision-language models (VLM) embed images and text in a unified embedding space, and various VLMs have been released~\cite{CLIP, CoCa, ALIGN, BLIP, G-DINO, FIS, PFIS}. A representative model, CLIP~\cite{CLIP}, is a foundational model learned through contrastive learning~\cite{contrastive, Vilbert} on 400 million image-text pairs. Its zero-shot capability has been widely utilized in a variety of computer vision tasks~\cite{StableDiffusion, CoOp, GLIP, Clip4clip, Wav2clip, LARE, PointCLIP}. However, for object detection and image segmentation tasks, pixel-text matching rather than CLS token-text matching like CLIP is required to capture object-level features. VLM with pixel-text matching~\cite{CLIP-Surgery, OWL-ViT, Denseclip, Zegclip} can capture spatial and semantic information of objects from each patch-text relationship. Notably, CLIP Surgery~\cite{CLIP-Surgery} enhances the class attention map (CAM)~\cite{cam, Grad-cam, Clims, CLIP-ES} of CLIP to create high-performance pixel-text correlations (attention maps) without additional training. This model retains the benefits of CLIP's large-scale foundational model while providing image-text embeddings that capture object-level semantic features. Therefore, in this study, we employ CLIP Surgery as VLM and CLIP as a comparison.
%---------------------------------------------------------------------

%---------------------------------------------------------------------
\section{Methodology: VLP-SAM}

\begin{figure*}[ht]
\centering
\includegraphics[width=0.85\linewidth]{VLP-SAM3.png}
\caption{Overview of our proposed method VLP-SAM. VLP-SAM introduces a novel prompt encoder for SAM, called the VLP encoder, which generates the prompt embedding $\bm{Q}'_t$ with vision-language information. In particular, the VLP encoder first embeds the target image $I_t$, the reference image $I_r$, and the text label into the unified embedding space using the VLM encoder. These embeddings are then concatenated with prototypes and masks that aggregate information about the target object and the text label. Finally, the prompt embedding $\bm{Q}'_t$, refined through Transformer-based attention, is fed into SAM's mask decoder, enabling VLP-SAM to produce a more accurate mask of the target object.}
\label{fig:3}
\end{figure*}

In this paper, we propose a novel few-shot segmentation model, \textit{Vision and Language reference Prompt into SAM} (VLP-SAM), that inputs not only reference images but also text labels as prompts to SAM. By using text labels with VLM, VLP-SAM can utilize both the visual information of reference images and the semantic information of text labels. 

An overview of VLP-SAM is shown in \cref{fig:3}. A novel SAM's prompt encoder, VLP encoder, generates prompt embeddings from the target image, the annotated reference image, and the text label with minimal learnable parameters. By inputting these prompt embeddings with vision-language information into SAM's mask decoder, VLP-SAM outputs the mask of the target object. The details of the VLP encoder and the training process are described below.

\subsection{VLP Encoder}
In the few-shot segmentation task, the target image $I_t$, the reference image $I_r$, and its annotation mask $M^i_r$, where $i$ denotes the category of the annotated object, are given. Additionally, this study prepares the class label of the annotated object $i$ and uses it as a text $T_i$, such as ``a photo of a [$i$].'' By utilizing the information from the annotated reference image $I_r + M^i_r$ and the text label $T_i$, VLP-SAM can accurately predict the mask $M^i_t$ of the category $i$ in the target image. 

First, using a multimodal VLM, $I_r$, $I_t$, and $T_i$ are embedded into the same latent space, and image embeddings $\bm{F}_r \in \mathbb{R}^{C\times{H}\times{W}}$, $\bm{F}_t \in \mathbb{R}^{C\times{H}\times{W}}$, text embedding $\bm{F}_\text{text} \in \mathbb{R}^{C\times{1}}$ are obtained via a $1\times{1}$ convolution layer~\cite{CNN}. Since image embeddings use not only CLS tokens but also $H\times{W}$ patches, CLIP Surgery~\cite{CLIP-Surgery}, which considers pixel-text matching, is used instead of CLIP~\cite{CLIP}. To prevent overfitting of the VLP encoder, the VLM is frozen during the training phase. Next, the visual prototype $\bm{P}_i$ that aggregates the features of the target object $i$ from $\bm{F}_r$ and $M^i_r$, and the textual prototype $\bm{F}_\text{text}$ that aggregates the semantic information of category $i$ are extracted. The visual prototype $\bm{P}_i$ is the average embedding of the target object from the reference image $\bm{F}_r$, and it is formulated as follows:

\begin{equation}
    \bm{P}_i={\rm MaskAvgPool}(\bm{F}_r, M^i_r).
    \label{eq:1}
\end{equation}

Furthermore, following previous work~\cite{VRP-SAM}, the masks of the reference and target images are utilized to enhance the information about objects of category $i$. The mask of the reference image uses the annotation mask $M^i_r$, while the mask of the target image uses a pseudo mask $M^{pseudo}_t$, which is obtained through a common training-free approach. The pseudo mask $M^{pseudo}_t \in \mathbb{R}^{H\times{W}}$ is created by the cosine similarity correlations $\bm{S}\in\mathbb{R}^{(H\times{W})\times{(H\times{W})}}$ between the target image and the annotated reference image. Additionally, following CLIP Surgery~\cite{CLIP-Surgery}, the attention masks $M^{attn}_r, M^{attn}_t \in \mathbb{R}^{H\times{W}}$ of the reference and target images are utilized. The attention masks are derived from the cosine similarity between the image and text embeddings, and they include semantic information with the text label. Enhanced image embeddings $\bm{F}'_r\in\mathbb{R}^{C\times{H}\times{W}}$, $\bm{F}'_t\in\mathbb{R}^{C\times{H}\times{W}}$ are obtained by concatenating the visual prototype $\bm{P}_i$, textual prototype $\bm{F}_\text{text}$, pseudo mask $M^i_r$, $M^{pseudo}_t$, and attention mask $M^{attn}_r$, $M^{attn}_t$ with $\bm{F}_r$ and $\bm{F}_t$ through a $1\times{1}$ convolution layer to reduce the dimension. These enhanced embeddings $\bm{F}'_r$, $\bm{F}'_t$ are formulated as follows:

\begin{align}
    \bm{F}'_r &= {\rm Conv}({\rm concat}(\bm{F}_r, \bm{P}_i, \bm{F}_\text{text}, M^i_r, M^{attn}_r)), \label{eq:2} \\
    \bm{F}'_t &= {\rm Conv}({\rm concat}(\bm{F}_t, \bm{P}_i, \bm{F}_\text{text}, M^{pseudo}_t, M^{attn}_t)). \label{eq:3}
\end{align}



\begin{table*}[ht]
\begin{center}
\caption{Performance of one-shot segmentation on PASCAL-$5^i$ and COCO-$20^i$. Backbone indicates the image encoder within the novel SAM's prompt encoder (VRP and VLP encoders). VLP-SAM outperforms the previous state-of-the-art model VRP-SAM (ResNet-50) by a large margin ($6.3\%$ and $9.5\%$ in mIoU, respectively).}
\label{tab:1}
\scalebox{0.95}{
\begin{tabular}{@{}ccccccccccccc@{}}
\toprule
\multirow{2}{*}{\makecell[c]{\textbf{Method}}} & \multirow{2}{*}{\makecell[c]{\textbf{Backbone}}} & \multirow{2}{*}{\makecell[c]{\textbf{\# Learnable}\\\textbf{params}}} & \multicolumn{5}{c}{\textbf{PASCAL-$5^i$}} & \multicolumn{5}{c}{\textbf{COCO-$20^i$}} \\ \cmidrule(lr){4-8} \cmidrule(lr){9-13}
& & & F-0 & F-1 & F-2 & F-3 & \textbf{\underline{Mean}} & F-0 & F-1 & F-2 & F-3 & \textbf{\underline{Mean}} \\ \midrule
\multirow{4}{*}{VRP-SAM} & ResNet-50 & 1.6M & 73.18 & 75.74 & 69.50 & 64.21 & 70.66 & 43.40 & 54.43 & 53.28 & 50.53 & 50.41\\
& CLIP (ResNet-50) & 1.5M & 64.44 & 70.38 & 64.44 & 57.72 & 64.25 & 30.45 & 41.51 & 43.74 & 40.06 & 38.94\\
& CLIP (ViT-B/16) & 1.3M & 68.64 & 72.33 & 65.78 & 60.38 & 66.78 & 37.56 & 48.11 & 50.84 & 46.83 & 45.84\\
& CLIP Surgery (ViT-B/16) & 1.3M & 70.50 & 75.88 & 65.44 & 61.63 & 68.36 & 40.62 & 52.28 & 52.77 & 49.87 & 48.89\\ \midrule
\multirow{2}{*}{\makecell[c]{VLP-SAM\\(ours)}} & CLIP (ViT-B/16) & 1.4M & 71.92 & 73.12 & 65.28 & 62.20 & 68.13 & 41.74 & 52.76 & 52.49 & 51.66 & 49.66\\
& CLIP Surgery (ViT-B/16) & 1.4M & \textbf{76.94} & \textbf{83.08} & \textbf{72.73} & \textbf{75.27} & \textbf{77.01} & \textbf{52.40} & \textbf{63.90} & \textbf{62.10} & \textbf{61.26} & \textbf{59.92}\\
\bottomrule
\end{tabular}
}
\end{center}
\end{table*}





Finally, the VLP encoder generates prompt embeddings for SAM using enhanced image embeddings $\bm{F}'_r$, $\bm{F}'_t$ and learnable queries $\bm{Q}\in\mathbb{R}^{N\times{C}}$, where $N$ denotes the number of tokens in the prompt embeddings. The learnable queries $\bm{Q}$ initially interact with the reference features $\bm{F}'_r$ through cross-attention and self-attention layers to acquire category-specific information. Subsequently, these queries interact with the target features $\bm{F}'_t$ to acquire foreground information in the target image. These processes are formulated as follows:

\begin{align}
    \bm{Q}'_r &= {\rm SelfAttn}({\rm CrossAttn}(\bm{Q}, \bm{F}'_r)), \label{eq:4} \\
    \bm{Q}'_t &= {\rm SelfAttn}({\rm CrossAttn}(\bm{Q}'_r, \bm{F}'_t)). \label{eq:5}
\end{align}

The final $\bm{Q}'_t$ represents the prompt embeddings for SAM, containing both visual and semantic information of the target object.

\subsection{Training}
The prompt embeddings obtained from the VLP encoder are input to SAM's mask decoder instead of geometric prompts. VLP-SAM can predict the mask $M^i_t$ for the category $i$ of the target image from the mask decoder by utilizing the reference images and the text labels. We employ Binary Cross-Entropy (BCE) loss and Dice loss to train the VLP encoder. The loss of VLP-SAM is formulated as follows:

\begin{equation}
\begin{aligned}
{\rm Loss} & =\underbrace{-\frac{1}{n} \sum_{j=1}^n\left[y_j \log \left(p_j\right)+\left(1-y_j\right) \log \left(1-p_j\right)\right]}_{\text {BCE Loss }} \\
& +\underbrace{1-\frac{2 \sum_{j=1}^n\left(p_j \cdot y_j\right) +1}{\sum_{j=1}^n\left(p_j+y_j\right)+1}}_{\text {Dice Loss }}.
\end{aligned}
\end{equation}
where $n$ represents the total number of pixels, $y_j$ denotes the pixel $j$ of the ground truth mask $M^i_{gt}$, and $p_j$ denotes the pixel $j$ of the predicted mask $M^i_t$. During training, the parameters of SAM's image encoder, mask encoder, and VLM encoder are frozen to prevent overfitting. By leveraging the weights of pre-trained foundational models, VLP-SAM enables few-shot segmentation into any classes not included in the training data.

%---------------------------------------------------------------------
\section{Experiment}
%---------------------------------------------------------------------
\subsection{Settings}
To demonstrate the effectiveness of VLP-SAM, we conducted experiments on the PASCAL-$5^i$~\cite{PASCAL} and COCO-$20^i$~\cite{COCO} datasets following the few-shot setting~\cite{VRP-SAM, BAM, CyCTR}. We validate the generalization capability of VLP-SAM by dividing both datasets into four folds: three folds are used for training, and the remaining one fold is used for testing. PASCAL-$5^i$ consists of 20 classes (15 for training and 5 for testing), while COCO-$20^i$ consists of 80 classes (60 for training and 20 for testing), with no class overlap between training and testing. In each fold, 1,000 reference-target pairs are randomly sampled for testing. 

We compare VLP-SAM with a SAM-based few-shot segmentation model, VRP-SAM~\cite{VRP-SAM}, which does not input text labels. We use ResNet-50~\cite{ResNet}, CLIP~\cite{CLIP} (ViT-B/16~\cite{ViT}, ResNet-50), CLIP Surgery~\cite{CLIP-Surgery} (ViT-B/16) as the image encoder backbone. ResNet-50 is a pre-trained model on ImageNet~\cite{ImageNet}, and is only used in VRP-SAM because it cannot accept text as input. To ensure a fair comparison, all methods employ the AdamW optimizer~\cite{AdamW} with an initial learning rate of 1e-4, a batch size of 8, 100 epochs for PASCAL-$5^i$, 50 epochs for COCO-$20^i$, and the number of learnable queries of 50. In addition, VLP-SAM sets the input text to the VLM as ``a photo of a [label].'' Following previous work~\cite{VRP-SAM, Mather, HDMNet}, the evaluation metric is mean intersection over union (mIoU).

\subsection{Results}
\cref{tab:1} shows the performance of one-shot segmentation on PASCAL-$5^i$ and COCO-$20^i$. As shown in \cref{tab:1}, VLP-SAM (CLIP Surgery) achieves mIoU scores of 77.01 on PASCAL-$5^i$ and 59.92 on COCO-$20^i$ with 1.4M learnable parameters. Furthermore, VLP-SAM (CLIP Suegery) outperforms the previous state-of-the-art model VRP-SAM (ResNet-50) by a large margin ($6.3\%$ and $9.5\%$, respectively). Additionally, compared to the text-free model VRP-SAM (CLIP Surgery) with the same backbone, our method achieves superior performance by increasing 8.6 and 11.0 mIoU on the respective datasets. These results demonstrate the high accuracy of VLP-SAM for few-shot segmentation and its generalization capability to unseen classes.

%---------------------------------------------------------------------
\section{Discussion}
This section discusses the effectiveness of the proposed method VLP-SAM. The proposed method significantly outperforms the previous state-of-the-art model by inputting not only reference images but also text labels as prompts to SAM. This improvement is attributed to its semantic similarities with text labels that lead to segmenting complex objects such as ``dog-like cats'' or ``fully clothed people.'' Furthermore, by leveraging prompt embeddings with multiple perspectives, such as visual and semantic features of target objects, VLP-SAM demonstrates powerful generalization capability to new objects.

In addition, VLP-SAM is a scalable SAM-based few-shot segmentation model with minimal learnable parameters. Traditional SAM only accepts geometric prompts such as points, boxes, or masks. On the other hand, by introducing a novel SAM's prompt encoder with VLM, VLP-SAM removes the limitations of geometric prompts, enabling the input of any modality like text. The introduction of text will expand to applications such as prompt engineering and open vocabulary.

%---------------------------------------------------------------------
%\section{Ablation}
%\subsection{n}

%\subsection{a}

%---------------------------------------------------------------------
\section{Conclusion and Future Work}
In this paper, we proposed VLP-SAM, a novel SAM-based few-shot segmentation model that inputs not only reference images but also text labels as prompts. By learning a novel prompt encoder for SAM using a multimodal VLM, VLP-SAM achieves high accuracy and generality. In particular, experiments on two datasets demonstrated that VLP-SAM outperforms previous state-of-the-art models by a large margin. Additionally, VLP-SAM proved its generalization to novel objects, making it applicable to various scenarios.

In future work, we will focus on enhancing and reducing reference information. By leveraging the simple and scalable structure of the VLP encoder, it is possible to add various modals of reference information, such as 3D images and depth information as input to SAM. Conversely, by exploiting the richness of large-scale foundational models like SAM and VLM, reducing annotation masks or reference images could enhance usability by enabling segmentation with minimal input information.

\section*{Acknowledgment}
This work was supported by JSPS, Japan KAKENHI Grant Numbers 21H04600 and 24H00370.


\section*{Declaration of Competing Interest}
The authors declare that they have no known competing financial interests or personal relationships that could have appeared to influence the work reported in this paper.
%--------------------------------------------------------------------------------------------------------------------------------

%%Bibliography
\bibliographystyle{unsrt}  
\bibliography{references}  

%% Template for ISBI paper; to be used with:
%          spconf.sty  - ICASSP/ICIP LaTeX style file, and
%          IEEEbib.bst - IEEE bibliography style file.
% --------------------------------------------------------------------------
\documentclass{article}
\usepackage{spconf,amsmath,graphicx}

% It's fine to compress itemized lists if you used them in the
% manuscript
\usepackage{enumitem}
\usepackage{multirow}
\setlist{nosep, leftmargin=14pt}
\usepackage{booktabs}
\usepackage{caption}
\usepackage{subcaption}
\usepackage{mwe} % to get dummy images
\usepackage{url}

% Example definitions.
% --------------------
\def\x{{\mathbf x}}
\def\L{{\cal L}}

% Title.  
% ------
\title{Anatomical Grounding Pre-training for Medical Phrase Grounding}
%
% Single address.
% ---------------
% \name{Wenjun Zhang, Shakes Chandra, Aaron Nicolson}
% \address{University of Queensland}
%
% For example:
% ------------
%\address{School\\
%	Department\\
%	Address}
%
% Two addresses (uncomment and modify for two-address case).
% ----------------------------------------------------------
%\twoauthors
%  {A. Author-one, B. Author-two\sthanks{Some author footnote.}}
%	{School A-B\\
%	Department A-B\\
%	Address A-B}
%  {C. Author-three, D. Author-four\sthanks{The fourth author performed the work
%	while at ...}}
%	{School C-D\\
%	Department C-D\\
%	Address C-D}
%
% More than two addresses
% -----------------------
\name{Wenjun Zhang$^{\star}$ \qquad Shekhar S. Chandra$^{\star}$ \qquad Aaron Nicolson$^{\dagger}$}

\address{$^{\star}$University of Queensland\\$^{\dagger}$Australian e-Health Research Centre, CSIRO Health and Biosecurity, Brisbane, Australia}
% \address{$^{\dagger}$}Australian e-Health Research Centre, CSIRO Health and Biosecurity, Brisbane, Australia \\}

\begin{document}
%\ninept
%
\maketitle

\begin{abstract}

Medical Phrase Grounding (MPG) maps radiological findings described in medical reports to specific regions in medical images. The primary obstacle hindering progress in MPG is the scarcity of annotated data available for training and validation. We propose anatomical grounding as an in-domain pre-training task that aligns anatomical terms with corresponding regions in medical images, leveraging large-scale datasets such as Chest ImaGenome. Our empirical evaluation on MS-CXR demonstrates that anatomical grounding pre-training significantly improves performance in both a zero-shot learning and fine-tuning setting, outperforming state-of-the-art MPG models. Our fine-tuned model achieved state-of-the-art performance on MS-CXR with an mIoU of 61.2, demonstrating the effectiveness of anatomical grounding pre-training for MPG.

% Phrase grounding models maps phrases to specific regions in an image, while for medical phrase grounding, the phrase 

\end{abstract}

\section{Introduction}
MPG involves mapping a descriptive phrase containing a radiological finding to a specific region in a medical image \cite{10.1007/978-3-031-43990-2_35}. An MPG model could be used to visually connect findings in a radiologist report---whether produced by radiologist or by automatic report generation model---to the corresponding regions in the images. Findings accompanied by their associated bounding boxes are easier to verify, enhancing the reliability of reported information \cite{bernstein_can_2023, 10204026, doi:10.1148/ryai.2020190043}.

MPG is a specialised application within the broader field of phrase grounding. State-of-the-art general-domain phrase grounding models are pre-trained on large-scale phrase-to-region datasets and demonstrate strong zero-shot learning and few-shot transferability on downstream localisation tasks \cite{9879567, 9710994, 10.1007/978-3-031-72970-6_3}. However, despite their success in general-domain tasks, these models struggle to generalise to MPG, especially in a zero-shot learning setting. One possible reason is the significant domain shift from general-domain to medical-domain data \cite{zhao-titov-2023-transferability}. Furthermore, large-scale pre-training is challenging in the medical domain due to the scarcity of annotated MPG datasets, with only a small public benchmark dataset available \cite{10.1007/978-3-031-20059-5_1}. 

To overcome the challenges of limited MPG training data and the large domain gap between MPG and the general phrase grounding data, we propose to leverage anatomical grounding as an in-domain pre-training task for MPG, as demonstrated in Figure \ref{fig:concept} (middle). Anatomical grounding involves aligning text describing an anatomical region with the corresponding region within a medical image. This approach leverages the extensive anatomical text-to-region data available in datasets such as Chest ImaGenome \cite{wu2021chestimagenomedatasetclinical}, enabling effective fine-tuning or zero-shot learning for MPG tasks, where data is more limited \cite{ 10.1007/978-3-031-20059-5_1}. This pre-training step equips the model to recognise common anatomical landmarks, which radiologists frequently reference when describing findings in radiological reports. For instance, by learning to localise the \textit{right apical zone} with the Chest ImaGenome dataset, the model is more capable of localising findings such as a \textit{small right apical pneumothorax}.


\begin{figure}
    \centering
    \includegraphics[width=1\linewidth]{concept.png}
    \caption{Anatomical grounding as an in-domain pre-training task for Medical Phrase Grounding (MPG).}
    \label{fig:concept}
\end{figure}

We evaluated the effectiveness of anatomical grounding pre-training on MS-CXR, a MPG dataset, using two pre-trained general-domain phrase grounding models, TransVG \cite{9710016} and MDETR \cite{9710994}. We also evaluate it in both a zero-shot learning and a fine-tuning setting. Figure \ref{fig:concept} describes the training process; TransVG or MDETR is first pre-trained on anatomical grounding. They are then fine-tuned on MPG (if they are not evaluated in a zero-shot learning setting). Our empirical evaluation demonstrates that anatomical grounding pre-training significantly improves performance in a zero-shot learning setting, and significantly improves the performance of MDETR in a fine-tuning setting. We compare our anatomically grounded pre-trained models to state-of-the-art MPG models from the literature, and demonstrate that our models achieve an improvement in performance. The pre-trained models, and demo for this work are available at: \url{https://github.com/Claire1217/AGPT}.


\section{Related Work}
\subsection{General-domain Phrase Grounding}
Vision-language models pre-trained on large-scale image-text datasets, such as CLIP, have shown strong zero-shot learning and few-shot learning capabilities on global image understanding tasks \cite{pmlr-v139-radford21a}. GLIP extends this by pre-training on large-scale phrase grounding data \cite{9879567}. The learned representations demonstrate strong transferability to various local-level recognition tasks. Current pre-trained general-domain phrase grounding models are typically applied to two primary tasks: phrase localisation and referring expression comprehension. Phrase localisation focuses on identifying and locating multiple objects mentioned in a sentence. MDETR is a phrase localisation model, associating sub-phrases within a sentence with multiple object queries \cite{9710994}. In contrast, TransVG is a referring expression comprehension model---it detects a single object or region in an image for a whole sentence \cite{9710016}.

\subsection{Medical Phrase Grounding}
Due to the scarcity of annotated data, MPG has received limited attention in the literature. Boecking \textit{et al.} introduced MS-CXR, a phrase grounding chest X-ray benchmark dataset \cite{10.1007/978-3-031-20059-5_1}. Their objective with the dataset was to evaluate the grounding performance of their self-supervised biomedical vision-language model (BioViL). BioViL demonstrates strong zero-shot learning capabilities, given that it is not trained for MPG. Recently, Chen \textit{et al.} directly fine-tuned TransVG on a split of MS-CXR in order to directly learn MPG, forming MedRPG \cite{10.1007/978-3-031-43990-2_35}. Here, a bounding box supervised loss and a specific contrastive loss were leveraged. Unlike these models, we pre-train on large-scale anatomical grounding data using Chest ImaGenome, in order to provide in-domain pre-training.

\subsection{Anatomical Information in Medical Imaging}
Anatomical information has been effectively used in tasks like pathology detection and classification to improve accuracy and localisation. For example, the Anatomy-Driven Pathology Detection (ADPD) model \cite{muller_anatomy-driven_2023} used easy-to-annotate anatomical regions as proxies for pathologies, helping to locate disease locations without detailed pathology-specific bounding boxes. AnaXNet \cite{agu_anaxnet_2021} used anatomical relationships to improve classification by identifying the exact regions where findings occur. Despite these successes, no work has applied anatomical information to medical phrase grounding. 

\section{Methodology}\label{sec:methodology}
Our work addresses \textbf{medical phrase grounding} (MPG),  which involves mapping a descriptive phrase containing radiological finding to a specific
region in a medical image. This can be defined as learning a function  \( f: P \times I \rightarrow B \), where \( P \) represents the set of medical phrases, \( I \) represents the set of medical images, and \( B \) represents the set of bounding boxes. Given a phrase \( p \in P \) and an image \( i \in I \), the model predicts a bounding box \( b \in B \) such that $b = f(p, i)$. Our approach introduces a novel training framework for MPG, which involves extending the pre-training of general phrase grounding models with an anatomical grounding pre-training. 

Anatomical grounding involves predicting bounding boxes for anatomical structures using textual descriptions of their locations. The task can be formulated as 
\( f_{\text{anat}}: A \times I \rightarrow B \). Specifically, for each anatomical term \( a \in A \) and image \( i \in I \), the model predicts a bounding box \( b \in B \) such that $b = f_{\text{anat}}(a, i; \theta_{\text{gen}})$, 
where \( \theta_{\text{gen}} \) are the initial general-domain pre-trained weights. Through anatomical grounding pre-training, we refine the weights to create anatomy-specific parameters  \( \theta_{\text{anat}} \). 

To enhance generalisation and robustness, we leverage GPT-4 to generate four additional synonymous variations for each anatomical location in the Chest ImaGenome dataset. This aligns with clinical practice, where radiologists frequently use interchangeable terms to describe the same region. For example, ``left lung base” might also be referred to as ``left basal lung” or ``left lower lung base”. The detailed augmentation of anatomical regions is included in the aforementioned code repository. 

\section{Datasets} \label{sec:dataset}

\paragraph*{Chest ImaGenome \cite{wu2021chestimagenomedatasetclinical}}
We use the Chest ImaGenome dataset for anatomical grounding pre-training. Chest ImaGenome is a scene graph-structured dataset that includes $242\,072$ images. It contains $1\,256$ combinations of relational annotations between 29 anatomical structures in chest X-rays, with bounding box coordinates and additional attributes organised as a scene graph per image. In this study, we use the names and bounding box coordinates of these 29 anatomical structures, focusing specifically on frontal images. Examples of anatomical structures include ``left lung base", ``left lung apical zone", and ``right hilar structures".

\paragraph*{MS-CXR \cite{10.1007/978-3-031-20059-5_1}} 
We use the MS-CXR dataset for the MPG task. It contains $1\,162$ medical phrase-bounding box pairs across eight pathologies, such as \textit{cardiomegaly} and \textit{pleural effusion}. The findings are manually annotated and described by radiologists, ensuring precise alignment between medical phrases and bounding boxes. Example phrases include ``Large right-sided pneumothorax", and ``Small bilateral pleural effusions". The whole dataset was used for testing for the zero-shot learning setting with the general-domain pre-trained and anatomical pre-trained phrase grounding models, while the train-test-val split from \cite{10.1007/978-3-031-43990-2_35} was used for the fine-tuning setting. 

\section{Experiment Setup}
\paragraph*{Model}
Experiments were conducted with two models, TransVG and MDETR. For TransVG, ResNet-50 and ClinicalBERT were used as the visual and text encoders, respectively, whereas ResNet-101 and RoBERTa-base were used for MDETR. Here, MDETR functions on a sentence-level, mapping a medical phrase to one region in an image. This differs from its standard function, where it maps multiple sub-phrases from a sentence to multiple regions in the image. Full-model anatomical grounding pre-training of MDETR resulted in an unstable training process, likely due to its multi-object detection task. To address this, we applied Low-Rank Adaptation (LoRA) \cite{Hu2021LoRA:Models} during anatomical grounding pre-training. This likely stabilised training by limiting trainable parameters to low-rank layers, preventing drastic weight updates and reducing instability during adaptation.

\paragraph*{Pre-training and Fine-Tuning}
For anatomical grounding pre-training, we process mini-batches of eight images, each paired with five anatomical regions chosen from five synonymous terms, creating 40 anatomical text-region pairs per mini-batch. For MPG fine-tuning, both models were trained on the MS-CXR training set with a mini-batch size of 12. During fine-tuning, all of the weights of MDETR were trainable, including the LoRA weights. The AdamW optimiser with a learning rate of 1e-4 and 1e-5 was used for pre-training and fine-tuning, respectively \cite{DBLP:conf/iclr/LoshchilovH19}. Each model was trained for 1 epoch during pre-training and 90 epochs during fine-tuning. Images were resized and padded to a size of 640$\times$640. During training, the images were augmented with colour jitter and Gaussian noise.

% When fine-tuning the anatomical pre-trained models on the training set of MS-CXR. The task is formulated as \( f_{\text{MPG}}: P \times I \rightarrow B \), where given a medical phrase \( p \in P \) and an image \( i \in I \), the task is to produce a bounding box \( b \in B \) as follows: $b = f_{\text{MPG}}(p, i; \theta_{\text{anat}})$. With fine-tuning, the weights are updated to \( \theta_{\text{MPG}} \). 

\paragraph*{Evaluation}
We used mIoU and accuracy (Acc) as metrics. For accuracy, a predicted bounding box was considered true if the mIoU with the ground truth bounding box was larger than 0.5. We evaluate the anatomical grounding pre-trained MDETR and TransVG models on the MS-CXR dataset in both zero-shot learning and fine-tuning settings. The self-supervised pre-trained models GLoRIA \cite{9710099} and BioViL \cite{10.1007/978-3-031-20059-5_1} were used for comparison. In the fine-tuning setting, we further fine-tuned the anatomical grounding pre-trained MDETR and TransVG models on the training split of MS-CXR (described in Section \ref{sec:dataset}). These were compared to MDETR and TransVG without anatomical grounding pre-training and MedRPG \cite{10.1007/978-3-031-43990-2_35}. For zero-shot learning and fine-tuning, the epoch with the highest validation mIoU was selected for testing.

\section{Results \& Discussion}
\subsection{Effectiveness of Anatomical Grounding Pre-training}
The performance of anatomical grounding pre-training is demonstrated in Table \ref{tab:anat_comparison}. Applying MDETR and TransVG to MPG in a zero-shot learning setting produced low scores on both metrics, underscoring the limitations of general-domain phrase grounding models for MPG. However, pre-training with anatomical grounding led to a statistically significant improvement in both models’ performance across both metrics for zero-shot learning of MPG. These results demonstrate that anatomical grounding pre-training improves the models’ ability to generalise to MPG.

\begin{table}[ht]
\small
\centering
\caption{Performance of \textbf{anatomical grounding pre-training (AGPT)} on MS-CXR. Underlined indicates a stat. sig. difference to the model without anatomical grounding pre-training ($p < 0.05)$.}
\renewcommand{\arraystretch}{0.85}
\begin{tabular}{lcccc}
\toprule
\multirow{2}{*}{\textbf{Model}} & \multicolumn{2}{c}{\textbf{Zero-shot}} & \multicolumn{2}{c}{\textbf{Fine-tuning}} \\
\cmidrule(lr){2-3} \cmidrule(lr){4-5}
                                & \textbf{Acc}   & \textbf{mIoU}   & \textbf{Acc}    & \textbf{mIoU}   \\
\midrule
TransVG           & 1.2         & 10.3         & 68.9          & \textbf{59.4}          \\
\quad+ AGPT   & \textbf{\underline{39.8}}         & \textbf{\underline{40.7}}         & \textbf{70.7}          & 59.2 \\
\addlinespace % Adds space between the two groups
MDETR              & 3.0         & 14.7         & 66.9          & 57.8         \\
\quad+ AGPT   &\textbf{ \underline{34.7}}         & \textbf{\underline{32.6}}         & \textbf{\underline{70.7} }         & \textbf{\underline{61.2}} \\
\bottomrule
\end{tabular}
\label{tab:anat_comparison}
\end{table}


When fine-tuning on the MS-CXR training set, anatomical grounding pre-training produced a statistically significant improvement across all metrics for MDETR. It also demonstrated an improvement with TransVG for Acc. This indicates that anatomical grounding pre-training is effective for MPG fine-tuning, particularly for certain types of models.

In Figure \ref{fig:viz}, we illustrate the models performing MPG in zero-shot learning settings on two examples: ``right apical pneumothorax" and ``mild cardiomegaly". Without anatomical grounding pre-training, both TransVG and MDETR fail to ground the phrases accurately. However, with anatomy pre-training, both models are able to ground the text to the correct anatomical region---the right apical zone for pneumothorax and the heart for cardiomegaly. Fine-tuning offers a further improvement in the grounding accuracy.

\begin{figure}[t]
    \centering
    \includegraphics[width=1\linewidth]{vizz.png}
    \caption{MPG with and without \textbf{anatomical grounding pre-training (AGPT)}. The top example contains the anatomical region within the text, whereas the bottom example does not. Blue and red boxes indicate the ground-truth and predicted bounding boxes, respectively.}
    \label{fig:viz}
\end{figure}



\begin{table*}[t]
\small
\centering
\caption{A comparison of \textbf{anatomical grounding pre-training (AGPT)} with other models in the literature in both zero-shot learning and fine-tuning settings with \textbf{mIoU} as the metric. $\dagger$ indicates scores sourced from the BioViL paper \cite{10.1007/978-3-031-20059-5_1}.}
\label{table:combined_iou_scores}
\renewcommand{\arraystretch}{0.85}
\begin{tabular}{l c c c c c c c c c c}
\toprule
\textbf{Model} & \textbf{Supervision} & \textbf{Cardio.} & \textbf{Opacity} & \textbf{Edema} & \textbf{Consol.} & \textbf{Pneu.} & \textbf{Atelect.} & \textbf{Pneumo.} & \textbf{Pl. Eff.} & \textbf{Avg} \\
\midrule
\multicolumn{11}{c}{\textbf{Zero-shot learning}} \\ 
\midrule
\textbf{GLoRIA \cite{9710099}}$^\dagger$ & Self-super. & 27.3 & 19.8 & 25.1 & 32.4 & 24.6 & 26.1 & 10.0 & 25.4 & 24.6 \\
\textbf{BioViL \cite{10.1007/978-3-031-20059-5_1}}$^\dagger$ & Self-super. & 37.5 & 20.9 & \textbf{27.5} & \textbf{34.6} & 31.5 & 30.2 & 13.5 & \textbf{31.5} & 28.4 \\
\cmidrule(lr){1-11}
\textbf{MDETR + AGPT} & Box-super. & 61.3 & 6.0 & 8.7 & 18.5 & 18.8 & 8.2 & 16.1 & 14.6 & 32.6 \\
\textbf{TransVG + AGPT} & Box-super. & \textbf{61.5} & \textbf{23.0} & 14.5 & 33.0 & \textbf{31.9} & \textbf{39.3} & \textbf{26.9} & 21.1 & \textbf{40.7} \\
\midrule
\multicolumn{11}{c}{\textbf{Fine-tuning}} \\ 
\midrule
\textbf{MedRPG \cite{10.1007/978-3-031-43990-2_35}} & Box-super. & 80.5 & 39.3 & \textbf{51.7} & 49.1 & 46.4 & \textbf{48.8} & 38.5 & 52.8 & 59.6 \\
\textbf{MDETR \cite{9710994}} & Box-super. & 79.6 & 43.1 & 45.5 & 45.8 & 40.1 & 36.0 & 39.1 & 50.5 & 57.8 \\
\textbf{TransVG \cite{9710016}} & Box-super. & 80.6 & \textbf{46.8} & 35.6 & 42.7 & \textbf{48.5} & 42.8 & 38.3 & 49.5 & 59.4 \\
\cmidrule(lr){1-11}
\textbf{MDETR + AGPT} & Box-super. & \textbf{81.2} & 45.1 & 25.2 & \textbf{56.3} & 38.9 & 47.4 & \textbf{43.1} & \textbf{57.2} & \textbf{61.2} \\
\textbf{TransVG + AGPT} & Box-super. & 79.1 & 37.6 & 43.0 & 45.4 & 45.9 & 47.7 & 41.9 & 54.1 & 59.2 \\
\bottomrule
\end{tabular}
\end{table*}
 
\subsection{Comparison to other MPG models}
First, we compare our anatomical grounding pre-trained MDETR and TransVG models (MDETR + AGPT and TransVG + AGPT, respectively) in a zero-shot learning setting, as shown in Table \ref{table:combined_iou_scores}. We compare these to two self-supervised models, GLoRIA \cite{9710099} and BioViL \cite{10.1007/978-3-031-20059-5_1}. Both MDETR + AGPT and TransVG + AGPT outperformed GLoRIA and BioViL. This indicates that anatomical grounding pre-training is more effective for zero-shot learning MPG than the self-supervised learning strategies of GLoRIA and BioViL. Furthermore, our fine-tuned MDETR + AGPT model attained an mIoU improvement of 1.6 over the current state-of-the-art model, MedRPG \cite{10.1007/978-3-031-43990-2_35}. 

% The baseline performance for the two models is from the BioViL paper.

\subsection{Effectiveness of Synonymous Anatomical Term Augmentation}
We conducted ablation studies to evaluate the impact of adding synonymous variations of the anatomical locations, as described in Section \ref{sec:methodology}. The results show that synonymous augmentation improved the scores for both TransVG and MDETR, with a stronger effect observed in TransVG. Notably, anatomical grounding pre-training with synonymous augmentation led to a 15.6\% improvement in zero-shot learning accuracy. This provides the model with a broader range of terms for the same anatomical location. This allows the model to better generalise to new phrases in a zero-shot learning setting.
\begin{table}[h!]
\small
\centering
\caption{Improvement in performance with when using synonymous variations of the anatomical locations. Underlined indicates a stat. sig. difference to the model without synonymous variations ($p < 0.05)$.}
\label{table:syn_augmentation_effect}
\renewcommand{\arraystretch}{0.85}
\begin{tabular}{lcccc}
\toprule
\multirow{2}{*}{\textbf{Model}}  & \multicolumn{2}{c}{\textbf{Zero-shot}} & \multicolumn{2}{c}{\textbf{Fine-tuning}} \\
\cmidrule(lr){2-3} \cmidrule(lr){4-5}
 & \textbf{Acc} & \textbf{mIoU} & \textbf{Acc} & \textbf{mIoU} \\
\midrule
TransVG   & \underline{+15.6} & \underline{+13.9} & \underline{+5.4} & \underline{+2.6} \\
MDETR     & \underline{+1.8}  & \underline{+1.2}   & +2.4 & +0.9 \\
\bottomrule
\end{tabular}
\end{table}

\vspace{-10pt}
\section{Conclusion}
In this paper, we introduced anatomical grounding pre-training to address the challenges of MPG, a task constrained by limited in-domain data and significant domain shifts from general-domain pre-trained models. Our methodology involved pre-training phrase grounding models on anatomical text-region pairs using the Chest ImaGenome dataset, followed by MPG-specific fine-tuning on the MS-CXR dataset. Empirical results demonstrated that anatomical grounding pre-training significantly improved zero-shot learning and fine-tuning performance on MPG, surpassing existing self-supervised and state-of-the-art MPG models. Additionally, our augmentation with synonymous anatomical terms further enhanced generalisability. This work demonstrates that leveraging anatomical grounding pre-training is an effective solution to the challenge of limited MPG data.

\section{Compliance with Ethical Standards}  
\vspace{-2mm} % Reduce space above section title
\noindent This study used public data from MIMIC-CXR (under PhysioNet’s credentialed license). Ethical approval was not required as confirmed by the license attached with the open access data.
\vspace{-2mm}

\section*{Acknowledgments}
\vspace{-2mm} % Reduce space before acknowledgments
\noindent No funding was received. The authors declare no competing interests.  

% Tighten bibliography spacing
\vspace{-5mm} % Reduce space before references


\bibliographystyle{IEEEtran}
\bibliography{isbi/ISBI_2024_template-master/references}
\end{document}



%--------------------------------------------------------------------------------------------------------------------------------
%\section*{Appendix}

\end{document}

\typeout{get arXiv to do 4 passes: Label(s) may have changed. Rerun}