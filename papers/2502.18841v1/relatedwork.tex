\section{Related work}
\label{RelatedWork}
A lot of work has been conducted in literature on movie reviews sentiment analysis. Starting with traditional machine learning, a step-by-step lexicon-based sentiment analysis using the R open-source software is presented in \cite{anandarajan2019sentiment}. In \cite{baid2017sentiment}, the authors implemented and compared traditional machine learning techniques like Naive Bayes (NB), K-Nearest Neighbours (KNN), and  Random Forests (RF) for sentiment analysis. Their results showed that Naive Bayes was the best classifier on the task. An ensemble generative approach for various machine learning approaches on sentiment analysis is used in \cite{mesnil2014ensemble}. KNN with the help of information gain technique was also used in \cite{daeli2020sentiment} on the task. In \cite{thongtan2019sentiment}, the authors proposed training document embeddings using cosine similarity, feature combination, and NB. KNN outperforms all other models in these works.
%The approaches had unsatisfactory accuracy, since they are typical traditional machine learning techniques. 

Deep learning approaches have also been implemented in movie reviews sentiment analysis. Recurrent Neural Network (RNN) and Convolutional Neural Network (CNN) architectures performances were explored for semantic analysis of movie reviews in \cite{shirani2014applications}. RNNs give satisfactory results, but they suffer from the problem of vanishing or exploding gradients when used with long sentences. Nonetheless, CNNs provide non-optimal accuracy on text classification. Coupled Oscillatory RNN (CoRNN), which is a time-discretization of a system of second-order ordinary differential equations, was proposed in \cite{rusch2020coupled} to mitigate the exploding and vanishing gradient problem, though the performance was still not convincing.  Bodapati et al.~\cite{bodapati2019sentiment} used LSTM on movie reviews sentiment analysis by investigating the impact of different hyper parameters like dropout, number of layers, and activation functions. Additionally, BiLSTM network for the task of text classification has also been applied via mixed objective function in \cite{singh2020revisiting}. BiLSTM achieved better results but at the expense of a very sophisticated architecture.  %These studies deployed very sophisticated architectures to attain good results. %in solving %the problem under study and still produced unsatisfactory results.

BERT has also been previously applied to sentiment analysis. BERT was used on SST-2 movie reviews benchmark for sentiment analysis in \cite{munikar2019fine}. In \cite{sousa2019bert}, the authors used BERT for stock market sentiment analysis. BERT was also applied on target-dependent sentiment classification in \cite{gao2019target}. However, there is still room for improvement considering their results. 

Therefore, in this work, BERT is fine-tuned  by coupling with BiLSTM for sentiment analysis on a 2-point scale. Afterwards, an application of sentiment  analysis is shown by computing overall polarity of movie reviews, which can also be utilised in recommending a movie.