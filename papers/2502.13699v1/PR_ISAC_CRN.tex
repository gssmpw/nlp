\documentclass[journal]{IEEEtran}

\IEEEoverridecommandlockouts

\usepackage{relsize}
\usepackage{moreverb}
\usepackage{mathrsfs}
\usepackage{amsmath}
\usepackage{amssymb}
\usepackage{stfloats}
\usepackage{graphicx}
\usepackage{makecell}
\usepackage{xcolor}
\usepackage{cite}
%\usepackage[colorlinks, linkcolor=red, anchorcolor=blue, citecolor=green]{hyperref}
%\usepackage{algorithm}      % format of the algorithm
%\usepackage{algorithmic}    % format of the algorithm
%\renewcommand{\algorithmicrequire}{ \textbf{Input:}} %Use Input in the format of Algorithm
%\renewcommand{\algorithmicensure}{ \textbf{Output:}} %UseOutput in the format of Algorithm

\usepackage{algorithm}

\usepackage{algorithmicx}
\usepackage{algpseudocode}

\usepackage{multirow}       % multirow for format of table
\usepackage{booktabs}
\usepackage{multirow}
\usepackage{tabularx,ragged2e,booktabs,caption}

\ifCLASSOPTIONcompsoc
\usepackage[tight,normalsize,sf,SF]{subfigure}
\else
\usepackage[tight,footnotesize]{subfigure}
\usepackage{subfigure}


\usepackage{amsmath}
\usepackage{amsthm}

\newtheorem{theorem}{Theorem}
\newtheorem{proposition}{Proposition}
%\interdisplaylinepenalty=2500
%\usepackage[cmintegrals]{newtxmath}
\hyphenation{op-tical net-works semi-conduc-tor}


\usepackage{setspace}
\usepackage[hidelinks]{hyperref}
\usepackage{graphicx}
\usepackage{flushend}
%\hypersetup{colorlinks=false, linktoc=all, linkcolor=black}
\UseRawInputEncoding

\begin{document}
	
	\title{\huge{Secure and Green Rate-Splitting Multiple Access Integrated Sensing and Communications}
		%\IEEEauthorrefmark{1}
%	\thanks{This work was supported in part by the National Natural Science Foundation of China (No.61701407, 61871327, and 61801218), in part by Guangdong Basic and Applied Basic Research Foundation (No.2021A1515110077), and in part by Natural Science Foundation of Shaanxi Province (No. 2022JQ-637). \emph{(Corresponding author: Rugui Yao.)}}
	\thanks{This work was supported by National Key Laboratory of Unmanned Aerial Vehicle Technology in NPU (Grant No. WR202404), Fundamental Research Funds for the Central Universities (Grant No. G2024WD0159, D5000240239), National Key Laboratory Fund
	Project for Space Microwave Communication (Grant
	No. HTKI2024KL504010), 2023 Zhongdian Tian’ao
	Innovation Theory and Technology Group Fund (Grant
	No. 2023JSQ0101), and Guangdong Basic and Applied Basic
	Research Foundation (No. 2021A1515110077). The work of A.-A. A. Boulogeorgos is supported by MINOAS. His research project MINOAS is implemented in the framework of H.F.R.I call “Basic research Financing (Horizontal support of all Sciences)” under the National Recovery and Resilience Plan “Greece 2.0” funded by the European Union –NextGenerationEU (H.F.R.I. Project Number: 15857).” \emph{(Corresponding author: Rugui Yao.)}}
	%
	%	\thanks{Copyright (c) 2015 IEEE. Personal use of this material is permitted. However, permission to use this material for any other purposes must be obtained from the IEEE by sending a request to pubs-permissions@ieee.org.}
    \thanks{Xudong Li is with the China Ship Development and Design Center, Wuhan 430064, China (e-mail: xudong\_good@mail.nwpu.edu.cn).}
	\thanks{Rugui Yao is with the School of Electronics and Information, Northwestern Polytechnical University, Xi’an 710072, China (e-mail: yaorg@nwpu.edu.cn).}
	\thanks{Theodoros A. Tsiftsis is with Department of Informatics Telecommunications, University of Thessaly, Lamia 35100, Greece, and also with the Department of Electrical and Electronic Engineering, University of Nottingham Ningbo China, Ningbo 315100, China (e-mail: tsiftsis@uth.gr).}	
	\thanks{Alexandros-Apostolos A. Boulogeorgos is with the Department of Electrical and Computer Engineering, University of Western Macedonia, Kozani 50100, Greece (email: al.boulogeorgos@ieee.org).}	
%		\thanks{Nan Qi is with the Key Laboratory of Dynamic Cognitive System of Electromagnetic Spectrum Space, Ministry of Industry and Information Technology, Nanjing University of Aeronautics and Astronautics, Nanjing 210016, China, and also with the National Mobile Communications Research Laboratory, Southeast University, Nanjing 210096, China (e-mail: nanqi.commun@gmail.com).}
}

{\Large{\author{\IEEEauthorblockN{Xudong Li, Rugui Yao,~\IEEEmembership{Senior Member,~IEEE}, Theodoros A. Tsiftsis,~\IEEEmembership{Senior Member,~IEEE}, \\and Alexandros-Apostolos A. Boulogeorgos,~\IEEEmembership{Senior Member,~IEEE}}}}}
%{\Large{\author{\IEEEauthorblockN{Xudong Li, Rugui Yao,Theodoros A. Tsiftsis, and Alexandros-Apostolos A. Boulogeorgos}}}}
		%
		%%\IEEEauthorblockA{\IEEEauthorrefmark{2}School of Electronics and Information, Northwestern Polytechnical University, Xi'an, Shaanxi, China}
	
	
	\maketitle
	
\begin{abstract}
%With researches on the \emph{integrated sensing and communication} (ISAC) increasingly rich, communication and sensing in the ISAC have received sufficient attention, whereas the exploration of security risk caused by open broadcast characteristics and energy consumption from massive data transmission is limited. In addition, diverse services in future \emph{sixth generation} (6G) scenarios put forward new requirements for ISAC architecture in complex network environments. Therefore, 
In this paper, we investigate the sensing, communication, security, and energy efficiency of \emph{integrated sensing and communication} (ISAC)-enabled \emph{cognitive radio networks} (CRNs) in a challenging scenario where communication quality, security, and sensing accuracy are affected by interference and eavesdropping. 
%In CRNs, imperfect \emph{channel state information} (CSI) of the eavesdropper is known and the green interference based on the ISAC sensing signal and the rate-splitting multiple access communication signal is constructed to jointly improve the above four vital functionalities. 
Specifically, we analyze the communication and sensing signals of ISAC as well as the communication signal consisting of common and private streams, based on \emph{rate-splitting multiple access} (RSMA) of multicast network. Then, the sensing signal-to-cluster-plus-noise ratio, the security rate, the communication rate, and the \emph{security energy efficiency} (SEE) are derived, respectively.
%To jointly enhance the aforementioned performances, we formulate a targeted optimization, i.e. SEE maximization through jointly optimizing the transmit signal \emph{beamforming} (BF) vector and the echo signal BF vector. Given that the resulting optimization problem is non-convex, an alternative optimization policy based on Taylor series expansion, majorization-minimization, semi-definite programming, and successive convex approximation is presented. The originally non-convex and intractable targeted optimization is decomposed into three simplified sub-optimization problems solved by alternating iterations. 
To simultaneously enhance the aforementioned performance metrics, we formulate a targeted optimization framework that aims to maximizing SEE by jointly optimizing the transmit signal \emph{beamforming} (BF) vectors and the echo signal BF vector to construct green interference using the echo signal, as well as common and private streams split by RSMA to refine security rate and suppress power consumption, i.e., achieving a higher SEE. Given the non-convex nature of the optimization problem, we present an alternative approach that leverages Taylor series expansion, majorization-minimization, semi-definite programming, and successive convex approximation techniques. Specifically, we decompose the original non-convex and intractable optimization problem into three simplified sub-optimization problems, which are iteratively solved using an alternating optimization strategy.
Simulations provide comparisons with state-of-the-art schemes, highlighting the superiority of the proposed joint multi-BF optimization scheme based on RSMA and constructed green interference in improving system performances.
\end{abstract}

\begin{IEEEkeywords}
	Beamforming, green interference, integrated sensing and communication, rate-splitting multiple access, security energy efficiency.
\end{IEEEkeywords}

\IEEEpeerreviewmaketitle


\section{Introduction}

%\subsection{Background}
\IEEEPARstart{W}{ith} the rapid promotion of \emph{fifth generation} (5G), \emph{beyond} (B5G) and even \emph{six generation} (6G) networks around the world, spectrum scarcity becomes the key limited factor of next-generation bandwidth-hungry applications \cite{ISAC_PLS_6_re}. Moreover, the current frequency band utilization is in the range of 15-20\% \cite{CR_data}. Therefore, it is necessary to explore emerging technologies that refine spectrum utilization and mitigate radio resource competitions, such as \emph{cognitive radios} (CRs) \cite{CR_intro_1,CR_intro_2}, and \emph{integrated sensing and communication} (ISAC) \cite{ISAC_PLS_4,ISAC_PLS_5,ISAC_PLS_52}. % and ISAC , as bright visions of 5G, B5G and \emph{six generation} (6G) , is undoubtedly a typical representative of high-spectrum-utilization technologies.

%CRs are characterized by flexibility, intelligence and reconfiguration. They sense the external environment and purposefully change operating parameters, in order to adapt their internal states to wireless channel alterations, thus, achieving highly reliable communication and efficient utilization of limited wireless spectrum resources \cite{ISAC_PLS_15_r,ISAC_PLS_16_r}. %Given researches on the CRN are rich, in what follows, we pay more attentions to the ISAC.
%ISAC refers to the unified design of sensing and communication functions through the joint schedule of air interface and protocol, time-frequency resource reuse, hardware equipment sharing and other means, then wireless networks carry out high-quality communication and obtain the sensing information of environment through the analysis of echo signals, fulfilling necessary applications like positioning, ranging, velocity measurement, imaging, detection, identification, leading to the enhancement of network performances and service capabilities of ISAC \cite{ISAC_PLS_15_r,ISAC_PLS_16_r,ISAC_PLS_7}.
%Therefore, ISAC has many remarkable advantages like multi-level discretion due to the use of diverse frequency bands, i.e., sub-6G to terahertz \cite{ISAC_PLS_8}; seamless and ubiquitous sensing driven by the universal availability of 5G networks \cite{ISAC_PLS_9}; high-accuracy sensing enabled by pencil-\emph{beamforming} (BF) \cite{ISAC_PLS_10}; cost-efficient sensing tasks \cite{ISAC_PLS_11};  and global perception through rich collaboration opportunities for sensing and communication \cite{ISAC_PLS_12}, respectively.
%The above characteristics of ISAC give it several advantages, such as: (1) rich frequency band resources, from sub-6GHz to millimeter and future terahertz can provide different degrees of sensing \cite{ISAC_PLS_8}; (2) 5G networks are widely deployed, and provide all-weather seamless ubiquitous sensing \cite{ISAC_PLS_9}; (3) 5G networks’ large bandwidth and large antenna array can be used to fulfill high-accuracy sensing \cite{ISAC_PLS_10}; (4) Reuse base station site, 5G base station equipment, and spectrum resources to achieve low cost perception \cite{ISAC_PLS_11}; (5) Rich collaboration opportunities for sensing and communication \cite{ISAC_PLS_12}. Sensing and communication share some limited resources, and provide more comprehensive and rich experience content for more scenarios while improving resource utilization.

%Therefore, owing to remarkable highlights of the ISAC, it provides communication and sensing, opening the application space beyond the traditional mobile communication network connections \cite{ISAC_PLS_13,ISAC_PLS_15,ISAC_PLS_16}. The sensing data is calculated and processed by the computing power base station, and is opened to industry applications through the ability to form a closed-loop application system that is deeply integrated with the industry, and can be modeled by different computing forces to power communication companies, navigation manufacturers and other units for business analysis and decision-making. In addition, sensing-assisted communication and communication-assisted sensing complement each other and are organically unified to achieve mutual benefit and win-win \cite{ISAC_PLS_17,ISAC_PLS_18}.


%\subsection{Literature Review}
How to optimize sensing and communication of the ISAC independently or jointly is a hotspot focused by scholars \cite{ISAC_PLS_15_r,ISAC_PLS_16_r,ISAC_PLS_7}. The authors of \cite{ISAC_PLS_30} studied the single-static sensing performance of the multi-target massive \emph{massive input massive output} (MIMO)-ISAC systems, and minimized the sum of \emph{Cramer-Rao lower bounds} (CRLBs) of target arrival directions under a communication rate constraint. The authors stated that their scheme achieved near-optimal
performance with less complexity through a single optimization of the sensing signal.
In \cite{ISAC_PLS_19}, the authors investigated a mobile-antenna-assisted ISAC system, aiming to improve the communication rate and echo signal \emph{signal-to-cluster-plus-noise ratio} (SCNR) by jointly optimizing the antenna coefficient and antenna position. An ISAC network assisted by a fluid antenna was designed in \cite{ISAC_PLS_24}. With the position and waveform of the fluid antenna jointly optimized, the sum communication rate of all users was improved. The authors of \cite{ISAC_PLS_20} presented an ISAC network in which \emph{unmanned aerial vehicle} (UAV) acted as a \emph{base station} (BS). To achieve a balance between sensing and communication, the position and transmit power of the UAV were optimized, and then the communication rate and CRLB of target sensing accuracy were maximized. %In a similar scenario, the authors of \cite{ISAC_PLS_25} further introduced an \emph{intelligent reflecting surface} (IRS) to jointly optimize sensing target scheduling, transmission symbol vector covariance matrix, IRS coefficient, and UAV flight path to maximize the communication rate while being subject to constraints on the echo signal SCNR.
Li \emph{et al.} analyzed an ISAC network in which a mobile UAV served as the sensing target. Under the communication \emph{quality of service} (QoS) and transmit power of the BS constraints, transmit signal BF vector and target allocation were optimized to improve the echo signal \emph{signal-to-interference-plus-noise ratio} (SINR) \cite{ISAC_PLS_23}. In \cite{ISAC_PLS_21}, the author studied ISAC composed of multiple BSs and multiple users acting as targets and communication users. %Owing to the optimization of the transmit signal BF vector, the interference among BSs and users was mitigated, and the SINRs of sensing and communications was improved. %A joint resource allocation scheme for \emph{vehicle-to-everything} (V2X) communication and sensing was reported in \cite{ISAC_PLS_22}, which adopted multi-agent \emph{deep deterministic policy gradient} (DDPG) algorithm to adjust network power, the communication rate and sensing accuracy get improved. %The authors of \cite{ISAC_PLS_26} articulated the optimization of the anti-interference resource allocation in ISAC networks. Under the anti-interference communication and sensing constraints, the channel allocation, power allocation and BF were optimized with \emph{deep reinforcement learning} (DRL) in order to jointly maximize the communication rate and sensing efficiency.

%The aforementioned contributions confront some hinders that limit both sensing accuracy and communication QoS, of which two of the most important are multiplicative fading and untreated mutual interference.
 Considering that the overall composition in ISAC is relatively complex and there is a non-negligible interference between the target sensing and the multicast communications, the scientific community turned its attention towards a variety of multiple access schemes to attain interference mitigation in ISAC \cite{MA_ISAC,Inter_ISAC}. The authors of \cite{ISAC_PLS_42} focused on a \emph{space division multiple access} (SDMA) scheme that used linear precoding to distinguish users in a spatial domain, relying entirely on treating any remaining multi-user interference as noise. The communication performance gain of the ISAC based on SDMA was evaluated in \cite{ISAC_PLS_43}. In contrast to SDMA, \emph{non-orthogonal multiple access} (NOMA) works with superimposing coding at the transmitter and \emph{successive interference cancellation} (SIC) coding at the receiver. NOMA superimposes users in the power domain, and forces users with better channel conditions to perform complete decoding through user grouping and sorting \cite{ISAC_PLS_42} to eliminate interference caused by other users. Different from NOMA, \emph{orthogonal multiple access} (OMA) assigns one resource to one user, resulting in lower resource utilization. 
The authors of \cite{ISAC_PLS_44} assessed the OMA-empowered and NOMA-empowered performance gains of communication and sensing of the semi-ISAC network based on the traversal rate and the traversal estimation of information rate. An uplink transmission scheme was articulated in \cite{ISAC_PLS_45} for NOMA-ISAC system to mitigate mutual interference between sensing and communication signals, and enhance communication convergence rate, reliability, and sensing accuracy. In \cite{ISAC_PLS_46_r}, the authors documented a joint optimization scheme of transmit signal BF, NOMA transmission time, and target sensing scheduling to maximize the sensing efficiency of ISAC systems, while ensuring a high communication QoS. 
A joint precoding optimization problem based on NOMA was solved in \cite{ISAC_PLS_47}, which maximized the security rate of multi-user through \emph{artificial noise} (AN), and achieved secure transmission, while satisfying the sensing performance constraint. 

Although studies on SDMA and NOMA to improve the ISAC performance is gradually deepening, there are some extremes to conventional multi-access architectures, such as SDMA and NOMA. Specifically, SDMA treats interference entirely as noise, seriously reducing the reliability. Instead, NOMA decodes interference one by one, implying that the effectiveness is hard to guarantee. Above shortcomings and deficiencies urge us to find a new scheme like \emph{rate-splitting multiple access} (RSMA)  \cite{ISAC_PLS_14}, adopting rate splitting based on linear precoding and SIC. As a consequence, RSMA decodes some interference and treats the remaining as noise, fully absorbing advantages of both SDMA and NOMA, and achieves high reliability and high effectiveness. Under the constraints of data rate and transmit power budget, RSMA structure and parameters were designed in \cite{ISAC_PLS_48} to minimize the CRLB of sensing response matrix at radar receivers. The authors of \cite{ISAC_PLS_49} presented an indicative example of an RSMA-ISAC waveform design that jointly optimized the minimum fairness rate among communication users and the CRLB of target detection under power constraints.

Coexistence of sensing and communication broadens the prospects for next-generation communication systems, while increasing the energy consumption. This necessitates the development of green ISAC systems simultaneously. In this direction, the authors of \cite{ISAC_PLS_29} introduced a power consumption minimization policy for the near-field ISAC system. In particular, the transmit signal BF vector was optimized to minimize network power consumption under the constraints of communication SINR, sensing target transmit beam pattern gain, and interference power. For \emph{intelligent reflecting surface} (IRS)-ISAC systems, the authors of \cite{ISAC_PLS_25_re,ISAC_PLS_36_re} maximized the \emph{energy efficiency} (EE) by jointly optimizing the transmit signal BF vector, the IRS reflection coefficient matrix, and the IRS deployment location. Energy-saving BF design of ISAC systems that aimed to maximize the EE by appropriately designing transmission waveforms in multi-user communication and target estimation scenarios was documented in \cite{ISAC_PLS_37} and \cite{ISAC_PLS_38}. %The authors of \cite{ISAC_PLS_39} investigated the efficient channel sharing auxiliary ISAC, and jointly optimized the transmit BF vector, receive BF vector, and multi-objective sensing scheduling to improve the sensing EE. 
In a multi-BS ISAC network, the energy consumption was reduced due to optimum task allocation, beam scheduling and transmit power control \cite{ISAC_PLS_40}. %The authors of \cite{ISAC_PLS_41} documented two joint spectrum division and power allocation schemes to increase channel mutual information, transmission rates, and EE of ISAC systems.

Additionally, due to the inherent open nature of downlink data transmission and broadcast mechanism, as well as the resource sharing between perception and communication of the ISAC network, it is vulnerable to security threats like eavesdropping and intercepting \cite{Intro_PLS_ISAC}. Consequently, it is of great significance and urgency to carry out researches on \emph{physical layer security} (PLS) in ISAC networks \cite{ISAC_PLS_50}. The effort that has been spent on the PLS-ISAC is limited. In an IRS-ISAC network, the authors of \cite{ISAC_PLS_27} maximized the minimum communication rate by optimizing the transmit BF vector, the receive BF vector, and the IRS reflection coefficient matrix under the constraints of echo signal power and security rate. For the same system model, BF was designed to maximize the minimum weighted beam pattern gain under security rate and transmit power constraints \cite{ISAC_PLS_33}. 
The authors of \cite{ISAC_PLS_28} focused on an ISAC network in which UAVs served as BSs to provided downlink data transmission for multiple users, sense and interfere with the \emph{eavesdropper} (Eve) to maximize security sum rate. 
%The authors of \cite{ISAC_PLS_31} and \cite{ISAC_PLS_51} extended the aforementioned contribution to IRS-ISAC systems. In detail, when the security outage probability of each user fell below a predetermined threshold, the transmit BF vector and the IRS reflection coefficient matrix were optimized in order to maximize the echo signal SINR as well as security rate. 
The authors of \cite{ISAC_PLS_32} used neural networks to optimize the transmit signal precoders to minimize the maximum SINR of the Eve. In a UAV-IRS-ISAC system, the DRL framework was employed to optimize the transmit signal BF vector and the coefficient matrix of the IRS loaded by a UAV to maximize the security sum rate \cite{ISAC_PLS_34}. In \cite{ISAC_PLS_35}, NOMA and AN were adopted to jointly optimize the radar correlation and transmit signal BF vector to maximize echo signal power. %The authors of reference [51] improve the safety rate of the IRS-ISAC system by jointly designing the BF vector of the transmitted signal and the reflection coefficient matrix of the active IRS.

%\subsection{Motivation and Contribution}
To sum up, \emph{state-of-the-art} (SOTA) ISAC works \cite{ISAC_PLS_23,ISAC_PLS_30,ISAC_PLS_33,ISAC_PLS_35,ISAC_PLS_48} improved sensing of the ISAC system by optimizing echo signal power, SINR, or target detection CRLB. In \cite{ISAC_PLS_25_re,ISAC_PLS_27,ISAC_PLS_28,ISAC_PLS_32,ISAC_PLS_34,ISAC_PLS_47}, the security or communication QoS was enhanced by optimizing the communication sum rate, security coordination rate, security rate, or eavesdropping SINR. Optimization frameworks for jointly enhancing sensing and communication. was articulated in \cite{ISAC_PLS_19,ISAC_PLS_20,ISAC_PLS_21,ISAC_PLS_44,ISAC_PLS_45,ISAC_PLS_49}. In \cite{ISAC_PLS_29,ISAC_PLS_36_re,ISAC_PLS_37,ISAC_PLS_38,ISAC_PLS_40}, the objective was to refine the EE. From the above, it gets obvious that joint optimization of multiple performances like sensing, communication, security, and EE of the ISAC has not been extensively investigated.

In PLS-ISAC, the dynamic change of channel state and the non-cooperative characteristics of unauthenticated Eve make the acquisition of the perfect CSI extremely difficult, while \cite{ISAC_PLS_28,ISAC_PLS_33,ISAC_PLS_34,ISAC_PLS_47} assume that Eve's perfect CSI can be ascertained. Undoubtedly, the assumption of perfect CSI provides tractability allow the derivation of performance bound with the cost accuracy and applicability as it. %there is a certain deviation between the calculation based on the perfect CSI hypothesis and the real situation, bringing non-negligible uncertainties to the analysis reliability. Hence, imperfect CSI that is more in line with the actual scenario deserves more attention.
The aforementioned contributions on the PLS-ISAC \cite{ISAC_PLS_27,ISAC_PLS_28,ISAC_PLS_32,ISAC_PLS_33,ISAC_PLS_34,ISAC_PLS_35} take insufficient advantage of the inherent interference in networks. From the perspective of security, AN improves the security rate. Meanwhile, it results in notable power overhead and computational complexity increase.

Based on the aforementioned discussions, this paper focuses on \emph{cognitive radio networks} (CRNs) where ISAC and multicast communications coexist. %For the sake of simplicity, we define that PR-ISAC CRNs refers to the networks studied in this paper. 
We articulate a joint BF optimization scheme based on RSMA and constructed green interference in order to jointly enhance sensing SCNR, communication capabilities, as well as \emph{security energy efficiency} (SEE) .
In more detail, our contributions can be summarized as follows:
\begin{enumerate} 
	\item Despite adverse conditions such as interference, eavesdropping, and Eve' imperfect CSI are confronted, we introduce novel PR-ISAC CRNs architecture supporting both ISAC and multicast communications, which increases the utilization of limited spectrum resources and improves performances. Building upon this, we quantify sensing, communication, security, and EE through echo signal SCNR, communication rate, security rate, and EE. %An optimization framework aiming at the SEE maximization is established to improve the above system performances.
	
	\item We propose an optimization framework designed to maximize the SEE by optimizing both the transmit BF vectors and the echo BF vector. Our approach leverages RSMA and ISAC signals to generate green interference from multicast communication signals. This green interference is strategically designed to minimize interference for Bob while increasing interference for Eve; thereby, enhancing security rate. Unlike AN, green interference is an inherent component of the system's energy, eliminating the need for additional power consumption and computational complexity. Consequently, this approach contributes to the overall improvement of SEE.
	%We formulate an optimization framework aiming at SEE maximization by optimizing the transmit BF vectors as well as the echo BF vector, and constructing green interference from multicast communication signal based on RSMA and ISAC sensing signal. Constructed green interference exerts less interference on Bob, while more interference radiates to Eve, achieving higher security rate. Different from AN, green interference is a part of system energy, which avoid causing additional power consumption and computational complexity, and is conducive to SEE improvement.
	
	\item We recognize that the original optimization objective and its associated constraints are inherently non-convex. To address this challenge, we propose an optimization algorithm that leverages Taylor series expansion, \emph{majorization-minimization} (MM), \emph{semi-definite programming} (SDP), and \emph{successive convex approximation} (SCA). This approach systematically transforms the originally non-convex and intractable problem into three more manageable sub-optimization problems. By employing an alternating iteration strategy, our method effectively converges toward a solution for the original optimization problem.
	%We consider that the original optimization objective and constraints are non-convex. Accordingly, we provide an optimization algorithm based on Taylor series expansion, \emph{majorization-minimization} (MM), \emph{semi-definite programming} (SDP), and \emph{successive convex approximation} (SCA) that transforms the originally non-convex and intractable optimization problem into three tractable sub-optimization problems, and perform alternating iterations to solve the original optimization problem.
	
	\item We carry out Monte Carlo simulations to validate the efficiency of the proposed multi-BF scheme. Full comparison with previously published works highlights the superiority of the proposed joint BF optimization scheme based on RSMA and green interference in enhancing sensing, communication, security, and EE.
	
\end{enumerate}
\begin{figure}[t]
	\centering
	\includegraphics[width=8.625cm,height=4.6cm]{system_model.pdf}
	\caption{PR-ISAC CRNs architecture.}\label{fig_system_model} 
\end{figure}
%\subsection{Organization and Notation}

The remainder of the paper is organized as follows: The channel model, PR-ISAC CRNs architecture, and signal models accompanied by performance indicators are provided in Section II. Section III formulates the optimization problem, and present the corresponding solutions. Numerical results and simulations are provided in Section IV. Section V concludes this paper by summarizing its main message and key remarks.

\textbf{Notations:} 
Matrices and vectors are denoted as uppercase boldface and lowercase boldface, respectively. ${\left(  \cdot  \right)^{\rm{H}}}$, ${\mathop{\rm Tr}\nolimits} \left(  \cdot  \right)$, ${\mathop{\rm rank}\nolimits} \left(  \cdot  \right)$, and ${\left\|  \cdot  \right\|_2}$ are the Hermitian transpose operation, trace operation, rank operation, and 2-norm operation. ${\mathbb{C}^{x \times y}}$ stands for the 2-dimension complex space. ${{\bf{I}}_Z}$ represents the $Z \times Z$ identity matrix, ${{\bf{I}}_{{Z \times 1}}}$ represents the $\left( {{Z} \times 1} \right)$-dimension unit vector, $ \otimes $ denotes the Kroneker product of two matrices, $\mathcal{O}$ is computational complexity, and ${\left[ x \right]^ + } = \max \left\{ {0,x} \right\}$.

\section{System, Channel, and Signal Models}
To facilitate the formulation and solution of the optimization problem, we describe the architecture, Rician-shadowed fading channel, legitimate communication signal and unauthenticated eavesdropping signal in the ISAC, the RSMA-based multicast communication signal, and the sensing signal in the ISAC.

\subsection{System Model}
As shown in Figure \ref{fig_system_model}, in CRNs, there are two sub-networks, namely ISAC as primary network and multicast communications as secondary network. 
In the primary network, $\rm BS_1$ is equipped with $M_1$ antennas provides the downlink data transmission service to Bob while sensing the target's location status information. Meanwhile, Eve, an unauthorized external eavesdropper equipped with a single antenna, attempts to wiretap confidential information that $\rm BS_1$ transmits to Bob.
In the secondary network, $\rm BS_2$ equipped with $M_2$ antennas employs the RSMA scheme to provide highly reliable and low-interference downlink data transmission for $N$ users. Obviously, in the PR-ISAC CRNs, the signal transmitted by $\rm BS_1$ interferes with $N$ communication users' received signals, while the signal transmitted by $\rm BS_2$ affects both Bob's and Eve's signal reception.

\subsection{Channel Model}
In the primary network, a portion of the \emph{line of sight} (LoS) component is refracted and scattered by buildings and trees, forming the \emph{non LoS} (NLoS) component that coexists with the LoS component.  Therefore, the downlink channel can be modeled as the superposition of a predominant LoS component and a sparse set of single-bounce
NLoS components. This situation can be accurately characterized by the Rician-shadowed fading channel model in \cite{ISAC_PLS_53}, which is taken into account to capture the statistical characteristics of communication and sensing. Specifically, we have
\begin{equation}\label{channel_model}
\begin{array}{l}
	{\bf{h}} = \sqrt {\rho \left( {{\vartheta _0},{\varphi _0}} \right)} {\alpha _{\rm{0}}}{{\bf{u}}_{\rm{h}}}\left( {{\vartheta _0},{\varphi _0}} \right) \otimes {{\bf{u}}_{\rm{e}}}\left( {{\vartheta _0},{\varphi _0}} \right) + \\ \displaystyle
	\frac{1}{{\sqrt T }}\sum\nolimits_{t = 1}^T {\sqrt {\rho \left( {{\vartheta _t},{\varphi _t}} \right)} {\alpha _t}{{\bf{u}}_{\rm{h}}}\left( {{\vartheta _t},{\varphi _t}} \right) \otimes {{\bf{u}}_{\rm{e}}}\left( {{\vartheta _t},{\varphi _t}} \right)} ,
\end{array}
\end{equation}
where $\rho\left( {\vartheta ,\varphi } \right)$ is the antenna directivity pattern, ${{\bf{u}}_{\rm{h}}}\left( {\vartheta ,\varphi } \right)$ and ${{\bf{u}}_{\rm{e}}}\left( {\vartheta ,\varphi } \right)$ are horizon and elevation steering vectors, respectively, $\alpha_0$ and $\alpha_t$ are path-loss parameters of the LoS and the $t$-th NLoS components, $\varphi $ and $\vartheta $ are the azimuth and elevation angles of departure, respectively, and $T$ is the number of NLoS components. ${{\bf{u}}_{\rm{h}}}\left( {{\vartheta _t},{\varphi _t}} \right)$ and ${{\bf{u}}_{\rm{e}}}\left( {{\vartheta _t},{\varphi _t}} \right)$ obtained as in (\ref{angle_1}) and (\ref{angle_2}), given at the top of this page, where $\eta$ is the phase constant, $l_{\rm x}$ and $l_{\rm y}$ are distances between two adjacent antenna feeds of horizon direction and elevation direction, and ${M_1} = {M_{{\rm{1,1}}}} \times {M_{{\rm{1,2}}}}$.
\begin{figure*}
	\begin{equation}
		\label{angle_1} \displaystyle
	{{\bf{u}}_{\rm{h}}}\left( {{\vartheta _t},{\varphi _t}} \right) = {\left[ {{e^{ - j\eta \left( {{M_{{\rm{1}},{\rm{1}}}} - 1} \right){2^{ - 1}}{l_{\rm{x}}}\sin {\vartheta _t}\cos {\varphi _t}}}, \cdots ,{e^{j\eta \left( {{M_{{\rm{1}},{\rm{1}}}} - 1} \right){2^{ - 1}}{l_{\rm{x}}}\sin {\vartheta _t}\cos {\varphi _t}}}} \right]^{\rm{T}}},
	\end{equation}
%	\hrulefill
	% \vspace*{4pt}
\end{figure*}
\begin{figure*}
	\begin{equation}
		\label{angle_2} \displaystyle
	{{\bf{u}}_{\rm{e}}}\left( {{\vartheta _t},{\varphi _t}} \right) = {\left[ {{e^{ - j\eta \left( {{M_{{\rm{1}},{\rm{2}}}} - 1} \right){2^{ - 1}}{l_{\rm{y}}}\cos {\vartheta _t}}}, \cdots ,{e^{j\eta \left( {{M_{{\rm{1}},{\rm{2}}}} - 1} \right){2^{ - 1}}{l_{\rm{y}}}\cos {\vartheta _t}}}} \right]^{\rm{T}}},
	\end{equation}
	\hrulefill
	% \vspace*{4pt}
\end{figure*}


In what follows, it is necessary to point out that unless otherwise specified, all channels considered are with the magnitudes of their elements following Rician-shadowed distributions by default.

\subsection{Communication Signal Model}
For $\rm BS_1$, let ${x_{{\rm{bo}}}}$ and ${x_{{\rm{ta}}}}$ represent the communication signal transmitted to Bob and the sensing signal transmitted to target, respectively. ${{\bf{w}}_{{\rm{bo}}}} \in {\mathbb{C}^{{M_1} \times 1}}$ and ${{\bf{w}}_{{\rm{ta}}}} \in {\mathbb{C}^{{M_1} \times 1}}$ are the BF vectors of ${x_{{\rm{bo}}}}$ and ${x_{{\rm{ta}}}}$, respectively. Before transmitting ${x_{{\rm{bo}}}}$ and ${x_{{\rm{ta}}}}$, $\rm BS_1$ superimposes ${x_{{\rm{bo}}}}$ and ${x_{{\rm{ta}}}}$ together. Thus, the final signal transmitted by $\rm BS_1$ can be expressed as
\begin{equation}\label{B1_signal}
{{\bf{x}}_1} = {{\bf{w}}_{{\rm{bo}}}}{x_{{\rm{bo}}}} + {{\bf{w}}_{{\rm{ta}}}}{x_{{\rm{ta}}}}.
\end{equation}
Additionally, given that $\rm BS_2$ uses the RSMA scheme to implement multicast communications, all signals transmitted  by $\rm BS_2$ to $N$ communication users can be divided into two parts, i.e.,  common stream and the private stream, namely $\left\{ {z_{\rm 1}^{\rm{c}},z_{\rm 1}^{\rm{p}}} \right\}$, $\left\{ {z_{\rm 2}^{\rm{c}},z_{\rm 2}^{\rm{p}}} \right\}$, $ \cdots  \cdots $, $\left\{ {z_{{\rm }N}^{\rm{c}},z_{{\rm },N}^{\rm{p}}} \right\}$, respectively. 
Then, $N$ common streams, $\left\{ {z_{\rm 1}^{\rm{c}},z_{\rm 2}^{\rm{c}}, \cdots ,z_{\rm \it N}^{\rm{c}}} \right\}$, are extracted,  combined, and encoded as the common information, $s_{\rm c}$ via a codebook shared by $N$ communication users.
Private streams, $\left\{ {z_{\rm 1}^{\rm{p}},z_{\rm 2}^{\rm{p}}, \cdots ,z_{\rm \it N}^{\rm{p}}} \right\}$, are encoded separately as private information $\left\{ {{s_{ 1}},{s_{ 2}}, \cdots ,{s_{ N}}} \right\}$, respectively. It is assumed that the signals $x_{\rm bo}$, $x_{\rm ta}$, $s_{\rm c}$, and $s_n$ are uncorrelated with zero mean and unit variance.
Then, we assume that ${{\bf{o}}_{\rm{c}}}\in {\mathbb{C}^{M_2 \times 1}}$ and ${{\bf{o}}_{{\it n}}}\in {\mathbb{C}^{M_2 \times 1}}$ are beamformers of the common information $s_{\rm i,c}$ and the private information of the $n$-th communication user ${s_{ n}}$, $n \in \left\{ {1,2, \cdots ,N} \right\}$, respectively. 
The signal transmitted by ${\rm BS_ 2}$ can be
\begin{equation}\label{BS2_signal}
{{\bf{x}}_2} = {{\bf{o}}_{\rm{c}}}{s_{\rm{c}}} + \sum\nolimits_{n = 1}^N {{{\bf{o}}_n}{s_n}}.
\end{equation}
Let ${{\bf{h}}_{{\rm{bo}}}} \in {\mathbb{C}^{{M_1} \times 1}}$, ${{\bf{h}}_{{\rm{e}}}} \in {\mathbb{C}^{{M_1} \times 1}}$, and ${{\bf{h}}_{{n}}} \in {\mathbb{C}^{{M_1} \times 1}}$ represent the channels from $\rm BS_1 $ to Bob, Eve, and the $n$-th user. Likewise, ${{\bf{g}}_{{\rm{bo}}}} \in {\mathbb{C}^{{M_2} \times 1}}$, ${{\bf{g}}_{{\rm{e}}}} \in {\mathbb{C}^{{M_2} \times 1}}$, and ${{\bf{g}}_{{n}}} \in {\mathbb{C}^{{M_2} \times 1}}$ represent the channels from $\rm BS_2 $ to Bob, Eve, and the $n$-th user. $n_{\rm bo}$, $n_{\rm e}$, and $n_{ n}$ are independent and identically distributed complex random Gaussian
noises with mean being 0 and variance being $\sigma^2$ at Bob, Eve, and $n$-th user.
The received signals at Bob, Eve, and $n$-th user are given by
\begin{equation}\label{Bob_signal}
\begin{array}{*{20}{l}} \displaystyle
	{{y_{{\rm{bo}}}} = {\bf{h}}_{{\rm{bo}}}^{\rm{H}}{{\bf{w}}_{{\rm{bo}}}}{x_{{\rm{bo}}}} + {n_{{\rm{bo}}}}}\\ \quad\;\;\, \displaystyle
	{ + {\bf{h}}_{{\rm{bo}}}^{\rm{H}}{{\bf{w}}_{{\rm{ta}}}}{x_{{\rm{ta}}}} + {\bf{g}}_{{\rm{bo}}}^{\rm{H}}{{\bf{o}}_{\rm{c}}}{s_{\rm{c}}} + \sum\nolimits_{n = 1}^N {{\bf{g}}_{{\rm{bo}}}^{\rm{H}}{{\bf{o}}_n}{s_n}} ,}
\end{array}
\end{equation}
\begin{equation}\label{Eve_signal}
\begin{array}{*{20}{l}} \displaystyle
	{{y_{\rm{e}}}{\rm{ = }}{\bf{h}}_{\rm{e}}^{\rm{H}}{{\bf{w}}_{{\rm{bo}}}}{x_{{\rm{bo}}}} + {n_{\rm{e}}}}\\ \displaystyle \quad
	{ + {\bf{h}}_{\rm{e}}^{\rm{H}}{{\bf{w}}_{{\rm{ta}}}}{x_{{\rm{ta}}}} + {\bf{g}}_{\rm{e}}^{\rm{H}}{{\bf{o}}_{\rm{c}}}{s_{\rm{c}}} + \sum\nolimits_{n = 1}^N {{\bf{g}}_{\rm{e}}^{\rm{H}}{{\bf{o}}_n}{s_n}} ,}
\end{array}
\end{equation}
\begin{equation}\label{n-th-user_signal}
\begin{array}{*{20}{l}} \displaystyle
	{{y_n} = {\bf{g}}_n^{\rm{H}}{{\bf{o}}_{\rm{c}}}{s_{\rm{c}}} + {\bf{g}}_n^{\rm{H}}{{\bf{o}}_n}{s_n} + {n_n}}\\ \displaystyle  \quad\;
+	{\sum\nolimits_{k \ne n} {{\bf{g}}_k^{\rm{H}}{{\bf{o}}_k}{s_k}}  + {\bf{h}}_n^{\rm{H}}{{\bf{w}}_{{\rm{bo}}}}{x_{{\rm{bo}}}} + {\bf{h}}_n^{\rm{H}}{{\bf{w}}_{{\rm{ta}}}}{x_{{\rm{ta}}}},}
\end{array}
\end{equation}
respectively. Since Eve is foreign, unauthorized, and non-cooperative to $\rm BS_1$ and $\rm BS_2$, they ascertain Eve's imperfect CSI. As such, we have 
\begin{equation}\label{BS1_Eve_CSI}
{{\bf{h}}_{\rm{e}}} = {{\bf{h}}_{{\rm{es}}}} + {{\bf{h}}_{{\rm{er}}}}, 
\end{equation}
\begin{equation}\label{BS2_Eve_CSI}
 {{\bf{g}}_{\rm{e}}} = {{\bf{g}}_{{\rm{es}}}} + {{\bf{g}}_{{\rm{er}}}},
\end{equation}
respectively, where ${{\bf{h}}_{{\rm{es}}}}$ and ${{\bf{g}}_{{\rm{es}}}}$ are the estimated CSIs of the $\rm BS_1$-Eve and $\rm BS_2$-Eve links, ${{\bf{h}}_{{\rm{er}}}}$ and ${{\bf{g}}_{{\rm{er}}}}$ represent the differences between the real and estimated CSIs of the $\rm BS_1$-Eve and $\rm BS_2$-Eve links, respectively, of which the 2-norms satisfy
\begin{equation}\label{BS1_Eve_CSI_er_value_range}
0 \le {\left\| {{{\bf{h}}_{{\rm{er}}}}} \right\|_2} \le {{\rm{e}}_{\rm{h}}},
\end{equation}
\begin{equation}\label{BS2_Eve_CSI_er_value_range}
0 \le {\left\| {{{\bf{g}}_{{\rm{er}}}}} \right\|_2} \le {{\rm{e}}_{\rm{g}}},
\end{equation}
respectively, where ${{\rm{e}}_{\rm{h}}}$ and ${{\rm{e}}_{\rm{g}}}$ are the upper bounds of 2-norms of ${{\bf{h}}_{{\rm{er}}}}$ and ${{\bf{g}}_{{\rm{er}}}}$.
To quantify the CSI uncertainty of the $\rm BS_1$-Eve link, there exist
\begin{equation}\label{BS1_Eve_bo_uncertainty}
{\left\| {{\bf{h}}_{\rm{e}}^{\rm{H}}{{\bf{w}}_{{\rm{bo}}}}} \right\|_2^2 = {\bf{w}}_{{\rm{bo}}}^{\rm{H}}\left( {{{\bf{h}}_{{\rm{es}}}}{\bf{h}}_{{\rm{es}}}^{\rm{H}} + {{\bf{\Delta }}_{\rm{h}}}} \right){{\bf{w}}_{{\rm{bo}}}},}
\end{equation}
\begin{equation}\label{BS1_Eve_ta_uncertainty}
	\begin{array}{l}
		\left\| {{\bf{h}}_{\rm{e}}^{\rm{H}}{{\bf{w}}_{{\rm{ta}}}}} \right\|_2^2 
		= {\bf{w}}_{{\rm{ta}}}^{\rm{H}}\left( {{{\bf{h}}_{{\rm{es}}}}{\bf{h}}_{{\rm{es}}}^{\rm{H}} + {{\bf{\Delta }}_{\rm{h}}}} \right){{\bf{w}}_{{\rm{ta}}}},
	\end{array}
\end{equation}
where ${{\bf{\Delta }}_{\rm{h}}} = {{\bf{h}}_{{\rm{es}}}}{\bf{h}}_{{\rm{er}}}^{\rm{H}} + {{\bf{h}}_{{\rm{er}}}}{\bf{h}}_{{\rm{es}}}^{\rm{H}} + {{\bf{h}}_{{\rm{er}}}}{\bf{h}}_{{\rm{er}}}^{\rm{H}}$ is the CSI error matrix of $\rm BS_1$-Eve link, and satisfies the compatibility and triangle inequality constraint. Then, we can get a range of values for the 2-norm of ${{\bf{\Delta }}_{\rm{h}}}$, i.e., ${{{\left\| {{{\bf{\Delta }}_{\rm{h}}}} \right\|}_2} \le {\rm{e}}_{\rm{h}}^{\rm{2}} + 2{{\rm{e}}_{\rm{h}}}{{\left\| {{{\bf{h}}_{{\rm{er}}}}} \right\|}_2} = {{\rm{e}}_{{\rm{h}},{\rm{UB}}}}}$,
where ${{\rm{e}}_{{\rm{h,UB}}}}$ is the upper bound of 2-norm of ${{\bf{\Delta }}_{\rm{h}}}$.
Similarly, we obtain $\left\| {{\bf{g}}_{\rm{e}}^{\rm{H}}{{\bf{o}}_{\rm{c}}}} \right\|_2^2 = {\bf{o}}_{\rm{c}}^{\rm{H}}\left( {{{\bf{g}}_{{\rm{es}}}}{\bf{g}}_{{\rm{es}}}^{\rm{H}} + {{\bf{\Delta }}_{\rm{g}}}} \right){{\bf{o}}_{\rm{c}}}$ and $\left\| {{\bf{g}}_{\rm{e}}^{\rm{H}}{{\bf{o}}_n}} \right\|_2^2 = {\bf{o}}_n^{\rm{H}}\left( {{{\bf{g}}_{{\rm{es}}}}{\bf{g}}_{{\rm{es}}}^{\rm{H}} + {{\bf{\Delta }}_{\rm{g}}}} \right){{\bf{o}}_n}$, where ${{\bf{\Delta }}_{\rm{g}}} = {{\bf{g}}_{{\rm{es}}}}{\bf{g}}_{{\rm{er}}}^{\rm{H}} + {{\bf{g}}_{{\rm{er}}}}{\bf{g}}_{{\rm{es}}}^{\rm{H}} + {{\bf{g}}_{{\rm{er}}}}{\bf{g}}_{{\rm{er}}}^{\rm{H}}$ is the CSI error matrix of the $\rm BS_2$-Eve link, and satisfies the compatibility and triangle inequality constraint. Then, we get a range of values for the 2-norm of ${{\bf{\Delta }}_{\rm{g}}}$, i.e., ${\left\| {{{\bf{\Delta }}_{\rm{g}}}} \right\|_2} \le {\rm{e}}_{\rm{g}}^{\rm{2}} + 2{{\rm{e}}_{\rm{g}}}{\left\| {{{\bf{g}}_{{\rm{er}}}}} \right\|_2} = {{\rm{e}}_{{\rm{g,UB}}}}$,
where ${{\rm{e}}_{{\rm{g,UB}}}}$ is the upper bound of 2-norm of ${{\bf{\Delta }}_{\rm{g}}}$.

In the ISAC, the decoding order is (i) sensing and (ii) communication signals. Hence, it is assumed that  the channel gain of the sensing signal is greater than that of the communication signal, i.e, $\left\| {{\bf{h}}_{{\rm{bo}}}^{\rm{H}}{{\bf{w}}_{{\rm{bo}}}}} \right\|_2^2 < \left\| {{\bf{h}}_{{\rm{bo}}}^{\rm{H}}{{\bf{w}}_{{\rm{ta}}}}} \right\|_2^2$ and $\left\| {{\bf{h}}_{{\rm{bo}}}^{\rm{H}}{{\bf{w}}_{{\rm{bo}}}}} \right\|_2^2 < \left\| {{\bf{h}}_{{\rm{bo}}}^{\rm{H}}{{\bf{w}}_{{\rm{ta}}}}} \right\|_2^2$
%In the ISAC, when the communication signal is decoded, the sensing signal is treated as interference. To remove the sensing signal before decoding the communication signal, it is necessary to ensure that the received sensing signal power is greater than the received communication signal power at Bob and Eve. Therefore, we assume that the channel gain of the sensing signal is greater than that of the communication signal, i.e

%(\ref{Bob_eliminate_sensing}) and (\ref{Eve_eliminate_sensing}) are the premises that Bob and Eve use the SIC to sense the target before decoding the communication signal.
The sensing SINR at Bob can be obtained as
\begin{equation}\label{Bob_sensing_SINR}
{\gamma _{{\rm{bo,ta}}}} = \frac{{\left\| {{\bf{h}}_{{\rm{bo}}}^{\rm{H}}{{\bf{w}}_{{\rm{ta}}}}} \right\|_2^2}}{{\left\| {{\bf{h}}_{{\rm{bo}}}^{\rm{H}}{{\bf{w}}_{{\rm{bo}}}}} \right\|_2^2 + \left\| {{\bf{g}}_{{\rm{bo}}}^{\rm{H}}{{\bf{o}}_{\rm{c}}}} \right\|_2^2 + \sum\nolimits_{n = 1}^N {\left\| {{\bf{g}}_{{\rm{bo}}}^{\rm{H}}{{\bf{o}}_n}} \right\|_2^2}  + {\sigma ^2}}}.
\end{equation}
To employ SIC and remove sensing signals and reserve communication signals, the following condition needs to be satisfied: $ {\gamma _{{\rm{bo,ta}}}} \ge {\gamma _{{\rm{th}}}}$,
where ${\gamma _{{\rm{th}}}}$ is the required SINR threshold for decoding sensing signals successfully. Then, with sensing signals eliminated at Bob, the communication SINR at Bob can be represented as 
\begin{equation}\label{Bob_communication_SINR}
{\gamma _{{\rm{bo,bo}}}} = \frac{{\left\| {{\bf{h}}_{{\rm{bo}}}^{\rm{H}}{{\bf{w}}_{{\rm{bo}}}}} \right\|_2^2}}{{\left\| {{\bf{g}}_{{\rm{bo}}}^{\rm{H}}{{\bf{o}}_{\rm{c}}}} \right\|_2^2 + \sum\nolimits_{n = 1}^N {\left\| {{\bf{g}}_{{\rm{bo}}}^{\rm{H}}{{\bf{o}}_n}} \right\|_2^2}  + {\sigma ^2}}}.
\end{equation}
Similarly, the sensing SINR at Eve is 
\begin{equation}\label{Eve_sensing_SINR}
{\gamma _{{\rm{e,ta}}}} = \frac{{\left\| {{\bf{h}}_{\rm{e}}^{\rm{H}}{{\bf{w}}_{{\rm{ta}}}}} \right\|_2^2}}{{\left\| {{\bf{h}}_{\rm{e}}^{\rm{H}}{{\bf{w}}_{{\rm{bo}}}}} \right\|_2^2 + \left\| {{\bf{g}}_{\rm{e}}^{\rm{H}}{{\bf{o}}_{\rm{c}}}} \right\|_2^2 + \sum\nolimits_{n = 1}^N {\left\| {{\bf{g}}_{\rm{e}}^{\rm{H}}{{\bf{o}}_n}} \right\|_2^2}  + {\sigma ^2}}}.
\end{equation}
In contrast to Bob, the Eve is constrained to be failed to decode
the target’s message, obtaining the communication signal with the sensing signal interference, and thus degradating the eavesdropping SINR, the following condition needs to be satisfied: $	{\gamma _{{\rm{bo,ta}}}} <  {\gamma _{{\rm{th}}}}$.
Therefore, the communication SINR at Eve can be respectively expressed as
 \begin{equation}\label{Eve_communication_SINR}
 {\gamma _{{\rm{e,bo}}}} = \frac{{\left\| {{{\bf{h}}_{\rm{e}}}{\bf{w}}_{{\rm{bo}}}^{\rm{H}}} \right\|_2^2}}{{\left\| {{{\bf{h}}_{\rm{e}}}{\bf{w}}_{{\rm{ta}}}^{\rm{H}}} \right\|_2^2 + \left\| {{{\bf{g}}_{\rm{e}}}{\bf{o}}_{\rm{c}}^{\rm{H}}} \right\|_2^2 + \sum\nolimits_{n = 1}^N {\left\| {{{\bf{g}}_{\rm{e}}}{\bf{o}}_n^{\rm{H}}} \right\|_2^2}  + {\sigma ^2}}}.
 \end{equation}
Thus, the security rate is given by ${R_{\rm{S}}} = {\left[ {{R_{{\rm{bo}}}} - {R_{\rm{e}}}} \right]^ + }$, with $	{R_{{\rm{bo}}}} = {\log _2}\left( {1 + {\gamma _{{\rm{bo,bo}}}}} \right)$ and $	{R_{\rm{e}}} = {\log _2}\left( {1 + {\gamma _{{\rm{e,bo}}}}} \right)$ being the legitimate rate at Bob and eavesdropping rate at Eve.

Meanwhile, $\rm BS_2$ provides RSMA-based downlink data transmission to $N$ users. According to RSMA, the common information $s_{\rm c}$ is decoded into a common stream first, and the private information $s_n$ is regarded as interference. Then, by applying SIC, the common stream is recoded, pre-encoded, and removed from the received signal. The private information $s_n$ of the $n$-th user is decoded into a private stream, and private informations of other users is treated as interference.
Consequently, the SINRs of the common and the private streams of the $n$-th user can be respectively expressed as
\begin{equation}\label{common_SINR}
{\gamma _{\rm{c}}} = \frac{{\left\| {{\bf{g}}_{n}^{\rm{H}}{{\bf{o}}_{\rm{c}}}} \right\|_2^2}}{{\sum\nolimits_{n = 1}^N {\left\| {{\bf{g}}_{n}^{\rm{H}}{{\bf{o}}_n}} \right\|_2^2}  + \left\| {{\bf{h}}_{n}^{\rm{H}}{{\bf{w}}_{{\rm{bo}}}}} \right\|_2^2 + \left\| {{\bf{h}}_{n}^{\rm{H}}{{\bf{w}}_{{\rm{ta}}}}} \right\|_2^2 + {\sigma ^2}}},
\end{equation}
\begin{equation}\label{private_SINR}
{\gamma _n} = \frac{{\left\| {{\bf{g}}_n^{\rm{H}}{{\bf{o}}_n}} \right\|_2^2}}{{\sum\nolimits_{j \ne n} {\left\| {{\bf{g}}_j^{\rm{H}}{{\bf{o}}_j}} \right\|_2^2}  + \left\| {{\bf{h}}_n^{\rm{H}}{{\bf{w}}_{{\rm{bo}}}}} \right\|_2^2 + \left\| {{\bf{h}}_n^{\rm{H}}{{\bf{w}}_{{\rm{ta}}}}} \right\|_2^2 + \sigma _n^2}}.
\end{equation}
The corresponding common and private stream rates can be respectively given by ${R_{\rm{c}}} = {\log _2}\left( {1 + {\gamma _{\rm{c}}}} \right)$ and ${R_{{n}}} = {\log _2}\left( {1 + {\gamma _{{n}}}} \right)$.

\subsection{Sensing Signal Model}
Since $\rm BS_1$ is fully aware of its own transmit signal ${{\bf{x}}_1}$, which is composed of the sensing signal ${{\bf{w}}_{{\rm{ta}}}}{x_{{\rm{ta}}}}$ and the communication signal ${{\bf{w}}_{{\rm{bo}}}}{x_{{\rm{bo}}}}$, ${{\bf{x}}_1}$ can be used to detect the target. Meanwhile, sensing is affected by clutter from the environment. Therefore, the echo signal received at $\rm BS_1$ can be expressed as
\begin{equation}\label{echo_signal}
\begin{array}{l} \displaystyle
	{{\bf{y}}_{{\rm{b1}}}} = \xi {{\bf{h}}_{{\rm{ta}}}}{\bf{h}}_{{\rm{ta}}}^{\rm{H}}\left( {{{\bf{w}}_{{\rm{bo}}}}{x_{{\rm{bo}}}} + {{\bf{w}}_{{\rm{ta}}}}{x_{{\rm{ta}}}}} \right)\\  \displaystyle \quad\;\;\;
	+ {{\bf{I}}_{{M_{1 \times 1}}}}{\bf{h}}_{{\rm{cl}}}^{\rm{H}}\left( {{{\bf{w}}_{{\rm{bo}}}}{x_{{\rm{bo}}}} + {{\bf{w}}_{{\rm{ta}}}}{x_{{\rm{ta}}}}} \right) + {{\bf{n}}_{{\rm{b1}}}},
\end{array}
\end{equation}
where ${{\bf{h}}_{{\rm{ta}}}} \in {\mathbb{C}^{{M_1} \times 1}}$ is the channel from $\rm BS_1$ to the target and ${{\bf{h}}_{{\rm{cl}}}} \in {\mathbb{C}^{{M_1} \times 1}}$ is the channel from the environment to $\rm BS_1$. 
Additionally, $\xi $ is the \emph{radar cross section} (RCS) coefficient with the mean square value being ${\kappa ^2}$, and ${{\bf{n}}_{{\rm{b1}}}} \sim CN\left( {0,{\sigma ^2}{{\bf{I}}_{{M_1} \times 1}}} \right)$ represents the complex random Gaussian noise at $\rm BS_1$ with a mean being 0 and a variance being $\sigma^2$. We model the clutter from the environment as complex random Gaussian noise with a mean being 0 and a variance being $\sigma^2$ \cite{ISAC_PLS_54}.

Then, a radar receiver filter vector at $\rm BS_1$ is defined as ${{\bf{a}}} \in {\mathbb{C}^{{M_1} \times 1}}$, and thus the further processed echo signal and SCNR can be respectively obtained as
\begin{equation}\label{a_echo_signal}
\begin{array}{l} \displaystyle
	{\bf{a}}_{{\rm{b1}}}^{\rm{H}}{{\bf{y}}_{{\rm{b1}}}} = \xi {\bf{a}}_{{\rm{b1}}}^{\rm{H}}{{\bf{h}}_{{\rm{ta}}}}{\bf{h}}_{{\rm{ta}}}^{\rm{H}}\left( {{{\bf{w}}_{{\rm{bo}}}}{x_{{\rm{bo}}}} + {{\bf{w}}_{{\rm{ta}}}}{x_{{\rm{ta}}}}} \right)\\ \displaystyle \qquad\quad \,
	+ {\bf{a}}_{{\rm{b1}}}^{\rm{H}}{{\bf{I}}_{{M_1} \times 1}}{\bf{h}}_{{\rm{cl}}}^{\rm{H}}\left( {{{\bf{w}}_{{\rm{bo}}}}{x_{{\rm{bo}}}} + {{\bf{w}}_{{\rm{ta}}}}{x_{{\rm{ta}}}}} \right) + {\bf{a}}_{{\rm{b1}}}^{\rm{H}}{{\bf{n}}_{{\rm{b1}}}},
\end{array}
\end{equation}
\begin{equation}\label{echo_SCNR}
{\gamma _{{\rm{b1}}}} = \frac{{{\kappa ^2}{{\bf{a}}^{\rm{H}}}{{\bf{h}}_{{\rm{ta}}}}{\bf{h}}_{{\rm{ta}}}^{\rm{H}}\left( {{{\bf{w}}_{{\rm{bo}}}}{\bf{w}}_{{\rm{bo}}}^{\rm{H}} + {{\bf{w}}_{{\rm{ta}}}}{\bf{w}}_{{\rm{ta}}}^{\rm{H}}} \right){\bf{h}}_{{\rm{ta}}}^{\rm{H}}{{\bf{h}}_{{\rm{ta}}}}{\bf{a}}}}{{2{\sigma ^2}{{\bf{a}}^{\rm{H}}}{\bf{a}}}}.
\end{equation}
Likewise, to successfully decode the echo signal at $\rm BS_1$, the following condition should be satisfied: ${\gamma _{{\rm{b1}}}} \ge {\gamma _{{\rm{th}}}}$.

\section{Optimization Formulation and Solution}
We aim at jointly enhancing the system's sensing, communication, security, and SEE. To this end, we design a joint BF optimization scheme to reduce the interference exerted on Bob, suppress Eve's eavesdropping, and improve the sensing accuracy and energy utilization based on RSMA and the constructed green interference. Specifically, our objective is to maximize the SEE of the entire PR-ISAC CRNs. The optimization objective SEE is related to $R_{\rm bo}$, $R_{\rm e}$, and $P_1+P_2+P_0$. Optimization of $\textbf{a}$ reduces $P_1$ while $\rm BS_1$ can successfully decode the echo signal. Optimization of $\textbf{w}_{\rm bo}$ leads to a larger $R_{\rm bo}$, optimizing $\textbf{w}_{\rm ta}$ can reduce $P_1$ and $R_{\rm e}$, $\textbf{o}_{\rm c}$ and $\textbf{o}_{n}$ are optimized to increase the multicast communication rate and reduce $R_{\rm e}$. The optimization problem can be mathematically expressed as

\begin{subequations} \label{P1}
	\begin{align}
		&{\bf{P1}}. \;\mathop {\max }\limits_{{\bf{a}},{{\bf{w}}_{{\rm{bo}}}},{{\bf{w}}_{{\rm{ta}}}},{{\bf{o}}_{\rm{c}}},{{\bf{o}}_n}} \frac{{{R_{\rm{S}}}\left( {{{\bf{w}}_{{\rm{bo}}}},{{\bf{w}}_{{\rm{ta}}}},{{\bf{o}}_{\rm{c}}},{{\bf{o}}_n}} \right)}}{{{P_1}\left( {{{\bf{w}}_{{\rm{bo}}}},{{\bf{w}}_{{\rm{ta}}}}} \right) + {P_2}\left( {{{\bf{o}}_{\rm{c}}},{{\bf{o}}_n}} \right) + {P_0}}}\\
		& {\rm{s}}{\rm{.t}}\quad {\rm{    }}\left\| {{{\bf{w}}_{{\rm{bo}}}}} \right\|_2^2 + \left\| {{{\bf{w}}_{{\rm{ta}}}}} \right\|_2^2 = {P_1} \le {P_{{\rm{1,max}}}},\\ \label{P1_C5}
		&  \quad\quad{\rm{    }}\left\| {{{\bf{o}}_{\rm{c}}}} \right\|_2^2 + \sum\nolimits_{n = 1}^N {\left\| {{{\bf{o}}_n}} \right\|_2^2}  = {P_2} \le {P_{{\rm{2,max}}}}, 	\\ \label{P1_C5-2}		
		&\quad\quad {R_{\rm{S}}} \ge {I_{\rm{S}}}\, \& \, {R_{\rm{c}}} \ge {I_{\rm{c}}}\, \&\, {R_n} \ge {I_{\rm p}},\\ \label{P1_C2}
		&\quad\quad  \left\| {{\bf{h}}_{{\rm{bo}}}^{\rm{H}}{{\bf{w}}_{{\rm{bo}}}}} \right\|_2^2 \le \left\| {{\bf{h}}_{{\rm{bo}}}^{\rm{H}}{{\bf{w}}_{{\rm{ta}}}}} \right\|_2^2,\\ \label{P1_C3}
		&\quad\quad \left\| {{\bf{h}}_{\rm{e}}^{\rm{H}}{{\bf{w}}_{{\rm{bo}}}}} \right\|_2^2 \le \left\| {{\bf{h}}_{\rm{e}}^{\rm{H}}{{\bf{w}}_{{\rm{ta}}}}} \right\|_2^2,\\ \label{P1_C3-2}
		&\quad\quad {\rm{     }}{\gamma _{{\rm{bo,ta}}}} \ge {\gamma _{{\rm{th}}}}\, \&\, {\gamma _{{\rm{e,ta}}}} < {\gamma _{{\rm{th}}}},		\\
		&\quad\quad \frac{{{\kappa ^2}{{\bf{a}}^{\rm{H}}}{{\bf{h}}_{{\rm{ta}}}}{\bf{h}}_{{\rm{ta}}}^{\rm{H}}\left( {{{\bf{w}}_{{\rm{bo}}}}{\bf{w}}_{{\rm{bo}}}^{\rm{H}} + {{\bf{w}}_{{\rm{ta}}}}{\bf{w}}_{{\rm{ta}}}^{\rm{H}}} \right){\bf{h}}_{{\rm{ta}}}^{\rm{H}}{{\bf{h}}_{{\rm{ta}}}}{\bf{a}}}}{{2{\sigma ^2}{{\bf{a}}^{\rm{H}}}{\bf{a}}}}\ge {\gamma _{{\rm{th}}}},	
	\end{align}	
\end{subequations}
where (\ref{P1}a) is the optimization objective on SEE maximization. (\ref{P1}b) and (\ref{P1}c) are constraints of $\rm BS_1$ and $\rm BS_2$ power consumptions, $P_{\rm 1,max}$ and $P_{\rm 2,max}$ are maximums of $P_1$ and $P_2$, $P_0$ is constant circuit power consumption. (\ref{P1}d) are constraints on security rate, common stream rate, and private stream rate, respectively. $I_{\rm S}$, $I_{\rm c}$, and $I_{\rm p}$ are thresholds of security rate, common stream rate, and private stream rate. (\ref{P1}e) and (\ref{P1}f) are constraints on the decoding sequence of the sensing signal and the communication signal. (\ref{P1}g) are constraints that Bob and $\rm BS_1$ successfully decode sensing signals and Eve fails to decode its received sensing signal, deemed as interference for Eve. What's more, we define ${P_{{\rm{sum}}}} = {P_1} + {P_2} + {P_0}$.

It can be observed that there is a deep coupling relationship between the variables to be optimized in \textbf{P1} and the optimization objective and corresponding constraints, which brings notable challenges for effectively and reliably solving \textbf{P1}. To this end, we present an alternative iterative optimization algorithm based on Taylor series expansion, MM, SDP, and SCA, respectively.
In particular, we decompose the original non-convex and intractable targeted optimization \textbf{P1} into three sub-optimization problems. The first sub-optimization problem focuses on the optimization of the echo signal BF vector at $\rm BS_1$. The second one considers the optimization of the transmit BF vector at $\rm BS_1$. The last one is the optimization of the transmit BF vector at $\rm BS_2$.
In what follows, it is defined ${{\bf{X}}_i} = {{\bf{x}}_i}{\bf{x}}_i^{\rm{H}}$.
The fractional form of the objective and the coupled variables {$\textbf{a}$, $\textbf{w}_{\rm bo}$, $\textbf{w}_{\rm ta}$, $\textbf{o}_{\rm c}$, $\textbf{o}_n$} render problem (\ref{P1}) intractable. Therefore, we split the original problem into three sub-problems, and propose an alternating optimization scheme to solve it.

\subsection{Optimization of the Echo Signal BF Vector at $\rm BS_1$}
To optimize the echo signal BF vector at $\rm BS_1$ $\textbf{a}$, we keep $\textbf{w}_{\rm bo}$, $\textbf{w}_{\rm ta}$, $\textbf{o}_{\rm c}$, and $\textbf{o}_n$ fixed, and we present the equivalent form of the constraint  (\ref{P1}h) as 
\begin{subequations} \label{PA1}
	\begin{align}
		&{\rm{        }}{\bf{P2.1}}{\rm{.}}\; \mathop {\max }\limits_{\bf{a}} \frac{{{\kappa ^2}}}{{2{\sigma ^2}}}\frac{{{{\bf{a}}^{\rm{H}}}\left( {{{\bf{H}}_{{\rm{ta - b1}}}}{{\bf{w}}_{{\rm{b1}}}}{\bf{w}}_{{\rm{b1}}}^{\rm{H}}{\bf{H}}_{{\rm{ta - b1}}}^{\rm{H}}} \right){\bf{a}}}}{{{{\bf{a}}^{\rm{H}}}{\bf{a}}}}\\
		&{\rm{s}}{\rm{.t}}\quad \, (\ref{P1}\rm b) \, \mbox{-}  \, (\ref{P1}\rm g).
	\end{align}	
\end{subequations}
where ${{\bf{H}}_{{\rm{ta,b1}}}} = {{\bf{h}}_{{\rm{ta}}}}{\bf{h}}_{{\rm{ta}}}^{\rm{H}} \in {\mathbb{C}^{{M_1} \times {M_1}}}$ and ${{\bf{w}}_{{\rm{b1}}}} = \left[ {{{\bf{w}}_{{\rm{bo}}}},{{\bf{w}}_{{\rm{ta}}}}} \right] \in {\mathbb{C}^{{M_1} \times 2}}$.

Herein, $\textbf{S}$ is a Hermitian matrix, which can be expressed as
\begin{equation}\label{Hermitian_matrix}
{{{\bf{S}}^{\rm{H}}} = {{\left[ {\frac{{{\kappa ^2}{{\bf{h}}_{{\rm{ta}}}}{\bf{h}}_{{\rm{ta}}}^{\rm{H}}\left( {{{\bf{w}}_{{\rm{bo}}}}{\bf{w}}_{{\rm{bo}}}^{\rm{H}} + {{\bf{w}}_{{\rm{ta}}}}{\bf{w}}_{{\rm{ta}}}^{\rm{H}}} \right){{\bf{h}}_{{\rm{ta}}}}{\bf{h}}_{{\rm{ta}}}^{\rm{H}}}}{{2{\sigma ^2}}}} \right]}^{\rm{H}}} = {\bf{S}}}.
\end{equation}
We diagonalize the Hermitian matrix $\textbf{S}$, define the eigenvalue diagonal matrix of $\textbf{S}$ as ${\bf{\tilde S}} = {\mathop{\rm diag}\nolimits} \left\{ {{\lambda _1},{\lambda _2}, \cdots ,{\lambda _{{M_1}}}} \right\} \in {\mathbb{C}^{{M_1} \times {M_1}}}$, define the eigenvector matrix of $\textbf{S}$ as ${\bf{Q}} = \left\{ {{{\bf{q}}_1},{{\bf{q}}_2}, \cdots ,{{\bf{q}}_{{M_1}}}} \right\} \in {\mathbb{C}^{{M_1} \times {M_1}}}$  with ${\bf{Q}}{{\bf{Q}}^{\rm{H}}} = {{\bf{I}}_{{M_1} \times {M_1}}}$. Then, (\ref{PA1}) can be rewritten as
\begin{subequations} \label{PA2}
	\begin{align}
		&{\rm{        }}{\bf{P2.2}}{\rm{.}}\;{\mathop {\max }\limits_{\bf{a}} \frac{{{{\bf{a}}^{\rm{H}}}{\bf{Q\tilde S}}{{\bf{Q}}^{\rm{H}}}{\bf{a}}}}{{{{\bf{a}}^{\rm{H}}}{\bf{Q}}{{\bf{Q}}^{\rm{H}}}{\bf{a}}}}}  \\
		&{\rm{s}}{\rm{.t}}\quad \, (\ref{P1}\rm b) \, \mbox{-} \, (\ref{P1}\rm g).
	\end{align}	
\end{subequations}
Besides, let ${\bf{\tilde Q}} = {{\bf{Q}}^{\rm{H}}}{\bf{a}} = {\left\{ {{{{\bf{\tilde q}}}_1},{{{\bf{\tilde q}}}_2}, \cdots ,{{{\bf{\tilde q}}}_{{M_1}}}} \right\}^{\rm{H}}}$. By applying ${\bf{\tilde Q}}$ to  (\ref{PA2}), we obtain
\begin{subequations} \label{PA3}
	\begin{align}
		&{\rm{        }}{\bf{P2.3}}{\rm{.}}\;{\mathop {\max }\limits_{\bf{a}} \frac{{\sum\nolimits_{j = 1}^{{M_1}} {{\lambda _j}{\bf{\tilde q}}_j^2} }}{{\sum\nolimits_{j = 1}^{{M_1}} {{\bf{\tilde q}}_j^2} }}}\\
		&{\rm{s}}{\rm{.t}}\quad \, (\ref{P1}\rm b) \, \mbox{-} \, (\ref{P1}\rm g).
	\end{align}	
\end{subequations}
Then, after defining ${\lambda _{\max }} = \max \left\{ {{\lambda _1},{\lambda _2}, \cdots ,{\lambda _{{M_1}}}} \right\}$ and ${\lambda _{\min }} = \min \left\{ {{\lambda _1},{\lambda _2}, \cdots ,{\lambda _{{M_1}}}} \right\}$, we obtain
\begin{equation}\label{PA3_max}
\frac{{\sum\nolimits_{j = 1}^{{M_1}} {{\lambda _j}{\bf{\tilde q}}_j^2} }}{{\sum\nolimits_{j = 1}^{{M_1}} {{\bf{\tilde q}}_j^2} }} \le \frac{{\sum\nolimits_{j = 1}^{{M_1}} {{\lambda _{\max }}{\bf{\tilde q}}_j^2} }}{{\sum\nolimits_{j = 1}^{{M_1}} {{\bf{\tilde q}}_j^2} }} = \frac{{{\lambda _{\max }}\sum\nolimits_{j = 1}^{{M_1}} {{\bf{\tilde q}}_j^2} }}{{\sum\nolimits_{j = 1}^{{M_1}} {{\bf{\tilde q}}_j^2} }} = {\lambda _{\max }},
\end{equation}
\begin{equation}\label{PA3_min}
\frac{{\sum\nolimits_{j = 1}^{{M_1}} {{\lambda _j}{\bf{\tilde q}}_j^2} }}{{\sum\nolimits_{j = 1}^{{M_1}} {{\bf{\tilde q}}_j^2} }} \ge \frac{{\sum\nolimits_{j = 1}^{{M_1}} {{\lambda _{\min }}{\bf{\tilde q}}_j^2} }}{{\sum\nolimits_{j = 1}^{{M_1}} {{\bf{\tilde q}}_j^2} }} = \frac{{{\lambda _{\min }}\sum\nolimits_{j = 1}^{{M_1}} {{\bf{\tilde q}}_j^2} }}{{\sum\nolimits_{j = 1}^{{M_1}} {{\bf{\tilde q}}_j^2} }} = {\lambda _{\min }}.
\end{equation}
Finally, (\ref{PA1}) can be simplified as
\begin{subequations} \label{PA4}
	\begin{align}
		&{\rm{        }}{\bf{P2}}{\rm{.}}\; \mathop {\max }\limits_{\bf{a}} \frac{{{{\bf{a}}^{\rm{H}}}{\bf{Sa}}}}{{{{\bf{a}}^{\rm{H}}}{\bf{a}}}} = {\lambda _{\max }}\\
		&{\rm{s}}{\rm{.t}}\quad \, (\ref{P1}\rm b) \, \mbox{-}  \, (\ref{P1}\rm g).
	\end{align}	
\end{subequations}
In this case, the optimization variable $\textbf{a}$ is equivalent to the eigenvector corresponding to the maximum eigenvalue of \textbf{S}, i.e., ${\bf{Sa}} = {\lambda _{\max }}{\bf{a}}$.
We use QR decomposition to compute the maximum of the target optimization, i.e., the maximum eigenvalue of \textbf{S}, and the corresponding eigenvector, i.e., the optimization variable $\textbf{a}$. The corresponding calculation process is provided in Algorithm \ref{algorithm_1}.


\subsection{Optimization of the Transmit BF Vectors at $\rm BS_1$}
For fixed $\textbf{a}$, $\textbf{o}_{\rm c}$, and $\textbf{o}_n$, we optimize $\textbf{w}_{\rm bo}$ and $\textbf{w}_{\rm ta}$ in \textbf{P1} jointly. 
The security rate in \textbf{P1} can be rewritten as
\begin{equation}\label{security_rewritten}
	\begin{array}{l} \displaystyle
		{R_{\rm{S}}} = {R_{\rm{1}}} - {R_{\rm{2}}} + {R_{\rm{3}}} - {R_{\rm{4}}}\\ \displaystyle \quad\;\;
		= {\log _2}\left( {{\mathop{\rm Tr}\nolimits} \left( {{{\bf{H}}_{{\rm{bo}}}}{{\bf{W}}_{{\rm{bo}}}}} \right) + {\alpha _{{\rm{bo}}}}} \right)\\ \displaystyle\quad\;\;
		- {\log _2}\left( {{\mathop{\rm Tr}\nolimits} \left( {{{\bf{H}}_{\rm{e}}}{{\bf{W}}_{{\rm{bo}}}}} \right) + {\mathop{\rm Tr}\nolimits} \left( {{{\bf{H}}_{\rm{e}}}{{\bf{W}}_{{\rm{ta}}}}} \right) + {\alpha _{\rm{e}}}} \right)\\ \displaystyle \quad\;\;
		+ {\log _2}\left( {{\mathop{\rm Tr}\nolimits} \left( {{{\bf{H}}_{\rm{e}}}{{\bf{W}}_{{\rm{ta}}}}} \right) + {\alpha _{\rm{e}}}} \right) - {\log _2}\left( {{\alpha _{{\rm{bo}}}}} \right),
	\end{array}
\end{equation}
where ${\alpha _{{\rm{bo}}}} = {\mathop{\rm Tr}\nolimits} \left( {{{\bf{G}}_{{\rm{bo}}}}{{\bf{O}}_{\rm{c}}}} \right) + \sum\nolimits_{n = 1}^N {{\mathop{\rm Tr}\nolimits} \left( {{{\bf{G}}_{{\rm{bo}}}}{{\bf{O}}_n}} \right)}  + {\sigma ^2}$, and ${\alpha _{\rm{e}}} = {\mathop{\rm Tr}\nolimits} \left( {{{\bf{G}}_{\rm{e}}}{{\bf{O}}_{\rm{c}}}} \right) + \sum\nolimits_{n = 1}^N {{\mathop{\rm Tr}\nolimits} \left( {{{\bf{G}}_{\rm{e}}}{{\bf{O}}_n}} \right)}  + {\sigma ^2}$ are constants.

\begin{algorithm}[t]
	\caption{Optimization of echo signal BF vector at $\rm BS_1$.}
	\label{algorithm_1}
	{\algorithmicrequire} Given $\textbf{w}_{\rm bo}$, $\textbf{w}_{\rm ta}$, $\textbf{o}_{\rm c}$, and $\textbf{o}_n$, and related variables in the defined models.
	
	{\algorithmicensure} Optimal echo signal BF vector $\textbf{a}$.
	\begin{algorithmic}[1] % remove [1] if you do not need row number in your algorithm
		\State Set the mean square value of RCS $\kappa$, iteration threshold $\delta$, $\varepsilon=0$, $\lambda_{0}=0$, $\lambda_{-1}=0$, and  tolerance threshold $\delta$;
		\While {all constraints in (\ref{PA4}\rm b) are satisfied}
		\Repeat 
		\State Solve the optimization objective in (\ref{PA4}) to obtain 
		\Statex \qquad \;\,\, the maximum eigenvalue $\lambda_\varepsilon$;
		\State Update ${{\bf{a}}_\varepsilon } = {\mathop{\rm QR}\nolimits} \left( {{\bf{S}},{\lambda _\varepsilon }} \right)$; 
		\State $\varepsilon  = \varepsilon  + 1$;
		\Until {${\lambda _\varepsilon } - {\lambda _{\varepsilon  - 1}} \le {\delta}$};
		\EndWhile
		\State Calculate optimal echo signal BF vector $\textbf{a}$ at $\rm BS_1$ with QR decomposition based on maximum eigenvalue, $\lambda_{\rm max}$;
	\end{algorithmic}
\end{algorithm} 

\begin{algorithm}[htbp]
	\caption{Optimization of the transmit BF vectors at $\rm BS_1$.}
	\label{algorithm_2}
	{\algorithmicrequire} Given $\textbf{a}$, $\textbf{o}_{\rm c}$, and $\textbf{o}_n$, and related variables in the defined models.
	
	{\algorithmicensure} Optimal transmit signal BF vectors $\textbf{w}_{\rm bo}$ and $\textbf{w}_{\rm ta}$.
	\begin{algorithmic}[1] % remove [1] if you do not need row number in your algorithm
		\State Set auxiliary variables $r_{\rm a}$ and $s_{\rm a}$, iteration threshold $\delta$, $\varepsilon=0$, ${\rm SEE}_{0}=0$, and ${\rm SEE}_{-1}=0$;
		\While {all constraints in (\ref{PBB-3}\rm b) are satisfied}
		\State Set $i = 0$, and initialize ${r_{{\rm a},0,i }}$ and ${s_{{\rm a},0,i }}$;
		\Repeat 
		\State $i  = i  + 1$;
		\State Solve the optimization objective in (\ref{PBB-3}) to obtain 
		\Statex \qquad \;\,\, maximum ${\rm SEE}=\left( {2{r_{\rm{a}}}\sqrt {{R_{{\rm{S}},\max }}\left( r \right)}  - r_{\rm{a}}^2{P_{{\rm{sum}}}}} \right)$;
		\State Update $\textbf{W}_{{\rm bo},{\varepsilon}}$, $\textbf{W}_{{\rm ta},{\varepsilon}}$, $r_{{\rm a, 0},i}$, and $s_{{\rm a, 0},i}$;
		\Until {${\rm{SE}}{{\rm{E}}_\varepsilon } - {\rm{SE}}{{\rm{E}}_{\varepsilon  - 1}} \le {\delta }$};
		\State Obtain solutions $\textbf{W}_{\rm bo}$ and $\textbf{W}_{\rm ta}$;
		\State Set ${{\bf{W}}_{{\rm{bo}},\varepsilon  + 1}}{\rm{ = }}{{\bf{W}}_{\rm{bo}}}$ and ${{\bf{W}}_{{{\rm ta}},\varepsilon  + 1}}{\rm{ = }}{{\bf{W}}_{{\rm ta}}}$;
		\While {${{\bf{W}}_{{\rm{bo}},\varepsilon  + 1}} \approx {{\bf{W}}_{{\rm{bo}},\varepsilon }}$, and ${{\bf{O}}_{{{n}},\varepsilon  + 1}} \approx {{\bf{O}}_{{{n}},\varepsilon }}$  are not \indent \indent  satisfied}
		\State $\varepsilon  = \varepsilon  + 1$;
		\EndWhile
		\EndWhile
		\State Employ \emph{singular value decomposition} (SVD) to $\textbf{W}_{\rm bo, \varepsilon}$ and $\textbf{W}_{\rm ta,\varepsilon}$, and then the transmit signal BF vectors $\textbf{w}_{\rm bo}$ and $\textbf{w}_{\rm ta}$ are ascertained, respectively;
	\end{algorithmic}
\end{algorithm} 


According to the MM, the following equality is satisfied: ${\log _2}\left( z \right) \le \frac{{z - {z_0}}}{{{z_0}}} + {\log _2}\left( {{z_0}} \right)$, where $z_0$ is a specific value of variable $z$. Therefore, the second item and the third item in (\ref{security_rewritten}) can be upper-bounded as
\begin{equation}\label{2nd_UB}
\begin{array}{l} \displaystyle
	{R_{\rm{2}}} \le \frac{{{\mathop{\rm Tr}\nolimits} \left( {{{\bf{H}}_{\rm{e}}}{{\bf{W}}_{{\rm{bo}}}}} \right) - {\mathop{\rm Tr}\nolimits} \left( {{{\bf{H}}_{\rm{e}}}{{\bf{W}}_{{\rm{bo,}}i}}} \right)}}{{{\mathop{\rm Tr}\nolimits} \left( {{{\bf{H}}_{\rm{e}}}{{\bf{W}}_{{\rm{bo,}}i}}} \right) + {\mathop{\rm Tr}\nolimits} \left( {{{\bf{H}}_{\rm{e}}}{{\bf{W}}_{{\rm{ta}}}}} \right) + {\alpha _{\rm{e}}}}}\\  \displaystyle \quad\,\,\,
	+ \frac{{{\mathop{\rm Tr}\nolimits} \left( {{{\bf{H}}_{\rm{e}}}{{\bf{W}}_{{\rm{ta}}}}} \right) - {\mathop{\rm Tr}\nolimits} \left( {{{\bf{H}}_{\rm{e}}}{{\bf{W}}_{{\rm{ta,}}j}}} \right)}}{{{\mathop{\rm Tr}\nolimits} \left( {{{\bf{H}}_{\rm{e}}}{{\bf{W}}_{{\rm{bo,}}i}}} \right) + {\mathop{\rm Tr}\nolimits} \left( {{{\bf{H}}_{\rm{e}}}{{\bf{W}}_{{\rm{ta,}}j}}} \right) + {\alpha _{\rm{e}}}}}\\  \displaystyle \quad\,\,\,
	+ {\log _2}\left( {{\mathop{\rm Tr}\nolimits} \left( {{{\bf{H}}_{\rm{e}}}{{\bf{W}}_{{\rm{bo,}}i}}} \right) + {\mathop{\rm Tr}\nolimits} \left( {{{\bf{H}}_{\rm{e}}}{{\bf{W}}_{{\rm{ta,}}j}}} \right) + {\alpha _{\rm{e}}}} \right),
\end{array}
\end{equation}
and
\begin{equation}\label{3rd_UB}
\begin{array}{l} \displaystyle
	{R_{\rm{3}}} \le \frac{{{\mathop{\rm Tr}\nolimits} \left( {{{\bf{H}}_{\rm{e}}}{{\bf{W}}_{{\rm{ta}}}}} \right) - {\mathop{\rm Tr}\nolimits} \left( {{{\bf{H}}_{\rm{e}}}{{\bf{W}}_{{\rm{ta,}}i}}} \right)}}{{{\mathop{\rm Tr}\nolimits} \left( {{{\bf{H}}_{\rm{e}}}{{\bf{W}}_{{\rm{ta,}}i}}} \right) + {\alpha _{\rm{e}}}}}\\ \displaystyle \quad\,\,\,
	+ {\log _2}\left( {{\mathop{\rm Tr}\nolimits} \left( {{{\bf{H}}_{\rm{e}}}{{\bf{W}}_{{\rm{ta,}}i}}} \right) + {\alpha _{\rm{e}}}} \right),
\end{array}
\end{equation}
where ${{{\bf{W}}_{{\rm{bo,}}i}}}$ is the $i$-th iteration result of ${{{\bf{W}}_{{\rm{bo}}}}}$, and ${{{\bf{W}}_{{\rm{ta,}}j}}}$ is the $j$-th iteration result of ${{{\bf{W}}_{{\rm{ta}}}}}$.
Based on (\ref{BS1_Eve_bo_uncertainty}) and (\ref{BS1_Eve_ta_uncertainty}), we obtain the minimum of $R_2$ in (\ref{R2_minimum}) and the maximum of $R_3$ in (\ref{R3_maximum}) at the top of this page, where ${{\bf{H}}_{{\rm{e,max}}}} = {{\bf{H}}_{{\rm{es}}}} + {\rm{e}}_{\rm{h,UB}}{{\bf{I}}_{{M_1} \times {M_1}}}$ and ${{\bf{H}}_{{\rm{e,min}}}} = {{\bf{H}}_{{\rm{es}}}} - {\rm{e}}_{\rm{h,UB}}{{\bf{I}}_{{M_1} \times {M_1}}}$.
\begin{figure*}
	\begin{equation}
		\label{R2_minimum} \displaystyle
\begin{array}{l} \displaystyle
	{R_{{\rm{2}},{\rm{min}}}} = \frac{{{\rm{Tr}}\left( {{{\bf{H}}_{{\rm{e}},{\rm{min}}}}{{\bf{W}}_{{\rm{bo}}}}} \right) - {\rm{Tr}}\left( {{{\bf{H}}_{{\rm{e}},{\rm{max}}}}{{\bf{W}}_{{\rm{bo}},i}}} \right)}}{{{\rm{Tr}}\left( {{{\bf{H}}_{{\rm{e}},{\rm{max}}}}{{\bf{W}}_{{\rm{bo}},i}}} \right) + {\rm{Tr}}\left( {{{\bf{H}}_{{\rm{e}},{\rm{max}}}}{{\bf{W}}_{{\rm{ta}}}}} \right) + {\alpha _{\rm{e}}}}} + \frac{{{\rm{Tr}}\left( {{{\bf{H}}_{{\rm{e}},{\rm{min}}}}{{\bf{W}}_{{\rm{ta}}}}} \right) - {\rm{Tr}}\left( {{{\bf{H}}_{{\rm{e}},{\rm{max}}}}{{\bf{W}}_{{\rm{ta}},j}}} \right)}}{{{\rm{Tr}}\left( {{{\bf{H}}_{{\rm{e}},{\rm{max}}}}{{\bf{W}}_{{\rm{bo}},i}}} \right) + {\rm{Tr}}\left( {{{\bf{H}}_{{\rm{e}},{\rm{max}}}}{{\bf{W}}_{{\rm{ta}},j}}} \right) + {\alpha _{\rm{e}}}}}\\ \displaystyle \qquad\quad
	+ {\log _2}\left( {{\rm{Tr}}\left( {{{\bf{H}}_{{\rm{e}},{\rm{min}}}}{{\bf{W}}_{{\rm{bo}},i}}} \right) + {\rm{Tr}}\left( {{{\bf{H}}_{{\rm{e}},{\rm{min}}}}{{\bf{W}}_{{\rm{ta}},j}}} \right) + {\alpha _{\rm{e}}}} \right),
\end{array}
	\end{equation}
%	\hrulefill
	% \vspace*{4pt}
\end{figure*}
\begin{figure*}
	\begin{equation}
		\label{R3_maximum} \displaystyle
	{R_{{\rm{3}},{\rm{max}}}} = \frac{{{\rm{Tr}}\left( {{{\bf{H}}_{{\rm{e}},{\rm{max}}}}{{\bf{W}}_{{\rm{ta}}}}} \right) - {\rm{Tr}}\left( {{{\bf{H}}_{{\rm{e}},{\rm{min}}}}{{\bf{W}}_{{\rm{ta}},i}}} \right)}}{{{\rm{Tr}}\left( {{{\bf{H}}_{{\rm{e}},{\rm{min}}}}{{\bf{W}}_{{\rm{ta}},i}}} \right) + {\alpha _{\rm{e}}}}} + {\log _2}\left( {{\rm{Tr}}\left( {{{\bf{H}}_{{\rm{e}},{\rm{max}}}}{{\bf{W}}_{{\rm{ta}},i}}} \right) + {\alpha _{\rm{e}}}} \right).
	\end{equation}
	\hrulefill
	% \vspace*{4pt}
\end{figure*}


The maximum $R_{\rm S}$ can be expressed as ${R_{{\rm{S}},\max }} = {R_1} - {R_{{\rm{2,min}}}} + {R_{{\rm{3,max}}}} - {R_{{\rm{4}}}}$.
 Then, \textbf{P1} is transformed into 
\begin{subequations} \label{PBBB1}
	\begin{align}
		&{\rm{        }}{\bf{P3.1}}{\rm{.}}\;\mathop {\max }\limits_{{{\bf{w}}_{{\rm{bo}}}},{{\bf{w}}_{{\rm{ta}}}}} \frac{{{R_{{\rm{S}},{\rm{max}}}}\left( {{{\bf{w}}_{{\rm{bo}}}},{{\bf{w}}_{{\rm{ta}}}}} \right)}}{{{P_1}\left( {{{\bf{w}}_{{\rm{bo}}}},{{\bf{w}}_{{\rm{ta}}}}} \right) + {P_2} + {P_0}}}\\
		&{\rm{s}}{\rm{.t}}\quad \, (\ref{P1}\rm b) \, \mbox{-} \, (\ref{P1}\rm h).
	\end{align}	
\end{subequations}
As shown in (\ref{PBBB1}a), we need to increase $R_{\rm S, max}$ and decrease $P_1$ to maximize SEE through optimization of $\textbf{w}_{\rm bo}$ and $\textbf{w}_{\rm ta}$. Given optimization function in (\ref{PBBB1}) is fractional and intractable, with a slack variable $r$ and an auxiliary variable $r_{\rm a}$ introduced, the equivalent sub-optimization of (\ref{PBBB1}) is given by
\begin{subequations} \label{PBB-2}
	\begin{align}
		&{\rm{        }}{\bf{P3.2}}{\rm{.}}\,\begin{array}{*{20}{l}}
			{\mathop {\max }\limits_{{{\bf{w}}_{{\rm{bo}}}},{{\bf{w}}_{{\rm{ta}}}},{r_{\rm{a}}}} 2{r_{\rm{a}}}\sqrt {{R_{{\rm{S}},\max }}}  - r_{\rm{a}}^2\left( {{P_1} + {P_2} + {P_0}} \right)}
		\end{array}\\
		&{\rm{s}}{\rm{.t}}\quad \, (\ref{P1}\rm b) \, \mbox{-} \, (\ref{P1}\rm h), \\
		& \quad \quad  \, {R_{\rm{S}}} \ge {\log _2}\left( {1 + r} \right).
	\end{align}	
\end{subequations}
Since (\ref{PBB-2}\rm c) is non-convex, we introduce an auxiliary variable $s_{\rm a}$ to transform inequality constraints into equality ones. Then, (\ref{PBB-2}\rm c) can be rewritten as
\begin{equation}\label{PB2_C2}
2{s_{\rm{a}}}\sqrt {1 + {\gamma _{{\rm{bo}},{\rm{bo}}}}}  - s_{\rm{a}}^2\left( {1 + {\gamma _{{\rm{e}},{\rm{bo}}}}} \right) \ge r.
\end{equation}
Let us define the $i$-th iteration of ${{\bf{w}}_{\rm{bo}}}$ and ${{\bf{w}}_{\rm ta}}$ are defined as ${{\bf{w}}_{{\rm{bo}},i}}$ and ${{\bf{w}}_{{\rm ta},i}}$; then corresponding approximate matrices are given by
\begin{equation}\label{w_bo_approximation}
	{{\bf{W}}_{{\rm{bo,}}i}} = {{\bf{w}}_{{\rm{bo,}}i}}{\bf{w}}_{\rm{bo}}^{\rm{H}} + {{\bf{w}}_{\rm{bo}}}{\bf{w}}_{{\rm{bo,}}i}^{\rm{H}} - {{\bf{w}}_{{\rm{bo,}}i}}{\bf{w}}_{{\rm{bo,}}i}^{\rm{H}},
\end{equation}
\begin{equation}\label{w_ta_approximation}
	{{\bf{W}}_{{\rm{ta,}}i}} = {{\bf{w}}_{{\rm{ta,}}i}}{\bf{w}}_{\rm{ta}}^{\rm{H}} + {{\bf{w}}_{\rm{ta}}}{\bf{w}}_{{\rm{ta,}}i}^{\rm{H}} - {{\bf{w}}_{{\rm{ta,}}i}}{\bf{w}}_{{\rm{ta,}}i}^{\rm{H}}.
\end{equation}
Finally, the non-convex targeted optimization \textbf{P1} can be rewritten in a convex form as follows
\begin{subequations} \label{PBB-3}
	\begin{align}
		&{\rm{        }}{\bf{P3}}{\rm{.}}\;\mathop {\max }\limits_{{{\bf{w}}_{{\rm{bo}}}},{{\bf{w}}_{{\rm{ta}}}},{r_{\rm a}},{s_{\rm a}}} 2{r_{\rm{a}}}\sqrt {{R_{{\rm{S}},\max }}\left( r \right)}  - r_{\rm{a}}^2{P_{{\rm{sum}}}}\\
		&{\rm{s}}{\rm{.t}}\quad \, (\ref{P1}\rm b) \, \mbox{-} \, (\ref{P1}\rm h), and\, (\ref{PB2_C2}).
	\end{align}	
\end{subequations}
Corresponding calculation process is summarized in Algorithm \ref{algorithm_2}.



\subsection{Optimization of the Transmit BF Vector at $\rm BS_2$}
For the third sub-optimization, we need to increase $R_{\rm S, max}$ and decrease $P_2$ to maximize SEE through optimization of $\textbf{o}_{\rm c}$ and $\textbf{o}_n$.
For given $\textbf{a}$, $\textbf{w}_{\rm bo}$, and $\textbf{w}_{\rm ta}$, and then $\textbf{o}_{\rm c}$ and $\textbf{o}_n$ are optimized  in \textbf{P1}. For the sake of simplicity, we define: ${{\bf{\tilde H}}_{{\rm{ta - b1}}}} = {{\bf{H}}_{{\rm{ta,b1}}}}{\bf{H}}_{{\rm{ta,b1}}}^{\rm{H}}$, ${{\bf{H}}_{{\rm{bo}}}} = {{\bf{h}}_{{\rm{bo}}}}{\bf{h}}_{{\rm{bo}}}^{\rm{H}}$, ${{\bf{H}}_{\rm{e}}} = {{\bf{h}}_{\rm{e}}}{\bf{h}}_{\rm{e}}^{\rm{H}}$, ${{\bf{W}}_{{\rm{bo}}}} = {{\bf{w}}_{{\rm{bo}}}}{\bf{w}}_{{\rm{bo}}}^{\rm{H}}$, and ${{\bf{W}}_{{\rm{ta}}}} = {{\bf{w}}_{{\rm{ta}}}}{\bf{w}}_{{\rm{ta}}}^{\rm{H}}$, respectively.
Then, we hold that
\begin{equation}\label{PB_C1}
\mathop {\max }\limits_{{{\bf{h}}_{{\rm{es}}}}} {\rm{Tr}}\left( {{{\bf{H}}_{\rm{e}}}{{\bf{W}}_{{\rm{bo}}}}} \right) = {\rm{Tr}}\left[ {\left( {{{\bf{H}}_{{\rm{es}}}} + {\rm{e}}_{\rm{h,UB}}{{\bf{I}}_{{M_1} \times {M_1}}}} \right){{\bf{W}}_{{\rm{bo}}}}} \right],
\end{equation}
\begin{equation}\label{PB_C2}
	\mathop {\min }\limits_{{{\bf{h}}_{{\rm{es}}}}} {\rm{Tr}}\left( {{{\bf{H}}_{\rm{e}}}{{\bf{W}}_{{\rm{bo}}}}} \right) = {\rm{Tr}}\left[ {\left( {{{\bf{H}}_{{\rm{es}}}} - {\rm{e}}_{\rm{h,UB}}{{\bf{I}}_{{M_1} \times {M_1}}}} \right){{\bf{W}}_{{\rm{bo}}}}} \right],
\end{equation}
where ${{\bf{I}}_{{M_{\rm{1}}}}}$ represents the identity matrix with rank being $M_1$.
\begin{proposition}\label{Proposition1}
The constraint on the security rate in (\ref{P1}d) can be converted to
\begin{equation}\label{SR_written}
\left( {\tau  - 1} \right){U_1}{U_2} \ge \tau {\rm{Tr}}\left[ {{{\bf{H}}_{{\rm{bo}}}}{{\bf{W}}_{\rm bo}}} \right]{\rm{Tr}}\left[ {{{\bf{H}}_{\rm{e}}}{{\bf{W}}_{\rm bo}}} \right],
\end{equation}
\begin{equation}\label{SR_1}
\begin{array}{l} \displaystyle
	{U_1} = {\rm{Tr}}\left( {{{\bf{H}}_{{\rm{bo}}}}{{\bf{W}}_{{\rm{bo}}}}} \right) + \left( {1 - \tau } \right){\sigma ^2}\\ \displaystyle \quad\;\,
	+ \left( {1 - \tau } \right){\rm{Tr}}\left( {{{\bf{G}}_{{\rm{bo}}}}{{\bf{O}}_{\rm{c}}} + {{\bf{G}}_{{\rm{bo}}}}\sum\nolimits_{n = 1}^N {{{\bf{O}}_n}} } \right),
\end{array}
\end{equation}
\begin{equation}\label{SR_2}
\begin{array}{l} \displaystyle
	{U_2} = \frac{\tau }{{\tau  - 1}}{\rm{Tr}}\left[ {{{\bf{H}}_{\rm{e}}}{{\bf{W}}_{{\rm{bo}}}}} \right] + \sigma _{\rm{e}}^2\\ \displaystyle \quad\;\,
	+ {\rm{Tr}}\left[ {{{\bf{G}}_{\rm{e}}}{{\bf{O}}_{\rm{c}}} + {{\bf{G}}_{\rm{e}}}\sum\nolimits_{n = 1}^N {{{\bf{O}}_n}} } \right] + {\rm{Tr}}\left( {{{\bf{H}}_{\rm{e}}}{{\bf{W}}_{{\rm{ta}}}}} \right),
\end{array}
\end{equation}
respectively, where $\tau  = {2^{{I_{\rm{S}}}}}$.
\end{proposition}
\begin{proof}
For brevity, the proof of proposition 1 is provided to Appendix A. 
\end{proof}
\begin{proposition}
	The right part of (\ref{SR_written}) can be rewritten as
	\begin{equation}\label{pro2}
	{\rm{Tr}}\left[ {{{\bf{H}}_{{\rm{bo}}}}{{\bf{W}}_{{\rm{bo}}}}} \right]{\rm{Tr}}\left[ {{{\bf{H}}_{\rm{e}}}{{\bf{W}}_{{\rm{bo}}}}} \right]{\rm{ = Tr}}{\left[ {{{\bf{h}}_{{\rm{bo}}}}{\bf{h}}_{\rm{e}}^{\rm{H}}{{\bf{W}}_{{\rm{bo}}}}} \right]^2}.
	\end{equation}
\end{proposition}
\begin{proof}
	Please refer to Appendix B.  
\end{proof}

\begin{proposition}
The following inequality holds: 
	\begin{equation}\label{pro3-1}
	{U_1} + {U_2} \ge {\left\| {\begin{array}{*{20}{c}}
				{2\sqrt {{\tau  \mathord{\left/
								{\vphantom {\tau  {\left( {\tau  - 1} \right)}}} \right.
								\kern-\nulldelimiterspace} {\left( {\tau  - 1} \right)}}} {\rm{Tr}}\left( {{{\bf{h}}_{{\rm{bo}}}}{\bf{h}}_{\rm{e}}^{\rm{H}}{{\bf{W}}_{\rm bo}}} \right)}\\ \displaystyle
				{ - {U_1} + {U_2}}
		\end{array}} \right\|_2}.
	\end{equation}
\end{proposition}
\begin{proof}
	Please refer to Appendix C.  
\end{proof}
Based on (\ref{BS1_Eve_CSI_er_value_range}) and (\ref{SR_2}), the range value of $U_2$ can be expressed as ${U_{2,{\rm{UB}}}} \ge {U_2} \ge {U_{2,{\rm{LB}}}}$, where
\begin{equation}\label{range_UB}
\begin{array}{l} \displaystyle
	{U_{2,{\rm{UB}}}} = \frac{\upsilon }{{\upsilon  - 1}}{\rm{Tr}}\left[ {\left( {{{\bf{H}}_{{\rm{es}}}} + {{\rm{e}}_{{\rm{h,UB}}}}{{\bf{I}}_{{M_1} \times {M_1}}}} \right){{\bf{W}}_{{\rm{bo}}}}} \right]\\ \displaystyle \quad\quad\;\;\,
	+ {\rm{Tr}}\left[ {{{\bf{T}}_{\rm{e}}}{{\bf{O}}_{\rm{c}}} + {{\bf{T}}_{\rm{e}}}\sum\nolimits_{n = 1}^N {{{\bf{O}}_n}} } \right] + {\sigma ^2}\\ \displaystyle \quad\quad\;\;\,
	+ {\rm{Tr}}\left[ {\left( {{{\bf{H}}_{{\rm{es}}}} + {{\rm{e}}_{{\rm{h,UB}}}}{{\bf{I}}_{{M_1} \times {M_1}}}} \right){{\bf{W}}_{{\rm{ta}}}}} \right],
\end{array}
\end{equation}
and
\begin{equation}\label{range_LB}
\begin{array}{l} \displaystyle
	{U_{2,{\rm{LB}}}} = \frac{\upsilon }{{\upsilon  - 1}}{\rm{Tr}}\left[ {\left( {{{\bf{H}}_{{\rm{es}}}} - {{\rm{e}}_{{\rm{h,UB}}}}{{\bf{I}}_{{M_1} \times {M_1}}}} \right){{\bf{W}}_{{\rm{bo}}}}} \right]\\ \displaystyle \quad\quad\;\;\,
	+ {\rm{Tr}}\left[ {{{\bf{T}}_{\rm{e}}}{{\bf{O}}_{\rm{c}}} + {{\bf{T}}_{\rm{e}}}\sum\nolimits_{n = 1}^N {{{\bf{O}}_n}} } \right] + {\sigma ^2}\\ \displaystyle \quad\quad\;\;\,
	+ {\rm{Tr}}\left[ {\left( {{{\bf{H}}_{{\rm{es}}}} - {{\rm{e}}_{{\rm{h,UB}}}}{{\bf{I}}_{{M_1} \times {M_1}}}} \right){{\bf{W}}_{{\rm{ta}}}}} \right].
\end{array}
\end{equation}

\begin{proposition}
Based on (\ref{pro3-1}), (\ref{range_UB}), and (\ref{range_LB}), the following inequality holds: 
\begin{equation}\label{pro4-1}
	{U_1} + {U_{{\rm{2,LB}}}} \ge {\left\| {\begin{array}{*{20}{c}}
				{2\sqrt {{\tau  \mathord{\left/
								{\vphantom {\tau  {\left( {\tau  - 1} \right)}}} \right.
								\kern-\nulldelimiterspace} {\left( {\tau  - 1} \right)}}} {\rm{Tr}}\left[ {{\bf{\tilde H}}{{\bf{W}}_{{\rm{bo}}}}} \right]}\\
				{ - {U_1} + {U_{{\rm{2,UB}}}}}
		\end{array}} \right\|_2},
\end{equation}
where ${\bf{\tilde H}} = {{\bf{h}}_{{\rm{bo}}}}{\bf{h}}_{{\rm{es}}}^{\rm{H}} + {{\rm{e}}_{{\rm{h,UB}}}}{{\bf{I}}_{{M_1} \times {M_1}}}$.
\end{proposition}
\begin{proof}
	Please refer to Appendix D.  
\end{proof}
Let $\chi  = \left\{ {{x_1},{x_2},{x_3},{x_4},{x_5},{x_6},{x_7}} \right\}$ be a set of auxiliary variables further introduced, and the equivalent form of \textbf{P1} can be expressed as
\begin{subequations} \label{PB1}
	\begin{align}
		&{\rm{        }}{\bf{P4.1}}{\rm{.}}\;\mathop {\max }\limits_{{{\bf{o}}_{\rm{c}}},{{\bf{o}}_n},\chi } {x_1}\left( {{{\bf{o}}_{\rm{c}}},{{\bf{o}}_n}} \right)\\
		&{\rm{s}}{\rm{.t}}\quad{x_2} - {x_3} - {x_4} \ge {x_1}{x_7},\\
		&\quad\quad  {\rm{Tr}}\left( {{{\bf{W}}_{{\rm{b1}}}}} \right) + {\rm{Tr}}\left( {{{\bf{O}}_{{\rm{c,}}n}}} \right) \le {x_7},\\
		& \quad\quad {\rm{Tr}}\left( {{{\bf{H}}_{{\rm{bo}}}}{{\bf{W}}_{{\rm{b1}}}}} \right) + {\rm{Tr}}\left[ {{{\bf{G}}_{{\rm{bo}}}}{{\bf{O}}_{{\rm{c,}}n}}} \right] \ge \left( {{2^{{x_2}}} - 1} \right){\sigma ^2},\\
		& \quad\quad{\rm{Tr}}\left[ {{{\bf{G}}_{{\rm{bo}}}}{{\bf{O}}_{{\rm{c,}}n}}} \right] \le \left( {{2^{{x_3}}} - 1} \right){\sigma ^2},\\
		&\quad\quad {\rm{Tr}}\left[ {\left( {{{\bf{G}}_{{\rm{es}}}} + {{\rm{e}}_{{\rm{g}},{\rm{UB}}}}{{\bf{I}}_{{M_2} \times {M_2}}}} \right){{\bf{O}}_{{\rm{c,}}n}}} \right] \le \left( {{2^{{x_6}}} - 1} \right){\sigma ^2},\\  
		&\quad\quad {\rm{Tr}}\left[ {{{\bf{G}}_{{\rm{es,min}}}}{{\bf{O}}_{{\rm{c,}}n}}} \right] + {\rm{Tr}}\left[ {{{\bf{H}}_{{\rm{es,min}}}}{{\bf{W}}_{{\rm{b1}}}}} \right] \ge \left( {{2^{{x_5}}} - 1} \right){\sigma ^2}, \\
	& \quad\quad   - \left( {{2^{{I_{\rm{c}}}}} - 1} \right)f\left( {{{\bf{O}}_n}} \right){\rm{ + Tr}}\left( {{{\bf{G}}_n}{{\bf{O}}_{\rm{c}}}} \right) \ge 0,\\
	&\quad\quad   - \left( {{2^{{I_{\rm{p}}}}} - 1} \right)f\left( {{{\bf{O}}_j}} \right){\rm{ + Tr}}\left( {\sum\nolimits_{n = 1}^N {{{\bf{G}}_n}{{\bf{O}}_n}} } \right) \ge 0,\\
	&\quad\quad  {x_5} - {x_6} \le {x_4},\\
	&\quad\quad {\rm rank}\left( {{{\bf{W}}_{{\rm{b1}}}}} \right) = {\rm rank}\left( {{{\bf{O}}_{\rm{c}}}} \right) ={\rm rank}\left( {{{\bf{O}}_n}} \right) = 1,\\
	&\quad\quad (\ref{pro4-1}), (\ref{P1} \rm b), (\ref{P1} \rm c),
	\end{align}	
\end{subequations}
where $f\left( {{{\bf{O}}_n}} \right) = \left[ {{\rm{Tr}}\left( {\sum\nolimits_{n = 1}^N {{{\bf{G}}_n}{{\bf{O}}_n}} } \right) + {\rm{Tr}}\left( {{{\bf{H}}_n}{{\bf{W}}_{{\rm{b1}}}}} \right) + \sigma _n^2} \right]$ and ${{\bf{O}}_{{\rm{c,}}n}} = {{\bf{O}}_{\rm{c}}} + \sum\nolimits_{n = 1}^N {{{\bf{O}}_n}}$.
Notice that (\ref{PB1}\rm b) is non-convex; thus another variable $x_8$ is provided to convert (\ref{PB1}\rm b) into ${x_2} - {x_3} - {x_4} \ge {x_8 ^2}$ and $f\left( {{x_7},{x_8}} \right) = x_1^{ - 1}x_7^{ - 1}x_8^2 \ge 1$, which leads to 
\begin{equation}\label{C1_transformation}
		\frac{{{x_2} - {x_3} - {x_4} + 1}}{2} \ge {\left\| {\begin{array}{*{20}{c}} \displaystyle
					{\frac{{{x_2} - {x_3} - {x_4} - 1}}{2}}\\ \displaystyle
					x_8 
			\end{array}} \right\|_2}.
\end{equation}
\begin{proposition}
For fixed $x_7$ and $x_8$, $x_{7,0}$ and $x_{8,0}$, taken into account, the following inequality holds: 
\begin{equation}\label{specific_value}
		2{x_{7,0}}{x_{8,0}}{x_8} \ge x_{7,0}^2{x_1} + x_{8,0}^2{x_7}.
\end{equation}
\end{proposition}
\begin{proof}
Please refer to Appendix E.  
\end{proof}
Derivation of (\ref{specific_value}) enables (\ref{PB1}\rm e) and (\ref{PB1}\rm f) to be respectively expressed as convex forms as follows:
\begin{equation}\label{PB1_e_convex}
\begin{array}{l} \displaystyle
	{\rm{Tr}}\left[ {{{\bf{G}}_{{\rm{bo}}}}\left( {{{\bf{O}}_{\rm{c}}} + {{\sum\nolimits_{n = 1}^N {\bf{O}} }_n}} \right)} \right] \le \\ \displaystyle
	{\sigma ^2}\left[ {{2^{{x_{3,0}}}}\left( {{x_3}\ln 2 - {x_{3,0}}\ln 2 + 1} \right) - 1} \right],
\end{array}
\end{equation}
\begin{equation}\label{PB1_f_convex}
\begin{array}{l} \displaystyle
	{\rm{Tr}}\left[ {\left( {{{\bf{G}}_{{\rm{es}}}} + {{\rm{e}}_{{\rm{g,UB}}}}{{\bf{I}}_{{M_2} \times {M_2}}}} \right)\left( {{{\bf{O}}_{\rm{c}}} + \sum\nolimits_{n = 1}^N {{{\bf{O}}_n}} } \right)} \right]\\ \displaystyle
	\le {\sigma ^2}\left[ {{2^{{x_{6,0}}}}\left( {{x_6}\ln 2 - {x_{6,0}}\ln 2 + 1} \right) - 1} \right].
\end{array}
\end{equation}

Let us define the $i$-th iteration of ${{\bf{o}}_{\rm{c}}}$ and ${{\bf{o}}_n}$ are defined as ${{\bf{o}}_{{\rm{c}},i}}$ and ${{\bf{o}}_{n,i}}$. Then, the corresponding approximate matrices can be obtained as ${{\bf{O}}_{{\rm{c,}}i}} = {{\bf{o}}_{{\rm{c,}}i}}{\bf{o}}_{\rm{c}}^{\rm{H}} + {{\bf{o}}_{\rm{c}}}{\bf{o}}_{{\rm{c,}}i}^{\rm{H}} - {{\bf{o}}_{{\rm{c,}}i}}{\bf{o}}_{{\rm{c,}}i}^{\rm{H}}$ and ${{\bf{O}}_{n,i}} = {{\bf{o}}_{n,i}}{\bf{o}}_n^{\rm{H}} + {{\bf{o}}_n}{\bf{o}}_{n,i}^{\rm{H}} - {{\bf{o}}_{n,i}}{\bf{o}}_{n,i}^{\rm{H}}$.
Finally, the non-convex targeted optimization \textbf{P1} can be rewritten in a convex-form as
\begin{subequations} \label{PB2}
	\begin{align}
		&{\rm{        }}{\bf{P4}}.\;\mathop {\max }\limits_{{{\bf{o}}_{\rm{c}}},{{\bf{o}}_n},\chi } {x_1}\\
		&{\rm{s}}{\rm{.t}}\quad (\ref{P1} \rm b), (\ref{P1} \rm c), (\ref{PB1}\rm c), (\ref{PB1}\rm d), (\ref{PB1}\rm g)~(\ref{PB1}\rm k),\\
		&\quad\quad (\ref{pro4-1}), (\ref{specific_value}), (\ref{PB1_f_convex}).
	\end{align}	
\end{subequations}
The calculation process of (\ref{PB2}) is presented in Algorithm \ref{algorithm_3}. 
%Finally, the complete solution process is summarized in Algorithm \ref{algorithm_4}.
Finally, the complete algorithm for
solving (\ref{P1}) is outlined in Algorithm \ref{algorithm_4}, where
Algorithms \ref{algorithm_1}-\ref{algorithm_3} are alternately executed in each iteration.

\begin{algorithm}[t]
	\caption{Optimization of the transmit BF vectors at $\rm BS_2$.}
	\label{algorithm_3}
	
	{\algorithmicrequire} Given $\textbf{a}$, $\textbf{w}_{\rm bo}$, and $\textbf{w}_{\rm ta}$, and related variables in the defined models.
	
	{\algorithmicensure} Optimal transmit signal BF vectors $\textbf{o}_{\rm c}$, and $\textbf{o}_n$.
	
	\begin{algorithmic}[1] % remove [1] if you do not need row number in your algorithm
		\State Set a set of auxiliary variables $\chi$ and $x_8$, iteration threshold $\delta$, $\varepsilon=0$, ${\rm SEE}_{0}=0$, and ${\rm SEE}_{-1}=0$;
		\State Set positive values ${\rm e _h}$, ${\rm e _g}$, ${\rm e _{\rm h, UB}}$, ${\rm e _{\rm g, UB}}$, ${I _{\rm{S}}}$, ${I _{\rm{c}}}$, ${I _{\rm{p}}}$, $P_{\rm 1, max}$, and $P_{\rm 2, max}$;
		\State Initialize  $\textbf{O}_{\rm c, \varepsilon}$ and $\textbf{O}_{n,\varepsilon}$ to satisfy constraints in (\ref{PB2}\rm b) and (\ref{PB2}\rm c);
		%\While {Eq. (48) is satisfied} 
		\While {all constraints in (\ref{PB2}\rm b) and (\ref{PB2}\rm c) are satisfied}
		\State Set $i = 0$, and initialize ${x_{2,0,i }}$, ${x_{6,0,i }}$, ${x_{7,0,i }}$, and ${x_{8,0,i }}$;
		\Repeat 
		\State $i  = i  + 1$;
		\State Solve the optimization objective in (\ref{PB2}) to obtain 
		\Statex \qquad \quad maximum SEE;
		\State Update $\textbf{O}_{\rm c, {\varepsilon}}$, $\textbf{O}_{n,\varepsilon}$, ${x_{2,0,i }}$, ${x_{6,0,i }}$, ${x_{7,0,i}}$, and ${x_{8,0,i }}$;
		\Until {${\rm{SE}}{{\rm{E}}_\varepsilon } - {\rm{SE}}{{\rm{E}}_{\varepsilon  - 1}} \le {\delta }$};
		\State Obtain solutions $\textbf{O}_{\rm c}$ and $\textbf{O}_{n}$;
		\State Set ${{\bf{O}}_{{\rm{c}},\varepsilon  + 1}}{\rm{ = }}{{\bf{O}}_{\rm{c}}}$ and ${{\bf{O}}_{{{n}},\varepsilon  + 1}}{\rm{ = }}{{\bf{O}}_{{n}}}$;
		
		\While {${{\bf{O}}_{{\rm{c}},\varepsilon  + 1}} \approx {{\bf{O}}_{{\rm{c}},\varepsilon }}$, and ${{\bf{O}}_{{{n}},\varepsilon  + 1}} \approx {{\bf{O}}_{{{n}},\varepsilon }}$  are not \indent \indent  satisfied}
		\State $\varepsilon  = \varepsilon  + 1$;
		\EndWhile
		
		\EndWhile
		\State Employ SVD to $\textbf{O}_{\rm c, \varepsilon}$ and $\textbf{O}_{n,\varepsilon}$, and then the transmit signal BF vectors $\textbf{o}_{\rm c}$ and $\textbf{o}_n$ are ascertained, respectively;
	\end{algorithmic}
\end{algorithm} 

\begin{algorithm}[t]
	\caption{Optimization of the SEE in (\ref{P1}) based on Algorithms 1, 2, and 3.}
	\label{algorithm_4}
	
	{\algorithmicrequire} Initialization values of $\textbf{a}$, $\textbf{w}_{\rm bo}$, $\textbf{w}_{\rm ta}$, $\textbf{o}_{\rm c}$, and $\textbf{o}_{n}$, and related variables in the defined models, the number of iterations $\varepsilon $, and tolerance threshold $\delta$, respectively.
	
	{\algorithmicensure} BF vectors $\textbf{a}$, $\textbf{w}_{\rm bo}$, $\textbf{w}_{\rm ta}$, $\textbf{o}_{\rm c}$, and $\textbf{o}_{n}$, respectively.
	
	\begin{algorithmic}[1] % remove [1] if you do not need row number in your algorithm
		\Repeat 
		\State Carry out Algorithm \ref{algorithm_1} to obtain $\textbf{a}_{ \varepsilon+1 }$ with $\textbf{w}_{\rm bo, \varepsilon }$, \Statex\qquad$\textbf{w}_{\rm ta, \varepsilon}$, $\textbf{o}_{\rm c, \varepsilon}$, and $\textbf{o}_{n, \varepsilon}$;
		\State Carry out Algorithm \ref{algorithm_2} to obtain $\textbf{w}_{\rm ta, \varepsilon+1}$ and $\textbf{o}_{\rm c, \varepsilon+1}$ 
		\Statex\qquad with $\textbf{a}_{ \varepsilon+1 }$, $\textbf{o}_{\rm c, \varepsilon}$, and $\textbf{o}_{n, \varepsilon}$;
		\State Carry out Algorithm \ref{algorithm_3} to obtain $\textbf{o}_{\rm c, \varepsilon+1}$ and $\textbf{o}_{n, \varepsilon+1}$ 
		\Statex\qquad with $\textbf{a}_{ \varepsilon+1 }$, $\textbf{w}_{\rm ta, \varepsilon+1}$, and $\textbf{o}_{\rm c, \varepsilon+1}$;
		\State $\varepsilon  = \varepsilon  + 1$;
		\Until Variation of adjacent iteration results is less than $\delta$;
		\State Employ QR decomposition to obtain echo signal BF vector $\textbf{a}$, employ SVD to $\textbf{W}_{\rm bo, \varepsilon}$, $\textbf{W}_{\rm ta,\varepsilon}$, $\textbf{O}_{\rm c, \varepsilon}$, and $\textbf{O}_{n,\varepsilon}$, and then the transmit signal BF vectors $\textbf{w}_{\rm bo}$, $\textbf{w}_{\rm ta}$, $\textbf{o}_{\rm c}$, and $\textbf{o}_n$ are ascertained, respectively;
	\end{algorithmic}
\end{algorithm} 

\subsection{Complexity analysis of Algorithm \ref{algorithm_4}}
For Algorithm \ref{algorithm_4}, there exist $M_1^2 + M_2^2\left( {N + 1} \right) + 1$ original variables, $\left( {N + 6} \right)$ slack variables, 2 $M_1$-size \emph{linear matrix inequality} (LMI) constraints, $(N+1)$ $M_2$-size LMI constraints, $(N+8)$ 1-size LMI constraints, and $(N+1)$ 3-size SOC constraints. Therefore, the computational complexity of Algorithm \ref{algorithm_4} is given in (\ref{Computational_complexity}) at the top of the next page, where $D  = {M_1^2 + \left( {N + 1} \right)M_2^2 + \left( {N +7} \right)}$.
\begin{figure*} 
	\begin{equation}\label{Computational_complexity}
{\cal O}\left\{ {\sqrt {2{M_1} + \left( {N + 1} \right){M_2} + \left( {N + 8} \right)} D\left[ {M_1^2\left( {{M_1} + D} \right) + \left( {N + 1} \right)M_2^2 \times \left( {{M_2} + D} \right) + \left( {N + 8} \right)\left( {1 + D} \right) + 3\left( {N + 1} \right) + {D^2}} \right]} \right\},
	\end{equation}
	\hrulefill
	% \vspace*{4pt}
\end{figure*}

%Iterative results, ${{\bf{O}}_{{\rm{c,}}i}}$ and ${{\bf{O}}_{n,i}}$ can be decomposed by \emph{singular value decomposition} (SVD), and the transmit BF vectors $\textbf{o}_{\rm c}$ and $\textbf{o}_n$ at $\rm BS_2$ are ascertained, respectively.

\section{Numerical Results and Discussions}
In this section, Monte Carlo simulations are performed to illustrate the effectiveness of the presented PR-ISAC CRNs and the BF optimization scheme. The key simulation parameters are set as: The bandwidth $f_{\rm m}$ = 80 MHz, the carrier frequency $f_{\rm c}$= 18 GHz, the coverage region of a cell $r$ = 120 m, numbers of transmit antennas $M_1$=12, $M_2$=12, number of communication users $N$=4, thresholds of security rate, common and private stream rates, decoding sensing SINR $I_{\rm S}$=1 bit/s/Hz, $I_{\rm c}$=1 bit/s/Hz, $I_{\rm p}$=1 bit/s/Hz, and $\gamma_{\rm th}$=0.5, noise power at $\rm BS_1$, Bob, Eve, $n$-th user $\sigma^2$=-80 dBmW, maximum transmit power at $\rm BS_1$ and $\rm BS_2$ $P_{\rm 1,max}$=10 dBW and $P_{\rm 2,max}$=10 dBW, respectively.
The rest of this section is structured as: Firstly, we show the convergence performance of Algorithm \ref{algorithm_4}. Next, the sensing SCNRs realized by different echo signal BF vector schemes are illustrated. Effects of different multiple access schemes on the security, power, and SEE of the PR-ISAC CRNs are provided. We give the realizable SEE under different Eve's CSI uncertainty parameters. We compare the achievable SEEs under different transmit signal BF optimization schemes. Finally, we provide the common and private stream beampatterns based on RSMA.

\begin{figure}[htbp]
	\centering
	\begin{minipage}{0.49\linewidth}
		\centering
		\includegraphics[width=1\linewidth]{fig.2_iteration.pdf}
		\caption{SEE v.s. iteration number for different $M_1$.}
		\label{iteration_com}%文中引用该图片代号
	\end{minipage}
	%\qquad
	\begin{minipage}{0.49\linewidth}
		\centering
			\includegraphics[width=1\linewidth]{fig.3_SCNR_com.pdf}
		\caption{$\gamma_{\rm b1}$ v.s. $P_{\rm 1, max}$ for different optimization schemes.}
		\label{SCNR_com} 
	\end{minipage}
\end{figure}

\begin{figure}[htbp]
	\centering
	\begin{minipage}{0.49\linewidth}
		\centering
	\includegraphics[width=1\linewidth]{fig.4_Rs_com.pdf}
\caption{$R_{\rm S}$ v.s. $P_{\rm 2, max}$ for different multiple access schemes.}\label{RS_com} 
	\end{minipage}
	%\qquad
	\begin{minipage}{0.49\linewidth}
		\centering
		\includegraphics[width=1\linewidth]{fig.5_power_com.pdf}
	\caption{$P_{\rm 2, max}$ v.s. $R_{\rm S}$ for different multiple access schemes.}\label{power_com} 
	\end{minipage}
\end{figure}

\begin{figure}[htbp]
	\centering
	\begin{minipage}{0.49\linewidth}
		\centering
		\includegraphics[width=1\linewidth]{fig.6_SEE_com.pdf}
	\caption{SEE v.s. $P_{\rm 2, max}$ for different multiple access schemes.}\label{SEE_com} 
	\end{minipage}
	%\qquad
	\begin{minipage}{0.49\linewidth}
		\centering
		\includegraphics[width=1\linewidth]{fig.7_SEE_ICSI_com.pdf}
	\caption{SEE v.s. $P_{\rm 2, max}$ for different CSI uncertainty parameters. }\label{SEE_ICSI_com} 
	\end{minipage}
\end{figure}

\begin{figure}[htbp]
	\centering
	\includegraphics[width=0.5\linewidth]{fig.8_SEE_opti_com.pdf}
	\caption{SEE v.s. $P_{\rm 1, max}$ for different transmit BF optimization schemes at $\rm BS_1$.}\label{SEE_opti_com} 
\end{figure}

\begin{figure}[htbp]
	\centering
	\begin{minipage}{0.49\linewidth}
		\centering
		\includegraphics[width=1\linewidth]{fig.9.RSMA_common_3D_re.pdf}
		\caption{3D BF pattern
			of the common stream.}\label{common_stream_3D} 
	\end{minipage}
	%\qquad
	\begin{minipage}{0.49\linewidth}
		\centering
		\includegraphics[width=1\linewidth]{fig.10.RSMA_common_2D_re.pdf}
		\caption{2D BF pattern
			of the common stream.}\label{common_stream_2D} 
	\end{minipage}
\end{figure}

\begin{figure}[htbp]
	\centering
	\begin{minipage}{0.49\linewidth}
		\centering
		\includegraphics[width=1\linewidth]{fig.11.RSMA_private_3D_re.pdf}
		\caption{3D BF pattern
			of the private stream.}\label{private_stream_3D} 
	\end{minipage}
	%\qquad
	\begin{minipage}{0.49\linewidth}
		\centering
		\includegraphics[width=1\linewidth]{fig.12.RSMA_private_2D_re.pdf}
		\caption{2D BF pattern
			of the private stream.}\label{private_stream_2D} 
	\end{minipage}
\end{figure}

%\subsection{Analysis of the Single Variable} 
%Here, it is defined that ${{\rm{P}}_{\rm{r}}} = {{\rm{P}}_{{\rm{CIP}}}}$.

%\subsubsection{Sensing SCNR versus iteration number for different $L$}
The convergence performance of the SEE improvement in the PR-ISAC CRNs is shown by the solid line in Fig. \ref{iteration_com}, and the traversal algorithm is depicted by the dashed line in contrast.
For a fixed  $M_1$, the SEE gradually increases with the increase of the number of iterations. After an acceptable number of iterations, SEE converges. For a given number of iterations, SEE increases gradually with the increase of $M_1$, since the transmit power carried by a single antenna is constant. The greater $M_1$ is, the greater the maximum transmit power becomes, which is translated to an increase of the spatial degrees of freedom and BF design accuracy and efficiency, leading to a greater $R_{\rm S}$ for a fixed transmit power. Then, SEE also increases. As $M_1$ increases, the number of iterations required to solve the SEE gradually increases. This is because as $M_1$ increases, the computational complexity of the solution also increases; thus the number of iterations increases. For Algorithm \ref{algorithm_4} and traversal algorithm, as $M_1$ increases, the numbers of iterations required to achieve convergence are 24, 30, 39, and 30, 40, and 52, respectively. Obviously, compared with the traversal algorithm, under similar reliability condition and for fixed $M_1$, Algorithm \ref{algorithm_4} requires fewer iterations to achieve convergence, achieving better convergence performance and effectiveness.

%\subsubsection{Sensing SCNR versus $P_{\rm max}$ for different $C_{\rm c,L}$}

In Fig. \ref{SCNR_com}, we show the relationship between $\gamma_{\rm b1}$ and $P_{\rm 1, max}$ under different BF optimization schemes. It can be seen that under the same BF scheme, $\gamma_{\rm b1}$ increases with the increase of $P_{\rm 1, max}$, since the signal power received by the target and reflected to $\rm BS_1$ increases with the increase of $P_{\rm 1, max}$ for a given RCS. Meanwhile, the clutter power and noise power in the environment are relatively stable, and thus $\gamma_{\rm b1}$ enhances. In addition, the proposed RSMA-based BF scheme outperforms NOMA in \cite{ISAC_PLS_46_r}, TDMA in \cite{ISAC_PLS_54}, and non-optimized echo signal in \cite{ISAC_PLS_34}. This is because the proposed SEE BF optimization scheme continuously receives the echo signal in the time dimension, optimizes the BF vector of the echo signal in the space dimension, and simultaneously uses the sensing signal and the communication signal for sensing the target in the power dimension. However, the other three schemes either neglect to optimize the BF vector of the echo signal, or do not continuously or stably receive the target information, or only use the sensing signal for target identification. Therefore, the proposed multi-BF optimization scheme outperforms the contributions of \cite{ISAC_PLS_46_r,ISAC_PLS_54,ISAC_PLS_34} in terms of sensing.

Fig. \ref{RS_com} illustrates the relationship between $R_{\rm S}$ and $P_{\rm 2, max}$ under four different multiple access schemes. It is obvious that for a fixed multiple access scheme, $R_{\rm S}$ increases as $P_{\rm 2, max}$ increases. This is because the multiple-access-based BF realizes that $\rm BS_2$ accounts for the downlink transmission services for $N$ users, while increasing the interference power towards Eve and reducing the interference to Bob. Therefore, with the increase of $P_{\rm 2, max}$, Bob is affected by relatively limited interference, and the change of Bob's receiving SINR is small, but the interference at Eve is significantly increased, and Eve's eavesdropping SINR also significantly deteriorates, leading to the improvement of $R_{\rm S}$.
Besides, given the fixed $P_{\rm 2, max}$, the RSMA scheme adopted in this paper has a superior $R_{\rm S}$ to NOMA \cite{ISAC_PLS_46_r}, OMA \cite{ISAC_PLS_44}, and SDMA \cite{ISAC_PLS_42}. %This phenomenon is consistent with the analysis in Section I.
 %Besides, when $P_2$ increases, although $R_{\rm S}$ also increases, the increase rate is slowing down, which is related to two situations. First, $R_{\rm S}$ is a logarithmic function with respect to P2, and the first derivative of RS with respect to P2 is a monotonically decreasing function. Second, the transmitted signal of BS2 has the same mechanism of improving secure communication as artificial noise, which only reduces the eavesdropping rate of Eve rather than increasing Bob, or reduces the legal communication rate of Bob to a limited extent. The improvement effect of the security communication rate has limitations, and it is an auxiliary scheme to improve the security communication rate. More intuitive and important is to increase the transmission power P1 of BS1.




%\subsection{Analysis of Beamforming Pattern} 




%\subsubsection{Common Data Stream}

%\subsubsection{Private Data Stream}



%\subsection{Analysis of Performance of the DS-RSMA ISAC System}

	
%\subsubsection{Comparison on Different Sensing Schemes} 


%\subsubsection{Comparison on different multiple access schemes.} 
Fig. \ref{power_com} depicts the relationship between $P_{\rm 2, max}$ and $R_{\rm S}$ for four different multiple access schemes. It is obvious that, for a given multiple access scheme, $P_{\rm 2, max}$ increases gradually as $R_{\rm S}$ increases. This is because the deterioration effect of green interference on Eve's eavesdropping channel is more powerful than that on Bob's side. Therefore, a higher $R_{\rm S}$ requires a larger $P_{\rm 2, max}$. However, with the increase of $R_{\rm S}$, the power consumption also increases, since $P_{\rm 2, max}$ is an exponential function of $R_{\rm S}$. This improvement is evident in a certain power range, rather than in the entire power range. Meanwhile, for a given $R_{\rm S}$, $P_{\rm 2, max}$ required for SDMA, OMA, NOMA, and RSMA decreases successively, because the RSMA rationalizes common and private streams to meet different needs with utilizing interference and cutting down power budget, which neither NOMA nor SDMA can do.

Fig. \ref{SEE_com} presents the relationship between SEE and $P_{\rm 2, max}$ under different multiple access schemes. Under the same multiple access scheme, with the increase of $P_{\rm 2, max}$, SEE shows a change process of first increasing and then decreasing. This phenomenon is reasonable and based on the definition of SEE \cite{ISAC_PLS_13}. %On the one hand, $R_{\rm S}$ is the logarithm of $P_{\rm 2, max}$, thus with the linear increase of $P_{\rm 2, max}$, $R_{\rm S}$ increases with a lower increase rate. Therefore, when the increase rate of $R_{\rm S}$ is greater than that of $P_{\rm 2, max}$, the SEE as the quotient of $R_{\rm S}$ and $P_{\rm 2, max}$, also increases. When the increase rate of $R_{\rm S}$ is less than that of $P_{\rm 2, max}$, the increase rate of SEE starts to decrease. When the increase rate of increase of $R_{\rm S}$ continues to decrease to a certain critical point, SEE starts to monotonically decrease. On the other hand, 
The second derivative of SEE with respect to $P_{\rm 2, max}$ is less than 0, implying SEE is a convex function of $P_{\rm 2, max}$. For a given $P_{\rm 2, max}$, RSMA obtains the best SEE among four multiple access schemes, as presented in Figs. \ref{RS_com} and \ref{power_com}. %RSMA solutions can use some interference to increase the safety rate and treat some interference as noise to reduce power consumption to meet the diverse and dynamic needs of secure communication. These advantages are not available in the other three multiple access schemes, so the RSMA scheme has the most significant effect on improving SEE.

Fig. \ref{SEE_ICSI_com} demonstrates the relationship between SEE and $P_{\rm 2, max}$ under different CSI uncertainty parameters. Herein, we define ${\rm{e}} = {{{{\rm{e}}_{\rm{h}}}} \mathord{\left/
		{\vphantom {{{{\rm{e}}_{\rm{h}}}} {{{\left\| {{{\bf{h}}_{{\rm{es}}}}} \right\|}_2}}}} \right.
		\kern-\nulldelimiterspace} {{{\left\| {{{\bf{h}}_{{\rm{es}}}}} \right\|}_2}}} = {{{{\rm{e}}_{\rm{g}}}} \mathord{\left/
		{\vphantom {{{{\rm{e}}_{\rm{g}}}} {{{\left\| {{{\bf{g}}_{{\rm{es}}}}} \right\|}_2}}}} \right.
		\kern-\nulldelimiterspace} {{{\left\| {{{\bf{g}}_{{\rm{es}}}}} \right\|}_2}}}$ to quantitatively characterize the uncertainty of the channel estimation. For a given $P_{\rm 2, max}$, SEE increases with the decrease of e. The smaller e is, the more accurate and reliable the channel estimation is, and the closer the calculation result is to the theoretical upper bound. The smaller the uncertainty of the Eve's CSI estimation is, the better knowledge of the Eve's CSI characteristics $\rm BS_1$ and $\rm BS_2$ has, and the stronger the suppression effect of BF optimization for Eve's channel is, improving $R_{\rm S}$ and SEE.

Fig. \ref{SEE_opti_com} presents the relationship between SEE and $P_{\rm 1, max}$ under different transmit BF optimization schemes. For a fixed multiple access scheme, as $P_{\rm 1, max}$ increases, the SEE shows a change process of first increasing and then decreasing. The effect of $P_{\rm 1, max}$ on SEE can be referred to that of $P_{\rm 2, max}$ on SEE. For a given $P_{\rm 1, max}$, compared with the optimization of $\textbf{w}_{\rm bo}$ in \cite{ISAC_PLS_48}, optimization of $\textbf{w}_{\rm ta}$ in \cite{ISAC_PLS_33}, and no optimization in \cite{ISAC_PLS_54}, the proposed joint optimization of $\textbf{w}_{\rm bo}$ and $\textbf{w}_{\rm ta}$ achieves a higher SEE. 
Optimization of $\textbf{w}_{\rm ta}$ transforms the harmful interference that affects communication into the green interference that affects eavesdropping with a smaller eavesdropping rate. Optimization of $\textbf{w}_{\rm bo}$ improves the signal quality received by Bob and $R_{\rm S}$. Finally, the proposed joint optimization of $\textbf{w}_{\rm bo}$ and $\textbf{w}_{\rm ta}$ fully demonstrates the advantage in enhancing SEE compared to schemes in \cite{ISAC_PLS_48,ISAC_PLS_33,ISAC_PLS_54}.

In Figs. \ref{common_stream_3D}, \ref{common_stream_2D}, \ref{private_stream_3D}, and \ref{private_stream_2D}, yellow, red, green, and blue dots are positions for Bob, Eve, users, and the user wanting this private stream.
Figs. \ref{common_stream_3D} and \ref{common_stream_2D} depict \emph{three-dimensional} (3D) and \emph{two-dimensional} (2D) beampatterns of the common stream. There are 4 significant and relatively close amplitude gains with values greater than $\rm 0\,dB$. This is because $\rm BS_2$ serves 4 communication users, all of whom require the common stream. Therefore, the amplitude gain of the common stream for 4 users is greater than $\rm 0\,dB$, ensuring that the power of the transmit common stream gets amplified. On the other hand,, amplitude gains of common stream transmitted to Bob and Eve is close to $\rm 0\,dB$. In summary, the optimization of $\textbf{o}_{\rm c}$ in this paper improves the signal power received by each user, and overcomes the fading problem in the signal transmission process. Therefore, each user obtains the common stream reliably while the interference from the common stream exerted on the signal reception of Bob and Eve is limited.

Figs. \ref{private_stream_3D} and \ref{private_stream_2D} illustrate 3D and 2D beampatterns of the private stream. In Figs. \ref{private_stream_3D} and \ref{private_stream_2D}, it can be found that among 6 amplitude gains, the amplitude gain of the fourth user is significantly greater than $\rm 0\,dB$, and the amplitude gains of other uses are close to or less than $\rm 0\,dB$. This is because a private stream is only useful for one user and harmful to other users. Therefore, when a private stream is provided to the fourth user, it is necessary to increase the amplitude gain of the fourth user and enhance the signal power received by the fourth user. Meanwhile, the amplitude gains of the other users is cut down, decreasing the received interference power, and thus improving the received SINR. In addition, the amplitude gain for Bob is less than $\rm 0\,dB$, and the interference received by Bob is suppressed, while the amplitude gain for Eve is greater than $\rm 0\,dB$, causing a stronger interference to Eve to improve the SEE. Therefore, the optimization of $\textbf{o}_n$ improves the SINR of users and SEE of the system.

%We depict the three-dimension beampatterns of the common stream, ${{\bf{o}}_{\rm{c}}}$, and the private stream, ${{\bf{o}}_n}$ in Figs. \ref{common_stream} and \ref{private_stream}, respectively. Gains of transmit power from $\rm BS_2$ to $\rm U_1$, $\rm U_2$, $\rm U_3$, and $\rm SU_4$ are both strengthened, while those from the base station to the PU and Eve are suppressed with ${{\bf{u}}_{\rm{c}}}$. This result ensures that both of SUs can receive the common signal, and interference to the PU is mitigated. Then, the interference to the $\rm SU_1$ and PU is restrained, and the channel quality from the base station to the $\rm SU_2$ is improved with ${{\bf{u}}_{{\rm{p}} , 2}}$. Moreover, the interference from the base station to the Eve is greater than that from the base station to the PU, which implies that the security of the CSTN is further enhanced. 

%the SEE obtained by the joint optimization of $\textbf{w}_{\rm bo}$ and $\textbf{w}_{\rm ta}$ is greater than those obtained in optimizing $\textbf{w}_{\rm bo}$ or $\textbf{w}_{\rm ta}$ alone, or if neither $\textbf{w}_{\rm bo}$ nor $\textbf{w}_{\rm ta}$ is optimized.

\section{Conclusion}
This paper presented the PR-ISAC CRNs employing ISAC wireless techniques,  multicast communication approaches. Given the challenges posed by interference and eavesdropping, we derived key performance indicators for sensing, communication, security, and EE. To enhance these metrics, we developed a green interference scheme by leveraging the ISAC sensing signal and the RSMA communication signal, which was realized in the form of multi-BF optimization at the mathematical level. %Then, we designed multi-BF optimization scheme to further improve system performance. 
Since the original optimization problem was highly non-convex, posing significant challenges for reliable and efficient solutions, we decomposed it into three tractable sub-optimization problems using Taylor series expansion, MM, SDP, and SCA, and then solved it with alternating optimization. Finally, simulations validated the superiority
of our designs in achieving secure and green sensing and communications in the challenging scenario with interference and eavesdropping to existing methods.

Studying green interference is promising, since interference in complex networks is inevitable, and energy allocated to the network gets increasingly limited. Thus, it becomes reasonable and urgent to turn some of interference into treasure, improving network performances, such as sensing accuracy, communication rate, security rate, and energy efficiency. This is the motivation as well as main contribution of this paper.

%Note that we firstly simulations provided comparative evaluations against baseline schemes, and demonstrated that the proposed multi-BF optimization scheme achieved superior performance in sensing, communication, security, and EE compared to existing methods. Note that the efforts on building green interference is 
%Given that the non-convex form of the original optimization problem brought great challenges to its reliable and efficient solution, we decomposed the original optimization problem into three sub-optimization problems based on Taylor series expansion, MM, SDP, and SCA, and solved the optimization problem through the alternating iteration. Finally, simulations verified the correctness of the analysis, provided comparisons with baseline schemes, and revealed that the proposed multi-BF optimization scheme outperforms prior published contributions in terms of sensing, communication, security, and EE.
 
\begin{appendices}
	
	\section{Proof of Proposition 1}
Based on ${R_{\rm{S}}} = {\left[ {{R_{{\rm{bo}}}} - {R_{\rm{e}}}} \right]^ + }$, (\ref{P1}\rm e) can be rewritten as
\begin{equation}\label{rate_to_SINR}
\begin{array}{l} \displaystyle
	\frac{{{\rm{Tr}}\left( {{{\bf{H}}_{{\rm{bo}}}}{{\bf{W}}_{{\rm{bo}}}}} \right)}}{{{\rm{Tr}}\left( {{{\bf{G}}_{{\rm{bo}}}}{{\bf{O}}_{{\rm{c,}}n}}} \right) + {\sigma ^2}}} - \left( {\tau  - 1} \right) \ge \\ \displaystyle
	\tau \frac{{{\rm{Tr}}\left( {{{\bf{H}}_{\rm{e}}}{{\bf{W}}_{{\rm{bo}}}}} \right)}}{{{\rm{Tr}}\left( {{{\bf{H}}_{\rm{e}}}{{\bf{W}}_{{\rm{ta}}}}} \right) + {\rm{Tr}}\left( {{{\bf{G}}_{\rm{e}}}{{\bf{O}}_{{\rm{c,}}n}}} \right) + {\sigma ^2}}}.
\end{array}
\end{equation}
Then, we transform (\ref{rate_to_SINR}) into a LMI as
\begin{equation}\label{non-LMI}
\begin{array}{*{20}{l}} \displaystyle
	{{\rm{Tr}}\left( {{{\bf{H}}_{{\rm{bo}}}}{{\bf{W}}_{{\rm{bo}}}}} \right)\left[ {{\rm{Tr}}\left( {{{\bf{H}}_{\rm{e}}}{{\bf{W}}_{{\rm{ta}}}}} \right) + {\sigma ^2} + {\rm{Tr}}\left( {{{\bf{G}}_{\rm{e}}}{{\bf{O}}_{{\rm{c,}}n}}} \right)} \right]}\\  \displaystyle
	{ - \left( {\tau  - 1} \right)\left[ {{\rm{Tr}}\left( {{{\bf{G}}_{{\rm{bo}}}}{{\bf{O}}_{{\rm{c,}}n}}} \right) + {\sigma ^2}} \right]}\\  \displaystyle
	{ \times \left[ {{\rm{Tr}}\left( {{{\bf{H}}_{\rm{e}}}{{\bf{W}}_{{\rm{ta}}}}} \right) + {\rm{Tr}}\left( {{{\bf{G}}_{\rm{e}}}{{\bf{O}}_{{\rm{c,}}n}}} \right) + {\sigma ^2}} \right]}\\  \displaystyle
	{ \ge \tau {\rm{Tr}}\left( {{{\bf{H}}_{\rm{e}}}{{\bf{W}}_{{\rm{bo}}}}} \right)\left[ {{\rm{Tr}}\left( {{{\bf{G}}_{{\rm{bo}}}}{{\bf{O}}_{{\rm{c,}}n}}} \right) + {\sigma ^2}} \right].}
\end{array}
\end{equation}
After some algebraic manipulations, we obtain
\begin{equation}\label{LMI-1}
{U_1}{U_2} \ge \frac{\tau }{{\tau  - 1}}{\rm{Tr}}\left( {{{\bf{H}}_{{\rm{bo}}}}{{\bf{W}}_{{\rm{bo}}}}} \right){\rm{Tr}}\left( {{{\bf{H}}_{\rm{e}}}{{\bf{W}}_{{\rm{bo}}}}} \right).
\end{equation}
This concludes the proof.

\section{Proof of Proposition 2}
Let ${{\bf{h}}_{{\rm{bo}}}} = {\left[ {{x_1},{x_2}, \cdots ,{x_{{M_{{\rm{2}}}}}}} \right]^{\rm{H}}}$, ${{\bf{h}}_{{\rm{e}}}} = {\left[ {{y_1},{y_2}, \cdots ,{y_{{M_{{\rm{2}}}}}}} \right]^{\rm{H}}}$, and ${{\bf{W}}_{{\rm{bo}}}} = {\left[ {{z_1},{z_2}, \cdots ,{z_{{M_{{\rm{2}}}}}}} \right]^{\rm{H}}}$; thus
\begin{equation}\label{pro2-1}
	{\rm{Tr}}\left[ {{{\bf{H}}_{{\rm{bo}}}}{{\bf{W}}_{{\rm{bo}}}}} \right]{\rm{Tr}}\left[ {{{\bf{H}}_{\rm{e}}}{{\bf{W}}_{{\rm{bo}}}}} \right] = {\left( {\sum\nolimits_{i = 1}^{{M_{\rm{2}}}} {{x_i}{z_i}} } \right)^2}{\left( {\sum\nolimits_{j = 1}^{{M_{\rm{2}}}} {{y_j}{z_j}} } \right)^2},
\end{equation}
\begin{equation}\label{pro2-2}
	{\rm{Tr}}\left[ {{{\bf{h}}_{{\rm{bo}}}}{\bf{h}}_{\rm{e}}^{\rm{H}}{{\bf{W}}_{{\rm{bo}}}}} \right] = \sum\nolimits_{i = 1}^{{M_{\rm{2}}}} {{x_i}{z_i}} \sum\nolimits_{j = 1}^{{M_{\rm{2}}}} {{y_j}{z_j}} .
\end{equation}
Obviously, based on (\ref{pro2-1}) and (\ref{pro2-2}), we obtain
\begin{equation}
{\rm{Tr}}\left[ {{{\bf{H}}_{{\rm{bo}}}}{{\bf{W}}_{{\rm{b1}}}}} \right]{\rm{Tr}}\left[ {{{\bf{H}}_{\rm{e}}}{{\bf{W}}_{{\rm{b1}}}}} \right]{\rm{ = Tr}}{\left[ {{{\bf{h}}_{{\rm{bo}}}}{\bf{h}}_{\rm{e}}^{\rm{H}}{{\bf{W}}_{{\rm{b1}}}}} \right]^2},
\end{equation}
which concludes the proof of Proposition 2. 

\section{Proof of Proposition 3}\label{pro3}
 Based on Proposition 1 and Proposition 2, we hold that
\begin{equation}\label{pro3-prove-1}
\begin{array}{l} \displaystyle
	{\left( {{U_1} + {U_2}} \right)^2} - {\left( { - {U_1} + {U_2}} \right)^2} = 4{U_1}{U_2}\\ \displaystyle
	\ge \frac{{4\tau }}{{\tau  - 1}}{\rm{Tr}}\left[ {{{\bf{H}}_{{\rm{bo}}}}{{\bf{W}}_{{\rm{bo}}}}} \right]{\rm{Tr}}\left[ {{{\bf{H}}_{\rm{e}}}{{\bf{W}}_{{\rm{bo}}}}} \right]\\ \displaystyle
	= \frac{{4\tau }}{{\tau  - 1}}{\rm{Tr}}{\left( {{{\bf{h}}_{{\rm{bo}}}}{\bf{h}}_{\rm{e}}^{\rm{H}}{{\bf{W}}_{{\rm{b1}}}}} \right)^2}\\ \displaystyle
	= {\left[ {2\sqrt {{\tau  \mathord{\left/
						{\vphantom {\tau  {\left( {\tau  - 1} \right)}}} \right.
						\kern-\nulldelimiterspace} {\left( {\tau  - 1} \right)}}} {\rm{Tr}}\left( {{{\bf{h}}_{{\rm{bo}}}}{\bf{h}}_{\rm{e}}^{\rm{H}}{{\bf{W}}_{{\rm{bo}}}}} \right)} \right]^2},
\end{array}
\end{equation}
or equivalently
\begin{equation}\label{pro3-prove-2}
\begin{array}{l}
	{U_1} + {U_2} \ge \\   \displaystyle
	\sqrt {{{\left[ {2\sqrt {{\tau  \mathord{\left/
								{\vphantom {\tau  {\left( {\tau  - 1} \right)}}} \right.
								\kern-\nulldelimiterspace} {\left( {\tau  - 1} \right)}}} {\rm{Tr}}\left( {{{\bf{h}}_{{\rm{bo}}}}{\bf{h}}_{\rm{e}}^{\rm{H}}{{\bf{W}}_{{\rm{bo}}}}} \right)} \right]}^2} + {{\left( { - {U_1} + {U_2}} \right)}^2}}.
\end{array}
\end{equation}
According to the definition of \emph{second order cone} (SOC) and (\ref{pro3-prove-2}), we obtain (\ref{pro3-1}). This concludes proof of Proposition 3.

\section{Proof of Proposition 4}
According to (\ref{PB_C1}) and (\ref{pro3-prove-2}), we get
\begin{equation}\label{pro4-prove-1}
\begin{array}{l} \displaystyle
	{\rm{Tr}}\left( {{{\bf{h}}_{{\rm{bo}}}}{\bf{h}}_{\rm{e}}^{\rm{H}}{{\bf{W}}_{{\rm{bo}}}}} \right) \le {\rm{Tr}}\left[ {\left( {{{\bf{h}}_{{\rm{bo}}}}{\bf{h}}_{{\rm{es}}}^{\rm{H}} + {{\bf{h}}_{{\rm{bo}}}}{\bf{h}}_{{\rm{er}}}^{\rm{H}}} \right){{\bf{W}}_{{\rm{bo}}}}} \right]\\ \displaystyle
	\le {\rm{Tr}}\left[ {\left( {{{\bf{h}}_{{\rm{bo}}}}{\bf{h}}_{{\rm{es}}}^{\rm{H}} + {{\rm{e}}_{{\rm{h,UB}}}}{{\bf{h}}_{{\rm{bo}}}}{{\bf{I}}_{1 \times {M_1}}}} \right){{\bf{W}}_{{\rm{bo}}}}} \right]\\ \displaystyle
	\le {\rm{Tr}}\left[ {\left( {{{\bf{h}}_{{\rm{bo}}}}{\bf{h}}_{{\rm{es}}}^{\rm{H}} + {{\rm{e}}_{{\rm{h,UB}}}}{{\bf{I}}_{{M_1} \times {M_1}}}} \right){{\bf{W}}_{{\rm{bo}}}}} \right] = {\rm{Tr}}\left( {{\bf{\tilde H}}{{\bf{W}}_{{\rm{bo}}}}} \right).
\end{array}
\end{equation}
To ensure that the constraint on the security rate in (\ref{pro3-1}) is always satisfied, based on \textbf{Proposition 3}, (\ref{range_UB}), (\ref{range_LB}), and (\ref{pro4-prove-1}), (\ref{pro3-1}) can be rewritten as
\begin{equation}\label{pro4-prove-2}
	\begin{array}{l}
		{U_1} + {U_{2,\rm LB}} \ge \\   \displaystyle
		\sqrt {{{\left[ {2\sqrt {{\tau  \mathord{\left/
									{\vphantom {\tau  {\left( {\tau  - 1} \right)}}} \right.
									\kern-\nulldelimiterspace} {\left( {\tau  - 1} \right)}}} {\rm{Tr}}\left( {{\bf{\tilde H}}{{\bf{W}}_{{\rm{bo}}}}} \right)} \right]}^2} + {{\left( { - {U_1} + {U_{2,\rm UB}}} \right)}^2}}.
	\end{array}
\end{equation}
Afterwards, based on the definition SOC, we can get (\ref{pro4-1}). This concludes the proof.

\section{Proof of Proposition 5}
Herein, considering a binary function 
\begin{equation}\label{binary_func}
	f\left( {{x_7} ,{x_8}} \right) = {{x_8^2} \mathord{\left/
			{\vphantom {{x_8^2} {\left( {{x_1}{x_7}} \right)}}} \right.
			\kern-\nulldelimiterspace} {\left( {{x_1}{x_7}} \right)}},
\end{equation}
which is presented before (\ref{C1_transformation}), we obtain the Taylor series expansion expression of (\ref{binary_func}). Then, with the first-order Taylor series expansion reserved and other items ignored, we get
\begin{equation}\label{pro5-prove-1}
\begin{array}{l} \displaystyle
	f\left( {{x_7},{x_8}} \right) \approx f\left( {{x_{7,0}},{x_{8,0}}} \right) + \left( {{x_7} - {x_{7,0}}} \right){f_{{x_7}}}^\prime \left( {{x_{7,0}},{x_{8,0}}} \right)\\ \displaystyle \qquad\qquad\,\,
	+ \left( {{x_8} - {x_{8,0}}} \right){f_{{x_8}}}^\prime \left( {{x_{7,0}},{x_{8,0}}} \right).
\end{array}
\end{equation}
By substituting the expression of (\ref{binary_func}) into (\ref{pro5-prove-1}), we obtain
\begin{equation}\label{pro5-prove-2}
\begin{array}{l} \displaystyle
	f\left( {{x_7},{x_8}} \right) = \frac{{x_{8,0}^2}}{{{x_{7,0}}{x_1}}} - \left( {{x_7} - {x_{7,0}}} \right)\frac{{x_{8,0}^2}}{{x_{7,0}^2{x_1}}}\\   \displaystyle \qquad\qquad\;
	+ \left( {{x_8} - {x_{8,0}}} \right)\frac{{2{x_{8,0}}}}{{{x_{7,0}}{x_1}}} \ge 1.
\end{array}
\end{equation}
By simplifying (\ref{pro5-prove-2}), we get (\ref{specific_value}).  This concludes the proof.

\end{appendices}

\bibliographystyle{IEEEtran}
\bibliography{PR-ISAC_CRN}


\end{document}









