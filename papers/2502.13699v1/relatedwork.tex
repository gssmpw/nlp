\section{Literature Review}
How to optimize sensing and communication of the ISAC independently or jointly is a hotspot focused by scholars \cite{ISAC_PLS_15_r,ISAC_PLS_16_r,ISAC_PLS_7}. The authors of \cite{ISAC_PLS_30} studied the single-static sensing performance of the multi-target massive \emph{massive input massive output} (MIMO)-ISAC systems, and minimized the sum of \emph{Cramer-Rao lower bounds} (CRLBs) of target arrival directions under a communication rate constraint. The authors stated that their scheme achieved near-optimal
performance with less complexity through a single optimization of the sensing signal.
In \cite{ISAC_PLS_19}, the authors investigated a mobile-antenna-assisted ISAC system, aiming to improve the communication rate and echo signal \emph{signal-to-cluster-plus-noise ratio} (SCNR) by jointly optimizing the antenna coefficient and antenna position. An ISAC network assisted by a fluid antenna was designed in \cite{ISAC_PLS_24}. With the position and waveform of the fluid antenna jointly optimized, the sum communication rate of all users was improved. The authors of \cite{ISAC_PLS_20} presented an ISAC network in which \emph{unmanned aerial vehicle} (UAV) acted as a \emph{base station} (BS). To achieve a balance between sensing and communication, the position and transmit power of the UAV were optimized, and then the communication rate and CRLB of target sensing accuracy were maximized. %In a similar scenario, the authors of \cite{ISAC_PLS_25} further introduced an \emph{intelligent reflecting surface} (IRS) to jointly optimize sensing target scheduling, transmission symbol vector covariance matrix, IRS coefficient, and UAV flight path to maximize the communication rate while being subject to constraints on the echo signal SCNR.
Li \emph{et al.} analyzed an ISAC network in which a mobile UAV served as the sensing target. Under the communication \emph{quality of service} (QoS) and transmit power of the BS constraints, transmit signal BF vector and target allocation were optimized to improve the echo signal \emph{signal-to-interference-plus-noise ratio} (SINR) \cite{ISAC_PLS_23}. In \cite{ISAC_PLS_21}, the author studied ISAC composed of multiple BSs and multiple users acting as targets and communication users. %Owing to the optimization of the transmit signal BF vector, the interference among BSs and users was mitigated, and the SINRs of sensing and communications was improved. %A joint resource allocation scheme for \emph{vehicle-to-everything} (V2X) communication and sensing was reported in \cite{ISAC_PLS_22}, which adopted multi-agent \emph{deep deterministic policy gradient} (DDPG) algorithm to adjust network power, the communication rate and sensing accuracy get improved. %The authors of \cite{ISAC_PLS_26} articulated the optimization of the anti-interference resource allocation in ISAC networks. Under the anti-interference communication and sensing constraints, the channel allocation, power allocation and BF were optimized with \emph{deep reinforcement learning} (DRL) in order to jointly maximize the communication rate and sensing efficiency.

%The aforementioned contributions confront some hinders that limit both sensing accuracy and communication QoS, of which two of the most important are multiplicative fading and untreated mutual interference.
 Considering that the overall composition in ISAC is relatively complex and there is a non-negligible interference between the target sensing and the multicast communications, the scientific community turned its attention towards a variety of multiple access schemes to attain interference mitigation in ISAC \cite{MA_ISAC,Inter_ISAC}. The authors of \cite{ISAC_PLS_42} focused on a \emph{space division multiple access} (SDMA) scheme that used linear precoding to distinguish users in a spatial domain, relying entirely on treating any remaining multi-user interference as noise. The communication performance gain of the ISAC based on SDMA was evaluated in \cite{ISAC_PLS_43}. In contrast to SDMA, \emph{non-orthogonal multiple access} (NOMA) works with superimposing coding at the transmitter and \emph{successive interference cancellation} (SIC) coding at the receiver. NOMA superimposes users in the power domain, and forces users with better channel conditions to perform complete decoding through user grouping and sorting \cite{ISAC_PLS_42} to eliminate interference caused by other users. Different from NOMA, \emph{orthogonal multiple access} (OMA) assigns one resource to one user, resulting in lower resource utilization. 
The authors of \cite{ISAC_PLS_44} assessed the OMA-empowered and NOMA-empowered performance gains of communication and sensing of the semi-ISAC network based on the traversal rate and the traversal estimation of information rate. An uplink transmission scheme was articulated in \cite{ISAC_PLS_45} for NOMA-ISAC system to mitigate mutual interference between sensing and communication signals, and enhance communication convergence rate, reliability, and sensing accuracy. In \cite{ISAC_PLS_46_r}, the authors documented a joint optimization scheme of transmit signal BF, NOMA transmission time, and target sensing scheduling to maximize the sensing efficiency of ISAC systems, while ensuring a high communication QoS. 
A joint precoding optimization problem based on NOMA was solved in \cite{ISAC_PLS_47}, which maximized the security rate of multi-user through \emph{artificial noise} (AN), and achieved secure transmission, while satisfying the sensing performance constraint. 

Although studies on SDMA and NOMA to improve the ISAC performance is gradually deepening, there are some extremes to conventional multi-access architectures, such as SDMA and NOMA. Specifically, SDMA treats interference entirely as noise, seriously reducing the reliability. Instead, NOMA decodes interference one by one, implying that the effectiveness is hard to guarantee. Above shortcomings and deficiencies urge us to find a new scheme like \emph{rate-splitting multiple access} (RSMA)  \cite{ISAC_PLS_14}, adopting rate splitting based on linear precoding and SIC. As a consequence, RSMA decodes some interference and treats the remaining as noise, fully absorbing advantages of both SDMA and NOMA, and achieves high reliability and high effectiveness. Under the constraints of data rate and transmit power budget, RSMA structure and parameters were designed in \cite{ISAC_PLS_48} to minimize the CRLB of sensing response matrix at radar receivers. The authors of \cite{ISAC_PLS_49} presented an indicative example of an RSMA-ISAC waveform design that jointly optimized the minimum fairness rate among communication users and the CRLB of target detection under power constraints.

Coexistence of sensing and communication broadens the prospects for next-generation communication systems, while increasing the energy consumption. This necessitates the development of green ISAC systems simultaneously. In this direction, the authors of \cite{ISAC_PLS_29} introduced a power consumption minimization policy for the near-field ISAC system. In particular, the transmit signal BF vector was optimized to minimize network power consumption under the constraints of communication SINR, sensing target transmit beam pattern gain, and interference power. For \emph{intelligent reflecting surface} (IRS)-ISAC systems, the authors of \cite{ISAC_PLS_25_re,ISAC_PLS_36_re} maximized the \emph{energy efficiency} (EE) by jointly optimizing the transmit signal BF vector, the IRS reflection coefficient matrix, and the IRS deployment location. Energy-saving BF design of ISAC systems that aimed to maximize the EE by appropriately designing transmission waveforms in multi-user communication and target estimation scenarios was documented in \cite{ISAC_PLS_37} and \cite{ISAC_PLS_38}. %The authors of \cite{ISAC_PLS_39} investigated the efficient channel sharing auxiliary ISAC, and jointly optimized the transmit BF vector, receive BF vector, and multi-objective sensing scheduling to improve the sensing EE. 
In a multi-BS ISAC network, the energy consumption was reduced due to optimum task allocation, beam scheduling and transmit power control \cite{ISAC_PLS_40}. %The authors of \cite{ISAC_PLS_41} documented two joint spectrum division and power allocation schemes to increase channel mutual information, transmission rates, and EE of ISAC systems.

Additionally, due to the inherent open nature of downlink data transmission and broadcast mechanism, as well as the resource sharing between perception and communication of the ISAC network, it is vulnerable to security threats like eavesdropping and intercepting \cite{Intro_PLS_ISAC}. Consequently, it is of great significance and urgency to carry out researches on \emph{physical layer security} (PLS) in ISAC networks \cite{ISAC_PLS_50}. The effort that has been spent on the PLS-ISAC is limited. In an IRS-ISAC network, the authors of \cite{ISAC_PLS_27} maximized the minimum communication rate by optimizing the transmit BF vector, the receive BF vector, and the IRS reflection coefficient matrix under the constraints of echo signal power and security rate. For the same system model, BF was designed to maximize the minimum weighted beam pattern gain under security rate and transmit power constraints \cite{ISAC_PLS_33}. 
The authors of \cite{ISAC_PLS_28} focused on an ISAC network in which UAVs served as BSs to provided downlink data transmission for multiple users, sense and interfere with the \emph{eavesdropper} (Eve) to maximize security sum rate. 
%The authors of \cite{ISAC_PLS_31} and \cite{ISAC_PLS_51} extended the aforementioned contribution to IRS-ISAC systems. In detail, when the security outage probability of each user fell below a predetermined threshold, the transmit BF vector and the IRS reflection coefficient matrix were optimized in order to maximize the echo signal SINR as well as security rate. 
The authors of \cite{ISAC_PLS_32} used neural networks to optimize the transmit signal precoders to minimize the maximum SINR of the Eve. In a UAV-IRS-ISAC system, the DRL framework was employed to optimize the transmit signal BF vector and the coefficient matrix of the IRS loaded by a UAV to maximize the security sum rate \cite{ISAC_PLS_34}. In \cite{ISAC_PLS_35}, NOMA and AN were adopted to jointly optimize the radar correlation and transmit signal BF vector to maximize echo signal power. %The authors of reference [51] improve the safety rate of the IRS-ISAC system by jointly designing the BF vector of the transmitted signal and the reflection coefficient matrix of the active IRS.

%