Consider an inner product space $(\universe, \innerprod)$. A \mydef{(generalized\footnote{When $\vioperset$ is singleton-valued a generalized variational inequality is simply called a variational inequality. As our computational results will be limited to generalized variational inequalities where $\vioperset$ is singleton-valued, for simplicity, we will refer to generalized variational inequalities simply as variational inequalities.}) variational inequality (VI)}, denoted $(\set, \vioperset)$, consists of a \mydef{constraint set} $\set \subseteq \universe$ and an \mydef{optimality operator} $\vioperset: \universe \rightrightarrows \universe^*$. For notational convenience, for any $\vartuple \in \set$, we denote any arbitrary element of $\vioperset(\vartuple)$ by $\vioper(\vartuple)$, and denote the variational inequality by $(\set, \vioper)$ if $\vioperset$ is singleton-valued.

\if 0
Any VI $(\set, \vioperset)$ defines a problem known as the 
\mydef{(generalized) Stampacchia variational inequality (SVI)} \cite{lions1967variational}: 
\begin{align}
        &\text{Find } \vartuple[][*] \in \set \text{ such that } &\innerprod[{\vioper(\vartuple[][*])}][{\vartuple - \vartuple[][*]}] \geq 0 && \text{for all } \vartuple \in \set\\
        &\text{and for some } &\vioper(\vartuple[][*]) \in \vioperset(\vartuple[][*])
\end{align}

\subsubsection{Solution Concepts}
A solution to a SVI is called a \mydef{strong solution} of the variational inequality $(\set, \vioperset)$. Just like in convex optimization settings (see Section 1.1.2 of \citet{nesterov1998introductory}), in practice, it is not possible to compute an exact strong solution to a VI $(\set, \vioperset)$, and as such we have to resort to approximate solutions which we call the $\vepsilon$-strong solution. Note that in the following definition, in line with the literature (see, for instance Section 1.2 of \citet{diakonikolas2020halpern}), the inequality is negated (and as such inverted). 
\fi

\subsection{Solution Concepts and Properties}
The canonical solution concept for VIs is the strong (or Stampacchia \cite{lions1967variational}) solution. In practice, it is not possible to compute an exact strong solution to a VI $(\set, \vioperset)$, and as such we have to resort to approximate solutions which we call the $\vepsilon$-strong solution. Given an \mydef{approximation parameter} $\vepsilon \geq 0$, a $\vepsilon$-\mydef{strong} (or \mydef{Stampacchia}) \mydef{solution} of the VI $(\set, \vioperset)$ is a $\vartuple[][*] \in \set$ s.t. for all $\vartuple \in \set$, there exists $\vioper(\vartuple[][*]) \in \vioperset(\vartuple[][*])$, $\innerprod[{\vioper(\vartuple[][*])}][{\vartuple[][*]  - \vartuple }] \leq \vepsilon$.
A $0$-strong solution is simply called a \mydef{strong solution}. 
We denote the set of $\varepsilon$-strong (resp. the set of strong) solutions a VI $(\set, \vioperset)$ by $\svi[\varepsilon](\set, \vioperset)$ (resp. $\svi(\set, \vioperset)$).
\if 0
\begin{definition}[Strong Solution]
    Given an \mydef{approximation parameter} $\vepsilon \geq 0$, a $\vepsilon$-\mydef{strong} (or \mydef{Stampacchia}) \mydef{solution} of the VI $(\set, \vioperset)$ is a $\vartuple[][*] \in \set$ that satisfies the following:
    \begin{align}
        &\exists \vioper(\vartuple[][*]) \in \vioperset(\vartuple[][*]), & \max_{\vartuple \in \set} \innerprod[{\vioper(\vartuple[][*])}][{\vartuple[][*]  - \vartuple }] \leq \vepsilon
    \end{align}

    A $0$-strong solution is simply called a \mydef{strong solution}. 
    We denote the set of $\varepsilon$-strong (resp. the set of strong) solutions a VI $(\set, \vioperset)$ by $\svi[\varepsilon](\set, \vioperset)$ (resp. $\svi(\set, \vioperset)$).
\end{definition}
\fi
% \deni{ADD PROOF TO THM BELOW:}
% \removed{
% \begin{proof}[Proof of \Cref{thm:exist_strong_sol}]
%     Given a continuous VI $(\set, \vioperset)$, define the operator $\fpoper(\vartuple) \doteq \proj[\set] \left[ \vartuple - \vioper(\vartuple) \right]$. Notice that any fixed point $\fpoper$, i.e., any point $\vartuple[][*] \in \set$, s.t. 
% \end{proof}
% }

Strong solutions can be shown to exist in a broad of class known as continuous. A \mydef{continuous} VI is a VI $(\set, \vioperset)$ s.t. $\set$ is non-empty, compact, and convex, and $\vioperset$ is  upper hemcontinuous, non-empty-, compact-, and convex-valued.
% 
The proof of existence of a strong solution in continuous VIs relies on a fixed-point argument applied to a mapping whose fixed points correspond to strong solutions of the VI, whose fixed points can in turn be shown to exist by the Glicksberg-Kakutani fixed point theorem. We refer the reader to Theorem 2.2.1 of \citet{facchinei2003finite} for a reference.

An alternative but related solution to a VI is the weak (or Minty) solution \cite{minty1967generalization}, for which similarly, we can define an approximate variant for computational purposes. Given a VI $(\set, \vioperset)$ and an \mydef{approximation parameter} $\vepsilon \geq 0$, a \mydef{$\vepsilon$-weak (or Minty) solution} is a $\vartuple[][*] \in \set$ s.t. for all $\vartuple \in \set, \vioper(\vartuple) \in \vioperset(\vartuple)$:
$\innerprod[{\vioper(\vartuple)}][{\vartuple[][*] - \vartuple}] \leq \vepsilon$.
A $0$-weak solution to the VI is simply called a \mydef{weak solution}. 
We denote the set of $\varepsilon$-weak (resp. the set of weak) solutions a VI $(\set, \vioperset)$ by $\mvi[\varepsilon](\set, \vioperset)$ (resp. $\mvi(\set, \vioperset)$).


\if 0 
\begin{definition}[Weak (or Minty) Solution]
    Given a VI $(\set, \vioperset)$ and an \mydef{approximation parameter} $\vepsilon \geq 0$, a \mydef{$\vepsilon$-weak (or Minty) solution}
    % \footnote{This solution is often referred to as the ``weak'' Minty solution as for all $\vartuple \in \set$, the condition has to hold only for some optimality condition $\vioper(\vartuple) \in \vioperset(\vartuple)$. One can further restrict the set of solutions by requiring condition to hold for all optimality condition $\vioper(\vartuple) \in \vioperset(\vartuple)$ at $\vartuple \in \set$, in which case the solution is called a strong Minty solution.}
    is a $\vartuple[][*] \in \set$ that satisfies the following:
    \begin{align}
    &\max_{\substack{\vartuple \in \set\\ \vioper(\vartuple) \in \vioperset(\vartuple)}} \innerprod[{\vioper(\vartuple)}][{\vartuple[][*] - \vartuple}] \leq \vepsilon
    \end{align}

    A $0$-weak solution to the VI is simply called a \mydef{weak solution}. 
    We denote the set of $\varepsilon$-weak (resp. the set of weak) solutions a VI $(\set, \vioperset)$ by $\mvi[\varepsilon](\set, \vioperset)$ (resp. $\mvi(\set, \vioperset)$).
\end{definition}
\fi 

In continuous VIs, the set of weak solutions is a subset of set of strong solutions, i.e., the MVI is a refinement of the SVI. However, we note that a weak solution is in general not guaranteed to exist in continuous VIs.
Additionally, if we assume that the optimality operator $\vioperset$ is monotone, then the set of strong and weak solutions are equal. Surprisingly, a $\vepsilon$-weak-solution is not guaranteed to be a $\vepsilon$-strong solution. However, if $\vioperset$ is assumed to be monotone then any $\vepsilon$-strong solution is also a $\vepsilon$-weak solution but not vice versa. 

The following additional properties of VIs will be relevant in the sequel, and will define important properties of the set of strong and weak solutions of VIs. 
A VI $(\set, \vioperset)$ is \{ \mydef{monotone}, \mydef{pseudomonotone}, \mydef{quasimonotone} \} iff the optimality operator $\vioperset$ is \{ monotone, pseudomonotone, quasimonotone \}.
% 
Another common more property for the analysis of VIs is known as the Minty condition. A VI $(\set, \vioperset)$ satisfies the \mydef{Minty condition} iff the set of weak solutions is non-empty, i.e., $\mvi(\set, \vioperset) \neq \emptyset$.
% 
With these definitions in order, we summarize the following known properties of the solution sets of VIs.

\begin{remark}[Solution Set Properties]
Let $\varepsilon \geq 0$, then the following implications hold:

    \begin{itemize}
        \item $(\set, \vioperset)$ is continuous $\implies$ $\svi(\set, \vioperset) \neq \emptyset$ (Theorem 2.2.1 of \citet{facchinei2003finite}))
        \item $(\set, \vioperset)$ is continuous $\implies$ $\mvi(\set, \vioperset) \subseteq \svi(\set, \vioperset)$ 
        \item $(\set, \vioperset)$ is monotone $\implies$ $\svi[\varepsilon](\set, \vioperset) \subseteq \mvi[\varepsilon](\set, \vioperset)$
        \item $(\set, \vioperset)$ is pseudomonotone $\implies$ $\svi(\set, \vioperset) \subseteq \mvi(\set, \vioperset)$
        \item $(\set, \vioperset)$ is quasimonotone with $\set$ non-empty, and compact $\implies$ $\mvi(\set, \vioperset) \neq \emptyset$ (Lemma 3.1 of \cite{he2017solvability})
        \item Suppose $\svi(\set, \vioperset) \neq \emptyset$, then: monotone $\implies$ pseudomonotone $\implies$ Minty's condition
    \end{itemize}
    
\end{remark}

\if 0
Note that while it has become common place to use the Minty condition in the analysis of VIs as it is much more general (see, for instance, \citet{he2022convergence}), the Minty condition can at the cost of generality be replaced by the assumption that the VI $(\set, \vioperset)$ is quasimonotone with $\set$ non-empty, and compact by Lemma 3.1 and Proposition 3.1 of \cite{he2017solvability}.
\fi