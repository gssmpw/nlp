We now turn our attention to the computation of solutions to variational inequalities. In what follows, for simplicity, we will restrict ourselves to VIs $(\set, \vioperset)$ in which $\vioperset$ is singleton-valued, which we will for simplicity denote as $(\set, \vioper)$. In future work, the algorithms and results provided in this chapter could be extended to the more general non-singleton-valued VI setting.
% 
We will in this paper consider first-order methods for computing strong solutions of VIs. We will hereafter restrict ourselves to singleton-valued optimality operators $\vioperset(\vartuple) \doteq \{\vioper(\vartuple))$.
Given a VI $(\set, \vioper)$, and an initial iterate $\vartuple[][][0] \in \set$, a \mydef{first-order method} $\kordermethod$ consists of an update function which generates the sequence of iterates $\{\vartuple[][][\numhorizon]\}_{\numhorizon}$ given for all $\numhorizon = 0, 1, \hdots$ by:
$
    \vartuple[][][\numhorizon + 1] \doteq  \kordermethod \left(\bigcup_{i = 0}^{\numhorizon} (\vartuple[][][i],  \vioper(\vartuple[][][i])) \right).
$

When $\kordermethod$ depends solely on the last item in the sequence, we simply write $ \vartuple[][][\numhorizon + 1] = \kordermethod(\vartuple[][][\numhorizon],  \vioper(\vartuple[][][\numhorizon]))$. As is standard in the literature (see, for instance, \citet{cai2022tight}), given a VI $(\set, \vioper)$, the computational complexity measures in this paper will take the number of evaluations of the optimality operator $\vioper$ as the unit of account.
% 
A common assumption for obtain polynomial-time computation for strong solutions of VIs, is Lipschitz-continuity. Given a modulus of continuity $\lsmooth \geq 0$, a $\lsmooth$-\mydef{Lipschitz-continuous} VI is a VI $(\set, \vioper)$ s.t. $\set$ is non-empty, compact, and convex, and $\vioper$ is $\lsmooth$-Lipschitz continuous.

