\begin{abstract}
    We study Walrasian economies (or general equilibrium models) and their solution concept, the Walrasian equilibrium. A key challenge in this domain is identifying price-adjustment processes that converge to equilibrium. One such process, tâtonnement, is an auction-like algorithm first proposed in 1874 by Léon Walras. While continuous-time variants of tâtonnement are known to converge to equilibrium in economies satisfying the Weak Axiom of Revealed Preferences (WARP), the process fails to converge in a pathological Walrasian economy known as the Scarf economy. To address these issues, we analyze Walrasian economies using variational inequalities (VIs), an optimization framework. We introduce the class of mirror extragradient algorithms, which, under suitable Lipschitz-continuity-like assumptions, converge to a solution of any VI satisfying the Minty condition in polynomial time. We show that the set of Walrasian equilibria of any balanced economy—which includes among others Arrow-Debreu economies—corresponds to the solution set of an associated VI that satisfies the Minty condition but is generally discontinuous. Applying the mirror extragradient algorithm to this VI we obtain a class of tâtonnement-like processes, which we call the mirror extratâtonnement process. While our VI formulation is generally discontinuous, it is Lipschitz-continuous in variationally stable Walrasian economies with bounded elasticity—including those satisfying WARP and the Scarf economy—thus establishing the polynomial-time convergence of mirror extratâtonnement in these economies. We validate our approach through experiments on large Arrow-Debreu economies with Cobb-Douglas, Leontief, and CES consumers, as well as the Scarf economy, demonstrating fast convergence in all cases without failure. Our results suggest that the lack of polynomial-time computability results for general Arrow-Debreu economies is largely a theoretical issue stemming from discontinuities rather than fundamental computational intractability. This provides one resolution to the challenge set by Herbert Scarf’s fifty-year-old agenda on applied general equilibrium—namely, providing “a general method for the explicit numerical solution of the neoclassical model.”
\end{abstract}