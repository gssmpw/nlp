\subsection{A Tractable Variational Inequality Framework for Walrasian Economies}
To address the challenge brought forward by the impossibility result of \citet{papadimitriou2010impossibility}, we provide a characterization of Walrasian equilibrium using the variational inequality (VI) optimization framework. To this end, we first introduce the class of \mydef{mirror extragradient algorithms} and prove the polynomial-time convergence of this method for VIs thta satisfy a computational tractability condition known as the Minty condition \cite{minty1967generalization}, and a generalization of Lipschitz-continuity known as Bregman-continuity. A Bregman-continuous (or relatively continuous \cite{lu2019relative}) function is one for which the change in the Euclidean distance of the function between any two points is proportional to Bregman divergence between those two points. 
% The Minty condition can be interpreted as an analog of a computational tractability assumption known as gradient dominance in non-convex optimization problem which states that any stationary point of the objective is also a global minimum.   

With these tools in place, we then demonstrate that the set of Walrasian equilibria of \mydef{balanced economies}---those Walrasian economies with an excess demand that is homogeneous of degree $0$, and satisfy weak Walras's (i.e., the value of the excess demand is less than or equal to $0$ at any price)---a class of Walrasian economies which among others includes Arrow and Debreu's competitive economies \cite{arrow-debreu}, is equal to the set of strong solutions of a VI that satisfies the Minty condition \cite{minty1967generalization}. With this characterization in hand, we apply the mirror extragradient algorithm to solving this VI, which gives rise to a novel natural price-adjustment process we call \mydef{mirror \emph{extrat\^atonnement}}.

An important property of the VI we introduce is that its search space for prices is \emph{not\/} restricted to the unit simplex as it is traditionally the case for competitive economies, but rather to the unit box. This fact offers us insight into understanding how we can overcome \citeauthor{papadimitriou2010impossibility}'s impossibility result on the exponential-time convergence of price-adjustment processes in general Walrasian economies. \citeauthor{papadimitriou2010impossibility}'s definition of a price-adjustment process restricts prices generated by the process to lie within the unit simplex; however, when the search space of the VI we introduce is restricted in this way, the VI fails to satisfy the Minty condition, and is thus computationally intractable. This suggests that relaxing the requirement that prices lie within the unit simplex can overcome the challenge of the exponential-time convergence of price-adjustment processes in Walrasian economies, and allow for the efficient computation of Walrasian equilibrium, at least in practice.

The reader might wonder what we mean by ``in practice''. As it turns out the VI characterization we provide is in general discontinuous at one point in its search space, namely when the prices for all commodities are $0$. As such, because it is not possible to ensure the Lipschitz-continuity or Bregman-continuity of the excess demand on the unit box in general, it is not possible to obtain polynomial-time convergence of our mirror extragradient to solve our VI without further assumptions. Nevertheless, as we discuss in the sequel, we observe the fast convergence of mirror \emph{extrat\^atonnement} process in a large class of competitive economies, including very large instances with Leontief consumers, for which the computation of a Walrasian equilibrium is known to be PPAD-complete \cite{codenotti2006leontief, deng2008computation}. This suggests the need for a novel assumption that would explain the convergence of process to a Walrasian equilibrium in practice. To this end, we introduce the pathwise Bregman-continuity assumption, a condition that requires the excess demand to be Bregman-continuous along the sequence of prices generated by the mirror \emph{extrat\^atonnement} process, which we show is sufficient to guarantee the polynomial-time convergence of our process.

% we introduce a family of price-adjustment processes which we call the \mydef{mirror \emph{extrat\^atonnement} processes} which we show converges to a Walrasian equilibrium in polynomial-time under a novel pathwise Bregman-continuity assumption in . A Bregman-continuous (or relatively continuous \cite{lu2019relative}) function is a function for which the change in the euclidean distance of the function between any two points is proportional to Bregman divergence between those two points. Accordingly, the pathwise Bregman-continuity assumption we make requires the excess demand to be Bregman-continuous along the sequence of prices generated by the mirror \emph{extrat\^atonnement} processes.

While the pathwise Bregman-continuity assumption provides intuition on the fast convergence of the mirror \emph{extrat\^atonnement} processes in practice, it is hard to very this assumption theoretically. \amy{theoretically? or in advance. i.e., without just running the process!} Thus, we subsequently restrict our search space for prices to the unit simplex, and restrict our attention to competitive economies that are variationally stable on the unit simplex (i.e., those economies for which the associated VI satisfies the Minty condition) and have a \mydef{bounded elasticity of excess demand} (i.e., the percentage change in the excess demand for a percentage change in prices is bounded across all price changes). We demonstrate that under these additional assumptions, the VI is guaranteed to satisfy the Minty condition, and show that for such economies the excess demand is Bregman-continuous, thus providing the first polynomial-time convergence result for a price adjustment processes in this class of Walrasian economies, which among others includes competitive economies that satisfy WGS, and more generally, WARP. 





\if 0
In \Cref{chap:vis}, after reviewing background material on variational inequalities, we introduce two new types of methods with polynomial-time convergence guarantees.

The first type of methods is a family of first-order methods known as the mirror extragradient method. We prove that this method converges to a strong solution of any variational inequality for which a weak solution exists. Furthermore, in the absence of a weak solution, we establish local convergence to a strong solution when the algorithm’s first iterate is initialized sufficiently close to a local weak solution. Since first-order methods are not guaranteed to converge beyond settings where a (local) weak solution exists, we then turn our attention to a class of second-order methods known as merit function methods. In particular, we introduce the primal mirror descent, which we show is guaranteed to converge to a local minimum of the regularized primal gap function of any Lipschitz-smooth variational inequality.

In \Cref{chap:walrasian_economies}, after reviewing background material on Walrasian economies, we demonstrate that the set of Walrasian equilibria of any Walrasian economy is equivalent to the set of strong solutions of an associated variational inequality. Additionally, we show that applying the gradient method to this variational inequality corresponds to solving the Walrasian economy through a well-known price-adjustment process known as \emph{tâtonnement}. Extending this analysis, we introduce a new family of price-adjustment processes, termed the mirror \emph{extratâtonnement} process, by applying the mirror extragradient method to the variational inequality formulation. Leveraging results from the prior section, we establish that this process converges to a Walrasian equilibrium in all Walrasian economies whose excess demand satisfies the Weak Axiom of Revealed Preferences (WARP).

As neither \emph{tâtonnement} nor \emph{extratâtonnement} processes are guaranteed to converge beyond this class of Walrasian economies, we introduce a class of merit function methods tailored for Walrasian economies with a Lipschitz-smooth excess demand. These methods are guaranteed to converge to a local minimum of the regularized primal gap function of any Lipschitz-smooth variational inequality. Our approach provides a novel perspective on Walrasian equilibria through the lens of variational inequalities, enabling the design of price-adjustment processes with strong theoretical guarantees. Additionally, we validate our framework through numerical experiments on large Arrow-Debreu economies with Cobb-Douglas, Leontief, and CES consumers, as well as the Scarf economy, demonstrating robust convergence in all cases. These results suggest that the computational intractability of general equilibrium models primarily arises from discontinuities rather than fundamental algorithmic limitations, addressing a long-standing challenge posed by Herbert Scarf regarding the explicit numerical solution of neoclassical equilibrium models.

\fi