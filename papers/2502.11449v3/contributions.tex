\subsection{Technical Contributions}

\paragraph{Variational Inequalities}
Our first major contribution is introducing the class of mirror extragradient algorithms, a generalization of Korpelevich's extragradient method \cite{korpelevich1976extragradient} for solving VIs. We establish best-iterate convergence of the class of mirror extragradient algorithms to a $\varepsilon$-strong solution of VIs that satisfy the Minty condition and are Bregman-continuous in $O(\nicefrac{1}{\varepsilon^2})$ evaluations of the optimality operator of the VI (\Cref{thm:mirror_extragradient_global_convergence}). Our result generalizes the results and proof techniques of \citet{huang2023beyond} for the extragradient method, and extends the convergence results of \citet{zhang2023mirror} for the unconstrained mirror extragradient method to constrained domains. In addition, to provide further justification for the convergence of the mirror extrat\^atonnement process in balanced economies, we establish suitable conditions for the local convergence of the mirror extragradient algorithm to an $\varepsilon$-strong solution of any Bregman-continuous VI that does \emph{not\/} satisfy the Minty condition---to the best of our knowledge, the first result of its kind (\Cref{thm:vi_mirror_extragrad_local}).



\paragraph{Walrasian Economies}

While a characterization of the set of Walrasian equilibria of any Walrasian economy as the solution set of an associated complementarity problem (i.e., a VI where the constraint set is the positive orthant) seems to have already been known \cite{dafermos1990exchange}, for balanced economies, we provide the first computationally tractable characterization of Walrasian equilibria as the set of strong solutions of a VI that satisfies the Minty condition and whose constraint set is given by the unit box. We then apply the mirror extragradient method to obtain a novel natural price-adjustment process we call the mirror \emph{extrat\^atonnement\/} process (\Cref{alg:mirror_extratatonnement}), and prove its convergence in all balanced economies that satisfy pathwise Bregman-continuity (\Cref{thm:bregman_mirror_exta_tatonn_convergence}).

We then restrict our attention to a novel class of competitive economies, namely those which are variationally stable on the unit simplex, and establish the polynomial-time convergence of the mirror \emph{extrat\^atonnement} process in all such economies assuming bounded elasticity of excess demand (\Cref{thm:mirror_extratatonn_var_stable}). Our convergence result also provides the first polynomial-time convergence result for price-adjustment processes in the class of economies that satisfy WARP, and generalizes the well-known \emph{t\^atonnement\/} convergence result in competitive economies with bounded elasticity of excess demand that satisfy WGS \cite{codenotti2005market}.

We then apply the mirror \emph{extrat\^atonnement} process to the Scarf economy, and prove its polynomial-time convergence to the unique Walrasian equilibrium of the economy (\Cref{thm:scarf_convergence}). As such, the mirror \emph{extrat\^atonnement\/} process is the first discrete-time \emph{natural} price adjustment process to converge in the Scarf economy.

Finally, we run a series of experiments on a variety of competitive economies where we verify that the pathwise Bregman-continuity assumption holds, and demonstrate that our algorithm converges to a Walrasian equilibrium at the rate predicted by our theory. Importantly, our experiments include examples of randomly initialized very large competitive economies ($\sim 500$ consumers and $\sim 500$ commodities) which are known to be PPAD-complete (e.g., Leontief economies), for which we show that our algorithm computes a Walrasian equilibrium fast without failure in all cases. 
% \amy{AND FAST!!! your results are non-asymptotic, n'est-ce pas?} \deni{Maybe add something on Scarf's challenge}