\begin{wraptable}{R}{0.70\textwidth}
\begin{minipage}{0.70\textwidth}
   \footnotesize
   \begin{table}[H]
   \caption{Summary of Setups for Arrow-Debreu Exchange Economy Experiments}\label{table:exp_summary}
   \begin{center}
   % \centering
    \begin{tabular}{p{1cm} | p{1.2cm} p{1.2cm} p{1.2cm} p{1.5cm} p{1.5cm} p{1.2cm}}
    \hline
    \makecell{Exp \\ No.} & 
    \makecell{Num. \\ Comm.} & 
    \makecell{Num. \\ Linear \\ Cons.} & 
    \makecell{Num. \\ Cobb\\-Doug. \\ Cons.} & 
    \makecell{Num. \\  CES \\ $\rho \! \in \! \!(\!0, 1\!)$  \\ Cons.} & 
    \makecell{Num. \\  CES \\ $\rho <  0$ \\ Cons.} & 
    \makecell{Num. \\ Leont. \\ Cons.} \\ \hline \hline
    1 & 500 & 0 & 0 & 0 & 0 & 600 \\
    2 & 500 & 0 & 0 & 0 & 600 & 0\\
    3 & 500 & 0 & 0 & 600 & 0 & 0\\
    4 & 500 & 0 & 600 & 0 & 0 & 0\\
    5 & 500 & 600 & 0 & 0 & 0 & 0\\
    6 & 1000 & 200 & 200 & 200 & 200 & 200\\
    7 & 1000 & 0 & 200 & 200 & 200 & 200\\ \hline
    \end{tabular}
   \end{center}
\end{table}
\end{minipage}
\end{wraptable}


In this section, we first apply the \emph{t\^atonnement} and mirror \emph{extrat\^atonnement} process with kernel function $\kernel(\price) \doteq \|\price\|^2$, first to solve the Scarf economy with the goal of illustrating the differing convergence behavior between the two price-adjustment processes. We then apply the mirror \emph{extrat\^atonnement} process with kernel function $\kernel(\price) \doteq \|\price\|^2$ to solve a number of Arrow-Debreu exchange economies \cite{arrow-debreu} with the goal of demonstrating that our pathwise Bregman-continuity assumption holds, and that the mirror \emph{extrat\^atonnement} process can efficiently solve very large Walrasian economies in practice.

\deni{REVIEWWWWWW}We record in the first 2 leftmost plots of \Cref{fig:exp_results} the movement of prices in the Scarf economy for the \emph{t\^atonnement} and mirror \emph{extrat\^atonnement} processes respectively. As is well-established, the sequence of prices generated by \emph{t\^atonnement}, despite starting very close to the equilibrium prices $(\nicefrac{1}{3}, \nicefrac{1}{3}, \nicefrac{1}{3})$ spiral away from the prices, converging to the $(0, 0, 1)$ price vector which is not a Walrasian equilibrium. In contrast, the prices generated by the mirror \emph{extrat\^atonnement} process spiral inwards towards the equilibrium price despite starting far way from the equilibrium prices. An intuitive explanation of this behavior is as follows, the continuous-time variant of \emph{t\^atonnement} is known to cycle around the equilibrium prices \cite{scarf1960instable}. Now, one way to intepret the discrete-time \emph{t\^atonnement} (resp. mirror \emph{extrat\^atonnement}) process is as an explicit (resp. implicit) discretization \cite{butcher2008numerical} of the continuous-time \emph{t\^atonnement} dynamics. A well-known fact is that explicit (resp. implicit) discretization methods are unstable (resp. stable) when the continuous-time dynamics cycle, thus explaining the observed behavior.

An Arrow-Debreu exchange economy $(\numbuyers, \numcommods, \consumptions, \consendow, \util)$ consists of $\numcommods \in \N$ commodities, $\numbuyers \in \N$ consumers each $\consumer \in [\numconsumers]$ with a consumption space $\consumptions[\buyer]$, an endowment of commodities  $\consendow[\consumer] \in \R^\numcommods_+$, and a utility function $\util[\consumer]: \consumptions[\buyer] \to \R$. An Arrow-Debreu exchange economy $(\numbuyers, \numcommods, \consumptions, \consendow, \util)$ can be represented as a bounded continuous competitive economy $(\numgoods, \excessset)$ where the excess demand correspondence is given as: $\excessset(\price) \doteq \sum_{\player \in \players} \argmax\limits_{\consumption[\consumer] \in \consumptions[\consumer]: \consumption[\consumer] \cdot \price \leq \consendow[\consumer] \cdot \price}  \util[\consumer](\consumption[\consumer]) - \sum_{\consumer \in \consumers} \consendow[\consumer]$.\footnote{We refer the reader to \Cref{sec_app:ad_comp} on additional background and definitions on Arrow-Debreu exchange economies.} 


\begin{figure}[htbp]
    % \centering
    \begin{subfigure}{0.3\textwidth}
        \centering
        % \hspace{1em}
        \includegraphics[height=45mm, width=45mm]{figures/scarf_tatonn.jpg}
        % \caption{$f_1(x) = \sin(x)$}
    \end{subfigure}
    \hfill
    \begin{subfigure}{0.3\textwidth}
        \centering
        \includegraphics[height=45mm, width=45mm]{figures/scarf_extratonn.jpg}
        % \caption{$f_1(x) = \sin(x)$}
    \end{subfigure}
    \hfill
    \begin{subfigure}{0.3\textwidth}
        \centering
        \hspace{1cm}
        \includegraphics[height=30mm, width=45mm]{figures/ex_1.jpg}    
        % \vspace{2em}
        % \caption{$f_1(x) = \sin(x)$}
    \end{subfigure}
    \vspace{0.3cm}
    \begin{subfigure}{0.3\textwidth}
        \centering
        \includegraphics[height=30mm, width=45mm]{figures/ex_2.jpg}
        % \caption{$f_2(x) = \cos(x)$}
    \end{subfigure}
    \hfill
    \begin{subfigure}{0.3\textwidth}
        \centering
        \includegraphics[height=30mm, width=45mm]{figures/ex_3.jpg}
        % \caption{$f_3(x) = x^2$}
    \end{subfigure}
    \hfill
    \begin{subfigure}{0.3\textwidth}
        \centering
        \includegraphics[height=30mm, width=45mm]{figures/ex_4.jpg}
        % \caption{$f_4(x) = x^3$}
    \end{subfigure}
    \hfill
    \begin{subfigure}{0.3\textwidth}
        \centering
        \includegraphics[height=30mm, width=45mm]{figures/ex_5.jpg}
        % \caption{$f_5(x) = e^x$}
    \end{subfigure}
    \hfill
    \begin{subfigure}{0.3\textwidth}
        \includegraphics[height=30mm, width=45mm]{figures/ex_6.jpg}
        % \caption{$f_6(x) = \ln(x)$}
    \end{subfigure}
    \hfill
    \begin{subfigure}{0.3\textwidth}
        \centering
        \includegraphics[height=30mm, width=45mm]{figures/ex_7.jpg}
        % 
        % \caption{$f_7(x) = \frac{1}{x}$}
    \end{subfigure}
    
    \caption{Phase Portraits  of \emph{T\^atonnement}, and \emph{Extrat\^atonnement} for the Scarf Economy, and Results of Experiments 1-7.}
    \label{fig:exp_results}
\end{figure}


% 
\begin{figure}[htbp]
    % \centering
    \includegraphics[scale=0.45, angle=-90]{figures/full_plots_arxiv.pdf}
    \caption{Phase Portraits  of \emph{T\^atonnement}, and \emph{Extrat\^atonnement} for the Scarf Economy, and Results of Experiments 1-7.}
    \label{fig:exp_results}
\end{figure}



We consider the following utility function classes to run our experiments:
1.~linear: $\util[\buyer](\allocation[\buyer]) = \sum_{\good \in \goods} \valuation[\buyer][\good] \allocation[\buyer][\good]$; 
2.~Cobb-Douglas:  $\util[\buyer](\allocation[\buyer]) = \prod_{\good \in \goods} \allocation[\buyer][\good][][{\valuation[\buyer][\good]}]$; 
3.~Leontief:  $\util[\buyer](\allocation[\buyer]) = \min_{\good \in \goods} \left\{ \nicefrac{\allocation[\buyer][\good]}{\valuation[\buyer][\good]}\right\}$; 
and 4.~CES: $\util[\buyer](\allocation[\buyer]) = \sqrt[{\rho_\buyer}]{ \sum_{\good \in \goods} \valuation[\buyer][\good] \allocation[\buyer][\good][][{\rho_\buyer}]}$ with each utility function parameterized by a vector of valuations $\valuation[\buyer] \in \mathbb{R}_+^{\numbuyers}$, where each $\valuation[\buyer][\good]$ quantifies the value of commodity $\good$ to consumer $\consumer$. 
We summarize the experiments we run in \Cref{table:exp_summary}. The parameters of each economy are initialized randomly according to the uniform random distribution.\footnote{ For reproducibility purposes, we include our code ready to run on \coderepo, and include all details of our experimental setup in \Cref{sec_ap:experiments}.} We record the results of our experiments in \Cref{fig:exp_results}, describing for what value of $\varepsilon \geq 0$, are the prices generated throughout the algorithm a $\varepsilon$-Walrasian equilibrium.

We observe that in all our experiments except in experiments 5 and 6---which include Linear consumers and are as such not covered by our theory as the excess demand for the economies is not singleton-valued---the mirror \emph{extrat\^atonnement} process converges to a Walrasian equilibrium. In all experiments, we verify and confirm that pathwise Bregman-continuity holds, thus justifying our assumption. Finally, while our experiments obey our theory which suggests a best-iterate convergence to a $\varepsilon$-Walrasian equilibrium in $\nicefrac{1}{\varepsilon^2}$ time-steps, we observe that a last-iterate convergence occurs only for experiments 5, corresponding to the case of Cobb-Douglas consumers, for which even \emph{t\^atonnement} is known to converge in last-iterates. This suggests that achieve convergence in last iterates might not be possible with the mirror \emph{extrat\^atonnement} process.


% 
\begin{figure}[htbp]
    % \centering
    \begin{subfigure}{0.3\textwidth}
        \centering
        % \hspace{1em}
        \includegraphics[height=45mm, width=45mm]{figures/scarf_tatonn.jpg}
        % \caption{$f_1(x) = \sin(x)$}
    \end{subfigure}
    \hfill
    \begin{subfigure}{0.3\textwidth}
        \centering
        \includegraphics[height=45mm, width=45mm]{figures/scarf_extratonn.jpg}
        % \caption{$f_1(x) = \sin(x)$}
    \end{subfigure}
    \hfill
    \begin{subfigure}{0.3\textwidth}
        \centering
        \hspace{1cm}
        \includegraphics[height=30mm, width=45mm]{figures/ex_1.jpg}    
        % \vspace{2em}
        % \caption{$f_1(x) = \sin(x)$}
    \end{subfigure}
    \vspace{0.3cm}
    \begin{subfigure}{0.3\textwidth}
        \centering
        \includegraphics[height=30mm, width=45mm]{figures/ex_2.jpg}
        % \caption{$f_2(x) = \cos(x)$}
    \end{subfigure}
    \hfill
    \begin{subfigure}{0.3\textwidth}
        \centering
        \includegraphics[height=30mm, width=45mm]{figures/ex_3.jpg}
        % \caption{$f_3(x) = x^2$}
    \end{subfigure}
    \hfill
    \begin{subfigure}{0.3\textwidth}
        \centering
        \includegraphics[height=30mm, width=45mm]{figures/ex_4.jpg}
        % \caption{$f_4(x) = x^3$}
    \end{subfigure}
    \hfill
    \begin{subfigure}{0.3\textwidth}
        \centering
        \includegraphics[height=30mm, width=45mm]{figures/ex_5.jpg}
        % \caption{$f_5(x) = e^x$}
    \end{subfigure}
    \hfill
    \begin{subfigure}{0.3\textwidth}
        \includegraphics[height=30mm, width=45mm]{figures/ex_6.jpg}
        % \caption{$f_6(x) = \ln(x)$}
    \end{subfigure}
    \hfill
    \begin{subfigure}{0.3\textwidth}
        \centering
        \includegraphics[height=30mm, width=45mm]{figures/ex_7.jpg}
        % 
        % \caption{$f_7(x) = \frac{1}{x}$}
    \end{subfigure}
    
    \caption{Phase Portraits  of \emph{T\^atonnement}, and \emph{Extrat\^atonnement} for the Scarf Economy, and Results of Experiments 1-7.}
    \label{fig:exp_results}
\end{figure}

