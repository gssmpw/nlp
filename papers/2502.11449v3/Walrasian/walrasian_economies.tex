


An \mydef{Walrasian economy} $(\numgoods, \excessset)$ consists of $\numgoods \in \N$ \mydef{commodities}\footnote{The ``commodity'' terminology is used here in the tradition of \citet{arrow-debreu}, and refers to any raw, intermediate, \& final commodities, labor \& services. }, with any quantity of any given commodity being exchangeable for a quantity of another. 
% \footnote{Units of commodities can be restricted to be discrete.} 
The exchange process is governed by a system of valuation called \mydef{prices} modeled as a vector $\price \in \R^\numgoods_+$ s.t. $\price[\good] \geq 0$ is the price of commodity $\good \in \goods$.\footnote{An observant reader might notice that in real-world economies the prices of certain commodities can be negative (e.g., prices of oil when storage of excess oil is not possible), and might rise the concern that the model does not account for the possibility negative prices. However, in these cases the price of the commodity is ``negative'' only colloquially speaking, and rather the price of an associated commodity is positive. For instance, when the price of oil is negative, companies are no more selling oil, rather they are buying a service: the storage of oil. As such, we include ``negative pricing'' in the real-world by adding the commodities with ``negative prices'' as additional commodities into the economy (e.g., including both oil and the sale of oil as commodities).} Prices $\price \in \R^\numgoods_+$ allow the \mydef{sale} of $x \in \R_+$ units of any commodity $\good \in \goods$ in exchange for the \mydef{purchase} of $\frac{x \price[\good]}{\price[k]}$ units of any other commodity $k \in \goods$.

For any price, the economy determines quantities of each commodity which can be bought and sold, with all admissible exchanges being summarized by an \mydef{excess demand correspondence} $\excessset: \R^\numgoods_+ \rightrightarrows \R^\numgoods$ which for any prices $\price \in \R^\numgoods_+$ outputs a \mydef{set of excess demands} $\excessset(\price) \subseteq \R^\numgoods$ with each excess demand denoted $\excess(\price) \in \excessset(\price)$. For any price $\price \in \R^\numgoods_+$ and excess demands $\excess(\price) \in \excessset(\price)$, $\excess[\good](\price) \geq 0$ denotes the number of units of commodity $\good \in \goods$ \mydef{demanded in excess} (i.e., more units of it are bought than sold) $\good \in \goods$, while $\excess[\good](\price) < 0$ denotes the number of units of commodity $\good \in \goods$ \mydef{supplied in excess} (i.e., more units of it are sold than bought). If $\excessset$ is singleton-valued, then we will for convenience represent $\excessset$ as a function and denote it $\excess$. 

A price vector $\price \in \R^\numgoods_+$ is said to be \mydef{feasible} if there exists a $\excess(\price) \in \excessset(\price)$ s.t. for all commodity $\good \in \goods$, $\excess[\good](\price) \leq 0$. Similarly, a price vector $\price \in \R^\numgoods_+$ is said to satisfy \mydef{Walras' law} if there exists a $\excess(\price) \in \excessset(\price)$ s.t. $\price \cdot \excess(\price) = 0$. 
% 
% 
The canonical solution concept for Walrasian equilibria is the approximate Walrasian equilibrium \cite{walras}. 
% In the sequel, we will introduce algorithms with polynomial-time convergence guarantees to a Walrasian equilibrium, and to analyze their convergence will use a computationally relevant generalization of Walrasian equilibrium, namely the approximate Walrasian equilibrium to account for the bounded accuracy of computational methods. 
% 
% \begin{definition}[Approximate Walrasian Equilibrium]

Given an approximation parameter $\varepsilon \geq 0$, a price vector $\price[][][*] \in [0, 1]^\numgoods$ is said to be a $\varepsilon$-\mydef{Walrasian} (or $\varepsilon$-\mydef{competitive}) \mydef{equilibrium} of a Walrasian economy $(\numgoods, \excessset)$ if there exists an excess demand $\excess(\price[][][*]) \in \excessset(\price[][][*])$ s.t. we have: \mydef{($\varepsilon-$Feasibility)} for all commodities $\good \in \goods$, $\excess[\good](\price[][][*]) \leq \varepsilon$; and \mydef{($\varepsilon$-Walras' law)} $-\varepsilon \leq \price[][][*] \cdot \excess(\price[][][*]) \leq \varepsilon$
% 
We denote the set of $\varepsilon$-Walrasian equilibria of any Walrasian economy $(\numgoods, \excessset)$ by $\we[\varepsilon](\numgoods, \excessset)$.
% 
A $0$-Walrasian equilibrium is simply called a \mydef{Walrasian equilibrium}, in which case we denote the set of Walrasian equilibria $\we(\numgoods, \excessset)$.
% \end{definition}
% 
\if 0
\begin{definition}[Walrasian Equilibrium]
    A price vector $\price[][][*] \in \R^\numgoods_+$ is said to be a \mydef{Walrasian} (or $\varepsilon$-\mydef{competitive}) \mydef{equilibrium} \cite{walras} if there exists an excess demand $\excess(\price[][][*]) \in \excessset(\price[][][*])$ s.t. 
    \begin{enumerate}[leftmargin=5cm, itemsep=0cm, labelsep=0.5cm]
        \item[{(Feasility)}] $\excess(\price[][][*]) \leq \zeros$
        \item[{(Walras' law)}] $\price[][][*] \cdot \excess(\price[][][*]) = 0$
    \end{enumerate}
    % for all commodities $\good \in \goods$:
    % \begin{align}
    %     &\price[\good][][*] > 0 \implies  \excess[\good](\price[][][*]) = 0 
    %     &\price[\good][][*] = 0 \implies \excess[\good](\price[][][*]) \leq 0
    % \end{align}
\end{definition}
\fi
% 
\if 0
Seen otherwise, a Walrasian equilibrium $\price[][][*] \in \R^\numgoods_+$ is a price vector s.t.
for all commodities $\good \in \goods$, $\price[\good][][*] > 0 \implies  \excess[\good](\price[][][*]) = 0$ and $\price[\good][][*] = 0 \implies \excess[\good](\price[][][*]) \leq 0$. 
Intuitively, a Walrasian equilibrium is a price vector which ensures that the exchange of any commodity with another can be implemented. More precisely, on the one hand, if the price of a commodity $\good \in \goods$ is strictly positive then at a Walrasian equilibrium commodity $\good$ will always find a buyer since its excess demand is zero, which makes sense since the exchange system dictates that $\good$ can be exchanged for strictly positive units of some other commodity $k \in \goods$. On the other hand, if the price of commodity $\good$ is zero, at a Walrasian equilibrium the commodity might not find a buyer, which makes sense since the price system dictates that the commodity $\good$ cannot be exchanged for any other good.
\fi 
% More precisely, on one hand feasibility implies that any commodity $\good \in \goods$ can be bought in exchange for some commodity $k \in \goods$ since it is in excess supply. On the other hand, weak Walras' law implies that the valuation system induced by the price vector is consistent in the sense that the value of all commodities bought is equal to the value of all commodities sold, and hence the price exchange system can be implemented.  That is, 
   % 
% 


\subsection{Walrasian Economies and Variational Inequalities}

With definitions in order, we now present the fundamental relationship there exists between Walrasian economies and VIs.
The following theorem due to \citet{dafermos1990exchange}, is to the best of our knowledge the first result exposing the connection between VIs and Walrasian equilibria, and states that the problem of computing a Walrasian equilibrium is equivalent to the problem of computing a strong solution of a VI whose set of constraints is given by the positive ortanth (a class of VIs known as \mydef{complementarity problems} \cite{cottle1968complementary}). For completeness, we include its proof, as well as all other omitted results and proofs of this section in \Cref{sec_app:walrasian}.

\if 0
\begin{theorem}[Walrasian economies as Complementarity Problems]\label{thm:we_equal_svi}
    The set of Walrasian equilibria of any Walrasian economy $(\numgoods, \excessset)$ is equal to the set of strong solutions of the VI $(\R^\numgoods_+, -\excessset)$, i.e., $\we(\numgoods, \excessset) = \svi(\R^\numgoods_+, -\excessset)$.
\end{theorem}
\fi

\begin{restatable}[Walrasian economies as Complementarity Problems]{theorem}{thmweequalsvi}\label{thm:we_equal_svi}
    The set of Walrasian equilibria of any Walrasian economy $(\numgoods, \excessset)$ is equal to the set of strong solutions of the VI $(\R^\numgoods_+, -\excessset)$, i.e., $\we(\numgoods, \excessset) = \svi(\R^\numgoods_+, -\excessset)$.
\end{restatable}


While \Cref{thm:we_equal_svi} is useful to approach any Walrasian equilibrium computation problem as a strong solution computation problem, as the domain of prices is unbounded, i.e., $\R^\numgoods_+$, to obtain existence and computational results, we have to restrict the class of Walrasian economies we study. To this end, we introduce two important classes of Walrasian economies. The first of these classes are balanced economies. 

A \mydef{balanced economy} is a Walrasian economy $(\numgoods, \excessset)$ whose excess demand correspondence satisfies (Homogeneity of degree $0$) for all $\lambda >0$, $\excessset(\lambda \price) = \excessset(\price)$; and (Weak Walras' law) for all $\price \in \R^\numgoods_+$ and $\excess(\price) \in \excessset(\price)$,  $\price \cdot \excess(\price) \leq 0$.
% 
Intuitively, homogeneity requires that prices have a meaning only relative to other prices, and have no absolute meaning of their own; weak Walras' law requires budget-balance to hold. While homogeneity of degree 0 is a standard assumption, weak Walras' law is significantly weaker than standard assumptions previously considered in the literature (see, for instance \citet{arrow-hurwicz} and \citet{debreu1974excess}), and are satisfied by Arrow-Debreu economies \cite{arrow-debreu}.



% Intuitively, in the above definition, homogeneity requires that prices have a meaning only relative to other prices, and have no absolute meaning of their own (i.e., if all prices get scaled by the same amount the excess demand remains unchanged); and weak Walras' law requires budget-balance to hold (i.e., at any given prices, the total value of what is being demanded cannot exceed the value of what is supplied). The above conditions are weaker than standard assumptions previously considered in the literature (see, for instance \citet{arrow-hurwicz} and \citet{debreu1974excess}), and are satisfied by Arrow-Debreu economies \cite{arrow-debreu} (see, \Cref{chap:arrow_debreu_economies} for additional details).

% As we will show, Walrasian equilibria of homogeneous economies lend themselves to even more interesting characterizations as VIs. 

% With this definition, in hand, some important remarks are in order.
% \begin{remark}[Homogeneity and Normalized price space]\label{remark:ad_normalization}

%     % The assumption that the excess demand is homogeneous of degree $0$ is a highly standard assumption which we also make for expositional ease. Nevertheless, we note that our results extend to settings for which the excess demand is non-homogeneous, by assuming instead that prices lie in a compact 
% \end{remark}

\if 0
\begin{remark}[Walrasian equilibrium in Arrow-Debreu economies]
    Note that Walras' law implies weak Walras' law, as such in Arrow-Debreu economies a price vector is a competitive equilibrium if and only if it is a feasible price vector. As we will see in \Cref{chap:arrow_debreu_economies}, the assumption that Walras' law is a very natural assumption which is ensured to hold in many general equilibrium models.
\end{remark}
\fi 


% \begin{definition}[Non-trivial Walrasian economy]
%     A \mydef{Walrasian economy} $(\numgoods, \goodset, \excessset)$ is said to be non-trivial iff the vector $\zeros \in \R^\numgoods_+$ is not a Walrasian equilibrium.
% \end{definition}
\if 0
\begin{remark}[Bounding Walrasian Equilibirum Prices]\label{remark:homo_we_scaling}
    Now, as the excess demand correspondence in homogeneous of degree $0$, if $\price[][][*]$ is a Walrasian equilibrium price then for any $\lambda > 0$, $\lambda \price[][][*]$ is also a Walrasian equilibrium. Hence, in homogeneous economies, without loss of generality, we can restrict Walrasian equilibrium prices to be bounded by 1, i.e., $\price \in [0, 1]^\numgoods$.    
\end{remark}
\fi 

We now provide a novel characterization of Walrasian equilibrium prices in balanced economies as a VI over $[0, 1]^\numgoods$ rather than $\R^\numgoods_+$ which will allow us to obtain polynomial-time algorithms for the computation of Walrasian equilibrium, as the computational guarantees of our algorithms for VIs depend on the diameter of the constraint space of the VIs. 

% \begin{theorem}[Balanced economies as VIs]\label{thm:we_balanced_equal_svi}
%     For any balanced economy $(\numgoods, \excessset)$, the set of Walrasian equilibria is equal to the strictly positive cone generated by the strong solutions of the continuous VI $([0, 1]^\numgoods, -\excessset)$, i.e., $\we(\numgoods, \excessset) = \bigcup_{\lambda \geq 1} \lambda \svi([0, 1]^\numgoods, -\excessset)$.
% \end{theorem}

\begin{restatable}[Balanced economies as VIs]{theorem}{thmwebalancedequalsvi}\label{thm:we_balanced_equal_svi}
    For any balanced economy $(\numgoods, \excessset)$, the set of Walrasian equilibria is equal to the strictly positive cone generated by the strong solutions of the continuous VI $([0, 1]^\numgoods, -\excessset)$, i.e., $\we(\numgoods, \excessset) = \bigcup_{\lambda \geq 1} \lambda \svi([0, 1]^\numgoods, -\excessset)$.
\end{restatable}

In the sequel, we will make use of the following lemma which states that for any balanced economy $(\numgoods, \excessset)$, any approximate strong solution of the VI $([0, 1]^\numgoods, -\excessset)$ is an approximate Walrasian equilibrium of $(\numgoods, \excessset)$.

% \begin{lemma}[$\varepsilon$-strong solution and $\varepsilon$-Walrasian equilibrium]\label{lemma:approx_svi_eq_approx_we}
%     For any balanced economy $(\numgoods, \excessset)$, any $\varepsilon$-strong solution of the VI $([0, 1]^\numgoods, -\excessset)$ is a $\varepsilon$-Walrasian equilibrium of $(\numgoods, \excessset)$. 
%     % Conversely, any $\varepsilon$-Walrasian of $(\numgoods, \excessset)$ is a $(\numgoods + 1)\varepsilon$-strong solution of the VI $([0, 1]^\numgoods, -\excessset)$
% \end{lemma}
\begin{restatable}[$\varepsilon$-strong solution $\implies$ $\varepsilon$-Walrasian equilibrium]{lemma}{lemmaapproxsvieqapproxwe}\label{lemma:approx_svi_eq_approx_we}
    For any balanced economy $(\numgoods, \excessset)$, any $\varepsilon$-strong solution of the VI $([0, 1]^\numgoods, -\excessset)$ is a $\varepsilon$-Walrasian equilibrium of $(\numgoods, \excessset)$. 
\end{restatable}

We now turn our attention to prove the existence of Walrasian equilibrium. In balanced economies, under the assumption that the excess demand correspondence $\excessset$ is upper hemicontinuous, non-empty-, compact-, and convex-valued, it is possible to prove the existence of a Walrasian equilibrium $\price[][][*] \in [0, 1]^\numgoods$ as a corollary of the existence of strong solutions in continuous VIs (Theorem 2.2.1 of \citet{facchinei2003finite}). 
% 
Unfortunately, this Walrasian equilibrium can be trivial, i.e.,  $\price[][][*] = \zeros[\numgoods]$, and to prove the existence of a non-trivial Walrasian equilibrium, we have to restrict our attention to a canonical subset of balanced Walrasian economies studied in the literature which we call competitive economies \cite{debreu1974excess, sonnenschein1972market}. 

A \mydef{competitive economy} is a balanced economy $(\numgoods, \excessset)$ whose excess demand correspondence satisfies (Non-Satiation) for all $\price \in \R^\numgoods_+$, and $\excess(\price) \in \excessset(\price)$, $\excess(\price) \leq \zeros[\numgoods] \implies \price \cdot \excess(\price) = 0$. Intuitively the additional non-satiation condition requires that whenever all goods are supplied in excess, it must be that the economy cannot spend any more money on purchasing commodities (i.e., the value of the excess demand is 0). 
% As such, the excess demand is non-satiated, in the sense that the economy cannot demand more of any commodity because it cannot afford it, and not because it is not supplied in sufficient quantity.
In competitive economies, an alternative VI characterization of Walrasian equilibrium holds over the constraint space $\simplex[\numgoods]$ rather than $[0, 1]^\numgoods$, which is more suitable for proving existence. The canonical example of a competitive economy is the \mydef{Arrow-Debreu competitive economy} \cite{arrow-debreu} (see, \Cref{lemma:ad_economies_are_comp_bounded}, \Cref{sec_app:ad_comp}).

% \begin{theorem}[Competitive economies as VIs]\label{thm:we_comp_equal_svi}
%     For any competitive economy $(\numgoods, \excessset)$, the set of Walrasian equilibria is equal to the strictly positive cone generated by the strong solutions of the continuous VI $(\simplex[\numgoods], -\excessset)$, i.e., $\we(\numgoods, \excessset) = \bigcup_{\lambda \geq 1} \lambda \svi(\simplex[\numgoods], -\excessset)$.
% \end{theorem}

\begin{restatable}[Competitive economies as VIs]{theorem}{thmwecompequalsvi}\label{thm:we_comp_equal_svi}
     For any competitive economy $(\numgoods, \excessset)$, the set of Walrasian equilibria is equal to the strictly positive cone generated by the strong solutions of the continuous VI $(\simplex[\numgoods], -\excessset)$, i.e., $\we(\numgoods, \excessset) = \bigcup_{\lambda > 0} \lambda \svi(\simplex[\numgoods], -\excessset)$.
\end{restatable}



To prove existence, it will be necessary to make assumptions on the continuity of the excess demand, which necessitates the definition of continuous economies. A \mydef{continuous economy} is a Walrasian economy $(\numgoods, \excessset)$ whose excess demand correspondence $\excessset$ is upper hemicontinuous on $\simplex[\numgoods]$, non-empty-, compact-, and convex-valued. We note that in the following definition we assume upper hemicontinuity only on $\simplex[\numgoods]$, since in competitive, and more generally balanced, economies it is too restrictive to assume that the excess demand $\excessset$ is upper hemicontinuous on $\R^\numgoods_+$ since any correspondence which is homogeneous of degree $0$ and continuous in the entirety of its domain is constant.\footnote{In more stylized applications such as Arrow-Debreu competitive economies \cite{arrow-debreu}, the excess demand correspondence is sometimes defined so as to be guaranteed to be continuous only on the interior of the unit simplex, i.e., $\interior(\simplex[\numgoods]) = \simplex[\numgoods]$, as the excess demand for a good can be infinite if the price of any goods is 0. However, this issue in these stylized models only arises from a modeling choice which allows the demand of commodities to be possibly greater than the total amount of the commodity that can be ever supplied. However, it is indeed possible to restrict the excess demand to bounded by the total amount of the commodity that can be ever supplied without modifying the Walrasian equilibria of the economy. This is indeed the approach that \citet{arrow-debreu} take in Section 3 of their paper for proving their seminal Walrasian equilibrium existence result. In \Cref{lemma:ad_economies_are_comp_bounded} (\Cref{sec_app:ad_comp}) we prove that any Arrow-Debreu competitive economy can be represented as a continuous competitive economy with a bounded excess demand. This restriction is also realistic from an economic perspective since it is not possible for the economy to consume more of a commodity that there can exist, and resources in the real-world are indeed scarce. Indeed, otherwise there would be no use for the economic sciences: the science of resource allocation under scarcity.} Intuitively, continuous economies are those economies in which changes in the proportions of prices lead to well-behaved changes in excess demands.

With the above theorem in hand, we can leverage the fact that a strong solution is guaranteed in continuous VIs to establish the existence of a Walrasian equilibrium in continuous competitive economies. We note that any Arrow-Debreu competitive economy is a continuous economy, as we show in \Cref{lemma:ad_economies_are_comp_bounded} (\Cref{sec_app:ad_comp}). As such, the following result provides an alterative proof of existence of a Walrasian equilibrium in Arrow-Debreu competitive economies.


\begin{restatable}[Existence of Walrasian Equilibrium]{theorem}{thmexistencewe}\label{thm:existence_we}
    The set of Walrasian equilibria of any continuous competitive economy $(\numgoods, \excessset)$ is non-empty, i.e., $\we(\numgoods, \excessset) \neq \emptyset$.
\end{restatable}

With our characterization of Walrasian equilibria as strong solutions of VIs complete, we now turn our attention to solving the VIs we have introduced.
% \if 0
% \begin{remark}[Existence of Non-Trivial Walrasian Equilibrium]
%     An observant reader might argue that the price vector $\zeros[\numgoods]$ is a trivial Walrasian equilibrium and as such the above existence result is not of great importance. However, allowing for equilibrium prices to be $\zeros[\numgoods]$ does not invalidate the above existence result as in many applications of Walrasian economies (e.g., see the Arrow-Debreu economy application in \Cref{chap:arrow_debreu_economies}), logically, at prices $\zeros[\numgoods]$, the excess demand is strictly positive, meaning that $\zeros[\numgoods]$ cannot be a Walrasian equilibrium, and as such the above result implies the existence of a non-trivial Walrasian equilibrium. 
    
%     For the unconvinced reader, we remark that by assuming in addition to  $(\numgoods, \excessset)$ being balanced, that for all $\price \in \R^\numgoods_+$, $\price \cdot \excess(\price) \geq 0$ (i.e., Walras' law holds), one can show that a straightforward modification of the statement and proof of \Cref{thm:we_balanced_equal_svi} which restricts the prices to lie in $\simplex[\numgoods]$ rather than $[0, 1]^\numgoods$ applies. This, in turn guarantees the existence of a Walrasian equilibrium which is not $\zeros[\numgoods]$. 
% \end{remark}



% \begin{remark}[Continuity of excess demand]
%      In more stylized applications (see, for instance, \Cref{chap:fisher_markets} or \Cref{chap:arrow_debreu_economies}), the excess demand correspondence is in general defined so as to be guaranteed to be continuous only on the interior of the price space, i.e., $\interior(\R^\numgoods_+) = \R^\numgoods_{++}$, as the excess demand for a commodity can be infinite if the price of any commodity is 0. However, this issue in these stylized models only arises from a modeling choice which allows the demand of commodities to be possibly greater than the total amount of the commodity that can be ever supplied. As such, it is indeed possible to restrict the excess demand to bounded by the total amount of the commodity that can be ever supplied without modifying the Walrasian equilibria of the economy. This is indeed the approach that \citet{arrow-debreu} take in Section 3 of their paper for proving their seminal Walrasian equilibrium existence result, and it is also the approach we will take in \Cref{chap:arrow_debreu_economies} to prove convergence of price adjustment processes in Arrow-Debreu economies. This restriction is also realistic from an economic perspective since it is not possible for the economy to consume more of a commodity that there can exist, and resources in the real-world are indeed scarce. Indeed, otherwise there would be no use for the economic sciences: the science of resource allocation under scarcity.

%      While we will discuss this issue in greater lengths in \Cref{chap:arrow_debreu_economies}, for the interested reader, we note that even when the excess demand is restricted so as to be bounded, the excess demand can be still discontinuous at $\zeros[\numgoods]$. Nevertheless, as we will show, it is possible to get around this issue by designating one of the comodities with strictly positive equilibrium price as a num\'eraire commodity (i.e. a commodity which is used as unit of account) and setting its price to be strictly positive to get around this issue.
% \end{remark}

% \begin{remark}[Normalized Price Space]
%     With \Cref{thm:we_balanced_equal_svi} in hand, we will for balanced economies henceforth restrict the price space to be $[0, 1]^\numgoods_+$. This restriction will be useful as it will allow us to extend the guarantees of algorithms to compute a strong solution of VIs developed in \Cref{chap:vis} to the computation of a Walrasian equilibrium, as these guarantees depend on the diameter of the constraint space of the VI.
% \end{remark}
% \fi
% With the question of existence of a Walrasian equilibrium out of the way, we now turn our attention to the computation of Walrasian equilibrium.


