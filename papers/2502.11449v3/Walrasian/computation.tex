

We will for the rest of this chapter, assume that the excess demand correspondence is singleton-valued unless otherwise noted. 
We will consider (first-order) price-adjustment processes, as first defined by \citet{papadimitriou2010impossibility}.
Given a Walrasian economy $(\numgoods, \excessset)$, and an initial iterate $\price[][0] \in \R^\numgoods_+$, a \mydef{(first-order) price adjustment process} $\priceupdate$ consists of an update function which generates the sequence of iterates $\{\price[][\numhorizon]\}_{\numhorizon}$ given for all $\numhorizon = 0, 1, \hdots$ by
$\vartuple[][][\numhorizon + 1] \doteq  \priceupdate \left(\bigcup_{i = 0}^{\numhorizon} (\price[][i],  \excess(\price[][i]) \right)$.
  % 
As is standard in the literature (see, for instance, \citet{papadimitriou2010impossibility}) the computational complexity measures in this paper will take the number of evaluations of the excess demand $\excess$ as the unit of account.
% \begin{theorem}\label{thm:existence_we}
%     The set of Walrasian equilibria of any continuous Walrasian economy is non-empty.
% \end{theorem}
% \begin{proof}[Proof of \Cref{thm:existence_we}]
%     By \Cref{thm:we_equal_svi}, we know that the set of Walrasian equilibria $(\numgoods, \R^\numgoods_+, \goodset, \excessset)$ of any coercive economy is equal to the set of strong solutions $\svi(\R^\numgoods_+, -\excessset)$ of the VI $(\R^\numgoods_+, -\excessset)$.

%     Now, notice that for a continuous Walrasian economy $([0, 1]^\numgoods, -\excessset)$ is a continuous VI. Hence, by \Cref{thm:exist_strong_sol} a strong solution is guaranteed to exist, which in turn implies the existence of a Walrasian equilibrium.
% \end{proof}
