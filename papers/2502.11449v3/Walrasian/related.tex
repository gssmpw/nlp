\subsection{Historical Background and Related Works}

\mydef{Competitive (or Walrasian) equilibrium} \cite{arrow-debreu,walras}, first studied by French economist L\'eon Walras in 1874, is the steady state of an economy---any 
%\sdeni{market}{}\deni{What is a market?} 
system governed by supply and demand \cite{walras}.
Walras assumed that each producer in an economy would act so as to maximize its profit, while consumers would make decisions that maximize their preferences over their available consumption choices; all this, while perfect competition prevails, meaning producers and consumers are unable to influence the prices that emerge.
%for different commodities.
Under these assumptions, the demand and supply of each commodity is a function of prices, as they are a consequence of the decisions made by the producers and consumers, having observed the prevailing prices.
A competitive equilibrium then corresponds to prices that solve the system of simultaneous equations with demand on one side and supply on the other, i.e., prices at which supply meets demand.
%for all markets in the system
Unfortunately, Walras did not provide conditions that guarantee the existence of such a solution, and the question of whether such prices exist remained open until \citeauthor{arrow-debreu}'s rigorous analysis of competitive equilibrium in their model of a competitive economy in the middle of last century \cite{arrow-debreu}.

The Arrow-Debreu model comprises a set of commodities; a set of firms, each deciding what quantity of each commodity to supply; and a set of consumers, each choosing a quantity of each commodity to demand in exchange for their endowment \cite{arrow-debreu}. 
\citeauthor{arrow-debreu} define a \mydef{competitive equilibrium} as a collection of consumptions, one per consumer, a collection of productions, one per firm, and prices, one per commodity, such that fixing equilibrium prices: (1) no consumer can increase their utility by deviating to an alternative affordable consumption, (2) no firm can increase profit by deviating to another production in their production set, and (3) the \mydef{aggregate demand} for each commodity (i.e., the sum of the commodity's consumption across all consumers) does not exceed to its \mydef{aggregate supply} (i.e., the sum of the commodity's production and endowment across firms and consumers, respectively), while the total value of the aggregate demand is equal to the total value of the aggregate supply, i.e., \mydef{Walras' law} holds.

\citeauthor{arrow-debreu} proceeded to show that their competitive economy could be seen as an \mydef{abstract economy}, which today is better known as a \mydef{pseudo-game} \cite{arrow-debreu, facchinei2010generalized}.
A pseudo-game is a generalization of a game in which the actions taken by each player impact not only the other players' payoffs, as in games, but also their set of permissible actions.
\citeauthor{arrow-debreu} proposed \mydef{generalized Nash equilibrium} as the solution concept for this model, an action profile from which no player can improve their payoff by unilaterally deviating to another action in the space of permissible actions determined by the actions of other players.
\citeauthor{arrow-debreu} further showed that any competitive economy could be represented as a pseudo-game inhabited by a fictional auctioneer, who sets prices so as to buy and resell commodities at a profit, as well as consumers and producers, who respectively, choose utility-maximizing consumptions of commodities in the budget sets determined by the prices set by the auctioneer, and profit-maximizing productions at the prices set by the auctioneer.
The elegance of the reduction from competitive economies to pseudo-games is rooted in a simple observation: the set of competitive equilibria of a competitive economy is equal to the set of generalized Nash equilibria of the associated pseudo-game, implying the existence of competitive equilibrium in competitive economies as a corollary of the existence of generalized Nash equilibria in pseudo-games, whose proof is a straightforward generalization of Nash's proof for the existence of Nash equilibria \cite{nash1950existence}.%
\footnote{\citet{mckenzie1959existence} would prove the existence of competitive equilibrium independently, but concurrently.
Much of his work, however, has gone unrecognized perhaps because his proof technique does not depend on this fundamental relationship between competitive and abstract economies.}

With the question of existence out of the way, this line of work on competitive equilibrium, which today is known as general equilibrium theory \cite{mckenzie2005classical}, turned its attention to questions of (1) efficiency, (under what assumptions are competitive equilibria Pareto-optimal?) (2) uniqueness (under what assumptions are competitive equilibria unique?), and (3) stability (under what conditions would a competitive economy settle into a competitive equilibrium?).
The first two questions were answered between the 1950s and 1970s \cite{arrow1951extension, arrow1958note, arrow-hurwicz, balasko1975some, debreu1951pareto, dierker1982unique, hahn1958gross, pearce193unique}, showing that (1) under suitable assumptions (e.g., see \citet{arrow-welfare}) competitive equilibrium demands are Pareto-optimal, and (2) competitive equilibria are unique in markets with an \mydef{excess demand function}, (i.e., the difference between the aggregate demand and supply functions), which satisfies the \mydef{gross substitutes} (GS) condition (i.e., the excess demand of any commodity increases if the price of any other commodity increases, fixing all other prices).
In regards to the question of stability, most relevant work is concerned with the convergence properties of a natural auction-like price-adjustment process, known as \mydef{t\^atonnement}, which mimics the behavior of the \mydef{law of supply and demand}, updating prices at a rate equal to the excess demand \cite{arrow1971general, kaldor1934classificatory}.
Research on \emph{t\^atonnement\/} in the economics literature is motivated by the fact that it can be understood as a plausible explanation of how prices move in real-world markets.
Hence, if one could prove convergence in all exchange economies, then perhaps it would be justifiable to claim real-world markets would also eventually settle at a competitive equilibrium.

\citet{walras}
%, who introduced \emph{t\^atonnement\/} in 1874, 
conjectured, albeit without conclusive evidence, that \emph{t\^atonnement\/} would converge to a competitive equilibrium.
While a handful of results guarantee the convergence of \emph{t\^atonnement\/} under mathematical conditions without widely agreed-upon economic interpretations \cite{nikaido1960stability, uzawa1960walras}, \citeauthor{arrow-hurwicz} [1958; 1960]
%(see also \cite{arrow1960competitive}) 
were the first to formally establish the convergence of \emph{t\^atonnement} to unique competitive equilibrium prices in a class of economically well-motivated competitive economies, namely those that satisfy the GS assumption.
Following this promising result, \citet{scarf1960instable} dashed all hope that \emph{t\^atonnement\/} would prove to be a universal price-adjustment process that converges in all economies, by showing that competitive equilibrium prices are unstable under \emph{t\^atonnement\/} dynamics in his eponymous competitive economy without firms, and with only three commodities and three consumers with Leontief preferences, i.e., \mydef{the Scarf exchange economy}.
Scarf's negative result seems to have discouraged further research by economists on the stability of competitive equilibrium \cite{fisher1975stability}.
Despite research on this question coming to a near halt, one positive outcome was achieved, on the convergence of a non-\emph{t\^atonnement\/} update rule known as \mydef{Smale's process} \cite{herings1997globally, kamiya1990globally, van1987convergent, smale1976convergent}, which updates prices at the rate of the product of the excess demand and the inverse of its Jacobian, in most competitive economies, even beyond GS, again suggesting the possibility that real-world economies could indeed settle at a competitive equilibrium. 

Nearly half a century after these seminal analyses of competitive economies, research on the stability of competitive equilibrium is once again coming to the fore, this time in computer science, perhaps motivated by applications of algorithms such as \emph{t\^atonnement\/} to load balancing over networks \cite{jain2013constrained}, or to pricing of transactions on crypotocurrency blockchains \cite{leonardos2021dynamical, liu2022empirical, reijsbergen2021transaction}.
% \deni{Does the rest of this paragraph feel out of place? Should we move to the related work?}
A detailed inquiry into the computational properties of market equilibria was initiated by \citet{devanur2008market}, who studied a special case of the Arrow-Debreu competitive economy known as the \mydef{Fisher market} \cite{brainard2000compute}.
This model, for which Irving Fisher computed equilibrium prices using a hydraulic machine in the 1890s, is essentially the Arrow-Debreu model of a competitive economy, but there are no firms, and buyers are endowed with only one type of commodity---hereafter good%
\footnote{In the context of Fisher markets, commodities are typically referred to as goods \citep{fisher-tatonnement}, as Fisher markets are often analyzed for a single time period only.
More generally, in Arrow-Debreu markets, where commodities vary by time, location, or state of the world, "an apple today" may be different than "an apple tomorrow". For consistency with the literature, we refer to commodities as goods. }---an artificial currency 
%\samy{but we note that}{because this distinction is unnecessary \samy{}{in a single time-period model}.\deni{I feel like this edit is too strong. In particular, a Fisher market is still an arrow-debreu market so you could think of goods as time and space stamped commodities but computer scientists just chose not to because their applications do often not have time and space stamped commodities. The point of the footnote is to say that computer scientists are not being precise but we stick to this terminology for consistency.}}
\cite{brainard2000compute, AGT-book}.
\citet{devanur2002market} exploited a connection first made by \citet{eisenberg1961aggregation} between the \mydef{Eisenberg-Gale program} and competitive equilibrium to solve Fisher markets assuming buyers with linear utility functions, thereby providing a (centralized) polynomial-time algorithm for equilibrium computation in these markets~\cite{devanur2002market,devanur2008market}.
Their work was built upon by \citet{jain2005market}, who extended the Eisenberg-Gale program to all Fisher markets in which buyers have \mydef{continuous, quasi-concave, and homogeneous} utility functions, and proved that the equilibrium of Fisher markets with such buyers can be computed in polynomial time by interior point methods. 
% Hereinafter, as is standard in the literature (see e.g., \cite{jain2005market}. functions are cont

Concurrent with this line of work on computing competitive equilibrium using centralized methods, a line of work on devising and proving 
% \samy{polynomial-time}{} \amy{maybe you don't want to stress comp'l efficiency until the next paragraph?}
convergence guarantees for price-adjustment processes (i.e., iterative algorithms that update prices according to a predetermined update rule) developed.
% \amy{why is this iterative process decentralized? the description doesn't imply decentralization.} \deni{These price adjustment processes are decentralized in the sense that the adjustment of the price of one good does not depend on the demand or supply of the other goods.} \amy{so it sounds more parallel than decentralized?}
This literature has focused on devising \emph{natural\/} price-adjustment processes, like \emph{t\^atonnement}, which might explain or imitate the movement of prices in real-world markets.
In addition to imitating the law of supply and demand, \emph{t\^atonnement} has been observed to replicate the movement of prices in lab experiments, where participants are given endowments and asked to trade with one another \cite{gillen2020divergence}.
%Beyond interest in understanding the convergence of natural price-adjustment processes for the aforementioned applications,
% \amy{this next sentence does not follow from the lab experiments before it:} 
Perhaps more importantly, the main premise of research on the stability of competitive equilibrium in computer science 
% \amy{this literature refers to the lab expts literature, and that is NOT the main premise of the experimental/behavioral literature!}
is that for competitive equilibrium to be justified, not only should it be backed by a natural price-adjustment process as economists have long argued, but it should also be computationally efficient \cite{AGT-book}.
% As Kamal Jain put it, ``If your laptop cannot find it, neither can the market'' \cite{AGT-book}.

The first result on this question is due to \citet{codenotti2005market}, who introduced a discrete-time version of \emph{t\^atonnement}, 
%proving an analog of \citeauthor{arrow-hurwicz}'s results in discrete time from a computational complexity perspective.
and showed that in exchange economies that satisfy \mydef{weak gross substitutes (WGS)}, the \emph{t\^atonnement\/} process converges to an approximate competitive equilibrium in a number of steps which is polynomial in the approximation factor and size of the problem.
Unfortunately, soon after this positive result appeared, \citet{papadimitriou2010impossibility} showed that it is impossible for a price-adjustment process based on the excess demand function to converge in polynomial time to a competitive equilibrium in general, ruling out the possibility of Smale's process (and many others)
%or any other excess-demand-based price adjustment 
justifying the notion of competitive equilibrium in all competitive economies.
Nevertheless, further study of the convergence of price-adjustment processes such as \emph{t\^atonnement\/} under stronger assumptions, or in simpler models than full-blown Arrow-Debreu competitive economies, remains worthwhile, as these processes are being deployed in practice \cite{jain2013constrained, leonardos2021dynamical, liu2022empirical, reijsbergen2021transaction}.


Following \citeauthor{codenotti2005market}'s [\citeyear{codenotti2005market}] initial analysis of \emph{t\^atonnement\/} in competitive economies that satisfy WGS, \citet{garg2004auction} introduced an auction algorithm that also converges in polynomial time for linear exchange economies.
More recently, \citet{bei2015tatonnement} established faster convergence bounds for \emph{t\^atonnement\/} in WGS exchange economies.

Another line of work considers price-adjustment processes in variants of Fisher markets.
\citet{cole2008fast} analyzed \emph{t\^atonnement\/} in a real-world-like model satisfying WGS called the ongoing market model.
In this model, \emph{t\^atonnement\/} once-again converges in polynomial-time \cite{cole2008fast, cole2010discrete}, and it has the advantage that it can be seen as an abstraction for market processes.
% \amy{is the reader supposed to know what in-market processes are?}\deni{it's not a really well-defined term, it just means that it is an abstraction of behavior that might occur in real world markets, such as companies storing under sold goods in storage facilities.}\amy{so how about we cross out ``in''? b/c i could imagine what market processes are. but in-market processes sounds like something technical.}
\citeauthor{cole2008fast}'s results were later extended by \citet{cheung2012tatonnement} to ongoing markets with \mydef{weak gross complements}, i.e., the excess demand of any commodity weakly increases if the price of any other commodity weakly decreases, fixing all other prices, and ongoing markets with a mix of WGC and WGS commodities.
The ongoing market model these two papers study contains as a special case the Fisher market; however \citet{cole2008fast} assume bounded own-price elasticity of Marshallian demand, and bounded income elasticity of Marshallian demand, while \citet{cheung2012tatonnement} assume, in addition to \citeauthor{cole2008fast}'s assumptions, bounded adversarial market elasticity, which can be seen as a variant of bounded cross-price elasticity of Marshallian demand, from below.
With these assumptions, these results cover Fisher markets with a small range of the well-known CES utilities, including CES Fisher markets with $\rho \in [0, 1)$ and WGC Fisher markets with $\rho \in (- 1, 0]$.%
\footnote{We refer the reader to \Cref{sec:prelim} for a definition of CES utilities in terms of the substitution parameter $\rho$.}

\citet{fisher-tatonnement} built on this work by establishing the convergence of \emph{t\^atonnement\/} in polynomial time in nested CES Fisher markets, excluding the limiting cases of linear and Leontief markets, but nonetheless extending polynomial-time convergence guarantees for \emph{t\^atonnement\/} to Leontief Fisher markets as well.
More recently, \citet{cheung2018amortized} showed that \citeauthor{fisher-tatonnement}'s [\citeyear{fisher-tatonnement}] result extends to an asynchronous version of \emph{t\^atonnement}, in which good prices are updated during different time periods. 
In a similar vein, \citet{cheung2019tracing} analyzed \emph{t\^atonnement\/} in online Fisher markets, determining that \emph{t\^atonnement\/} tracks competitive equilibrium prices closely provided the market changes slowly.

Another price-adjustment process that has been shown to converge to market equilibria in Fisher markets is \mydef{proportional response dynamics}, first introduced by \citet{first-prop-response} for linear utilities; then expanded upon and shown to converge by \cite{proportional-response} for all CES utilities; and very recently shown to converge in Arrow-Debreu exchange economies with linear and CES ($\rho \in [0,1)$) utilities by \citeauthor{branzei2021proportional}. 
The study of the proportional response process was proven fundamental when \citeauthor{fisher-tatonnement} noticed its relationship to gradient descent.
This discovery opened up a new realm of possibilities in analyzing the convergence of market equilibrium processes.
For example, it allowed \citet{cheung2018dynamics} to generalize the convergence results of proportional response dynamics to Fisher markets for buyers with mixed CES utilities.
This same idea was applied by \citet{fisher-tatonnement} to prove the convergence of \emph{t\^atonnement\/} in Leontief Fisher markets, using the equivalence between mirror descent \cite{boyd2004convex}
on the dual of the Eisenberg-Gale program 
%\sdeni{}{(without explicitly constructing the dual)} %\deni{Are we sure we need a caveat here?} \amy{we are taking the position that no one knew the dual before this paper (although we are not making a big deal about it), so how did they prove this equivalence?} 
and \emph{t\^atonnement}, first observed by \citet{devanur2008market}.
%\amy{didn't one of the reviews say that Devanur was not the first to observe this?}\deni{No, they said that Devanur was the first to observe this.}
% \amy{comment from reviewer: Page 2: Devanur et al. [31] discovered ..." -> It seems this connection was known earlier, e.g., see Eisenberg (1961).}
More recently, \citet{gao2020first} developed 
%first-order 
methods to solve the Eisenberg-Gale convex program in the case of linear, quasi-linear, and Leontief Fisher markets.

An alternative to the (global) competitive economy model, in which an agent's trading partners are unconstrained, is the \citet{kakade2004graphical} model of a graphical economies.
This model features local markets, in which each agent can set its own prices for purchase only by neighboring agents, and likewise can purchase only from neighboring agents. 
Auction-like price-adjustment processes have been shown to converge in variants of this model assuming WGS \cite{andrade2021graphical}.
