Walrasian economies (or general equilibrium models), first studied by French economist L\'eon Walras in 1874, are a broad mathematical framework for modeling any economic system governed by supply and demand \cite{walras}. A Walrasian economy consists of a finite set of commodities, characterized by an excess demand function that maps values for commodities, called \mydef{prices}, to positive (resp. negative) quantities of each commodity demanded (resp. supplied) in excess. Walras proposed a steady-state solution of his economy, namely a Walrasian (or competitive) equilibrium, represented by a collection of per-commodity prices which is \mydef{feasible}, i.e., there is no excess demand for any commodity, and for which \mydef{Walras' law} holds, i.e., the value of the excess demand is equal to 0. 

Walras did not establish conditions ensuring the existence of an equilibrium, leaving the question unresolved until the 1950s \cite{arrow-debreu}, but argued, albeit without conclusive evidence, that his economy would settle at a Walrasian equilibrium via a \mydef{price-adjustment process} (i.e., any process that generates a sequence of prices based on prior prices and associated excess demands), known as \mydef{\emph{t\^atonnement}}, which mimics the behavior of the \mydef{law of supply and demand}, updating prices at a rate equal to the excess demand \cite{walras, uzawa1960walras, arrow-hurwicz}. To motivate the relevance of \emph{t\^atonnement} to real-world economies, Walras argued that \emph{t\^atonnement} is a \mydef{natural price-adjustment process}, in the sense that if each each commodity is owned by a different seller, then each seller can update the price of its commodity without coordinating with other sellers, using only information about the excess demand of its commodity, hence making it plausible that \emph{t\^atonnement} could explain the movement of prices in real-world economies where sellers again do not coordinate with one another. 

Nearly half a century after Walras' initial foray into general equilibrium analysis, a group of academics brought together by the Cowles Commission in 1939 reinitiated a study of Walras' economic model with the purpose of bringing rigorous mathematics to the analysis of markets. One of the earliest and most important outputs of this collaborative effort was the introduction of a broad and well-justified class of Walrasian economies known as \mydef{competitive economies} \cite{arrow-debreu}, for which the existence of Walrasian equilibrium was established by a novel application of fixed point theorems to economics. With the question of existence thus resolved, the field subsequently turned its focus to investigating questions on the \mydef{stability} of Walrasian equilibrium
i.e., which price-adjustment processes can settle at a Walrasian equilibrium and under what assumptions? \cite{uzawa1960walras,balasko1975some,arrow-hurwicz, cole2008fast, cheung2018dynamics,fisher-tatonnement, jain2005market, codenotti2005market, codenotti2006leontief, chen2009spending}.

Most relevant work on stability has been concerned with the convergence properties of \emph{t\^atonnement}.
Beyond Walras' justification for \emph{t\^atonnement}'s relevance to real-world economies, research on \emph{t\^atonnement} in the post-world war II economics literature is motivated by the fact that it can be understood as a plausible explanation of how prices move in real-world markets \cite{gillen2020divergence}. Hence, if one could prove that \emph{t\^atonnement\/} is a \mydef{universal price-adjustment process} (i.e., a price-adjustment process that converges to a Walrasian equilibrium in all competitive economies), then perhaps it would be justifiable to claim real-world economies would also eventually settle at a Walrasian equilibrium.

In 1958, \citet{arrow-hurwicz} 
%were among the first to address questions of stability. They 
established the convergence of a continuous-time variant of \emph{t\^atonnement\/} in Walrasian economies with an excess demand function satisfying the weak axiom of revealed preferences (WARP) \cite{afriat1967construction}, which among others, includes Walrasian economies satisfying the GS condition \cite{arrow1959stability, arrow1960competitive}. This result was complemented by \citeauthor{nikaido1960stability}'s \cite{nikaido1960stability} result on the convergence of a discrete-time variant of \emph{t\^atonnement\/} in Walrasian economies satisfying WARP---albeit without any non-asymptotic convergence guarantees. These initial results sparked hopes that \emph{t\^atonnement\/} could be a universal price-adjustment process.

Furthermore, as there in general exists no closed-form formulas for Walrasian equilibria, these results ignited further interest in discovering algorithms to compute a Walrasian equilibrium, as \emph{t\^atonnement} could be implemented on a computer to obtain numerical approximations of Walrasian equilibria in Walrasian economies.
Indeed, these early results on the stability of \emph{t\^atonnement} inspired a new line of work on \mydef{applied general equilibrium} \cite{scarf1967computation, scarf1967approximation, scarf1973book, scarf1982computation} initiated by Herbert Scarf \cite{scarf-eaves}, whose goal was to establish ``a general method for the explicit numerical solution of the neoclassical [Walrasian economy] model'' \cite{scarf1973book}. The motivation behind this research agenda was a desire to predict the impact of economic policy on an economy by estimating the parameters of a parametric Walrasian economy from empirical data, and then running a comparative static analysis to compare the numerical solution of the Walrasian economy before and after the implementation of the policy.

Unfortunately, soon after initiating this research agenda, Scarf dashed all hopes that \emph{t\^atonnement\/} could be a universal price-adjustment process by showing that the sequence of prices generated by a continuous-time variant of \emph{t\^atonnement\/} can cycle ad infinitum around the Walrasian equilibrium of his eponymous competitive economy, with only three commodities and an excess demand function generated by three consumers with Leontief preferences, i.e., \mydef{the Scarf economy} \cite{scarf1960instable}. Even more discouragingly, when applied to the Scarf economy, the prices generated by discrete-time variants of \emph{t\^atonnement\/} spiral away from the Walrasian equilibrium, moving further and further away from equilibrium.

Scarf's negative result seems to have discouraged further research by economists on the stability of Walrasian equilibrium \cite{fisher1975stability}.
Despite research on this question coming to a near halt, one positive outcome was achieved, on the convergence of a non-\emph{t\^atonnement\/} update rule known as \mydef{Smale's process} \cite{herings1997globally, kamiya1990globally, van1987convergent, smale1976convergent}, which updates prices at the rate of the product of the excess demand and the inverse of its Jacobian, to a Walrasian equilibrium in competitive economies which have an excess demand that has a non-singular Jacobian, including Scarf economies. Unfortunately, this convergence result for Smale's process comes with two caveats: 1) Smale's process is not a ``natural" price-adjustment process, as it updates the price of each commodity using information about not only the excess demand of the commodity but also
the derivative of the excess demand function with respect to each commodity in the economy, 2) convergence of discrete time-variants of Smale's process require the excess demand to satisfy the law of supply and demand, which even Walrasian economies that satisfy the GS or WARP conditions do not satisfy. 

Nearly half a century after these seminal analyses of competitive economies, research on the stability and efficient computation of Walrasian equilibrium is once again coming to the fore, motivated by applications of algorithms to compute Walrasian equilibrium in dynamic stochastic general equilibrium models in macroeconomics \cite{geanakoplos1990introduction,sargent2000recursive,taylor1999handbook,FernandezVillaverde2023CompMethodsMacro}, and the use of algorithms such as \emph{t\^atonnement\/} to solve models of transactions on crypotocurrency blockchains \cite{leonardos2021dynamical, liu2022empirical, reijsbergen2021transaction} and load balancing over networks \cite{jain2013constrained}.
In contrast to the prior literature on the stability of \emph{t\^atonnement}, which was primarily concerned with proving asymptotic convergence of price-adjustment processes to a Walrasian equilibrium, this line of work is also concerned with obtaining non-asymptotic convergence rates, and hence computing approximate Walrasian equilibria in polynomial time.

The first result on this question is due to \citet{codenotti2005market}, who introduced a discrete-time version of \emph{t\^atonnement}, and showed that in exchange economies that satisfy \mydef{weak gross substitutes (WGS)} (i.e., the excess demand of any commodity \emph{weakly} increases if the price of any other commodity increases, fixing all other prices), the \emph{t\^atonnement\/} process converges to an approximate Walrasian equilibrium in a number of steps which is polynomial in the inverse of the approximation factor and size of the problem.
Unfortunately, soon after this positive result appeared, \citet{papadimitriou2010impossibility} showed that it is impossible for a price-adjustment process based on the excess demand function to converge in polynomial time to a Walrasian equilibrium in general, ruling out the possibility of Smale's process (and many others), justifying the notion of Walrasian equilibrium in all competitive economies.
Nevertheless, further study of the convergence of price-adjustment processes such as \emph{t\^atonnement\/} under stronger assumptions, or in simpler models than full-blown Arrow-Debreu competitive economies, continues, as these processes are being deployed in practice \cite{jain2013constrained, leonardos2021dynamical, liu2022empirical, reijsbergen2021transaction}.%
\footnote{We refer the reader to \Cref{sec_app:related} for additional related works on algorithms for Walrasian Economies and VIs.}
