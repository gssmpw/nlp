\paragraph{Computational Resources}
Our experiments were run on MacOS machine with 8GB RAM and an Apple M1 chip, and took about 10 minutes to run. Only CPU resources were used.

\paragraph{Programming Languages, Packages, and Licensing}
We ran our experiments in Python 3.7 \cite{van1995python}, using NumPy \cite{numpy},  Jax \cite{jax2018github}, and  JaxOPT \cite{jaxopt_implicit_diff}.
All figures were graphed using Matplotlib \cite{matplotlib}. 

Python software and documentation are licensed under the PSF License Agreement. Numpy is distributed under a liberal BSD license. Pandas is distributed under a new BSD license. Matplotlib only uses BSD compatible code, and its license is based on the PSF license. 


\paragraph{Experimental Setup Details}


Each economy is initialized using a random seed to ensure reproducibility. Each consumer is assigned an initial endowment, drawn from a uniform distribution:
$
\consendow[][][][\prime] \sim \mathrm{Unif}(10^{-6}, 1), \quad \forall \consumer \in [\numconsumers], \good \in [\numgoods].
$
For numerical stability, we restrict the total economy-wide aggregate supply of each commodity to remain fixed at $10$\footnote{This is without loss of generality since commodities are divisible.}, to this end we normalize the endowments of consumers for all $\good \in \goods$,  $\consumer \in \consumers$ to obtain their final endowment:
\[
\consendow[\consumer][\good] \doteq \frac{10 \consendow[\consumer][\good][][\prime]}{\sum_{\consumer \in \consumers} \consendow[\consumer][\good][][\prime]}.
\]

Each consumer’s valuation of each commodity is drawn from a uniform distribution, i.e., for all $\good \in \goods$, $\consumer \in \consumers$:
\[
\valuation[\consumer][\good] \sim \mathrm{Unif}(0, 1) .
\]

For any CES consumer $\consumer \in \consumers$, the elasticity of substitution parameter $\rho_\consumer$, is drawn as follows from the uniform distribution for substitutes and complements consumers respectively:
\begin{align*}
&\rho_\consumer^{\text{substitutes}} \sim \mathrm{Unif}(0.6, 0.9) \
&\rho_\consumer^{\text{complements}} &\sim \mathrm{Unif}(-1000, -1)
\end{align*}



The initial price vector $\price[][0]$ for the algorithms is drawn from a uniform distribution s.t. for all $\good \in \goods$:
\[
\price[\good][0] \sim \mathrm{Unif}(1, 10) .
\]
We note that while we initialize the prices between $1$ and $10$ for numerical stability, this choice is without loss of generality since the excess demand is homogeneous of degree $0$.

To summarize. Given a random seed, the initialization process consists of:
1) Sampling endowments from a uniform distribution and normalizing them to ensure total supply constraints; 2) sampling valuations from a uniform distribution; 3) sampling substitution parameters for CES consumers, 4) generating an initial price vector.