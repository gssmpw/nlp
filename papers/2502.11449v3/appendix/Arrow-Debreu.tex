An \mydef{Arrow-Debreu economy} $(\numbuyers, \numcommods, \consumptions, \consendow, \util)$, denoted $(\consumptions, \consendow, \util)$ when clear from context, comprises a finite set of $\numcommods \in \N_+$ divisible \mydef{commodities} and $\numconsumers \in \N_+$ \mydef{consumers}.
Each consumer $\consumer \in \consumers$ is characterized by a \mydef{set of consumptions} $\consumptions[\consumer] \subseteq \R^{\numcommods}$, an \mydef{endowment} of commodities $\consendow[\consumer] = \left(\consendow[\consumer][1], \dots, \consendow[\consumer][\numcommods] \right) \in \R^\numconsumers$, and a \mydef{utility function} $\util[\consumer]: \R^{\numcommods} \to \mathbb{R}$ which for any \mydef{consumption} $\consumption[\consumer] \in \consumptions[\consumer]$ describes the utility $\util[\consumer](\consumption[\consumer])$ consumer $\consumer$ derives.\footnote{In line with the literature (see, for instance, \cite{debreu1954representation}), the  value of this utility function should not be interpreted to have any meaning, and the utility function $\util[\consumer]$ should be understood to represent a preference relation $\prefer[\consumer]$ on the space of consumptions $\consumptions[\consumer]$ so that for any two consumptions $\consumption[\consumer], \consumption[\consumer][][][\prime] \in \consumptions$, $\util[\consumer](\consumption[\consumer]) \geq \util[\consumer](\consumption[\consumer][][][\prime]) \implies \consumption[\consumer] \prefer[\consumer] \consumption[\consumer][][][\prime]$. }
We define any collection of per-consumer consumptions $\consumption \doteq (\consumption[1], \hdots, \consumption[\numconsumers]) \in \consumptions$ a \mydef{consumption profile}, where $\consumptions \doteq \bigtimes_{\consumer \in \consumers} \consumptions[\consumer]$ is the \mydef{set of consumption profiles}, and any collection of per-consumer endowments an \mydef{endowment profile} $\consendow \doteq \left(\consendow[1], \hdots, \consendow[\numbuyers] \right) \in \R^{\numconsumers \numcommods}$. 

\begin{remark}
    For ease of exposition, without loss of generality, we restrict ourselves to Arrow-Debreu exchange economies and opt to not present Arrow-Debreu competitive economies (see \citet{arrow-debreu}) which in addition to consumers also contain firms. Nevertheless, our focus on Arrow-Debreu exchange economies is without loss of generality since any firm can be represented as a consumer in an Arrow-Debreu exchange economy by adding an additional commodity into the economy which represents ownership of the firm, setting the consumption space of the new consumer to be equal to the production space of the firm, and its utility function so that it seeks to maximize its consumption of the commodity associated with the firm's ownership. The commodity associated with ownership of the firm should further appear in the endowments of consumers that are supposed to have a contractual claim over the profits of the firms. A similar, albeit much more complicated reduction than described here was proposed earlier by \citet{garg2015markets} which refer the reader to for additional details.
\end{remark}


\begin{definition}
    A \mydef{Arrow-Debreu equilibrium} $(\consumption[][][][*], \price[][][*])$ is a tuple comprising consumptions $\consumption[][][][*] \in \R_+^{\numconsumers \times \numbuyers}$ and prices $\price[][][*] \in \simplex[\numcommods]$ s.t.
        \begin{enumerate}
        % \item[{(Homogeneity of degree $0$)}] For all $\lambda >0$, $\excessset(\lambda \price) = \excessset(\price)$
        \item (Utility maximization)  all consumers $\consumer \in \consumers$, maximize their utility constrained by the value of their endowment: $ \max\limits_{\consumption[\consumer] \in \consumptions[\consumer]: \consumption[\consumer] \cdot \price[][][*] \leq \consendow[\consumer] \cdot \price[][][*]} \util[\consumer](\consumption[\consumer]) \leq \util[\consumer](\consumption[\consumer][][][*]) $;
        \item (Feasibility) the consumptions are feasible, i.e., $\sum_{\consumer \in \consumers} \consumption[\consumer][][][*]  \leq \sum_{\consumer \in \consumers} \consendow[\consumer]$
        \item (Walras' law) the value of the demand and the supply are equal, i.e., $\price[][][*] \cdot \left( \sum_{\consumer \in \consumers} \consumption[\consumer][][][*]  - \sum_{\consumer \in \consumers} \consendow[\consumer] \right) = 0$
    \end{enumerate}
\end{definition}

\begin{assumption}\label{assum:ad_economy}
    Any Arrow-Debreu economy $(\consumptions, \consendow, \util)$ satisfies the following conditions for all consumers $\consumer \in \consumers$:
    \begin{enumerate}
        \item (Closed consumption set) $\consumptions[\consumer]$ is non-empty, bounded from below, closed, and convex 
        \item (Feasible budget set)  There exists a consumption that is strictly less than the consumer's endowment, i.e., for all $\consumer \in \consumers$,  $\exists \consumption[\consumer]  \in \consumptions[\consumer], \quad   \consumption[\consumer] < \consendow[\consumer ]$
        \item (Continuity) $\util[\consumer]$ is continuous
        \item (Quasiconcavity) $\util[\consumer]$ is quasi-concave, i.e., forall $\consumption[\consumer], \consumption[\consumer][][][\prime] \in \R^\numcommods, \lambda \in(0,1), \ \  \util[\consumer](\lambda \consumption[\consumer] + (1-\lambda) \consumption[\consumer][][][\prime]) \geq \min \left\{\util[\consumer](\consumption[\consumer]), \util[\consumer](\consumption[\consumer][][][\prime]) \right\}$ , 
        \item (Non-satiation) $\util[\consumer]$ is non-satiated, i.e.,  $\forall \consumption[\consumer] \in \consumptions[\consumer]$, there exists $\consumption[\consumer][][][\prime] \in \consumptions[\consumer]$ s.t. $\util[\consumer](\consumption[\consumer][][][\prime]) > \util[\consumer](\consumption[\consumer][][][\prime])$ 
    \end{enumerate}
\end{assumption}


\begin{definition}[Walrasian Arrow-Debreu Competitive Economy]
    Given an Arrow-Debreu economy $(\numconsumers, \numcommods, \consumptions, \consendow, \util)$, the \mydef{Walrasian Arrow-Debreu competitive economy} $(\numcommods, \excessset)$ is a Walrasian economy with the excess demand correspondence given as:
    \begin{align*}
        \excessset(\price) &= \sum_{\player \in \players} \left[\argmax\limits_{\consumption[\consumer] \in \consumptions[\consumer][\prime]: \consumption[\consumer] \cdot \price \leq \consendow[\consumer] \cdot \price}  \util[\consumer](\consumption[\consumer]) \right] - \sum_{\consumer \in \consumers} \consendow[\consumer] \enspace ,
    \end{align*}
    where $\consumptions[\consumer][\prime] \doteq \left\{\consumption[\consumer] \mid \sum_{k \in \consumers} \consumption[k] \leq \sum_{k \in \consumers} \consendow[k], \forall k \in \consumers, \consumption[{k}] \in \consumptions[{k}] \right\}$.
\end{definition}


From the proof of Theorem 1 of \citet{arrow-debreu}, we can infer that any Walrasian equilibrium $\price[][][*] \in \simplex[\numgoods]$ of the  Walrasian Arrow-Debreu competitive economy $(\numcommods, \excessset)$ is an Arrow-Debreu equilibrium price of $(\numconsumers, \numcommods, \consumptions, \consendow, \util)$. Further, as shown in the following lemma, it is straightforward to verify that the Walrasian economy $(\numcommods, \excessset)$, as the name suggests, gives rise to a Walrasian Arrow-Debreu economy. 

\begin{lemma}[Arrow-Debreu Economies are Walrasian competitive Economies]\label{lemma:ad_economies_are_comp_bounded}
    Consider the Walrasian Arrow-Debreu competitive economy $(\numcommods, \excessset)$ associated with the Arrow-Debreu economy $(\numconsumers, \numcommods, \consumptions, \consendow, \util)$. Then, $\excessset$ satisfies the following:
    \begin{enumerate}
        % \item[{(Homogeneity of degree $0$)}] For all $\lambda >0$, $\excessset(\lambda \price) = \excessset(\price)$
        \item (Homogeneity of degree $0$) For all $\lambda >0$, $\excessset(\lambda \price) = \excessset(\price)$
        \item (Weak Walras' law) For all $\price \in \R^\numgoods_+$ and $\excess(\price) \in \excessset(\price)$,  $\price \cdot \excess(\price) \leq 0$
        \item (Non-Satiation) for all $\price \in \R^\numgoods_+$, and $\excess(\price) \in \excessset(\price)$, $\excess(\price) \leq \zeros[\numgoods] \implies \price \cdot \excess(\price) = 0$
        \item (Continuity) The excess demand correspondence $\excessset$ is upper hemicontinuous on $\simplex[\numgoods]$, non-empty-, compact-, and convex-valued.
        \item (Boundedness) For all $\price \in \R^\numgoods_+$, and $\excess(\price) \in \excessset(\price)$, $\| \excess(\price)\|_\infty < \infty$
    \end{enumerate}
    
    That is, the Walrasian Arrow-Debreu competitive economy $(\numcommods, \excessset)$ associated with the Arrow-Debreu economy $(\numconsumers, \numcommods, \consumptions, \consendow, \util)$, is a continuous competitive economy which is bounded.
    
\end{lemma}

\begin{proof}[Proof of \Cref{lemma:ad_economies_are_comp_bounded}]

    \textbf{Homogeneity.}
    For all $\lambda >0$, we have:
    \begin{align*}
        \excessset(\lambda \price) &= \sum_{\player \in \players} \left[\argmax\limits_{\consumption[\consumer] \in \consumptions[\consumer][\prime]: \consumption[\consumer] \cdot (\lambda \price) \leq \consendow[\consumer] \cdot (\lambda \price)}  \util[\consumer](\consumption[\consumer]) \right] - \sum_{\consumer \in \consumers} \consendow[\consumer]\\
        &= \sum_{\player \in \players} \left[\argmax\limits_{\consumption[\consumer] \in \consumptions[\consumer][\prime]: \lambda \consumption[\consumer] \cdot \price \leq \lambda \consendow[\consumer] \cdot  \price}  \util[\consumer](\consumption[\consumer]) \right] - \sum_{\consumer \in \consumers} \consendow[\consumer]\\
        &= \sum_{\player \in \players} \left[\argmax\limits_{\consumption[\consumer] \in \consumptions[\consumer][\prime]: \consumption[\consumer] \cdot \price \leq \consendow[\consumer] \cdot  \price}  \util[\consumer](\consumption[\consumer]) \right] - \sum_{\consumer \in \consumers} \consendow[\consumer] = \excessset(\price)
    \end{align*}

    \textbf{Walras' law.}
    Fix any $\price \in \R^\numcommods_+$, and let for all consumers $\consumer \in \consumers$, $\consumption[\consumer][][][*] \in \argmax\limits_{\consumption[\consumer] \in \consumptions[\consumer][\prime]: \consumption[\consumer] \cdot \price \leq \consendow[\consumer] \cdot  \price}  \util[\consumer](\consumption[\consumer])$. Then, we have:
    \begin{align*}
        \consumption[\consumer][][][*] \cdot \price  \leq \consendow[\consumer][][][]  \cdot \price 
    \end{align*}
    Summing up across all consumers, and re-organizing, we have:
    \begin{align*}
        \price \cdot \left( \sum_{\consumer \in \consumers} \consumption[\consumer][][][*]   - \sum_{\consumer \in \consumers} \consendow[\consumer][][][]  \right)  \leq 0
    \end{align*}
    Hence, we have for all $\price \in \R^\numgoods_+$ and $\excess(\price) \in \excessset(\price)$,  $\price \cdot \excess(\price) \leq 0$.
    
    \textbf{Non-Satiation}
    Fix any $\price \in \simplex[\numcommods]$, and let for all consumers $\consumer \in \consumers$, $\consumption[\consumer][][][*] \in \argmax\limits_{\consumption[\consumer] \in \consumptions[\consumer][\prime]: \consumption[\consumer] \cdot \price \leq \consendow[\consumer] \cdot  \price}  \util[\consumer](\consumption[\consumer])$. Suppose by contradiction that $\excess(\price) \leq \zeros[\numgoods]$ but there exists some consumer $\consumer \in \consumers$ s.t.:
    \begin{align*}
        \consumption[\consumer][][][*] \cdot \price  < \consendow[\consumer][][][]  \cdot \price 
    \end{align*}
     
    
    Now, by non-satiation, there exists $\consumption[\consumer][][][\prime] \in \consumptions[\consumer]$ s.t. $\util[\consumer](\consumption[\consumer][][][\prime]) > \util[\consumer](\consumption[\consumer][][][*])$. As a result, there must also exist $\lambda \in (0, 1)$ s.t. for the consumption $\consumption[\consumer][][][\dagger] \doteq \lambda \consumption[\consumer][][][\prime] + (1- \lambda) \consumption[\consumer][][][*]$, we have: 
    \begin{enumerate}
        \item $\consumption[\consumer][][][\dagger] \in \consumptions[\consumer][\prime]$ since $\consumption[\consumer][][][*] \in \interior(\consumptions[\buyer][\prime])$;
        \item $\util[\consumer](\consumption[\consumer][][][\dagger]) > \util[\consumer](\consumption[\consumer][][][*])$ since $\util[\consumer]$ is quasiconcave;
        \item $\consumption[\consumer][][][\dagger] \cdot \price  \leq \consendow[\consumer][][][]  \cdot \price$ since the function $\consumption[\consumer] \mapsto \consumption[\consumer] \cdot \price$ is continuous.
    \end{enumerate}
    However, this is a contradiction since $\consumption[\consumer][][][*] \in \argmax\limits_{\consumption[\consumer] \in \consumptions[\consumer][\prime]: \consumption[\consumer] \cdot \price \leq \consendow[\consumer] \cdot \price} \util[\consumer](\consumption[\consumer])$.
    
    Hence, for all consumers $\consumer \in \consumers$ we must have:
    \begin{align}
        \consumption[\consumer][][][*] \cdot \price  = \consendow[\consumer][][][*]  \cdot \price
    \end{align}

    Summing the above across $\consumer \in \consumers$, and re-organizing the expression, we have for all $\excess(\price) \in \excessset(\price)$:
    \begin{align*}
        0 = \price[][][*] \cdot \left( \sum_{\consumer \in \consumers} \consumption[\consumer][][][*]  - \sum_{\consumer \in \consumers} \consendow[\consumer] \right) = \price[][][*] \cdot \excess(\price)
    \end{align*}

    \textbf{Continuity.}

    Since $\consumptions[][\prime]$ is non-empty, compact, and convex, and for all consumers $\consumer \in \consumers$, $\util[\consumer]$ is continuous and quasiconcave, and $\exists \consumption[\consumer]  \in \consumptions[\consumer], \quad   \consumption[\consumer] < \consendow[\consumer ]$ the assumptions of Berge's maximum theorem \cite{berge1997topological} hold, and the excess demand $\excessset$ is upper hemicontinuous, non-empty, compact, and convex-valued over $\simplex[\numcommods]$.

    \textbf{Boundedness}
    Since for all consumers $\consumer \in \consumers$, $\consumptions[\consumer]$ is bounded from below, $\consumptions[\consumer][\prime]$ must be bounded as it is bounded from above by $\sum_{\consumer \in \consumers} \consendow[\consumer]$. Hence, for all consumers $\consumer \in \consumers$, $\consumptions[\consumer][\prime]$ is compact. Hence, we must have for all $\price \in \R^\numgoods_+$, and $\excess(\price) \in \excessset(\price)$, $\| \excess(\price)\|_\infty < \diam(\consumptions[\consumer][\prime])$.
\end{proof}