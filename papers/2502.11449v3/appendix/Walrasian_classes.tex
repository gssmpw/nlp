
We now discuss some important classes of Walrasian economies which are variationally stable on $\simplex[\numgoods]$. The most basic class of Walrasian economies which are variationally stable on $\simplex[\numgoods]$ are those which satisfy the law of supply and demand. Intuitively, these Walrasian economies are those for which the excess demand is downward sloping. 


\begin{definition}[Law of supply and demand economies]\label{def:law_of_supply_and_demand}
    Given a Walrasian economy $(\numgoods, \excessset)$, an excess demand correspondence is said to satisfy the \mydef{law of supply and demand} iff     
    \begin{align}
        \innerprod[{\excess(\otherprice) - \excess(\price)}][{\otherprice - \price}] \leq 0 && \text{ for all $\excess(\price) \in \excessset(\price), \excess(\otherprice) \in \excessset(\otherprice)$}
    \end{align}
\end{definition}

We note that the excess demand of a Walrasian economies satisfies the law of supply and demand iff $-\excessset$ is monotone. This implies that $-\excessset$ is quasimonotone, and hence for any non-empty and compact price space $\pricespace \subseteq \R^\numgoods_+$ the VI $(\pricespace, -\excessset)$ satisfies the Minty condition (see Lemma 3.1 of \citet{he2017solvability}), meaning that any Walrasian economy which satisfies the law of supply and demand is variationally stable on $\pricespace$. 

Another important class of Walrasian economies which are variationally stable on $\simplex[\numgoods]$ is the class of Walrasian economies which satisfy the weak gross substitutes condition. Intuitively, these are Walrasian economies for which the excess demand for a given good only increases when the price of some other good increases. 
% We note that the following definition is generalized for balanced economies which might not necessarily satisfy Walras' law, and as such, requires that the excess demand for the good whose price went up to also not increase. When the economy is assumed to satisfy this additional condition is implied by the first one and as such can be dropped.
While we omit the proof as it is involved, we note that any continuous balanced weak gross substitutes Walrasian economy $(\numgoods, \excessset)$ which satisfies Walras' law, i.e., (for all $\price \in \R^\numgoods_+, \excess(\price) \in \excessset(\price)$, $\price \cdot \excess(\price)$) is a subset of the class of variationally stable economies on $\pricespace \subseteq \R^\numgoods_+$ for any non-empty and compact price space $\pricespace$ (see, for instance lemma 5 of \citet{arrow1959stability}).


\begin{definition}[Weak Gross Substitutes economies]\label{def:wgs}
    Given a Walrasian economy $(\numgoods, \excessset)$, an excess demand correspondence is said to satisfy the \mydef{weak gross substitutes condition} iff for all $\price, \otherprice \in \R^\numgoods_+$ s.t. for some $k \in \goods$, $\otherprice[k] > \price[k]$ and for all $\good \neq k, \otherprice[\good] = \price[\good]$, we have:
    \begin{align}
        \excess[\good](\otherprice) \geq \excess[\good](\price) 
        % \text{ and } \excess[k](\otherprice) \leq \excess[k](\price)
        && \text{ for all $\excess(\price) \in \excessset(\price), \excess(\otherprice) \in \excessset(\otherprice)$}
    \end{align}
\end{definition}

\if 0
\begin{lemma}
    Suppose that $(\numgoods, \excessset)$ is a Walrasian economy with excess demand correspondence which satisfies the weak gross substitutes condition and which has a non-empty set of Walrasian equilibria. Then, $(\numgoods, \excessset)$ is variationally stable on any non-empty and compact set $\pricespace \subseteq \R^\numgoods_+$.
\end{lemma}
\begin{proof}
    Let $(\numgoods, \excessset)$ be a Walrasian economy which satisfies the weak gross substitutes condition, and $\price[][][*] \in \we(\numgoods, \excessset)$ be a Walrasian equilibrium of $(\numgoods, \excessset)$. Without loss of generality, given any price $\price \in \R^\numgoods$ relabel the commodities, suppose that $\price[1] - \price[1][][*] \leq \price[2] - \price[2][][*] \leq \hdots \leq \price[\numgoods] - \price[\numgoods][][*]$ and let $\good^*$ the last index s.t. $\price[{\good^*}] - \price[{\good^*}][][*] \leq 0$ if such an index exists and $0$ otherwise. Then, define the following associated price vectors:
    \begin{align*}
        \price[][0] &= (\price[1][][*], \price[2][][*], \price[3][][*], \hdots, \price[\numgoods][][*])\\
        \price[][1] &= (\price[1], \price[2][][*], \price[3][][*] \hdots, \price[1][][*])\\
        \price[][2] &= (\price[1], \price[2], \price[3][][*] \hdots, \price[\numgoods][][*])\\
        \price[][3] &= (\price[1], \price[2], \price[3] \hdots, \price[\numgoods][][*])\\
        &\vdots\\
        \price[][\numgoods] &= (\price[1], \price[2], \hdots, \price[\numgoods])
    \end{align*}


    Observe then we have:
    \begin{align}
        \innerprod[{\excess(\price[][][*])}][{\price - \price[][][*]}] &= \innerprod[{\excess(\price[][][*])}][{\price}] - \underbrace{\innerprod[{\excess(\price[][][*])}][{\price[][][*]}]}_{= 0} \notag\\
        &= \underbrace{\innerprod[{\excess(\price[][][*])}][{\price}]}_{\leq 0} \notag\\
        &\leq 0 \label{eq:wgs_bound_1}
    \end{align}
    where the penultimate line follows from Walras' law holding at the Walrasian equilibrium $\price[][][*]$, and the last line from feasibility holding at the Walrasian equilibrium $\price[][][*]$, and the positive of prices $\price$.

    
    Further, notice that we have:
    \begin{align}
        \price - \price[][][*] = \price[][\numgoods] - \price[][0] = \price[][\numgoods] - \price[][0] + \sum_{k = 1}^{\numgoods -1} [\price[][k] - \price[][k - 1]] = \sum_{k = 1}^{\numgoods} [\price[][k] - \price[][k - 1 ]] \label{eq:wgs_cycle_prices}
    \end{align}

    Similarly, we also have
    \begin{align}
        \excess(\price) - \excess(\price[][][*]) = \excess(\price[][\numgoods]) - \excess(\price[][0]) = \sum_{k = 1}^{\numgoods} [\excess(\price[][k]) - \excess(\price[][k -1])] \label{eq:wgs_cycle_excess}
    \end{align}

    Hence, we have:
    \begin{align*}
         &\innerprod[{ \excess(\price)}][{\price - \price[][][*]}] \\
        &\leq \innerprod[{ \excess(\price)}][{\price - \price[][][*]}] - \underbrace{\innerprod[{\excess(\price[][][*])}][{\price - \price[][][*]}]}_{\leq 0} && \text{(\Cref{eq:wgs_bound_1})}\\
        &= \innerprod[{ \excess(\price) - \excess(\price[][][*])}][{\price - \price[][][*]}]\\
        &=\innerprod[{ \sum_{k =1}^{\numgoods} \excess(\price[][k]) - \excess(\price[][k-1])}][{ \price[][m] - \price[][0]}] && \text{(\Crefrange{eq:wgs_cycle_prices}{eq:wgs_cycle_excess})}\\
        &= \sum_{k = 1}^{\numgoods} \sum_{l = 1}^{\numgoods}  \innerprod[{  \excess(\price[][k]) - \excess(\price[][k-1])}][{ \price[][l] - \price[][l-1]}]\\
        &= \sum_{k = 1}^{\numgoods} \sum_{\good = 1}^{\numgoods} [\excess[\good](\price[][k]) - \excess[\good](\price[][k - 1])] [\price[\good][\good] - \price[\good][\good - 1]]\\
        &= \sum_{\good = 1}^\numgoods \sum_{k = 1}^{\good -1 }[\excess[\good](\price[][k]) - \excess[\good](\price[][k - 1])] [\price[\good][\good] - \price[\good][\good - 1]]\\
        & \quad + \sum_{\good = 1}^\numgoods \sum_{k = \good + 1}^{\numgoods} [\excess[\good](\price[][k]) - \excess[\good](\price[][k - 1])] [\price[\good][\good] - \price[\good][\good - 1]] \\
        &\quad +   \underbrace{\sum_{\good \in \goods}[\excess[\good](\price[][\good]) - \excess[\good](\price[][\good - 1])] [\price[\good][\good] - \price[\good][\good - 1]]}_{\leq 0 \text{  by Weak Gross Substitutes}}\\
        &= \sum_{\good = 1}^\numgoods \sum_{k = 1}^{\good -1 }[\excess[\good](\price[][k]) - \excess[\good](\price[][k - 1])] [\price[\good] - \price[\good][][*]]\\
        & \quad + \sum_{\good = 1}^\numgoods \sum_{k = \good + 1}^{\numgoods} [\excess[\good](\price[][k]) - \excess[\good](\price[][k - 1])] [\price[\good] - \price[\good][][*]]\\
        % &= \sum_{\good = 1}^\numgoods [\excess[\good](\price[][\good - 1]) - \excess[\good](\price[][][*])] [\price[\good] - \price[\good][][*]]\\
        % & \quad + \sum_{\good = 1}^\numgoods  [\excess[\good](\price) - \excess[\good](\price[][\good + 1][])] [\price[\good] - \price[\good][][*]]\\
        % &= \sum_{\good = 1}^\numgoods [\excess[\good](\price[][\good - 1]) - \excess[\good](\price[][\good + 1][])] [\price[\good] - \price[\good][][*]]\\
        % & \quad + \sum_{\good = 1}^\numgoods  [\excess[\good](\price) - \excess[\good](\price[][][*])] [\price[\good] - \price[\good][][*]]
        % &= \sum_{\good = 1}^{\numgoods} \sum_{k = 1}^{\good} [\excess[\good](\price[][k]) - \excess[\good](\price[][k - 1])] [\price[\good][\good] - \price[\good - 1][\good - 1]] + \sum_{\good = 1}^{\numgoods} \sum_{k = \good + 1}^{\numgoods}[\excess[\good](\price[][k]) - \excess[\good](\price[][k - 1])] [\price[\good][\good] - \price[\good - 1][\good - 1]]\\
        % &= \sum_{k = 1}^{\good^*} \sum_{\good = 1}^{k} [\excess[\good](\price[][k]) - \excess[\good](\price[][k - 1])] [\price[\good][\good] - \price[\good - 1][\good - 1]] + \underbrace{\sum_{k = 1}^{\good^*} \sum_{\good = k + 1}^{\numgoods}[\excess[\good](\price[][k]) - \excess[\good](\price[][k - 1])] [\price[\good][\good] - \price[\good - 1][\good]] }_{\leq 0}\notag\\
        % & + \underbrace{\sum_{k = \good^* + 1}^{\numgoods} \sum_{\good = 1}^{k - 1} [\excess[\good](\price[][k]) - \excess[\good](\price[][k - 1])] [\price[\good][\good] - \price[\good][\good - 1]]}_{\leq 0} + \sum_{k = \good^* + 1}^{\numgoods} \sum_{\good = k }^{\numgoods}[\excess[\good](\price[][k]) - \excess[\good](\price[][k - 1])] [\price[\good][\good] - \price[\good][\good - 1]]\\
        % &\leq \sum_{k = 1}^{\good^*} \sum_{\good = 1}^{k} [\excess[\good](\price[][k]) - \excess[\good](\price[][k - 1])] [\price[\good][\good] - \price[\good][\good - 1]] + \sum_{k = \good^* + 1}^{\numgoods} \sum_{\good = k }^{\numgoods}[\excess[\good](\price[][k]) - \excess[\good](\price[][k - 1])] [\price[\good][\good] - \price[\good][\good - 1]]\\
        % &\leq \sum_{k = 1}^{\good^*} \sum_{\good = 1}^{k}  -[ \price[\good][\good - 1]] [\excess[\good](\price[][k]) - \excess[\good](\price[][k - 1])]  + \sum_{k = \good^* + 1}^{\numgoods} \sum_{\good = k }^{\numgoods} [\price[\good][\good]] [\excess[\good](\price[][k]) - \excess[\good](\price[][k - 1])] \\
        % &\leq  \sum_{k = 1}^{\good^*} \sum_{\good = 1}^{k}  \price[\good][\good - 1] \excess[\good](\price[][k - 1]) - \sum_{k = 1}^{\good^*} \sum_{\good = 1}^{k} \price[\good][\good - 1]\excess[\good](\price[][k])+ \sum_{k = \good^* + 1}^{\numgoods} \sum_{\good = k }^{\numgoods} \price[\good][\good] \excess[\good](\price[][k]) - \sum_{k = \good^* + 1}^{\numgoods} \sum_{\good = k }^{\numgoods} \price[\good][\good] \excess[\good](\price[][k - 1]) \\
        % &=\sum_{k = 1}^{\good^*} \sum_{\good = 1}^{k} \langle \excess(\price[][k]) - \excess(\price[][k - 1]), \price[][\good] - \price[][\good - 1] \rangle +  \sum_{k = \good^* + 1}^{\numgoods} \sum_{\good = k }^{\numgoods} \langle \excess(\price[][k]) - \excess(\price[][k - 1]), \price[][\good] - \price[][\good - 1] \rangle\\
        % &=\sum_{k = 1}^{\good^*} \sum_{\good = 1}^{k} \langle \excess(\price[][k]) - \excess(\price[][k - 1]), \price[][\good] - \price[][\good - 1] \rangle +  \sum_{k = \good^* + 1}^{\numgoods} \sum_{\good = k }^{\numgoods} \langle \excess(\price[][k]) - \excess(\price[][k - 1]), \price[][\good] - \price[][\good - 1] \rangle\\
        % &=\sum_{k = 1}^{\good^*}  \langle \excess(\price[][k]) - \excess(\price[][k - 1]), \price[][k] - \price[][0] \rangle +  \sum_{k = \good^* + 1}^{\numgoods}  \langle \excess(\price[][k]) - \excess(\price[][k - 1]), \price - \price[][k - 1] \rangle
        % &= \sum_{k = 0}^{\numgoods - 1}   \innerprod[{  \excess(\price[][k+1]) - \excess(\price[][k])}][{\price[][m]}] - \sum_{k = 0}^{\numgoods - 1}   \innerprod[{  \excess(\price[][k+1]) - \excess(\price[][k])}][{  \price[][0]}]\\
    \end{align*}

    Now, notice that we must have $\sum_{\good = 1}^\numgoods \sum_{k = 1}^{\good -1 }[\excess[\good](\price[][k]) - \excess[\good](\price[][k - 1])] [\price[\good][\good] - \price[\good][\good - 1]]+ \sum_{\good = 1}^\numgoods \sum_{k = \good + 1}^{\numgoods} [\excess[\good](\price[][k]) - \excess[\good](\price[][k - 1])] [\price[\good][\good] - \price[\good][\good - 1]]\leq 0$.
    where the last inequality follows from the weak gross substitutes assumption and the fact that $\price[1][1] - \price[1][0] = \price[1][][] - \price[1][][*]  = \price[1] - \price[1][][*] \leq \hdots  \price[\numgoods] - \price[\numgoods][][*] = \price[\numgoods][\numgoods] - \price[\numgoods][\numgoods - 1]$.
%     \if 0
%     \begin{align*}
%         &\innerprod[{ \excess(\price)}][{\price - \price[][][*]}] \\
%         &\leq \innerprod[{ \excess(\price)}][{\price - \price[][][*]}] - \underbrace{\innerprod[{\excess(\price[][][*])}][{\price - \price[][][*]}]}_{\leq 0} && \text{(\Cref{eq:wgs_bound_1})}\\
%         &= \innerprod[{ \excess(\price) - \excess(\price[][][*])}][{\price - \price[][][*]}]\\
%         &=\innerprod[{ \sum_{k = 0}^{\numgoods - 1} \excess(\price[][k+1]) - \excess(\price[][k])}][{ \price[][m] - \price[][0]}] && \text{(\Crefrange{eq:wgs_cycle_prices}{eq:wgs_cycle_excess})}\\
%         &= \sum_{k = 0}^{\numgoods - 1}   \innerprod[{  \excess(\price[][k+1]) - \excess(\price[][k])}][{ [\price[][m] - \price[][0]]}]\\
%         &= \sum_{k = 0}^{\numgoods - 1}   \innerprod[{  \excess(\price[][k+1]) - \excess(\price[][k])}][{\price[][m]}] - \sum_{k = 0}^{\numgoods - 1}   \innerprod[{  \excess(\price[][k+1]) - \excess(\price[][k])}][{  \price[][0]}]\\
%         &= \sum_{k = 0}^{\numgoods - 1}  \sum_{\good = 1}^\numgoods \left[ \left(\excess[\good](\price[][k+1]) - \excess[\good](\price[][k]) \right) \price[\good][m] - \left(\excess[\good](\price[][k+1]) - \excess[\good](\price[][k]) \right) \price[\good][0] \right]\\
%         &=  \sum_{\good = 1}^\numgoods \left[ \underbrace{\sum_{k = 0}^{\good - 1}\left(\excess[\good](\price[][k+1]) - \excess[\good](\price[][k]) \right) (\price[\good][m] - \price[\good][0])} \notag \\
%         & \quad  + \sum_{k = \good}^{\numgoods} \left(\excess[\good](\price[][k+1]) - \excess[\good](\price[][k]) \right) (\price[\good][m] - \price[\good][0]) \right]
%         % &=  \left[ \sum_{\good \in \goods: \price[\good][m] \geq \price[\good][0]} \sum_{k = 0}^{\numgoods - 1} \left(\excess[\good](\price[][k+1]) - \excess[\good](\price[][k]) \right) (\price[\good][m] - \price[\good][0]) \right.  \\
%         % & \quad \left. + \sum_{\good \in \goods: \price[\good][m] < \price[\good][0]} \left(\excess[\good](\price[][k+1]) - \excess[\good](\price[][k]) \right) (\price[\good][\good + 1] - \price[\good][\good])  \right]\\
%         % &= \sum_{k = 0}^{\numgoods - 1} \left[ \sum_{\good \in \goods: \price[\good][m] \geq \price[\good][0]} \left(\excess[\good](\price[][k+1]) - \excess[\good](\price[][k]) \right) (\price[\good][m] - \price[\good][0]) \right.  \\
%         % & \quad \left. + \sum_{\good \in \goods: \price[\good][m] < \price[\good][0]} \left(\excess[\good](\price[][k+1]) - \excess[\good](\price[][k]) \right) (\price[\good][m] - \price[\good][0])  \right]
%     \end{align*}
    
%     \begin{align*}
%         \innerprod[{ \excess(\price)}][{\price - \price[][][*]}] &\leq \innerprod[{ \excess(\price)}][{\price - \price[][][*]}] - \underbrace{\innerprod[{\excess(\price[][][*])}][{\price - \price[][][*]}]}_{\leq 0} && \text{(\Cref{eq:wgs_bound_1})}\\
%         &= \innerprod[{ \excess(\price) - \excess(\price[][][*])}][{\price - \price[][][*]}]\\
%         &=\innerprod[{ \sum_{k = 0}^{\numgoods - 1} [\excess(\price[][k+1]) - \excess(\price[][k])] }][{ \sum_{k = 0}^{\numgoods - 1} [\price[][k+1] - \price[][k]]}] && \text{(\Crefrange{eq:wgs_cycle_prices}{eq:wgs_cycle_excess})}\\
%         &= \sum_{k = 0}^{\numgoods - 1} \sum_{l = 0}^{\numgoods - 1}  \innerprod[{  [\excess(\price[][k+1]) - \excess(\price[][k])] }][{ [\price[][l+1] - \price[][l]]}]
%     \end{align*}
%     Then, it is well-known  \citet{kuga1965weak}

    
    
%     Suppose that  for all $\innerprod[{\excess(\otherprice)}][{\price - \otherprice}]  < 0$, we then have for all $\otherprice \in \R^\numgoods_+$: 
% % 
%     \begin{align*}
%         &\innerprod[{ \excess(\price)}][{\price[][][*] - \price}]\\
%         &= \innerprod[{ \excess(\price)}][{\price[][][*]}] - \underbrace{\innerprod[{ \excess(\price)}][{\price}]}_{\leq 0}\\
%         &\geq \innerprod[{ \excess(\price)}][{\price[][][*]}]\\
%         &= \innerprod[{ \excess(\price)}][{\price[][][*] - \otherprice}] + \innerprod[{ \excess(\price)}][{ \otherprice}]\\
%         &= \sum_{\good \neq k} \excess[\good](\price) \price[\good][][*] +  \excess[k](\price) \price[k][][*]
%         % &\innerprod[{\excess(\price)}][{\price - \otherprice}]\\
%         % &< \innerprod[{\excess(\price)}][{\price - \otherprice}] - \innerprod[{\excess(\otherprice)}][{\price - \otherprice}]\\
%         % &= \sum_{\good \in \goods} (\excess[\good](\price) - \excess[\good](\otherprice) (\price[\good] - \otherprice[\good]) \\
%         % &= \sum_{\good \in \goods: \otherprice[\good] \geq \price[\good]} (\excess[\good](\price) - \excess[\good](\otherprice) (\price[\good] - \otherprice[\good]) + \sum_{\good \in \goods: \otherprice[\good] < \price[\good]} (\excess[\good](\price) - \excess[\good](\otherprice) (\price[\good] - \otherprice[\good])
%     \end{align*}

%     For all $\good \in \goods$, s.t. $\otherprice[\good] \geq \price[\good]$

%     \fi
\end{proof}
\fi 

Going further, we can show that any Walrasian economy which satisfies the well-known weak axiom of revealed preferences \citet{afriat1967construction, arrow-hurwicz}, is variationally stable on $\simplex[\numgoods]$ (and more generally on any non-empty and compact price space $\pricespace \subseteq \R^\numgoods$). To this end, let us first define the weak axiom of revealed preferences for balanced economies. 

\begin{definition}[WARP excess demand]\label{def:warp}
    Given a Walrasian economy $(\numgoods, \excessset)$, an excess demand correspondence is said to satisfy the \mydef{weak axiom of revealed preferences} (\mydef{WARP}) iff for all $\excess(\price) \in \excessset(\price), \excess(\otherprice) \in \excessset(\otherprice)$:     
    \begin{align*}
        \innerprod[{\excess(\otherprice)}][{\price}] \leq \innerprod[{\excess(\otherprice)}][{\otherprice}] \text{ and } \excess(\price) \neq \excess(\otherprice)  \implies \innerprod[{\excess(\price)}][{\otherprice}] > \innerprod[{\excess(\price)}][{\price}]
    \end{align*}
    % \begin{align}
    %     \innerprod[{\excess(\otherprice)}][{\price}] < \innerprod[{\excess(\otherprice)}][{\otherprice}] \implies \innerprod[{\excess(\price)}][{\otherprice}] \geq \innerprod[{\excess(\price)}][{\price}] && \text{ for all $\excess(\price) \in \excessset(\price), \excess(\otherprice) \in \excessset(\otherprice)$}
    % \end{align}
\end{definition}

\begin{remark}
    This definition of (WARP) is adapted to arbitrary Walrasian economies and as such is a generalization of the usual definition for economies which satisfy Walras' law (i.e., for all $\price \in \R^\numgoods_+$, $\price \cdot \excess(\price)$), which requires that $\excessset$ is singleton-valued, and $\innerprod[{\excess(\otherprice)}][{\price}] \leq 0 
    \text{ and } \excess(\price) \neq \excess(\otherprice)  
    \implies \innerprod[{\excess(\price)}][{\otherprice}] > 0$ (i.e., for all $\price \in \R^\numgoods_+$, $\price \cdot \excess(\price)$). 
    % This definition of (WARP) is weaker than the usual definition (see, for instance \citet{arrow-hurwicz}), which requires that $\excessset$ is singleton-valued, and $\innerprod[{\excess(\otherprice)}][{\price}] \leq \innerprod[{\excess(\otherprice)}][{\otherprice}] \text{ and } \excess(\price) \neq \excess(\otherprice)  \implies \innerprod[{\excess(\price)}][{\otherprice}] > \innerprod[{\excess(\price)}][{\price}]$, as well as a recent a weaker definition provided by \citet{quah2008existence}. Additionally, note that this definition is more general as it applies to any Walrasian economy, and not only Walrasian economies which satisfy Walras' law (i.e., for all $\price \in \R^\numgoods_+$, $\price \cdot \excess(\price)$) as it is traditionally assumed for Walrasian economies in these prior definitions. As such these other definitions of WARP imply the one we consider here.
\end{remark}

As we show next, WARP implies that $-\excessset$ is pseudomonotone in balanced economies.\footnote{To be more precise, we note that an excess demand function $\excessset$ satisfies WARP iff $-\excessset$ is strictly pseudomonotone. However, as this result will not be used we present the more general result.} 

\begin{lemma}[WARP $\implies$ pseudomonotone ]\label{lemma:warp_implies_pseudomonotone}
    If the excess demand correspondence $\excessset$ of a Walrasian economy $(\numgoods, \excessset)$ satisfies WARP, then $-\excessset$ is pseudomonotone.
\end{lemma}
\begin{proof}
    Suppose that $\excessset$ satisfies WARP, and that $\innerprod[{-\excess(\otherprice)}][{\otherprice - \price}] = \innerprod[{\excess(\otherprice)}][{\price - \otherprice}] \leq 0$, then we have $\innerprod[{-\excess(\price)}][{\otherprice - \price}] = \innerprod[{\excess(\price)}][{\price - \otherprice}]$

    If $\excess(\price) \neq \excess(\otherprice)$, then, by WARP, we have $\innerprod[{\excess(\price)}][{\price - \otherprice}] < 0$. 
    
    Otherwise, if $\excess(\price) = \excess(\otherprice)$, then we have:
    \begin{align*}
        \innerprod[{\excess(\price)}][{\price - \otherprice}] =  \innerprod[{\excess(\otherprice)}][{\price - \otherprice}] \leq 0
    \end{align*}

    That is, if $\excessset$ satisfies WARP, we have:
    \begin{align*}
        \innerprod[{-\excess(\otherprice)}][{\otherprice - \price}] \leq 0 \implies \innerprod[{-\excess(\price)}][{\otherprice - \price}] \leq 0
    \end{align*}
    
    Hence, $-\excessset$ is pseudomonotone.
\end{proof}

An important consequence of \Cref{lemma:warp_implies_pseudomonotone} is that since $-\excessset$ is pseudomonotone, for any non-empty and compact price space $\pricespace \subseteq \R^\numgoods_+$ the VI $(\pricespace, -\excessset)$ satisfies the Minty condition (see Lemma 3.1 of \citet{he2017solvability}). As such, we have the following corollary of \Cref{lemma:warp_implies_pseudomonotone}. 
% a Walrasian economy is variationally stable on any non-empty and compact price space $\pricespace \subseteq \R^\numgoods_+$ if it satisfies WARP. 

\begin{corollary}[WARP $\implies$ Minty's condition]\label{cor:warp_iff_minty}
    Any Walrasian economy which satisfies WARP is variationally stable on any non-empty and compact price space $\pricespace \subseteq \R^\numgoods_+$. 
\end{corollary}
