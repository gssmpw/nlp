\subsection{Walrasian Economies and Variational Inequalities}
\thmweequalsvi*
\begin{proof}[Proof of \Cref{thm:we_equal_svi}]
    $(\implies)$
    Let $\price[][][*] \in \we(\numgoods, \excessset)$ be a Walrasian equilibrium. Then, for some $\excess(\price[][][*]) \in \excessset(\price)$, we have:
    \begin{align*}
        % &\innerprod[{-\excess(\price[][][*])}][{\price[][][*] - \price}] && \forall \price \in \R^\numgoods_+\\
        &\innerprod[{\excess(\price[][][*])}][{\price - \price[][][*]}] && \forall \price \in \R^\numgoods_+\\
        &= \innerprod[{\excess(\price[][][*])}][{\price}] - \underbrace{\innerprod[{\excess(\price[][][*])}][{\price[][][*] }]}_{= 0} && \forall \price \in \R^\numgoods_+\\
        &= \underbrace{\innerprod[{\excess(\price[][][*])}][{\price}]}_{\leq 0} && \forall \price \in \R^\numgoods_+\\
        &\leq 0
    \end{align*}
    where the last line follows from the feasibility of $\excess(\price[][][*])$, i.e., $\excess(\price[][][*]) \leq 0$, and the positivity of $\price$.
    % By Corollary 2.1 of \citet{nagurney2009network}, we know that the set of strong solutions of the VI $(\R^\numgoods_+, -\excessset)$ is non-empty since $-\excessset$ is coercive since the economy $(\numgoods, \R^\numgoods_+, \goodset, \excessset)$ is coercive. 

    $(\impliedby)$
    Let $\price[][][*] \in \svi(\R^\numgoods_+, -\excessset)$. Then, for some $\excess(\price[][][*]) \in \excessset(\price)$, we have:
    \begin{align*}
        0&\geq  \innerprod[{\excess(\price[][][*])}][{\price - \price[][][*]}] && \forall \price \in \R^\numgoods_+
    \end{align*}
    Substituting $\price \doteq \price[][][*] + \basis[\good]$, we have:
    \begin{align*}
        0 &\geq \innerprod[{\excess(\price[][][*])}][{\price[][][*] + \basis[\good] - \price[][][*]}]\\
        &= \innerprod[{\excess(\price[][][*])}][{\basis[\good]}]\\
        &\geq \excess[\good](\price[][][*]) && \forall \good \in \goods
    \end{align*}
    That is, $\price[][][*]$ is feasible.
    
    Similarly, substituting in $\price \doteq \zeros$ and $\price \doteq 2\price[][][*]$, we have:
    \begin{align*}
        0&\leq  \innerprod[{\excess(\price[][][*])}][{\price[][][*]}] 
    \end{align*}
    and 
    \begin{align}
        0&\geq  \innerprod[{\excess(\price[][][*])}][{\price[][][*]}] 
    \end{align}

    That is, $\price[][][*]$ satisfies weak Walras' law.
    Hence, $\price[][][*]$ is a Walrasian equilibrium.
\end{proof}


\thmwebalancedequalsvi*
\begin{proof}[Proof of \Cref{thm:we_balanced_equal_svi}]
    $(\implies)$
    Let $\price[][][*] \in \we(\numgoods, \excessset)$ be a Walrasian equilibrium. Let $\alpha \doteq \frac{1}{\max\{1, \|\price[][][*]\|_\infty \}}$. Then, we have $\alpha \price[][][*] \in [0, 1]^\numgoods$. Further, for some $\excess(\alpha \price[][][*]) \in \excessset(\alpha \price[][][*])$, we have:
    \begin{align*}
        &\innerprod[{-\excess(\alpha\price[][][*])}][{\alpha\price[][][*] - \price}] && \forall \price \in [0, 1]^\numgoods\\
        &= \innerprod[{\excess(\price[][][*])}][{\price - \alpha\price[][][*]}] && \forall \price \in [0, 1]^\numgoods && \text{(Homogeneity of $\excess$)}\\
        &= \innerprod[{\excess(\price[][][*])}][{\price}] - \alpha \underbrace{\innerprod[{\excess(\price[][][*])}][{\price[][][*] }]}_{= 0} && \forall \price \in [0, 1]^\numgoods\\
        &= \innerprod[{\excess(\price[][][*])}][{\price}]  && \forall \price \in [0, 1]^\numgoods\\
        &\leq 0
    \end{align*}
    where the penultimate line follows from Walras' law holding at a Walrasian equilibrium, and the last line follows from the feasibility of $\excess(\price[][][*])$, i.e., $\excess(\price[][][*]) \leq \zeros$, and the positivity of $\price$. Hence, $\alpha \price[][][*]$ is a strong solution of the VI $([0, 1]^\numgoods, - \excessset)$, which means that $\price[][][*] \in \frac{1}{\alpha} \svi([0, 1]^\numgoods, - \excessset)$.
    
    Now, notice that by homogeneity of the excess demand in balanced economies since for all $\lambda > 0$, $\excessset(\lambda \price[][][*]) = \excessset(\price[][][*])$, if $\price[][][*]$ is a Walrasian equilibrium, then so is $\lambda \price[][][*]$. Hence, $\alpha$ takes values in $(0, 1]$, implying $\frac{1}{\alpha} \in [1, \infty)$, and as such we must have $\we(\numgoods, \excessset) \subseteq  \bigcup_{\lambda \geq 1} \lambda \svi([0, 1]^\numgoods, -\excessset)$.
    
    $(\impliedby)$
    Let $\price[][][*] \in \svi([0, 1]^\numgoods, -\excessset)$ and $\lambda \geq 1$. Then, for some $\excess(\price[][][*]) \in \excessset(\price[][][*])$, we have:
    \begin{align}
        0&\geq \innerprod[{-\excess(\price[][][*])}][{\price[][][*] - \price}] && \forall \price \in [0, 1]^\numgoods \notag\\
        &= \innerprod[{\excess(\price[][][*])}][{\price - \price[][][*]}] && \forall \price \in [0, 1]^\numgoods \notag\\
        &= \innerprod[{\excess(\price[][][*])}][{\price}] - \innerprod[{\excess(\price[][][*])}][{\price[][][*] }] && \forall \price \in [0, 1]^\numgoods \label{eq:we_svi_eq_left}
    \end{align}
    Plugging $\price = \zeros[\numgoods]$ in \Cref{eq:we_svi_eq_left}, we then have: 
    \begin{align*}
        0&\geq \underbrace{\innerprod[{\excess(\price[][][*])}][{\zeros[\numgoods]}]}_{= 0} - \innerprod[{\excess(\price[][][*])}][{\price[][][*] }]\\
        0&\geq -  \innerprod[{\excess(\price[][][*])}][{\price[][][*]}]\\
        0&\leq \innerprod[{\excess(\price[][][*])}][{\price[][][*]}]\\
        0&\leq \innerprod[{\excess(\lambda \price[][][*])}][{\price[][][*]}] && \text{(Homogeneity of $\excess$)}\\
        0&\leq \innerprod[{\excess(\lambda \price[][][*])}][{\lambda \price[][][*]}] 
    \end{align*}
    Further, since $(\numgoods, \excessset)$ is balanced, we have $\lambda \price[][][*] \cdot \excess(\lambda \price[][][*]) = \price[][][*] \cdot \excess(\price[][][*]) \leq 0$, hence, combining it with the above inequality, we must have $\lambda \price[][][*] \cdot \excess(\lambda \price[][][*]) = 0$, meaning that $\lambda \price[][][*]$ satisfies Walras' law.

    In addition, continuing from \Cref{eq:we_svi_eq_left} again, we have:
    \begin{align}
        0&\geq \innerprod[{\excess(\price[][][*])}][{\price}] - \underbrace{\innerprod[{\excess(\price[][][*])}][{\price[][][*] }]}_{\leq 0} \notag\\
        &\geq \innerprod[{\excess(\price[][][*])}][{\price}] && \forall \price \in [0, 1]^\numgoods\\
        &\geq \innerprod[{\excess(\lambda \price[][][*])}][{\price}] && \forall \price \in [0, 1]^\numgoods\label{eq:weak_walras_law_application}
    \end{align}
     where the penultimate line follows from the fact that balanced economies satisfy weak Walras' law, and the last line from homogeneity of degree $0$ of the excess demand.

    Now, plugging $\price = \basis[\good]$ for all $\good \in \goods$ in \Cref{eq:weak_walras_law_application}, we have:
    \begin{align*}
        0 &\geq \innerprod[{\excess(\lambda \price[][][*])}][{\basis[\good]}] && \forall \good \in \goods \notag\\
        &\geq \excess[\good](\lambda \price[][][*]) && \forall \good \in \goods \notag \enspace .
    \end{align*}

   That is, $\lambda \price[][][*]$ is feasible. Putting it all together, $\lambda \price[][][*]$ must be a Walrasian equilibrium. As such we must have $ \bigcup_{\lambda \geq 1} \lambda \svi([0, 1]^\numgoods, -\excessset) \subseteq \we(\numgoods, \excessset)$
\end{proof}



\lemmaapproxsvieqapproxwe*
\begin{proof}[Proof of \Cref{lemma:approx_svi_eq_approx_we}]
    For any $\varepsilon \geq 0$, let $\price[][][*] \in \svi[\varepsilon]([0, 1]^\numgoods, -\excessset)$. Then, for some $\excess(\price[][][*]) \in \excessset(\price[][][*])$, we have:
    \begin{align}
        \varepsilon &\geq \innerprod[{-\excess(\price[][][*])}][{\price[][][*] - \price}] && \forall \price \in [0, 1]^\numgoods \notag\\
        &= \innerprod[{\excess(\price[][][*])}][{\price - \price[][][*]}] && \forall \price \in [0, 1]^\numgoods \notag\\
        &= \innerprod[{\excess(\price[][][*])}][{\price}] - \innerprod[{\excess(\price[][][*])}][{\price[][][*] }] && \forall \price \in [0, 1]^\numgoods \label{eq:we_svi_eq_left1}
    \end{align}
    Plugging $\price = \zeros[\numgoods]$ in \Cref{eq:we_svi_eq_left}, we then have: 
    \begin{align*}
        \varepsilon &\geq \underbrace{\innerprod[{\excess(\price[][][*])}][{\zeros[\numgoods]}]}_{= 0} - \innerprod[{\excess(\price[][][*])}][{\price[][][*] }]\\
        \varepsilon &\geq -  \innerprod[{\excess(\price[][][*])}][{\price[][][*]}]\\
        -\varepsilon &\leq \innerprod[{\excess(\price[][][*])}][{\price[][][*]}]
    \end{align*}
    Further, since $(\numgoods, \excessset)$ is balanced, we have $\price[][][*] \cdot \excess(\price[][][*]) \leq 0 \leq \varepsilon$, hence, combining it with the above inequality, we must have that $\price[][][*]$ satisfies $\varepsilon$-Walras' law.

    In addition, continuing from \Cref{eq:we_svi_eq_left1} again, we have:
    \begin{align}
        \varepsilon &\geq \innerprod[{\excess(\price[][][*])}][{\price}] - \underbrace{\innerprod[{\excess(\price[][][*])}][{\price[][][*] }]}_{\leq 0} \notag\\
        &\geq \innerprod[{\excess(\price[][][*])}][{\price}] && \forall \price \in [0, 1]^\numgoods \label{eq:weak_walras_law_application2}
    \end{align}
     where the last line follows from the fact that balanced economies satisfy weak Walras' law.

    Now, plugging $\price = \basis[\good]$ for all $\good \in \goods$ in \Cref{eq:weak_walras_law_application2}, we have:
    \begin{align*}
        \varepsilon &\geq \innerprod[{\excess(\price[][][*])}][{\basis[\good]}] && \forall \good \in \goods \notag\\
        &\geq \excess[\good](\price[][][*]) && \forall \good \in \goods \notag \enspace .
    \end{align*}

   That is, $\price[][][*]$ is $\varepsilon$-feasible. Putting it all together, $\price[][][*]$ must be a $\varepsilon$-Walrasian equilibrium.

   % \paragraph{($\varepsilon$-Walrasian equilibrium $\implies$ $2\varepsilon$-strong solution)}
   % Let $\price[][][*] \in \we[\varepsilon](\numgoods, \excessset) \cap [0, 1]^\numgoods$ be a Walrasian equilibrium in the unit box. Then, for some $\excess(\price[][][*]) \in \excessset(\price[][][*])$, we have:
   %  \begin{align*}
   %      &\innerprod[{-\excess(\price[][][*])}][{\price[][][*] - \price}] && \forall \price \in [0, 1]^\numgoods\\
   %      &= \innerprod[{\excess(\price[][][*])}][{\price - \price[][][*]}] && \forall \price \in [0, 1]^\numgoods\\
   %      &= \innerprod[{\excess(\price[][][*])}][{\price}] - \underbrace{\innerprod[{\excess(\price[][][*])}][{\price[][][*] }]}_{\geq -\varepsilon} && \forall \price \in [0, 1]^\numgoods\\
   %      &= \underbrace{\innerprod[{\excess(\price[][][*])}][{\price}]}_{\leq \numgoods\varepsilon}  + \varepsilon && \forall \price \in [0, 1]^\numgoods\\
   %      &\leq (\numgoods + 1) \varepsilon
   %  \end{align*}
   %  where the penultimate line follows from $\varepsilon$-Walras' law holding at a $\varepsilon$-Walrasian equilibrium, and the last line follows from the $\varepsilon$-feasibility of $\excess(\price[][][*])$, i.e., $\excess(\price[][][*]) \leq \varepsilon$, and the positivity of $\price$.
\end{proof}



\thmwecompequalsvi*
\begin{proof}[Proof of \Cref{thm:we_comp_equal_svi}]
    $(\implies)$
    Let $\price[][][*] \in \we(\numgoods, \excessset)$ be a Walrasian equilibrium. Let $\alpha \doteq \frac{1}{\|\price[][][*]\|_1}$. Then, we have $\alpha \price[][][*] \in \simplex[\numgoods]$. Further, for some $\excess(\alpha \price[][][*]) \in \excessset(\alpha \price[][][*])$, we have:
    \begin{align*}
        &\innerprod[{-\excess(\alpha\price[][][*])}][{\alpha\price[][][*] - \price}] && \forall \price \in \simplex[\numgoods]\\
        &= \innerprod[{\excess(\price[][][*])}][{\price - \alpha\price[][][*]}] && \forall \price \in \simplex[\numgoods] && \text{(Homogeneity of $\excess$)}\\
        &= \innerprod[{\excess(\price[][][*])}][{\price}] - \alpha \underbrace{\innerprod[{\excess(\price[][][*])}][{\price[][][*] }]}_{= 0} && \forall \price \in \simplex[\numgoods]\\
        &= \innerprod[{\excess(\price[][][*])}][{\price}]  && \forall \price \in \simplex[\numgoods]\\
        &\leq 0
    \end{align*}
    where the penultimate line follows from Walras' law holding at a Walrasian equilibrium, and the last line follows from the feasibility of $\excess(\price[][][*])$, i.e., $\excess(\price[][][*]) \leq \zeros$, and the positivity of $\price$. Hence, $\alpha \price[][][*]$ is a strong solution of the VI $(\simplex[\numgoods], - \excessset)$, which means that $\price[][][*] \in \frac{1}{\alpha} \svi(\simplex[\numgoods], - \excessset)$.
    
    Now, notice that by homogeneity of the excess demand in competitive economies since for all $\lambda > 0$, $\excessset(\lambda \price[][][*]) = \excessset(\price[][][*])$, if $\price[][][*]$ is a Walrasian equilibrium, then so is $\lambda \price[][][*]$. Hence, $\alpha$ takes values in $(0, \infty)$, implying $\frac{1}{\alpha} \in [1, \infty)$, and as such we must have $\we(\numgoods, \excessset) \subseteq  \bigcup_{\lambda > 0} \lambda \svi(\simplex[\numgoods], -\excessset)$.
    
    $(\impliedby)$
    Let $\price[][][*] \in \svi(\simplex[\numgoods], -\excessset)$ and $\lambda > 1$. Then, for some $\excess(\price[][][*]) \in \excessset(\price[][][*])$, we have:
    \begin{align*}
        0&\geq \innerprod[{-\excess(\price[][][*])}][{\price[][][*] - \price}] && \forall \price \in \simplex[\numgoods] \notag\\
        &= \innerprod[{\excess(\price[][][*])}][{\price - \price[][][*]}] && \forall \price \in \simplex[\numgoods] \notag\\
        % &= \innerprod[{\excess(\price[][][*])}][{\price}] - \innerprod[{\excess(\price[][][*])}][{\price[][][*] }] && \forall \price \in \simplex[\numgoods] \\
        &= \innerprod[{\excess(\price[][][*])}][{\price}] - \underbrace{\innerprod[{\excess(\price[][][*])}][{\price[][][*] }]}_{\leq 0} \notag\\
        &\geq \innerprod[{\excess(\price[][][*])}][{\price}] && \forall \price \in \simplex[\numgoods]\\
        &\geq \innerprod[{\excess(\lambda \price[][][*])}][{\price}] && \forall \price \in \simplex[\numgoods]
    \end{align*}
     where the penultimate line follows from the fact that competitive economies satisfy weak Walras' law, and the last line from homogeneity of degree $0$ of the excess demand.

    Now, plugging $\price = \basis[\good]$ for all $\good \in \goods$ in the above, we have:
    \begin{align*}
        0 &\geq \innerprod[{\excess(\lambda \price[][][*])}][{\basis[\good]}] && \forall \good \in \goods \notag\\
        &\geq \excess[\good](\lambda \price[][][*]) && \forall \good \in \goods \notag \enspace .
    \end{align*}

   That is, $\lambda \price[][][*]$ is feasible. Now by non-satiation, since $\excess(\lambda \price[][][*]) \leq \zeros[\numgoods]$, we must have $\lambda \price[][][*] \cdot \excess(\lambda \price[][][*]) \geq 0$. As by weak Walras' law $\lambda \price[][][*] \cdot \excess(\lambda \price[][][*]) \leq 0$, we must have $\lambda \price[][][*] \cdot \excess(\lambda \price[][][*]) = 0$, meaning that $\lambda \price[][][*]$ satisfies Walras' law. Putting it all together, $\lambda \price[][][*]$ must be a Walrasian equilibrium. As such we must have $ \bigcup_{\lambda > 0} \lambda \svi(\simplex[\numgoods], -\excessset) \subseteq \we(\numgoods, \excessset)$
\end{proof}


\thmexistencewe*
\begin{proof}[Proof of \Cref{thm:existence_we}]
    By \Cref{thm:we_comp_equal_svi}, we know that the set of strong solutions $\svi(\simplex[\numgoods], -\excessset)$ of the VI $(\simplex[\numgoods], -\excessset)$ is a subset of the set of Walrasian equilibria $(\numgoods, \excessset)$.

    Now, notice that for a continuous Walrasian economy $(\simplex[\numgoods], -\excessset)$ is a continuous VI. Hence, by (Theorem 2.2.1 of \citet{facchinei2003finite}) a strong solution to $(\simplex[\numgoods], -\excessset)$ is guaranteed to exist, which in turn implies the existence of a Walrasian equilibrium in continuous competitive Walrasian economies.
\end{proof}

\subsection{Results for Balanced Economies}

\lemmabalancedisminty*
\begin{proof}[Proof of \Cref{lemma:balanced_is_minty}]
    % We will show that choosing $\price[][][*] \doteq \zeros[\numgoods]$, for all prices $\price \in [0, 1]^\numgoods$ and some $\excess(\price) \in \excessset(\price)$:
    % \begin{align*}
    %     \innerprod[{ \excess(\price)}][{\price[][][*] - \price}] \geq 0
    % \end{align*}

    Let $(\numgoods, \excessset)$ be a balanced economy, then setting $\price[][][*] \doteq \zeros[\numgoods]$, we have:
    \begin{align*}
        \innerprod[{ \excess(\price)}][{\price[][][*] - \price}] &= \innerprod[{ \excess(\price)}][{\zeros[\numgoods] - \price}]\\
        &= \underbrace{\innerprod[{ \excess(\price)}][{\zeros[\numgoods]}]}_{= 0} - \innerprod[{ \excess(\price)}][{\price}]\\
        &= - \underbrace{\innerprod[{ \excess(\price)}][{\price}]}_{\leq 0}\\
        &\geq 0
    \end{align*}
    where the last line follow from the weak Walras' law assumption holding in balanced economies, i.e., for all prices $\price \in [0, 1]^\numgoods$, $\innerprod[{ \excess(\price)}][{\price}] \leq 0$, which implies $-\innerprod[{ \excess(\price)}][{\price}] \geq 0$. 
    % Hence, the VI $([0, 1]^\numgoods, -\excessset)$ associated with any balanced economy $(\numgoods, \excessset)$ satisfies the Minty condition.
\end{proof}




\thmmirrorextratatonnconvergence*
\begin{proof}[Proof of \Cref{thm:mirror_extra_tatonn_convergence}]
    Since $(\numgoods, \excess)$ is a balanced economy, by \Cref{lemma:balanced_is_minty}, $(\numgoods, \excess)$ is variationally stable on $[0, 1]^\numgoods$, and hence the VI $([0, 1]^\numgoods, -\excessset)$ satisfies the Minty condition. Hence, as the mirror \emph{extrat\^atonnement} process is simply the mirror extragradient method run on the VI $([0, 1]^\numgoods, -\excessset)$, the assumptions of \Cref{thm:mirror_extragradient_global_convergence} are satisfied, and we obtain the result. 
    
    \if 0
    we have the following bound:

    \begin{align*}
         \min_{k = 0, \hdots, \numhorizons} \max_{\price \in \simplex} \langle -\excess(\price[][k+0.5]),  \price[][k+0.5] - \price \rangle &\leq  \frac{2 (1 + \kernelsmooth)\diam(\simplex[\numgoods])}{\learnrate[ ][ ]} \frac{\sqrt{\max_{\price \in \simplex}\divergence[\kernel][{\price}][{\price[][0]}]}}{\sqrt{\numhorizons}}\\
         \min_{k = 0, \hdots, \numhorizons} \max_{\price \in \simplex} \langle \excess(\price[][k+0.5]),  \price \rangle &\leq  \frac{2 (1 + \kernelsmooth)\diam(\simplex[\numgoods])}{\learnrate[ ][ ]} \frac{\sqrt{\max_{\price \in \simplex}\divergence[\kernel][{\price}][{\price[][0]}]}}{\sqrt{\numhorizons}}
    \end{align*}
    Now, note that $\diam(\simplex[\numgoods]) = \sqrt{2}$, hence, we have:
    \begin{align}
        \min_{k = 0, \hdots, \numhorizons} \max_{\price \in \simplex} \langle \excess(\price[][k+0.5]),  \price \rangle \leq  \frac{2 \sqrt{2} (1 + \kernelsmooth)}{\learnrate[ ][ ]} \frac{\sqrt{\max_{\price \in \simplex}\divergence[\kernel][{\price}][{\price[][0]}]}}{\sqrt{\numhorizons}} \notag\\
        \min_{k = 0, \hdots, \numhorizons}  \langle \excess(\price[][k+0.5]),  \basis[\good] \rangle \leq  \frac{2 \sqrt{2} (1 + \kernelsmooth)}{\learnrate[ ][ ]} \frac{\sqrt{\max_{\price \in \simplex}\divergence[\kernel][{\price}][{\price[][0]}]}}{\sqrt{\numhorizons}} && \forall \good \in \goods \notag\\
        \min_{k = 0, \hdots, \numhorizons}  \excess[\good](\price[][k+0.5]) \leq  \frac{2 \sqrt{2} (1 + \kernelsmooth)}{\learnrate[ ][ ]} \frac{\sqrt{\max_{\price \in \simplex}\divergence[\kernel][{\price}][{\price[][0]}]}}{\sqrt{\numhorizons}} && \forall \good \in \goods \label{eq:we_convergence_bound}
    \end{align}

    Now, let $\min_{k = 0, \hdots, \numhorizons}  \excess[\good](\price[][k+0.5]) \leq \frac{2 \sqrt{2} (1 + \kernelsmooth)}{\learnrate[ ][ ]} \frac{\sqrt{\max_{\price \in \simplex}\divergence[\kernel][{\price}][{\price[][0]}]}}{\sqrt{\numhorizons}} \leq \varepsilon$, we have: 
    \begin{align*}
        \frac{2 \sqrt{2}(1 + \kernelsmooth)}{\learnrate[ ][ ]} \frac{\sqrt{\max_{\price \in \simplex}\divergence[\kernel][{\price}][{\price[][0]}]}}{\sqrt{\numhorizons}} \leq \varepsilon\\
        \frac{2 \sqrt{2}(1 + \kernelsmooth)}{\learnrate[ ][ ]} \frac{\sqrt{\max_{\price \in \simplex}\divergence[\kernel][{\price}][{\price[][0]}]}}{\varepsilon} \leq \sqrt{\numhorizons}\\
        \frac{8(1 + \kernelsmooth)^2}{\learnrate[ ][ ]^2} \frac{\max_{\price \in \simplex}\divergence[\kernel][{\price}][{\price[][0]}]}{\varepsilon^2} \leq \numhorizons
    \end{align*}

    Further, by \Cref{thm:mirror_extragradient_global_convergence}, letting $\bestiter[{\price}][\numhorizons] \in \argmin_{\vartuple[][][k+0.5] : k = 0, \hdots, \numhorizons} \|\price[][k+0.5] - \price[][k]\|$, for some $\numhorizons \in O(\frac{\lsmooth}{\varepsilon^2})$, 
    $\bestiter[{\price}][\numhorizons]$ is a $\varepsilon$-strong solution of $([0, 1]^\numgoods, -\excessset)$. Then, by \Cref{lemma:approx_svi_eq_approx_we}, $\bestiter[{\price}][\numhorizons]$ is also a $\varepsilon$-Walrasian equilibrium.
    
    Finally, going back to \Cref{eq:we_convergence_bound}, and taking $\numhorizons \to \infty$, we obtain the last part of the theorem.
    \fi
\end{proof}


\lemmabregmancontelastic*
\begin{proof}[Proof of \Cref{lemma:bregman_cont_elastic}]
    By the assumption of the theorem, we have for all $\price, \otherprice \in \simplex[\numgoods]$, $\good, k \in \goods$:
    \begin{align*}
        \left|\frac{\demandfunc[\good](\otherprice) - \demandfunc[\good](\price)}{\demandfunc[\good](\price)} \frac{\price[k]}{\otherprice[k] - \price[k]} \right| &\leq \elastic\\
        \frac{\left|\demandfunc[\good](\otherprice) - \demandfunc[\good](\price)\right| }{|\demandfunc[\good](\price)|} \frac{|\price[k]|}{\otherprice[k] - \price[k]} 
        &\leq \elastic\\
        \left|\demandfunc[\good](\otherprice) - \demandfunc[\good](\price)  \right| &\leq \frac{\elastic |\demandfunc[\good](\price)|}{\price[k]} |\otherprice[k] - \price[k]|\\
        \left|\demandfunc[\good](\otherprice) - \demandfunc[\good](\price)  \right|^2 &\leq \frac{\elastic^2 |\demandfunc[\good](\price)|^2}{(\price[k])^2} \left|\otherprice[k] - \price[k]\right|^2
    \end{align*}
    Summing up over $\good \in \goods$, we have for all $k \in \goods$:
    \begin{align*}
        \left\|\demandfunc(\otherprice) - \demandfunc(\price)  \right\|^2 &\leq \frac{\elastic^2 \|\demandfunc(\price)\|^2}{(\price[k])^2} |\otherprice[k] - \price[k]|^2\\
        &\leq \frac{\elastic^2 \|\demandfunc(\price)\|^2}{(\price[k])^2} \left\|\otherprice - \price\right\|^2
    \end{align*}

    Since $\kernel$ is 1-strongly-convex, $\forall \vartuple, \othervartuple \in \set$, $\divergence[\kernel][{\vartuple}][{\othervartuple}] \geq \nicefrac{1}{2}\| \vartuple - \othervartuple\|^2$, hence, we have:
    \begin{align*}
        \left\|\demandfunc(\otherprice) - \demandfunc(\price)  \right\|^2 \leq \frac{2\elastic^2 \|\demandfunc(\price)\|^2}{(\price[k])^2} \divergence[\kernel](\otherprice,\price) 
    \end{align*}

    Taking the square root of both sides and the taking a minimum over $k \in \goods$, we have:
    \begin{align*}
        \left\|\demandfunc(\otherprice) - \demandfunc(\price)  \right\| &\leq \min_{k \in \goods}\frac{\elastic \|\demandfunc(\price)\|}{\price[k]} \sqrt{2\divergence[\kernel](\otherprice,\price)} \\
        &= \frac{\elastic \|\demandfunc(\price)\|}{\max_{k \in \goods} \price[k]} \sqrt{2\divergence[\kernel](\otherprice,\price)}\\
        &= \frac{\elastic \|\demandfunc(\price)\|}{\|\price\|_\infty} \sqrt{2\divergence[\kernel](\otherprice,\price)}
    \end{align*}

    By a similar argument, note that we also have:
    \begin{align*}
        \left\|\supplyfunc(\otherprice) - \supplyfunc(\price)  \right\| &\leq \frac{\elastic \|\supplyfunc(\price)\|}{\|\price\|_\infty} \sqrt{2\divergence[\kernel](\otherprice,\price)}
    \end{align*}

    Combining the two above bounds, we then have:
    \begin{align*}
        \| \excess(\otherprice) - \excess(\price) \| &=  \| \demandfunc(\otherprice)  - \supplyfunc(\otherprice) - \demandfunc(\price) + \supplyfunc(\price) \|\\
        &\leq \|\demandfunc(\otherprice) - \demandfunc(\price) \| + \| \supplyfunc(\otherprice) - \supplyfunc(\price) \|\\
        &\leq \frac{\elastic \|\demandfunc(\price)\|}{\|\price\|_\infty} \sqrt{2 \divergence[\kernel](\otherprice,\price)}  + \frac{\elastic \|\supplyfunc(\price)\|}{\|\price\|_\infty} \sqrt{2 \divergence[\kernel](\otherprice,\price)}\\
        % &\leq \frac{\elastic \|\excess\|_\infty}{ \max_{k \in \goods} (\price[k])} \sqrt{2 \divergence[\kernel](\otherprice,\price)}  +  \frac{\elastic \|\excess\|_\infty}{\max_{k \in \goods}(\price[k])} \sqrt{2\divergence[\kernel](\otherprice,\price)}\\
        &\leq \frac{\elastic \|\demandfunc(\price)\| + \|\supplyfunc(\price)\|}{\|\price\|_\infty} \sqrt{2 \divergence[\kernel](\otherprice,\price)}  
    \end{align*}
    
    Squaring both sides and re-organizing expressions, we then have:
    \begin{align*}
        \frac{1}{2}\| \excess(\otherprice) - \excess(\price) \|^2 \leq \left(\frac{\elastic \left(\|\demandfunc(\price)\| + \|\supplyfunc(\price)\| \right)}{\|\price\|_\infty} \right)^2 \divergence[\kernel](\otherprice,\price)
    \end{align*}
\end{proof}

\subsection{Results for Variationally Stable Economies}


\thmmirrorextratatonnvarstable*
\begin{proof}[Proof of \Cref{thm:mirror_extratatonn_var_stable}]
       Since $(\numgoods, \excess)$ is variationally stable on $\simplex[\numgoods]$, the VI $(\simplex[\numgoods], -\excessset)$ satisfies the Minty condition. In addition, since by the assumption of the theorem the economy is $\lelastic$-elastic and $\lbounded$-bounded, by \Cref{lemma:bregman_cont_elastic}, $\excess$ is $\left(2\numgoods \elastic \lbounded \right)$-Bregman-continuous on $\simplex[\numgoods]$. That is, we have \begin{align*}
        \frac{1}{2}\| \excess(\otherprice) - \excess(\price) \|^2 &\leq \left(\frac{\elastic \left(\|\demandfunc(\price)\| + \|\supplyfunc(\price)\| \right)}{\|\price\|_\infty} \right)^2 \divergence[\kernel](\otherprice,\price)\\
        &\leq \max_{\price \in \simplex[\numgoods]} \left(\frac{\elastic \left(\|\demandfunc(\price)\| + \|\supplyfunc(\price)\| \right)}{\|\price\|_\infty} \right)^2 \divergence[\kernel](\otherprice,\price)\\
        &\leq  \left(\frac{\elastic \left(\|\demandfunc\|_\infty + \|\supplyfunc\|_\infty \right)}{\min_{\price \in \simplex[\numgoods]} \|\price\|_\infty} \right)^2 \divergence[\kernel](\otherprice,\price)\\
        &\leq \left(\frac{2\elastic \lbounded }{\frac{1}{\numgoods}} \right)^2 \divergence[\kernel](\otherprice,\price)\\
        &\leq \left(2\numgoods \elastic \lbounded  \right)^2 \divergence[\kernel](\otherprice,\price) \enspace . 
       \end{align*}

       That is, 

        Suppose that under the assumptions of the theorem the mirror generates the sequence of prices $\{\price[][\numhorizon], \price[][\numhorizon + 0.5] \}_{\numhorizon}$. Let $\bestiter[{\price}][\numhorizons] \in \argmin_{\vartuple[][][k+0.5] : k = 0, \hdots, \numhorizons} \divergence[\kernel](\price[][k+0.5], \price[][k])$. As the mirror \emph{extrat\^atonnement} process is simply the mirror extragradient method run on the VI $(\simplex[\numgoods], -\excessset)$, and the assumptions of \Cref{thm:mirror_extragradient_global_convergence} are satisfied and hence we have the following bound:

        \begin{align*}
             \min_{k = 0, \hdots, \numhorizons} \max_{\price \in \simplex} \langle -\excess(\price[][k+0.5]),  \price[][k+0.5] - \price \rangle &\leq  \frac{2 (1 + \kernelsmooth)\diam(\simplex[\numgoods])}{\learnrate[ ][ ]} \frac{\sqrt{\max_{\price \in \simplex}\divergence[\kernel][{\price}][{\price[][0]}]}}{\sqrt{\numhorizons}}\\
             \min_{k = 0, \hdots, \numhorizons} \max_{\price \in \simplex} \langle \excess(\price[][k+0.5]),  \price - \price[][k+0.5] \rangle &\leq  \frac{2 \sqrt{2}(1 + \kernelsmooth)}{\learnrate[ ][ ]} \frac{\sqrt{\max_{\price \in \simplex}\divergence[\kernel][{\price}][{\price[][0]}]}}{\sqrt{\numhorizons}}
        \end{align*}

        Further, $\lim_{\numhorizon} \price[][\numhorizon] = \lim_{\numhorizon \to \infty} \price[][][\numhorizon + 0.5] = \price[][][*]$ is a Walrasian equilibrium.

\end{proof}


\subsection{The Scarf Economy}


The following lemma states that any Scarf economy is a balanced economy.
\begin{lemma}[Scarf Economies are Balanced]
    The Scarf economy is a balanced economy which satisfies Walras' law, i.e., for all $\price \in \R^\numgoods_+$, $\price \cdot \excess^{\mathrm{scarf}}(\price) = 0$. Further, the set of Walrasian equilibrium of the Scarf economy $\excess^{\mathrm{scarf}}$ is given by $\we(\excess^{\mathrm{scarf}}) \doteq \{\lambda \ones[3] \mid \lambda > 0\}$.
\end{lemma}

\begin{proof}
    First, notice that the scarf economy is homogeneous of degree $0$. That is, for all $\lambda \geq 0$, we have:
    \begin{align*}
        \excess^{\mathrm{scarf}}(\lambda \price) \doteq \begin{pmatrix}
            \frac{\lambda\price[1]}{\lambda\price[1] + \lambda\price[2]} + \frac{\lambda\price[3]}{\lambda\price[1] + \lambda\price[3]} - 1\\
            \frac{\lambda\price[1]}{\lambda \price[1] + \lambda \price[2]} + \frac{\lambda \price[2]}{\lambda \price[2] + \lambda \price[3]} - 1\\
            \frac{\lambda \price[2]}{\lambda \price[2] + \lambda \price[3]} + \frac{\lambda \price[3]}{\lambda \price[1] + \lambda \price[3]} - 1
            \end{pmatrix} =  \begin{pmatrix}
            \frac{\price[1]}{\price[1] + \price[2]} + \frac{\price[3]}{\price[1] + \price[3]} - 1\\
            \frac{\price[1]}{\price[1] + \price[2]} + \frac{\price[2]}{\price[2] + \price[3]} - 1\\
            \frac{\price[2]}{\price[2] + \price[3]} + \frac{\price[3]}{\price[1] + \price[3]} - 1
            \end{pmatrix} 
            = \excess^{\mathrm{scarf}}(\price)
    \end{align*}

    Second, for all $\price \in \R^\numgoods$, notice we have:
    \begin{align*}
        \price \cdot \excess^{\mathrm{scarf}}(\price) &= \frac{\price[1][][2]}{\price[1] + \price[2]} + \frac{\price[1]\price[3]}{\price[1] + \price[3]} - \price[1] + 
            \frac{\price[1]\price[2]}{\price[1] + \price[2]} + \frac{\price[2][][2]}{\price[2] + \price[3]} - \price[2] +
            \frac{\price[2]\price[3]}{\price[2] + \price[3]} + \frac{\price[3][][2]}{\price[1] + \price[3]} - \price[3]\\
            &= \frac{\price[1][][2] + \price[1]\price[2]}{\price[1] + \price[2]} + \frac{\price[2][][2] + \price[2]\price[3]}{\price[2] + \price[3]} + \frac{\price[3][][2] + \price[1]\price[3]}{\price[1] + \price[3]} - \price[1] - \price[2] - \price[3]\\
            &= \frac{\price[1] (\price[1] + \price[2])}{\price[1] + \price[2]} + \frac{\price[2] (\price[2] + \price[3])}{\price[2] + \price[3]} + \frac{\price[3](\price[3] + \price[1])}{\price[1] + \price[3]} - \price[1] - \price[2] - \price[3]\\
            &= 0
    \end{align*}

    Finally, observe that for $\price[][][*] = \ones[\numgoods]$, we have $\excess^{\mathrm{scarf}}(\price[][][*]) = \zeros[\numgoods]$, and, we have $\price[][][*] \cdot \excess^{\mathrm{scarf}}(\price[][][*])$. Notice that this equilibrium is unique up to positive scaling since if the price of any commodity is changed from $\price[][][*]$, then the excess demand for another commodity is guaranteed to decrease while the excess demand of some other commodity increases. 
\end{proof}



\begin{restatable}[Variational stability and Bregman-continuity of the Scarf Economy]{lemma}{lemmascarfvarstablebregcont}\label{lemma:scarf_var_stable_breg_cont}
        Any Scarf economy $\excess^{\mathrm{scarf}}$ is variationally stable on any non-empty price space $\pricespace \subseteq \R^3_+$. Further, for any $\underline{\price[ ][]} \in (0, \nicefrac{1}{3})$ and any $1$-strongly-convex kernel function $\kernel: \R^3_+ \to \R$, the Scarf economy $\excess^{\mathrm{scarf}}$ is variationally stable and $(\nicefrac{3}{\underline{\price[ ][]}^2}, \kernel)$-Bregman-continuous on $[\underline{\price[ ][]}, 1]^3$.
\end{restatable}
\begin{proof}[Proof of \Cref{lemma:scarf_var_stable_breg_cont}]
    \textbf{Part 1: Variational stability on $\simplex[\numgoods]$.}
    
    We claim that for $\price[][][*] = \left(\nicefrac{1}{3}, \nicefrac{1}{3}, \nicefrac{1}{3} \right)$, the Scarf economy is variationally stable on $\simplex[\numgoods]$, i.e., we have for all prices $\price \in \simplex[3]$ and all $\excess(\price) \in \excessset(\price)$, we have $\innerprod[{ \excess^{\mathrm{scarf}}(\price)}][{\price[][][*] - \price}] \geq 0$.

    First, notice that expanding the expression $\innerprod[{ \excess^{\mathrm{scarf}}(\price)}][{\price[][][*] - \price}]$ we have for all $\price \in \pricespace$:
    \begin{align*}
        \innerprod[{ \excess^{\mathrm{scarf}}(\price)}][{\price[][][*] - \price}] &= \innerprod[{ \excess^{\mathrm{scarf}}(\price)}][{\price[][][*] }] - \underbrace{\innerprod[{ \excess^{\mathrm{scarf}}(\price)}][{\price}]}_{= 0} \\
        &= \innerprod[{ \excess^{\mathrm{scarf}}(\price)}][{\price[][][*] }]\\
        &= 2\frac{\price[1]}{\price[1] + \price[2]} + 2\frac{\price[2] }{\price[2] + \price[3]} + 2\frac{\price[3]}{\price[1] + \price[3]} - \underbrace{\nicefrac{1}{3} - \nicefrac{1}{3} - \nicefrac{1}{3}}_{ = -1}\\
        &= 2\frac{\price[1]}{\price[1] + \price[2]} + 2\frac{\price[2] }{\price[2] + \price[3]} + 2\frac{\price[3]}{\price[1] + \price[3]} - 1
    \end{align*}

    We proceed by proof by cases.

    \textbf{Case 1 $\price[1] \geq \price[2]$.}
    \begin{align*}
        \innerprod[{ \excess^{\mathrm{scarf}}(\price)}][{\price[][][*] - \price}] &= 2\frac{\price[1]}{\price[1] + \price[2]} + \underbrace{2\frac{\price[2] }{\price[2] + \price[3]}}_{\geq 0} + \underbrace{2 \frac{\price[3]}{\price[1] + \price[3]}}_{\geq 0} - 1\\
        &\geq  2\frac{\price[1]}{\price[1] + \underbrace{\price[2]}_{\leq \price[1]}} - 1\\
        &\geq  2\frac{\price[1]}{\price[1] + \price[1]}  - 1\\
        &=  1 - 1\\
        &= 0
    \end{align*}
    \textbf{Case 2 $\price[1] < \price[2]$.}
    \begin{align*}
        \innerprod[{ \excess^{\mathrm{scarf}}(\price)}][{\price[][][*] - \price}]
        &= \underbrace{2\frac{\price[1]}{\price[1] + \price[2]}}_{\geq 0} + 2\frac{\price[2] }{\price[2] + \price[3]} + 2 \frac{\price[3]}{\price[1] + \price[3]} - 1\\
        &= 2\frac{\price[2] }{\price[2] + \price[3]} + 2 \frac{\price[3]}{\underbrace{\price[1]}_{< \price[2]} + \price[3]} - 1\\
        &= 2\frac{\price[2] }{\price[2] + \price[3]} + 2 \frac{\price[3]}{\price[2] + \price[3]} - 1\\
        &= 2\frac{\price[2] + \price[3]}{\price[2] + \price[3]} - 1\\
        &= 2-1\\
        &\geq 0
        % &= 2\frac{\price[1]}{\underbrace{\price[1]}_{\leq \price[2]} + \price[2]} + 2\frac{\price[2] }{\price[2] + \price[3]} + 2 \frac{\price[3]}{\price[1] + \price[3]} - 1\\
    \end{align*}

    Hence, we must have: $\innerprod[{ \excess^{\mathrm{scarf}}(\price)}][{\price[][][*] - \price}] \geq 0$, and the Scarf economy is variationally stable on $\simplex[\numgoods]$. 

    \textbf{Part 2: Variational stability and Bregman-continuity on $[\underline{\price[ ][]}, 1]^3$.}
    First, for variational stability on $[\underline{\price[ ][]}, 1]$, observe that the proof provided in part 1 applies directly by replacing $\simplex[\numgoods]$ by $[\underline{\price[ ][]}, 1]$. 
    
    Second, notice that the excess demand is differentiable with its Jacobian matrix given by:
    % \begin{align*}
    %     \excess^{\mathrm{scarf}}(\price) \doteq \begin{pmatrix}
    %         \frac{\price[1]}{\price[1] + \price[2]} + \frac{\price[3]}{\price[1] + \price[3]} - 1\\
    %         \frac{\price[1]}{\price[1] + \price[2]} + \frac{\price[2]}{\price[2] + \price[3]} - 1\\
    %         \frac{\price[2]}{\price[2] + \price[3]} + \frac{\price[3]}{\price[1] + \price[3]} - 1
    %         \end{pmatrix}
    % \end{align*}
    % 
    \begin{align*}
        \grad \excess(\price) = 
        \begin{bmatrix}
            -\frac{\price[1]}{(\price[1] + \price[2])^2} - \frac{\price[3]}{(\price[1] + \price[3])^2} 
            & -\frac{\price[1]}{(\price[1] + \price[2])^2} 
            & -\frac{\price[3]}{(\price[1] + \price[3])^2} \\[10pt]
            -\frac{\price[1]}{(\price[1] + \price[2])^2} 
            & -\frac{\price[1]}{(\price[1] + \price[2])^2} - \frac{\price[3]}{(\price[2] + \price[3])^2} 
            & -\frac{\price[2]}{(\price[2] + \price[3])^2} \\[10pt]
            -\frac{\price[3]}{(\price[1] + \price[3])^2} 
            & -\frac{\price[3]}{(\price[2] + \price[3])^2}
            & -\frac{\price[2]}{(\price[2] + \price[3])^2} - \frac{\price[3]}{(\price[1] + \price[3])^2}
        \end{bmatrix}      
    \end{align*}
    
    Thus, the Jacobian consists of entries of the form of $f(x,y) \doteq \frac{x}{(x+y)^2}$. For $x, y \in [\underline{\price[ ][]}, 1]$, we then have $|f(x,y)| \leq \frac{1}{4 \underline{\price[ ][]}^2}$. This means that the absolute value of the off diagonal entries of $\grad \excess(\price)$ are bounded by $\frac{1}{4 \underline{\price[ ][]}^2}$, while the diagonal entries are bounded by $\frac{1}{2\underline{\price[ ][]}^2}$. Hence, for all $\price \in [\underline{\price[ ][]}, 1]^3$, we have $\|\grad \excess(\price)\|_1 \leq \frac{3}{2\underline{\price[ ][]}^2} + \frac{6}{4 \underline{\price[ ][]}^2} = \frac{3}{\underline{\price[ ][]}^2}$. Then, by the mean value theorem, $\excess^{\mathrm{scarf}}$ is $\frac{3}{\underline{\price[ ][]}^2}$-Lipschitz-continuous on $[\underline{\price[ ][]}, 1]^3$, i.e., for all $\price, \otherprice \in [\underline{\price[ ][]}, 1]^3$, $\|\excess(\price) - \excess(\otherprice)\| \leq \frac{3}{\underline{\price[ ][]}^2} \| \otherprice - \price \|$. Now, by the assumptions of the theorem, since $\kernel$ is $1$-strongly-convex, we have for all $\price, \otherprice \in \R^3_+$, $\frac{1}{2}\|\price - \otherprice \|^2 \leq \divergence[\kernel](\price, \otherprice)$. Hence, this implies that for all for all $\price, \otherprice \in [\underline{\price[ ][]}, 1]^3$, 
    \begin{align*}
        \nicefrac{1}{2}\|\excess(\price) - \excess(\otherprice)\|^2 &\leq \frac{1}{2} \left(\frac{3}{\underline{\price[ ][]}^2} \right)^2   \| \price - \otherprice \|^2\\ 
        &\leq \left(\frac{3}{\underline{\price[ ][]}^2} \right)^2 \divergence[\kernel](\price, \otherprice)
        % &\leq \left(\frac{3}{2\underline{\price[ ][]}^2} \right)^2 \divergence[\kernel](\price, \otherprice)    
    \end{align*}
\end{proof}