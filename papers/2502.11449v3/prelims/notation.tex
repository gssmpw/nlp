\paragraph{Notation.} We use caligraphic uppercase letters to denote sets (e.g., $\calX$), bold uppercase letters to denote matrices (e.g., $\allocation$), bold lowercase letters to denote vectors (e.g., $\price$), lowercase letters to denote scalar quantities (e.g., $x$). 
We denote the $i$th row vector of a matrix (e.g., $\allocation$) by the corresponding bold lowercase letter with subscript $i$ (e.g., $\allocation[\buyer])$. 
Similarly, we denote the $j$th entry of a vector (e.g., $\price$ or $\allocation[\buyer]$) by the corresponding lowercase letter with subscript $j$ (e.g., $\price[\good]$ or $\allocation[\buyer][\good]$).
We denote functions by a letter determined by the value of the function, e.g., $f$ if the mapping is scalar valued, $\f$ if the mapping is vector valued, and $\calF$ if the mapping is set valued (i.e., $\calF$ is a correspondence).
We denote the set $\left\{1, \hdots, n\right\}$ by $[n]$, the set of natural numbers by $\N$, and the set of real numbers by $\R$. 
We denote the positive and strictly positive elements of a set using a $+$ or $++$ subscript, respectively, e.g., $\R_+$ and $\R_{++}$. 

For any $n \in \N$, we denote the  $n$-dimensional vector of zeros and ones by $\zeros[n]$ and $\ones[n]$, respectively, and the $i^{th}$ basis vector in $\R^n$ by $\basis[i]$.
We let $\simplex[n] = \{\x \in \R_+^n \mid \sum_{i = 1}^n x_i = 1 \}$ denote the unit simplex in $\R^n$.
Unless otherwise noted, we denote the 2-norm $\| \cdot \| \doteq \| \cdot \|_2$.
Finally, we denote the Euclidean projection operator onto a set $C$ by $\project[C]$, i.e., $\project[C](\x) \doteq \argmin_{\y \in C} \left\|\x - \y \right\|^2$. Given a metric space $(\metricspace, \metric)$ and $\varepsilon \geq 0$, we write $\closedball[\varepsilon][\var] = \{ \var[\prime] \in \metricspace \mid \metric(\var, \var[\prime]) \leq \varepsilon \}$ to denote the closed $\varepsilon$-ball centered at $\var \in \metricspace$.
The multiplication of a scalar and a set is defined as the Minkowksi product, i.e., for all $a \in \R$ and $\set \subseteq \R^\numgoods$, we define $a \set \doteq \{a \vartuple \mid \vartuple \in \set \}$.
