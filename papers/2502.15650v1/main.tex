% Preamble
\documentclass[11pt]{article}

% Packages
\usepackage{amsmath}
\usepackage{amsthm}
\usepackage{amssymb}
\usepackage{amsfonts}
\usepackage{forest}
\usepackage{mathrsfs}
\usepackage{bbm}
\usepackage{bbold}
\usepackage{setspace}
\usepackage{thmtools}
\usepackage{thm-restate}
\usepackage{fullpage}
\usepackage{tcolorbox}
\usepackage[all]{xy}
\usepackage[title, titletoc]{appendix}
\usepackage[ruled,linesnumbered]{algorithm2e}
\usepackage{mathalfa}
\usepackage{amsbsy}
\usepackage{comment}
\usepackage{xr}
\usepackage{xifthen}
\usepackage{hyperref}
\usepackage{enumitem}
\usepackage{mathtools}
%\usepackage{latex}
\usepackage{mathalpha}
\usepackage{tikz-cd}
%\usepackage{stmaryrd}
\usepackage{bm}
%\usepackage{enumerate}
\usepackage{empheq}
\usepackage{hypbmsec}


\newtheorem{theorem}{Theorem}[section]
\newtheorem{definition}[theorem]{Definition}
\newtheorem{lemma}[theorem]{Lemma}
\newtheorem{claim}[theorem]{Claim}
\newtheorem{question}[theorem]{Question}
\newtheorem{proposition}[theorem]{Proposition}
\newtheorem{fact}[theorem]{Fact}
\newtheorem{corollary}[theorem]{Corollary}
\newtheorem{observation}[theorem]{Observation}
\newtheorem{notation}[theorem]{Notation}
\newtheorem{example}[theorem]{Example}
\newtheorem{remark}[theorem]{Remark}
\newtheorem{note}[theorem]{Note}
\newtheorem*{notation*}{Notation}
\newtheorem*{lemma*}{Lemma}
\newtheorem*{proposition*}{Proposition}
\newtheorem*{note*}{Note}
\newtheorem*{theorem*}{Theorem}
\newtheorem*{corollary*}{Corollary}




% Logic

\newcommand{\innerifequals}[3]{\ifthenelse{\equal{#1}{#2}}{#3}{}}
\newenvironment{switch}[1]{\newcommand{\case}{\innerifequals{#1}}}{}
\newcommand{\definedas}[0]{\coloneqq}

% Linear Algebra
\DeclareMathOperator*{\im}{Im}
\newcommand{\vectorfield}[0]{V}
% \renewcommand{\vec}[1]{\overrightarrow{#1}}


\newcommand{\alignparens}[2]{\left( #1 \vphantom{#2} \right. & \left. \vphantom{#1} #2 \right)}
\newcommand{\alignset}[2]{\left\{ #1 \vphantom{#2} \suchthat \right. & \left. \vphantom{#1} #2 \right\}}
\newcommand{\multilinenorm}[2]{\left\| #1 \vphantom{#2} \right. \\ & \quad \left. \vphantom{#1} #2 \right\|}

\newcommand{\parens}[1]{\left( #1 \right)}
\newcommand{\sparens}[1]{\left[ #1 \right]}
\newcommand{\leftsparens}[1]{\left[ #1 \right.}
\newcommand{\rightsparens}[1]{\left. #1 \right]}
\newcommand{\innerprod}[1]{\left< #1 \right>}

\newcommand{\rhalfopen}[2]{\ensuremath{[#1,#2)}}
\newcommand{\lhalfopen}[2]{\ensuremath{(#1,#2]}}


% Probability
\newcommand{\expectation}[2]{\mathbbm{E}_{#1} \sparens{#2}}
% Set theory
\newcommand{\set}[1]{\left\{ #1 \right\}}
\newcommand{\suchthat}[0]{\middle|}

% Proof parameters
\newcommand{\basefield}[0]{\mathbbm{F}}
\newcommand{\blocklength}[0]{n}
\newcommand{\field}[0]{\basefield^{\blocklength}}
\newcommand{\titlefield}[0]{\texorpdfstring{$\field$}{Fn}}
\newcommand{\degree}[0]{d}
\newcommand{\basefieldsize}[0]{p}

\newcommand{\genfinitefield}[0]{\mathbbm{F}_{q}}

% Constants and indexes
\newcommand{\series}[3]{#1_{#2},\ldots,#1_{#3}}
\newcommand{\seriesdep}[4]{#1_{#3}(#2),\ldots,#1_{#4}(#2)}

% bias
\newcommand{\charfunc}[1]{e \sparens{#1}}
\newcommand{\genbias}[0]{\mu}

\newcommand{\genprime}[0]{p}
% Sets
\NewDocumentCommand{\genset}{o}{\IfNoValueTF{#1}{A}{
    \begin{switch}{#1}
        \case{1}{A}
        \case{2}{B}
        \case{3}{C}
    \end{switch}
}}
% unctions
\NewDocumentCommand{\onvarfunc}{o}{\IfNoValueTF{#1}{f}{
    \begin{switch}{#1}
        \case{1}{f}
        \case{2}{g}
        \case{3}{h}
    \end{switch}
}}
\NewDocumentCommand{\onvarfuncset}{o}{\IfNoValueTF{#1}{\mathfrak{f}}{
    \begin{switch}{#1}
        \case{1}{\mathfrak{f}}
        \case{2}{\mathfrak{g}}
        \case{3}{\mathfrak{h}}
    \end{switch}
}}

\NewDocumentCommand{\genfunc}{o}{\IfNoValueTF{#1}{F}{
    \begin{switch}{#1}
        \case{1}{F}
        \case{2}{G}
        \case{3}{H}
    \end{switch}
}}

\NewDocumentCommand{\genfuncset}{o}{\IfNoValueTF{#1}{\mathfrak{F}}{
    \begin{switch}{#1}
        \case{1}{\mathfrak{F}}
        \case{2}{\mathfrak{G}}
        \case{3}{\mathfrak{H}}
    \end{switch}
}}

\newcommand{\funcdef}[3]{#1:#2\rightarrow#3}
\newcommand{\restrictfunc}[2]{#1|_{#2}}

% Polynomials
\NewDocumentCommand{\genpoly}{o}{\IfNoValueTF{#1}{P}{
    \begin{switch}{#1}
        \case{1}{P}
        \case{2}{Q}
        \case{3}{H}
    \end{switch}
}}
\NewDocumentCommand{\genpolyset}{o}{\IfNoValueTF{#1}{\mathcal{P}}{
    \begin{switch}{#1}
        \case{1}{\mathcal{P}}
        \case{2}{\mathcal{Q}}
        \case{3}{\mathcal{H}}
    \end{switch}
}}

\newcommand{\factor}[0]{\mathcal{B}}
\newcommand{\semrefine}[0]{\succeq_{\text{sem}}}
\newcommand{\semrefineex}[1]{\succeq_{\text{sem}|#1}}
\newcommand{\relsemrefine}[1]{\succeq^{#1}_{\text{sem}}}
\newcommand{\relsemrefineex}[1]{\succeq^{#1}_{\text{sem}}}
\newcommand{\synrefine}[0]{\succeq_{\text{syn}}}
\newcommand{\cubesfactor}[3]{C_{#1,#2}({#3})}

\newcommand{\allpolyset}[3]{Poly_{#1}(#2\rightarrow#3)}

\newcommand{\cube}[2]{(#1|#2)}
\newcommand{\almostcube}[2]{(#1|#2)^{\prime}}
\newcommand{\cubes}[2]{C_{#1}(#2)}
\newcommand{\almostcubes}[3]{C^{\prime}_{{#1},{#2}}({#3})}


\NewDocumentCommand{\lowdegpoly}{o}{\IfNoValueTF{#1}{\alpha}{
    \begin{switch}{#1}
        \case{1}{\alpha}
        \case{2}{\beta}
    \end{switch}
}}

\newcommand{\depth}[1]{depth(#1)}


% Constant Sets
\newcommand{\torus}[0]{\mathbbm{T}}
\newcommand{\naturalnumbersset}[0]{\mathbbm{N}}
\newcommand{\realnumbersset}[0]{\mathbbm{R}}
\newcommand{\complextnumberset}[0]{\mathbbm{C}}


% Varieties
\NewDocumentCommand{\varpoly}{o}{\IfNoValueTF{#1}{L}{
    \begin{switch}{#1}
        \case{1}{L}
        \case{2}{M}
    \end{switch}
}}
\NewDocumentCommand{\varpolyset}{o}{\IfNoValueTF{#1}{\tilde{\mathcal{L}}}{
    \begin{switch}{#1}
        \case{1}{\matchcal{L}}
        \case{2}{\matchcal{M}}
    \end{switch}
}}

\newcommand{\relativeremainder}[1]{\overline{#1}}
\newcommand{\remainderpoly}[0]{R}

\NewDocumentCommand{\onvarpoly}{o}{\IfNoValueTF{#1}{p}{
    \begin{switch}{#1}
        \case{1}{p}
        \case{2}{q}
        \case{3}{h}
    \end{switch}
}}
\NewDocumentCommand{\onvarpolyset}{o}{\IfNoValueTF{#1}{\mathfrak{p}}{
    \begin{switch}{#1}
        \case{1}{\mathfrak{p}}
        \case{2}{\mathfrak{q}}
        \case{3}{\mathfrak{h}}
    \end{switch}
}}

% Specific variety notations
\newcommand{\variety}{{\tilde{X}}}
\newcommand{\titlevariety}{\texorpdfstring{$\variety$}{X}}
%\newcommand{\varietypolycount}{\tilde{c}}
\newcommand{\varietypolycount}{{\tilde{c}}}
\newcommand{\varietydeg}[0]{\tilde{\degree}}
\newcommand{\varietycodeminsion}[0]{\tilde{L}}
\newcommand{\varietyminepsilon}[1]{\tilde{\epsilon}_{#1}}


\newcommand{\lift}[1]{\widehat{#1}}
\newcommand{\homopart}[1]{\mathrm{h}({#1})}
\newcommand{\derivative}[1]{{#1}^{\prime}}
\newcommand{\directionderivative}[0]{D}
\newcommand{\derivativepoly}[1]{\vartheta{#1}}

\newcommand{\zerofunc}[1]{Z \parens{#1}}
\newcommand{\existfunc}[1]{1_{#1}}

% Factors


% Properties
\newcommand{\rank}[1]{rank \parens{#1}}
\newcommand{\drank}[2]{rank_{#1} \parens{#2}}
\newcommand{\rankex}[2]{rank_{#1} \parens{#2}}
\newcommand{\schmrank}[1]{schmrank \parens{#1}}
\newcommand{\lemmarank}[1]{rank^{#1}}
\newcommand{\relschmrank}[2]{shcmrank_{#1}\parens{{#2}}}
\newcommand{\relrank}[2]{rank_{#1} \parens{#2}}
\newcommand{\drelrank}[3]{rank_{{#1}, {#2}} \parens{#3}}


%Constants Notations
\newcommand{\rankval}[0]{{{r}}}
\newcommand{\rankfunc}[0]{{{{r}}}}
\newcommand{\rankbiasfunc}[0]{{\tilde{r}}}
\newcommand{\varietyrankval}[0]{{\bar{r}}}
\newcommand{\dominatedrankfunc}[0]{{r_{\square}}}
\newcommand{\epsilonlimitedrankbias}[0]{{\tilde{\epsilon}}}
\newcommand{\constlimiteddominated}[0]{{c_{\square}}}


\newcommand{\bias}[1]{bias({#1})}
\newcommand{\relbias}[2]{bias_{#1}{({#2})}}

% Norms
\newcommand{\norm}[1]{\left\lVert{#1}\right\rVert}
\newcommand{\abs}[1]{\left| #1 \right|}
\newcommand{\gowersnorm}[2]{{\norm{#2}}_{U^{#1}}}
\newcommand{\infnorm}[1]{{\norm{#1}}_{\infty}}
\newcommand{\dist}[1]{dist \parens{#1}}


%Probability
\newcommand{\pr}[1]{\Pr\left[ #1 \right]}
\newcommand{\prex}[2]{\Pr_{#1}\left[ #2 \right]}
\newcommand\given[1][]{\:#1\vert\:}

% Constraints
\newcommand{\linearformset}[0]{L}
\newcommand{\linearformsetvaribables}[0]{\tilde{\constraintsset}}
\NewDocumentCommand{\linearform}{o}{\IfNoValueTF{#1}{l}{
    \begin{switch}{#1}
        \case{0}{l}
        \case{1}{r}
    \end{switch}
}}
\newcommand{\uniformityerror}[0]{\tau}
\newcommand{\uniformitcomplexity}[0]{\xi}
\newcommand{\density}[0]{\delta}


\newcommand{\gencode}{\mathfrak{C}}
\newcommand{\generatormatrix}{\boldsymbol{G}}
\newcommand{\paritymatrix}{\boldsymbol{H}}

\newcommand{\quotientcode}{\gencode_{\variety}}

\newcommand{\reedmullercode}[3]{RM_{#1}(#2, #3)}
\newcommand{\reedmullercodeex}[3]{RM_{#1, #2}(#3)}
\newcommand{\listpolycount}[4]{\ell_{#1, #2}(#3, #4)}
\newcommand{\listdecodingradiusex}[3]{LDR_{#1, #2}(#3)}
\newcommand{\normalizedcodedistance}[2]{\delta_{#1}(#2)}
\newcommand{\normalizedcodedistanceex}[3]{\delta_{#1, #2}(#3)}
\newcommand{\trycommand}[3][2]{(#2 + #3)^#1}

\DeclareRobustCommand{\topbot}{\genfrac{}{}{0pt}{}}



\title{List Decoding Quotient Reed-Muller Codes}
\author{
Omri Gotlib
\footnote{Department of Computer Science, Bar-Ilan University, Email: gotlib.omri@gmail.com.}
\and
Tali Kaufman
\footnote{Department of Computer Science, Bar-Ilan University, Email: kaufmant@mit.edu, supported by ISF.}
\and
Shachar Lovett
\footnote{Department of Computer Science and Engineering, UC San Diego, Email: shachar.lovett@gmail.com, supported by NSF award 2425349 and a Simons investigator award.}
}


% Document
\begin{document}
    \maketitle
    \begin{abstract}
        Reed-Muller codes consist of evaluations of
        $\blocklength$-variate polynomials over a finite field $\basefield$ with degree at most $\degree$.
        Much like every linear code, Reed-Muller codes can be characterized by constraints, where a codeword is valid if and only if it satisfies all \emph{degree-$\degree$} constraints.

        For a subset $\variety \subseteq \field$,
        we introduce the notion of \emph{$\variety$-quotient} Reed-Muller code.
        A function $\funcdef{\genfunc}{\variety}{\basefield}$ is a valid codeword in the quotient code if it satisfies all the constraints of degree-$\degree$ polynomials \emph{lying in $\variety$}.
        This gives rise to a novel phenomenon: a quotient codeword may have \emph{many} extensions to original codewords.
        This weakens the connection between original codewords and quotient codewords which introduces a richer range of behaviors along with substantial new challenges.

        Our goal is to answer the following question: what properties of $\variety$ will imply that the quotient code inherits its distance and list-decoding radius from the original code?
        \newline
        We address this question using techniques developed by Bhowmick and Lovett~\cite{bhowmick2014list},
        identifying key properties of $\field$ used in their proof and extending them to general subsets $\variety \subseteq \field$.
        By introducing a new tool, we overcome the novel challenge in analyzing the quotient code that arises from the weak connection between original and quotient codewords.
        This enables us to apply known results from additive combinatorics and algebraic geometry~\cite{kazhdan2018polynomial, kazhdan2019extendingweaklypolynomialfunctions, lampert2021relative}
        to show that when $\variety$ is a \emph{high rank variety}, $\variety$-quotient Reed-Muller codes inherit the distance and list-decoding parameters from the original Reed-Muller codes.
    \end{abstract}
    In recent years, the growing use of deep learning across various fields~\cite{yang2021intelligent, macas2022survey, grigorescu2020survey} has highlighted the need for efficient, safe, and private deployment methods, driving the adoption of edge computing. By bringing computation closer to the data source, edge computing reduces latency, saves bandwidth, and enhances privacy and security. However, deploying deep neural networks (DNNs) on edge devices poses challenges due to limited computational resources and energy constraints~\cite{alvear2023edge}.

To tackle these challenges, researchers have focused on improving the accuracy and efficiency of DNNs for edge applications. Traditional approaches, such as manual model optimization~\cite{Marchisio_2018IJCNN_PruNet, vadera2021methods, Hanif_2022MICPRO_EfficientEmbeddedDL, chen2021quantization, matsubara2020head}, often fail to balance multiple objectives effectively. The emergence of Hardware-aware Neural Architecture Search (HW-NAS) has automated the search for optimal architectures, improving model efficiency and accuracy in resource-constrained settings~\cite{benmeziane2021comprehensive}.

HW-NAS uses machine learning to explore architecture spaces and identify designs that balance performance and resource consumption, accelerating edge DNN deployment~\cite{zhang2020fast}. Unlike traditional NAS, which prioritizes accuracy~\cite{he2021automl, elsken2019neural}, HW-NAS employs multi-objective optimization to enhance both accuracy and efficiency (e.g., latency, size)\cite{wistuba2019survey, Marchisio_2020ICCAD_NASCaps, Prabakaran_2021JIOT_BioNetExplorer}, generating Pareto-optimal solutions\cite{Kaisa1999MultiObjective}.



\subsection{Target Research Problem and Associated Challenges}

While advancements in edge DNN design have improved accuracy and computational efficiency, critical performance metrics like fairness, robustness, and generalization remain underexplored~\cite{sheng2022larger}. Fairness ensures equitable performance across diverse user groups, robustness measures reliability under varying conditions (e.g., lighting, weather, visibility)\cite{drenkow2021systematic, porrello2020robust}, and generalization evaluates performance on unseen data\cite{zhou2022domain}.

Enhancing these metrics is essential for edge DNNs, particularly in safety-critical applications like medical diagnostics~\cite{esteva2021deep}. However, achieving this faces challenges such as (1) ensuring data quality and balance~\cite{mehrabi2021survey}, (2) addressing insufficient data diversity~\cite{recht2019imagenet}, and (3) overcoming the computational limitations of edge devices~\cite{ibrahim2022robustness, feuerriegel2020fair, hickey2021fairness}.

Most existing work addresses these challenges by improving data quality~\cite{pitoura2020social} or training procedures~\cite{jain2024fairness}, with limited focus on architecture design. Studies targeting architecture often prioritize a single metric, such as fairness~\cite{sheng2022larger}. Comprehensive approaches that consider fairness, robustness, and generalization together are scarce. To fill this gap, we aim to propose a design methodology ensuring edge DNNs are not only efficient and high-performing but also fair and robust.



\subsection{Analysis: Fairness, Robustness, Generalization in Edge DNNs} \label{intro:pre-analysis}
To highlight the limitations of edge DNN designs in addressing fairness, robustness, and generalization, we evaluated 12 state-of-the-art (SOTA) edge DNNs on person classification (binary classification of images with or without a person) using a COCO dataset subset~\cite{lin2014microsoft}. We measured overfitting, sensitivity to light, and accuracy across different skin tones using the FACET dataset~\cite{gustafson2023facet}.

Figure~\ref{fig:sota_rob} presents robustness and generalization results, revealing notable gaps: validation and test accuracies differ significantly, and accuracies under varying lighting conditions drop by up to 15\%. This indicates potential overfitting and poor robustness to light changes in SOTA models. Figure~\ref{fig:sota_fair} shows a concerning fairness issue: average accuracy declines from 85.0\% for the lightest skin tone category to 70.9\% for the darkest.

Further analysis (Figure~\ref{fig:sota_graph}) demonstrates the influence of model architecture and size on fairness. Models with similar sizes but different architectures exhibit varying fairness scores, indicating that fairness is highly architecture-dependent. Moreover, increasing model size tends to enhance test accuracy but worsens fairness disparities, as larger channel sizes (depth) and more feature representations (width) amplify biases. These findings reveal the need for novel scaling strategies that explicitly target fairness and robustness without solely relying on traditional size scaling techniques.

\begin{figure}
\centering
\begin{subfigure}[b]{0.48\linewidth}
\includegraphics[width=1.\linewidth]{figures/sota_rob-gen.pdf}
    \caption{Overfitting and Sensitivity to light conditions of edge SOTA DNNs.}
    \label{fig:sota_rob}
    \end{subfigure}
%\hspace{.3cm}
\hfill
\begin{subfigure}[b]{0.48\linewidth}
\includegraphics[width=1.\linewidth]{figures/res_plot_skintone_.pdf}
    \caption{Average Accuracy of SOTA edge DNNs across 10 levels of skin tones (from lightest=1 to darkest=10) on FACET~\cite{gustafson2023facet}. }
    \label{fig:sota_fair}
\end{subfigure} 
    \caption{Evaluation of Fairness, Robustness, and Generalization of SOTA edge DNNs on FACET.}
    \label{fig:sota}
    %\vspace{-10pt}
\end{figure}





%\textit{These findings highlight the unintended consequence of current edge DNN designs: a significant bias that disadvantages individuals with darker skin tones and exhibits poor robustness and generalization capabilities.} This consequence not only raises serious ethical and reliability concerns about edge DNNs performance, but also underscores the need for a shift in the design paradigm towards creating edge DNNs that are high-performing, inclusive, and robust.


%models sharing the same architecture shown in 
%In Figure~\ref{fig:sota_graph}, we illustrate the impact of model architecture and size on its skin fairness. \textit{(1) The results reveal that model architecture and size indeed influence model fairness score}. Models with different architectures but similar sizes exhibit varied skin fairness scores. Moreover, \textit{ (2) we observe that increasing model size tends to enhance test accuracy, but leads to a reduction in the model`s skin fairness despite the benefits of larger channel sizes (depth) and more features (width)}. Observations (1) and (2) underscore the need for different model scaling approaches that can improve performance without compromising fairness in edge DNNs.

\begin{figure}
    \centering
    \includegraphics[width=\linewidth]{figures/sota_fair_ppt2.pdf}
    %\vspace{-8pt}
    \caption{Test Accuracy vs. Skin Fairness of SOTA edge DNNs: Models sharing the same architecture are connected by straight lines. The Pareto front is illustrated with a dashed line. }
    \label{fig:sota_graph}
    %\vspace{-10pt}
\end{figure}

\subsection{Contributions}

Our key contributions (listed below and summarized in Figure~\ref{fig:MoENAS_overview}), enable the design of fair, robust, and general edge DNNs.
%
%\vspace{-8pt}
%
\begin{enumerate}[leftmargin=*]
%
    \item \textbf{Dynamic Feature Extraction via Model Scaling}: We propose a scaling approach that varies the number of feature extractors dynamically (inspired by Mixture of Experts (MoE) and Switch layer architectures~\cite{fedus2022switch}), enabling adaptive feature extraction tailored to specific inputs. This flexibility indirectly benefits fairness and robustness by allowing more efficient and context-aware use of resources.
%\vspace{-5pt}    
    \item \textbf{HW-NAS with Fairness and Robustness Optimization}: We propose a HW-NAS method with a search space over switching architectures, varying the number of experts per block. Using machine-learning-based performance predictors, the search strategy identifies architectures optimized for accuracy, fairness, robustness, and model size.
%\vspace{-5pt}  
    \item \textbf{Expert Pruning for Efficiency}: We introduce an expert pruning method that helps reduce the sizes of models discovered by the search method by iteratively pruning the least used experts. This approach aims to enhance the efficiency of the resulting models without sacrificing performance.
\end{enumerate}
%\vspace{-5pt}  
%
\begin{figure*}[ht]
    \centering \includegraphics[width=.9\linewidth]{figures/BmoENAS_introver_1.pdf}
    \caption{Summary of MoENAS contributions: (1) Replace the FFN layer with a Switch FFN layer, (2) Search within the expert mixing space for optimal architectures (according to accuracy, fairness, robustness, generalization). (3) Pruning based on expert importance for better efficiency. }
    \label{fig:MoENAS_overview}
    %\vspace{-10pt}
\end{figure*}

%The rest of this paper will be structured as follows: Section 2 discusses the current state of the art in HW-NAS, MoE, fairness, and DNN robustness. Section 3 details our methodology and key contributions. Section 4 compares our model results with SOTA edge DNNs, followed by an ablation study and conclusion.

    
\section{Preliminaries}\label{sec:preliminaries}



%We denote by $(\Ac(x_\Ac),\Bc(x_\Bc))(z)$ a random execution of $\pi$ with private inputs $(x_\Ac,y_\Ac)$, and common input $z$.

%\Jnote{Move to DP}
% At the end of such an execution, the protocol outputs a public transcript denoted by the random variable $\trans_\pi(x_\Ac,x_\Ac,z)$ we denotes the common as $\out(\trans_\pi(x_\Ac,x_\Ac,z)$, and each party $\Pc \in \set{\Ac,\Bc}$ obtains his view denoted $\view^\Pc_\pi(x_\Ac,x_\Bc,z)$, which may also contain a ``local output'' \Jnote{Local} $\out^\Pc(x_\Ac,x_\Bc,z)$ (if the protocol specifies such an output). \Jnote{Common output, and parties output}


\subsection{Distributions and Random Variables}\label{sec:prelim:dist}
The support of a distribution $P$ over a finite set $\cS$ is defined by $\Supp(P) \eqdef \set{x\in \cS: P(x)>0}$. For a distribution or a random variable $D$, let $d\from D$ denote that $d$ was sampled according to $D$. Similarly,  for a set $\cS$, let $x \from \cS$ denote that $x$ is drawn uniformly from $\cS$, and denote by $\cU_{\cS}$ the uniform distribution over $\cS$. For a finite set $\cX$ and a distribution $C_X$ over $\cX$, we use the capital letter $X$ to denote the random variable that takes values in $\cX$ and is sampled according to $C_X$. The {\sf statistical distance} (\aka {\sf~variation distance}) of two distributions $P$ and $Q$ over a discrete domain $\cX$ is defined by $\sdist{P}{Q} \eqdef \max_{\cS\subseteq \cX} \size{P(\cS)-Q(\cS)} = \frac{1}{2} \sum_{x \in \cS}\size{P(x)-Q(x)}$. 
For a vector $x = (x_1,\ldots,x_n)$ and index $i\in [n]$, we let $x_{-i} = (x_1,\ldots,x_{i-1},x_{i+1},\ldots,x_n)$ and $x^{(i)} = (x_1,\ldots,x_{i-1}, -x_i, x_{i+1},\ldots,x_n)$, for a set $\cS \subseteq [n]$ we let $x_{\cS} = (x_i)_{i \in \cS}$ and $x_{-\cS} = (x_i)_{i \in [n]\setminus \cS}$, and for a vector $r \in \zo^n$ we let $x_r = (x_i)_{\set{i \colon r_i = 1}}$ and $x_{-r} = (x_i)_{\set{i \colon r_i = 0}}$.

%For $n \in \N$ we let $U_n$ be the uniform distribution over $\oo^n$, and let $S_n$ be the distribution induces by the sum of $n$ i.i.d.\ random variables, each is distributed according to $U_1$. Let $\cN(0,1)$ be the standard normal distribution.
%For a distribution $\cD$ and a function $f$, we define by $f(\cD)$ the distribution that is induced by the output of $f(x)$ for $x \from \cD$. 





% \begin{theorem}[\cite{McGregorMPRTV10}]\label{thm:sv-extracotr}
% 	\Enote{Remove if not needed}
% 	There is a constant $c$ to make the following holds. Let $X$ be an $\alpha$-SV source on $\{0,1\}^n$, let $Y$ be a source on $\{0,1\}^n$ with min-entropy at least $\beta n$ (independent from $X$), and let $Z=\ip{X,Y}\mbox{mod m}$ for some $m\in\mathbb{N}$. Then for every $\delta\in[0,1]$, the random variable $(Y,Z)$ is $\delta$-close to $(Y,U)$ where $U$ is uniform on $\mathbb{Z}_m$ and independent of $Y$, provided that
% 	$$
% 	n\geq c\cdot\frac{m^2}{\alpha\beta}\cdot\log(\frac{m}{\beta})\cdot\log(\frac{m}{\delta}).
% 	$$
% \end{theorem}



\Enote{I removed the definition of DP since it already appears in the intro}
\remove{
\subsection{Differential Privacy}\label{sec:prelim:DP}
We use the following standard definition of (information theoretic) differential privacy, due to \citet{DMNS06}. For notational convenience, we focus on databases over $\oo$.
\begin{definition}[Differentially private mechanisms]\label{def:mech}
	A randomized function $f\colon\oo^n\mapsto \zs$ is an {\sf $n$-size, $(\eps,\delta)$-differentially private mechanism} (denoted $(\eps,\delta)$-\DP) if for every neighboring $w,w'\in \oo^n$ and every function $g\colon \zs\mapsto \zo$, it holds that 
	$$
	\pr{g(f(w))=1}\leq e^{\eps}\cdot \pr{g(f(w'))=1} +\delta.
	$$ 	
	If $\delta=0$, we omit it from the notation.
\end{definition}
}


\subsubsection{Computational Differential Privacy}
There are several ways for defining computational differential privacy (see \cref{sec:related-works}). We use the most relaxed version due to \cite{BNO08}. For notational convenience, we focus on databases over $\oo$.
\begin{definition}[Computational differentially private mechanisms]\label{def:ComMech}
	A randomized function ensemble $f=\set{f_\pk\colon\oo^{n(\pk)}\mapsto \zs}$ is an {\sf $n$-size, $(\eps,\delta)$-computationally differentially private} (denoted $(\eps,\delta)$-$\CDP$) if for every poly-size circuit family $\set{\Ac_\pk}_{\pk\in \N}$, the following holds for every large enough $\pk$ and every neighboring $w,w'\in\oo^{n(\pk)}$:
	$$
	\pr{\Ac_\pk(f_\pk(w))=1}\leq e^{\eps(\pk)}\cdot \pr{\Ac_\pk(f_\pk(w'))=1} +\delta(\pk).
	$$ 
	If $\delta(\pk) = \negl(\pk)$, we omit it from the notation. 
\end{definition}



\subsubsection{Two-Party Differential Privacy}\label{sec:DP}
In this section we formally define distributed differential privacy mechanism (\ie protocols). %For the ease of notation, we consider protocol with no common input.

\begin{definition}\label{def:DP}%\Nnote{fix security parameter}
	A two-party protocol $\Pi=(\Ac,\Bc)$ is {\sf $(\eps,\delta)$-differentially private}, denoted $(\eps,\delta)$-$\DP$, if the following holds for every algorithm $\Dc$: let $\V^\Pc(x,y)(\pk)$ be the view of party $\Pc$ in a random execution of $\Pi(x,y)(1^\pk)$. Then for every $\pk,n \in \N$, $x\in \oo^n$ and neighboring $y,y'\in\oo^n$:
	\begin{align*}
	\pr{\Dc(V^\Ac(x,y)(\pk))=1}\le e^{\eps(\pk)}\cdot \pr{\Dc(V^\Ac (x,y')(\pk))=1}+\delta(\pk),
	\end{align*} 
	and for every $y\in \oo^n$ and neighboring $x,x'\in\oo^{n}$:
	\begin{align*}
	\pr{\Dc(V^\Bc(x,y)(\pk))=1}\le e^{\eps(\pk)}\cdot \pr{\Dc(V^\Bc (x',y)(\pk))=1}+\delta(\pk).
	\end{align*} 	
	Protocol $\Pi$ is {\sf $(\eps,\delta)$-computational differentially private}, denoted $(\eps,\delta)$-$\CDP$, if the above inequalities only hold for a non-uniform \ppt $\Dc$ and large enough $\pk$. We omit $\delta = \negl(\pk)$ from the notation. \footnote{Note that define we give for two-party differentially private protocols is a semi-honest definition, in which we ask for the security to hold when the parties interact in an honest execution of the protocol. Since we are proving a lower bound, starting from this weaker guarantee (as opposed to security against malicious players), yields a stronger result.}
\end{definition}
%We omit $\delta$ from the notation if $\delta$ is a negligible function of $n$.

%\Enote{simulation-based}
\begin{remark}[The definition for computational differential privacy we use]\label{rem:comDPChannel} 
	An alternative, stronger definition of computational differential privacy, known as simulation-based computational differential privacy, requires that the distribution of each party’s view be computationally indistinguishable from a distribution that ensures privacy in an information-theoretic sense. \cref{def:DP} is a weaker notion in comparison. Consequently, establishing a lower bound for a protocol that satisfies this weaker guarantee (as we do in this work) yields a stronger result.%Actually, our lower bound only requires the privacy to hold against \emph{uniform} external observer.
	%\Nnote{Maybe add: When only interesting in \Dp against external observer, the two definitions can be achieve using key-agreement and (single-party) \Dp mechanism. }
\end{remark}




\subsection{Useful Claims}
\remove{
In this section, we state generic lemmas and propositions that we will use later in our proofs.

The following lemma which we prove in \cref{sec:missing-proofs:distance-I}, measures the distance between two uniform stings conditioned one a random index $i$ either being fixed to $0$ or to $1$.

\def\distanceILemma{
    Let $R \la \zo^n$. For any (randomized) function $f:\{0,1\}^n\rightarrow \{0,1\}$ and $\alpha > 0$, it holds that
    \begin{align}\label{eq:f-alpha}
        \ppr{i \la [n]}{\size{\:\ex{f(R) \mid R_i = 0}-\ex{f(R) \mid R_i = 1}\:}\geq \alpha} \leq \frac{2}{n \alpha^2},
    \end{align}
    where the expectations are taken over $R$ and the randomness of $f$.
}

\begin{lemma}\label{lem:distance-I}
    \distanceILemma
\end{lemma}
}

The following two propositions state that given the output of a differentially private function, it is not possible to predict well even a random index (even if all other indexes are leaked). The first proposition handles the information-theoretic case and the second handles the computation case. Both propositions are proven in \cref{sec:missing-proofs:hard-to-guess}. 

\def\propHardToGuessInf{
    Let $f\colon \oo^n \rightarrow \cY$ be an $(\eps,\delta)$-\DP function, let $g \colon [n] \times \oo^{n-1} \times \cY \rightarrow \set{-1,1,\bot}$ be a (randomized) function, and let $X = (X_1,\ldots,X_n) \la \oo^n$. Then the following holds for every $i \in [n]$ where $X_i^* = g(i,X_{-i},f(X_1,\ldots,X_n))$:
    \begin{align*}
        \pr{X_i^* = X_i} \leq e^{\eps}\cdot \pr{X_i^* = -X_i} + \delta.
    \end{align*}
}

\begin{proposition}\label{prop:hard-to-guess-inf}
    \propHardToGuessInf
\end{proposition}


\def\propHardToGuessComp{
    Let $f = \set{f_{\pk} \colon \oo^{n(\pk)} \rightarrow \zo^{m(\pk)}}_{\pk \in \bbN}$ be an $(\eps,\delta)$-\CDP function ensemble, and let $\set{g_{\pk}}_{\pk \in \bbN}$ be a poly-size circuit family. Then, for large enough $\pk$ and $X = (X_1,\ldots,X_{n(\pk)}) \la \oo^{n(\pk)}$, the following holds for every $i \in [n(\pk)]$ where $X_i^* = g_{\pk}(i,X_{-i},f_{\pk}(X_1,\ldots,X_n))$:
    \begin{align*}
        \pr{X_i^* = X_i} \leq e^{\eps(\pk)}\cdot \pr{X_i^* = -X_i} + \delta(\pk).
    \end{align*}
}

\begin{proposition}\label{prop:hard-to-guess-comp}
    \propHardToGuessComp
\end{proposition}





\remove{
\Enote{Chao's old statement:}
\begin{lemma}\label{lem:distance-I-old}
        Let $R \la \zo^n$. 
	For any function $f:\{0,1\}^n\rightarrow \{0,1\}$ and $\alpha<0.01$, it holds that
	$$
	\Pr_{i\la[n]}\left[\: \size{\:\mathbb{E}[f(R) \mid R_i = 0]-\mathbb{E}[f(R) \mid R_i = 1]\:}\geq \alpha\right]\leq \frac{2+2\log(\frac{1}{\alpha})}{n\alpha^2}.
	$$
\end{lemma}
\begin{proof}
	Define $S_1=\{r \in \zo^n \colon f(r)=1\}$. Then for any $i\in[n]$, we have
	$$
	\begin{array}{rl}
		\size{\mathbb{E}[f(R) \mid R_i = 0]-\mathbb{E}[f(R) \mid R_i = 1]}
		&=\size{\Pr[R\in S_1|R_i=0]-\Pr[R\in S_1|R_i=1]}\\
		&=\size{\frac{\Pr[R_i=0|R\in S_1]\cdot\Pr[R\in S_1]}{\Pr[R_i=0]}-\frac{\Pr[R_i=1|R\in S_1]\cdot\Pr[R\in S_1]}{\Pr[R_i=1]}}\\
		&=\frac{2\size{S_1}}{2^n}\size{\Pr[R_i=0|R\in S_1]-\Pr[R_i=1|R\in S_1]}
	\end{array}
	$$
	When $|S_1|\leq \alpha\cdot 2^{n-1}$, we have $\size{\mathbb{E}[f(R) \mid R_i = 0]-\mathbb{E}[f(R) \mid R_i = 1]}\leq\frac{2\size{S_1}}{2^n}\leq \alpha$ for any $i\in[n]$. Hence, in the following, we assume $|S_1|> \alpha\cdot 2^{n-1}$.

	%Define $I_{bad}=\{i|\size{\Pr[R_i=0|R\in S_1]-\Pr[R_i=1|R\in S_1]}>2\alpha\}$ and $k=\size{I_{bad}}$, then for any $i\notin I_{bad}$, we have 
    %$$
    %\begin{array}{rl}
    %    2\alpha&\geq \size{\Pr[R_i=0|R\in S_1]-\Pr[R_i=1|R\in S_1]}\\
    %    &=\size{\frac{\Pr[R\in S_1|R_i=0]\cdot\Pr[R_i=0]}{\Pr[R\in S_1]}-\frac{\Pr[R\in S_1|R_i=1]\cdot\Pr[R_i=1]}{\Pr[R\in S_1]}}\\
    %    &=\size{\Pr[R\in S_1|R_i=0]-\Pr[R\in S_1|R_i=1]}\cdot\frac{1}{2\Pr[R\in S_1]}\\
    %    &\geq \size{\mathbb{E}[f(R) \mid R_i = 0]-\mathbb{E}[f(R) \mid R_i = 1]}\cdot \frac{1}{2},
    %\end{array}
    %$$ 
    %where the last inequality is because $\Pr[R\in S_1]\leq 1$. So that $\size{\mathbb{E}}[f(R) \mid R_i = 0]-\mathbb{E}[f(R) \mid R_i = 1]\leq %4\alpha$.
    Define $I_{bad}=\{i \colon \size{\Pr[R_i=0|R\in S_1]-\Pr[R_i=1|R\in S_1]} \geq 2\alpha\}$ and $k=\size{I_{bad}}$, and denote $I_{bad}=\{i_1,\dots,i_k\}$. Define $(X_{i_1}, \ldots X_{i_k}) = (R_{i_1},\dots,R_{i_k})\mid_{R \in S_1}$. 
    Consider the min-entropy
	$$
	\begin{array}{rl}
		H_{min}(X_{i_1},\dots,X_{i_k})&\leq H(X_{i_1},\dots,X_{i_k})\\
		&\leq \sum_{j=1}^k H(X_{i_j})\\
		&\leq k\cdot \left(-(\frac{1}{2}+2\alpha)\cdot\log(\frac{1}{2}+2\alpha)-(\frac{1}{2}-2\alpha)\cdot\log(\frac{1}{2}-2\alpha)\right)\\
            &=k\cdot \left(-(\frac{1}{2}+2\alpha)\cdot(\log(1+4\alpha)-1)-(\frac{1}{2}-2\alpha)\cdot(\log(1-4\alpha)-1)\right)\\
            &=k\cdot \left(1-(\frac{1}{2}+2\alpha)\cdot\log(1+4\alpha)-(\frac{1}{2}-2\alpha)\cdot\log(1-4\alpha)\right),
		
	\end{array}
	$$
	where $H_{min}(Y)$ is the minimum entropy of $Y$ and $H(Y)$ is the Shannon entropy of $Y$.\Enote{add to preliminaries.}
        The third inequality holds since by the definition of $I_{bad}$, for every $j \in [k]$ it holds that $\size{\pr{X_{i_j} = 1}-\pr{X_{i_j} = 0}} > 2\alpha$, and therefore $H(X_{i_j}) \leq H(1/2 + 2\alpha)$\Enote{define}.
	
	Therefore, there exists $b_1,\dots,b_k\in\{0,1\}$, such that 
	
	\begin{align}\label{eq:min-entropy-result}
		\Pr\left[(R_{i_1},\ldots,R_{i_k}) = (b_1,\ldots,b_k) \mid R\in S_1\right]
		&= \pr{(X_{i_1},\ldots,X_{i_k}) = (b_1,\ldots,b_k)}\\
		&= 2^{-H_{min}(X_{i_1},\dots,X_{i_k})}\nonumber\\
		&\geq 2^{k\cdot \left(-1+(\frac{1}{2}+2\alpha)\cdot\log(1+4\alpha)+(\frac{1}{2}-2\alpha)\cdot\log(1-4\alpha)\right)}.\nonumber
	\end{align}
	
	Let $S_{bad}=\{r \in \zo^n  \colon \set{(r_{i_1},\ldots,r_{i_k}) = (b_1,\ldots,b_k)} \land \set{r\in S_1}\}$.
	It holds that
	\begin{align*}
		|S_{bad}|
		&= \size{S_1} \cdot \Pr\left[(R_{i_1},\ldots,R_{i_k}) = (b_1,\ldots,b_k) \mid R\in S_1\right]\\
		&\geq \alpha\cdot 2^{n-1}\cdot2^{k\cdot \left(-1+(\frac{1}{2}+2\alpha)\cdot\log(1+4\alpha)+(\frac{1}{2}-2\alpha)\cdot\log(1-4\alpha)\right)},
	\end{align*} 
	where the inequality holds by \cref{eq:min-entropy-result} and since $\size{S_1} \geq \alpha\cdot 2^{n-1}$.
	Notice that any string in $S_{bad}$ depends on at most $n-k$ bits. It implies that $|S_{bad}|\leq 2^{n-k}$. Therefore, we have
	$$
	\begin{array}{rl}
		&2^{n-k}\geq \alpha\cdot 2^{n-1}\cdot2^{k\cdot \left(-1+(\frac{1}{2}+2\alpha)\cdot\log(1+4\alpha)+(\frac{1}{2}-2\alpha)\cdot\log(1-4\alpha)\right)} \\
		\Rightarrow& n-k \geq \log \alpha+n-1+k\cdot \left(-1+(\frac{1}{2}+2\alpha)\cdot\log(1+4\alpha)+(\frac{1}{2}-2\alpha)\cdot\log(1-4\alpha)\right)\\
		\Rightarrow& 1-\log \alpha \geq k\cdot((\frac{1}{2}+2\alpha)\cdot\log(1+4\alpha)+(\frac{1}{2}-2\alpha)\cdot\log(1-4\alpha))\\
		\Rightarrow& 1-\log \alpha \geq k\cdot(4\alpha\cdot\log(1+4\alpha)+(\frac{1}{2}-2\alpha)\cdot\log(1-16\alpha^2))\\
        \Rightarrow& 1-\log\alpha \geq k\cdot(15.9\alpha^2-8\alpha^2+32\alpha^3)=k\cdot(7.9\alpha^2+32\alpha^3)>0.5k\alpha^2\\
		\Rightarrow& k\leq \frac{2-2\log \alpha}{\alpha^2} = \frac{2+2\log (1/\alpha)}{\alpha^2},
	\end{array}
	$$
	Where the third transition holds since 
	\begin{align*}
		\lefteqn{(\frac{1}{2}+2\alpha)\cdot\log(1+4\alpha)+(\frac{1}{2}-2\alpha)\cdot\log(1-4\alpha)}\\
		&= 4\alpha\cdot\log(1+4\alpha) + (\frac{1}{2}-2\alpha)\paren{\log(1+4\alpha)+\log(1-4\alpha)}\\
		&= 4\alpha\cdot\log(1+4\alpha)+(\frac{1}{2}-2\alpha)\cdot\log(1-16\alpha^2),
	\end{align*}
	and the forth transition holds since $4\alpha\cdot\log(1+4\alpha)+(\frac{1}{2}-2\alpha)\cdot\log(1-16\alpha^2) > 15.9\alpha^2-8\alpha^2+32\alpha^3$ for $\alpha < 0.01$.
	Thus, we conclude that 
	$$
	\Pr_{i\la[n]}\left[\size{\mathbb{E}[f(R) \mid R_i=0]-\mathbb{E}[f(R) \mid R_i = 1]}\geq \alpha\right]\leq \frac{k}{n}\leq \frac{2+2\log (1/\alpha)}{n\alpha^2}.
	$$
\end{proof}
}


\subsection{Channels and Two-Party Protocols}\label{sec:protocol}

\paragraph{Channels.}A channel is simply a distribution of a pair of tuples defined as follows. 
\begin{definition}[Channels]\label{def:channel} A {\sf channel} $C_{(X,U)(Y,V)}$ of size $\isize$ over alphabet $\Sigma$ is a probability distribution over $(\Sigma^\isize \times\zo^\ast) \times(\Sigma^\isize \times\zo^\ast)$. The ensemble $C_{(X,U)(Y,V)}= \set{C_{(X_\pk,U_\pk)(Y_\pk,V_\pk)}}_{\pk\in \N}$ is an $\isize$-size channel ensemble, if for every $\pk\in \N$, $C_{(X_\pk,U_\pk)(Y_\pk,V_\pk)}$ is an $\isize(\pk)$-size channel. %We denote a channel of size one by a \emph{single-bit} channel. 
We refer to $X$ and $Y$ as the {\sf local outputs}, and to $U$ and $V$ as the {\sf views}.	
\end{definition}

We view a  channel as the experiment in which there are two parties $\Ac$ and $\Bc$.  Party $\Ac$ receives ``output'' $X$ and ``view'' $U$, and party $\Bc$ receives ``output'' $Y$ and ``view'' $V$. Unless stated otherwise, the channels we consider are over the alphabet $\Sigma = \oo$. We naturally identify channels with the distribution that characterizes their output.








\subsubsection{Two-Party Protocols}

A two-party protocol $\Pi=(\Ac,\Bc)$ is \ppt if the running time of both parties is polynomial in their input length. We let $\Pi(x,y)(z)$ or $(\Ac(x),\Bc(y))(z)$ denote a random execution of $\Pi$ on a common input $z$, and private inputs $x,y$.%We assume \wlg that a protocol has a common output (part of its transcript).\Jnote{This is not really the case we consider in this paper..}

\begin{definition}[Oracle-aided protocols]\label{def:ChannelAidedProtocol}
	In a two-party protocol $\Pi$ with oracle access to a {\sf protocol} $\Psi$, denoted $\Pi^\Psi$, the parties make use of the \textit{next-message function} of $\Psi$.\footnote{The function that on a partial view of one of the parties, returns its next message.} In a two-party protocol $\Pi$ with oracle access to a {\sf channel} $C_{Z W}$, denoted $\Pi^C$, the parties can jointly invoke $C$ for several times. In each call, an independent pair $(z,w)$ is sampled according to $C_{Z W}$, one party gets $z$, the other gets $w$.
\end{definition}


\begin{definition}[The channel of a protocol]\label{def:ChannlOfProtocol}
	For a no-input two-party protocol $\Pi= (\Ac,\Bc)$, we associate the channel $C_\Pi$, defined by $\C_\Pi= C_{(X, U),(Y, V)}$, where $X$ and $Y$ are the local outputs of $\Ac$ and $\Bc$ (respectively) and
	$U$ and $V$ are the local views of $\Ac$ and $\Bc$ (respectively).
    
	For a two-party protocol $\Pi$ that gets a security parameter $1^\pk$ as its (only, common) input, we associate the channel ensemble $ \set{C_{\Pi(1^\pk)}}_{\pk\in \N}$. 
\end{definition}

\begin{definition}[$(\alpha,\gamma)$-Accurate channel]\label{def:accurate-func}
	A channel $C = C_{(X, U),(Y, V)}$ is {\sf $(\alpha,\gamma)$-accurate for the function $f$}, if $\ppr{C}{\size{\out(V)-f(X,Y)}\leq \alpha}\ge \gamma$, where $\out(V)$ is the designated output.
    A channel ensemble $C_{(X, U),(Y, V)}= \set{C_{(X_\pk, U_\pk),(Y_\pk, V_\pk)}}_{\pk\in \N}$ is  $(\alpha,\gamma)$-accurate for  $f$ if $C_{(X_\pk, U_\pk),(Y_\pk, V_\pk)}$ is $(\alpha(\pk),\gamma(\pk))$-accurate for $f$, for every $\pk \in \N$.
\end{definition}

\subsubsection{Differentially Private Channels}\label{sec:DPChannel}
Differentially private channels are naturally defined as follows:
\begin{definition}[Differentially private channels]\label{def:DPChannel}
	An $n$-size channel $C = C_{(X, U),(Y, V)}$ with $X, Y$ over $\oo^n$ 
	is {\sf$(\eps,\delta)$-differentially private} (denoted $(\eps,\delta)$-$\DP$) if for every $x \in \Supp(X)$ there exists an $n$-size $(\eps,\delta)$-$\DP$ mechanisms $\Mc_x$ such that $(X,Y,U) \equiv (X,Y,\Mc_X(Y))$, and for every $y \in \Supp(Y)$ there exists an $n$-size $(\eps,\delta)$-$\DP$ mechanisms $\Mc_y'$ such that $(X,Y,V) \equiv (X,Y,\Mc_Y'(X))$. In addition, we say that the channel is \emph{uniform} if $X$ and $Y$ are independent random variables uniformly distributed in $\oo^n$. 
\end{definition}

\begin{definition}[Computational differentially private channels]\label{def:CDPChannel}
	An $n$-size channel ensemble $C = \set{C_{(X_\pk, U_\pk),(Y_\pk, V_\pk)}}_{\pk\in\N}$ with $X_\pk, Y_\pk$ over $\oo^n$ 
	is {\sf$(\eps,\delta)$-computationally differentially private} (denoted $(\eps,\delta)$-$\CDP$) if for every ensemble $\set{x_\pk \in \Supp(X_\pk)}_{\pk\in\N}$ there exists an $n$-size $(\eps,\delta)$-\CDP mechanisms ensemble $\set{\Mc_{x_\pk}}_{\pk\in\N}$ such that $(X_\pk,Y_\pk,U_\pk) \equiv (X_\pk,Y_\pk,\Mc_{X_\pk}(Y_\pk))$, for every $\pk\in\N$, and for every ensemble $\set{y_\pk \in \Supp(Y_\pk)}_{\pk\in\N}$ there exists an $n$-size $(\eps,\delta)$-$\CDP$ mechanisms ensemble $\set{\Mc'_{y_\pk}}_{\pk\in\N}$ such that $(X_\pk,Y_\pk,V_\pk) \equiv (X_\pk,Y_\pk,\Mc_{Y_\pk}'(X_\pk))$ for every $\pk\in \N$. In addition, we say that the channel is \emph{uniform} if $X_\pk$ and $Y_\pk$ are independent random variables uniformly distributed in $\{\pm 1\}^n$ for all $\pk\in\N$.
\end{definition}




% \begin{lemma}~\label{lem:dp-sv-source}
% 	Let $P$ be an $\varepsilon$-DP randomized protocol. Let $X$ and $Y$ be independent random variables uniformly distributed in $\{\pm 1\}^n$ and let random variable $\Pi(X,Y)$ denote the transcript of running $P(X,y)$. Then for every $\pi\in Supp(\Pi)$, the random variables corresponding to the inputs conditioned on transcript $\pi$, $X_\pi$ and $Y_\pi$, are independent $e^{-\varepsilon}$-strong SV source.
% \end{lemma}





\subsubsection{Weak Erasure Channel (\WEC)}

\begin{definition}[\WEC]\label{def:WEC}
	A channel $((O_A,V_A), (O_B,V_B))$ with $O_A \in \set{0,1}$ and $O_B \in \set{0,1,\bot}$ is a {\sf weak erasure channel}, denoted $(\alpha,p,q)$-$\WEC$, if:
	\begin{itemize}
		%\item $O_A\in \set{-1,1}$ and $O_B\in \set{-1,1,\bot}$.
		\item Random erasure: $\pr{O_B = \perp} = 1/2$.
		
		\item Agreement: $\pr{O_A\ne O_B\mid O_B\ne \bot}\le \alpha$.
		
		\item Secrecy:
		
		\begin{enumerate}
			\item For every algorithm $\Dc$ it holds that\label{WEC:item:A}
			\begin{align*}
				%\size{\pr{\Ac(O_A,V_A) = 1 \mid O_B \neq \perp} - \pr{\Ac(O_A,V_A) = 1 \mid O_B = \perp}} \le p
				\size{\pr{\Dc(V_A) = 1 \mid O_B \neq \perp} - \pr{\Dc(V_A) = 1 \mid O_B = \perp}} \le p
			\end{align*}
			(Alice doesn't know if $O_B = \perp$.)
			
			\item For every algorithm $\Dc$ it holds that\label{WEC:item:B}
			\begin{align*}
				\pr{\Dc(V_B) = O_A \mid O_B=\bot} \leq \frac{1+q}{2}.
			\end{align*}
			(i.e., if $O_B=\bot$, Bob don't know what is the value of $O_A$).
			
			%\item $SD((O_A U|O_B=\bot),(O_A U|O_B\ne \bot))\le p$ (The sender don't know if $O_B=\bot$).
			
			%\item $SD(V O_A|O_B=\bot,V(-O_A)|O_B=\bot)\le q$ (If $O_B=\bot$, Bob don't know what the value of $O_A$).
		\end{enumerate}
	\end{itemize}
   We say that a channel ensemble $C=\set{C_\pk}_{\pk\in N}$ is a {\sf computational weak erasure channel}, denoted $(\alpha,p,q)$-\CompWEC, if for every \ppt algorithm $\Dc$ and every sufficiently large $\pk\in\N$, $C_\pk$ satisfies the properties stated in the items above, where the secrecy property holds with respect to a \ppt algorithm $\Dc$. A protocol $\Lambda$ is said to be $(\alpha,p,q)$-$\CompWEC$, if the ensemble induces by the protocol (that is, $C=\set{C_{\Lambda(\pk)}}_{\pk\in\N}$) is $(\alpha,p,q)$-$\CompWEC$.  
\end{definition}



\subsubsection{Approximate Weak Erasure Channel (\AWEC)}\label{sec:AWEC}

\begin{definition}[\AWEC]\label{def:AWEC}
	A channel $C = ((O_A,V_A), (O_B,V_B))$ over $([-n,n] \times \zo^*) \times (([-n,n] \cup \bot)  \times \zo^*)$ is an {\sf approximate weak erasure channel}, denoted $(\ell,\alpha,p,q)$-\AWEC if:
	\begin{itemize}
		
		\item Random erasure: $\pr{O_B = \perp} = 1/2$.
		
		\item Accuracy: $\pr{\size{O_A - O_B} > \ell \mid O_B \ne \bot}\le \alpha$.
		
		\item Secrecy:
		
		\begin{enumerate}
			\item For every algorithm $\Dc$ it holds that\label{AWEC:item:A}
			\begin{align*}
				%\size{\pr{\Ac(O_A,V_A) = 1 \mid O_B \neq \perp} - \pr{\Ac(O_A,V_A) = 1 \mid O_B = \perp}} \le p
				\size{\pr{\Dc(V_A) = 1 \mid O_B \neq \perp} - \pr{\Dc(V_A) = 1 \mid O_B = \perp}} \le p
			\end{align*}
			(Alice doesn't know if $O_B=\bot$).
			
			\item For every algorithm $\Dc$ it holds that\label{AWEC:item:B}
			\begin{align*}
				\pr{\size{\Dc(V_B) - O_A} \leq 1000 \ell \mid O_B=\bot} \leq q.
			\end{align*}
			(i.e., if $O_B=\bot$, Bob can't estimate the value of $O_A$ with error $\leq 1000 \ell$).
		\end{enumerate}
	\end{itemize}
     We say that a channel ensemble $C=\set{C_\pk}_{\pk\in N}$ is a {\sf computational approximate weak erasure channel}, denoted $(\ell,\alpha,p,q)$-\CompAWEC, if for every \ppt algorithm $\Dc$ and every sufficiently large $\pk\in\N$, $C_\pk$ satisfies the properties stated in the items above. A protocol $\Gamma$ is said to be $(\ell,\alpha,p,q)$-$\CompAWEC$, if the ensemble induced by the protocol (that is, $C=\set{C_{\Gamma(\pk)}}_{\pk\in\N}$) is $(\ell,\alpha,p,q)$-$\CompAWEC$.  
\end{definition}

We will make use of the following lemma, which shows that for some choices of the parameters, \AWEC implies \WEC. The lemma is proven in \cref{sec:AWEC-to-WEC}.

\begin{lemma}\label{lemma:AWEC-to-WEC}
	For every $\ell> 0$, there exists a \ppt protocol $\Lambda = (\Pc_1,\Pc_2)$ such that given an oracle access to an $(\ell,\alpha,p,q)$-\AWEC $C$, the channel $\tilde{C}$ induced by $\Lambda^C$ is $(\alpha'=\alpha+0.001,\: p' = p ,\:  q' = 1/2 + 2(q+0.01))$-\WEC.
	Furthermore, the proof is constructive in a black-box manner:
	\begin{enumerate}
		\item There exists an oracle-aided \ppt algorithm $\Ec_1$ such that for every channel $C = ((\OA,\VA), (\OB,\VB))$ and algorithm $\Dc$ violating the \WEC secrecy property~\ref{WEC:item:A} of $\tilde{C}$, algorithm $\Ec_1^{\Dc}$ violates the \AWEC secrecy property~\ref{AWEC:item:A} of $C$.
		
		\item There exists an oracle-aided \ppt algorithm $\Ec_2$ such that for every channel $C = ((\OA,\VA), (\OB,\VB))$ and algorithm $\Dc$ violating the \WEC secrecy property~\ref{WEC:item:B} of $\tilde{C}$, algorithm $\Ec_2^{\Dc}$ violates the \AWEC secrecy property~\ref{AWEC:item:B} of $C$.
	\end{enumerate}
\end{lemma}

Since \cref{lemma:AWEC-to-WEC} is constructive, the following is an immediate corollary.
\begin{corollary}\label{cor:CompAWEC to CompWEC}
There exists an oracle aided \ppt protocol $\Lambda$, such that given a protocol $\Gamma$ that induces $(\ell,\alpha,p,q)$-\CompAWEC, it holds that $\Lambda^\Gamma$ is $(\alpha'=\alpha+0.001,\: p' = p ,\:  q' = 1/2 + 2(q+0.01))$-\CompWEC.  
\end{corollary}
\begin{proof}[Proof of \ref{cor:CompAWEC to CompWEC}]
Let $\Lambda$ be the \ppt algorithm guaranteed  by Lemma \ref{lemma:AWEC-to-WEC}. Given an $(\ell,\alpha,p,q)$-\CompAWEC protocol $\Gamma$, we define $\Lambda(\pk)={\Lambda^{\Gamma(\pk)}(\pk)}$. Assume towards a contradiction that $\Lambda$ is not a $(\alpha',p',q')$-\CompWEC. It follows that there exists a \ppt $\Dc$ that for infinity many $\pk\in\N$ contradicts one of the \WEC secrecy properties of channel ensemble $\set{C_{\Lambda(\pk)}}_{\pk\in\N}$. Fix $\pk\in\N$ for which this holds. By Lemma \ref{lemma:AWEC-to-WEC}, there exists a \ppt $\Ec^\Dc$ that for every such $\pk$  contradicts one of the secrecy properties of the channel $C_{\Gamma(\pk)}$. This implies that for infinity many $\pk\in\N$, $\Ec^\Dc$  contradict the secrecy of the channel ensemble $\set{C_{\Gamma(\pk)}}_{\pk\in\N}$, which is a contradiction since this would means that $\Gamma$ is not a $(\ell,\alpha,p,q)$-\CompAWEC.       
\end{proof}



\subsection{Oblivious Transfer (\OT)}

\paragraph{Secure Computation.}
We use the standard notion of securely computing a functionality, \cf  \cite{Goldreich04}.
\begin{definition}[Secure computation]\label{def:SFE}
	A two-party protocol {\sf securely computes a functionality $f$}, if it does so according to the real/ideal paradigm.   We add the term perfectly/statistically/computationally/non-uniform computationally, if the simulator's output is  perfect/statistical/computationally indistinguishable/  non-uniformly indistinguishable from  the real distribution.  The protocol have the above notions of security {\sf against semi-honest  adversaries}, if its security only  guaranteed to holds against an adversary that follows the prescribed protocol.   Finally, for the case of perfectly secure computation, we naturally apply the above notion also to the non-asymptotic case: the protocol with no security parameter perfectly  compute a functionality $f$.
	
	A two-party protocol {\sf securely computes a functionality ensemble $f$ with oracle to a channel $C$}, if it does so according to the above definition when the parties have access to a trusted party computing $C$. All the above adjectives naturally extend to this setting.
\end{definition}

\paragraph{Oblivious Transfer.}
The (one-out-of-two) oblivious transfer functionality is defined as follows.
\begin{definition}[oblivious transfer functionality $f_{\OT}$]\label{def:OTfunc}
	The oblivious transfer functionality over $\zo \times (\zs)^2$ is defined by  $f_{\OT} (i,(\sigma_0,\sigma_1)) = (\perp,\sigma_i)$.
\end{definition}
A protocol is $\ast$ secure OT,   for \\$\ast\in \set{\text{semi-honest statistically/computationally/computationally non-uniform}}$, if it  compute the $f_{\OT}$  functionality with $\ast$ security.





% \begin{definition}[Computational oblivious transfer, semi-honest model]
% A protocol $\Pi=(\Ac,\Bc)$ is a semi-honest 1-out-of-2 computational oblivious transfer (comp-OT) protocol if the following holds. Given a common input $1^{\pk}$, the parties $\Ac$ and $\Bc$ run the protocol $\Pi(1^\pk)$ (in an honest manner) and    
% $\Ac$ outputs $X=(m_1,m_2)\in \zo\times\zo$ and has a view $U$ and $\Bc$ outputs $Y=(i,\hat{m})\in\zo\times\zo$ and has a view $V$, and the following properties are satisfied:
% \begin{enumerate}
%     \item \textbf{Correctness:} 
%     $\pr{\hat{m}\neq m_i}<\negl(\pk).$ 
    
%     \item \textbf{A's Privacy:} For every \ppt $\Dc$ and every sufficiently large $\pk$:
%     $\pr{\Dc(V)=m_{i-1}}<(1+\negl(\pk))/2$
    
%     \item \textbf{B's Privacy:} For every \ppt $\Dc$ and every sufficiently large $\pk$:
%     $\pr{\Dc(U)=i}<(1+\negl(\pk))/2$  
% \end{enumerate}
% \end{definition}

We make use of the following useful results by Wullschleger on oblivious transfer amplification from weak channels.
\begin{theorem}[\cite{Wullschleger09}, from \WEC to statistically secure \OT]\label{thm:WEC TO OT IT}
    There exists an oracle aided protocol $\Pi$ such that the following holds: Given a $(\alpha,p,q)$-\WEC $C$, if $44(\alpha+p)\le 1-q$ then $\Pi^{C}(1^\pk)$ is a semi-honest statistically secure \OT.
\end{theorem}

The following computational version of \cref{thm:WEC TO OT IT} is implicit in \cite{Wullschleger09} and is based on the computational proof explicitly stated in \cite{Wul07} (see Section 6 in \cite{Wullschleger09} for discussion).   

\begin{theorem}[\cite{Wullschleger09,   Wul07}, from \CompWEC to computinally secure \OT]\label{thm:WEC TO OT Comp}
    There exists an oracle aided protocol $\Pi$ such that the following holds: Given a $(\alpha,p,q)$-\CompWEC protocol $\Lambda$, if $44(\alpha+p)\le 1-q$ then $\Pi^{\Lambda}$ is a semi-honest computational secure \OT.
\end{theorem}



% \begin{definition}[Computational 1-out-of-2 Oblivious Transfer, semi-honest model]
% A protocol $\Pi=(\Ac,\Bc)$ is a semi-honest 1-out-of-2 $(\eps,\alpha,\beta)$-oblivious transfer (OT) protocol if the following holds. 

% The parties $\Ac$ and $\Bc$ run the protocol (in an honest manner) and    
% $\Ac$ outputs $X=(m_1,m_2)\in \zo\times\zo$ and has a view $U$ and $\Bc$ outputs $Y=(i,\hat{m})\in\zo\times\zo$ and has a view $V$, and following properties are satisfied:
% \begin{enumerate}
%     \item \textbf{Correctness:} 
%     $\pr{\hat{m}\neq m_i}<\eps.$ 
    
%     \item \textbf{A's Privacy:} For every adversary $\Dc$:
%     $\pr{\Dc(V)=m_{i-1}}<(1+\alpha)/2$
    
%     \item \textbf{B's Privacy:} For every adversary $\Dc$: $\pr{\Dc(U)=i}<(1+\beta)/2$  
% \end{enumerate}
% \end{definition}
    \section[Polynomials in \titlevariety]{Polynomials in \titlevariety}\label{sec:polynomials_in_X}
In this section we wish to generalize the definition of degree-$\degree$ polynomials for functions $\funcdef{\onvarfunc}{\variety}{\basefield}$.
Note that we wish to define it using a property of $\onvarfunc$ that is intrinsic to $\variety$: given a function $\funcdef{\onvarfunc}{\variety}{\basefield}$,
we wish be able to determine its degree only using values of $\variety$, without considering any value outside of $\variety$ (such as values of $\field \setminus \variety$).
\newline
To define such property, we generalize the local definition of a degree that is defined for polynomials in $\field$.
We remind the reader that in $\field$, we said a function over $\field$ is a polynomial of degree $\leq \degree$ if and only if its $(\degree+1)$-derivative in every direction is $\equiv 0$.
Thus, in order to determine the $(\degree+1)$-derivative of a function in directions $y_1,...,y_{\degree+1}$, one needs to evaluate the function over all the points of the cube generated by $x,y_1,...,y_{\degree+1}$,
which is the set of points $\set{x + \sum_{i \in S}{y_i}}_{S \subseteq [\degree+1]}$.
This raises a challnge in extending this definition for functions defined over $\variety \subseteq \field$:
depending on $\variety$, the function $\funcdef{\onvarfunc}{\variety}{\basefield}$ is not be defined to all points in all the cubes of $\field$, because some of those points do not lie in $\variety$.
\newline
Therefore, to generalize the definition of a polynomial to $\variety$, we start by giving the formal definition and notation of the set of cubes in $\variety$:
\begin{definition}[Cubes]
    Let $k \in \naturalnumbersset$ be an integer and let $x, y_1,...,y_{k} \in \field$.
    We define the cube $\cube{x}{y_1,...y_{k}}$ as follows:
    \[
        \cube{x}{y_1,...,y_{k}} \definedas \set{x + \sum_{i \in S}{y_i}}_{S \subseteq [k]}
    \]
    We refer to $x$ as the \emph{offset} of the cube, and $y_1,...,y_{k}$ as the \emph{directions} of the cube.
    \newline
    Moreover, Let $\variety \subseteq \field$ be a subset.
    We define the \emph{set of cubes of $\variety$ of size $k$} as follows:
    \[
        \cubes{k}{\variety} \definedas
            \set{\cube{x}{y_1,...,y_{k}} \suchthat \forall S \subseteq[k]: (x + \sum_{i \in S}{y_i}) \in \variety}
    \]
\end{definition}

Using this definition, we can define a polynomial of degree $\leq \degree$ for subsets of $\field$:
\begin{definition}[Polynomials in $\variety$]
    Let $\degree \in \naturalnumbersset$ be an integer, and let $\variety \subseteq \field$.
    We say the degree of a function $\funcdef{\onvarfunc}{\variety}{\basefield}$ is $\degree$
    if $\degree$ is the smallest integer such that $\onvarfunc$ vanishes over all cubes of size $(\degree+1)$, i.e:
    \[
        \forall {\cube{x}{y_1,...,y_{\degree + 1}} \in \cubes{\degree+1}{\variety}}:
            \directionderivative_{y_{\degree + 1}}...\directionderivative_{y_1} \onvarpoly(x) = 0
    \]
    A function over $\variety$ of degree $\leq \degree$ is also called a \emph{polynomial of degree $\leq \degree$}.
    We denote the set of polynomials of degree $\leq \degree$ over $\variety$ by $\allpolyset{\leq \degree}{\variety}{\basefield}$.
\end{definition}
\begin{note*}
    For $\variety = \field$, the definition above coincides with the local definition of polynomials.
\end{note*}

\subsection[Lifting Polynomials]{Lifting Polynomials}\label{subsec:lifting-polynomials}
Our goal to achieve good properties for polynomials over $\variety$.
To do so, we wish to connect the desired properties of polynomials defined over $\variety$, to properties known for polynomials over $\field$.
Following such strategy raises a question: given a polynomial $\funcdef{\onvarpoly}{\variety}{\basefield}$, which polynomial over $\field$ should we consider to deduce properties of $\onvarpoly$?
To find such a polynomial over $\field$, it would have been useful that all polynomials over $\variety$ actually "came from" polynomials over $\field$.
More formally, it would have been useful that all polynomials $\funcdef{\onvarpoly}{\variety}{\basefield}$ would be equal to a restriction of some polynomial $\funcdef{\genpoly}{\field}{\basefield}$ of degree $\leq \degree$, to the set $\variety$.
This would give us a "good candidate" (or candidates) to polynomials over $\field$, that using their known properties, we could achieve the properties we desire for polynomials over $\variety$.
\newline
Generally speaking, the existence of such polynomial $\funcdef{\genpoly}{\field}{\basefield}$ is not trivial by itself, and it mapy depend on the polynomial $\onvarpoly$ and the set $\variety$.
In this subsection, we discuss sets $\variety \subseteq \field$ that have this property for every polynomial $\funcdef{\onvarpoly}{\variety}{\basefield}$.
Before formulating the notion above, we start by a simple remark:
\begin{remark}
    By the local criteria for $\field$, we have that a restriction of a polynomial of degree $\leq \degree$ over $\field$ to $\variety$ is a polynomial of degree $\leq \degree $ over $\variety$.
    Therefore, the other direction is true: every restriction of a polynomial over $\field$ to $\variety$ is a polynomial over $\variety$.
\end{remark}
%First, let us present an exmaple that shows that not every set has this property~\cite[Example 1.4]{kazhdan2019extendingweaklypolynomialfunctions}:
%\begin{example}
%    Fix $\blocklength = 2$.
%    Define $\varpoly(x_1, x_2) = x_1 \cdot x_2 \cdot (x_1 - x_2)$,
%    and consider:
%    \[
%        \variety \definedas \zerofunc{\varpoly} = \set{x = (x_1, x_2) \in \basefield^2 \suchthat x_1 = 0 \vee x_2 = 0 \vee x_1 = x_2}
%    \]
%    Now, consider the function $\funcdef{\onvarpoly}{\variety}{\basefield}$ defined as:
%    $\onvarpoly(x_1, 0) = \onvarpoly(0, x_2) = 0, \onvarpoly(x_1, x_1) = x_1$.
%
%\end{example}

Next, let us define subsets $\variety \subseteq \field$ that have the desired property, which we call \emph{$\degree$-lift-enabler variety}.
\begin{definition}[$\degree$-lift-enabler Subset]
    Let $\basefield$ be a field, and $\blocklength>0$ be an integer.
    For an integer $\degree > 0$, we say a subset $\variety\subseteq\field$ is \emph{$\degree$-lift-enabler} if for every $\degree^{\prime} \leq \degree$,
    for every polynomial $\onvarpoly \in \allpolyset{\degree^{\prime}}{\variety}{\basefield}$
    there exist a polynomial $\lift{\onvarpoly} \in \allpolyset{\degree^{\prime}}{\field}{\basefield}$ such that $\restrictfunc{\onvarpoly}{\variety}=\restrictfunc{\lift{\onvarpoly}}{\variety}$.
\end{definition}
\begin{remark}
    Using the local criterion of polynomials and the fact that that $\cubes{\degree+1}{\variety} \subseteq \cubes{\degree+1}{\field}$,
    one can see that for a polynomial $\funcdef{\onvarpoly}{\variety}{\basefield}$ with $\deg(\onvarpoly) = \degree$,
    every extension $\funcdef{\genpoly}{\field}{\basefield}$ with $\onvarpoly = \restrictfunc{\genpoly}{\variety}$
    holds the bound $\deg(\lift{\onvarpoly}) \geq \degree$.
    The other direction is not true in the general case, but it is specifically promised when the variety is $\degree$-lift-enabler.
\end{remark}

This definition naturally raises the following definition:
\begin{definition}[The Lift Operator]
    Let $\degree \in \naturalnumbersset$ be an integer.
    Let $\variety \subseteq \field$ be a $\degree$-lift-enabler subset.
    We define \emph{the $\degree$-lift operator} to be an operator $\funcdef{\lift{\square}}{\allpolyset{\leq \degree}{\variety}{\basefield}}{\allpolyset{\leq \degree}{\field}{\basefield}}$ the following way:
    \newline
    Let $\degree^\prime \leq \degree$.
    Given a polynomial $\funcdef{\onvarpoly}{\variety}{\basefield}$ of degree $\degree^\prime$, the operator $\lift{\square}$ returns a polynomial $\funcdef{\lift{\onvarpoly}}{\field}{\basefield}$ of degree $\degree^\prime$
    such that $\onvarpoly = \restrictfunc{\lift{\onvarpoly}}{\variety}$.
    Note that we did not require the lift to be unique.
    Thus, in case there are multiple valid lifts for a polynomial $\onvarpoly \in \allpolyset{\leq \degree}{\variety}{\basefield}$, the lift operator picks a single (consistent) one of them.
    Moreover, the lift always exists because the subset $\variety$ is $\degree$-lift-enabler.
    \newline
    In addition, for a collection $\onvarpolyset = (\onvarpoly_1,...,\onvarpoly_c)$ of polynomials $\onvarpoly_i \in \allpolyset{\leq \degree}{\variety}{\basefield}$,
    we denote $\lift{\onvarpolyset} \definedas (\lift{\onvarpoly_1},...,\lift{\onvarpoly_c})$
\end{definition}

In the following subsections, we give example to two concrete sets $\variety \subseteq \field$ that are $\degree$-lift-enablers.
Before doing so, we define an algebraic variety:
\begin{definition}[Algebraic Variety]
    For a collection of functions $\genfuncset \definedas \set{\genfunc_1,...\genfunc_c}$ such that $\funcdef{\genfunc_i}{\field}{\basefield}$,
    we denote $\zerofunc{\genfuncset} \definedas \set{x \in \field \suchthat \forall i: \genfunc_i(x) = 0}$.
    \newline
    If the collection is a collection of polynomials, we call $\zerofunc{\genfuncset}$ an \emph{algebraic variety}, shorthand by \emph{variety}.
    \newline
    The degree of the variety is defined to be the maximal degree of polynomials in the collection that defines it.
    The rank of the variety is defined to be the rank of the collection that defines the variety (as a collection).
\end{definition}

\subsection[High Rank Varieties of High Minimal Degree]{High Rank Varieities of High Minimal Degree}\label{subsec:high-rank-varieities-of-high-degree}
We now present a theorem proved in \cite[Corollary 1.10]{kazhdan2018polynomial}, that shows that high rank varieties are $\degree$-lift-enabler when the polynomials defining the variety are of degree $>\degree$:
\begin{theorem}\label{subsec:high-rank-varities-are-d-lift-enabler}
    Let $\basefield$ be a finite field, and let $\varietydeg$, $\varietypolycount>0$ representing parameters of a variety.
    Let $\degree < \varietydeg$ a positive integer representing a degree of a polynomial which we wish to lift.
    There exists $\varietyrankval=\varietyrankval(\basefield,\varietydeg,\varietypolycount)>0$ such that for for all $\blocklength \in \naturalnumbersset$, any variety $\variety = \zerofunc{\varpolyset} \subseteq \field$ for $\varpolyset = (\varpoly_1,...,\varpoly_{\varietypolycount})$
    such that $\rank{\varpolyset} > \varietyrankval$, degree $\deg(\varpolyset) = \varietydeg$, with all defining polynomials of degree $\deg(\varpoly_i)> \degree$,
    it holds that $\variety$ is a $\degree$-lift-enabler subset.
\end{theorem}
\begin{remark}
    Under the conditions stated above, it was proved in ~\cite{kazhdan2018polynomial} that the lift is in fact \emph{unique}.
    Formally, if $\funcdef{\onvarpoly}{\variety}{\basefield}$ is a polynomial of degree $\leq \degree$,
    then there exists a \emph{unique} polynomial $\funcdef{\genpoly }{\field}{\basefield}$ such that $\restrictfunc{\genpoly}{\variety} \equiv \onvarpoly$.
\end{remark}

\subsection[High Rank Varieties on a Large Field]{High Rank Varieties on a Large Field}\label{subsec:high-rank-varieties-on-a-large-field}
In this subsection, we recall a theorem proved by~\cite[Theorem 1.7]{kazhdan2019extendingweaklypolynomialfunctions} regarding high rank varieties that are defined on "large" fields.
We note that the fields are large in respect of the degree $\degree$ one wish to lift, but still does not depend on $\blocklength$.

Next we define a weakly polynomial, which generalizes our definition of a polynomial in $\variety$, that was used in~\cite[Definition 1.1]{kazhdan2019extendingweaklypolynomialfunctions}:
\begin{definition}
    Let $\variety \subseteq \field$ be a set.
    We say a function $\funcdef{\genfunc}{\variety}{\basefield}$ is a \emph{weakly polynomial of degree $\leq \degree$}
    if for any affine subspace $L \subseteq \variety$, the restriction $\restrictfunc{\genfunc}{L}$ is a polynomial of degree $\leq \degree$.
\end{definition}
\begin{remark}
    By the local criteria of a polynomial, it is easy to see that every $\genpoly \in \allpolyset{\leq \degree}{\variety}{\basefield}$ is a weakly polynomial of degree $\leq \degree$.
\end{remark}
And now, we can present the lifting theorem for large fields, as proved in~\cite[Theorem 2.17]{kazhdan2020propertieshighranksubvarieties}.
\begin{theorem}\label{high-rank-varieties-over-large-fields-are-d-lift-enablers}\cite[Theorem 2.17]{kazhdan2020propertieshighranksubvarieties}
    Let $\degree, \varietydeg \in \naturalnumbersset$,
    and let $\basefield$ be a finite field such that $\abs{\basefield} > \degree \cdot \varietydeg$.
    There exists $\rankfunc_{\ref{high-rank-varieties-over-large-fields-are-d-lift-enablers}} = \rankfunc_{\ref{high-rank-varieties-over-large-fields-are-d-lift-enablers}}(\varietydeg, \degree)$
    such that for any variety $\variety \subseteq \field$ of maximal degree $\leq \varietydeg$ and rank $\geq \rankfunc_{\ref{high-rank-varieties-over-large-fields-are-d-lift-enablers}}$,
    have the following property:
    Every weakly polynomial function $\funcdef{\onvarpoly }{\variety}{\basefield}$ of degree $\leq \degree$
    can be lifted to a polynomial function $\funcdef{\genpoly}{\field}{\basefield}$ of degree $\leq \degree$.
\end{theorem}
\begin{note*}
    Note that we stated the theorem above to finite fields, but it is also valid for infinite algebraically closed fields.
\end{note*}
The theorem above implies the following corollary:
\begin{corollary}
    Let $\degree, \varietydeg \in \naturalnumbersset$,
    and let $\basefield$ be a finite field such that $\abs{\basefield} > \degree \cdot \varietydeg$.
    There exists $\rankfunc_{\ref{high-rank-varieties-over-large-fields-are-d-lift-enablers}} = \rankfunc_{\ref{high-rank-varieties-over-large-fields-are-d-lift-enablers}}(\varietydeg, \degree)$
    such that for any variety $\variety \subseteq \field$ of degree $\varietydeg$ and rank $\geq \rankfunc_{\ref{high-rank-varieties-over-large-fields-are-d-lift-enablers}}$
    is a $\degree$-lift-enabler.
\end{corollary}


    \section[Relative Rank-Bias Property]{Relative Rank-Bias Property}\label{sec:relative-rank-bias-property}
In this section, we generalize the relation between rank and bias that is known for $\field$ also for $\variety \subseteq \field$.
Specifically, in Lemma~\ref{high-rank-implies-low-bias}, it was shown that high rank factors have low bias in $\field$.
We wish to define an alternative definition of rank for $\variety \subseteq \field$, called \emph{$\variety$-relative rank},
such that high $\variety$-relative rank implies low bias in $\variety$.
This type of relation (and definition) was shown previously to a few sets;
in~\cite[Theorem 1.8]{lampert2021relative} for sets $\variety = \zerofunc{\genpolyset[2]}$ where $\genpolyset[2]$ is a collection of polynomials of high rank;
and in~\cite[Theorem 1.4]{gowers2022equidistributionhighrankpolynomialsvariables} for sets $\variety = S^\blocklength$ for $S \subset \basefield$.

To understand this notion, we first introduce a simple example that demonstrates the need for a different definition of rank to achieve equidistribution properties in subsets of $\field$.
\begin{example}
    Let $\variety = \set{x \in \field \suchthat x_1 = 0}$.
    Define $\funcdef{\genpoly}{\field}{\basefield}$ by $\genpoly(x) \definedas x_1$.
    \newline
    In $\field$, $\genpoly$ has rank $\infty$ as it can not be decomposed polynomials of degree $< 1$ (constants).
    Additionally, it is perfectly equidistributed.
    This is the simplest example of the rank-bias relation in $\field$.
    \newline
    However, when restricting $\genpoly$ to $\variety$, we get $\restrictfunc{\genpoly}{\variety} \equiv 0$.
    As $0$ is a constant function, it is the \emph{least} equidistributed possible in $\variety$.
    Therefore, we see that the way we defined rank in $\field$ does not imply the desired equidistribution in $\variety$:
    we found a polynomial with high rank (infinity) that has a very high bias \emph{in $\variety$} (the maximal).
\end{example}
The reason the definition of rank in $\field$ fails to capture equidistribution even on subsets that are really similar to $\field$ (isomorphic to $\basefield^k$), is because of the following reason:
Even though our polynomial $\genpoly$ does \emph{not} have a decomposition to a few lower-degree polynomials by itself,
there \emph{exists} a \emph{$\variety$-equivalent} polynomial that has such structured decomposition.
Here, by $\variety$-equivalent we mean a polynomial in $\field$ that is bounded by the same degree bound, and is equal to $\genpoly$ in $\variety$.
In the example described above, this equivalent polynomial is the constant function $0$, and its decomposition is the trivial one (any function decomposes a constant function).
An alternative perspective which we use throughout this paper to $\variety$-equivalence is that both polynomials are equal up to a \emph{valid $\variety$-remainder}: a bounded degree polynomial that is $\equiv 0$ in $\variety$.

Generally speaking, high $\variety$-relative rank may not imply low bias in $\variety$.
Therefore, this structure-randomness relation is not true for a general subset $\variety \subseteq \field$,
but is a \emph{property} of the subset $\variety$.
Thus, we say that a subset has the \emph{relative rank-bias property} if this relation holds,
i.e. if high $\variety$-relative rank implies equidistribution in $\variety$.

Let us now formally define our definition for relative rank, inspired by the two different definitions of relative rank presented in~\cite[Definition 1.6]{lampert2021relative}
and in~\cite[Definition 1.3]{gowers2022equidistributionhighrankpolynomialsvariables}:
\begin{definition}[Relative Rank of a Polynomial]\label{def:relative-rank-of-polynomial}
    Let $\variety \subseteq \field$ and let $\degree \in \naturalnumbersset$.
    For an integer $\degree \in \naturalnumbersset$ and a polynomial $\funcdef{\genpoly}{\field}{\basefield}$, we define its \emph{$\degree$-relative rank} in respect of $\variety$ as:
    \[
        \drelrank{\degree}{\variety}{\genpoly} \definedas \min \set{\drank{\degree}{\genpoly - \relativeremainder{\genpoly}} \suchthat
        \relativeremainder{\genpoly} \in \allpolyset{\leq \deg(\genpoly)}{\field}{\basefield}, \restrictfunc{\relativeremainder{\genpoly}}{\variety} \equiv 0}
    \]
    For a polynomial $\genpoly$ of degree $\deg(\genpoly) = \degree$ we denote $\relrank{\variety}{\genpoly} \definedas \drelrank{\degree}{\variety}{\genpoly}$.
\end{definition}
\begin{definition}[$\variety$-equivalent and $\variety$-remainder]
    Moreover, we say a polynomial is \emph{$\variety$-equivalent} to $\genpoly$ if its restriction to $\variety$ is $\equiv \restrictfunc{\genpoly}{\variety}$.
    We say it is \emph{valid $\variety$-equivalent to $\genpoly$} if it is $\variety$-equivalent to $\genpoly$ and its of the \emph{same} degree of $\genpoly$.
    \newline
    Similarly, we say a polynomial is \emph{$\variety$-remainder} of $\genpoly$ if its restriction to $\variety$ is \emph{$\equiv 0$}.
    We say it is \emph{valid $\variety$-remainder} if it is $\variety$-remainder of $\genpoly$ and its of degree \emph{smaller or equal} of the degree of $\genpoly$.
    \newline
    %TODO: Denote also the equivalent polynomial and make the notations in the other sections consistent with it.
    We typically denote such polynomial as $\relativeremainder{\genpoly}$.
\end{definition}
In other words, the $\degree$-relative rank of a polynomial $\genpoly$ is the smallest $\degree$-rank of all valid $\variety$-equivalents of $\genpoly$.

\begin{note}\label{note:comparison-to-gowers-rank}
    Note that~\cite[Definition 1.3]{gowers2022equidistributionhighrankpolynomialsvariables} defines rank in a substantially different way than our definition,
    and consequentially our results will not apply to the sets they presented.
    One of the main differences in the definition of rank occurs for $\degree = 1$.
    In the definition we use for rank, the rank of every (non-constant) degree-$1$ polynomial is $\infty$, where in the definition used in~\cite{gowers2022equidistributionhighrankpolynomialsvariables} it is a finite number (which is possibly very small).
    This difference is crucial, as for example, it makes regularization according to their definition not-trivially possible,
    where it is known to be possible when rank is defined by the definition we use (Lemma~\ref{regularization-in-Fn-lemma}).
\end{note}
\begin{definition}[Relative Rank of a Factor]
    Let $\variety \subseteq \field$.
    Let $\genpolyset$ be a set of polynomials $\genpolyset = \set{\genpoly_1,...,\genpoly_c}$.
    The rank of the polynomial set $\genpolyset$ relative to the subset $\variety$ is defined as:
    \[
        \relrank{\variety}{\genpolyset} \definedas \min \set{\drelrank{\degree}{\variety}{\sum_{i=1}^c{\lambda_i \genpoly_i}} \suchthat 0 \neq \vec{\lambda} \in \basefield^c, \degree = \max_{i \in \sparens{c}}{\deg(\lambda_i \genpoly_i)}}
    \]
    For a factor $\factor$ defined by a collection of polynomials, we define its relative rank relative to $\variety$ to be the relative rank of the collection of polynomials defining it, relative to the set $\variety$.
    \newline
    For a non-decreasing function $\funcdef{\rankfunc}{\naturalnumbersset}{\naturalnumbersset}$, a factor $\factor$ is called $\rankfunc$-$\variety$-regular if its relative rank in respect to $\variety$ is at least $\rankfunc(\abs{\factor})$.
\end{definition}

\subsection[Relative Rank-Bias Property]{Relative Rank-Bias Property}
%TODO: Add introduction to this subsection
\begin{definition}[Relative Rank-Bias property]
    Let $\basefield$ be a finite field, and let $\degree \in \naturalnumbersset$ be an integer.
    Let $\funcdef{\rankbiasfunc}{\realnumbersset^{+}}{\naturalnumbersset}$ be a function that represents the rank-bias relation for a fixed $\degree, \basefield$.
    \newline
    We say a set $\variety \subseteq \field$ has the \emph{$(\rankbiasfunc, \basefield, \degree)$-relative rank-bias property} if
    for every $\epsilon > 0$,
    for every polynomial $\genpoly$ of degree $\leq \degree$ with $\relrank{\variety}{\genpoly} \geq \rankbiasfunc(\epsilon)$ we have:
    \[
        \relbias{\variety}{\genpoly} < \epsilon
    \]
\end{definition}

As an immediate corollary of Lemma~\ref{high-rank-implies-low-bias} that shows that high rank implies low bias, we have that $\variety = \field$ has the relative rank-bias property.
\begin{corollary}[$\field$ has the relative rank-bias property]
    For every finite field $\basefield$ and $\degree \in \naturalnumbersset$, let $\funcdef{\rankbiasfunc}{\realnumbersset^{+}}{\naturalnumbersset}$ defined as $\rankbiasfunc(\epsilon) \definedas \rankval_{\ref{high-rank-implies-low-bias}}(\basefield, \degree, \epsilon)$.
    Then, we have that the set $\variety = \field$ has the $(\rankbiasfunc, \basefield, \degree)$-relative rank-bias property.
\end{corollary}
\begin{proof}
    This is a simple usage of Lemma~\ref{high-rank-implies-low-bias}:
    Note that when $\variety = \field$, we have that $\relrank{\variety}{\genpoly} = \rank{\genpoly}$.
    Now, if $\genpoly$ is a polynomial of degree $\leq \degree$ and $\relrank{\variety}{\genpoly} = \rank{\genpoly} \geq \rankbiasfunc(\epsilon) = \rankbiasfunc_{\ref{high-rank-implies-low-bias}}(\basefield, \degree, \epsilon)$,
    then:
    \[
        \relbias{x \in \field}{\genpoly(x)} < \epsilon
    \]
\end{proof}


\subsection[Limited Relative Rank-Bias Property]{Limited-Relative Rank-Bias Property}\label{subsec:limited-relative-rank-bias-property}
Sometimes, however, we can not request $\variety$ to be such that high relative rank implies low bias for every $\epsilon > 0$,
but only for $\epsilon^\prime \geq \epsilon$ for some constant $\epsilon > 0$.
This leads to defining the \emph{limited relative rank-bias property}, which will be used to discuss such sets $\variety \subseteq \field$.
\newline
As we will later see, this definition raises naturally where $\variety$ is a high rank variety,
in which for the relative rank-bias property to hold for some $\epsilon > 0$, the rank of the variety should be greater than a value that is dependent of $\epsilon$.
Thus, to have the relative rank-bias property for a high rank variety but without requiring an infinitely large rank, we must limit the relative rank-bias property for $\epsilon^\prime \geq \epsilon$
We formulate the definition of this property as follows:
\begin{definition}[Limited Relative Rank-bias property]
    Let $\basefield$ be a finite field, let $\degree \in \naturalnumbersset$ be an integer, and let $\epsilon > 0$ be a constant.
    Let $\funcdef{\rankbiasfunc}{[\epsilon, \infty]}{\naturalnumbersset}$ be a function that represents the limited-relative-rank-bias relation.
    \newline
    We say a set $\variety \subseteq \field$ has the \emph{$(\rankbiasfunc, \basefield, \degree, \epsilon)$-limited-relative-rank-bias property} if
    for every $\epsilon^\prime \geq \epsilon$,
    for every polynomial $\genpoly$ of degree $\leq \degree$ with $\relrank{\variety}{\genpoly} \geq \rankbiasfunc(\epsilon^\prime)$ we have:
    \[
        \relbias{\variety}{\genpoly} < \epsilon^{\prime}
    \]
    As a convention, we denote by $\epsilonlimitedrankbias$ the $\epsilon$ such that the limited-relative-rank-bias property holds for $\variety$.
\end{definition}

%TODO: Change relative rank to be rank_X instead of rank_L.
\subsubsection[High Rank Varieities]{High Rank Varieities}
In this subsection, we are discussing specifically $\variety \subseteq \field$ that are in the form $\variety = \zerofunc{\genpolyset[2]}$ for a set of polynomials $\genpolyset[2]$ that form a high rank factor.
Let us present some known results of the relative rank-bias relation for high rank varieites:
In the scenario when we are working relative to $\variety$, the equivalent for Theorem~\ref{high-rank-implies-low-bias} is also known when we assume $\degree < char(\basefield)$, as shown in~\cite[Theorem 1.8]{lampert2021relative}:
\begin{theorem}[High relative rank implies low bias in high rank varieties]\label{high-relative-schmidt-rank-implies-low-relative-bias}
    Let $\basefield$ be a finite field and let $0 \leq \degree < char(\basefield)$.
    Let $\epsilon > 0$ be a constant, and let $\varietypolycount \in \naturalnumbersset$.
    There exist $\varietyrankval^{\ref{high-relative-schmidt-rank-implies-low-relative-bias}} = \varietyrankval^{\ref{high-relative-schmidt-rank-implies-low-relative-bias}}(\basefield, \degree, \varietypolycount, \epsilon)$
    and $\rankval^{\ref{high-relative-schmidt-rank-implies-low-relative-bias}} = \rankval^{\ref{high-relative-schmidt-rank-implies-low-relative-bias}}(\basefield, \degree, \epsilon)$ such that the following holds:
    \newline
    Let $\varpolyset = (\varpoly_1,...,\varpoly_{\varietypolycount})$ be a collection of polynomials of degrees $\leq \degree$ with $\rank{\varpolyset} \geq \varietyrankval^{\ref{high-relative-schmidt-rank-implies-low-relative-bias}}$ and let $\genpoly$ be a polynomial of degree $\leq \degree$.
    \newline
    Then, if $\relrank{\varpolyset}{\genpoly} \geq \rankval^{\ref{high-relative-schmidt-rank-implies-low-relative-bias}}$, we have:
    \[
        \relbias{\varpolyset}{\genpoly} < \epsilon
    \]
\end{theorem}
\begin{note*}
    Note that the original statements in~\cite{lampert2021relative} are stated for a different definition of rank, noted as \emph{schmidt rank}.
    In the appendix~\ref{sec:comparing-ranks} we compare the two different definitions, and show that our definition of rank is comprehensive
    enough in a sense that a polynomial with high rank also has high schmidt rank.
    Additionally, we show that for a given $\rankval \in \naturalnumbersset$, the lower bound of rank required for a polynomial to be of schmidt rank $\geq \rankval$,
    is only $c \cdot \rankval$ for some constant $c \in \naturalnumbersset$.
\end{note*}
\begin{remark}
    Note that in the original statement of theorem~\ref{high-relative-schmidt-rank-implies-low-relative-bias} as stated in~\cite[Theorem 1.8]{lampert2021relative},
    there are good bounds on the rank needed for $\varpolyset$ and $\genpoly$ for the theorem to hold.
    \newline
    Specifically, there exist constants $A(\degree), B(\degree)$ such that for an error $\epsilon = \abs{\basefield^{-s}}$,
    if $\varietyrankval^{\ref{high-relative-schmidt-rank-implies-low-relative-bias}} = A(\varietypolycount + s)^B$ and $\rankval^{\ref{high-relative-schmidt-rank-implies-low-relative-bias}} = A(1 + s)^B$,
    then we have:
    \[
        \relbias{\zerofunc{\varpolyset}}{\genpoly} < \abs{\basefield}^{-s}
    \]
    \newline
    In our proof, it is enough that the bounds on $\rankval$ and $\varietyrankval$ are independent of $\blocklength$, thus we omit the exact bounds stated above and use the statement as stated in Theroem~\ref{high-relative-schmidt-rank-implies-low-relative-bias}.
\end{remark}

\begin{remark}\label{in-high-relative-rank-implies-low-relative-bias-epsilon-increasing-rank-requirment-decreasing}
Note that both $\rankval^{\ref{high-relative-schmidt-rank-implies-low-relative-bias}}(\basefield, \degree, \epsilon)$
and $\varietyrankval^{\ref{high-relative-schmidt-rank-implies-low-relative-bias}}(\basefield, \degree, \varietypolycount, \epsilon)$
are decreasing when $\epsilon$ is increasing.
This means for example, that for all $\epsilon^\prime \geq \epsilon$,
a variety that satisfies the theorem's rank condition for $\epsilon$ also satisfies the theorem's rank condition for $\epsilon^\prime$.
Therefore, a polynomial with rank $\geq \rankval^{\ref{high-relative-schmidt-rank-implies-low-relative-bias}}(\basefield, \degree, \epsilon^\prime)$
will have a bias $< \epsilon^\prime$.
\end{remark}
As a corollary of Theorem~\ref{high-relative-schmidt-rank-implies-low-relative-bias} and Corollary~\ref{relative-schimdt-rank-equals-schmidt-rank-if-the-variety-is-of-high-degree}, we have that
high rank varieties has the limited-relative-rank-bias property.
Formally, we have:
\begin{corollary}[High Rank Varieties Have the Limited-Relative Rank-Bias Property]\label{high-rank-variety-has-limited-rank-relative-bias-property}
    Let $\basefield$ be a finite field, and let $\varietydeg \in \naturalnumbersset$ such that $0 < \varietydeg < \abs{\basefield}$.
    Let $\epsilonlimitedrankbias > 0$ be a constant which represents the desired relative rank-bias limit.
    There exists $\funcdef{\rankbiasfunc_{\ref{high-rank-variety-has-limited-rank-relative-bias-property}}}{[\epsilonlimitedrankbias, \infty]}{\naturalnumbersset}$ with $\rankbiasfunc_{\ref{high-rank-variety-has-limited-rank-relative-bias-property}} \definedas \rankbiasfunc_{\ref{high-rank-variety-has-limited-rank-relative-bias-property}}(\basefield, \varietydeg)$
    such that the following holds:
    \newline
    Let $\varietypolycount \in \naturalnumbersset$ be an integer.
    There exists $\varietyrankval_{\ref{high-rank-variety-has-limited-rank-relative-bias-property}} \definedas \varietyrankval_{\ref{high-rank-variety-has-limited-rank-relative-bias-property}}(\basefield, \varietydeg, \varietypolycount, \epsilonlimitedrankbias)$ such that
    for every $\varpolyset = (\varpoly_1,...,\varpoly_{\varietypolycount})$ polynomial factor of degree $\leq \varietydeg$ with $\rank{\varpolyset} \geq \varietyrankval$, we have:
    \newline
    The variety $\variety = \zerofunc{\varpolyset}$ has the $(\rankbiasfunc_{\ref{high-rank-variety-has-limited-rank-relative-bias-property}}, \basefield, \varietydeg, \epsilonlimitedrankbias)$-limited-relative-rank-bias property.
\end{corollary}
\begin{proof}
    Let $\basefield$ be a finite field, and let $\varietydeg \in \naturalnumbersset$ such that $0 < \varietydeg < \abs{\basefield}$.
    Let $\epsilonlimitedrankbias > 0$.
    We choose:
    \[
        \rankbiasfunc_{\ref{high-rank-variety-has-limited-rank-relative-bias-property}}(\epsilon) \definedas
        \rankval_{\ref{high-relative-schmidt-rank-implies-low-relative-bias}}(\basefield, \varietydeg, \epsilon)
    \]
    Note that for every $\epsilon$ in its domain, $\rankbiasfunc_{\ref{high-rank-variety-has-limited-rank-relative-bias-property}}$ does not depend on $\epsilonlimitedrankbias$.
    Let $\varietypolycount \in \naturalnumbersset$.
    Now, we choose:
    \[
        \varietyrankval_{\ref{high-rank-variety-has-limited-rank-relative-bias-property}}(\basefield, \varietydeg, \varietypolycount, \epsilonlimitedrankbias) \definedas
        \varietyrankval_{\ref{high-relative-schmidt-rank-implies-low-relative-bias}}(\basefield, \varietydeg, \varietypolycount, \epsilonlimitedrankbias)
    \]
    Using Theorem~\ref{high-relative-schmidt-rank-implies-low-relative-bias} that shows high rank implies low bias in $\variety$,
    and the assumption that $\epsilon \geq \epsilonlimitedrankbias$ (specifically Remark~\ref{in-high-relative-rank-implies-low-relative-bias-epsilon-increasing-rank-requirment-decreasing})
    concludes the proof.
\end{proof}


    \section[Regularization Relative to \titlevariety]{Regularization Relative to \titlevariety}\label{sec:regularization-relative-to-X}

In this section, we generalize the definitions and statements regarding factors and regularization in $\field$,
to their corresponding definitions and statements to relative rank in respect of $\variety \subseteq \field$.
\newline
Note that in oppose to the previous chapter that we discussed a general $U$ and $A \subseteq U$,
in this chapter we discuss only $U = A = \field$.
This is done for clearance and to avoid defining definitions we will not use in our main proof.
\begin{definition}[Measurable Relative to $\variety$]
    Let $\genfuncset = \set{\genfunc_1,...,\genfunc_c}$ be a set of functions $\funcdef{\genfunc_i}{\field}{\basefield}$.
    We say a function $\funcdef{\genfunc[2]}{\field}{\basefield}$ is \emph{measurable in respect of $\genfuncset$ relative to $\variety$},
    or \emph{$\variety$-relative $\genfuncset$-measurable},
    if there exists a function $\funcdef{\relativeremainder{\genfunc[2]}}{\field}{\basefield}$ with $\restrictfunc{\relativeremainder{\genfunc[2]}}{\variety} \equiv 0$
    and a function $\funcdef{\Gamma}{\basefield^c}{\basefield}$ such that:
    \[
        \forall a \in A: \genfunc[2](a) = \Gamma(\genfunc_1(a),...,\genfunc_c(a)) + \relativeremainder{\genfunc[2]}(a)
    \]
    And:
    \[
        \deg(\genfunc[2] - \relativeremainder{\genfunc[2]}) \leq \deg(\genfunc[2])
    \]
    We sometimes refer to $\Gamma$ as the \emph{$\variety$-relative measurement function}.
\end{definition}

\begin{note*}
    Note that if $\deg(\genfunc[2] - \relativeremainder{\genfunc[2]}) \leq \deg(\genfunc[2])$ as discussed above,
    then the same bound  also bounds the degree of the remainder, i.e. $\deg(\relativeremainder{\genfunc[2]}) \leq \deg(\genfunc[2])$.
    Therefore $\relativeremainder{\genfunc[2]}$ is a valid $\variety$-remainder of $\genfunc[2]$.
    Moreover, this requirement is equivalent to the definition above,
    as if $\deg(\relativeremainder{\genfunc[2]}) \leq \deg(\genfunc[2])$, then we also have $\deg(\genfunc[2] - \relativeremainder{\genfunc[2]}) \leq \deg(\genfunc[2])$.
\end{note*}
\begin{note*}
    Also note that without the bound on the degree of the remainder,
    being measurable relative to $\variety$ is in fact equivalent for being a measurable in $A = \variety$.
    This is true because under these conditions, the remainder $\relativeremainder{\genfunc[2]}$ has no constraints but $\restrictfunc{\relativeremainder{\genfunc[2]}}{\variety} \equiv 0$,
    thus the condition left on the measurement is just being a measurement to $\genfunc[2]$ in $\variety$.
\end{note*}
\begin{remark}
    If $\genfunc[2]$ is a function that it is $\genfuncset$-measurable relative to $\variety$,
    then every value of $\genfunc[2]$ can be determined by the values of $\genfunc_1,...,\genfunc_c$ up to a remainder $\relativeremainder{\genfunc[2]}$ of degree $\leq \degree$.
    Thus, perhaps we do not know that the function $\genfunc[2]$ is constant inside every atom of $\genfuncset$ as in a regular semantic refinement,
    but we do know that there exists a function $(\genfunc[2] - \relativeremainder{\genfunc[2]})$ that equals to $\genfunc[2]$ on $\variety$, is constant on every atom of $\genfuncset$ and it is a function with a bounded degree i.e. $\deg(\genfunc[2] - \relativeremainder{\genfunc[2]})\leq \deg(\genfunc[2])$.
\end{remark}

Next, we present a new type of refinement, which is a relaxation of semantic refinement.
This relaxation will allow us to discuss the corresponding claim of the polynomial regularity lemma (Lemma~\ref{regularization-in-Fn-lemma}) for relative rank (instead of rank).
\begin{definition}[Semantic Refinement Relative to $\variety$]
    Let $\factor$ and $\factor^\prime$ be polynomial factors on $\field$, defined by sets of polynomials $\genpolyset, \genpolyset^{\prime}$ respectively,
    and let $\degree \in \naturalnumbersset$.
    We say a factor $\factor^\prime$ is a \emph{semantic refinement relative to $\variety$} of the factor $\factor$,
    or \emph{$\variety$-relative semantic refinement},
    if the following holds:
    Every function $\funcdef{\genfunc}{\field}{\basefield}$ that is $\genpolyset$-measurable relative to $\variety$,
    is also $\genpolyset^{\prime}$-measurable relative to $\variety$.
    If the definition above holds, we denote $\factor^{\prime} \relsemrefineex{\variety} \factor$.
\end{definition}
\begin{note*}
    It is easy to see that this relation is transitive, i.e. if
    $\factor^{\prime} \relsemrefineex{\variety} \factor$ and
    $\factor^{\prime\prime} \relsemrefineex{\variety} \factor^{\prime}$,
    then $\factor^{\prime\prime} \relsemrefineex{\variety} \factor$.
\end{note*}
\begin{remark}
    In $\variety$, semantic refinements relative to $\variety$ behave the same as regular semantic refinements in the perspective of being measurable:
    every function that is $\genpolyset$-measurable in $\variety$ is also $\genpolyset^{\prime}$-measurable in $\variety$.
    However, the two definitions behave differently in the perspective of being measurable in $\field$.
    Specifically, in relative semantic refinements,
    if $\genfunc[2]$ is a $\genpolyset$-measurable function it is not necessarily $\genpolyset^{\prime}$-measurable.
    However, it is measurable up to a remainder $\relativeremainder{\genfunc[2]}$ of degree $\leq \deg(\genfunc[2])$ such that $\restrictfunc{\relativeremainder{\genfunc[2]}}{\variety} \equiv 0$.
\end{remark}
\begin{corollary}\label{relative-semantic-refinement-is-restricted-semantic-refinement}
    If $\factor^{\prime} \relsemrefineex{\variety} \factor$, then in $\variety$ it is a regular semantic refinement, i.e. $\factor^{\prime} \semrefineex{\variety} \factor$.
\end{corollary}

Next, we present a new regularization process that allows us to increase the \emph{relative} rank of a factor without increasing the size of the factor too much (independent of $\blocklength$).
This regularization process generalizes the regularization process in $\field$, which was first presented by~\cite[2.3]{green2007distribution}.
We call this type of regularization process a \emph{relative-regularization process} relative to $\variety$, shorthand by $\variety$-regularization
For a specific function $\rankfunc$, we will sometimes call applying this lemma a \emph{$\rankfunc$-$\variety$-regularization}.
Note that to allow such a relative-regularization process to hold, we must use the relaxed definition of semantic refinement that is presented above.
\begin{theorem}\label{theorem:regularization-in-X}
    Let $\funcdef{\rankfunc}{\naturalnumbersset}{\naturalnumbersset}$ be a non-decreasing function and let $\degree \in \naturalnumbersset$.
    There exists $\funcdef{C_{\rankfunc, \degree}^{\ref{theorem:regularization-in-X}}}{\naturalnumbersset}{\naturalnumbersset}$ such that the following holds:
    Let $\factor$ be a factor defined by polynomials $\genpolyset = (\genpoly_1,...,\genpoly_c)$ where for all $i \in [c]$: $\funcdef{\genpoly_i}{\field}{\basefield}$ and $\deg(\genpoly_i) \leq \degree$.
    Then, there is an $\rankfunc$-$\variety$-regular factor $\factor^\prime$ defined by polynomials $\genpolyset^{\prime} = (\genpoly^{\prime}_{1},...,\genpoly^{\prime}_{c^\prime})$ where
    for all $i \in [c]$: $\funcdef{\genpoly^{\prime}_i}{\field}{\basefield}$ and $\deg(\genpoly^{\prime]}_{i}) \leq \degree$ such that
    $\factor^\prime \relsemrefineex{\variety} \factor$ and $c^\prime \leq C_{\rankfunc, \degree}^{\ref{theorem:regularization-in-X}}(c)$.
    \newline
    Moreover, if $\factor \synrefine \bar{\factor}$ for some polynomial factor $\bar{\factor}$ with
    relative rank of at least $\rankfunc(c^\prime)+c^\prime+1$ and rank of at least ${\rankfunc_{\ref{preserving-degree-starting-field}}(\basefield, \degree, c^{\prime})} + c^\prime + 1$,
    then we can require that $\factor^\prime \synrefine \bar{\factor}$.
\end{theorem}
\begin{proof}
    We follow the lines of the proof given by~\cite{book}[Lemma 7.29], but here, we wish to increase the \emph{relative} rank of the factor instead of its rank.
    We present an iterative process, which will eventually lead us to a factor of size $c^{\prime}$ with relative rank higher than $\rankfunc(c^\prime)$,
    that is a semantic refinement relative to $\variety$.
    Let $\degree \in \naturalnumbersset$,
    and let $\factor$ be a polynomial factor defined by $\genpolyset = (\genpoly_1,...,\genpoly_c)$ such that $\funcdef{\genpoly_i}{\field}{\basefield}$ of degree $\leq \degree$.
    Define $M(\factor) \definedas (M_{\degree},...,M_1) \in \naturalnumbersset^{\degree}$,
    where $M_i$ denotes the number of polynomials in $\genpolyset$ that have degree exactly $i$.
    Thus, $\sum_{i=1}^{\degree}M_i = c$.
    We define the lexicographical order on $\naturalnumbersset^{\degree}$ where $M > M^{\prime}$ if and only if $M_i > M^{\prime}_i$ for some $1 \leq i \leq \degree$,
    and $M_j = M^{\prime}_j$ for all $j > i$.
    This proof will be by transfinite induction on $M$ under the lexicographical order.
    Next we describe a step of the regularization process.
    \newline
    Let $\factor$ be a polynomial factor defined by $\genpolyset = (\genpoly_1,...,\genpoly_c)$.
    Note that this is an abuse of notations: the factor $\factor$ and the set $\genpolyset$ refer to the original factor in the first step, and also to the current factor in the middle of the relative-regularization process.
    If $\factor$ is $\rankfunc$-$\variety$-regular, then we are done.
    Otherwise, we change $\factor$ as follows:
    First, we denote $\rankfunc_{\ref{preserving-degree-starting-field}}^{\basefield, \degree}(c) \definedas \rankfunc_{\ref{preserving-degree-starting-field}}(\basefield, \degree, c)$,
    and we $\rankfunc_{\ref{preserving-degree-starting-field}}$-regularize $\genpolyset$ using lemma~\ref{regularization-in-Fn-lemma}
    to get a set of polynomials $\genpolyset_1 = (\genpoly^1_1,...,\genpoly^1_{c_1})$ of degree $\leq \degree$,
    which defines a factor $\factor_1$ and has a rank $\geq \rankfunc_{\ref{preserving-degree-starting-field}}^{\basefield, \degree}(c_1)$.
    %TODO: Do I need to prove the regular regularization in order to show that?
    Note that $M(\factor_1) \leq M(\factor)$.
    Then, again, if somehow $\factor_1$ is now $\rankfunc$-$\variety$-regular, we are done.
    \newline
    Otherwise, by definition, there exists some linear combination of the polynomials in $\genpolyset_{1}$ that
    has $\degree^\star$-relative rank less than $\rankfunc(c_1)$,
    where $\degree^\star$ is the maximal degree that participates in the linear combination.
    Let $\vec{\genpoly}(x) = \sum_{i=0}^{c_1}{\lambda_i \genpoly^{1}_i(x)}$ where $\vec{0} \neq \vec{\lambda}\in \basefield^{c_1}$,
    be the linear combination with $\drelrank{\degree^\star}{\variety}{\vec{\genpoly}} \leq \rankfunc(c_1)$ where $\degree^\star \definedas \max_{i \in \sparens{c_1}}{\deg(\lambda_i \genpoly^1_i)}$.
    By definition of relative rank, there exists $\relativeremainder{\genpoly} \in \allpolyset{\leq \deg(\vec{\genpoly})}{\field}{\basefield}$ with $\restrictfunc{\relativeremainder{\genpoly}}{\variety} \equiv 0$ such that
    $\drank{\degree^\star}{\vec{\genpoly} - \relativeremainder{\genpoly}} \leq \rankfunc(c_1)$.
    Note that $\deg(\relativeremainder{\genpoly}) \leq \degree^\star$.
    By definition of $\degree^\star$-rank, we have that we can decompose $\vec{\genpoly} - \relativeremainder{\genpoly}$ as a function of $\rankfunc(c_1)$ polynomials of degree $\leq \degree^\star - 1$.
    In other words, there exist a measurement function $\funcdef{\vec{\Gamma}}{\basefield^{\rankfunc(c_1)}}{\basefield}$ and polynomials $\genpoly[2]_1,...,\genpoly[2]_{\rankfunc(c_1)}$
    with $\deg(\genpoly[2]_i) \leq \degree^\star - 1$ such that:
    \[
        \forall a \in \field: \vec{\genpoly}(a) - \relativeremainder{\genpoly}(a) = \vec{\Gamma} \parens {\genpoly[2]_1(a),...,\genpoly[2]_{\rankfunc(c_1)}(a)}
    \]
    Now, let $\genpolyset^{\star} \subseteq \genpolyset_1$ be the set of all such maximal-degree polynomials,
    and let $i^{\star}$ be chosen such that $\genpoly^{1}_{i^\star} \in \genpolyset^{\star}$.
    Note that the set $\genpolyset^\star$ is non empty, as by definition, $\degree^{\star}$ is the maximal degree of polynomial in the expression $\sum_{i=1}^{c_1}{\lambda_{i} \genpoly^{1}_i}$ such that $\lambda_i \neq 0$.
    \newline
    For the next step, define the polynomial factor $\factor_2$ be the polynomial factor defined by the set:
    \[
        \genpolyset_2 \definedas \genpolyset_1 \setminus \set{\genpoly^1_{i^\star}} \cup \set{\genpoly[2]_1,...,\genpoly[2]_{\rankfunc(c_1)}}
    \]
    Finally, the factor $\factor_2$ will be the factor returned from the relative-regularization step.
    \newline
    It is easy to see that if the process above halts, we get a $\rankfunc$-$\variety$-regular factor.
    Now, we prove the first part of the lemma by showing the following claims:
    \begin{claim}
        The factor generated from the regularization above is of bounded size: a bound that may depend on $\rankfunc, \degree, c$, but does not depend on $\blocklength$.
        Formally, we claim that there exists $\funcdef{C^{\ref{theorem:regularization-in-X}}_{\rankfunc, \degree}}{\naturalnumbersset}{\naturalnumbersset}$
        such that we have $c^{\prime} \leq C^{\ref{theorem:regularization-in-X}}_{\rankfunc, \degree}(c)$.
    \end{claim}
    \begin{proof}
        It is enough to prove the following:
        \begin{enumerate}
            \item~\label{relative-regularization-step-factor-size-is-bounded}
            In each step, the amount of polynomials there are in $\genpolyset_1, \genpolyset_2$ are bounded by a bound that depend only on $\rankfunc, \degree, c$ (independent of $\blocklength$).
            \item~\label{relative-regularization-amount-of-steps-is-bounded}
            The number of steps of the relative-regularization process is also bounded by a bound that depends only on $\rankfunc, \degree, c$ (independent of $\blocklength$).
        \end{enumerate}
        The combination of these two will obtain the desired bound of the amount of polynomials in the last-step regularized factor, which is $C^{\ref{theorem:regularization-in-X}}_{\rankfunc, \degree}(c)$.
        Note that the bound on the last-step relative-regularized factor in not simply the multiplication of the two bounds,
        but a recursively-substitution of the bound in~\ref{relative-regularization-step-factor-size-is-bounded},
        a bounded amount of times (bounded by the bound in~\ref{relative-regularization-amount-of-steps-is-bounded}).
        \newline
        For~\ref{relative-regularization-step-factor-size-is-bounded}, we first notice that the number of polynomials in the regular regularization process is bounded,
        specifically we have $\abs{\genpolyset^1} = c_{1} \leq C^{\ref{regularization-in-Fn-lemma}}_{\rankfunc_{\ref{preserving-degree-starting-field}}, \degree}(c)$.
        Moreover, the polynomial factor $\factor_2$ is generated by adding at most $\rankfunc(c_1)$ polynomials to the factor, and thus we have $\abs{\genpolyset_2} \leq c_1 + \rankfunc(c_1)$ which is also bounded by substituting the bound on $c_1$.
        \newline
        For~\ref{relative-regularization-amount-of-steps-is-bounded}, we use the transfinite induction on $M$ we mentioned earlier to show that the process must halt after a bounded number of steps.
        Formally, we show that there exist $M^{\prime}$ which depends only on $M(\factor)$ such that $M(\factor_2) \leq M^{\prime} < M(\factor)$.
        This will bound the number of steps by a value that depend only on $M(\factor)$, which depends only on $\rankfunc, \degree, c$.
        To do so, we first notice that the regular regularization does not increase the value of $M$, i.e. $M(\factor_1) \leq M(\factor)$.
        Thus, we can focus on the second part of the relative-regularization.
        In this part, we replace a single degree $\degree^{\star}$ polynomial by at most $\rankfunc(c_1)$ polynomials of degree $\leq \degree^{\star} - 1$.
        Therefore, by choosing $M^{\prime} \definedas (M_{\degree}, ..., M_{\degree^{\star}+1}, M_{\degree^{\star}}-1, M_{\degree^{\star}-1}+\rankfunc(c_1),...,M_1+\rankfunc(c_1))$
        we get that $M(\factor_2) \leq M^{\prime} < M(\factor_1) \leq M(\factor)$, which concludes~\ref{relative-regularization-amount-of-steps-is-bounded}.
    \end{proof}
    \begin{claim}
        The factor generated from the regularization above is a $\variety$-relative semantic refinement of the original factor, i.e $\factor^\prime \relsemrefineex{\variety} \factor$.
    \end{claim}
    \begin{proof}
        It is enough to show that in each step, the factors generated by the relative-regularization process are semantic refinements relative to $\variety$ of the previous step's factor.
        Specifically, we show $\factor_{2} \relsemrefineex{\variety} \factor_{1} \relsemrefineex{\variety} \factor$ and the claim will follow from transitivity of relative semantic refinements.
        \newline
        We start by proving $\factor_1 \relsemrefineex{\variety} \factor$.
        Let $\funcdef{\genfunc}{\field}{\basefield}$ be a function that is $\genpolyset$-measurable relative to $\variety$.
        We denote $\degree_{\genfunc} \definedas \deg(\genfunc)$.
        By definition, there exists $\funcdef{\Gamma}{\basefield^c}{\basefield}$,
        $\funcdef{\relativeremainder{\genfunc}}{\field}{\basefield}$ where $\deg(\relativeremainder{\genfunc}), \deg(\genfunc - \relativeremainder{\genfunc}) \leq \degree_{\genfunc}$ and $\restrictfunc{\relativeremainder{\genfunc}}{\variety} \equiv 0$, such that:
        \[
            \forall a \in \field: \genfunc(a) = \Gamma(\genpoly_1(a),...,\genpoly_c(a)) + \relativeremainder{\genfunc}(a)
        \]
        Clearly, the function $\Gamma(\genpoly_1(a),...,\genpoly_c(a))$ is $\genpolyset$-measurable in $\field$,
        and because we have $\factor \semrefine \factor_1$, it is also $\genpolyset_1$-measurable in $\field$.
        Thus there exists $\funcdef{\Gamma_1}{\basefield^{c_1}}{\basefield}$ such that:
        \[
            \forall a \in \field: \genfunc(a) = \Gamma_1(\genpoly^{1}_{1}(a),...,\genpoly^{1}_{c_1}(a)) + \relativeremainder{\genfunc}(a)
        \]
        And therefore we have $\factor_1 \relsemrefineex{\variety} \factor$.
        \newline
        Now, we prove $\factor_2 \relsemrefineex{\variety} \factor_1$.
        Let $\funcdef{\genfunc}{\field}{\basefield}$ be a function that is $\genpolyset_1$-measurable relative to $\variety$.
        Again, we denote $\degree_{\genfunc} \definedas \deg(\genfunc)$,
        and by definition there exists $\funcdef{\Gamma_1}{\basefield^c}{\basefield}$,
        $\funcdef{\relativeremainder{\genfunc_1}}{\field}{\basefield}$ where $\deg(\genfunc - \relativeremainder{\genfunc_1}) \leq \degree_{\genfunc}$ and $\restrictfunc{\relativeremainder{\genfunc_1}}{\variety} \equiv 0$, such that:
        \begin{equation} \label{eq:f-decomposition-a}
            \forall a \in \field: \genfunc(a) = \Gamma_1(\genpoly^{1}_{1}(a),...,\genpoly^{1}_{c_1}(a)) + \relativeremainder{\genfunc}_1(a)
        \end{equation}
        Note that we also have $\deg(\relativeremainder{\genfunc_1}) \leq \degree_{\genfunc}$.
        We will refer this equation, and its simplifications we do throughout the proof, as \emph{the $\genpolyset_1$-decomposition of $\genfunc$}.
        \newline
        We wish to show that there exists $\funcdef{\Gamma_2}{\basefield^{c_2}}{\basefield}$ and
        $\funcdef{\relativeremainder{\genfunc}_2}{\field}{\basefield}$ where $\deg(\genfunc - \relativeremainder{\genfunc}_2) \leq \degree_{\genfunc}$ and $\restrictfunc{\relativeremainder{\genfunc}_2}{\variety} \equiv 0$, such that:
        \[
            \forall a \in \field: \genfunc(a) = \Gamma_2 \parens {\genpoly^1_1(a),...\genpoly^1_{i^\star - 1}(a), \genpoly^1_{i^\star + 1}(a),..., \genpoly^1_c(a), \genpoly[2]_1(a),...,\genpoly[2]_{\rankfunc(c_1)}(a)} + \relativeremainder{\genfunc}_2(a)
        \]
        We will do so using the $\genpolyset_1$-decomposition of $\genfunc$.
        Note that showing $\deg(\genfunc - \relativeremainder{\genfunc}) \leq \degree_{\genfunc}$ is equivalent of showing $\deg(\relativeremainder{\genfunc}_2) \leq \degree_{\genfunc}$.
        \newline
        First, by the way we built $\genpolyset_2$, using the same notations in the regularization step, we have:
        \[
            \forall a \in \field: \genpoly_{i^\star}^1(a) =
                \vec{\Gamma} \parens {\genpoly[2]_1(a),...,\genpoly[2]_{\rankfunc(c_1)}(a)}
                + \relativeremainder{\genpoly}(a)
                - \sum_{i \neq i^\star}{\genpoly_{i}^1(a)}
        \]
        Next, we substitute the value of $\genpoly_{i^\star}^1$ in the $\genpolyset_1$-decomposition of $\genfunc$ (\ref{eq:f-decomposition-a}),
        and get another decomposition of $\genfunc$ that does not depend on $\genpoly_{i^\star}^1$.
        Specifically we have:
        \begin{align} \label{eq:f-decomposition-b}
            \forall a \in \field: \genfunc(a) &=
        \Gamma_1 \parens
            {\genpoly^{1}_{1}(a)
                ,...,
                \parens{
                    \vec{\Gamma} \parens {\genpoly[2]_1(a),...,\genpoly[2]_{\rankfunc(c_1)}(a)}
                    + \relativeremainder{\genpoly}(a)
                    - \sum_{i \neq i^\star}{\genpoly_{i}^1(a)}}
                ,...,
                \genpoly^{1}_{c_1}(a))}\\
            &+ \relativeremainder{\genfunc}_1(a)
        \end{align}
        We wish to use the equation above to show that $\genfunc$ is $\genpolyset_2$-measurable relative to $\variety$.
        However, in order to show that the equation above is in the desired structure that proves that $\genfunc$ is $\genpolyset_2$-measurable,
        the expression inside $\Gamma_1$ must not depend on $\relativeremainder{\genpoly}$ because $\relativeremainder{\genpoly} \notin \genpolyset_2$.
        Note that this is enough as the rest of the polynomials in the expression above are in $\genpolyset_2$,
        and therefore without $\relativeremainder{\genpoly}$ the expression is $\genpolyset_2$-measurable.
        \newline
        To do so, we start by simplifying some of the notations.
        We denote:
        \[
            \vec{\genpoly}_2(a) \definedas \vec{\Gamma} \parens {\genpoly[2]_1(a),...,\genpoly[2]_{\rankfunc(c_1)}(a)} - \sum_{i \neq i^\star}{\genpoly_{i}^1(a)}
        \]
        This is the part of the sum that decomposes $\genpoly_{i^\star}^1(a)$ that is $\genpolyset_2$-measurable,
        thus the following equality applies:
        \[
            \genpoly_{i^\star}^1(a) = \vec{\genpoly}_2(a) + \relativeremainder{\genpoly}(a)
        \]
        where $\deg(\vec{\genpoly_2}), \deg(\relativeremainder{\genpoly}) \leq \degree^{\star}$.
        Using this notation, we write the $\genpolyset_1$-decomposition of $\genfunc$ (\ref{eq:f-decomposition-b}), and get:
        \begin{equation} \label{eq:f-decomposition-c}
            \forall a \in \field: \genfunc(a) = \Gamma_1 \parens
            {\genpoly^{1}_{1}(a)
                ,...,
                \parens{
                    \vec{\genpoly}_2(a)
                    - \relativeremainder{\genpoly}(a)}
                ,...,
                \genpoly^{1}_{c_1}(a))}
            + \relativeremainder{\genfunc}_1(a)
        \end{equation}
        Now, we use the following key observation:
        $\rank{\genpolyset_1} \geq \rankfunc_{\ref{preserving-degree-starting-field}}(\basefield, \degree, c_1)$,
        and as $\deg(\Gamma_1(\genpoly_1^1,...,\genpoly^1_{c_1})) \leq \degree_{\genfunc}$
        we can use Lemma~\ref{preserving-degree-starting-field} to achieve that $\Gamma_1$ is a polynomial of the form:
        \begin{equation*} \label{eq:regularization-gamma-is-a-polynomial}
            \Gamma_1(z_1,...,z_{c_1})  =
            \sum_{\alpha \in \sparens{\basefieldsize - 1}^{c_1}}
            {C_{\alpha} \cdot {\prod_{i = 1}^{c_1}}{z_i^{\alpha_i}}}
        \tag{$\star$}
        \end{equation*}
        where $C_{\alpha} = 0$ whenever $\sum_{i = 1}^{c_1}(\alpha_i \cdot \deg(\genpoly_i^1)) > \degree_{\genfunc}$.
        \newline
        Next, we substitute the polynomial structure of $\Gamma_1$ \eqref{eq:regularization-gamma-is-a-polynomial}
        in the $\genpolyset_1$-decomposition of $\genfunc$~\eqref{eq:f-decomposition-c},
        and observe what happens to each summand monomial with non-zero coefficients of $\Gamma_1$ in the expression after the substitution.
        \newline
        We will show that each such monomial is either $\genpolyset_2$-measurable,
        or a sum of a $\genpolyset_2$-measurable function with a valid $\variety$-remainder, i.e. a polynomial of degree $\leq \degree_{\genfunc}$ that is $\equiv 0$ in $\variety$.
        Note that if this is true for each monomial,
        every linear combination of such monomials is also a sum of $\genpolyset_2$-measurable function with a valid $\variety$-remainder.
        Thus, this will also be true for the entire decomposition of $\genfunc$, as it is a linear combination of such monomials summed with a valid remainder $\relativeremainder{\genfunc}_1$.
        This will conclude the proof.
        \newline
        Let $\alpha = (\alpha_1,...,\alpha_{c_1})$ be a vector of degrees that represents such a monomial.
        If $\alpha_{i^\star} = 0$, then the monomial is in the form:
        \[
            \prod_{i \in [c_1]}{{\genpoly_i}^{\alpha_i}} =
            \prod_{i \in [c_1] \setminus \set{i^{\star}}}{{\genpoly_i}^{\alpha_i}}
        \]
        and therefore it is clearly $\genpolyset_2$-measurable as all the polynomials in the expression above are in $\genpolyset_2$.
        \newline
        Next, if $\alpha_{i^\star} \neq 0$, then the monomial is in the form:
       \begin{equation} \label{eq:monomial-of-gamma}
            \prod_{i \in [c_1]}{{\genpoly_i}^{\alpha_i}} =
            (\vec{\genpoly_2} + \relativeremainder{\genpoly})^{\alpha_{i^\star}} \cdot
                \parens{\prod_{i \in [c_1] \setminus \set{i^{\star}}}{{\genpoly_i}^{\alpha_i}}}
       \end{equation}
        where $\sum_{i \in [c_1]}(\alpha_i \cdot \deg(\genpoly_i^1)) \leq \degree_{\genfunc}$.
        As $\deg(\vec{\genpoly_2} + \relativeremainder{\genpoly}) = \deg(\genpoly_{i^\star}) =\degree^{\star}$, we have:
        \[
            \deg \parens{\prod_{i \in [c_1] \setminus \set{i^{\star}}}{{\genpoly_i}^{\alpha_i}}} =
                \sum_{i \in [c_1] \setminus {i^{\star}}}(\alpha_i \cdot \deg(\genpoly_i^1))
                \leq \degree_{\genfunc} - \alpha_{i^\star} \cdot \degree^{\star}
        \]
        Now, we open the left brackets in (\ref{eq:monomial-of-gamma}), i.e $(\vec{\genpoly_2} + \relativeremainder{\genpoly})^{\alpha_{i^\star}}$.
        This enables us to separate the monomial to the part that only depend on $\vec{\genpoly_2}$ summed with a polynomial with bounded degree multiplied by $\relativeremainder{\genpoly}$ (and therefore a valid remainder).
        To be more specific, the monomial is in the form:
        \[
            (\vec{\genpoly_2} + \relativeremainder{\genpoly})^{\alpha_{i^\star}} = \vec{\genpoly_2}^{\alpha_{i^\star}} + \relativeremainder{\genpoly_{\alpha}}
        \]
        for some polynomial $\relativeremainder{\genpoly_{\alpha}}$ such that:
        \begin{enumerate}
            \item $\relativeremainder{\genpoly_{\alpha}}$ is of degree
                    $\deg(\relativeremainder{\genpoly_{\alpha}}) \leq \max \set{\deg(\vec{\genpoly_2}), \deg(\relativeremainder{\genpoly)}} \cdot \alpha_{i^\star} \leq \alpha_{i^\star} \cdot \degree^{\star}$
            \item $\relativeremainder{\genpoly_{\alpha}}$ is a multiple of $\relativeremainder{\genpoly}$, and therefore $\restrictfunc{\relativeremainder{\genpoly_{\alpha}}}{\variety} \equiv 0$
        \end{enumerate}
        Therefore, by substituting the left brackets back to the equation (\ref{eq:monomial-of-gamma}) and as $\vec{\genpoly_2}$ and $\genpoly_i$ for $i \neq i^{\star}$ are $\genpolyset_2$-measurable,
        one can see that the monomial is a sum of a $\genpolyset_2$-measurable polynomial with a valid remainder.
        Specifically, the remainder $\equiv 0$ in $\variety$, and its degree is $\leq \alpha_{i^\star} \cdot \degree^{\star} + \degree_{\genfunc} - \alpha_{i^\star} \cdot \degree^{\star} = \degree_{\genfunc}$.
        This concludes the proof of the claim.
        \end{proof}
    Now, it remains to prove the second part of the Theorem~{\ref{theorem:regularization-in-X}}.
    \begin{claim}
        If $\factor \synrefine \bar{\factor}$ for some polynomial factor $\bar{\factor}$ with
        relative rank of at least $\rankfunc(c^\prime)+c^\prime+1$ and rank of at least ${\rankfunc_{\ref{preserving-degree-starting-field}}(\basefield, \degree, c^{\prime})} + c^\prime + 1$,
        then we can require that $\factor^\prime \synrefine \bar{\factor}$.
    \end{claim}
    \begin{proof}
        We will show claim step-by-step.
        We denote by $\genpolyset, \bar{\genpolyset}, \genpolyset_1, \genpolyset_2$ the polynomial sets that generate the factors $\factor, \bar{\factor}, \factor_1, \factor_2$.
        Note that $\factor_1, \factor_2$ are the factors in the current step of the regularization process, and thus change in each step of the proof.
        We show that in each step, if $\factor \synrefine \bar{\factor}$ for some polynomial factor $\bar{\factor}$ with
        relative rank of at least $\rankfunc(c^\prime)+c^\prime+1$ and rank of at least ${\rankfunc_{\ref{preserving-degree-starting-field}}(\basefield, \degree, c^{\prime})} + c^\prime + 1$,
        then we can require that $\factor_1 \synrefine \bar{\factor}$, and also that $\factor_2 \synrefine \bar{\factor}$.
        \newline
        For the first part, we have $\factor_1 \synrefine \bar{\factor}$ by a simple usage of the second part of lemma~\ref{regularization-in-Fn-lemma},
        as:
        \[
            \rank{\bar{\genpolyset}}
            > \rankfunc_{\ref{preserving-degree-starting-field}}(\basefield, \degree, c^{\prime}) + c^\prime + 1
            \geq \rankfunc_{\ref{preserving-degree-starting-field}}(\basefield, \degree, c_1) + c_1 + 1
        \]
        \newline
        Now we prove the second part.
        We show that in the current regularization step, we could replace $\genpoly^1_{i^\star} \in \genpolyset_1$ such that $\genpoly^1_{i^\star} \notin \bar{\genpolyset}$.
        Note that this is possible whenever $\genpolyset^\star\cap \bar{\genpolyset} \neq \emptyset$ as the choice of $i^\star$ is arbitrary in polynomials which are in $\genpolyset^\star$.
        \newline
        Assume that is not possible and the factor $\genpolyset_1$ is still not $\rankfunc$-$\variety$-regular.
        Then, we have a linear combination $\vec{\genpoly}(x) \definedas \sum_{i=0}^{c_1}{\lambda_{i}\genpoly^1_i(x)}$ with $\drelrank{\degree^\star}{\variety}{\vec{\genpoly}} \leq \rankfunc(c_1)$
        where $\degree^\star = \max_{i \in \sparens{c_1}} {\deg(\lambda_i \genpoly^1_i)}$.
        We denote by $I^{\star} \subseteq [c_1]$ the set of indexes of such maximal-degree polynomials.
        By this notation, our assumption states that for all $i \in I^{\star}$ we have $\genpoly^1_i \in \bar{\genpolyset}$.
        Additionally, note that for all $i \notin I^{\star}$ we have $\deg(\genpoly^1_i) < \degree^{\star}$.
        Therefore, as the linear combination is of $\degree^\star$-relative rank $\leq \rankfunc(c_1)$,
        there exists a polynomial $\relativeremainder{\genpoly}$ of degree $\leq \deg(\vec{\genpoly}) \leq \degree^\star$ with $\restrictfunc{\relativeremainder{\genpoly}}{\variety} \equiv 0$
        such that $\drank{\degree^\star}{\vec{\genpoly} - \relativeremainder{\genpoly}} \leq \rankfunc(c_1)$.
        In other words, there exist a measurement function $\funcdef{\vec{\Gamma}}{\basefield^{\rankfunc(c_1)}}{\basefield}$ and polynomials $\genpoly[2]_1,...,\genpoly[2]_{\rankfunc(c_1)}$
        with $\deg(\genpoly[2]_i) \leq \degree^\star$ such that:
        \[
            \forall a \in \field: \vec{\genpoly}(a) - \relativeremainder{\genpoly}(a) = \vec{\Gamma} \parens {\genpoly[2]_1(a),...,\genpoly[2]_{\rankfunc(c_1)}(a)}
        \]
        By a simple calculation we have:
        \[
            \forall a \in \field: \sum_{i \in I^{\star}}{\genpoly^1_i(a)} - \relativeremainder{\genpoly}(a) =  \vec{\Gamma} \parens {\genpoly[2]_1(a),...,\genpoly[2]_{\rankfunc(c_1)}(a)} +  \sum_{i \notin I^\star}{\genpoly^1_i(a)}
        \]
        and by this we found a linear combination of polynomials in $\bar{\genpolyset}$ with maximal degree $\degree^\star$,
        that has $\degree^\star$-relative-rank $\leq \rankfunc(c_1) + c_1 + 1$.
        This is a contradiction to our assumptions on $\bar{\factor}$, which completes the proof of the claim.
    \end{proof}
    This completes the proof of the lemma.
\end{proof}

    \section[Radius of Reed-Muller over \titlevariety]{Radius of Reed-Muller over \titlevariety}\label{sec:radius-of-RM-over-X}
We recall that the normalized distances of Reed-Muller codes over $\field$ and over $\variety$
are denoted by $\normalizedcodedistance{\basefield}{\degree}$ and $\normalizedcodedistanceex{\basefield}{\variety}{\degree}$ respectively.
We present a theorem that shows that Reed-Muller codes over a subset $\variety \subseteq \field$ that is $\degree$-lift-enabler and has the (limited) relative rank-bias property,
has (approximately) an \emph{equal} normalized distance as Reed-Muller codes over $\field$.
%TODO: Change to +- epsilon instead of >
\begin{theorem}\label{thm:distance-of-RM-in-X}
    There exist a function $\epsilon_1(\basefield, \degree, \rankbiasfunc, \epsilonlimitedrankbias)$  such that the following holds:
    Let $\basefield$ be a finite field, and let $\degree \in \naturalnumbersset$ be an integer that represents a degree.
    Let $\epsilonlimitedrankbias > 0$, and let $\funcdef{\rankbiasfunc}{[\epsilonlimitedrankbias, \infty]}{\naturalnumbersset}$ be a limited-relative-rank-bias function.
    \newline
    Let $\variety \subseteq \field$ be a set with the following properties
    \begin{enumerate}
        \item $\variety$ is $\degree$-lift-enabler with a lift operator $\lift{\square}$.
        \item $\variety$ has the $(\rankbiasfunc, \basefield, \degree, \epsilonlimitedrankbias)$-relative-rank-bias property.
    \end{enumerate}
    Then, for $\epsilon_1 \definedas \epsilon_1(\basefield, \degree, \rankbiasfunc, \epsilonlimitedrankbias)$ we have that for all $\blocklength \in \naturalnumbersset$:
    \[
        \normalizedcodedistanceex{\basefield}{\variety}{\degree} \geq \
        \normalizedcodedistance{\basefield}{\degree} - \epsilon_1
    \]
\end{theorem}
\begin{proof}
    We wish to do a reduction of our question regarding the radius of Reed-Muller in $\variety$ to the same question about Reed-Muller in $\field$.
    Let $\basefield$ be a finite field, and let $\degree \in \naturalnumbersset$ be an integer that represents a degree.
    Let $\epsilonlimitedrankbias > 0$, and let $\funcdef{\rankbiasfunc}{[\epsilonlimitedrankbias, \infty]}{\naturalnumbersset}$ be a limited-rank-relative-bias function.
    Let $\epsilon_1 \definedas \epsilon_1(\basefield, \degree, \rankbiasfunc, \epsilonlimitedrankbias)$ be a function we will specify later.
    Let $\variety \subseteq \field$ be a set with the properties defined above.
    \newline
    Moreover, let $\epsilon > \epsilon_1$ be some positive value.
    We will show that:
    \[
        \normalizedcodedistanceex{\basefield}{\variety}{\degree} > \
        \normalizedcodedistance{\basefield}{\degree} - \epsilon
    \]
    This will be enough as if the above holds for every $\epsilon > \epsilon_1$, we get that in fact
    $\normalizedcodedistanceex{\basefield}{\variety}{\degree} \geq \normalizedcodedistance{\basefield}{\degree} - \epsilon_1$.
    \newline
    For start, we note a simple observation: as Reed-Muller over $\variety$ is a linear code, we have
    \[
        \normalizedcodedistanceex{\basefield}{\variety}{\degree} =
        \min \set{\prex{x \in \variety}{\onvarpoly(x) \neq 0} \suchthat \onvarpoly \in \allpolyset{\leq \degree}{\variety}{\basefield}}
    \]
    Now, let $\onvarpoly \in \allpolyset{\leq \degree}{\variety}{\basefield}$ be a polynomial over $\variety$,
    and denote $\degree_{\onvarpoly} \definedas \deg(\onvarpoly)$.
    We wish to lower-bound the value of $\prex{x \in \variety}{\onvarpoly(x) \neq 0}$.
    To do so, we will equivalently upper-bound the value of $\prex{x \in \variety}{\onvarpoly(x) = 0}$.
    Precisely, to complete the proof all we need to show is:
    \[
        \prex{x \in \variety}{\onvarpoly(x) = 0} \leq 1 - \normalizedcodedistance{\basefield}{\degree} + \epsilon
    \]
    \newline
    Now we begin the proof itself.
    First, we lift the polynomial $\onvarpoly$ and get a polynomial $\funcdef{\lift{\onvarpoly}}{\field}{\basefield}$
    such that $\restrictfunc{\lift{\onvarpoly}}{\variety} \equiv \onvarpoly$ and $\deg(\lift{\onvarpoly}) = \degree_{\onvarpoly}$.
    Next, denote by $\factor_{\lift{\onvarpoly}}$ the factor defined by the set of single polynomial $\genpolyset = \set {\lift{\onvarpoly}}$.
    Trivially, the polynomial $\lift{\onvarpoly}$ is measurable in respect of $\genpolyset$.
    \newline
    We define the rank function:
    \[
        \rankfunc(m) \definedas \max \set{
            \rankbiasfunc \parens {\dfrac{\epsilon / 2}{\abs{\basefield}^m}},
            \rankfunc_{\ref{high-rank-implies-low-bias}} \parens{\basefield, \degree, \dfrac{\epsilon / 2}{\abs{\basefield}^m}}}
    \]
    Then, we $\rankfunc$-$\variety$-regularize $\genpolyset$ using Lemma~{\ref{theorem:regularization-in-X}}.
    This gives us a $\rankfunc$-$\variety$-regular factor $\factor^\prime$, which is defined by a set of polynomials $\genpolyset^{\prime} \definedas \set{\genpoly^\prime_1,...,\genpoly^\prime_{c^\prime}}$
    of degree $\leq \degree$ such that $\factor^\prime \relsemrefine{\variety}\factor_{\lift{\onvarpoly}}$ with $\relrank{\variety}{\genpolyset^\prime} \geq \rankfunc$
    and with bounded amount of polynomials defining it i.e,$c^\prime \leq C_{\rankfunc, \degree}^{\ref{theorem:regularization-in-X}}(1)$.
    Therefore, from definition we have that $\lift{\onvarpoly}$ is $\genpolyset^\prime$-measurable relative to $\variety$.
    Thus, there exists a measurement function $\funcdef{\Gamma}{\basefield^{c^\prime}}{\basefield}$
    and a remainder $\funcdef{\relativeremainder{\Gamma}}{\field}{\basefield}$ with $\restrictfunc{\relativeremainder{\Gamma}}{\variety} \equiv 0$
    and degree bounded by $\degree_{\onvarpoly}$, such that:
    \[
        \forall a \in \field:
        \lift{\onvarpoly}(a) =
        \Gamma(\genpoly^\prime_1(a),...,\genpoly^\prime_{c^\prime}(a))
        + \relativeremainder{\Gamma}(a)
    \]
    Next, we denote $\genpoly^\prime \definedas \lift{\onvarpoly} - \relativeremainder{\Gamma}$.
    By definition of remainder function, we have that $\restrictfunc{\genpoly^\prime}{\variety} \equiv \onvarpoly$.
    Additionally, note that $\genpoly^\prime$ is a polynomial over $\field$ of degree $\deg(\genpoly^\prime) = \degree_{\onvarpoly} \leq \degree$,
    and hence by the definition of $\normalizedcodedistance{\basefield}{\degree}$:
    \begin{equation}\label{eq:polynomials-in-fn-are-bounded-away-from-zero-with-high-probability}
    \prex{a \in \field}{\genpoly^\prime(a) = 0} \leq 1 - \normalizedcodedistance{\basefield}{\degree}
    \end{equation}
    For the next step, we claim that $\genpoly^\prime$ equals $0$ in $\field$ approximately with the same probability it equals $0$ in $\variety$.
    Note that this is the heart of the proof: it allows use properties known in $\field$ to new properties in $\variety$.
    This is formulated as follows:
    \begin{claim}\label{claim:p-and-P-have-the-same-approximation}
        We have:
        \[
            \abs{\prex{a \in \field}{\genpoly^\prime(a)= 0} -
            \prex{x \in \variety}{\genpoly^\prime(x) = 0}} \leq  \epsilon
        \]
    \end{claim}
    \begin{proof}
        Denote $S \definedas \basefield^{c^\prime}$, and for all $s \in S$, denote:
        \[
            p_1(s) \definedas \prex{a \in \field}{(\genpoly^{\prime}_1(a),...,\genpoly^{\prime}_{c^\prime}(a)) = s}
        \]
        As of our choice of $\rankfunc$, we have $\rank{\genpolyset^{\prime}} \geq \rankfunc_{\ref{high-rank-implies-low-bias}} \parens {\basefield, \degree, \dfrac{\epsilon/2}{\abs{\basefield}^{c^\prime}}}$.
        By combining Lemma~{\ref{high-rank-implies-low-bias}} with Lemma~{\ref{every-linear-combination-has-low-bias-implies-equidistribution}},
        we have that $p_1$ is ($\epsilon/2\abs{S}$)-equidistributed, i.e:
        \[
            p_1(s) = \dfrac{1 \pm \epsilon/2}{\abs{S}}
        \]
        Similarly, denote:
        \[
            p_2(s) \definedas \prex{x \in \variety}{(\genpoly^{\prime}_1(x),...,\genpoly^{\prime}_{c^\prime}(x)) = s}
        \]
        As of our choice of $\rankfunc$, we have $\relrank{\variety}{\genpolyset^\prime} \geq \rankbiasfunc(\epsilon / 2\abs{S})$.
        Now, we wish to use the relative rank-bias relation with Lemma~\ref{every-linear-combination-has-low-bias-implies-equidistribution}
        to conclude similarly that $p_2$ is ($\epsilon/2\abs{S}$)-equidistributed, i.e:
        \[
            p_2(s) = \dfrac{1 \pm \epsilon/2}{\abs{S}}
        \]
        However, in order to so, we must first ensure that $(\epsilon/2\abs{S}) \geq \epsilonlimitedrankbias$.
        This is done by choosing a correct $\epsilon_1$, and formulated in the following claim:
        \begin{claim}
            One can choose $\epsilon_1 \definedas \epsilon_1(\basefield, \degree, \rankbiasfunc, \epsilonlimitedrankbias)$ such that if $\epsilon \geq \epsilon_1$ we have that $\epsilon/2\abs{S} \geq \epsilon_1$.
        \end{claim}
        \begin{proof}
            We need that:
            \[
                \dfrac{\epsilon}{2 \abs{\basefield}^{c^\prime}} \geq \epsilonlimitedrankbias
            \]
            As $c^\prime \leq C^{\ref{theorem:regularization-in-X}}_{\rankfunc, \degree}(1)$,
            for the term above to hold it is enough that the following will be true:
            \[
                \epsilon \geq \epsilonlimitedrankbias \cdot 2 \abs{\basefield}^{C^{\ref{theorem:regularization-in-X}}_{\rankfunc, \degree}(1)}
            \]
            and as $\rankfunc$ and thus also $C^{\ref{theorem:regularization-in-X}}_{\rankfunc, \degree}(1)$ are independent of $\blocklength$,
            we can pick $\epsilon_1 = \epsilon_1(\basefield, \degree, \rankbiasfunc, \epsilonlimitedrankbias)$ and get what we aimed for.
        \end{proof}
        Now, under that assumption of $\epsilon_1$ written above, we have that $p_2$ is ($\epsilon/2\abs{S}$)-equidistributed.
        This allows us to use the similar distributions of $\genpolyset^\prime$ in $\field$ and in $\variety$ to conclude
        that $\genpoly^\prime$ behaves similar in $\field$ and in $\variety$:
        \begin{flalign*}
            \prex{a \in \field}{\genpoly^\prime(a)= 0}
            &=\sum_{s \in S} {p_1(s) \cdot \existfunc{\Gamma(s) = 0}} \\
            &=\sum_{s \in S} {p_2(s) \cdot \existfunc{\Gamma(s) = 0}} \pm \epsilon \\
            &=\prex{x \in \variety}{\genpoly^\prime(x)= 0} \pm \epsilon
        \end{flalign*}
        which concludes the proof of the claim.
    \end{proof}

    Finally, as $\restrictfunc{\genpoly^\prime}{\variety} \equiv \onvarpoly$, we have that $\prex{x \in \variety}{\genpoly^\prime(x) = 0} = \prex{x \in \variety}{\onvarpoly(x) = 0}$.
    Thus, the claim above combining with~\eqref{eq:polynomials-in-fn-are-bounded-away-from-zero-with-high-probability}
    shows that the probability we wished to bound is bounded as we aimed for:
    \[
        \prex{x \in \variety}{\onvarpoly(x) = 0} \leq 1 - \normalizedcodedistance{\basefield}{\degree} + \epsilon
    \]
    This concludes the proof of the theorem.
\end{proof}
\begin{remark}
    Under the same conditions,
    the distance of Reed-Muller codes in $\variety$ is also bounded \emph{from above} by the distance of Reed-Muller codes in $\field$,
    and we have:
    \[
        \normalizedcodedistanceex{\basefield}{\variety}{\degree} \leq \normalizedcodedistance{\basefield}{\degree} + \epsilon_1
    \]
    \begin{proof}
        Let $\funcdef{\genfunc}{\field}{\basefield}$ be the polynomial in $\field$ with the \emph{smallest} distance from $0$ as possible, that is $\normalizedcodedistance{\basefield}{\degree}$.
        Denote $\onvarpoly \definedas \restrictfunc{\genpoly}{\variety}$.
        Note that $\onvarpoly$ is a polynomial in $\variety$.
        Now repeat the proof using these two polynomials, and by Claim~\ref{claim:p-and-P-have-the-same-approximation}, we have that a random input of $\genpoly$ yields $0$
        (approximately) the same as a random input of $\onvarpoly$ yields $0$.
        Thus as we have $\prex{x \in \field}{\genpoly(x) = 0} = 1 - \normalizedcodedistance{\basefield}{\degree}$
        we also get:
        \[
            \prex{x \in \variety}{\onvarpoly(x) = 0} \geq 1 - \normalizedcodedistance{\basefield}{\degree} - \epsilon_1
        \]
        This bounds \emph{from above} the distance of Reed-Muller code in $\variety$ and we have:
        \[
            \normalizedcodedistanceex{\basefield}{\variety}{\degree} \leq \normalizedcodedistance{\basefield}{\degree} + \epsilon_1
        \]
    \end{proof}
\end{remark}
\begin{corollary}
    If we assume $\variety$ has the limited-relative rank-bias property to \emph{any extent} (or just the relative rank-bias property),
    then the theorem above proves an exact equality $\normalizedcodedistanceex{\basefield}{\variety}{\degree} = \normalizedcodedistance{\basefield}{\degree}$.
\end{corollary}

    \section[List Decoding Reed Muller Over \titlevariety]{List Decoding Reed Muller Over \titlevariety}\label{sec:list-decoding-reed-muller-over-X}
In this section, we prove our main theorem:
we prove the list decoding radius of Reed-Muller codes in is $\variety$ \emph{at least}
the list decoding radius of Reed-Muller codes in $\field$,
assuming $\variety$ is lift-enabler and has the relative rank-bias property.
We start by presenting formally the list decoding radius in $\variety$.
\begin{definition}[List Decoding in $\variety$]
    Let $\basefield$ be a finite field.
    Let $\degree, \blocklength \in \naturalnumbersset$, and let $\variety \subseteq \field$.
    \newline
    We define the reed muller list-decoding count in $\variety$ at distance $\tau$ as follows:
    \[
        \listpolycount{\basefield}{\variety}{\degree}{\tau} \definedas
        \max_{\funcdef{\genfunc}{\variety}{\basefield}}
            {\abs{\set{\genpoly \in \allpolyset{\leq \degree}{\variety}{\basefield} \suchthat {\dist{\genpoly, \genfunc} \leq \tau}}}}
    \]
    Additionally, we define $\listdecodingradiusex{\basefield}{\variety}{\degree}$ to be the \emph{list decoding radius}, which is
    the maximum $\tau$ for which $\listpolycount{\basefield}{\variety}{\degree}{\tau - \epsilon}$ is bounded by a \emph{constant} depending only on $\epsilon, \abs{\basefield}, \degree$.
\end{definition}

We recall that it was shown in~\cite[Theorem 1]{bhowmick2014list} that the list decoding radius of Reed Muller is $\normalizedcodedistance{\basefield}{\degree}$.
To be more precise, it was shown that for every $\epsilon > 0$, the list-decoding count is constant (independent of $\blocklength$) in distance $\tau = \normalizedcodedistance{\basefield}{\degree} - \epsilon$.
Formally, they have shown the following theorem:
\begin{theorem}[List Decoding RM in $\field$]\label{list-decoding-RM-in-Fn}
    There exists a function $c(\basefield, \degree, \epsilon)$ such that the following holds:
    Let $\basefield$ be a finite field, let $\epsilon > 0$, and let $\degree, \blocklength \in \naturalnumbersset$.
    Then, we have:
    \[
        \listpolycount{\basefield}{\field}{\degree}{\normalizedcodedistance{\basefield}{\degree} - \epsilon}
        \leq c(\basefield, \degree, \epsilon)
    \]
\end{theorem}
Additionally, we recall a lemma that was presented in~\cite[Corollary 3.3]{bhowmick2014list}, and was used in the analysis of the list decoding radius of Reed-Muller codes in $\field$:
\begin{lemma}[Low Complexity Approximation]~\cite[Corollary 3.3]{bhowmick2014list}\label{every-function-can-be-approximated-by-a-few-functions}
Let $\funcdef{\genfunc[2]}{A}{B}$, and let $\epsilon > 0$.
Let $\genfuncset \subseteq B^A$ be a collection of functions from $A$ to $B$.
Then there exists $c \leq 1/\epsilon^2$ functions $\genfunc_1,...,\genfunc_c \in \genfuncset$ such that
for every $\genfunc \in \genfuncset$, there is a function $\funcdef{\Gamma_{\genfunc}}{B^c}{B}$ such that:
\[
    \prex{x \in A}{\Gamma_{\genfunc}(\genfunc_1(x),...,\genfunc_c(x)) = \genfunc(x)}
    \geq \prex{x \in A}{\genfunc[2](x) = \genfunc(x)} - \epsilon
\]
\end{lemma}
The lemma shows that $\genfunc[2]$ can ``estimated'' by a only a few functions from $\genfuncset$.
Note that the estimation is close to $\genfunc[2]$ in compare to every $\genfunc \in \genfuncset$ and not necessarily close to $\genfunc[2]$ itself.

Finally, we present our main theorem, which shows that under assumptions on the subset $\variety \subseteq \field$, the list decoding radius of polynomials in $\variety$ will be similar to the list decoding radius in $\field$.
\newline
In more details (and informally), we show that if $\variety \subseteq \field$ is lift-enabler, has the limited-relative-rank-bias-property,
the list-decoding count is constant (independent of $\blocklength$) for every valid $\epsilon$ in distance $\tau = \normalizedcodedistance{\basefield}{\degree} - \epsilon$.
Note that not every $\epsilon > 0$ will be valid: the valid values of $\epsilon$ will depend on the limitations of the rank-bias property.
Formally, we show the following:
\begin{theorem}[List Decoding RM in $\variety$]\label{thm:list-decoding-RM-in-X}
    There exist functions $c_1(\basefield, \degree, \rankbiasfunc, \epsilonlimitedrankbias)$ and $c_2(\basefield, \degree, \rankbiasfunc, \epsilon)$ such that the following holds:
    Let $\basefield$ be a finite field, and let $\degree \in \naturalnumbersset$ be an integer that represents a degree.
    Let $\epsilonlimitedrankbias > 0$, and let $\funcdef{\rankbiasfunc}{[\epsilonlimitedrankbias, \infty]}{\naturalnumbersset}$ be a limited-relative-rank-bias function.
    \newline
    Let $\variety \subseteq \field$ be a set with the following properties
    \begin{enumerate}
        \item $\variety$ is $\degree$-lift-enabler with a lift operator $\lift{\square}$.
        \item $\variety$ has the $(\rankbiasfunc, \basefield, \degree, \epsilonlimitedrankbias)$-relative-rank-bias property.
    \end{enumerate}
    Then, for every $\epsilon \geq c_1(\basefield, \degree, \rankbiasfunc, \epsilonlimitedrankbias)$ it holds:
    \[
        \listpolycount{\basefield}{\variety}{\degree}{\normalizedcodedistanceex{\basefield}{\field}{\degree} - \epsilon} \leq
        c_2(\basefield, \degree, \rankbiasfunc, \epsilon)
    \]
\end{theorem}
\begin{proof}
    We follow the lines of the proof of~\cite[Theorem 1]{bhowmick2014list}.
    Let $\basefield$ be a finite field, and let $\degree \in \naturalnumbersset$ be an integer that represents a degree.
    Let $\epsilonlimitedrankbias > 0$, and let $\funcdef{\rankbiasfunc}{[\epsilonlimitedrankbias, \infty]}{\naturalnumbersset}$ be a limited-rank-relative-bias function.
    Let $c_1(\basefield, \degree, \rankbiasfunc, \epsilonlimitedrankbias)$ be a function we will specify later.
    Let $\variety \subseteq \field$ be a set with the properties defined above.
    \newline
    Finally, let $\epsilon \geq c_1(\basefield, \degree, \rankbiasfunc, \epsilonlimitedrankbias)$ for $c_1$ that we will specify later, and let $\funcdef{\onvarfunc}{\variety}{\basefield}$ be a received word.
    We wish to bound the amount of polynomials in $\allpolyset{\leq \degree}{\variety}{\basefield}$ that are $(\normalizedcodedistance{\basefield}{\degree} - \epsilon)$-close to $\onvarfunc$.
    \newline
    Apply Lemma~\ref{every-function-can-be-approximated-by-a-few-functions} with $A = \variety$, $B = \basefield$, $\genfunc[2] = \onvarfunc$, $\genfuncset = {\allpolyset{\leq \degree}{\variety}{\basefield}}$
    and approximation parameter $\epsilon / 2$ to obtain $\onvarpolyset[3] \subset \allpolyset{\leq \degree}{\variety}{\basefield}$, defined by $\onvarpolyset[3] = (\onvarpoly[3]_1,...,\onvarpoly[3]_c)$ where $c \leq 4/\epsilon^2$,
    such that for every $\onvarpoly \in \allpolyset{\leq \degree}{\variety}{\basefield}$ there is a function $\funcdef{\Gamma_{\onvarpoly}}{\basefield^c}{\basefield}$ that approximates $\onvarfunc$ in $\variety$ relative to $\allpolyset{\leq \degree}{\variety}{\basefield}$ i.e.:
    \[
        \forall \onvarpoly \in \allpolyset{\leq \degree}{\variety}{\basefield}: \prex{x \in \variety}{\Gamma_{\onvarpoly}(\onvarpoly[3]_1(x),...,\onvarpoly[3]_c(x)) = \onvarpoly(x)} \geq \prex{x \in \variety}{\onvarfunc(x) = \onvarpoly(x)} - \epsilon / 2
    \]
    \newline
    Let $\funcdef{\rankfunc_1, \rankfunc_2}{\naturalnumbersset}{\naturalnumbersset}$ be two non-decreasing functions that represents rank that we will specify later.
    For $\rankfunc_1$, we will require that for all $m \geq 1$:
    \[
        \rankfunc_1(m) \geq \max { \set{
        \topbot{
            {\rankfunc_2(C_{\rankfunc_2, \degree}^{\ref{theorem:regularization-in-X}}(m + 1)) + C_{\rankfunc_2, \degree}^{\ref{theorem:regularization-in-X}}(m + 1) + 1,}}
            {\rankfunc_2(C_{\rankfunc_{\ref{preserving-degree-starting-field}}, \degree}^{\ref{theorem:regularization-in-X}}(m + 1))
            + C_{\rankfunc_{\ref{preserving-degree-starting-field}}, \degree}^{\ref{theorem:regularization-in-X}}(m + 1) + 1}
        }}
    \]
    Note that in the expression above, we denote $\funcdef{\rankfunc_{\ref{preserving-degree-starting-field}}}{\naturalnumbersset}{\naturalnumbersset}$,
    as follows: $\rankfunc_{\ref{preserving-degree-starting-field}}(c) \definedas \rankfunc_{\ref{preserving-degree-starting-field}}(\basefield, \degree, c)$.
    \newline
    The reason we chose this $\rankfunc_1$, is that by our choice of $\rankfunc_1$ we can use the second part of Lemma~\ref{theorem:regularization-in-X}.
    Specifically, if we start with $\rankfunc_1$-$\variety$-regular factor and we $\rankfunc_2$-$\variety$-regularize it,
    we get that the $\rankfunc_2$-$\variety$-regular factor that we received is a syntactic refinement of the $\rankfunc_1$-$\variety$-regular factor we started with.
    \newline
    As a first step, we lift the polynomial factor to get $\genpolyset[3] \definedas \lift{\onvarpolyset[3]}$.
    Note that because $\forall x \in \variety: \lift{\onvarpoly[3]_i}(x) = \onvarpoly[3]_i(x)$, for all $\onvarpoly \in F$ we have:
    \[
        \prex{x \in \variety}{\Gamma_{\onvarpoly}(\lift{\onvarpoly[3]_1}(x),...,\lift{\onvarpoly[3]_c}(x)) = \onvarpoly(x)} \geq \prex{x \in \variety}{\onvarfunc(x) = \onvarpoly(x)} - \epsilon / 2
    \]
    Next, we $\rankfunc_1$-$\variety$-regularize the factor $\genpolyset[3]$ by Theorem~\ref{theorem:regularization-in-X}.
    This gives us a $\rankfunc_1$-$\variety$-regular factor $\factor^\prime$, which is defined by a set of polynomials $\genpolyset[3]^\prime \definedas (\genpoly[3]^\prime_1,...,\genpoly[3]^\prime_{c^\prime})$ of degree $\leq \degree$
    such that $\factor^\prime \relsemrefine{\variety} \factor$,
    with $\relrank{\variety}{\genpolyset[3]^\prime} \geq \rankfunc(c^\prime)$
    and with bounded amount of polynomials defining it i.e. $c^\prime \leq C_{\rankfunc_1, \degree}^{\ref{theorem:regularization-in-X}}(c)$.
    We apply Corollary~\ref{relative-semantic-refinement-is-restricted-semantic-refinement}
    and get that $\factor^{\prime} \semrefineex{\variety} \factor$.
    We then use the fact that $\Gamma_{\onvarpoly}(\lift{\onvarpoly[3]_1}(x),...,\lift{\onvarpoly[3]_c}(x))$ is measurable in respect of $\genpolyset$ \emph{in $\variety$},
    and deduce we have a similar approximation of $\onvarpoly$ using $\genpolyset^\prime$ as the approximation of $\onvarpoly$ using $\genpolyset$.
    Formally, there exists a function $\funcdef{\Gamma_{\onvarpoly}^\prime}{\basefield^{c^\prime}}{\basefield}$ such that:
    \[
        \prex{x \in \variety}
        {\Gamma^{\prime}_{\onvarpoly}(\genpoly[3]^{\prime}_1(x)),...,\genpoly[3]^{\prime}_{c^\prime}(x)) = \onvarpoly(x)} \geq
            \prex{x \in \variety}{\onvarfunc(x) = \onvarpoly(x)} - \epsilon / 2
    \]
    Now we recall that we wished to bound the amount of polynomials $\onvarpoly \in \allpolyset{\leq \degree}{\variety}{\basefield}$ such that
    $\prex{x \in \variety}{\onvarfunc(x) \neq \onvarpoly(x)} < \normalizedcodedistance{\basefield}{\degree} - \epsilon$.
    Let $\onvarpoly \in \allpolyset{\leq \degree}{\variety}{\basefield}$ be a polynomial as we just described.
    We will show that such $\onvarpoly$ is measurable with respect to $\genpolyset[3]^\prime$ in $\variety$.
    This will upper bound the amount of possible polynomials $\onvarpoly$ by the amount of possible different $\funcdef{\Gamma^{\prime}_{\onvarpoly}}{\basefield^{c^\prime}}{\basefield}$,
    which is $\abs{\basefield}^{\norm{\factor^\prime}} = \basefieldsize^{(\basefieldsize^{c^\prime})}$, and thus $c_2(\basefield, \degree, \rankbiasfunc, \epsilon) \leq \basefieldsize^{(\basefieldsize^{c^\prime})}$.
    \newline
    By our choice of $c^\prime$ we have that $c^\prime \leq C_{\rankfunc_1, \degree}^{\ref{theorem:regularization-in-X}}(4/\epsilon^2)$, and thus $c_2$ is bounded by a function of $(\basefield, \degree, \rankfunc_1, \epsilon)$.
    Note that we have not yet specified the value of $\rankfunc_1$, because it is determined by the choice of $\rankfunc_2$ that we will later define its exact values.
    The important thing about our future choice of $\rankfunc_2$ is that the value of $\rankfunc_2$ must be independent of $\blocklength$,
    but can depend on $(\basefield, \degree, \rankbiasfunc, \epsilon)$.
    This will conclude the proof.
    \newline
    Now, consider a lift of $\onvarpoly$, i.e $\genpoly \definedas \lift{\onvarpoly}$.
    Note that by the definition of lift $\forall x \in \variety: \genpoly(x) = \onvarpoly(x)$.
    We will show that $\genpoly$ is measurable in respect of $\genpolyset[3]^\prime$ in $\variety$.
    \newline
    We consider the factor $\factor_{\genpoly}$ that is generated by $\genpolyset[3]_{\genpoly} \definedas \genpolyset[3]^\prime \cup \set{\genpoly}$.
    By using Theorem~\ref{theorem:regularization-in-X}, we can $\rankfunc_2$-$\variety$-regularize it and get the polynomial factor $\factor^{\prime\prime}$ that relative-refines $\factor_{\genpoly}$.
    We denote the set of polynomials in the factor as $\genpolyset[3]^{\prime\prime}$.
    \newline
    Next, notice that the factor $\factor^{\prime\prime}$ is a $\rankfunc_2$-regular factor, therefore by our choice of $\rankfunc_1$ and the second part of Theorem~\ref{theorem:regularization-in-X},
    we in fact have $\factor^{\prime\prime} \synrefine \factor^{\prime}$.
    This is true because by our choice of $\rankfunc_1$:
    \[
        \relrank{\variety}{\genpolyset[3]^\prime} \geq
        \rankfunc_1(c^\prime) \geq
        \rankfunc_2(C_{\rankfunc_2, \degree}^{\ref{theorem:regularization-in-X}}(c^\prime + 1)) + C_{\rankfunc_2, \degree}^{\ref{theorem:regularization-in-X}}(c^\prime + 1) + 1 \geq
        \rankfunc_2(\abs{\factor^{\prime\prime}}) + \abs{\factor^{\prime\prime}}+1
    \]
    And as rank is always bigger than relative rank, we also have:
    \[
        \rank{\genpolyset[3]^\prime} \geq
        \rankfunc_1(c^\prime) \geq
        \rankfunc_2(C_{\rankfunc_{\ref{preserving-degree-starting-field}}, \degree}^{\ref{theorem:regularization-in-X}}(c^\prime + 1))
        + C_{\rankfunc_{\ref{preserving-degree-starting-field}}, \degree}^{\ref{theorem:regularization-in-X}}(c^\prime + 1) + 1
    \]
    Thus, the polynomials defining $\factor^{\prime\prime}$ are in the form $\genpolyset[3]^{\prime\prime} \definedas \genpolyset[3]^{\prime} \cup \set{\genpoly[3]^{\prime\prime}_1,...,\genpoly[3]^{\prime\prime}_{c^{\prime\prime}}}$.
    Note that as promised in Theorem~{\ref{theorem:regularization-in-X}}, we have $\abs{\genpolyset[3]^{\prime\prime}} = c^\prime+c^{\prime\prime} \leq C^{\ref{theorem:regularization-in-X}_{\rankfunc_2, \degree}}(c^\prime)$.
    \newline
    Additionally, by the way we built $\genpolyset[3]_{\genpoly}$, the function $\genpoly$ is measurable in respect of it.
    Therefore, as $\factor^{\prime\prime} \relsemrefine{\variety} \factor_{\genpoly}$, we have that $\genpoly$ is $\genpolyset[3]^{\prime\prime}$-measurable relative to $\variety$.
    In other words, there exists $\funcdef{\Phi}{\basefield^{c^\prime + c^{\prime\prime}}}{\basefield}$
    and $\funcdef{\relativeremainder{\genpoly}}{\field}{\basefield}$ with $\deg(\relativeremainder{\genpoly}), \deg(\genpoly - \relativeremainder{\genpoly}) \leq \deg(\genpoly) \leq \degree$ and $\restrictfunc{\relativeremainder{\genpoly}}{\variety} \equiv 0$ such that:
    \[
        \forall a \in \field: \genpoly(a) = \Phi(\genpoly[3]^\prime_1(a),...,\genpoly[3]^\prime_{c^\prime}(a), \genpoly[3]^{\prime\prime}_1(a),...,\genpoly[3]^{\prime\prime}_{c^{\prime\prime}}(a))) + \relativeremainder{\genpoly}(a)
    \]
    And specifically in $\variety$ we have:
    \[
        \forall x \in \variety: \genpoly(x) = \Phi(\genpoly[3]^\prime_1(x),...,\genpoly[3]^\prime_{c^\prime}(x), \genpoly[3]^{\prime\prime}_1(x),...,\genpoly[3]^{\prime\prime}_{c^{\prime\prime}}(x)))
    \]
    Denote $\genpoly^{\prime} \definedas \genpoly - \relativeremainder{\genpoly}$.
    We will show the polynomial $\genpoly^{\prime}$ does not depend on its last $c^{\prime\prime}$ variables, and thus $\Phi$ does not depend on its last $c^{\prime\prime}$ variables.
    This will imply that $\genpoly$ is measurable in respect of $\genpolyset[3]^\prime$ in $\variety$, which will conclude the proof.
    \newline
    Now, we choose $\rankfunc_2$ to be such that:
    \[
        \rankfunc_2(m) \geq \max \set {
            \rankbiasfunc \parens{\dfrac{\epsilon / 4}{\abs{\basefield}^m}},
            \rankval_{\ref{high-rank-implies-low-bias}}\parens{\dfrac{\epsilon / 4}{\abs{\basefield}^m}},
            \rankval_{\ref{preserving-degree-starting-field}}(m)
        }
    \]
    Note that in the expression above we are discussing fixed field and degree, i.e. $\basefield, \degree$.
    Therefore we denote $\funcdef{\rankfunc_{\ref{preserving-degree-starting-field}}}{\naturalnumbersset}{\naturalnumbersset}$
    as $\rankfunc_{\ref{preserving-degree-starting-field}}(c) \definedas \rankfunc_{\ref{preserving-degree-starting-field}}(\basefield, \degree, c)$
    and $\funcdef{\rankfunc_{\ref{high-rank-implies-low-bias}}}{\naturalnumbersset}{\naturalnumbersset}$
    as $\rankfunc_{\ref{high-rank-implies-low-bias}}(\epsilon) \definedas \rankfunc_{\ref{high-rank-implies-low-bias}}(\basefield, \degree, \epsilon)$.
    \newline
    Next, we show that even if we change the polynomials in the factor to have a disjoint set of inputs in $\field$,
    we still obtain a polynomial in the same degree, which have an approximation close to the approximation we had in $\variety$.
    Note that after this step, the proof becomes very similar to the proof of list decoding Reed Muller in $\field$~\cite[Theorem 1]{bhowmick2014list}:
    we omit the dependence of $\variety$ and get the same approximation by functions of multiple variables,
    as we had in $\field$.
    This is done by the following lemma:
    \begin{lemma}
        Let $\set{a^{i}, b^{j}}, i \in [c^\prime], j \in [c^{\prime\prime}]$  be pairwise disjoint sets of $\blocklength$ variables each.
        Let $\blocklength^{\prime} \definedas \blocklength(c^\prime + c^{\prime\prime})$.
        Let $\funcdef{\vec{\genpoly}}{\basefield^{\blocklength^\prime}}{\basefield}$ and $\funcdef{\vec{\onvarfunc}}{\basefield^{\blocklength^{\prime}}}{\basefield}$
        be functions of $\blocklength^\prime$ variables defined as follows:
        \[
            \vec{\genpoly^{\prime}}(\vec{a}) \definedas
                \Phi \parens{\genpoly[3]^\prime_1(a^1),...,\genpoly[3]^\prime_{c^\prime}(a^{c^\prime}), \genpoly[3]^{\prime\prime}_1(b^{1}),...,\genpoly[3]^{\prime\prime}_{c^{\prime\prime}}(b^{c^{\prime\prime}})}
        \]
        and:
        \[
            \vec{\onvarfunc}(\vec{a}) \definedas \Gamma^{\prime}_{\onvarpoly}(\genpoly[3]^{\prime}_1(a^{1})),...,\genpoly[3]^{\prime}_{c^\prime}(a^{c^\prime}))
        \]
        Note that $\vec{\onvarfunc}$ is a function that receives $\blocklength^\prime$ variables, and ignores its last $c^{\prime\prime}$ variables.
        \newline
        Then:
        \begin{enumerate}
            \item The degree of $\vec{\genpoly^\prime}$ remains bounded, i.e. $\deg(\vec{\genpoly^\prime}) \leq \degree$.
            \item The approximation of $\vec{\onvarfunc}$ to $\vec{\genpoly^\prime}$ in $\basefield^{n^\prime}$ is close to the approximation of $\Gamma^{\prime}_{\onvarpoly}$ to $\onvarpoly$ in $\variety$.
            Specifically, we show:
            \[
                \abs{
                    \prex{\vec{a} \in \basefield^{\blocklength^\prime}}{\vec{\onvarfunc}(\vec{a}) = \vec{\genpoly^\prime}(\vec{a})} -
                    \prex{x \in \variety}{\Gamma^{\prime}_{\onvarpoly}(\genpoly[3]^{\prime}_1(x)),...,\genpoly[3]^{\prime}_{c^\prime}(x)) =\onvarpoly(x)}
                }
                \leq \epsilon/4
            \]
        \end{enumerate}
        \begin{proof}
            We start by proving the first part of the lemma: bounding the degree of $\vec{\genpoly^\prime}$ by $\degree$.
            First, we recall that $\genpoly^{\prime} = \genpoly - \relativeremainder{\genpoly}$ where $\relativeremainder{\genpoly}$ is a valid remainder.
            Specifically, we have $\deg(\genpoly^{\prime}) = \deg(\genpoly - \relativeremainder{\genpoly}) \leq \deg(\genpoly) \leq \degree$.
            In addition, by the way we built $\Phi$ we have:
            \[
                \forall a \in \field: \genpoly^\prime(a) = \Phi(\genpoly[3]^\prime_1(a),...,\genpoly[3]^\prime_{c^\prime}(a), \genpoly[3]^{\prime\prime}_1(a),...,\genpoly[3]^{\prime\prime}_{c^{\prime\prime}}(a)))
            \]
            Thus the function above is of degree $\leq \degree$.
            Moreover, we have:
            \[
                \rank{\genpolyset[3]^{\prime\prime}} \geq
                \relrank{\variety}{\genpolyset[3]^{\prime\prime}} \geq
                \rankfunc_2(\abs{\genpolyset[3]^{\prime\prime}})\geq
                \rankfunc_{\ref{preserving-degree-starting-field}}(\abs{\genpolyset[3]^{\prime\prime}})
            \]
            Therefore we can use Lemma~\ref{preserving-degree-starting-field} to get that $\deg(\vec{\genpoly^{\prime}}) \leq \deg(\genpoly^\prime) \leq \degree$.
            Note that in order to use the lemma formally,
            we had to extend the input space of $\genpoly^{\prime}$ to be of $\blocklength^{\prime}$ variables (and make it depend only on the first $\blocklength$ variables as it used to).
            Because lemma~\ref{preserving-degree-starting-field} require bounds independent of $\blocklength$, this is done smoothly.
            \newline
            Now we move to the second part of the lemma: bounding the approximation of $\vec{\onvarfunc}$ to $\vec{\genpoly^\prime}$.
            Denote $S \definedas \basefield^{c^\prime + c^{\prime\prime}}$, and for each $s \in S$ denote:
            \[
                p_1(s) \definedas \prex{x \in \variety}{
                \parens{\genpoly[3]^\prime_1(x),...,\genpoly[3]^\prime_{c^\prime}(x), \genpoly[3]^{\prime\prime}_1(x),...,\genpoly[3]^{\prime\prime}_{c^{\prime\prime}}(x)} = s}
            \]
            and as of our choice of $\rankfunc_2$, we have $\rank{\genpolyset[3]^{\prime\prime}} \geq \rankbiasfunc(\epsilon/8\abs{S})$.
            Therefore, if we require that the relative rank-bias relation holds for $\epsilon / 8\abs{S}$, we can use Lemma~\ref{every-linear-combination-has-low-bias-implies-equidistribution} with $A = \variety$
            to get that $p_1$ is $(\epsilon /8\abs{S})$-almost uniform, i.e:
            \[
                p_1(s) = \dfrac{1 \pm \epsilon / 8}{\abs{S}}
            \]
            We show that this can be done in the following claim by choosing a proper $c_1$:
            \begin{claim}
                One can choose $c_1 \definedas c_1(\basefield, \degree, \rankbiasfunc, \epsilonlimitedrankbias)$ such that if $\epsilon \geq c_1$
                we have that $\epsilon / 8\abs{S} \geq c_1$.
            \end{claim}
            \begin{proof}
                This is done by using the bound we already know.
                We need that:
                \[
                    \epsilonlimitedrankbias \leq \dfrac{\epsilon}{8 \abs{\basefield}^{c^\prime + c^{\prime\prime}}}
                \]
                As $c^\prime + c^{\prime\prime} \leq C^{\ref{theorem:regularization-in-X}}_{\rankfunc_2, \degree}(c^\prime)$,
                for the term above to hold it is enough that the following will be true:
                \[
                    \epsilon \geq \epsilonlimitedrankbias \cdot 8 \abs{\basefield}^{C^{\ref{theorem:regularization-in-X}}_{\rankfunc_2, \degree}(c^\prime)}
                \]
                and as $\rankfunc_2, c^\prime$ and thus also $C^{\ref{theorem:regularization-in-X}}_{\rankfunc_2, \degree}(c^\prime)$ are independent of $\blocklength$,
                we can pick $c_1 = c_1(\basefield, \degree, \rankbiasfunc, \epsilonlimitedrankbias)$ and get what we aimed for.
            \end{proof}
            Thus, we can assume that $p_1$ is $(\epsilon/8\abs{S})$-almost uniform.
            Now, let:
            \[
                p_2(s) \definedas \prex{\vec{a} \in \basefield^{\blocklength^\prime}}
                {\parens{\genpoly[3]^\prime_1(a^1),...,\genpoly[3]^\prime_{c^\prime}(a^{c^\prime}), \genpoly[3]^{\prime\prime}_1(b^{1}),...,\genpoly[3]^{\prime\prime}_{c^{\prime\prime}}(b^{c^{\prime\prime}})} = s}
            \]
            Note that the rank of $\vec{\genpolyset[3]}^{\prime\prime} = {\set{\genpoly[3]^\prime_1(a^1),...,\genpoly[3]^\prime_{c^\prime}(a^{c^\prime}), \genpoly[3]^{\prime\prime}_1(b^{1}),...,\genpoly[3]^{\prime\prime}_{c^{\prime\prime}}(b^{c^{\prime\prime}})}}$,
            as a factor defined over $\basefield^{\blocklength^\prime}$, can not be lower than the rank of $\genpolyset[3]^{\prime\prime}$
            and thus we have $\rank{\vec{\genpolyset[3]}^{\prime\prime}} \geq \rankval_{\ref{high-rank-implies-low-bias}}\parens{\dfrac{\epsilon / 8}{\abs{\basefield}^m}}$.
            By using lemma~{\ref{high-rank-implies-low-bias}}, which shows the rank-bias relation for $\basefield^{\blocklength^\prime}$,
            we can similarly use Lemma~\ref{every-linear-combination-has-low-bias-implies-equidistribution} with $A = \basefield^{\blocklength^\prime}$
            to get that $p_2$ is also $(\epsilon/8\abs{S})$-almost-uniform, i.e:
            \[
                p_2(s) = \dfrac{1 \pm \epsilon / 8}{\abs{S}}
            \]
            Now, we show the approximations are the same.
            Denote by $s^\prime$ the restriction of $s$ to its first $c^\prime$ coordinates, and consider the approximation:
            \begin{flalign*}
                \prex{\vec{a} \in \basefield^{\blocklength^\prime}}{\vec{\onvarfunc}(\vec{a}) = \vec{\genpoly}^\prime (\vec{a})} = \\
                &=\sum_{s \in S} {p_2(s) \cdot \existfunc{\Phi(s) = \Gamma_{\genpoly}^\prime (s^\prime)}} \\
                &=\sum_{s \in S} {p_1(s) \cdot \existfunc{\Phi(s) = \Gamma_{\genpoly}^\prime (s^\prime)}} \pm \epsilon / 4 \\
                &=\prex{x \in \variety}{\Gamma^{\prime}_{\onvarpoly}(\genpoly[3]^{\prime}_1(x)),...,\genpoly[3]^{\prime}_{c^\prime}(x)) =\onvarpoly(x)} \pm \epsilon/4
            \end{flalign*}
            This completes the proof the lemma.
        \end{proof}

        The proof is followed by the same methods used in~\cite{bhowmick2014list}.
        We repeat if for completeness.
        We next restate a lemma proved in~\cite[Claim 4.2]{bhowmick2014list}, which is a varaiant of the Schwartz-Zippel lemma~\cite{10.1145/322217.322225,Zippel1979ProbabilisticAF}:
        \begin{lemma}\label{lemma-schwarz-zippel-for-comparing-polynomial-to-function-with-less-variables}
            Let $\degree$, $\blocklength_1$, $\blocklength_2 \in \naturalnumbersset$ be integers.
            Let $\genpoly_1 \in \allpolyset{\leq \degree}{\basefield^{\blocklength_1 + \blocklength_2}}{\basefield}$,
            and let $\funcdef{\genfunc_1}{\basefield^{\blocklength_1}}{\basefield}$ be a function.
            Assume the polynomial is $\normalizedcodedistance{\basefield}{\degree}$-close to the function, i.e:
            \[
                \prex{x_1,...,x_{\blocklength_1+\blocklength_2} \in \basefield}
                        {\genpoly_1(x_1,...,x_{\blocklength_1+\blocklength_2}) = \genfunc_1(x_1,...,x_n)} > 1 - \normalizedcodedistance{\basefield}{\degree}
            \]
            Then, $\genpoly_1$ does not depend on $x_{\blocklength_1 + 1},...,x_{\blocklength_1 + \blocklength_2}$.
        \end{lemma}
        Now, apply Lemma~\ref{lemma-schwarz-zippel-for-comparing-polynomial-to-function-with-less-variables} to
        $\genpoly_1 = \vec{\genpoly^{\prime}}$, $\genfunc_1 = \vec{\onvarfunc}$, $\blocklength_1 = \blocklength c^\prime$, $\blocklength_2 = \blocklength c^{\prime\prime}$.
        We obtain that $\vec{\genpoly^{\prime}}$ does not depend on its last $c^{\prime\prime}$ variables, and thus by denoting $C_{i} \definedas \genpoly[3]^{\prime\prime}_i(0)$ for $i \in \sparens{c^{\prime\prime}}$ we have:
        \[
            \vec{\genpoly^{\prime}}(\vec{a}) = \Phi \parens{\genpoly[3]^\prime_1(a^1),...,\genpoly[3]^\prime_{c^\prime}(a^{c^\prime}), C_1,...,C_{c^{\prime\prime}}}
        \]
        Now, for every $a \in \field$, if we substitute $a$ in the $i$-th component of $\vec{a}$ for every $i \in \sparens{c^{\prime}}$ in the equation above, we get the following is true:
        \[
            \genpoly^{\prime}(a) = \Phi \parens{\genpoly[3]^\prime_1(a),...,\genpoly[3]^\prime_{c^\prime}(a), C_1,...,C_{c^{\prime\prime}}}
        \]
        Hence $\genpoly^{\prime}$ does not depend on its last $c^{\prime\prime}$ variables.
        As explained earlier, this implies that $\genpoly$ is measurable in respect of $\genpolyset[3]^{\prime}$ in $\variety$.
        This completes the proof of the theorem.
    \end{lemma}

\end{proof}


    \begin{appendices}
        \section[Equidistribution of Functions]{Equidistribution of Functions}\label{sec:equidistribution-of-functions}
Assume we have a collection of functions $(\genfunc_1,...,\genfunc_c$), where $\funcdef{\genfunc_i}{A}{\basefield}$ for some finite set $A$.
We are interested in showing that the functions are equidistributed, which means that their values behave close to independent random variables.
We begin by formulating this definition:
\begin{definition}[Equidistribution of Functions]
    Given $\epsilon > 0$ and $A \subseteq \field$,
    we say a collection of functions $\genfuncset = (\genfunc_1,...,\genfunc_c)$ where $\funcdef{\genfunc_i}{\genset}{\basefield}$ is $\epsilon$-equidistributed in $A$ if for all $\vec{\alpha} = (\alpha_1,...,\alpha_c) \in \basefield^c$ we have:
    \[
        \prex{x \in A}{(\genfunc_1(x),...,\genfunc_c(x)) = \vec{\alpha}} = \frac{1}{\abs{\basefield}^c} \pm \epsilon
    \]
\end{definition}

The following is a standard lemma that shows that if every linear combination of a collection of functions has low bias, the collection is equidistributed.
We repeat the steps of the proof of \cite[Lemma 7.24]{book}, but here, we think of $A$ as any finite set (and not particularly $\field$):
\begin{lemma}\label{every-linear-combination-has-low-bias-implies-equidistribution}
Let $\epsilon > 0$, and let $A$ be a finite set.
Let $\genfuncset = (\genfunc_1,...,\genfunc_c)$ be a collection of functions defined over $A$, i.e. $\funcdef{\genfunc_i}{A}{\basefield}$.
Assume each linear combination of the collection has low bias, i.e for each $\lambda = (\lambda_1,...,\lambda_c) \in \basefield^c$ such that $\lambda \neq \vec{0}$ we have:
\[
    \relbias{x \in A}{\sum_{i=1}^{c}{\lambda_i \genfunc_i}} < \epsilon
\]
Then, the collection $\genfuncset$ is $\epsilon$-equidistributed over $A$.
\newline
In particular, for $\epsilon < \frac{1}{\abs{\basefield}^c}$, the lemma shows that each atom of $\genfuncset$ is not empty i.e for all $\vec{\alpha}$ there is some $x \in A$ such that $(\genfunc_1(x),...,\genfunc_c(x)) = \vec{\alpha}$.
\end{lemma}
\begin{proof}
    We wish to show that for each $\vec{\alpha} \in \basefield^c$ we have:
    \[
        \prex{x \in A}{(\genfunc_1(x),...,\genfunc_c(x)) = \vec{\alpha}} = \frac{1}{\abs{\basefield}} \pm \epsilon
    \]
    We express the fraction of inputs that are in the atom $\vec{\alpha}$ the following way:
    \[
        \prex{x \in A}{(\genfunc_1(x),...,\genfunc_c(x))} =
        \expectation{x \in A}{\prod_{i=1}^c {1_{[\genfunc_i(x) = \alpha_i]}}}
    \]
    We use the fact that for every $0 \neq x \in \basefield$, we have $\sum_{\lambda = 0}^{\basefieldsize - 1} \charfunc{\lambda x} = 0$,
    and if $x = 0$ we have $\sum_{\lambda = 0}^{\basefieldsize - 1} \charfunc{\lambda x} = \basefieldsize$.
    Therefore, the expression above equals:
    \[
        =\expectation{x \in A}{\prod_{i=1}^c \parens {{\frac{1}{\basefieldsize} \cdot \sum_{\lambda_i = 0}^{\basefieldsize - 1} {\charfunc{\lambda_i (\genfunc_i(x) - \alpha_i)}}}}} =\\
        \frac{1}{\basefieldsize^c} \cdot {\expectation{x \in A}{\prod_{i=1}^c \sum_{\lambda_i = 0}^{\basefieldsize - 1} {\charfunc{\lambda_i (\genfunc_i(x) - \alpha_i)}}}}
    \]
    By the definition of character functions, we have that $\charfunc{a+b} = \charfunc{a} \cdot \charfunc{b}$, and therefore the expression above equals:
    \[
        \frac{1}{\basefieldsize^c} \cdot \sum_{(\lambda_1,...,\lambda_c) \in \prod_{i=1}^c [0, \basefieldsize - 1]} \parens{\expectation{x \in A}{\charfunc{\sum_{i = 0}^{c} {\lambda_i (\genfunc_i(x) - \alpha_i)}}}}
    \]
    Now, we use the fact that:
    \[
        \relbias{x \in A}{\sum_{i=1}^c (\lambda_i (\genfunc_i(x) -\alpha_i)} = \relbias{x \in A}{\sum_{i=1}^c (\lambda_i \genfunc_i(x))} < \epsilon
    \]
    and get that:
    \[
        \prex{x \in A}{(\genfunc_1(x),...,\genfunc_c(x)) = \vec{\alpha}} = \frac{1}{\basefieldsize^c} \cdot \parens{1 \pm \epsilon \prod_{i=1}^c{\basefieldsize}} = \frac{1}{\abs{\basefield}^c} \pm \epsilon
    \]
\end{proof}

%Now, if we have $A = \basefield^{\blocklength}$ for some $\blocklength \in \naturalnumbersset$, we have the following corollary:
%\begin{corollary}[High rank implies equidistribution in $\field$]
%    TODO %TODO
%\end{corollary}
%
%Moreover, if we have that $A = \variety$ when $\variety \subseteq \field$ has the relative rank-bias property, we have:
%\begin{corollary}[High relative rank implies equidistribution in $\variety$]
%    TODO %TODO
%\end{corollary}
%
        
\section[Comparing Ranks]{Comparing Ranks}\label{sec:comparing-ranks}
In this section, we compare the definition of rank we used in this paper to another definition of rank used implicitly throughout this paper.
This comparison is crucial, as there is no universally accepted definition of rank;
different theorems presented throughout this paper employ distinct definitions.
We demonstrate that our definition is sufficiently comprehensive, in that a polynomial (or a factor) classified as having high rank according to our criteria
also exhibits high rank according to the second implicitly-used definition.
While in many cases the comparison may appear straightforward, we include it for the sake of completeness.
\newline
Specifically, we compare our definition of rank with the definition established in ~\cite{lampert2021relative}.
The paper~\cite{lampert2021relative} extended the original definition of rank that was presented in \cite{10.1007/BF02392473},
to include also the concept of relative rank.
It is important to note that this definition is specifically defined to subsets $\variety \subseteq \field$ that can be expressed as sets in the form $\variety = \zerofunc{\varpolyset}$ for some set of polynomials $\varpolyset$,
and not to a general set $\variety \subseteq \field$.
\newline
First, we present a useful notation that is used in the definition presented in ~\cite{lampert2021relative}:
\begin{notation*}[Largest Degree Homogenous Part]
    For a polynomial $\genpoly$ of degree $\degree$, we denote by $\homopart{\genpoly}$ its degree-$\degree$ homogenous component.
    In other words, $\homopart{\genpoly}$ is the sum of all the monomials of $\genpoly$ of degree exactly $\degree$.
    For a set of polynomials $\genpolyset = \set{\genpoly_1,...,\genpoly_c}$, we define $\homopart{\genpolyset} \definedas \set{\homopart{\genpoly_i} \suchthat i = 1,...,c}$.
\end{notation*}
Next, we present the exact definition of rank for a polynomial:
\begin{definition}[Schmidt Rank of a Polynomial]
    The schmidt rank of a homogenous polynomial $\funcdef{\genpoly[1]}{\field}{\basefield}$, noted as $\schmrank{\genpoly[1]}$, is the minimal $r$ such that there exist $(\genpoly[2]_i, \genpoly[3]_i)_{i\in[r]}$
    with $\deg{\genpoly[2]_i}, \deg{\genpoly[3]_i} < \deg{\genpoly[1]}$ such that:
    \[
        \genpoly[1](x) = \sum_{i=1}^{r}(\genpoly[2](x) \cdot \genpoly[3](x))
    \]
    For a general polynomial $\genpoly$ of degree $\degree$, we set its rank to be the rank of its degree-$\degree$ homogenous component, i.e. $\schmrank{\genpoly} \definedas \schmrank{\homopart{\genpoly}}$.
\end{definition}

\begin{remark}[High rank implies high schmidt rank]\label{high-rank-implies-high-schmidt-rank}
If $\rank{\genpoly} \geq 2 \cdot r + 1$ for some constant $r \in \naturalnumbersset$, then $\schmrank{\genpoly} \geq r$.
\end{remark}
\begin{proof}
    For homogenous polynomial $\genpoly$, assume $\schmrank{\genpoly} < r$.
    Then, there exist $r^\prime < r$ such that there exist $(\genpoly[2]_i, \genpoly[3]_i)_{i=1}^{r^\prime}$ with $\deg{\genpoly[2]_i}, \deg{\genpoly[3]}_i < \deg \genpoly$ such that:
    \[
        \genpoly[1](x) = \sum_{i=1}^{r^\prime}(\genpoly[2](x) \cdot \genpoly[3](x))
    \]
    Then we can choose $\funcdef{\Gamma}{\basefield^{2r^\prime}}{\basefield}$ to be a sum of multiples of each two consecutive variables to get that
    $\genpoly(x) = \Gamma(\genpoly[2]_1(x), \genpoly[3]_1(x),...,\genpoly[2]_{r^\prime}(x), \genpoly[3]_{r^\prime}(x))$, where the polynomials are from a degree $< \deg(\genpoly)$.
    This means that $\rank{\genpoly} \leq 2r^{\prime} < 2r$ as we requested.
    \newline
    If we do not assume $\genpoly$ is homogenous, by adding $\genpoly - \homopart{\genpoly}$ as an input to $\Gamma$,
    one can create a $\funcdef{\Gamma^\prime}{\basefield^{2r^\prime+1}}{\basefield}$
    which equals to $\genpoly$ when substituting the inputs with some polynomials with degree $< \deg{\genpoly}$,
    which concludes the proof in a similar way.
\end{proof}

Next, we present the definition of Schmidt rank of a factor as defined in ~\cite{lampert2021relative}.
\begin{definition}[Schmidt Rank of a Factor]
    For a factor of homogenous polynomials $\genpolyset = (\genpoly_1,...,\genpoly_{c})$, the schmidt rank of the factor is defined as:
    \[
        \schmrank{\genpolyset} \definedas \min \parens{\schmrank{\sum_{i=1}^{c}\lambda_i \genpoly_i} \suchthat 0 \neq (\lambda_1,...,\lambda_c) \in \basefield^c}
    \]
    Similarly, for a factor of general polynomials $\genpolyset$, we set its rank to be the rank of its matching homogenous-factor,
    i.e. $\schmrank{\genpolyset} \definedas \schmrank{\homopart{\genpolyset}}$
    For a factor $\factor$ generated by $\genpolyset$, we define $\schmrank{\factor} \definedas \schmrank{\genpolyset}$.
\end{definition}

To establish the equivalence of this definition with the one employed throughout the paper, we must first acknowledge two key distinctions between the definitions.
The first distinction is that this definition focuses on the largest-degree homogeneous components of the polynomials involved in the factor, rather than considering linear combinations of polynomials from the factor.
The second distinction pertains to the treatment of $\degree$ in the computation of $\degree$-rank of each linear combination.
This definition uses the degree of the linear combination directly to calculate the rank that participates in the minimum, in contrast to our definition which uses $\max_{i}{\deg(\lambda_i \genpoly_i)}$.
Despite these differences, we will demonstrate that both definitions ultimately yield a similar rank assessment, thereby affirming their equivalence.
\begin{remark}[High Rank Implies High Schmidt Rank for Factors]
    Let $\genpolyset = \parens{\genpoly_1,...,\genpoly_c}$ be a set of polynomials and let $\rankval \in \naturalnumbersset$ be a positive integer, i.e. $\rankval > 0$.
    If $\rank{\genpolyset} \geq 2 \cdot \rankval + 1$, then $\schmrank{\genpolyset} \geq \rankval$.
    \begin{proof}
        Assume that $\schmrank{\genpolyset} \leq \rankval$ for $\rankval > 0$.
        We will show that $\rank{\genpolyset} \leq 2 \rankval + 1$.
        By definition, there exists a linear combination of polynomials in $\homopart{\genpolyset}$ with rank $\leq \rankval$.
        In other words, there exists $\vec{0} \neq \lambda \in \basefield^{c}$ such that $\schmrank{\sum_{i=1}^{c} {\lambda_i \homopart{\genpoly_i}}} \leq \rankval$.
        Denote $\vec{\genpoly_h} \definedas \sum_{i=1}^{c} {\lambda_i \homopart{\genpoly_i}} $.
        As was shown in a previous remark, a rank of a polynomial is smaller than its schmidt rank up to a constant factor,
        thus $\rank{\vec{\genpoly_h}} \leq 2\rankval + 1$ (see Remark~\ref{high-rank-implies-high-schmidt-rank}).
        \newline
        Next, we denote $\vec{\genpoly} \definedas \sum_{i=1}^{c} {\lambda_i \genpoly_i}$, and $\degree_M \definedas \max_{i \in \sparens{c}}{\lambda_i \genpoly_i}$.
        Note that  $\deg(\vec{\genpoly}) \leq \degree_M$.
        We wish to show that $\drank{\degree_M}{\vec{\genpoly}} \leq 2 \rankval + 1$.
        First, we observe that the $\degree_M$-degree homogenous component of $\vec{\genpoly_h}$ equals the $\degree_M$-degree homogenous component of $\vec{\genpoly}$.
        This is true because every highest-degree component of polynomials in the linear combination that generated $\vec{\genpoly}$,
        also exists in the linear combination that generates $\vec{\genpoly_h}$.
        In particular, all homogenous components of degree $\degree_M$ exists in both linear combinations $\vec{\genpoly_h}$ and $\vec{\genpoly}$.
        Therefore, if the degree of $\vec{\genpoly}$ equals $\degree_M$, we have that $\drank{\degree_M}{\vec{\genpoly}} = \rank{\vec{\genpoly}} \geq 2 \rankval + 1$.
        Otherwise, if $\deg(\vec{\genpoly}) < \degree_M$, then $\drank{\degree_M}{\vec{\genpoly}} = 1 \leq 2 \rankval + 1$.
        This completes the proof.
        \newline
    \end{proof}
\end{remark}
\begin{note*}
    In the case discussed above, if $\deg(\vec{\genpoly}) < \degree_M$, then $\schmrank{\genpolyset} = 0$.
    \begin{proof}
        Assume that $\deg(\vec{\genpoly}) < \degree_M$.
        Therefore, the degree of the linear combination $\vec{\genpoly} = \sum_{i=1}^{c} {\lambda_i \genpoly_i}$ is strictly smaller than the degree of at least one of the polynomials participating in it.
        Denote by $\vec{\lambda}^\star$ the sub-combination of $\vec{\lambda}$ that consists only the polynomials that participated in $\vec{\genpoly}$ that are of degree $= \degree_M$.
        Trivially, $\vec{\lambda}^\star \neq \vec{0}$.
        Additionally, we have $\deg(\sum_{i=1}^{c} {\lambda_i^\star \genpoly_i}) < \degree_M$.
        Now, we use the following observation: the linear combination above, when summing only the homogenous components of each polynomial, equals $0$, i.e. $\sum_{i=1}^{c} {\lambda_i^\star \homopart{\genpoly_i}} \equiv 0$.
        By this, we found a linear combination of $\homopart{\genpolyset}$ that is $\equiv 0$.
        Thus by definition, we have $\schmrank{\genpolyset} = 0$.
    \end{proof}
\end{note*}
\begin{note*}
    This shows that if we compare only the differences in the definition of rank of a factor, i.e. the focus on linear combinations of the largest-degree homogenous components in contrast to the use of the maximal degree $\degree$-rank,
    the two definitions for a rank of a factor are equal up to $\pm 1$
    (in case we use the same definition of rank for a single polynomial).
    To avoid confusion, we omit the exact definitions and respective proof.
\end{note*}

We now present the definition of relative rank as stated in ~\cite[Definition 1.6]{lampert2021relative}:
We remind the reader that this definition is specifically defined to subsets $\variety \subseteq \field$ that can be expressed by $\variety = \zerofunc{\varpolyset}$ for some set of polynomials $\varpolyset$, and not to a general set $\variety \subseteq \field$.
\begin{definition}[Relative Schmidt Rank of a Polynomial]
    The relative schmidt rank of a homogeneous polynomial $\genpoly[1]$ relative to a collection of homogeneous polynomials $\varpolyset=(\series{\varpoly}{1}{\varietypolycount})$ is
    \[
        \relschmrank{\varpolyset}{\genpoly[1]} \definedas
        \min \set{\schmrank{P+\sum_{i=1}^{\varietypolycount}{\remainderpoly_{i}\varpoly_{i}}}
            \suchthat
            \deg(\varpoly_{i})+\deg(\remainderpoly_{i}) \leq \deg(\genpoly[1]), \forall i \in [\varietypolycount]}
    \]
    Note that whenever $\deg{\varpoly_{i}}>\deg{\genpoly}$, this implies $\remainderpoly_{i}=0$.
    \newline
    For general polynomial $\genpoly$ and general collection of polynomials $\varpolyset$, we define
    the schmidt rank of the former in respect to the latter by the relative rank of their largest-degree homogenous component,
    i.e. $\relschmrank{\varpolyset}{\genpoly} \definedas \relschmrank{\homopart{\varpolyset}}{\homopart{\genpoly}}$.
\end{definition}
\begin{remark}[High Relative Rank $\Rightarrow$ High Relative Schmidt Rank]\label{remark-high-relative-rank-implies-high-relative-schmidt-rank}
    Let $\genpoly$ and $\varpolyset = \set{\varpoly_1,...,\varpoly_{\varietypolycount}}$ be polynomials,
    and let $\variety \subseteq \field$ be defined as $\variety = \zerofunc{\varpolyset}$.

    If $\relrank{\variety}{\genpoly} \geq 2 \cdot \rankval + 2$ for some constant $\rankval \in \naturalnumbersset$,
    then $\relschmrank{\varpolyset}{\genpoly} \geq r$.
\end{remark}
\begin{proof}
    Let $\genpoly$ and $\varpoly_1,...,\varpoly_{\varietypolycount}$ be polynomials.
    Assume that $\relschmrank{\varpolyset}{\genpoly} \leq \rankval$.
    Then, there exists $\remainderpoly_1,...,\remainderpoly_\varietypolycount$ with
    $\deg(\varpoly_{i})+\deg(\remainderpoly_{i}) \leq \deg(\genpoly[1])$ for all $i \in \sparens{\varietypolycount}$
    such that:
    \[
        \schmrank{\homopart{\genpoly} + \sum_{i=1}^{\varietypolycount}{\remainderpoly_i \homopart{\varpoly_i}}} \leq \rankval
    \]
    Denote $\relativeremainder{\genpoly_h} \definedas \sum_{i=1}^{\varietypolycount}{\remainderpoly_i \homopart{\varpoly_i}}$.
    As we have shown earlier, a rank of a polynomial is smaller than its schmidt rank up to a constant factor (See Remark~\ref{high-rank-implies-high-schmidt-rank}).
    Thus:
    \[
        \rank{\homopart{\genpoly} + \relativeremainder{\genpoly_h}} \leq
        2 \cdot \schmrank{\homopart{\genpoly} + \relativeremainder{\genpoly_h}} + 1 \leq
        2 \cdot \relschmrank{\variety}{\genpoly} + 1 =
        2 \rankval + 1
    \]
    Next, we denote the respective remainder polynomial for the non-homogenous analogue, i.e $\relativeremainder{\genpoly} \definedas \sum_{i=1}^{\varietypolycount}{\remainderpoly_i \varpoly_i}$.
    By observing the highest degree homogenous component of each summand, one can see that $\homopart{\genpoly + \relativeremainder{\genpoly}} = \homopart{\homopart{\genpoly} + \relativeremainder{\genpoly_h}}$.
    Therefore, by adding to the decomposition the non higest-degree-homogenous-component, one can see that:
    \[
        \rank{\genpoly + \relativeremainder{\genpoly}} \leq
        \rank{\homopart{\genpoly} + \relativeremainder{\genpoly_h}} + 1 \leq
        2 \rankval + 2
    \]
    This completes the proof as $\relrank{\variety}{\genpoly} \leq \rank{\genpoly + \relativeremainder{\genpoly}} \leq 2 \rankval + 2$.
\end{proof}

\begin{remark}[Relative Schmidt Rank over Varieties of High Degree]\label{relative-schimdt-rank-equals-schmidt-rank-if-the-variety-is-of-high-degree}
If the polynomials defining the variety $\varpolyset = (\varpoly_1,...,\varpoly_{\varietypolycount})$ are of degree $> \deg(\genpoly)$,
then, $\relschmrank{\varpolyset}{\genpoly} = \schmrank{\genpoly}$.
This is true because in this case, in the calculation of the minimum in the definition of relative schmidt rank, we must have $\remainderpoly_i = 1$ for all $i \in [\varietypolycount]$ and therefore the minimum above is simply $\rank{\genpoly}$.
\newline
Note that a similar statement holds for factors aswell.
If $\genpolyset = (\genpoly_1,...,\genpoly_c)$ is a factor of degree $\degree$, then if all the polynomials in $\varpolyset$ have degree $> \degree$, then the statement above is also true i.e $\relschmrank{\varpolyset}{\genpolyset} = \schmrank{\genpolyset}$.
This is true because for every linear combination of $\genpolyset$ has degree $\leq \degree$ and therefore its relative schimdt rank equals its rank.
\end{remark}

%TODO: Add a note, is the same property holds in our definition?

Finally, we present the extension of the definition of relative rank for polynomials factors:
\begin{definition}[Relative Schmidt Rank of a Factor]
    The relative rank of a set of homogenous polynomials $\genpolyset = \set{\genpoly_1,...,\genpoly_c}$
    relative to another collection of polynomials $\varpolyset = \set{\varpoly_1,...,\varpoly_\varietypolycount}$ is defined as:
    \[
        \relschmrank{\varpolyset}{\genpolyset} \definedas
        \min \set{\relschmrank{\varpolyset}{\sum_{i=1}^c{\lambda_i \genpoly_i}} \suchthat \vec{0} \neq (\lambda_1,...,\lambda_c) \in \basefield^c}
    \]
    If $\genpolyset$ is a general collection of polynomials, then $\relschmrank{\varpolyset}{\genpolyset} \definedas \relschmrank{\varpolyset}{\homopart{\genpolyset}}$.
    \newline
    For a factor $\factor$ generated by a set of polynomials $\genpolyset$, we define its schmidt rank relative to $\variety = \zerofunc{\varpolyset}$
    to be $\relschmrank{\variety}{\factor} \definedas \relschmrank{\varpolyset}{\genpolyset}$.
\end{definition}


\begin{remark}
    Let $\genpolyset = \set{\genpoly_1,...,\genpoly_c}$ and $\varpolyset = \set{\varpoly_1,...,\varpoly_{\varietypolycount}}$ be sets of polynomials,
    and let $\variety \subseteq \field$ be defined as $\variety = \zerofunc{\varpolyset}$.
    Additionally, let $\rankval \in \naturalnumbersset$ such that $\rankval > 0$.
    If $\relrank{\variety}{\genpolyset} \geq 2 \cdot \rankval + 2$ for some constant $\rankval \in \naturalnumbersset$, then $\relschmrank{\varpolyset}{\genpoly} \geq r$.
\end{remark}
\begin{proof}
    Assume that $\relschmrank{\varpolyset}{\genpolyset} \leq \rankval$.
    We will show that $\relrank{\variety}{\genpolyset} \leq 2\rankval + 2$.
    Let $\vec{0} \neq \vec{\lambda} \in \basefield^c$ be some vector of coefficients.
    Let $\vec{\genpoly} \definedas \sum_{i=1}^{c} {\lambda_i \genpoly_i}$ and $\vec{\genpoly_h} \definedas \sum_{i=1}^{c} {\lambda_i \homopart{\genpoly_i}}$
    be the linear combinations of polynomials in $\genpolyset$ and $\homopart{\genpolyset}$ with coefficients $\vec{\lambda}$ respectively,
    and let $\degree_M \definedas \max_{i \in \sparens{c}}{\deg(\lambda_i \genpoly_i)}$.
    Additionally, denote $\hat{\rankval} \definedas \relschmrank{\varpolyset}{\vec{\genpoly_h}} \leq \rankval$.
    It is enough to show that $\drelrank{\degree_M}{\variety}{\vec{\genpoly}} \leq 2 \hat{\rankval} + 2$,
    If $\deg(\vec{\genpoly}) < \degree_M$, then $\drelrank{\degree_M}{\variety}{\vec{\genpoly}} = 1 \leq 2 \rankval + 2$.
    Otherwise, if $\deg(\vec{\genpoly}) = \degree_M$, then the remark follows from Remark~\ref{remark-high-relative-rank-implies-high-relative-schmidt-rank}
    as:
    \[
        \drelrank{\degree_M}{\variety}{\vec{\genpoly}} =
        \relrank{\variety}{\vec{\genpoly}} \leq
        2 \cdot \relschmrank{\varpolyset}{\vec{\genpoly}} + 2
    \]
    Where:
    \[
        \relschmrank{\varpolyset}{\vec{\genpoly}}
        \relschmrank{\varpolyset}{\homopart{\vec{\genpoly}}} =
        \relschmrank{\varpolyset}{\homopart{\vec{\genpoly_h}}} =
        \relschmrank{\varpolyset}{\vec{\genpoly_h}} =
        \hat{\rankval}
    \]
\end{proof}


    \end{appendices}

    \bibliography{main}
    \bibliographystyle{alpha}


\end{document}
