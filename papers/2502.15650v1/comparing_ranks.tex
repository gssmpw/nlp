
\section[Comparing Ranks]{Comparing Ranks}\label{sec:comparing-ranks}
In this section, we compare the definition of rank we used in this paper to another definition of rank used implicitly throughout this paper.
This comparison is crucial, as there is no universally accepted definition of rank;
different theorems presented throughout this paper employ distinct definitions.
We demonstrate that our definition is sufficiently comprehensive, in that a polynomial (or a factor) classified as having high rank according to our criteria
also exhibits high rank according to the second implicitly-used definition.
While in many cases the comparison may appear straightforward, we include it for the sake of completeness.
\newline
Specifically, we compare our definition of rank with the definition established in ~\cite{lampert2021relative}.
The paper~\cite{lampert2021relative} extended the original definition of rank that was presented in \cite{10.1007/BF02392473},
to include also the concept of relative rank.
It is important to note that this definition is specifically defined to subsets $\variety \subseteq \field$ that can be expressed as sets in the form $\variety = \zerofunc{\varpolyset}$ for some set of polynomials $\varpolyset$,
and not to a general set $\variety \subseteq \field$.
\newline
First, we present a useful notation that is used in the definition presented in ~\cite{lampert2021relative}:
\begin{notation*}[Largest Degree Homogenous Part]
    For a polynomial $\genpoly$ of degree $\degree$, we denote by $\homopart{\genpoly}$ its degree-$\degree$ homogenous component.
    In other words, $\homopart{\genpoly}$ is the sum of all the monomials of $\genpoly$ of degree exactly $\degree$.
    For a set of polynomials $\genpolyset = \set{\genpoly_1,...,\genpoly_c}$, we define $\homopart{\genpolyset} \definedas \set{\homopart{\genpoly_i} \suchthat i = 1,...,c}$.
\end{notation*}
Next, we present the exact definition of rank for a polynomial:
\begin{definition}[Schmidt Rank of a Polynomial]
    The schmidt rank of a homogenous polynomial $\funcdef{\genpoly[1]}{\field}{\basefield}$, noted as $\schmrank{\genpoly[1]}$, is the minimal $r$ such that there exist $(\genpoly[2]_i, \genpoly[3]_i)_{i\in[r]}$
    with $\deg{\genpoly[2]_i}, \deg{\genpoly[3]_i} < \deg{\genpoly[1]}$ such that:
    \[
        \genpoly[1](x) = \sum_{i=1}^{r}(\genpoly[2](x) \cdot \genpoly[3](x))
    \]
    For a general polynomial $\genpoly$ of degree $\degree$, we set its rank to be the rank of its degree-$\degree$ homogenous component, i.e. $\schmrank{\genpoly} \definedas \schmrank{\homopart{\genpoly}}$.
\end{definition}

\begin{remark}[High rank implies high schmidt rank]\label{high-rank-implies-high-schmidt-rank}
If $\rank{\genpoly} \geq 2 \cdot r + 1$ for some constant $r \in \naturalnumbersset$, then $\schmrank{\genpoly} \geq r$.
\end{remark}
\begin{proof}
    For homogenous polynomial $\genpoly$, assume $\schmrank{\genpoly} < r$.
    Then, there exist $r^\prime < r$ such that there exist $(\genpoly[2]_i, \genpoly[3]_i)_{i=1}^{r^\prime}$ with $\deg{\genpoly[2]_i}, \deg{\genpoly[3]}_i < \deg \genpoly$ such that:
    \[
        \genpoly[1](x) = \sum_{i=1}^{r^\prime}(\genpoly[2](x) \cdot \genpoly[3](x))
    \]
    Then we can choose $\funcdef{\Gamma}{\basefield^{2r^\prime}}{\basefield}$ to be a sum of multiples of each two consecutive variables to get that
    $\genpoly(x) = \Gamma(\genpoly[2]_1(x), \genpoly[3]_1(x),...,\genpoly[2]_{r^\prime}(x), \genpoly[3]_{r^\prime}(x))$, where the polynomials are from a degree $< \deg(\genpoly)$.
    This means that $\rank{\genpoly} \leq 2r^{\prime} < 2r$ as we requested.
    \newline
    If we do not assume $\genpoly$ is homogenous, by adding $\genpoly - \homopart{\genpoly}$ as an input to $\Gamma$,
    one can create a $\funcdef{\Gamma^\prime}{\basefield^{2r^\prime+1}}{\basefield}$
    which equals to $\genpoly$ when substituting the inputs with some polynomials with degree $< \deg{\genpoly}$,
    which concludes the proof in a similar way.
\end{proof}

Next, we present the definition of Schmidt rank of a factor as defined in ~\cite{lampert2021relative}.
\begin{definition}[Schmidt Rank of a Factor]
    For a factor of homogenous polynomials $\genpolyset = (\genpoly_1,...,\genpoly_{c})$, the schmidt rank of the factor is defined as:
    \[
        \schmrank{\genpolyset} \definedas \min \parens{\schmrank{\sum_{i=1}^{c}\lambda_i \genpoly_i} \suchthat 0 \neq (\lambda_1,...,\lambda_c) \in \basefield^c}
    \]
    Similarly, for a factor of general polynomials $\genpolyset$, we set its rank to be the rank of its matching homogenous-factor,
    i.e. $\schmrank{\genpolyset} \definedas \schmrank{\homopart{\genpolyset}}$
    For a factor $\factor$ generated by $\genpolyset$, we define $\schmrank{\factor} \definedas \schmrank{\genpolyset}$.
\end{definition}

To establish the equivalence of this definition with the one employed throughout the paper, we must first acknowledge two key distinctions between the definitions.
The first distinction is that this definition focuses on the largest-degree homogeneous components of the polynomials involved in the factor, rather than considering linear combinations of polynomials from the factor.
The second distinction pertains to the treatment of $\degree$ in the computation of $\degree$-rank of each linear combination.
This definition uses the degree of the linear combination directly to calculate the rank that participates in the minimum, in contrast to our definition which uses $\max_{i}{\deg(\lambda_i \genpoly_i)}$.
Despite these differences, we will demonstrate that both definitions ultimately yield a similar rank assessment, thereby affirming their equivalence.
\begin{remark}[High Rank Implies High Schmidt Rank for Factors]
    Let $\genpolyset = \parens{\genpoly_1,...,\genpoly_c}$ be a set of polynomials and let $\rankval \in \naturalnumbersset$ be a positive integer, i.e. $\rankval > 0$.
    If $\rank{\genpolyset} \geq 2 \cdot \rankval + 1$, then $\schmrank{\genpolyset} \geq \rankval$.
    \begin{proof}
        Assume that $\schmrank{\genpolyset} \leq \rankval$ for $\rankval > 0$.
        We will show that $\rank{\genpolyset} \leq 2 \rankval + 1$.
        By definition, there exists a linear combination of polynomials in $\homopart{\genpolyset}$ with rank $\leq \rankval$.
        In other words, there exists $\vec{0} \neq \lambda \in \basefield^{c}$ such that $\schmrank{\sum_{i=1}^{c} {\lambda_i \homopart{\genpoly_i}}} \leq \rankval$.
        Denote $\vec{\genpoly_h} \definedas \sum_{i=1}^{c} {\lambda_i \homopart{\genpoly_i}} $.
        As was shown in a previous remark, a rank of a polynomial is smaller than its schmidt rank up to a constant factor,
        thus $\rank{\vec{\genpoly_h}} \leq 2\rankval + 1$ (see Remark~\ref{high-rank-implies-high-schmidt-rank}).
        \newline
        Next, we denote $\vec{\genpoly} \definedas \sum_{i=1}^{c} {\lambda_i \genpoly_i}$, and $\degree_M \definedas \max_{i \in \sparens{c}}{\lambda_i \genpoly_i}$.
        Note that  $\deg(\vec{\genpoly}) \leq \degree_M$.
        We wish to show that $\drank{\degree_M}{\vec{\genpoly}} \leq 2 \rankval + 1$.
        First, we observe that the $\degree_M$-degree homogenous component of $\vec{\genpoly_h}$ equals the $\degree_M$-degree homogenous component of $\vec{\genpoly}$.
        This is true because every highest-degree component of polynomials in the linear combination that generated $\vec{\genpoly}$,
        also exists in the linear combination that generates $\vec{\genpoly_h}$.
        In particular, all homogenous components of degree $\degree_M$ exists in both linear combinations $\vec{\genpoly_h}$ and $\vec{\genpoly}$.
        Therefore, if the degree of $\vec{\genpoly}$ equals $\degree_M$, we have that $\drank{\degree_M}{\vec{\genpoly}} = \rank{\vec{\genpoly}} \geq 2 \rankval + 1$.
        Otherwise, if $\deg(\vec{\genpoly}) < \degree_M$, then $\drank{\degree_M}{\vec{\genpoly}} = 1 \leq 2 \rankval + 1$.
        This completes the proof.
        \newline
    \end{proof}
\end{remark}
\begin{note*}
    In the case discussed above, if $\deg(\vec{\genpoly}) < \degree_M$, then $\schmrank{\genpolyset} = 0$.
    \begin{proof}
        Assume that $\deg(\vec{\genpoly}) < \degree_M$.
        Therefore, the degree of the linear combination $\vec{\genpoly} = \sum_{i=1}^{c} {\lambda_i \genpoly_i}$ is strictly smaller than the degree of at least one of the polynomials participating in it.
        Denote by $\vec{\lambda}^\star$ the sub-combination of $\vec{\lambda}$ that consists only the polynomials that participated in $\vec{\genpoly}$ that are of degree $= \degree_M$.
        Trivially, $\vec{\lambda}^\star \neq \vec{0}$.
        Additionally, we have $\deg(\sum_{i=1}^{c} {\lambda_i^\star \genpoly_i}) < \degree_M$.
        Now, we use the following observation: the linear combination above, when summing only the homogenous components of each polynomial, equals $0$, i.e. $\sum_{i=1}^{c} {\lambda_i^\star \homopart{\genpoly_i}} \equiv 0$.
        By this, we found a linear combination of $\homopart{\genpolyset}$ that is $\equiv 0$.
        Thus by definition, we have $\schmrank{\genpolyset} = 0$.
    \end{proof}
\end{note*}
\begin{note*}
    This shows that if we compare only the differences in the definition of rank of a factor, i.e. the focus on linear combinations of the largest-degree homogenous components in contrast to the use of the maximal degree $\degree$-rank,
    the two definitions for a rank of a factor are equal up to $\pm 1$
    (in case we use the same definition of rank for a single polynomial).
    To avoid confusion, we omit the exact definitions and respective proof.
\end{note*}

We now present the definition of relative rank as stated in ~\cite[Definition 1.6]{lampert2021relative}:
We remind the reader that this definition is specifically defined to subsets $\variety \subseteq \field$ that can be expressed by $\variety = \zerofunc{\varpolyset}$ for some set of polynomials $\varpolyset$, and not to a general set $\variety \subseteq \field$.
\begin{definition}[Relative Schmidt Rank of a Polynomial]
    The relative schmidt rank of a homogeneous polynomial $\genpoly[1]$ relative to a collection of homogeneous polynomials $\varpolyset=(\series{\varpoly}{1}{\varietypolycount})$ is
    \[
        \relschmrank{\varpolyset}{\genpoly[1]} \definedas
        \min \set{\schmrank{P+\sum_{i=1}^{\varietypolycount}{\remainderpoly_{i}\varpoly_{i}}}
            \suchthat
            \deg(\varpoly_{i})+\deg(\remainderpoly_{i}) \leq \deg(\genpoly[1]), \forall i \in [\varietypolycount]}
    \]
    Note that whenever $\deg{\varpoly_{i}}>\deg{\genpoly}$, this implies $\remainderpoly_{i}=0$.
    \newline
    For general polynomial $\genpoly$ and general collection of polynomials $\varpolyset$, we define
    the schmidt rank of the former in respect to the latter by the relative rank of their largest-degree homogenous component,
    i.e. $\relschmrank{\varpolyset}{\genpoly} \definedas \relschmrank{\homopart{\varpolyset}}{\homopart{\genpoly}}$.
\end{definition}
\begin{remark}[High Relative Rank $\Rightarrow$ High Relative Schmidt Rank]\label{remark-high-relative-rank-implies-high-relative-schmidt-rank}
    Let $\genpoly$ and $\varpolyset = \set{\varpoly_1,...,\varpoly_{\varietypolycount}}$ be polynomials,
    and let $\variety \subseteq \field$ be defined as $\variety = \zerofunc{\varpolyset}$.

    If $\relrank{\variety}{\genpoly} \geq 2 \cdot \rankval + 2$ for some constant $\rankval \in \naturalnumbersset$,
    then $\relschmrank{\varpolyset}{\genpoly} \geq r$.
\end{remark}
\begin{proof}
    Let $\genpoly$ and $\varpoly_1,...,\varpoly_{\varietypolycount}$ be polynomials.
    Assume that $\relschmrank{\varpolyset}{\genpoly} \leq \rankval$.
    Then, there exists $\remainderpoly_1,...,\remainderpoly_\varietypolycount$ with
    $\deg(\varpoly_{i})+\deg(\remainderpoly_{i}) \leq \deg(\genpoly[1])$ for all $i \in \sparens{\varietypolycount}$
    such that:
    \[
        \schmrank{\homopart{\genpoly} + \sum_{i=1}^{\varietypolycount}{\remainderpoly_i \homopart{\varpoly_i}}} \leq \rankval
    \]
    Denote $\relativeremainder{\genpoly_h} \definedas \sum_{i=1}^{\varietypolycount}{\remainderpoly_i \homopart{\varpoly_i}}$.
    As we have shown earlier, a rank of a polynomial is smaller than its schmidt rank up to a constant factor (See Remark~\ref{high-rank-implies-high-schmidt-rank}).
    Thus:
    \[
        \rank{\homopart{\genpoly} + \relativeremainder{\genpoly_h}} \leq
        2 \cdot \schmrank{\homopart{\genpoly} + \relativeremainder{\genpoly_h}} + 1 \leq
        2 \cdot \relschmrank{\variety}{\genpoly} + 1 =
        2 \rankval + 1
    \]
    Next, we denote the respective remainder polynomial for the non-homogenous analogue, i.e $\relativeremainder{\genpoly} \definedas \sum_{i=1}^{\varietypolycount}{\remainderpoly_i \varpoly_i}$.
    By observing the highest degree homogenous component of each summand, one can see that $\homopart{\genpoly + \relativeremainder{\genpoly}} = \homopart{\homopart{\genpoly} + \relativeremainder{\genpoly_h}}$.
    Therefore, by adding to the decomposition the non higest-degree-homogenous-component, one can see that:
    \[
        \rank{\genpoly + \relativeremainder{\genpoly}} \leq
        \rank{\homopart{\genpoly} + \relativeremainder{\genpoly_h}} + 1 \leq
        2 \rankval + 2
    \]
    This completes the proof as $\relrank{\variety}{\genpoly} \leq \rank{\genpoly + \relativeremainder{\genpoly}} \leq 2 \rankval + 2$.
\end{proof}

\begin{remark}[Relative Schmidt Rank over Varieties of High Degree]\label{relative-schimdt-rank-equals-schmidt-rank-if-the-variety-is-of-high-degree}
If the polynomials defining the variety $\varpolyset = (\varpoly_1,...,\varpoly_{\varietypolycount})$ are of degree $> \deg(\genpoly)$,
then, $\relschmrank{\varpolyset}{\genpoly} = \schmrank{\genpoly}$.
This is true because in this case, in the calculation of the minimum in the definition of relative schmidt rank, we must have $\remainderpoly_i = 1$ for all $i \in [\varietypolycount]$ and therefore the minimum above is simply $\rank{\genpoly}$.
\newline
Note that a similar statement holds for factors aswell.
If $\genpolyset = (\genpoly_1,...,\genpoly_c)$ is a factor of degree $\degree$, then if all the polynomials in $\varpolyset$ have degree $> \degree$, then the statement above is also true i.e $\relschmrank{\varpolyset}{\genpolyset} = \schmrank{\genpolyset}$.
This is true because for every linear combination of $\genpolyset$ has degree $\leq \degree$ and therefore its relative schimdt rank equals its rank.
\end{remark}

%TODO: Add a note, is the same property holds in our definition?

Finally, we present the extension of the definition of relative rank for polynomials factors:
\begin{definition}[Relative Schmidt Rank of a Factor]
    The relative rank of a set of homogenous polynomials $\genpolyset = \set{\genpoly_1,...,\genpoly_c}$
    relative to another collection of polynomials $\varpolyset = \set{\varpoly_1,...,\varpoly_\varietypolycount}$ is defined as:
    \[
        \relschmrank{\varpolyset}{\genpolyset} \definedas
        \min \set{\relschmrank{\varpolyset}{\sum_{i=1}^c{\lambda_i \genpoly_i}} \suchthat \vec{0} \neq (\lambda_1,...,\lambda_c) \in \basefield^c}
    \]
    If $\genpolyset$ is a general collection of polynomials, then $\relschmrank{\varpolyset}{\genpolyset} \definedas \relschmrank{\varpolyset}{\homopart{\genpolyset}}$.
    \newline
    For a factor $\factor$ generated by a set of polynomials $\genpolyset$, we define its schmidt rank relative to $\variety = \zerofunc{\varpolyset}$
    to be $\relschmrank{\variety}{\factor} \definedas \relschmrank{\varpolyset}{\genpolyset}$.
\end{definition}


\begin{remark}
    Let $\genpolyset = \set{\genpoly_1,...,\genpoly_c}$ and $\varpolyset = \set{\varpoly_1,...,\varpoly_{\varietypolycount}}$ be sets of polynomials,
    and let $\variety \subseteq \field$ be defined as $\variety = \zerofunc{\varpolyset}$.
    Additionally, let $\rankval \in \naturalnumbersset$ such that $\rankval > 0$.
    If $\relrank{\variety}{\genpolyset} \geq 2 \cdot \rankval + 2$ for some constant $\rankval \in \naturalnumbersset$, then $\relschmrank{\varpolyset}{\genpoly} \geq r$.
\end{remark}
\begin{proof}
    Assume that $\relschmrank{\varpolyset}{\genpolyset} \leq \rankval$.
    We will show that $\relrank{\variety}{\genpolyset} \leq 2\rankval + 2$.
    Let $\vec{0} \neq \vec{\lambda} \in \basefield^c$ be some vector of coefficients.
    Let $\vec{\genpoly} \definedas \sum_{i=1}^{c} {\lambda_i \genpoly_i}$ and $\vec{\genpoly_h} \definedas \sum_{i=1}^{c} {\lambda_i \homopart{\genpoly_i}}$
    be the linear combinations of polynomials in $\genpolyset$ and $\homopart{\genpolyset}$ with coefficients $\vec{\lambda}$ respectively,
    and let $\degree_M \definedas \max_{i \in \sparens{c}}{\deg(\lambda_i \genpoly_i)}$.
    Additionally, denote $\hat{\rankval} \definedas \relschmrank{\varpolyset}{\vec{\genpoly_h}} \leq \rankval$.
    It is enough to show that $\drelrank{\degree_M}{\variety}{\vec{\genpoly}} \leq 2 \hat{\rankval} + 2$,
    If $\deg(\vec{\genpoly}) < \degree_M$, then $\drelrank{\degree_M}{\variety}{\vec{\genpoly}} = 1 \leq 2 \rankval + 2$.
    Otherwise, if $\deg(\vec{\genpoly}) = \degree_M$, then the remark follows from Remark~\ref{remark-high-relative-rank-implies-high-relative-schmidt-rank}
    as:
    \[
        \drelrank{\degree_M}{\variety}{\vec{\genpoly}} =
        \relrank{\variety}{\vec{\genpoly}} \leq
        2 \cdot \relschmrank{\varpolyset}{\vec{\genpoly}} + 2
    \]
    Where:
    \[
        \relschmrank{\varpolyset}{\vec{\genpoly}}
        \relschmrank{\varpolyset}{\homopart{\vec{\genpoly}}} =
        \relschmrank{\varpolyset}{\homopart{\vec{\genpoly_h}}} =
        \relschmrank{\varpolyset}{\vec{\genpoly_h}} =
        \hat{\rankval}
    \]
\end{proof}

