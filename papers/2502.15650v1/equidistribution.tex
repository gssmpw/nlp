\section[Equidistribution of Functions]{Equidistribution of Functions}\label{sec:equidistribution-of-functions}
Assume we have a collection of functions $(\genfunc_1,...,\genfunc_c$), where $\funcdef{\genfunc_i}{A}{\basefield}$ for some finite set $A$.
We are interested in showing that the functions are equidistributed, which means that their values behave close to independent random variables.
We begin by formulating this definition:
\begin{definition}[Equidistribution of Functions]
    Given $\epsilon > 0$ and $A \subseteq \field$,
    we say a collection of functions $\genfuncset = (\genfunc_1,...,\genfunc_c)$ where $\funcdef{\genfunc_i}{\genset}{\basefield}$ is $\epsilon$-equidistributed in $A$ if for all $\vec{\alpha} = (\alpha_1,...,\alpha_c) \in \basefield^c$ we have:
    \[
        \prex{x \in A}{(\genfunc_1(x),...,\genfunc_c(x)) = \vec{\alpha}} = \frac{1}{\abs{\basefield}^c} \pm \epsilon
    \]
\end{definition}

The following is a standard lemma that shows that if every linear combination of a collection of functions has low bias, the collection is equidistributed.
We repeat the steps of the proof of \cite[Lemma 7.24]{book}, but here, we think of $A$ as any finite set (and not particularly $\field$):
\begin{lemma}\label{every-linear-combination-has-low-bias-implies-equidistribution}
Let $\epsilon > 0$, and let $A$ be a finite set.
Let $\genfuncset = (\genfunc_1,...,\genfunc_c)$ be a collection of functions defined over $A$, i.e. $\funcdef{\genfunc_i}{A}{\basefield}$.
Assume each linear combination of the collection has low bias, i.e for each $\lambda = (\lambda_1,...,\lambda_c) \in \basefield^c$ such that $\lambda \neq \vec{0}$ we have:
\[
    \relbias{x \in A}{\sum_{i=1}^{c}{\lambda_i \genfunc_i}} < \epsilon
\]
Then, the collection $\genfuncset$ is $\epsilon$-equidistributed over $A$.
\newline
In particular, for $\epsilon < \frac{1}{\abs{\basefield}^c}$, the lemma shows that each atom of $\genfuncset$ is not empty i.e for all $\vec{\alpha}$ there is some $x \in A$ such that $(\genfunc_1(x),...,\genfunc_c(x)) = \vec{\alpha}$.
\end{lemma}
\begin{proof}
    We wish to show that for each $\vec{\alpha} \in \basefield^c$ we have:
    \[
        \prex{x \in A}{(\genfunc_1(x),...,\genfunc_c(x)) = \vec{\alpha}} = \frac{1}{\abs{\basefield}} \pm \epsilon
    \]
    We express the fraction of inputs that are in the atom $\vec{\alpha}$ the following way:
    \[
        \prex{x \in A}{(\genfunc_1(x),...,\genfunc_c(x))} =
        \expectation{x \in A}{\prod_{i=1}^c {1_{[\genfunc_i(x) = \alpha_i]}}}
    \]
    We use the fact that for every $0 \neq x \in \basefield$, we have $\sum_{\lambda = 0}^{\basefieldsize - 1} \charfunc{\lambda x} = 0$,
    and if $x = 0$ we have $\sum_{\lambda = 0}^{\basefieldsize - 1} \charfunc{\lambda x} = \basefieldsize$.
    Therefore, the expression above equals:
    \[
        =\expectation{x \in A}{\prod_{i=1}^c \parens {{\frac{1}{\basefieldsize} \cdot \sum_{\lambda_i = 0}^{\basefieldsize - 1} {\charfunc{\lambda_i (\genfunc_i(x) - \alpha_i)}}}}} =\\
        \frac{1}{\basefieldsize^c} \cdot {\expectation{x \in A}{\prod_{i=1}^c \sum_{\lambda_i = 0}^{\basefieldsize - 1} {\charfunc{\lambda_i (\genfunc_i(x) - \alpha_i)}}}}
    \]
    By the definition of character functions, we have that $\charfunc{a+b} = \charfunc{a} \cdot \charfunc{b}$, and therefore the expression above equals:
    \[
        \frac{1}{\basefieldsize^c} \cdot \sum_{(\lambda_1,...,\lambda_c) \in \prod_{i=1}^c [0, \basefieldsize - 1]} \parens{\expectation{x \in A}{\charfunc{\sum_{i = 0}^{c} {\lambda_i (\genfunc_i(x) - \alpha_i)}}}}
    \]
    Now, we use the fact that:
    \[
        \relbias{x \in A}{\sum_{i=1}^c (\lambda_i (\genfunc_i(x) -\alpha_i)} = \relbias{x \in A}{\sum_{i=1}^c (\lambda_i \genfunc_i(x))} < \epsilon
    \]
    and get that:
    \[
        \prex{x \in A}{(\genfunc_1(x),...,\genfunc_c(x)) = \vec{\alpha}} = \frac{1}{\basefieldsize^c} \cdot \parens{1 \pm \epsilon \prod_{i=1}^c{\basefieldsize}} = \frac{1}{\abs{\basefield}^c} \pm \epsilon
    \]
\end{proof}

%Now, if we have $A = \basefield^{\blocklength}$ for some $\blocklength \in \naturalnumbersset$, we have the following corollary:
%\begin{corollary}[High rank implies equidistribution in $\field$]
%    TODO %TODO
%\end{corollary}
%
%Moreover, if we have that $A = \variety$ when $\variety \subseteq \field$ has the relative rank-bias property, we have:
%\begin{corollary}[High relative rank implies equidistribution in $\variety$]
%    TODO %TODO
%\end{corollary}
%