\section[Polynomials in \titlevariety]{Polynomials in \titlevariety}\label{sec:polynomials_in_X}
In this section we wish to generalize the definition of degree-$\degree$ polynomials for functions $\funcdef{\onvarfunc}{\variety}{\basefield}$.
Note that we wish to define it using a property of $\onvarfunc$ that is intrinsic to $\variety$: given a function $\funcdef{\onvarfunc}{\variety}{\basefield}$,
we wish be able to determine its degree only using values of $\variety$, without considering any value outside of $\variety$ (such as values of $\field \setminus \variety$).
\newline
To define such property, we generalize the local definition of a degree that is defined for polynomials in $\field$.
We remind the reader that in $\field$, we said a function over $\field$ is a polynomial of degree $\leq \degree$ if and only if its $(\degree+1)$-derivative in every direction is $\equiv 0$.
Thus, in order to determine the $(\degree+1)$-derivative of a function in directions $y_1,...,y_{\degree+1}$, one needs to evaluate the function over all the points of the cube generated by $x,y_1,...,y_{\degree+1}$,
which is the set of points $\set{x + \sum_{i \in S}{y_i}}_{S \subseteq [\degree+1]}$.
This raises a challnge in extending this definition for functions defined over $\variety \subseteq \field$:
depending on $\variety$, the function $\funcdef{\onvarfunc}{\variety}{\basefield}$ is not be defined to all points in all the cubes of $\field$, because some of those points do not lie in $\variety$.
\newline
Therefore, to generalize the definition of a polynomial to $\variety$, we start by giving the formal definition and notation of the set of cubes in $\variety$:
\begin{definition}[Cubes]
    Let $k \in \naturalnumbersset$ be an integer and let $x, y_1,...,y_{k} \in \field$.
    We define the cube $\cube{x}{y_1,...y_{k}}$ as follows:
    \[
        \cube{x}{y_1,...,y_{k}} \definedas \set{x + \sum_{i \in S}{y_i}}_{S \subseteq [k]}
    \]
    We refer to $x$ as the \emph{offset} of the cube, and $y_1,...,y_{k}$ as the \emph{directions} of the cube.
    \newline
    Moreover, Let $\variety \subseteq \field$ be a subset.
    We define the \emph{set of cubes of $\variety$ of size $k$} as follows:
    \[
        \cubes{k}{\variety} \definedas
            \set{\cube{x}{y_1,...,y_{k}} \suchthat \forall S \subseteq[k]: (x + \sum_{i \in S}{y_i}) \in \variety}
    \]
\end{definition}

Using this definition, we can define a polynomial of degree $\leq \degree$ for subsets of $\field$:
\begin{definition}[Polynomials in $\variety$]
    Let $\degree \in \naturalnumbersset$ be an integer, and let $\variety \subseteq \field$.
    We say the degree of a function $\funcdef{\onvarfunc}{\variety}{\basefield}$ is $\degree$
    if $\degree$ is the smallest integer such that $\onvarfunc$ vanishes over all cubes of size $(\degree+1)$, i.e:
    \[
        \forall {\cube{x}{y_1,...,y_{\degree + 1}} \in \cubes{\degree+1}{\variety}}:
            \directionderivative_{y_{\degree + 1}}...\directionderivative_{y_1} \onvarpoly(x) = 0
    \]
    A function over $\variety$ of degree $\leq \degree$ is also called a \emph{polynomial of degree $\leq \degree$}.
    We denote the set of polynomials of degree $\leq \degree$ over $\variety$ by $\allpolyset{\leq \degree}{\variety}{\basefield}$.
\end{definition}
\begin{note*}
    For $\variety = \field$, the definition above coincides with the local definition of polynomials.
\end{note*}

\subsection[Lifting Polynomials]{Lifting Polynomials}\label{subsec:lifting-polynomials}
Our goal to achieve good properties for polynomials over $\variety$.
To do so, we wish to connect the desired properties of polynomials defined over $\variety$, to properties known for polynomials over $\field$.
Following such strategy raises a question: given a polynomial $\funcdef{\onvarpoly}{\variety}{\basefield}$, which polynomial over $\field$ should we consider to deduce properties of $\onvarpoly$?
To find such a polynomial over $\field$, it would have been useful that all polynomials over $\variety$ actually "came from" polynomials over $\field$.
More formally, it would have been useful that all polynomials $\funcdef{\onvarpoly}{\variety}{\basefield}$ would be equal to a restriction of some polynomial $\funcdef{\genpoly}{\field}{\basefield}$ of degree $\leq \degree$, to the set $\variety$.
This would give us a "good candidate" (or candidates) to polynomials over $\field$, that using their known properties, we could achieve the properties we desire for polynomials over $\variety$.
\newline
Generally speaking, the existence of such polynomial $\funcdef{\genpoly}{\field}{\basefield}$ is not trivial by itself, and it mapy depend on the polynomial $\onvarpoly$ and the set $\variety$.
In this subsection, we discuss sets $\variety \subseteq \field$ that have this property for every polynomial $\funcdef{\onvarpoly}{\variety}{\basefield}$.
Before formulating the notion above, we start by a simple remark:
\begin{remark}
    By the local criteria for $\field$, we have that a restriction of a polynomial of degree $\leq \degree$ over $\field$ to $\variety$ is a polynomial of degree $\leq \degree $ over $\variety$.
    Therefore, the other direction is true: every restriction of a polynomial over $\field$ to $\variety$ is a polynomial over $\variety$.
\end{remark}
%First, let us present an exmaple that shows that not every set has this property~\cite[Example 1.4]{kazhdan2019extendingweaklypolynomialfunctions}:
%\begin{example}
%    Fix $\blocklength = 2$.
%    Define $\varpoly(x_1, x_2) = x_1 \cdot x_2 \cdot (x_1 - x_2)$,
%    and consider:
%    \[
%        \variety \definedas \zerofunc{\varpoly} = \set{x = (x_1, x_2) \in \basefield^2 \suchthat x_1 = 0 \vee x_2 = 0 \vee x_1 = x_2}
%    \]
%    Now, consider the function $\funcdef{\onvarpoly}{\variety}{\basefield}$ defined as:
%    $\onvarpoly(x_1, 0) = \onvarpoly(0, x_2) = 0, \onvarpoly(x_1, x_1) = x_1$.
%
%\end{example}

Next, let us define subsets $\variety \subseteq \field$ that have the desired property, which we call \emph{$\degree$-lift-enabler variety}.
\begin{definition}[$\degree$-lift-enabler Subset]
    Let $\basefield$ be a field, and $\blocklength>0$ be an integer.
    For an integer $\degree > 0$, we say a subset $\variety\subseteq\field$ is \emph{$\degree$-lift-enabler} if for every $\degree^{\prime} \leq \degree$,
    for every polynomial $\onvarpoly \in \allpolyset{\degree^{\prime}}{\variety}{\basefield}$
    there exist a polynomial $\lift{\onvarpoly} \in \allpolyset{\degree^{\prime}}{\field}{\basefield}$ such that $\restrictfunc{\onvarpoly}{\variety}=\restrictfunc{\lift{\onvarpoly}}{\variety}$.
\end{definition}
\begin{remark}
    Using the local criterion of polynomials and the fact that that $\cubes{\degree+1}{\variety} \subseteq \cubes{\degree+1}{\field}$,
    one can see that for a polynomial $\funcdef{\onvarpoly}{\variety}{\basefield}$ with $\deg(\onvarpoly) = \degree$,
    every extension $\funcdef{\genpoly}{\field}{\basefield}$ with $\onvarpoly = \restrictfunc{\genpoly}{\variety}$
    holds the bound $\deg(\lift{\onvarpoly}) \geq \degree$.
    The other direction is not true in the general case, but it is specifically promised when the variety is $\degree$-lift-enabler.
\end{remark}

This definition naturally raises the following definition:
\begin{definition}[The Lift Operator]
    Let $\degree \in \naturalnumbersset$ be an integer.
    Let $\variety \subseteq \field$ be a $\degree$-lift-enabler subset.
    We define \emph{the $\degree$-lift operator} to be an operator $\funcdef{\lift{\square}}{\allpolyset{\leq \degree}{\variety}{\basefield}}{\allpolyset{\leq \degree}{\field}{\basefield}}$ the following way:
    \newline
    Let $\degree^\prime \leq \degree$.
    Given a polynomial $\funcdef{\onvarpoly}{\variety}{\basefield}$ of degree $\degree^\prime$, the operator $\lift{\square}$ returns a polynomial $\funcdef{\lift{\onvarpoly}}{\field}{\basefield}$ of degree $\degree^\prime$
    such that $\onvarpoly = \restrictfunc{\lift{\onvarpoly}}{\variety}$.
    Note that we did not require the lift to be unique.
    Thus, in case there are multiple valid lifts for a polynomial $\onvarpoly \in \allpolyset{\leq \degree}{\variety}{\basefield}$, the lift operator picks a single (consistent) one of them.
    Moreover, the lift always exists because the subset $\variety$ is $\degree$-lift-enabler.
    \newline
    In addition, for a collection $\onvarpolyset = (\onvarpoly_1,...,\onvarpoly_c)$ of polynomials $\onvarpoly_i \in \allpolyset{\leq \degree}{\variety}{\basefield}$,
    we denote $\lift{\onvarpolyset} \definedas (\lift{\onvarpoly_1},...,\lift{\onvarpoly_c})$
\end{definition}

In the following subsections, we give example to two concrete sets $\variety \subseteq \field$ that are $\degree$-lift-enablers.
Before doing so, we define an algebraic variety:
\begin{definition}[Algebraic Variety]
    For a collection of functions $\genfuncset \definedas \set{\genfunc_1,...\genfunc_c}$ such that $\funcdef{\genfunc_i}{\field}{\basefield}$,
    we denote $\zerofunc{\genfuncset} \definedas \set{x \in \field \suchthat \forall i: \genfunc_i(x) = 0}$.
    \newline
    If the collection is a collection of polynomials, we call $\zerofunc{\genfuncset}$ an \emph{algebraic variety}, shorthand by \emph{variety}.
    \newline
    The degree of the variety is defined to be the maximal degree of polynomials in the collection that defines it.
    The rank of the variety is defined to be the rank of the collection that defines the variety (as a collection).
\end{definition}

\subsection[High Rank Varieties of High Minimal Degree]{High Rank Varieities of High Minimal Degree}\label{subsec:high-rank-varieities-of-high-degree}
We now present a theorem proved in \cite[Corollary 1.10]{kazhdan2018polynomial}, that shows that high rank varieties are $\degree$-lift-enabler when the polynomials defining the variety are of degree $>\degree$:
\begin{theorem}\label{subsec:high-rank-varities-are-d-lift-enabler}
    Let $\basefield$ be a finite field, and let $\varietydeg$, $\varietypolycount>0$ representing parameters of a variety.
    Let $\degree < \varietydeg$ a positive integer representing a degree of a polynomial which we wish to lift.
    There exists $\varietyrankval=\varietyrankval(\basefield,\varietydeg,\varietypolycount)>0$ such that for for all $\blocklength \in \naturalnumbersset$, any variety $\variety = \zerofunc{\varpolyset} \subseteq \field$ for $\varpolyset = (\varpoly_1,...,\varpoly_{\varietypolycount})$
    such that $\rank{\varpolyset} > \varietyrankval$, degree $\deg(\varpolyset) = \varietydeg$, with all defining polynomials of degree $\deg(\varpoly_i)> \degree$,
    it holds that $\variety$ is a $\degree$-lift-enabler subset.
\end{theorem}
\begin{remark}
    Under the conditions stated above, it was proved in ~\cite{kazhdan2018polynomial} that the lift is in fact \emph{unique}.
    Formally, if $\funcdef{\onvarpoly}{\variety}{\basefield}$ is a polynomial of degree $\leq \degree$,
    then there exists a \emph{unique} polynomial $\funcdef{\genpoly }{\field}{\basefield}$ such that $\restrictfunc{\genpoly}{\variety} \equiv \onvarpoly$.
\end{remark}

\subsection[High Rank Varieties on a Large Field]{High Rank Varieties on a Large Field}\label{subsec:high-rank-varieties-on-a-large-field}
In this subsection, we recall a theorem proved by~\cite[Theorem 1.7]{kazhdan2019extendingweaklypolynomialfunctions} regarding high rank varieties that are defined on "large" fields.
We note that the fields are large in respect of the degree $\degree$ one wish to lift, but still does not depend on $\blocklength$.

Next we define a weakly polynomial, which generalizes our definition of a polynomial in $\variety$, that was used in~\cite[Definition 1.1]{kazhdan2019extendingweaklypolynomialfunctions}:
\begin{definition}
    Let $\variety \subseteq \field$ be a set.
    We say a function $\funcdef{\genfunc}{\variety}{\basefield}$ is a \emph{weakly polynomial of degree $\leq \degree$}
    if for any affine subspace $L \subseteq \variety$, the restriction $\restrictfunc{\genfunc}{L}$ is a polynomial of degree $\leq \degree$.
\end{definition}
\begin{remark}
    By the local criteria of a polynomial, it is easy to see that every $\genpoly \in \allpolyset{\leq \degree}{\variety}{\basefield}$ is a weakly polynomial of degree $\leq \degree$.
\end{remark}
And now, we can present the lifting theorem for large fields, as proved in~\cite[Theorem 2.17]{kazhdan2020propertieshighranksubvarieties}.
\begin{theorem}\label{high-rank-varieties-over-large-fields-are-d-lift-enablers}\cite[Theorem 2.17]{kazhdan2020propertieshighranksubvarieties}
    Let $\degree, \varietydeg \in \naturalnumbersset$,
    and let $\basefield$ be a finite field such that $\abs{\basefield} > \degree \cdot \varietydeg$.
    There exists $\rankfunc_{\ref{high-rank-varieties-over-large-fields-are-d-lift-enablers}} = \rankfunc_{\ref{high-rank-varieties-over-large-fields-are-d-lift-enablers}}(\varietydeg, \degree)$
    such that for any variety $\variety \subseteq \field$ of maximal degree $\leq \varietydeg$ and rank $\geq \rankfunc_{\ref{high-rank-varieties-over-large-fields-are-d-lift-enablers}}$,
    have the following property:
    Every weakly polynomial function $\funcdef{\onvarpoly }{\variety}{\basefield}$ of degree $\leq \degree$
    can be lifted to a polynomial function $\funcdef{\genpoly}{\field}{\basefield}$ of degree $\leq \degree$.
\end{theorem}
\begin{note*}
    Note that we stated the theorem above to finite fields, but it is also valid for infinite algebraically closed fields.
\end{note*}
The theorem above implies the following corollary:
\begin{corollary}
    Let $\degree, \varietydeg \in \naturalnumbersset$,
    and let $\basefield$ be a finite field such that $\abs{\basefield} > \degree \cdot \varietydeg$.
    There exists $\rankfunc_{\ref{high-rank-varieties-over-large-fields-are-d-lift-enablers}} = \rankfunc_{\ref{high-rank-varieties-over-large-fields-are-d-lift-enablers}}(\varietydeg, \degree)$
    such that for any variety $\variety \subseteq \field$ of degree $\varietydeg$ and rank $\geq \rankfunc_{\ref{high-rank-varieties-over-large-fields-are-d-lift-enablers}}$
    is a $\degree$-lift-enabler.
\end{corollary}

