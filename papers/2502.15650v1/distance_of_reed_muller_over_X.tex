\section[Radius of Reed-Muller over \titlevariety]{Radius of Reed-Muller over \titlevariety}\label{sec:radius-of-RM-over-X}
We recall that the normalized distances of Reed-Muller codes over $\field$ and over $\variety$
are denoted by $\normalizedcodedistance{\basefield}{\degree}$ and $\normalizedcodedistanceex{\basefield}{\variety}{\degree}$ respectively.
We present a theorem that shows that Reed-Muller codes over a subset $\variety \subseteq \field$ that is $\degree$-lift-enabler and has the (limited) relative rank-bias property,
has (approximately) an \emph{equal} normalized distance as Reed-Muller codes over $\field$.
%TODO: Change to +- epsilon instead of >
\begin{theorem}\label{thm:distance-of-RM-in-X}
    There exist a function $\epsilon_1(\basefield, \degree, \rankbiasfunc, \epsilonlimitedrankbias)$  such that the following holds:
    Let $\basefield$ be a finite field, and let $\degree \in \naturalnumbersset$ be an integer that represents a degree.
    Let $\epsilonlimitedrankbias > 0$, and let $\funcdef{\rankbiasfunc}{[\epsilonlimitedrankbias, \infty]}{\naturalnumbersset}$ be a limited-relative-rank-bias function.
    \newline
    Let $\variety \subseteq \field$ be a set with the following properties
    \begin{enumerate}
        \item $\variety$ is $\degree$-lift-enabler with a lift operator $\lift{\square}$.
        \item $\variety$ has the $(\rankbiasfunc, \basefield, \degree, \epsilonlimitedrankbias)$-relative-rank-bias property.
    \end{enumerate}
    Then, for $\epsilon_1 \definedas \epsilon_1(\basefield, \degree, \rankbiasfunc, \epsilonlimitedrankbias)$ we have that for all $\blocklength \in \naturalnumbersset$:
    \[
        \normalizedcodedistanceex{\basefield}{\variety}{\degree} \geq \
        \normalizedcodedistance{\basefield}{\degree} - \epsilon_1
    \]
\end{theorem}
\begin{proof}
    We wish to do a reduction of our question regarding the radius of Reed-Muller in $\variety$ to the same question about Reed-Muller in $\field$.
    Let $\basefield$ be a finite field, and let $\degree \in \naturalnumbersset$ be an integer that represents a degree.
    Let $\epsilonlimitedrankbias > 0$, and let $\funcdef{\rankbiasfunc}{[\epsilonlimitedrankbias, \infty]}{\naturalnumbersset}$ be a limited-rank-relative-bias function.
    Let $\epsilon_1 \definedas \epsilon_1(\basefield, \degree, \rankbiasfunc, \epsilonlimitedrankbias)$ be a function we will specify later.
    Let $\variety \subseteq \field$ be a set with the properties defined above.
    \newline
    Moreover, let $\epsilon > \epsilon_1$ be some positive value.
    We will show that:
    \[
        \normalizedcodedistanceex{\basefield}{\variety}{\degree} > \
        \normalizedcodedistance{\basefield}{\degree} - \epsilon
    \]
    This will be enough as if the above holds for every $\epsilon > \epsilon_1$, we get that in fact
    $\normalizedcodedistanceex{\basefield}{\variety}{\degree} \geq \normalizedcodedistance{\basefield}{\degree} - \epsilon_1$.
    \newline
    For start, we note a simple observation: as Reed-Muller over $\variety$ is a linear code, we have
    \[
        \normalizedcodedistanceex{\basefield}{\variety}{\degree} =
        \min \set{\prex{x \in \variety}{\onvarpoly(x) \neq 0} \suchthat \onvarpoly \in \allpolyset{\leq \degree}{\variety}{\basefield}}
    \]
    Now, let $\onvarpoly \in \allpolyset{\leq \degree}{\variety}{\basefield}$ be a polynomial over $\variety$,
    and denote $\degree_{\onvarpoly} \definedas \deg(\onvarpoly)$.
    We wish to lower-bound the value of $\prex{x \in \variety}{\onvarpoly(x) \neq 0}$.
    To do so, we will equivalently upper-bound the value of $\prex{x \in \variety}{\onvarpoly(x) = 0}$.
    Precisely, to complete the proof all we need to show is:
    \[
        \prex{x \in \variety}{\onvarpoly(x) = 0} \leq 1 - \normalizedcodedistance{\basefield}{\degree} + \epsilon
    \]
    \newline
    Now we begin the proof itself.
    First, we lift the polynomial $\onvarpoly$ and get a polynomial $\funcdef{\lift{\onvarpoly}}{\field}{\basefield}$
    such that $\restrictfunc{\lift{\onvarpoly}}{\variety} \equiv \onvarpoly$ and $\deg(\lift{\onvarpoly}) = \degree_{\onvarpoly}$.
    Next, denote by $\factor_{\lift{\onvarpoly}}$ the factor defined by the set of single polynomial $\genpolyset = \set {\lift{\onvarpoly}}$.
    Trivially, the polynomial $\lift{\onvarpoly}$ is measurable in respect of $\genpolyset$.
    \newline
    We define the rank function:
    \[
        \rankfunc(m) \definedas \max \set{
            \rankbiasfunc \parens {\dfrac{\epsilon / 2}{\abs{\basefield}^m}},
            \rankfunc_{\ref{high-rank-implies-low-bias}} \parens{\basefield, \degree, \dfrac{\epsilon / 2}{\abs{\basefield}^m}}}
    \]
    Then, we $\rankfunc$-$\variety$-regularize $\genpolyset$ using Lemma~{\ref{theorem:regularization-in-X}}.
    This gives us a $\rankfunc$-$\variety$-regular factor $\factor^\prime$, which is defined by a set of polynomials $\genpolyset^{\prime} \definedas \set{\genpoly^\prime_1,...,\genpoly^\prime_{c^\prime}}$
    of degree $\leq \degree$ such that $\factor^\prime \relsemrefine{\variety}\factor_{\lift{\onvarpoly}}$ with $\relrank{\variety}{\genpolyset^\prime} \geq \rankfunc$
    and with bounded amount of polynomials defining it i.e,$c^\prime \leq C_{\rankfunc, \degree}^{\ref{theorem:regularization-in-X}}(1)$.
    Therefore, from definition we have that $\lift{\onvarpoly}$ is $\genpolyset^\prime$-measurable relative to $\variety$.
    Thus, there exists a measurement function $\funcdef{\Gamma}{\basefield^{c^\prime}}{\basefield}$
    and a remainder $\funcdef{\relativeremainder{\Gamma}}{\field}{\basefield}$ with $\restrictfunc{\relativeremainder{\Gamma}}{\variety} \equiv 0$
    and degree bounded by $\degree_{\onvarpoly}$, such that:
    \[
        \forall a \in \field:
        \lift{\onvarpoly}(a) =
        \Gamma(\genpoly^\prime_1(a),...,\genpoly^\prime_{c^\prime}(a))
        + \relativeremainder{\Gamma}(a)
    \]
    Next, we denote $\genpoly^\prime \definedas \lift{\onvarpoly} - \relativeremainder{\Gamma}$.
    By definition of remainder function, we have that $\restrictfunc{\genpoly^\prime}{\variety} \equiv \onvarpoly$.
    Additionally, note that $\genpoly^\prime$ is a polynomial over $\field$ of degree $\deg(\genpoly^\prime) = \degree_{\onvarpoly} \leq \degree$,
    and hence by the definition of $\normalizedcodedistance{\basefield}{\degree}$:
    \begin{equation}\label{eq:polynomials-in-fn-are-bounded-away-from-zero-with-high-probability}
    \prex{a \in \field}{\genpoly^\prime(a) = 0} \leq 1 - \normalizedcodedistance{\basefield}{\degree}
    \end{equation}
    For the next step, we claim that $\genpoly^\prime$ equals $0$ in $\field$ approximately with the same probability it equals $0$ in $\variety$.
    Note that this is the heart of the proof: it allows use properties known in $\field$ to new properties in $\variety$.
    This is formulated as follows:
    \begin{claim}\label{claim:p-and-P-have-the-same-approximation}
        We have:
        \[
            \abs{\prex{a \in \field}{\genpoly^\prime(a)= 0} -
            \prex{x \in \variety}{\genpoly^\prime(x) = 0}} \leq  \epsilon
        \]
    \end{claim}
    \begin{proof}
        Denote $S \definedas \basefield^{c^\prime}$, and for all $s \in S$, denote:
        \[
            p_1(s) \definedas \prex{a \in \field}{(\genpoly^{\prime}_1(a),...,\genpoly^{\prime}_{c^\prime}(a)) = s}
        \]
        As of our choice of $\rankfunc$, we have $\rank{\genpolyset^{\prime}} \geq \rankfunc_{\ref{high-rank-implies-low-bias}} \parens {\basefield, \degree, \dfrac{\epsilon/2}{\abs{\basefield}^{c^\prime}}}$.
        By combining Lemma~{\ref{high-rank-implies-low-bias}} with Lemma~{\ref{every-linear-combination-has-low-bias-implies-equidistribution}},
        we have that $p_1$ is ($\epsilon/2\abs{S}$)-equidistributed, i.e:
        \[
            p_1(s) = \dfrac{1 \pm \epsilon/2}{\abs{S}}
        \]
        Similarly, denote:
        \[
            p_2(s) \definedas \prex{x \in \variety}{(\genpoly^{\prime}_1(x),...,\genpoly^{\prime}_{c^\prime}(x)) = s}
        \]
        As of our choice of $\rankfunc$, we have $\relrank{\variety}{\genpolyset^\prime} \geq \rankbiasfunc(\epsilon / 2\abs{S})$.
        Now, we wish to use the relative rank-bias relation with Lemma~\ref{every-linear-combination-has-low-bias-implies-equidistribution}
        to conclude similarly that $p_2$ is ($\epsilon/2\abs{S}$)-equidistributed, i.e:
        \[
            p_2(s) = \dfrac{1 \pm \epsilon/2}{\abs{S}}
        \]
        However, in order to so, we must first ensure that $(\epsilon/2\abs{S}) \geq \epsilonlimitedrankbias$.
        This is done by choosing a correct $\epsilon_1$, and formulated in the following claim:
        \begin{claim}
            One can choose $\epsilon_1 \definedas \epsilon_1(\basefield, \degree, \rankbiasfunc, \epsilonlimitedrankbias)$ such that if $\epsilon \geq \epsilon_1$ we have that $\epsilon/2\abs{S} \geq \epsilon_1$.
        \end{claim}
        \begin{proof}
            We need that:
            \[
                \dfrac{\epsilon}{2 \abs{\basefield}^{c^\prime}} \geq \epsilonlimitedrankbias
            \]
            As $c^\prime \leq C^{\ref{theorem:regularization-in-X}}_{\rankfunc, \degree}(1)$,
            for the term above to hold it is enough that the following will be true:
            \[
                \epsilon \geq \epsilonlimitedrankbias \cdot 2 \abs{\basefield}^{C^{\ref{theorem:regularization-in-X}}_{\rankfunc, \degree}(1)}
            \]
            and as $\rankfunc$ and thus also $C^{\ref{theorem:regularization-in-X}}_{\rankfunc, \degree}(1)$ are independent of $\blocklength$,
            we can pick $\epsilon_1 = \epsilon_1(\basefield, \degree, \rankbiasfunc, \epsilonlimitedrankbias)$ and get what we aimed for.
        \end{proof}
        Now, under that assumption of $\epsilon_1$ written above, we have that $p_2$ is ($\epsilon/2\abs{S}$)-equidistributed.
        This allows us to use the similar distributions of $\genpolyset^\prime$ in $\field$ and in $\variety$ to conclude
        that $\genpoly^\prime$ behaves similar in $\field$ and in $\variety$:
        \begin{flalign*}
            \prex{a \in \field}{\genpoly^\prime(a)= 0}
            &=\sum_{s \in S} {p_1(s) \cdot \existfunc{\Gamma(s) = 0}} \\
            &=\sum_{s \in S} {p_2(s) \cdot \existfunc{\Gamma(s) = 0}} \pm \epsilon \\
            &=\prex{x \in \variety}{\genpoly^\prime(x)= 0} \pm \epsilon
        \end{flalign*}
        which concludes the proof of the claim.
    \end{proof}

    Finally, as $\restrictfunc{\genpoly^\prime}{\variety} \equiv \onvarpoly$, we have that $\prex{x \in \variety}{\genpoly^\prime(x) = 0} = \prex{x \in \variety}{\onvarpoly(x) = 0}$.
    Thus, the claim above combining with~\eqref{eq:polynomials-in-fn-are-bounded-away-from-zero-with-high-probability}
    shows that the probability we wished to bound is bounded as we aimed for:
    \[
        \prex{x \in \variety}{\onvarpoly(x) = 0} \leq 1 - \normalizedcodedistance{\basefield}{\degree} + \epsilon
    \]
    This concludes the proof of the theorem.
\end{proof}
\begin{remark}
    Under the same conditions,
    the distance of Reed-Muller codes in $\variety$ is also bounded \emph{from above} by the distance of Reed-Muller codes in $\field$,
    and we have:
    \[
        \normalizedcodedistanceex{\basefield}{\variety}{\degree} \leq \normalizedcodedistance{\basefield}{\degree} + \epsilon_1
    \]
    \begin{proof}
        Let $\funcdef{\genfunc}{\field}{\basefield}$ be the polynomial in $\field$ with the \emph{smallest} distance from $0$ as possible, that is $\normalizedcodedistance{\basefield}{\degree}$.
        Denote $\onvarpoly \definedas \restrictfunc{\genpoly}{\variety}$.
        Note that $\onvarpoly$ is a polynomial in $\variety$.
        Now repeat the proof using these two polynomials, and by Claim~\ref{claim:p-and-P-have-the-same-approximation}, we have that a random input of $\genpoly$ yields $0$
        (approximately) the same as a random input of $\onvarpoly$ yields $0$.
        Thus as we have $\prex{x \in \field}{\genpoly(x) = 0} = 1 - \normalizedcodedistance{\basefield}{\degree}$
        we also get:
        \[
            \prex{x \in \variety}{\onvarpoly(x) = 0} \geq 1 - \normalizedcodedistance{\basefield}{\degree} - \epsilon_1
        \]
        This bounds \emph{from above} the distance of Reed-Muller code in $\variety$ and we have:
        \[
            \normalizedcodedistanceex{\basefield}{\variety}{\degree} \leq \normalizedcodedistance{\basefield}{\degree} + \epsilon_1
        \]
    \end{proof}
\end{remark}
\begin{corollary}
    If we assume $\variety$ has the limited-relative rank-bias property to \emph{any extent} (or just the relative rank-bias property),
    then the theorem above proves an exact equality $\normalizedcodedistanceex{\basefield}{\variety}{\degree} = \normalizedcodedistance{\basefield}{\degree}$.
\end{corollary}
