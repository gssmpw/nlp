\section[Regularization Relative to \titlevariety]{Regularization Relative to \titlevariety}\label{sec:regularization-relative-to-X}

In this section, we generalize the definitions and statements regarding factors and regularization in $\field$,
to their corresponding definitions and statements to relative rank in respect of $\variety \subseteq \field$.
\newline
Note that in oppose to the previous chapter that we discussed a general $U$ and $A \subseteq U$,
in this chapter we discuss only $U = A = \field$.
This is done for clearance and to avoid defining definitions we will not use in our main proof.
\begin{definition}[Measurable Relative to $\variety$]
    Let $\genfuncset = \set{\genfunc_1,...,\genfunc_c}$ be a set of functions $\funcdef{\genfunc_i}{\field}{\basefield}$.
    We say a function $\funcdef{\genfunc[2]}{\field}{\basefield}$ is \emph{measurable in respect of $\genfuncset$ relative to $\variety$},
    or \emph{$\variety$-relative $\genfuncset$-measurable},
    if there exists a function $\funcdef{\relativeremainder{\genfunc[2]}}{\field}{\basefield}$ with $\restrictfunc{\relativeremainder{\genfunc[2]}}{\variety} \equiv 0$
    and a function $\funcdef{\Gamma}{\basefield^c}{\basefield}$ such that:
    \[
        \forall a \in A: \genfunc[2](a) = \Gamma(\genfunc_1(a),...,\genfunc_c(a)) + \relativeremainder{\genfunc[2]}(a)
    \]
    And:
    \[
        \deg(\genfunc[2] - \relativeremainder{\genfunc[2]}) \leq \deg(\genfunc[2])
    \]
    We sometimes refer to $\Gamma$ as the \emph{$\variety$-relative measurement function}.
\end{definition}

\begin{note*}
    Note that if $\deg(\genfunc[2] - \relativeremainder{\genfunc[2]}) \leq \deg(\genfunc[2])$ as discussed above,
    then the same bound  also bounds the degree of the remainder, i.e. $\deg(\relativeremainder{\genfunc[2]}) \leq \deg(\genfunc[2])$.
    Therefore $\relativeremainder{\genfunc[2]}$ is a valid $\variety$-remainder of $\genfunc[2]$.
    Moreover, this requirement is equivalent to the definition above,
    as if $\deg(\relativeremainder{\genfunc[2]}) \leq \deg(\genfunc[2])$, then we also have $\deg(\genfunc[2] - \relativeremainder{\genfunc[2]}) \leq \deg(\genfunc[2])$.
\end{note*}
\begin{note*}
    Also note that without the bound on the degree of the remainder,
    being measurable relative to $\variety$ is in fact equivalent for being a measurable in $A = \variety$.
    This is true because under these conditions, the remainder $\relativeremainder{\genfunc[2]}$ has no constraints but $\restrictfunc{\relativeremainder{\genfunc[2]}}{\variety} \equiv 0$,
    thus the condition left on the measurement is just being a measurement to $\genfunc[2]$ in $\variety$.
\end{note*}
\begin{remark}
    If $\genfunc[2]$ is a function that it is $\genfuncset$-measurable relative to $\variety$,
    then every value of $\genfunc[2]$ can be determined by the values of $\genfunc_1,...,\genfunc_c$ up to a remainder $\relativeremainder{\genfunc[2]}$ of degree $\leq \degree$.
    Thus, perhaps we do not know that the function $\genfunc[2]$ is constant inside every atom of $\genfuncset$ as in a regular semantic refinement,
    but we do know that there exists a function $(\genfunc[2] - \relativeremainder{\genfunc[2]})$ that equals to $\genfunc[2]$ on $\variety$, is constant on every atom of $\genfuncset$ and it is a function with a bounded degree i.e. $\deg(\genfunc[2] - \relativeremainder{\genfunc[2]})\leq \deg(\genfunc[2])$.
\end{remark}

Next, we present a new type of refinement, which is a relaxation of semantic refinement.
This relaxation will allow us to discuss the corresponding claim of the polynomial regularity lemma (Lemma~\ref{regularization-in-Fn-lemma}) for relative rank (instead of rank).
\begin{definition}[Semantic Refinement Relative to $\variety$]
    Let $\factor$ and $\factor^\prime$ be polynomial factors on $\field$, defined by sets of polynomials $\genpolyset, \genpolyset^{\prime}$ respectively,
    and let $\degree \in \naturalnumbersset$.
    We say a factor $\factor^\prime$ is a \emph{semantic refinement relative to $\variety$} of the factor $\factor$,
    or \emph{$\variety$-relative semantic refinement},
    if the following holds:
    Every function $\funcdef{\genfunc}{\field}{\basefield}$ that is $\genpolyset$-measurable relative to $\variety$,
    is also $\genpolyset^{\prime}$-measurable relative to $\variety$.
    If the definition above holds, we denote $\factor^{\prime} \relsemrefineex{\variety} \factor$.
\end{definition}
\begin{note*}
    It is easy to see that this relation is transitive, i.e. if
    $\factor^{\prime} \relsemrefineex{\variety} \factor$ and
    $\factor^{\prime\prime} \relsemrefineex{\variety} \factor^{\prime}$,
    then $\factor^{\prime\prime} \relsemrefineex{\variety} \factor$.
\end{note*}
\begin{remark}
    In $\variety$, semantic refinements relative to $\variety$ behave the same as regular semantic refinements in the perspective of being measurable:
    every function that is $\genpolyset$-measurable in $\variety$ is also $\genpolyset^{\prime}$-measurable in $\variety$.
    However, the two definitions behave differently in the perspective of being measurable in $\field$.
    Specifically, in relative semantic refinements,
    if $\genfunc[2]$ is a $\genpolyset$-measurable function it is not necessarily $\genpolyset^{\prime}$-measurable.
    However, it is measurable up to a remainder $\relativeremainder{\genfunc[2]}$ of degree $\leq \deg(\genfunc[2])$ such that $\restrictfunc{\relativeremainder{\genfunc[2]}}{\variety} \equiv 0$.
\end{remark}
\begin{corollary}\label{relative-semantic-refinement-is-restricted-semantic-refinement}
    If $\factor^{\prime} \relsemrefineex{\variety} \factor$, then in $\variety$ it is a regular semantic refinement, i.e. $\factor^{\prime} \semrefineex{\variety} \factor$.
\end{corollary}

Next, we present a new regularization process that allows us to increase the \emph{relative} rank of a factor without increasing the size of the factor too much (independent of $\blocklength$).
This regularization process generalizes the regularization process in $\field$, which was first presented by~\cite[2.3]{green2007distribution}.
We call this type of regularization process a \emph{relative-regularization process} relative to $\variety$, shorthand by $\variety$-regularization
For a specific function $\rankfunc$, we will sometimes call applying this lemma a \emph{$\rankfunc$-$\variety$-regularization}.
Note that to allow such a relative-regularization process to hold, we must use the relaxed definition of semantic refinement that is presented above.
\begin{theorem}\label{theorem:regularization-in-X}
    Let $\funcdef{\rankfunc}{\naturalnumbersset}{\naturalnumbersset}$ be a non-decreasing function and let $\degree \in \naturalnumbersset$.
    There exists $\funcdef{C_{\rankfunc, \degree}^{\ref{theorem:regularization-in-X}}}{\naturalnumbersset}{\naturalnumbersset}$ such that the following holds:
    Let $\factor$ be a factor defined by polynomials $\genpolyset = (\genpoly_1,...,\genpoly_c)$ where for all $i \in [c]$: $\funcdef{\genpoly_i}{\field}{\basefield}$ and $\deg(\genpoly_i) \leq \degree$.
    Then, there is an $\rankfunc$-$\variety$-regular factor $\factor^\prime$ defined by polynomials $\genpolyset^{\prime} = (\genpoly^{\prime}_{1},...,\genpoly^{\prime}_{c^\prime})$ where
    for all $i \in [c]$: $\funcdef{\genpoly^{\prime}_i}{\field}{\basefield}$ and $\deg(\genpoly^{\prime]}_{i}) \leq \degree$ such that
    $\factor^\prime \relsemrefineex{\variety} \factor$ and $c^\prime \leq C_{\rankfunc, \degree}^{\ref{theorem:regularization-in-X}}(c)$.
    \newline
    Moreover, if $\factor \synrefine \bar{\factor}$ for some polynomial factor $\bar{\factor}$ with
    relative rank of at least $\rankfunc(c^\prime)+c^\prime+1$ and rank of at least ${\rankfunc_{\ref{preserving-degree-starting-field}}(\basefield, \degree, c^{\prime})} + c^\prime + 1$,
    then we can require that $\factor^\prime \synrefine \bar{\factor}$.
\end{theorem}
\begin{proof}
    We follow the lines of the proof given by~\cite{book}[Lemma 7.29], but here, we wish to increase the \emph{relative} rank of the factor instead of its rank.
    We present an iterative process, which will eventually lead us to a factor of size $c^{\prime}$ with relative rank higher than $\rankfunc(c^\prime)$,
    that is a semantic refinement relative to $\variety$.
    Let $\degree \in \naturalnumbersset$,
    and let $\factor$ be a polynomial factor defined by $\genpolyset = (\genpoly_1,...,\genpoly_c)$ such that $\funcdef{\genpoly_i}{\field}{\basefield}$ of degree $\leq \degree$.
    Define $M(\factor) \definedas (M_{\degree},...,M_1) \in \naturalnumbersset^{\degree}$,
    where $M_i$ denotes the number of polynomials in $\genpolyset$ that have degree exactly $i$.
    Thus, $\sum_{i=1}^{\degree}M_i = c$.
    We define the lexicographical order on $\naturalnumbersset^{\degree}$ where $M > M^{\prime}$ if and only if $M_i > M^{\prime}_i$ for some $1 \leq i \leq \degree$,
    and $M_j = M^{\prime}_j$ for all $j > i$.
    This proof will be by transfinite induction on $M$ under the lexicographical order.
    Next we describe a step of the regularization process.
    \newline
    Let $\factor$ be a polynomial factor defined by $\genpolyset = (\genpoly_1,...,\genpoly_c)$.
    Note that this is an abuse of notations: the factor $\factor$ and the set $\genpolyset$ refer to the original factor in the first step, and also to the current factor in the middle of the relative-regularization process.
    If $\factor$ is $\rankfunc$-$\variety$-regular, then we are done.
    Otherwise, we change $\factor$ as follows:
    First, we denote $\rankfunc_{\ref{preserving-degree-starting-field}}^{\basefield, \degree}(c) \definedas \rankfunc_{\ref{preserving-degree-starting-field}}(\basefield, \degree, c)$,
    and we $\rankfunc_{\ref{preserving-degree-starting-field}}$-regularize $\genpolyset$ using lemma~\ref{regularization-in-Fn-lemma}
    to get a set of polynomials $\genpolyset_1 = (\genpoly^1_1,...,\genpoly^1_{c_1})$ of degree $\leq \degree$,
    which defines a factor $\factor_1$ and has a rank $\geq \rankfunc_{\ref{preserving-degree-starting-field}}^{\basefield, \degree}(c_1)$.
    %TODO: Do I need to prove the regular regularization in order to show that?
    Note that $M(\factor_1) \leq M(\factor)$.
    Then, again, if somehow $\factor_1$ is now $\rankfunc$-$\variety$-regular, we are done.
    \newline
    Otherwise, by definition, there exists some linear combination of the polynomials in $\genpolyset_{1}$ that
    has $\degree^\star$-relative rank less than $\rankfunc(c_1)$,
    where $\degree^\star$ is the maximal degree that participates in the linear combination.
    Let $\vec{\genpoly}(x) = \sum_{i=0}^{c_1}{\lambda_i \genpoly^{1}_i(x)}$ where $\vec{0} \neq \vec{\lambda}\in \basefield^{c_1}$,
    be the linear combination with $\drelrank{\degree^\star}{\variety}{\vec{\genpoly}} \leq \rankfunc(c_1)$ where $\degree^\star \definedas \max_{i \in \sparens{c_1}}{\deg(\lambda_i \genpoly^1_i)}$.
    By definition of relative rank, there exists $\relativeremainder{\genpoly} \in \allpolyset{\leq \deg(\vec{\genpoly})}{\field}{\basefield}$ with $\restrictfunc{\relativeremainder{\genpoly}}{\variety} \equiv 0$ such that
    $\drank{\degree^\star}{\vec{\genpoly} - \relativeremainder{\genpoly}} \leq \rankfunc(c_1)$.
    Note that $\deg(\relativeremainder{\genpoly}) \leq \degree^\star$.
    By definition of $\degree^\star$-rank, we have that we can decompose $\vec{\genpoly} - \relativeremainder{\genpoly}$ as a function of $\rankfunc(c_1)$ polynomials of degree $\leq \degree^\star - 1$.
    In other words, there exist a measurement function $\funcdef{\vec{\Gamma}}{\basefield^{\rankfunc(c_1)}}{\basefield}$ and polynomials $\genpoly[2]_1,...,\genpoly[2]_{\rankfunc(c_1)}$
    with $\deg(\genpoly[2]_i) \leq \degree^\star - 1$ such that:
    \[
        \forall a \in \field: \vec{\genpoly}(a) - \relativeremainder{\genpoly}(a) = \vec{\Gamma} \parens {\genpoly[2]_1(a),...,\genpoly[2]_{\rankfunc(c_1)}(a)}
    \]
    Now, let $\genpolyset^{\star} \subseteq \genpolyset_1$ be the set of all such maximal-degree polynomials,
    and let $i^{\star}$ be chosen such that $\genpoly^{1}_{i^\star} \in \genpolyset^{\star}$.
    Note that the set $\genpolyset^\star$ is non empty, as by definition, $\degree^{\star}$ is the maximal degree of polynomial in the expression $\sum_{i=1}^{c_1}{\lambda_{i} \genpoly^{1}_i}$ such that $\lambda_i \neq 0$.
    \newline
    For the next step, define the polynomial factor $\factor_2$ be the polynomial factor defined by the set:
    \[
        \genpolyset_2 \definedas \genpolyset_1 \setminus \set{\genpoly^1_{i^\star}} \cup \set{\genpoly[2]_1,...,\genpoly[2]_{\rankfunc(c_1)}}
    \]
    Finally, the factor $\factor_2$ will be the factor returned from the relative-regularization step.
    \newline
    It is easy to see that if the process above halts, we get a $\rankfunc$-$\variety$-regular factor.
    Now, we prove the first part of the lemma by showing the following claims:
    \begin{claim}
        The factor generated from the regularization above is of bounded size: a bound that may depend on $\rankfunc, \degree, c$, but does not depend on $\blocklength$.
        Formally, we claim that there exists $\funcdef{C^{\ref{theorem:regularization-in-X}}_{\rankfunc, \degree}}{\naturalnumbersset}{\naturalnumbersset}$
        such that we have $c^{\prime} \leq C^{\ref{theorem:regularization-in-X}}_{\rankfunc, \degree}(c)$.
    \end{claim}
    \begin{proof}
        It is enough to prove the following:
        \begin{enumerate}
            \item~\label{relative-regularization-step-factor-size-is-bounded}
            In each step, the amount of polynomials there are in $\genpolyset_1, \genpolyset_2$ are bounded by a bound that depend only on $\rankfunc, \degree, c$ (independent of $\blocklength$).
            \item~\label{relative-regularization-amount-of-steps-is-bounded}
            The number of steps of the relative-regularization process is also bounded by a bound that depends only on $\rankfunc, \degree, c$ (independent of $\blocklength$).
        \end{enumerate}
        The combination of these two will obtain the desired bound of the amount of polynomials in the last-step regularized factor, which is $C^{\ref{theorem:regularization-in-X}}_{\rankfunc, \degree}(c)$.
        Note that the bound on the last-step relative-regularized factor in not simply the multiplication of the two bounds,
        but a recursively-substitution of the bound in~\ref{relative-regularization-step-factor-size-is-bounded},
        a bounded amount of times (bounded by the bound in~\ref{relative-regularization-amount-of-steps-is-bounded}).
        \newline
        For~\ref{relative-regularization-step-factor-size-is-bounded}, we first notice that the number of polynomials in the regular regularization process is bounded,
        specifically we have $\abs{\genpolyset^1} = c_{1} \leq C^{\ref{regularization-in-Fn-lemma}}_{\rankfunc_{\ref{preserving-degree-starting-field}}, \degree}(c)$.
        Moreover, the polynomial factor $\factor_2$ is generated by adding at most $\rankfunc(c_1)$ polynomials to the factor, and thus we have $\abs{\genpolyset_2} \leq c_1 + \rankfunc(c_1)$ which is also bounded by substituting the bound on $c_1$.
        \newline
        For~\ref{relative-regularization-amount-of-steps-is-bounded}, we use the transfinite induction on $M$ we mentioned earlier to show that the process must halt after a bounded number of steps.
        Formally, we show that there exist $M^{\prime}$ which depends only on $M(\factor)$ such that $M(\factor_2) \leq M^{\prime} < M(\factor)$.
        This will bound the number of steps by a value that depend only on $M(\factor)$, which depends only on $\rankfunc, \degree, c$.
        To do so, we first notice that the regular regularization does not increase the value of $M$, i.e. $M(\factor_1) \leq M(\factor)$.
        Thus, we can focus on the second part of the relative-regularization.
        In this part, we replace a single degree $\degree^{\star}$ polynomial by at most $\rankfunc(c_1)$ polynomials of degree $\leq \degree^{\star} - 1$.
        Therefore, by choosing $M^{\prime} \definedas (M_{\degree}, ..., M_{\degree^{\star}+1}, M_{\degree^{\star}}-1, M_{\degree^{\star}-1}+\rankfunc(c_1),...,M_1+\rankfunc(c_1))$
        we get that $M(\factor_2) \leq M^{\prime} < M(\factor_1) \leq M(\factor)$, which concludes~\ref{relative-regularization-amount-of-steps-is-bounded}.
    \end{proof}
    \begin{claim}
        The factor generated from the regularization above is a $\variety$-relative semantic refinement of the original factor, i.e $\factor^\prime \relsemrefineex{\variety} \factor$.
    \end{claim}
    \begin{proof}
        It is enough to show that in each step, the factors generated by the relative-regularization process are semantic refinements relative to $\variety$ of the previous step's factor.
        Specifically, we show $\factor_{2} \relsemrefineex{\variety} \factor_{1} \relsemrefineex{\variety} \factor$ and the claim will follow from transitivity of relative semantic refinements.
        \newline
        We start by proving $\factor_1 \relsemrefineex{\variety} \factor$.
        Let $\funcdef{\genfunc}{\field}{\basefield}$ be a function that is $\genpolyset$-measurable relative to $\variety$.
        We denote $\degree_{\genfunc} \definedas \deg(\genfunc)$.
        By definition, there exists $\funcdef{\Gamma}{\basefield^c}{\basefield}$,
        $\funcdef{\relativeremainder{\genfunc}}{\field}{\basefield}$ where $\deg(\relativeremainder{\genfunc}), \deg(\genfunc - \relativeremainder{\genfunc}) \leq \degree_{\genfunc}$ and $\restrictfunc{\relativeremainder{\genfunc}}{\variety} \equiv 0$, such that:
        \[
            \forall a \in \field: \genfunc(a) = \Gamma(\genpoly_1(a),...,\genpoly_c(a)) + \relativeremainder{\genfunc}(a)
        \]
        Clearly, the function $\Gamma(\genpoly_1(a),...,\genpoly_c(a))$ is $\genpolyset$-measurable in $\field$,
        and because we have $\factor \semrefine \factor_1$, it is also $\genpolyset_1$-measurable in $\field$.
        Thus there exists $\funcdef{\Gamma_1}{\basefield^{c_1}}{\basefield}$ such that:
        \[
            \forall a \in \field: \genfunc(a) = \Gamma_1(\genpoly^{1}_{1}(a),...,\genpoly^{1}_{c_1}(a)) + \relativeremainder{\genfunc}(a)
        \]
        And therefore we have $\factor_1 \relsemrefineex{\variety} \factor$.
        \newline
        Now, we prove $\factor_2 \relsemrefineex{\variety} \factor_1$.
        Let $\funcdef{\genfunc}{\field}{\basefield}$ be a function that is $\genpolyset_1$-measurable relative to $\variety$.
        Again, we denote $\degree_{\genfunc} \definedas \deg(\genfunc)$,
        and by definition there exists $\funcdef{\Gamma_1}{\basefield^c}{\basefield}$,
        $\funcdef{\relativeremainder{\genfunc_1}}{\field}{\basefield}$ where $\deg(\genfunc - \relativeremainder{\genfunc_1}) \leq \degree_{\genfunc}$ and $\restrictfunc{\relativeremainder{\genfunc_1}}{\variety} \equiv 0$, such that:
        \begin{equation} \label{eq:f-decomposition-a}
            \forall a \in \field: \genfunc(a) = \Gamma_1(\genpoly^{1}_{1}(a),...,\genpoly^{1}_{c_1}(a)) + \relativeremainder{\genfunc}_1(a)
        \end{equation}
        Note that we also have $\deg(\relativeremainder{\genfunc_1}) \leq \degree_{\genfunc}$.
        We will refer this equation, and its simplifications we do throughout the proof, as \emph{the $\genpolyset_1$-decomposition of $\genfunc$}.
        \newline
        We wish to show that there exists $\funcdef{\Gamma_2}{\basefield^{c_2}}{\basefield}$ and
        $\funcdef{\relativeremainder{\genfunc}_2}{\field}{\basefield}$ where $\deg(\genfunc - \relativeremainder{\genfunc}_2) \leq \degree_{\genfunc}$ and $\restrictfunc{\relativeremainder{\genfunc}_2}{\variety} \equiv 0$, such that:
        \[
            \forall a \in \field: \genfunc(a) = \Gamma_2 \parens {\genpoly^1_1(a),...\genpoly^1_{i^\star - 1}(a), \genpoly^1_{i^\star + 1}(a),..., \genpoly^1_c(a), \genpoly[2]_1(a),...,\genpoly[2]_{\rankfunc(c_1)}(a)} + \relativeremainder{\genfunc}_2(a)
        \]
        We will do so using the $\genpolyset_1$-decomposition of $\genfunc$.
        Note that showing $\deg(\genfunc - \relativeremainder{\genfunc}) \leq \degree_{\genfunc}$ is equivalent of showing $\deg(\relativeremainder{\genfunc}_2) \leq \degree_{\genfunc}$.
        \newline
        First, by the way we built $\genpolyset_2$, using the same notations in the regularization step, we have:
        \[
            \forall a \in \field: \genpoly_{i^\star}^1(a) =
                \vec{\Gamma} \parens {\genpoly[2]_1(a),...,\genpoly[2]_{\rankfunc(c_1)}(a)}
                + \relativeremainder{\genpoly}(a)
                - \sum_{i \neq i^\star}{\genpoly_{i}^1(a)}
        \]
        Next, we substitute the value of $\genpoly_{i^\star}^1$ in the $\genpolyset_1$-decomposition of $\genfunc$ (\ref{eq:f-decomposition-a}),
        and get another decomposition of $\genfunc$ that does not depend on $\genpoly_{i^\star}^1$.
        Specifically we have:
        \begin{align} \label{eq:f-decomposition-b}
            \forall a \in \field: \genfunc(a) &=
        \Gamma_1 \parens
            {\genpoly^{1}_{1}(a)
                ,...,
                \parens{
                    \vec{\Gamma} \parens {\genpoly[2]_1(a),...,\genpoly[2]_{\rankfunc(c_1)}(a)}
                    + \relativeremainder{\genpoly}(a)
                    - \sum_{i \neq i^\star}{\genpoly_{i}^1(a)}}
                ,...,
                \genpoly^{1}_{c_1}(a))}\\
            &+ \relativeremainder{\genfunc}_1(a)
        \end{align}
        We wish to use the equation above to show that $\genfunc$ is $\genpolyset_2$-measurable relative to $\variety$.
        However, in order to show that the equation above is in the desired structure that proves that $\genfunc$ is $\genpolyset_2$-measurable,
        the expression inside $\Gamma_1$ must not depend on $\relativeremainder{\genpoly}$ because $\relativeremainder{\genpoly} \notin \genpolyset_2$.
        Note that this is enough as the rest of the polynomials in the expression above are in $\genpolyset_2$,
        and therefore without $\relativeremainder{\genpoly}$ the expression is $\genpolyset_2$-measurable.
        \newline
        To do so, we start by simplifying some of the notations.
        We denote:
        \[
            \vec{\genpoly}_2(a) \definedas \vec{\Gamma} \parens {\genpoly[2]_1(a),...,\genpoly[2]_{\rankfunc(c_1)}(a)} - \sum_{i \neq i^\star}{\genpoly_{i}^1(a)}
        \]
        This is the part of the sum that decomposes $\genpoly_{i^\star}^1(a)$ that is $\genpolyset_2$-measurable,
        thus the following equality applies:
        \[
            \genpoly_{i^\star}^1(a) = \vec{\genpoly}_2(a) + \relativeremainder{\genpoly}(a)
        \]
        where $\deg(\vec{\genpoly_2}), \deg(\relativeremainder{\genpoly}) \leq \degree^{\star}$.
        Using this notation, we write the $\genpolyset_1$-decomposition of $\genfunc$ (\ref{eq:f-decomposition-b}), and get:
        \begin{equation} \label{eq:f-decomposition-c}
            \forall a \in \field: \genfunc(a) = \Gamma_1 \parens
            {\genpoly^{1}_{1}(a)
                ,...,
                \parens{
                    \vec{\genpoly}_2(a)
                    - \relativeremainder{\genpoly}(a)}
                ,...,
                \genpoly^{1}_{c_1}(a))}
            + \relativeremainder{\genfunc}_1(a)
        \end{equation}
        Now, we use the following key observation:
        $\rank{\genpolyset_1} \geq \rankfunc_{\ref{preserving-degree-starting-field}}(\basefield, \degree, c_1)$,
        and as $\deg(\Gamma_1(\genpoly_1^1,...,\genpoly^1_{c_1})) \leq \degree_{\genfunc}$
        we can use Lemma~\ref{preserving-degree-starting-field} to achieve that $\Gamma_1$ is a polynomial of the form:
        \begin{equation*} \label{eq:regularization-gamma-is-a-polynomial}
            \Gamma_1(z_1,...,z_{c_1})  =
            \sum_{\alpha \in \sparens{\basefieldsize - 1}^{c_1}}
            {C_{\alpha} \cdot {\prod_{i = 1}^{c_1}}{z_i^{\alpha_i}}}
        \tag{$\star$}
        \end{equation*}
        where $C_{\alpha} = 0$ whenever $\sum_{i = 1}^{c_1}(\alpha_i \cdot \deg(\genpoly_i^1)) > \degree_{\genfunc}$.
        \newline
        Next, we substitute the polynomial structure of $\Gamma_1$ \eqref{eq:regularization-gamma-is-a-polynomial}
        in the $\genpolyset_1$-decomposition of $\genfunc$~\eqref{eq:f-decomposition-c},
        and observe what happens to each summand monomial with non-zero coefficients of $\Gamma_1$ in the expression after the substitution.
        \newline
        We will show that each such monomial is either $\genpolyset_2$-measurable,
        or a sum of a $\genpolyset_2$-measurable function with a valid $\variety$-remainder, i.e. a polynomial of degree $\leq \degree_{\genfunc}$ that is $\equiv 0$ in $\variety$.
        Note that if this is true for each monomial,
        every linear combination of such monomials is also a sum of $\genpolyset_2$-measurable function with a valid $\variety$-remainder.
        Thus, this will also be true for the entire decomposition of $\genfunc$, as it is a linear combination of such monomials summed with a valid remainder $\relativeremainder{\genfunc}_1$.
        This will conclude the proof.
        \newline
        Let $\alpha = (\alpha_1,...,\alpha_{c_1})$ be a vector of degrees that represents such a monomial.
        If $\alpha_{i^\star} = 0$, then the monomial is in the form:
        \[
            \prod_{i \in [c_1]}{{\genpoly_i}^{\alpha_i}} =
            \prod_{i \in [c_1] \setminus \set{i^{\star}}}{{\genpoly_i}^{\alpha_i}}
        \]
        and therefore it is clearly $\genpolyset_2$-measurable as all the polynomials in the expression above are in $\genpolyset_2$.
        \newline
        Next, if $\alpha_{i^\star} \neq 0$, then the monomial is in the form:
       \begin{equation} \label{eq:monomial-of-gamma}
            \prod_{i \in [c_1]}{{\genpoly_i}^{\alpha_i}} =
            (\vec{\genpoly_2} + \relativeremainder{\genpoly})^{\alpha_{i^\star}} \cdot
                \parens{\prod_{i \in [c_1] \setminus \set{i^{\star}}}{{\genpoly_i}^{\alpha_i}}}
       \end{equation}
        where $\sum_{i \in [c_1]}(\alpha_i \cdot \deg(\genpoly_i^1)) \leq \degree_{\genfunc}$.
        As $\deg(\vec{\genpoly_2} + \relativeremainder{\genpoly}) = \deg(\genpoly_{i^\star}) =\degree^{\star}$, we have:
        \[
            \deg \parens{\prod_{i \in [c_1] \setminus \set{i^{\star}}}{{\genpoly_i}^{\alpha_i}}} =
                \sum_{i \in [c_1] \setminus {i^{\star}}}(\alpha_i \cdot \deg(\genpoly_i^1))
                \leq \degree_{\genfunc} - \alpha_{i^\star} \cdot \degree^{\star}
        \]
        Now, we open the left brackets in (\ref{eq:monomial-of-gamma}), i.e $(\vec{\genpoly_2} + \relativeremainder{\genpoly})^{\alpha_{i^\star}}$.
        This enables us to separate the monomial to the part that only depend on $\vec{\genpoly_2}$ summed with a polynomial with bounded degree multiplied by $\relativeremainder{\genpoly}$ (and therefore a valid remainder).
        To be more specific, the monomial is in the form:
        \[
            (\vec{\genpoly_2} + \relativeremainder{\genpoly})^{\alpha_{i^\star}} = \vec{\genpoly_2}^{\alpha_{i^\star}} + \relativeremainder{\genpoly_{\alpha}}
        \]
        for some polynomial $\relativeremainder{\genpoly_{\alpha}}$ such that:
        \begin{enumerate}
            \item $\relativeremainder{\genpoly_{\alpha}}$ is of degree
                    $\deg(\relativeremainder{\genpoly_{\alpha}}) \leq \max \set{\deg(\vec{\genpoly_2}), \deg(\relativeremainder{\genpoly)}} \cdot \alpha_{i^\star} \leq \alpha_{i^\star} \cdot \degree^{\star}$
            \item $\relativeremainder{\genpoly_{\alpha}}$ is a multiple of $\relativeremainder{\genpoly}$, and therefore $\restrictfunc{\relativeremainder{\genpoly_{\alpha}}}{\variety} \equiv 0$
        \end{enumerate}
        Therefore, by substituting the left brackets back to the equation (\ref{eq:monomial-of-gamma}) and as $\vec{\genpoly_2}$ and $\genpoly_i$ for $i \neq i^{\star}$ are $\genpolyset_2$-measurable,
        one can see that the monomial is a sum of a $\genpolyset_2$-measurable polynomial with a valid remainder.
        Specifically, the remainder $\equiv 0$ in $\variety$, and its degree is $\leq \alpha_{i^\star} \cdot \degree^{\star} + \degree_{\genfunc} - \alpha_{i^\star} \cdot \degree^{\star} = \degree_{\genfunc}$.
        This concludes the proof of the claim.
        \end{proof}
    Now, it remains to prove the second part of the Theorem~{\ref{theorem:regularization-in-X}}.
    \begin{claim}
        If $\factor \synrefine \bar{\factor}$ for some polynomial factor $\bar{\factor}$ with
        relative rank of at least $\rankfunc(c^\prime)+c^\prime+1$ and rank of at least ${\rankfunc_{\ref{preserving-degree-starting-field}}(\basefield, \degree, c^{\prime})} + c^\prime + 1$,
        then we can require that $\factor^\prime \synrefine \bar{\factor}$.
    \end{claim}
    \begin{proof}
        We will show claim step-by-step.
        We denote by $\genpolyset, \bar{\genpolyset}, \genpolyset_1, \genpolyset_2$ the polynomial sets that generate the factors $\factor, \bar{\factor}, \factor_1, \factor_2$.
        Note that $\factor_1, \factor_2$ are the factors in the current step of the regularization process, and thus change in each step of the proof.
        We show that in each step, if $\factor \synrefine \bar{\factor}$ for some polynomial factor $\bar{\factor}$ with
        relative rank of at least $\rankfunc(c^\prime)+c^\prime+1$ and rank of at least ${\rankfunc_{\ref{preserving-degree-starting-field}}(\basefield, \degree, c^{\prime})} + c^\prime + 1$,
        then we can require that $\factor_1 \synrefine \bar{\factor}$, and also that $\factor_2 \synrefine \bar{\factor}$.
        \newline
        For the first part, we have $\factor_1 \synrefine \bar{\factor}$ by a simple usage of the second part of lemma~\ref{regularization-in-Fn-lemma},
        as:
        \[
            \rank{\bar{\genpolyset}}
            > \rankfunc_{\ref{preserving-degree-starting-field}}(\basefield, \degree, c^{\prime}) + c^\prime + 1
            \geq \rankfunc_{\ref{preserving-degree-starting-field}}(\basefield, \degree, c_1) + c_1 + 1
        \]
        \newline
        Now we prove the second part.
        We show that in the current regularization step, we could replace $\genpoly^1_{i^\star} \in \genpolyset_1$ such that $\genpoly^1_{i^\star} \notin \bar{\genpolyset}$.
        Note that this is possible whenever $\genpolyset^\star\cap \bar{\genpolyset} \neq \emptyset$ as the choice of $i^\star$ is arbitrary in polynomials which are in $\genpolyset^\star$.
        \newline
        Assume that is not possible and the factor $\genpolyset_1$ is still not $\rankfunc$-$\variety$-regular.
        Then, we have a linear combination $\vec{\genpoly}(x) \definedas \sum_{i=0}^{c_1}{\lambda_{i}\genpoly^1_i(x)}$ with $\drelrank{\degree^\star}{\variety}{\vec{\genpoly}} \leq \rankfunc(c_1)$
        where $\degree^\star = \max_{i \in \sparens{c_1}} {\deg(\lambda_i \genpoly^1_i)}$.
        We denote by $I^{\star} \subseteq [c_1]$ the set of indexes of such maximal-degree polynomials.
        By this notation, our assumption states that for all $i \in I^{\star}$ we have $\genpoly^1_i \in \bar{\genpolyset}$.
        Additionally, note that for all $i \notin I^{\star}$ we have $\deg(\genpoly^1_i) < \degree^{\star}$.
        Therefore, as the linear combination is of $\degree^\star$-relative rank $\leq \rankfunc(c_1)$,
        there exists a polynomial $\relativeremainder{\genpoly}$ of degree $\leq \deg(\vec{\genpoly}) \leq \degree^\star$ with $\restrictfunc{\relativeremainder{\genpoly}}{\variety} \equiv 0$
        such that $\drank{\degree^\star}{\vec{\genpoly} - \relativeremainder{\genpoly}} \leq \rankfunc(c_1)$.
        In other words, there exist a measurement function $\funcdef{\vec{\Gamma}}{\basefield^{\rankfunc(c_1)}}{\basefield}$ and polynomials $\genpoly[2]_1,...,\genpoly[2]_{\rankfunc(c_1)}$
        with $\deg(\genpoly[2]_i) \leq \degree^\star$ such that:
        \[
            \forall a \in \field: \vec{\genpoly}(a) - \relativeremainder{\genpoly}(a) = \vec{\Gamma} \parens {\genpoly[2]_1(a),...,\genpoly[2]_{\rankfunc(c_1)}(a)}
        \]
        By a simple calculation we have:
        \[
            \forall a \in \field: \sum_{i \in I^{\star}}{\genpoly^1_i(a)} - \relativeremainder{\genpoly}(a) =  \vec{\Gamma} \parens {\genpoly[2]_1(a),...,\genpoly[2]_{\rankfunc(c_1)}(a)} +  \sum_{i \notin I^\star}{\genpoly^1_i(a)}
        \]
        and by this we found a linear combination of polynomials in $\bar{\genpolyset}$ with maximal degree $\degree^\star$,
        that has $\degree^\star$-relative-rank $\leq \rankfunc(c_1) + c_1 + 1$.
        This is a contradiction to our assumptions on $\bar{\factor}$, which completes the proof of the claim.
    \end{proof}
    This completes the proof of the lemma.
\end{proof}
