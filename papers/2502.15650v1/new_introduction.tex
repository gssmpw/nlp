\section{Introduction}\label{sec:introduction}
% We introduce a new question: the properties of the quotient code, and its connection to the original code
% Contribution: introduce a new definition
Let $\basefield$ be a finite field, $\blocklength \in \naturalnumbersset$, and let $\variety \subseteq \field$ be a subset
\footnote{As a convention, we use $\tilde{\square}$ to denote properties of the subset, and thus also the subset itself.}
.
We begin by introducing a new definition applicable to any linear code over $\basefield$: the \emph{$\variety$-quotient code}.
We then illustrate this novel definition using Reed-Muller codes, and present a property of $\variety$ which we use to show that $\variety$-quotient Reed-Muller code \emph{inherits its distance and list decoding radius} from the original Reed-Muller code.
Finally, leveraging known results from additive combinatorics and algebraic geometry, we establish as a corollary that this inheritance holds when $\variety$ is a \emph{high-rank variety}.

\paragraph{The Quotient Code}
Let $\gencode$ be a linear code over $\basefield$.
Each codeword of $\gencode$ can be described as a function $\funcdef{\genfunc}{\field}{\basefield}$ that is in the span of the columns of the code's \emph{generator matrix}.
An equivalent way to describe $\gencode$ is using a \emph{parity check matrix}, where a function $\genfunc$ is a codeword if and only if it satisfies the constraints represented by parity-check matrix.
Each such constraint can be thought of as a requirement over a few inputs of $\genfunc$ from $\field$: the requirement that their weighted sum will equal $0$.

The first novel definition we introduce is the definition of the \emph{$\variety$-induced} code:
\begin{definition}[The $\variety$-Induced Code]
    We define the \emph{$\variety$-induced code $\quotientcode$} to be
    the set of all functions $\funcdef{\onvarfunc}{\variety}{\basefield}$
    \footnote{By convention, we use uppercase letters to denote functions with domain $\field$ and lowercase letters to denote functions with domain $\variety$.}
    that satisfy all the constraints \emph{that lie in $\variety$}.
\end{definition}

Let us briefly describe the connection between codewords in $\field$ and $\variety$-induced codewords.
One can easily verify that each original codeword \emph{restricted} to $\variety$ is a valid codeword in the induced code.
\newline
We call an extension of an $\variety$-induced codeword $\funcdef{\onvarfunc}{\variety}{\basefield}$ to valid codeword in the original code (extending its domain to $\field$), a \emph{lift} of $\onvarfunc$.
When each induced codeword has a unique lift, there is a natural 1-to-1 correspondence between the original and induced codeword.
This becomes substantially more interesting for subsets $\variety$ in which induced codewords have \emph{multiple} lifts.
This non-uniqueness weakens the connection between the original codewords and induced codewords, and leads to a richer range of phenomena (and interesting new challenges).

We also note that the other direction is not always true: For a general subset $\variety$, there might be an induced codeword (a valid codeword in the induced code) that \emph{cannot be lifted} to a valid codeword in $\field$.
We are interested to better understand $\quotientcode$ using $\gencode$ and vice-versa, and therefore we introduce a new notion, which is the notion of the \emph{$\variety$-quotient code}:
\begin{definition}[The $\variety$-Quotient Code]
    Let $\gencode$ be a linear code, and let $\quotientcode$ be the $\variety$-induced code of $\gencode$.
    We say $\quotientcode$ is a \emph{$\variety$-quotient code}
    if every quotient codeword $\onvarfunc \in \quotientcode$ has a lift to $\field$.
\end{definition}
In the case described above, we also say that $\variety$ is a \emph{lift-enabler} for $\gencode$ and that the code $\gencode$ is a \emph{covering code} for the code $\quotientcode$.
\newline
The novelty of this definition is that it captures subsets in which \emph{there is} a correspondence between codewords in $\variety$ and in $\field$,
and the correspondence may be \emph{1-to-many}.

\paragraph{Importance of Definition}
This timely definition extends a fundamental and useful concept previously introduced for graphs and complexes—namely, the notion of a \emph{covering graph} or alternatively, the \emph{quotient graph}.
This concept gained an increasing prominence in theoretical computer science, where it was recently employed to construct \emph{high dimensional expanders}~\cite{dikstein2022newhighdimensionalexpanders, yaacov2024sparsehighdimensionalexpanders}
and achieve improved \emph{local testing} results~\cite{gotlib2022listagreementexpansioncoboundary, dikstein2024agreementtheoremshighdimensional, bafna2024characterizingdirectproducttesting},
where the latter also played a crucial role in constructions of PCPs.
Consequently, the study of covering spaces for graphs has found usages in theoretical computer science and specifically in development of PCPs with enhanced properties.
We believe our question, which explores the analogous question for codes, will similarly lead to meaningful applications in theoretical computer science.

In addition to that, the question of \emph{puncturing} of codes has caught much attention recently, in a line of work~\cite{brakensiek2024genericreedsolomoncodesachieve, alrabiah2024randomlypuncturedreedsolomoncodes, brakensiek2024generalizedgmmdspolynomialcodes, brakensiek2024agcodesachievelistdecoding},
followed by the resolution of the GM-MDS conjecture~\cite{DBLP:journals/corr/abs-1803-02523, DBLP:journals/corr/abs-1803-03752}.
Where the question of puncturing is focused exclusively on the case where the lift is \emph{unique},
the study of quotient codes also tackles subsets $\variety \subseteq \field$ where the lift is \emph{not unique}.
Notably, in the unique-lift case there are well-established lower-bounds for the size of $\variety$ such as~\cite[Theorem 1.1]{DBLP:journals/cc/DoronTT22}.
In contrast, the size of $\variety$ in quotient codes may be \emph{much smaller} than its lower-bound in punctured code (for example in Reed-Muller codes), suggesting the potential for new insights and improved results.

\paragraph{Our Question}
Our goal is to answer the following question:
what properties of $\variety$ will imply that the quotient code inherits its distance and list-decoding radius from the original code?

This question is analogous to the study of quotients of expander graphs—just as not all quotients of an expander necessarily preserve expansion,
not all subsets $\variety$ necessarily yield a well-behaved quotient code.
Understanding the conditions under which expansion is preserved has been a fundamental problem in the study of expanders,
and similarly, identifying the conditions under which a quotient code retains key properties of the original code is a central challenge in our work.
Given this parallel, we believe our question may have broader implications for future research in both coding theory and theoretical computer science.

We answer this question in the context of \emph{Reed-Muller codes}.
Notably, our approach does \emph{not only} address the case of where there are multiple lifts,
but also introduces a novel framework for analyzing unique-lift (puncturing) setting when the field size is constant-a scenario that is typically considered more challenging.

\paragraph{Reed-Muller Codes}
Let $\basefield$ be a finite field, and let $\blocklength, \degree$ be integers.
Each codeword in Reed-Muller code $\reedmullercodeex{\basefield}{\field}{\degree}$,
is defined by a polynomial over $\basefield$ in $\blocklength$ variables with total degree $\leq \degree$
\footnote{We focus on the regime where $\degree, \abs{\basefield}$ are considered constants and $\blocklength$ is considered very large.}
.
The message that one wishes to encode is represented in the code as a polynomial $\funcdef{\genpoly}{\field}{\basefield}$, whose coefficients are the different message characters.
The encoding of the message is a vector of the different evaluation of $\genpoly$ over \emph{all} possible points in $\field$.

Alternatively, one can describe Reed-Muller codes using a set of local constraints.
A function $\funcdef{\genfunc}{\field}{\basefield}$ is a polynomial of degree $\leq \degree$
if and only if the (alternating) sum of each possible \emph{cube}, which is a set of points of the form $\set{x + \sum_{i \in S} y_i}_{S \subseteq \sparens{\degree + 1}}$ for $x, y_1,...,y_{\degree+1} \in \field$, equals $0$.
The set of all cubes is \emph{the set of constraints of degree-$\degree$ polynomials}.

Next, we present our notations for the induced Reed-Muller code:
\begin{notation}[The $\variety$-Induced Reed-Muller Code]
    We say a function $\funcdef{\genfunc}{\variety}{\basefield}$ is a \emph{polynomial of degree $\leq \degree$ \emph{in $\variety$}}
    if it satisfies all the constraints of degree-$\degree$ polynomials \emph{that lie in $\variety$}.
    \newline
    We denote the $\variety$-induced Reed-Muller code:
    \[
        \reedmullercodeex{\basefield}{\variety}{\degree} = \set{\funcdef{\onvarpoly}{\variety}{\basefield} \suchthat \onvarpoly \text{ is a polynomial of degree } \leq \degree \text{ in } \variety}
    \]
\end{notation}

\paragraph{Properties of Induced Reed-Muller Codes}
A study of Ziegler and Kazhdan~\cite{kazhdan2018polynomial, kazhdan2019extendingweaklypolynomialfunctions, kazhdan2020propertieshighranksubvarieties}
shows that if $\variety$ is a \emph{high rank variety}
\footnote{Under some conditions we describe later.}
, then $\variety$ is a \emph{lift-enabler} for $\reedmullercodeex{\basefield}{\field}{\degree}$.
In other words, the authors showed that the $\variety$-induced Reed-Muller code is in fact a \emph{$\variety$-quotient Reed-Muller code}.
%More accurately, Ziegler and Kazhdan showed that every polynomial in a high-rank variety
%can be lifted to a polynomial in $\field$ with the same degree
%\footnote{This is a stronger variant of being a quotient code: every polynomial of degree $\degree^\prime \leq \degree$ in $\variety$ is liftable to a polynomial of \emph{the same degree} $\degree^\prime$ in $\field$.}
%.
%\newline
%We call a subset $\variety \subseteq \field$ that has this property for polynomials of degree $\leq \degree$ a \emph{$\degree$-lift-enabler},
We rely on this property of $\variety$ as a black-box.
See Section~\ref{sec:polynomials_in_X} for more details in this regard.

An additional property of $\variety \subseteq \field$ we rely on is the connection between \emph{algebraic structure} and \emph{random behavior (equidistribution)} of polynomials in $\variety$.
\newline
For $\field$, this connection is a well-studied result~\cite{green2007distribution, kaufman2008worst, DBLP:journals/corr/0001L15}.
%which is formally described by the relation of \emph{rank} and \emph{bias} accordingly.
%We clarify that rank is a measure of algebraic structure, where low rank implies being structured,
%and bias is a measure of lack of random behavior, where low bias implies being equidistributed.
It lies in the heart of many results in higher-order Fourier analysis,
and specifically was used in~\cite{bhowmick2014list} to analyze the list decoding radius of Reed-Muller code in $\field$.
\newline
The equivalent of this relation for subsets $\variety \subseteq \field$ was studied in~\cite{lampert2021relative, gowers2022equidistributionhighrankpolynomialsvariables}.
These works captured the measure of algebraic-structure in $\variety$ by a definition called \emph{relative rank},
and captured the lack of random behavior in $\variety$ by a definition called \emph{relative bias}.
We note that for subsets, the definition of algebraic structure of a polynomial in $\variety$ considers the algebraic structure of \emph{all its possible} lifts.
It was shown in~\cite{lampert2021relative} that when $\variety$ is a high-rank variety, high relative rank implies low relative bias
\footnote{
    Note that even though Gowers and Karam~\cite{gowers2022equidistributionhighrankpolynomialsvariables} also acheived a similar relation for a type of subsets,
    the definition of rank they used is slightly different than the standard definition of rank.
    While this difference may seem unharmful at first, it is, to our knowledge, does not allow to do a \emph{regularization} process
    (note that a generalization of this process is the heart of our proof).
}
.
\newline
We use this property as a black box as well.
When a subset $\variety \subseteq \field$ has such property for polynomials of degree $\leq \degree$, we say that it has the \emph{$\degree$-relative rank-bias property}.
See Section~\ref{sec:relative-rank-bias-property} for more details.

\paragraph{Our Results}
Next, let us present our main theorem more concretely.
Our work focuses on the regime where $\degree < \abs{\basefield}$ for prime finite fields $\basefield = \basefield_p$.
Throughout this paper, we always assume these two assumptions.
Denote the \emph{minimum normalized distance of $\reedmullercodeex{\basefield}{\field}{\degree}$} by $\normalizedcodedistanceex{\basefield}{\field}{\degree}$,
shorthand by $\normalizedcodedistance{\basefield}{\degree}$.
We have:
\[
    \normalizedcodedistance{\basefield}{\degree} = 1 - \degree/\abs{\basefield}
\]
Moreover, we define the \emph{list decoding count} of $\reedmullercodeex{\basefield}{\field}{\degree}$ by:
\[
    \listpolycount{\basefield}{\field}{\degree}{\tau} \definedas
    \max_{\funcdef{\genfunc}{\field}{\basefield}}
        {\abs{\set{\genpoly \in \allpolyset{\leq \degree}{\field}{\basefield} \suchthat {\dist{\genpoly, \genfunc} \leq \tau}}}}
\]
Let $\listdecodingradiusex{\basefield}{\field}{\degree}$ be the \emph{list decoding radius} of $\reedmullercodeex{\basefield}{\field}{\degree}$,
which is the maximum $\tau$ for which $\listpolycount{\basefield}{\field}{\degree}{\tau - \epsilon}$ is bounded by a \emph{constant} depending only on $\epsilon, \abs{\basefield}, \degree$.
\newline
In the paper~\cite{bhowmick2014list} it was shown that for constant field size and degree, the list decoding radius \emph{reaches the distance of the code}, as conjectured earlier by~\cite{10.1145/1374376.1374417}
\footnote{Note that it is known that $\listdecodingradiusex{\basefield}{\field}{\degree} \leq \normalizedcodedistance{\basefield}{\degree}$,
    and therefore, in a sense, their result is \emph{optimal in $\field$} assuming $\degree, \abs{\basefield}$ are considered as constants.}
.
We denote the corresponding distance parameter of $\variety \subseteq \field$ by $\normalizedcodedistanceex{\basefield}{\variety}{\degree}$ and $\listdecodingradiusex{\basefield}{\variety}{\degree}$ respectively.

We next present our main theorem, which establishes that the \emph{list decoding radius} of the quotient Reed-Muller code is \emph{at least as good} as the that of the original code:
\begin{theorem*}[List Decoding Quotient Reed-Muller Code]
\footnote{Informal, for formal see Theorem~\ref{thm:list-decoding-RM-in-X}.}
Let $\basefield$ be a finite (prime) field of constant size, let $\degree \in \naturalnumbersset$ be a constant such that $\degree < \abs{\basefield}$,
and let $\blocklength \in \naturalnumbersset$ be an integer.
\newline
Let $\variety \subseteq \field$ be a subset that is a lift-enabler for $\reedmullercodeex{\basefield}{\field}{\degree}$ and has the $\degree$-relative rank-bias property.
\newline
Then, $\reedmullercodeex{\basefield}{\variety}{\degree}$ inherits its \emph{list decoding radius} from $\reedmullercodeex{\basefield}{\field}{\degree}$, i.e:
\[
    \listdecodingradiusex{\basefield}{\variety}{\degree} \geq \listdecodingradiusex{\basefield}{\field}{\degree}
\]
\end{theorem*}

In addition, we also achieve a (simpler) result regarding the \emph{distance} of the quotient Reed-Muller code (Theorem~\ref{thm:distance-of-RM-in-X}):
Under the conditions described above,
$\reedmullercodeex{\basefield}{\variety}{\degree}$ also inherits its \emph{distance} from $\reedmullercodeex{\basefield}{\field}{\degree}$, i.e
$\normalizedcodedistanceex{\basefield}{\variety}{\degree} \geq \normalizedcodedistanceex{\basefield}{\field}{\degree}$
\footnote{Our techniques also show that also the other direction is true, which yields an \emph{equality} in the distance of the two codes.}
.

As a corollary, using results studied in~\cite{kazhdan2018polynomial, kazhdan2019extendingweaklypolynomialfunctions, lampert2021relative} regarding high-rank varieties, we obtain the following:
%TODO: Add this theorem formally in the end.
\begin{corollary*}[List Decoding Quotient Reed-Muller Code: High Rank Variety]
%TODO: [Informal. For formal see...]
    Let $\variety \subseteq \field$ be a \emph{high rank variety},
    that is, $\variety$ is the set of common zeros of a collection of polynomials $\varpolyset = (\varpoly_1,...,\varpoly_{\varietypolycount})$
    that is of \emph{high rank}
    \footnote{We note that the higher the rank of the collection is, the more accurate the greater or equal in the theorem is.}
    \footnote{We also note that for this result some assumptions are needed regarding the field size or the degree of the polynomials in the collection.}
    , i.e. $\variety = \zerofunc{\varpolyset} = \set{x \suchthat \forall i: \varpoly_i(x) = 0}$.
%    \newline
%    Assume that either $\abs{\basefield} > \degree \cdot \varietydeg$ or all polynomials of the collection $\varpolyset$ are of degree $> \degree$.
    \newline
    Then, $\reedmullercodeex{\basefield}{\variety}{\degree}$ inherits its distance parameters from $\reedmullercodeex{\basefield}{\field}{\degree}$, i.e:
    \begin{enumerate}
        \item $\normalizedcodedistanceex{\basefield}{\variety}{\degree} \geq \normalizedcodedistanceex{\basefield}{\field}{\degree}$.
        \item $\listdecodingradiusex{\basefield}{\variety}{\degree} \geq \listdecodingradiusex{\basefield}{\field}{\degree}$.
    \end{enumerate}
\end{corollary*}

\paragraph{Main Technical Challenge}
We achieve these results by combining the two black-box properties of subsets $\variety \subseteq \field$ we presented.
Analysis of the polynomials in $\variety$ raises a new challenge, as previous techniques that were used to analyze low-degree polynomials,
both regarding $\field$~\cite{green2007distribution} and regarding subsets $\variety$~\cite{lampert2021relative},
were focused on maintaining the behavior of polynomials \emph{in the set they work on} ($\field$ and $\variety$ accordingly).
\newline
The novelty of our new technique is that it uses a similar approach to analyze polynomials $\variety$ as commonly used in $\field$,
\emph{while simultaneously maintaining a connection} between polynomials in $\variety$ to polynomials in $\field$.
This connection allows us to deduce that polynomials in $\variety$ behave similarly to polynomials in $\field$.
Informally, given a question regarding a polynomial in $\variety$, our new technique allows us to
associate it with a ``correct'' lift of it, and answer the question \emph{using properties of its lift}.
We emphasize that the correct lift (the one we later choose to use) \emph{depend} on the question,
thus we cannot pick a single canonical lift to \emph{generally} describe each polynomial in $\variety$.

Next we describe this challenge in more detail.
\newline
Analyses of polynomials in $\field$ were commonly based on the structure-randomness connection of polynomials in $\field$.
To use this connection, a procedure introduced by~\cite{green2007distribution}, which is called the \emph{regularization process}, is often used~\cite{kaufman2008worst, tao2011inverse, hatami2011higher, bhattacharyya2013locally, bhattacharyya2013algorithmic, DBLP:journals/corr/0001L15}.
This procedure takes any collection of polynomials, and constructs from it another collection of polynomials that has \emph{equidistriubtion in $\field$}
and ``captures'' all functions ``captured'' by the previous collection.
This notion of ``capturing'' is formulated by a definition called \emph{measurable},
and thus it is required that every function measurable by the old collection will be measurable by the new collection.

We note that the regularization procedure achieves random behavior in $\field$ by requiring the collection to have an \emph{extremely low algebraic structure}:
This implies the new collection has random behavior (equidistributed) as it is a property of $\field$.
The notion of structure is captured by a definition called \emph{rank}, where a polynomial with high rank has extremely low structure.
Additionally, the notion of lack of random behavior is captured by a definition called \emph{bias}, where a polynomial with low bias behaves randomly (equidistributed).
Therefore, the equidistribution is achieved in the regularization process by constructing a collection with high rank, as \emph{in $\field$ high rank implies low bias}.

To generalize these ideas to $\variety$, one must achieve a similar result in $\variety \subseteq \field$:
Given any collection of polynomials, construct a new collection of polynomials that is both equidistributed in $\variety$
and captures every function in $\variety$ that was previously captured.
In our case, however, we must also ensure that the new collection also captures all functions that were previously-captured \emph{in $\field$},
as in our case we wish to use the connection of polynomials in $\variety$ to polynomials in $\field$.
This can be summarized by 3 requirements:
\begin{enumerate}
    \item The polynomials in the new collection will behave random in $\variety$.
    \item Every function that was measurable in $\variety$ by the old collection will be measurable by the new collection in $\variety$.
    \item Every function that was measurable \emph{in $\field$} by the old collection will be measurable by the new collection \emph{in $\field$}.
\end{enumerate}
Alas, this third-requirement is incompatible with the way we achieve the first requirement.
Achieving the first requirement, which is the random behavior in $\variety$, is done by requiring an extremely low algebraic structure \emph{according to relative rank}.
This requires one to consider all possible lifts of polynomials in the collection to avoid any structure.
\newline
More accurately
\footnote{As the polynomials we have here are polynomials in $\field$ we can not discuss their lift.}
, for a polynomial $\funcdef{\genpoly}{\field}{\basefield}$,
we define an \emph{$\variety$-equivalent polynomial for $\genpoly$} to be
a polynomial in $\field$ that coincides with $\genpoly$ on $\variety$ and has the same degree bound as $\genpoly$
\footnote{This is the same as considering all lifts of the polynomial $\restrictfunc{\genpoly}{\variety}$, assuming such lift exist.}
.
Using this definition, the definition of relative rank requires examining all possible \emph{$\variety$-equivalent} polynomials,
and ensuring non of them exhibit structure.
\newline
Typically (in $\field$ for example), avoiding structure is achieved by replacing every structured polynomial by a \emph{small}
set of less-structured polynomials that capture it.
We note that it is \emph{crucial} that the set is small, and from reasons we did not explain here (see definition~\ref{definition:rank}), it is promised because the polynomial we wish to replace is structured.
\newline
For $\variety$, we aim to avoid \emph{all} $\variety$-equivalent polynomials of a polynomial from being structured.
Achieving this, while keeping the collection small,
requires one to replace the polynomial by a set of less-structured polynomials that capture a \emph{structured-$\variety$-equivalent} of it.
Therefore, this process creates a new collection that captures this $\variety$-equivalent polynomial,
but does not necessarily capture the original polynomial!
\newline
In summary, the challenge is that avoiding the structure of \emph{all} the lifts of a polynomial to achieve equidistribution in $\variety$,
without adding too many polynomials, may harm the functions we capture in $\field$.

\paragraph{Introducing New Tools}
We overcome this challenge by presenting a new definition that relaxes the notion of \emph{measurable} we required for functions in $\field$,
which we call \emph{$\variety$-measurable}.
This enables us to describe a relaxed version of the regularization process,
in which we require that every function in $\field$ that was $\variety$-measurable by the old collection will still be $\variety$-measurable by the new collection.
In contrast to the original regularization process, which mandated that functions that were measurable by the old collection will be measurable by the collection,
this relaxed definition only requires such functions to be \emph{$\variety$-measurable} by the new collection.

Even though we no longer need to capture all previously captured functions in $\field$,
it is important that the new relaxed-definition is strict enough to keep the connection between polynomials in $\variety$ and in $\field$.
Therefore, maintaining the $\variety$-measurable functions throughout the regularization process cannot be done trivially,
and this is handled in a procedure we call \emph{the $\variety$-relative regularization process} which is a stronger-version of the regularization process that is used in $\field$.
This new definition and procedure are thoroughly described in Section~\ref{sec:regularization-relative-to-X}.

We note that these new definition and procedure are a novel contribution of this work, and we believe they can
be useful in future research of the quotient Reed-Muller code.


\subsection{Comparison to Related Work}\label{subsec:previous-work}
In~\cite{bhowmick2014list} the authors studied the list decoding radius of Reed Muller codes $\field$.
They proved that, for prime fields, the list decoding radius \emph{reaches the distance of the code}, as conjectured earlier by~\cite{10.1145/1374376.1374417}
\footnote{Note that it is known that $\listdecodingradiusex{\basefield}{\field}{\degree} \leq \normalizedcodedistance{\basefield}{\degree}$,
    and therefore, in a sense, their result is \emph{optimal in $\field$} assuming $\degree, \abs{\basefield}$ are considered as constants.}
\footnote{We also note that their work also apply to the regime $\degree \geq \abs{\basefield}$. }
.
Formally, they showed the following theorem:
\begin{theorem}~\cite[Theorem 1]{bhowmick2014list}
Let $\basefield$ be a prime field.
Let $\epsilon > 0$ and $\degree, \blocklength \in \naturalnumbersset$.
There exists a constant
\footnote{It is important to note that $c$ is \emph{independent of $\blocklength$}.}
$c \definedas c(\abs{\basefield}, \degree, \epsilon)$ such that:
\[
    \listpolycount{\basefield}{\field}{\degree}{\normalizedcodedistance{\basefield}{\degree}- \epsilon} \leq c
\]
\end{theorem}
Our work gives new tools for analyzing polynomials in $\variety \subseteq \field$,
which we later use to follow their line of proof and show equivalent result \emph{in $\variety$}.

We next present related work regarding the study of polynomial codes in subsets $\variety \subseteq \field$.
Before presenting them specifically, we note that our work has a \emph{fundamental difference} than that of the previous study of polynomials in subsets.
%The essence of the difference lies in the way properties are deduced regarding polynomials in $\variety$.
Most works which studied polynomials over subsets $\variety \subseteq \field$ were focused on subsets in which every polynomial has a \emph{unique} lift.
This ensures that there is a 1-to-1 correspondence between polynomials in $\variety$ and in $\field$
and therefore allows easier connection between polynomials in $\variety$ and in $\field$.
%This connection is the most common approach to deduce properties regarding polynomials in $\variety$:
%such properties are inferred from the characteristics of their lifts,
%which are more comprehensively understood, as each lift is a polynomial in $\field$
%they are deduced from the properties of their lifts, which we better understand (as each lift is a polynomial in $\field$).
%\footnote{A possible reason for that focus is their focus on the \emph{extrinsic} definition of a polynomial,
%    that that is more focused on the behaviour of the origianl codewords \emph{restricted} to $\variety$
%    rather than the intrinsic definition, which captures the code \emph{quotient} by $\variety$.}
%.
\newline
We note that our work is non-trivial even in this case:
it extracts the properties of $\field$ that were used in~\cite{bhowmick2014list}, in a way they can be used to analyze quotient Reed-Muller codes.
However, as described earlier, our work addresses an additional substantial challenge which arise when the lift is \emph{not} unique.
Thus our work is only comparable to other works in the unique-lift case, which is the less-challenging case we address.

The first line of work we mention is this regard is the study of hitting sets for low degree polynomials~\cite{6243404, 10.1145/2554797.2554828, 6875485},
and a stronger variant of it which is the study of pseudorandom-generators against low degree polynomials~\
    \cite{10.1145/1060590.1060594, 4389478, 10.1145/1374376.1374455, 4558816,  Cohen2013PseudorandomGF, derksen2022fooling, dwivedi2024optimalpseudorandomgeneratorslowdegree}
Both definitions capture subsets
\footnote{Sometimes this subset is allowed to be a \emph{multiset}.}
$\variety \subseteq \field$ such that every polynomial over $\field$ has a non-negligible distance from $0$ \emph{when restricted to $\variety$}.
This requirement implicitly implies that every low degree polynomial over $\variety$ has at most a \emph{single} lift.

Another line of work worth mentioning in this regard is~\cite{4558818, guruswami2017efficientlylistdecodablepuncturedreedmuller},
which studied \emph{puncturing of Reed-Muller codes}.
This line of work studied the construction of sets $\variety \subseteq \field$,
such that puncturing Reed-Muller codes over $\variety$, that is, taking every original codeword and \emph{restricting} it to $\variety$, will yield a good error-correction code.
To perform their analysis, it was important that every polynomial in $\variety$ has at most a single lift,
and therefore it was an assumption in their work.
%We also note that our work answers similar questions regarding the distance parameter of the constructed code,
%but instead of focusing on a specific construction,
%it describes general properties for subsets that achieve the desired distance parameters (with an explicit construction that has those properties).

%TODO: change the order of references to be all increasing
The papers~\cite{brakensiek2024genericreedsolomoncodesachieve, alrabiah2024randomlypuncturedreedsolomoncodes, brakensiek2024generalizedgmmdspolynomialcodes}
also studied similar questions.
This line of work is followed by the resolution of the \emph{GM-MDS conjecture}, which was proved by~\cite{DBLP:journals/corr/abs-1803-02523, DBLP:journals/corr/abs-1803-03752}.
%After its resolution, it was proved by~\cite{brakensiek2024genericreedsolomoncodesachieve} that over \emph{a field of size $2^{O(\blocklength)}$}, randomly punctured Reed-Solomon codes are combinatorially list-decodable all the way to the list decoding capacity.
%The paper~\cite{brakensiek2024generalizedgmmdspolynomialcodes} generalizes this result and achieves similar results for \emph{all} polynomial codes over exponentially large fields.
%Following this line of work~\cite{alrabiah2024randomlypuncturedreedsolomoncodes} showed that for Reed-Solomon codes, a similar result can be attained to a field of a smaller size.
%Specifically, \emph{linear in $\blocklength$}.
\newline
We note that these works
were focused on the regime where the field is \emph{large}.
More specifically,
they require that the field is \emph{large in respect of $\blocklength$}, i.e $\Omega(\blocklength)$.
We emphasize that our work is focused on \emph{constant fields}.
Moreover, their results were regarding \emph{random} puncturing, while our result makes an \emph{explicit} puncturing.

%Another line of work in this regard studies puncturing of AG codes~\cite{DBLP:journals/corr/abs-1708-01070, guo2021efficientlistdecodingconstantalphabet, brakensiek2024agcodesachievelistdecoding}.
%The newest paper in this line of work~\cite{brakensiek2024agcodesachievelistdecoding} showed that random puncturing of AG codes achieve list decoding capacity over constant fields,
%and as a corollary they showed that AG codes (without puncturing) achieve list decoding capacity over constant fields.
%\newline
%We note that AG codes are a generalization of Reed-Solomon codes,
%where our work is focused on Reed-Muller codes in subsets $\variety \subseteq \field$, which can be thought of as a generalization of AG codes to multiple transcendental variables.
%
We also note that most studies presented above also achieved results regarding the \emph{rate} of the punctured code.
This property of the code can be analyzed naturally when each polynomial over $\variety$ has a \emph{unique} lift, as such assumption implies that the number of polynomials remains the same in $\variety$ as of in $\field$.
As our work does \emph{not} assume such uniqueness, the rate of the code we consider is not analyzed in our work, and thus remained \emph{an open problem}
\footnote{Note that is highly dependent on $\variety$, as additional assumptions are needed to acheive good results in this regard.}
.

\subsection{Proof Overview}\label{subsec:our-work}
In this subsection we present our main technical contribution, which is how we address the challenge of \emph{non-unique lift}.
This is done by introducing the definition of being \emph{$\variety$-measurable}, and by presenting a new tool which is the \emph{relative regularization process}.

To describe them clearly, we first elaborate more on two definitions we described briefly.
\paragraph{Measurable}
Suppose we have a collection of polynomials
\footnote{In this context we think of $c$ as a small (constant for example).}
$\genpolyset[1] = \parens{\genpoly_1,...,\genpoly_c}$ where $\funcdef{\genpoly_i}{\field}{\basefield}$ is a polynomial of degree $\leq \degree$.
We say a function $\funcdef{\genfunc}{\field}{\basefield}$ is \emph{measurable in respect of $\genpolyset[1]$} if it can be determined by the values of $\genpoly_1,...,\genpoly_c$:
if one knows the values of $\genpoly_1(x),...,\genpoly_c(x)$, then she also knows the value of $\genfunc(x)$.
This mathematical-analysis notion, which was first used in a similar context in~\cite{green2007primescontainarbitrarilylong}, is formally defined as follows:
\begin{definition}[Measurable]
    We say a function $\funcdef{\genfunc}{\field}{\basefield}$ is \emph{measurable in respect of $\genpolyset[1] = \parens{\genpoly_1,...,\genpoly_c}$} if
    there exists $\funcdef{\Gamma_{\genfunc}}{\basefield^c}{\basefield}$ such that:
    \[
        \genfunc(x) = \Gamma_{\genfunc}(\genpoly_1(x),...,\genpoly_c(x))
    \]
\end{definition}
This definition can be thought of as the collection $\genpolyset$ ``captures'' the function $\genfunc$
\footnote{Note that this definition also generalizes to every collection of functions.
For now, one can think of the collcetion as a collection of bounded degree polynomials.}
\footnote{One can think of this definition as a generalization of linear span: the collection \emph{spans} the function, where $\Gamma$ is some notion of a span.}
.

Moreover, it would have been useful had this collection of polynomials been ``pseudo-random'', i.e the vector $\parens{\genpoly_1(x),...,\genpoly_c(x)}$ would be equidistributed over a random input $x \in \field$.
This equidistribution would allow us to better understand functions $\genfunc$ that are measurable in respect of $\genpolyset$.

As $\field$ has the rank-bias property, this equidistribution can be achieved by requiring $\genpolyset$ to be a collection of high-rank.
This is a fundamental idea behind the regularization process, first presented in~\cite{green2007distribution}.
Given a collection of polynomials $\genpolyset$, the regularization process constructs another collection $\genpolyset[2]$ of polynomials (with the same degree bound),
such that $\genpolyset[2]$ is a collection of high-rank (and therefore equidistributed) that \emph{refines} $\genpolyset$.
By refine, we mean that every function that was measurable by the first collection $\genpolyset$ is also measurable by the new collection $\genpolyset[2]$ (See definition~\ref{def:semantic-refinement}).

\paragraph{Relative Rank}
We remind the reader that $\variety$-relative rank is a notion that measures the algebraic structure of a polynomial in a subset $\variety \subseteq \field$, by considering the structure of all of its $\variety$-equivalent polynomials.
This notion was presented by~\cite{gowers2022equidistributionhighrankpolynomialsvariables, lampert2021relative}, and is used to achieve equidistribution in $\variety$ assuming $\variety$ has relative rank-bias property.
It is defined as follows:
\begin{definition}[Relative Rank, informal.
See definition~\ref{def:relative-rank-of-polynomial}]
    Let $\variety \subseteq \field$ be a subset,
    let $\degree \in \naturalnumbersset$, and let $\funcdef{\genpoly}{\field}{\basefield}$ be a polynomial of degree $= \degree$.
    The $\variety$-relative rank of $\genpoly$ is defined as follows:
    \[
        \relrank{\variety}{\genpoly} \definedas \min \set{\rank{\genpoly - \relativeremainder{\genpoly}} \suchthat
        \relativeremainder{\genpoly} \in \allpolyset{\leq \degree}{\field}{\basefield}, \restrictfunc{\relativeremainder{\genpoly}}{\variety} \equiv 0}
    \]
\end{definition}
%\paragraph{Concrete Sets Examples}
%In recent years, a few works have studied subsets such that have the relative rank-bias property, specifically in the regime where $\degree < \abs{\basefield}$.
%\newline
%One of them is a paper~\cite{lampert2021relative}, in the line of work that studies subsets $\variety \subseteq \field$ that are \emph{high rank varieties}~\cite{kazhdan2020propertieshighranksubvarieties, kazhdan2018polynomial, kazhdan2019extendingweaklypolynomialfunctions, kazhdan2017extendinglinearquadraticfunctions}.
%An algebraic variety is defined as the zero set of a collection of polynomials $\set{\varpoly_1,...,\varpoly_{\varietypolycount}}$, i.e. $\variety = \set{x \suchthat \varpoly_1(x)=...=\varpoly_{\varietypolycount}(x)=0}$.
%When the collection of polynomials that generate the algebraic variety has a high rank (as a collection),
%we say that the variety is a \emph{high rank variety}
%\footnote{
%    Note that the rank of a collcetion of polynomials is a rather different from the rank of a single polynomial,
%    but as it captures a similar idea, we leave the exact definitions to later (see definition~\ref{definition:factor-rank}).
%}
%.
%TODO: Always put the names of the authors after an cite. Instead saying "they", always call them by name of say "the authors".
%In~\cite[Theorem 1.8]{lampert2021relative}, Lampert and Ziegler have shown that \emph{high rank varieties have the relative rank-bias property}.
%We translate their statement to our point of view in Corollary~\ref{high-rank-variety-has-limited-rank-relative-bias-property}.

%Additionally, in~\cite{gowers2022equidistributionhighrankpolynomialsvariables}, Gowers and Karam studied subsets $\variety \subseteq \field$ of \emph{restricted alphabets}.
%Formally, let $S \subset \basefield$ be a set that restricts the alphabet, and denote $\variety \definedas S^{\blocklength}$.
%In other words, the codewords of the code are polynomials over $\field$, restricted only to inputs from $S^\blocklength$.
%In~\cite[Theorem 1.4]{gowers2022equidistributionhighrankpolynomialsvariables} it was shown that \emph{restricted alphabet subsets have the relative rank-bias property}.
%\newline
%Note that the definition of rank presented by Gowers and Karam is slightly definition than the standard definition of rank.
%While this difference may seem unharmful at first, it is, to our knowledge, incompatible to the \emph{relative regularization} process (which is the heart of our proof).
%A more detailed explanation regarding this difference is given in Note~\ref{note:comparison-to-gowers-rank}.
%
%\paragraph{Formulating the Notion}
%This property is parametrized by the following parameters:
%The field $\basefield$ which we work on, an integer representing the degree of evaluated polynomials $\degree$,
%and a function $\funcdef{\rankbiasfunc}{\realnumbersset^{+}}{\naturalnumbersset}$ that for every $\epsilon$ returns the relative rank needed so that the bias in $\variety$ will be $< \epsilon$.
%In this paper, we think about the first two parameters as constants.
%We think of the last parameter as \emph{how well the subset is regarding the relative rank-bias property}.
%Under this parameterization, we may say a subset $\variety \subseteq \field$ has $(\rankbiasfunc, \basefield, \degree)$-relative rank-bias property.
%For formal definition, see Subsection~\ref{sec:relative-rank-bias-property}.
%\newline
%Next, we note that sometimes, the requirement that a subset $\variety \subseteq \field$ will have the relative rank-bias property to \emph{all} extent is \emph{impossible} to achieve.
%Here, by ``extent'' we mean how small $\epsilon > 0$ can be, such that high relative rank will imply the bias in $\variety$ is $< \epsilon$.
%We see this challenge raised with subsets that are not described by a ``boolean'' property (where a subset either has it or not, such as restricted alphabets),
%but by subsets that are described by a more ``continuous'' property (properties that subsets can have to some extent).
%An example of the latter is the case where $\variety$ is a high rank variety: the rank of the variety is a natural number, thus being a subset of ``high rank'' is not a boolean property; it ``gets better'' the higher the rank of variety is.
%Subsets that have such properties but still have the relative rank-bias property to some extent, are captured by a definition we call \emph{limited relative rank-bias property}.
%We formulate this notion in Subsection~\ref{subsec:limited-relative-rank-bias-property}.

%\subsubsection{Measurable and The Regularization Process}

\subsubsection{\titlevariety-measurable and The \titlevariety-Relative Regularization Process}
In this subsection we discuss the generalization of the regularization process to subsets $\variety \subseteq \field$ using the equivalent of rank-bias relation in $\variety$.
We name this tool \emph{the relative regularization process}.
%This is a tool that enables us to achieve similar equidistribution properties of $\genpolyset$ in $\variety$ as the regularization process achieved in $\field$.
%Note that, as in $\field$ it was achieved using the rank-bias property of $\field$,
%we now require that $\variety$ has the \emph{relative} rank-bias property.
%Additionally, in this case, instead of refining the collection $\genpolyset$ to a collection of high rank, we refine the collection $\genpolyset$ to be a collection of high \emph{$\variety$-relative rank}.
%This requirement adds an extra challenge, thus requiring us to relax our definition of what is a \emph{refinement}, or what is being ``measurable in respect of $\genpolyset$''.

%We note that this is our main tool that enables us to use tools from high-order Fourier analysis to also analyze polynomials in $\variety$.
Practically, we use this tool to show that given a specific question in mind, every $\funcdef{\onvarpoly}{\variety}{\basefield}$ has some polynomial $\funcdef{\genpoly}{\field}{\basefield}$ that behave ``similarly'' in respect to this question.
This allows us to pull properties of $\genpoly$ to better understand $\onvarpoly$.
The perfect candidate for such $\genpoly$ is a \emph{lift} of $\onvarpoly$.
\newline
In order to use $\genpoly$ to deduce properties of $\onvarpoly$, we use the well-studied properties of polynomials in $\field$ to acheive properties of $\genpoly$, and relate these to properties of $\onvarpoly$.
More specifically, assume that $\onvarpoly$ and $\genpoly$ are measurable in respect of a collection of polynomials $\genpolyset$ (each in its domain).
Our strategy is to use $\genpoly$ to deduce properties of $\Gamma_{\genpoly}$, and then use the properties of $\Gamma_{\genpoly}$ to deduce properties of $\onvarpoly$.

Now let us describe the extra challenge.
We start by following the ideas of the regularization process we described for $\field$.
Assuming the collection is not a collection of $\variety$-relative high rank, then there must exist a polynomial in the collection that has low \emph{relative} rank, which we denote by $\genpoly^\star$
\footnote{More precisely, some linear combination of polynomials has low relative rank.}
.
Note that in relative rank, this does not necessarily mean that $\genpoly^\star$ is of low rank, but that there exists another $\variety$-equivalent polynomial that has a low rank.
Thus, even if we remove the low-rank $\variety$-equivalent polynomial and add to the collection all the polynomials that decomposed it,
we cannot require that every function that was measurable by the old collection will still be measurable by the new collection:
even the polynomial we removed is not necessarily measurable by the new collection!
\newline
To allow such regularization process to still apply, we note that while $\genpoly$ might not be measurable in respect of the new collection, a $\variety$-equivalent polynomial of $\genpoly$ \emph{is} measurable with respect of it.
Therefore, we relax the notion of being measurable to being \emph{$\variety$-measurable}.
\newline
We say a function $\genfunc$ is $\variety$-measurable in respect of $\genpolyset$ if it can be determined by the polynomials of $\genpolyset$
\emph{up to a valid $\variety$-remainder}.
We first describe an incomplete definition, then present the challenge that rises with it, and finally present its resolution.
\begin{definition}[$\variety$-measurable, Incomplete Definition]
\footnote{This incomplete definition lacks the requirement of the \emph{validity} of the $\variety$-remainder}
We say a function $\genfunc$ is $\variety$-measurable
if there exists a function $\funcdef{\Gamma}{\basefield^c}{\basefield}$
and a $\variety$-remainder, i.e a function $\funcdef{\relativeremainder{\genfunc}}{\field}{\basefield}$ with $\restrictfunc{\relativeremainder{\genfunc}}{\variety} \equiv 0$
such that:
\[
    \forall a \in \field: \genfunc(a) = \Gamma(\genpoly_1(a),...,\genpoly_c(a)) + \relativeremainder{\genfunc}(a)
\]
\end{definition}

Previous works analyzing polynomials in $\field$ were able to deduce two things from $\genfunc$ being measurable by $\genpolyset$:
that the structure of $\Gamma$ is similar to the structure of $\genfunc$, and that a random input of $\Gamma$ behave similarly to a random input of $\genfunc$.
\newline
To study polynomials in $\variety$, we wish to connect $\onvarpoly$ to $\genpoly$ (which is a lift of $\onvarpoly$).
Thus, we think of $\genfunc = \genpoly$, and require two similar things.
Firstly, we want the structure of $\Gamma$ to be similar to the structure of $\genfunc$ (in this case, $\genpoly$), which we understand as $\genfunc$ is a polynomial in $\field$.
Secondly, we want a random input of $\Gamma$ to behave similarly to a random input of $\onvarpoly$, as $\onvarpoly$ is the polynomial we wish to understand.
The latter is easily achieved using the fact high $\variety$-relative rank implies equidistribution in $\variety$.
The former, however, might be damaged by the remainder as we defined it: we can only learn the structure of $\Gamma$ using the structure of $\genfunc - \relativeremainder{\genfunc} = \Gamma(\genpoly_1,...,\genpoly_c)$.
However, the structure of $\genfunc - \relativeremainder{\genfunc}$ can be very different from the structure of $\genfunc$,
as we did not require any structure of the $\variety$-remainder $\relativeremainder{\genfunc}$.
Thus, we can not deduce the structure of $\Gamma$ via the structure of $\genfunc$ using the incomplete definition described above.

To handle this issue, we add one more requirement regarding the $\variety$-remainder,
which ensures that the structure of $\genfunc$ can be understood via the structure of $\Gamma$:
\[
    \deg(\genfunc - \relativeremainder{\genfunc}) \leq \deg(\genfunc)
\]
If the $\variety$-remainder also has this property, we say it is a \emph{valid} $\variety$-remainder for $\genfunc$.
This can be summarized by the following (complete) definition:
\begin{definition}[$\variety$-measurable]
    We say a function $\genfunc$ is $\variety$-measurable
    if there exists a function $\funcdef{\Gamma}{\basefield^c}{\basefield}$
    and a \emph{valid} $\variety$-remainder, i.e a function $\funcdef{\relativeremainder{\genfunc}}{\field}{\basefield}$
    with $\restrictfunc{\relativeremainder{\genfunc}}{\variety} \equiv 0$ and $\deg(\genfunc - \relativeremainder{\genfunc}) \leq \deg(\genfunc)$
    such that:
    \[
        \forall a \in \field: \genfunc(a) = \Gamma(\genpoly_1(a),...,\genpoly_c(a)) + \relativeremainder{\genfunc}(a)
    \]
\end{definition}
We use this new definition the following way:
Instead of using $\genfunc$ to understand $\Gamma$, we use $\genfunc - \relativeremainder{\genfunc}$ to do so.
We choose $\genfunc - \relativeremainder{\genfunc}$ as it has the same structure as $\genfunc$, but it is ``closer'' to the function $\Gamma$ as $\genfunc - \relativeremainder{\genfunc} = \Gamma(\genpoly_1,...,\genpoly_c)$
\footnote{One can think of this step as "taking the right $\variety$-equivalent" in respect of $\genpolyset$.}
.
Finally, as $\Gamma$ behaves similarly to $\onvarpoly$ for random inputs, we can use $\Gamma$ to deduce properties regarding $\onvarpoly$.
\newline
With this in hand, let us finish describing the relative-regularization process.
The requirement on the validity of the $\variety$-remainder raises a new challenge in the $\variety$-relative regularization process:
we need to somehow control the structure of the $\variety$-remainder, even though this ``error''
is substituted in $\Gamma$ each time we wish to replace a polynomial in our collection.
We address this challenge using a Lemma proved in~\cite{DBLP:journals/corr/0001L15} called the ``faithful composition lemma'',
which allows us to deduce strong properties regarding the structure of $\Gamma$ given the collection was of a high (regular) rank in the first place.
Therefore, we add to each step of the relative-regularization process a (regular) regularization, which ensures $\Gamma$ is very structured.
This strong structure of $\Gamma$ is later used to control the error and deduce it is in the form of a valid $\variety$-remainder.
For the exact details, see Theorem~\ref{theorem:regularization-in-X}.
We conclude this by informally stating our main technical theorem, which is the relative regularization process we just described:
\begin{theorem}[Relative Regularization Process, Informal, See Theorem~\ref{theorem:regularization-in-X}]
Let $\rankval, \degree \in \naturalnumbersset$ be integers that represents a requested rank and degree respectively,
and let $\genpoly_1,...,\genpoly_c$ be a collection of polynomials of degree $\leq \degree$.
Then, there is another collection $\genpoly^{\prime}_1,...,\genpoly^{\prime}_{c^\prime}$ of polynomials of degree $\leq \degree$,
such that:
\begin{enumerate}
    \item Every function that is $\variety$-measurable in respect to the first collection is also $\variety$-measurable in respect to the new collection.
    \item The new collection is of $\variety$-relative rank $\geq \rankval$.
    \item The new collection is of bounded size, i.e $c^\prime \leq C_{\rankfunc, \degree, c}$.
\end{enumerate}
\end{theorem}

\subsubsection{List Decoding in \titlevariety via \titlevariety-Relative Regularization}
In this subsection, we demonstrate how to use the relative regularization process to achieve our main theorem: analysis of the list decoding radius of $\reedmullercodeex{\basefield}{\variety}{\degree}$.

We follow the line of proof of~\cite{bhowmick2014list}, but this time, we are interested in bounding the amount of polynomials \emph{in $\variety$} around every function \emph{in $\variety$}.
More specifically, we wish to show that there is a constant number of words that are $(\normalizedcodedistance{\basefield}{\degree} - \epsilon)$-close to any fixed function in $\variety$.


%
%Let $\funcdef{\onvarfunc}{\variety}{\basefield}$ be a function.
%We apply Lemma~\ref{every-function-can-be-approximated-by-a-few-functions} with $A = \variety$, $B = \basefield$, $F = {\allpolyset{\leq \degree}{\variety}{\basefield}}$.
%We obtain a collection of a few polynomials $\onvarpolyset[3] \subset \allpolyset{\leq \degree}{\variety}{\basefield}$ defined by $\onvarpolyset[3] = (\onvarpoly[3]_1,...,\onvarpoly[3]_c)$
%that approxiamtes every polynomial of degree $\leq \degree$ in $\variety$.
%This reduces the question to only count polynomials in the radius of functions $\funcdef{\onvarfunc}{\variety}{\basefield}$ of the form:
%\[
%    \onvarfunc(x) = \Gamma (\onvarpoly[3]_1(x),...,\onvarpoly[3]_c(x))
%\]
%for some $\funcdef{\Gamma}{\basefield^c}{\basefield}$.
%
%Next, we lift every polynomial in the collection $\onvarpolyset[3]$ and get a collection of polynomials in $\field$, denoted by $\genpolyset[3]$.
%We note that $\genpolyset[3]$, when restricting each of its functions to $\variety$, measures exactly the same functions that were measurable by $\onvarpolyset[3]$.
%
%Now, we apply the $\variety$-relative regularization process to $\genpolyset[3]$.
%This yields a new collection of polynomials $\genpolyset[3]^\prime = \set{\genpoly[3]_1^{\prime},...,\genpoly[3]_{c^\prime}^{\prime}}$ that is equidistributed in $\variety$
%and captures in $\variety$ the same functions that were captured in $\variety$.
%More formally, the latter states that every function that was measurable by $\genpolyset[3]$ restricted to $\variety$, is still measurable by the new collection restricted to $\variety$.
%We note that here, we do not state the ``validity'' (the structure) of the remainder, which is an important promise given by the definition of $\variety$-measurable.
%This will play a crucial role in a second relative regularization process which we will do during the analysis.

%The strategy of the proof is to show that every polynomial in $\variety$ that is $\normalizedcodedistance{\basefield}{\degree}$-close to $\onvarfunc$ is \emph{measurable} by the collection $\genpolyset[3]^\prime$ in $\variety$.
%This is similar to the strategy of the proof of~\cite{bhowmick2014list}, with difference in the domain of the functions.
%This will bound the number of polynomials in the radius of $\onvarfunc$ by the amount of possible functions measurable by a $c^\prime$-sized collection, that is $\abs{\basefield}^{\abs{\basefield}^{c^\prime}}$.

Let $\funcdef{\onvarfunc}{\variety}{\basefield}$ be a received word.
First, we apply a lemma proved in~\cite[Corollary 3.3]{bhowmick2014list}.
The lemma shows that there is a constant-sized (depending on $\epsilon$) collection of polynomials in $\variety$, denoted by $\onvarpolyset[3]$,
such that the distance of $\onvarfunc$ to \emph{any} polynomial can be approximated by the distance of $\onvarfunc$ to some function that is measurable by $\onvarpolyset[3]$ in $\variety$.
This means that instead of bounding the number of polynomials in the radius of $\onvarfunc$, one can bound the number of polynomials in the radius of some function measurable by $\onvarpolyset[3]$.
Thereby, every polynomial-specific measurable function can be thought of as a \emph{low complexity proxy} for $\onvarfunc$ in respect to the polynomial.

Next, we lift each polynomial from $\onvarpolyset[3]$ and apply the \emph{relative regularization process}.
This yields a new collection of polynomials in $\field$ that is constant sized and randomly-behaving (in $\field$).
Denote this new collection by $\genpolyset[3]^{\prime}$
\footnote{We use the same notations as the original proof for clearannce.}
.
Thereby, the question of list decoding is reduced to the following question:
We have a specific constant-sized randomly-behaving collection of polynomials $\genpolyset[3]^\prime = \set{\genpoly[3]_1^{\prime},...,\genpoly[3]^{\prime}_{c^\prime}}$
that was constructed using the function $\onvarfunc$.
We need to bound the amount of polynomials in $\variety$ that are $(\normalizedcodedistance{\degree}{\basefield} - \epsilon / 2)$-close to be measurable by this collection in $\variety$.
Note that the randomly-behaving property was achieved using the \emph{relative rank-bias property} of $\variety$.
Additionally, we note the collection $\genpolyset[3]^\prime$ is a collection of polynomials in $\field$ which we obtained by using the \emph{lift-enabler property} of $\variety$.

From there (and similarly to the analysis in $\field$),
the strategy is to show that polynomials that are that close to being measurable by the randomly-behaving collection $\genpolyset[3]^\prime$, are in fact \emph{measurable} by it.
This will bound the number of such polynomials by the amount of possible functions that are measurable by $\genpolyset[3]^\prime$, which is constant as the collection is of constant size.


Let $\funcdef{\onvarpoly}{\variety}{\basefield}$ be a polynomial of degree $\leq \degree$, and consider a lift of it $\funcdef{\genpoly}{\field}{\basefield}$.
Consider the collection $\genpolyset[3]^\prime \cup \set{\genpoly}$.
Surely, $\genpoly$ is measurable by this collection in $\field$.
Applying $\variety$-relative-regularization to this collection yields a new collection $\genpolyset[3]^{\prime\prime}$ that is equidistributed in $\variety$, such that every $\variety$-measurable function by the old collection is $\variety$-measurable by the new collection.
By a reason we have not explained in this brief explanation, we can ensure this collection is of the form $\genpolyset[3]^{\prime\prime} = \genpolyset[3]^\prime \cup \set{\genpoly[3]_{1}^{\prime\prime},...,\genpoly[3]_{c^{\prime\prime}}^{\prime\prime}}$.

As $\genpoly$ was $\variety$-measurable by $\genpolyset[3]^\prime \cup \set{\genpoly}$ (it was even measurable), $\genpoly$ is $\variety$-measurable by the new collection $\genpolyset[3]^{\prime\prime}$:
That is, $\genpoly$ is measurable by $\genpolyset[3]^{\prime\prime}$ up to a \emph{valid} remainder, denoted by $\relativeremainder{\genpoly}$.
\newline
This means there exists $\funcdef{\Phi}{\basefield^{c^\prime + c^{\prime\prime}}}{\basefield}$ such that:
\[
    \forall a \in \field: \genpoly(a) = \Phi(\genpoly[3]^\prime_1(a),...,\genpoly[3]^\prime_{c^\prime}(a), \genpoly[3]^{\prime\prime}_1(a),...,\genpoly[3]^{\prime\prime}_{c^{\prime\prime}}(a))) + \relativeremainder{\genpoly}(a)
\]

In $\field$, the proof would follow by studying the structure of the function $\Phi$ and use it to induce that $\Phi$ does not depend on its last $c^{\prime\prime}$ variables.
This implies that $\genpoly$ is measurable by the original collection $\genpolyset[3]^{\prime}$ which concludes the proof
\footnote{Note that in $\field$ there is no remainder, so the equation above (with the last $c^{\prime\prime}$ variables as constants) implies measurability by $\genpolyset[3]^{\prime}$.}
.

More accurately, the analysis in $\field$ used the fact that substituting \emph{randomly behaving} polynomials in $\Phi$ yields a structured function
\footnote{In our notations, this structured function is $\genpoly$, which is a polynomial of degree $\leq \degree$ and thus structured}
.
This is used to show that $\Phi$ as a function by itself, with inputs from $\basefield^{c^\prime + c^{\prime\prime}}$, is a very structured function.
The strong structure of $\Phi$, with the fact that $\Phi$ (with inputs substitued to be the functions of $\genpolyset[3]^{\prime\prime}$) is close to the function $\onvarfunc$,
are then combined to deduce that $\Phi$ does not depend on its last $c^{\prime\prime}$ variables.

This paradigm can not be extended effortlessly to our case.
In $\variety$, deducing that $\Phi$ is very structured requires a one-more major step.
This is because we do \emph{not} know any correspondence in the behavior of $\Phi$ (which we want to understand) with the behavior of $\genpoly$ (which we know is structured).
We only know there is a correspondence between $\Phi$ to another function $\genpoly - \relativeremainder{\genpoly}$, which apriori we do not know is structured!

Fortunately, the relative regularization process (Theorem~\ref{theorem:regularization-in-X}) mandates that the remainder of the measurement is \emph{valid}.
That is, if $\genpoly$ was structured (a polynomial of degree $\leq \degree$), then so does $\genpoly - \relativeremainder{\genpoly}$.
This is \emph{crucial}, as it allows us to use the relation between $\Phi$ and $\genpoly - \relativeremainder{\genpoly}$ to deduce that $\Phi$ is structured,
and continue the original outline of the proof of~\cite{bhowmick2014list}.
For more details in this regard, see Theorem~\ref{thm:list-decoding-RM-in-X}.


\subsection{Organization}\label{subsec:organization}
In Section~\ref{sec:preliminaries} we present some basic notations and conventions,
and define the preliminaries we have regarding high-order Fourier analysis in $\field$: polynomials, rank and regularization.
We later generalize each component we presented in Section~\ref{sec:preliminaries} to study polynomials in $\field$ to also study polynomials in $\variety$:
in Section~\ref{sec:polynomials_in_X} we present the set of polynomials in $\variety$ and present the \emph{lift-enabler property};
in Section~\ref{sec:relative-rank-bias-property} we present the \emph{$\variety$-relative rank-bias property};
and in Section~\ref{sec:regularization-relative-to-X} we present the $\variety$-measurable notion, and our main tool, which is the \emph{$\variety$-relative regularization process}.
Next, we present two applications regarding the distance parameters of Reed-Muller codes in $\variety$:
In Section~\ref{sec:radius-of-RM-over-X} we prove the inheritance of the \emph{distance} of the code;
and in Section~\ref{sec:list-decoding-reed-muller-over-X} we prove the inheritance of the \emph{list decoding distance} of it (which is much more involved).

%\subsection{Acknowledgments}\label{subsec:acknowledgments}
%We express our gratitude to Schahar Lovett for his invaluable consultation on this work and for engaging in several insightful discussions.
%Furthermore, the first author recommends the survey~\cite{book} as an excellent resource for readers seeking an introduction to higher-order Fourier analysis.
