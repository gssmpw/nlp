\section[List Decoding Reed Muller Over \titlevariety]{List Decoding Reed Muller Over \titlevariety}\label{sec:list-decoding-reed-muller-over-X}
In this section, we prove our main theorem:
we prove the list decoding radius of Reed-Muller codes in is $\variety$ \emph{at least}
the list decoding radius of Reed-Muller codes in $\field$,
assuming $\variety$ is lift-enabler and has the relative rank-bias property.
We start by presenting formally the list decoding radius in $\variety$.
\begin{definition}[List Decoding in $\variety$]
    Let $\basefield$ be a finite field.
    Let $\degree, \blocklength \in \naturalnumbersset$, and let $\variety \subseteq \field$.
    \newline
    We define the reed muller list-decoding count in $\variety$ at distance $\tau$ as follows:
    \[
        \listpolycount{\basefield}{\variety}{\degree}{\tau} \definedas
        \max_{\funcdef{\genfunc}{\variety}{\basefield}}
            {\abs{\set{\genpoly \in \allpolyset{\leq \degree}{\variety}{\basefield} \suchthat {\dist{\genpoly, \genfunc} \leq \tau}}}}
    \]
    Additionally, we define $\listdecodingradiusex{\basefield}{\variety}{\degree}$ to be the \emph{list decoding radius}, which is
    the maximum $\tau$ for which $\listpolycount{\basefield}{\variety}{\degree}{\tau - \epsilon}$ is bounded by a \emph{constant} depending only on $\epsilon, \abs{\basefield}, \degree$.
\end{definition}

We recall that it was shown in~\cite[Theorem 1]{bhowmick2014list} that the list decoding radius of Reed Muller is $\normalizedcodedistance{\basefield}{\degree}$.
To be more precise, it was shown that for every $\epsilon > 0$, the list-decoding count is constant (independent of $\blocklength$) in distance $\tau = \normalizedcodedistance{\basefield}{\degree} - \epsilon$.
Formally, they have shown the following theorem:
\begin{theorem}[List Decoding RM in $\field$]\label{list-decoding-RM-in-Fn}
    There exists a function $c(\basefield, \degree, \epsilon)$ such that the following holds:
    Let $\basefield$ be a finite field, let $\epsilon > 0$, and let $\degree, \blocklength \in \naturalnumbersset$.
    Then, we have:
    \[
        \listpolycount{\basefield}{\field}{\degree}{\normalizedcodedistance{\basefield}{\degree} - \epsilon}
        \leq c(\basefield, \degree, \epsilon)
    \]
\end{theorem}
Additionally, we recall a lemma that was presented in~\cite[Corollary 3.3]{bhowmick2014list}, and was used in the analysis of the list decoding radius of Reed-Muller codes in $\field$:
\begin{lemma}[Low Complexity Approximation]~\cite[Corollary 3.3]{bhowmick2014list}\label{every-function-can-be-approximated-by-a-few-functions}
Let $\funcdef{\genfunc[2]}{A}{B}$, and let $\epsilon > 0$.
Let $\genfuncset \subseteq B^A$ be a collection of functions from $A$ to $B$.
Then there exists $c \leq 1/\epsilon^2$ functions $\genfunc_1,...,\genfunc_c \in \genfuncset$ such that
for every $\genfunc \in \genfuncset$, there is a function $\funcdef{\Gamma_{\genfunc}}{B^c}{B}$ such that:
\[
    \prex{x \in A}{\Gamma_{\genfunc}(\genfunc_1(x),...,\genfunc_c(x)) = \genfunc(x)}
    \geq \prex{x \in A}{\genfunc[2](x) = \genfunc(x)} - \epsilon
\]
\end{lemma}
The lemma shows that $\genfunc[2]$ can ``estimated'' by a only a few functions from $\genfuncset$.
Note that the estimation is close to $\genfunc[2]$ in compare to every $\genfunc \in \genfuncset$ and not necessarily close to $\genfunc[2]$ itself.

Finally, we present our main theorem, which shows that under assumptions on the subset $\variety \subseteq \field$, the list decoding radius of polynomials in $\variety$ will be similar to the list decoding radius in $\field$.
\newline
In more details (and informally), we show that if $\variety \subseteq \field$ is lift-enabler, has the limited-relative-rank-bias-property,
the list-decoding count is constant (independent of $\blocklength$) for every valid $\epsilon$ in distance $\tau = \normalizedcodedistance{\basefield}{\degree} - \epsilon$.
Note that not every $\epsilon > 0$ will be valid: the valid values of $\epsilon$ will depend on the limitations of the rank-bias property.
Formally, we show the following:
\begin{theorem}[List Decoding RM in $\variety$]\label{thm:list-decoding-RM-in-X}
    There exist functions $c_1(\basefield, \degree, \rankbiasfunc, \epsilonlimitedrankbias)$ and $c_2(\basefield, \degree, \rankbiasfunc, \epsilon)$ such that the following holds:
    Let $\basefield$ be a finite field, and let $\degree \in \naturalnumbersset$ be an integer that represents a degree.
    Let $\epsilonlimitedrankbias > 0$, and let $\funcdef{\rankbiasfunc}{[\epsilonlimitedrankbias, \infty]}{\naturalnumbersset}$ be a limited-relative-rank-bias function.
    \newline
    Let $\variety \subseteq \field$ be a set with the following properties
    \begin{enumerate}
        \item $\variety$ is $\degree$-lift-enabler with a lift operator $\lift{\square}$.
        \item $\variety$ has the $(\rankbiasfunc, \basefield, \degree, \epsilonlimitedrankbias)$-relative-rank-bias property.
    \end{enumerate}
    Then, for every $\epsilon \geq c_1(\basefield, \degree, \rankbiasfunc, \epsilonlimitedrankbias)$ it holds:
    \[
        \listpolycount{\basefield}{\variety}{\degree}{\normalizedcodedistanceex{\basefield}{\field}{\degree} - \epsilon} \leq
        c_2(\basefield, \degree, \rankbiasfunc, \epsilon)
    \]
\end{theorem}
\begin{proof}
    We follow the lines of the proof of~\cite[Theorem 1]{bhowmick2014list}.
    Let $\basefield$ be a finite field, and let $\degree \in \naturalnumbersset$ be an integer that represents a degree.
    Let $\epsilonlimitedrankbias > 0$, and let $\funcdef{\rankbiasfunc}{[\epsilonlimitedrankbias, \infty]}{\naturalnumbersset}$ be a limited-rank-relative-bias function.
    Let $c_1(\basefield, \degree, \rankbiasfunc, \epsilonlimitedrankbias)$ be a function we will specify later.
    Let $\variety \subseteq \field$ be a set with the properties defined above.
    \newline
    Finally, let $\epsilon \geq c_1(\basefield, \degree, \rankbiasfunc, \epsilonlimitedrankbias)$ for $c_1$ that we will specify later, and let $\funcdef{\onvarfunc}{\variety}{\basefield}$ be a received word.
    We wish to bound the amount of polynomials in $\allpolyset{\leq \degree}{\variety}{\basefield}$ that are $(\normalizedcodedistance{\basefield}{\degree} - \epsilon)$-close to $\onvarfunc$.
    \newline
    Apply Lemma~\ref{every-function-can-be-approximated-by-a-few-functions} with $A = \variety$, $B = \basefield$, $\genfunc[2] = \onvarfunc$, $\genfuncset = {\allpolyset{\leq \degree}{\variety}{\basefield}}$
    and approximation parameter $\epsilon / 2$ to obtain $\onvarpolyset[3] \subset \allpolyset{\leq \degree}{\variety}{\basefield}$, defined by $\onvarpolyset[3] = (\onvarpoly[3]_1,...,\onvarpoly[3]_c)$ where $c \leq 4/\epsilon^2$,
    such that for every $\onvarpoly \in \allpolyset{\leq \degree}{\variety}{\basefield}$ there is a function $\funcdef{\Gamma_{\onvarpoly}}{\basefield^c}{\basefield}$ that approximates $\onvarfunc$ in $\variety$ relative to $\allpolyset{\leq \degree}{\variety}{\basefield}$ i.e.:
    \[
        \forall \onvarpoly \in \allpolyset{\leq \degree}{\variety}{\basefield}: \prex{x \in \variety}{\Gamma_{\onvarpoly}(\onvarpoly[3]_1(x),...,\onvarpoly[3]_c(x)) = \onvarpoly(x)} \geq \prex{x \in \variety}{\onvarfunc(x) = \onvarpoly(x)} - \epsilon / 2
    \]
    \newline
    Let $\funcdef{\rankfunc_1, \rankfunc_2}{\naturalnumbersset}{\naturalnumbersset}$ be two non-decreasing functions that represents rank that we will specify later.
    For $\rankfunc_1$, we will require that for all $m \geq 1$:
    \[
        \rankfunc_1(m) \geq \max { \set{
        \topbot{
            {\rankfunc_2(C_{\rankfunc_2, \degree}^{\ref{theorem:regularization-in-X}}(m + 1)) + C_{\rankfunc_2, \degree}^{\ref{theorem:regularization-in-X}}(m + 1) + 1,}}
            {\rankfunc_2(C_{\rankfunc_{\ref{preserving-degree-starting-field}}, \degree}^{\ref{theorem:regularization-in-X}}(m + 1))
            + C_{\rankfunc_{\ref{preserving-degree-starting-field}}, \degree}^{\ref{theorem:regularization-in-X}}(m + 1) + 1}
        }}
    \]
    Note that in the expression above, we denote $\funcdef{\rankfunc_{\ref{preserving-degree-starting-field}}}{\naturalnumbersset}{\naturalnumbersset}$,
    as follows: $\rankfunc_{\ref{preserving-degree-starting-field}}(c) \definedas \rankfunc_{\ref{preserving-degree-starting-field}}(\basefield, \degree, c)$.
    \newline
    The reason we chose this $\rankfunc_1$, is that by our choice of $\rankfunc_1$ we can use the second part of Lemma~\ref{theorem:regularization-in-X}.
    Specifically, if we start with $\rankfunc_1$-$\variety$-regular factor and we $\rankfunc_2$-$\variety$-regularize it,
    we get that the $\rankfunc_2$-$\variety$-regular factor that we received is a syntactic refinement of the $\rankfunc_1$-$\variety$-regular factor we started with.
    \newline
    As a first step, we lift the polynomial factor to get $\genpolyset[3] \definedas \lift{\onvarpolyset[3]}$.
    Note that because $\forall x \in \variety: \lift{\onvarpoly[3]_i}(x) = \onvarpoly[3]_i(x)$, for all $\onvarpoly \in F$ we have:
    \[
        \prex{x \in \variety}{\Gamma_{\onvarpoly}(\lift{\onvarpoly[3]_1}(x),...,\lift{\onvarpoly[3]_c}(x)) = \onvarpoly(x)} \geq \prex{x \in \variety}{\onvarfunc(x) = \onvarpoly(x)} - \epsilon / 2
    \]
    Next, we $\rankfunc_1$-$\variety$-regularize the factor $\genpolyset[3]$ by Theorem~\ref{theorem:regularization-in-X}.
    This gives us a $\rankfunc_1$-$\variety$-regular factor $\factor^\prime$, which is defined by a set of polynomials $\genpolyset[3]^\prime \definedas (\genpoly[3]^\prime_1,...,\genpoly[3]^\prime_{c^\prime})$ of degree $\leq \degree$
    such that $\factor^\prime \relsemrefine{\variety} \factor$,
    with $\relrank{\variety}{\genpolyset[3]^\prime} \geq \rankfunc(c^\prime)$
    and with bounded amount of polynomials defining it i.e. $c^\prime \leq C_{\rankfunc_1, \degree}^{\ref{theorem:regularization-in-X}}(c)$.
    We apply Corollary~\ref{relative-semantic-refinement-is-restricted-semantic-refinement}
    and get that $\factor^{\prime} \semrefineex{\variety} \factor$.
    We then use the fact that $\Gamma_{\onvarpoly}(\lift{\onvarpoly[3]_1}(x),...,\lift{\onvarpoly[3]_c}(x))$ is measurable in respect of $\genpolyset$ \emph{in $\variety$},
    and deduce we have a similar approximation of $\onvarpoly$ using $\genpolyset^\prime$ as the approximation of $\onvarpoly$ using $\genpolyset$.
    Formally, there exists a function $\funcdef{\Gamma_{\onvarpoly}^\prime}{\basefield^{c^\prime}}{\basefield}$ such that:
    \[
        \prex{x \in \variety}
        {\Gamma^{\prime}_{\onvarpoly}(\genpoly[3]^{\prime}_1(x)),...,\genpoly[3]^{\prime}_{c^\prime}(x)) = \onvarpoly(x)} \geq
            \prex{x \in \variety}{\onvarfunc(x) = \onvarpoly(x)} - \epsilon / 2
    \]
    Now we recall that we wished to bound the amount of polynomials $\onvarpoly \in \allpolyset{\leq \degree}{\variety}{\basefield}$ such that
    $\prex{x \in \variety}{\onvarfunc(x) \neq \onvarpoly(x)} < \normalizedcodedistance{\basefield}{\degree} - \epsilon$.
    Let $\onvarpoly \in \allpolyset{\leq \degree}{\variety}{\basefield}$ be a polynomial as we just described.
    We will show that such $\onvarpoly$ is measurable with respect to $\genpolyset[3]^\prime$ in $\variety$.
    This will upper bound the amount of possible polynomials $\onvarpoly$ by the amount of possible different $\funcdef{\Gamma^{\prime}_{\onvarpoly}}{\basefield^{c^\prime}}{\basefield}$,
    which is $\abs{\basefield}^{\norm{\factor^\prime}} = \basefieldsize^{(\basefieldsize^{c^\prime})}$, and thus $c_2(\basefield, \degree, \rankbiasfunc, \epsilon) \leq \basefieldsize^{(\basefieldsize^{c^\prime})}$.
    \newline
    By our choice of $c^\prime$ we have that $c^\prime \leq C_{\rankfunc_1, \degree}^{\ref{theorem:regularization-in-X}}(4/\epsilon^2)$, and thus $c_2$ is bounded by a function of $(\basefield, \degree, \rankfunc_1, \epsilon)$.
    Note that we have not yet specified the value of $\rankfunc_1$, because it is determined by the choice of $\rankfunc_2$ that we will later define its exact values.
    The important thing about our future choice of $\rankfunc_2$ is that the value of $\rankfunc_2$ must be independent of $\blocklength$,
    but can depend on $(\basefield, \degree, \rankbiasfunc, \epsilon)$.
    This will conclude the proof.
    \newline
    Now, consider a lift of $\onvarpoly$, i.e $\genpoly \definedas \lift{\onvarpoly}$.
    Note that by the definition of lift $\forall x \in \variety: \genpoly(x) = \onvarpoly(x)$.
    We will show that $\genpoly$ is measurable in respect of $\genpolyset[3]^\prime$ in $\variety$.
    \newline
    We consider the factor $\factor_{\genpoly}$ that is generated by $\genpolyset[3]_{\genpoly} \definedas \genpolyset[3]^\prime \cup \set{\genpoly}$.
    By using Theorem~\ref{theorem:regularization-in-X}, we can $\rankfunc_2$-$\variety$-regularize it and get the polynomial factor $\factor^{\prime\prime}$ that relative-refines $\factor_{\genpoly}$.
    We denote the set of polynomials in the factor as $\genpolyset[3]^{\prime\prime}$.
    \newline
    Next, notice that the factor $\factor^{\prime\prime}$ is a $\rankfunc_2$-regular factor, therefore by our choice of $\rankfunc_1$ and the second part of Theorem~\ref{theorem:regularization-in-X},
    we in fact have $\factor^{\prime\prime} \synrefine \factor^{\prime}$.
    This is true because by our choice of $\rankfunc_1$:
    \[
        \relrank{\variety}{\genpolyset[3]^\prime} \geq
        \rankfunc_1(c^\prime) \geq
        \rankfunc_2(C_{\rankfunc_2, \degree}^{\ref{theorem:regularization-in-X}}(c^\prime + 1)) + C_{\rankfunc_2, \degree}^{\ref{theorem:regularization-in-X}}(c^\prime + 1) + 1 \geq
        \rankfunc_2(\abs{\factor^{\prime\prime}}) + \abs{\factor^{\prime\prime}}+1
    \]
    And as rank is always bigger than relative rank, we also have:
    \[
        \rank{\genpolyset[3]^\prime} \geq
        \rankfunc_1(c^\prime) \geq
        \rankfunc_2(C_{\rankfunc_{\ref{preserving-degree-starting-field}}, \degree}^{\ref{theorem:regularization-in-X}}(c^\prime + 1))
        + C_{\rankfunc_{\ref{preserving-degree-starting-field}}, \degree}^{\ref{theorem:regularization-in-X}}(c^\prime + 1) + 1
    \]
    Thus, the polynomials defining $\factor^{\prime\prime}$ are in the form $\genpolyset[3]^{\prime\prime} \definedas \genpolyset[3]^{\prime} \cup \set{\genpoly[3]^{\prime\prime}_1,...,\genpoly[3]^{\prime\prime}_{c^{\prime\prime}}}$.
    Note that as promised in Theorem~{\ref{theorem:regularization-in-X}}, we have $\abs{\genpolyset[3]^{\prime\prime}} = c^\prime+c^{\prime\prime} \leq C^{\ref{theorem:regularization-in-X}_{\rankfunc_2, \degree}}(c^\prime)$.
    \newline
    Additionally, by the way we built $\genpolyset[3]_{\genpoly}$, the function $\genpoly$ is measurable in respect of it.
    Therefore, as $\factor^{\prime\prime} \relsemrefine{\variety} \factor_{\genpoly}$, we have that $\genpoly$ is $\genpolyset[3]^{\prime\prime}$-measurable relative to $\variety$.
    In other words, there exists $\funcdef{\Phi}{\basefield^{c^\prime + c^{\prime\prime}}}{\basefield}$
    and $\funcdef{\relativeremainder{\genpoly}}{\field}{\basefield}$ with $\deg(\relativeremainder{\genpoly}), \deg(\genpoly - \relativeremainder{\genpoly}) \leq \deg(\genpoly) \leq \degree$ and $\restrictfunc{\relativeremainder{\genpoly}}{\variety} \equiv 0$ such that:
    \[
        \forall a \in \field: \genpoly(a) = \Phi(\genpoly[3]^\prime_1(a),...,\genpoly[3]^\prime_{c^\prime}(a), \genpoly[3]^{\prime\prime}_1(a),...,\genpoly[3]^{\prime\prime}_{c^{\prime\prime}}(a))) + \relativeremainder{\genpoly}(a)
    \]
    And specifically in $\variety$ we have:
    \[
        \forall x \in \variety: \genpoly(x) = \Phi(\genpoly[3]^\prime_1(x),...,\genpoly[3]^\prime_{c^\prime}(x), \genpoly[3]^{\prime\prime}_1(x),...,\genpoly[3]^{\prime\prime}_{c^{\prime\prime}}(x)))
    \]
    Denote $\genpoly^{\prime} \definedas \genpoly - \relativeremainder{\genpoly}$.
    We will show the polynomial $\genpoly^{\prime}$ does not depend on its last $c^{\prime\prime}$ variables, and thus $\Phi$ does not depend on its last $c^{\prime\prime}$ variables.
    This will imply that $\genpoly$ is measurable in respect of $\genpolyset[3]^\prime$ in $\variety$, which will conclude the proof.
    \newline
    Now, we choose $\rankfunc_2$ to be such that:
    \[
        \rankfunc_2(m) \geq \max \set {
            \rankbiasfunc \parens{\dfrac{\epsilon / 4}{\abs{\basefield}^m}},
            \rankval_{\ref{high-rank-implies-low-bias}}\parens{\dfrac{\epsilon / 4}{\abs{\basefield}^m}},
            \rankval_{\ref{preserving-degree-starting-field}}(m)
        }
    \]
    Note that in the expression above we are discussing fixed field and degree, i.e. $\basefield, \degree$.
    Therefore we denote $\funcdef{\rankfunc_{\ref{preserving-degree-starting-field}}}{\naturalnumbersset}{\naturalnumbersset}$
    as $\rankfunc_{\ref{preserving-degree-starting-field}}(c) \definedas \rankfunc_{\ref{preserving-degree-starting-field}}(\basefield, \degree, c)$
    and $\funcdef{\rankfunc_{\ref{high-rank-implies-low-bias}}}{\naturalnumbersset}{\naturalnumbersset}$
    as $\rankfunc_{\ref{high-rank-implies-low-bias}}(\epsilon) \definedas \rankfunc_{\ref{high-rank-implies-low-bias}}(\basefield, \degree, \epsilon)$.
    \newline
    Next, we show that even if we change the polynomials in the factor to have a disjoint set of inputs in $\field$,
    we still obtain a polynomial in the same degree, which have an approximation close to the approximation we had in $\variety$.
    Note that after this step, the proof becomes very similar to the proof of list decoding Reed Muller in $\field$~\cite[Theorem 1]{bhowmick2014list}:
    we omit the dependence of $\variety$ and get the same approximation by functions of multiple variables,
    as we had in $\field$.
    This is done by the following lemma:
    \begin{lemma}
        Let $\set{a^{i}, b^{j}}, i \in [c^\prime], j \in [c^{\prime\prime}]$  be pairwise disjoint sets of $\blocklength$ variables each.
        Let $\blocklength^{\prime} \definedas \blocklength(c^\prime + c^{\prime\prime})$.
        Let $\funcdef{\vec{\genpoly}}{\basefield^{\blocklength^\prime}}{\basefield}$ and $\funcdef{\vec{\onvarfunc}}{\basefield^{\blocklength^{\prime}}}{\basefield}$
        be functions of $\blocklength^\prime$ variables defined as follows:
        \[
            \vec{\genpoly^{\prime}}(\vec{a}) \definedas
                \Phi \parens{\genpoly[3]^\prime_1(a^1),...,\genpoly[3]^\prime_{c^\prime}(a^{c^\prime}), \genpoly[3]^{\prime\prime}_1(b^{1}),...,\genpoly[3]^{\prime\prime}_{c^{\prime\prime}}(b^{c^{\prime\prime}})}
        \]
        and:
        \[
            \vec{\onvarfunc}(\vec{a}) \definedas \Gamma^{\prime}_{\onvarpoly}(\genpoly[3]^{\prime}_1(a^{1})),...,\genpoly[3]^{\prime}_{c^\prime}(a^{c^\prime}))
        \]
        Note that $\vec{\onvarfunc}$ is a function that receives $\blocklength^\prime$ variables, and ignores its last $c^{\prime\prime}$ variables.
        \newline
        Then:
        \begin{enumerate}
            \item The degree of $\vec{\genpoly^\prime}$ remains bounded, i.e. $\deg(\vec{\genpoly^\prime}) \leq \degree$.
            \item The approximation of $\vec{\onvarfunc}$ to $\vec{\genpoly^\prime}$ in $\basefield^{n^\prime}$ is close to the approximation of $\Gamma^{\prime}_{\onvarpoly}$ to $\onvarpoly$ in $\variety$.
            Specifically, we show:
            \[
                \abs{
                    \prex{\vec{a} \in \basefield^{\blocklength^\prime}}{\vec{\onvarfunc}(\vec{a}) = \vec{\genpoly^\prime}(\vec{a})} -
                    \prex{x \in \variety}{\Gamma^{\prime}_{\onvarpoly}(\genpoly[3]^{\prime}_1(x)),...,\genpoly[3]^{\prime}_{c^\prime}(x)) =\onvarpoly(x)}
                }
                \leq \epsilon/4
            \]
        \end{enumerate}
        \begin{proof}
            We start by proving the first part of the lemma: bounding the degree of $\vec{\genpoly^\prime}$ by $\degree$.
            First, we recall that $\genpoly^{\prime} = \genpoly - \relativeremainder{\genpoly}$ where $\relativeremainder{\genpoly}$ is a valid remainder.
            Specifically, we have $\deg(\genpoly^{\prime}) = \deg(\genpoly - \relativeremainder{\genpoly}) \leq \deg(\genpoly) \leq \degree$.
            In addition, by the way we built $\Phi$ we have:
            \[
                \forall a \in \field: \genpoly^\prime(a) = \Phi(\genpoly[3]^\prime_1(a),...,\genpoly[3]^\prime_{c^\prime}(a), \genpoly[3]^{\prime\prime}_1(a),...,\genpoly[3]^{\prime\prime}_{c^{\prime\prime}}(a)))
            \]
            Thus the function above is of degree $\leq \degree$.
            Moreover, we have:
            \[
                \rank{\genpolyset[3]^{\prime\prime}} \geq
                \relrank{\variety}{\genpolyset[3]^{\prime\prime}} \geq
                \rankfunc_2(\abs{\genpolyset[3]^{\prime\prime}})\geq
                \rankfunc_{\ref{preserving-degree-starting-field}}(\abs{\genpolyset[3]^{\prime\prime}})
            \]
            Therefore we can use Lemma~\ref{preserving-degree-starting-field} to get that $\deg(\vec{\genpoly^{\prime}}) \leq \deg(\genpoly^\prime) \leq \degree$.
            Note that in order to use the lemma formally,
            we had to extend the input space of $\genpoly^{\prime}$ to be of $\blocklength^{\prime}$ variables (and make it depend only on the first $\blocklength$ variables as it used to).
            Because lemma~\ref{preserving-degree-starting-field} require bounds independent of $\blocklength$, this is done smoothly.
            \newline
            Now we move to the second part of the lemma: bounding the approximation of $\vec{\onvarfunc}$ to $\vec{\genpoly^\prime}$.
            Denote $S \definedas \basefield^{c^\prime + c^{\prime\prime}}$, and for each $s \in S$ denote:
            \[
                p_1(s) \definedas \prex{x \in \variety}{
                \parens{\genpoly[3]^\prime_1(x),...,\genpoly[3]^\prime_{c^\prime}(x), \genpoly[3]^{\prime\prime}_1(x),...,\genpoly[3]^{\prime\prime}_{c^{\prime\prime}}(x)} = s}
            \]
            and as of our choice of $\rankfunc_2$, we have $\rank{\genpolyset[3]^{\prime\prime}} \geq \rankbiasfunc(\epsilon/8\abs{S})$.
            Therefore, if we require that the relative rank-bias relation holds for $\epsilon / 8\abs{S}$, we can use Lemma~\ref{every-linear-combination-has-low-bias-implies-equidistribution} with $A = \variety$
            to get that $p_1$ is $(\epsilon /8\abs{S})$-almost uniform, i.e:
            \[
                p_1(s) = \dfrac{1 \pm \epsilon / 8}{\abs{S}}
            \]
            We show that this can be done in the following claim by choosing a proper $c_1$:
            \begin{claim}
                One can choose $c_1 \definedas c_1(\basefield, \degree, \rankbiasfunc, \epsilonlimitedrankbias)$ such that if $\epsilon \geq c_1$
                we have that $\epsilon / 8\abs{S} \geq c_1$.
            \end{claim}
            \begin{proof}
                This is done by using the bound we already know.
                We need that:
                \[
                    \epsilonlimitedrankbias \leq \dfrac{\epsilon}{8 \abs{\basefield}^{c^\prime + c^{\prime\prime}}}
                \]
                As $c^\prime + c^{\prime\prime} \leq C^{\ref{theorem:regularization-in-X}}_{\rankfunc_2, \degree}(c^\prime)$,
                for the term above to hold it is enough that the following will be true:
                \[
                    \epsilon \geq \epsilonlimitedrankbias \cdot 8 \abs{\basefield}^{C^{\ref{theorem:regularization-in-X}}_{\rankfunc_2, \degree}(c^\prime)}
                \]
                and as $\rankfunc_2, c^\prime$ and thus also $C^{\ref{theorem:regularization-in-X}}_{\rankfunc_2, \degree}(c^\prime)$ are independent of $\blocklength$,
                we can pick $c_1 = c_1(\basefield, \degree, \rankbiasfunc, \epsilonlimitedrankbias)$ and get what we aimed for.
            \end{proof}
            Thus, we can assume that $p_1$ is $(\epsilon/8\abs{S})$-almost uniform.
            Now, let:
            \[
                p_2(s) \definedas \prex{\vec{a} \in \basefield^{\blocklength^\prime}}
                {\parens{\genpoly[3]^\prime_1(a^1),...,\genpoly[3]^\prime_{c^\prime}(a^{c^\prime}), \genpoly[3]^{\prime\prime}_1(b^{1}),...,\genpoly[3]^{\prime\prime}_{c^{\prime\prime}}(b^{c^{\prime\prime}})} = s}
            \]
            Note that the rank of $\vec{\genpolyset[3]}^{\prime\prime} = {\set{\genpoly[3]^\prime_1(a^1),...,\genpoly[3]^\prime_{c^\prime}(a^{c^\prime}), \genpoly[3]^{\prime\prime}_1(b^{1}),...,\genpoly[3]^{\prime\prime}_{c^{\prime\prime}}(b^{c^{\prime\prime}})}}$,
            as a factor defined over $\basefield^{\blocklength^\prime}$, can not be lower than the rank of $\genpolyset[3]^{\prime\prime}$
            and thus we have $\rank{\vec{\genpolyset[3]}^{\prime\prime}} \geq \rankval_{\ref{high-rank-implies-low-bias}}\parens{\dfrac{\epsilon / 8}{\abs{\basefield}^m}}$.
            By using lemma~{\ref{high-rank-implies-low-bias}}, which shows the rank-bias relation for $\basefield^{\blocklength^\prime}$,
            we can similarly use Lemma~\ref{every-linear-combination-has-low-bias-implies-equidistribution} with $A = \basefield^{\blocklength^\prime}$
            to get that $p_2$ is also $(\epsilon/8\abs{S})$-almost-uniform, i.e:
            \[
                p_2(s) = \dfrac{1 \pm \epsilon / 8}{\abs{S}}
            \]
            Now, we show the approximations are the same.
            Denote by $s^\prime$ the restriction of $s$ to its first $c^\prime$ coordinates, and consider the approximation:
            \begin{flalign*}
                \prex{\vec{a} \in \basefield^{\blocklength^\prime}}{\vec{\onvarfunc}(\vec{a}) = \vec{\genpoly}^\prime (\vec{a})} = \\
                &=\sum_{s \in S} {p_2(s) \cdot \existfunc{\Phi(s) = \Gamma_{\genpoly}^\prime (s^\prime)}} \\
                &=\sum_{s \in S} {p_1(s) \cdot \existfunc{\Phi(s) = \Gamma_{\genpoly}^\prime (s^\prime)}} \pm \epsilon / 4 \\
                &=\prex{x \in \variety}{\Gamma^{\prime}_{\onvarpoly}(\genpoly[3]^{\prime}_1(x)),...,\genpoly[3]^{\prime}_{c^\prime}(x)) =\onvarpoly(x)} \pm \epsilon/4
            \end{flalign*}
            This completes the proof the lemma.
        \end{proof}

        The proof is followed by the same methods used in~\cite{bhowmick2014list}.
        We repeat if for completeness.
        We next restate a lemma proved in~\cite[Claim 4.2]{bhowmick2014list}, which is a varaiant of the Schwartz-Zippel lemma~\cite{10.1145/322217.322225,Zippel1979ProbabilisticAF}:
        \begin{lemma}\label{lemma-schwarz-zippel-for-comparing-polynomial-to-function-with-less-variables}
            Let $\degree$, $\blocklength_1$, $\blocklength_2 \in \naturalnumbersset$ be integers.
            Let $\genpoly_1 \in \allpolyset{\leq \degree}{\basefield^{\blocklength_1 + \blocklength_2}}{\basefield}$,
            and let $\funcdef{\genfunc_1}{\basefield^{\blocklength_1}}{\basefield}$ be a function.
            Assume the polynomial is $\normalizedcodedistance{\basefield}{\degree}$-close to the function, i.e:
            \[
                \prex{x_1,...,x_{\blocklength_1+\blocklength_2} \in \basefield}
                        {\genpoly_1(x_1,...,x_{\blocklength_1+\blocklength_2}) = \genfunc_1(x_1,...,x_n)} > 1 - \normalizedcodedistance{\basefield}{\degree}
            \]
            Then, $\genpoly_1$ does not depend on $x_{\blocklength_1 + 1},...,x_{\blocklength_1 + \blocklength_2}$.
        \end{lemma}
        Now, apply Lemma~\ref{lemma-schwarz-zippel-for-comparing-polynomial-to-function-with-less-variables} to
        $\genpoly_1 = \vec{\genpoly^{\prime}}$, $\genfunc_1 = \vec{\onvarfunc}$, $\blocklength_1 = \blocklength c^\prime$, $\blocklength_2 = \blocklength c^{\prime\prime}$.
        We obtain that $\vec{\genpoly^{\prime}}$ does not depend on its last $c^{\prime\prime}$ variables, and thus by denoting $C_{i} \definedas \genpoly[3]^{\prime\prime}_i(0)$ for $i \in \sparens{c^{\prime\prime}}$ we have:
        \[
            \vec{\genpoly^{\prime}}(\vec{a}) = \Phi \parens{\genpoly[3]^\prime_1(a^1),...,\genpoly[3]^\prime_{c^\prime}(a^{c^\prime}), C_1,...,C_{c^{\prime\prime}}}
        \]
        Now, for every $a \in \field$, if we substitute $a$ in the $i$-th component of $\vec{a}$ for every $i \in \sparens{c^{\prime}}$ in the equation above, we get the following is true:
        \[
            \genpoly^{\prime}(a) = \Phi \parens{\genpoly[3]^\prime_1(a),...,\genpoly[3]^\prime_{c^\prime}(a), C_1,...,C_{c^{\prime\prime}}}
        \]
        Hence $\genpoly^{\prime}$ does not depend on its last $c^{\prime\prime}$ variables.
        As explained earlier, this implies that $\genpoly$ is measurable in respect of $\genpolyset[3]^{\prime}$ in $\variety$.
        This completes the proof of the theorem.
    \end{lemma}

\end{proof}

