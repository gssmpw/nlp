\section[Preliminaries]{Preliminaries}\label{sec:preliminaries}

\subsection{Basic Definitions and Notations}\label{subsec:definitions_and_notations}
We denote by $\naturalnumbersset$ the set set of integers, i.e natural numbers (excluding 0).
For an integer $k$ we denote $\sparens{k} \definedas \set{1,2,...,k}$.
We use $y = x \pm \epsilon$ to denote $y \in \sparens{x - \epsilon, x + \epsilon}$, and similarly $y = x \mp \lambda$ to denote $y \in \sparens{x + \lambda, x - \lambda}$ (usually when $\lambda < 0$).
\newline
Fix a prime field $\basefield = \basefield_{\genprime}$.
Denote by $\abs{\cdot}$ the natural map of $\basefield$ to $\set{1,...,\genprime - 1} \in \naturalnumbersset$.
We denote the character from $\basefield$ by $\charfunc{x} \definedas e^{2 \pi i \cdot \abs{x}}$

Generally speaking and unless stated otherwise, we use the following conventions:
We use $\blocklength \in \naturalnumbersset$ to denote the number of variables in Reed-Muller code.
We use $\degree$ to denote a degree (typically the degree of the polynomials in our code), and $\variety$ to denote the subset of $\field$ we work in i.e $\variety \subseteq \field$.
Properties of the subset $\variety$ will usually be denoted with $\tilde{\square}$.
We use $\genfunc[1], \genfunc[2], \genfunc[3]$ to denote general functions with domain $\field$, and $\onvarfunc[1], \onvarfunc[2], \onvarfunc[3]$ to denote functions with domain $\variety$.
We use $\genfuncset[1], \genfuncset[2], \genfuncset[3]$ and $\onvarfuncset[1], \onvarfuncset[2], \onvarfuncset[3]$ respectively to denote sets of such functions.
Similarly, we use $\genpoly[1], \genpoly[2], \genpoly[3]$ to denote polynomials with domain $\field$, and $\onvarpoly[1], \onvarpoly[2], \onvarpoly[3]$ polynomials with domain $\variety$.
We use $\genpolyset[1], \genpolyset[2], \genpolyset[3]$ and $\onvarpolyset[1], \onvarpolyset[2], \onvarpolyset[3]$ respectively to denote sets of such polynomials.

\subsection[Polynomials in \titlefield]{Polynomials in \titlefield}\label{subsec:polynomials_in_Fn}
We start by presenting a standard definition for a polynomial over a finite field.
\begin{definition}[Polynomial: Global Definition]
    Let $\degree \in \naturalnumbersset$ be a constant.
    A function $\funcdef{\genpoly}{\field}{\basefield}$ is called a \emph{polynomial of degree $\leq \degree$} if
    it is of the following form:
    \[
        \genpoly(x_1,...,x_{\blocklength}) =
            \sum_{0 \leq \degree_1,...,\degree_{n} : \sum_{i=1}^{\blocklength}{\degree_i} \leq \degree}
                {c_{\degree_1,...,\degree_n} \prod_{i=1}^{\blocklength} x_i^{\degree_i}}
    \]
    We denote the set of all polynomials of degree $\leq \degree$ by $\allpolyset{\leq \degree}{\field}{\basefield}$.
    The value $\degree$ in the definition above is called the \emph{global degree} of the function $\genpoly$, shorthand by the \emph{degree} of $\genpoly$,
    and it is denoted by $\deg(\genpoly) = \degree$.
    \newline
    Additionally, the set of all polynomials from $\field$ to $\basefield$ of degree $\leq \degree$ is denoted by:
    \[
        \allpolyset{\leq \degree}{\field}{\basefield}
    \]
\end{definition}
Next, we present a known equivalent definition for a polynomial using derivatives.
To do so, we first define a derivative in the case of finite fields.
\begin{definition}[Derivative]
    Given a function $\funcdef{\genfunc}{\field}{\basefield}$ and $a \in \field$,
    we define the derivative of $\genfunc$ in direction $a$ as a function $\funcdef{\directionderivative_{a}\genfunc}{\field}{\basefield}$
    defined as follows:
    \[
        \directionderivative_a \genfunc(x) \definedas \genfunc(x + a) - \genfunc(x)
    \]
\end{definition}
\begin{lemma}
    Let $\degree \in \naturalnumbersset$.
    A function $\funcdef{\genfunc}{\field}{\basefield}$ is a polynomial of degree $\leq \degree$
    if and only if $\directionderivative_{a}\genfunc$ is a polynomial of degree $\leq \degree - 1$ for all $a \in \field$.
\end{lemma}
This leads us to a natural definition of a \emph{degree} of a function using derivatives.
\begin{definition}[Local Degree]
    For a function $\funcdef{\genfunc}{\field}{\basefield}$, we define its \emph{local degree},
    to be the least integer $\degree \in \naturalnumbersset$ such that
    for all $a_1,...,a_{\degree + 1}, x \in \field$:
    \[
        \directionderivative_{a_{\degree + 1}}...\directionderivative_{a_{1}} \genfunc(x) = 0
    \]
\end{definition}
In $\field$, the two definitions of degree coincide, and we get a single definition of a degree:
\begin{lemma}[Equivalance of definitions of a degree]\label{lemma:alternative-definition-for-polynomial-using-derivatives}
    Let $\funcdef{\genfunc}{\field}{\basefield}$ be a function, and let $\degree \in \naturalnumbersset$ be an integer.
    Then, the global degree of a $\genfunc$ \emph{equals} its local degree.
\end{lemma}
\begin{remark}
    We sometimes refer to the requirement that the local degree of a function is $\leq \degree$,
    as the \emph{local criteria} of degree $\leq \degree$ polynomials.
\end{remark}

\subsection[Rank-bias in \titlefield]{Rank-bias in \titlefield}
We start by defining the notion of \emph{bias}, which is a measure of how the function is far from being equidistributed (see Appendix~\ref{sec:equidistribution-of-functions} for the exact details).
\begin{definition}[Bias]
    Let $\funcdef{\genfunc}{\field}{\basefield}$.
    The \emph{bias} of the function $\genfunc$ is defined in the following way:
    \[
        \bias{\genfunc} \definedas 1 / \abs{\field} \cdot \sum_{x \in \field}{\charfunc{\genfunc(x)}}
    \]
    Moreover, for a subset $\variety \subseteq \field$, we define the \emph{bias of $\genfunc$ in $\variety$} to be:
    \[
        \relbias{\variety}{\genfunc} \definedas 1 / \abs{\variety} \cdot \sum_{x \in \variety}{\charfunc{\genfunc(x)}}
    \]
\end{definition}
Next, we present a standard definition of rank of a polynomial, which is a notion that measures how \emph{structured} is the function.
Note that low rank implies the polynomial is highly structured.
Formally we have the following definition:
\begin{definition}[Rank of a Polynomial]\label{definition:rank}
    Given a constant $\degree \in \naturalnumbersset$ and a polynomial $\genpoly$,
    the \emph{$\degree$-rank} of $\genpoly$, denoted as $\drank{\degree}{\genpoly}$ is defined to be
    the smallest integer $\rankval$ such that $\genpoly$ can be computed given $\rankval$ polynomials of degree $< \degree$.
    In other wards, we say $\drank{\degree}{\genpoly} = \rankval$ if $\rankval$ is the smallest integer such that
    there exists $\rankval$ polynomials $\genpoly[2]_1,...,\genpoly[2]_\rankval \in \allpolyset{\leq \degree - 1}{\field}{\basefield}$
    and a function $\funcdef{\Gamma}{\field}{\basefield}$ such that:
    \[
        \genpoly(x) = \Gamma \parens{\genpoly[2]_1(x),...,\genpoly[2]_\rankval(x)}
    \]
    If $\degree = 1$, then $1$-rank is defined to be $\infty$ if $\genpoly$ is non constant, and $0$ otherwise.
    \newline
    Moreover, for a polynomial $\genpoly$ of degree $\deg(\genpoly) = \degree$ we denote $\rank{\genpoly} \definedas \drank{\degree}{\genpoly}$.
    \newline
    We call such function $\Gamma$ a \emph{decomposition} or a \emph{computation} of $\genpoly$ using lower-degree polynomials.
\end{definition}
Let us now define a factor.
Note that we focus our discussion to factors in $\field$,
but define the basic definitions over a general set $U$ so they will apply for factors over a general sets.
This is necessary as we will later use them also for other sets such as $\variety \subseteq \field$.
\begin{definition}[Factor]
    Let $U$ be a set.
    A \emph{factor over $U$}, denoted by $\factor$, is simply a partition of the set $U$.
    Each subset in the partition is called an \emph{atom}.
    \newline
    A collection of function $\funcdef{\genfunc_1,...,\genfunc_c}{U}{\basefield}$ defines a factor $\factor_{\genfunc_1,...,\genfunc_c}$ over $U$
    with atoms:
    \[
        \set{u \in U \suchthat f_1(u)=b_1, ..., f_c(u)=b_c}
    \]
    for all $b_1,...,b_c \in \basefield$.
    \newline
    Additionally, we use $\factor$ to also denote the map $\factor(u) \rightarrow (\genfunc_1(u),...,\genfunc_c(u))$.
\end{definition}

\begin{notation*}
    Let $\funcdef{\genfunc_1,...,\genfunc_c}{U}{\basefield}$ be a collection of functions.
    For a factor $\factor \definedas \factor_{\genfunc_1,...,\genfunc_c}$, we denote by $\abs{\factor}$ the amount of functions that define it, i.e. $\abs{\factor} = c$.
    Moreover, we denote $\norm{\factor} \definedas \abs{\basefield}^c$, which is the maximal amount of (possibly empty) atoms.
    Additionally, the rank of the factor is defined to be the rank of the polynomials that generate it.
\end{notation*}
\begin{definition}[Polynomial Factor]
    We say a factor $\factor$ over $\field$ is a \emph{polynomial factor} if it is defined by a collection of polynomials $\funcdef{\genpoly_1,...,\genpoly_c}{\field}{\basefield}$,
    i.e. $\factor = \factor_{\genpoly_1,...,\genpoly_c}$.
    The degree of the factor, denote as $\deg(\factor)$ is the maximal degree of the polynomials $\genpoly_1,...,\genpoly_c$.
\end{definition}
Note that the notion of degree (and polynomial) are defined only for functions over $\field$,
therefore this definition is well-defined only for $U = \field$.

%TODO: Avoid using "factor" before it is defined.
\begin{definition}[Rank of a Factor]\label{definition:factor-rank}
    Let $\genpolyset$ be a collection of polynomials $\funcdef{\genpoly_1,...,\genpoly_c}{\field}{\basefield}$.
    The rank of the polynomial collection is defined as:
    \[
        \rank{\genpolyset} \definedas \min \set{\drank{\degree}{\sum_{i=1}^c{\lambda_i \genpoly_i}} \suchthat 0 \neq \vec{\lambda} \in \basefield^c, \degree = \max_{i\in\sparens{c}}{\deg(\lambda_i \genpoly_i)}}
    \]
    For a factor $\factor$ defined by a collection of polynomials $\genpolyset$, we define its rank to be the rank of the collection of polynomials defining it.
    For a non-decreasing function $\funcdef{\rankfunc}{\naturalnumbersset}{\naturalnumbersset}$, a factor $\factor$ is called $\rankfunc$-regular if its rank is at least $\rankfunc(\abs{\factor})$.
\end{definition}
\begin{note*}
    Note that in the definition above, the rank of each linear combination is calculated as the $\degree$-rank,
    where $\degree$ is the maximal degree of a polynomial that participates in the linear combination non-trivially.
    This is crucial as it ensures that a high rank factor do not have linear dependence in the largest-degree homogenous component of any of its polynomials.
\end{note*}


We now present a fundamental property of high rank polynomials, that was first proved by~\cite{green2007distribution} when $\degree < \abs{\basefield}$,
later extended to general fields by~\cite{kaufman2008worst}, and further extended also to large fields by~\cite{DBLP:journals/corr/0001L15}.
This property of high rank polynomials is that they they have low bias:
\begin{theorem}[Rank-bias in $\field$]\label{high-rank-implies-low-bias}
    Let $\basefield$ be a finite field.
    Let $\epsilon > 0$ and $\degree \in \naturalnumbersset$.
    There exists $r_{\ref{high-rank-implies-low-bias}} \definedas r_{\ref{high-rank-implies-low-bias}}(\basefield, \degree, \epsilon)$,
    such that for every degree-$\degree$ polynomial $\funcdef{\genpoly}{\field}{\basefield}$:
    if $\rank{\genpoly} \geq r_{\ref{high-rank-implies-low-bias}}$ then $\bias{\genpoly} < \epsilon$.
\end{theorem}
\begin{remark}
    This property implies that a collection of polynomials that have high rank is \emph{equidistributed}.
    See Appendix~\ref{sec:equidistribution-of-functions} for more details in this regard.
\end{remark}

%
%Moreover, high rank polynomial factors do not just have low bias, they also have low $\degree$-gowers norm:
%\begin{theorem}\cite[Theorem 1.20]{tao2011inverse}\label{high-rank-implies-low-gowers-norm}
%    Let $\basefield$ be a finite field.
%    Let $\epsilon > 0$ and $\degree \in \naturalnumbersset$.
%    There exists $r_{\ref{high-rank-implies-low-gowers-norm}} \definedas r_{\ref{high-rank-implies-low-gowers-norm}}(\basefield, \degree, \epsilon)$ such that for every non-classical polynomial $\funcdef{\genpoly}{\field}{\basefield}$
%    of degree $\leq \degree$, if $\rank{\genpoly} \geq r_{\ref{high-rank-implies-low-gowers-norm}}$ then $\gowersnorm{\degree}{\charfunc{\genpoly}} < \epsilon$
%\end{theorem}
%
%Furthermore, when looking on the gowers norm, the other direction is also true: uniform polynomials have high rank:
%\begin{theorem}\cite[Claim 7.16]{book}\label{low-gowers-norm-implies-high-rank}
%    Let $\degree, r \geq 1$ be integers.
%    There exists $\epsilon_{\ref{low-gowers-norm-implies-high-rank}}(\basefield, \degree, r)$ such that every non-classical polynomial $\funcdef{\genpoly}{\field}{\basefield}$
%    of degree $\leq \degree$, if $\gowersnorm{\degree}{\charfunc{\genpoly}} < \epsilon_{\ref{low-gowers-norm-implies-high-rank}}$ then $\rank{\genpoly} \geq r$
%\end{theorem}
%
%%TODO: Delete torus and write to F.


\subsection[Regularization in \titlefield]{Regularization in \titlefield}\label{subsec:regularization-in-Fn}
In this subsection we define the regularization process in $\field$.
Before doing so, let us present some definitions in this regard.
Note that we define the basic definitions over a general set $U$ so they will apply for factors over a general sets,
as this is necessary as we will later use them also for other sets such as $\variety \subseteq \field$.
\begin{definition}[Measureable]
    Let $U$ be a set, and let $A \subseteq U$.
    Let $\genfuncset = \set{\genfunc_1,...,\genfunc_c}$ be a collection of functions $\funcdef{\genfunc_i}{U}{\basefield}$.
    We say a function $\funcdef{\genfunc[2]}{U}{\basefield}$ is \emph{measurable in respect of $\genfuncset$ in $A$}, shorthand by \emph{$\genfuncset$-measurable in $A$},
    if there exists a function $\funcdef{\Gamma}{\basefield^c}{\basefield}$ such that:
    \[
        \forall a \in A: g(a) = \Gamma(\genfunc_1(a),...,\genfunc_c(a))
    \]
    When discussing the factor over $A$ defined by $\factor = \factor_{\genfunc_1,...,\genfunc_c}$,
    we also say $\genfunc[2]$ is \emph{measurable in resepct of $\factor$}.
    The function $\Gamma$ will be denoted as the \emph{measurement function} of $\genfunc[2]$ in respect of $\genfuncset$.
    Additionally, when $A = U$, we sometimes omit the specification of the domain, and say $\genfunc[2]$ is measurable in respect of $\genfuncset$.
    \newline
    Note that in this paper, we usually think of $U = \field$, and $A$ is either $\field$ or $\variety \subseteq \field$.
\end{definition}
\begin{remark}
    If $\genfunc[2]$ is $\genfuncset$-measurable in $A$, then every value of $\genfunc[2]$ in $A$
    can be determined by the values of $\genfunc_1,...,\genfunc_c$.
    In other words, the function $\genfunc[2]$ is constant inside every atom of the factor defined by $\genfuncset$.
\end{remark}

\begin{definition}[Syntactic Refinement]
    Let $\factor$ and $\factor^\prime$ be polynomial factors over $U$.
    We say a factor $\factor^\prime$ is a \emph{syntactic refinement} of the factor $\factor$, if the collection of functions defining $\factor$ is a subset of the set
    of functions defining $\factor^\prime$.
    We denote this property of $\factor^\prime$ by $\factor^\prime \synrefine \factor$.
\end{definition}
We now present a standard generalized definition of refinement, where we only require the atoms induced by the refined factors are sub-atoms of those that are induced by the original factor.
Note that in this refinement, we allow the refined factor to include completely different polynomials than the original factor.
\begin{definition}[Semantic Refinement]\label{def:semantic-refinement}
    Let $\factor$ and $\factor^\prime$ be polynomial factors on $U$ defined by $\genpolyset$ and $\genpolyset^{\prime}$ respectively.
    We say the factor $\factor^\prime$ is a \emph{semantic refinement} of the factor $\factor$ in $A \subseteq U$,
    if $x, y \in A$ with $\factor^\prime(x) = \factor^\prime(y)$ implies that $\factor(x)=\factor(y)$.
    We denote this property of $\factor^\prime$ by $\factor^\prime \semrefineex{A} \factor$.
    When $A = U$, we sometimes omit $A$ from the syntax and denote it with $\factor^{\prime} \semrefine \factor$
    \newline
    Note that $\factor^\prime \synrefine \factor$ implies $\factor^\prime \semrefineex{A} \factor$ for every $A \subseteq U$.
\end{definition}
\begin{remark}
    A handy property of semantic refinement is that if $\funcdef{\genfunc}{A}{\basefield}$ is $\genpolyset$-measurable,
    then it is also $\genpolyset^{\prime}$-measurable in $A$.
    Moreover, the other direction is also true:
    If every $\genpolyset$-measurable function $\funcdef{\genfunc}{A}{\basefield}$ in $A$ is also $\genpolyset^{\prime}$-measurable in $A$, then $\factor^{\prime} \semrefineex{A} \factor$.
\end{remark}

Next, we recall a lemma that was presented in~\cite[Theorem 4.1]{bhattacharyya2013locally}, that allows us, given a polynomial that is measurable by a high rank factor in $\field$,
to replace the polynomials in the measurement function to any collection of polynomials with smaller or equal degree, and preserve the degree of the original polynomial.
Note that we state the lemma under the constraint that $\degree < \abs{\basefield}$, but it is also valid for when $\degree \geq \abs{\basefield}$ with proper generalization of definitions to this case (See~\cite[Theorem 4.1]{bhattacharyya2013locally} for the exact statement).
\begin{lemma}[Preserving Degree in $\field$]\label{preserving-degree-starting-field}
Let $\degree>0$ an integer such that $\degree < \abs{\basefield}$, and let $\funcdef{\genpoly_1,...,\genpoly_c}{\field}{\basefield}$ be polynomials of degree at most $\degree$, that form a factor of rank $\geq  \rankfunc^{\ref{preserving-degree-starting-field}}(\basefield, \degree, c)$.
Assume that for $\funcdef{\Gamma}{\basefield^{c}}{\basefield}$, the function $\funcdef{\gamma}{\field}{\basefield}$ defined as $\gamma(a) \definedas \Gamma(\genpoly_1(a),...,\genpoly_c(a))$ is of $\deg(\gamma)=\degree^{\prime}$.
\newline
Then, for every collection of polynomials $\funcdef{\genpoly[2]_1,...,\genpoly[2]_c}{\field}{\basefield}$ that satisfy $\deg(\genpoly[2]_i) \leq \deg(\genpoly_{i})$,
the function $\gamma^{\prime}$ defined as $\gamma^{\prime}(a)=\Gamma(\genpoly[2]_1(a),...,\genpoly[2]_c(a))$ is a polynomial of $\deg(\gamma^{\prime})\leq\degree^{\prime}$.
\end{lemma}
Next, we restate a useful lemma from~\cite[Lemma 4.17]{DBLP:journals/corr/0001L15} that shows that under the conditions
above, $\Gamma$ is as a low-degree polynomial (with even stronger conditions).
Formally, they showed:
\begin{lemma}[Faithful Composition]
    In the case discussed above, the structure of $\Gamma$ is as follows:
    \[
        \Gamma(z_1,...,z_{c_1})  =
        \sum_{\alpha \in \sparens{\basefieldsize - 1}^{c}} {{C_{\alpha}} \cdot {\prod_{i = 1}^{c}}{z_i^{\alpha_i}}}
    \]
    where $C_{\alpha} = 0$ whenever $\sum_{i = 1}^{c_1}(\alpha_i \cdot \deg(\genpoly_i)) > \degree^\prime$.
    \newline
    In other words, this means that $\Gamma$ as a function $\funcdef{\Gamma}{\basefield^{c}}{\basefield}$,
    is a polynomial of degree $\leq \degree^{\prime}$, even when substituting its $i$-th input by any polynomial of degree $\leq \deg(\genpoly_i)$.
\end{lemma}


Finally, we restate the regularization process, that was first presented by~\cite[Lemma 2.3]{green2007distribution}.
The regularization process shows that every factor have a high-rank factor that semantically refines it,
without increasing the size of the factor too much (its new size is independent of $\blocklength$).
\begin{lemma}[Regularization in $\field$]\label{regularization-in-Fn-lemma}
    Let $\funcdef{\rankfunc}{\naturalnumbersset}{\naturalnumbersset}$ be a non-decreasing function and let $\degree \in \naturalnumbersset$.
    There exists $\funcdef{C_{\rankfunc, \degree}^{\ref{regularization-in-Fn-lemma}}}{\naturalnumbersset}{\naturalnumbersset}$ such that the following holds:
    Let $\factor$ be a factor on $\field$ defined by polynomials $\genpolyset = (\genpoly_1,...,\genpoly_c)$ where for all $i \in [c]$: $\funcdef{\genpoly_i}{\field}{\basefield}$ and $\deg(\genpoly_i) \leq \degree$.
    Then, there is an $\rankfunc$-regular factor $\factor^\prime$ defined by polynomials $\genpolyset[2] = (\genpoly[2]_1,...,\genpoly[2]_{c^\prime})$ where
    for all $i \in [c]$: $\funcdef{\genpoly[2]_i}{\field}{\basefield}$ and $\deg(\genpoly[2]_i) \leq \degree$ such that
    $\factor^\prime \semrefine \factor$ and $c^\prime \leq C_{\rankfunc, \degree}^{\ref{regularization-in-Fn-lemma}}(c)$.
    \newline
    Moreover, if $\factor \synrefine \bar{\factor}$ for some polynomial factor $\bar{\factor}$ with rank at least $\rankfunc(c^\prime)+c^\prime+1$,
    then we can require that $\factor^\prime \synrefine \bar{\factor}$.
\end{lemma}


