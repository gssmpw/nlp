\section[Relative Rank-Bias Property]{Relative Rank-Bias Property}\label{sec:relative-rank-bias-property}
In this section, we generalize the relation between rank and bias that is known for $\field$ also for $\variety \subseteq \field$.
Specifically, in Lemma~\ref{high-rank-implies-low-bias}, it was shown that high rank factors have low bias in $\field$.
We wish to define an alternative definition of rank for $\variety \subseteq \field$, called \emph{$\variety$-relative rank},
such that high $\variety$-relative rank implies low bias in $\variety$.
This type of relation (and definition) was shown previously to a few sets;
in~\cite[Theorem 1.8]{lampert2021relative} for sets $\variety = \zerofunc{\genpolyset[2]}$ where $\genpolyset[2]$ is a collection of polynomials of high rank;
and in~\cite[Theorem 1.4]{gowers2022equidistributionhighrankpolynomialsvariables} for sets $\variety = S^\blocklength$ for $S \subset \basefield$.

To understand this notion, we first introduce a simple example that demonstrates the need for a different definition of rank to achieve equidistribution properties in subsets of $\field$.
\begin{example}
    Let $\variety = \set{x \in \field \suchthat x_1 = 0}$.
    Define $\funcdef{\genpoly}{\field}{\basefield}$ by $\genpoly(x) \definedas x_1$.
    \newline
    In $\field$, $\genpoly$ has rank $\infty$ as it can not be decomposed polynomials of degree $< 1$ (constants).
    Additionally, it is perfectly equidistributed.
    This is the simplest example of the rank-bias relation in $\field$.
    \newline
    However, when restricting $\genpoly$ to $\variety$, we get $\restrictfunc{\genpoly}{\variety} \equiv 0$.
    As $0$ is a constant function, it is the \emph{least} equidistributed possible in $\variety$.
    Therefore, we see that the way we defined rank in $\field$ does not imply the desired equidistribution in $\variety$:
    we found a polynomial with high rank (infinity) that has a very high bias \emph{in $\variety$} (the maximal).
\end{example}
The reason the definition of rank in $\field$ fails to capture equidistribution even on subsets that are really similar to $\field$ (isomorphic to $\basefield^k$), is because of the following reason:
Even though our polynomial $\genpoly$ does \emph{not} have a decomposition to a few lower-degree polynomials by itself,
there \emph{exists} a \emph{$\variety$-equivalent} polynomial that has such structured decomposition.
Here, by $\variety$-equivalent we mean a polynomial in $\field$ that is bounded by the same degree bound, and is equal to $\genpoly$ in $\variety$.
In the example described above, this equivalent polynomial is the constant function $0$, and its decomposition is the trivial one (any function decomposes a constant function).
An alternative perspective which we use throughout this paper to $\variety$-equivalence is that both polynomials are equal up to a \emph{valid $\variety$-remainder}: a bounded degree polynomial that is $\equiv 0$ in $\variety$.

Generally speaking, high $\variety$-relative rank may not imply low bias in $\variety$.
Therefore, this structure-randomness relation is not true for a general subset $\variety \subseteq \field$,
but is a \emph{property} of the subset $\variety$.
Thus, we say that a subset has the \emph{relative rank-bias property} if this relation holds,
i.e. if high $\variety$-relative rank implies equidistribution in $\variety$.

Let us now formally define our definition for relative rank, inspired by the two different definitions of relative rank presented in~\cite[Definition 1.6]{lampert2021relative}
and in~\cite[Definition 1.3]{gowers2022equidistributionhighrankpolynomialsvariables}:
\begin{definition}[Relative Rank of a Polynomial]\label{def:relative-rank-of-polynomial}
    Let $\variety \subseteq \field$ and let $\degree \in \naturalnumbersset$.
    For an integer $\degree \in \naturalnumbersset$ and a polynomial $\funcdef{\genpoly}{\field}{\basefield}$, we define its \emph{$\degree$-relative rank} in respect of $\variety$ as:
    \[
        \drelrank{\degree}{\variety}{\genpoly} \definedas \min \set{\drank{\degree}{\genpoly - \relativeremainder{\genpoly}} \suchthat
        \relativeremainder{\genpoly} \in \allpolyset{\leq \deg(\genpoly)}{\field}{\basefield}, \restrictfunc{\relativeremainder{\genpoly}}{\variety} \equiv 0}
    \]
    For a polynomial $\genpoly$ of degree $\deg(\genpoly) = \degree$ we denote $\relrank{\variety}{\genpoly} \definedas \drelrank{\degree}{\variety}{\genpoly}$.
\end{definition}
\begin{definition}[$\variety$-equivalent and $\variety$-remainder]
    Moreover, we say a polynomial is \emph{$\variety$-equivalent} to $\genpoly$ if its restriction to $\variety$ is $\equiv \restrictfunc{\genpoly}{\variety}$.
    We say it is \emph{valid $\variety$-equivalent to $\genpoly$} if it is $\variety$-equivalent to $\genpoly$ and its of the \emph{same} degree of $\genpoly$.
    \newline
    Similarly, we say a polynomial is \emph{$\variety$-remainder} of $\genpoly$ if its restriction to $\variety$ is \emph{$\equiv 0$}.
    We say it is \emph{valid $\variety$-remainder} if it is $\variety$-remainder of $\genpoly$ and its of degree \emph{smaller or equal} of the degree of $\genpoly$.
    \newline
    %TODO: Denote also the equivalent polynomial and make the notations in the other sections consistent with it.
    We typically denote such polynomial as $\relativeremainder{\genpoly}$.
\end{definition}
In other words, the $\degree$-relative rank of a polynomial $\genpoly$ is the smallest $\degree$-rank of all valid $\variety$-equivalents of $\genpoly$.

\begin{note}\label{note:comparison-to-gowers-rank}
    Note that~\cite[Definition 1.3]{gowers2022equidistributionhighrankpolynomialsvariables} defines rank in a substantially different way than our definition,
    and consequentially our results will not apply to the sets they presented.
    One of the main differences in the definition of rank occurs for $\degree = 1$.
    In the definition we use for rank, the rank of every (non-constant) degree-$1$ polynomial is $\infty$, where in the definition used in~\cite{gowers2022equidistributionhighrankpolynomialsvariables} it is a finite number (which is possibly very small).
    This difference is crucial, as for example, it makes regularization according to their definition not-trivially possible,
    where it is known to be possible when rank is defined by the definition we use (Lemma~\ref{regularization-in-Fn-lemma}).
\end{note}
\begin{definition}[Relative Rank of a Factor]
    Let $\variety \subseteq \field$.
    Let $\genpolyset$ be a set of polynomials $\genpolyset = \set{\genpoly_1,...,\genpoly_c}$.
    The rank of the polynomial set $\genpolyset$ relative to the subset $\variety$ is defined as:
    \[
        \relrank{\variety}{\genpolyset} \definedas \min \set{\drelrank{\degree}{\variety}{\sum_{i=1}^c{\lambda_i \genpoly_i}} \suchthat 0 \neq \vec{\lambda} \in \basefield^c, \degree = \max_{i \in \sparens{c}}{\deg(\lambda_i \genpoly_i)}}
    \]
    For a factor $\factor$ defined by a collection of polynomials, we define its relative rank relative to $\variety$ to be the relative rank of the collection of polynomials defining it, relative to the set $\variety$.
    \newline
    For a non-decreasing function $\funcdef{\rankfunc}{\naturalnumbersset}{\naturalnumbersset}$, a factor $\factor$ is called $\rankfunc$-$\variety$-regular if its relative rank in respect to $\variety$ is at least $\rankfunc(\abs{\factor})$.
\end{definition}

\subsection[Relative Rank-Bias Property]{Relative Rank-Bias Property}
%TODO: Add introduction to this subsection
\begin{definition}[Relative Rank-Bias property]
    Let $\basefield$ be a finite field, and let $\degree \in \naturalnumbersset$ be an integer.
    Let $\funcdef{\rankbiasfunc}{\realnumbersset^{+}}{\naturalnumbersset}$ be a function that represents the rank-bias relation for a fixed $\degree, \basefield$.
    \newline
    We say a set $\variety \subseteq \field$ has the \emph{$(\rankbiasfunc, \basefield, \degree)$-relative rank-bias property} if
    for every $\epsilon > 0$,
    for every polynomial $\genpoly$ of degree $\leq \degree$ with $\relrank{\variety}{\genpoly} \geq \rankbiasfunc(\epsilon)$ we have:
    \[
        \relbias{\variety}{\genpoly} < \epsilon
    \]
\end{definition}

As an immediate corollary of Lemma~\ref{high-rank-implies-low-bias} that shows that high rank implies low bias, we have that $\variety = \field$ has the relative rank-bias property.
\begin{corollary}[$\field$ has the relative rank-bias property]
    For every finite field $\basefield$ and $\degree \in \naturalnumbersset$, let $\funcdef{\rankbiasfunc}{\realnumbersset^{+}}{\naturalnumbersset}$ defined as $\rankbiasfunc(\epsilon) \definedas \rankval_{\ref{high-rank-implies-low-bias}}(\basefield, \degree, \epsilon)$.
    Then, we have that the set $\variety = \field$ has the $(\rankbiasfunc, \basefield, \degree)$-relative rank-bias property.
\end{corollary}
\begin{proof}
    This is a simple usage of Lemma~\ref{high-rank-implies-low-bias}:
    Note that when $\variety = \field$, we have that $\relrank{\variety}{\genpoly} = \rank{\genpoly}$.
    Now, if $\genpoly$ is a polynomial of degree $\leq \degree$ and $\relrank{\variety}{\genpoly} = \rank{\genpoly} \geq \rankbiasfunc(\epsilon) = \rankbiasfunc_{\ref{high-rank-implies-low-bias}}(\basefield, \degree, \epsilon)$,
    then:
    \[
        \relbias{x \in \field}{\genpoly(x)} < \epsilon
    \]
\end{proof}


\subsection[Limited Relative Rank-Bias Property]{Limited-Relative Rank-Bias Property}\label{subsec:limited-relative-rank-bias-property}
Sometimes, however, we can not request $\variety$ to be such that high relative rank implies low bias for every $\epsilon > 0$,
but only for $\epsilon^\prime \geq \epsilon$ for some constant $\epsilon > 0$.
This leads to defining the \emph{limited relative rank-bias property}, which will be used to discuss such sets $\variety \subseteq \field$.
\newline
As we will later see, this definition raises naturally where $\variety$ is a high rank variety,
in which for the relative rank-bias property to hold for some $\epsilon > 0$, the rank of the variety should be greater than a value that is dependent of $\epsilon$.
Thus, to have the relative rank-bias property for a high rank variety but without requiring an infinitely large rank, we must limit the relative rank-bias property for $\epsilon^\prime \geq \epsilon$
We formulate the definition of this property as follows:
\begin{definition}[Limited Relative Rank-bias property]
    Let $\basefield$ be a finite field, let $\degree \in \naturalnumbersset$ be an integer, and let $\epsilon > 0$ be a constant.
    Let $\funcdef{\rankbiasfunc}{[\epsilon, \infty]}{\naturalnumbersset}$ be a function that represents the limited-relative-rank-bias relation.
    \newline
    We say a set $\variety \subseteq \field$ has the \emph{$(\rankbiasfunc, \basefield, \degree, \epsilon)$-limited-relative-rank-bias property} if
    for every $\epsilon^\prime \geq \epsilon$,
    for every polynomial $\genpoly$ of degree $\leq \degree$ with $\relrank{\variety}{\genpoly} \geq \rankbiasfunc(\epsilon^\prime)$ we have:
    \[
        \relbias{\variety}{\genpoly} < \epsilon^{\prime}
    \]
    As a convention, we denote by $\epsilonlimitedrankbias$ the $\epsilon$ such that the limited-relative-rank-bias property holds for $\variety$.
\end{definition}

%TODO: Change relative rank to be rank_X instead of rank_L.
\subsubsection[High Rank Varieities]{High Rank Varieities}
In this subsection, we are discussing specifically $\variety \subseteq \field$ that are in the form $\variety = \zerofunc{\genpolyset[2]}$ for a set of polynomials $\genpolyset[2]$ that form a high rank factor.
Let us present some known results of the relative rank-bias relation for high rank varieites:
In the scenario when we are working relative to $\variety$, the equivalent for Theorem~\ref{high-rank-implies-low-bias} is also known when we assume $\degree < char(\basefield)$, as shown in~\cite[Theorem 1.8]{lampert2021relative}:
\begin{theorem}[High relative rank implies low bias in high rank varieties]\label{high-relative-schmidt-rank-implies-low-relative-bias}
    Let $\basefield$ be a finite field and let $0 \leq \degree < char(\basefield)$.
    Let $\epsilon > 0$ be a constant, and let $\varietypolycount \in \naturalnumbersset$.
    There exist $\varietyrankval^{\ref{high-relative-schmidt-rank-implies-low-relative-bias}} = \varietyrankval^{\ref{high-relative-schmidt-rank-implies-low-relative-bias}}(\basefield, \degree, \varietypolycount, \epsilon)$
    and $\rankval^{\ref{high-relative-schmidt-rank-implies-low-relative-bias}} = \rankval^{\ref{high-relative-schmidt-rank-implies-low-relative-bias}}(\basefield, \degree, \epsilon)$ such that the following holds:
    \newline
    Let $\varpolyset = (\varpoly_1,...,\varpoly_{\varietypolycount})$ be a collection of polynomials of degrees $\leq \degree$ with $\rank{\varpolyset} \geq \varietyrankval^{\ref{high-relative-schmidt-rank-implies-low-relative-bias}}$ and let $\genpoly$ be a polynomial of degree $\leq \degree$.
    \newline
    Then, if $\relrank{\varpolyset}{\genpoly} \geq \rankval^{\ref{high-relative-schmidt-rank-implies-low-relative-bias}}$, we have:
    \[
        \relbias{\varpolyset}{\genpoly} < \epsilon
    \]
\end{theorem}
\begin{note*}
    Note that the original statements in~\cite{lampert2021relative} are stated for a different definition of rank, noted as \emph{schmidt rank}.
    In the appendix~\ref{sec:comparing-ranks} we compare the two different definitions, and show that our definition of rank is comprehensive
    enough in a sense that a polynomial with high rank also has high schmidt rank.
    Additionally, we show that for a given $\rankval \in \naturalnumbersset$, the lower bound of rank required for a polynomial to be of schmidt rank $\geq \rankval$,
    is only $c \cdot \rankval$ for some constant $c \in \naturalnumbersset$.
\end{note*}
\begin{remark}
    Note that in the original statement of theorem~\ref{high-relative-schmidt-rank-implies-low-relative-bias} as stated in~\cite[Theorem 1.8]{lampert2021relative},
    there are good bounds on the rank needed for $\varpolyset$ and $\genpoly$ for the theorem to hold.
    \newline
    Specifically, there exist constants $A(\degree), B(\degree)$ such that for an error $\epsilon = \abs{\basefield^{-s}}$,
    if $\varietyrankval^{\ref{high-relative-schmidt-rank-implies-low-relative-bias}} = A(\varietypolycount + s)^B$ and $\rankval^{\ref{high-relative-schmidt-rank-implies-low-relative-bias}} = A(1 + s)^B$,
    then we have:
    \[
        \relbias{\zerofunc{\varpolyset}}{\genpoly} < \abs{\basefield}^{-s}
    \]
    \newline
    In our proof, it is enough that the bounds on $\rankval$ and $\varietyrankval$ are independent of $\blocklength$, thus we omit the exact bounds stated above and use the statement as stated in Theroem~\ref{high-relative-schmidt-rank-implies-low-relative-bias}.
\end{remark}

\begin{remark}\label{in-high-relative-rank-implies-low-relative-bias-epsilon-increasing-rank-requirment-decreasing}
Note that both $\rankval^{\ref{high-relative-schmidt-rank-implies-low-relative-bias}}(\basefield, \degree, \epsilon)$
and $\varietyrankval^{\ref{high-relative-schmidt-rank-implies-low-relative-bias}}(\basefield, \degree, \varietypolycount, \epsilon)$
are decreasing when $\epsilon$ is increasing.
This means for example, that for all $\epsilon^\prime \geq \epsilon$,
a variety that satisfies the theorem's rank condition for $\epsilon$ also satisfies the theorem's rank condition for $\epsilon^\prime$.
Therefore, a polynomial with rank $\geq \rankval^{\ref{high-relative-schmidt-rank-implies-low-relative-bias}}(\basefield, \degree, \epsilon^\prime)$
will have a bias $< \epsilon^\prime$.
\end{remark}
As a corollary of Theorem~\ref{high-relative-schmidt-rank-implies-low-relative-bias} and Corollary~\ref{relative-schimdt-rank-equals-schmidt-rank-if-the-variety-is-of-high-degree}, we have that
high rank varieties has the limited-relative-rank-bias property.
Formally, we have:
\begin{corollary}[High Rank Varieties Have the Limited-Relative Rank-Bias Property]\label{high-rank-variety-has-limited-rank-relative-bias-property}
    Let $\basefield$ be a finite field, and let $\varietydeg \in \naturalnumbersset$ such that $0 < \varietydeg < \abs{\basefield}$.
    Let $\epsilonlimitedrankbias > 0$ be a constant which represents the desired relative rank-bias limit.
    There exists $\funcdef{\rankbiasfunc_{\ref{high-rank-variety-has-limited-rank-relative-bias-property}}}{[\epsilonlimitedrankbias, \infty]}{\naturalnumbersset}$ with $\rankbiasfunc_{\ref{high-rank-variety-has-limited-rank-relative-bias-property}} \definedas \rankbiasfunc_{\ref{high-rank-variety-has-limited-rank-relative-bias-property}}(\basefield, \varietydeg)$
    such that the following holds:
    \newline
    Let $\varietypolycount \in \naturalnumbersset$ be an integer.
    There exists $\varietyrankval_{\ref{high-rank-variety-has-limited-rank-relative-bias-property}} \definedas \varietyrankval_{\ref{high-rank-variety-has-limited-rank-relative-bias-property}}(\basefield, \varietydeg, \varietypolycount, \epsilonlimitedrankbias)$ such that
    for every $\varpolyset = (\varpoly_1,...,\varpoly_{\varietypolycount})$ polynomial factor of degree $\leq \varietydeg$ with $\rank{\varpolyset} \geq \varietyrankval$, we have:
    \newline
    The variety $\variety = \zerofunc{\varpolyset}$ has the $(\rankbiasfunc_{\ref{high-rank-variety-has-limited-rank-relative-bias-property}}, \basefield, \varietydeg, \epsilonlimitedrankbias)$-limited-relative-rank-bias property.
\end{corollary}
\begin{proof}
    Let $\basefield$ be a finite field, and let $\varietydeg \in \naturalnumbersset$ such that $0 < \varietydeg < \abs{\basefield}$.
    Let $\epsilonlimitedrankbias > 0$.
    We choose:
    \[
        \rankbiasfunc_{\ref{high-rank-variety-has-limited-rank-relative-bias-property}}(\epsilon) \definedas
        \rankval_{\ref{high-relative-schmidt-rank-implies-low-relative-bias}}(\basefield, \varietydeg, \epsilon)
    \]
    Note that for every $\epsilon$ in its domain, $\rankbiasfunc_{\ref{high-rank-variety-has-limited-rank-relative-bias-property}}$ does not depend on $\epsilonlimitedrankbias$.
    Let $\varietypolycount \in \naturalnumbersset$.
    Now, we choose:
    \[
        \varietyrankval_{\ref{high-rank-variety-has-limited-rank-relative-bias-property}}(\basefield, \varietydeg, \varietypolycount, \epsilonlimitedrankbias) \definedas
        \varietyrankval_{\ref{high-relative-schmidt-rank-implies-low-relative-bias}}(\basefield, \varietydeg, \varietypolycount, \epsilonlimitedrankbias)
    \]
    Using Theorem~\ref{high-relative-schmidt-rank-implies-low-relative-bias} that shows high rank implies low bias in $\variety$,
    and the assumption that $\epsilon \geq \epsilonlimitedrankbias$ (specifically Remark~\ref{in-high-relative-rank-implies-low-relative-bias-epsilon-increasing-rank-requirment-decreasing})
    concludes the proof.
\end{proof}

