\section{Related Work}
\subsection{Text Style Transfer}
The objective of text style transfer (TST) is to endow text with a different style (\eg, positive $\rightarrow$ negative) while preserving its semantic content. 
The traditional paradigm explicitly divides text into content and style information and then employs a target style for desired text generation ~\cite{yl-sen-con,storytrans,tst-sc2}.
Specifically, \citet{storytrans} address the task of author-style transfer and implement content-style disentanglement and stylization at the discourse level.
\citet{tst-sc2} develop a multilayer Joint Style-Content Weighed (JSCW) module along with a style consistency loss to ensure both content preservation and consistent style across generated sentences.
Recently, LLMs have shown promising results on TST through fine-tuning ~\cite{tst-ft-1, tst-ft-2}, in-context learning \cite{tst-icl-1,tst-icl-3} and promt-based editing \cite{p-r,tst-pd-2}.
However, previous methods primarily transfer the text styles related to sentiment and formality.
This study adapts Chinese literary classics into a child-friendly style, emphasizing vivid character descriptions and concise narrative structure tailored to children’s reading levels.
\begin{figure*}[]
  \centering
  \includegraphics[scale=0.23]{./model_tmp_1.pdf}
  \caption{Overview of InstructChild.
  (a) fine-grained instruction tuning, which incorporates the characters' personalities and narrative structure to fine-tune the large language model (LLM) with LoRA.
  (b) refinement with reinforcement learning, which considers a readability metric (\ie, Red-CN) as a reward to further optimize 
  the LLM to align with the children's reading level.
  (c) lookahead decoding strategy, which extends the traditional decoding strategy with the readability score during inference. } 
  \label{framework}
\end{figure*}


\subsection{Text Simplification}
Text simplification aims to reduce the complexity of the text, 
which can be categorized into two main branches (\ie, editing operations and lexical-syntactic rules).
Specifically, 
the editing operations simplify the text through various editing techniques, such as replacing difficult words and reordering sentence components \cite{edit-ts-1,edit-ts-2}. 
Moreover, some works introduce the lexical and syntactic rules for simplification \cite{ts-lsbert,child-ts}.
For instance, \citet{child-ts} incorporate the LLM with the lexical simplification model to simplify the children's story.
In contrast to text simplification, which aims to simplify vocabulary and syntax, Chinese literary classics often feature classical language that is challenging for children to understand. 
Our approach involves clearly explaining complex expressions while maintaining children's engagement through vivid character descriptions and an age-appropriate narrative structure.