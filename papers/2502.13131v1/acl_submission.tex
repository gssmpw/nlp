% This must be in the first 5 lines to tell arXiv to use pdfLaTeX, which is strongly recommended.
\pdfoutput=1
% In particular, the hyperref package requires pdfLaTeX in order to break URLs across lines.

\documentclass[11pt]{article}

% Change "review" to "final" to generate the final (sometimes called camera-ready) version.
% Change to "preprint" to generate a non-anonymous version with page numbers.
\usepackage[review]{acl}

% Standard package includes
\usepackage{times}
\usepackage{latexsym}
% For proper rendering and hyphenation of words containing Latin characters (including in bib files)
\usepackage[T1]{fontenc}
% For Vietnamese characters
% \usepackage[T5]{fontenc}
% See https://www.latex-project.org/help/documentation/encguide.pdf for other character sets

% This assumes your files are encoded as UTF8
\usepackage[utf8]{inputenc}

% This is not strictly necessary, and may be commented out,
% but it will improve the layout of the manuscript,
% and will typically save some space.
\usepackage{microtype}

% This is also not strictly necessary, and may be commented out.
% However, it will improve the aesthetics of text in
% the typewriter font.
\usepackage{inconsolata}

%Including images in your LaTeX document requires adding
%additional package(s)
\usepackage{graphicx}

% If the title and author information does not fit in the area allocated, uncomment the following
%
%\setlength\titlebox{<dim>}
%
% and set <dim> to something 5cm or larger.

\usepackage{booktabs}  % 用于三线表
\usepackage{graphicx}  % 如果需要插入图片
\usepackage{xcolor}    % 颜色支持
\usepackage{colortbl}
\usepackage{tabularx}
\usepackage{multirow}
\usepackage{multicol}
\usepackage{amsmath}
\usepackage{amssymb}
\usepackage{mathtools}
\usepackage{amsthm}
\usepackage{subcaption}
\newcommand{\cy}[1]{{\color{brown} [{\bf Chunyuan}: #1]}}




\title{Rethinking Diverse Human Preference Learning \\through Principal Component Analysis}

\author{First Author \\
  Affiliation / Address line 1 \\
  Affiliation / Address line 2 \\
  Affiliation / Address line 3 \\
  \texttt{email@domain} \\\And
  Second Author \\
  Affiliation / Address line 1 \\
  Affiliation / Address line 2 \\
  Affiliation / Address line 3 \\
  \texttt{email@domain} \\}

%\author{
%  \textbf{First Author\textsuperscript{1}},
%  \textbf{Second Author\textsuperscript{1,2}},
%  \textbf{Third T. Author\textsuperscript{1}},
%  \textbf{Fourth Author\textsuperscript{1}},
%\\
%  \textbf{Fifth Author\textsuperscript{1,2}},
%  \textbf{Sixth Author\textsuperscript{1}},
%  \textbf{Seventh Author\textsuperscript{1}},
%  \textbf{Eighth Author \textsuperscript{1,2,3,4}},
%\\
%  \textbf{Ninth Author\textsuperscript{1}},
%  \textbf{Tenth Author\textsuperscript{1}},
%  \textbf{Eleventh E. Author\textsuperscript{1,2,3,4,5}},
%  \textbf{Twelfth Author\textsuperscript{1}},
%\\
%  \textbf{Thirteenth Author\textsuperscript{3}},
%  \textbf{Fourteenth F. Author\textsuperscript{2,4}},
%  \textbf{Fifteenth Author\textsuperscript{1}},
%  \textbf{Sixteenth Author\textsuperscript{1}},
%\\
%  \textbf{Seventeenth S. Author\textsuperscript{4,5}},
%  \textbf{Eighteenth Author\textsuperscript{3,4}},
%  \textbf{Nineteenth N. Author\textsuperscript{2,5}},
%  \textbf{Twentieth Author\textsuperscript{1}}
%\\
%\\
%  \textsuperscript{1}Affiliation 1,
%  \textsuperscript{2}Affiliation 2,
%  \textsuperscript{3}Affiliation 3,
%  \textsuperscript{4}Affiliation 4,
%  \textsuperscript{5}Affiliation 5
%\\
%  \small{
%    \textbf{Correspondence:} \href{mailto:email@domain}{email@domain}
%  }
%}

\begin{document}
\maketitle
\begin{abstract}
Understanding human preferences is crucial for improving foundation models and building personalized AI systems. However, preferences are inherently diverse and complex, making it difficult for traditional reward models to capture their full range. While fine-grained preference data can help, collecting it is expensive and hard to scale. In this paper, we introduce Decomposed Reward Models (DRMs), a novel approach that extracts diverse human preferences from binary comparisons without requiring fine-grained annotations. Our key insight is to represent human preferences as vectors and analyze them using Principal Component Analysis (PCA). By constructing a dataset of embedding differences between preferred and rejected responses, DRMs identify orthogonal basis vectors that capture distinct aspects of preference. These decomposed rewards can be flexibly combined to align with different user needs, offering an interpretable and scalable alternative to traditional reward models. We demonstrate that DRMs effectively extract meaningful preference dimensions (e.g., helpfulness, safety, humor) and adapt to new users without additional training. Our results highlight DRMs as a powerful framework for personalized and interpretable LLM alignment.
\end{abstract}




% \begin{figure*}[t!]
% \vspace{-0.6cm}
%     \centering
%     \includegraphics[width=1.0\textwidth]{Fig/framework-DRM.pdf}
%     \caption{Illustration of the decomposition pipeline in DRMs.}
%     \label{fig: Framework.}
% \vspace{-0.2cm}
% \end{figure*}

\begin{figure*}[t!]
\vspace{-0.6cm}
    \centering
    \includegraphics[width=1.0\textwidth]{Fig/DRM_framework.pdf}
    \caption{Illustration of the decomposition pipeline in DRMs. In the original single-dimensional head, a prompt–response pair can be predicted incorrectly. In contrast, DRMs capture preferences along multiple dimensions, aligning more effectively with the complex and multifaceted nature of human preferences.}
    \label{fig: Framework.}
\vspace{-0.2cm}
\end{figure*}




\section{Introduction}
Reinforcement Learning from Human Feedback (RLHF) \cite{stiennon2020learning,ouyang2022training,bai2022training,team2023gemini,grattafiori2024llama3herdmodels,liu2024deepseek} has proven to be a powerful approach for fine-tuning large language models (LLMs) to serve as better general assistants \cite{team2023gemini,achiam2023gpt}, and AI agents \cite{nakano2021webgpt,wu2023autogen,zhao2024expel}. Typically, LLMs are optimized using a scalar reward model trained on human preference data, acting as a proxy for overall user satisfaction. However, this approach has two key limitations: (1) it often reflects the preferences of the majority, potentially marginalizing underrepresented groups \cite{chakraborty2024maxmin,chidambaram2024direct} and failing to capture the full diversity of human preferences, and (2) it struggles to represent the complex, multifaceted, and sometimes conflicting nature of human preferences with a single scalar reward \cite{jang2023personalized,rame2024rewarded,yang2024rewards,zhou2024beyond}.


As demand grows for more personalized LLMs \cite{zhang2024personalization}, researchers have explored ways to capture fine-grained, multidimensional human preferences. Some studies have introduced datasets that evaluate multiple aspects such as relevance, correctness, completeness, helpfulness, and harmlessness \cite{wu2023fine,wang2023helpsteer,cui2023ultrafeedback,pitis2024improving}. Building on these datasets, others have proposed multi-objective optimization methods to accommodate diverse user needs \cite{yang2024rewards,yang2024metaaligner,wang2024conditional}. However, these approaches face significant challenges: collecting fine-grained human annotations is expensive, and using GPT-generated labels can introduce biases. Considering that large-scale binary preference datasets—where users simply compare two responses—are easier to collect and more widely available, we ask the question: 
\emph{\textbf{Can we infer multidimensional human preferences directly from large-scale binary comparisons?}}

To address this question, we propose Decomposed Reward Models (DRMs), a framework that extracts fine-grained human preferences using binary comparison data. Unlike traditional probabilistic models such as Bradley-Terry (BT) \cite{bradley1952rank} or pairwise preference models \cite{jiang2023llm}, our approach represents human preferences as $d$-dimensional vectors, corresponding to the learned weights of the final linear layer in a reward model. We show that this vector-based representation establishes a connection between Principal Component Analysis (PCA) and preference learning.
Our method constructs a dataset of embeddings for all prompt-response pairs and applies PCA to the differences between chosen and rejected response embeddings. This process identifies orthogonal basis vectors, each capturing a distinct human preference direction. These basis vectors can be combined with their corresponding feature extractor to construct reward models. The framework is shown in Figure \ref{fig: Framework.}.
DRMs are particularly effective for personalized preference learning. Given a small adaptation dataset from a new user, we compute coefficients for the basis vectors and form a linear combination that best aligns with the user’s preferences. This approach allows for flexible, scalable, and interpretable preference modeling without requiring additional training.

In our experiments, we first show that the learned DRM basis vectors capture diverse human preference attributes. For instance, one vector may strongly correlate with safety, while another may reflect humor. Next, we evaluate DRM’s effectiveness in adapting to user preferences at test time. Our results show that DRMs outperform both the single-head reward model and previous test-time alignment methods based on reward ensembles. These findings emphasize DRM’s advantages in modeling and adapting to diverse human preferences. In addition, we also provide an explainability study to better understand human preferences.

In summary, our work makes three key contributions. (1) We propose a vector-based representation of human preferences, providing a more structured and interpretable approach to preference learning. (2) We establish a connection between preference learning and PCA, enabling the extraction of diverse and meaningful reward components without additional training. (3) We empirically validate DRMs, demonstrating their effectiveness in capturing diverse human preferences and adapting to new users at test time. These findings highlight DRMs as a scalable and flexible alternative to traditional reward models, with promising applications in interpretable AI and LLM personalization.





\section{Preliminary}
% \subsection{Reward Modeling}
The Bradley-Terry (BT) model \cite{bradley1952rank} is a probabilistic framework commonly used in preference learning. Given a prompt \( x \) and two responses, \( y_1 \) and \( y_2 \), the BT model defines the probability of \( y_1 \) being preferred over \( y_2 \) as:  
$$ 
P(y_1 \succ y_2 | x) = \frac{\exp(r(x,y_1))}{\exp(r(x,y_1)) + \exp(r(x,y_2))}  
$$
where \( r(x,y) \) represents the reward model, which assigns a score to a given prompt-response pair.  
Given a dataset of comparisons \(\mathcal{D} = \{(x_i, y_i^c, y_i^r)\}_{i=1}^{N}\), where \( y_i^c \) is the preferred response and \( y_i^r \) is the rejected one, the reward model is trained by maximizing the likelihood of human preferences. This results in the following objective:  
\begin{equation}  
\begin{aligned} \label{eq:bt_loss}
    \max_{\theta} \mathbb{E}_{(x_i, y_i^c, y_i^r) \sim \mathcal{D}} \left[\log \sigma \left( r_\theta(x_i, y_i^c) - r_\theta(x_i, y_i^r) \right) \right]  
\end{aligned}
\end{equation}  
where \( r_\theta(x, y) \) is the reward score parameterized by \( \theta \), and \( \sigma(\cdot) \) denotes the sigmoid function.  
By minimizing this loss, the reward model learns to assign higher scores to human-preferred responses, making it a useful proxy for human preferences.  


% \subsection{Test Time Adaptation}



\section{Methodology}\label{sec:method}

The standard approach to preference learning relies on a scalar-valued reward model, which may not adequately capture the full complexity of human preferences. To address this limitation, we first introduce a vector representation of human preference and establish its connection to PCA. Building on this, we propose a PCA-based method that decomposes human preference into multiple basis vectors, allowing any preference to be represented as a linear combination of these vectors. This approach enables a novel paradigm for diverse reward modeling without 
%requiring 
additional training.

\subsection{Vector Representation of Preferences}
In conventional reward modeling approaches, human preferences are transferred into \textit{scalar scores} using the BT models, as illustrated in Eq. \eqref{eq:bt_loss}. In practice, modern reward models are typically fine-tuned from pretrained or instruction-fine-tuned language models, with a reward head that maps hidden states to a scalar reward prediction. %Those scalar reward values fall short in capturing multiple aspects of human preferences.

In this paper, we consider a \textbf{vector representation} of human preference --- denoting \( \phi(x,y) \in \mathbb{R}^d \) as a \( d \)-dimensional feature extractor (e.g., the penultimate layer outputs), and a final linear layer \( w \) in the reward model, where  $r_{\theta}(x,y)=w^T\phi(x,y)$. 
Under this formulation, the BT objective can be rewritten as:  
\begin{equation*}
\begin{aligned}
  &\max_w  \mathbb{E}_i \left[ \log \sigma \left(w^T \phi(x_i,y_i^c) - w^T \phi(x_i,y_i^r) \right) \right]  \\
  &= \max_w \mathbb{E}_i \left[ \log \sigma \left(w^T (\phi(x_i,y_i^c) - \phi(x_i,y_i^r))\right) \right]
\end{aligned}
\end{equation*}
where \( (x_i,y_i^c), (x_i,y_i^r) \) denote the chosen and rejected samples, respectively. 

This reformulation is particularly interesting because it allows human preference to be captured using a vector \( w \in \mathbb{R}^d \) instead of relying on a large set of model parameters. Consequently, human preference can be interpreted as a direction in the \( d \)-dimensional space. When the feature space difference \( \phi(x_i, y_i^c) - \phi(x_i, y_i^r) \) aligns with such a direction, it means the response pair is aligned with human preference (i.e., \(y_i^c \succ y_i^r\) ); otherwise, it indicates a contradiction. 
Since the distances between preference vectors provide a natural way of comparing preferences, \textit{such a vector-based representation lays a foundation for a more interpretable framework for understanding human preference. }
% Additionally, this vector-based approach enables an intuitive way to compare preferences by simply measuring the distance between preference vectors. \emph{Therefore, this vector-based representation provides a more interpretable framework for understanding human preferences.}


\subsection{Rethinking Preference Learning with Principal Component Analysis (PCA)}\label{sec:rethink_pca}
The next step is selecting a \textit{good basis} to represent any vector preferences. The basis should be universal, interpretable, and have a strong connection with human preferences. Our choice of the basis is motivated by the following observations.

%Based on our representation of human preference, 
Formally, we start by defining \( z_i = \phi(x_i,y_i^c) - \phi(x_i,y_i^r) \) and letting it be \( \{z_i\}_{i=1}^{N} \) zero-centered, (i.e., $\mathbb{E}_i [z_i]=0$, which can be ensured via normalization). Without loss of generality, we consider the unified preference vector \( \|w\|_2^2 =1 \). Then the objective of reward modeling becomes
\begin{equation}
\begin{aligned}
      \max_w    \mathbb{E}_i \left[ \log \sigma \left(w^T z_i\right)\right], \quad \text{s.t.} \quad \|w\|_2^2 =1.
\end{aligned}
\end{equation}
Here, \( w \) is optimized to find a direction that best distinguishes the preference data, which bears a resemblance to PCA: both methods seek a meaningful projection of the data onto a dimension. To further explore this connection, consider the following regularized objective ($\lambda > 0$):
\begin{equation}
    J(w) = \mathbb{E}_i  \left[ \log \sigma \left(w^T z_i\right)  \right] - \lambda \|w\|_2^2.
\end{equation}
Taking the gradient, we obtain:
\begin{equation}
    \nabla_w  J(w) = \mathbb{E}_i \left[ (1-\sigma(w^T z_i)) z_i \right] - 2\lambda w.
\end{equation}
For small \( w^T z_i \), we have \( 1-\sigma(x) \approx \frac{1}{2} + c x \) according to Taylor expansion, where \( c \) is a constant. Therefore, we have
\begin{equation}
\begin{aligned}
    \nabla_w  J(w) &\approx c \mathbb{E}_i \left[(z_i z_i^T) w \right] - 2\lambda w \\
    &= c \Sigma w - 2\lambda w,
\end{aligned}
\end{equation}
where \( \Sigma = \mathbb{E}_i \left[ z_i z_i^T \right] \) is the covariance matrix of \( \{z_i\} \). Setting the gradient to zero gives:
\begin{equation}
    c \Sigma w = 2\lambda w \quad \Rightarrow \quad \Sigma w = \frac{2\lambda}{c} w.
\end{equation}
\noindent\textbf{Discussion.} This equation suggests that under certain conditions, the learned preference direction \( w \) aligns with an eigenvector of the covariance matrix \( \Sigma \). While this does not imply a direct equivalence between preference learning and PCA, it highlights an interesting connection: both methods extract a principal direction from the data. Unlike PCA, which maximizes variance in an unsupervised manner, preference learning optimizes a supervised ranking objective, making the relationship approximate rather than exact.

Additionally, while PCA eigenvectors are direction-agnostic—meaning their signs are arbitrary—human preference is inherently directional. As a result, when deriving a preference vector \( w \) from PCA, both \( w \) and \( -w \) need to be considered to ensure an accurate representation.


\subsection{From Scalar Reward to Diverse Rewards}
Our analysis above suggests a connection between the eigenvectors of the covariance matrix and human preferences. A key observation is that the covariance matrix has \( d \) eigenvectors—for example, \( d = 2048 \) for gemma-2B~\cite{team2024gemma} and \( d = 4096 \) for Llama-3.1-8B~\cite{grattafiori2024llama3herdmodels}. This means we can extract a large number of meaningful preference vectors \( w \) from PCA applied to the embedding dataset \( \{z_i\}_{i=1}^{N} \).  

These eigenvectors form an orthonormal basis in the \( d \)-dimensional space, meaning they are mutually orthogonal and span the entire space. Mathematically, the eigendecomposition of the covariance matrix is given by:  
\(
\Sigma = W \Lambda W^T
\)
where \( W = [w_1, w_2, \dots, w_d] \) is an orthonormal matrix whose columns are the eigenvectors of \( \Sigma \), \( \Lambda = \text{diag}(\lambda_1, \lambda_2, \dots, \lambda_d) \) is a diagonal matrix of eigenvalues.  
This ensures that \( w_1, w_2, \dots, w_d \) represent diverse preference directions. Any human preference can then be expressed as a linear combination of these basis vectors:  
\[
w = \sum_{i=1}^d k_i w_i,
\]
where \( k_1, \ldots, k_d \) are weight parameters. This formulation enables a flexible and expressive human preference representation as a combination of these decomposed rewards.  


\subsection{Decomposed Reward Models (DRMs)} \label{sec:drm}
As illustrated in Figure \ref{fig: Framework.}, we propose Decomposed Reward Models (DRMs) by applying PCA to a human preference dataset $\mathcal{D} = \{(x_i, y_i^c, y_i^r)\}_{i=1}^{N}$ using an embedding extractor $\phi(x,y)$. $\phi$ can be any language models (pretrained or instruction-tuned) or reward models that produce hidden states of dimension \( d \). DRMs consist of two main steps:  
(1) \emph{Embedding Extraction:} We run inference over the preference dataset \( \mathcal{D} \) using \( \phi \) to obtain a dataset of embedding differences:  
   \[
   \mathcal{D}_e = \{z_i\}_{i=1}^{N}, \quad \text{where } z_i = \phi(x_i,y_i^c) - \phi(x_i,y_i^r).
   \]
(2) \emph{Principal Decomposition:} Given the dataset \( \mathcal{D}_e \) of shape \( N \times d \), we perform PCA to obtain a set of basis vectors \( W = [w_1, w_2, \dots, w_d] \). Each eigenvector \( w_i \) captures a distinct directional human preference and can be combined with the feature extractor \( \phi(x,y) \) to construct a reward model, leading to a total of $d$ rewards.

To fully utilize DRMs, we aim to represent any human preference as a linear combination of these decomposed rewards. Various methods can achieve this, such as mixture-of-experts \cite{quan2024dmoerm,wang2024interpretable} and test-time alignment \cite{lee2024test}. Since both approaches are fundamentally similar, and test-time alignment can better leverage our method without additional fine-tuning, we adopt it as our implementation. However, \emph{DRMs are compatible with any downstream method that requires multiple diverse reward heads.}

We implement test-time alignment based on HyRe \cite{lee2024test}, which adapts to diverse human preferences by optimizing reward weights given a small adaptation set \( \mathcal{D}_{\rm adapt} \). This is practical in real-world scenarios where new users' data helps refine the strategy to better align with individual preferences. The weights \( k_i \) are computed based on the loss function over the adaptation set:  
\begin{equation}
    k_i = \frac{\exp{(-\mathcal{L}(w_i, \mathcal{D_{\rm adapt}}))}}{ \sum_j\exp{(-\mathcal{L}(w_j, \mathcal{D_{\rm adapt}}))}}
\end{equation}
where
\begin{equation}
    \begin{aligned}
        \mathcal{L}(w_i, \mathcal{D_{\rm adapt}})= - \mathbb{E}_{(x_i,y_i^c, y_i^r)\sim \mathcal{D}_{\rm adapt}} \\ \left[  \log \sigma \left(w^T (\phi(x_i,y_i^c) - \phi(x_i,y_i^r))\right) \right]
    \end{aligned}
\end{equation}
Here, we also normalize the features based on $\mathcal{D}_{\rm adapt}$ to prevent the loss $\mathcal{L}$ from being dominated by samples with large-scale features.
This formulation ensures that weights \( k_i \) dynamically adjust based on the dataset, assigning higher importance to directions \( w_i \) with lower loss and reducing the influence of those with higher loss. As a result, we can obtain a soft, weighted sum of DRMs that better aligns with the desired human preference.  


\paragraph{Advantages of DRMs.}
(1) \emph{Simplicity:} Our proposed DRMs offer a simple yet effective approach to reward modeling without requiring additional training. (2) \emph{Diversity:} Unlike traditional scalar reward models that struggle to represent heterogeneous human preferences, DRMs leverage a diverse set of basis vectors to capture a wide range of preferences. (3) \emph{Adaptivity:} By decomposing human preference data through PCA, DRMs naturally extract meaningful directional preferences, allowing them to adapt flexibly to different user needs. This adaptability can be leveraged by test-time alignment, which dynamically adjusts reward weights to better suit specific preferences. As a result, DRMs provide a scalable and interpretable solution for modeling complex and diverse human preferences.


\vspace{-5pt}
\begin{table*}[h!]
    \centering
    \renewcommand{\arraystretch}{1.2}
    \setlength{\tabcolsep}{4pt}
    \resizebox{0.95\linewidth}{!}{%
    \begin{tabular}{p{2.2cm}|l|c|c|c|c|c|c}
    \toprule
    \multirow{2}{*}{\textbf{Benchmark}} & \multirow{2}{*}{\textbf{Attribute}} & \multicolumn{3}{c|}{\textbf{Gemma-2B-RM}} & \multicolumn{3}{c}{\textbf{Llama3-8B-RM}} \\
    \cline{3-8}
    % \rowcolor{gray!15}
    & & Single Head & Max Value & Max Head & Single Head & Max Value & Max Head \\
    \midrule
    \multirow{5}{*}{RewardBench} 
    & Overall & 0.733 & \textbf{0.735} & head\_0 & 0.862 & \textbf{0.869} & head\_0 \\
    & Chat & 0.944 & \textbf{0.950} & head\_0 & 0.983 & \textbf{0.986} & head\_0 \\
    & Chat Hard & 0.467 & \textbf{0.660} & head\_3 & 0.684 & \textbf{0.695} & head\_3 \\
    & Safety & \textbf{0.759} & 0.745 & head\_0, head\_8 & 0.868 & \textbf{0.886} & head\_0 \\
    & Reasoning & 0.759 & \textbf{0.821} & head\_32 & 0.912 & \textbf{0.923} & head\_0 \\
    \midrule
    \multirow{6}{*}{RPR} 
    & Overall & 0.714 & \textbf{0.735} & head\_0 & \textbf{0.853} & 0.839 & head\_0 \\
    & User-Friendliness & 0.506 & \textbf{0.798} & head\_9, head\_26 & 0.719 & \textbf{0.899} & head\_10 \\
    & Narrative \& Storytelling & 0.662 & \textbf{0.825} & head\_12 & 0.838 & \textbf{0.912} & head\_5 \\
    & Linguistic Creativity & 0.817 & \textbf{0.885} & head\_12 & 0.875 & \textbf{0.981} & head\_37 \\
    & Scientific Rigor & \textbf{0.881} & \textbf{0.881} & head\_34 & 0.940 & \textbf{0.964} & head\_0 \\
    & Humor \& Entertainment & 0.690 & \textbf{0.964} & head\_9 & 0.893 & \textbf{0.952} & head\_37, head\_74 \\
    \bottomrule
    \end{tabular}
    }
    \vspace{-5pt}
    \caption{Performance of top 100 decomposed reward heads. ``Single Head" is the trained single-head baseline. ``Max Value" refers to the highest score achieved for each attribute, while ``Max Head"  indicates which specific head attains this maximum score. ``Overall" represents the average accuracy of a single head across all attributes.}
    \label{single_head_analysis}
    \vspace{-10pt}
\end{table*}



\section{Experiment}
In this section, we conduct extensive experiments to evaluate the effectiveness of DRMs, focusing on the \emph{diversity and interpretability} of the decomposed heads, as well as their \emph{adaptivity} to downstream human preferences.

\subsection{Experimental Setup}

\textbf{Dataset.} We choose the \textit{mixture2\_and\_safe\_pku} dataset\footnote{\url{https://huggingface.co/datasets/weqweasdas/preference_dataset_mixture2_and_safe\_pku}}~\cite{dong2024rlhf}, a collection of 550k pairwise preference samples. This dataset combines diverse data sources, including human-labeled preferences from HH-RLHF~\cite{bai2022training} and GPT-labeled preferences from UltraFeedback~\cite{cui2023ultrafeedback}, making it well-suited for studying diverse preference decomposition. 
% We utilize the same dataset for PCA to train baseline reward models.  
To evaluate the effectiveness of the decomposed reward heads, we test models on two unseen benchmarks with multiple attributes: (1) RewardBench~\cite{lambert2024rewardbench}, a dataset designed to evaluate reward models across various dimensions, including chat quality, safety, and reasoning. (2) Reasonable Preference Reversal (RPR) test set~\cite{pitis2024improving} focuses on personalized context-aware preference evaluation. From RPR, we sample five fine-grained categories (i.e., User-Friendliness, Narrative Quality, Linguistic Creativity, Scientific Rigor, and Humor), each with over 80 annotated samples. Other categories were excluded due to insufficient data for reliable evaluation.  


\noindent\textbf{Base Model.} 
For experiments on decomposed reward heads and test-time adaptation, we use two open-source reward models, \href{https://huggingface.co/Ray2333/Gemma-2B-rewardmodel}{Gemma-2B-RM} and \href{https://huggingface.co/Ray2333/GRM-llama3-8B-distill}{Llama3-8B-RM}~\cite{yang2024regularizing}, along with an instruction-tuned language model, \href{https://huggingface.co/google/gemma-2-9b-it}{gemma-2-9b-it} \cite{team2024gemma}, as our base models.  
To analyze their performance, we keep the backbone fixed as our feature extractors while generating multiple new reward heads for them. 


\noindent\textbf{Baselines.} We compare the performance of DRMs against several baselines. (1) \textit{Single-Head RM} fine-tunes a single reward head on top of a fixed backbone using the same dataset as DRMs. (2) \textit{Share-Base RM}~\cite{lee2024test} is a training-based method that employs multiple learnable reward heads while incorporating a frozen prior network\cite{osband2023epistemic} to maintain diversity, with the final output derived from their combination. (3) \textit{Random Head} uses multiple reward heads with randomly initialized weights to capture diverse but largely unstructured preferences. Specifically, we experiment with both uniform and Gaussian initialization, followed by L2 normalization to ensure consistency with DRM vectors. 

% Additional implementation details of the baseline models can be found in \textcolor{red}{Appendix}.

\begin{table*}[h]
 \vspace{-10pt}
    \centering
    \renewcommand{\arraystretch}{1.2}
    \setlength{\tabcolsep}{6pt}  % 调整列间距
    \resizebox{0.93\linewidth}{!}{  % 让表格自适应
    \begin{tabular}{p{2.45cm} l c c c c c c}  
        \toprule
        \textbf{Base Model} & \textbf{Method} & \shortstack{\textbf{User} \\ \textbf{Friendliness}} & \shortstack{\textbf{Narrative} \\ \textbf{\& Storytelling}} & \shortstack{\textbf{Linguistic} \\ \textbf{Creativity}} & \shortstack{\textbf{Scientific} \\ \textbf{Rigor}} & \shortstack{\textbf{Humor} \\ \textbf{\& Entertainment}} & \textbf{Overall} \\

        \midrule
        % ------------------- 2B 模型 -------------------
        \multirow{4}{*}{\textbf{Gemma-2B-RM}} 
        & Single Head
            & 0.506 & 0.662 & 0.817 & 0.881 & 0.690 & 0.714 \\
        & Shared-Base  
            & 0.517(0.000) & 0.688(0.000) & 0.817(0.000) & 0.881(0.000) & 0.690(0.000) & 0.721(0.000) \\
        % & Random (Kaiming)  
        %     & 0.698(0.066) & 0.838(0.027) & 0.877(0.060) & 0.802(0.053) & 0.965(0.026) & 0.836(0.022) \\
        & Random (Uniform)  
            & 0.713(0.062) & 0.782(0.068) & 0.920(0.045) & 0.907(0.043) & 0.907(0.026) & 0.848(0.024) \\
        & Random (Gaussian)  
            & 0.582(0.039) & 0.760(0.060) & 0.823(0.063) & 0.873(0.025) & 0.817(0.053) & 0.771(0.022) \\
        \rowcolor{gray!15}
        & \textbf{DRM (Ours)}  
            & \textbf{0.789(0.062)} & \textbf{0.871(0.033)} & \textbf{0.953(0.034)} & \textbf{0.907(0.019)} & \textbf{0.975(0.017)} & \textbf{0.900(0.017)} \\
        \midrule
        % ------------------- 8B 模型 -------------------
        \multirow{4}{*}{\textbf{Llama3-8B-RM}}  
        % & Single Head  
        %     & 0.719 & 0.838 & 0.875 & 0.940 & 0.893 & 0.853 \\ % original head
        & Single Head  
            & 0.685 & 0.825 & 0.846 & 0.964 & 0.905 & 0.844 \\ % fine-tune on 500k
        & Shared-Base  
            & 0.674(0.000) & 0.825(0.000) & 0.827(0.000) & 0.964(0.000) & 0.881(0.000) & 0.832(0.000) \\
        % & Random (Kaiming)  
        %     & 0.698(0.063) & 0.909(0.042) & 0.910(0.035) & 0.953(0.006) & 0.935(0.011) & 0.880(0.016) \\
        & Random (Uniform)  
            & 0.616(0.104) & 0.860(0.037) & 0.798(0.103) & 0.958(0.007) & 0.906(0.031) & 0.823(0.041) \\
        & Random (Gaussian)  
            & 0.730(0.100) & 0.892(0.027) & 0.887(0.055) & 0.956(0.008) & 0.919(0.032) & 0.875(0.028) \\
        \rowcolor{gray!15}
        & \textbf{DRM (Ours)}  
            & \textbf{0.812(0.063)} & \textbf{0.946(0.029)} & \textbf{0.945(0.015)} & \textbf{0.969(0.010)} & \textbf{0.991(0.011)} & \textbf{0.931(0.016)} \\
        \bottomrule
    \end{tabular}
    }
     \vspace{-5pt}
    \caption{Evaluation Results on RPR ($n=5$). We compare DRMs with trained baselines (``Single Head" and ``Shared-Base"), and randomly generated multi-head baselines (``Random"). Except for single-head baseline, other methods use HyRe for test-time adaptation. Standard deviation over 20 sampled adaptation sets are reported.}\label{tab:RPR_res}
    \vspace{-5pt}
\end{table*}



\begin{table*}[h]
    \centering
    \small
    \renewcommand{\arraystretch}{1.2}
    \setlength{\tabcolsep}{6pt}  % 调整列间距
    \resizebox{0.9\textwidth}{!}{
    \begin{tabular}{p{2.2cm} l c c c c c}  
        \toprule
        \textbf{Base Model} & \textbf{Method} & \textbf{Chat} & \textbf{Chat Hard} & \textbf{Safety} & \textbf{Reasoning} & \textbf{Overall} \\
        \midrule
        % --- 2B 模型 ---
        \multirow{6}{*}{\textbf{Gemma-2B-RM}} 
        % ------------------- 2B 模型 -------------------
        & Single Head 
            & 0.944 & 0.467 & 0.759 & 0.759 & 0.733 \\
        & Shared-Base 
            & 0.947(0.000) & 0.476(0.000) & 0.765(0.000) & 0.774(0.000) & 0.740(0.000) \\
        % & RandomA (Kaiming) 
        %     & 0.948(0.008) & 0.594(0.030) & 0.787(0.014) & 0.783(0.057) & 0.778(0.017) \\
        & Random (Uniform)  
            & 0.940(0.005) & 0.567(0.029) & \textbf{0.800(0.010)} & 0.843(0.019) & 0.787(0.009) \\
        & Random (Gaussian)  
            & 0.951(0.005) & 0.573(0.033) & 0.781(0.015) & 0.839(0.021) & 0.786(0.008) \\
        \rowcolor{gray!15}
        & DRMs(Ours) 
            & \textbf{0.953(0.003)} & \textbf{0.650(0.028)} & 0.783(0.030) & \textbf{0.872(0.025)} & \textbf{0.814(0.013)} \\
        \midrule
        % ------------------- 8B 模型 -------------------
        \multirow{4}{*}{\textbf{Llama3-8B-RM}}  
        % \rowcolor{gray!15} 
        % & Single Head  & 0.983 & 0.684 & 0.868 & 0.912 & 0.862 \\ % original head
        & Single Head  & \textbf{0.989} & 0.684 & 0.891 & 0.920 & 0.871 \\ % after fine-tune on 500k
        & Shared-Base   & 0.986(0.000) & 0.684(0.000) & 0.895(0.000) & 0.927(0.000) & 0.873(0.000) \\ 
        % & RandomA (Kaiming) 
        %     & 0.980(0.003) & 0.649(0.088) & 0.895(0.008) & 0.927(0.011) & 0.863(0.022) \\
        & Random (Uniform)  
            & 0.985(0.003) & 0.623(0.089) & \textbf{0.903(0.010)} & 0.915(0.014) & 0.857(0.023) \\
        & Random (Gaussian)  
            & 0.982(0.004) & 0.663(0.096) & 0.889(0.009) & \textbf{0.936(0.011)} & 0.868(0.024) \\
        \rowcolor{gray!15}
        & DRMs(Ours) & 0.986(0.002) & \textbf{0.755(0.032)} & 0.885(0.036) & 0.914(0.036) & \textbf{0.885(0.012)} \\
        \bottomrule
    \end{tabular}}
     \vspace{-5pt}
    \caption{Evaluation Results ($n=15$), comparing different methods across two base models. }\label{tab:eval_rmbench}
     \vspace{-10pt}
\end{table*}



\begin{table}[h]
    \centering
    \renewcommand{\arraystretch}{1.2}
    \setlength{\tabcolsep}{8pt}
    \resizebox{\linewidth}{!}{%
    % 原先是 {l|l|c|c},这里加一列,所以要改为 {l|l|c|c|c}
    % 并在 'Single Head' 后插入 'kaiming_uniform'
    \begin{tabular}{l|l|c|c|c}
    \toprule
    \textbf{Benchmark} & \textbf{Attributes} & \textbf{Single Head} & {\shortstack{\textbf{Random}\\ \textbf{(Uniform)}}} & \textbf{DRM(Ours)}\\
    \midrule
    \multirow{5}{*}{\cellcolor{white}\textbf{RewardBench}} 
    \rowcolor{gray!15}
        & Overall    & 0.759 & 0.770 & \textbf{0.830} \\
        & Chat       & 0.905 & 0.897 &\textbf{ 0.920 }\\
        & Chat Hard  & 0.621 & 0.600 & \textbf{0.692} \\
        & Safety     & 0.699 & 0.753 & \textbf{0.786} \\
        & Reasoning  & 0.813 & 0.832 & \textbf{0.920 } \\
    \midrule
    \multirow{6}{*}{\cellcolor{white}\textbf{RPR}} 
    \rowcolor{gray!15}
        & Overall                     & 0.746 & 0.630 & \textbf{0.796} \\
        & User-Friendliness          & 0.640 & 0.555 & \textbf{0.657} \\
        & Narrative \& Storytelling & 0.713 & 0.610 & \textbf{0.763} \\
        & Linguistic Creativity      & 0.808 & 0.595 & \textbf{0.843} \\
        & Scientific Rigor           & 0.762 & 0.661 & \textbf{0.806} \\
        & Humor \& Entertainment    & 0.798 & 0.744 & \textbf{0.905} \\
    \bottomrule
    \end{tabular}%
    }
     \vspace{-5pt}
    \caption{Performance of DRMs on \textbf{instruction-tuned model} 
    gemma-2-9b-it. Single head baseline is trained with the same dataset used for DRMs. Aligned with the previous setting, we use $n=15$ and $n=5$ for RewardBench and RPR respectively.}\label{tab:gemma-2-9b-it_exp}
     \vspace{-15pt}
\end{table}



\subsection{What information is Captured by DRMs?}
\label{single_head}
We aim to better understand the decomposed reward heads in DRMs. To achieve this, we evaluate the performance of the top 100 reward vectors, ranked by eigenvalue, on both RewardBench and RPR’s fine-grained subsets.
Table~\ref{single_head_analysis} reports the scores of a trained single-head baseline. The ``Max Value" column shows the highest score achieved for each attribute, while the ``Max Head" column indicates which reward head achieves this score. We also compare the results with the single-head baseline. The results reveal the following findings:

\noindent (1) \textbf{Diversity and Interpretability:} DRMs effectively capture diverse human preferences, with different reward heads excelling at different attributes. For instance, in Gemma-2B-RM, head\_9 performs best on ``User-Friendliness" (accuracy: 0.798) and ``Humor and Entertainment" (0.964), while head\_12 excels in ``Narrative and Storytelling" (0.825) and ``Linguistic Creativity" (0.885). In contrast, the single-head RM fails to capture this diversity, yielding suboptimal performance on attributes such as ``User-Friendliness" (0.506) and ``Humor and Entertainment" (0.690) for Gemma-2B-RM. These results indicate that DRMs not only capture a broader range of human preferences but also provide interpretable representations that align well with certain preference attributes.

\noindent (2) \textbf{The first head is the most informative:} An interesting observation is that the head head\_0 consistently achieves the highest overall accuracy for both models. This aligns with expectations, as head\_0 corresponds to the eigenvector with the largest variance, i.e., the most informative direction. Furthermore, among the top 100 heads, most of the high-performing heads appear before index 40, which aligns with PCA’s property that variance decreases as the head index increases. This finding further supports our argument that PCA can approximate preference learning.

In summary, our results show that a single reward head is insufficient to represent the full spectrum of human preferences. Instead, DRMs provide high-quality and interpretable estimations of diverse human preferences, supporting our analysis in Section \ref{sec:rethink_pca}.


\vspace{-5pt}
\subsection{Test-time Preference Adaptation}
\label{TTA}\vspace{-3pt}
A natural application of DRMs is test-time adaptation—deriving linear combinations to match new user preferences. Following the adaptation method in Section \ref{sec:drm}, we use a small subset of test data for each attribute (e.g., $n=15$ for RewardBench and $n=5$ for RPR), which corresponds to less than $4\%$ of the available data per attribute in RewardBench and less than $6\%$ for RPR. We compare DRMs against several baselines, including the single-head and ensemble-head baselines trained on the same dataset, and two random-head baselines that sample random heads. To ensure efficiency, we limit all models, including DRMs, to using only 100 heads. The results for the two benchmarks, shown in Table~\ref{tab:RPR_res} and Table~\ref{tab:eval_rmbench}, include the standard deviation over 20 repetitions of sampled adaptation sets.


The results demonstrate that DRMs achieve the best overall performance across different base models and test sets. The improvement is particularly significant for Gemma-2B-RM, where DRMs improve the single-head baseline from 0.733 to 0.814 on RewardBench and from 0.714 to 0.90 on RPR. Interestingly, the ``Shared-Base" ensemble baseline does not outperform the single-head baseline, suggesting that it lacks diversity in its learned reward heads. In contrast, the random-head baseline offers some improvement over the single-head baseline but remains inferior to DRMs. This confirms that DRMs provide a diverse and well-structured set of basis vectors that enable efficient test-time adaptation to user preferences.

Another important question is whether a language model can serve as a feature extractor instead of a reward model. To explore this, we conduct experiments using Gemma-2-9B-it as the feature extractor. The results, presented in Table~\ref{tab:gemma-2-9b-it_exp}, show that DRMs can successfully exploit a language model for this purpose, outperforming the single-head trained baseline by 7.8\% on RewardBench and 26\% on RPR. Furthermore, comparing results across models reveals that while DRMs perform well with language models as feature extractors, reward models remain a more effective choice. 


\begin{figure}[t]
\vspace{-10pt}
    \centering
    \includegraphics[trim={0 0cm 0 2cm}, clip, width=1\linewidth]{Fig/TTA_weight_rb.pdf}
    \vspace{-15pt}
    \caption{Weight distributions of the top 100 decomposed reward heads on RewardBench for DRMs using Gemma-2B-RM as the backbone.}
    \label{fig:weight_distribution}
    \vspace{-15pt}
\end{figure}


\begin{figure}[t]
\vspace{-15pt}
    \centering
    \includegraphics[trim={2cm 0.5cm 0 0}, clip, width=1.2\linewidth]{Fig/correlation.pdf}
    \vspace{-20pt}
    \caption{Correlation among weight vectors for DRMs. The feature extractor is Gemma-2B-RM.}
    \label{fig:Correlation}
    \vspace{-15pt}
\end{figure}



\subsection{Quantitative Attribute Explainability}\label{sec:attribute_interp}
Beyond its adaptability, DRMs offer a significant advantage in interpretability, helping not only to understand human preferences but also to analyze the multiple attributes present in current test sets. Specifically, for each attribute subset, we can obtain the weight parameters $\textbf{k} = [k_1, \ldots,k_d]$ corresponding to each basis vector. Figure~\ref{fig:weight_distribution} visualizes some of these weight vectors in RewardBench, revealing distinct patterns. The \emph{Chat} subset primarily relies on the first few basis vectors, which capture the most data variance. In contrast, the \emph{Chat Hard} and \emph{Safety} subsets exhibit a more uniform weight distribution, while the \emph{Reasoning} subset depends more heavily on basis vectors indexed between 1 and 50. These variations highlight the differing preference requirements across subsets.  


Using the weight vectors $\textbf{k}$ for all attributes, we compute their Pearson correlation, providing a quantitative explanation of the dataset's attributes. The resulting correlation matrix, shown in Figure~\ref{fig:Correlation}, reveals meaningful relationships between attributes. For instance, ``\emph{Narrative \& Storytelling}" strongly correlates with ``\emph{Humor \& Entertainment}" and ``\emph{Linguistic Creativity}" (with a correlation of approximately 0.87), which aligns with the idea that humor and creativity enhance storytelling. This correlation suggests that some attributes may be redundant. 
% Similarly, ``\emph{User-Friendliness}" and ``\emph{Humor \& Entertainment}" are moderately correlated (\(\text{Pearson } r = 0.6\)), indicating that humor can enhance user experience.  
On the other hand, ``\emph{Scientific Rigor}" negatively correlates with several attributes, including ``\emph{Chat}" and ``\emph{Narrative \& Storytelling}" (\(\text{Pearson } r = -0.46\) and \(-0.35\), respectively), suggesting that scientific rigor may conflict with other human preferences. Additionally, many attributes show weak or negligible correlations (with absolute Pearson \(r < 0.1\)).
% , reinforcing the necessity of multi-attribute evaluation.  
Overall, DRMs provide a structured framework for quantitatively explaining attribute relationships, offering deeper insights into benchmark design and multi-attribute evaluation.


\subsection{Ablation Study}\label{sec:ablation}
We analyze two key factors affecting test-time adaptation: adaptation set size and the number of DRM heads used. Using Gemma-2B-RM as the feature extractor, we present results on RPR and RewardBench in Figure \ref{fig:ablation}.  
Our findings show that performance improves with a larger adaptation set, converging on RewardBench at \( n \geq 15 \). Similarly, increasing the number of heads enhances performance but saturates beyond 100, likely because the most meaningful PCA directions lie within the first 100 heads.  
When the adaptation set is small (e.g., \( n = 3 \)), performance is unstable, and fewer heads can yield better results. This may be due to difficulty in correctly weighting heads with limited data, whereas using more heads increases the risk of assigning incorrect weights. However, with sufficient data, more heads eventually lead to better performance. These results suggest that a slightly larger adaptation set and a carefully chosen number of heads are key to optimizing performance.

% In this section, we examine two key factors affecting test-time adaptation performance: the adaptation set size and the number of DRMs' heads used for adaptation. We use Gemma-2B-RM as the feature extractor, and the results on RPR and RewardBench are shown in Figure \ref{fig:ablation}.  
% The results reveal a clear trend: as the adaptation set size increases, overall performance improves. On RewardBench, performance gradually converges when \( n \geq 15 \). Similarly, increasing the number of heads generally leads to better performance. However, performance saturates once the number of heads exceeds 100, likely because the most meaningful directions captured by PCA are within the first 100 heads. Moreover, when the adaptation set is small, e.g., $n=3$, the performance is not stable, with smaller number of heads achieves better results, possibly due to the ambiguity when using very small number of adaptation data makes it hard to correctly weight different heads, while more heads can make it less likely to correctly assign the weights to correct heads. Although finally larger number of head performs better. This suggest using slightly larger adaptation set and a suitable number of head according to the specific setting.


% \begin{figure}[t]
%     \centering
%     \includegraphics[width=1\linewidth]{Fig/ablation_2b.pdf}
%     \caption{Ablations on the adaptation set size and number of reward heads for test-time adaptation. }
%     \label{fig:ablation}
% \end{figure}

\begin{figure}[t]
    \centering
    \begin{subfigure}[b]{0.49\linewidth}
        \centering
        \includegraphics[width=\linewidth]{Fig/gemma_2b_rewardbench_Overall_heads-as-lines.pdf}
        % \caption{xxxxxxx.}
        \label{fig:fig1}
    \end{subfigure}
    \begin{subfigure}[b]{0.49\linewidth}
        \centering
        \includegraphics[width=\linewidth]{Fig/gemma_2b_rpr_multi_class_test_Overall_heads-as-lines.pdf}
        % \includegraphics[width=\linewidth, trim=20mm 0 0 0, clip]{Fig/gemma_2b_rpr_multi_class_test_Overall_heads-as-lines.pdf}
        % \caption{xxxxxxx.}
        \label{fig:fig2}
    \end{subfigure}
    \vspace{-27pt}
    \caption{Ablations on the adaptation set size and number of reward heads for test-time adaptation based on Gemma-2B-RM. }
    \label{fig:ablation}
    \vspace{-15pt}
\end{figure}

\vspace{-5pt}
\section{Related work}
\vspace{-5pt}
\section{Related Work}
% \subsection{Vision Language Model}
% 시각장애인에서 상황을 설명할 DB가 없으니 만들었다. 그리고 이를 VLM에 튜닝했다.
\subsection{Technical approaches for assisting the visually-impaired}


\subsection{Datasets for visual instruction tuning}


\vspace{-5pt}
\section{Conclusion}
\vspace{-5pt}
In this paper, we establish the connection between preference learning and PCA, introducing Decomposed Reward Models (DRMs). DRMs represent diverse human preferences as a set of orthogonal basis vectors using a novel vector-based formulation of preference. This approach enables efficient test-time adaptation to user preferences without requiring additional training, making it both scalable and practical.
Beyond the efficiency, DRMs provide a structured way to understand human preferences. By decomposing complex preferences into interpretable components, they reveal how preferences are formed and interact. We hope this work inspires further research into the fundamentals of human preference learning while promoting more transparent and personalized AI systems.


\section{Limitations}
In this paper, we obtain a large number of decomposed rewards from DRMs. However, due to the large scale (e.g., 2048 or 4096 reward heads), we did not manually examine each head to identify its corresponding preference attribute. Future work could focus on developing automated methods to analyze these rewards, such as recognizing patterns in the first 100 reward heads and assessing whether the last 100 primarily capture noise or meaningful subtleties.  
Additionally, we did not incorporate interdisciplinary study by collaborating with psychology experts to explore human preferences in depth. Future research could benefit from such collaboration to bridge the gap between computational models and cognitive science.



\section{Ethics Statement}
This paper introduces Decomposed Reward Models (DRMs), a step toward improving multi-objective alignment in LLMs. Here, we discuss the potential benefits of our approach while acknowledging the associated risks.  

Our method enhances LLM alignment with diverse human preferences. DRMs are lightweight, flexible, and easily adaptable to new users and evolving preferences. Their efficiency reduces resource demands and broadens accessibility, paving the way for personalized preference learning and scalable LLM alignment. By offering an interpretable framework, DRMs promote greater transparency and customization in human preference modeling.  


We carefully follow the license for all datasets and models used in our paper. Human preference datasets often contain biases, reflecting the perspectives and prejudices of their sources. If not properly managed, these biases could propagate through the model, influencing decomposition and principal components, potentially leading to unintended consequences. Mitigating this risk requires careful curation, filtering, and bias reduction before large-scale deployment. Additionally, our method does not inherently control the meaning of each reward head, which could unintentionally capture harmful human preferences. Therefore, thorough evaluation is necessary before deployment to ensure ethical and responsible use.


% \section*{Checklist}

% A. For every submission:
% A1. Did you describe the limitations of your work? - Point out any strong assumptions and how robust your results are to violations of these assumptions (e.g., independence assumptions, noiseless settings, model well-specification, asymptotic approximations only held locally). Reflect on how these assumptions might be violated in practice and what the implications would be. - Reflect on the scope of your claims, e.g., if you only tested your approach on a few datasets, languages, or did a few runs. In general, empirical results often depend on implicit assumptions, which should be articulated. Reflect on the factors that influence the performance of your approach. For example, a speech-to-text system might not be able to be reliably used to provide closed captions for online lectures because it fails to handle technical jargon. - If you analyze model biases: which definition of bias are you using? Did you state the motivation and definition explicitly? See the discussion in Blodgett et al. (2020). - We understand that authors might fear that complete honesty about limitations might be used by reviewers as grounds for rejection. It is worth keeping in mind that a worse outcome might be if reviewers discover limitations that aren’t acknowledged in the paper. In general, we advise authors to use their best judgement and recognize that individual actions in favor of transparency play an important role in developing norms that preserve the integrity of the community. Reviewers will be specifically instructed to not penalize honesty concerning limitations.

% A2. Did you discuss any potential risks of your work? - Examples of risks include potential malicious or unintended harmful effects and uses (e.g., disinformation, generating fake profiles, surveillance), environmental impact (e.g., training huge models), fairness considerations (e.g., deployment of technologies that could further disadvantage or exclude historically disadvantaged groups), privacy considerations (e.g., a paper on model/data stealing), and security considerations (e.g., adversarial attacks). See discussion in Leins et. al. (2020) as examples. - Does the research contribute to overgeneralization, bias confirmation, under or overexposure of specific languages, topics, or applications at the expense of others? See Hovy and Spruit (2016) for examples. - We expect many papers to be foundational research and not tied to particular applications, let alone deployments. However, we encourage authors to discuss potential risks if they see a path to any positive or negative applications. For example, the authors can emphasize how their systems are intended to be used, how they can safeguard their systems against misuse, or propose future research directions. - Consider different stakeholders that could be impacted by your work. Is it possible that research benefits some stakeholders while harming others? Does it pay special attention to vulnerable or marginalized communities? Does the research lead to exclusion of certain groups? See Dev et. al (2021) for examples. - Consider dual use, i.e, possible benefits or harms that could arise when the technology is being used as intended and functioning correctly, benefits or harms that could arise when the technology is being used as intended but gives incorrect results, and benefits or harms following from (intentional or unintentional) misuse of the technology. - Consider citing previous work on relevant mitigation strategies for the potential risks of the work (e.g., gated release of models, providing defenses in addition to attacks, mechanisms for monitoring misuse, mechanisms to monitor how a system learns from feedback over time, improving the efficiency and accessibility of NLP).

% B. Did you use or create scientific artifacts?
% Scientific artifacts may include code, data, models or other artifacts.
% Most NLP research uses scientific artifacts.
% Many NLP papers are accompanied by new scientific artifacts.
% B1. Did you cite the creators of artifacts you used? - For composite artifacts like the GLUE benchmark, this means all creators. - Cite the original paper that produced the code package or dataset. - Remember to state which version of the asset you’re using. - If possible, include a URL.

% B2. Did you discuss the license or terms for use and / or distribution of any artifacts? - State the name of the license (e.g., CC-BY 4.0) for each asset. - If you scraped or collected data from a particular source (e.g., website or social media API), you should state the copyright and terms of service of that source. Please note that some sources do not allow inference of protected categories like gender, sexual orientation, health status, etc. - The data might be in public domain and licensed for research purposes. - The data might be used with consent of its creators or copyright holders. - If the data is used without consent, the paper makes the case to justify its legal basis (e.g., research performed in the public interest under GDPR). - If you are releasing assets, you should include a license, copyright information, and terms of use in the package. - If you are repackaging an existing dataset, you should state the original license as well as the one for the derived asset (if it has changed). - If you cannot find this information online, you are encouraged to reach out to the asset’s creators.

% B3. Did you discuss if your use of existing artifact(s) was consistent with their intended use, provided that it was specified? For the artifacts you create, do you specify intended use and whether that is compatible with the original access conditions (in particular, derivatives of data accessed for research purposes should not be used outside of research contexts)? - Data and/or pretrained models are released under a specified license that is compatible with the conditions under which access to data was granted (in particular, derivatives of data accessed for research purposes should not be deployed in the real world as anything other than a research prototype, especially commercially) - The paper specifies the efforts to limit the potential use to circumstances in which the data/models could be used safely (such as an accompanying data/model statement). - The data is sufficiently anonymized to make identification of individuals impossible without significant effort. If this is not possible due to the research type, please state so explicitly and explain why. - The paper discusses the harms that may ensue from the limitations of the data collection methodology, especially concerning marginalized/vulnerable populations, and specifies the scope within which the data can be used safely.

% B4. Did you discuss the steps taken to check whether the data that was collected / used contains any information that names or uniquely identifies individual people or offensive content, and the steps taken to protect / anonymize it? - There are some settings where the existence of offensive content is not necessarily bad (e.g., swear words occur naturally in text), or part of the research question (i.e., hate speech). This question is just to encourage discussion of potentially undesirable properties. - Explain how you checked for offensive content and identifiers (e.g., with a script, manually on a sample, etc.). - Explain how you anonymized the data, i.e., removed identifying information like names, phone and credit card numbers, addresses, user names, etc. Examples are monodirectional hashes, replacement, or removal of data points. If anonymization is not possible due to the nature of the research (e.g., author identification), explain why. - List any further privacy protection measures you are using: separation of author metadata from text, licensing, etc. - If any personal data is used: the paper specifies the standards applied for its storage and processing, and any anonymization efforts. - If the individual speakers remain identifiable via search: the paper discusses possible harms from misuse of this data, and their mitigation.

% B5. Did you provide documentation of the artifacts, e.g., coverage of domains, languages, and linguistic phenomena, demographic groups represented, etc.? - Be sure to report the language of any language data, even if it is commonly-used benchmarks. - Describe basic information about the data that was used, such as the domain of the text, any information about the demographics of the authors, etc.

% B6. Did you report relevant statistics like the number of examples, details of train / test / dev splits, etc. for the data that you used / created? Even for commonly-used benchmark datasets, include the number of examples in train / validation / test splits, as these provide necessary context for a reader to understand experimental results. For example, small differences in accuracy on large test sets may be significant, while on small test sets they may not be.

% C. Did you run computational experiments?
% C1. Did you report the number of parameters in the models used, the total computational budget (e.g., GPU hours), and computing infrastructure used? - Even for commonly-used models like BERT, reporting the number of parameters is important because it provides context necessary for readers to understand experimental results. The size of a model has an impact on performance, and it shouldn’t be up to a reader to have to go look up the number of parameters in models to remind themselves of this information. - The total computational budget can be presented however is most appropriate for the paper – if most experiments were run on GPUs, then important information would likely include the total number of GPU hours, the amount of parallelism across GPUs, the size of the GPUs, etc. to allow a reader to estimate the computational requirements to reproduce, use, or build upon the work. - Note that this should include information about all experiments, not just the final runs that led to the results presented in the paper. If exact numbers are not available, an estimate is better than nothing.

% C2. Did you discuss the experimental setup, including hyperparameter search and best-found hyperparameter values? - The experimental setup should include information about exactly how experiments were set up, like how model selection was done (e.g., early stopping on validation data, the single model with the lowest loss, etc.), how data was preprocessed, etc. - Many research projects involve manually tuning hyperparameters until some “good” values are found, and then running a final experiment which is reported in the paper. Other projects involve using random search or grid search to find hyperparameters. In all cases, report the results of such experiments, even if they were stopped early or didn’t lead to your best results, as it allows a reader to know the process necessary to get to the final result and to estimate which hyperparameters were important to tune. - Be sure to include the best-found hyperparameter values (e.g., learning rate, regularization, etc.) as these are critically important for others to build on your work. - The experimental setup should likely be described in the main body of the paper, as that is important for reviewers to understand the results, but large tables of hyperparameters or the results of hyperparameter searches could be presented in the main paper or appendix.

% C3. Did you report descriptive statistics about your results (e.g., error bars around results, summary statistics from sets of experiments), and is it transparent whether you are reporting the max, mean, etc. or just a single run? - Error bars can be computed by running experiments with different random seeds, Clopper–Pearson confidence intervals can be placed around the results (e.g., accuracy), or expected validation performance can be useful tools here. - In all cases, when a result is reported, it should be clear if it is from a single run, the max across N random seeds, the average, etc. - When reporting a result on a test set, be sure to report a result of the same model on the validation set (if available) so others reproducing your work don’t need to evaluate on the test set to confirm a reproduction.

% C4. If you used existing packages (e.g., for preprocessing, for normalization, or for evaluation, such as NLTK, Spacy, ROUGE, etc.), did you report the implementation, model, and parameter settings used? - The version number or reference to specific implementation is important because different implementations of the same metric can lead to slightly different results (e.g., ROUGE). - The paper cites the original work for the model or software package. If no paper exists, a URL to the website or repository is included. - If you modified an existing library, explain what changes you made.

% D. Did you use human annotators (e.g., crowdworkers) or research with human participants?
% D1. Did you report the full text of instructions given to participants, including e.g., screenshots, disclaimers of any risks to participants or annotators, etc.? - Examples of risks include a crowdsourcing experiment which might show offensive content or collect personal identifying information (PII). Ideally, the participants should be warned. - Including this information in the supplemental material is fine, but if the main contribution of your paper involves human subjects, then we strongly encourage you to include as much detail as possible in the main paper.

% D2. Did you report information about how you recruited (e.g., crowdsourcing platform, students) and paid participants, and discuss if such payment is adequate given the participants’ demographic (e.g., country of residence)? - Be explicit about how you recruited your participants. For instance, mention the specific crowdsourcing platform used. If participants are students, give information about the population (e.g., graduate/undergraduate, from a specific field), and how they were compensated (e.g., for course credit or through payment). - In case of payment, provide the amount paid for each task (including any bonuses), and discuss how you determined the amount of time a task would take. Include discussion on how the wage was determined and how you determined that this was a fair wage.

% D3. Did you discuss whether and how consent was obtained from people whose data you’re using/curating? For example, if you collected data via crowdsourcing, did your instructions to crowdworkers explain how the data would be used?

% D4. Was the data collection protocol approved (or determined exempt) by an ethics review board? - Depending on the country in which research is conducted, ethics review (e.g., from an IRB board in the US context) may be required for any human subjects research. If an ethics review board was involved, you should clearly state it in the paper. However, stating that you obtained approval from an ethics review board does not imply that the societal impact of the work does not need to be discussed. - For initial submissions, do not include any information that would break anonymity, such as the institution conducting the review.

% D5. Did you report the basic demographic and geographic characteristics of the annotator population that is the source of the data? - State if your data include any protected information (e.g., sexual orientation or political views under GDPR). - The paper is accompanied by a data statement (see Bender and Friedman, 2018) describing the basic demographic and geographic characteristics of the author population that is the source of the data, and the population that it is intended to represent. - If applicable: the paper describes whether any characteristics of the human subjects were self-reported (preferably) or inferred (in what way), justifying the methodology and choice of description categories.

% E. Did you use AI assistants (e.g., ChatGPT, Copilot) in your research, coding, or writing?
% E1. Did you include information about your use of AI assistants?

% E1. Elaboration For Yes Or No. For yes, provide a section number. For no, justify why not.

% E1. Section Or Justification




\bibliography{custom}

\newpage

\newpage
\appendix

\section{Appendix}
\label{sec:appendix}

\begin{figure}[h]
    \centering
    \begin{subfigure}[b]{0.49\linewidth}
        \centering
        \includegraphics[width=\linewidth]{Fig/llama_8b_rpr_multi_class_test_Overall_heads-as-lines.pdf}
        % \caption{xxxxxxx.}
        \label{fig:fig_2_1}
    \end{subfigure}
    \begin{subfigure}[b]{0.49\linewidth}
        \centering
        \includegraphics[width=\linewidth]{Fig/llama_8b_rewardbench_Overall_heads-as-lines.pdf}
        % \caption{xxxxxxx.}
        \label{fig:fig_2_2}
    \end{subfigure}
    \caption{Ablations on the adaptation set size and number of reward heads for test-time adaptation on Llama3-8B-RM. }
    \label{fig:ablation_8b}
\end{figure}


\subsection{Ablations on Llama3-8B-RM}
We add the ablation results on Llama3-8B-RM in Figure \ref{fig:ablation_8b}. The trend is similar to the ablations in our main paper.

\subsection{Weight Distributions on RPR Using Gemma-2B-RM}
We also visualize the weight distributions of the top 100 decomposed reward heads on RPR in Figure \ref{fig:TTA_weight_rpr}. For different attributes, weight distributions have diverse patterns.

\begin{figure}[h]
    \centering
    \includegraphics[width=\linewidth]{Fig/TTA_weight_rpr.pdf}
    \caption{Weight distributions of the top 100 decomposed reward heads on RPR for DRMs using Gemma-2B-RM as the backbone.}
    \label{fig:TTA_weight_rpr}
\end{figure}

\subsection{Implementation Details}
For all training-based reward head models, including both the Single Head and Share-Base variants, we train them on the \textit{mixture2\_and\_safe\_pku} dataset\footnote{\url{https://huggingface.co/datasets/weqweasdas/preference_dataset_mixture2_and_safe_pku}}~\cite{dong2024rlhf} for one epoch with a batch size of 16. 

For our proposed Decomposed Reward Models (DRMs), we apply Principal Component Analysis (PCA) using scikit-learn's default settings to get the component vectors. We experiment with 100 heads for all methods. Note that DRMs utilize 50 distinct reward heads, and including their negative counterparts results in a total of 100 heads.

\end{document}
