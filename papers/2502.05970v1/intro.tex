\section{Introduction}\label{sec1}

Designing new materials and molecules is essential for the development of new technologies. Traditionally, this design process involves extensive experimental trial and error or high-throughput simulations to screen databases, both of which are time-consuming and resource-intensive \cite{axelrod2022learning, sanchez2018inverse}. As a result, there is increasing interest in applying machine learning (ML) techniques to accelerate the discovery of materials and molecules with desired properties \cite{axelrod2022learning, sanchez2018inverse, bilodeau2022generative, noh2020machine, kim2021deep}.

One strategy for finding materials and molecules with desired properties is inverse design through conditional generation \cite{noh2020machine, kim2021deep, zeni2023mattergen, yang2023scalable, xie2021crystal, sanchez2018inverse}.
A complementary approach is screening candidates through property prediction \cite{walters2020applications, dunn2020benchmarking, wang2021compositionally, zhuo2018predicting, de2021materials, ward2016general}. 
However, both approaches typically struggle when property values fall outside the training distribution \cite{noh2020machine,wang2021compositionally,de2021materials, kauwe2020can, zhao2022limitations, omee2024structure} \pulkit{Its good to explain why the properties will fall out of training distribution -- because we are looking for compounds with properties that currently don't exist -- stress this bit.}. Enhancing extrapolative capabilities in property prediction would improve large dataset screening by identifying promising compounds and molecules with exceptional properties. This approach could help guide further synthesis and computational efforts, ultimately advancing materials and molecular design.

Bilinear Transduction \cite{netanyahu2023transduction} has been successful in extrapolation in robotics domains w.r.t. the input space by making predictions based on analogies \pulkit{TOo many details about -- its on the input space is not well explained nor what it means by analogies}. 
% 
In this work, we empirically investigate whether Bilinear Transduction improves extrapolation in materials science (Figure~\ref{fig:ood-kde-plot}).
Specifically, extrapolation in the output space, i.e. the property value, for finding high-performance materials \pulkit{State this is as the problem -- extrapolation in the property space -- and then state that you use method that has been successful in extrapolation in other problems to this problem -- no need to say how the method works just yet}.


Our main contributions are (1) adapting a method for extrapolation to a new application in materials science and (2) an extensive evaluation of our approach on common high-throughput calculated and experimental solid-state materials and molecules benchmarks, showcasing improved extrapolation to out-of-support (\oos) \pulkit{I would expect the main contribution is to show that we can make OOS predictions? Has previous work done this well? Or did previous work end up at chance performance? If at chance performance, then the result is quite significant. Or if there is some other way to show significance of the result, instead of saying that our contribution is a new application of an existing method.}.