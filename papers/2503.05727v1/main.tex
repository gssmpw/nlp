\documentclass[manuscript,nonacm]{acmart}

\AtBeginDocument{%
  \providecommand\BibTeX{{%
    Bib\TeX}}}

\usepackage{xcolor}
\newcommand{\gray}{\cellcolor{gray!20}}
\usepackage{wrapfig}
\usepackage{colortbl}
\usepackage{multirow}
\usepackage{booktabs}
\newcommand{\cbox}[1]{\raisebox{\depth}
{\fcolorbox{black}{#1}{\null}}}
\newcommand{\ctriangle}[1]{\raisebox{0.2ex}{\textcolor{#1}{$\blacktriangle$}}}
\newcolumntype{P}[1]{>{\centering\arraybackslash}p{#1}}

\begin{document}

\title{Toward Integrated Solutions: A Systematic Interdisciplinary Review of Cybergrooming Research}

\author{Heajun An}
\email{heajun@vt.edu}
\orcid{0009-0007-6124-3750}
\affiliation{
  \institution{Virginia Tech}
  \city{Blacksburg}
  \state{Virginia}
  \country{USA}}

\author{Marcos Silva}
\email{marcos.silva@aluno.cefetmg.br}
\orcid{0009-0006-9420-9535}
\affiliation{
  \institution{Federal Center for Technological Education of Minas Gerais}
  \city{Belo Horizonte}
  \state{Minas Gerais}
  \country{Brazil}
}

\author{Qi Zhang}
\email{qiz21@vt.edu}
\orcid{0000-0002-3607-3258}
\affiliation{
  \institution{Virginia Tech}
  \city{Falls Church}
  \state{Virginia}
  \country{USA}}

\author{Arav Singh}
\email{aravsingh@vt.edu}
\orcid{0009-0007-3582-9600}
\affiliation{
  \institution{Virginia Tech}
  \city{Blacksburg}
  \state{Virginia}
  \country{USA}}

\author{Minqian Liu}
\email{minqianliu@vt.edu}
\orcid{0009-0001-6014-3949}
\affiliation{
  \institution{Virginia Tech}
  \city{Blacksburg}
  \state{Virginia}
  \country{USA}}

\author{Xinyi Zhang}
\email{xinyizhang@vt.edu}
\orcid{0009-0003-4188-0104}
\affiliation{
  \institution{Virginia Tech}
  \city{Blacksburg}
  \state{Virginia}
  \country{USA}}

\author{Sarvech Qadir}
\email{sarvech.qadir@vanderbilt.edu}
\orcid{0009-0006-7962-5792}
\affiliation{
  \institution{Vanderbilt University}
  \city{Nashville}
  \state{Tennessee}
  \country{USA}}

\author{Sang Won Lee}
\email{sangwonlee@vt.edu}
\orcid{0000-0002-1026-315X}
\affiliation{
  \institution{Virginia Tech}
  \city{Blacksburg}
  \state{Virginia}
  \country{USA}}

\author{Lifu Huang}
\email{lfuhuang@ucdavis.edu}
\orcid{0000-0002-2743-7718}
\affiliation{
  \institution{University of California, Davis}
  \city{Davis}
  \state{California}
  \country{USA}}

\author{Pamela Wisnieswski}
\email{pamela.wisniewski@vanderbilt.edu}
\orcid{0000-0002-6223-1029}
\affiliation{
  \institution{Vanderbilt University}
  \city{Nashville}
  \state{Tennessee}
  \country{USA}}

\author{Jin-Hee Cho}
\email{jicho@vt.edu}
\orcid{0000-0002-5908-4662}
\affiliation{
  \institution{Virginia Tech}
  \city{Falls Church}
  \state{Virginia}
  \country{USA}}
  

\renewcommand{\shortauthors}{An et al.}

\begin{abstract}
Cybergrooming exploits minors through online trust-building, yet research remains fragmented, limiting holistic prevention. Social sciences focus on behavioral insights, while computational methods emphasize detection, but their integration remains insufficient. This review systematically synthesizes both fields using the PRISMA framework to enhance clarity, reproducibility, and cross-disciplinary collaboration. Findings show that qualitative methods offer deep insights but are resource-intensive, machine learning models depend on data quality, and standard metrics struggle with imbalance and cultural nuances. By bridging these gaps, this review advances interdisciplinary cybergrooming research, guiding future efforts toward more effective prevention and detection strategies.
\end{abstract}

\begin{CCSXML}
<ccs2012>
   <concept>
       <concept_id>10002978.10003029</concept_id>
       <concept_desc>Security and privacy~Human and societal aspects of security and privacy</concept_desc>
       <concept_significance>500</concept_significance>
       </concept>
 </ccs2012>
\end{CCSXML}

\ccsdesc[500]{Security and privacy~Human and societal aspects of security and privacy}

\keywords{Cybergrooming, Online Grooming, Cyber Exploitation, Internet Grooming, Predator Detection}

\maketitle

\section{Introduction} \label{sec:introduction}

The rapid expansion of digital communication has facilitated unprecedented social connectivity, but it has also introduced new threats, including \textit{cybergrooming}, where perpetrators manipulate online trust-building to exploit minors~\cite{mladenovic2021cyber}. Reports indicate a significant rise in cybergrooming cases, with law enforcement agencies observing a sharp increase in online sexual exploitation incidents during the COVID-19 pandemic~\cite{ringenberg2022prepost, whittle2013character}. As minors spend more time on social platforms, the risk of exposure to predatory behaviors continues to escalate, underscoring the urgent need for effective prevention and intervention strategies.

Despite its severity, cybergrooming research remains fragmented, with social sciences focusing on psychological and behavioral aspects, such as risk factors and victim vulnerability, while computational sciences prioritize automated detection methods. However, these disciplines rarely intersect, limiting the development of holistic solutions. Without interdisciplinary collaboration, efforts to detect and mitigate cybergrooming remain constrained by isolated methodologies, preventing the integration of behavioral insights into computational models and the validation of automated techniques through social science frameworks. Addressing this divide is crucial for developing comprehensive strategies that effectively prevent, detect, and disrupt cybergrooming.

This paper bridges these gaps by systematically synthesizing cybergrooming research across social and computational sciences, examining research objectives, evaluation methods, metrics, datasets, bias mitigation approaches, findings, limitations, and lessons learned. Through this interdisciplinary synthesis, we aim to advance prevention and detection strategies, identify existing research limitations, and outline future directions that foster collaboration between fields.

This review makes the following {\bf key contributions}:
\begin{enumerate}
    \item {\bf Interdisciplinary Analysis:} We systematically analyze cybergrooming research across both social and computational sciences, highlighting strengths, limitations, insights, and lessons learned to guide future interdisciplinary research and collaboration.
    \item {\bf Methodological Rigor:} Using the Preferred Reporting Items for Systematic Reviews and Meta-Analyses (PRISMA) methodology~\cite{page2021prisma}, we enhance clarity, transparency, and reproducibility, ensuring a structured analysis of cybergrooming methods, datasets, evaluation metrics, challenges, and future research needs.
    \item {\bf Research Gaps:} Our exclusive focus on cybergrooming enables the identification of distinct research gaps, advocating for stronger interdisciplinary collaboration to enhance understanding and mitigation strategies.
    \item {\bf Multidisciplinary Integration:} We emphasize multidisciplinary integration, demonstrating how insights from social sciences can refine computational models, improve detection accuracy, and support culturally adaptive, real-world solutions for law enforcement, policymakers, and online safety initiatives.
\end{enumerate}


We aim to answer the following {\bf research questions}:
\begin{enumerate}
\setlength{\itemindent}{2em}
    \item[\bf RQ1] \textit{What are the key approaches used to study cybergrooming in social and computational sciences?}
    \item[\bf RQ2] \textit{What are the primary contributions of existing cybergrooming research in both fields?}
    \item[\bf RQ3] \textit{What limitations exist in the methodologies used across different disciplines?}
    \item[\bf RQ4] \textit{How do social and computational studies on cybergrooming differ, and what interdisciplinary insights can enhance mitigation of cybergrooming risks?}
    \item[\bf RQ5] \textit{What gaps and challenges remain in cybergrooming research, and what future directions are needed?}
\end{enumerate}

We will discuss the answers to these research questions in Section~\ref{sec:discussions}.


\section{Review Methodology} \label{sec:review-methodology}

\subsection{The Preferred
Reporting Items for Systematic reviews
and Meta-Analyses (PRISMA)}

The Preferred Reporting Items for Systematic Reviews and Meta-Analyses (PRISMA)~\cite{page2021prisma} is a widely adopted guideline designed to promote transparent, comprehensive reporting in systematic reviews and meta-analyses across diverse research fields. Developed to enhance clarity and reproducibility, PRISMA provides a structured checklist ensuring rigorous methodology, supporting both critical appraisal and synthesis of evidence. While originally designed for health interventions, PRISMA has been increasingly applied to interdisciplinary research, facilitating the integration of findings across varied study designs and methodologies.

The PRISMA 2020 statement~\cite{page2021prisma} is particularly well-suited for cybergrooming research, which spans both social and computational sciences. Given the fragmented nature of existing studies, PRISMA's structured framework enables a systematic synthesis of diverse methodologies, including behavioral analyses, machine learning-based detection models, and policy evaluations. By standardizing inclusion criteria, data extraction, and quality assessment, PRISMA enhances the rigor of interdisciplinary reviews, ensuring a comprehensive and reproducible evaluation of cybergrooming studies. This approach fosters a more cohesive understanding of cybergrooming dynamics, supporting the development of integrated prevention and intervention strategies.


\subsection{Information Sources \& Search Methods} \label{subsec:info-sources}
This section provides a comprehensive overview of the information sources and search strategies employed to identify studies relevant to cybergrooming, covering two primary domains: social and computational sciences. We describe the databases and registries consulted, detailing the latest searches conducted between Jan. and Oct. 2024.

To ensure comprehensive coverage across both domains, we systematically searched multiple databases to each field, prioritizing sources with established relevance to cybergrooming research. For social sciences, we selected databases specializing in psychological, sociological, and criminological studies, as these disciplines extensively explore behavioral patterns, risk factors, and victim-offender dynamics. Computational sciences databases were chosen for their robust repositories on cybersecurity, artificial intelligence, and machine learning, which are central to automated detection methodologies. While these databases offered broad coverage, we acknowledge potential limitations, such as language bias (English-dominant sources) and the exclusion of gray literature that may contain valuable practitioner insights.

\textbf{\em Social science} is an academic discipline focused on studying human behavior, social relationships, and societal structures to reveal patterns shaping social systems. Cybergrooming, explored within psychology, sociology, education, and criminology, involves multifaceted social and psychological factors. To capture a wide range of perspectives, we conducted a systematic literature search using PsycINFO, PubMed, SocINDEX, Education Resources Information Center (ERIC), and Web of Science. These databases were selected for their interdisciplinary scope, ensuring coverage of empirical studies, theoretical discussions, and policy-oriented research. To filter relevant studies, we applied the following Boolean search strategy to the title and abstract fields: \textit{("online grooming" OR "cybergrooming" OR "online sexual grooming" OR "internet grooming" OR "cyber grooming")}. We limited results to peer-reviewed articles where possible, ensuring methodological rigor. This approach yielded 360 papers across the selected databases.

\textbf{\em Computational science} examines complex phenomena through models, simulations, and data analysis to identify patterns and predict behaviors. In the context of cybergrooming, computational approaches focus on automated detection using natural language processing (NLP), machine learning, and cybersecurity frameworks. To identify relevant studies, we searched the ACM Digital Library, IEEE Xplore, Computers \& Applied Sciences Complete, and ScienceDirect, which collectively provide extensive coverage of technical innovations in cyber threat detection and AI-driven solutions. The search applied the same Boolean strategy used in social sciences: \textit{("online grooming" OR "cybergrooming" OR "online sexual grooming" OR "internet grooming" OR "cyber grooming")}. In the ACM Digital Library, we restricted searches to titles only to prevent retrieval of broadly related works, such as studies on grooming in machine learning contexts. Where possible, we prioritized peer-reviewed publications to ensure research quality. This search yielded 410 relevant papers.

To ensure completeness, we conducted a supplementary search to capture any relevant publications that may have been initially overlooked. This included a citation analysis of key studies, identifying recent works citing foundational sources, and manually reviewing reference lists. Additionally, we searched Google Scholar and arXiv for non-peer-reviewed studies, preprints, and emerging research that might not yet be indexed in traditional databases. This supplementary search yielded 25 additional relevant papers, further strengthening the comprehensiveness of our review.


\subsection{Review Criteria and PRISMA Result} \label{subsec:review-criteria}

\begin{wrapfigure}{r}{0.7\textwidth}
    \centering
    \includegraphics[width=\linewidth]{figs/Prisma_0216.png}
    \caption{PRISMA flow diagram}
    \label{fig:PRISMA}
\end{wrapfigure}
This section outlines the inclusion and exclusion criteria, detailing our focus on {\em cybergrooming} while excluding other topics, such as cyberbullying.  To ensure relevance, we included only studies that primarily address cybergrooming. While papers referencing related issues, such as cyberbullying, sexting, and online harassment, were not automatically excluded, our focus remained on those where cybergrooming was central. This approach ensures that the review remains specific to cybergrooming, a phenomenon involving unique manipulation and trust-building tactics that pose distinct risks. By concentrating on cybergrooming, our review aims to highlight its specific challenges and better inform prevention and intervention strategies tailored to its unique dynamics.

\textbf{Scope definition.} We prioritized studies examining various aspects of cybergrooming, with a focus on research addressing risk factors, coping strategies, detection methods, and metrics for evaluating performance and security. Specifically, we included studies on topics closely related to cybergrooming, such as online grooming, cyber predators, internet grooming, online child grooming, online child exploitation, cyber exploitation, online sexual abuse, internet predators, digital predators, and online predatory behavior.

\textbf{Exclusion based on terminology.} Our search strategy initially identified 795 publications. After removing duplicates and retracting articles using Zotero 7 and performing manual checks, we retained 582 articles for further screening. In the first phase, we reviewed titles and abstracts to exclude studies unrelated to cybergrooming. During this process, 408 publications were excluded. The high exclusion rate can be attributed to the varied meanings of the term \textit{grooming}. Specifically, most excluded articles came from network-related fields, where grooming refers to \textit{traffic grooming}, a technique used to optimize the flow of data in a network to improve network performance. In addition, several articles were from the social sciences, where grooming typically refers to \textit{social grooming}, which is the process of building relationships or trust with others. These articles focused on social relationships rather than the sexual grooming that was the target of our review, leading to their removal.

\textbf{Trends in excluded studies.} During the screening process, we identified several common reasons for excluding papers, highlighting gaps and challenges in existing cybergrooming research. First, a substantial number of exclusions stemmed from studies using the term \textit{grooming} in unrelated contexts, such as network traffic optimization (traffic grooming) in computer networking and social grooming in behavioral psychology. These fields, while relevant in their own domains, do not align with the focus of our review on online sexual grooming. Second, a notable portion of rejected studies examined adjacent but distinct topics, including cyberbullying, online harassment, and sexting, without explicitly addressing cybergrooming as a primary research focus. Although these issues share behavioral and technological overlaps with cybergrooming, they were excluded to maintain a targeted scope. Third, methodological concerns contributed to exclusions, including studies with anecdotal or non-empirical evidence, lacking clear data collection methodologies or systematic evaluation metrics. Some rejected papers focused exclusively on offline grooming, limiting their applicability to digital contexts. Finally, language barriers led to the exclusion of several non-English papers due to translation limitations and inconsistent full-text access. These exclusion patterns underscore critical research gaps, such as the need for more precise terminology in computational studies, greater emphasis on interdisciplinary integration, and expanded multilingual investigations to capture global variations in cybergrooming behaviors.

\textbf{Final inclusion criteria.} The remaining articles were then fully retrieved, with further exclusions applied based on specific criteria: non-English language, review articles, incorrect format, limited focus on cybergrooming, emphasis on offline grooming, and subjects other than victims or groomers. This process resulted in 79 papers for inclusion in the review, comprising 43 papers from social sciences and 36 from computational sciences. This count excludes review papers discussed in Section~\ref{subsec:other-cybergrooming-review}. Each PRISMA process stage is illustrated in Fig.~\ref{fig:PRISMA}.

\begin{wrapfigure}{r}{0.7\textwidth}
    \vspace{-3mm}
    \centering
    \includegraphics[width=\linewidth]{figs/Articles_0216.png}
    \vspace{-5mm}
    \caption{Trends in social and computational science articles on cybergrooming (2007 -- 2024).}
    \label{fig:Literatures_numbers}
    \vspace{-3mm}
\end{wrapfigure}
\textbf{Research trends.} Fig.~\ref{fig:Literatures_numbers} illustrates the growth of cybergrooming research over time. Both social and computational sciences show increasing contributions, with a sharp rise in computational publications from 2015 onward. Social science research has steadily expanded, emphasizing behavioral insights, while computational studies have accelerated, likely driven by advances in AI and technology. This trend highlights an interdisciplinary shift, with both fields addressing cybergrooming through complementary perspectives.

\subsection{Other Related Review Papers} \label{subsec:other-cybergrooming-review}

\paragraph{\bf Review papers in social sciences domain} 
\citet{ringenberg2022prepost} reviewed grooming techniques before and after the Internet's advent, noting similar strategies with differing concepts and timing. However, their review focused only on groomers' strategies without addressing detection methods. Our paper fills this gap by exploring potential solutions, explicitly targeting online grooming.  Similarly, \citet{whittle2013character} reviewed the characteristics of online grooming and the internet behavior of younger generations as studied in social sciences. They found consistency in the grooming process despite variations in techniques but did not propose solutions, focusing only on describing the phenomenon. Our paper contributes by offering insights into strategies for addressing the problem.

\citet{schittenhelm2024cybergrooming} conducted a systematic review of quantitative research on cybergrooming victimization, examining the prevalence, risk factors, and outcomes through a modified {\em General Aggression Model}. A limitation of this review is its reliance on cross-sectional, self-reported data, which may overlook minority populations and limit causal insights. Our paper adopts a broader interdisciplinary approach, encompassing both social and computational sciences, in contrast to their review, which is primarily grounded in social science.

\paragraph{\bf Review papers in computational sciences domain}

\citet{Ngejane_etal_2018} reviewed cybersecurity methodologies focusing on machine learning techniques to combat online sexual grooming, primarily highlighting the dominance of supervised learning approaches. However, their review is largely technical and centers exclusively on machine learning methods. In contrast, our review takes an interdisciplinary approach, integrating computational and social science perspectives for a more holistic view of cybergrooming research. \citet{Razi_etal_2021} examined computational methodologies for detecting online sexual risks from a human-centered perspective, emphasizing high-quality datasets, real-world applications, and contextual information in detection models. However, their review covers a broad range of sexual risks, including grooming, trafficking, harassment, and abuse. Our review narrows this scope to cybergrooming, offering a more focused analysis relevant to researchers in this field.

\citet{borj2023chatlog} analyzed online grooming detection methods targeted at minors using chat log data, evaluating detection algorithms and datasets. Their focus is primarily on grooming detection, including identifying grooming phases, applying psychological theories, and profiling predators. Our review extends beyond these aspects by covering broader research fields, integrating cybergrooming studies from both computational and social sciences for a multidisciplinary perspective. \citet{mladenovic2021cyber} surveyed definitions of cyber-aggression, cyberbullying, and cybergrooming, as well as linguistic features, datasets, and models. However, their focus is broad, with cybergrooming one of several topics within cyber abuse. In contrast, our review is specifically dedicated to cybergrooming, providing a focused resource for researchers studying online sexual exploitation across social media platforms.

\paragraph{\bf Key research gaps to fill by our review paper} 

Existing reviews in social and computational sciences often address cybergrooming from isolated perspectives. First, \citet{ringenberg2022prepost} and \citet{whittle2013character} focus on offenders' strategies and victims' behaviors, providing valuable real-world insights. However, these findings are difficult to translate into computational approaches. Conversely, \citet{Ngejane_etal_2018} emphasize machine learning techniques but overlook the broader social context. Our review bridges these gaps by adopting an interdisciplinary approach, integrating technical and social science insights to explore comprehensive solutions for cybergrooming.

Second, reviews like \citet{Razi_etal_2021} and \citet{mladenovic2021cyber} cover a wide array of online risks, such as grooming, trafficking, and harassment, which dilutes the focus on cybergrooming. In contrast, our review specifically targets cybergrooming, offering a comprehensive analysis that identifies distinct research gaps and provides targeted insights to guide future studies. By applying the PRISMA methodology, we enhance both clarity and reproducibility, effectively addressing the limitations of previous, less transparent reviews in this area.

Finally, studies such as \citet{borj2023chatlog} examine specific aspects like psychological theories and detection methods without connecting them to broader interdisciplinary trends. Our review synthesizes insights from both social and computational sciences, delivering a comprehensive understanding of cybergrooming that informs more robust and effective prevention, detection, and intervention strategies.

For easy access to the reviews, Table~\ref{tab:sota-review-papers-analysis} summarizes their key contributions, limitations, and distinctions from ours.
\begin{table}[t]
\small 
\centering
\caption{Summary of Related Review Papers on Cybergrooming Research} \label{tab:sota-review-papers-analysis}
\vspace{-3mm}
\begin{tabular}{p{3cm} p{4cm} p{3cm} p{4cm}}
\hline
\multicolumn{1}{c}{\bf Review Paper} & \multicolumn{1}{c}{\bf Key Contribution} & \multicolumn{1}{c}{\bf Limitations} & \multicolumn{1}{c}{\bf Differences with Our Review} \\ \hline
\multicolumn{4}{c}{\textbf{\gray Social Sciences}}
\\ \hline
\citet{ringenberg2022prepost} & Reviewed grooming techniques before and after the advent of the Internet, highlighting similarities in strategies. & Focused only on groomers' strategies without addressing detection methods. & Our review explores potential solutions for online grooming, including detection strategies, offering a more solution-oriented perspective. \\ \hline
\citet{whittle2013character} & Analyzed online grooming characteristics and younger generations' behavior and consistency in the grooming techniques. & Descriptive in nature, without proposing solutions to counter cybergrooming. & Our review provides actionable insights and strategies for addressing cybergrooming, filling the gap left by descriptive studies. \\ \hline
\citet{schittenhelm2024cybergrooming} & Conducted a systematic review of cybergrooming victimization, examining prevalence, risk factors, and outcomes. & Relies on cross-sectional, self-reported data, potentially overlooking minority groups and limiting causal insights. & Our review takes an interdisciplinary approach, integrating computational methods with social science insights, broadening the scope beyond victimization. \\ \hline
\multicolumn{4}{c}{\textbf{\gray Computational Sciences}}
\\ \hline
\citet{Ngejane_etal_2018} & Reviewed machine learning techniques in cybersecurity to combat online grooming, emphasizing supervised learning. & Focusing exclusively on machine learning without considering the broader social context. & We take an interdisciplinary approach, combining computational and social sciences for a holistic view of cybergrooming. \\ \hline
\citet{Razi_etal_2021} & Examined computational methodologies for detecting online sexual risks, emphasizing dataset quality and real-world applicability. & Broad scope, covering various online risks (grooming, trafficking, harassment). & Our review specifically targets cybergrooming, providing a focused analysis relevant for researchers interested in this area. \\ \hline
\citet{borj2023chatlog} & Analyzed detection methods for online grooming using chat log data, focusing on detection algorithms and datasets. & Primarily focuses on psychological theories and grooming detection. & Our review synthesizes insights across social and computational sciences, offering a multidisciplinary understanding of cybergrooming. \\ \hline
\citet{mladenovic2021cyber} & Surveyed definitions and linguistic features in cyber-aggression, covering cybergrooming as part of broader cyber abuse. & Broad coverage of cyber abuse topics, with cybergrooming as a minor focus. & Our review focuses on cybergrooming, providing an in-depth resource for studying online sexual exploitation on social media. \\ \hline
\end{tabular}
\vspace{-5mm}
\end{table}


\section{Cybergrooming Research in Social Sciences} \label{sec:cybergrooming-SC}

Cyber grooming has become an extensively researched online exploitation topic in social sciences. 
The following presentation of the social science literature reviews the goals, methods, evaluation methods, metrics, results, and limitations of the existing approaches surveyed in this paper.

\subsection{Research Objectives in Social Science Cybergrooming Research} \label{subsec:ss-research-objectives}

\begin{table}[t] % Adjusted float positioning
\small
    \caption{Research Objectives of the Reviewed Social Science Research Works on Cybergrooming}
    \label{tab:research-objects-ss}
    \vspace{-3mm}
    \begin{tabular}{p{4cm} P{2.5cm} P{2.5cm}} % Adjusted column widths
    \hline
        \textbf{Research Objectives} & \textbf{Percentage} & \textbf{Number of Articles} \\
    \hline
        \raisebox{-0.05cm}{\cbox{yellow!90!black}} Risk Analysis & 44.19\% & 19 \\
    \hline
        \raisebox{-0.05cm}{\cbox{teal!20}} Offender Profiling & 41.86\% & 18 \\
    \hline
        \raisebox{-0.05cm}{\cbox{blue!60}} Prevention and Awareness & 20.93\% & 9 \\
    \hline
        \raisebox{-0.05cm}{\cbox{red!70}} Behavioral Stages & 9.30\% & 4 \\
    \hline
    \end{tabular}
    \vspace{1mm}

    \small{(Note: An article may appear multiple times if it addresses multiple objectives, each counted individually.)}
    \vspace{-5mm}
\end{table}

\cbox{yellow!90!black} \textbf{Risk Analysis.} Social science research extensively examines the underlying risk factors contributing to cybergrooming victimization. Studies have explored correlations between cybergrooming, sexting, sextortion, and parental control measures, particularly during the COVID-19 lockdowns, highlighting vulnerabilities in adolescent populations \cite{almeida2024online}. Other research identifies demographic and behavioral factors influencing victim susceptibility, including age, gender, and routine online behaviors \cite{Finkelhor22-online-sexual-offense, wachs2020routine}. The emotional consequences of grooming, such as shame, guilt, and psychological distress, have also been investigated to understand how victims cope with exploitation \cite{gamez2023stability, HERNANDEZ2021106569}. These findings inform policies aimed at reducing risk exposure and strengthening online safety measures.

\cbox{teal!20} \textbf{Offender Profiling.} Research focuses on cybergrooming patterns and offender behaviors, categorizing stages and tactics used by predators. Studies have examined linguistic strategies, manipulation techniques, and grooming progression to build offender typologies \cite{Anggraeny2023, Fernandes_et_al_2023}. Specific tactics, such as strategic praise and emotional manipulation, reveal how predators gain victims’ trust before initiating exploitation \cite{Lorenzo-Dus_Izura_2017, gamez2021unraveling}. %This knowledge aids early detection and informs law enforcement efforts.

\cbox{blue!60} \textbf{Prevention and Awareness.} Raising awareness and educating victims, parents, and educators about cybergrooming is a key focus of social science research. Studies evaluate educational programs that enhance online safety knowledge among minors and caregivers \cite{Calvete_Orue_GamezGuadi_2022}. Other research highlights parental supervision strategies in mitigating grooming risks \cite{Dorasamy_etal_2021, kamar2022parental, christiansen2024hidden, khotimah2024child}. Additionally, scholars assess awareness campaigns and school-based digital literacy initiatives in reducing victimization rates \cite{craven_current_2007}.

\cbox{red!70} \textbf{Behavioral Stages.} Understanding the sequential phases of cybergrooming is crucial for both intervention and detection. Research has categorized grooming into distinct stages, such as access, relationship-building, manipulation, exploitation, and concealment \cite{Anggraeny2023}. Linguistic and behavioral analyses have been conducted to compare online grooming strategies with offline tactics, identifying unique features of digital grooming interactions \cite{black2015linguistic, grant_assuming_2016}. These findings aid in refining detection models and improving response strategies for law enforcement and child protection agencies.

Table~\ref{tab:research-objects-ss} shows that most research in social sciences on cybergrooming focuses on \textit{risk analysis} (44.19\%) and \textit{offender profiling} (41.86\%). \textit{Prevention and awareness} (20.93\%) and \textit{behavioral stages} (9.30\%) are less frequently addressed, indicating a primary focus on understanding risks and offender profiles over awareness initiatives and intervention planning.

A limitation of existing research is the relatively lower emphasis on \textit{prevention and awareness} and \textit{behavioral stages}, critical for proactive intervention. This gap highlights the need to bridge theoretical findings with practical applications, such as integrating awareness programs into law enforcement training and developing real-time intervention tools.

To address these gaps, social science research on cybergrooming could benefit from techniques developed in computational sciences, where detection-focused approaches dominate. Computational methods, including machine learning models and natural language processing, can enhance the identification of grooming stages and provide scalable tools for raising awareness. By integrating these techniques, social science research could strengthen preventive strategies and foster a more proactive response to cybergrooming.

\begin{table}[t]
\small
    \centering
    \caption{Evaluation Methods in the Reviewed Social Science Research Works on Cybergrooming}
    \label{tab:evaluation-methods-social}    
    \vspace{-3mm}
    \begin{tabular}{p{2.5cm} p{8.5cm} p{3cm}}
%    \hline
 %   \multicolumn{3}{c}{{\textbf{\gray Social Sciences}}} \\
    \hline
        \multicolumn{1}{c}{\textbf{Evaluation Method}} & \multicolumn{1}{c}{\textbf{Description}} & \multicolumn{1}{c}{\textbf{Research Work}} \\
    \hline
        Questionnaires & Collect quantitative data through structured questions; repeated to assess short- and long-term effects. Used to study online behaviors, victimization, demographics, personality, and other factors. & \cite{almeida2024online, Calvete_Orue_GamezGuadi_2022, Finkelhor22-online-sexual-offense, gamez2023stability, HERNANDEZ2021106569, wachs2020routine, Wachs2012, Weingraber2020, sani_social_2021, gamez-guadix_construction_2018, wachs2016cross, gamez2018persuasion, schoeps2020risk, carmo2023knowledge, villacampa2017online} \\
    \hline
        Document Analysis & Qualitative review of documents to gather insights on research topics. Analyzed legal texts, books, journals, and other documents on cybergrooming and child protection laws. & \cite{Anggraeny2023, Fernandes_et_al_2023, ringenberg2022prepost, de_santisteban_progression_2018, choo_responding_2009, dolev-cohen_parental_2024, craven_current_2007, shannon2008sweden, seymour2021discursive} \\
    \hline
        Interviews & Collect in-depth information via conversations. Explored themes like parental awareness, criminal motives, and social media risks. & \cite{Anggraeny2023, Dorasamy_etal_2021, Fernandes_et_al_2023, de_santisteban_progression_2018, kloess2019case, Chiu2022onlinegrooming, christiansen2024hidden, khotimah2024child} \\
    \hline
        Comparative Analysis & Compare items to highlight similarities/differences. Analyzed phases of cybergrooming investigations. & \cite{Fernandes_et_al_2023, soldino_criminological_2024, broome_psycho-linguistic_2020} \\
    \hline
        Deductive Technique & Draw specific conclusions from general principles. Investigated stages of cyber child grooming. & \cite{Anggraeny2023, soldino_criminological_2024} \\
    \hline
        Data Visualization Techniques & Methods for visual representation of data. Treemap Diagrams and Word Clouds were used to highlight key terms and visualize word frequencies related to cybergrooming. & \cite{Dorasamy_etal_2021} \\
    \hline
        Statistical Analysis & Various methods to analyze and interpret quantitative data, including statistical software. ANCOVA, ANOVA, chi-square test, t-test, logistic regression, multiple linear regression analysis, and Pearson’s correlation used for hypothesis testing. & \cite{almeida2024online, Finkelhor22-online-sexual-offense, gamez2023stability, HERNANDEZ2021106569, wachs2020routine, Wachs2012, Weingraber2020, Calvete_Orue_GamezGuadi_2022, soldino_criminological_2024, gamez-guadix_construction_2018, craven_current_2007,wachs2016cross, van2016behavioural, gamez2018persuasion, schoeps2020risk} \\
    \hline
        Other Tools & Tools to interpret social science data and phenomena. Computer-Mediated Discourse Analysis, Structural Equation Modeling, thematic analysis, discourse analysis, and generic qualitative inquiry. & \cite{Lorenzo-Dus_Izura_2017, Dorasamy_etal_2021, HERNANDEZ2021106569, broome_psycho-linguistic_2020, grant_assuming_2016, broome_investigation_2024, schoeps2020risk, carmo2023knowledge} \\
    \hline
    \end{tabular}
    \vspace{-5mm}
\end{table}

\begin{table}[t]
\small 
    \centering
    \caption{Evaluation Metrics in the Reviewed Social Science Research Works on Cybergrooming}
    \label{tab:evaluation-metrics-social}    
\vspace{-3mm}    
\begin{tabular}{p{3.5cm} p{7.5cm} p{3cm}} 
    \hline
   % \multicolumn{3}{c}{\textbf{\gray Social Sciences}} \\
   % \hline
        \multicolumn{1}{c}{\textbf{Evaluation Metric}} & \multicolumn{1}{c}{\textbf{Description}} & \multicolumn{1}{c}{\textbf{Research Work}} \\
    \hline
        Chi-square Statistics & An assessment of variable association, with higher values indicating greater significance. & \cite{Finkelhor22-online-sexual-offense, gamez2023stability, HERNANDEZ2021106569, wachs2020routine, Wachs2012, Weingraber2020} \\
    \hline
        Frequency & A measure of theme frequency, identifying prevalent patterns. & \cite{Anggraeny2023, Dorasamy_etal_2021, Fernandes_et_al_2023, Lorenzo-Dus_Izura_2017, broome_psycho-linguistic_2020, sani_social_2021, broome_investigation_2024} \\
    \hline
        $t$-statistics & An evaluation of the statistical significance of differences between group means. & \cite{Calvete_Orue_GamezGuadi_2022, HERNANDEZ2021106569, wachs2020routine, Weingraber2020, soldino_criminological_2024, sani_social_2021, broome_investigation_2024} \\
    \hline
        Comparative Fit Index (CFI) & An indicator of model fit, with values closer to one suggesting a better fit. & \cite{HERNANDEZ2021106569, wachs2020routine, gamez-guadix_construction_2018, schoeps2020risk} \\
    \hline
        Composite Reliability Coefficient & Testing the internal consistency in Structural Equation Modeling models. & \cite{HERNANDEZ2021106569, wachs2020routine} \\
    \hline
        Confidence Interval & A range that likely contains the true parameter, indicating the level of precision. & \cite{wachs2020routine, Wachs2012} \\
    \hline
        Correlation Coefficients & Shows direction and strength of relationships between variables. & \cite{Calvete_Orue_GamezGuadi_2022, Weingraber2020} \\
    \hline
        Cronbach's Alpha & Measuring the internal consistency of responses. & \cite{HERNANDEZ2021106569, Wachs2012, gamez-guadix_construction_2018, schoeps2020risk} \\
    \hline
        Prevalence Rate & Measuring the proportion of population meeting a condition, indicating a phenomenon extent. & \cite{Finkelhor22-online-sexual-offense, Wachs2012} \\
    \hline
        RMSEA & Evaluating model fit; lower values suggest better fit. & \cite{HERNANDEZ2021106569, wachs2020routine, schoeps2020risk} \\
    \hline
        Standard Error & Estimating the variability of sample statistics, indicating precision. & \cite{Finkelhor22-online-sexual-offense, wachs2020routine} \\
    \hline
        Tucker-Lewis Index (TLI) & Assessing model fit; values close to one indicate good fit. & \cite{HERNANDEZ2021106569, wachs2020routine} \\
    \hline
        Average Variance Extracted (AVE) & Evaluating measurement model quality in Structural Equation Modeling. & \cite{HERNANDEZ2021106569} \\
    \hline
        Contingency Coefficient & Measuring association strength between two categorical variables. & \cite{Weingraber2020} \\
    \hline
        $f$-statistics & Evaluating overall significance of a regression model. & \cite{Wachs2012} \\
    \hline
        Guttman's Lambda 6 (G6) & Assessing data consistency; values near one indicate reliability. & \cite{Wachs2012} \\
    \hline
        Odds Ratio & Measuring association between two events; indicates their relationship. & \cite{Wachs2012} \\
    \hline
        Partial Eta Squared & Measuring the effect size and variance explained by a model. & \cite{Wachs2012, broome_psycho-linguistic_2020, sani_social_2021, broome_investigation_2024} \\
    \hline
        Regression Coefficients & Evaluating strength and direction of relationships between variables. & \cite{Calvete_Orue_GamezGuadi_2022} \\
    \hline
    \end{tabular}
    \vspace{-5mm}
\end{table}

\subsection{Risk Factors of Victims to Cybergrooming Identified in Social Science Cybergrooming Research} \label{subsec:ss-risk-factors}

Research on cybergrooming has identified several victim characteristics that may predict cybergrooming risk. These risk factors can be broadly categorized into \textbf{behavioral and social}, \textbf{personal and demographic}, \textbf{psychological and emotional}, and \textbf{technological} factors, each playing a role in increasing adolescent vulnerability. Figure~\ref{fig:risk-factors-model} presents a conceptual model illustrating the interactions among these risk categories, demonstrating how individual, environmental, and technological influences collectively contribute to cybergrooming susceptibility.

\begin{itemize}
    \item \textbf{Behavioral and Social Factors:} Engaging in \textit{Online Disclosure}~\cite{wachs2020routine, whittle2014their}, such as sharing private information and maintaining open profiles, increases cybergrooming risk. Adolescents under restrictive \textit{Parental Mediation}~\cite{wachs2020routine} tend to disclose more personal information online compared to those under instructive mediation, thereby heightening their vulnerability. Additionally, individuals with a \textit{Past Experience} of cyberbullying~\cite{Wachs2012} or a \textit{Willingness to Meet Strangers}~\cite{Wachs2012, whittle2014their} face significantly higher risks. Those who \textit{Spend extensive time online}~\cite{whittle2014their} also show an increased likelihood of encountering groomers.

    \item \textbf{Personal and Demographic Factors:} \textit{Gender}~\cite{Finkelhor22-online-sexual-offense, Wachs2012} and \textit{Age}~\cite{gamez2023stability} play a crucial role, with older adolescents and females generally facing greater risks. Additional demographic factors include \textit{Foreign birth}~\cite{gamez2023stability} and \textit{Sexual minority identification}~\cite{gamez2023stability}, both of which have been associated with higher vulnerability. Family circumstances, such as having \textit{Parents with lower education levels}~\cite{gamez2023stability}, \textit{Living with separated or divorced parents}~\cite{gamez2023stability, whittle2014their}, or experiencing \textit{Difficult family relations}~\cite{whittle2014their}, further elevate susceptibility. A \textit{Lack of education about online safety}~\cite{whittle2014their}, whether at home or in school, also contributes to increased exposure to online grooming risks.

    \item \textbf{Psychological and Emotional Factors:} Adolescents at higher risk often experience elevated \textit{anxiety}, \textit{depression}~\cite{gamez2023stability, gamez2021unraveling}, \textit{shame}~\cite{gamez2023stability}, and \textit{guilt}~\cite{gamez2023stability}, making them more susceptible to manipulative grooming tactics. \textit{Low self-esteem}~\cite{wachs2016cross, whittle2014their} and \textit{loneliness}~\cite{whittle2014their} further increase vulnerability. \textit{Disinhibition}~\cite{HERNANDEZ2021106569}, reducing self-restraint in online interactions, heightens susceptibility. Specific personality traits such as \textit{extraversion} and \textit{lack of empathy}~\cite{HERNANDEZ2021106569} in boys and \textit{narcissism}~\cite{HERNANDEZ2021106569} in girls have also been linked to greater grooming risk.
    
    \item \textbf{Technological Factors:} Adolescents with \textit{Internet access in their bedroom} or a \textit{Personal device}~\cite{wachs2020routine, whittle2014their} face heightened cybergrooming risk. Unsupervised internet use, especially in private spaces, enables groomers to establish prolonged, unmonitored interactions, complicating early intervention.
\end{itemize}
\begin{wrapfigure}{r}{0.6\textwidth}
    \centering
    \includegraphics[width=\linewidth]{figs/risk_factors_model.pdf}
    \caption{Conceptual model of cybergrooming risk factors: Interaction between individual, environmental, and technological influences.}
    \label{fig:risk-factors-model}
\end{wrapfigure}
The identified characteristics provide valuable insights into the risk factors associated with cybergrooming, highlighting how behavioral patterns, social environments, demographic variables, psychological traits, and technological access collectively shape adolescent vulnerability. Frequent online disclosure and a lack of parental guidance are significant behavioral risks, while emotional instability, such as low self-esteem and loneliness, further exacerbates susceptibility. Additionally, access to private internet spaces increases the opportunity for prolonged grooming attempts. 

Figure~\ref{fig:risk-factors-model} illustrates the conceptual model of cybergrooming risk factors, highlighting how behavioral, demographic, psychological, and technological influences contribute to increased online exposure, reduced parental supervision, and heightened psychological vulnerability. These intermediate factors collectively increase the overall risk of cybergrooming victimization.


However, despite these insights, certain limitations exist. Most available data rely on cross-sectional, self-reported measures, which may introduce biases and limit causal inference. Furthermore, existing research primarily focuses on Western adolescent populations, raising concerns about the generalizability of findings to diverse cultural contexts. Notably, there is a lack of research on how {\em age-specific and cultural variations} influence grooming susceptibility. For example, younger children may be more trusting, whereas older adolescents might engage in riskier online behaviors due to increased autonomy. Similarly, cultural differences in parenting styles and internet use norms may alter vulnerability patterns. Addressing these gaps requires more longitudinal and cross-cultural studies to better understand cybergrooming risks across different social and geographical contexts.

\subsection{Evaluation Methods and Metrics in Social Science Cybergrooming Research} \label{subsec:ss-evaluation-methods}

Research on cybergrooming employs diverse social science methods to understand behaviors, motivations, and social dynamics. These methods can be broadly classified into \textbf{quantitative} and \textbf{qualitative} approaches, each offering distinct advantages and limitations. While quantitative techniques excel at identifying patterns and generalizable trends, qualitative methods provide a deeper contextual understanding. A combination of both is often necessary for a comprehensive analysis of cybergrooming. Table~\ref{tab:evaluation-methods-social} provides a detailed comparison of these methods.

\textbf{Quantitative methods}: Surveys and structured questionnaires are frequently used to collect large-scale data on victimization experiences, online behaviors, and demographic influences. Longitudinal and cross-sectional studies track behavioral trends over time, enhancing understanding of risk factors. For instance, \citet{almeida2024online} used surveys to examine online sexual behaviors, while \citet{gamez2023stability} analyzed the link between cybergrooming prevalence and mental health in 746 adolescents. \citet{Finkelhor22-online-sexual-offense} conducted a large-scale survey of 2,639 adolescents to assess online sexual offenses, and \citet{wachs2020routine} collected data from 5,938 teenagers across six countries to study risk factors. These methods enable statistical analyses such as chi-square tests, regression models, and correlation assessments to determine relationships between key variables. However, their reliance on self-reported data may introduce bias, and imbalanced datasets can affect result accuracy.

\textbf{Qualitative methods}: Document analysis and interviews provide deep insights into the social and psychological aspects of cybergrooming. Document analysis examines legal policies, existing research, and judicial records to identify gaps in regulation and enforcement. For example, \citet{Anggraeny2023} analyzed court decisions, child protection laws, and criminal codes to understand the legal framework surrounding cyber child grooming. Interviews with victims, parents, educators, and law enforcement offer firsthand perspectives on online risks and grooming tactics. \citet{Dorasamy_etal_2021} interviewed parents to explore their awareness of online grooming risks, while \citet{Fernandes_et_al_2023} interviewed a law enforcement officer to develop a specialized investigative framework. Thematic analysis~\cite{Dorasamy_etal_2021} is frequently applied to extract recurring themes in interview data, enhancing interpretability. While qualitative research provides depth and contextual richness, findings are often difficult to generalize due to smaller sample sizes.

\textbf{Comparing Methods:} Quantitative approaches are most effective for identifying \textit{prevalence rates} and \textit{demographic trends}, while qualitative methods are superior for exploring \textit{psychological motivations}, \textit{perpetrator strategies}, and \textit{legal challenges}. However, an integrated approach combining both methods offers the most robust insights. For example, mixed-methods research can use surveys to establish statistical trends, followed by interviews to explain underlying causes. \citet{Fernandes_et_al_2023} combined literature reviews, thematic analysis, and comparative analysis to construct a comprehensive cybergrooming investigation framework. Similarly, \citet{Anggraeny2023} used deductive reasoning to analyze grooming stages, complementing legal document reviews with offender interviews.

\textbf{Evaluation Metrics:} Statistical tools validate findings and quantify relationships between key variables. Commonly used software includes R~\cite{gamez2023stability, Wachs2012}, SPSS~\cite{HERNANDEZ2021106569, wachs2020routine}, Stata~\cite{Finkelhor22-online-sexual-offense}, and Mplus~\cite{HERNANDEZ2021106569, wachs2020routine}. Chi-square tests, frequency counts, and t-tests assess group differences, while more advanced models rely on Comparative Fit Index (CFI) and Root Mean Square Error of Approximation (RMSEA) to evaluate model fit. Table~\ref{tab:evaluation-metrics-social} details these and additional metrics, such as Cronbach’s alpha for internal consistency and regression coefficients for predictive analysis.

A key takeaway is that \textbf{\em neither method alone is sufficient} to fully understand cybergrooming. While surveys can detect statistical patterns, interviews uncover emotional and behavioral nuances. Therefore, \textbf{\em future research should adopt interdisciplinary, mixed-methods approaches} to integrate large-scale quantitative analyses with rich qualitative insights, improving both the validity and applicability of cybergrooming studies.

\subsection{Evaluation Datasets Used in Scial Science Cybergrooming Research} \label{subsec:ss-eval-datasets}

This section describes the datasets commonly used in cybergrooming research within the social sciences:

\ctriangle{red} \textbf{Questionnaire Data}~\cite{almeida2024online, Calvete_Orue_GamezGuadi_2022, Finkelhor22-online-sexual-offense, wachs2020routine, HERNANDEZ2021106569, Wachs2012, Weingraber2020, wachs2016cross, gamez2018persuasion, schoeps2020risk, carmo2023knowledge, villacampa2017online, gamez-guadix_construction_2018}: Social science research often leverages questionnaire data, particularly longitudinal datasets, to investigate specific research questions and track changes over time. For instance, one study collected data from 2,639 respondents using online questionnaires to explore aspects of online sexual offenses \cite{Finkelhor22-online-sexual-offense}. Another used pre-test, 3-month, and 6-month follow-up data to evaluate the effects of an educational intervention on cybergrooming awareness \cite{Calvete_Orue_GamezGuadi_2022}. While these datasets provide large-scale insights, reliance on self-reported measures introduces biases like recall errors and social desirability effects. To mitigate these issues, researchers have employed \textit{data triangulation} by combining survey responses with behavioral logs or parental reports \cite{wachs2020routine}.

\ctriangle{cyan} \textbf{Interview Data}~\cite{Anggraeny2023, Dorasamy_etal_2021, Fernandes_et_al_2023, kloess2019case, Chiu2022onlinegrooming, christiansen2024hidden, khotimah2024child}: Interviews are widely used to collect structured demographic information and open-ended responses on cybergrooming experiences, offering rich qualitative insights. For example, \citet{Dorasamy_etal_2021} conducted interviews on social media use, sex education, and grooming risks, while \citet{Fernandes_et_al_2023} interviewed law enforcement officers to explore investigative challenges. To enhance credibility, some studies apply \textit{investigator triangulation}, where multiple researchers analyze transcripts independently to reduce subjective bias \cite{Fernandes_et_al_2023}. Additionally, member checking—where participants verify their responses—helps validate qualitative findings.


\begin{table}[t]
\small
\centering
\caption{Datasets Used in the Reviewed Social Science Cybergrooming Research}
\label{tab:datasets-social}
\vspace{-3mm}
\begin{tabular}{p{3cm} p{5cm} p{5cm} P{1cm}}
\toprule
\textbf{Source} & \multicolumn{1}{c}{\textbf{Dataset Content}} & \multicolumn{1}{c}{\textbf{Dataset Description}} & \textbf{Year}\\ 
\hline

\textbf{Childhood Online Sexual Abuse in the US} & 
Self-reported experiences of online sexual abuse from individuals aged 18–28, covering 11 abuse types. Data collected from 2,639 participants. & 
Survey-based dataset assessing the prevalence and dynamics of online sexual abuse, including grooming and image-based abuse, to inform prevention strategies. & 
2021 \textcolor{red}{$\blacktriangle$} \\
\multicolumn{4}{l}{\textbf{Link:} \url{https://www.icpsr.umich.edu/web/pages/index.html}}  \\
\midrule
\textbf{Online Child Sexual Abuse in Bangladesh} & 
Binary and continuous variables on internet use, online activity, and exposure to risks. & 
Mixed-method study with survey and interview data from grade 9–10 students in urban and rural schools. & 
2023 \textcolor{red}{$\blacktriangle$} \\
\multicolumn{4}{l}{\textbf{Link:} \url{https://data.mendeley.com/datasets/t8tm3ygfff/1}} \\
\bottomrule
\end{tabular}
\vspace{-5mm}
\end{table}

\ctriangle{lime} \textbf{Perverted Justice (PJ) Dataset}~\cite{black2015linguistic, Lorenzo-Dus_Izura_2017, van2016behavioural, lorenzo2016understanding, thomas2023offenders, lorenzo2020modus}: The PJ dataset~\cite{PJ-dataset}, consisting of chat logs between online predators and decoys, is one of the few available resources for studying grooming tactics. It has been used to analyze manipulation strategies, such as the strategic use of compliments to build trust \cite{Lorenzo-Dus_Izura_2017}. However, a key limitation is its reliance on decoy interactions, which may not fully replicate real victim-predator dynamics. Some studies attempt to address this gap by integrating findings from PJ chat logs with victim narratives obtained from legal case studies \cite{lorenzo2016understanding}, improving external validity.

\ctriangle{pink} \textbf{Document Analysis}~\cite{Anggraeny2023, Fernandes_et_al_2023, craven_current_2007, shannon2008sweden, seymour2021discursive}: Qualitative studies often use document analysis of books, academic literature, theses, and legal documents to complement primary data sources. For instance, \citet{Fernandes_et_al_2023} combined legal document analysis with interview data to construct a comprehensive framework for cybergrooming investigations. \citet{Anggraeny2023} analyzed child protection laws alongside offender interviews to assess how legal frameworks shape offender behaviors. While document analysis helps contextualize findings, it often depends on secondary sources, which may be outdated or lack direct relevance to current grooming behaviors.

While these datasets provide valuable insights into cybergrooming, they each have inherent biases and limitations. Questionnaire and interview data rely on self-reported responses, which can be influenced by memory recall, social desirability, or underreporting of sensitive information. To enhance reliability, some studies employ \textit{triangulation techniques}, such as combining interviews with observations from multiple sources~\cite{khotimah2024child}. Longitudinal datasets, while useful for tracking behavioral changes, often suffer from participant attrition, which can reduce data completeness over time. The PJ dataset, though widely used, may not capture real-world predator-victim interactions due to its decoy-based structure. Document analysis, while effective in examining legal and policy frameworks, often lacks direct empirical validation.

Table~\ref{tab:datasets-social} presents datasets commonly used in social science cybergrooming research, primarily relying on self-reported surveys and mixed-method studies to analyze victim experiences, risk factors, and online grooming prevalence.

These limitations highlight the need for more diverse, longitudinal, and context-rich datasets integrating multiple evidence sources. Future research should prioritize mixed-method approaches combining large-scale surveys, qualitative insights, case studies, and technological monitoring tools to enhance cybergrooming research depth and accuracy.

\subsection{Methods to Mitigate the Risk of Bias in Social Science Cybergrooming Research} \label{subsec:ss-methods-mitigate-risk-bias}

Biases in social science research on cybergrooming can affect study validity and reliability. To improve methodological rigor, researchers employ various mitigation strategies, each with strengths and limitations. Below, we discuss key biases and best practices to minimize their impact.

First, \textbf{\em experimental design biases}~\cite{Calvete_Orue_GamezGuadi_2022, Dorasamy_etal_2021, Wachs2012} arise from flaws in study design that influence outcomes. Best practices include {\em double-blind, randomized trials}~\cite{Calvete_Orue_GamezGuadi_2022}, where neither experimenters nor participants know the treatment assignment, reducing researcher influence. \citet{Calvete_Orue_GamezGuadi_2022} used this approach to assess an educational intervention on cybergrooming awareness. Another key strategy is {\em pre-tests}~\cite{Wachs2012} to identify design flaws before full data collection. \citet{Wachs2012} refined their questionnaire through pilot testing for clarity and reliability. Ensuring \textit{transparent participant selection}~\cite{Dorasamy_etal_2021} also prevents selection bias. \citet{Dorasamy_etal_2021} adopted a first-come, first-served approach in interviews to avoid preferential sampling. However, achieving fully neutral studies remains challenging, as transparency protocols must be rigorously maintained to prevent unintended biases.

Second, \textbf{\em missing data biases}~\cite{Finkelhor22-online-sexual-offense, HERNANDEZ2021106569} occur when participants leave survey items incomplete, leading to gaps in datasets. Common best practices include {\em Full Information Maximum Likelihood (FIML)}~\cite{HERNANDEZ2021106569, schoeps2020risk} and {\em data weighting}~\cite{Finkelhor22-online-sexual-offense} to estimate missing responses and adjust survey results. \citet{HERNANDEZ2021106569} applied FIML in their structural models to address missing responses, while \citet{Finkelhor22-online-sexual-offense} used weighted adjustments to improve result reliability. Another strategy involves {\em multiple imputation}, which replaces missing values with statistically plausible estimates, reducing bias from systematic data loss. Despite these methods, missing data remain a challenge, particularly when non-random dropout patterns introduce systematic errors. To further improve data completeness, researchers should adopt proactive approaches, such as conducting follow-up surveys and incentivizing full participation.

Third, \textbf{\em measurement biases}~\cite{Lorenzo-Dus_Izura_2017, wachs2020routine, black2015linguistic} arise from inaccuracies in data collection tools or procedures, affecting study reliability. Best practices include {\em inter-coder reliability checks}~\cite{Lorenzo-Dus_Izura_2017, black2015linguistic}, where multiple coders independently assess qualitative data for consistency. In multilingual research, a {\em standardized translation and back-translation process}~\cite{wachs2020routine} ensures conceptual consistency across languages. \citet{wachs2020routine} implemented a rigorous approach involving translation, independent back-translation, and version comparison to enhance accuracy. However, biases persist due to cultural nuances, varying survey interpretations, and self-report inaccuracies. To mitigate this, researchers should apply \textit{triangulation}, integrating multiple data sources (e.g., surveys, behavioral logs, and interviews) to cross-validate findings.

Finally, \textbf{\em sampling biases}~\cite{gamez2023stability, HERNANDEZ2021106569, Lorenzo-Dus_Izura_2017, thomas2023offenders} occur when certain groups are over- or underrepresented, limiting generalizability. Best practices include {\em random sampling}~\cite{gamez2023stability, HERNANDEZ2021106569, Lorenzo-Dus_Izura_2017}, giving all individuals in the target population an equal chance of selection. \citet{gamez2023stability} used a randomized approach to select 37 schools in Spain, ensuring regional diversity. \textit{stratified random sampling} improves representativeness by dividing participants into subgroups (e.g., age, gender, socioeconomic status) before selection. \citet{Lorenzo-Dus_Izura_2017} combined stratified and random sampling for a balanced PJ dataset. However, demographic imbalances and participant accessibility can still introduce bias. To address this, researchers should consider \textit{quota sampling} to maintain proportional demographic representation.

In summary, while these mitigation strategies significantly enhance research validity, adopting \textbf{\em best practices such as triangulation, stratified sampling, pre-testing, and standardized translation} can further reduce biases. Future studies should prioritize culturally adaptable research designs, integrate diverse data sources, and employ advanced statistical techniques to ensure the robustness and generalizability of cybergrooming research findings.

\subsection{Summary of Key Findings in Social Science Cybergrooming Research} \label{subsec:ss-summary-key-findings}

Our comprehensive review of cybergrooming studies in social sciences revealed several critical findings, highlighting actionable insights for policymakers, educators, and practitioners.

First, \textbf{\em education programs are essential} for effective cybergrooming prevention, as they increase awareness of risks and preventive strategies, significantly reducing incidents~\cite{Calvete_Orue_GamezGuadi_2022, whittle2014their}. These programs equip children with the skills to recognize and avoid risky online behaviors, emphasizing the need for widespread implementation. Policymakers should prioritize mandatory digital safety education in school curricula, ensuring children are taught how to identify and respond to grooming attempts from an early age.

Second, \textbf{\em family protection plays a key role} in preventing cybergrooming, with open communication between parents, caregivers, and children crucial~\cite{whittle2014their}. Active parental involvement in internet use and early safety education fosters a safer environment, encouraging children to share online experiences. Research shows that perceived parental supervision discourages grooming attempts and strengthens defense against predators~\cite{kamar2022parental}. Practitioners should promote parental training programs that teach effective internet monitoring and child-focused communication strategies.

Third, \textbf{\em training for adults working with children} is crucial, as many professionals, including teachers and child protection workers, are motivated to prevent cybergrooming but lack specific training on this type of abuse~\cite{carmo2023knowledge}. Targeted training on grooming signs and tactics would enable these adults to act as early responders, detecting warning signs more accurately and intervening effectively before grooming escalates. Governments should invest in professional development programs that integrate cybergrooming awareness into teacher and childcare training modules.

Fourth, \textbf{\em gender differences influence} cybergrooming dynamics, underscoring the need for prevention and intervention strategies~\cite{Finkelhor22-online-sexual-offense, HERNANDEZ2021106569, Wachs2012, Weingraber2020, gamez2021unraveling}. Grooming approaches and victim vulnerability often vary based on gender, with certain tactics more commonly used against specific groups. Prevention programs should incorporate gender-specific risk factors, ensuring that awareness campaigns address unique grooming strategies employed against different populations.

Fifth, \textbf{\em identifying risk factors for potential victims} is vital for designing targeted prevention measures~\cite{Finkelhor22-online-sexual-offense, gamez2023stability, HERNANDEZ2021106569, wachs2020routine, Wachs2012, Weingraber2020}. Awareness of \textbf{\em common grooming strategies}, such as praise~\cite{Anggraeny2023, Lorenzo-Dus_Izura_2017, gamez2018persuasion, black2015linguistic}, deception, gift-giving, and sexualization, is essential for developing countermeasures~\cite{gamez2021unraveling}. Online platforms should integrate AI-driven content moderation tools that flag grooming tactics in real time, helping to protect minors from predatory behaviors.

Sixth, \textbf{\em online grooming strategies differ from offline methods}, with offenders often pretending to be younger to reduce perceived threat~\cite{gamez2018persuasion, oconnell2003typology}. Offenders assess risks early and exploit vulnerabilities from the start of interactions~\cite{black2015linguistic, shannon2008sweden}, presenting unique challenges for online protection. Law enforcement agencies should develop specialized units trained to detect and respond to digital grooming tactics, using proactive monitoring and AI-driven forensic analysis.

Seventh, \textbf{\em grooming tactics vary by gender}, with offenders engaging in lengthy emotional conversations with girls to build trust, while boys often face more direct, explicit chats involving age deception~\cite{van2016behavioural, seymour2021discursive}. These differing approaches suggest tailored interventions are necessary, with particular attention to how groomers manipulate trust in one group and exploit vulnerability more directly in the other. Digital platforms should develop algorithmic detection models that recognize gender-specific grooming tactics, enabling more precise intervention strategies.

Finally, the COVID-19 lockdowns saw a surge in online grooming incidents, with increased links to sexting and sextortion, revealing new vulnerabilities during times of high internet use~\cite{almeida2024online}. \textbf{\em Peer/sibling victimization and sexual victimization} emerged as major predictors, underscoring the need for careful monitoring during such times. Policymakers should establish emergency digital safety protocols for crises such as pandemics, ensuring that increased online exposure does not leave minors more vulnerable to exploitation.

These findings underscore the urgent need for a multi-stakeholder approach involving policymakers, educators, parents, technology companies, and law enforcement agencies. Future research should focus on culturally adaptive prevention strategies, real-time detection systems, and cross-border collaboration to combat cybergrooming effectively.
\subsection{Limitations \& Lessons Learned in Social Science Cybergrooming Research} \label{subsec:ss-discussions}

Our comprehensive review of cybergrooming studies in social sciences revealed several limitations and key lessons learned. One major limitation is the \textbf{\em over-reliance on self-reported data}, which can introduce recall bias, social desirability bias, or underreporting, particularly in studies on sensitive topics like cybergrooming~\cite{almeida2024online, black2015linguistic, Calvete_Orue_GamezGuadi_2022, Finkelhor22-online-sexual-offense, gamez2023stability, HERNANDEZ2021106569, Weingraber2020, wachs2016cross, quayle2014rapid, gamez2018persuasion, christiansen2024hidden}. \textbf{\em Privacy concerns} further complicate data collection, particularly when involving minors, as ethical restrictions limit researchers’ access to critical but sensitive information~\cite{Dorasamy_etal_2021, Fernandes_et_al_2023, Lorenzo-Dus_Izura_2017, Wachs2012, van2016behavioural}. To address these challenges, future research should incorporate \textit{behavioral logging methods} and \textit{secure, anonymized data collection platforms} to improve data accuracy while ensuring ethical compliance.

Another key limitation is the \textbf{\em limited generalizability} of findings due to small sample sizes and reliance on potentially outdated data, restricting applicability to broader populations~\cite{Anggraeny2023, Dorasamy_etal_2021, Fernandes_et_al_2023, Finkelhor22-online-sexual-offense, gamez2023stability, HERNANDEZ2021106569, wachs2020routine, Weingraber2020, whittle2014their, wachs2016cross, van2016behavioural, gamez2018persuasion, kloess2019case, carmo2023knowledge, kamar2024relevance, Chiu2022onlinegrooming, christiansen2024hidden}. Future research should prioritize \textit{multi-country, large-scale studies} that include diverse age groups and socioeconomic backgrounds to enhance external validity. Additionally, longitudinal studies can track behavioral changes over time, providing deeper insights into grooming dynamics.

\textbf{\em Cultural and demographic gaps} remain a significant issue, as much of the existing research is conducted in Western contexts, limiting the understanding of cybergrooming risks in other cultural settings~\cite{wachs2016cross, villacampa2017online}. Additionally, the \textbf{\em underrepresentation of sexual minorities} highlights a critical gap, as these groups may experience distinct grooming patterns that current research fails to capture~\cite{Calvete_Orue_GamezGuadi_2022, Wachs2012, gamez2023stability}. Future studies should employ \textit{intersectional approaches} that examine how factors such as gender identity, ethnicity, and cultural norms influence grooming susceptibility. Expanding research to underrepresented regions, including low- and middle-income countries, would improve global understanding and inform more inclusive prevention strategies.

The lessons learned emphasize several key areas for enhancing prevention and intervention strategies. There is strong consensus on the need for \textbf{\em educational programs} to raise awareness about cybergrooming risks and prevention techniques, with studies showing that structured digital literacy programs can significantly reduce victimization rates~\cite{Dorasamy_etal_2021, wachs2020routine, wachs2016cross, villacampa2017online}. \textbf{\em Early education and parental involvement} are particularly crucial, as open discussions between parents and adolescents can foster safer online behaviors and increase reporting of suspicious interactions~\cite{Dorasamy_etal_2021, wachs2016cross}. 

Providing targeted training for professionals who work with children, such as teachers and social workers, could enable them to act as \textbf{\em first responders} in cybergrooming cases~\cite{carmo2023knowledge}. Future policies should encourage mandatory cybergrooming training for educators and youth service providers, equipping them with the tools to detect and intervene in early-stage grooming interactions. Additionally, tailored programs that consider \textbf{\em cultural differences} and \textbf{\em gender dynamics} are essential, as these factors influence the effectiveness of prevention efforts~\cite{wachs2016cross, villacampa2017online}. Governments and international organizations should support research collaborations to develop culturally adaptive intervention models.

Another lesson learned is the sophistication of \textbf{\em grooming tactics}, including deception, emotional manipulation, and trust-building strategies. Many grooming interactions now occur across multiple digital platforms, making detection more challenging~\cite{gamez2021unraveling}. Researchers and policymakers should advocate for \textit{enhanced AI-driven content monitoring tools} that can detect early grooming patterns and alert moderators in real time. Collaborations between academic institutions and tech companies could facilitate the development of automated tools that detect cross-platform grooming behaviors.

These limitations and lessons highlight the urgent need for \textbf{\em culturally inclusive, longitudinal, and multi-disciplinary approaches} in cybergrooming research. Future studies should prioritize diverse population samples, leverage emerging technologies for data collection and analysis, and strengthen collaborations between researchers, educators, and policymakers to enhance prevention, detection, and response strategies.

\section{Cybergrooming Research in Computational Sciences} \label{sec:cybergrooming-CS}

Over the past decade, cybergrooming has attracted increasing attention in computational sciences, leading to the development of technological solutions aimed at identifying and mitigating online grooming threats. Unlike social science research, which focuses on behavioral risk factors, victim experiences, and intervention strategies, computational studies primarily address detection through machine learning (ML), natural language processing (NLP), and data mining techniques. The emphasis on detection stems from the urgent need to automate identification on digital platforms, where vast volumes of online interactions make manual monitoring impractical. This section reviews computational studies, highlighting key research objectives, methods, datasets, and challenges.

\subsection{Research Objectives in Computational Science Cybergrooming Research} \label{subsec:cs-research-objectives}

\begin{table}[t]
\small 
    \caption{Research Objectives in Computational Sciences for Cybergrooming} 
    \label{tab:research-objects-cs}
    \vspace{-3mm}
    \begin{tabular}{p{5cm}P{2.5cm}P{2.5cm}}
    \hline
         \multicolumn{1}{c}{\textbf{Research Objectives}} & \textbf{Percentage} & \textbf{Number of Articles} \\
    \hline
        \raisebox{-0.05cm}{\cbox{green!60!black}} Detecting Cybergrooming & 83.33\% & 30 \\
    \hline
        \raisebox{-0.05cm}{\cbox{blue!60}}  Educating on Cybergrooming & 11.11\% & 4 \\
    \hline
        \raisebox{-0.05cm}{\cbox{red!70}} Identifying Grooming Stages & 11.11\% & 4 \\
    \hline
        \raisebox{-0.05cm}{\cbox{yellow!90}} Evaluating Victim Traits & 2.78\% & 1 \\
    \hline
    \end{tabular}
    \vspace{1mm}
    
    \small{(Note: An article may appear multiple times if it addresses multiple objectives, each counted individually.)}
    \vspace{-5mm}
\end{table}

Computational approaches to cybergrooming research center on four primary objectives:

\cbox{green!60!black} \textbf{Detecting Cybergrooming}~\cite{kim2020analysis, Ashcroft_Kaati_Meyer_2015, bours2019detection, sulaiman2019classification, pranoto2015logistic, bogdanova2014exploring, cano2014detecting, isaza2022classifying, amer2021detection, Borj_Bours_2019, Eilifsen_Shrestha_Bours_2023, fauzi2020ensemble, Munoz_Isaza_Castillo_2021, cook2023protecting, rezaee2023detecting, Gunawan16, Lykousas_Patsakis_2022, gupta2012characterizing, anderson2019intelligent, elzinga2012analyzing, ringenberg2024assessing, ALKHATEEB201614, zuo2018grooming, michalopoulos2010towards, milon2022take, waezi2024osprey, amuchi2012identifying, fauzi2023identifying, Michalopoulos14-gars, prosser2024helpful}: Most computational research focuses on detecting grooming behaviors, using ML and NLP to analyze chat logs for patterns of predatory behavior. Sections~\ref{subsec:cs-features} and~\ref{subsec:cs-eval} discuss detection techniques in greater detail.

\cbox{blue!60} \textbf{Educating on Cybergrooming}~\cite{Wang21-eancs, guo23-is, kim2023ai, rita2021chatbot}: Some studies aim to develop educational tools that raise awareness and promote prevention. \citet{Wang21-eancs} introduced SERI, a chatbot that simulates grooming conversations to educate users, while \citet{rita2021chatbot} created an interactive website providing resources and reporting mechanisms. \citet{kim2023ai} designed an AI-based simulation to enhance children's cybersecurity awareness.

\cbox{red!70} \textbf{Identifying Grooming Stages}~\cite{cano2014detecting, oconnell2003typology, elzinga2012analyzing, farag2023enhanced}: Understanding how grooming unfolds over time is another research goal. \citet{oconnell2003typology} proposed a six-stage grooming model, which has been leveraged in ML-based analyses~\cite{cano2014detecting}, applying machine learning to detect these stages in chat logs. \citet{farag2023enhanced} used the Luring Communication Theory Model (LCTM)~\cite{olson2007entrapping} to classify grooming patterns.

\cbox{yellow!90!} \textbf{Evaluating Victim Traits}~\cite{guo2023text}: A smaller subset of research examines victim behavior in online interactions. \citet{guo2023text} utilized text-mining techniques such as Linguistic Inquiry and Word Count (LIWC) to analyze chat data and identify social-psychological traits of potential victims.

Table~\ref{tab:research-objects-cs} summarizes the distribution of research objectives. The dominance of detection-based studies (83.33\%) highlights the field’s focus on automating cybergrooming identification, likely due to the scalability of ML models in processing vast amounts of chat data. In contrast, educational interventions and stage-based analysis each constitute only 11.11\% of studies, while research on victim traits is notably scarce (2.78\%).

This emphasis on detection reflects an urgent need to equip platforms with real-time monitoring tools to prevent grooming before escalation. However, the limited focus on education, behavioral analysis, and victim profiling suggests critical gaps in proactive prevention strategies. Unlike social science research, which examines root causes and social interventions, computational approaches remain largely reactive, prioritizing post-interaction detection rather than preemptive intervention. Future research should explore integrating \textbf{\em detection with prevention}, incorporating ML-driven risk assessments into educational platforms, and leveraging AI-generated conversational simulations to prepare children and caregivers for identifying early grooming tactics. Expanding research on victim vulnerability factors can also improve predictive modeling, allowing for more targeted intervention strategies.

To create a more comprehensive approach to cybergrooming prevention and mitigation, future computational research should balance detection efforts with investment in \textbf{\em prevention, intervention, and victim behavior analysis}. Collaborations between computational scientists and social scientists can help bridge this gap, ensuring that technical solutions align with behavioral insights to maximize impact.

\subsection{Key Features of Cybergrooming Used in Computational Science Cybergrooming Research} \label{subsec:cs-features}

Developing detection algorithms to identify grooming activities is central to cybergrooming research. Key features used in grooming detection include:

\begin{itemize}
    \item \textbf{Content Features}: Analyzing shared text material, including {\em content patterns}~\cite{cano2014detecting, farag2023enhanced} for text complexity, readability, and sentence length; {\em discourse patterns}~\cite{cano2014detecting} to assess idea flow and semantic profiling of predatory behaviors; specific {\em grooming characteristics}~\cite{Ashcroft_Kaati_Meyer_2015, Gunawan16, cook2023protecting} to identify grooming behaviors; and tracking {\em frequency of images and URLs}~\cite{Ashcroft_Kaati_Meyer_2015} for external content sharing. These features are widely applied in industry settings, such as automated moderation tools in social media platforms and chat-based detection systems used by law enforcement agencies to flag potential grooming interactions in real time.

    \item \textbf{Emotional Features}: Capturing tone and emotions to reveal hidden intentions, including {\em emoticons}~\cite{Ashcroft_Kaati_Meyer_2015, Borj_Bours_2019} used to express emotions in chat interactions and {\em sentiment analysis}~\cite{bogdanova2014exploring, cano2014detecting} to detect sentiments like joy, sadness, or anger. These techniques have been integrated into AI-driven monitoring systems, such as child protection software that alerts parents or moderators when concerning sentiment patterns emerge in conversations.

    \item \textbf{Psychological Features}: Reflecting personality traits that signal grooming tendencies, including {\em psycho-linguistic patterns}~\cite{Borj_Bours_2019, cano2014detecting, gupta2012characterizing, kim2023ai} (often using LIWC for emotional and cognitive cues) and {\em psychological traits}~\cite{bogdanova2014exploring}, such as neuroticism, which may indicate predatory behavior. Law enforcement agencies utilize these analyses in forensic cybercrime investigations, where suspect chat histories are examined for manipulation tactics and personality-driven behavioral patterns.

    \item \textbf{Textual Features}: Derived from word frequency, phrases, structure, and style, textual techniques include Bag of Words for analyzing word and character patterns~\cite{bogdanova2014exploring, Borj_Bours_2019, bours2019detection, cano2014detecting, fauzi2020ensemble, Munoz_Isaza_Castillo_2021}; TF-IDF (Term Frequency-Inverse Document Frequency) with ML classifiers for identifying predators in chats~\cite{Borj_Bours_2019, Eilifsen_Shrestha_Bours_2023, fauzi2020ensemble, Gunawan16, pranoto2015logistic, anderson2019intelligent, farag2023enhanced}; Part of Speech analysis to detect predators’ lexical behaviors~\cite{Ashcroft_Kaati_Meyer_2015, cano2014detecting, fauzi2023identifying}; and TF with ML algorithms to measure suspect word frequency~\cite{Eilifsen_Shrestha_Bours_2023, fauzi2020ensemble}. These features are extensively used in industry-grade cyber defense tools, such as AI-powered content moderation systems deployed by social media companies to automatically filter predatory content in messaging platforms.

\end{itemize}

\textbf{\em Real-World Applications}: These features have enabled practical cybergrooming detection tools. Law enforcement agencies like the FBI and INTERPOL use ML-based systems to analyze chat data and identify grooming. Social media platforms such as Facebook, Instagram, and Discord deploy NLP-based moderation to detect and prevent grooming conversations, enabling faster intervention. Additionally, nonprofit organizations and educational platforms use AI-driven chatbots to educate children about grooming risks through interactive simulations with real-time detection.

\textbf{\em Limitations and Future Directions}: While cybergrooming detection research has made significant strides, notable challenges remain. {\em Generalizability across languages and cultures} is a key limitation, as most NLP models are trained on English-language datasets and may not accurately detect grooming behaviors in other linguistic or cultural contexts. Grooming tactics vary across regions, and certain conversational nuances may not be effectively captured by current models, leading to false positives or missed detections. Future research should focus on developing {\em multilingual and culturally adaptive models}, ensuring that detection techniques are robust across diverse populations.

Another limitation is the reliance on {\em text-based analysis}, which may overlook crucial non-verbal cues such as voice tone, video behaviors, or even behavioral metadata (e.g., frequency and time of messages). Addressing this gap requires integrating {\em multimodal detection approaches} that incorporate video, audio, and behavioral analytics into cybergrooming detection frameworks. Collaboration between researchers, industry professionals, and policymakers is essential to refine detection models and enhance the effectiveness of digital safety measures.

These insights emphasize the importance of balancing technological detection with ethical and practical considerations. Future efforts should explore {\em privacy-preserving AI models} that respect user data rights while improving the accuracy and effectiveness of cybergrooming detection across global digital platforms.

\subsection{Evaluation Methods and Metrics in in Computational Science Cybergrooming Research}\label{subsec:cs-eval}

Researchers in computational sciences studying cybergrooming use various machine learning (ML), text-mining, and natural language processing (NLP) techniques to develop detection tools. ML models, particularly those using NLP, dominate due to the \textbf{\em text-heavy nature} of grooming interactions.  Techniques like {\em tokenization} and {\em sentiment analysis} aid detection by capturing nuanced textual details but require {\em high contextual accuracy} and {\em quality data}, posing challenges in handling {\em cultural nuances} and {\em implicit meanings} that demand careful model tuning. \citet{amer2021detection} employed word embeddings and pre-trained models like GloVe for effective text feature extraction, enhancing detection accuracy. \citet{anderson2019intelligent} used the {\em bag of words (BoW)} method to identify key lexical features in online child grooming texts. \citet{bogdanova2014exploring} examined high-level feature extraction techniques, such as {\em sentiment analysis} and {\em parsing}, to analyze textual features reflecting emotional and psychological states, identifying predatory behavior.

Beyond traditional NLP techniques, advanced models like \textbf{\em Support Vector Machines (SVMs)} and \textbf{\em Convolutional Neural Networks (CNNs)} have been applied. \citet{Gunawan16} deployed SVM alongside {\em k-nearest neighbors (KNN)} to detect online child grooming conversations, achieving high accuracy in classifying conversations based on predefined grooming characteristics. \citet{rezaee2023detecting} explored {\em contrastive chat embeddings} to enhance the effectiveness of SVMs by distinguishing typical from atypical conversation patterns. Deep learning models such as CNNs have been leveraged for their ability to detect complex patterns in unstructured text data. \citet{Munoz_Isaza_Castillo_2021} used CNNs to extract invariant features across chat logs, highlighting their utility in monitoring online interactions. However, CNNs require {\em large, annotated datasets}, which are often costly and scarce. Simpler models, such as {\em Naïve Bayes}, remain relevant for their {\em ease of interpretation}, although they may oversimplify linguistic dependencies. 

The effectiveness of ML models in cybergrooming detection is heavily influenced by \textbf{\em dataset composition}. Many datasets (e.g., PJ), focus on {\em predator-centric interactions}, leading to biases where models excel at identifying offenders but struggle to detect vulnerable victims. \citet{guo2023text} highlighted this limitation, showing that current models classify groomer intent more accurately than victim susceptibility. Additionally, {\em linguistic and cultural biases} in predominantly English-language datasets reduce accuracy across different languages and contexts. To address these issues, researchers should adopt {\em data augmentation techniques} and develop {\em multilingual training corpora} to improve model generalizability.

To evaluate these tools, studies employ various metrics, summarized in Table~\ref{tab:evaluation-metrics-computational}. Key metrics include {\em precision and recall}, which are vital for detection accuracy. Precision minimizes false positives, essential in social and legal contexts, while recall ensures comprehensive case coverage. The {\em F-Score} balances both but may be sensitive to data imbalances, prompting the use of {\em ROC curves and AUROC scores} for a refined evaluation of model discrimination capabilities. Additionally, \citet{Eilifsen_Shrestha_Bours_2023} utilized {\em Youden’s Index} to optimize sensitivity and specificity, ensuring effective differentiation between grooming and non-grooming conversations. 

\textbf{\em Human evaluation} is also employed in computational cybergrooming research. \citet{Wang21-eancs} randomly selected 200 conversation samples to be evaluated by three human graders, who compared chatbot-generated responses to those from the PJ dataset, determining which response was more appropriate based on historical conversational context. This approach ensures that chatbot responses are both {\em realistic} and {\em appropriate for preventing cybergrooming}.

Addressing these challenges requires a more \textbf{\em comprehensive, multidisciplinary approach}. Future research should focus on developing {\em multilingual NLP models} to improve detection across linguistic and cultural settings. Additionally, {\em integrating multimodal features}, such as behavioral metadata and voice sentiment analysis, could enhance model robustness beyond text-based analysis. Ethical considerations should also be prioritized, ensuring that AI-driven detection systems balance {\em privacy concerns} with effective intervention strategies. Collaboration between computational and social scientists is essential to refine detection frameworks and integrate them into real-world applications, bridging the gap between technological advancements and practical interventions.

\begin{table}[t]
\small 
    \centering
    \caption{Evaluation Methods in Computational Science Studies on Cybergrooming}
    \label{tab:evaluation-methods-computational}
\vspace{-3mm}
    \begin{tabular}{p{3.5cm} p{7.5cm} p{3cm}}
%    \hline
%    \multicolumn{3}{c}{\textbf{\gray Computational Sciences}} \\
    \hline
         \multicolumn{1}{c}{\textbf{Evaluation Method}} &  \multicolumn{1}{c}{\textbf{Description and Application}} & \multicolumn{1}{c}{\textbf{Research Work}} \\
    \hline
        Natural Language Processing (NLP) & Techniques for processing and analyzing natural language data, including tokenization, parsing, and sentiment analysis to be applied in various studies on text data & \cite{amer2021detection, gupta2012characterizing, bogdanova2014exploring, cano2014detecting, Borj_Bours_2019, Munoz_Isaza_Castillo_2021, Lykousas_Patsakis_2022, Eilifsen_Shrestha_Bours_2023, guo2023text, isaza2022classifying, anderson2019intelligent, cook2023protecting, farag2023enhanced, ringenberg2024assessing} \\
    \hline
        Support Vector Machine (SVM) & A supervised learning model for classification and regression, identifying hyperplanes that separate classes in feature space & \cite{bogdanova2014exploring, Gunawan16, Borj_Bours_2019, sulaiman2019classification, anderson2019intelligent, rezaee2023detecting} \\
    \hline
        Convolutional Neural Networks (CNNs) & Deep neural networks primarily for visual imagery, applied to text for feature extraction and classification & \cite{Munoz_Isaza_Castillo_2021, isaza2022classifying, guo23-is} \\
    \hline
        LIWC (Linguistic Inquiry and Word Count) & A text analysis tool measuring word category usage across texts, useful for psychological and social research & \cite{gupta2012characterizing, cano2014detecting, kim2023ai} \\
    \hline
        Naïve Bayes & A probabilistic classifier based on Bayes' theorem to be effective for text classification tasks due to its assumption of feature independence & \cite{Gunawan16, sulaiman2019classification} \\
    \hline
        Agile Software Development & Methodologies for efficient project management in software development, emphasizing flexibility and customer feedback & \cite{rita2021chatbot} \\
    \hline
        Large Language Models (LLMs) & Advanced transformer models such as BERT, T5, RoBERTa, and BERT for NLP tasks, known for their contextual understanding of language & \cite{Wang21-eancs, rezaee2023detecting, prosser2024helpful} \\
    \hline
        Binary Logistic Regression & A model predicting binary outcomes by estimating probabilities, useful for classifying observations into two categories & \cite{pranoto2015logistic} \\
    \hline
        $k$-Nearest Neighbors & An instance-based learning algorithm classifying data by comparing it to similar data points in the training set & \cite{Gunawan16} \\
    \hline
        Linear Discriminant Analysis & A method finding linear combinations of features that best separate classes, used in pattern recognition tasks & \cite{Lykousas_Patsakis_2022} \\
    \hline
        Psycho-linguistic profiling & Analysis of text to understand psychological states and traits, often using tools like LIWC & \cite{gupta2012characterizing} \\
    \hline
        Random Forest & Combining multiple decision trees for accurate predictions while reducing overfitting & \cite{Gunawan16} \\
    \hline
        Recurrent Neural Networks & Neural networks with directed connections capable of handling temporal sequences for dynamic behavior & \cite{kim2020analysis, guo23-is} \\
    \hline
        Universal Sentence Encoder & Using LSTM to generate message vectors, preserving word relationships for conversational context classification & \cite{kim2020analysis} \\
    \hline
        Word2Vec & Converting words into vector representations to capture semantic similarities, aiding in classification and clustering & \cite{Munoz_Isaza_Castillo_2021} \\
    \hline
        Temporal Concept Analysis (TCA) & TCA with Temporal Relational Semantic Systems, conceptual scaling, and nested line diagrams to analyze chat conversations, providing insights into the temporal dynamics and context of conversations for potential risk detection & \cite{elzinga2012analyzing} \\
    \hline
        Game Theory & Proposing game-theoretic approaches to model and simulate threat scenarios & \cite{kim2023ai} \\
    \hline
    \end{tabular}
    \vspace{-5mm}
\end{table}

\begin{table}[t]
\small 
    \centering
    \caption{Evaluation Metrics in Computational Science Studies on Cybergrooming}
    \label{tab:evaluation-metrics-computational}
\vspace{-3mm}
    
    \begin{tabular}{p{3.5cm} p{7.5cm} p{3cm}}
%    \hline
%    \multicolumn{3}{c}{\textbf{\gray Computational Sciences}} \\
    \hline
        \multicolumn{1}{c}{\textbf{Evaluation Metric}} & \multicolumn{1}{c}{\textbf{Description and Application}} & \multicolumn{1}{c}{\textbf{Research Work}} \\
    \hline
        Precision & Proportion of true positives among all positive predictions, helping identify relevant cases in classification tasks. & \cite{cano2014detecting, Ashcroft_Kaati_Meyer_2015, Borj_Bours_2019, bours2019detection, fauzi2020ensemble, kim2020analysis, Munoz_Isaza_Castillo_2021, Eilifsen_Shrestha_Bours_2023, amer2021detection, isaza2022classifying, pranoto2015logistic, sulaiman2019classification, cook2023protecting, rezaee2023detecting} \\
    \hline
        Accuracy & Measures the proportion of correctly identified cases (both positives and negatives) out of the total cases examined, indicating overall model performance. & \cite{bogdanova2014exploring, Ashcroft_Kaati_Meyer_2015, Gunawan16, Borj_Bours_2019, Munoz_Isaza_Castillo_2021, fauzi2020ensemble, Eilifsen_Shrestha_Bours_2023, amer2021detection, isaza2022classifying, pranoto2015logistic, sulaiman2019classification, cook2023protecting, rezaee2023detecting} \\
    \hline
        Recall & Proportion of actual positives correctly identified, demonstrating the model’s ability to capture all relevant cases. & \cite{cano2014detecting, Ashcroft_Kaati_Meyer_2015, Borj_Bours_2019, bours2019detection, fauzi2020ensemble, kim2020analysis, Munoz_Isaza_Castillo_2021, Eilifsen_Shrestha_Bours_2023, isaza2022classifying, pranoto2015logistic, sulaiman2019classification, cook2023protecting, rezaee2023detecting} \\
    \hline
        F Score & Combines precision and recall into a single metric, with the F1 score balancing them equally and the $F\beta$ score adjusting this balance based on specific needs. & \cite{cano2014detecting, Borj_Bours_2019, bours2019detection, kim2020analysis, fauzi2020ensemble, Munoz_Isaza_Castillo_2021, Eilifsen_Shrestha_Bours_2023, amer2021detection, isaza2022classifying, sulaiman2019classification, cook2023protecting, rezaee2023detecting} \\
    \hline
        ROC Curve and AUROC & The ROC curve plots true positive rates against false positive rates; AUROC represents the area under the curve, indicating the model's class distinction ability. & \cite{Munoz_Isaza_Castillo_2021, Eilifsen_Shrestha_Bours_2023, isaza2022classifying, amer2021detection} \\
    \hline
        Confusion Matrix & A table summarizing the performance of a classification model by showing counts of true positives, true negatives, false positives, and false negatives. & \cite{pranoto2015logistic, amer2021detection, Ashcroft_Kaati_Meyer_2015} \\
    \hline
        BLEU, ROUGE, BERTScore & Metrics evaluating machine-generated text quality, focusing on n-gram precision, recall, and contextual similarity. & \cite{Wang21-eancs} \\
    \hline
        Specificity & Measures the proportion of true negatives correctly identified by the model, relevant in classification tasks. & \cite{Munoz_Isaza_Castillo_2021} \\
    \hline
        Youden's Index & A single metric that balances sensitivity and specificity, aiding in the evaluation of diagnostic tests. & \cite{Eilifsen_Shrestha_Bours_2023} \\
    \hline
        Human Evaluation & Involves annotators comparing machine-generated content with human-created content to assess quality and naturalness. & \cite{Wang21-eancs} \\
    \hline
        Topic Coherence & Measures coherence of topics generated by modeling algorithms, indicating alignment with human understanding. & \cite{Lykousas_Patsakis_2022} \\
    \hline
    \end{tabular}
    \vspace{-5mm}
\end{table}


\begin{table}[t]
\small 
\centering
\caption{Datasets Used in the Reviewed Computational Science Cybergrooming Research}
\label{tab:datasets-computational}
\vspace{-3mm}
\begin{tabular}{p{3cm} p{5cm} p{5cm} P{1cm}}
\toprule
\textbf{Source} & \multicolumn{1}{c}{\textbf{Dataset Content}} & \multicolumn{1}{c}{\textbf{Dataset Description}} & \textbf{Year}\\ 
\hline

\textbf{Literotica} & 
User-generated erotic literature and conversation scripts. & 
Online platform hosting 1.25 billion words of erotic content, attracting 2.6–3.1 million unique visitors annually. & 
1996 \textcolor{green}{$\blacktriangle$} \\
\multicolumn{4}{l}{\textbf{Link:} \url{https://literotica.com/}} \\
\hline

\textbf{NPS Chat} & 
Chat messages with metadata on dates, age groups, and part-of-speech/dialog-act annotations. & 
Chat-based dataset for studying language patterns and behaviors associated with grooming and online risks. & 
2006 \textcolor{orange}{$\blacktriangle$} \\
\multicolumn{4}{l}{\textbf{Link 1:} \url{https://www.kaggle.com/datasets/nltkdata/nps-chat}}\\
\multicolumn{4}{l}{\textbf{Link 2:} \url{https://catalog.ldc.upenn.edu/LDC2010T05}} \\
\hline

\textbf{Perverted Justice} & 
Chat logs in CSV format with metadata on conversation dates and user interactions. & 
Dataset containing decoy-predator chat logs used for exposing online predators. & 
2012 \textcolor{lime}{$\blacktriangle$} \\
\multicolumn{4}{l}{\textbf{Link 1:} \url{https://web.archive.org/web/20240127030926/}} \\ 
\multicolumn{4}{l}{\textbf{Link 2:} \url{http://www.perverted-justice.com/}}\\
\multicolumn{4}{l}{\textbf{Link 3:} \url{https://ieee-dataport.org/documents/curated-pj-dataset}} \\
\hline

\textbf{PAN 2012} & 
XML file with 60,000 training and 155,000 test chat logs, plus predator identification files. & 
Dataset from PAN 2012 competition containing predator and non-predator conversations from IRC, Omegle, and PJ datasets. & 
2012 \textcolor{brown}{$\blacktriangle$} \\
\multicolumn{4}{l}{\textbf{Link:} \url{https://zenodo.org/records/3713280}} \\
\hline

\textbf{PAN 2013} & 
Blog posts labeled with author demographics (age, gender) in English and Spanish. & 
Dataset from PAN 2013 competition for author profiling and linguistic analysis across demographic groups. & 
2013 \textcolor{black}{$\blacktriangle$} \\
\multicolumn{4}{l}{\textbf{Link:} \url{https://zenodo.org/records/3715864}} \\
\hline

\textbf{LiveMe SLSS} & 
39 million chat messages from 1.4 million users during 293,271 live broadcasts (2016–2018). & 
Dataset analyzing viewer-streamer interactions and potential grooming behaviors in live streams. & 
2018 \textcolor{violet}{$\blacktriangle$} \\
\multicolumn{4}{l}{\textbf{Link:} \url{https://zenodo.org/records/3560365}} \\
\hline

\textbf{ChatCoder} & 
Chat transcripts between convicted predators and adults posing as minors, with labeled metadata. & 
Annotated dataset for analyzing cyber-predatory behavior and cyberbullying patterns. & 
2020 \textcolor{yellow}{$\blacktriangle$} \\
\multicolumn{4}{l}{\textbf{Link:} \url{https://chatcoder.com/}} \\
\bottomrule
\end{tabular}
\vspace{-5mm}
\end{table}

\subsection{Evaluation Datasets in Computational Science Cybergrooming Research} \label{subsec:eval-datasets}
Computational science research utilizes several key datasets to detect cybergrooming in online communications. These datasets provide essential training and evaluation resources for machine learning models, enabling the identification of predatory behaviors in digital interactions.

\ctriangle{lime} \textbf{Perverted Justice (PJ)}~\cite{pranoto2015logistic, sulaiman2019classification, Ashcroft_Kaati_Meyer_2015, bogdanova2014exploring, bours2019detection, cano2014detecting, Gunawan16, guo2023text, gupta2012characterizing, Wang21-eancs, anderson2019intelligent, elzinga2012analyzing, cook2023protecting, rezaee2023detecting, milon2022take, prosser2024helpful}: Collected by the Perverted Justice organization, this dataset comprises chat logs where adult decoys interact with potential predators posing as minors. The dataset is stored in CSV format and contains rich metadata, facilitating the study of grooming behaviors and language patterns in online conversations~\cite{PJ-dataset}. Despite its extensive use, its reliance on decoy interactions may introduce biases not entirely reflective of real-world grooming tactics.

\ctriangle{brown} \textbf{PAN 2012}~\cite{amer2021detection, Ashcroft_Kaati_Meyer_2015, Borj_Bours_2019, bours2019detection, Eilifsen_Shrestha_Bours_2023, fauzi2020ensemble, isaza2022classifying, Munoz_Isaza_Castillo_2021, rezaee2023detecting, milon2022take}: The PAN 2012 dataset~\cite{inches2012pan12} integrates chat logs from PJ, IRC logs, and Omegle, offering a diverse set of conversations. It includes both predator and non-predator interactions, making it valuable for training machine learning models to distinguish grooming behaviors. However, its platform-specific content may limit generalizability across different online environments.

\ctriangle{black} \textbf{PAN 2013}~\cite{anderson2019intelligent}: Originally designed for age and gender profiling, the PAN 2013 dataset~\cite{PAN13} contains blog posts labeled with author demographics. While not explicitly intended for grooming detection, it is often combined with PJ data to enhance linguistic diversity and improve models’ ability to generalize across various forms of online communication.

\ctriangle{green} \textbf{Literotica}~\cite{pranoto2015logistic, sulaiman2019classification}: The Literotica dataset~\cite{literotica} comprises user-generated erotic stories, totaling over a billion words. It provides a resource for studying language patterns associated with grooming and predatory narratives. However, as it consists primarily of fictional content, its application to real-world grooming detection remains limited.

\ctriangle{yellow} \textbf{ChatCoder2}~\cite{kim2020analysis, waezi2024osprey, amuchi2012identifying, michalopoulos2010towards, milon2022take, Michalopoulos14-gars, prosser2024helpful}: Developed as part of the NSF-funded "Tracking Predators" project, ChatCoder2 includes annotated chat logs of convicted predators interacting with decoys~\cite{chatcoder2}. This dataset allows for in-depth behavioral analysis, providing structured metadata that facilitates the identification of grooming patterns. However, like PJ, its reliance on decoy interactions may not fully capture the complexity of real grooming behaviors.

\ctriangle{lightgray} \textbf{Experimental Results}~\cite{oconnell2003typology, raihana2024implementation}: These datasets stem from experimental setups where researchers simulate child-user interactions in chat environments to study grooming stages. While these controlled environments allow for targeted data collection, they may lack the spontaneity and variability of real-world grooming attempts.

\ctriangle{red} \textbf{Questionnaire Data}~\cite{rita2021chatbot}: Surveys and self-reported data provide insights into public awareness, chatbot effectiveness in cybergrooming detection, and user experiences with online safety~\cite{rita2021chatbot}. While valuable, self-reporting biases and limited sample sizes may affect the reliability and generalizability of findings.

\ctriangle{violet} \textbf{LiveMe SLSS}~\cite{Lykousas_Patsakis_2022, fauzi2023identifying, michalopoulos2010towards, milon2022take, guo23-is, Michalopoulos14-gars}: The LiveMe SLSS dataset~\cite{lykousas2021large} comprises over 39 million chat messages collected from live broadcasts on the LiveMe platform. This dataset enables the study of real-time grooming behaviors and audience-streamer interactions. However, live-stream data is often noisy and unstructured, posing challenges for accurate annotation and analysis.

\ctriangle{orange} \textbf{NPS Chat}~\cite{bogdanova2014exploring, amuchi2012identifying}: The NPS Chat corpus~\cite{forsyth2010nps} contains chat room posts labeled with age information, part-of-speech tags, and dialogue acts. This dataset aids in understanding conversational patterns and linguistic cues in online interactions but does not specifically focus on grooming detection.


Table~\ref{tab:datasets-computational} in Section~\ref{sec:datasets_twofields} summarizes these datasets. Computational cybergrooming research relies heavily on annotated chat logs to train and evaluate detection algorithms. Key datasets such as PJ, PAN 2012, and PAN 2013 provide realistic conversational data, enabling researchers to develop and refine machine learning models. These datasets are particularly valuable for supervised learning, where labeled interactions help models distinguish between predatory and benign conversations. However, their reliance on decoy-based interactions introduces limitations in authenticity and may not fully capture the nuanced tactics of real predators.

Newer datasets, such as LiveMe SLSS and ChatCoder2, reflect a shift toward dynamic, real-world interactions. LiveMe SLSS focuses on live-streaming environments, introducing challenges related to real-time grooming detection and the evolving nature of grooming tactics. ChatCoder2, with its structured annotations, enhances behavioral analysis and facilitates model refinement. Despite their advantages, these datasets also present challenges, such as noise in live-streaming data and platform-specific biases that may limit generalizability.

In summary, computational datasets for cybergrooming research are instrumental in advancing detection models but are constrained by artificial setups, ethical concerns, and data authenticity limitations. Future research should prioritize access to more diverse, real-world datasets while maintaining ethical safeguards to capture the evolving nature of grooming tactics across different platforms.


\subsection{Methods to Mitigate the Risk of Bias in Computational Science Cybergrooming Research} \label{subsec:cs-methods-risk-bias}

Sampling bias and class imbalance bias are notable challenges in computational cybergrooming research. \textbf{\em Sampling bias} occurs when the sampling process disproportionately favors certain groups or outcomes, limiting generalizability~\cite{bogdanova2014exploring, Eilifsen_Shrestha_Bours_2023, gupta2012characterizing, Gunawan16, Munoz_Isaza_Castillo_2021, Wang21-eancs, amer2021detection}. To reduce this bias, researchers employ \textbf{\em random sampling} to select diverse subsets, improving representativeness. For instance, \citet{bogdanova2014exploring} randomly sampled 20 chat logs from the Cybersex and NPS chat corpora to build a testing dataset for detecting cyberpedophilia. \citet{Eilifsen_Shrestha_Bours_2023} implemented a \textbf{\em dynamic sampling method} that adjusts sliding window sizes based on conversation flow to capture a broader range of grooming behaviors. Similarly, \citet{Gunawan16} randomly selected grooming conversations from the PJ dataset and non-grooming conversations from Literotica to enhance model robustness. \citet{amer2021detection} applied \textbf{\em stratified sampling}, ensuring training data for artificial neural networks covered diverse sexual harassment scenarios and chat predator detection. These approaches minimize bias by creating datasets that better represent various online interactions.

\textbf{\em Class imbalance bias} arises when one class (e.g., predatory or non-predatory conversations) dominates, skewing model performance by favoring the majority class~\cite{Borj_Bours_2019, cano2014detecting, fauzi2020ensemble}. Studies address this issue using \textbf{\em fixed sampling}, \textbf{\em data augmentation}, and \textbf{\em synthetic data generation}. \citet{cano2014detecting} applied fixed sampling, adding a predefined number of samples, such as randomly selecting 3,304 sentences, to balance datasets. \citet{Borj_Bours_2019} leveraged the PAN12 dataset, which contains 64,911 regular and 2,016 predatory conversations, and attempted to reduce imbalance by selectively removing overrepresented conversations. Additionally, researchers integrate datasets such as ChatCoder2~\cite{kim2020analysis, milon2022take, Michalopoulos14-gars} and PAN2013~\cite{anderson2019intelligent} to enhance the representation of underrepresented classes. However, while these methods improve balance, they may oversimplify the complexity of real-world grooming interactions.

Models trained using \textbf{\em generative adversarial networks (GANs)} can simulate synthetic grooming conversations that mimic real-world interactions, providing more diverse training samples. This method reduces overfitting to specific linguistic patterns, ensuring that models remain effective across varied datasets. For instance, GANs have been employed to generate synthetic data in fraud detection, enhancing classifier performance by addressing class imbalance issues~\cite{strelcenia2023survey}.  Further, \textbf{\em reinforcement learning-based bias correction} can adjust decision boundaries dynamically, making classification systems more adaptable to nuanced grooming behaviors. A reinforcement learning framework has been introduced to mitigate biases acquired during data collection, thereby improving model fairness and adaptability~\cite{lewis2013reinforcement}. 


This analysis shows that random and fixed sampling techniques enhance data diversity and model generalization. However, relying on artificially balanced subsets may oversimplify real-world complexities. Future research should integrate \textbf{\em adversarial training, GAN-based data augmentation, and reinforcement learning strategies}~\cite{li2025novel} to refine class distributions without compromising the intricacies of grooming conversations. Combining these advanced techniques with traditional sampling methods can improve the fairness, accuracy, and adaptability of cybergrooming detection models.


\subsection{Summary of Key Findings in Computational Science Cybergrooming Research} \label{subsec:cs-summary-key-findings}

First, \textbf{\em high-level features} are crucial for detecting online grooming, including emotional markers, fixated discourse, and deceptive behavior, significantly improving accuracy in distinguishing grooming from other interactions~\cite{Ashcroft_Kaati_Meyer_2015, bogdanova2014exploring, kim2020analysis}. Computational research trends show a shift toward deep learning-based models, particularly recurrent neural networks (RNNs) and transformer architectures, which process sequential chat data with better contextual understanding. Advanced models, such as GRU and BiLSTM combinations, have achieved up to 99.12\% accuracy in harassment detection, underscoring their effectiveness in identifying harmful content~\cite{amer2021detection}. These findings highlight the increasing reliance on adaptive learning systems that evolve with grooming strategies.

Second, \textbf{\em context-aware detection systems} effectively identify predatory behaviors early. These systems use real-time NLP processing and sequential pattern analysis to monitor conversations and classify them as predatory or non-predatory~\cite{Ashcroft_Kaati_Meyer_2015, Borj_Bours_2019, Eilifsen_Shrestha_Bours_2023, bours2019detection, gupta2012characterizing, fauzi2020ensemble, Gunawan16, isaza2022classifying}. Visual representation methods, such as conversational heatmaps and time-series progression models, provide a dynamic view of evolving risks, enabling faster intervention and improved law enforcement tracking~\cite{elzinga2012analyzing}. The integration of explainable AI (XAI) into detection systems is also emerging, enhancing transparency and trust in automated solutions.

Third, \textbf{\em human-machine collaboration} helps reduce bias and prevent over-prediction in ML-based detection of predatory communication. While ML approaches are efficient, they struggle to differentiate deceptive grooming tactics from benign interactions. Incorporating \textit{human-in-the-loop validation}, where annotators intervene in ambiguous cases, improves precision with minimal effort compared to full human annotation~\cite{cook2023protecting}. This approach ensures edge cases, such as context-dependent conversations, are accurately classified, reducing false positives and negatives. The shift toward \textit{hybrid AI-human moderation} aligns with industry trends, where social media monitoring and child safety applications integrate human oversight to refine detection models.

Fourth, \textbf{\em victim characteristics} significantly influence grooming vulnerability. Identity and personality traits affect susceptibility, while cognitive resilience and strong social support serve as protective factors~\cite{guo2023text}. Studies highlight predictive grooming behaviors, such as reframing conversations, soliciting explicit images, and escalating interactions to sexual content, aiding detection and forensic investigations~\cite{pranoto2015logistic}. Computational models reveal distinct grooming strategies when targeting real adolescents versus adult impersonators (e.g., undercover officers or chatbots). Differences in risk assessment, meeting arrangements, and conversational control suggest that groomers tailor tactics based on perceived victim profiles~\cite{ringenberg2024assessing}. This insight informs law enforcement, emphasizing the need for \textit{adaptive decoy models} that mimic adolescent behaviors to enhance grooming intervention success.


Fifth, \textbf{\em interactive platforms and chatbots} are effective in preventing grooming and facilitating abuse reporting, demonstrating their growing role in educational and real-time intervention systems~\cite{rita2021chatbot, Wang21-eancs}. AI-driven conversational agents are being developed to engage potential victims in simulated grooming interactions, helping them recognize risks before they escalate~\cite{raihana2024implementation}. In addition to intervention, gamified educational platforms are emerging as a key trend in prevention efforts, enhancing children's awareness of grooming risks through interactive training modules.

Finally, \textbf{\em relationship formation} is a key grooming phase, with groomers initially building trust before introducing sexual content. The latest computational research suggests that early-stage relational analysis should be prioritized in detection models, focusing on sentiment shifts, grooming-specific linguistic markers, and progression modeling to identify risk before escalation~\cite{cano2014detecting, gupta2012characterizing, Lykousas_Patsakis_2022}. Practical applications include preemptive alert systems for parents and educators, where AI models detect early grooming attempts and generate warnings before explicit content is introduced.

These findings reflect \textbf{\em a broader shift in computational cybergrooming research toward real-time prevention, hybrid human-AI collaboration, and explainable AI systems}. Future research should focus on enhancing cross-platform grooming detection, improving bias mitigation techniques, and integrating multilingual capabilities to ensure broader applicability across diverse digital environments.

\subsection{Limitations \& Lessons Learned in Computational Science Cybergrooming Research}

Our review of cybergrooming research highlights numerous limitations and critical lessons that underline the complexity of addressing grooming behavior in computational settings. Researchers emphasize the \textbf{\em importance of awareness}, advocating for educational initiatives in schools to help students and educators recognize cybergrooming risks and equip them with preventative strategies~\cite{oconnell2003typology, rita2021chatbot, Wang21-eancs}. However, a significant limitation lies in the \textbf{\em limited generalizability} of detection models, which are often trained on narrow, specific datasets, challenging their applicability in varied real-world scenarios~\cite{bours2019detection, Eilifsen_Shrestha_Bours_2023, fauzi2020ensemble, gupta2012characterizing, elzinga2012analyzing, waezi2024osprey, amuchi2012identifying, fauzi2023identifying, milon2022take}. This limitation suggests a need for more \textbf{\em representative and diverse datasets} to capture the full scope of online interactions.

\textbf{\em Cross-cultural differences} further complicate detection efforts, as the interpretation and meaning of certain language patterns vary significantly across cultural contexts~\cite{rezaee2023detecting}. This calls for models adaptable to cultural nuances, which is essential in global contexts. Additionally, many computational approaches for cybergrooming detection demand \textbf{\em significant computing resources}, emphasizing the need for more resource-efficient algorithms, especially for use in low-resource environments where access to advanced hardware may be restricted~\cite{fauzi2020ensemble, Munoz_Isaza_Castillo_2021}.

The \textbf{\em lack of high-quality datasets} remains a major barrier in this field, with data scarcity, privacy constraints, and poor readability affecting training quality~\cite{Ashcroft_Kaati_Meyer_2015, bogdanova2014exploring, Borj_Bours_2019, bours2019detection, Eilifsen_Shrestha_Bours_2023, guo2023text, gupta2012characterizing, Wang21-eancs, amer2021detection, isaza2022classifying, sulaiman2019classification, pranoto2015logistic, anderson2019intelligent}. Additionally, model adaptability across \textbf{\em languages and linguistic variations} remains a challenge, as models must handle slang, regional dialects, and common misspellings to maintain accuracy in diverse online communities~\cite{Ashcroft_Kaati_Meyer_2015, Munoz_Isaza_Castillo_2021, Wang21-eancs, pranoto2015logistic, guo23-is}.

A critical yet often overlooked challenge is the \textbf{\em ethical concerns in machine learning-based grooming detection}. False positives, where benign conversations are mistakenly flagged as grooming, pose serious risks, particularly in legal and social contexts. Wrongful accusations can lead to reputational damage and legal consequences. Conversely, false negatives, or missed detections, allow predatory behavior to persist, endangering victims. To address these risks, researchers should explore \textbf{\em explainable AI (XAI)} to enhance transparency in model decisions, improving human oversight. Incorporating \textbf{\em human-in-the-loop systems}, where moderators validate high-risk predictions, can mitigate false positives while maintaining efficiency.

To overcome these limitations, \textbf{\em interdisciplinary collaboration} is essential. Integrating insights from psychology, sociology, and linguistics can enhance model development and refine detection capabilities. Partnering with educators, law enforcement, and child protection agencies can foster a holistic approach, combining technology with social awareness to more effectively counteract cybergrooming~\cite{ALKHATEEB201614, Ashcroft_Kaati_Meyer_2015, guo2023text, gupta2012characterizing, Munoz_Isaza_Castillo_2021, rita2021chatbot}. Additionally, future research should focus on differential privacy techniques to ensure user data protection while improving detection reliability. This cross-disciplinary engagement can help address the unique complexities of grooming detection and support the development of robust, adaptable, and ethically responsible solutions.

\section{Discussions}
\subsection{Multidisciplinary Approaches}

The study of cybergrooming from social and computational perspectives offers varied approaches with unique contributions to understanding and mitigating grooming risks. A \textbf{\em multidisciplinary approach} combines insights, theories, and methods across fields, aiming to address cybergrooming through integrated, comprehensive strategies that leverage both human expertise and algorithmic advancements.

First, \textbf{\em integrated models merge social science insights}, such as grooming behaviors and victim vulnerabilities, into computational algorithms to refine detection tools. By incorporating qualitative data into machine learning models, these frameworks increase the sensitivity and specificity of detection systems, resulting in more accurate classification of grooming behaviors~\cite{Wang21-eancs, street2024enhanced, Munoz_Isaza_Castillo_2021}. Successful interdisciplinary models, such as those integrating psycho-linguistic profiling with deep learning techniques, demonstrate that combining behavioral theories with computational classification improves the detection of grooming attempts. This review focuses on computationally grounded integrated models (Section~\ref{sec:cybergrooming-CS}) and their potential for broader application.

Second, \textbf{\em human-machine collaboration enhances detection accuracy} by combining machine learning efficiency with expert validation. While ML models can process vast datasets, they often struggle with context-dependent conversations and nuanced deception tactics. Incorporating expert review of ML outputs can improve precision in cybergrooming detection, reducing false positives and negatives~\cite{cook2023protecting}. One example is hybrid AI-human monitoring systems, where human annotators validate high-risk flagged interactions, refining algorithmic outputs. Future research should explore integrating XAI into these hybrid systems to provide transparency in detection decision-making.

Third, \textbf{\em collaboration between social and computational scientists strengthens policy and education.} Social studies provide context-specific insights, ensuring computational tools are culturally sensitive and empirically grounded. Computational findings, in turn, support policy-making and educational initiatives, equipping parents, educators, and policymakers with actionable data to safeguard vulnerable populations~\cite{kim2023ai}. Successful joint efforts, such as AI-driven educational chatbots used in prevention training programs, demonstrate the potential for interdisciplinary research to inform both detection and awareness strategies. Future initiatives should prioritize collaborative projects that integrate social science-driven risk assessments with real-time computational detection models.

To encourage more interdisciplinary research, funding initiatives such as the National Science Foundation (NSF) Secure and Trustworthy Cyberspace (SaTC) program and the European Horizon Cybersecurity Research Initiative offer potential avenues for supporting joint projects. Establishing research centers focused on AI ethics in online safety or cross-disciplinary cybersecurity research hubs can further promote cooperation between social and computational scientists. By fostering collaboration through dedicated funding and institutional support, future cybergrooming detection strategies can become more adaptable, culturally informed, and ethically sound, strengthening global efforts in prevention, detection, and policy development.


\subsection{Answers to Research Questions}
\label{sec:discussions}

\textbf{RQ1.} \textbf{\em What are the key approaches used to study cybergrooming in social and computational sciences?}

Social science research employs \textbf{\em qualitative methods}, such as interviews and literature reviews, to investigate victim profiles, grooming tactics, and the psychological impacts of cybergrooming. These approaches provide deep, contextual insights, uncovering perceptions and behaviors that quantitative methods might overlook. \textbf{\em Quantitative methods}, including surveys and statistical analyses, enable large-scale data collection and pattern identification, helping researchers understand risk factors and victim-offender dynamics across diverse demographics.

Computational science, in contrast, focuses on \textbf{\em machine learning (ML)} and \textbf{\em natural language processing (NLP)} to detect grooming behaviors and analyze communication patterns. ML models efficiently process vast datasets, identifying patterns and anomalies indicative of grooming activity, reducing the burden on human moderators. Techniques such as \textbf{\em sentiment analysis} assess emotional cues, \textbf{\em chatbot simulations} replicate grooming scenarios for training and prevention, and \textbf{\em automated detection algorithms} provide real-time monitoring. NLP enables the identification of subtle linguistic cues, such as persuasion and manipulation tactics. The methods used in cybergrooming research are summarized in Table~\ref{tab:evaluation-methods-social} and Table~\ref{tab:evaluation-methods-computational}, with Fig.~\ref{fig:datasets} providing an overview of the datasets employed.

\begin{wrapfigure}{r}{0.6\textwidth}
    \centering
    \includegraphics[width=\linewidth]{figs/Dataset.png}
    \caption{Overall trends of datasets in the reviewed cybergrooming research.}
    \label{fig:datasets}
    \vspace{-3mm}
\end{wrapfigure}
\textbf{RQ2.} \textbf{\em What are the primary contributions of existing cybergrooming research in both fields?}

Social science research provides a detailed understanding of the \textbf{\em personal and psychological aspects} of grooming, shaping targeted prevention programs, and uncovering social dynamics that increase vulnerability. These studies support \textbf{\em policy development and education initiatives} by informing intervention strategies tailored to groomers' and victims' behaviors and motivations.

Computational research contributes by developing \textbf{\em scalable, real-time monitoring and detection systems} that process vast amounts of online interactions, identifying grooming behaviors and supporting law enforcement efforts. The integration of \textbf{\em machine learning models and NLP tools} enhances detection accuracy and efficiency, enabling timely intervention. While social science research refines our understanding of grooming behaviors, computational methods operationalize this knowledge into technological solutions that can be deployed at scale.

\begin{wrapfigure}{r}{0.6\textwidth}
    \centering
    \includegraphics[width=\linewidth]{figs/Limitation.png}
    \caption{Key limitations in the reviewed cybergrooming research.}
    \label{fig:limitations}
\vspace{-3mm}
\end{wrapfigure}
\textbf{RQ3.} \textbf{\em What limitations exist in the methodologies used across different disciplines?}

Social science research is often constrained by \textbf{\em reliance on self-reported data and small samples}, introducing biases and limiting generalizability. While qualitative methods offer rich insights, they are resource-intensive and susceptible to researchers' biases. Quantitative methods enable large-scale data collection. However, they depend on self-reports and require extensive resources for longitudinal studies.

Computational science faces challenges related to \textbf{\em dataset availability, biases, and model limitations}. Effective model training requires \textbf{\em extensive, diverse datasets}, which are often scarce due to privacy concerns. ML models may inherit biases from training data, leading to skewed results. Furthermore, these models risk \textbf{\em overfitting}, excelling in training but struggling with unseen data, particularly in different linguistic or cultural contexts. Algorithms also face difficulty in interpreting nuanced human interactions, such as sarcasm, idioms, and contextual deception. Risks of \textbf{\em false positives and false negatives} complicate the deployment of detection tools. Fig.~\ref{fig:limitations} provides an overview of these limitations.

\textbf{RQ4.} \textbf{\em How do social and computational studies on cybergrooming differ, and what interdisciplinary insights can enhance the mitigation of cybergrooming risks?}

Cybergrooming studies in social and computational sciences differ in \textbf{\em focus and methodologies}. Social sciences examine psychological, behavioral, and societal dimensions through interviews and surveys, providing insights into victim and perpetrator profiles. Computational sciences rely on \textbf{\em data-driven techniques}, including ML and NLP, for automated grooming detection.

Interdisciplinary collaboration enhances mitigation strategies by integrating \textbf{\em social science insights into computational models}. Social science research provides knowledge on grooming tactics, which can refine detection algorithms by improving sensitivity to subtle linguistic cues. Conversely, computational approaches offer \textbf{\em scalable tools for analyzing large datasets}, facilitating broader monitoring and intervention efforts. A combined approach fosters \textbf{\em ethical, culturally aware solutions}, ensuring adaptive detection systems that evolve with changing grooming tactics.

\textbf{RQ5.} \textbf{\em What gaps and challenges remain in cybergrooming research, and what future directions are needed?}

A critical gap is the need for \textbf{\em systematic integration of social and computational approaches} to leverage the strengths of both fields. While social sciences contribute behavioral insights, computational methods provide \textbf{\em scalable detection capabilities}. Groomers continually adapt their strategies, requiring \textbf{\em flexible, evolving detection models} that remain effective over time. Additionally, the \textbf{\em global nature of online interactions} highlights the importance of culturally inclusive solutions that can be applied across diverse contexts.

Future research should focus on \textbf{\em interdisciplinary frameworks} that facilitate collaboration between social scientists and computational researchers. Expanding \textbf{\em data availability and sharing mechanisms}, while ensuring privacy and ethical compliance, is essential for improving model training and validation. Developing \textbf{\em culturally adaptive machine learning models} that account for linguistic diversity will enhance detection accuracy across global digital environments. Furthermore, \textbf{\em community engagement and stakeholder involvement} in prevention strategy development can lead to more practical, widely accepted solutions. Lastly, integrating \textbf{\em multimodal data analysis}, including visual and auditory cues alongside text-based detection, can provide a more comprehensive understanding of grooming behaviors.

By bridging the gap between social science insights and computational scalability, future cybergrooming research can develop more effective, ethical, and adaptable mitigation strategies.

\subsection{Differences in the Used Datasets in Social and Computational Science Cybergrooming Research}
\label{sec:datasets_twofields}

We provided details of the datasets used in cybergrooming research, summarized in Tables~\ref{tab:datasets-social} and~\ref{tab:datasets-computational}. Datasets in both categories combine quantitative and qualitative data to offer comprehensive insights into online interaction patterns, focusing on the behaviors of both predators and victims. However, they differ in key aspects:

\begin{itemize}
    \item \textbf{What data to collect:} Social science focuses on personal surveys and self-reported experiences to capture victim perspectives and prevalence rates (e.g., Childhood Online Sexual Abuse in the US). In contrast, computational science relies on automated data collection from online platforms, emphasizing behavioral analysis and pattern recognition (e.g., chat logs from Literotica and NPS Chat).
    \item \textbf{Data usage:} Social science datasets provide detailed individual responses, offering in-depth psychological and sociological insights into grooming risks and victim vulnerability. Computational science datasets, however, are structured for large-scale processing, enabling broad pattern detection, real-time risk analysis, and machine learning-driven classification.
    \item \textbf{How to collect data:} Social science relies on direct interactions with participants through surveys and interviews, ensuring subjective depth and contextual understanding. Computational science, in contrast, employs automated techniques such as data mining, text analysis, and machine learning to extract insights from large digital communication datasets.
\end{itemize}

To further distinguish these datasets, we introduce a taxonomy based on their \textbf{\em scope and purpose}:

\begin{itemize}
    \item \textbf{General-purpose datasets:} These datasets capture a broad range of interactions and are used for diverse research questions. Examples include the NPS Chat dataset, which contains chat-based conversations used for linguistic analysis, and PAN12, providing predatory and non-predatory conversations for cybercrime research.
    \item \textbf{Domain-specific datasets:} These datasets focus on particular aspects of cybergrooming, such as law enforcement case studies or psychological assessments. For example, the Childhood Online Sexual Abuse in the US dataset specifically examines victimization experiences, while the Perverted Justice dataset contains decoy-predator chat logs primarily used in cybercrime detection.
    \item \textbf{Real-time datasets:} These datasets are continuously updated and reflect ongoing online interactions. Examples include the LiveMe SLSS dataset, which collects live-stream chat data, allowing researchers to study grooming behaviors in real-time environments.
\end{itemize}

Understanding these dataset categories provides a clearer framework for selecting appropriate datasets based on research objectives. Social science research benefits from domain-specific and general-purpose datasets to study human behavior and victimization trends, while computational science relies heavily on large-scale general-purpose and real-time datasets to train and refine machine learning models. The integration of insights from both fields can enhance cybergrooming research by balancing behavioral depth with automated scalability.

\section{Conclusions \& Future Research Directions}

In this work, we systematically examined cybergrooming research across both social and computational sciences, providing a comprehensive assessment of their strengths, limitations, insights, and lessons learned. This analysis supported future interdisciplinary research and fostered collaborative efforts to advance the field. Using the Preferred Reporting Items for Systematic Reviews and Meta-Analyses (PRISMA) methodology~\cite{page2021prisma}, we ensured clarity, transparency, and reproducibility, facilitating a structured evaluation of cybergrooming methods, datasets, evaluation metrics, challenges, and future research directions. By exclusively focusing on cybergrooming, we identified distinct research gaps and strongly advocated for interdisciplinary collaboration to deepen understanding and enhance mitigation strategies. Furthermore, we emphasized the importance of multidisciplinary integration, demonstrating how insights from social sciences could refine computational approaches, leading to culturally adaptive and more effective solutions for detecting and preventing cybergrooming.

\subsection{Summary of the Key Findings}

Our comprehensive review of cybergrooming research across social and computational sciences reveals several critical research findings that are essential for advancing both fields and guiding future research, policy, and practice.

{\em First}, \textbf{\em qualitative methods in social sciences} offer valuable but resource-intensive insights. Interviews and questionnaires help explore victim experiences, grooming tactics, and psychological impacts, but they are prone to self-report biases and require significant time and resources to conduct and analyze. The sensitive nature of cybergrooming also limits response reliability, as participants may withhold details. To improve data accuracy, researchers should incorporate {\em mixed-method approaches}, combining self-reports with behavioral data and longitudinal studies to track evolving victim experiences. Ethical considerations, including participant confidentiality and culturally sensitive research protocols, should remain a priority in designing future studies.

{\em Second}, \textbf{\em machine learning models} are effective yet highly dependent on data quality. Models like SVM, RNN, and CNN successfully detect grooming behaviors through text analysis but require large, well-labeled datasets, which can be difficult to obtain due to privacy concerns. Biases in training data can affect model fairness, underscoring the need for {\em transparent data collection, bias mitigation strategies, and adversarial testing frameworks}. Future research should focus on creating {\em privacy-preserving data-sharing mechanisms}, ensuring that detection models generalize across diverse digital platforms, languages, and conversational styles.

{\em Third}, \textbf{\em standard metrics have limitations} in handling imbalanced datasets. Accuracy, F1-score, precision, and recall must be carefully selected and weighted to mitigate false positives, which are particularly problematic in social and legal contexts. More tailored evaluation frameworks are needed to address challenges posed by skewed data distributions. Beyond standard performance metrics, {\em explainability and interpretability} should be prioritized to enhance trust in ML-based detection systems. Researchers and practitioners should adopt {\em XAI techniques} to improve transparency and facilitate the ethical deployment of automated detection models in law enforcement and online safety platforms.

{\em Lastly}, \textbf{\em NLP techniques}, though helpful, struggle with interpreting cultural and contextual subtleties. Sentiment analysis and linguistic profiling reveal emotional and psychological cues, yet models often miss nuances across cultures, underscoring the need for {\em adaptive, context-aware NLP tools}. Informal online language, slang, and code-switching further complicate detection, necessitating the development of {\em multilingual, real-world conversational datasets} to improve model robustness. Future research should explore the integration of {\em multimodal analysis}, incorporating voice tone, facial expressions, and behavioral cues alongside text analysis to enhance grooming detection accuracy.

These findings reinforce the need for an \textbf{\em interdisciplinary approach} that bridges social and computational sciences. Social science research provides the foundational understanding of grooming tactics and victim vulnerabilities, while computational models offer scalable, real-time detection capabilities. By integrating behavioral theories with advanced AI-driven methodologies, researchers can develop more effective, ethical, and contextually adaptive detection and prevention strategies. Policymakers should support cross-disciplinary collaborations through funding initiatives, ethical AI frameworks, and standardized data-sharing agreements, ensuring that technological solutions align with legal and societal needs. Practitioners, including educators and law enforcement, should leverage these interdisciplinary insights to implement more targeted intervention and prevention measures. Future efforts should prioritize collaborative research that enhances data accessibility, promotes cross-sector partnerships, and ensures the ethical deployment of AI-based grooming detection systems.

\subsection{Future Research Directions}

Building on this, we suggest the following \textbf{future research directions} with actionable steps and global considerations:

{\em First}, \textbf{\em broad stakeholder engagement} is critical for effective prevention strategies. Social scientists, educators, technology companies, and policymakers must collaborate to create prevention programs that fit real-world needs, ensuring practical implementation and increased public awareness. Establishing cross-sector partnerships can facilitate knowledge exchange, resource sharing, and the development of adaptive policies that evolve with emerging threats. Future research should explore the effectiveness of {\em community-driven intervention models}, addressing the question: \textit{How can localized, culturally specific prevention programs improve grooming awareness and mitigation efforts?} Empirical studies comparing interventions across different regions can help tailor policies to specific socio-cultural contexts.

{\em Second}, a \textbf{\em multidisciplinary approach} combining social and computational sciences is essential. Social science insights into behavior and grooming tactics can inform computational models, leading to technically robust and socially informed systems that enhance detection accuracy. Developing hybrid frameworks that integrate psychological theories with AI-driven analysis can improve predictive capabilities while fostering ethical AI design. Future studies should investigate {\em how AI models can be trained on linguistically and culturally diverse datasets}, ensuring applicability across multiple jurisdictions. A key research question is: \textit{How can machine learning models account for cultural and linguistic variations in grooming strategies while minimizing bias?} Experimental methodologies involving transfer learning and adversarial testing could be applied to assess model adaptability across different populations.

{\em Third}, \textbf{\em integrating qualitative and behavioral data} offers a fuller understanding of grooming. Merging interview and survey insights with behavioral indicators like digital footprints allows for a comprehensive view of grooming dynamics, improving detection and intervention. Future efforts should focus on scalable data fusion techniques to link self-reported experiences with large-scale behavioral datasets while addressing privacy concerns. A crucial research avenue involves {\em privacy-preserving behavioral analysis}, addressing the question: \textit{How can behavioral signals be used in cybergrooming detection while maintaining user privacy and ethical compliance?} Developing federated learning approaches that allow distributed data analysis without centralizing sensitive information could be a viable solution.

{\em Fourth}, \textbf{\em longitudinal studies} are needed to track grooming behavior evolution and the efficacy of interventions. Long-term studies help researchers understand lasting impacts and adapt strategies as grooming tactics evolve. Establishing continuous monitoring frameworks can enable early detection of emerging trends and allow for iterative refinement of countermeasures. Future research should focus on developing {\em real-time grooming behavior tracking systems}, investigating the question: \textit{How do grooming strategies change over time in response to increased detection efforts and policy interventions?} Methodologies such as real-time text analysis and adaptive monitoring systems that adjust detection thresholds based on evolving conversational patterns could provide valuable insights.

{\em Finally}, \textbf{\em interdisciplinary research} should guide policy development, balancing protection with privacy. Empirical insights from both fields can help policymakers design regulations that protect users while respecting privacy, creating safer digital spaces for vulnerable populations. Collaborative efforts between academia, industry, and regulatory bodies are necessary to develop adaptive, evidence-based policies that address dynamic cyber threats effectively. Future research should focus on {\em global legal harmonization in cybergrooming detection}, answering the question: \textit{How can international collaboration ensure that detection technologies align with legal frameworks across different jurisdictions?} Comparative studies on privacy laws and ethical AI governance in different countries could inform best practices for implementing effective yet legally compliant detection systems.

By addressing these research questions and leveraging interdisciplinary methodologies, future efforts can bridge the gap between theory and practice, ensuring that cybergrooming detection and prevention strategies are robust, ethical, and adaptable to diverse global contexts.

\begin{acks}
This work is partially supported by the Commonwealth Cyber Initiative (CCI) Southwest Virginia (SWVA) Cybersecurity Research Program and the National Science Foundation (NSF) Secure and Trustworthy Cyberspace (SaTC) program under grants 2330940 and 2330941.
\end{acks}

\bibliographystyle{ACM-Reference-Format}
\bibliography{main}

\appendix

\end{document}
