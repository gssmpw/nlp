\section{Limitations and Future Work}
\label{sec:limitations}

Just like previous exemplar-based generative methods, \ourmethod is limited in the shape variations it can generate: while our approach is not strictly patch-based, it is similarly restricted in its ability to consider widely different variants. %This makes it more likely to maintain the style, semantics, and aesthetics of the exemplar. 
Extending its range of alterations through data augmentation or more involved (equivariant) features remains an intriguing possibility that would broaden the applicability of our method. Moreover, we focused our approach on generating high-quality, detailed geometry and did not consider fine texture generation. While existing exemplar-based methods have proposed approaches to generate textures for meshes that we could apply as-is, we believe there may be other exciting possibilities to explore, such as fitting 2D Gaussian splats~\cite{Huang2DGS2024} within our finest voxels to enrich our geometry with radiance field reconstruction.

Now that we have proven the efficacy of explicit geometry encoding through colored points and normals for creating shapes in our exemplar context, it would be interesting to study its adequacy in the more general case of generative modeling from large datasets: its lightweight, surface-based nature may circumvent a number of issues plaguing current state-of-the-art approaches. 

\section{Conclusion}
\label{sec:conclusion}
We proposed a novel generative approach for generating high-quality and detailed 3D models from a single exemplar. Our approach stands out as the first 3D generative method based on an explicit encoding of geometry through points, normals, and optionally colors. Combined with sparse voxel grid, we demonstrated that both training and inference times are (at times drastically) reduced compared to previous methods, despite a significantly improved quality of our geometric outputs and an ability to deal seamlessly with closed or open surfaces alike. We thus believe that \ourmethod sets a new standard for the quality of geometric outputs in generative modeling. 

\section{Acknowledgments}
This work was supported by the French government through the 3IA Cote d’Azur Investments in the project managed by the National Research Agency (ANR-23-IACL-0001), Ansys, Adobe Research, and a Choose France Inria chair. \vspace*{-2mm}