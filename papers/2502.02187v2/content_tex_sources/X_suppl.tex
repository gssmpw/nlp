\clearpage
% \setcounter{page}{1}
\maketitlesupplementary

This supplementary material provides additional details, results, and comparisons.
% \tableofcontents % Automatically generates the table of contents



\begin{figure}[h] \vspace*{-2mm}
\centering
        % \includegraphics[width=.85\linewidth]{figures/ours_big_canyon.pdf}
        \begin{subfigure}{.32\linewidth}
        \centering
        \includegraphics[width=\linewidth]{figures/supp-fig1/gt_canyon.jpg}
        \caption{Input Geometry}
    \end{subfigure}
    \begin{subfigure}{.32\linewidth}
        \centering
        \includegraphics[width=\linewidth]{figures/supp-fig1/ours_canyon.jpg}
        \caption{Ours}
    \end{subfigure}
    \begin{subfigure}{.32\linewidth}
        \centering
        \includegraphics[width=\linewidth]{figures/supp-fig1/rodin-canyon.jpg}
         \caption{Rodin}
    \end{subfigure}
    
        \vspace*{-2mm}   
        \caption{\textbf{\ourmethod.} Given a 3D exemplar (left), we train a hierarchical diffusion model to create novel variations that preserve the geometric details and styles of the exemplar (center), whereas a large generative model such as Rodin~\cite{zhang2024clay}
        % , trained with thousands of shapes, loses all the details present in the input (right).
        tends to lose the geometric details present in the input (right).}
    \label{fig:ours_0}
      \vspace*{-3mm}   
\end{figure}

% m@: Nissim wants to kill this section; am ambivalent.

\section{Additional results and renderings}


We provide more results in Fig.~\ref{fig:ours_0}, \ref{fig:all_renderings0}, and~\ref{fig:all_renderings} to better illustrate the outputs of ShapeShifter on a variety of reference models. 
%(including new ones compared to the main paper). 
Note that we also show that \ourmethod can generate purely geometric variants from untextured meshes, see last result in Fig.~\ref{fig:all_renderings}.

\subsection{Comparison to SSG}
\label{sec:add_method}
We provide additional comparisons with SSG~\cite{wu2022learning}, which is a 3D generalization of SinGAN~\cite{shaham2019singan} trained on multiscale triplane occupancy fields. Tab.~\ref{tab:ssfid2} shows quantitative evaluation on models for which SSG provides publicly available outputs, demonstrating the higher quality of \ourmethod. Furthermore, we demonstrate in Fig.~\ref{fig:ssg} that the typical results of this GAN-based method exhibit exaggerated smoothness like in all existing techniques, and often suffer from voxelized artifacts as well.

\begin{figure}[!h]
\centering
 \begin{subfigure}{.32\linewidth}
        \centering
        \includegraphics[width=\linewidth]{figures/supp-ssg2/gt_acropolis.jpg}
    \end{subfigure}
    \begin{subfigure}{.32\linewidth}
        \centering
        \includegraphics[width=\linewidth]{figures/supp-ssg2/ours_acropolis.jpg}
    \end{subfigure}
    \begin{subfigure}{.32\linewidth}
        \centering
        \includegraphics[width=\linewidth]{figures/supp-ssg2/ssg_acropolis.jpg}
    \end{subfigure}

    \begin{subfigure}{.32\linewidth}
        \centering
        \includegraphics[width=\linewidth]{figures/supp-ssg2/gt_acropolis_detail.jpg}
    \end{subfigure}
     \begin{subfigure}{.32\linewidth}
        \centering
        \includegraphics[width=\linewidth]{figures/supp-ssg2/ours_acropolis_details.jpg}
    \end{subfigure}
    \begin{subfigure}{.32\linewidth}
        \centering
        \includegraphics[width=\linewidth]{figures/supp-ssg2/ssg_acropolis_details.jpg}
    \end{subfigure}
    
 \begin{subfigure}{.32\linewidth}
        \centering
        \includegraphics[width=\linewidth]{figures/supp-ssg2/gt_wood.jpg}
    \end{subfigure}
    \begin{subfigure}{.32\linewidth}
        \centering
        \includegraphics[width=\linewidth]{figures/supp-ssg2/ours_wood.jpg}
    \end{subfigure}
    \begin{subfigure}{.32\linewidth}
        \centering
        \includegraphics[width=\linewidth]{figures/supp-ssg2/ssg_wood.jpg}
    \end{subfigure}

\begin{subfigure}{.32\linewidth}
        \centering
        \includegraphics[width=\linewidth]{figures/supp-ssg2/gt_wood_detail.jpg}
        \caption{Input Geometry}
    \end{subfigure}
 \begin{subfigure}{.32\linewidth}
        \centering
        \includegraphics[width=\linewidth]{figures/supp-ssg2/ours_wood_detail.jpg}
        \caption{\ourmethod{}}
    \end{subfigure}
    \begin{subfigure}{.32\linewidth}
        \centering
        \includegraphics[width=\linewidth]{figures/supp-ssg2/ssg_wood_detail.jpg}
        \caption{SSG}
    \end{subfigure}

    \caption{\textbf{Visual inspection of SSG~\cite{wu2022learning} results.} While SSG can generate 3D outputs with very short inference time, results are typically blobby or overly smooth, with spurious artifacts often visible due to its voxel-based generation process. In contrast, our method generates better sharp edges and subtle details.}
    \label{fig:ssg} 
\end{figure}


\begin{table}[!h]
    \centering
    \resizebox{1.\linewidth}{!}{\begin{tabular}{ l |c|  c  c  c  c }\toprule
Metric & Method & acropolis  & house & small-town & wood \\\midrule

\multirow{2}{*}{G-Qual. $\downarrow$} & SSG  &  2.81 & 0.91 & 1.71 & 0.07 \\
&\ourmethod{}  &  \textbf{0.01}   & \textbf{0.01}   &    \textbf{1.00}   &  \textbf{0.02}\\

\midrule
\multirow{2}{*}{G-Div. $\uparrow$} & SSG  &  \textbf{0.081}   &    \textbf{0.01}   &    0.19   &    \textbf{0.11}   \\
&\ourmethod{}  &  0.04   &    \textbf{0.01}    &   \textbf{0.60}   &     0.08 \\

\bottomrule

\end{tabular}

}
    \vspace*{-1mm}
    \caption{\textbf{Evaluating geometric quality and diversity using SSFID and pairwise IoU scores.} As we discussed in the main paper and in \cref{sec:add_metrics}, both metrics have their blindspots: SSFID tends to overlook geometric details, while pairwise IoU systematically rewards artifacts.} 
    \label{tab:ssfid2}
\end{table}

% \subsection{\!Data-intensive {\small \bf vs.} \!exemplar-based {\small \bf 3D} generation}
\subsection{Data-intensive vs. exemplar-based generation}

In this section, we discuss the value of exemplar-based 3D generation in light of the recent advancements in 3D generation models trained on millions of examples. The latter can be used to create highly diverse 3D assets and provide intuitive user controls through simple text and images. However, such models require immense computational resources for training and inference. Yet, as shown in \cref{fig:ours_0}, 
% even with its astounding 1.5 billion parameters, 
the state-of-the-art generator Rodin~\cite{zhang2024clay} (1.5B parameters) fails to create convincing geometric details comparable to those generated by our model.

Furthermore, the control provided by such models is limited, as the generation can only adhere to extremely coarse guidance. For example, in ~\cref{fig:ours_0}, we use the exemplar mesh as part of the inputs to Rodin for a conditioned generation. However, the output (right) completely loses the styles and details present in the exemplar mesh.

% \begin{figure}[h] \vspace*{-3mm}
% \centering
%         \includegraphics[width=.7\linewidth]{figures/supp-ssg/006.jpeg}
%         \vspace*{-2mm}
%       \caption{\textbf{Text-to-3D Generative Modeling.} While a prompt in Rodin~\cite{Wang2022RODINAG} asking for a ``\emph{a highly detailed canyon}'' returns a very convincing 3D canyon example, the output 3D mesh has little to no geometric details. From this limitation stems the need for one-shot methods, that generate variants from a highly-detailed, curated input model, to return high-quality output surfaces. \vspace*{-2mm}}
%     \label{fig:Rodin}
% \end{figure}


% For context, Fig.~\ref{fig:Rodin} shows a typical result of a text-to-3D approach (here, we used Rodin~\cite{Wang2022RODINAG}). Even a text prompt asking for ``\emph{a highly detailed canyon}'' only provides a very smooth 3D model, which does represent a canyon but contains only limited geometric details. 
% This limitation has led to the design of one-shot methods, using a single high-quality and curated input model, which can generate new variants while preserving the fine geometric details of the input. 

\section{Additional comments on metrics}
\label{sec:add_metrics}

While we use the two commonly-used metrics (geometric quality and diversity through SSFID and pairwise IoU scores) to evaluate our results and compare them to prior art, a few comments are in order. 

First, the validity of these two scores is debatable. While geometric quality is arguably fair but cannot really gauge the diversity of the results, the measure of diversity itself is quite delicate to analyze. In a sense, the diversity score rewards noise, not just real diversity. For instance, ten grids of random binary values would get a diversity of 0.66, while ten grids of axis-aligned planes that are not overlapping would have a score of 1.0 --- so a diversity score mixes different properties. This partial inadequacy of the score is the reason why we state in the main paper that geometric quality and geometric diversity should really be considered together to infer the success of an approach. Moreover, we also point out that the diversity scores should be clearly smaller for very structured models (like the acropolis model) than for free-form or organic shapes; our results have scores in line with this expected behavior, which seems more meaningful than systematic high scores which would point to noise artifacts instead of good results. 

Second, we wish to point out that our scores of Sin3DM~\cite{wu2024sindm} are \emph{different} from the ones they publish. The reason is that Sin3DM applies a pre-processing step to make the input meshes watertight. \emph{This initial step systematically inflates small details and thin surfaces} such as the roof of the house or the entablature of the acropolis, which negates many of the advantages of one-shot generative modeling: it degrades (at times severely) the input, losing the very reason why creating variants of a carefully-designed input model is highly sought after, i.e., the high-quality geometry of the exemplar. So we compared their results to the unprocessed input models, and did not re-train their neural network because we assumed that they made their best efforts to fit ground-truth shapes. So one should be aware that the low geometric quality scores we provide reflect \emph{both} the degradations of the pre-processing step and of their SDF-based generative approach --- again, to account for the real use of these generative approaches. 

%\begin{itemize}
%    \item Sin3DM~\cite{wu2024sindm} use a pre-processing step to make the meshes watertight. 
%    \item It systematically inflates small details/thin surfaces such as the roof of the house
%    \item SDF also smooth small details
%    \item We did not re-train their method because we assume that they made their best efforts to fit GT shapes
%    \item We compute our metrics w.r.t. GT meshes because we want/can fit closely to the G.T., hence the higher scores than in their paper.
%\end{itemize}
%ABOUT IOU:
%\begin{itemize}
%    \item 10 grids of random binary values with probability .5 have a score of 0.6666
%    \item 10 grids with non-overlapping planes have a score of 1.0
%    \item Fidelity diversity is a tradeoff! 
%\end{itemize}


\begin{table}[t]
    \centering
   \resizebox{1.\linewidth}{!}{\begin{tabular}{ l |c  c  c  c  c }\toprule
 Method & Level 0  & Level 1 & Level 2 & Level 3 & Level 4  \\\midrule
\ourmethod{}  &  0.49 & 0.17 & 0.18 & 0.26 &  0.78 \\
Sin3DM  &  - &5.18  & - & - &- \\

% \ourmethod{} (DDPM)  &  4.77 & 1.66 & 1.75 & 2.63 &  8.13 \\
% Sin3DM (DDPM)  &  - &14.3 & - & - &- \\

\bottomrule

\end{tabular}

}
    \vspace*{-3mm}
    \caption{\textbf{Inference timing for generating a single variation.} We report the inference time at each level for generating a single variation and compare it with Sin3DM, which has a grid resolution equivalent to our second level (level 1). Note that  in the main paper, we reported the inference time for 10 variations instead. DDIM sampling is used for both methods.\vspace*{-3mm}} 
    \label{tab:timings}
\end{table}

\section{Inference timings}
\label{sec:add_timings}

In the main paper, the inference times for \ourmethod and Sin3DM are reported for the generation of 10 variants. Here, we provide the inference timing for generating a single variant (i.e., using a batch size of 1 instead of 10) as shown in ~\cref{tab:timings}.

Our method generates a single variant in less than 2 seconds: approximately 0.5 seconds for the coarsest level, followed by less than 1.5 seconds in total for the four finer levels. In comparison, Sin3DM requires around 5 seconds, while Sin3DGen takes over 3 minutes, excluding the time needed for optimizing the input plenoxels and converting them to a mesh. Notably, our method produces the coarsest level in under half a second, which can be directly splatted using \cite{ravi2020pytorch3d} (see the video for live demonstrations). In contrast, Sin3DM\cite{wu2024sindm} takes 5.18 seconds to process an equivalent grid size ($32^3$).

%NISSIM: I'M NOT SURE HOW WE CAN UNIFORMIZE TIMINGS WITH THE MAIN TEXT. In the main paper we said 10.7 sec for our method vs 15.8 = (3.8 (diffusion) + 4 (meshing) + 8 (texturing)) for SIN3DM with DDIM on a batch size 10.
\begin{figure}[h] \vspace*{-3mm}
\centering
    \begin{subfigure}{.49\linewidth}
        \centering
        \includegraphics[width=\linewidth]{figures/supp-QEM/yes_QEM.jpg}
        \caption{QEM averaging}
    \end{subfigure}
    \begin{subfigure}{.49\linewidth}
        \centering
        \includegraphics[width=\linewidth]{figures/supp-QEM/no_QEM.jpg}
         \caption{standard averaging}
    \end{subfigure}
    \caption{\textbf{QEM-averaging ablation.} While QEM-averaging (proposed in~\cite{maruani_ponq_2024}) keeps sharp features (like corners or spikes) in place helping our generative approach maintain these local details, a usual averaging would move the ``corner'' points inwards, increasing the probability of smoothing features out in generated variants.\vspace*{-1mm}}
    \label{fig:qem-ablation2}
\end{figure}


\section{QEM averaging}
\label{sec:qem}

Finally, we demonstrate why our use of QEM averaging during our fine-to-coarse analysis of the input models helps preserve sharp features of the ground truth. As Fig.~\ref{fig:qem-ablation2} demonstrates, standard scale-by-scale averaging of the points and normals from the finest sparse voxel grid all the way to the coarsest grid leads to drifts of the salient features: for instance, the bottom left corner of the house has migrated inwards, which may create rounding of the corner. Instead, applying the QEM averaging defined in the PoNQ method~\cite{maruani_ponq_2024} places the coarsest point on the corner, and of the intermediate points to remain right there as well --- resulting in outputs which will better preserve this geometric feature.


\begin{figure*}[!h]
\centering
    \begin{subfigure}{.16\linewidth}
        \centering
        \includegraphics[width=\linewidth]{figures/supp_all_renderings/acropolis_gt.jpeg}
    \end{subfigure}
    \begin{subfigure}{.16\linewidth}
        \centering
        \includegraphics[width=\linewidth]{figures/supp_all_renderings/acropolis_gt_tex.jpeg}
    \end{subfigure}
    \unskip\ \vrule\ 
    \begin{subfigure}{.16\linewidth}
        \centering
        \includegraphics[width=\linewidth]{figures/supp_all_renderings/acropolis_0_gen.jpeg}
    \end{subfigure}
    \begin{subfigure}{.16\linewidth}
        \centering
        \includegraphics[width=\linewidth]{figures/supp_all_renderings/acropolis_0_gen_texture.jpeg}
    \end{subfigure}
    \begin{subfigure}{.16\linewidth}
        \centering
        \includegraphics[width=\linewidth]{figures/supp_all_renderings/acropolis_1_gen.jpeg}
    \end{subfigure}
    \begin{subfigure}{.16\linewidth}
        \centering
        \includegraphics[width=\linewidth]{figures/supp_all_renderings/acropolis_1_gen_texture.jpeg}
    \end{subfigure}
    \vspace*{-5mm} 
    \begin{subfigure}{.16\linewidth}
        \centering
        \includegraphics[width=\linewidth]{figures/supp_all_renderings/canyon_gt.jpeg}
    \end{subfigure}
    \begin{subfigure}{.16\linewidth}
        \centering
        \includegraphics[width=\linewidth]{figures/supp_all_renderings/canyon_gt_tex.jpeg}
    \end{subfigure}
    \unskip\ \vrule\ 
    \begin{subfigure}{.16\linewidth}
        \centering
        \includegraphics[width=\linewidth]{figures/supp_all_renderings/canyon_0_gen.jpeg}
    \end{subfigure}
    \begin{subfigure}{.16\linewidth}
        \centering
        \includegraphics[width=\linewidth]{figures/supp_all_renderings/canyon_0_gen_texture.jpeg}
    \end{subfigure}
    \begin{subfigure}{.16\linewidth}
        \centering
        \includegraphics[width=\linewidth]{figures/supp_all_renderings/canyon_1_gen.jpeg}
    \end{subfigure}
    \begin{subfigure}{.16\linewidth}
        \centering
        \includegraphics[width=\linewidth]{figures/supp_all_renderings/canyon_1_gen_texture.jpeg}
    \end{subfigure}
    \vspace*{-5mm} 
    \begin{subfigure}{.16\linewidth}
        \centering
        \includegraphics[width=\linewidth]{figures/supp_all_renderings/fighting-pillar_gt.jpeg}
    \end{subfigure}
    \begin{subfigure}{.16\linewidth}
        \centering
        \includegraphics[width=\linewidth]{figures/supp_all_renderings/fighting-pillar_gt_tex.jpeg}
    \end{subfigure}
    \unskip\ \vrule\ 
    \begin{subfigure}{.16\linewidth}
        \centering
        \includegraphics[width=\linewidth]{figures/supp_all_renderings/fighting-pillar_0_gen.jpeg}
    \end{subfigure}
    \begin{subfigure}{.16\linewidth}
        \centering
        \includegraphics[width=\linewidth]{figures/supp_all_renderings/fighting-pillar_0_gen_texture.jpeg}
    \end{subfigure}
    \begin{subfigure}{.16\linewidth}
        \centering
        \includegraphics[width=\linewidth]{figures/supp_all_renderings/fighting-pillar_1_gen.jpeg}
    \end{subfigure}
    \begin{subfigure}{.16\linewidth}
        \centering
        \includegraphics[width=\linewidth]{figures/supp_all_renderings/fighting-pillar_1_gen_texture.jpeg}
    \end{subfigure}
    \vspace*{-5mm} 
    \begin{subfigure}{.16\linewidth}
        \centering
        \includegraphics[width=\linewidth]{figures/supp_all_renderings/house_gt.jpeg}
    \end{subfigure}
    \begin{subfigure}{.16\linewidth}
        \centering
        \includegraphics[width=\linewidth]{figures/supp_all_renderings/house_gt_tex.jpeg}
    \end{subfigure}
    \unskip\ \vrule\ 
    \begin{subfigure}{.16\linewidth}
        \centering
        \includegraphics[width=\linewidth]{figures/supp_all_renderings/house_0_gen.jpeg}
    \end{subfigure}
    \begin{subfigure}{.16\linewidth}
        \centering
        \includegraphics[width=\linewidth]{figures/supp_all_renderings/house_0_gen_texture.jpeg}
    \end{subfigure}
    \begin{subfigure}{.16\linewidth}
        \centering
        \includegraphics[width=\linewidth]{figures/supp_all_renderings/house_1_gen.jpeg}
    \end{subfigure}
    \begin{subfigure}{.16\linewidth}
        \centering
        \includegraphics[width=\linewidth]{figures/supp_all_renderings/house_1_gen_texture.jpeg}
    \end{subfigure}
    % \vspace*{-5mm} 
    \begin{subfigure}{.16\linewidth}
        \centering
        \includegraphics[width=\linewidth]{figures/supp_all_renderings/ruined-tower_gt.jpeg}
    \end{subfigure}
    \begin{subfigure}{.16\linewidth}
        \centering
        \includegraphics[width=\linewidth]{figures/supp_all_renderings/ruined-tower_gt_tex.jpeg}
    \end{subfigure}
    \unskip\ \vrule\ 
    \begin{subfigure}{.16\linewidth}
        \centering
        \includegraphics[width=\linewidth]{figures/supp_all_renderings/ruined-tower_0_gen.jpeg}
    \end{subfigure}
    \begin{subfigure}{.16\linewidth}
        \centering
        \includegraphics[width=\linewidth]{figures/supp_all_renderings/ruined-tower_0_gen_texture.jpeg}
    \end{subfigure}
    \begin{subfigure}{.16\linewidth}
        \centering
        \includegraphics[width=\linewidth]{figures/supp_all_renderings/ruined-tower_1_gen.jpeg}
    \end{subfigure}
    \begin{subfigure}{.16\linewidth}
        \centering
        \includegraphics[width=\linewidth]{figures/supp_all_renderings/ruined-tower_1_gen_texture.jpeg}
    \end{subfigure}
    % \vspace*{-5mm} 
    \begin{subfigure}{.16\linewidth}
        \centering
        \includegraphics[width=\linewidth]{figures/supp_all_renderings/small-town_gt.jpeg}
    \end{subfigure}
    \begin{subfigure}{.16\linewidth}
        \centering
        \includegraphics[width=\linewidth]{figures/supp_all_renderings/small-town_gt_tex.jpeg}
    \end{subfigure}
    \unskip\ \vrule\ 
    \begin{subfigure}{.16\linewidth}
        \centering
        \includegraphics[width=\linewidth]{figures/supp_all_renderings/small-town_0_gen.jpeg}
    \end{subfigure}
    \begin{subfigure}{.16\linewidth}
        \centering
        \includegraphics[width=\linewidth]{figures/supp_all_renderings/small-town_0_gen_texture.jpeg}
    \end{subfigure}
    \begin{subfigure}{.16\linewidth}
        \centering
        \includegraphics[width=\linewidth]{figures/supp_all_renderings/small-town_1_gen.jpeg}
    \end{subfigure}
    \begin{subfigure}{.16\linewidth}
        \centering
        \includegraphics[width=\linewidth]{figures/supp_all_renderings/small-town_1_gen_texture.jpeg}
    \end{subfigure}
    % \vspace*{-5mm} 
    \begin{subfigure}{.16\linewidth}
        \centering
        \includegraphics[width=\linewidth]{figures/supp_all_renderings/wood_gt.jpeg}
    \end{subfigure}
    \begin{subfigure}{.16\linewidth}
        \centering
        \includegraphics[width=\linewidth]{figures/supp_all_renderings/wood_gt_tex.jpeg}
    \end{subfigure}
    \unskip\ \vrule\ 
    \begin{subfigure}{.16\linewidth}
        \centering
        \includegraphics[width=\linewidth]{figures/supp_all_renderings/wood_0_gen.jpeg}
    \end{subfigure}
    \begin{subfigure}{.16\linewidth}
        \centering
        \includegraphics[width=\linewidth]{figures/supp_all_renderings/wood_0_gen_texture.jpeg}
    \end{subfigure}
    \begin{subfigure}{.16\linewidth}
        \centering
        \includegraphics[width=\linewidth]{figures/supp_all_renderings/wood_1_gen.jpeg}
    \end{subfigure}
    \begin{subfigure}{.16\linewidth}
        \centering
        \includegraphics[width=\linewidth]{figures/supp_all_renderings/wood_1_gen_texture.jpeg}
    \end{subfigure}
        \vspace*{-5mm} 

    \subfloat[Input Shape]{\hspace{.333\linewidth}}
      \subfloat[Generated Shape (ours)]{\hspace{.333\linewidth}}
        \subfloat[Generated Shape (ours)]{\hspace{.333\linewidth}}
    \caption{\textbf{Samples of our results I.} This figure shows a variety of input models and some of the generated variants (both shown without and with texture to facilitate visual inspection) ShapeShifter outputs.}
\label{fig:all_renderings0}
\end{figure*}



    
\begin{figure*}[!h]
    \centering
    \begin{subfigure}{.16\linewidth}
        \centering
        \includegraphics[width=\linewidth]{figures/supp_all_renderings/desert-rock_gt.jpeg}
    \end{subfigure}
    \begin{subfigure}{.16\linewidth}
        \centering
        \includegraphics[width=\linewidth]{figures/supp_all_renderings/desert-rock_gt_tex.jpeg}
    \end{subfigure}
    \unskip\ \vrule\ 
    \begin{subfigure}{.16\linewidth}
        \centering
        \includegraphics[width=\linewidth]{figures/supp_all_renderings/desert-rock_0_gen.jpeg}
    \end{subfigure}
    \begin{subfigure}{.16\linewidth}
        \centering
        \includegraphics[width=\linewidth]{figures/supp_all_renderings/desert-rock_0_gen_texture.jpeg}
    \end{subfigure}
    \begin{subfigure}{.16\linewidth}
        \centering
        \includegraphics[width=\linewidth]{figures/supp_all_renderings/desert-rock_1_gen.jpeg}
    \end{subfigure}
    \begin{subfigure}{.16\linewidth}
        \centering
        \includegraphics[width=\linewidth]{figures/supp_all_renderings/desert-rock_1_gen_texture.jpeg}
    \end{subfigure}
    \vspace*{-5mm} 
    \begin{subfigure}{.16\linewidth}
        \centering
        \includegraphics[width=\linewidth]{figures/supp_all_renderings/tree_gt.jpeg}
    \end{subfigure}
    \begin{subfigure}{.16\linewidth}
        \centering
        \includegraphics[width=\linewidth]{figures/supp_all_renderings/tree_gt_tex.jpeg}
    \end{subfigure}
    \unskip\ \vrule\ 
    \begin{subfigure}{.16\linewidth}
        \centering
        \includegraphics[width=\linewidth]{figures/supp_all_renderings/tree_0_gen.jpeg}
    \end{subfigure}
    \begin{subfigure}{.16\linewidth}
        \centering
        \includegraphics[width=\linewidth]{figures/supp_all_renderings/tree_0_gen_texture.jpeg}
    \end{subfigure}
    \begin{subfigure}{.16\linewidth}
        \centering
        \includegraphics[width=\linewidth]{figures/supp_all_renderings/tree_1_gen.jpeg}
    \end{subfigure}
    \begin{subfigure}{.16\linewidth}
        \centering
        \includegraphics[width=\linewidth]{figures/supp_all_renderings/tree_1_gen_texture.jpeg}
    \end{subfigure}
    \vspace*{-5mm} 
    \begin{subfigure}{.315\linewidth}
        \centering
        \includegraphics[width=0.7\linewidth]{figures/supp_all_renderings/vase_gt.jpeg}
    \end{subfigure}
    %\begin{subfigure}{.16\linewidth}
    %    \centering
    %    \includegraphics[width=\linewidth]{figures/supp_all_renderings/blank.jpeg}
    %\end{subfigure}
    \unskip\ \vrule\ 
    \begin{subfigure}{.32\linewidth}
        \centering
        \includegraphics[width=0.7\linewidth]{figures/supp_all_renderings/vase_0_gen.jpeg}
    \end{subfigure}
    % \begin{subfigure}{.16\linewidth}
    %     \centering
    %     \includegraphics[width=\linewidth]{figures/supp_all_renderings/blank.jpeg}
    % \end{subfigure}
    \begin{subfigure}{.32\linewidth}
        \centering
        \includegraphics[width=0.7\linewidth]{figures/supp_all_renderings/vase_1_gen.jpeg}
    \end{subfigure}
    % \begin{subfigure}{.16\linewidth}
    %     \centering
    %     \includegraphics[width=\linewidth]{figures/supp_all_renderings/blank.jpeg}
    % \end{subfigure}
     \centering
    \begin{subfigure}{.32\linewidth}
        \centering
        \includegraphics[width=\linewidth]{figures/supp_all_renderings/pig_gt.png}
    \end{subfigure}
    %\begin{subfigure}{.16\linewidth}
    %    \centering
    %    \includegraphics[width=\linewidth]{figures/supp_all_renderings/blank.jpeg}
    %\end{subfigure}
    \unskip\ \vrule\ 
    \begin{subfigure}{.32\linewidth}
        \centering
        \includegraphics[width=\linewidth]{figures/supp_all_renderings/pig_1_gen.png}
    \end{subfigure}
    %\begin{subfigure}{.16\linewidth}
    %   \centering
    %    \includegraphics[width=\linewidth]{figures/supp_all_renderings/blank.jpeg}
    %\end{subfigure}
    \begin{subfigure}{.32\linewidth}
        \centering
        \includegraphics[width=\linewidth]{figures/supp_all_renderings/pig_2_gen.png}
    \end{subfigure}
    %\begin{subfigure}{.16\linewidth}
    %    \centering
    %    \includegraphics[width=\linewidth]{figures/supp_all_renderings/blank.jpeg}
    %\end{subfigure}
    \vspace*{3mm} 

    \subfloat[Input Shape]{\hspace{.333\linewidth}}
      \subfloat[Generated Shape (ours)]{\hspace{.333\linewidth}}
        \subfloat[Generated Shape (ours)]{\hspace{.333\linewidth}}
\caption{\textbf{Samples of our results II.} This figure shows input models that were not used in the main paper, and some of the generated variants (both shown without and with texture to facilitate visual inspection) ShapeShifter outputs. Note that the last two examples (vase and pig) are an ablation test where we do not use colors among the per-voxel features in our approach.}
\label{fig:all_renderings}
\end{figure*}