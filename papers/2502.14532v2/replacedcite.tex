\section{Related work}
\label{sec:related}

The densest-subgraph problem has received considerable attention in the literature. Here, we discuss only the results that are most relevant to our work. 
We refer to the recent surveys by ____ and ____ for more discussion.

The problem of the densest subgraph with $k$~refinements (\dskc), as introduced in this paper, is a generalization of \emph{densest $k$-subgraph} (\dks)____.
Despite significant attention in the theory community, there is still a large gap between the best approximation algorithms and hardness results for 
\dks____,
where the best polynomial-time algorithm for \dks offers an approximation guarantee of $\bigO(n^{1/4+\varepsilon})$ for any constant $\varepsilon >0$____.

The unconstrained version of the densest-subgraph problem has been used in practice to find communities in graphs. 
____ has shown that this problem can be solved in polynomial
time and____ provides a $2$-approximation algorithm.  
Besides the case of static graphs, efficient methods have been designed for  dynamic____ and streaming settings____.

Many variants of the densest-subgraph problem have been studied. For instance, ____ consider finding a densely connected subgraph that contains a set of query nodes.
____ consider finding a dense subgraph whose vertices are close to a set of reference nodes and contain a small set of anchored nodes.
____ observe that, in practice, the densest subgraph is typically large and 
aim to find smaller, denser subgraphs by modifying the objective~function. 
Another line of work aims at finding subgraphs that are dense for multiple graph snap\-shots ____.

____ introduce the \kdense problem, where the goal is to add
$k$~vertices to a given subgraph to increase its density. 
They obtain $O(\sigma)$-approximation algorithms for graphs that have a $\sigma$-quasi-elimination~ordering. 
This class of graphs includes, for instance, chordal graphs. However, general graphs may not have a small $\sigma$-quasi-elimination~ordering.\footnote{Their algorithm requires a predefined $\sigma$.}
In contrast, our algorithms from Section~\ref{sec:densest-subgraph} provide constant-factor approximations for general graphs, but we need the assumption~$k=\Omega(n)$. 
As our hardness results show, this assumption is necessary to obtain $O(1)$-approximation algorithms assuming the 
Small Set Expansion Conjecture____.

The problem of max-cut with $k$~refinements (\maxcutkc) is a generalization of \emph{max-cut with cardinality constraints} ({\ccmaxcut})____, in which the size of the cut is restricted to be equal to $k$, for a given $k \in [n]$.
The most studied version of {\ccmaxcut} is known as \maxbis, in which $k=n/2$
____. 
The best polynomial-time approximation algorithm for {\ccmaxcut} achieves an approximation ratio of approximately $0.858$____.
This ratio is improved to approximately $0.8776$ for \maxbis 
____.
With respect to hardness of approximation for {\ccmaxcut}, the best result is due to____ who give hardness of approximation with $k=\tau n$  as a function of~$\tau$, assuming the Unique Games Conjecture____.
A matching approximation algorithm for $\tau \in (0,0.365) \cup (0.635,1)$ is given by ____.
Our approximation algorithm for \maxcutkc obtains the same approximation ratio. As we show that \maxcutkc generalizes \ccmaxcut, in the parameter setting above, our algorithms are also optimal under the Unique Games Conjecture____.

The problem of graph partitioning with $k$ refinements (\gpkc) is a
generalization of \emph{maximum graph partitioning}
(\maxgp)____, which generalizes several graph partitioning
	problems.
Besides the two problems \dks, \ccmaxcut that we generalize in this paper, two
other problems, namely, \emph{max-uncut with cardinality constraints} (\ccmaxuncut),
	  and \emph{vertex cover with cardinality constraints} (\ccvc) are also presented
	  as applications of \maxgp. 
All of the problems can be solved by \sdp-based approaches, and constant
approximation results are obtained assuming $k \in \Omega(n)$, where the
concrete approximation ratios depend on the value of~$k$. 

Finally, we remark that independently and concurrently with our project, ____ and ____ propose reconfiguration frameworks that have similarities with our OptiRefine framework.
In addition, ____ study the $r$-move-$k$-partition problem, where the goal is to \emph{minimize} the multi\-way cut among $k$ partitions by moving $r$ nodes' positions. 
We note that this line of research focuses on developing fixed-parameter
algorithms for different sets of problems. Hence, their results apply in the
setting of small values of~$k$, whereas we concentrate on large values of~$k$.
Therefore, the concrete results of these papers are incomparable with ours, but
highlight the importance of studying refinement problems.

\vspace{1mm}
\para{Structure of the paper.} 
This paper is structured as follows.
In Section~\ref{sec:densest-subgraph}, we define \dskc, and we illustrate its
connections to \kdense and \dks. Moreover, we present a black-box
solution for \dskc by applying a solver for \dks. 
Section~\ref{sec:max-cut} presents \maxcutkc, and we illustrate its connections
to \ccmaxcut. Moreover, we consider a black-box solution for \maxcutkc by applying a solver for \maxcut. 
In Section~\ref{sec:general-framework}, we use a general problem, graph
partitioning with $k$ refinements (\gpkc), that captures both \dskc and
\maxcutkc. We then propose an \sdp-based algorithm and
obtain constant approximation results assuming $k \in \Omega(n)$. 
In Section~\ref{sec:max-cut:sos}, we describe a sum-of-squares algorithm to optimally solve \maxcutkc under certain regimes of $k \in \Omega(n)$.
Section~\ref{sec:experiments} presents our experiments, in which we evaluate our
algorithms for \dskc using multiple datasets. Moreover, in the appendix, we present more
experimental evaluations for \dskc and \maxcutkc, as well as all
omitted proofs from the main text.