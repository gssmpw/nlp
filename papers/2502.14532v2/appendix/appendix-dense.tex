\section{Omitted content of Section~\ref{sec:densest-subgraph}}
\label{appendix:ds-proofs}

\subsection{Proof of Lemma~\ref{lem:densest-subgraph-generalization}}
Let $G = (V, E)$ and $k$ be the input for the \dks problem. 
Suppose that we have an algorithm~$\algo$ for the \dskc problem. 
We apply $\algo$ on~$G$ with~$U = \emptyset$ and with the same parameter $k$. 
The output of $\algo$ is a set~$C$ of size~$|C|=k$, which we use as the solution~$U$ for the \dks problem. 
Since we did not make any changes to $G$, the running time of $\algo$ on~$G$ is $\bigO(T(m,n,k))$. 
We obtain our approximation guarantees by observing that 
both problems have feasible solutions consisting of exactly 
$k$ vertices, and since $U = \emptyset$, the optimal solutions 
for both problems coincide.

\subsection{Proof of Corollary~\ref{cor:densest-subgraph-hardness}}
We conclude from Lemma~\ref{lem:densest-subgraph-generalization} that if there is no 
$\beta_k$-approximation algorithm for the \dks problem, 
then there also does not exist a $\beta_k$-approximation algorithm 
for the \dskc problem with the same 
parameter $k$.

\citet{DBLP:conf/stoc/LeeG22} showed
that the \dks problem for $k = \tau \cdot n$ is 
difficult to approximate within a factor of~$O(\tau \log(1/\tau))$ under the \emph{Small Set Expansion Conjecture}. 
This implies our corollary.

\subsection{Proof of Lemma~\ref{lemma:k-densify-local-changes-connection}}
Let $C_1^*$ be the optimal set of vertices selected for the \kdense problem.
Let $C_1$ be the set of vertices chosen by an $\alpha$-approximation algorithm for the \kdense problem.
Observe that $|C_1^*| = |C_1| = k$ and $\abs{E[U \cup C_1]} \geq \alpha \abs{E[U \cup C_1^*]}$.

Let $C^*$ be the optimal set of vertices selected for the \dskc problem. Notice that $\abs{C^*} = k$. 
Using the condition $k \leq c\abs{U}$, we find that $|U \symm C^*|\geq (1-c)|U|$ and $|U| \geq \frac{1}{(1 + c)} \abs{U \symm C^*}$. 
Consequently, $|U \symm C^*| \geq \frac{(1 - c)}{(1+c)}\abs{U \symm C_1^*} = \frac{(1 - c)}{(1+c)}\abs{U \cup C_1^*}$.
Notice that the equality holds because $k \leq n - \abs{U}$, and hence the
optimal set of vertices~$C_1^*$ for \kdense does not intersected with $U$,
implying that $U \symm C_1^* = U \cup C_1^*$. 

Since $C_1^*$ is the optimal set of vertices for the \kdense problem, we have $\abs{E[U \cup C_1^*]} \geq \abs{E[U \symm C^*]}$.

Based on the above analysis, we can derive the following inequality that illustrates the performance of the $\alpha$-approximation algorithm for the \kdense problem,

\begin{equation}
\begin{aligned}
\frac{\abs{E[U \symm C^*]}}{\abs{U \symm C^*}} 
&\leq \frac{(1+c)}{(1-c)} \cdot \frac{\abs{E[U \cup C_1^*]}}{\abs{U \cup C_1^*}} \\
&\leq \frac{1}{\alpha} \cdot \frac{(1+c)}{(1-c)} \cdot \frac{\abs{E[U \cup C_1]}}{\abs{U \cup C_1}} .\\
\end{aligned}
\end{equation}

The second inequality holds since $C_1$ is an $\alpha$-approximation solution for the \kdense problem. 
This concludes the proof, demonstrating that an $\alpha$-approximate solution for the \kdense problem implies a $\frac{1-c}{1+c}\alpha$-approximate solution for the \dskc problem.

\subsection{Pseudocode for solving \kdense and \dskc with a \dks solver (in Section~\ref{sec:densest-subgraph:black-box})}
\label{appendix:densest-subgraph:black-box:pseudocode}

The pseudocode of our black-box method on \kdense to \dks
from Section~\ref{sec:densest-subgraph:black-box}, 
is illustrated as Algorithm~\ref{algo:dense-black}.

\begin{algorithm}[H]
\caption{Solve \kdense using a solver for \dks}
\label{algo:dense-black}
\begin{algorithmic}[1]
\Require{$(G=(V,E,w),\OldSet,k)$, access to a solver for \dksplus}
\State Create a new graph $G_1 = ((V\setminus \OldSet)\cup \{i^*\}, E_1, w_1)$,
	where $i^*$ corresponds to the contracted vertices in $U$  
\State Set $E_1 \gets E[V\setminus \OldSet] \cup \{ (i^*,j) \colon (i,j)\in E \text{ for } i\in \OldSet, j\in V\setminus \OldSet\}$
\State Set $w_1(i,j) \gets w(i,j)$ for alledges $(i,j) \in E[V \setminus \OldSet]$
\ForAll{$(i^*,j) \in E_1$}
\State $w_1(i^*,j) \gets \sum_{i\in \OldSet \colon (i,j)\in E} w(i,j)$
\EndFor
\State Solve \dksplus in $G_1$ to obtain a set of vertices $U'$ of size $k+1$
\If{$i^* \in U'$}
\State $U' \gets U'\setminus \{i^*\}$
\Else
\State $j \gets \argmin_{j\in U'} \text{weighted degree of } j \text{ in } G[U \cup (U' \setminus \{i^*\})]$
\State $U' \gets U' \setminus \{j\}$
\EndIf
\State \Return{$U'$}
\end{algorithmic}
\end{algorithm}

\subsection{Proof of Lemma~\ref{lemma:k-densify-dks}}
\label{appendix:ds-proofs:kdensifydks}

We use the reduction from \kdense to \dksplus that we described in
Section~\ref{sec:relationships-densest}.  We present the pseudocode of the
reduction in Algorithm~\ref{algo:dense-black}.

We start by defining notation.
Let $G=(V,E,w)$ be the input graph for $\kdense$.
We let $G_1 = (V_1, E_1, w_1)$ be the graph defined in the reduction.
Formally, we set
$V_1 = (V \setminus U) \cup \{i^*\}$, where $i^*$ is the special vertex we newly
added to the graph, which represents the contracted vertices in $U$, and
$E_1 = E[V \setminus U] \cup \{(i^*, j)\colon (i,j) \in E \mbox{ for } i\in U, j \in V \setminus U\}$. 
We set $w_1(i^*,j) = \sum_{i \in U \colon j \in V \setminus U, (i,j)\in E} w(i,j)$. 

Now let $C_2^*$ be the optimal set of vertices selected for \dksplus
on $G_1$.  Let $C_2$ be the set of vertices selected for 
\dksplus on $G_1$ by an
$\alpha_{k+1}$-approximation algorithm.  Notice that $|C_2^*| = |C_2| = k+1$. 
Furthermore, we let $C_1^*$ denote the optimal set of vertices selected for
\kdense in $G$.

By definition $\frac{\abs{E_1[C_2]}}{k+1} \geq \alpha_{k+1} \frac{\abs{E_1[C_2^*]}}{k+1} \geq \alpha_{k+1} \frac{\abs{E_1[C_1^* \cup \{i^*\}]}}{k+1}$.
The first inequality follows from the $\alpha_{k+1}$-approximation algorithm,
the second inequality follows from
$\abs{E_1[C_2^*]} \geq \abs{E_1[C_1^* \cup \{i^*\}]}$
since $C_2^*$ is the optimal solution for \dksplus in $G_1$ and since
$C_1^*\cup\{i^*\}$ is a feasible solution for \dksplus in $G_1$.
It follows that $\abs{E_1[C_2]} \geq \alpha_{k+1} \abs{E_1[C_1^* \cup \{i^*\}]}$. 

Notice that if $i^* \in C_2$, then $\abs{C_2 \setminus \{i^*\}} = k$ and $C_2 \setminus \{i^*\}$ is a feasible solution for \kdense in $G$. 
If $i^* \notin C_2$, we have to pick a node $v \in C_2$ so that $C_2 \setminus \{v\}$ becomes a feasible solution for \kdense.
To obtain our approximation ratios, we we make the following case distinctions.

\emph{Case~1:} $i^* \in C_2$. The induced edge weights in the graph $G$ after adding
$C_2\setminus \{i^*\}$ to $U$ are given by $|E[U \cup (C_2 \setminus \{i^*\})]|$.
In addition, for the sum of edge weights induced by $C_2$ on $G_1$ we have that
$|E_1[C_2]|= |E[C_2 \setminus \{i^*\}]| + |E[U, C_2 \setminus \{i^*\}]|$. 
We then derive the following bound, 

\begin{equation}
    \begin{aligned}
        |E[U \cup (C_2 \setminus \{i^*\})]| &= |E[U]| + |E[C_2 \setminus \{i^*\}]| + |E[U, C_2 \setminus \{i^*\}]| \\
		&= |E[U]| + | E_1[C_2]| \\
		&\geq |E[U]| + \alpha_{k+1}|E_1[C_1^* \cup \{i^*\}]| \\
		&\geq \alpha_{k+1} (|E[U]| + |E_1[C_1^* \cup \{i^*\}]|) \\
		&= \alpha_{k+1} |E[U \cup C_1^*]|. 
    \end{aligned}
\end{equation}

\emph{Case~2:} $i^* \notin C_2$. 
Let $a$ be a node in $C_2$ with smallest weighted degree in $G_1[C_2]$, i.e., $a = \arg\min_{u \in C_2} \sum_{v \in C_2}w_1(u, v)$.
The weight of induced edges after adding $C_2 \setminus \{a\}$ to $U$, i.e.,
$|E[U \cup (C_2 \setminus \{a\})]|$, is then at least
$|E[U]| + \frac{k-1}{k+1}\abs{E_1[C_2 \cup \{i^*\}]}$. 
Now we obtain that:
\begin{equation}
    \begin{aligned}
		|E[U \cup (C_2 \setminus \{a\})]|&=|E[U]| + |E[C_2 \setminus \{a\}]| + |E[U, C_2 \setminus \{a\}]| \\
		&= |E[U]| + |E_1[(C_2 \setminus \{a\}) \cup \{i^*\}]| \\
        &\geq |E[U]| + \abs{E_1[C_2 \setminus \{a\}]} \\
		&\geq |E[U]| + \frac{k-1}{k+1}\abs{E_1[C_2]} \\
		&\geq  |E[U]| + \frac{k-1}{k+1} \alpha_{k+1}\abs{E_1[C_2^*]} \\
        &= |E[U]| + \frac{k-1}{k+1} \alpha_{k+1}\abs{E_1[C_1^* \cup \{i^*\}]} \\
		&\geq \frac{k-1}{k+1} \alpha_{k+1} (|E[U]| + \abs{E_1[C_1^* \cup \{i^*\}]}).\\
		&= \frac{k-1}{k+1} \alpha_{k+1} \abs{E[U \cup C_1^*]}.
    \end{aligned}
\end{equation}
As $|E[U\cup C_1^*]|$ is the optimal objective function value for \kdense, we
have finished the proof. 

\subsection{Proof of Theorem~\ref{thm:densest-subgraph-reduction}}
\label{proof:densest-subgraph-reduction}

The theorem follows directly from combining Lemmas~\ref{lemma:k-densify-dks}
and~\ref{lemma:k-densify-local-changes-connection}.
