\section{Discussion} \label{discussion}
\begin{wraptable}{r}{0.5\textwidth}
  \vspace{-1.0em}
  \caption{\textbf{ConvNova performance on Histone tasks in NT Benchmark.} Results are reported for models with 15\% and 100\% full sequence receptive fields. The best values are in bold.}
  \label{table:localdependency}
  \centering
  \resizebox{0.5\textwidth}{!}{
  \begin{tabular}{l>{\columncolor{low}}rrr}
    \toprule
    \rowcolor{gray!20} \textbf{Task}                &   \textbf{15\% Receptive Field} & \textbf{100\% Receptive Field}            \\
    % \multicolumn{7}{c}{xxx}                   \\
    % \cmidrule(r){1-7}
    \midrule
    H3K4me2             &      \textbf{57.45} \small{±2.27}  &  53.72 \small{±2.42}              \\
    H3K4me3             &     \textbf{67.15} \small{±0.93}   &     60.20 \small{±1.91}            \\ 
    H3K14ac             &     \textbf{70.71} \small{±2.32}   &     66.19 \small{±1.84} \\ 
    H3K9ac             &     65.49 \small{±1.83}   &     \textbf{68.10} \small{±1.91} \\
    \bottomrule
  \end{tabular}}
  \vspace{-1.1em}
\end{wraptable}
We utilize dilation to control the receptive field in histone tasks from the NT-benchmark. H3K4me2, H3K4me3, and H3K14ac demonstrate significantly improved classification accuracy when the receptive field covers approximately 15\% of the input length, compared to a global receptive field.

The enhancement effects of H3K14ac (Figure  \ref{fig:H3K14ac}) and H3K4me3 (Figure  \ref{fig:H3K4me3}) align with previous biological research~\citep{ramakrishnan2016counteracting, regadas2021unique}, which has shown that these histone marks are highly enriched around transcription start sites (TSS), indicating their strong localized characteristics. As illustrated in Figure  \ref{fig:H3K14ac}, the pronounced enrichment of H3K14ac and H3K4me3 within this localized region supports the idea that a smaller receptive field is sufficient for effective classification in these tasks.

In contrast, H3K9ac (Figure  \ref{fig:H3K9ac}) exhibits a more uniform distribution across the gene body, which explains the suboptimal performance of a small receptive field in this case.

Interestingly, H3K4me2 (Figure  \ref{fig:H3K4me2}) also shows an enhancement effect with a small receptive field despite its enrichment being predominantly located in the middle of genes.



This observation raises two intriguing possibilities:

\textbf{Gene Position Context:} The enrichment of H3K4me2 in the middle of genes may indicate its role in transcriptional regulation during elongation. This localization could facilitate the binding of transcriptional machinery, enhancing transcriptional efficiency. Additionally, since our data were collected through ChIP-chip experiments on cells before and after oxidative stress, this enrichment might reflect interactions with genes associated with the histone methyltransferase Set1, suggesting a complex relationship that requires further exploration.

\textbf{Contextual Locality:} Alternatively, the improved performance of H3K4me2 in small receptive fields may not be tied to its position within genes but rather could stem from its involvement in other biological processes not captured during the experimental conditions. The sequences associated with H3K4me2 might exhibit strong local characteristics in contexts outside the immediate transcriptional environment, potentially involving regulatory mechanisms that are yet to be fully understood.

These possibilities highlight the need for further exploration to clarify the specific roles of H3K4me2 and its implications in various biological contexts.