\section{Conclusion}\label{conclusion}
In this study, we revisit the convolutional paradigm for DNA modeling. Through extensive analysis and experimentation, we show that convolutional neural networks maintain a competitive edge in DNA modeling tasks. This reevaluation prompts a reconsideration of CNNs, a relatively early method, for such functions within the community. We
%'ve
have 
introduced ConvNova, featuring three key design elements: dilated convolution, gated convolution, and a dual-branch structure. With these innovations, ConvNova achieves state-of-the-art performance on multiple benchmarks with small model sizes.
Additionally, we have investigated the varying demands for receptive field sizes across different tasks. While some tasks exhibit expected behaviors, others may suggest the presence of undiscovered biological phenomena.

%The
One \textbf{limitation} is that this work does not consider the multi-species pre-training dataset. Furthermore, we have not considered the specific genomic regions used in pretraining; it is possible that training exclusively in known functional regions could lead to improved performance. Furthermore, the current foundation model benchmarks are primarily limited to classification tasks, which results in a lack of diversity in the types of tasks evaluated. The \textbf{ positive social impact} is that this work can accelerate DNA research. No \textbf{ negative social impact} is perceived.