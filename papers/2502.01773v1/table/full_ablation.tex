
\begin{table*}[!ht]
\centering
\scriptsize
% \hspace*{-1.2cm}
\begin{tabular}{lccccccccccccccccccccc}
\toprule
                    & avg.          & task SR      & GPU              & SGD                & stack     & sweep to  & put in    & close & drag  & screw  & put in   & place & put in      \\
Method              & SR $\uparrow$ &STD $\downarrow$& mem $\downarrow$ & time $\downarrow$  & cups      & dustpan   & drawer    & jar   & stick & bulb   & safe     & wine  & cupboard   \\ \midrule
% Method              &               &                  &                    & 100       & 100       & 100   & 100   & 100    & 100      & 100   & 100       & 100       \\

Ours                & 83            & $\pm14$         & 11GB              & 0.7s              &  87       &  84       &  56       &  80   &  93   &  63    &  98      &  99   &  88       \\
no coarse-to-fine \footnotemark &26 & $\pm27$         & 12GB              & 2.2s              &  20       & 100       &   4       &  29   &  12   &  18    &  27      &  10   &  13       \\
no cross, C2F, seg  & 39            & $\pm27$         & 16GB              & 0.9s              &   2       & 100       &  66       &  49   &  49   &   0    &  38      &  13   &  30       \\
no segmentation     & 80            & $\pm22$         & 11GB              & 0.7s              &  68       &  95       &  27       &  90   & 100   &  65    &  94      &  94   &  89       \\
no augmentation     & 77            & $\pm13$         & 11GB              & 0.7s              &  79       &  56       &  87       &  78   &  97   &  56    &  76      &  82   &  84       \\
\bottomrule
\end{tabular}
\caption{\textbf{Ablation study.} 
% on 9 RLBench tasks shows the importance of each component of our method, i.e., keyframe action through cross-correlation in Section \ref{sec:place_only}, coarse-to-fine action inference in Section \ref{sec: cross_correlation}, bi-equivariant data augmentation in Section \ref{sec:data_aug}, and in-hand segmentation in Section \ref{sec:in_hand_segmentation}. 
\underline{avg. SR} shows the average success rate. \underline{task SR STD} shows the standard deviation of the success rate. \underline{GPU mem} and \underline{SGD time} show the GPU memory consumption and time for 1 SGD step during training.}
\vspace{-0.5cm}
\label{table:full_ablation}
\end{table*}
